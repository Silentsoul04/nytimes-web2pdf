Sections

SEARCH

\protect\hyperlink{site-content}{Skip to
content}\protect\hyperlink{site-index}{Skip to site index}

\href{https://www.nytimes3xbfgragh.onion/section/style}{Style}

\href{https://myaccount.nytimes3xbfgragh.onion/auth/login?response_type=cookie\&client_id=vi}{}

\href{https://www.nytimes3xbfgragh.onion/section/todayspaper}{Today's
Paper}

\href{/section/style}{Style}\textbar{}A Brief History of `Karen'

\url{https://nyti.ms/2DjFEPC}

\begin{itemize}
\item
\item
\item
\item
\item
\item
\end{itemize}

Advertisement

\protect\hyperlink{after-top}{Continue reading the main story}

Supported by

\protect\hyperlink{after-sponsor}{Continue reading the main story}

\hypertarget{a-brief-history-of-karen}{%
\section{A Brief History of `Karen'}\label{a-brief-history-of-karen}}

In 1965, it was the third-most-popular baby name in the United States.
In 2018, it was the 635th --- and today it's even less popular. How did
Karens fall so far?

\includegraphics{https://static01.graylady3jvrrxbe.onion/images/2020/08/02/fashion/31NAMED-KAREN-art/00KAREN-articleLarge.jpg?quality=75\&auto=webp\&disable=upscale}

By Henry Goldblatt

\begin{itemize}
\item
  July 31, 2020
\item
  \begin{itemize}
  \item
  \item
  \item
  \item
  \item
  \item
  \end{itemize}
\end{itemize}

Ask a woman named Karen what she used to think of her name and you'll
hear phrases like ``generic,'' ``perfectly serviceable'' and ``an easy
name.''

In 2020, Karen is no longer ``an easy name.'' Once popular for girls
born in the 1960s, it then became a pseudonym for a middle-aged busybody
with a blond choppy bob who asks to speak to the manager. Now, the
moniker has most recently morphed into a symbol of racism and white
privilege.

A ``Karen'' now roams restaurants and stores, often without a mask
during this coronavirus era, spewing venom and calling the authorities
to tattle, usually on people of color and often putting them in
dangerous situations. And while this archetype had previously been
called
``\href{https://www.cnn.com/2018/06/25/us/permit-patty-san-francisco-trnd/index.html}{Permit
Patty}'' or
``\href{https://www.newsweek.com/bbq-becky-white-woman-who-called-cops-black-bbq-911-audio-released-im-really-1103057}{BBQ
Becky},'' ``Karen'' has stuck.

In fact, many news reports don't even bother to use a woman's actual
given name. \href{https://www.youtube.com/watch?v=qHqzg3vgW4I}{Whitefish
Karen} (named for her town in Montana) coughed on a couple when they
called her out for not wearing a mask inside a grocery store.
\href{https://www.metrotimes.com/news-hits/archives/2020/06/17/metro-detroits-own-kroger-karen-prevents-black-customer-from-leaving-the-parking-lot-in-viral-video}{Kroger
Karen}, named after the supermarket chain, blocked an African-American
mother's car so the woman couldn't leave the market's parking lot.
\href{https://sfist.com/2020/06/14/sf-karen-filmed-confronting-pacific-heights-man-over-writing-black-lives-matter-on-his-property/}{San
Francisco Karen} called the police on a Filipino man stenciling ``Black
Lives Matter'' \emph{on his own property}.

And, of course, the Queen of Karens ---
\href{https://www.nytimes3xbfgragh.onion/2020/06/14/nyregion/central-park-amy-cooper-christian-racism.html}{Amy
Cooper}, also known as Central Park Karen --- threatened and fabricated
accusations against a Black man after he politely asked her to put her
dog on a leash, as park rules stated.

For some women with the name Karen, these videos have made them
outraged, of course, but also, at times, ashamed.

``I remember hearing about names like Becky and thinking, `What if this
was my name, how would it feel?''' said Karen Scholl, a 47-year-old
writer in Columbus, Ohio, with whom I worked at a college newspaper more
than 20 years ago. ``It's just so embarrassing, honestly. But I can't
get bent out of shape. I have no control over it. There are people
losing their lives every day. If it's the only thing I have to be upset
about in this world, then good for me.''

Karen Chang, a Bay Area resident who works in business management, had
shrugged off early memes, but then the Amy Cooper video changed
everything for her.

``It was very upsetting, but I would sacrifice my name for the
visibility and awareness that incident generated,'' said Ms. Chang, who
is Asian-American. Indeed, she may do just that. She said she's
considering changing her name to ``KC'' after she and her fiancé
eventually wed. ``It has always been a term of endearment.''

Ms. Chang may be able to change her name, but if one San Francisco Board
of Supervisors member, Shamann Walton, **** has his way, a version of
``Karen'' will be immortalized into city law.

In early July, Mr. Walton
\href{https://www.nytimes3xbfgragh.onion/2020/07/24/briefing/caren-act-911-san-francisco.html}{introduced
the CAREN (Caution Against Racially Exploitative Non-Emergencies) Act}
(presumably he couldn't come up with a suitable word that began with K).
The bill would change the city's code to punish people who call 911 and
file false, racially biased complaints.

That's a step too far for Karen Ortiz-Orband, a Boston-area nurse who is
of Puerto Rican and Dominican descent. She supports the contents of the
proposal, but emailed Mr. Walton's office urging him to reconsider its
title.

``I asked him to be mindful of the fact that there are women named Karen
and people aren't differentiating between the two. And by naming this
bill as he has, he's doing exactly what the metaphorical Karen is doing
--- creating an opportunity for discrimination,'' said Ms. Ortiz-Orband,
who is in her late 40s.

And before you think, ``that's so Karen to complain about CAREN,'' Ms.
Ortiz-Orband asks you to imagine if it were your name: ``It's one thing
to make memes,'' she said. ``It's another when you start applying it to
laws. You're villainizing a name that people actually have and you're
putting these people at risk. When a woman acts like that name, you
should use her correct name.''

Karen Gormandy, a literary agent and arts studio manager in New York
City, said she doesn't take it personally when she hears her name used
in these contexts, ``because I assign that meme to white people. I'm
totally disconnected from it. I'm on the receiving end of this
misbehavior.'' Ms. Gormandy, 61, said, ``I feel as a person of color I
don't need to apologize and explain my name.''

Yet she said people sometimes avoid using her name when speaking to her:
``Some people use Becky instead,'' she said, laughing.

\hypertarget{the-origins-of-karen}{%
\subsection{The Origins of `Karen'}\label{the-origins-of-karen}}

But why the name Karen?

Robin Queen, the chairwoman of the linguistics department at University
of Michigan, has
\href{https://theconversation.com/how-karen-went-from-a-popular-baby-name-to-a-stand-in-for-white-entitlement-139644}{looked
closely} at this question and her exploration led her to, of all people,
Dane Cook.

His 2005 comedy album contains a riff called ``The Friend Nobody
Likes'': ``There is one person in a group of friends that nobody
likes,'' Mr. Cook says, using an expletive to emphasize how much they
are, in fact, disliked. ``They basically keep them there to hate their
guts. When that person is not around the rest of your little base camp,
your hobby is cutting that person down.'' As an ``example'' of this
person, he describes a woman named Karen.

Other antecedents include Amanda Seyfried's vacant Karen in ``Mean
Girls,'' who racistly spouts to Lindsay Lohan's Cady: ``If you're from
Africa, why are you white?'' A parody account on Reddit from late 2017
based on the rants of a spurned husband is also often cited as an early
driver, and highlights the sexism of the ``Karen'' trope.

Karen Grigsby Bates, the senior correspondent for the ``Code Switch''
podcast on NPR, said Karen's roots are anchored deep in American
folklore. Ms. Bates ---
\href{https://www.npr.org/transcripts/891177904}{who embarked on this
research} not because of her name, but because the phenomenon was ``a
convergence of gender, race, class, social upheaval and social media in
this great big tornado'' --- pointed to the term ``Miss Ann'' from the
antebellum and Jim Crow periods.

African-Americans used the term as code ``to refer to these unreasonable
white women,'' Ms. Bates said. She described Miss Ann as ``a woman who
knew her place in society, was complicit in maintaining it, and who was
at the upper end of the hierarchy. Even if she was a nice Miss Ann, she
was still upholding this system that said: `White womanhood above all
else, except white manhood.'''

Researchers also point to the demographic characteristics of the name
Karen. \href{https://www.ssa.gov/OACT/babynames/index.html}{According to
Social Security data}, Karen soared in popularity in the 1960s, peaking
as the third-most-popular baby name of 1965, but never had a resurgence.
The archetype is meant to evoke a woman of a certain age, but then again
Linda, Cynthia or Susan would, too.

That's where the Karen theories get geekily fascinating. Miriam Eckert,
who has a Ph.D. in linguistics and lives in Boulder, Colo., said that
the word ``Karen'' contains what's known as a ``voiceless plosive.''

``That's the K sound at the beginning of the word,'' Ms. Eckert said.
``When you say some consonants, like K or a T, there's a complete
blockage of airflow and a sudden release --- whereas a name like Cynthia
has no stops at all. Karen is kind of a harsh sound that you can really
spit out. And that aligns with the kind of person we are thinking of
when we talk about a `Karen.'''

\hypertarget{the-future-of-karen}{%
\subsection{The Future of `Karen'}\label{the-future-of-karen}}

But will it always? In 2018 --- the latest year for which data is
available --- Karen ranked as the 635th most popular girl's name,
alongside Elaine and Dallas. ``Nobody is going to name their kid this
now,'' Ms. Gormandy said. ``It's just going to disappear and then
somebody not knowing the history of any of this might decide it's a cool
name.''

Ms. Queen, the linguistics expert, agreed. ``Maybe in 50 years or so it
might come back.''

In the meantime, she thinks it could at some point fade from the
lexicon. ``The meaning gets so broad that it's going to stop having the
same power to make a particular critique,'' she said, pointing to
examples like ``basic,'' ``hot mess'' and ``Negative Nancy'' that faded
from the lexicon.

As a moniker, she said, ``I would be surprised to find it around a
decade from now.''

Advertisement

\protect\hyperlink{after-bottom}{Continue reading the main story}

\hypertarget{site-index}{%
\subsection{Site Index}\label{site-index}}

\hypertarget{site-information-navigation}{%
\subsection{Site Information
Navigation}\label{site-information-navigation}}

\begin{itemize}
\tightlist
\item
  \href{https://help.nytimes3xbfgragh.onion/hc/en-us/articles/115014792127-Copyright-notice}{©~2020~The
  New York Times Company}
\end{itemize}

\begin{itemize}
\tightlist
\item
  \href{https://www.nytco.com/}{NYTCo}
\item
  \href{https://help.nytimes3xbfgragh.onion/hc/en-us/articles/115015385887-Contact-Us}{Contact
  Us}
\item
  \href{https://www.nytco.com/careers/}{Work with us}
\item
  \href{https://nytmediakit.com/}{Advertise}
\item
  \href{http://www.tbrandstudio.com/}{T Brand Studio}
\item
  \href{https://www.nytimes3xbfgragh.onion/privacy/cookie-policy\#how-do-i-manage-trackers}{Your
  Ad Choices}
\item
  \href{https://www.nytimes3xbfgragh.onion/privacy}{Privacy}
\item
  \href{https://help.nytimes3xbfgragh.onion/hc/en-us/articles/115014893428-Terms-of-service}{Terms
  of Service}
\item
  \href{https://help.nytimes3xbfgragh.onion/hc/en-us/articles/115014893968-Terms-of-sale}{Terms
  of Sale}
\item
  \href{https://spiderbites.nytimes3xbfgragh.onion}{Site Map}
\item
  \href{https://help.nytimes3xbfgragh.onion/hc/en-us}{Help}
\item
  \href{https://www.nytimes3xbfgragh.onion/subscription?campaignId=37WXW}{Subscriptions}
\end{itemize}
