Sections

SEARCH

\protect\hyperlink{site-content}{Skip to
content}\protect\hyperlink{site-index}{Skip to site index}

\href{https://www.nytimes3xbfgragh.onion/section/technology}{Technology}

\href{https://myaccount.nytimes3xbfgragh.onion/auth/login?response_type=cookie\&client_id=vi}{}

\href{https://www.nytimes3xbfgragh.onion/section/todayspaper}{Today's
Paper}

\href{/section/technology}{Technology}\textbar{}Microsoft Said to Be in
Talks to Buy TikTok, as Trump Weighs Curtailing App

\url{https://nyti.ms/3ffaaHO}

\begin{itemize}
\item
\item
\item
\item
\item
\item
\end{itemize}

Advertisement

\protect\hyperlink{after-top}{Continue reading the main story}

Supported by

\protect\hyperlink{after-sponsor}{Continue reading the main story}

\hypertarget{microsoft-said-to-be-in-talks-to-buy-tiktok-as-trump-weighs-curtailing-app}{%
\section{Microsoft Said to Be in Talks to Buy TikTok, as Trump Weighs
Curtailing
App}\label{microsoft-said-to-be-in-talks-to-buy-tiktok-as-trump-weighs-curtailing-app}}

The discussions come as TikTok's ownership by a Chinese company is under
scrutiny by the White House and lawmakers.

\includegraphics{https://static01.graylady3jvrrxbe.onion/images/2020/08/01/business/31JPtiktok-print/merlin_170133723_92af7f95-2132-4ee6-bc4d-638fcf0dc8cd-articleLarge.jpg?quality=75\&auto=webp\&disable=upscale}

\href{https://www.nytimes3xbfgragh.onion/by/mike-isaac}{\includegraphics{https://static01.graylady3jvrrxbe.onion/images/2018/02/16/multimedia/author-mike-isaac/author-mike-isaac-thumbLarge.jpg}}\href{https://www.nytimes3xbfgragh.onion/by/ana-swanson}{\includegraphics{https://static01.graylady3jvrrxbe.onion/images/2018/12/10/multimedia/author-ana-swanson/author-ana-swanson-thumbLarge.png}}\href{https://www.nytimes3xbfgragh.onion/by/alan-rappeport}{\includegraphics{https://static01.graylady3jvrrxbe.onion/images/2018/06/12/multimedia/author-alan-rappeport/author-alan-rappeport-thumbLarge-v2.png}}

By \href{https://www.nytimes3xbfgragh.onion/by/mike-isaac}{Mike Isaac},
\href{https://www.nytimes3xbfgragh.onion/by/ana-swanson}{Ana Swanson}
and \href{https://www.nytimes3xbfgragh.onion/by/alan-rappeport}{Alan
Rappeport}

\begin{itemize}
\item
  July 31, 2020
\item
  \begin{itemize}
  \item
  \item
  \item
  \item
  \item
  \item
  \end{itemize}
\end{itemize}

SAN FRANCISCO --- TikTok, the Chinese-owned video app that has been
under scrutiny from the Trump administration, is in talks to sell itself
to Microsoft and other companies as President Trump weighs harsh actions
against the business, including forcing TikTok to divorce itself from
its parent company, ByteDance, said people with knowledge of the
discussions.

The powerful Committee on Foreign Investment in the United States, or
Cfius, has been examining ByteDance's 2017 purchase of Musical.ly, an
app that eventually morphed to become TikTok. The committee has decided
to order ByteDance to divest TikTok, and the government is engaged in
negotiations over the terms of the separation, according to a person
familiar with the administration's plans, who spoke on the condition of
anonymity. White House officials have said TikTok may pose a national
security threat because of its Chinese ownership.

On Friday, Treasury Secretary Steven T. Mnuchin, who leads the
committee, briefed the president on the divestment plan. But it remains
unclear what the president will do, including whether the U.S. would
apply a divestment order to all of TikTok's American operations and
whether its actions would affect the app's global business as well.

Mr. Trump is weighing several other courses of action, including an
executive order that could use the vast powers of the International
Emergency Economic Powers Act to bar certain foreign apps from American
app stores. The Trump administration has also considered whether to add
TikTok's parent to a so-called ``entity list,'' which would prevent it
from purchasing American products and services without a special
license, said people with knowledge of the matter. Discussions are
expected to continue into this weekend.

In his comments on Friday, Mr. Trump told reporters that there were ``a
couple of options'' with TikTok, including ``banning'' it. He added,
``But a lot of things are happening, so we'll see what happens. But we
are looking at a lot of alternatives with respect to TikTok.''

Later on Friday, Mr. Trump said he planned to take action as soon as
Saturday. He added that he was not leaning toward allowing an American
company to buy TikTok's U.S. operations.

It's unclear how advanced TikTok's talks to sell itself to Microsoft and
other companies are, but changing ownership is crucial for the app. The
United States is one of TikTok's major markets, so continued operations
in the country are a priority.

TikTok has discussed other scenarios to alleviate concerns by U.S.
officials. In one scenario, non-Chinese investors like Sequoia Capital,
SoftBank and General Atlantic could
\href{https://www.nytimes3xbfgragh.onion/2020/07/23/business/dealbook/tiktok-bytedance-investors-trump.html}{purchase
a majority stake in the app from ByteDance}, people familiar with the
discussions have said.

Any deal would likely be expensive. ByteDance's valuation recently stood
at around \$100 billion, according to the research firm PitchBook.

In a statement, TikTok did not address Mr. Trump's comments or any deal
talks. A spokeswoman said the app was confident in its long-term success
and that it was committed to protecting the privacy and safety of its
creators so they could ``bring joy to families.''

Microsoft declined to comment.

The discussions between Microsoft and TikTok were earlier reported by
Fox Business. Bloomberg earlier reported that President Trump was poised
to announce an order to force ByteDance to sell TikTok's U.S.
operations.

The developments reflect the increasing pressure on TikTok. For months,
lawmakers and the Trump administration have questioned whether the app
is susceptible to influence from the Chinese government, including
potential requests to censor material shared on the platform or to share
American user data with Chinese officials.

``It is well established at this point that apps that have granular
access to user data and location and other sensitive personal data are
very much on the radar of Cfius and can cause significant national
security concerns,'' said John P. Kabaelo, a lawyer who represents
companies in Cfius reviews.

TikTok generally collects similar amounts of data from mobile phones as
other social media apps, said security experts. But Christoph Hebeisen,
the director of security intelligence research at Lookout, a company
that focuses on the security of mobile devices, said U.S. officials are
concerned by the app because ``if the parent company is Chinese, which
it is in this case, they are under Chinese security law.''

He added, ``I don't think it is a stretch to think if China wanted to
access that data they would have a means to do so.''

TikTok is used by more than 800 million people around the world and is
especially popular with young people. Users can easily add music and
other audio tracks to their videos, which then often travel virally
across Facebook and Twitter.

As the app has become more popular, TikTok's Chinese offices have
swollen to thousands of employees. The company has also maintained a
U.S. presence, with offices in New York and Los Angeles.

In response to the heightened scrutiny from Washington, TikTok in May
hired a top
\href{https://www.nytimes3xbfgragh.onion/2020/05/18/business/media/tiktok-ceo-kevin-mayer.html}{Disney
executive, Kevin Mayer} to be its chief executive. The app has also
pledged to publicly reveal the algorithm that powers its app.

In addition, TikTok has bulked up its lobbying operation in Washington.
With help from prominent investors like SoftBank and General Atlantic,
it has hired the former head of the Internet Association, a trade group
that represents companies like Google and Facebook, and staff members
from prominent members from Congress.

The company has signed on more than 35 lobbyists, including David J.
Urban, a former West Point classmate of Secretary of State Mike Pompeo
and an ally of Mr. Trump. The company's lobbyists have highlighted
TikTok's American investors and Mr. Mayer's hire.

Sensing weakness, rivals like Facebook have homed in on lawmakers'
distrust of TikTok's Chinese ownership. Mark Zuckerberg, Facebook's
chief executive, has said that American companies like his would suffer
if the government put them at a competitive disadvantage against TikTok.

On Wednesday, with the chief executives of Amazon, Apple, Facebook and
Google
\href{https://www.nytimes3xbfgragh.onion/2020/07/29/technology/big-tech-hearing-apple-amazon-facebook-google.html}{testifying
in front of Congress} about their market power, Mr. Mayer defended
TikTok while pledging to do right by the U.S. government.

``The entire industry has received scrutiny, and rightly so. Yet we have
received even more scrutiny due to the company's Chinese origins,'' he
said in a
\href{https://newsroom.tiktok.com/en-us/fair-competition-and-transparency-benefits-us-all}{statement}.
``We believe it is essential to show users, advertisers, creators, and
regulators that we are responsible and committed members of the American
community that follows U.S. laws.''

Cfius has previously ordered companies to divest their acquisitions.
Congress had expanded the panel's purview in 2018 to include reviews of
transactions involving ``sensitive user data,'' The change was spurred
by concerns that foreign ownership of data gathered by apps and internet
sites could threaten national security.

In 2019, the Trump administration ordered a Chinese firm to
\href{https://www.nytimes3xbfgragh.onion/2019/03/28/us/politics/grindr-china-national-security.html}{relinquish
its control of Grindr}, the gay dating app, concerned that China might
use the information to blackmail American officials. The Chinese
company, Beijing Kunlun Technology,
\href{https://www.reuters.com/article/us-grindr-m-a-investors-exclusive/exclusive-grindrs-chinese-owner-nears-deal-to-sell-gay-dating-app-sources-idUSKBN20T0IR}{said
it reached a deal} with Cfius earlier this year to sell the app to an
investment group, San Vicente Acquisition LLC, though
\href{https://www.reuters.com/article/us-grindr-m-a-sanvicente-exclusive/exclusive-winning-bidder-for-grindr-has-ties-to-chinese-owner-idUSKBN2391AI}{Reuters
later reported} that the buyer had ties to the Chinese owner.

Mike Isaac reported from San Francisco, and Ana Swanson and Alan
Rappeport from Washington. David McCabe and Julian Barnes contributed
reporting from Washington.

Advertisement

\protect\hyperlink{after-bottom}{Continue reading the main story}

\hypertarget{site-index}{%
\subsection{Site Index}\label{site-index}}

\hypertarget{site-information-navigation}{%
\subsection{Site Information
Navigation}\label{site-information-navigation}}

\begin{itemize}
\tightlist
\item
  \href{https://help.nytimes3xbfgragh.onion/hc/en-us/articles/115014792127-Copyright-notice}{©~2020~The
  New York Times Company}
\end{itemize}

\begin{itemize}
\tightlist
\item
  \href{https://www.nytco.com/}{NYTCo}
\item
  \href{https://help.nytimes3xbfgragh.onion/hc/en-us/articles/115015385887-Contact-Us}{Contact
  Us}
\item
  \href{https://www.nytco.com/careers/}{Work with us}
\item
  \href{https://nytmediakit.com/}{Advertise}
\item
  \href{http://www.tbrandstudio.com/}{T Brand Studio}
\item
  \href{https://www.nytimes3xbfgragh.onion/privacy/cookie-policy\#how-do-i-manage-trackers}{Your
  Ad Choices}
\item
  \href{https://www.nytimes3xbfgragh.onion/privacy}{Privacy}
\item
  \href{https://help.nytimes3xbfgragh.onion/hc/en-us/articles/115014893428-Terms-of-service}{Terms
  of Service}
\item
  \href{https://help.nytimes3xbfgragh.onion/hc/en-us/articles/115014893968-Terms-of-sale}{Terms
  of Sale}
\item
  \href{https://spiderbites.nytimes3xbfgragh.onion}{Site Map}
\item
  \href{https://help.nytimes3xbfgragh.onion/hc/en-us}{Help}
\item
  \href{https://www.nytimes3xbfgragh.onion/subscription?campaignId=37WXW}{Subscriptions}
\end{itemize}
