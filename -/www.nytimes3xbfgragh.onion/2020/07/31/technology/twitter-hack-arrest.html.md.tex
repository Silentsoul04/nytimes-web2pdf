Sections

SEARCH

\protect\hyperlink{site-content}{Skip to
content}\protect\hyperlink{site-index}{Skip to site index}

\href{https://www.nytimes3xbfgragh.onion/section/technology}{Technology}

\href{https://myaccount.nytimes3xbfgragh.onion/auth/login?response_type=cookie\&client_id=vi}{}

\href{https://www.nytimes3xbfgragh.onion/section/todayspaper}{Today's
Paper}

\href{/section/technology}{Technology}\textbar{}Florida Teenager Is
Charged as `Mastermind' of Twitter Hack

\url{https://nyti.ms/3gi8w9n}

\begin{itemize}
\item
\item
\item
\item
\item
\item
\end{itemize}

Advertisement

\protect\hyperlink{after-top}{Continue reading the main story}

Supported by

\protect\hyperlink{after-sponsor}{Continue reading the main story}

\hypertarget{florida-teenager-is-charged-as-mastermind-of-twitter-hack}{%
\section{Florida Teenager Is Charged as `Mastermind' of Twitter
Hack}\label{florida-teenager-is-charged-as-mastermind-of-twitter-hack}}

The authorities arrested a 17-year-old who they said ran a scheme that
targeted the accounts of celebrities, including former President Barack
Obama and Elon Musk. Two others were also charged.

\includegraphics{https://static01.graylady3jvrrxbe.onion/images/2020/08/01/business/31twitter2-print/merlin_161161578_5dd24641-dd88-4782-a57d-fad4dd7bb08b-articleLarge.jpg?quality=75\&auto=webp\&disable=upscale}

By \href{https://www.nytimes3xbfgragh.onion/by/kate-conger}{Kate Conger}
and
\href{https://www.nytimes3xbfgragh.onion/by/nathaniel-popper}{Nathaniel
Popper}

\begin{itemize}
\item
  July 31, 2020
\item
  \begin{itemize}
  \item
  \item
  \item
  \item
  \item
  \item
  \end{itemize}
\end{itemize}

OAKLAND, Calif. --- One by one, the celebrity Twitter accounts posted
the same strange message: Send Bitcoin and they would send back double
your money. Elon Musk. Bill Gates. Kanye West. Joseph R. Biden Jr.
Former President Barack Obama. They, and dozens of others, were being
hacked, and Twitter appeared powerless to stop it.

While some initially thought the hack was the work of professionals, it
turns out the ``mastermind'' of one of the most high-profile hacks in
recent years was a 17-year-old recent high school graduate from Florida,
the authorities said on Friday.

Graham Ivan Clark was arrested in his Tampa apartment, where he lived by
himself, early Friday, state officials said. He faces 30 felony charges
in the hack, including fraud, and is being charged as an adult.

Two other people, Mason John Sheppard, 19, of the United Kingdom, and
Nima Fazeli, 22, of Orlando, Fla., were accused of helping Mr. Clark
during the takeover. Prosecutors said the two appeared to have aided the
central figure in the attack, who went by the name Kirk. Documents
released on Friday do not provide the real identity of Kirk, but they
suggest that it was Mr. Clark.

Mr. Clark was skilled enough to go unnoticed inside Twitter's network,
said Andrew Warren, the Florida state attorney handling the case.

``This was not an ordinary 17-year-old,'' Mr. Warren said.

Mr. Clark convinced one of the company's employees that he was a
co-worker in the technology department who needed the employee's
credentials to access the customer service portal, a criminal affidavit
from Florida said. By the time the hackers were done, they had broken
into 130 accounts and raised significant new questions about Twitter's
security.

Despite the hackers' cleverness, their plan quickly fell apart,
according to court documents. They left hints about their real
identities and scrambled to hide the money they'd made once the hack
became public. Their mistakes allowed law enforcement to quickly track
them down.

Less than a week after the incident, federal agents, search warrant in
hand, went to a home in Northern California, according to the documents.
There, they interviewed another youngster who admitted participating in
the scheme. The individual, who is not named in the documents because he
or she is a minor, gave authorities information that helped them
identify Mr. Sheppard and said that Mr. Sheppard had discussed turning
himself in to law enforcement.

Because Mr. Clark is under 18, he was charged by the Florida state
attorney in Tampa, rather than by federal authorities. His age also
means that many details of his case are being kept under wraps.

Federal authorities were already tracking Mr. Clark's online activity
before the Twitter hack, according to legal documents. In April, the
Secret Service seized over \$700,000 worth of Bitcoin from him, but it
was unclear why.

The documents released on Friday largely repeat what several hackers
involved in the attack
\href{https://www.nytimes3xbfgragh.onion/2020/07/17/technology/twitter-hackers-interview.html}{told
The New York Times} two weeks ago: The hack began early on July 15 as a
quiet scheme to steal and sell unusual user names.

But as the day wore on, the attack, led by Kirk, took over dozens of
accounts belonging to cryptocurrency companies and celebrities. Bitcoin
flowed into the hackers' accounts. The scheme netted Bitcoin worth more
than \$180,000, according to a New York Times estimate.

A special agent with an Internal Revenue Service investigative unit said
in a court filing that Mr. Sheppard participated in the hack while using
the screen name ``ever so anxious.'' A person using that name told The
Times a few days after the attack that he got involved because he wanted
to acquire unique Twitter user names.

``i just kinda found it cool having a username that other people would
want,'' ``ever so anxious'' said in a chat with The Times. He ultimately
brokered the sale of at least 10 addresses, such as @drug, @w and @L,
according to the indictment against him.

Mr. Fazeli is also accused of serving as a middleman, helping to sell
stolen Twitter accounts on the day of the attack under the user name
``Rolex.'' But the indictment provides few details on Mr. Fazeli's work
as a middleman.

By the time Twitter finally managed to stop the attack, the hackers had
tweeted from 45 of the accounts they had broken into, gained access to
the direct messages of 36 accounts, and downloaded full information from
seven accounts, the company said.

Mr. Fazeli and Mr. Clark were arrested on Friday. Mr. Sheppard has not
been arrested but is expected to be taken into custody, the F.B.I. said.

``While investigations into cyber breaches can sometimes take years, our
investigators were able to bring these hackers into custody in a matter
of weeks,'' said John Bennett, a special agent in charge with the F.B.I.
The investigation is still underway, and it is possible there will be
additional arrests, a bureau spokeswoman said.

The young men who participated in the breach come from a loose-knit
community of hackers who focus on account takeovers, cybersecurity
experts said. Using a practice known as SIM-swapping, they often target
telecom companies to compromise victims' phone numbers and intercept
login credentials.

The attackers targeted Twitter employees, stealing their account
credentials in order to gain access to an internal system that allowed
them to reset the passwords of most Twitter users. (Some users, like
President Trump, have extra security on their accounts to prevent
takeovers.)

``These people come trained to be efficient and creative at their attack
methods,'' said Allison Nixon, the chief research officer of the
security firm Unit 221B. ``They've realized there's this world of soft
targets.''

These hackers often focus on financial fraud, but their ability to gain
access to the accounts of political figures could attract new and
dangerous customers, Ms. Nixon said.

``One of the things that concerns me is that, as these actors continue
to refine their techniques and learn, they're going to realize that
there are other customers who will pay a lot more for things other than
a single-character user name,'' she said. ``I don't think they've even
scratched the surface of how much damage they could cause.''

In a
\href{https://twitter.com/TwitterComms/status/1289267856333402112}{statement},
Twitter thanked law enforcement for its ``swift actions'' and said it
would continue to cooperate with the investigation.

The relatively young age of the hackers did not come as a surprise to
security professionals who monitor the SIM-swapper community. Many of
the people drawn to it are teenagers who pursue unique user names
because controlling them conveys a sense of importance and clout.

``This activity is addictive in a way, it's a thrill,'' Ms. Nixon.
``Breaking into gigantic companies and stealing ridiculous amounts of
money is a huge thrill for them.''

Advertisement

\protect\hyperlink{after-bottom}{Continue reading the main story}

\hypertarget{site-index}{%
\subsection{Site Index}\label{site-index}}

\hypertarget{site-information-navigation}{%
\subsection{Site Information
Navigation}\label{site-information-navigation}}

\begin{itemize}
\tightlist
\item
  \href{https://help.nytimes3xbfgragh.onion/hc/en-us/articles/115014792127-Copyright-notice}{©~2020~The
  New York Times Company}
\end{itemize}

\begin{itemize}
\tightlist
\item
  \href{https://www.nytco.com/}{NYTCo}
\item
  \href{https://help.nytimes3xbfgragh.onion/hc/en-us/articles/115015385887-Contact-Us}{Contact
  Us}
\item
  \href{https://www.nytco.com/careers/}{Work with us}
\item
  \href{https://nytmediakit.com/}{Advertise}
\item
  \href{http://www.tbrandstudio.com/}{T Brand Studio}
\item
  \href{https://www.nytimes3xbfgragh.onion/privacy/cookie-policy\#how-do-i-manage-trackers}{Your
  Ad Choices}
\item
  \href{https://www.nytimes3xbfgragh.onion/privacy}{Privacy}
\item
  \href{https://help.nytimes3xbfgragh.onion/hc/en-us/articles/115014893428-Terms-of-service}{Terms
  of Service}
\item
  \href{https://help.nytimes3xbfgragh.onion/hc/en-us/articles/115014893968-Terms-of-sale}{Terms
  of Sale}
\item
  \href{https://spiderbites.nytimes3xbfgragh.onion}{Site Map}
\item
  \href{https://help.nytimes3xbfgragh.onion/hc/en-us}{Help}
\item
  \href{https://www.nytimes3xbfgragh.onion/subscription?campaignId=37WXW}{Subscriptions}
\end{itemize}
