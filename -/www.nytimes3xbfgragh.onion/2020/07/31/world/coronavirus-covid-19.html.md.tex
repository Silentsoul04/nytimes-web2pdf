Sections

SEARCH

\protect\hyperlink{site-content}{Skip to
content}\protect\hyperlink{site-index}{Skip to site index}

\href{https://www.nytimes3xbfgragh.onion/section/world}{World}

\href{https://myaccount.nytimes3xbfgragh.onion/auth/login?response_type=cookie\&client_id=vi}{}

\href{https://www.nytimes3xbfgragh.onion/section/todayspaper}{Today's
Paper}

\href{/section/world}{World}\textbar{}Coronavirus Live Updates: A
\$600-a-Week Lifeline for Unemployed Americans Expires After an Impasse
in Washington

\url{https://nyti.ms/2CW4m96}

\begin{itemize}
\item
\item
\item
\item
\item
\item
\end{itemize}

\href{https://www.nytimes3xbfgragh.onion/news-event/coronavirus?action=click\&pgtype=Article\&state=default\&region=TOP_BANNER\&context=storylines_menu}{The
Coronavirus Outbreak}

\begin{itemize}
\tightlist
\item
  live\href{https://www.nytimes3xbfgragh.onion/2020/07/31/world/coronavirus-covid-19.html?action=click\&pgtype=Article\&state=default\&region=TOP_BANNER\&context=storylines_menu}{Latest
  Updates}
\item
  \href{https://www.nytimes3xbfgragh.onion/interactive/2020/us/coronavirus-us-cases.html?action=click\&pgtype=Article\&state=default\&region=TOP_BANNER\&context=storylines_menu}{Maps
  and Cases}
\item
  \href{https://www.nytimes3xbfgragh.onion/interactive/2020/science/coronavirus-vaccine-tracker.html?action=click\&pgtype=Article\&state=default\&region=TOP_BANNER\&context=storylines_menu}{Vaccine
  Tracker}
\item
  \href{https://www.nytimes3xbfgragh.onion/interactive/2020/07/29/us/schools-reopening-coronavirus.html?action=click\&pgtype=Article\&state=default\&region=TOP_BANNER\&context=storylines_menu}{What
  School May Look Like}
\item
  \href{https://www.nytimes3xbfgragh.onion/live/2020/07/31/business/stock-market-today-coronavirus?action=click\&pgtype=Article\&state=default\&region=TOP_BANNER\&context=storylines_menu}{Economy}
\end{itemize}

Advertisement

\protect\hyperlink{after-top}{Continue reading the main story}

Supported by

\protect\hyperlink{after-sponsor}{Continue reading the main story}

LIVE UPDATES

Updated~

Aug. 1, 2020, 12:54 a.m. ET

Aug. 1, 2020, 12:54 a.m. ET

\hypertarget{coronavirus-live-updates-a-600-a-week-lifeline-for-unemployed-americans-expires-after-an-impasse-in-washington}{%
\section{Coronavirus Live Updates: A \$600-a-Week Lifeline for
Unemployed Americans Expires After an Impasse in
Washington}\label{coronavirus-live-updates-a-600-a-week-lifeline-for-unemployed-americans-expires-after-an-impasse-in-washington}}

California became the first state to reach 500,000 total cases. Once the
site of a major outbreak, Italy now offers lessons for keeping the virus
in check.

Right Now

The leader of a religious sect in South Korea was arrested on charges of
embezzling and conspiring to impede efforts to fight the coronavirus.

\hypertarget{heres-what-you-need-to-know}{%
\subsubsection{Here's what you need to
know:}\label{heres-what-you-need-to-know}}

\begin{itemize}
\tightlist
\item
  \protect\hyperlink{link-7c4d159d}{Tens of millions of jobless
  Americans are losing a benefit that helped keep them afloat.}
\item
  \protect\hyperlink{link-4e17d805}{California's summer outbreak makes
  it the first state with half a million cases.}
\item
  \protect\hyperlink{link-65fa7f74}{Giroir, Trump's testing czar, said
  most virus test results were coming back quickly. Public health
  experts disagree.}
\item
  \protect\hyperlink{link-2b88e858}{South Korea arrests the leader of a
  church where the virus spread rapidly.}
\item
  \protect\hyperlink{link-3bb771a7}{Florida, already reeling from the
  virus, faces a new threat from Hurricane Isaias.}
\item
  \protect\hyperlink{link-747b61fb}{Contact tracing, a process critical
  for managing the virus, falters from testing shortages and backlogs.}
\item
  \protect\hyperlink{link-19b57b6f}{A large outbreak at a Georgia summer
  camp adds to the evidence that children are susceptible to the virus.}
\end{itemize}

\includegraphics{https://static01.graylady3jvrrxbe.onion/images/2020/07/31/us/31virus-briefing-relief/merlin_175181316_602a1744-c139-44e8-9e19-20e8bd9d39cb-articleLarge.jpg?quality=75\&auto=webp\&disable=upscale}

\subsection{}

Tens of millions of jobless Americans are losing a benefit that helped
keep them afloat.

A \$600 weekly jobless benefit from the federal government that became a
lifeline for tens of millions of unemployed Americans, while also
helping prop up the coronavirus-ravaged economy, expired at midnight as
officials in Washington failed to agree on a new relief bill.

The loss of the aid will leave millions struggling to make ends meet at
a precarious moment when nearly 11 percent of Americans have said that
they live in households where there is not enough to eat, according to a
\href{https://www.census.gov/programs-surveys/household-pulse-survey/data.html?utm_campaign=20200727mspuls1ccdtanl\&utm_medium=email\&utm_source=govdelivery}{recent
Census Bureau survey}, and more than a quarter have missed a rent or
mortgage payment.

And it comes as unemployment remains at record levels. More than 1.4
million Americans filed new for state unemployment benefits last week,
the \href{https://oui.doleta.gov/press/2020/073020.pdf}{Labor Department
said Thursday}. It was the
\href{https://www.nytimes3xbfgragh.onion/2020/07/30/business/economy/q2-gdp-coronavirus-economy.html}{19th
straight week that the tally exceeded one million,}an unheard-of figure
before the pandemic. Some 30 million people are receiving unemployment
benefits.

The benefit's expiration will force Louise Francis, who worked as a
banquet cook at the Sheraton Hotel in New Orleans for nearly two decades
before being furloughed last
spring,\href{https://www.nytimes3xbfgragh.onion/2020/07/30/business/economy/q2-gdp-coronavirus-economy.html}{to
get by on just state unemployment benefits, which for her come to \$247
a week.}

``With the \$600, you could see your way a little bit,'' said Ms.
Francis, 59. ``You could feel a little more comfortable. You could pay
three or four bills and not feel so far behind.''

The aid lapsed as Republicans and Democrats in Washington
\href{https://www.nytimes3xbfgragh.onion/2020/07/28/us/politics/coronavirus-relief-bills-house-senate.html}{remained
far apart on what the next round of virus relief should look like}.

Democrats wanted to extend the \$600 weekly payments through the end of
the year, as part of an expansive \$3 trillion aid package that would
also prop up state and local governments that are weighing layoffs and
service cuts to offset dwindling tax revenues. Republicans, worried that
the \$600 benefit left some people with more money than they earned when
they were working, sought to scale it back to \$200 per week as part of
a \$1 trillion proposal.

White House officials and Democrats blamed each other on Friday for the
benefit's expiration.

At a White House news conference, Mark Meadows, President Trump's chief
of staff, accused Democrats of playing ``politics as usual.'' At the
Capitol, Nancy Pelosi, the House Speaker, declared that administration
officials ``do not understand the gravity of the situation.''

Both said they planned to continue discussions, possibly into the
weekend, to find a compromise. But the talks will come too late to help
laid-off workers set to lose their aid.

As the deadline neared, Republicans proposed continuing the \$600
benefit for one week while talks continue. Democrats rejected the
short-term extension.

``When you have a six-day, one-week extension on a provision, it is
usually --- has always been --- to accommodate a legislative topic if
you're on the verge of having an agreement,'' Ms. Pelosi said. ``Why
don't we just get the job done? Why don't we just get the job done?''

\hypertarget{-1}{%
\subsection{}\label{-1}}

California's summer outbreak makes it the first state with half a
million cases.

Image

A line outside a coronavirus testing site in Los Angeles this
month.Credit...Philip Cheung for The New York Times

California passed a grim milestone on Friday, becoming the first state
to report more than 500,000 cases of the coronavirus, according to a
\href{https://www.nytimes3xbfgragh.onion/interactive/2020/us/coronavirus-us-cases.html\#states}{New
York Times database}.

In per capita terms, both the infections and deaths in California ---
the country's most populous state, with 40 million residents --- remain
lower than in many other states, including Florida, where the
concentration of cases is the worst in the nation. Three more states
have reported more than 400,000 cases --- Texas, Florida and New York
--- and no other had more than 200,000 as of Friday.

And though California has the third-highest number of
coronavirus-related deaths, with slightly over 9,000, its total is
significantly lower than that of New York, which has over 32,000. New
Jersey has the country's second-highest total, with more than 15,000.
With numbers still arriving on Friday, California officials reported 209
new deaths for its single-day record, surpassing the previous high, 192,
recorded on Wednesday.

California locked down its residents relatively early, on March 19,
buying time for hospitals and public health workers to prepare for an
expected onslaught. The state's weekly average number of infections in
late April was less than 20 percent of what it is today.

But while the restrictions led to
\href{https://www.nytimes3xbfgragh.onion/2020/04/14/us/california-coronavirus-shutdown.html}{early
success} in the state, which has the world's fifth-largest economy, they
eventually wore on residents reeling from spikes in unemployment.
Resistance mounted to the restrictions.

After a phased reopening that began in May, which some health officials
warned was premature, the number of infections
\href{https://www.nytimes3xbfgragh.onion/2020/06/29/us/california-coronavirus-reopening.html}{began
to soar}. Gov. Gavin Newsom has since
\href{https://www.nytimes3xbfgragh.onion/interactive/2020/07/17/upshot/coronavirus-face-mask-map.html}{made
face masks mandatory}, closed the state's bars and banned indoor dining,
\href{https://www.nytimes3xbfgragh.onion/2020/07/14/us/california-counties-reopening.html}{rolled
back reopening plans} for most Californians and begun withholding
federal relief funds from cities that refuse to enforce public health
orders.

Municipalities have stepped up enforcement as well. Los Angeles County
this week
\href{https://www.latimes.com/california/story/2020-07-29/county-shuts-three-businesses-for-failing-to-report-coronavirus-outbreaks}{shut
down} three food distribution facilities for failing to report
outbreaks, and Palm Springs ordered a
\href{https://www.palmspringsca.gov/home/showdocument?id=75670}{midnight
curfew}.

Nonetheless, the state reported a record 197 new coronavirus deaths on
Wednesday. The average weekly fatalities have doubled since the
beginning of July. **** The virus also officially spread to the last of
the state's 58 counties, with two cases reported in remote Modoc County,
which is at the Nevada and Oregon borders.

``It's here,'' the county's director of health services
\href{http://modochealthservices.org/corona-virus}{said in a news
release}, ``and we could see the number of cases increase in the next
few weeks.''

\hypertarget{-2}{%
\subsection{}\label{-2}}

Giroir, Trump's testing czar, said most virus test results were coming
back quickly. Public health experts disagree.

\includegraphics{https://static01.graylady3jvrrxbe.onion/images/2020/07/31/business/31virus-briefing-erin/31virus-briefing-erin-videoSixteenByNine3000-v2.jpg}

As schools, universities and businesses struggle to reopen without the
coronavirus testing they need to curb outbreaks, the Trump
administration's testing czar testified to Congress Friday that it was
currently impossible to get all tests back within three days.

The testing czar, Adm. Brett P. Giroir, told lawmakers that getting all
coronavirus tests back between 48 and 72 hours, which many health
officials have said is critical, ``is not a possible benchmark we can
achieve today, given the demand and the supply.''

Admiral Giroir said that it would be ``absolutely'' achievable in the
future, and that half of all test results were being processed within 24
hours. While not all tests can be turned around within three days, he
said, the average wait time for the rest was around that time or less
--- an assessment that is sharply at odds with what patients and health
professionals around the country say they are experiencing.

He told lawmakers that the nation was now averaging about 820,000 tests
each day, and that roughly half were ``done in either point-of-care
technologies with results in 15 minutes or less or at local hospitals
for which the turnaround time is generally within 24 hours.''

And he said that three-quarters of tests from commercial labs were
coming back within five days.

The remainder, he said, are processed by commercial labs like Quest
Diagnostics and LabCorp. Three-quarters of those tests were coming back
within five days, he said.

Admiral Giroir spoke alongside Dr. Anthony S. Fauci, the nation's top
infectious disease expert, and Dr. Robert R. Redfield, the director of
the Centers for Disease Control and Prevention, during a hearing of the
House Select Subcommittee on the Coronavirus Crisis, a special panel
created by Speaker Nancy Pelosi to oversee the Trump administration's
coronavirus response.

His comments on testing turnaround times were met with puzzlement by
public health experts, who say that even if the figures are accurate,
they do not reflect the reality on the ground. Reporting test results
and wait times in aggregate, these experts say, does not indicate things
are getting better. Testing shortages persist. And in some places, tests
cannot be processed at all because of a lack of reagents --- the
chemicals needed to detect whether the virus is present --- or lab
capacity.

``Across the board, the supply chain is still fragile and fragmented,''
said Amanda Harrington, director of the Clinical Microbiology Laboratory
at Loyola University Medical Center in Maywood, Ill. ``We have assays we
don't know if we can run tomorrow.

Dr. Michael T. Osterholm, director of the Center for Infectious Disease
Research and Policy at the University of Minnesota, said the
administration needed a ``national dashboard for testing'' where data is
collected and made publicly available.

Later Friday in an evening briefing in Florida with President Trump,
Gov. Ron DeSantis of Florida noted: ``We're doing so many tests,
sometimes it takes seven to ten days to get the results back, ''He said
that the state was trying to speed tests for symptomatic people, and
that new point-of-care tests from the federal government should help the
state get faster results.

In Alabama, the average wait time for coronavirus test results is
currently seven days --- significantly longer than the two or three-day
turnaround window advised by public health officials for making
quarantine and care decisions.

In \href{https://www.alabamapublichealth.gov/news/2020/07/31e.html}{a
statement} released by the state's department of public health on
Friday, officials asked health care providers to limit testing to ``the
elderly, those in congregate living settings, health care personnel,
those with symptoms consistent with COVID-19 and those with underlying
medical conditions that place them most at risk.''

Democrats on the House panel wasted little time in pointing out that the
caseload is much lower in Europe and Asia than in the United States. Dr.
Fauci said countries in those parts of the world were more aggressive
about shutting down as the pandemic raged.

``When they shut down, they shut down to the tune of about 95 percent,
getting their baseline down to tens or hundreds of cases a day,'' Dr.
Fauci said. By contrast, he said, only about 50 percent of the United
States shut down, and the baseline of daily cases was much higher --- as
many as 20,000 new cases a day --- even at its lowest. More recently,
the United States has recorded as many as 70,000 new cases a day.

Dr. Fauci also cast doubt on a study promoted by Mr. Trump and other
conservatives. Conducted by Henry Ford Hospital in Detroit, it showed an
apparent benefit for hydroxychloroquine, the anti-malaria drug that
President Trump has touted as a Covid-19 treatment. ``That study is a
flawed study,'' Mr. Fauci said.
(\href{https://www.nytimes3xbfgragh.onion/interactive/2020/science/coronavirus-drugs-treatments.html}{Read
more about the most-talked-about treatments for the coronavirus.})

\href{https://www.nytimes3xbfgragh.onion/interactive/2020/us/coronavirus-us-cases.html}{Tracking
the Coronavirus~›}

\href{https://www.nytimes3xbfgragh.onion/interactive/2020/us/coronavirus-us-cases.html}{}

\hypertarget{where-cases-are-rising-fastest}{%
\subsubsection{\texorpdfstring{Where cases are \textbf{rising}
fastest}{Where cases are rising fastest}}\label{where-cases-are-rising-fastest}}

\href{https://www.nytimes3xbfgragh.onion/interactive/2020/us/hawaii-coronavirus-cases.html}{}

Hawaii

\href{https://www.nytimes3xbfgragh.onion/interactive/2020/us/alaska-coronavirus-cases.html}{}

Alaska

\href{https://www.nytimes3xbfgragh.onion/interactive/2020/us/new-jersey-coronavirus-cases.html}{}

N.J.

\href{https://www.nytimes3xbfgragh.onion/interactive/2020/us/missouri-coronavirus-cases.html}{}

Mo.

\href{https://www.nytimes3xbfgragh.onion/interactive/2020/us/rhode-island-coronavirus-cases.html}{}

R.I.

\href{https://www.nytimes3xbfgragh.onion/interactive/2020/us/massachusetts-coronavirus-cases.html}{}

Mass.

\href{https://www.nytimes3xbfgragh.onion/interactive/2020/us/mississippi-coronavirus-cases.html}{}

Miss.

\href{https://www.nytimes3xbfgragh.onion/interactive/2020/us/maryland-coronavirus-cases.html}{}

Md.

\href{https://www.nytimes3xbfgragh.onion/interactive/2020/us/oklahoma-coronavirus-cases.html}{}

Okla.

\href{https://www.nytimes3xbfgragh.onion/interactive/2020/us/south-dakota-coronavirus-cases.html}{}

S.D.

\href{https://www.nytimes3xbfgragh.onion/interactive/2020/us/kentucky-coronavirus-cases.html}{}

Ky.

\href{https://www.nytimes3xbfgragh.onion/interactive/2020/us/nebraska-coronavirus-cases.html}{}

Neb.

\href{https://www.nytimes3xbfgragh.onion/interactive/2020/us/coronavirus-us-cases.html}{}

\hypertarget{us-hot-spots-}{%
\subsubsection{U.S. hot spots~›}\label{us-hot-spots-}}

\includegraphics{https://static01.graylady3jvrrxbe.onion/newsgraphics/2020/03/16/coronavirus-maps/8eab50b66d044aee484bb3f3e9dba618661f2851/images/orphan_usa-threeByTwoSmallAt2X.png}

\href{https://www.nytimes3xbfgragh.onion/interactive/2020/world/coronavirus-maps.html}{}

\hypertarget{worldwide-}{%
\subsubsection{Worldwide~›}\label{worldwide-}}

\includegraphics{https://static01.graylady3jvrrxbe.onion/newsgraphics/2020/03/16/coronavirus-maps/8eab50b66d044aee484bb3f3e9dba618661f2851/images/orphan_world-threeByTwoSmallAt2X.png}

\hypertarget{-3}{%
\subsection{}\label{-3}}

South Korea arrests the leader of a church where the virus spread
rapidly.

Image

Lee Man-hee, founder of the Shincheonji Church of Jesus, during a news
conference in March.~Credit...Yonhap/Reuters

The leader of a secretive religious sect in South Korea was arrested
Saturday on charges of embezzling church money and conspiring to impede
the government's efforts to fight the coronavirus.

\href{https://www.nytimes3xbfgragh.onion/2020/03/02/world/asia/coronavirus-south-korea-shincheonji.html?searchResultPosition=1}{Lee
Man-hee,} the founder of the Shincheonji Church of Jesus, was taken to
jail in Suwon, south of Seoul, early Saturday after a judge issued a
warrant for prosecutors to arrest him.

The rapid spread of the virus this winter among the church's worshipers
in Daegu, a city in the southeast, briefly made South Korea home to the
world's largest coronavirus outbreak outside China. As of Friday, more
than 36 percent of the country's 14,300 coronavirus patients were
members of Shincheonji or their contacts, according to government data.

Prosecutors say Mr. Lee and other church officials obstructed the
government's efforts to fight the epidemic by not fully disclosing the
number of worshipers and their gathering places. Seven church officials
were indicted last month on the same charge.

Mr. Lee, 88, has also been accused of embezzling 5.6 billion won, or
\$4.7 million, from church funds to build a luxurious ``peace palace''
north of Seoul. The Shincheonji church has broadly denied all the
charges against him, and he could face years in prison if convicted.

Intense criticism from the South Korean public
\href{https://www.nytimes3xbfgragh.onion/2020/03/02/world/asia/coronavirus-south-korea-shincheonji.html}{forced
Mr. Lee to apologize} in March.

In a statement on Saturday, the church said that Mr. Lee never intended
to hamper the government's efforts to control the epidemic, and that he
had only expressed concern over what he felt were excessive demands for
personal data on church worshipers.

``He has emphasized the importance of disease control and urged the
church members to cooperate with the authorities,'' the church said.
``We will do our best to let the truth be known through trial.''

But parents who accused the church of luring and brainwashing their
children with its unorthodox teachings welcomed his arrest on Saturday,
calling Mr. Lee a ``religious con artist.''

\hypertarget{-4}{%
\subsection{}\label{-4}}

Florida, already reeling from the virus, faces a new threat from
Hurricane Isaias.

Image

In preparation on Friday for the storm, people filled sand bags for
distribution to the residents of Palmetto Bay, near
Miami.~Credit...Chandan Khanna/Agence France-Presse --- Getty Images

Florida's Atlantic coast braced for the arrival of Hurricane Isaias this
weekend after the storm raked the Bahamas, parts of Puerto Rico and the
Dominican Republic on Friday.

Preparations for the storm were complicated by the state's battle with
the coronavirus, which could make evacuating homes and entering
community shelters especially risky. Friday was the third consecutive
day that Florida set its record for the most deaths reported in a single
day, according to a New York Times database.

Gov. Ron DeSantis said at a news conference on Friday that the division
of emergency management had been working at
\href{https://www.floridadisaster.org/sert/eoc-activation-levels/}{its
most active level} since March, ``allowing them to actively plan for
hurricane season even while responding to the Covid-19 pandemic.''

Early on in the pandemic, the governor said, the division created a
reserve of protective equipment for hurricane season, including 20
million masks, 22 million gloves and 1.6 million face shields.

\hypertarget{-5}{%
\subsection{}\label{-5}}

Contact tracing, a process critical for managing the virus, falters from
testing shortages and backlogs.

Image

El, who worked as a contact tracer in New York, said, ``I have never had
a more dysfunctional workplace.''Credit...Hiroko Masuike/The New York
Times

Considered a cornerstone of the public health arsenal to suppress the
virus,
\href{https://www.nytimes3xbfgragh.onion/2020/07/31/health/covid-contact-tracing-tests.html}{contact
tracing has largely failed in the United States}, as the virus's
pervasiveness and major lags in testing have rendered the system almost
pointless.

The
\href{https://www.cdc.gov/coronavirus/2019-ncov/php/contact-tracing/contact-tracing-plan/contact-tracing.html}{goal}of
contact tracing is to reach people who have spent more than 15 minutes
within six feet of an infected person and ask them to voluntarily
quarantine at home for two weeks, even
\href{https://www.cdc.gov/coronavirus/2019-ncov/symptoms-testing/testing.html}{if
they test negative}, monitoring themselves for symptoms during that
time. On Friday, Dr. Fauci said that if someone gets tested, ``they
should assume that it might be positive and should essentially isolate
themselves before they go back and get the result of the test.''

In some of the hardest-hit regions, contact-tracing efforts seem futile,
as many people have refused to participate or cannot even be located,
further hampering health care workers.

In Arizona's most populated region, for example, the virus is so
\href{https://www.azfamily.com/news/continuing_coverage/coronavirus_coverage/contact-tracing-important-but-less-useful-with-spiking-cases-maricopa-county-says/article_57d55328-bb4b-11ea-8718-8b1cf4ab4137.html}{ubiquitous}
that contact tracers have been unable to reach a fraction of those
infected. In Austin, Texas, the story is much the same. Cities in
Florida, which has been seeing an average of more than 10,000 new cases
a day in the past week, have
largely\href{https://www.nbcmiami.com/news/local/miami-beach-mayor-urges-desantis-to-address-failures-of-floridas-contact-tracing-program/2268324/}{given
up on contact tracing}. Things are equally dismal in California. And in
\href{https://www.nytimes3xbfgragh.onion/2020/07/29/nyregion/new-york-contact-tracing.html}{New
York City's tracing program}, workers have complained of crippling
communication and training problems.

From the very beginning, states and cities have struggled to detect the
prevalence of the virus because of spotty and sometimes rationed
diagnostic testing and long delays in getting results. For the tests
currently available and in high demand, there is not a consensus on
\href{https://www.nytimes3xbfgragh.onion/2020/07/31/health/coronavirus-test-ethics.html}{who
should get them.} Some experts say everyone should get tested, even
those without symptoms. Others say the tests should be reserved for the
people who have symptoms or are more vulnerable to infection.

There is broad consensus, however, that more tests are needed.

On Friday, the
\href{https://www.nih.gov/news-events/news-releases/nih-delivering-new-covid-19-testing-technologies-meet-us-demand}{National
Institutes of Health announced} that seven companies have received
\$248.7 million to ramp up test production and deliver millions more
weekly tests as early as September.

The tests, which include three simple ``point of care'' tests
that\href{https://www.nytimes3xbfgragh.onion/2020/07/06/health/fast-coronavirus-tests.html}{don't
need to be shipped to laboratories}, were selected as promising
candidates to address the serious shortages that have plagued testing
efforts since March.

\hypertarget{-6}{%
\subsection{}\label{-6}}

A large outbreak at a Georgia summer camp adds to the evidence that
children are susceptible to the virus.

As schools and universities plan for the new academic year, and
administrators grapple with complex questions about how to keep young
people safe, a
new\href{https://www.cdc.gov/mmwr/volumes/69/wr/mm6931e1.htm?s_cid=mm6931e1_w}{report
about a coronavirus outbreak at a sleepaway camp in Georgia} provides
fresh reasons for concern.

The camp implemented several precautionary measures against the virus,
but stopped short of requiring campers to wear masks. The virus blazed
through the community of about 600 campers and counselors, the Centers
for Disease Control and Prevention reported on Friday.

The study is notable because few outbreaks in schools or child care
settings have been described to date, said Caitlin Rivers, an
epidemiologist at the Johns Hopkins Bloomberg School of Public Health.

``The study affirms that group settings can lead to large outbreaks,
even when they are primarily attended by children,'' she said.

``The fact that so many children at this camp were infected after just a
few days together underscores the importance of mitigation measures in
schools that do reopen for in person learning,'' Dr. Rivers added.

While the role children play in the spread of the virus has been
questioned, the authors of the report said the research adds to evidence
that children of all ages are not only susceptible to infection, but may
play an important role in transmission.

Of the 344 campers and staff for whom test results were available, 260
tested positive, meaning at least 43 percent were infected, though the
figure may well be higher, the C.D.C. said.

Of children ages 6 to 10, over half were infected; 44 percent of those
ages 11 to 17 were infected, as were one-third of those ages 18 to 21.
Only seven staffers were older than 22, and two of them tested positive.

Those who had been at the camp longest had the highest rate of
infection; overall, more than half of the staff, who had arrived before
the campers, were infected.

\hypertarget{-7}{%
\subsection{}\label{-7}}

Fitch Ratings downgrades its outlook on U.S. debt as the deficit soars.

The credit rating firm Fitch left the United States' AAA rating
untouched, but downgraded its outlook on what's effectively the national
credit score, suggesting the country's status as one of the world's most
trustworthy borrowers could be put at risk by the enormous deficits the
federal government is running to combat fallout from the pandemic.

``The outlook has been revised to negative to reflect the ongoing
deterioration in the U.S. public finances and the absence of a credible
fiscal consolidation plan,''
\href{https://www.fitchratings.com/research/sovereigns/fitch-revises-united-states-outlook-to-negative-affirms-at-aaa-31-07-2020}{Fitch
analysts wrote} on Friday in a report announcing the decision.

Cratering tax revenues and surging expenditures have driven record
levels of red ink for the federal government in recent months. The
\href{https://www.nytimes3xbfgragh.onion/live/2020/07/13/business/stock-market-today-coronavirus\#the-us-budget-deficit-hits-another-monthly-record}{United
States budget deficit hit a record} \$864 billion in June as the
government continued pumping money into the economy to support workers
and businesses slammed by the pandemic. Some analysts expect monthly
deficits to soon top \$1 trillion.

Ballooning deficits have led to an explosion of new borrowing. Fitch
noted that the Treasury Department borrowed just under \$3 trillion
dollars from the end of February to the end of June.

Much of the supply of new government bonds was,
\href{https://www.nytimes3xbfgragh.onion/2020/04/15/business/coronavirus-stimulus-money.html}{essentially,
purchased by the Federal Reserve}, which has bought \$2.6 trillion in
financial assets since the middle of March, Fitch noted.

The presence of the Federal Reserve, which can essentially create
whatever money it wants and use it to buy assets, such as U.S.
government debt, has depressed yields on government bonds even as its
debts and deficits rise sharply.

On Friday, the yield on the 10-year note fell to 0.53 percent, one of
\href{https://www.marketwatch.com/story/10-year-treasury-yield-plunged-to-its-lowest-in-234-years-says-deutsche-bank-11596214464\#:~:text=The\%2010\%2Dyear\%20Treasury\%20note,scurrying\%20into\%20safe\%20haven\%20assets.}{the
lowest levels in recorded history}, suggesting there is virtually no
concern among investors about the country's ability to service its
growing debts.

Global roundup

\hypertarget{-8}{%
\subsection{}\label{-8}}

Once an out-of-control center, Italy now offers lessons for keeping the
virus in check.

Image

Italy has consolidated, or at least maintained, the rewards of a tough
nationwide lockdown through a mix of vigilance and painfully gained
medical expertise.Credit...Gianni Cipriano for The New York Times

When the virus erupted in the West, Italy was
\href{https://www.nytimes3xbfgragh.onion/interactive/2020/03/27/world/europe/coronavirus-italy-bergamo.html}{the
nightmarish epicenter}, a place to avoid at all costs and a shorthand in
the United States and much of Europe for uncontrolled contagion.

Fast forward a few months, and the United States has had tens of
thousands more deaths than any other country in the world. European
states that once looked smugly at Italy are facing new flare-ups.

And Italy? Its hospitals are basically empty of Covid-19 patients. Daily
deaths attributed to the virus in Lombardy, the region that bore the
brunt of the pandemic, hover around zero. The number of new daily cases
has plummeted to ``one of the lowest in Europe and the world,'' said
Giovanni Rezza, director of the infective illness department at the
National Institute of Health.

How Italy has gone from being a global pariah to a model --- however
imperfect --- of viral containment holds fresh lessons for the rest of
the world, including the United States.

Italy has consolidated, or at least maintained, the rewards of a tough
nationwide lockdown through a mix of vigilance and painfully gained
medical expertise.

\begin{itemize}
\item
  Its government has been guided by scientific and technical committees.
\item
  The country set aside economic pressures and only began easing its
  exceptionally tight lockdown based on case counts.
\item
  Italy continues to limit travel from elsewhere.
\item
  Local doctors, hospitals and health officials collect more than 20
  indicators on the virus daily and send them to regional authorities,
  who then forward them to the National Institute of Health.
\end{itemize}

The result is a weekly X-ray of the country's health upon which policy
decisions are based.

Here are other developments from around the globe:

\begin{itemize}
\item
  Across \textbf{Europe}, the economy tumbled into its worst recession
  on record in the second quarter. From April to June, gross domestic
  product fell by 11.9 percent from the first quarter in the European
  Union, and by 12.1 percent in the core group of countries that use the
  euro currency. On an annualized basis, European Union economies shrank
  by 14.4 percent, and eurozone economies by 15 percent, the sharpest
  contractions since statistics started being kept in 1995.
\item
  Britain has barred millions of people in northern \textbf{England}
  from meeting other members of other households at their homes, paused
  reopenings set for Aug. 1 and moved to make face masks mandatory in
  more places, after a day on which it reported 38
  \href{https://www.nytimes3xbfgragh.onion/interactive/2020/world/europe/united-kingdom-coronavirus-cases.html}{new
  coronavirus deaths} and nearly 900 known new infections, its highest
  case numbers in a month.
\end{itemize}

\includegraphics{https://static01.graylady3jvrrxbe.onion/images/2020/07/31/business/31virus-video-boris/31virus-video-boris-videoSixteenByNine3000.jpg}

\begin{itemize}
\item
  \href{https://www.santepubliquefrance.fr/maladies-et-traumatismes/maladies-et-infections-respiratoires/infection-a-coronavirus/documents/bulletin-national/covid-19-point-epidemiologique-du-30-juillet-2020}{French
  health authorities} on Friday reported a 54 increase in new cases over
  the past week and a rise in hospitalizations. Community transmission
  is accelerating most among young adults aged 20 to 30, according to
  Santé Publique \textbf{France}. The authority called for increased
  vigilance in preventive measures and cited a decline in social
  distancing and avoidance of hand-shaking and hugs, while reporting an
  increase in public mask-wearing.
\item
  A stark lack of testing in many \textbf{African} countries has kept
  officials from being able to track the pandemic, prompting fears that
  a recent surge in cases across the continent may be just the ``tip of
  the iceberg,'' according to the International Rescue Committee. The
  organization said many African nations needed international support to
  increase their testing capacity or the continent could face ``an
  undetected and uncontrolled spread.''
\item
  \textbf{Vietnam}, which has been
  \href{https://www.nytimes3xbfgragh.onion/2020/07/29/world/asia/coronavirus-vietnam.html}{fighting
  a fresh virus outbreak} after more than three months without reporting
  a locally transmitted case, has announced its first death from the
  coronavirus. The victim was a 70-year-old resident of the city of Hoi
  An who had been living with kidney disease for more than a decade. The
  man was admitted to a hospital on July 9 with chest tightness and
  fatigue, and tested positive for the virus on Sunday. He died Friday
  morning.
\item
  The \textbf{Hong Kong} government said on Friday that it would
  \href{https://www.nytimes3xbfgragh.onion/2020/07/31/world/asia/hong-kong-election-delayed.html}{postpone
  the city's September legislative election}by a year because of the
  coronavirus pandemic, a decision seen by the pro-democracy opposition
  as a brazen attempt to thwart its electoral momentum and avoid the
  defeat of pro-Beijing candidates.

  ``It is a really tough decision to delay but we want to ensure
  fairness, public safety and public health,'' said Carrie Lam, Hong
  Kong's chief executive.
\item
  On Saturday, \textbf{Japan} announced 1,579 new cases, breaking a
  record set the day before. The country now has more than 1,000 deaths
  related to the coronavirus, reporting 1,011 on Saturday.
\end{itemize}

\hypertarget{-9}{%
\subsection{}\label{-9}}

Health care workers who described their P.P.E. as inadequate had a
higher infection rate, a study says.

Image

Lab jackets and personal protective equipment hanging from a fence at a
coronavirus testing site in Milwaukee this month.Credit...Joshua Lott
for The New York Times

As the virus surged this spring, health care workers in the United
States and the United Kingdom scrambled to make do with scarce personal
protective equipment. The consequences to their own health were stark.

According to
a\href{https://www.thelancet.com/journals/lanpub/article/PIIS2468-2667(20)30164-X/fulltext}{new
study,} these workers were 3.4 times more likely to report a positive
coronavirus test than the general population. Workers who described
their equipment --- including masks, gloves and gowns --- as
insufficient were 1.3 times more likely to report positive tests than
their colleagues who deemed their equipment appropriate.

Using self-reported data collected
through\href{https://covid.joinzoe.com/us}{a Covid-symptom monitoring
app}, researchers at Harvard Medical School and Massachusetts General
Hospital surveyed 99,795 front-line health care workers and two million
other people from March 24 through April 23. Health care workers who
were Black, Asian or other races were 1.81 times more likely to report a
positive test result than non-Hispanic white health care workers, the
study found.

``Minority front-line health care workers tend to be in higher-risk
settings and have less access to protective equipment,'' said
\href{https://cgvh.harvard.edu/people/erica-warner}{Dr. Erica T.
Warner,} a co-author of the study and an assistant professor of medicine
at Harvard Medical School. ``This is a microcosm of the larger disparity
we see in health care in general.''

Even though protective gear is now more readily available, shortages are
\href{https://www.nytimes3xbfgragh.onion/2020/07/08/health/coronavirus-masks-ppe-doc.html}{still
common}.

U.S. ROUNDUP

\hypertarget{-10}{%
\subsection{}\label{-10}}

N.Y.C. sets a positivity rate threshold for reopening schools and a
strategy if someone tests positive at school.

Image

P.S. 241 in the Crown Heights neighborhood of Brooklyn closed in April
because of the pandemic.Credit...Kirsten Luce for The New York Times

New York City public schools, the nation's largest school system, will
be able to reopen its school buildings in September only if the city
maintains a test positivity rate below 3 percent, Mayor Bill de Blasio
announced on Friday. That conservative threshold is even lower than the
5 percent test positivity rate which has been set by Gov. Andrew M.
Cuomo as
\href{https://www.nytimes3xbfgragh.onion/2020/07/14/us/coronavirus-schools-fall.html}{a
cut-off for school reopening and recommended by public health experts}.

The average positivity rate for New York City has generally remained
lower even than the new city threshold, according to city and state
figures. But even a modest uptick in cases over the next few weeks could
nudge that rate higher, which raises fresh questions about
\href{https://www.nytimes3xbfgragh.onion/2020/07/08/nyregion/nyc-schools-reopening-plan.html}{whether
city schools will open part-time on Sept. 10 as planned}. On Friday, the
school system
\href{https://infohub.nyced.org/docs/default-source/default-document-library/nyc-doe---state-doh-reopening-plan-7-31.pdf}{submitted
its reopening plan to the state}.

New York is one of the only large districts in the country that is
planning to reopen its buildings at all: Children will report to school
one to three days a week to allow for social distancing. All staff
members will be asked to take tests before the start of school, with
expedited results. Education officials in the city laid out a plan on
Thursday for what would happen in the seemingly inevitable event that
cases are confirmed in a classroom.

The protocol means it is likely that at many of the city's 1,800
schools, individual classrooms or even entire buildings will be closed
at points during the school year.

Officials said confirmed infections among students, teachers and staff
members would be treated the same. One or two cases in a single
classroom would require those classes to close for 14 days; all students
and staff members in that classroom would be ordered to self-quarantine,
and students would learn remotely. The rest of the school would continue
to operate.

But if two or more people in different classrooms in the same school
tested positive, the entire building would close while city disease
detectives were brought in to investigate the cases, which could take
several days. Depending on the results of the investigation, the
building could reopen, but the classrooms with positive cases would
remain closed for 14 days.

Elsewhere in the U.S.:

\begin{itemize}
\item
  \textbf{Cases in New Jersey}, which just a week ago had plunged to
  their lowest levels since the pandemic began,
  \href{https://www.nytimes3xbfgragh.onion/2020/07/30/nyregion/coronavirus-cases-nj.html}{are
  rising again}, fueled in part by outbreaks among young adults along
  the Jersey Shore. As of Thursday, the state had recorded an average of
  434 cases per day over the last week, an increase of 35 percent from
  the average two weeks earlier, according to a Times database. On
  Friday, there were 699 new cases, the governor said.
\item
  Airbnb will start cracking down on house parties in \textbf{New
  Jersey} after state health officials warned that the parties were
  leading to Covid-19 clusters. The vacation rental company said it
  would remove 35 listings from the site, according to
  \href{https://apnews.com/c64053bb7f7b60001a314526da06732e}{The
  Associated Press}. It took police nearly five hours to break up a
  gathering of more than 700 people at an Airbnb rental property in
  Jackson, N.J., last weekend.
\item
  The U.S. Food and Drug Administration authorized the first two tests
  capable of estimating the quantity of coronavirus antibodies in a
  patient's blood. Both tests were developed by Siemens, according to
  \href{https://www.fda.gov/news-events/press-announcements/coronavirus-covid-19-update-fda-authorizes-first-tests-estimate-patients-antibodies-past-sars-cov-2}{a
  statement} released by the F.D.A. on Friday. The agency cautioned
  against interpreting the results from the tests, or any serology test,
  as a sign of immunity to the virus.
\item
  The French drug maker Sanofi said on Friday that it had secured an
  agreement of up to \$2.1 billion to supply the \textbf{U.S. federal
  governmen}t with 100 million doses of its experimental coronavirus
  vaccine, the largest such deal announced to date. The arrangement with
  Sanofi and its partner, the British pharmaceutical company
  GlaxoSmithKline, brings the Trump administration's investment in
  coronavirus vaccine projects
  \href{https://medicalcountermeasures.gov/app/barda/coronavirus/COVID19.aspx?filter=vaccine}{to
  more than \$8 billion}. This effort, known as Operation Warp Speed, is
  placing bets on multiple vaccines and is paying companies to
  manufacture millions of doses before clinical trials have been
  completed.
\item
  \textbf{The Trump administration} wasted around \$500 million by
  overpaying for ventilators through negotiations that were ``inept,'' a
  panel of the House Oversight and Reform Committee said in a report
  released Friday. It faulted Peter Navarro, Mr. Trump's top trade
  adviser, and Jared Kushner, his son-in-law and senior adviser, for
  negotiating a deal in which the panel said they paid almost five times
  the price per device than under a previous contract with the same
  vendor.
\item
  \textbf{Florida} broke a record --- the most deaths the state reported
  in a single day --- for the fourth day in a row: On Friday, the state
  announced 257 additional fatalities. \textbf{Mississippi} reported its
  most deaths in a single day from the virus, 52, as did Idaho, with 13.
  \textbf{North Dakota} reported a new single-day case record, 164.
\item
  Even with significant gaps in the available data, there are strong
  indications that \textbf{Native American people}
  \href{https://www.nytimes3xbfgragh.onion/2020/07/30/us/native-americans-coronavirus-data.html}{have
  been disproportionately affected by the virus}. The rate of known
  cases in the eight counties with the largest populations of Native
  Americans is nearly double the national average, a Times analysis has
  found.
\item
  \textbf{Greenwich, Conn.}, one of the wealthiest suburbs in the
  country, is experiencing what health officials have called a ``mini
  surge'' of infections, an outbreak that has cascaded through the
  community and underscored how social gatherings among young people are
  posing fresh challenges to containing the virus. More than 20 people
  between the ages 16 and 21 have tested positive for the virus, with
  more cases expected as testing continues, according to Greenwich
  health officials.
\item
  \textbf{\href{https://www.nytimes3xbfgragh.onion/2020/07/30/us/politics/juvenile-detainees-coronavirus.html}{Black
  youth}}\href{https://www.nytimes3xbfgragh.onion/2020/07/30/us/politics/juvenile-detainees-coronavirus.html}{detained
  in juvenile justice facilities} have been released at a far slower
  rate than their white peers in response to the coronavirus, according
  to a new report that also found that the gap in release rates between
  the two groups had nearly doubled over the course of the pandemic. The
  \href{https://www.aecf.org/blog/youth-detention-admissions-remain-low-but-releases-stall-despite-covid-19/}{report},
  released this month by the Annie E. Casey Foundation, illustrates one
  more disparity the coronavirus has exacerbated for Black children, who
  are disproportionately funneled into the juvenile justice system.
\end{itemize}

Reporting was contributed by Liz Alderman, Ian Austen, Luke Broadwater,
Julia Calderone, Emily Cochrane, Kate Conger, Michael Cooper, Michael
Crowley, Johnny Diaz, Robert Gebeloff, Erica L. Green, Jan Hoffman,
Rebecca Halleck, Jan Hoffman, Shawn Hubler, Michael Levenson, Giulia
McDonnell Nieto del Rio, Eshe Nelson, Richard A. Oppel Jr., Richard C.
Paddock, Elian Peltier, Matt Phillips, Austin Ramzy, Motoko Rich, Amanda
Rosa, Eliza Shapiro, Megan Specia, Sheryl Gay Stolberg, Eileen Sullivan,
Katie Thomas, Tracey Tulley, Hisako Ueno, Neil Vigdor, Katherine J. Wu
****** and Mihir Zaveri.

Advertisement

\protect\hyperlink{after-bottom}{Continue reading the main story}

\hypertarget{site-index}{%
\subsection{Site Index}\label{site-index}}

\hypertarget{site-information-navigation}{%
\subsection{Site Information
Navigation}\label{site-information-navigation}}

\begin{itemize}
\tightlist
\item
  \href{https://help.nytimes3xbfgragh.onion/hc/en-us/articles/115014792127-Copyright-notice}{©~2020~The
  New York Times Company}
\end{itemize}

\begin{itemize}
\tightlist
\item
  \href{https://www.nytco.com/}{NYTCo}
\item
  \href{https://help.nytimes3xbfgragh.onion/hc/en-us/articles/115015385887-Contact-Us}{Contact
  Us}
\item
  \href{https://www.nytco.com/careers/}{Work with us}
\item
  \href{https://nytmediakit.com/}{Advertise}
\item
  \href{http://www.tbrandstudio.com/}{T Brand Studio}
\item
  \href{https://www.nytimes3xbfgragh.onion/privacy/cookie-policy\#how-do-i-manage-trackers}{Your
  Ad Choices}
\item
  \href{https://www.nytimes3xbfgragh.onion/privacy}{Privacy}
\item
  \href{https://help.nytimes3xbfgragh.onion/hc/en-us/articles/115014893428-Terms-of-service}{Terms
  of Service}
\item
  \href{https://help.nytimes3xbfgragh.onion/hc/en-us/articles/115014893968-Terms-of-sale}{Terms
  of Sale}
\item
  \href{https://spiderbites.nytimes3xbfgragh.onion}{Site Map}
\item
  \href{https://help.nytimes3xbfgragh.onion/hc/en-us}{Help}
\item
  \href{https://www.nytimes3xbfgragh.onion/subscription?campaignId=37WXW}{Subscriptions}
\end{itemize}
