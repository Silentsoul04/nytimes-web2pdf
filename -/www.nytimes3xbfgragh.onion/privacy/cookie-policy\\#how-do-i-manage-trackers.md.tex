The Times and Your Data

\hypertarget{cookie-policy}{%
\section{Cookie Policy}\label{cookie-policy}}

The Times and Your Data

\hypertarget{the-trust-of-our-readers-is-essential}{%
\section{The trust of our readers is
essential.}\label{the-trust-of-our-readers-is-essential}}

\hypertarget{the-new-york-times-company-cookie-policy}{%
\section{The New York Times Company Cookie
Policy}\label{the-new-york-times-company-cookie-policy}}

Last revised on June 30, 2020

This Cookie Policy explains how we use cookies and similar tracking
methods (such as pixels, SDKs, JavaScript, device identifiers, etc.,
which we refer to as ``trackers'') when you visit our site. ``Our site''
means our websites, emails, apps and other services that link to this
Cookie Policy. For a complete list of our cookie policies covering all
Times properties,
\href{https://www.nytimes3xbfgragh.onion/subscription/dg-cookie-policy/cookie-policy.html}{visit
this page}.

This policy explains what various tracking methods are, and why they are
used. It also explains your right to control their use.

We may change this Cookie Policy at any time. Check the ``last revised''
date at the top of this page to see when this Cookie Policy was last
revised. Any change becomes effective when we post the revised Cookie
Policy on or through our site.

If you have any questions, please contact us by email at
\href{mailto:privacy@NYTimes.com}{\nolinkurl{privacy@NYTimes.com}}. You
can also write to us: The New York Times Company, 620 Eighth Avenue, New
York, N.Y. 10018, attn.: Privacy Counsel.

\begin{enumerate}
\def\labelenumi{\arabic{enumi}.}
\item
  1

  \href{what-is-a-tracker}{What Is a Tracker?}
\item
  2

  \href{what-trackers-do-we-use}{What Trackers Do We Use?}
\item
  3

  \href{how-do-i-manage-trackers}{How Do I Manage Trackers?}
\end{enumerate}

1.

What Is a Tracker?

We use a variety of tracker methods; six of the main types are explained
below.

\textbf{A) Cookies}

A cookie is a small string of text that a website (or online service)
stores on a user's browser. It saves data on your browser about your
visit to our site or other sites. It often includes a unique identifier
(e.g., cookie \#123).

``First-party cookies'' are cookies set by us (or on our behalf) on our
site. ``Third-party cookies'' are cookies set by other companies whose
functionality is embedded into our site (e.g., google.com).

``Session cookies'' are temporary cookies stored on your device while
you visit our site. They expire when you close your browser.
``Persistent cookies'' are stored on your browser for a period of time
after you leave our site. Persistent cookies expire on a set expiration
date, or when they are deleted manually.

You can choose whether to accept cookies by editing your browser
settings. However, if cookies are refused, your experience on our site
may be worse, and some features may not work as intended.

\textbf{B) Pixels}

Pixels (also known as ``web beacons,'' ``GIFs'' or ``bugs'') are
one-pixel transparent images located on web pages or messages. They
track whether you have opened these web pages or messages. Upon firing,
a pixel logs a visit to the current page or message and may read or set
cookies.

Pixels often rely on cookies to work, so turning off cookies can impair
them. But even if you turn off cookies, pixels can still detect a web
page visit.

\textbf{C) Javascript}

JavaScript is a programming language. It can be used to write trackers
that, when embedded into a page, allow us to measure how you interact
with our site and other sites.

\textbf{D) Software Development Kits (or SDKs)}

SDKs are pieces of code provided by our digital vendors (e.g.,
third-party advertising companies, ad networks and analytics providers)
in our mobile apps to collect and analyze certain device and user data.

\textbf{E) Device Identifiers}

Device identifiers are user-resettable identifiers comprised of numbers
and letters. They are unique to a specific device. They are stored
directly on the device. These include Apple's ID For Advertisers (IDFA)
and Google's Android Advertising ID (AAID). They are used to recognize
you and/or your devices(s) on, off and across different apps and devices
for marketing and advertising purposes.

\textbf{F) ID Synching}

In order to decide what type of ad might interest you, our digital and
marketing vendors sometimes link data --- inferred from your browsing of
other sites or collected from other sources --- using a method knowns as
``ID synching'' or ``cookie synching.'' To do this, they match the
tracker ID they have assigned to you with one or more tracker IDs that
are held in another company's database and that are likely also
associated with you. Any of the linked trackers may have certain
interests and other demographic information attributed to it. That
information is then used to determine which ad to show you.

Back to top

2.

What Trackers Do We Use?

Below is a list of the types of trackers that appear on our site.

\textbf{Essential Trackers}

Essential trackers are required for our site to operate. They allow you
to navigate our site and use its services and features (e.g., cookies
that help you stay logged in). Without essential trackers, our site will
not run smoothly; in fact, our site (or certain services or features)
might not even be available to you simply because of technical
limitations.

\textbf{Preference Trackers}

Preference trackers allow us to store information about your choices,
settings and preferences. They also help us recognize you when you
return to our site, remember your language settings (among others) and
customize our site accordingly. They are not essential to the
functioning of our site.

\textbf{Analytics Trackers}

Analytics trackers collect or use information about your site use, which
helps us improve our site. Among the uses of analytics trackers are to
show us which pages are most frequently visited, help us record
difficulties you have with our site, track subscription purchases and
behaviors leading to subscription purchases, and measure how well ads
perform.

These trackers add up our readers' visits to show us larger patterns in
our audience. We look at these larger patterns to analyze site traffic.

\textbf{Marketing Trackers}

These trackers help us determine which ads to show you for Times
properties --- both on our site and on other sites. To do this, these
trackers use information about your behavior on various sites to target
our ads.

These trackers allow us to limit the number of times you see our ad
across your devices. They help us personalize the ads we show you. They
also enable us to measure the effectiveness of our marketing campaigns
(e.g., measure if you subscribe after seeing our ads).

\textbf{Advertising Trackers}

Advertising trackers help us determine which ads from third parties are
selected for you. Some of these trackers collect or use information
about your behavior on various sites to aid this targeting. These
trackers sometimes limit the number of times you see an ad, make an ad
more relevant to you or measure the effectiveness of an ad campaign.

In the European Economic Area (E.E.A.), advertising is not personalized
or targeted by third parties through personal data given to them.
Instead, the ads you see are either not personalized, or personalized
using only information that we have about you and that is not shared
with third parties.

We work with advertisers, ad agencies and other vendors to serve these
ads. The ads served can include additional trackers.

Back to top

3.

How Do I Manage Trackers?

When you first come to our site, you may receive a notification that
trackers are present. By clicking or tapping ``accept,'' you agree to
the use of these trackers as described here.

You can manage your tracker settings by opting out of specific (or all)
trackers.

In addition to the options above, you can refuse or accept trackers from
our site (or any other site) in your browser's settings. If you refuse
trackers, you might not be able to sign in or use other
tracker-dependent features of our site.

Most browsers automatically accept cookies, but this is typically
something you can adjust. Information for each browser can be found in
the links below:

\begin{itemize}
\item
  \href{https://support.apple.com/guide/safari/manage-cookies-and-website-data-sfri11471/mac}{Safari
  on desktop} and \href{https://support.apple.com/en-us/HT201265}{Safari
  Mobile (iPhone and iPads)}: Note that, by default, Safari is
  engineered to protect you from being tracked from site to site unless
  you disable Intelligent Tracking Prevention (ITP).
\item
  \href{https://support.mozilla.org/en-US/kb/clear-cookies-and-site-data-firefox?redirectlocale=en-US\&redirectslug=delete-cookies-remove-info-websites-stored}{Firefox}:
  By default, Firefox protects you from cross-site tracking so long as
  you have not disabled Enhanced Tracking Protection (ETP). There is
  therefore less need to manage cookies to protect your privacy.
\item
  \href{https://support.google.com/chrome/answer/95647?hl=en}{Chrome}
\item
  \href{https://support.microsoft.com/en-us/help/4468242/microsoft-edge-browsing-data-and-privacy-microsoft-privacy}{Microsoft
  Edge}: Enabling tracking prevention with Edge will protect you from
  being tracked between sites, such that there will be less of a need to
  manage your cookies in order to protect your privacy.
\item
  \href{https://aboutdevice.com/clear-cookies-history-cache-on-samsung-internet-browser-android/}{Samsung
  Internet Browser}
\item
  Brave: Brave has several mechanisms to keep you from being tracked
  online, but you can
  \href{https://support.brave.com/hc/en-us/articles/360017989132-How-do-I-change-my-Privacy-Settings-}{change
  your privacy settings} if you wish to have greater control over its
  decisions.
\end{itemize}

For more information about other browsers, please refer to this
\href{https://www.allaboutcookies.org/manage-cookies/}{``All About
Cookies'' guide}.

To opt out of Google Analytics data collection, follow
\href{https://tools.google.com/dlpage/gaoptout}{these Google
instructions}.

To reset your device identifier, follow
\href{https://support.google.com/googleplay/android-developer/answer/6048248?hl=en}{Google
instructions} and \href{https://support.apple.com/en-us/HT205223}{Apple
instructions}.

The third-party advertisers, ad agencies and other vendors with which we
work may be members of the Network Advertising Initiative, the Digital
Advertising Alliance Self-Regulatory Program for Online Behavioural
Advertising and/or the European Digital Advertising Alliance. To opt out
of interest-based advertising from the participating companies, please
visit \href{http://optout.aboutads.info/?c=2\&lang=EN}{AboutAds.info} or
\href{http://www.youronlinechoices.eu/}{the European Digital Advertising
Alliance} for laptops and
\href{https://www.networkadvertising.org/mobile-choice/}{NAI Mobile
Choices} or \href{https://youradchoices.com/appchoices}{AppChoices} for
mobile devices. Note that opting out through these channels does not
mean you will no longer see ads. You will still receive other types of
ads from these companies, and any type of ad from nonparticipating
companies. The sites you visit may still collect your information for
other purposes.

Back to top

Something went wrong. Please try again later.

©2020 The New York Times Company

\href{/privacy}{Privacy F.A.Q.}\href{/privacy/privacy-policy}{Privacy
Policy}\href{/privacy/cookie-policy}{Cookie
Policy}\href{/privacy/california-notice}{California
Notice}\href{https://help.nytimes3xbfgragh.onion/hc/en-us/articles/115014893428-Terms-of-service}{Terms
of Service}

The Times and your Data

\hypertarget{main-menu}{%
\subsection{Main Menu}\label{main-menu}}

\href{/privacy}{Privacy F.A.Q.}\href{/privacy/privacy-policy}{Privacy
Policy}\href{/privacy/cookie-policy}{Cookie Policy}
