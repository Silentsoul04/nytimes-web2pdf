Sections

SEARCH

\protect\hyperlink{site-content}{Skip to
content}\protect\hyperlink{site-index}{Skip to site index}

\href{https://www.nytimes3xbfgragh.onion/section/opinion/sunday}{Sunday
Review}

\href{https://myaccount.nytimes3xbfgragh.onion/auth/login?response_type=cookie\&client_id=vi}{}

\href{https://www.nytimes3xbfgragh.onion/section/todayspaper}{Today's
Paper}

\href{/section/opinion/sunday}{Sunday Review}\textbar{}How to Give
Children Joy, Even During a Pandemic

\url{https://nyti.ms/3eXkkMN}

\begin{itemize}
\item
\item
\item
\item
\item
\item
\end{itemize}

Advertisement

\protect\hyperlink{after-top}{Continue reading the main story}

\href{/section/opinion}{Opinion}

Supported by

\protect\hyperlink{after-sponsor}{Continue reading the main story}

\hypertarget{how-to-give-children-joy-even-during-a-pandemic}{%
\section{How to Give Children Joy, Even During a
Pandemic}\label{how-to-give-children-joy-even-during-a-pandemic}}

The coronavirus forces parents to weigh their kids' safety against the
need for freedom --- a tension Black parents have been contending with
for generations.

\href{https://www.nytimes3xbfgragh.onion/by/esau-mccaulley}{\includegraphics{https://static01.graylady3jvrrxbe.onion/images/2020/07/21/opinion/author-esau-mccaulley/author-esau-mccaulley-thumbLarge.png}}

By \href{https://www.nytimes3xbfgragh.onion/by/esau-mccaulley}{Esau
McCaulley}

Contributing Opinion Writer

\begin{itemize}
\item
  July 23, 2020
\item
  \begin{itemize}
  \item
  \item
  \item
  \item
  \item
  \item
  \end{itemize}
\end{itemize}

\includegraphics{https://static01.graylady3jvrrxbe.onion/images/2020/07/26/opinion/sunday/26mccaulley/26mccaulley-articleLarge.jpg?quality=75\&auto=webp\&disable=upscale}

I drove my oldest son, a middle schooler, to his baseball game a few
miles down the road. There was a slight breeze, a perfect setting for
summer activity.

On the field, it looked like a standard summer of boys learning the
nuances of the sport, some further along the road to adult coordination
than others. What stood out were the masks on all their faces: a visual
reminder that we are in the summer of Covid-19. Joy and sadness,
normalcy and profound change competed among the young athletes for our
attention.

During the game we parents stood at least six feet away from one
another. We discussed the opening of school in the fall, the
politicization of wearing masks in public, and how quickly life had
changed. We talked about how the last time we saw one another it was at
the tryouts for the team in early March, before the world shut down.
They asked me how I planned to teach my college students with all the
changes, and I answered that I did not know.

As we talked, I wondered, as I have many times during the pandemic, how
much to tell my children. Does a 9-year-old or a 12-year-old need to
know how many have died?

This mixture of safety and peril and difficult decisions about a child's
freedom to play: It is familiar to me. Covid-19 has given all parents a
small taste of what it is like to be a Black parent.

Having our bodies as potential threats because of the coronavirus has
introduced all of America to what it is like to be perceived as a
problem merely by our presence. The major difference is that some of us
do carry an unknown virus, while blackness is simply one manifestation
of God's creativity. Nonetheless, the perceived danger has given others
insight into what it is like for Black bodies, even children's bodies,
to be a source of fear.

Pandemic parenting involves a similarly challenging calculus that those
of us who raise Black and brown children have faced for centuries. How
do we balance the need to protect from danger with the desire to let
them be young and free?

Dr. Martin Luther King Jr. once recounted the story of not knowing what
to say when his 6-year-old daughter asked to go to
\href{https://www.africa.upenn.edu/Articles_Gen/Letter_Birmingham.html}{a
local theme park called Funtown}. He did not want to tell her that she
could not go because Black people were forbidden. He said that
explaining segregation to his daughter was more daunting than the
speeches he gave all over the country.

People often mention ``the talk'' as if the only conversation Black
parents have to have with their children is about the complex
interaction with the police. That is hard, but not the only thing. At
some point we have to tell them about Funtown: the limits society wants
to place on them and the struggle to tear those limits down.

My son is in the midst of the transition from early middle schooler to
emerging teenager. In the strange moral reasoning of the United States,
this will mean a move from cute to dangerous. His Black body and his
increasing size could, in certain circumstances, be weaponized against
him. Citizens and officers merely have to utter the words ``I feared for
my life'' and his Black life could be in peril.

When do I warn him about wandering around our largely white neighborhood
in the evening? How long do I let him remain a child? Am I negligent if
in my attempts to give him a ``normal'' childhood I leave him unprepared
for the challenges he faces?

I was initially hesitant to have my son return to the baseball diamond,
even though it is a sport well suited for distancing. I reluctantly
agreed. My son's coach said that some teams would follow the safety
rules and others would not, but that he would do the best he could to
keep them safe.

At that first game, our team dutifully wore their masks. The other team
did not. Had politics sneaked on to the baseball field? Were some
families and teams simply not as worried as us? I did the math again.
Should I interrupt the game or remove my son? We already barely had
enough players to field a full squad. I decided to let him play, to be a
child.

There are no easy answers as to how to parent Black children in America
inside or outside a pandemic. It is not my job to tell someone how to do
it.

My wife and I have drifted to a bias toward joy. We tell our children
about some major events; other burdens we carry ourselves. Our children
know much of the history of this country, but the focus is on Black
triumph over suffering, not the suffering itself. I immerse them in the
soul, hip-hop and gospel music that has lifted many a weary soul even
when they would rather listen to Kidz Bop.

I have told them of Moses and the Israelites, of Mary Jesus' mother and
her dramatic yes to God. They know about Sojourner and her railroad and
Martin and his dream of Mother Pollard and her rested feet. I remind
them that God has looked upon their Black skin, hair and bodies and
called it good.

I am making deposit after deposit of Black joy and faith in the hope
that it will be with them when the inevitable struggle comes. I do so
because that is what my mother did for me.

My oldest has a favorite saint, the North African church father
Athanasius. He was called Athanasius Contra Mundum --- against the
world, a name he received for standing against seemingly insurmountable
foes, even at great cost, because of his convictions. My son loves the
defiance. Given that being Black in America can sometimes feel against
the world, that is a great trait to admire.

My son's team failed to emerge victorious in that first game back on the
field. A socially distanced wave from across the diamond replaced the
customary handshakes. Some two weeks later, it seems that we avoided
infection.

I'm glad baseball is back in our lives. Even with masks covering the
kids' faces and parents shouting encouragements from afar, it is still
baseball in the summer. There are still kids in the outfield more
interested in the cloud formations than a pop fly. The clean double play
remains the stuff of legend.

We parents had a brief moment of shared victory. We had given our
children the gift that is often only available to the young: the chance
for uncomplicated joy. We who looked on wondered what was next.

\emph{The Times is committed to publishing}
\href{https://www.nytimes3xbfgragh.onion/2019/01/31/opinion/letters/letters-to-editor-new-york-times-women.html}{\emph{a
diversity of letters}} \emph{to the editor. We'd like to hear what you
think about this or any of our articles. Here are some}
\href{https://help.nytimes3xbfgragh.onion/hc/en-us/articles/115014925288-How-to-submit-a-letter-to-the-editor}{\emph{tips}}\emph{.
And here's our email:}
\href{mailto:letters@NYTimes.com}{\emph{letters@NYTimes.com}}\emph{.}

\emph{Follow The New York Times Opinion section on}
\href{https://www.facebookcorewwwi.onion/nytopinion}{\emph{Facebook}}\emph{,}
\href{http://twitter.com/NYTOpinion}{\emph{Twitter (@NYTopinion)}}
\emph{and}
\href{https://www.instagram.com/nytopinion/}{\emph{Instagram}}\emph{.}

Advertisement

\protect\hyperlink{after-bottom}{Continue reading the main story}

\hypertarget{site-index}{%
\subsection{Site Index}\label{site-index}}

\hypertarget{site-information-navigation}{%
\subsection{Site Information
Navigation}\label{site-information-navigation}}

\begin{itemize}
\tightlist
\item
  \href{https://help.nytimes3xbfgragh.onion/hc/en-us/articles/115014792127-Copyright-notice}{©~2020~The
  New York Times Company}
\end{itemize}

\begin{itemize}
\tightlist
\item
  \href{https://www.nytco.com/}{NYTCo}
\item
  \href{https://help.nytimes3xbfgragh.onion/hc/en-us/articles/115015385887-Contact-Us}{Contact
  Us}
\item
  \href{https://www.nytco.com/careers/}{Work with us}
\item
  \href{https://nytmediakit.com/}{Advertise}
\item
  \href{http://www.tbrandstudio.com/}{T Brand Studio}
\item
  \href{https://www.nytimes3xbfgragh.onion/privacy/cookie-policy\#how-do-i-manage-trackers}{Your
  Ad Choices}
\item
  \href{https://www.nytimes3xbfgragh.onion/privacy}{Privacy}
\item
  \href{https://help.nytimes3xbfgragh.onion/hc/en-us/articles/115014893428-Terms-of-service}{Terms
  of Service}
\item
  \href{https://help.nytimes3xbfgragh.onion/hc/en-us/articles/115014893968-Terms-of-sale}{Terms
  of Sale}
\item
  \href{https://spiderbites.nytimes3xbfgragh.onion}{Site Map}
\item
  \href{https://help.nytimes3xbfgragh.onion/hc/en-us}{Help}
\item
  \href{https://www.nytimes3xbfgragh.onion/subscription?campaignId=37WXW}{Subscriptions}
\end{itemize}
