Sections

SEARCH

\protect\hyperlink{site-content}{Skip to
content}\protect\hyperlink{site-index}{Skip to site index}

\href{https://myaccount.nytimes3xbfgragh.onion/auth/login?response_type=cookie\&client_id=vi}{}

\href{https://www.nytimes3xbfgragh.onion/section/todayspaper}{Today's
Paper}

\href{/section/opinion}{Opinion}\textbar{}A `Safety Net' That's a
Kafkaesque Mess

\url{https://nyti.ms/2Byq9Tu}

\begin{itemize}
\item
\item
\item
\item
\item
\end{itemize}

Advertisement

\protect\hyperlink{after-top}{Continue reading the main story}

\href{/section/opinion}{Opinion}

Supported by

\protect\hyperlink{after-sponsor}{Continue reading the main story}

Disability

\hypertarget{a-safety-net-thats-a-kafkaesque-mess}{%
\section{A `Safety Net' That's a Kafkaesque
Mess}\label{a-safety-net-thats-a-kafkaesque-mess}}

The Supplemental Security Income program, on which millions of disabled
Americans depend, is daunting by design.

By Ari Ne'eman

Mr. Ne'eman is a disability rights activist and author.

\begin{itemize}
\item
  July 25, 2020
\item
  \begin{itemize}
  \item
  \item
  \item
  \item
  \item
  \end{itemize}
\end{itemize}

\includegraphics{https://static01.graylady3jvrrxbe.onion/images/2020/07/25/opinion/25disability-social-security/25disability-social-security-articleLarge.jpg?quality=75\&auto=webp\&disable=upscale}

In July 1988, a crowd of roughly 40 protesters gathered outside the Los
Angeles Federal Building. As they chanted, their leader, Dr. Paul
Longmore, a visiting scholar at Stanford, set fire to a biography of
George Washington on a small barbecue grill.

Dr. Longmore did not hate our first president --- far from it. The
\href{https://www.npr.org/sections/health-shots/2010/08/11/129127432/paul-longmore-historian-and-advocate-for-disabled-dies}{book
he was burning was in fact one he had written}. A survivor of polio who
relied on a ventilator and personal care aides paid for by Medicaid, Dr.
Longmore had organized the demonstration to protest Social Security
Administration policies that prevented him from being paid for writing
it.

Supplemental Security Income --- a cash assistance program whose
beneficiaries also get Medicaid coverage --- sharply limits the income
and assets of those who receive it. If Dr. Longmore were to receive
royalties from his publisher, he stood to lose over \$20,000 a year in
services, without which he could not survive.

Dr. Longmore would later have an illustrious academic career, becoming
\href{https://longmoreinstitute.sfsu.edu/paul-k-longmore}{a founder of
disability history}. In time, his activism would lead to modest changes
to Social Security rules on book royalties. But even as we celebrate the
30th anniversary of the Americans with Disabilities Act, disabled people
still struggle with byzantine rules for work.

Today, over eight million people receive S.S.I. benefits (not to be
confused with Social Security Disability Insurance, which offers
assistance on less onerous terms to those with sufficient work history).
\href{https://www.ssa.gov/policy/docs/statcomps/ssi_monthly/2020-06/table02.html}{Slightly
more than half} are working-age adults with disabilities. Of these,
\href{https://www.ssa.gov/policy/docs/statcomps/ssi_asr/2018/sect07.html\#_blank}{approximately
340,000} work. While impairment plays some role in this low employment
rate, so too does the complexity of the system and the dire consequences
of making a wrong move.

\href{https://www.ssa.gov/pubs/EN-05-10095.pdf}{Working beneficiaries}
of S.S.I. generally keep their first \$85 in income each month, after
which they lose \$1 in benefits for every \$2 of earnings, what amounts
to a 50 percent marginal ``tax rate.'' Receiving food and shelter from
family or friends
\href{https://www.ssa.gov/ssi/text-living-ussi.htm\#_blank}{cuts the
amount of the benefit by one-third}. Beneficiaries are prohibited from
saving more than \$2,000. This
\href{https://www.ssa.gov/policy/docs/chartbooks/disability_trends/sect05.html}{limit
has been the same since 1984.}

Beneficiaries can keep additional assets and income by setting up a Plan
to Achieve Self Support --- a written plan that must be submitted to and
approved by the agency --- but doing so is sufficiently complex
\href{https://www.ssa.gov/policy/docs/statcomps/ssi_asr/2018/sect07.html\#_blank}{that
only 568 people} in the entire country made use of this option in 2018.
Nor are benefits so generous that recipients have no need to work ---
the \href{https://www.ssa.gov/oact/cola/SSI.html}{monthly federal S.S.I.
payment for 2020} was a mere \$783. While many states supplement this,
recipients still face terrible poverty.

Keeping track of these arcane details would be difficult for anyone. For
low-income adults with intellectual disabilities and mental illnesses,
\href{https://www.ssa.gov/policy/docs/statcomps/ssi_asr/2018/sect07.html\#_blank}{who
make up most} working S.S.I. recipients, it is nearly impossible.
Failure to do so can have awful consequences. The agency often overpays
working disabled recipients,
\href{https://newrepublic.com/article/158095/civil-legal-system-no-money-no-lawyer-no-justice}{only
to notify them that they must pay back thousands of dollars}, sometimes
years after the fact.

Looking at the mess facing S.S.I. recipients who try to work, one feels
that a terrible mistake has been made. But history tells a different
story: this Kafkaesque nightmare was a deliberate choice.

When the Social Security Act passed in 1935, its architects had a
problem. Amid the Great Depression, the elderly faced horrifying
poverty. And yet, because Social Security's Old Age Insurance program
required beneficiaries to have paid into the system in order to get
benefits, the existing generation of seniors could never qualify for the
program. To rectify this, Congress funded states to establish public
assistance programs for seniors, single mothers and the blind.

But offering immediate cash assistance presented a risk. If recipients
of public assistance could get help on the same easy terms as ``earned
benefits'' like Old Age Insurance, what motivation would they have to
pay into Social Security?

Before Social Security's passage, blind activists had convinced several
states to establish blindness pension programs that gave a flat cash
grant with limited asset and income restrictions. States
\href{https://www.ssa.gov/history/reports/ces/ces2witte2.html}{established
similar benefits for the elderly} too. Many wished to let those
receiving benefits participate in the work force and receive help from
family and friends, seeing government help as perfectly compatible with
living a ``normal'' life.

In the law's early years, federal officials, including the Social
Security Board's chairman Arthur Altmeyer, feared that generous state
public assistance programs would build momentum for replacing Old Age
Insurance with a more progressive alternative. In the 1930s, Social
Security still faced attacks from left-wing critics like the
\href{https://www.finance.senate.gov/imo/media/doc/77HrgPensions.pdf}{Congress
of Industrial Organizations} and
\href{https://www.ssa.gov/history/towns5.html}{the populist Townsend
movement}. Social Security paid higher benefits to higher-paid workers
and relied on a payroll tax that let the wealthiest pay a lower
percentage of their income (a problem that still exists today). Critics
wanted equal benefits for all, financed by a redistributive payroll tax.

To protect against this possibility, Altmeyer made getting public
assistance as unpleasant as he possibly could. States were told that
they could not receive federal money unless they conducted intrusive
investigations of every applicant, reducing benefits to those who
received food or shelter from family or friends. Programs that permitted
beneficiaries to work and save were told to adopt more restrictive
eligibility standards or be denied funding.

Inevitably, this sparked backlash. In 1940, the National Federation of
the Blind was founded to fight ``the oppression of the social worker and
the arrogance of the governmental administrator.'' It lobbied Congress
to let beneficiaries benefit from their labor, securing incremental
improvements.

But Altmeyer's vision remained largely intact. Public assistance
maintained an aggressive means test. When disability and aging programs
were federalized into the Supplemental Security Income program in 1971,
these restrictions came with them.

Today, economists \href{https://www.nber.org/papers/w26041}{refer} to
Altmeyer's strategy as an ``ordeal'' --- a burden imposed on those
receiving benefits that yields no benefit to others. The purpose of an
ordeal is not to help the beneficiary or others in society. Instead,
ordeals deliberately make a program or service worse in order to
discourage people from using it.

Ordeals are a common attribute of programs serving the poor. Since
beneficiaries cannot be charged money they do not have, policymakers
impose a cost in time and suffering. Such indignities are designed to
weed out those who would rather go without than brave the bureaucratic
gantlet.

Ordeals are also typical in other disability programs: Medicaid's
\href{https://readingroom.law.gsu.edu/cgi/viewcontent.cgi?article=2416\&context=gsulr}{bias
in favor of institutional care} serves a similar purpose, deterring
people from seeking services by elevating institutions over
higher-quality community supports. Even if in-home care is less costly
for most, institutions are prioritized since many are willing to go
without rather than place themselves or their loved ones inside one. The
program is deliberately made worse to discourage use.

Ordeals likely do save some taxpayer money. But they impose tremendous
human cost. They are also only politically viable when imposed upon the
marginalized --- more powerful constituencies are able to demand a
higher class of service from their government.

The Covid-19 pandemic exposed many middle-class Americans
\href{https://www.theledger.com/news/20200706/state-feds-probe--jobless-benefits-system-mdash-but-who-will-fix-it}{to
some of the}
\href{https://madison.com/wsj/news/local/govt-and-politics/lives-on-hold-covid-19-pandemic-exposes-failures-of-wisconsin-unemployment-insurance-system/article_96810941-894d-5bb9-9485-16912cb2be57.html}{indignities
in public programs}
\href{https://coronavirus.jhu.edu/from-our-experts/can-the-us-safety-net-handle-the-covid-19-pandemic-and-recession}{more
typically relied upon by the poor}. One hopes that this experience will
spark empathy, an all too rare phenomenon in American politics, and
perhaps reform. In one promising sign, Joe Biden recently proposed to
increase S.S.I. benefits while easing asset and income restrictions
\href{https://joebiden.com/disabilities/}{in his campaign's disability
plan}.

Still, ordeals will likely remain a common feature in both disability
and poverty policy well after the pandemic is over. As we enter the
A.D.A.'s fourth decade, challenging them will be one of the disability
rights movement's most important tasks.

Ari Ne'eman, a visiting scholar at the Lurie Institute for Disability
Policy at Brandeis University and a doctoral student in health policy at
Harvard, is at work on a book on the history of American disability
advocacy.

\emph{Now in print: ``}\href{https://www.aboutusbook.com/}{\emph{About
Us: Essays From the Disability Series of The New York Times}}\emph{,''
edited by Peter Catapano and Rosemarie Garland-Thomson, published by
Liveright.}

\emph{The Times is committed to publishing}
\href{https://www.nytimes3xbfgragh.onion/2019/01/31/opinion/letters/letters-to-editor-new-york-times-women.html}{\emph{a
diversity of letters}} \emph{to the editor. We'd like to hear what you
think about this or any of our articles. Here are some}
\href{https://help.nytimes3xbfgragh.onion/hc/en-us/articles/115014925288-How-to-submit-a-letter-to-the-editor}{\emph{tips}}\emph{.
And here's our email:}
\href{mailto:letters@NYTimes.com}{\emph{letters@NYTimes.com}}\emph{.}

\emph{Follow The New York Times Opinion section on}
\href{https://www.facebookcorewwwi.onion/nytopinion}{\emph{Facebook}}\emph{,}
\href{http://twitter.com/NYTOpinion}{\emph{Twitter (@NYTopinion)}}
\emph{and}
\href{https://www.instagram.com/nytopinion/}{\emph{Instagram}}\emph{.}

Advertisement

\protect\hyperlink{after-bottom}{Continue reading the main story}

\hypertarget{site-index}{%
\subsection{Site Index}\label{site-index}}

\hypertarget{site-information-navigation}{%
\subsection{Site Information
Navigation}\label{site-information-navigation}}

\begin{itemize}
\tightlist
\item
  \href{https://help.nytimes3xbfgragh.onion/hc/en-us/articles/115014792127-Copyright-notice}{©~2020~The
  New York Times Company}
\end{itemize}

\begin{itemize}
\tightlist
\item
  \href{https://www.nytco.com/}{NYTCo}
\item
  \href{https://help.nytimes3xbfgragh.onion/hc/en-us/articles/115015385887-Contact-Us}{Contact
  Us}
\item
  \href{https://www.nytco.com/careers/}{Work with us}
\item
  \href{https://nytmediakit.com/}{Advertise}
\item
  \href{http://www.tbrandstudio.com/}{T Brand Studio}
\item
  \href{https://www.nytimes3xbfgragh.onion/privacy/cookie-policy\#how-do-i-manage-trackers}{Your
  Ad Choices}
\item
  \href{https://www.nytimes3xbfgragh.onion/privacy}{Privacy}
\item
  \href{https://help.nytimes3xbfgragh.onion/hc/en-us/articles/115014893428-Terms-of-service}{Terms
  of Service}
\item
  \href{https://help.nytimes3xbfgragh.onion/hc/en-us/articles/115014893968-Terms-of-sale}{Terms
  of Sale}
\item
  \href{https://spiderbites.nytimes3xbfgragh.onion}{Site Map}
\item
  \href{https://help.nytimes3xbfgragh.onion/hc/en-us}{Help}
\item
  \href{https://www.nytimes3xbfgragh.onion/subscription?campaignId=37WXW}{Subscriptions}
\end{itemize}
