Sections

SEARCH

\protect\hyperlink{site-content}{Skip to
content}\protect\hyperlink{site-index}{Skip to site index}

\href{https://www.nytimes3xbfgragh.onion/section/us}{U.S.}

\href{https://myaccount.nytimes3xbfgragh.onion/auth/login?response_type=cookie\&client_id=vi}{}

\href{https://www.nytimes3xbfgragh.onion/section/todayspaper}{Today's
Paper}

\href{/section/us}{U.S.}\textbar{}Hurricane Hanna Hits Texas Region
Struggling With Coronavirus

\url{https://nyti.ms/3hG7lRL}

\begin{itemize}
\item
\item
\item
\item
\item
\end{itemize}

\href{https://www.nytimes3xbfgragh.onion/news-event/coronavirus?action=click\&pgtype=Article\&state=default\&module=styln-coronavirus-national\&region=TOP_BANNER\&context=storylines_menu}{The
Coronavirus Outbreak}

\begin{itemize}
\tightlist
\item
  live\href{https://www.nytimes3xbfgragh.onion/2020/07/25/world/coronavirus-covid-19.html?action=click\&pgtype=Article\&state=default\&module=styln-coronavirus-national\&region=TOP_BANNER\&context=storylines_menu}{Latest
  Updates}
\item
  \href{https://www.nytimes3xbfgragh.onion/interactive/2020/us/coronavirus-us-cases.html?action=click\&pgtype=Article\&state=default\&module=styln-coronavirus-national\&region=TOP_BANNER\&context=storylines_menu}{Maps
  and Cases}
\item
  \href{https://www.nytimes3xbfgragh.onion/interactive/2020/07/23/us/coronavirus-hotspots-countries.html?action=click\&pgtype=Article\&state=default\&module=styln-coronavirus-national\&region=TOP_BANNER\&context=storylines_menu}{How
  the U.S. Compares}
\item
  \href{https://www.nytimes3xbfgragh.onion/interactive/2020/science/coronavirus-drugs-treatments.html?action=click\&pgtype=Article\&state=default\&module=styln-coronavirus-national\&region=TOP_BANNER\&context=storylines_menu}{Possible
  Treatments}
\item
  \href{https://www.nytimes3xbfgragh.onion/live/2020/07/24/business/stock-market-updates-coronavirus?action=click\&pgtype=Article\&state=default\&module=styln-coronavirus-national\&region=TOP_BANNER\&context=storylines_menu}{Economy}
\end{itemize}

Advertisement

\protect\hyperlink{after-top}{Continue reading the main story}

Supported by

\protect\hyperlink{after-sponsor}{Continue reading the main story}

\hypertarget{hurricane-hanna-hits-texas-region-struggling-with-coronavirus}{%
\section{Hurricane Hanna Hits Texas Region Struggling With
Coronavirus}\label{hurricane-hanna-hits-texas-region-struggling-with-coronavirus}}

Some of the communities in the storm's path have seen a sudden spike in
Covid-19 cases and hospitalizations as Texas has become one of the
largest hot spots in the country.

\includegraphics{https://static01.graylady3jvrrxbe.onion/images/2020/07/25/us/25texas-storm-4/merlin_174939237_2d995a3b-8bcf-483c-a8e9-87382a68603e-articleLarge.jpg?quality=75\&auto=webp\&disable=upscale}

By \href{https://www.nytimes3xbfgragh.onion/by/edgar-sandoval}{Edgar
Sandoval},
\href{https://www.nytimes3xbfgragh.onion/by/nicholas-bogel-burroughs}{Nicholas
Bogel-Burroughs} and
\href{https://www.nytimes3xbfgragh.onion/by/manny-fernandez}{Manny
Fernandez}

\begin{itemize}
\item
  July 25, 2020
\item
  \begin{itemize}
  \item
  \item
  \item
  \item
  \item
  \end{itemize}
\end{itemize}

CORPUS CHRISTI, Texas --- Already battered by the coronavirus pandemic,
southeast Texas faced a new but no less frightening foe on Saturday, as
Hurricane Hanna slammed the coast with heavy rains and powerful winds.

Hanna strengthened from a tropical storm to a Category 1 hurricane on
Saturday, becoming the first hurricane to hit the southern coastal
region of Texas since Hurricane Harvey struck the area in August 2017
and caused the worst rainstorm in United States history.

Hanna's eye made landfall on Padre Island, about 60 miles north of the
U.S.-Mexico border, at about 5 p.m. on Saturday, with winds of 90 m.p.h.
As the National Weather Service
\href{https://forecast.weather.gov/showsigwx.php?warnzone=TXZ242\&warncounty=TXC273\&firewxzone=TXZ242\&local_place1=3\%20Miles\%20NNE\%20Kingsville,\%20Naval\%20Air\%20Station\%20TX\&product1=Hurricane+Warning\#.XxxWW55KjIV}{warned}
that the strong winds could peel roofs from homes, mangle trees and
cause power failures, mayors and local officials turned from one crisis
mode to another.

The cities and counties in the path of Hanna are some of the same
communities that have seen a sudden spike in Covid-19 cases and
hospitalizations as Texas has
\href{https://www.nytimes3xbfgragh.onion/interactive/2020/us/texas-coronavirus-cases.html}{become
one of the largest hot spots in the country}. In a state that is no
stranger to bad weather, the typical hurricane-prep ritual was altered
by social distancing and face coverings, with fever checks required to
enter officials' news briefings and sandbag distribution provided by
workers who covered their faces in masks and bandannas.

\href{https://www.nytimes3xbfgragh.onion/interactive/2020/07/25/us/hurricane-hanna-tracker-map.html}{}

\includegraphics{https://static01.graylady3jvrrxbe.onion/images/2020/07/25/us/hurricane-hanna-tracker-map-promo-1595693859974/hurricane-hanna-tracker-map-promo-1595693859974-articleLarge-v4.jpg}

\hypertarget{live-hurricane-hanna-tracking-map}{%
\subsection{Live Hurricane Hanna Tracking
Map}\label{live-hurricane-hanna-tracking-map}}

The storm made landfall Saturday afternoon on the southern coast of
Texas.

When natural disasters strike, swift and effective disaster response by
local, state and federal officials has always been difficult. The
pandemic has made it even harder.

``Hurricanes can be deadly events,'' Gov. Greg Abbott said at a news
conference on Saturday afternoon. ``We cannot allow this hurricane to
lead to a more catastrophically deadly event by stoking additional
spread of Covid-19.''

Mr. Abbott said he was issuing a disaster declaration for 32 counties.
Chief W. Nim Kidd of the Texas Division of Emergency Management said the
Freeman Coliseum in San Antonio, usually used for concerts, rodeos and
other events, would be converted into a reception center for people who
had fled their homes. San Antonio officials were quick to make clear
that the arena was not serving as a shelter, but only as a location
where evacuees would be given vouchers to stay at hotels in the area.

It was another example of how social distancing is changing where, when
and how emergency shelters operate. Officials in California announced
recently that the coronavirus was affecting their wildfire preparations,
including by having fewer inmates available to assist because of
infections and quarantines. But there have been unexpected benefits,
too. In late March, a destructive tornado tore through Jonesboro, Ark.,
and the heart of its commercial district, but
\href{https://www.nytimes3xbfgragh.onion/2020/03/29/us/tornado-coronavirus-arkansas.html}{not
a single person died}, in part because businesses were closed and
residents were already sheltered at home from the virus.

As Hanna's outer bands reached Nueces County, which includes Corpus
Christi and has had nearly 2,000 new virus cases and 47 deaths in the
past seven days alone, officials reminded residents fleeing low-lying
areas to bring their face masks with them.

\includegraphics{https://static01.graylady3jvrrxbe.onion/images/2020/07/25/us/25texas-storm-1/merlin_174927330_520049a4-5613-498b-be92-433c30df7aa5-articleLarge.jpg?quality=75\&auto=webp\&disable=upscale}

``We're riding two horses right now, so be sure to take care of what
keeps you safe through the Covid experience,'' Mayor Joe McComb of
Corpus Christi said at a news conference on Friday. Moments later, a
woman in a mask and gloves carefully wiped down the podium to sanitize
it before the city manager stepped forward.

On Saturday, the mayor told reporters that people who had welcomed
friends or relatives into their homes to ride out the hurricane should
wear masks while indoors. ``Wear the mask in the house,'' Mr. McComb
said. ``I know that probably sounds kind of crazy, but keeping safe
sounds pretty good.''

\hypertarget{latest-updates-global-coronavirus-outbreak}{%
\section{\texorpdfstring{\href{https://www.nytimes3xbfgragh.onion/2020/07/24/world/coronavirus-covid-19.html?action=click\&pgtype=Article\&state=default\&module=styln-coronavirus-national\&region=MAIN_CONTENT_1\&context=storylines_live_updates}{Latest
Updates: Global Coronavirus
Outbreak}}{Latest Updates: Global Coronavirus Outbreak}}\label{latest-updates-global-coronavirus-outbreak}}

Updated 2020-07-25T09:43:39.358Z

\begin{itemize}
\tightlist
\item
  \href{https://www.nytimes3xbfgragh.onion/2020/07/24/world/coronavirus-covid-19.html?action=click\&pgtype=Article\&state=default\&module=styln-coronavirus-national\&region=MAIN_CONTENT_1\&context=storylines_live_updates\#link-4331c0b7}{The
  U.S. nears a single-day record as infections show no sign of slowing.}
\item
  \href{https://www.nytimes3xbfgragh.onion/2020/07/24/world/coronavirus-covid-19.html?action=click\&pgtype=Article\&state=default\&module=styln-coronavirus-national\&region=MAIN_CONTENT_1\&context=storylines_live_updates\#link-625c8263}{Pelosi
  again rules out a short extension of \$600 unemployment benefits,
  pushing for broader aid.}
\item
  \href{https://www.nytimes3xbfgragh.onion/2020/07/24/world/coronavirus-covid-19.html?action=click\&pgtype=Article\&state=default\&module=styln-coronavirus-national\&region=MAIN_CONTENT_1\&context=storylines_live_updates\#link-6ec043c9}{A
  new C.D.C. statement on schools calls for reopening and downplays the
  potential health risks.}
\end{itemize}

\href{https://www.nytimes3xbfgragh.onion/2020/07/24/world/coronavirus-covid-19.html?action=click\&pgtype=Article\&state=default\&module=styln-coronavirus-national\&region=MAIN_CONTENT_1\&context=storylines_live_updates}{See
more updates}

More live coverage:
\href{https://www.nytimes3xbfgragh.onion/live/2020/07/24/business/stock-market-updates-coronavirus?action=click\&pgtype=Article\&state=default\&module=styln-coronavirus-national\&region=MAIN_CONTENT_1\&context=storylines_live_updates}{Markets}

The mayor's message to the public on Saturday was simple: Keep doing
what you have been doing during the pandemic --- stay at home.

``I know we've been staying at home for five months because of the
corona,'' Mr. McComb said, adding, ``And so staying home doesn't sound
real popular, but right now this is a real important matter.''

The rise in cases in Nueces County was fueled, in part, by visitors from
Houston and elsewhere who
\href{https://www.nytimes3xbfgragh.onion/2020/07/11/us/coronavirus-texas-corpus-christi.html}{flocked
to its beaches} when it had a low case count. More than 10,000 people in
Nueces County have
\href{https://www.nytimes3xbfgragh.onion/interactive/2020/us/texas-coronavirus-cases.html}{been
infected with the virus}. At least 129 people have died, and 430 people
were being treated in hospitals on Friday, the highest total since the
pandemic began.

The storm arrived in Corpus Christi on Saturday morning with light rain
and strengthening wind that rattled windows and threatened to uproot
swaying palm trees.

Image

Employees with the City of Corpus Christi loaded sandbags into people's
cars on Friday.Credit...Courtney Sacco/Corpus Christi Caller-Times, via
Associated Press

By early Saturday evening, a storm surge slammed the city's Art Museum
of South Texas, which faces the bay, drawing dozens of concerned
onlookers.

``They have art from all over the world,'' said Andrew Gonzalez, 30, who
used his cellphone to record the rising waters splashing around the
museum's lower levels. ``I hope they removed all the art from the first
floor to the upper floors before the storm got here.''

Webb County, which includes the border city of Laredo and is about 150
miles inland from Corpus Christi, has
\href{https://www.lmtonline.com/local/article/City-officials-About-150-Hurricane-Harvey-11960437.php}{traditionally
provided shelter} to people fleeing the Texas coast, as it did for
hundreds of people during the devastation that Harvey wrought along the
coastline in 2017.

On Saturday, officials were grappling with how to handle a potential
influx of evacuees without worsening the spread of the coronavirus,
which has surged there in recent weeks. More than a quarter of Webb
County's nearly 5,000 cases have come in just the past week, during
which 38 people died from the virus.

Tano E. Tijerina, the county executive, said that if evacuees came to
the area, officials would take their temperatures, provide them with
protective gear and try to separate people with virus symptoms from
those who appeared healthy.

``We're going to do whatever we can to help anyone, that's with Covid or
without Covid,'' Mr. Tijerina said. ``We're taking a risk, but we're
being neighborly and we're never going to turn people away.''

He said that schools, community centers and even the county fairgrounds
could serve as shelters if necessary. He said the daily coronavirus
reports were increasingly dire, with four or five people dying each day.

\href{https://www.nytimes3xbfgragh.onion/news-event/coronavirus?action=click\&pgtype=Article\&state=default\&module=styln-coronavirus-national\&region=MAIN_CONTENT_3\&context=storylines_faq}{}

\hypertarget{the-coronavirus-outbreak-}{%
\subsubsection{The Coronavirus Outbreak
›}\label{the-coronavirus-outbreak-}}

\hypertarget{frequently-asked-questions}{%
\paragraph{Frequently Asked
Questions}\label{frequently-asked-questions}}

Updated July 23, 2020

\begin{itemize}
\item ~
  \hypertarget{what-is-school-going-to-look-like-in-september}{%
  \paragraph{What is school going to look like in
  September?}\label{what-is-school-going-to-look-like-in-september}}

  \begin{itemize}
  \tightlist
  \item
    It is unlikely that many schools will return to a normal schedule
    this fall, requiring the grind of
    \href{https://www.nytimes3xbfgragh.onion/2020/06/05/us/coronavirus-education-lost-learning.html?action=click\&pgtype=Article\&state=default\&module=styln-coronavirus-national\&region=MAIN_CONTENT_3\&context=storylines_faq}{online
    learning},
    \href{https://www.nytimes3xbfgragh.onion/2020/05/29/us/coronavirus-child-care-centers.html?action=click\&pgtype=Article\&state=default\&module=styln-coronavirus-national\&region=MAIN_CONTENT_3\&context=storylines_faq}{makeshift
    child care} and
    \href{https://www.nytimes3xbfgragh.onion/2020/06/03/business/economy/coronavirus-working-women.html?action=click\&pgtype=Article\&state=default\&module=styln-coronavirus-national\&region=MAIN_CONTENT_3\&context=storylines_faq}{stunted
    workdays} to continue. California's two largest public school
    districts --- Los Angeles and San Diego --- said on July 13, that
    \href{https://www.nytimes3xbfgragh.onion/2020/07/13/us/lausd-san-diego-school-reopening.html?action=click\&pgtype=Article\&state=default\&module=styln-coronavirus-national\&region=MAIN_CONTENT_3\&context=storylines_faq}{instruction
    will be remote-only in the fall}, citing concerns that surging
    coronavirus infections in their areas pose too dire a risk for
    students and teachers. Together, the two districts enroll some
    825,000 students. They are the largest in the country so far to
    abandon plans for even a partial physical return to classrooms when
    they reopen in August. For other districts, the solution won't be an
    all-or-nothing approach.
    \href{https://bioethics.jhu.edu/research-and-outreach/projects/eschool-initiative/school-policy-tracker/}{Many
    systems}, including the nation's largest, New York City, are
    devising
    \href{https://www.nytimes3xbfgragh.onion/2020/06/26/us/coronavirus-schools-reopen-fall.html?action=click\&pgtype=Article\&state=default\&module=styln-coronavirus-national\&region=MAIN_CONTENT_3\&context=storylines_faq}{hybrid
    plans} that involve spending some days in classrooms and other days
    online. There's no national policy on this yet, so check with your
    municipal school system regularly to see what is happening in your
    community.
  \end{itemize}
\item ~
  \hypertarget{is-the-coronavirus-airborne}{%
  \paragraph{Is the coronavirus
  airborne?}\label{is-the-coronavirus-airborne}}

  \begin{itemize}
  \tightlist
  \item
    The coronavirus
    \href{https://www.nytimes3xbfgragh.onion/2020/07/04/health/239-experts-with-one-big-claim-the-coronavirus-is-airborne.html?action=click\&pgtype=Article\&state=default\&module=styln-coronavirus-national\&region=MAIN_CONTENT_3\&context=storylines_faq}{can
    stay aloft for hours in tiny droplets in stagnant air}, infecting
    people as they inhale, mounting scientific evidence suggests. This
    risk is highest in crowded indoor spaces with poor ventilation, and
    may help explain super-spreading events reported in meatpacking
    plants, churches and restaurants.
    \href{https://www.nytimes3xbfgragh.onion/2020/07/06/health/coronavirus-airborne-aerosols.html?action=click\&pgtype=Article\&state=default\&module=styln-coronavirus-national\&region=MAIN_CONTENT_3\&context=storylines_faq}{It's
    unclear how often the virus is spread} via these tiny droplets, or
    aerosols, compared with larger droplets that are expelled when a
    sick person coughs or sneezes, or transmitted through contact with
    contaminated surfaces, said Linsey Marr, an aerosol expert at
    Virginia Tech. Aerosols are released even when a person without
    symptoms exhales, talks or sings, according to Dr. Marr and more
    than 200 other experts, who
    \href{https://academic.oup.com/cid/article/doi/10.1093/cid/ciaa939/5867798}{have
    outlined the evidence in an open letter to the World Health
    Organization}.
  \end{itemize}
\item ~
  \hypertarget{what-are-the-symptoms-of-coronavirus}{%
  \paragraph{What are the symptoms of
  coronavirus?}\label{what-are-the-symptoms-of-coronavirus}}

  \begin{itemize}
  \tightlist
  \item
    Common symptoms
    \href{https://www.nytimes3xbfgragh.onion/article/symptoms-coronavirus.html?action=click\&pgtype=Article\&state=default\&module=styln-coronavirus-national\&region=MAIN_CONTENT_3\&context=storylines_faq}{include
    fever, a dry cough, fatigue and difficulty breathing or shortness of
    breath.} Some of these symptoms overlap with those of the flu,
    making detection difficult, but runny noses and stuffy sinuses are
    less common.
    \href{https://www.nytimes3xbfgragh.onion/2020/04/27/health/coronavirus-symptoms-cdc.html?action=click\&pgtype=Article\&state=default\&module=styln-coronavirus-national\&region=MAIN_CONTENT_3\&context=storylines_faq}{The
    C.D.C. has also} added chills, muscle pain, sore throat, headache
    and a new loss of the sense of taste or smell as symptoms to look
    out for. Most people fall ill five to seven days after exposure, but
    symptoms may appear in as few as two days or as many as 14 days.
  \end{itemize}
\item ~
  \hypertarget{whats-the-best-material-for-a-mask}{%
  \paragraph{What's the best material for a
  mask?}\label{whats-the-best-material-for-a-mask}}

  \begin{itemize}
  \tightlist
  \item
    Scientists around the country
    \href{https://www.nytimes3xbfgragh.onion/article/coronavirus-homemade-mask-material-DIY-face-mask-ppe.html?action=click\&pgtype=Article\&state=default\&module=styln-coronavirus-national\&region=MAIN_CONTENT_3\&context=storylines_faq}{have
    tried to identify everyday materials that do a good job of filtering
    microscopic particles.} In recent tests, HEPA
    \href{https://www.nytimes3xbfgragh.onion/article/coronavirus-homemade-mask-material-DIY-face-mask-ppe.html?action=click\&action=click\&pgtype=Article\&state=default\&module=styln-coronavirus-national\&region=MAIN_CONTENT_3\&context=storylines_faq\&module=Spotlight\&pgtype=Homepage}{furnace
    filters} scored high, as did vacuum cleaner bags, fabric similar to
    flannel pajamas and those of 600-count pillowcases. Other materials
    tested included layered coffee filters and scarves and bandannas.
    These scored lower, but still captured a small percentage of
    particles.
  \end{itemize}
\item ~
  \hypertarget{does-asymptomatic-transmission-of-covid-19-happen}{%
  \paragraph{Does asymptomatic transmission of Covid-19
  happen?}\label{does-asymptomatic-transmission-of-covid-19-happen}}

  \begin{itemize}
  \tightlist
  \item
    So far, the evidence seems to show it does. A widely cited
    \href{https://www.nature.com/articles/s41591-020-0869-5}{paper}
    published in April suggests that people are most infectious about
    two days before the onset of coronavirus symptoms and estimated that
    44 percent of new infections were a result of transmission from
    people who were not yet showing symptoms. Recently, a top expert at
    the World Health Organization stated that transmission of the
    coronavirus by people who did not have symptoms was ``very rare,''
    \href{https://www.nytimes3xbfgragh.onion/2020/06/09/world/coronavirus-updates.html?action=click\&pgtype=Article\&state=default\&module=styln-coronavirus-national\&region=MAIN_CONTENT_3\&context=storylines_faq\#link-1f302e21}{but
    she later walked back that statement.}
  \end{itemize}
\end{itemize}

``We're going to get through it --- it's just a matter of how,'' Mr.
Tijerina said of the dual threat of the virus and the hurricane.

Hanna, in some ways, provided echoes of its destructive predecessor,
Hurricane Harvey.

Hanna was set to strike to the south of Corpus Christi; Harvey hit to
the northeast, near the bay town of Rockport. Hanna was expected to make
landfall on Saturday, exactly one month before the third anniversary of
Harvey, which first hit Rockport on Aug. 25, 2017.

Image

Storm clouds passed over a farm near Corpus Christi on
Friday.Credit...Tamir Kalifa for The New York Times

Harvey was one of the worst disasters in American history, causing \$125
billion in damage in Texas with winds at 130 m.p.h. and record-breaking
flooding. More than 100 people died
\href{https://www.nhc.noaa.gov/data/tcr/AL092017_Harvey.pdf}{directly
and indirectly} from Harvey, officials said. More than a quarter of a
million homes were damaged and another nearly 16,000 were destroyed.

Hanna was not expected to be nearly as devastating, but the unease grew
as the hurricane gathered strength on Saturday.

Corpus Christi residents who live on the coastline of scenic Whitecap
Beach kept a watchful eye on the swelling sea waters threatening their
condominiums. On Saturday afternoon, a woman on the beach fought the
full force of the wind to take a photograph of the rising tide, but
quickly turned around when the water rose to her waist. Other residents
nervously watched from a boardwalk, and waited.

``These are pretty impressive, aggressive waves,'' said Zack Smith, 36,
as he tried to stand still. ``Once they start reaching my feet, then
I'll leave. I don't want to get hit by one of those and get washed
out.''

The National Hurricane Center issued a hurricane warning for a swath of
Texas coast that spanned nearly 100 miles, from Port Mansfield to Port
Aransas. A storm surge warning reached even farther north, to about 75
miles south of Houston.

In addition to the wind, forecasters with the hurricane center said the
storm would bring up to a foot of rain over the weekend, with some
isolated areas getting up to 18 inches. The sudden downpour could lead
to ``life-threatening flash flooding,'' they said. Tornadoes could also
pose a threat on Saturday and overnight into Sunday over parts of the
coast.

``There could be damage to buildings and structures, especially if there
are any tornadoes,'' said Brian Field, a forecaster at the National
Weather Service office in Corpus Christi.

Hanna is the eighth named storm of the Atlantic hurricane season, which
runs from June 1 to Nov. 30. In the Pacific Ocean, a Category 3
hurricane continued to churn toward Hawaii, although forecasters said
that hurricane, named Douglas, was likely to weaken significantly as it
got closer to the islands.

Edgar Sandoval reported from Corpus Christi, Texas, Nicholas
Bogel-Burroughs from New York and Manny Fernandez from Houston. Derrick
Bryson Taylor contributed reporting from New York and Marie Fazio from
Jacksonville, Fla.

Advertisement

\protect\hyperlink{after-bottom}{Continue reading the main story}

\hypertarget{site-index}{%
\subsection{Site Index}\label{site-index}}

\hypertarget{site-information-navigation}{%
\subsection{Site Information
Navigation}\label{site-information-navigation}}

\begin{itemize}
\tightlist
\item
  \href{https://help.nytimes3xbfgragh.onion/hc/en-us/articles/115014792127-Copyright-notice}{©~2020~The
  New York Times Company}
\end{itemize}

\begin{itemize}
\tightlist
\item
  \href{https://www.nytco.com/}{NYTCo}
\item
  \href{https://help.nytimes3xbfgragh.onion/hc/en-us/articles/115015385887-Contact-Us}{Contact
  Us}
\item
  \href{https://www.nytco.com/careers/}{Work with us}
\item
  \href{https://nytmediakit.com/}{Advertise}
\item
  \href{http://www.tbrandstudio.com/}{T Brand Studio}
\item
  \href{https://www.nytimes3xbfgragh.onion/privacy/cookie-policy\#how-do-i-manage-trackers}{Your
  Ad Choices}
\item
  \href{https://www.nytimes3xbfgragh.onion/privacy}{Privacy}
\item
  \href{https://help.nytimes3xbfgragh.onion/hc/en-us/articles/115014893428-Terms-of-service}{Terms
  of Service}
\item
  \href{https://help.nytimes3xbfgragh.onion/hc/en-us/articles/115014893968-Terms-of-sale}{Terms
  of Sale}
\item
  \href{https://spiderbites.nytimes3xbfgragh.onion}{Site Map}
\item
  \href{https://help.nytimes3xbfgragh.onion/hc/en-us}{Help}
\item
  \href{https://www.nytimes3xbfgragh.onion/subscription?campaignId=37WXW}{Subscriptions}
\end{itemize}
