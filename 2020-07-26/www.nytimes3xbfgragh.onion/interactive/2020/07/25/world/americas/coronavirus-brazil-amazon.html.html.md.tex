Sections

SEARCH

\protect\hyperlink{site-content}{Skip to
content}\protect\hyperlink{site-index}{Skip to site index}

\hypertarget{comments}{%
\subsection{\texorpdfstring{\protect\hyperlink{commentsContainer}{Comments}}{Comments}}\label{comments}}

\href{}{The Coronavirus Unleashed Along the Amazon River}\href{}{Skip to
Comments}

The comments section is closed. To submit a letter to the editor for
publication, write to
\href{mailto:letters@NYTimes.com}{\nolinkurl{letters@NYTimes.com}}.

\hypertarget{the-coronavirus-unleashed-along-the-amazon-river}{%
\section{The Coronavirus Unleashed Along the Amazon
River}\label{the-coronavirus-unleashed-along-the-amazon-river}}

By \href{https://www.nytimes3xbfgragh.onion/by/tyler-hicks}{Tyler
Hicks},
\href{https://www.nytimes3xbfgragh.onion/by/julie-turkewitz}{Julie
Turkewitz}, Manuela Andreoni and
\href{https://www.nytimes3xbfgragh.onion/by/jeremy-white}{Jeremy
White}July 25, 2020

\begin{itemize}
\item
\item
\item
\item
\item
  \emph{114}
\end{itemize}

Brazil has been battered by the pandemic, with the second-highest death
toll in the world.

The Amazon has been hit particularly hard. Even in remote towns, people
have been as likely to get sick as in New York City.

A New York Times photographer, Tyler Hicks, traveled the river for
weeks, documenting how the virus spread.

\hypertarget{the-amazon-giver-of-life-unleashes-the-pandemic}{%
\section{The Amazon, Giver of Life, Unleashes the
Pandemic}\label{the-amazon-giver-of-life-unleashes-the-pandemic}}

Photographs by
\href{https://www.nytimes3xbfgragh.onion/by/tyler-hicks}{Tyler Hicks}

Written by
\href{https://www.nytimes3xbfgragh.onion/by/julie-turkewitz}{Julie
Turkewitz} and Manuela Andreoni

Graphics by
\href{https://www.nytimes3xbfgragh.onion/by/jeremy-white}{Jeremy White}

July 25, 2020

The virus swept through the region like past plagues that have traveled
the river with colonizers and corporations.

It spread with the dugout canoes carrying families from town to town,
the fishing dinghies with rattling engines, the ferries moving goods for
hundreds of miles, packed with passengers sleeping in hammocks, side by
side, for days at a time.

The Amazon River is South America's essential life source, a glittering
superhighway that cuts through the continent. It is the central artery
in a vast network of tributaries that sustains some 30 million people
across eight countries, moving supplies, people and industry deep into
forested regions often untouched by road.

But once again, in a painful echo of history, it is also bringing
disease.

\includegraphics{https://static01.graylady3jvrrxbe.onion/packages/flash/multimedia/ICONS/transparent.png}

\includegraphics{https://static01.graylady3jvrrxbe.onion/newsgraphics/2020/07/07/amazon-covid/assets/images/brazil002-2000.jpg}

Hammocks have become stretchers, carrying the sick from communities with
no doctors.

\includegraphics{https://static01.graylady3jvrrxbe.onion/packages/flash/multimedia/ICONS/transparent.png}

\includegraphics{https://static01.graylady3jvrrxbe.onion/newsgraphics/2020/07/07/amazon-covid/assets/images/brazil003-2000.jpg}

A family mourning their matriarch, Gertrude Ferreira dos Santos, who
spent her life along the river.

\includegraphics{https://static01.graylady3jvrrxbe.onion/packages/flash/multimedia/ICONS/transparent.png}

\includegraphics{https://static01.graylady3jvrrxbe.onion/newsgraphics/2020/07/07/amazon-covid/assets/images/brazil004-2000.jpg}

Boat ambulances travel for hours to reach a single patient.

As the pandemic assails Brazil, overwhelming it with more than two
million infections and more than 84,000 deaths --- second only to the
United States --- the virus is taking an exceptionally high toll on the
Amazon region and the people who have depended on its abundance for
generations.

In Brazil, the six cities with the highest coronavirus exposure
are\href{https://www.medrxiv.org/content/10.1101/2020.05.30.20117531v1}{}\href{https://www.medrxiv.org/content/10.1101/2020.05.30.20117531v1}{all
on the Amazon River,} according to an expansive new study from Brazilian
researchers that measured antibodies in the population.

The epidemic has spread so quickly and thoroughly along the river that
in remote fishing and farming communities like Tefé, people have been as
likely to get the virus as in New York City, home to one of the world's
worst outbreaks.

``It was all very fast,'' said Isabel Delgado, 34, whose father,
Felicindo, died of the virus shortly after falling ill in the small city
of Coari. He had been born on the river, raised his family by it and
built his life crafting furniture from the timber on its banks.

In the past four months, as the epidemic traveled from the biggest city
in the Brazilian Amazon, Manaus, with its high-rises and factories, to
tiny, seemingly isolated villages deep in the interior, the fragile
health care system has buckled under the onslaught.

Cities and towns along the river have some of the highest deaths per
capita in the country --- often several times the national average. In
Manaus, there were periods when every Covid ward was full and 100 people
were dying a day, pushing the city to cut new burial grounds out of
thick forest. Grave diggers lay rows of coffins in long trenches carved
in the freshly turned earth.

Down the river, hammocks have become stretchers, carrying the sick from
communities with no doctors to boat ambulances that careen through the
water. In remote reaches of the river basin, medevac planes land in tiny
airstrips sliced into the lush landscape only to find that their
patients died while waiting for help.

The virus is exacting
an\href{https://www.nytimes3xbfgragh.onion/2020/07/19/world/americas/coronavirus-brazil-indigenous.html?smid=tw-share}{}\href{https://www.nytimes3xbfgragh.onion/2020/07/19/world/americas/coronavirus-brazil-indigenous.html?smid=tw-share}{especially
high toll} on Indigenous people, a parallel to the past. Since the
1500s, waves of explorers have traveled the river, seeking gold, land
and converts --- and later, rubber, a resource that helped fuel the
Industrial Revolution, changing the world. But with them, these
outsiders brought violence and diseases like smallpox and measles,
killing millions and wiping out entire communities.

``This is a place that has generated so much wealth for others,'' said
Charles C. Mann, a journalist who has written extensively on the history
of the Americas, ``and look at what's happening to it.''

Indigenous people have been roughly six times as likely to be infected
with the coronavirus as white people, according to the Brazilian study,
and are dying in far-flung river villages untouched by electricity.

The Amazon is the central artery in a network of tributaries that
sustains 30 million people across eight countries, moving goods and
industry deep into the forest.

Only a few cases had been confirmed in the state of Amazonas when April
began.

But by July, the six Brazilian cities with the highest exposure rates to
the virus were all on the Amazon River.

The first case in Manaus came from someone who had traveled from
England.

The virus spread quickly along the river.

Many river cities have far more deaths per capita than the national
average.

The Times tracked the spread of the virus from the region's capital city
to small villages with no doctors.

Tefé, several days' journey by boat from Manaus, had one of the highest
death rates in the country.

Even in the best of times, the Amazon was among the most neglected parts
of the country, a place where the helping hand of the government can
feel distant, even nonexistent.

But the region's ability to confront the virus has been further weakened
under President Jair Bolsonaro, whose
public\href{https://www.nytimes3xbfgragh.onion/2020/06/08/world/americas/brazil-coronavirus-statistics.html}{}\href{https://www.nytimes3xbfgragh.onion/2020/06/08/world/americas/brazil-coronavirus-statistics.html}{dismissals
of the epidemic} have verged at times
on\href{https://www.nytimes3xbfgragh.onion/2020/05/01/world/americas/brazil-bolsonaro-coronavirus-crisis.html}{}\href{https://www.nytimes3xbfgragh.onion/2020/05/01/world/americas/brazil-bolsonaro-coronavirus-crisis.html}{mockery},
even though
he\href{https://www.nytimes3xbfgragh.onion/2020/07/07/world/americas/brazil-bolsonaro-coronavirus.html}{}\href{https://www.nytimes3xbfgragh.onion/2020/07/07/world/americas/brazil-bolsonaro-coronavirus.html}{tested
positive himself}.

The virus has surged on his
government's\href{https://www.nytimes3xbfgragh.onion/2020/06/10/world/americas/bolsonaro-coup-coronavirus-brazil.html}{}\href{https://www.nytimes3xbfgragh.onion/2020/06/10/world/americas/bolsonaro-coup-coronavirus-brazil.html}{disorganized
and lackluster} watch, tearing through the nation. From his first days
in office, Mr. Bolsonaro has made it clear that protecting the welfare
of Indigenous communities was not his priority, cutting their
funding,\href{https://www.nytimes3xbfgragh.onion/2020/04/19/world/americas/bolsonaro-brazil-amazon-indigenous.html}{}\href{https://www.nytimes3xbfgragh.onion/2020/04/19/world/americas/bolsonaro-brazil-amazon-indigenous.html}{whittling
away at their protections}
and\href{https://www.nytimes3xbfgragh.onion/2018/11/10/world/americas/brazil-indigenous-mining-bolsonaro.html}{}\href{https://www.nytimes3xbfgragh.onion/2018/11/10/world/americas/brazil-indigenous-mining-bolsonaro.html}{encouraging
illegal encroachments} into their territory.

To the outsider, the thickly forested region along the Amazon River
appears impenetrable, disconnected from the rest of the world.

But that isolation is deceptive, said Tatiana Schor, a Brazilian
geography professor who lives off one of the river's tributaries.

``There is no such thing as isolated communities in the Amazon,'' she
said, ``and the virus has shown that.''

The boats
that\href{https://www.nytimes3xbfgragh.onion/2016/11/19/world/americas/brazil-amazon-pirates-riverboats-rain-forest.html}{}\href{https://www.nytimes3xbfgragh.onion/2016/11/19/world/americas/brazil-amazon-pirates-riverboats-rain-forest.html}{nearly
everyone relies on}, sometimes crowded with more than 100 passengers for
many days, are behind the spread of the
virus,\href{https://journals.openedition.org/confins/30072}{}\href{https://journals.openedition.org/confins/30072}{researchers
say}. And even as local governments have officially limited travel,
people have continued to take to the water because almost everything ---
food, medicine, even the trip to the capital to pick up emergency aid
--- depends on the river.

\includegraphics{https://static01.graylady3jvrrxbe.onion/packages/flash/multimedia/ICONS/transparent.png}

\includegraphics{https://static01.graylady3jvrrxbe.onion/newsgraphics/2020/07/07/amazon-covid/assets/images/brazil006-2000.jpg}

Life along the Amazon is sometimes called an ``amphibious way of
being.''

\includegraphics{https://static01.graylady3jvrrxbe.onion/packages/flash/multimedia/ICONS/transparent.png}

\includegraphics{https://static01.graylady3jvrrxbe.onion/newsgraphics/2020/07/07/amazon-covid/assets/images/brazil007-2000.jpg}

Sick people traveled on the river to get tested for the virus.

\includegraphics{https://static01.graylady3jvrrxbe.onion/packages/flash/multimedia/ICONS/transparent.png}

\includegraphics{https://static01.graylady3jvrrxbe.onion/newsgraphics/2020/07/07/amazon-covid/assets/images/brazil008-2000.jpg}

A woman being tested last month in Manacapuru, where cases had exploded.

Scholars have long referred to life on the Amazon as an ``amphibious way
of being.''

The crisis in the Brazilian Amazon began in Manaus, a city of 2.2
million that has risen out of the forest in a jarring eruption of
concrete and glass, tapering at its edges to clusters of wooden homes
perched on stilts, high above the water.

Manaus, the capital of Amazonas state, is now an industrial powerhouse,
a major producer of motorcycles, with many foreign businesses. It is
intimately connected to the rest of the world --- its international
airport sees about 250,000 passengers a month --- and, through the
river, to much of the Amazon region.

Manaus's first documented case, confirmed on March 13, came from
England. The patient had mild symptoms and quarantined at home, in a
wealthier part of town, according to city health officials.

Soon, though, the virus seemed to be everywhere.

``We didn't have any more beds --- or even armchairs,'' Dr. Álvaro
Queiroz, 26, said of the days when his public hospital in Manaus was
completely full. ``People never stopped coming.''

Gertrude Ferreira Dos Santos lived on the city's eastern edge, in a
neighborhood pressed against the water. She used to say that her
favorite thing in the world was to travel the river by boat. With the
breeze on her face, she said, she felt free.

Then, in May, Ms. dos Santos, 54, fell ill. Days later, she called her
children to her bed, making them promise to stick together. She seemed
to know that she was about to die.

\includegraphics{https://static01.graylady3jvrrxbe.onion/packages/flash/multimedia/ICONS/transparent.png}

\includegraphics{https://static01.graylady3jvrrxbe.onion/newsgraphics/2020/07/07/amazon-covid/assets/images/brazil015-2000.jpg}

Funeral teams worked around the clock to collect bodies, including that
of Ms. dos Santos.

\includegraphics{https://static01.graylady3jvrrxbe.onion/packages/flash/multimedia/ICONS/transparent.png}

\includegraphics{https://static01.graylady3jvrrxbe.onion/newsgraphics/2020/07/07/amazon-covid/assets/images/brazil014-2000.jpg}

Many people with symptoms of the virus prefer to stay home, scared of
the hospital and of dying alone.

\includegraphics{https://static01.graylady3jvrrxbe.onion/packages/flash/multimedia/ICONS/transparent.png}

\includegraphics{https://static01.graylady3jvrrxbe.onion/newsgraphics/2020/07/07/amazon-covid/assets/images/brazil016-2000.jpg}

There were so many deaths in Manaus that the city cut new burial grounds
out of thick forest.

Eduany, 22, her youngest daughter, stayed with her that night. In early
morning, as Eduany got up to take a break, her sister Elen, 28, begged
her to come back.

Their mother had stopped breathing. The sisters, in desperation,
attempted mouth-to-mouth resuscitation. At 6 a.m., the sun rising above
the city, Ms. dos Santos died in their arms.

When men in white protective suits arrived later to carry away her body,
the sisters began to wail.

Ms. dos Santos had been a single mother. Life had not always been easy.
But she had maintained a sense of wonder, something her daughters
admired. ``In everything she did,'' Elen said, ``she was joyful.''

Her mother's death certificate listed many underlying conditions,
including longstanding breathing problems, according to the women. It
also listed respiratory failure, a key indicator that a person has died
of the coronavirus.

But her daughters didn't believe she was a victim of the pandemic. She
had certainly died of other causes, they said. God would not have given
her such an ugly disease.

Along the river, people said similar things over and over, reluctant to
admit to possible contagion, even as the health of their siblings and
parents declined. Many seemed to think their families would be shunned,
that a diagnosis would somehow tarnish an otherwise dignified life.

But as this stigma led people to play down symptoms of the virus out of
fear, doctors said, the pandemic was spreading quickly.

After Manaus, the virus traveled east and west, racing away from the
region's health care center.

\includegraphics{https://static01.graylady3jvrrxbe.onion/packages/flash/multimedia/ICONS/transparent.png}

\includegraphics{https://static01.graylady3jvrrxbe.onion/newsgraphics/2020/07/07/amazon-covid/assets/images/brazil026-2000.jpg}

Medical teams traveling to test people.

\includegraphics{https://static01.graylady3jvrrxbe.onion/packages/flash/multimedia/ICONS/transparent.png}

\includegraphics{https://static01.graylady3jvrrxbe.onion/newsgraphics/2020/07/07/amazon-covid/assets/images/brazil027-2000.jpg}

A community center was transformed into a walk-in clinic.

\includegraphics{https://static01.graylady3jvrrxbe.onion/packages/flash/multimedia/ICONS/transparent.png}

\includegraphics{https://static01.graylady3jvrrxbe.onion/newsgraphics/2020/07/07/amazon-covid/assets/images/brazil028-2000.jpg}

In places far from the capital, basic supplies, including disinfectant,
are shipped in.

In Manacapuru, more than an hour from the capital, Messias Nascimento
Farias, 40, carried his ailing wife to their car and sped down one of
the region's few country roads to meet the ambulance that could carry
her to a hospital.

His wife, Sandra Machado Dutra, 36, gasped in his truck.

``The Lord is my shepherd, I shall not want,'' he prayed over and over
until he handed her to health care workers. They were lucky. She
survived.

But for most people living along the river, hundreds of boat miles from
Manaus, the fastest way to a major hospital is by plane.

Even before the virus arrived, people in far-flung communities with a
life-threatening emergency could make a frantic call for an airplane
ambulance that would take them to a hospital in the capital.

But the small planes turned out to be dangerous for people with
Covid-19, sometimes causing blood oxygen levels to plummet as the
aircraft rose. Very few of the airlift patients seemed to be surviving,
doctors said.

Instead, physicians and nurses found themselves flying their patients to
painful deaths far from everything and everyone they had loved.

One morning in May, a white plane touched down at the airport in Coari,
about 230 miles from Manaus.

\includegraphics{https://static01.graylady3jvrrxbe.onion/packages/flash/multimedia/ICONS/transparent.png}

\includegraphics{https://static01.graylady3jvrrxbe.onion/newsgraphics/2020/07/07/amazon-covid/assets/images/brazil010-2000.jpg}

Sandra Machado Dutra passed out before being lifted into an ambulance.

\includegraphics{https://static01.graylady3jvrrxbe.onion/packages/flash/multimedia/ICONS/transparent.png}

\includegraphics{https://static01.graylady3jvrrxbe.onion/newsgraphics/2020/07/07/amazon-covid/assets/images/brazil011-2000.jpg}

The family of Felicindo Delgado, the furniture maker, waved goodbye as
he was loaded onto a plane in Coari.

\includegraphics{https://static01.graylady3jvrrxbe.onion/packages/flash/multimedia/ICONS/transparent.png}

\includegraphics{https://static01.graylady3jvrrxbe.onion/newsgraphics/2020/07/07/amazon-covid/assets/images/brazil012-2000.jpg}

The flight took a toll on Mr. Delgado.

On the tarmac on a stretcher was Mr. Delgado, 68, the furniture maker,
barefoot and barely breathing.

Dr. Daniel Sérgio Siqueira and a nurse, Walci Frank, exhausted after
weeks of constant work, loaded him into the small cabin. As the plane
rose, his oxygen levels began to dive.

Mr. Delgado's daughter Isabel turned to the doctor in a panic. ``My
father is very strong,'' she told him. ``He is going to make it.''

When the Delgados finally reached the hospital in Manaus, Isabel was
stunned by the scenes around her. Despairing relatives held up loved
ones who had crumpled under the burden of disease, hurrying them in for
treatment.

At the same time, patients who had managed to survive Covid-19 staggered
out, into the jubilant arms of family and friends.

``I was just there,'' she said, ``praying that God would save my
father.''

Mr. Delgado died a few days later. When Isabel found out, the doctor
started crying with her.

She had no doubt that the river her father loved had also brought him
the virus. Soon, she and five other family members fell ill, too.

\includegraphics{https://static01.graylady3jvrrxbe.onion/packages/flash/multimedia/ICONS/transparent.png}

\includegraphics{https://static01.graylady3jvrrxbe.onion/newsgraphics/2020/07/07/amazon-covid/assets/images/brazil030-2000.jpg}

A family in Manacapuru gathered to hear a doctor's assessment.

\includegraphics{https://static01.graylady3jvrrxbe.onion/packages/flash/multimedia/ICONS/transparent.png}

\includegraphics{https://static01.graylady3jvrrxbe.onion/newsgraphics/2020/07/07/amazon-covid/assets/images/brazil032-2000.jpg}

A doctor treating a frail patient at her home in Manacapuru.

\includegraphics{https://static01.graylady3jvrrxbe.onion/packages/flash/multimedia/ICONS/transparent.png}

\includegraphics{https://static01.graylady3jvrrxbe.onion/newsgraphics/2020/07/07/amazon-covid/assets/images/brazil029-2000.jpg}

Some people who became sick waited until they were very weak to go to
the hospital.

When the coronavirus
arrived\href{https://www.nytimes3xbfgragh.onion/2020/07/11/world/americas/coronavirus-latin-america-inequality.html}{}\href{https://www.nytimes3xbfgragh.onion/2020/07/11/world/americas/coronavirus-latin-america-inequality.html}{in
the Americas}, there was widespread fear that it would take a
devastating toll on Indigenous communities across the region.

In many places along the Amazon River, those fears appear to be coming
true.

At least 570 Indigenous people in Brazil have died of the disease since
March, according to an association that represents the country's
Indigenous people. The vast majority of those deaths were in places
connected to the river.

More than 18,000 Indigenous people have been infected. Community leaders
have reported entire villages confined to their hammocks, struggling to
rise even to feed their children.

In many instances, the very health workers sent to help
them\href{https://www.nytimes3xbfgragh.onion/2020/07/19/world/americas/coronavirus-brazil-indigenous.html}{}\href{https://www.nytimes3xbfgragh.onion/2020/07/19/world/americas/coronavirus-brazil-indigenous.html}{have
inadvertently spread the virus}.

In the riverside hamlet of São José da Fortaleza, Chief Iakonero
Apurinã's relatives sent word, one by one, that they couldn't eat,
that\href{https://www.nytimes3xbfgragh.onion/2020/06/28/health/coronavirus-delirium-hallucinations.html}{}\href{https://www.nytimes3xbfgragh.onion/2020/06/28/health/coronavirus-delirium-hallucinations.html}{they
heard voices}, that they were too sick to get up.

Soon, it seemed to the chief that everyone in her community was sick.

\includegraphics{https://static01.graylady3jvrrxbe.onion/packages/flash/multimedia/ICONS/transparent.png}

\includegraphics{https://static01.graylady3jvrrxbe.onion/newsgraphics/2020/07/07/amazon-covid/assets/images/brazil023-2000.jpg}

Apurinã families had survived generations of violence and forced labor.
The virus tested them anew.

\includegraphics{https://static01.graylady3jvrrxbe.onion/packages/flash/multimedia/ICONS/transparent.png}

\includegraphics{https://static01.graylady3jvrrxbe.onion/newsgraphics/2020/07/07/amazon-covid/assets/images/brazil022-2000.jpg}

The virus hit during the rainy season, swelling waterways.

\includegraphics{https://static01.graylady3jvrrxbe.onion/packages/flash/multimedia/ICONS/transparent.png}

\includegraphics{https://static01.graylady3jvrrxbe.onion/newsgraphics/2020/07/07/amazon-covid/assets/images/brazil024-2000.jpg}

Ferries continued to ply the region, with people sleeping side-by-side
for days at a time.

Chief Apurinã, 54, said her group of 35 Apurinã families had survived
generations of violence and forced labor. They had arrived in São José
da Fortaleza decades ago, believing that they would finally be safe.

It was the river, said the chief, that had sustained them, feeding,
washing and cleansing them spiritually.

Then the new disease came, and the chief was ferrying traditional teas
from home to home. Soon came her own cough and exhaustion. A test in
Coari confirmed that she had caught the virus.

Chief Apurinã didn't blame the river. She blamed the people who traveled
it.

``The river to us is purification,'' she said. ``It's the most beautiful
thing there is.''

Miraculously, she said in mid-July, not a single person among the 35
families had died.

In Tefé, a city of 60,000 people nearly 400 miles along the river from
Manaus, the virus had arrived with gale force.

At the small public hospital, where officials initially planned to
accommodate 12 patients, nearly 50 crowded the makeshift Covid-19 unit.
Dr. Laura Crivellari, 31, the hospital's only infectious disease expert,
took them in, doing what she could with two respirators, no intensive
care unit, many sick colleagues --- and no one to replace them.

At one of the worst moments, she was the only physician on duty for two
days, overseeing dozens of critically ill patients.

\includegraphics{https://static01.graylady3jvrrxbe.onion/packages/flash/multimedia/ICONS/transparent.png}

\includegraphics{https://static01.graylady3jvrrxbe.onion/newsgraphics/2020/07/07/amazon-covid/assets/images/brazil018-2000.jpg}

Patients quickly overwhelmed the Covid-19 ward of the hospital in Tefé.

\includegraphics{https://static01.graylady3jvrrxbe.onion/packages/flash/multimedia/ICONS/transparent.png}

\includegraphics{https://static01.graylady3jvrrxbe.onion/newsgraphics/2020/07/07/amazon-covid/assets/images/brazil019-2000.jpg}

The constant death in Tefé pushed one doctor to the breaking point.

\includegraphics{https://static01.graylady3jvrrxbe.onion/packages/flash/multimedia/ICONS/transparent.png}

\includegraphics{https://static01.graylady3jvrrxbe.onion/newsgraphics/2020/07/07/amazon-covid/assets/images/brazil020-2000.jpg}

Awaiting burial in Tefé.

The constant death pushed Dr. Crivellari to her breaking point. Some
days she barely stopped to eat or drink.

At home, she shared her anguish with her partner. She was thinking of
giving up medicine, she said. ``I can't carry on like this,'' she told
him.

The pandemic has
been\href{https://www.nytimes3xbfgragh.onion/2020/05/28/world/americas/virus-mexico-doctors.html}{}\href{https://www.nytimes3xbfgragh.onion/2020/05/28/world/americas/virus-mexico-doctors.html}{brutal
on medical workers} around the world, and it has been particularly
difficult for the doctors and nurses navigating the vast distances,
frequent communication cuts and deep supply scarcity along the Amazon.

Without proper training or equipment, many nurses and doctors along the
river have died. Others have infected their families.

Dr. Crivellari knew her city was vulnerable. It's a three-day boat ride
from Manaus to Tefé, with ferries often carrying 150 people at a time.

``Our fear was that an infected person would contaminate the whole
boat,'' she said, ``and that's what ended up happening.''

By early July, the daily deaths in Tefé were dropping, and Dr.
Crivellari began to celebrate the patients she had been able to save.
She no longer thinks of quitting medicine.

Tefé, as a whole, took a cautious collective breath.

The virus, at least for the moment, had moved to a new place on the
river.

\includegraphics{https://static01.graylady3jvrrxbe.onion/packages/flash/multimedia/ICONS/transparent.png}

\includegraphics{https://static01.graylady3jvrrxbe.onion/newsgraphics/2020/07/07/amazon-covid/assets/images/brazil034-2000.jpg}

Tending to the body of Gauldino da Silva. With so many dying at home,
untested, the virus's true toll in the region may never be known.

Related Stories

\href{https://www.nytimes3xbfgragh.onion/2020/07/19/world/americas/coronavirus-brazil-indigenous.html}{\includegraphics{https://static01.graylady3jvrrxbe.onion/images/2020/07/17/world/Brazil-Indigenous/merlin_174536157_e4698abf-20cb-4346-bce1-1da0bdd37af1-slide.jpg}}

\hypertarget{brazil-health-workers-may-have-spread-coronavirus-to-indigenous-people}{%
\subsubsection{Brazil Health Workers May Have Spread Coronavirus to
Indigenous
People}\label{brazil-health-workers-may-have-spread-coronavirus-to-indigenous-people}}

Many people assigned to care for the Indigenous have been infected,
exposing remote communities to the virus. Health workers say they have
been plagued by insufficient testing and protective gear.

\href{https://www.nytimes3xbfgragh.onion/2020/07/07/world/americas/brazil-bolsonaro-coronavirus.html}{\includegraphics{https://static01.graylady3jvrrxbe.onion/images/2020/07/07/world/07bolsonaro-sub/merlin_173631339_3e5fd5ae-7098-477c-b454-03eee6d780c4-slide.jpg}}

\hypertarget{president-bolsonaro-of-brazil-tests-positive-for-coronavirus}{%
\subsubsection{President Bolsonaro of Brazil Tests Positive for
Coronavirus}\label{president-bolsonaro-of-brazil-tests-positive-for-coronavirus}}

After months of denying the seriousness of the pandemic and brushing
aside protective measures, Mr. Bolsonaro felt symptoms of Covid-19. More
than 65,000 Brazilians have died of the virus.

\href{https://www.nytimes3xbfgragh.onion/2020/06/13/world/americas/virus-brazil-bolsonaro-chloroquine.html}{\includegraphics{https://static01.graylady3jvrrxbe.onion/images/2020/06/13/world/13brazil-hydroxychloroquine/13brazil-hydroxychloroquine-slide.jpg}}

\hypertarget{brazil-president-embraces-unproven-cure-as-pandemic-surges}{%
\subsubsection{Brazil President Embraces Unproven `Cure' as Pandemic
Surges}\label{brazil-president-embraces-unproven-cure-as-pandemic-surges}}

President Jair Bolsonaro hailed hydroxychloroquine as a godsend while he
railed against quarantine measures and other best practices, undermining
the country's coronavirus response.

Sources: Cases data from Ministério da Saúde (Ministry of Health) of
Brazil. Death rates from Brasil.io.

João Castellano and Letícia Casado contributed reporting. Alain
Delaquérière contributed research.

Read 114 Comments

\begin{itemize}
\item
\item
\item
\item
\end{itemize}

Advertisement

\protect\hyperlink{after-bottom}{Continue reading the main story}

\hypertarget{site-index}{%
\subsection{Site Index}\label{site-index}}

\hypertarget{site-information-navigation}{%
\subsection{Site Information
Navigation}\label{site-information-navigation}}

\begin{itemize}
\tightlist
\item
  \href{https://help.nytimes3xbfgragh.onion/hc/en-us/articles/115014792127-Copyright-notice}{©~2020~The
  New York Times Company}
\end{itemize}

\begin{itemize}
\tightlist
\item
  \href{https://www.nytco.com/}{NYTCo}
\item
  \href{https://help.nytimes3xbfgragh.onion/hc/en-us/articles/115015385887-Contact-Us}{Contact
  Us}
\item
  \href{https://www.nytco.com/careers/}{Work with us}
\item
  \href{https://nytmediakit.com/}{Advertise}
\item
  \href{http://www.tbrandstudio.com/}{T Brand Studio}
\item
  \href{https://www.nytimes3xbfgragh.onion/privacy/cookie-policy\#how-do-i-manage-trackers}{Your
  Ad Choices}
\item
  \href{https://www.nytimes3xbfgragh.onion/privacy}{Privacy}
\item
  \href{https://help.nytimes3xbfgragh.onion/hc/en-us/articles/115014893428-Terms-of-service}{Terms
  of Service}
\item
  \href{https://help.nytimes3xbfgragh.onion/hc/en-us/articles/115014893968-Terms-of-sale}{Terms
  of Sale}
\item
  \href{https://spiderbites.nytimes3xbfgragh.onion}{Site Map}
\item
  \href{https://help.nytimes3xbfgragh.onion/hc/en-us}{Help}
\item
  \href{https://www.nytimes3xbfgragh.onion/subscription?campaignId=37WXW}{Subscriptions}
\end{itemize}
