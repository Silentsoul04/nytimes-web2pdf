Sections

SEARCH

\protect\hyperlink{site-content}{Skip to
content}\protect\hyperlink{site-index}{Skip to site index}

\href{https://myaccount.nytimes.com/auth/login?response_type=cookie\&client_id=vi}{}

\href{https://www.nytimes.com/section/todayspaper}{Today's Paper}

\href{/section/opinion}{Opinion}\textbar{}How Do You Know a Human Wrote
This?

\url{https://nyti.ms/39CC1QR}

\begin{itemize}
\item
\item
\item
\item
\item
\item
\end{itemize}

Advertisement

\protect\hyperlink{after-top}{Continue reading the main story}

\href{/section/opinion}{Opinion}

Supported by

\protect\hyperlink{after-sponsor}{Continue reading the main story}

\hypertarget{how-do-you-know-a-human-wrote-this}{%
\section{How Do You Know a Human Wrote
This?}\label{how-do-you-know-a-human-wrote-this}}

Machines are gaining the ability to write,~and they are getting
terrifyingly good at it.

\href{https://www.nytimes.com/by/farhad-manjoo}{\includegraphics{https://static01.nyt.com/images/2019/01/08/opinion/farhad-manjoo-opinion/farhad-manjoo-opinion-thumbLarge.png}}

By \href{https://www.nytimes.com/by/farhad-manjoo}{Farhad Manjoo}

Opinion Columnist

\begin{itemize}
\item
  July 29, 2020, 5:00 a.m. ET
\item
  \begin{itemize}
  \item
  \item
  \item
  \item
  \item
  \item
  \end{itemize}
\end{itemize}

\includegraphics{https://static01.nyt.com/images/2020/07/29/opinion/29manjooWeb/29manjooWeb-articleLarge.jpg?quality=75\&auto=webp\&disable=upscale}

I've never really worried that computers might be gunning for my job. To
tell the truth, often, I pray for it. How much better would my life be
--- how much better would \emph{my editor's} life be, to say nothing of
the poor readers --- if I could ask an all-knowing machine to suggest
the best way to start this column? It would surely beat my usual writing
process, which involves clawing at my brain with a rusty pickax in the
dim hope that a few flakes of wisdom and insight might, like dandruff,
settle on the page.

See what I mean? A computer might have helped there. (Like
\emph{dandruff}? That's what you're going with, Farhad?) But we writers
can be a cocky bunch. Writing is something of an inexplicable trick, and
it feels, like telling a joke or making a soufflé, like an inviolably
human endeavor.

I've never really worried that a computer might take my job because it's
never seemed remotely possible. Not infrequently, my phone thinks I
meant to write the word ``ducking.'' A computer writing a newspaper
column? That'll be the day.

Well, writer friends, the day is nigh. This month, OpenAI, an
artificial-intelligence research lab based in San Francisco, began
allowing limited access to a piece of software that is at once amazing,
spooky, humbling and more than a little terrifying.

OpenAI's new software, called
\href{https://openai.com/blog/openai-api/}{GPT-3}, is by far the
\href{https://towardsdatascience.com/gpt-3-the-new-mighty-language-model-from-openai-a74ff35346fc}{most
powerful ``language model'' ever created}. A language model is an
artificial intelligence system that has been trained on an enormous
corpus of text; with enough text and enough processing, the machine
begins to
\href{https://www.analyticsvidhya.com/blog/2019/08/comprehensive-guide-language-model-nlp-python-code/}{learn
probabilistic connections between words}. More plainly: GPT-3 can read
and write. And not badly, either.

Software like GPT-3 could be enormously useful. Machines that can
understand and respond to humans in our own language could create more
helpful digital assistants, more realistic video game characters, or
virtual teachers personalized to every student's learning style. Instead
of writing code, one day you might create software just by telling
machines what to do.

OpenAI has given just a few hundred software developers access to GPT-3,
and many have been filling Twitter over the last few weeks with
demonstrations of its surprising capabilities, which range from the
mundane to the sublime to the possibly quite dangerous.

To appreciate the potential danger, it helps to understand how GPT-3
works. Language models often need to be trained for specific uses ---~a
customer-service bot used by a retailer might need to be fine-tuned with
data about products, while a bot used by an airline would need to learn
about flights. But GPT-3 doesn't need much extra training. Give GPT-3 a
natural-language prompt --- ``I hereby resign from Dunder-Mifflin'' or
``Dear John, I'm leaving you'' --- and the software will fill in the
rest with text that is eerily close to what a human would produce.

These aren't canned responses. GPT-3 is capable of generating entirely
original, coherent and sometimes even factual prose. And not just prose
--- it can write \href{https://www.gwern.net/GPT-3\#poetry}{poetry},
dialogue,
\href{https://aiweirdness.com/post/624445709546029056/choose-your-quarantine-house}{memes},
computer code and who knows what else.

GPT-3's flexibility is a key advance. Matt Shumer, the chief executive
of a company called OthersideAI, is using GPT-3 to build a service that
responds to email on your behalf --- you write the gist of what you'd
like to say, and the computer creates a full, nuanced, polite email out
of your bullet points.

Another company, Latitude, is using GPT-3 to build realistic,
interactive characters in
\href{https://play.aidungeon.io}{text-adventure games}. It works
surprisingly well --- the software is not only coherent but also can be
quite inventive, absurd and even funny.

Stew Fortier, a writer, created a zany satire using the software as a
kind of muse.

Fortier fed GPT-3 a strange prompt: ``Below is a transcript from an
interview where Barack Obama explained why he was banned from Golden
Corral for life.'' The system then filled in the rest of the interview,
running with the concept that Obama had been banned from an
all-you-can-eat buffet.

\begin{quote}
\textbf{Obama}: Yes. It's true. I am no longer allowed in Golden Corral.

\textbf{Interviewer}: Is this because of your extensive shrimp-n-crab
legs policy?

\textbf{Obama}: Absolutely.

\textbf{Interviewer}: What is your extensive shrimp-n-crab legs policy?

\textbf{Obama}: Oh, well, in brief, they were offering an
all-you-can-eat shrimp-n-crab leg buffet, and I did not hesitate. After
I ate so much shrimp and crab that my stomach hurt, I would quietly
sneak in and throw more shrimp and crab onto my plate. I did this over
and over again until I had cleaned out the buffet and was full of
shrimp-n-crab.
\end{quote}

Yet software like GPT-3 raises the prospect of frightening misuse. If
computers can produce large amounts of humanlike text, how will we ever
be able to tell humans and machines apart? In a research paper detailing
GPT-3's power, its creators cite a litany of dangers, including
``misinformation, spam, phishing, abuse of legal and governmental
processes, fraudulent academic essay writing and social engineering
pretexting.''

There are other problems. Because it was trained on text found online,
it's likely that GPT-3 mirrors many biases found in society. How can we
make sure the text it produces is not racist or sexist? GPT-3 also isn't
good at telling fact from fiction.
``\href{https://aiweirdness.com/post/623543644426829825/facts-about-whales}{I
gave it my own original three sentences about whales}, and it added
original text --- and the way I could tell it was original was that it
was pretty much dead wrong,'' Janelle Shane, who runs a blog called AI
Weirdness, told me.

To its credit, OpenAI has put in place many precautions. For now, the
company is letting only a small number of people use the system, and it
is vetting each application produced with it. The company also prohibits
GPT-3 from impersonating humans --- that is, all text produced by the
software must disclose that it was written by a bot. OpenAI has also
invited outside researchers to study the system's biases, in the hope of
mitigating them.

These precautions may be enough for now. But GPT-3 is so good at aping
human writing that it sometimes gave me chills. Not too long from now,
your humble correspondent might be put out to pasture by a machine ---
and you might even miss me when I'm gone.

\hypertarget{office-hours-with-farhad-manjoo}{%
\subsection{Office Hours With Farhad
Manjoo}\label{office-hours-with-farhad-manjoo}}

\emph{Farhad wants to}
\href{https://www.nytimes.com/2019/05/16/opinion/farhad-office-hours.html?module=inline}{\emph{chat
with readers on the phone}}\emph{. If you're interested in talking to a
New York Times columnist about anything that's on your mind, please fill
out this form. Farhad will select a few readers to call.}

\emph{The Times is committed to publishing}
\href{https://www.nytimes.com/2019/01/31/opinion/letters/letters-to-editor-new-york-times-women.html}{\emph{a
diversity of letters}} \emph{to the editor. We'd like to hear what you
think about this or any of our articles. Here are some}
\href{https://help.nytimes.com/hc/en-us/articles/115014925288-How-to-submit-a-letter-to-the-editor}{\emph{tips}}\emph{.
And here's our email:}
\href{mailto:letters@nytimes.com}{\emph{letters@nytimes.com}}\emph{.}

\emph{Follow The New York Times Opinion section on}
\href{https://www.facebook.com/nytopinion}{\emph{Facebook}}\emph{,}
\href{http://twitter.com/NYTOpinion}{\emph{Twitter (@NYTopinion)}}
\emph{and}
\href{https://www.instagram.com/nytopinion/}{\emph{Instagram}}\emph{.}

Advertisement

\protect\hyperlink{after-bottom}{Continue reading the main story}

\hypertarget{site-index}{%
\subsection{Site Index}\label{site-index}}

\hypertarget{site-information-navigation}{%
\subsection{Site Information
Navigation}\label{site-information-navigation}}

\begin{itemize}
\tightlist
\item
  \href{https://help.nytimes.com/hc/en-us/articles/115014792127-Copyright-notice}{©~2020~The
  New York Times Company}
\end{itemize}

\begin{itemize}
\tightlist
\item
  \href{https://www.nytco.com/}{NYTCo}
\item
  \href{https://help.nytimes.com/hc/en-us/articles/115015385887-Contact-Us}{Contact
  Us}
\item
  \href{https://www.nytco.com/careers/}{Work with us}
\item
  \href{https://nytmediakit.com/}{Advertise}
\item
  \href{http://www.tbrandstudio.com/}{T Brand Studio}
\item
  \href{https://www.nytimes.com/privacy/cookie-policy\#how-do-i-manage-trackers}{Your
  Ad Choices}
\item
  \href{https://www.nytimes.com/privacy}{Privacy}
\item
  \href{https://help.nytimes.com/hc/en-us/articles/115014893428-Terms-of-service}{Terms
  of Service}
\item
  \href{https://help.nytimes.com/hc/en-us/articles/115014893968-Terms-of-sale}{Terms
  of Sale}
\item
  \href{https://spiderbites.nytimes.com}{Site Map}
\item
  \href{https://help.nytimes.com/hc/en-us}{Help}
\item
  \href{https://www.nytimes.com/subscription?campaignId=37WXW}{Subscriptions}
\end{itemize}
