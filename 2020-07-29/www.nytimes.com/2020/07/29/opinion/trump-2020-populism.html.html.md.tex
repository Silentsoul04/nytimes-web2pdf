Sections

SEARCH

\protect\hyperlink{site-content}{Skip to
content}\protect\hyperlink{site-index}{Skip to site index}

\href{https://myaccount.nytimes.com/auth/login?response_type=cookie\&client_id=vi}{}

\href{https://www.nytimes.com/section/todayspaper}{Today's Paper}

\href{/section/opinion}{Opinion}\textbar{}Trump Is Trying to Bend
Reality to His Will

\url{https://nyti.ms/3gd9yUq}

\begin{itemize}
\item
\item
\item
\item
\item
\item
\end{itemize}

Advertisement

\protect\hyperlink{after-top}{Continue reading the main story}

\href{/section/opinion}{Opinion}

Supported by

\protect\hyperlink{after-sponsor}{Continue reading the main story}

\hypertarget{trump-is-trying-to-bend-reality-to-his-will}{%
\section{Trump Is Trying to Bend Reality to His
Will}\label{trump-is-trying-to-bend-reality-to-his-will}}

Can his aggressive version of ethnonationalist populism prevail in 2020?
The answer is not obvious.

\href{https://www.nytimes.com/by/thomas-b-edsall}{\includegraphics{https://static01.nyt.com/images/2018/04/02/opinion/thomas-b-edsall/thomas-b-edsall-thumbLarge-v2.png}}

By \href{https://www.nytimes.com/by/thomas-b-edsall}{Thomas B. Edsall}

Mr. Edsall contributes a weekly column from Washington, D.C. on
politics, demographics and inequality.

\begin{itemize}
\item
  July 29, 2020
\item
  \begin{itemize}
  \item
  \item
  \item
  \item
  \item
  \item
  \end{itemize}
\end{itemize}

\includegraphics{https://static01.nyt.com/images/2020/07/29/opinion/29edsall1a/29edsall1a-articleLarge.jpg?quality=75\&auto=webp\&disable=upscale}

Disruption, disorder and disease are gripping the United States as the
2020 election draws near, leading to an unusual degree of
unpredictability about our political future. Despite current state and
national \href{https://projects.fivethirtyeight.com/polls/}{polling that
favors} Democrats, we still can't say for sure whether the nation will
tip left or right.

``Modern democracies are currently experiencing destabilizing events,''
three Danish political scientists,
\href{https://pure.au.dk/portal/en/persons/michael-bang-petersen(7998cc16-75d5-4065-8b6e-395d73e22151).html}{Michael
Bang Petersen},
\href{https://pure.au.dk/portal/en/persons/mathias-osmundsen(a453964f-daa7-4f40-94d0-fde773a485d4).html}{Mathias
Osmundsen} and
\href{https://pure.au.dk/portal/en/persons/alexander-bor(df35f529-baf4-4bbf-916e-9dac13baf052).html}{Alexander
Bor}, write, ``including the emergence of demagogic leaders, the onset
of street riots, circulation of misinformation and extremely hostile
political engagements on social media.''

Driving this destabilization, according their new paper,
``\href{https://psyarxiv.com/puqzs}{Beyond Populism},'' is the feeling
millions of voters continue to have of being left behind, of ```losing
out' in a world marked by, on the one hand, traditional gender-and
race-based hierarchies, which limits the mobility of minority groups,
and, on the other hand, globalized competition, which puts a premium on
human capital'' --- especially on ``learning capacity,'' roughly
measured by the presence or absence of a college degree.

The crucial role of human capital is illustrated in a 2011 study
published in the American Economic Review, ``Sources of Lifetime
Inequality,'' by
\href{https://sites.google.com/georgetown.edu/mark-huggett/home}{Mark
Huggett}, \href{http://www.gustavoventura.com/index.html}{Gustavo
Ventura} and \href{https://fnce.wharton.upenn.edu/profile/yarona/}{Amir
Yaron}, economists at Georgetown, Arizona State and the University of
Pennsylvania.

The authors found that human capital, including learning skills,
accounted for ``61.2, 62.4, and 66.0 percent of the variation in
lifetime earnings, lifetime wealth, and
\href{https://www.economicshelp.org/blog/26552/concepts/measuring-utility/}{lifetime
utility}'' --- a measure of life satisfaction.

Petersen and his colleagues found that those experiencing rising levels
of frustration are motivated to turn to the relative extremes of the
political spectrum reflecting ``discontent with one's own personal
standing.''

This phenomenon, they continue, is concentrated

\begin{quote}
among individuals for whom prestige-based pathways to status are, at
least in their own perception, unlikely to be successful. Despite their
political differences, this perception may be the psychological
commonality of, on the one hand, race- or gender-based grievance
movements and, on the other hand, white lower-middle class right-wing
voters.
\end{quote}

The traditional avenue to standing in society --- ``tangible benefits
including income and job access'' and
``\href{https://books.google.com/books?id=GL80BQR6pk0C\&q=taggible\#v=snippet\&q=intangible\&f=false}{intangible
benefits} including cultural hegemony, prestige, authority and social
space'' --- requires the ``human capital'' I mentioned above, which what
Petersen described in an email as ``the stock of skills and competencies
that allow people to produce economic value'' that ``involve the
cultivation of talents and skills that are valuable for others and,
hence, based on a reciprocal relationship wherein status is granted in
exchange for service.''

When inequality increases, the issue of status becomes sharper, and
``people will simultaneously feel that (a) it is important to get status
and (b) that it is very difficult to do so.'' In such a situation, at
the extremes, ``some people will feel that the use of fear and
intimidation is an attractive shortcut to getting recognition,''
Petersen wrote by email.

``It would be wrong to exclusively think of this as a right-wing
phenomenon. People on the extremes of both the left-wing and the
right-wing are likely to be high in dominance motivations,'' Petersen
continued, adding that

\begin{quote}
we should expect dominance to be a key motivational factor among people
supporting or advocating the use of violence for political purposes.
While such supporters may appeal to a number of higher-order ideological
principles, a personal craving for status seems to be a key motivational
factor according to our research.
\end{quote}

The difficulty of rising up the economic ladder is reflected in the
decline in mobility in the United States. Research by
\href{http://www.rajchetty.com/}{Raj Chetty} and colleagues has
demonstrated that the percentage of children who make more than their
parents \href{https://www.nber.org/papers/w22910}{has fallen} from just
over 90 percent for those born in 1940 to 50 percent for those born in
1984. The declines have been sharpest in the South and Midwest, as shown
in the accompanying map --- in many of the areas that provided key
support to Donald Trump in 2016.

The frustration over the lack of mobility is particularly acute for
those without college degrees.

\hypertarget{upward-mobility-in-the-united-states}{%
\subsection{Upward Mobility in the United
States}\label{upward-mobility-in-the-united-states}}

One measure of a society's success is the ability of its low-income
children to climb the economic ladder. This tends to happen less in
Southern states like Georgia, South Carolina and Alabama. This map
measures where children who grew up with low-income parents fall in the
income distribution as adults, on average.

Avg. child percentile rank for parents at 25th percentile

More upward mobility

26

37

39

40

42

43

45

46

48

52+

Not enough data

Higher upward mobility

Lower

upward

mobility

Average child percentile rank for parents at 25th percentile

More upward mobility

26

37

39

40

42

43

45

46

48

52+

Not enough data

Higher upward mobility

Lower

upward

mobility

Source: Opportunity Insights \textbar{} By The New York Times

In a 2019 paper,
``\href{https://papers.ssrn.com/sol3/papers.cfm?abstract_id=3421153}{The
College Wealth Divide: Education and Inequality in America,
1956-2016},'' three German economists,
\href{https://www.bgse.uni-bonn.de/en/people/student-directory/2017/alina-bartscher}{Alina
Bartscher},
\href{https://www.bgse.uni-bonn.de/en/people/faculty-directory/moritz-kuhn}{Moritz
Kuhn} and Moritz Schularick, all of the University of Bonn, determined
that in the United States since the since the 1970s ``the real income of
non-college households stagnated, while the real income of college
households has risen by around 50 percent.'' The income data is,
however, dwarfed by the findings on wealth:

\begin{quote}
While non-college households were treading water in terms of wealth,
college households have increased their net worth by a factor of three
compared to 1971.
\end{quote}

The case made by Petersen and his collaborators that more Americans are
becoming marginalized gets strong support from
\href{https://en.politics.huji.ac.il/people/noam-gidron}{Noam Gidron}
and \href{https://scholar.harvard.edu/hall/home}{Peter A. Hall},
political scientists at Hebrew University and Harvard, in their paper,
``\href{https://scholar.harvard.edu/files/hall/files/gidronhallmay2018.pdf}{Populism
as a Problem of Social Integration}.'' They write:

\begin{quote}
Our key contention is that populist politics reflects problems of social
integration. That is to say, support for radical parties is likely to be
especially high among people who feel they have been socially
marginalized, i.e. deprived of the roles and respect normally accorded
members of mainstream society. From this perspective, the sources of
social marginalization may lie in economic or cultural developments and
in how they combine.
\end{quote}

Large segments of the population, they continue, have been

\begin{quote}
``left behind'' --- relegated to vulnerable economic and social
positions, increasingly alienated from the values prominent in elite
discourse, and sensing that they are no longer recognized as valued
members of society.
\end{quote}

As a result, the authors argue, ``subjective social status'' --- that
is, ``people's own beliefs about where they stand relative to others
within this status hierarchy'' --- has become a crucial factor in
shaping political commitments:

\begin{quote}
There is a consistent association between levels of subjective social
status and voting for parties of the populist right and radical left.
The more socially marginalized people feel, the more likely they are to
gravitate toward the fringes of the political spectrum.
\end{quote}

Voters who feel a loss of standing, who experience themselves as
marginalized, often turn left or right --- whites in this category may
turn to the right; African-American, Latino and other minority voters
can find that the left has more to offer.

``Changes in cultural frameworks,'' the authors write, are

\begin{quote}
leading people who hold traditional social attitudes to feel socially
marginalized as a result of incongruence between their values and the
discourse of mainstream elites. The growing prominence of cultural
frameworks promoting gender equality, multiculturalism, secular values
and LGBTQ rights is the most notable of such changes.
\end{quote}

They go on:

\begin{quote}
Steps toward inclusion are double-sided: they can lead people who hold
more traditional values to feel marginalized vis-à-vis the main-currents
of society.
\end{quote}

Gidron and Hall observe that extensive research has shown

\begin{quote}
that support for the far right is often strongest, not among people
suffering the greatest economic distress, but among people who are
somewhat better-off if still facing economic difficulties.
\end{quote}

This constituency of voters ``a few rungs up the socioeconomic ladder''
is, in turn,

\begin{quote}
susceptible to ``\href{https://www.nber.org/papers/w17234}{last place
aversion},'' namely, a fear of falling even farther down it; and they
often erect social boundaries separating `respectable' people like
themselves from others seen as lower down on that social ladder. Thus,
the anti-immigrant and anti-ethnic appeals of populist right parties may
be especially attractive to them, because they emphasize such
boundaries.
\end{quote}

Those drawn to the progressive left, in Gidron and Hall's view, are
``voters with status concerns but universalistic values that incline
them against ethnonationalist appeals.'' These voters ``are more often
found among sociocultural professionals and people with higher levels of
education'' and, because they have universalistic values, they are
likely to support ``left parties'' that ``promise redistribution.''

Another major difference between voters who back left or right parties
is the kind of work they do. According to Gidron and Hall,

\begin{quote}
There are stark differences in the occupational bases of support for the
radical left and right. Left parties seem to have particular appeal for
people in professional occupations who may nonetheless feel that they
are not receiving the social respect they deserve. By contrast, radical
right parties appeal most strongly to people in low-status occupations:
manual workers and low-skill service employees.
\end{quote}

What about socially marginalized voters who are conflicted --- holding,
for example, conservative values on cultural, moral and racial issues,
but more liberal and pro-redistribution economic views? In these
circumstances, the right has the advantage.

In his 2016 dissertation at Harvard,
``\href{https://dash.harvard.edu/bitstream/handle/1/33493265/GIDRON-DISSERTATION-2016.pdf?isAllowed=y\&sequence=4}{Many
Ways to be Right: The Unbundling of European Mass Attitudes and Partisan
Asymmetries across the Ideological Divide},'' Gidron reported that:

\begin{quote}
Cross-pressured voters (those who are conservative on some issues but
progressive on other issues) are more likely to support the right; while
support for the left requires progressive attitudes on all issues, it is
enough to be conservative on one issue to support the right.
\end{quote}

Gidron continued:

\begin{quote}
Put differently, the right is likely to attract all those who are
conservative on some issues --- and not only those who are conservative
on all issues. Those who oppose state intervention in the economy, those
who oppose progressive reforms on cultural questions such as gender
norms, and those who oppose greater openness toward immigration should
all gravitate toward the right, regardless of whether they are also
progressive on some other issues.
\end{quote}

How many voters can be described as cross-pressured by conservative
cultural views and liberal economic views?

A
\href{https://www.voterstudygroup.org/publication/political-divisions-in-2016-and-beyond}{Voter
Study Group analysis} of the 2016 election by
\href{https://www.newamerica.org/our-people/lee-drutman/}{Lee Drutman}
found that just under 30 percent of voters feel this way. In addition,
Drutman's study provided support for Gidron's view that these culturally
conservative and economically liberal voters lean decisively to the
right. Among the 24.3 percent of voters who fit this category and voted
for either Hillary Clinton or Donald Trump, 75.2 percent cast ballots
for Trump and 24.8 percent for Clinton, a 3 to 1 split.

Reinforcing the work of Petersen, Gidron and their colleagues are the
findings of four political scientists,
\href{https://www.yu.edu/faculty/pages/malka-ariel}{Ariel Malka},
\href{https://www.ylelkes.com/}{Yphtach Lelkes},
\href{http://www.bertbakker.com/wp-content/uploads/2012/04/Bakker_CV_Aug2015.pdf}{Bert
N. Bakker} in
\href{https://malkaresearch.files.wordpress.com/2020/06/malka-lelkes-bakker-spivack-inpress-pop-2020-withfigures.pdf}{a
paper} published in May.

The four argue that the focus on Democratic and Republican
identification masks another key divide between voters whose prime
concern is protection from adverse cultural and economic forces and
voters whose agenda is personal autonomy and economic freedom. They call
these two constituencies the ``protection-based'' and the ``freedom
based.''

The authors describe those with ``a `protection-based' attitude
package'' as voters who combine ``cultural conservatism with left
economic attitudes.'' These voters prioritize ``social order and
economic stability, which, in the minds of citizens, may be satisfied by
leadership and policy action that are unconstrained by democratic
rules.''

Malka and his co-authors elaborate:

\begin{quote}
Citizens who combine a culturally conservative worldview with an
economically redistributive and interventionist set of preferences often
place high priority on security, certainty, and stability. These
citizens seem to apply a mind-set to the political domain that attracts
them to policies that maintain cultural tradition and uniformity (social
conservatism) and that also entail top-down provision of material
security (left-wing economic views). This type of worldview has been
referred to as a `protection-based' attitude package, because it
involves strong government intervention to provide protection against
cultural and economic sources of insecurity.
\end{quote}

The protection-based constituency is, thus, made up of those who feel
under assault by liberal cultural trends and their challenges to
traditional morality and by economic forces that are shifting rewards to
those with higher levels of education and
``\href{http://www.act.org/content/act/en/research/reports/act-publications/beyond-academics/core-academic-skills/core-academic-skills-framework.html}{learning
skills}.''

On the other side is ``a freedom-based'' attitude package that
``combines left-wing cultural with right-wing economic views, reflecting
acceptance of cultural and economic risk rooted in the value of
freedom.'' Malka and his colleagues suggest that ``Anglophone
democracies are more likely to associate free market economics with
political freedom and democratic liberalism'' and note that

\begin{quote}
citizens of English-speaking countries tend to score higher than
citizens of other countries in self-report measures of individualism,
which tap a focus on personal as opposed to collective goals, individual
autonomy, self-differentiation, and competition.
\end{quote}

In addition, the authors suggest that in these democracies,

\begin{quote}
consistent support for procedural democratic rules is linked with a
classically liberal mind-set focused on individual autonomy to pursue
economic interests and cultural preferences without government
interference.
\end{quote}

Four years ago, Trump won by decisively carrying what Malka and his
colleagues call the protection-based constituency --- voters whose
privileged status as white Americans Trump has promised to protect,
despite the implausibility of that promise in a world of increasing
racial and ethnic diversity.

These cultural conservatives are now ``Trump's to lose,'' Malka said,
noting that ``he cannot afford to lose many of these whose economic
attitudes are well to the left of the Republican Party's.'' During the
current crisis, Trump has addressed core economic anxieties of these
voters with legislation funneling an extra
\href{https://fortune.com/2020/07/27/unemployment-600-extra-benefits-extension-cut-republican-plan-stimulus-package-bill-200-per-week-how-much-update/}{\$600-a-week}
to the unemployed, a
\href{https://www.marketplace.org/2020/03/19/white-house-may-send-out-1000-checks-to-adults-500-to-children/}{\$1,200
grant} to adults in household with incomes below \$150,000. a
\href{https://nlihc.org/federal-moratoriums}{moratorium} on evictions
from federally supported housing and
\href{https://smallbusiness.house.gov/uploadedfiles/cares_flow_chart_edit.pdf}{\$350
billion for loans to small businesses}. In this context, Malka wrote me,
Trump's ``efforts to stoke cultural conflict with authoritarian actions
make sense from the standpoint of keeping this group on his side, but it
comes with other electoral costs.''

Who Trump doesn't have on his side are the millions of Black and Latino
voters frustrated by even greater economic hurdles than their white
counterparts, compounded by a history of segregation and discrimination.
These voters, an ever-growing share of the electorate, loom even larger
now than they did four years ago, and they are squarely in the
Democratic camp.

\includegraphics{https://static01.nyt.com/images/2020/07/29/opinion/29edsall2/merlin_172933416_6425b773-f134-4277-a544-697c1a6b13b7-articleLarge.jpg?quality=75\&auto=webp\&disable=upscale}

``It's already clear that the 2020 electorate will be unique in several
ways. Nonwhites will account for a third of eligible voters --- their
largest share ever --- driven by long-term increases among certain
groups, especially Hispanics,'' according to a
\href{https://www.pewsocialtrends.org/essay/an-early-look-at-the-2020-electorate/}{Pew
Research report}. Trump, who is exceptionally dependent on white
support, faces the prospect of trying to win over an electorate in which
the share of white voters has fallen, over the past 20 years, from 76.4
percent to 66.6 percent.

Trump's approval level, always low among Black and Hispanic voters, has
plummeted as he has sharpened his ever-present appeals to bigotry. In
the first months of 2020, 16 percent of African-Americans approved of
Trump,
\href{https://news.gallup.com/poll/313454/trump-job-approval-rating-steady-lower-level.aspx}{according
to Gallup}. In the period from late May to June, that fell to 10
percent. Among Hispanics, Trump's approval over the same period fell
from 34 to 26 percent.

For Trump, this is his 2020 dilemma: As of July 28, Covid-19 cases
reached 16.3 million worldwide, with 4.3 million in the United States
Total deaths have reached 650,805, 147,672 of them in this country,
according to the
\href{https://www.who.int/docs/default-source/coronaviruse/situation-reports/20200727-covid-19-sitrep-189.pdf?sfvrsn=b93a6913_2}{World
Health Organization} and the
\href{https://www.cdc.gov/coronavirus/2019-ncov/cases-updates/cases-in-us.html}{Centers
for Disease Control} and Prevention. According to Politico, a mere third
of Americans (32 percent) say they support Trump's handling of the
pandemic.

Trump
\href{https://www.usatoday.com/story/news/politics/2020/07/27/trump-says-he-wont-pay-respects-john-lewis-he-lies-state/5520694002/}{pointedly
declined} to attend services for Representative John Lewis as mourners
lined up for blocks outside the east front of the U.S. Capitol in 90
degree heat on Monday. He
\href{https://www.washingtonpost.com/politics/more-federal-agents-dispatched-to-portland-as-protests-rise-in-other-cities/2020/07/27/20a717be-d03c-11ea-8d32-1ebf4e9d8e0d_story.html}{increased}
the number of
\href{https://www.theguardian.com/us-news/2020/jul/26/portland-federal-agents-teargas-protesters-black-lives-matter}{federal
paramilitaries} in Portland as he attempted to confront Black Lives
Matter protests he described as ``anarchy.''

Trump is trapped between pressures: to keep feeding red meat to the
white working and middle class voters who continue to support him while
struggling to slow the defection of white, well-educated suburbanites
who delivered 42 House seats to the Democrats in 2018 and who now
threaten to restore Democratic rule across the board.

Facing these cross-pressures, Trump has clearly staked out where his
allegiance lies: with conservative white America.

At a time when even the state of Mississippi is removing Confederate
emblems from its state flag, Trump told Chris Wallace ten days ago on
\href{https://www.foxnews.com/politics/transcript-fox-news-sunday-interview-with-president-trump}{Fox
News Sunday}:

\begin{quote}
When people proudly have their Confederate flags, they're not talking
about racism. They love their flag, it represents the South, they like
the South. People right now like the South. I'd say it's freedom of, of,
of many things, but it's freedom of speech.
\end{quote}

Asked about changing the names of military installations currently
honoring Confederate generals --- a proposal that has support from the
Armed Forces and some Republicans in Congress --- Trump replied:

\begin{quote}
Excuse me, excuse me. I don't care what the military says. I'm supposed
to make the decision \ldots{}. We're going to name it after the Reverend
Al Sharpton?
\end{quote}

Trump knows where he is going. The question is whether enough of the
electorate will follow.

\emph{The Times is committed to publishing}
\href{https://www.nytimes.com/2019/01/31/opinion/letters/letters-to-editor-new-york-times-women.html}{\emph{a
diversity of letters}} \emph{to the editor. We'd like to hear what you
think about this or any of our articles. Here are some}
\href{https://help.nytimes.com/hc/en-us/articles/115014925288-How-to-submit-a-letter-to-the-editor}{\emph{tips}}\emph{.
And here's our email:}
\href{mailto:letters@nytimes.com}{\emph{letters@nytimes.com}}\emph{.}

\emph{Follow The New York Times Opinion section on}
\href{https://www.facebook.com/nytopinion}{\emph{Facebook}}\emph{,}
\href{http://twitter.com/NYTOpinion}{\emph{Twitter (@NYTopinion)}}
\emph{and}
\href{https://www.instagram.com/nytopinion/}{\emph{Instagram}}\emph{.}

Advertisement

\protect\hyperlink{after-bottom}{Continue reading the main story}

\hypertarget{site-index}{%
\subsection{Site Index}\label{site-index}}

\hypertarget{site-information-navigation}{%
\subsection{Site Information
Navigation}\label{site-information-navigation}}

\begin{itemize}
\tightlist
\item
  \href{https://help.nytimes.com/hc/en-us/articles/115014792127-Copyright-notice}{©~2020~The
  New York Times Company}
\end{itemize}

\begin{itemize}
\tightlist
\item
  \href{https://www.nytco.com/}{NYTCo}
\item
  \href{https://help.nytimes.com/hc/en-us/articles/115015385887-Contact-Us}{Contact
  Us}
\item
  \href{https://www.nytco.com/careers/}{Work with us}
\item
  \href{https://nytmediakit.com/}{Advertise}
\item
  \href{http://www.tbrandstudio.com/}{T Brand Studio}
\item
  \href{https://www.nytimes.com/privacy/cookie-policy\#how-do-i-manage-trackers}{Your
  Ad Choices}
\item
  \href{https://www.nytimes.com/privacy}{Privacy}
\item
  \href{https://help.nytimes.com/hc/en-us/articles/115014893428-Terms-of-service}{Terms
  of Service}
\item
  \href{https://help.nytimes.com/hc/en-us/articles/115014893968-Terms-of-sale}{Terms
  of Sale}
\item
  \href{https://spiderbites.nytimes.com}{Site Map}
\item
  \href{https://help.nytimes.com/hc/en-us}{Help}
\item
  \href{https://www.nytimes.com/subscription?campaignId=37WXW}{Subscriptions}
\end{itemize}
