Sections

SEARCH

\protect\hyperlink{site-content}{Skip to
content}\protect\hyperlink{site-index}{Skip to site index}

\href{https://myaccount.nytimes.com/auth/login?response_type=cookie\&client_id=vi}{}

\href{https://www.nytimes.com/section/todayspaper}{Today's Paper}

\href{/section/opinion}{Opinion}\textbar{}We Lost the Battle for the
Republican Party's Soul Long Ago

\url{https://nyti.ms/2DawxRf}

\begin{itemize}
\item
\item
\item
\item
\item
\item
\end{itemize}

Advertisement

\protect\hyperlink{after-top}{Continue reading the main story}

\href{/section/opinion}{Opinion}

Supported by

\protect\hyperlink{after-sponsor}{Continue reading the main story}

\hypertarget{we-lost-the-battle-for-the-republican-partys-soul-long-ago}{%
\section{We Lost the Battle for the Republican Party's Soul Long
Ago}\label{we-lost-the-battle-for-the-republican-partys-soul-long-ago}}

Only fear will motivate the party to change --- the cold fear only
defeat can bring.

By Stuart Stevens

Mr. Stevens is a Republican political consultant.

\begin{itemize}
\item
  July 29, 2020
\item
  \begin{itemize}
  \item
  \item
  \item
  \item
  \item
  \item
  \end{itemize}
\end{itemize}

\includegraphics{https://static01.nyt.com/images/2020/07/28/opinion/28Stevens2/merlin_174249147_b661b850-6097-4221-bd90-92f37ce14c77-articleLarge.jpg?quality=75\&auto=webp\&disable=upscale}

After Mitt Romney lost the 2012 presidential race, the Republican
National Committee chairman, Reince Priebus, commissioned an
\href{https://www.nytimes.com/2013/03/19/us/politics/republicans-plan-overhaul-for-2016-primary-season.html}{internal
party study} to examine why the party had won the popular vote only once
since 1988.

The results of that so-called autopsy were fairly obvious: The party
needed to appeal to more people of color, reach out to younger voters,
become more welcoming to women. Those conclusions were presented as not
only a political necessity but also a moral mandate if the Republican
Party were to be a governing party in a rapidly changing America.

Then Donald Trump emerged and the party threw all those conclusions out
the window with an almost audible sigh of relief: \emph{Thank God we can
win without pretending we really care about this stuff.} That reaction
was sadly predictable.

I spent decades
\href{https://www.nytimes.com/2020/07/16/us/politics/trump-republicans.html}{working
to elect Republicans,} including Mr. Romney and four other presidential
candidates, and I am here to bear reluctant witness that Mr. Trump
didn't hijack the Republican Party. He is the logical conclusion of
\href{https://www.nytimes.com/2020/03/18/opinion/trump-republicans-racism.html}{what
the party became} over the past 50 or so years, a natural product of the
seeds of race-baiting, self-deception and anger that now dominate it.
Hold Donald Trump up to a mirror and that bulging, scowling orange face
is today's Republican Party.

\includegraphics{https://static01.nyt.com/images/2020/07/28/opinion/28Stevens/28Stevens-articleLarge.jpg?quality=75\&auto=webp\&disable=upscale}

I saw the warning signs but ignored them and chose to believe what I
wanted to believe: The party wasn't just
\href{https://www.nytimes.com/2020/03/18/opinion/trump-republicans-racism.html}{a
white grievance party}; there was still a big tent; the others guys were
worse. Many of us in the party saw this dark side and told ourselves it
was a recessive gene. We were wrong. It turned out to be the dominant
gene.

What is most telling is that the Republican Party actively embraced,
supported, defended and now enthusiastically identifies with a man who
eagerly exploits the nation's racial tensions. In our system, political
parties should serve a circuit breaker function. The Republican Party
never pulled the switch.

Racism is the original sin of the modern Republican Party. While many
Republicans today like to mourn the absence of an intellectual voice
like William Buckley, it is often overlooked that Mr. Buckley began his
career as
\href{https://www.politico.com/magazine/story/2017/05/13/william-f-buckley-civil-rights-215129}{a
racist defending segregation}.

In the Richard Nixon White House, Pat Buchanan and Kevin Phillips wrote
a re-election campaign memo
headed\href{https://www.cnn.com/2010/POLITICS/01/11/nixon.racial.strategy/index.html}{``Dividing
the Democrats''} in which they outlined what would come to be known as
the Southern Strategy. It assumes there is little Republicans can do to
attract Black Americans and details a two-pronged strategy: Utilize
Black support of Democrats to alienate white voters while trying to
decrease that support by sowing dissension within the Democratic Party.

That strategy has worked so well that it was copied by the Russians in
their
\href{https://www.nytimes.com/2019/04/18/us/politics/the-mueller-report-excerpts.html}{2016
efforts} to help elect Mr. Trump.

In the 2000 George W. Bush campaign, on which I worked, we acknowledged
the failures of Republicans to attract significant nonwhite support.
When Mr. Bush called himself a
\href{https://www.nytimes.com/2000/06/12/us/bush-draws-campaign-theme-from-more-than-the-heart.html}{``compassionate
conservative,''}some on the right attacked him, calling it an admission
that conservatism had not been compassionate. That was true; it had not
been. Many of us believed we could steer the party to that ``kinder,
gentler'' place his father described. We were wrong.

Reading Mr. Bush's
\href{http://movies2.nytimes.com/library/politics/camp/080400wh-bush-speech.html}{2000
acceptance speech} at the Republican National Convention now is like
stumbling across a document from a lost civilization, with its calls for
humility, service and compassion. That message couldn't attract 20
percent in a Republican presidential primary today. If there really was
a battle for the soul of the Republican Party, we lost.

There is a collective blame to be shared by those of us who have created
the modern Republican Party that has so egregiously betrayed the
principles it claimed to represent. My j'accuse is against us all, not a
few individuals who were the most egregious.

How did this happen? How do you abandon deeply held beliefs about
character, personal responsibility, foreign policy and the national debt
in a matter of months? You don't. The obvious answer is those beliefs
weren't deeply held. What others and I thought were bedrock values
turned out to be mere marketing slogans easily replaced. I feel like the
guy working for
\href{https://www.nytimes.com/topic/person/bernard-l-madoff}{Bernie
Madoff} who thought they were actually beating the market.

Mr. Trump has served a useful purpose by exposing the deep flaws of a
major American political party. Like a heavy truck driven over a bridge
on the edge of failure, he has made it impossible to ignore the
long-developing fault lines of the Republican Party. A party rooted in
decency and values does not embrace the anger that Mr. Trump peddles as
patriotism.

This collapse of a major political party as a moral governing force is
unlike anything we have seen in modern American politics. The closest
parallel is the demise of the Communist Party in the Soviet Union, when
the dissonance between what the party said it stood for and what
citizens actually experienced was so great that it was unsustainable.

This election should signal a day of reckoning for the party and all who
claim it as a political identity. Will it? I've given up hope that there
are any lines of decency or normalcy that once crossed would move
Republican leaders to act as if they took their oath of office more
seriously than their allegiance to party. Only fear will motivate the
party to change --- the cold fear only defeat can bring.

That defeat is looming. Will it bring desperately needed change to the
Republican Party? I'd like to say I'm hopeful. But that would be a lie
and there have been too many lies for too long.

Stuart Stevens is a Republican political consultant and the author of
the forthcoming book ``It Was All a Lie: How the Republican Party Became
Donald Trump,'' from which this essay is adapted.

\emph{The Times is committed to publishing}
\href{https://www.nytimes.com/2019/01/31/opinion/letters/letters-to-editor-new-york-times-women.html}{\emph{a
diversity of letters}} \emph{to the editor. We'd like to hear what you
think about this or any of our articles. Here are some}
\href{https://help.nytimes.com/hc/en-us/articles/115014925288-How-to-submit-a-letter-to-the-editor}{\emph{tips}}\emph{.
And here's our email:}
\href{mailto:letters@nytimes.com}{\emph{letters@nytimes.com}}\emph{.}

\emph{Follow The New York Times Opinion section on}
\href{https://www.facebook.com/nytopinion}{\emph{Facebook}}\emph{,}
\href{http://twitter.com/NYTOpinion}{\emph{Twitter (@NYTopinion)}}
\emph{and}
\href{https://www.instagram.com/nytopinion/}{\emph{Instagram}}\emph{.}

Advertisement

\protect\hyperlink{after-bottom}{Continue reading the main story}

\hypertarget{site-index}{%
\subsection{Site Index}\label{site-index}}

\hypertarget{site-information-navigation}{%
\subsection{Site Information
Navigation}\label{site-information-navigation}}

\begin{itemize}
\tightlist
\item
  \href{https://help.nytimes.com/hc/en-us/articles/115014792127-Copyright-notice}{©~2020~The
  New York Times Company}
\end{itemize}

\begin{itemize}
\tightlist
\item
  \href{https://www.nytco.com/}{NYTCo}
\item
  \href{https://help.nytimes.com/hc/en-us/articles/115015385887-Contact-Us}{Contact
  Us}
\item
  \href{https://www.nytco.com/careers/}{Work with us}
\item
  \href{https://nytmediakit.com/}{Advertise}
\item
  \href{http://www.tbrandstudio.com/}{T Brand Studio}
\item
  \href{https://www.nytimes.com/privacy/cookie-policy\#how-do-i-manage-trackers}{Your
  Ad Choices}
\item
  \href{https://www.nytimes.com/privacy}{Privacy}
\item
  \href{https://help.nytimes.com/hc/en-us/articles/115014893428-Terms-of-service}{Terms
  of Service}
\item
  \href{https://help.nytimes.com/hc/en-us/articles/115014893968-Terms-of-sale}{Terms
  of Sale}
\item
  \href{https://spiderbites.nytimes.com}{Site Map}
\item
  \href{https://help.nytimes.com/hc/en-us}{Help}
\item
  \href{https://www.nytimes.com/subscription?campaignId=37WXW}{Subscriptions}
\end{itemize}
