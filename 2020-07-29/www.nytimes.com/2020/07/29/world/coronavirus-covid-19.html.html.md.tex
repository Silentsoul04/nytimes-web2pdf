Sections

SEARCH

\protect\hyperlink{site-content}{Skip to
content}\protect\hyperlink{site-index}{Skip to site index}

\href{https://www.nytimes.com/section/world}{World}

\href{https://myaccount.nytimes.com/auth/login?response_type=cookie\&client_id=vi}{}

\href{https://www.nytimes.com/section/todayspaper}{Today's Paper}

\href{/section/world}{World}\textbar{}Coronavirus Live Updates: Deaths
Top 150,000 in the United States

\url{https://nyti.ms/2DeTB1m}

\begin{itemize}
\item
\item
\item
\item
\item
\item
\end{itemize}

\href{https://www.nytimes.com/news-event/coronavirus?action=click\&pgtype=Article\&state=default\&region=TOP_BANNER\&context=storylines_menu}{The
Coronavirus Outbreak}

\begin{itemize}
\tightlist
\item
  live\href{https://www.nytimes.com/2020/07/29/world/coronavirus-covid-19.html?action=click\&pgtype=Article\&state=default\&region=TOP_BANNER\&context=storylines_menu}{Latest
  Updates}
\item
  \href{https://www.nytimes.com/interactive/2020/us/coronavirus-us-cases.html?action=click\&pgtype=Article\&state=default\&region=TOP_BANNER\&context=storylines_menu}{Maps
  and Cases}
\item
  \href{https://www.nytimes.com/interactive/2020/science/coronavirus-vaccine-tracker.html?action=click\&pgtype=Article\&state=default\&region=TOP_BANNER\&context=storylines_menu}{Vaccine
  Tracker}
\item
  \href{https://www.nytimes.com/interactive/2020/07/27/upshot/coronavirus-pooled-testing.html?action=click\&pgtype=Article\&state=default\&region=TOP_BANNER\&context=storylines_menu}{Understand
  Pooled Testing}
\item
  \href{https://www.nytimes.com/live/2020/07/29/business/stock-market-today-coronavirus?action=click\&pgtype=Article\&state=default\&region=TOP_BANNER\&context=storylines_menu}{Economy}
\end{itemize}

Advertisement

\protect\hyperlink{after-top}{Continue reading the main story}

Supported by

\protect\hyperlink{after-sponsor}{Continue reading the main story}

LIVE UPDATES

Updated~

July 29, 2020, 2:24 p.m. ET

July 29, 2020, 2:24 p.m. ET

\hypertarget{coronavirus-live-updates-deaths-top-150000-in-the-united-states}{%
\section{Coronavirus Live Updates: Deaths Top 150,000 in the United
States}\label{coronavirus-live-updates-deaths-top-150000-in-the-united-states}}

Federal officials urged states to take aggressive action to slow the
spread of the virus while the president talked up nonexistent
``virus-free'' areas. Big retailers are mandating masks, but enforcement
is an issue.

Right Now

The chairman of a House panel investigating the government's coronavirus
response is accusing the White House of suppressing its own dire
state-by-state assessments of the pandemic.

\hypertarget{heres-what-you-need-to-know}{%
\subsubsection{Here's what you need to
know:}\label{heres-what-you-need-to-know}}

\begin{itemize}
\tightlist
\item
  \protect\hyperlink{link-1fc03c4a}{The virus death toll in the U.S.
  reaches 150,000.}
\item
  \protect\hyperlink{link-6644b9da}{Gohmert tests positive for the virus
  (and blames a mask for it), sending Capitol Hill racing to trace
  contacts.}
\item
  \protect\hyperlink{link-73760ee2}{Trump says `we really don't care'
  about negotiating a big recovery bill, instead pushing for a narrower
  aid package.}
\item
  \protect\hyperlink{link-531300e7}{There's a wrinkle in stores' mask
  policies: Enforcement.}
\item
  \protect\hyperlink{link-43ec24d5}{The chairman of a House panel
  examining the U.S. virus response accuses the White House of
  suppressing reports.}
\item
  \protect\hyperlink{link-69d86eaf}{A study asserts school closures in
  the spring saved lives. Experts caution about applying the findings
  now.}
\item
  \protect\hyperlink{link-4a99ca45}{More than 6,300 cases have been
  linked to U.S. colleges.}
\end{itemize}

\includegraphics{https://static01.nyt.com/images/2020/07/28/us/28virus-briefing-death-toll-swap/28virus-briefing-death-toll-swap-articleLarge.jpg?quality=75\&auto=webp\&disable=upscale}

Key Data of the day

\subsection{}

The virus death toll in the U.S. reaches 150,000.

More than 150,000 people have died in the United States from the
coronavirus,
\href{https://www.nytimes.com/interactive/2020/us/coronavirus-us-cases.html}{according
to a New York Times database}, as the rate of deaths continues to rise
on the heels of ballooning infections and hospitalizations in many
areas.

An average of about 1,000 virus-related deaths a day have been reported
over the past week, the worst rate since early June, when the number of
people dying seemed to be falling. Now, daily death counts are rising in
24 states and Puerto Rico.

The nation's overall death toll reached the grim figure **** on ****
Wednesday, five months after the
\href{https://www.nytimes.com/2020/02/29/us/coronavirus-washington-death.html}{first
reported virus death} in the United States in February. The nation
passed the 50,000 mark on April 27 and 100,000 on May 27, a milestone
whose approach
\href{https://www.nytimes.com/interactive/2020/05/24/us/us-coronavirus-deaths-100000.html}{The
Times commemorated} by filling its front page with names of the dead.

During the early peak of the U.S. epidemic in late April, the national
death toll was driven by a surge in New York State, which at its worst
was reporting about 1,000 deaths a day, roughly half the national total
at that time.

These days, the toll is being felt much more widely across many states,
especially in the South, while New York is reporting about 16 deaths a
day on average. For example, more than 2,100 deaths have been reported
in the past week in Texas, the state with the highest recent
\href{https://www.nytimes.com/interactive/2020/us/coronavirus-us-cases.html}{death
toll relative to its population}, followed by Arizona and South
Carolina. On Wednesday, Florida again set its single-day record for
deaths, reporting 216 fatalities, bringing the state's total to 6,332.

The trend in virus deaths generally lags the trend in infections,
reflecting the delays between when people test positive, when they die
and when those deaths are reported. Daily death tolls kept falling for a
while after daily case reports began to climb significantly in June.
Since early July, though, the death numbers have been rising, while
infection reports have begun to level off at around 65,000 a day.

\hypertarget{-1}{%
\subsection{}\label{-1}}

Gohmert tests positive for the virus (and blames a mask for it), sending
Capitol Hill racing to trace contacts.

Image

Representative Louie Gohmert on Capitol Hill on Tuesday.~While adjusting
his mask, he said, ``I am bound to have put some virus on the mask that
I sucked in.''Credit...Matt McClain, Press Pool

Representative Louie Gohmert, a Texas Republican who has frequently
refused to don a face covering in the Capitol, confirmed on Wednesday
that he had tested positive for the coronavirus ahead of a planned trip
with President Trump on Air Force One, and blamed his diagnosis on
wearing a mask.

The results immediately sent a shudder through the Capitol, where Mr.
Gohmert has actively participated in multiple congressional hearings
this week, including Tuesday's Judiciary Committee session with Attorney
General William P. Barr and a hearing held by the Natural Resources
Committee, during which he did not wear a mask. Mr. Gohmert said he was
not experiencing symptoms but had notified colleagues he may have come
into contact with.

Lawmakers and Mr. Barr were seated more than six feet apart during the
hearing, but reporters spotted an unmasked Mr. Gohmert outside the
hearing room exchanging words with Mr. Barr and in proximity to him. A
Justice Department spokesman, Kerri Kupec, said that the attorney
general would be tested on Wednesday.

Mr. Gohmert is among a group of House Republicans who have pointedly
refused to wear masks in many instances while in the Capitol in recent
weeks despite warnings from public health experts and an outbreak in his
home state. In an interview from his office later Wednesday,
\href{https://www.easttexasmatters.com/news/local-news/watch-now-gohmert-to-isolate-for-next-10-days-still-believes-in-personal-freedom-for-wearing-masks/}{he
told KETK TV}, a Texas Fox affiliate, that he would isolate for 10 days
on the advice of doctors and would wear a mask ``religiously'' until he
was cleared. But he said his diagnosis had vindicated his skepticism
about wearing facial coverings to guard against the spread of the virus.

Medical experts overwhelmingly say that wearing a mask is one of the
most effective ways to limit the spread of the virus, though they warn
that using your hands to adjust your mask improperly can pose a risk.

``There are an awful lot of people who think it's the great thing to do
all the time, but I can't help but think if I hadn't been wearing a mask
so much in the last 10 days or so, I really wonder if I would have
gotten it,'' Mr. Gohmert said. ``Moving the mask around, getting it
sitting just right, I am bound to have put some virus on the mask that I
sucked in. That is most likely what happened.''

Democrats were furious at the news, and both parties spent Wednesday
morning scrambling to retrace Mr. Gohmert's steps. The House Judiciary
Committee was waiting for official guidance from Congress's attending
physician. It is a daunting task, since Mr. Gohmert is a frequent
schmoozer who could have come into close contact with dozens of fellow
lawmakers and aides this week alone.

``I'm concerned about the irresponsible behavior of many of the
Republicans who have chosen to consistently flout well-established
public health guidance,'' said Representative Hakeem Jeffries, Democrat
of New York and a member of the Judiciary Committee. He pleaded with
Republicans like Mr. Gohmert to put on masks or go home.

Members of Congress have been flying weekly between Washington and their
home states --- some of which are experiencing serious outbreaks --- and
they are not required to be tested. Mr. Gohmert received a test only
because he was scheduled to be in proximity to the president.

\hypertarget{-2}{%
\subsection{}\label{-2}}

Trump says `we really don't care' about negotiating a big recovery bill,
instead pushing for a narrower aid package.

Image

``You work on the payments for the people,'' President Trump said,
referring to another round of direct payments, ``and the rest of it ---
we're so far apart, we don't care.''Credit...Doug Mills/The New York
Times

President Trump on Wednesday indicated that he did not care about the
fate of a broad economic recovery package that lawmakers in both
parties, along with members of his own administration, are scrambling to
put together before tens of millions of Americans formally lose their
jobless benefits on Friday, telling reporters he would rather see a
narrow package.

``You work on the payments for the people,'' Mr. Trump said, referring
to another round of direct payments, ``and the rest of it --- we're so
far apart, we don't care.''

``We really don't care,'' Mr. Trump added.

Mr. Trump suggested that he wanted to renew a federal moratorium on
evictions that expired earlier this month for millions of Americans,
saying, ``We want to stop the evictions.'' But the Republican proposal
his administration helped draft has no measure to do so.

Steven Mnuchin, the Treasury secretary, said the president ``is very
focused on evictions and unemployment'' --- though Mr. Trump made no
mention of the \$600-per-week enhanced unemployment benefits set to
formally expire Friday. Mr. Mnuchin said if the administration cannot
reach agreement with Democrats by then on a broader economic
stabilization plan, ``the president wants to look at giving us more time
to negotiate this.''

Mr. Mnuchin and Mark Meadows, the White House chief of staff, are
expected to huddle with Senate Republicans during their weekly policy
lunch and meet for the third consecutive day with Speaker Nancy Pelosi
of California and Senator Chuck Schumer of New York, the minority
leader, later Wednesday afternoon. Democrats have so far rejected the
prospect of a narrow package, insisting on a comprehensive package, and
Mr. Trump has dismissed the Republican package as ``semi-irrelevant.''

On Wednesday, he slammed Republicans for distancing themselves from his
efforts to secure funding for a new F.B.I. headquarters in Washington as
part of the recovery package, saying that, ``Republicans should go back
to school and learn.''

\includegraphics{https://static01.nyt.com/images/2017/01/29/podcasts/the-daily-album-art/the-daily-album-art-articleInline-v2.jpg?quality=75\&auto=webp\&disable=upscale}

\hypertarget{listen-to-the-daily-the-battle-over-unemployment-benefits}{%
\subsubsection{Listen to `The Daily': The Battle Over Unemployment
Benefits}\label{listen-to-the-daily-the-battle-over-unemployment-benefits}}

As Republicans consider the extension of existing unemployment benefits,
the November election looms large.

transcript

Back to The Daily

bars

0:00/26:13

-26:13

transcript

\hypertarget{listen-to-the-daily-the-battle-over-unemployment-benefits-1}{%
\subsection{Listen to `The Daily': The Battle Over Unemployment
Benefits}\label{listen-to-the-daily-the-battle-over-unemployment-benefits-1}}

\hypertarget{hosted-by-michael-barbaro-produced-by-rachel-quester-and-daniel-guillemette-with-help-from-robert-jimison-and-stella-tan-and-edited-by-mj-davis-lin}{%
\subsubsection{Hosted by Michael Barbaro; produced by Rachel Quester and
Daniel Guillemette; with help from Robert Jimison and Stella Tan; and
edited by M.J. Davis
Lin}\label{hosted-by-michael-barbaro-produced-by-rachel-quester-and-daniel-guillemette-with-help-from-robert-jimison-and-stella-tan-and-edited-by-mj-davis-lin}}

\hypertarget{as-republicans-consider-the-extension-of-existing-unemployment-benefits-the-november-election-looms-large}{%
\paragraph{As Republicans consider the extension of existing
unemployment benefits, the November election looms
large.}\label{as-republicans-consider-the-extension-of-existing-unemployment-benefits-the-november-election-looms-large}}

\begin{itemize}
\item
  {[}music{]}
\item
  michael barbaro\\
  From The New York Times, I'm Michael Barbaro. This is ``The Daily.''

  Today: A fight has erupted among congressional Republicans over how
  long and how generously government should help the unemployed during
  the pandemic. Nick Fandos on what that battle is really about.

  It's Tuesday, July 28.

  Nick, tell me about this deadline coming up on Friday.
\item
  nick fandos\\
  So on Friday, at the end of July, one of the key programs in the \$2
  trillion economic relief package, called the CARES Act, that Congress
  passed this spring to deal with the coronavirus pandemic, is set to
  expire. This is the federal unemployment benefit, this extra \$600
  that the federal government has been putting into unemployment checks,
  on top of whatever states give the tens of millions of Americans that
  are out of work.
\item
  michael barbaro\\
  Right. And the thinking was that state unemployment benefits, which is
  how most people get by when they are laid off, are kind of stingy. And
  because these layoffs were so widespread, the federal government
  needed to step in an unusual way.
\item
  nick fandos\\
  That's right. And you know, \$600 was arrived at by congressional
  Democrats and the Treasury Secretary, Steve Mnuchin, as something like
  a kind of average wage that they thought might be lost across the
  board. And though some Republicans were uneasy ---
\item
  archived recording\\
  Mr. President, the majority leader of the Senate.
\end{itemize}

nick fandos

--- they ultimately set aside their concerns and ended up voting
unanimously to put this program and others in place.

\begin{itemize}
\item
  archived recording (mitch mcconnell)\\
  Our nation needed us to go big and go fast. And they did.

  So today, Mr. President, the Senate will act to help the people of
  this country weather this storm.
\end{itemize}

michael barbaro

Right. And I think for many Americans the sense was that this program
--- \$600 a week from the federal government --- would probably last as
long as widespread unemployment lasted, stemming from the pandemic.

nick fandos

I think that that's right, that that was the assumption of many
Americans. But Republicans never quite viewed it that way.

\begin{itemize}
\tightlist
\item
  archived recording (john cornyn)\\
  We have spent a lot of money in the last couple of months. But we've
  done so in the face of an emergency, kind of like the civilian
  equivalent of World War II.
\end{itemize}

nick fandos

They saw the whole stimulus bill, including this benefit, as a kind of
extraordinary measure for extraordinary circumstances. And that this was
kind of a bridge to float the economy and float the American people
through this period where the government was asking them to stay home,
so that we could get the virus under control.

\begin{itemize}
\tightlist
\item
  archived recording (ted cruz)\\
  Look, I supported every one of these bills that has come through. I
  agree that we need emergency relief to help people, to help people
  through the crisis as a short-term bridge loan.
\end{itemize}

nick fandos

But you know, if that was a gamble --- and it was, that this is going to
be a temporary thing --- Republicans do not come out where they want to.
The virus has resurged in many states now across the South and West, you
know, in states that are traditionally red states and are represented by
Republicans.

\begin{itemize}
\tightlist
\item
  archived recording (mitch mcconnell)\\
  So the question today is where are we? And where do we go from here?
\end{itemize}

nick fandos

And the party now has to kind of come to terms with the fact that what
they hoped would be a bridge is going to be a lot longer than they
initially thought.

\begin{itemize}
\tightlist
\item
  archived recording (mitch mcconnell)\\
  We had hoped we'd be on the way to saying goodbye to this health care
  pandemic. Clearly, it is not over.
\end{itemize}

michael barbaro

Right. Which brings us back to this Friday expiration date. So do
Republicans have intrinsic objections to just renewing the \$600 a week?

nick fandos

So for most Republicans, the answer is yes.

michael barbaro

Hm.

nick fandos

That \$600 figure, as we said, was arrived at honestly, but somewhat
hastily back in March. And Republicans started voicing concerns at the
time.

\begin{itemize}
\tightlist
\item
  archived recording (ted cruz)\\
  For 68 percent of people receiving it right now, they are being paid
  more on unemployment than they made in their job.
\end{itemize}

nick fandos

And they've grown a lot louder since. That \$600 from the federal
government, on top of whatever states were giving people that were out
of work, was simply too generous.

\begin{itemize}
\tightlist
\item
  archived recording (ted cruz)\\
  And I'll tell you, I've spoken to small business owners all over the
  state of Texas who are trying to reopen.
\end{itemize}

nick fandos

And actually was disincentivizing and has disincentivized many Americans
from going back to work.

\begin{itemize}
\tightlist
\item
  archived recording (ted cruz)\\
  --- and they're calling their waiters and waitresses, they're calling
  their busboys. And they won't come back. And of course they won't come
  back. Because the federal government is paying, in some instances,
  twice as much money to stay home.
\end{itemize}

nick fandos

So ideologically, many Republicans in Congress were never comfortable
with this \$600 benefit at that level in the first place. And then,
they're certainly not comfortable with extending it into perpetuity.

michael barbaro

So Nick, with this program running out of time, how is this playing out
among the Republicans?

nick fandos

So as Republicans are approaching these deadlines at the end of July,
they're looking around and seeing a bunch of different inputs that are
really difficult for them. On the one hand, Democrats are, you know,
unabashedly and enthusiastically pushing to extend this \$600 benefit
through the end of the year and as long as it's needed.

michael barbaro

Mhm.

nick fandos

And at the same time, Republicans are having to reconcile themselves to
the fact that the virus is spreading around the country. There are signs
in the last few weeks that the economy, which was recovering, is
starting to potentially soften again. And they recognize for a variety
of reasons --- economically, for the livelihood of the country, and
politically, as they're looking ahead to November's elections --- that
it's simply not going to be an option not to have a plan.

michael barbaro

Mhm.

nick fandos

And so Republicans start trying to put together their own proposal for
how to fix unemployment benefits going forward and a range of other
programs to keep the economy afloat. And it turns out it's a lot harder
than they think it's going to be.

michael barbaro

What do you mean?

nick fandos

Well, it turns out, as they try to unpack this and get into the details
of what might we do next, that there's a pretty big split between two
different camps of Republicans.

\begin{itemize}
\tightlist
\item
  archived recording (ted cruz)\\
  I asked my Republican colleagues, what in the hell are we doing?
\end{itemize}

nick fandos

So one of them are the kind of arch conservatives that are really
worried about federal spending. People like Ted Cruz.

\begin{itemize}
\tightlist
\item
  archived recording (ted cruz)\\
  A number of senators at lunch get up and say, well gosh, we need \$20
  billion for this. We need \$100 billion for this. And they're just
  really eager to spend money. I'm, like, what are you guys doing?
\end{itemize}

nick fandos

Or Rand Paul, who compared his colleagues to a bunch of Bernie bros with
the way they were talking.

\begin{itemize}
\tightlist
\item
  archived recording (rand paul)\\
  I find it extraordinary that I just came from a Republican caucus
  meeting that could be sort of the Bernie bros progressive caucus.
\end{itemize}

nick fandos

And that is a sharp pejorative in the Senate Republican conference.

michael barbaro

I would think.

\begin{itemize}
\tightlist
\item
  archived recording (rand paul)\\
  This is insane. It's got to stop. We're ruining the country. And there
  has to be some voice left for fiscal conservatism in this country.
\end{itemize}

nick fandos

This group is just, frankly, uneasy about the \$2 trillion that they
spent back in the spring and is not interested in seeing the federal
government add to the deficit, add to the debt and further involve
itself in the U.S. economy.

\begin{itemize}
\tightlist
\item
  archived recording (rand paul)\\
  I, for one, am alarmed at where the country is heading. I'm also
  alarmed that my party has forgotten what they actually stand for.
  There is no difference now between the two parties on spending.
\end{itemize}

nick fandos

Now, at the other end of the spectrum are a group of more moderate or
middle-of-the-road Republicans, who are up for re-election this fall and
are actually having to face the voters, in many cases, in swing states
or blue states where President Trump and the Republican response to the
pandemic have been deeply unpopular. People like Cory Gardner or Thom
Tillis ---

\begin{itemize}
\tightlist
\item
  archived recording (thom tillis)\\
  Well, I think we have to build on what we did with the CARES Act,
  almost \$3 trillion dollars to help individuals, to provide a
  supplement for unemployment.
\end{itemize}

nick fandos

--- who have really staked their re-election on the government's
response to this crisis, and on showing that they are effectively
leading the country through one of its most challenging periods in
anybody's memory. And joining with them on that side ---

\begin{itemize}
\tightlist
\item
  archived recording (mitch mcconnell)\\
  This crisis is far from over.
\end{itemize}

nick fandos

--- are some of the best known leaders of the Republican Party on
Capitol Hill.

michael barbaro

Hm.

\begin{itemize}
\tightlist
\item
  archived recording (mitch mcconnell)\\
  For weeks now, I have made it clear that further legislation out of
  the Senate will be a serious response to the crisis.
\end{itemize}

nick fandos

So Mitch McConnell, the majority leader from Kentucky, and John Cornyn,
a Republican from Texas who's one of his longtime deputies ---

\begin{itemize}
\tightlist
\item
  archived recording (john cornyn)\\
  But as the impact of Covid-19 has grown, so has the need for
  assistance.
\end{itemize}

nick fandos

--- seem to recognize that not only are the fates of individual senators
up in the air, but the Republican Party's prospects up and down the
ticket this fall may well be tied up into how they are judged to have
handled this crisis. And doing what the conservatives want and basically
stopping now and saying, ``we've done what we need to do'' is not an
option for that group.

michael barbaro

Nick, how much of that debate you just described is being informed by
the political realities surrounding the single most important person in
the party at this moment, which is President Trump?

nick fandos

I think it's inescapable for elected Republicans. And it's not just the
way that the public seems to be viewing President Trump and giving him
very poor grades on handling the pandemic, which could hurt the whole
Republican Party in November. It's also the kind of erratic nature of
his leadership and engagement on this issue itself. And so they're
working with his Treasury secretary to iron out the details. But this is
not a negotiation that President Trump is leading or even all that
active in. They're trying to do whatever they can to bail out the party,
not to please President Trump in this case.

michael barbaro

Hm.

nick fandos

And that has added another kind of layer of interest and
unpredictability to this whole thing which, you know, we have not seen a
lot of in the last three and a half years.

michael barbaro

And what does that tell you, that they're choosing this moment to do
that?

nick fandos

Well, I think whether they want to acknowledge it or not, Republicans
are starting to sense that their party is really in trouble. That if
things aren't turned around quickly, they may not only lose the White
House, but really get wiped out in November. And are thinking in
different ways about why that is and what the party may need to look
like in a world that's just starting to dawn on them as a possibility of
being kind of post-Trump.

michael barbaro

So in other words, this battle over \$600 a week and what this entire
new version of a relief package looks like, it's not really just about
what's in a piece of legislation like this. It's about the identity of
the Republican Party at a time where it may need a new identity. Because
theoretically, Donald Trump could lose. And the Republican Party would
no longer be just the party of Donald Trump.

nick fandos

That's right. So while they're very much focused on how is the party
going to be viewed in November, they're really kind of foreshadowing or
staking out positioning for this potentially larger battle to come, over
what Republicanism really looks like after Donald Trump has defined it
for four or five years.

{[}music{]}

And you know, some of these folks are not new to their positions. But
they recognize that there may soon be more of a need to kind of assert
their views, and the primacy of those views, against others in the
Republican Party.

michael barbaro

We'll be right back.

{[}music{]}

So Nick, where does this very high stakes ideological battle within the
Republican Party, where does it leave this economic relief package?

nick fandos

So it's up to Mitch McConnell, basically, to try and pull together these
different factions and arrive at a bill that deals with the expiring
unemployment benefits and a host of other kind of programs and
priorities. Basically, to try and reconcile those differences and put
together a bill that can be Republicans' starting point when they go to
the negotiating table with Democrats.

michael barbaro

Mhm.

nick fandos

And so that's where we were by the middle of last week. And as he tries
to work out those details with the White House and run it by his
Republican colleagues, there's a bunch of snafus along the way. They
push past some small deadlines. But in the end, they're unable to
introduce their bill, because those differences turn out to have been
more significant than Republicans even wanted to let on.

michael barbaro

So the Republicans cannot come up with any kind of consensus bill to
salvage this program that we've been talking about?

nick fandos

So as of Thursday morning, no. And as lawmakers head for the exits for
the weekend, without a proposal for how to fix a whole host of programs,
they have not arrived at a solution on a range of issues, including what
to do about this expiring \$600 unemployment benefit. But their staff
and Treasury Secretary Mnuchin, Meadows, the White House chief of staff,
work through the weekend to try and iron out some of these details.

\begin{itemize}
\tightlist
\item
  archived recording (mitch mcconnell)\\
  Well, good afternoon, everyone. The Senate Republicans and the
  administration have been consulting over the last few weeks.
\end{itemize}

nick fandos

By Monday afternoon, what they finally introduce ---

\begin{itemize}
\tightlist
\item
  archived recording (mitch mcconnell)\\
  --- with what we think is an appropriate amount of additional debt to
  be added. We think it is about a trillion dollars.
\end{itemize}

nick fandos

--- is a plan that is roughly a trillion dollars.

\begin{itemize}
\tightlist
\item
  archived recording (mitch mcconnell)\\
  And we've allocated that in a way that we think makes the most sense.
\end{itemize}

nick fandos

Some of that goes to schools to help them reopen and for more testing
and contact tracing.

\begin{itemize}
\tightlist
\item
  archived recording (mitch mcconnell)\\
  So with that, I'm going to call on my colleagues who have developed
  the various ---
\end{itemize}

nick fandos

And on this key question of unemployment benefits, Republicans propose a
real overhaul to the way that they would work conceptually.

\begin{itemize}
\item
  archived recording (mitch mcconnell)\\
  Do we know who's next?
\item
  archived recording\\
  Chairman Grassley.
\item
  archived recording (mitch mcconnell)\\
  Senator Grassley.
\item
  archived recording (chuck grassley)\\
  Number one, we're going to continue ---
\end{itemize}

nick fandos

So they say that for the short term, we're going to cut that \$600 down
to \$200 a week.

michael barbaro

Big cut.

nick fandos

A pretty dramatic cut.

\begin{itemize}
\tightlist
\item
  archived recording (chuck grassley)\\
  So we want to continue to help the unemployed. But we want to
  encourage work. And we've learned a very tough lesson, that when you
  pay people not to work, what do you expect?
\end{itemize}

nick fandos

And they say, that's just going to buy us time over the next few months
for us to basically help states set up a new system, where what we're
going to try and do is make sure that every individual that's
unemployed, between the state government and the federal government ends
up getting about 70 percent of what their old wages would have been.

\begin{itemize}
\tightlist
\item
  archived recording (chuck grassley)\\
  We're going to have further tax relief for businesses to encourage
  hiring and rehiring. And we want to do that to encourage people to get
  back to work and help the employer, in the process, support people in
  the meantime.
\end{itemize}

nick fandos

And so what Republicans are trying to do here is keep a safety net in
place, but remove what they think is hindering people from going back to
work.

\begin{itemize}
\tightlist
\item
  archived recording (chuck grassley)\\
  Lastly, I hope that Democrats will come to the table and we can work
  out a bipartisan agreement. Thank you very much.
\end{itemize}

nick fandos

So in other words, if they can get this program up and operating, it
will always make sense from a financial point of view for somebody to go
and take their old job back or take a new job back, but not be so
draconian that they're making the economic situation drastically worse,
or can be accused of forcing people towards soup kitchens or the
streets.

michael barbaro

So this is a classic compromise. In other words, we're going to keep the
benefits but not at \$600 a week, because they see that as not
conservative and not incentivizing an economic recovery.

nick fandos

That's right. But remember, this is just kind of the first step. This
should have been the easy part for Republicans. Because what they have
coming is negotiations with Democrats, who are in favor of keeping the
benefit totally as it is, and are already lining up to say basically
that Republicans are giving a massive economic financial hit to
individuals and the economy right when they need it most, and at this
moment where the country's recovery seems to be teetering. Is it going
to keep going up? Or is it about to collapse again? And Democrats are
not going to settle for \$200 for any period of time.

michael barbaro

So given all that, what is likely to happen to this Republican bill in
the Senate?

nick fandos

So the interesting thing about where Republicans find themselves is,
this bill that they're introducing probably couldn't even pass the
Senate just on Republican votes. And that leaves them in a pretty weak
position as they head into negotiations with the Democrats. Because
remember, to pass anything into law, even if there's a Republican
president or a Republican Senate, you need the Democrats to get it
through Congress. And they have a very long and expensive wish list of
things that they'd like to see in legislation. And they're not going to
be easy on the Republicans.

michael barbaro

Nick, this may sound like a strange question. But do you think
Republicans now regret ever agreeing to these enhanced unemployment
benefits? I'm mindful of the fact that it was not a Republican idea. It
was Democrats who pushed for it. As you have said, it cuts against a lot
of Republican principles. But they agreed to it as a short-term fix. And
it turns out it's not going to be a short-term term fix, because there's
nothing short-term about this pandemic. And it is inevitably hard to
take something like this away from people once you give it to them. So
is it possible Republicans look back and think we should have never
agreed to do this?

nick fandos

I think there may be a small subset of fiscally conservative Republicans
that feel that way. But my guess is that the vast majority felt like,
hey, we did what we had to do back then in the springtime. I mean, the
economy was in freefall, remember. And the course of the virus was
highly uncertain. And the fundamental problem for them is that they
envisioned the federal government having a relatively short-term role to
play in getting the country back on its feet and ready to fight against
this virus. And it's just turned out to be, for a lot of different
reasons, a much more complicated, prolonged, expensive fight than they
wanted.

And honestly, Michael, at this point, it's hard to see how this
situation resolves itself. Usually, when you cover Congress for a while,
you can kind of see the pattern of how these negotiations will work. But
Republicans really find themselves pretty far up the stream without a
paddle right now. And there seem to be risks for them and consequences
in every direction.

And it's going to be a pretty fascinating next couple of weeks to see
how and if they can reach an agreement with Democrats --- and one that
some members of the party feel like doesn't completely undermine what
they stand for.

michael barbaro

Of course, weeks is not what people who are on this program have. They
have days. Because this thing really does expire on Friday.

nick fandos

That's right. Many of the people receiving these benefits are living
paycheck to paycheck or don't have a lot of savings to fall back on.
There can and will be very real consequences to this delay. And that's
not to mention the whole host of other programs that are being debated
by Congress right now that are touching different aspects of people's
lives.

{[}music{]}

The longer this goes on, the effects just get magnified. Bigger and
bigger and bigger. And it frankly makes the problem even harder to
solve.

michael barbaro

Thank you, Nick.

nick fandos

Thank you, Michael.

michael barbaro

On Monday night, Democratic leaders, including House Speaker Nancy
Pelosi, met with White House officials to begin negotiations over a new
economic relief package, including federal unemployment benefits.

\begin{itemize}
\tightlist
\item
  archived recording (nancy pelosi)\\
  Suffice to say that we hoped that we would be able to reach an
  agreement. We clearly do not have shared values.
\end{itemize}

michael barbaro

Little progress was made during the two-hour session. But afterward, the
Democratic leaders made one thing clear. Congressional Republicans lack
the votes to pass their own bill.

We'll be right back.

Here's what else you need to know today.

On Monday, the pandemic touched the worlds of politics, business and
sports. The Trump administration said that its national security
adviser, Robert O'Brien, had contracted the virus, becoming the most
senior White House official yet to test positive.

Meanwhile, the parent company of Google --- Alphabet --- told employees
that they would not be expected to return to the office until next
summer, suggesting that work-from-home policies will extend well past
the end of the year.

Finally, the Miami Marlins canceled two upcoming baseball games after 12
players and two coaches tested positive for the coronavirus. The
outbreak was disclosed just four days after the beginning of the
baseball season.

\begin{itemize}
\tightlist
\item
  archived recording (dave martinez)\\
  My level of concern went from about an eight to a 12. You know, it
  hits home now that you see half a team get infected and it go from one
  city to another. So ---
\end{itemize}

michael barbaro

During a news conference, the manager of the Washington Nationals
expressed alarm over the news.

\begin{itemize}
\tightlist
\item
  archived recording (dave martinez)\\
  Yeah, I got friends on that Miami team. And it really stinks. Now I'm
  not going to lie. I'm not going to sugarcoat it. Seeing those guys go
  down like that, it's not good for them. It's not good for anybody.
\end{itemize}

michael barbaro

That's it for ``The Daily.'' I'm Michael Barbaro. See you tomorrow.

\hypertarget{-3}{%
\subsection{}\label{-3}}

There's a wrinkle in stores' mask policies: Enforcement.

Image

Acme markets' parent company said it was not insisting on masks ``to
avoid conflicts that would put the store director or other employees and
customers at risk.''Credit...Natalie Keyssar for The New York Times

Big retailers have made strong statements recently about their new rules
requiring customers to
\href{https://www.nytimes.com/article/which-stores-require-masks.html?searchResultPosition=3}{wear
face masks} when shopping, saying that the health of their workers and
customers is paramount. But the companies are taking a decidedly
hands-off approach to enforcing those mandates.

Walmart has told employees that they should not prevent customers from
entering the store if they refuse to wear a mask. Walgreens said that
``for the safety of our team members,'' the company would not bar
customers without masks from its stores. Lowes also said it would ``not
ask our associates to put their safety at risk by confronting customers
about wearing masks.''

Many shoppers and workers say the retailers' reluctance to police mask
wearing
\href{https://www.nytimes.com/2020/07/29/business/coronavirus-masks-stores-walmart.html}{ultimately
renders their rules toothless}, and will perpetuate the spread of the
coronavirus. And workers find themselves thrust onto the front line of a
cultural and political war
\href{https://www.nytimes.com/2020/05/03/us/coronavirus-masks-protests.html}{over
masks} that can lead to
\href{https://www.nytimes.com/2020/05/15/us/coronavirus-masks-violence.html?searchResultPosition=105}{ugly
confrontations} and sometimes violence.

Last weekend, two episodes stood out. In one, a video of an altercation
involving two shoppers in Walmart wearing masks with a Nazi swastika
went viral. In the other, a man
\href{https://www.facebook.com/palmbeachcountysheriff/}{was arrested}
after he pulled a gun on another shopper who had asked him to put on his
mask in a Walmart in Palm Beach County, Fla.

Stuart Appelbaum, the president of the Retail, Wholesale, and Department
Store Union, representing workers at Macy's and Bloomingdales in New
York, said retailers needed to invest in more security guards or empower
management to confront shoppers, not leave it up to rank-and-file
workers. But not enforcing the rules, when they are challenged, was not
effective, he said.

``A rule that isn't enforced,'' Mr. Appelbaum said, ``is not a rule.''

\href{https://www.nytimes.com/interactive/2020/world/coronavirus-maps.html}{}

\includegraphics{https://static01.nyt.com/images/2020/03/03/world/coronavirus-map-promo/coronavirus-map-promo-articleLarge-v661.png}

\hypertarget{coronavirus-map-tracking-the-global-outbreak}{%
\subsection{Coronavirus Map: Tracking the Global
Outbreak}\label{coronavirus-map-tracking-the-global-outbreak}}

The virus has infected more than 16,789,000 people and has been detected
in nearly every country.

\hypertarget{-4}{%
\subsection{}\label{-4}}

The chairman of a House panel examining the U.S. virus response accuses
the White House of suppressing reports.

Image

The chairman of the House select committee investigating the
government's coronavirus response is accusing the White House of
suppressing its own dire state-by-state assessments of the virus's
spread and keeping science-based public health recommendations a secret
as Mr. Trump insists the pandemic is under control.

The chairman, Representative Jim Clyburn, Democrat of South Carolina,
\href{https://int.nyt.com/data/documenttools/clyburn-letter-to-pence/5eaf7827a6dbb331/full.pdf}{sent
a letter Wednesday} to the White House Coronavirus Task force, demanding
that it make its internal assessments public. On Tuesday, The New York
Times
\href{https://www.nytimes.com/interactive/2020/07/28/us/states-report-virus-response-july-26.html}{published
the most recent task force report}, which identified 21 ``red zone''
states and offered public health guidance like imposing statewide mask
orders or close bars and gyms.

(Read
\href{https://int.nyt.com/data/documenttools/clyburn-letter-to-pence/5eaf7827a6dbb331/full.pdf}{the
letter to Vice President Mike Pence}.)

``We are primarily concerned right now with the difference that seems to
be existing between what the White House is saying publicly and what it
is saying and doing privately,'' Mr. Clyburn said in an interview,
adding, ``Covid-19 is recognized by this White House as being much more
serious in their private dealing with it.''

Mr. Clyburn also sent letters to the Republican governors of four ``red
zone'' states --- Tennessee, Florida, Georgia and Oklahoma --- asking
them to produce internal correspondence with the administration, as well
as proof that they are following the task force's recommendations. The
letter sent to the task force was addressed to Vice President Mike Pence
and Dr. Deborah L. Birx, the administration's coronavirus response
coordinator.

``This unpublished report recommends far stronger public health measures
than the Trump Administration has called for in public --- including
requiring face masks, closing bars, and strictly limiting gatherings,''
Mr. Clyburn wrote. ``Yet many states do not appear to be following these
unpublished recommendations and are instead pursuing policies more
consistent with the Administration's contradictory public statements.''

Mr. Clyburn does not have the power to compel the documents, unless he
issues a subpoena --- and even then, the Trump White House has ignored
such legally binding requests. Mr. Clyburn stopped short of saying he
would subpoena the documents, but his committee, created by Speaker
Nancy Pelosi, has broad authority to investigate the government's
response and will hear from three top health officials, including Dr.
Anthony S. Fauci at a hearing on Friday. Dr. Birx is not scheduled to
testify.

\hypertarget{-5}{%
\subsection{}\label{-5}}

A study asserts school closures in the spring saved lives. Experts
caution about applying the findings now.

Image

An empty elementary school classroom in Maryland in April.Credit...Erin
Schaff/The New York Times

In a new analysis, pediatric researchers have estimated that the states'
decisions to close schools last spring
\href{https://www.nytimes.com/2020/07/29/health/covid-school-reopening.html}{likely
saved tens of thousands of lives from Covid-19} and prevented many more
coronavirus infections. Still, the authors acknowledged that their
findings are not broadly applicable today.

The findings come amid a worldwide debate on whether, when and how to
reopen schools, including for some 56 million American students,
kindergarten through high school. Outside experts cautioned that the
effect of school closings is extremely difficult to predict because of
unknowns regarding how infectious children are and because of the
difficulty in separating out the effect of school closures from other
measures that states took to control the virus, like stay-at-home
orders, business closures and limits on large social gatherings.

In addition, early in the pandemic, testing was especially limited and
spotty, raising questions about how well the number of confirmed cases
reflected actual infections.

The
study,\href{https://jamanetwork.com/journals/jama/fullarticle/10.1001/jama.2020.14348}{published
Wednesday in JAMA}, focused on a six-week period in the spring, when
there were still many unknowns.

``At the time, there wasn't any masking in schools, there wasn't
physical distancing, there wasn't an increase in hygiene and that sort
of thing,'' said Dr. Katherine Auger, an associate professor of
pediatrics at Cincinnati Children's Hospital and the lead author of the
study.

Some experts expressed concern that the study's estimates about the
impact of closing schools early in the pandemic would be seized upon as
an argument that schools should remain closed. Experts on public health
and education have recommended that communities and schools should work
toward reopening with strong health precautions in place, because
in-person schooling has such tremendous value for children's academic,
social and emotional development.

``I do worry that these large estimates of the effect of school closures
will lead people to give up because it is going to be challenging to
open schools,'' said Julie Donohue, a professor of public health at the
University of Pittsburgh who
co-wrote\href{https://jamanetwork.com/journals/jama/fullarticle/10.1001/jama.2020.13092}{an
editorial about the study}. ``I do worry that some districts will look
at these numbers and say, well, it's just too hard and it's not safe to
reopen.''

U.S. Roundup

\hypertarget{-6}{%
\subsection{}\label{-6}}

More than 6,300 cases have been linked to U.S. colleges.

Image

A Times survey of every public four-year college in the country, as well
as every private institution that competes in Division I sports or is a
member of an
\href{https://www.aau.edu/sites/default/files/AAU-Files/Who-We-Are/AAU-Member-List.pdf}{elite
group of research universities}, revealed at least 6,300 cases tied to
about 270 colleges over the course of the pandemic. And the new academic
year has not yet begun at most schools.

There is no standardized reporting method for cases and deaths at
colleges, and the information is not being publicly tracked at a
national level. Of nearly 1,000 institutions contacted by The Times,
some had already posted case information online, some provided full or
partial numbers and others refused to answer basic questions, citing
privacy concerns. Hundreds of colleges did not respond at all.

Still, the Times survey represents the most comprehensive look at the
toll the virus has taken on the country's colleges and universities.

Among the colleges that provided information, many offered no details
about who contracted the virus, when they became ill or whether a case
was connected to a larger outbreak. It is possible that some of the
cases were identified months ago, in the early days of the U.S. outbreak
before in-person learning was cut short, and that others involved
students and employees who had not been on campus recently. Here's a
look at other developments from around the U.S.:

\begin{itemize}
\tightlist
\item
  After calling off plans to give his convention speech next month in
  Jacksonville, Fla., because of the pandemic, Mr. Trump said Wednesday
  he may give it from the \textbf{White House} instead. ``It's something
  we're thinking about,'' Mr. Trump told reporters.
  \href{https://www.nytimes.com/2020/07/29/us/elections/biden-vs-trump.html?action=click\&module=Top\%20Stories\&pgtype=Homepage\#link-7eba6945}{Legal
  experts said that such a plan could run afoul of the Hatch Act}, which
  prohibits federal employees from engaging in political activities
  while on the job.
\end{itemize}

\begin{itemize}
\tightlist
\item
  In a move long sought by advocates, \textbf{California} has stepped up
  its efforts to track whether the virus is affecting L.G.B.T.Q. people
  at disproportionate rates. State health officials announced Tuesday
  that health care providers and labs would be required to collect and
  report to the state data that patients give voluntarily about their
  gender identity and sexual orientation, in addition to their age and
  ethnicity.
\end{itemize}

\hypertarget{-7}{%
\subsection{}\label{-7}}

Assessing the virus in the United States: The epidemic is splintering
into deadly pieces.

Image

The coronavirus has infected at least 4.3 million people in the United
States, killing more than 150,000.Credit...Jenna Schoenefeld for The New
York Times

Once again, the coronavirus is ascendant. As
\href{https://www.nytimes.com/interactive/2020/us/coronavirus-us-cases.html}{infections
mount across the country}, it is dawning on Americans that the epidemic
is now unstoppable, and that no corner of the nation will be left
untouched.

The pathogen has infected at least 4.3 million Americans, killing over
150,000. Many experts fear the virus could kill
\href{https://www.forbes.com/sites/mattperez/2020/07/07/imhe-model-projects-208255-us-deaths-by-november-but-estimate-falls-sharply-if-mask-use-increases/\#3c8ee9616f2e}{200,000}or
\href{https://www.cnbc.com/2020/07/22/dr-scott-gottlieb-us-coronavirus-deaths-may-hit-300000-by-year-end.html}{even
300,000} by year's end. Even Mr. Trump has donned a mask, after
resisting for months, and has
\href{https://www.nytimes.com/2020/07/23/us/politics/jacksonville-rnc.html}{canceled
the Republican National Convention} celebrations in Florida.

Each state, each city has its own crisis driven by its own risk factors:
vacation crowds in one, bars reopened too soon in another, a revolt
against masks in a third.

``We are in a worse place than we were in March,'' when the virus
coursed through New York, said
\href{https://www.gwumc.edu/smhs/facultydirectory/profile.cfm?empName=Leana\%20Wen\&FacID=2073685428}{Dr.
Leana S. Wen}, a former Baltimore health commissioner. ``Back then we
had one epicenter. Now we have lots.''

To assess where the country is heading now,
\href{https://www.nytimes.com/2020/07/29/health/coronavirus-future-america.html}{The
New York Times interviewed 20 public health experts} --- clinicians and
epidemiologists, historians and sociologists, because the spread of the
virus is now influenced as much by human behavior as it is by the
pathogen.

Over all, the scientists conveyed a pervasive sense of sadness and
exhaustion. Where
\href{https://www.nytimes.com/2020/03/22/health/coronavirus-restrictions-us.html}{once
there was defiance},
\href{https://www.nytimes.com/2020/04/18/health/coronavirus-america-future.html}{and
then a growing sense of dread}, now there seems to be sorrow and
frustration, a feeling that so many funerals never had to happen and
that nothing was going well.

``We're all incredibly depressed and in shock at how out of control the
virus is in the U.S.,'' said
\href{https://profiles.stanford.edu/michele-barry}{Dr. Michele Barry},
the director of the Center for Innovation in Global Health at Stanford
University.

\hypertarget{-8}{%
\subsection{}\label{-8}}

Latin America is facing `a decline of democracy' under the pandemic.

Image

A mural in downtown Caracas shows former leaders Hugo Chávez and Simón
Bolívar alongside President Nicolás Maduro. Mr. Maduro has cracked down
on dissent during the pandemic.Credit...Adriana Loureiro Fernandez for
The New York Times

Postponed elections. Sidelined courts. A persecuted opposition.

As the virus tears through Latin America and the Caribbean, killing more
than 180,000 and destroying the livelihoods of tens of millions in the
region,
\href{https://www.nytimes.com/2020/07/29/world/americas/latin-america-democracy-pandemic.html}{it
is also undermining democratic norms} that were already under strain.

Leaders from the center right to the far left have seized on the crisis
to extend their time in office, weaken oversight of government actions
and silence critics --- actions that under different circumstances would
be described as authoritarian and antidemocratic but that now are being
billed as lifesaving measures to curb the spread of the disease.

``It's not a matter of left or right, it's a general decline of
democracy across the region,'' said Alessandra Pinna, a Latin America
researcher at Freedom House, an independent Washington-based research
organization that measures global political liberties.

President Nicolás Maduro of
\href{https://www.nytimes.com/2020/06/19/world/americas/venezuela-forced-disappearances-Maduro.html}{Venezuela
has detained or raided the homes} of dozens of journalists, social
activists and opposition leaders for questioning the government's
dubious virus figures.

In Nicaragua, President Daniel Ortega released thousands of inmates
because of the threat posed by the virus, but
\href{https://www.barrons.com/news/nicaragua-excludes-political-prisoners-from-mass-release-01586430304}{kept
political prisoners} behind bars. In Guyana, a lockdown prevented
protests against the government's attempt to stay in power despite
losing an election.

And in Bolivia, a caretaker government has used the pandemic to postpone
elections, tap into emergency aid to bolster its electoral campaign and
threaten to ban the main opposition candidate from running.

The gradual undermining of democratic rules during an economic crisis
and public health catastrophe could leave Latin America primed for
slower growth and an increase in corruption and human rights abuses,
experts warned.

\hypertarget{-9}{%
\subsection{}\label{-9}}

New York City praises its contact-tracing program. Workers call the
rollout `a disaster.'

Image

El, who worked as a contact tracer in New York, said, ``I have never had
a more dysfunctional workplace.''Credit...Hiroko Masuike/The New York
Times

New York City's contact-tracing program
\href{https://www.nytimes.com/2020/07/29/nyregion/new-york-contact-tracing.html}{seems
to have been especially plagued by problems}.

Only a few weeks into the rollout of the city's much-heralded program,
which began on June 1 and was a vital initiative in the effort to
contain the virus and to reopen the local economy, the newly hired
contact tracers were already expressing growing misgivings about their
work.

One said the city was ``putting out propaganda'' about the program's
effectiveness.

Another wrote, ``I don't think this is the type of job we should just
`wing it,' and that's the sense I've been getting sometimes.''

Mayor Bill de Blasio has declared that the city's new Test and Trace
Corps, which has hired about 3,000 contact tracers, case monitors and
others, will make a difference in curbing the virus now that the
outbreak that devastated New York in the spring has waned.

The de Blasio administration acknowledged that the program had gotten
off to a troubled start, but said that improvements had been made.

``All signs indicate that the program has been effective in helping the
city avoid the resurgence we're seeing in other states,'' Avery Cohen, a
spokeswoman for the mayor, said.

Still, some contact tracers described the program's first six weeks as
poorly run and disorganized, leaving them frustrated and fearful that
their work would not have much of an impact.

They spoke of a confusing training regimen and priorities, and of newly
hired supervisors who were unable to provide guidance. They said
computer problems had sometimes caused patient records to disappear. And
they said their performances were being tracked by call-center-style
``adherence scores'' that monitor the length of coffee breaks but did
not account for how well tracers were building trust with clients.

Elsewhere in New York:

\begin{itemize}
\tightlist
\item
  The state's Department of Motor Vehicles will allow driving schools,
  starting on Wednesday, to conduct remote learning for pre-license
  driving courses, Gov. Andrew M. Cuomo said. Driving schools can hold
  courses over video chat programs like Zoom and Skype.
\end{itemize}

\hypertarget{-10}{%
\subsection{}\label{-10}}

A Fed meeting could provide fresh clues on how policymakers see the
economy.

Image

On Tuesday the Fed extended its emergency lending programs through the
end of 2020.Credit...T.J. Kirkpatrick for The New York Times

Federal Reserve officials will conclude a two-day policy meeting on
Wednesday that is likely to yield little action --- rates are already at
near-zero and are almost certain to stay there for an extended period
--- but could provide a fresh read on how policymakers are thinking
about the economic outlook, and hints about their plans.

On Tuesday the Fed
\href{https://www.nytimes.com/2020/07/28/business/economy/coronavirus-federal-reserve-policy.html}{extended
its emergency lending programs} through the end of 2020, a three-month
addition that, while not surprising, signaled how lasting the economic
damage from the coronavirus is proving.

The chair, Jerome H. Powell, who will hold a remote news conference at
2:30 p.m., is sure to field questions on the newly extended programs,
which were introduced to try to keep markets functioning and credit
flowing.

The Fed took unprecedented actions in March and April to provide a first
line of defense for the economy as coronavirus cases swept the nation
and shut down entire business sectors. Most of
\href{https://www.nytimes.com/2020/03/23/business/economy/coronavirus-fed-bond-buying.html}{the
nine programs} were set to expire on or around the end of September, a
sign that officials thought normal conditions might return by fall.

That optimism has been upended by a surge in infections, which has
continued to depress economic activity. While state and local economies
have reopened, many have had to roll back or delay their plans, and
experts warn that the situation could worsen if the virus takes hold
more deeply.

Here's what else is happening in the business world:

\begin{itemize}
\item
  U.S.
  \href{https://www.nytimes.com/live/2020/07/29/business/stock-market-today-coronavirus}{stocks
  ticked higher} and global markets were mixed on Wednesday as investors
  waded through corporate earnings reports.
\item
  \href{https://www.nytimes.com/live/2020/07/29/business/stock-market-today-coronavirus}{Boeing
  lost \$2.4 billion} in the second quarter, the company said Wednesday,
  adding that it plans to slow plane production and could cut more jobs
  as it reels from the grounding of the 737 Max and the devastating
  aviation slowdown brought on by the pandemic.
\end{itemize}

global roundup

\hypertarget{-11}{%
\subsection{}\label{-11}}

Pilots want FedEx to suspend Hong Kong operations, citing `difficult'
quarantine requirements.

Image

A FedEx pilot tested positive in Hong Kong on July 11, after visiting a
popular restaurant.Credit...Paul J. Richards/Agence France-Presse ---
Getty Images

A union representing FedEx pilots called on the delivery company on
Tuesday to suspend operations in Hong Kong after its members were
subject to quarantine facilities under ``extremely difficult
conditions.''

Hong Kong began testing all airline workers who were previously exempt
from mandatory coronavirus tests this month, prompting United Airlines
and American Airlines to suspend flights to the city. A FedEx pilot who
had arrived from the United States and visited a popular restaurant
tested positive on July 11.

The Air Line Pilots Association International said on Tuesday that three
FedEx pilots who had tested positive for the coronavirus but were
asymptomatic were ``forced into mandated hospital facilities.'' Those
who tested negative but had been in close contact with an infected
person ``were put into government camps under extremely difficult
conditions.''

``Pilots who test positive for Covid-19 face compulsory admission and
treatment in government-selected public hospitals, with as many as five
patients to a room with one shared bathroom,'' the union
\href{https://www.alpa.org/news-and-events/news-room/2020-07-28-fedex-pilots-face-unacceptable-conditions-in-hong-kong}{said
in a statement}.

``Not only do these situations pose unacceptable risks to our pilots'
safety and well-being, but they also create added stress and distraction
for flight operations,'' it added.

Hong Kong has had the same quarantining and hospitalization requirements
for residents.

The semiautonomous Chinese territory is fighting its biggest surge in
coronavirus infections yet, reporting more than 100 new cases in each of
the past seven days. Health officials believe the spike was caused by
people who had been exempted from quarantine rules to help boost the
economy, including airline workers, seafarers and business executives.

Hong Kong planned to tighten testing and quarantine arrangements for air
and sea crew members starting on Wednesday.

Reports about Hong Kong's quarantine facilities have varied. Some camps
have been compared to a
\href{https://edition.cnn.com/2020/04/09/homepage2/hong-kong-coronavirus-quarantine-diary-intl-hnk/index.html}{``cozy
university dorm''} with new Ikea furniture, but people in others have
complained about
\href{https://hongkongfp.com/2020/04/18/coronavirus-hong-kong-quarantine-arrivals-complain-of-disorderly-unsanitary-govt-facilities/}{unsanitary
and moldy environments}.

On Wednesday, Carrie Lam, the city's top leader, warned that the sharp
rise in infections could lead to a ``collapse'' of the hospital system.
Health officials reported 118 new cases on Wednesday, bringing the total
tally past 3,000.

Here are other developments from around the globe:

\begin{itemize}
\tightlist
\item
  In \textbf{Vietnam}, the largest country in the world without a single
  confirmed fatality from the virus, a 100-day streak with no reported
  local transmissions
  \href{https://www.nytimes.com/2020/07/29/world/asia/coronavirus-vietnam.html}{was
  broken after a weekend outbreak}. Officials said on Wednesday that
  cases had been discovered in Hanoi, Ho Chi Minh City and two central
  provinces. Hours after clusters of cases were confirmed in Danang
  hospitals earlier this week, officials said they would be shutting the
  city's airport and up to 80,000 local tourists who had traveled to the
  city would be evacuated. \textbf{Japan, China, Australia and South
  Korea}, all of which seemed to have their outbreaks reasonably under
  control, also recorded dramatic spikes on Wednesday.
\end{itemize}

\begin{itemize}
\item
  Across the \textbf{Middle East}, celebrations for Eid al-Adha, the
  festival of sacrifice that marks the end of the hajj this weekend,
  will be tamer this year. About 2.5 million Muslims from around the
  world performed the pilgrimage to Mecca last year. This year,
  \textbf{Saudi Arabia} said it would allow as few as 1,000 pilgrims,
  all from within the kingdom.
\item
  The agriculture minister of \textbf{Zimbabwe}, Perrance Shiri, who led
  a military unit that massacred thousands of civilians during civil
  strife in the 1980s and helped plot the coup that overthrew the
  country's longtime strongman leader,
  \href{https://www.nytimes.com/2019/09/06/obituaries/robert-mugabe-dead.html}{Robert
  Mugabe}, in 2017, has died of coronavirus, according to local media
  reports. Mr. Shiri was 55, and was thought to have contracted the
  virus from his driver, who also died recently.
\item
  The federal government in \textbf{Australia} said this week that it
  would send a specialist medical team usually deployed to disaster
  zones to help manage an outbreak in the state of Victoria. The state
  of Queensland said it would bar entry to travelers from Sydney and
  surrounding regions in New South Wales after recording new cases from
  travelers who had passed through the city.
\item
  Sending patients from hospitals to nursing homes to free up hospital
  beds early in the pandemic has been described as ``reckless'' by
  lawmakers in \textbf{Britain},
  \href{https://www.bbc.com/news/uk-politics-53574265}{the BBC reports}.
  The
  \href{https://www.nytimes.com/2020/05/25/world/europe/coronavirus-uk-nursing-homes.html}{death
  toll in British care homes} has been a defining scandal of the
  pandemic for Prime Minister Boris Johnson.
\item
  Drinkers in Ottawa, the capital of \textbf{Canada}, now must make
  reservations for seats on patios. The measure was introduced after Dr.
  Vera Etches, the city's medical officer of health, expressed concern
  that a rise in cases among people in their 20s was partly related to
  long lines outside bars.
\end{itemize}

\hypertarget{-12}{%
\subsection{}\label{-12}}

Want tips on how to talk about money?

Talking about money is always difficult, but new financial hardships may
be hitting those closest to you, making these conversations all the more
important. It doesn't have to be awkward.

Reporting was contributed by Ian Austen, Hannah Beech, **** Pam Belluck,
Nicholas Bogel-Burroughs, Weiyi Cai, Julia Calderone, Benedict Carey,
Michael Cooper, Michael Corkery, Chau Doan, **** Nicholas Fandos, Lauryn
Higgins, Danielle Ivory, Anatoly Kurmanaev, Isabella Kwai, Alex
Lemonides, Donald G. McNeil Jr., Claire Moses, Jeffrey Moyo, Sharon
Otterman, Amanda Rosa, Jeanna Smialek, Mitch Smith, Eileen Sullivan,
Neil Vigdor, and Elaine Yu.

Advertisement

\protect\hyperlink{after-bottom}{Continue reading the main story}

\hypertarget{site-index}{%
\subsection{Site Index}\label{site-index}}

\hypertarget{site-information-navigation}{%
\subsection{Site Information
Navigation}\label{site-information-navigation}}

\begin{itemize}
\tightlist
\item
  \href{https://help.nytimes.com/hc/en-us/articles/115014792127-Copyright-notice}{©~2020~The
  New York Times Company}
\end{itemize}

\begin{itemize}
\tightlist
\item
  \href{https://www.nytco.com/}{NYTCo}
\item
  \href{https://help.nytimes.com/hc/en-us/articles/115015385887-Contact-Us}{Contact
  Us}
\item
  \href{https://www.nytco.com/careers/}{Work with us}
\item
  \href{https://nytmediakit.com/}{Advertise}
\item
  \href{http://www.tbrandstudio.com/}{T Brand Studio}
\item
  \href{https://www.nytimes.com/privacy/cookie-policy\#how-do-i-manage-trackers}{Your
  Ad Choices}
\item
  \href{https://www.nytimes.com/privacy}{Privacy}
\item
  \href{https://help.nytimes.com/hc/en-us/articles/115014893428-Terms-of-service}{Terms
  of Service}
\item
  \href{https://help.nytimes.com/hc/en-us/articles/115014893968-Terms-of-sale}{Terms
  of Sale}
\item
  \href{https://spiderbites.nytimes.com}{Site Map}
\item
  \href{https://help.nytimes.com/hc/en-us}{Help}
\item
  \href{https://www.nytimes.com/subscription?campaignId=37WXW}{Subscriptions}
\end{itemize}
