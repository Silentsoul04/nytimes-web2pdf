Sections

SEARCH

\protect\hyperlink{site-content}{Skip to
content}\protect\hyperlink{site-index}{Skip to site index}

\hypertarget{comments}{%
\subsection{\texorpdfstring{\protect\hyperlink{commentsContainer}{Comments}}{Comments}}\label{comments}}

\href{}{Destroying a Way of Life to Save Louisiana}\href{}{Skip to
Comments}

The comments section is closed. To submit a letter to the editor for
publication, write to
\href{mailto:letters@nytimes.com}{\nolinkurl{letters@nytimes.com}}.

\hypertarget{destroying-a-way-of-life-to-save-louisiana}{%
\section{Destroying a Way of Life to Save
Louisiana}\label{destroying-a-way-of-life-to-save-louisiana}}

By Nathaniel RichJuly 21, 2020

\begin{itemize}
\item
\item
\item
\item
\item
  \emph{77}
\end{itemize}

The state's \$50 billion plan to re-engineer its coastline may wash some
fishing communities off the map.

\hypertarget{destroying-a-way-of-life-to-save-louisiana-1}{%
\section{Destroying a Way of Life to Save
Louisiana}\label{destroying-a-way-of-life-to-save-louisiana-1}}

The state's \$50 billion plan to re-engineer its coastline may wash some
fishing communities off the map.

By Nathaniel Rich

July 21, 2020

SHARE

\hypertarget{listen-to-this-article}{%
\subsection{Listen to This Article}\label{listen-to-this-article}}

\emph{To hear more audio stories from publishers like The New York
Times,}
\emph{\href{https://www.audm.com/?utm_source=nytmag\&utm_medium=embed\&utm_campaign=sacrificial_lands_louisiana}{download
Audm for iPhone or Android}.}

Kindra Arnesen's middle school was a plot of marsh a hundred yards off
the southern coast of Louisiana. At 12, after her mother lost her job,
Arnesen began skipping school to walk to the harbor in Buras, a town
near the mouth of the Mississippi River. A dredge boat ferried her to
Bay Adams, where she met a crew of oystermen. They gave her a flatboat,
rubber boots, burlap sacks and a hatchet. With a rope looped around her
waist, she trudged through the marsh, between the mud banks and the
tufts of saw grass, tugging the boat behind her.

``It was just beautiful out there,'' Arnesen says today. ``Serenity.''

It wasn't hard to find oysters then --- they were everywhere. She bent
into the water, yanked out a cluster, shook off the mud, tossed it in
the boat. When the boat was full, she climbed onto it. She cleaned the
oysters, hacking off debris and dead shells, and fed them into the
sacks. By the end of the day, she would have filled 10, earning about
\$100, the cash placed in an envelope with her name written on it that
she picked off the hood of the foreman's truck. She supported her entire
family, with enough left over for Girbauds jeans, Z Cavariccis
high-waisted pants and white K-Swiss Classics.

After oyster season, she fished for mullet, shoveled ice at Wet Willie's
Seafood or worked the deck on shrimp boats that left after dark and
returned at dawn. The summer she turned 14, she and a girlfriend
unloaded 100-pound sacks for aging Vietnamese oystermen. The girls
hauled as many as 800 a day, for a dollar a sack.

``As a young girl in a port town, a lot of bad stuff could've happened
to me,'' she says. ``Instead of getting in trouble, I worked on an
oyster boat. The men and women I worked with taught me to stick up for
myself. They saved me from the big bad world.''

Arnesen has since devoted herself to protecting those fishermen from
that big bad world. She now runs her own fishing business, bringing
amberjack, mullet, pompano, sheepshead and shrimp to distributors that
service restaurants in New Orleans and ship up the East Coast. Most
days, when not at sea, she drives between Venice, the last town before
the Mississippi emptied into the gulf, and New Orleans, about 90 minutes
north, buying parts for her fleet, signing paperwork and unloading
thousands of pounds of fish from the back of her Chevrolet Silverado
3500 pickup. She has nevertheless found the time to become one of the
most prominent national advocates for gulf fishermen. Since the BP oil
spill, she has attended just about every public meeting or legislative
session concerning the future of the Louisiana fishing industry, which
provides about a third of all seafood caught in the continental United
States.

``If we don't fight for these fishing families, if we lose a couple of
links out of the generational chain,'' she says, ``we lose a whole way
of life in this country.''

\textbf{This was how} Arnesen found herself on a pre-pandemic afternoon
at Belle Chasse Auditorium, 17 river miles south of New Orleans, ready
to confront the architects of the world's largest environmental
engineering project: the 50-year, \$50 billion
\href{http://coastal.la.gov/our-plan/2017-coastal-master-plan/}{Coastal
Master Plan}, developed by the state of Louisiana, to manage the coast's
furious retreat from the Gulf of Mexico. The plan contains 124 projects
designed to build tens of thousands of acres of new land, preserve what
remains and protect the coast from hurricanes and sea-level rise. The
state did not characterize the plan as the world's most expensive, and
most ambitious, climate-change-adaptation plan, though that could be one
way to describe it.

Much of the shrinkage has occurred in Plaquemines Parish, where Arnesen
lives and works and which in the last century has withered to almost
half its original size. Ten miles downriver of New Orleans, dangling
into the Gulf of Mexico, bisected by the Mississippi River but
bridgeless, Plaquemines has atrophied for more than seven decades. The
parish suffers one of the planet's fastest rates of relative sea-level
rise, thanks to a confluence of worst-case scenarios: As the Gulf of
Mexico rises because of global warming, the coastal marsh, scored by
oil-and-gas canals and starved of fresh sediment by the encasement of
the Mississippi River and the damming of its upriver tributaries,
subsides unmitigated. Should nothing be done, Plaquemines will lose more
than half of its remaining land, and one of the world's most productive
ecosystems, in the next 50 years.

This is not just a problem for Plaquemines, or Louisiana. It is a crisis
for the United States. The threatened three million acres of marsh,
approximately the landmass of Connecticut, are the coast's first line of
defense against the ouroboric perils of hurricanes and sea-level rise.
The marsh defends 17 percent of the nation's crude-oil production; 8
percent of its natural-gas reserves; a port connected to more than half
of the nation's oil-refining capacity; the city of New Orleans and its
port; the homes of more than 1.5 million people; and the integrity of
the lower Mississippi River, which conveys nearly 40 percent of the
nation's agricultural exports. The effort to rebuild the coast is one of
this country's first critical tests of the climate age.

In one of the reddest states in the nation, the master plan enjoys
thunderous bipartisan support; in 2017, when it last came up for a vote,
a single state legislator opposed it. Initial funding concerns were
alleviated by an infusion of \$4 billion received from
\href{https://www.nytimes.com/2012/03/03/us/accord-reached-settling-lawsuit-over-bp-oil-spill.html}{the
settlement of the BP oil-spill lawsuits}, which enabled the early stages
of the
\href{http://coastal.la.gov/our-work/key-initiatives/diversion-program/}{diversion
projects} to commence. But the plan will not help everybody. It may
require the government to use private land to build transformative
engineering projects that will render parts of the coast unrecognizable
--- or rather, to distinguish from the coast's current anomalous
appearance, unrecognizable in new ways. The Coastal Master Plan will not
only test the limits of our species' capacity to engineer our
environment; it will also test the government's capacity for compelling
even a small, relatively powerless group of people, against their will,
to suffer in the name of climate policy. Should the master plan succeed,
it would benefit the many. It would also harm the few.

Kindra Arnesen spoke for the few. She tried not to miss one of the
monthly public meetings, at which the state's engineers struggled to
reassure those who feared that their lives would be ruined by the
projects in the master plan. The engineers, who were far more
comfortable speaking about hydraulic design and earthen containment
dikes than climate ethics, greeted her with wary decorousness. ``We have
to be present, or they can say there's `no opposition,''' Arnesen said.
``I see this as doomsday. This will end us.''

\emph{This} was the centerpiece of the master plan: the construction of
new, man-made diversions in Plaquemines Parish. The state would cut open
the federal levee, creating powerful new distributaries of the
Mississippi River that will flush sediment into the marsh, building
land. These engineered floods would simulate the geological process that
created the Mississippi Delta in the first place. The initial two
incisions are to be made in the levees about 25 and 35 river miles south
of New Orleans. Construction on the mid-Barataria Diversion on the west
bank, could begin as soon as the end of 2022, followed by the mid-Breton
Diversion on the east bank. Once running at full capacity, the
diversions would themselves rank among the nation's largest rivers. Both
will flow at more than two times the volume of the Hudson River. Over
the course of years and decades, it is hoped, the gargantuan volume of
sediment borne by the diversions will patch the holes in the marsh's
moth-eaten fabric. Lost species will return and biological diversity
will increase. The local fisheries might ultimately become even more
productive.

In the short term, however, the diversions will transform the delicate
estuarine ecosystems. They would likely massacre giant populations of
oysters, brown shrimp, blue crab and dozens of species of fish. Areas of
brackish water will turn fresh, and saltwater vegetation will die.
Plaquemines Parish has the largest commercial fishing fleet in the
continental United States. Arnesen worried that the diversions would
destroy it.

The engineers' responses to Arnesen's concerns were pallid,
technocratic. They noted that, if they failed to build land, not only
the fisheries but the parish itself would, in the coming decades, vanish
entirely. They pointed out that the presence of brackish water so close
to the river was a historical anomaly. And they argued that the
freshwater would bring new species to fish.

The high-minded dismissiveness of the engineers gave Arnesen fits.
``Most of the people living here don't have the options that my family
has,'' she said. ``These boats aren't cheap.'' A small fishing skiff can
cost \$30,000, a larger shrimp trawler tops \$750,000, without
accounting for gear and licenses. Most fishermen could not afford to
diversify or wouldn't know how. A shrimper would lose his house if he
had to run a catfish business; the tourists who come to southern
Louisiana from around the world to hire speckled-trout charter captains
would not travel for shad. Oystermen were in the most precarious
position of all. Oyster leases, which the state rents at \$3 per acre
per year, have terms of 15 years. They are bequeathed to heirs like any
other real estate. An oyster farmer is as bound to his submerged plot as
a dairy farmer to his pasture. It is about as easy for an oysterman to
start fishing largemouth bass as it would be for an alfalfa farmer to
raise pigs. It is possible but difficult, risky and usually
cost-prohibitive.

The master plan aimed to ``balance'' a suite of objectives: to ``provide
flood protection, use natural processes, provide habitat for commercial
and recreational activities, sustain our unique cultural heritage and
support our working coast.'' This was unimpeachable in theory, offering
something for everyone. But the plan was silent on what to do when these
objectives came into direct conflict. What happened when
flood-protection measures threatened cultural heritage? Or when
``natural processes'' interfered with a ``working coast''? It drove
Arnesen crazy, the refusal to acknowledge that not all objectives were
treated equally. It was obvious to her that the state cared more about
the oil-and-gas industry than the fisheries, that it worried a lot more
about keeping New Orleans dry than Buras, that expanding the river's
shipping capacity was more important than preserving the heritage of
generational fishing communities. Most maddening of all was the plan's
emphasis on the future over the present.

``I don't just do this because it's my living,'' Arnesen said as she
left the meeting, trailing an entourage of well-wishers. ``They've made
our community feel like we're the trade-off and we don't matter. It's
easy for the state to say they're going to come up with an adaptation
plan. But what's the point of an adaptation plan if the end goal isn't
the survival of the people you're trying to save?''

\emph{The survival of the people.} How to characterize the way of life
threatened by the diversions? It did not simply entail the right to fish
the same species that your grandfather fished, or to inhabit the same
half-acre, or to live off the fruit of the sea and land, though that was
all part of it. It wouldn't be at all accurate to say that the lowest
stretch of the Mississippi was more remote than any other rural area of
America, though it often feels that way. To an interloper from New
Orleans, the lower Mississippi looks like the end of the world, a
wilderness untouched by human interference --- despite the fact that the
land in its current form owes its existence to human interference. You
could say, at the very least, that it remained possible to live a life
of wildness and freedom there. This was especially true for those living
on the wrong side of the Wall.

It was not technically a wall, but that's how it was known in
Plaquemines. Officially it was
\href{https://www.mvn.usace.army.mil/Missions/Mississippi-River-Flood-Control/Bonnet-Carre-Spillway-Overview/}{the
Hurricane and Storm Damage Risk Reduction System}, known by the
punishing acronym HSDRRS (``his dress''). The U.S. Army Corps of
Engineers built the \$14.5 billion interlocking network of gates, levees
and flood walls in response to the levee failures after Hurricane
Katrina --- New Orleans's answer to the Netherlands' Delta Works. It was
designed with the explicit goal of protecting the New Orleans
metropolitan area from a catastrophic hurricane. The system drew a line
separating those who would be kept safe from those who would be
abandoned to the furies. New Orleans metro would be saved. The entire
east bank of Plaquemines Parish was consigned to the sacrifice zone.
Today those scattered beyond the Wall are regarded by those inside the
Wall, if they are regarded at all, with an uneasy combination of
bafflement and pity.

In Plaquemines it is a matter of faith, if not scientific proof, that
the Wall was responsible for the devastation wrought by Hurricane Isaac
in 2012. While New Orleans experienced only minor street flooding, parts
of Plaquemines beyond the Wall lay below 17 feet of water. The Corps
blamed the disparity on the path taken by the storm, holding the Wall
entirely innocent of blame. Plaqueminers believe the Wall trapped the
storm surge in their parish, as a dammed river will turn a valley into a
lake.

``We never had water here before the Wall,'' said Kermit Williams Jr.,
standing on his family land at Wills Point, the site of the proposed
mid-Breton Diversion. This far down the river, the parish ran a single
property wide. The backyards terminated at the parish levee, which
defended against the encroaching marshes of Breton Basin, and where the
parish's coyote, wild boar, deer, cattle, rabbits and buzzards
congregated during hurricanes, seeking high land. The front yards met
the highway that trimmed the base of the federal levee. Here the
Mississippi was both invisible and oppressive, a tiger in a cloaked
cage. It could not be seen from the ground, even though, on this
afternoon, it was 17 feet higher than the ground. Often the crown of an
oil tanker or cruise ship, passing like spaceships, crested above the
levee's rim. The saturated flood wall oozed water that collected in
ominous puddles along the highway.

The assemblage of structures on Williams's parcel made a living tableau
of the parish's century. Several hundred yards back from the road lurked
the ruined husk of a three-bedroom house overgrown with dead vines,
cypress trees and a vibrating nest of honeybees. It had belonged to his
grandfather; Williams's father was born in its living room in 1910.
Before it stood a green stucco house, built in 1949 on a low foundation
of concrete blocks, where Williams lived until Hurricane Isaac. Williams
now lives with his daughter in a third house, closest to the highway.
Like most of the occupied properties along this Lorax-like stretch of
the parish, it stood on stilts more than 20 feet high.

Williams commiserated about the brutal history of heavy-handed
interventions in the parish with his neighbors, the brothers Danny and
John Hunter. The men, who were in their late 50s, agreed that the parish
could survive Nature but it might not survive the State of Louisiana.
Their litany of grievances began in 1927, with the dynamiting of the
levee at Caernarvon, a few miles upriver, a shortsighted gambit to spare
New Orleans from flooding. The explosions lasted 10 days and created a
flow of a quarter-million cubic feet of water per second through the
parish --- a Superdome of water every eight minutes and 20 seconds. For
the next six decades, most of the local marsh was perpetually inundated,
the greatest part of it a brackish pond called Big Mar that grew saltier
each year. In the 1960s came the construction of the Mississippi
River-Gulf Outlet, a canal offering a shorter route from the gulf to New
Orleans --- a dagger through the heart of the marsh that accelerated
land loss and contributed to the 13 feet of water that flooded the
Hunters' childhood home in St. Bernard after the assault of Hurricane
Betsy in 1965. (``It was pretty traumatic,'' John says more than 50
years later.) Over the following decades, a series of smaller diversions
on the east bank were abandoned, despite promises from the state. This
made the construction of the Wall feel less like a fresh betrayal than
the physical manifestation of a psychic boundary between the haves and
have-nots that has existed for nearly a century.

The Hunters spoke about the master plan in the cadence of undecided
voters. They recognized the need to fortify the marshes, particularly
given the ever-increasing projections of sea-level rise in the parish.
``If it builds land, I'm for it,'' John said. Danny agreed, saying,
``Everyone's for building land.'' They could remember a time, not very
long ago, when you could catch mangrove snapper in New Orleans East,
before speckled trout appeared in the shipping canals, when oyster were
plentiful on the east bank.

Yet they had internalized the arguments made by the Save Louisiana
Coalition, the only nonprofit opposed to the master plan, which
represented the movement of fishermen desperate to stop the government
from trying to save them. Although the fishermen had been making
arguments against the diversions for years, the winter of 2019 --- the
wettest winter the Mississippi Valley had in 124 years --- provided hard
evidence for their apocalyptic predictions. Everything they feared from
the diversions came true, just a few miles upriver.

As the Mississippi River rose at a terrifying rate, the corps opened a
different kind of diversion:
\href{https://www.mvn.usace.army.mil/Missions/Mississippi-River-Flood-Control/Bonnet-Carre-Spillway-Overview/}{the
Bonnet Carré Spillway}, which functions as a release valve when the
lower Mississippi comes close to overtopping its levees. Before last
year, the corps had opened the spillway a dozen times since 1927 and
never in consecutive years. In 2019 it was opened twice, for a
cumulative 123 days, easily a record. (This year it has already been
open 29 days.) Oyster populations collapsed, hundreds of dolphins were
stranded across the Gulf Coast and the Department of Commerce declared a
federal fisheries disaster. Lawsuits against the corps were filed by the
state of Mississippi, Biloxi and several other cities on the Mississippi
coast and two environmental groups that contended the corps failed to
consider the diverted water's effects on leatherback sea turtles and
West Indian manatees.

The Hunters worried that the diversions would make the spillway's
ecological outrages permanent. They were concerned about the toxicity of
the Mississippi River, the second-most-polluted river in the United
States (after the Ohio, its largest tributary). They couldn't stomach
the thought of the river poisoning the marsh. And they couldn't
understand the state's emphasis on long-term benefits. ``The scientists
might be right, but the plan looks ahead to 50 years in the future,''
John said. ``We don't have 50 years. We need it done now.''

Even the 50-year window is misleading, as every six years the clock
starts anew. The master plan is perpetually, implacably,
forward-looking. It is a model for the kind of governmental response
that galloping climate change demands: an agenda that combines
mitigation and adaptation, while retaining the flexibility to respond to
unforeseen developments, whether positive or catastrophic. It is the
rare example of legislation in which the costs are borne immediately and
the greatest benefits will not accrue until after the deaths of
currently elected officials. The Mississippi River has been managed by
the Army Corps of Engineers for nearly a century. The master plan
intends to manage the coast for longer.

The mid-Breton diversion will require the state to go through the
Hunters' land. The diversion would be dug right beside the property line
and the state highway would be rerouted through his backyard. The
brothers used the word ``reparations.'' But the value of Danny Hunter's
land far surpassed its real estate value. The rich alluvial soil had
given Hunter what he figured was one of America's most fertile backyard
gardens, with its profusion of creole tomatoes, eggplant, snap bean,
cucumbers and squash. Every spring his grove of satsuma and navel-orange
trees produced such an abundance of fruit that branches snapped under
the weight. In the back of the property, Danny had dug a long pond that
he stocked with crawfish, which the raccoons liked to poach. Owls,
cardinals and blue jays nested in the live oak trees. When Danny wanted
to escape, he strolled along the back levee, fringed with yellow
wildflowers, and gazed into the marsh. Mattresses of clover dotted with
white buds plunged into a plaid of Roseau cane, yellow palm and skeletal
stands of cypress, killed by saltwater intrusion. The ground looked
solid, but in most places it would melt under the pressure of a boot.

``This is my peaceful place,'' Danny said. ``When I'm in the right frame
of mind, I get on my knees.''

\textbf{A few miles} upriver, along the southern flank of the Wall,
\href{http://mississippiriverdelta.org/staff/john-lopez/}{John Lopez,}
who is as responsible as anyone for concocting the strategy behind the
Coastal Master Plan, and Theryn Henkel, a coastal ecologist, were
speeding on an airboat from the levee down a man-made channel to visit a
man-made swamp that was becoming a man-made forest. Airboat travel feels
less like boat travel than air travel: You glide over a changeable
terrain of open water, swamp, grass, island, rarely feeling so much as a
bump or jostle. A Kris Kristofferson type with a pleasingly gruff
demeanor, Lopez wore heavy earmuffs to mute the clamor of the dual fan
engines. At the airboat's approach, a succession of alligators, startled
out of their ruminations, flopped into the water like divers in a Busby
Berkeley musical.

Lopez began to worry about the inadequacy of the U.S. government's
response to the slow-motion disaster of the disintegrating Louisiana
coast in 2005. While working for the Army Corps of Engineers'
coastal-restoration program, he began to realize that the corps'
approach wasn't nearly ambitious enough. On nights and weekends, he
developed his own strategy to save the coast.

He thought that it was too late merely to preserve what remained. More
land had to be built. Lopez's solution resembled the one that has been
reached repeatedly by climate experts in the last 40 years. Nothing less
than a maximalist approach --- borne by desperation, terror and an
unshakable belief in human will and ingenuity --- would suffice, price
be damned. The financial cost would be severe, particularly up front,
though it would be cheaper than the alternative: the swift, unmitigated
collapse of the coast. The personal cost was more difficult to quantify.
Lopez, like the state and the federal government, fell back on
utilitarian arguments: the many prioritized above the few. If the
Louisianian fishermen were one of the first groups to be passed over by
climate policy, they wouldn't be alone. They would soon be joined by
coal miners, offshore roughnecks, long-haul truck drivers, Sonoran
farmers, Miami Beach condo owners. Lopez felt bad for them, he did. But
he felt for everyone else more.

Lopez called it
\href{https://meridian.allenpress.com/jcr/article-abstract/doi/10.2112/SI54-020.1/192267/The-Multiple-Lines-of-Defense-Strategy-to-Sustain?redirectedFrom=fulltext}{the
Multiple Lines of Defense Strategy.} He delivered his moonshot to his
colleagues in June 2005. It was greeted politely. ``There weren't a lot
of questions,'' Lopez says. ``My boss said, `John, this is a really good
idea, but the corps can't do this.''' Lopez agreed. The corps'
bureaucracy was too Balkanized to allow for the kind of systemic
campaign that Lopez knew was required. In July he presented his paper at
a national meeting in New Orleans sponsored by the National Oceanic and
Atmospheric Administration. He was awarded a prize named for the
environmental activist Orville T. Magoon, honoring the greatest
contribution to the public understanding of coastal threats. Still
nobody took him seriously. Louisiana's bureaucracy was as sclerotic as
the corps'; how could it possibly address a problem of this scale? And
where would the impoverished state find the billions to fund such a
plan? A month later, Katrina hit.

In the subsequent period of dread and opportunity, Louisiana merged its
coastal-restoration and flood-control divisions, creating a centralized
entity called the Coastal Protection and Restoration Authority. CPRA ---
``sip rah'' --- set about formulating a grand plan to preserve the
coast. Lopez's Multiple Lines of Defense Strategy became its organizing
principle. Implicit in the plan was the acknowledgment that a stable
coast required constant, and profound, human intervention. Lopez
believed this approach was required not only in Louisiana but globally.
``There's no such thing as a pristine environment, an environment that
can be left on its own without being managed,'' Lopez says. ``We're past
that point. It's like Colin Powell said: `You break it, you own it.'
Well, we own it now.''

The master plan spoke of ``coastal restoration,'' of ``rebuilding'' and
``saving'' the wetlands. But these were euphemisms. What was lost cannot
be restored, no more than the past can be relived. The best that now can
be achieved is an artful simulacrum of a river delta that serves the
same ecological and economic ends. The master plan is the blueprint for
this simulacrum.

The canal followed by Lopez's airboat ran perpendicularly from the
Mississippi River. The Army Corps of Engineers had blasted through the
levee here in 1991, creating an earlier diversion of the river, formally
known as the Caernarvon Freshwater Diversion Structure. The diversion
opened at the very site of the 1927 crevasse, blasting open both an
earthen and a psychological wound. In a reversal of the current river
politics, the diversion was heavily supported by local oystermen, whose
crop had declined for decades because of saltwater intrusion. After
Caernarvon opened, the inundation of river water destroyed oyster
production in the neighboring marsh. But it boosted production across
the larger area, as salinity levels in the Breton Basin returned to
historic averages. Caernarvon remains a sore point for local fishermen;
Lopez uses it as a showcase for the power of human ingenuity to build an
environment that is, by all appearances, natural.

It was expected that the salinity of the marsh would plunge after the
river began to flood it. The great shock of the Caernarvon diversion was
that it also began to produce land. Caernarvon, unlike the diversions in
the master plan, was not designed to capture sediment; for land-building
purposes, it could not have been more poorly designed. It was situated
at a bend in the river where the water flowed quickly, limiting sediment
buildup; it was operated intermittently, at low volumes; and it siphoned
from the river's surface, where sediment concentration was lowest and
finest. Yet Caernarvon had nevertheless managed to perform a stunning
magic trick.

The airboat pivoted sharply into a narrow watery trail through denser
forest. Lopez and Henkel have taken to calling this passage Bayou
Bonjour. NOAA has delisted more than 40 place names from nautical maps
of the parish since 2011, among them Bayous Long, Caiman and Tony; Lopez
and Henkel hope that Bayou Bonjour will be the first of many place names
to be added by the diversions. Bonjour was not technically a tributary
or a creek; it was the final vestige of a lake remaining between two
lobes of land that had grown toward each other.

Bayou Bonjour debouched into a steaming marsh indistinguishable from
thousands of others in southern Louisiana. Suspended in the shallow
water were what botanists call S.A.V., or submerged aquatic vegetation,
that took the form of green Mardi Gras wigs, tattered velvet,
elephantine dill. The roots trap sediment like weirs. The mud clots
until it surfaces as islets. The virgin land sprouts exclamatory tufts
of giant cut-grass, named for its razor-edged leaves, which draw blood.
Saplings colonize the accreting land, led by black willow, which shoot
up 30 feet within a few years. ``I don't think anyone in their wildest
dreams imagined that there'd be a forest here,'' Lopez said. But a
forest stood before him. It stood at the edge of the marsh, on land that
15 years ago was the open water of Big Mar pond. The Caernarvon
diversion has created more than 800 solid acres in Big Mar alone.

There are still puddles and elongated pools in Big Mar, though none more
than three feet in depth. ``The whole Big Mar is really restored,''
Lopez said. ``It never was going to be entirely solid land. You want
ponds, grass, forest, S.A.V.'' The proximity of so many habitats has
attracted a wide range of species. That afternoon Henkel and Lopez
spotted a blue heron, a white ibis, roseate spoonbills, redwing black
birds, egrets and a twitching mullet clutched in an osprey's talons. The
air swarmed with midges; monarch butterflies played around the bull
tongue; and iridescent blue dragonflies browsed the hyacinth.

``When I'm out here,'' Henkel said, ``I feel like I'm in Jurassic
Park.''

To secure what territorial gains have been made and hasten the
maturation of the forest, the Pontchartrain Conservancy, a nonprofit
environmental group that Lopez helped found in 1989, led the effort to
plant 36,000 trees in the Caernarvon area, mainly bald cypress, as well
as water tupelo, swamp red maple, green ash and black gum. Most of these
have been planted by migrant workers, a dozen of whom will plant 5,000
trees in a week. (There are also planting days staffed by local
volunteers, though they perform a service less efficient than
educational.) The cypress saplings wore plastic collars as a defense
from the nutria, which gnawed them to death. After bursting from the
collars, the trees were crowded by an understory of elephant ears,
arrowleaf and black gum. Deeper in the forest skulked muskrat, raccoons,
coyote, deer and wild boar. Henkel did Dr. Frankenstein: ``It's alive!''

The diversion and the land it built were an ecological monster --- a
product of human engineering, compromise and brute force. Like most
man-made things it was unruly, even clumsy; its charms, and its dangers,
were accidental and unforeseen. It was not, by any conventional
definition, natural. But it was alive.

Asked whether this was the future he saw for southern Louisiana ---
man-made rivers, man-made marshes, man-made forests --- Lopez
acknowledged that he had a provisional view of the problem. Even under
the rosiest projections for the diversions, the future coast would be
``skeletal'' compared to its current form, but ``functional,'' at least
in economic and ecological terms. He expected to preserve the state's
highway systems, railroads, port and energy facilities. He was even
optimistic about its fisheries, albeit in some altered condition. ``We
can expect that there will be crab, shrimp and oysters,'' Lopez said.
``We just can't say where or how much.'' He had fond memories of family
fishing trips to Plaquemines Parish as a child, catching speckled trout
off the coast near Buras, where Kindra Arnesen dredged for oysters. He
respected the fears that Arnesen and her allies expressed about the
threats posed by the diversions to the local fisheries. Still he did not
have a tremendous amount of sympathy for their plight. ``I'm not saying
the transition is going to be easy, or that it won't cost money,'' he
said. ``But I'm also not saying that the state or anyone else is
responsible for helping.'' Just because your father was an electrician,
Lopez said, by way of example, and your father's father was an
electrician, doesn't mean that \emph{you} need to be an electrician or
that the government should incentivize you to be one. Some people in the
local fishing industry, like Arnesen, might be able to adjust, buying
new boats and fishing new species. Others won't.

\textbf{In May, a team} of Tulane researchers led by the geologist
Torbjorn Tornqvist published
\href{https://advances.sciencemag.org/content/6/21/eaaz5512}{a study in
the journal Science Advances} that showed that Louisiana's remaining
6,000 square miles of coastal wetlands were far closer to collapse than
previously recognized. The present-day rate of relative sea-level rise
has already surpassed the tipping point at which the drowning of the
marsh is unstoppable.
\href{https://www.nola.com/news/environment/article_577f61aa-9c26-11ea-8800-0707002d333a.html}{Asked
by Mark Schleifstein of The Times-Picayune-The New Orleans Advocate to
translate his scientific findings,} Tornqvist said, ``We're screwed.''

Still Tornqvist was an ardent supporter of the diversions. So was his
report's second author, Krista Jankowski, who as a graduate student
conducted much of the research for the paper and joined CPRA two years
ago. The master plan, as John Lopez was quick to point out after the
publication of their study, has already taken its findings into account.
If Tornqvist and his colleagues were right, the southern marsh will
ultimately succumb to the rising sea, and if New Orleans exists, it will
be as an island city. But they did expect the diversions to extend the
life of the coast by decades. ``Having a few more decades could mean the
difference between something that looks like managed retreat to
something that looks like complete chaos,'' Tornqvist says. ``That's
easily worth a couple of billion dollars, because it's not hard to
imagine the amount of suffering that would be the result of leaving
everyone in this whole region to fight for themselves.''

We may all be screwed in the long term, but even central planners
worried about the medium term. Engineers often planned in 50-year
increments: The design life of most structures, whether buildings or
bridges or levees, tended to be 50 years. That was why the master plan
was a 50-year plan. By 2070, if Lopez's calculations were right, and the
sediment diversions worked as designed, we wouldn't all be screwed. But
a few of us would be. It was not written in any official document, but
that was part of the plan, too.

\hypertarget{the-great-climate-migrationthe-teenagers-at-the-end-of-the-worlddestroying-a-way-of-life-to-save-louisianathe-fearsome-thunderstorms-of-cuxf3rdoba-provincelearning-from-the-kariba-dam}{%
\paragraph{\texorpdfstring{\href{https://www.nytimes.com/interactive/2020/07/23/magazine/climate-migration.html}{The
Great Climate
Migration}\href{https://www.nytimes.com/interactive/2020/07/21/magazine/teenage-activist-climate-change.html}{The
Teenagers at the End of the
World}\href{https://www.nytimes.com/interactive/2020/07/21/magazine/louisiana-coast-engineering.html}{Destroying
a Way of Life to Save
Louisiana}\href{https://www.nytimes.com/interactive/2020/07/22/magazine/worst-storms-argentina.html}{The
Fearsome Thunderstorms of Córdoba
Province}\href{https://www.nytimes.com/interactive/2020/07/22/magazine/zambia-kariba-dam.html}{Learning
From the Kariba
Dam}}{The Great Climate MigrationThe Teenagers at the End of the WorldDestroying a Way of Life to Save LouisianaThe Fearsome Thunderstorms of Córdoba ProvinceLearning From the Kariba Dam}}\label{the-great-climate-migrationthe-teenagers-at-the-end-of-the-worlddestroying-a-way-of-life-to-save-louisianathe-fearsome-thunderstorms-of-cuxf3rdoba-provincelearning-from-the-kariba-dam}}

\begin{center}\rule{0.5\linewidth}{\linethickness}\end{center}

Nathaniel Rich is the author of the book ``Losing Earth: A Recent
History,'' based on
\href{https://www.nytimes.com/interactive/2018/08/01/magazine/climate-change-losing-earth.html}{an
article that appeared in the magazine.}

The Climate Issue

\begin{itemize}
\tightlist
\item
  \href{https://www.nytimes.com/interactive/2020/07/23/magazine/climate-migration.html}{The
  Great Climate Migration}
\item
  The Teenagers at the End of the World
\item
  \href{https://www.nytimes.com/interactive/2020/07/21/magazine/louisiana-coast-engineering.html}{Destroying
  a Way of Life to Save Louisiana}
\item
  \href{https://www.nytimes.com/interactive/2020/07/22/magazine/zambia-kariba-dam.html}{Learning
  From the Kariba Dam}
\item
  \href{https://www.nytimes.com/interactive/2020/07/22/magazine/worst-storms-argentina.html}{What's
  Going on Inside the Fearsome Thunderstorms of Córdoba Province?}
\end{itemize}

\protect\hyperlink{}{} \protect\hyperlink{}{}

\includegraphics{https://static01.nyt.com/newsgraphics/2020/07/26/climate/fbd0a5f2a975dc16f0d1a24d64f70f4d843e50a3/caret.svg}

Read 77 Comments

\begin{itemize}
\item
\item
\item
\item
\end{itemize}

Advertisement

\protect\hyperlink{after-bottom}{Continue reading the main story}

\hypertarget{site-index}{%
\subsection{Site Index}\label{site-index}}

\hypertarget{site-information-navigation}{%
\subsection{Site Information
Navigation}\label{site-information-navigation}}

\begin{itemize}
\tightlist
\item
  \href{https://help.nytimes.com/hc/en-us/articles/115014792127-Copyright-notice}{©~2020~The
  New York Times Company}
\end{itemize}

\begin{itemize}
\tightlist
\item
  \href{https://www.nytco.com/}{NYTCo}
\item
  \href{https://help.nytimes.com/hc/en-us/articles/115015385887-Contact-Us}{Contact
  Us}
\item
  \href{https://www.nytco.com/careers/}{Work with us}
\item
  \href{https://nytmediakit.com/}{Advertise}
\item
  \href{http://www.tbrandstudio.com/}{T Brand Studio}
\item
  \href{https://www.nytimes.com/privacy/cookie-policy\#how-do-i-manage-trackers}{Your
  Ad Choices}
\item
  \href{https://www.nytimes.com/privacy}{Privacy}
\item
  \href{https://help.nytimes.com/hc/en-us/articles/115014893428-Terms-of-service}{Terms
  of Service}
\item
  \href{https://help.nytimes.com/hc/en-us/articles/115014893968-Terms-of-sale}{Terms
  of Sale}
\item
  \href{https://spiderbites.nytimes.com}{Site Map}
\item
  \href{https://help.nytimes.com/hc/en-us}{Help}
\item
  \href{https://www.nytimes.com/subscription?campaignId=37WXW}{Subscriptions}
\end{itemize}
