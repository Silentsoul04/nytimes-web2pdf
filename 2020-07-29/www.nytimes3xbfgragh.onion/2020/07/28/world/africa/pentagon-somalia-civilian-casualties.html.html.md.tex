Sections

SEARCH

\protect\hyperlink{site-content}{Skip to
content}\protect\hyperlink{site-index}{Skip to site index}

\href{https://www.nytimes3xbfgragh.onion/section/world/africa}{Africa}

\href{https://myaccount.nytimes3xbfgragh.onion/auth/login?response_type=cookie\&client_id=vi}{}

\href{https://www.nytimes3xbfgragh.onion/section/todayspaper}{Today's
Paper}

\href{/section/world/africa}{Africa}\textbar{}Pentagon Admits to
Civilian Casualties in Somalia for a Third Time

\url{https://nyti.ms/3jJKwyp}

\begin{itemize}
\item
\item
\item
\item
\item
\end{itemize}

Advertisement

\protect\hyperlink{after-top}{Continue reading the main story}

Supported by

\protect\hyperlink{after-sponsor}{Continue reading the main story}

\hypertarget{pentagon-admits-to-civilian-casualties-in-somalia-for-a-third-time}{%
\section{Pentagon Admits to Civilian Casualties in Somalia for a Third
Time}\label{pentagon-admits-to-civilian-casualties-in-somalia-for-a-third-time}}

Africa Command's admission of the death comes in the wake of its slow
move toward better accountability after years of criticism from human
rights groups and lawmakers.

\includegraphics{https://static01.graylady3jvrrxbe.onion/images/2020/07/28/us/politics/28dc-military/merlin_75255760_b5d85ea8-b7df-47db-829e-c5b0c45c6b21-articleLarge.jpg?quality=75\&auto=webp\&disable=upscale}

\href{https://www.nytimes3xbfgragh.onion/by/thomas-gibbons-neff}{\includegraphics{https://static01.graylady3jvrrxbe.onion/images/2018/07/12/multimedia/author-thomas-gibbons-neff/author-thomas-gibbons-neff-thumbLarge.png}}

By
\href{https://www.nytimes3xbfgragh.onion/by/thomas-gibbons-neff}{Thomas
Gibbons-Neff}

\begin{itemize}
\item
  July 28, 2020
\item
  \begin{itemize}
  \item
  \item
  \item
  \item
  \item
  \end{itemize}
\end{itemize}

WASHINGTON --- The Pentagon has admitted for the third time that its
bombing campaign against terrorist groups in Somalia, which has been
underway for more than a decade, had caused civilian casualties there, a
\href{https://www.africom.mil/civilian-casualty-report-and-allegations}{military
report} said on Tuesday.

The announcement, by United States Africa Command, substantiated
\href{https://www.amnestyusa.org/press-releases/zero-accountability-as-deaths-mount-in-somalia-from-u-s-strikes/}{reports
by Amnesty International} that a U.S. airstrike on Feb. 2 in the Somali
town of Jilib killed Nurto Kusow Omar Abukar, 18, and injured her two
younger sisters and grandmother. The strike was targeting members of the
Shabab, an extremist group linked to Al Qaeda.

``Our goal is to always minimize impact to civilians,'' Gen. Stephen J.
Townsend, the commander of Africa Command, said in the report.
``Unfortunately, we believe our operations caused the inadvertent death
of one person and injury to three others who we did not intend to
target.''

As is common with almost every U.S. airstrike in Somalia, a military
statement released after the Jilib bombing said an ``initial assessment
concluded the airstrike killed one (1) terrorist. We currently assess no
civilians were injured or killed as a result of this airstrike.''

Africa Command's admission of the death comes in the wake of its slow
move toward better accountability after years of criticism from human
rights groups and lawmakers who frequently accused the command of
covering up civilian deaths and, at the least, not investigating claims
from local residents.

Last year, Africa Command pledged to review all military operations in
Somalia since 2017. After the review, it admitted to the deaths of four
other civilians in two different airstrikes in 2018 and 2019.

``Now that there has been an acknowledgment of their actions, there must
be accountability and reparations for the victims and their families,''
Brian Castner, the senior crisis adviser for arms and military
operations at Amnesty International, said in a statement.

The U.S. military has carried out more than 180 airstrikes in Somalia
since 2017, 42 of them in 2020. Amnesty International has assessed that
more than 20 civilians have been killed in a small number of those
strikes.

The increase in air attacks and ground raids has been attributed to more
lax rules of engagement under the Trump administration, giving
commanders more leeway to find and attack Shabab and Islamic State
targets.

The military campaign there, like most American conflicts since the
Sept. 11 attacks, has been locked in a stalemate as Shabab fighters
continue to hold and influence territory in the hinterlands while U.S.
and allied forces try to train and empower local troops to fight for
themselves.

Advertisement

\protect\hyperlink{after-bottom}{Continue reading the main story}

\hypertarget{site-index}{%
\subsection{Site Index}\label{site-index}}

\hypertarget{site-information-navigation}{%
\subsection{Site Information
Navigation}\label{site-information-navigation}}

\begin{itemize}
\tightlist
\item
  \href{https://help.nytimes3xbfgragh.onion/hc/en-us/articles/115014792127-Copyright-notice}{©~2020~The
  New York Times Company}
\end{itemize}

\begin{itemize}
\tightlist
\item
  \href{https://www.nytco.com/}{NYTCo}
\item
  \href{https://help.nytimes3xbfgragh.onion/hc/en-us/articles/115015385887-Contact-Us}{Contact
  Us}
\item
  \href{https://www.nytco.com/careers/}{Work with us}
\item
  \href{https://nytmediakit.com/}{Advertise}
\item
  \href{http://www.tbrandstudio.com/}{T Brand Studio}
\item
  \href{https://www.nytimes3xbfgragh.onion/privacy/cookie-policy\#how-do-i-manage-trackers}{Your
  Ad Choices}
\item
  \href{https://www.nytimes3xbfgragh.onion/privacy}{Privacy}
\item
  \href{https://help.nytimes3xbfgragh.onion/hc/en-us/articles/115014893428-Terms-of-service}{Terms
  of Service}
\item
  \href{https://help.nytimes3xbfgragh.onion/hc/en-us/articles/115014893968-Terms-of-sale}{Terms
  of Sale}
\item
  \href{https://spiderbites.nytimes3xbfgragh.onion}{Site Map}
\item
  \href{https://help.nytimes3xbfgragh.onion/hc/en-us}{Help}
\item
  \href{https://www.nytimes3xbfgragh.onion/subscription?campaignId=37WXW}{Subscriptions}
\end{itemize}
