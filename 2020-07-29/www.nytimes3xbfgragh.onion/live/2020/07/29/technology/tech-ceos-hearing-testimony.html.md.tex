Sections

SEARCH

\protect\hyperlink{site-content}{Skip to
content}\protect\hyperlink{site-index}{Skip to site index}

\href{https://myaccount.nytimes3xbfgragh.onion/auth/login?response_type=cookie\&client_id=vi}{}

\href{https://www.nytimes3xbfgragh.onion/section/todayspaper}{Today's
Paper}

LiveUpdated~

July 29, 2020, 2:07 p.m. ET

July 29, 2020, 2:07 p.m. ET

\hypertarget{big-tech-hearing-live-updates-lawmakers-take-aim-at-executives}{%
\section{Big Tech Hearing Live Updates: Lawmakers Take Aim at
Executives}\label{big-tech-hearing-live-updates-lawmakers-take-aim-at-executives}}

RIGHT NOW

\protect\hyperlink{watch-live-ceos-begin-to-answer-lawmaker-questions}{Watch
live: C.E.O.s begin to answer lawmaker questions.}

\hypertarget{heres-what-you-need-to-know}{%
\subsubsection{Here's what you need to
know:}\label{heres-what-you-need-to-know}}

\begin{itemize}
\item
  \protect\hyperlink{lawmaker-americans-should-not-bow-before-the-emperors-of-the-online-economy}{}

  Lawmaker: Americans should not `bow before the emperors of the online
  economy.'
\item
  \protect\hyperlink{republicans-immediately-raise-bias-concerns-about-platforms}{}

  Republicans immediately raise bias concerns about platforms.
\item
  \protect\hyperlink{bezos-gets-his-first-experience-in-the-hot-seat}{}

  Bezos gets his first experience in the hot seat.
\item
  \protect\hyperlink{the-ceos-are-testifying-remotely-using-ciscos-webex-videoconferencing}{}

  The C.E.O.s are testifying remotely, using Cisco's Webex
  videoconferencing.
\item
  \protect\hyperlink{big-techs-rivals-spoke-out-ahead-of-the-hearing}{}

  Big Tech's rivals spoke out ahead of the hearing.
\item
  \protect\hyperlink{dont-only-blame-congress-if-this-hearing-goes-off-the-rails}{}

  Don't (only) blame Congress if this hearing goes off the rails.
\end{itemize}

\hypertarget{watch-live-ceos-begin-to-answer-lawmaker-questions}{%
\subsection{\texorpdfstring{\protect\hyperlink{watch-live-ceos-begin-to-answer-lawmaker-questions}{Watch
live: C.E.O.s begin to answer lawmaker
questions.}}{Watch live: C.E.O.s begin to answer lawmaker questions.}}\label{watch-live-ceos-begin-to-answer-lawmaker-questions}}

Copied to clipboard.

\includegraphics{https://static01.graylady3jvrrxbe.onion/images/2020/07/29/business/29techhearing-video/29techhearing-video-videoSixteenByNine3000.jpg}

\hypertarget{lawmaker-americans-should-not-bow-before-the-emperors-of-the-online-economy}{%
\subsection{\texorpdfstring{\protect\hyperlink{lawmaker-americans-should-not-bow-before-the-emperors-of-the-online-economy}{Lawmaker:
Americans should not `bow before the emperors of the online
economy.'}}{Lawmaker: Americans should not `bow before the emperors of the online economy.'}}\label{lawmaker-americans-should-not-bow-before-the-emperors-of-the-online-economy}}

Copied to clipboard.

Representative David Cicilline, the chairman of the antitrust
subcommittee, began the hearing on Wednesday with a broadside against
the tech companies, saying their dominance harms the economy and leaves
consumers with no choice but to use their products.

Mr. Cicilline has led the investigation into the tech giants for more
than a year, and on Wednesday morning framed the four as singularly
influential and powerful over many facets of American life.

``Any single action by one of these companies can affect hundreds of
millions of us in profound and lasting ways,'' Mr. Cicilline, a Rhode
Island Democrat, said in his opening statement. ``Simply put: They have
too much power.''

Mr. Cicilline will be in control of many aspects of the hearing,
including how many rounds of questions lawmakers get. That may allow him
to extend lines of questioning in an attempt to dig deeper than an
initial five minutes allows.

Mr. Cicilline, who used to be the mayor of Providence,
\href{https://www.nytimes3xbfgragh.onion/2019/12/08/technology/David-Cicilline-antitrust-tech.html}{has
become a prominent foe} of the technology platform from his perch as the
top Democrat on the once-quiet subcommittee. For more than a year, his
staff has led the investigation, conducting hundreds of hours of
interviews and collecting 1.3 million documents. The team has grown to
include Lina Khan, a legal scholar who wrote a major law review note on
Amazon's power, and Phillip Berenbroick, previously the policy director
at the consumer group Public Knowledge.

Mr. Cicilline spent recent months negotiating to secure the appearance
by the chief executives. It has not always been friendly. When the
committee demanded that Mr. Bezos testify, Amazon responded with a
noncommittal letter. Mr. Cicilline threatened to subpoena Mr. Bezos
before the company agreed to make him available to answer the panel's
questions.

``Our founders would not bow before a king,'' Mr. Cicilline said
Wednesday. ``Nor should we bow before the emperors of the online
economy.''

--- \href{https://www.nytimes3xbfgragh.onion/by/david-mccabe}{David
McCabe}

\hypertarget{advertisement}{%
\subsubsection{Advertisement}\label{advertisement}}

\protect\hyperlink{after-dfp-ad-mid1}{Continue reading the main story}

\hypertarget{republicans-immediately-raise-bias-concerns-about-platforms}{%
\subsection{\texorpdfstring{\protect\hyperlink{republicans-immediately-raise-bias-concerns-about-platforms}{Republicans
immediately raise bias concerns about
platforms.}}{Republicans immediately raise bias concerns about platforms.}}\label{republicans-immediately-raise-bias-concerns-about-platforms}}

Copied to clipboard.

Top Republicans on the committee immediately raised concerns that the
tech giants are systemically suppressing conservative views, an unproven
claim that has grown popular on the right and moved the hearing away
from the central antitrust questions of the hearings.

Representative Jim Jordan of Ohio, the top Republican on the Judiciary
Committee, spent his opening statements listing anecdotes where
Republican officials had been subject to enforcement actions by the
platforms' rules. (He did not mention that conservative publications and
figures routinely rank among the top performing pages on Facebook and
other platforms.)

``I'll just cut to the chase, Big Tech's out to get conservatives,''
said Mr. Jordan. He later accused the companies of ``trying to impact
elections'' and ``censoring conservatives.''

The claims are a persistent, if largely unproven, gripe among
Republicans. President Trump, Attorney General William P. Barr and
lawmakers like Mr. Jordan and Senator Ted Cruz of Texas have all raised
concerns that Facebook, Twitter, and YouTube purposely downplay or
remove conservative voices on their sites.

The suspicions rise from the accurate perception that Silicon Valley is
dominated by liberal-leaning workers. In November 2018, Facebook removed
an ad by an anti-abortion group endorsing Republican Senator Marsha
Blackburn of Tennessee. Facebook said it did so because an image on the
ad that appeared to violate its community norms. That example and others
have fueled suspicion of conservative censorship.

Mr. Trump recently issued an executive order curtailing safe harbors for
internet companies in retaliation against his perceptions of bias. The
order was issued after
\href{https://slack-redir.net/link?url=https\%3A\%2F\%2Fwww.nytimes3xbfgragh.onion\%2F2020\%2F05\%2F26\%2Ftechnology\%2Ftwitter-trump-mail-in-ballots.html}{Twitter
labeled} a set of his tweets in late May for misinformation.

--- \href{https://www.nytimes3xbfgragh.onion/by/david-mccabe}{David
McCabe} and
\href{https://www.nytimes3xbfgragh.onion/by/cecilia-kang}{Cecilia Kang}

\hypertarget{bezos-gets-his-first-experience-in-the-hot-seat}{%
\subsection{\texorpdfstring{\protect\hyperlink{bezos-gets-his-first-experience-in-the-hot-seat}{Bezos
gets his first experience in the hot
seat.}}{Bezos gets his first experience in the hot seat.}}\label{bezos-gets-his-first-experience-in-the-hot-seat}}

Copied to clipboard.

\includegraphics{https://static01.graylady3jvrrxbe.onion/images/2020/07/29/business/29tech-hearing-bezopreview/merlin_161066715_389fe742-254c-4e71-abe2-54f2b577a380-articleLarge.jpg?quality=75\&auto=webp\&disable=upscale}

Jeff Bezos, Amazon's chief executive, has a couple of unique
distinctions among the chief executives testifying on Wednesday: He had
never faced Congress before, and he is by far the world's richest
person.

Those two facts may put him under a particularly intense spotlight at
the hearing, giving lawmakers the first chance to force him to publicly
respond to a range of criticisms. Some will probably hew close to
antitrust law, but others may veer to a broader critique of the
disparity between his wealth --- now hovering around \$180 billion ---
and the pay and working conditions of the labor making at or around \$15
an hour that forms Amazon's logistical advantage.

Mr. Bezos has never been particularly open to the press and has not
faced tough public questioning for years. He has held just a few brief
sessions about Amazon with journalists recently, and public
question-and-answer sessions have hardly included fierce critics. In
one, he sat opposite his brother, and in another, he was with the
private equity titan David Rubenstein.

In
\href{https://blog.aboutamazon.com/policy/statement-by-jeff-bezos-to-the-u-s-house-committee-on-the-judiciary}{prepared
testimony} for the hearing, Mr. Bezos presented himself as an American
success story, raised by a plucky young mother and adopted at four by
his father, a Cuban immigrant. He said he took a risk to start Amazon,
using seed funding from his parents.

``Amazon's success was anything but preordained,'' he said.

While he said he invited scrutiny, Mr. Bezos also said, ``Unlike many
other countries around the world, this great nation we live in supports
and does not stigmatize entrepreneurial risk-taking.''

Mr. Bezos did not address his own wealth head on, but he did say that 80
percent of Amazon's shares are held by outsiders, including pension
funds and mutual funds for retirement, that he said have benefited from
the rise in the company's share price.

While Mr. Bezos has
\href{https://www.nytimes3xbfgragh.onion/2020/07/27/business/jeff-bezos-amazon-congress.html}{a
growing presence} in Washington, he has let his top deputies do the
day-to-day work of lobbying and influencing lawmakers to its lobbying
and communications team, which has grown to more than 800 people
globally. Amazon also spent \$16.8 million on federal lobbying last
year.

--- \href{https://www.nytimes3xbfgragh.onion/by/karen-weise}{Karen
Weise}

\hypertarget{the-ceos-are-testifying-remotely-using-ciscos-webex-videoconferencing}{%
\subsection{\texorpdfstring{\protect\hyperlink{the-ceos-are-testifying-remotely-using-ciscos-webex-videoconferencing}{The
C.E.O.s are testifying remotely, using Cisco's Webex
videoconferencing.}}{The C.E.O.s are testifying remotely, using Cisco's Webex videoconferencing.}}\label{the-ceos-are-testifying-remotely-using-ciscos-webex-videoconferencing}}

Copied to clipboard.

\includegraphics{https://static01.graylady3jvrrxbe.onion/images/2020/07/29/business/29tech-hearing-webex/merlin_174831297_dbba0a25-3406-4981-8177-40356c08d534-articleLarge.jpg?quality=75\&auto=webp\&disable=upscale}

Congressional hearings usually involve witnesses appearing in dark
suits, with their entourages sitting behind them and lawmakers
questioning them from above as phalanxes of photographers snap pictures
and videographers stream the proceedings from a cavernous room at the
Capitol.

Not this time.

The C.E.O.s of Amazon, Apple, Facebook and Google are all appearing on
Wednesday before a House subcommittee virtually because of the
coronavirus pandemic. Remotely beaming into the hearing adds a wrinkle
of digital complexity, with any note-passing from aides and underlings
most likely happening off-camera.

And while many of the tech giants make
\href{https://www.nytimes3xbfgragh.onion/2020/04/24/technology/zoom-rivals-virus-facebook-google.html}{their
own video-calling software}, none will be using their own tools.
Instead, they will all be joining via Cisco's Webex videoconferencing
service.

Webex has been the go-to service for Congress since the pandemic began.
It has been certified by the House's administration committee for being
secure and meeting ``business and technical requirements,'' a House
administration spokesman, Peter Whippy, said.

In that time, Webex has been used for more than 100 congressional
hearings, said Jean Rosauer, Webex's head of government sector. Cisco
added that it had experienced more than triple its normal volume of
virtual meetings through Webex in recent months.

``Congressional hearings --- such as the upcoming House Judiciary
Committee hearing --- have traditions, policies and procedures, and we
had to ensure those could be conducted virtually and securely,'' Ms.
Rosauer said in a statement. She added that Cisco was ``incredibly
proud'' to play a role in keeping Congress connected.

--- \href{https://www.nytimes3xbfgragh.onion/by/kellen-browning}{Kellen
Browning}

\hypertarget{advertisement-1}{%
\subsubsection{Advertisement}\label{advertisement-1}}

\protect\hyperlink{after-dfp-ad-mid2}{Continue reading the main story}

\hypertarget{big-techs-rivals-spoke-out-ahead-of-the-hearing}{%
\subsection{\texorpdfstring{\protect\hyperlink{big-techs-rivals-spoke-out-ahead-of-the-hearing}{Big
Tech's rivals spoke out ahead of the
hearing.}}{Big Tech's rivals spoke out ahead of the hearing.}}\label{big-techs-rivals-spoke-out-ahead-of-the-hearing}}

Copied to clipboard.

\includegraphics{https://static01.graylady3jvrrxbe.onion/images/2020/07/29/business/29tech-hearing-rivals/merlin_164279103_480dfcec-5c97-492c-b524-1e7885551ae2-articleLarge.jpg?quality=75\&auto=webp\&disable=upscale}

Many competitors to Google, Facebook, Apple and Amazon have been busy
talking to House lawmakers for months about those companies' power. And
some deliberately spoke out this week to position themselves for how
they would be portrayed in the hearing and to influence the questioning.

TikTok, the Chinese-owned video app, issued a statement from
\href{https://www.nytimes3xbfgragh.onion/2020/05/18/business/media/tiktok-ceo-kevin-mayer.html}{its
chief executive}, Kevin Mayer, on Wednesday morning. In it, he addressed
how the app --- which Facebook is likely to cite in the hearing as an
example of how competition in social networking is thriving --- has been
dealing with scrutiny because of its Chinese ownership.

``We have received even more scrutiny due to the company's Chinese
origins,'' Mr. Mayer said
\href{https://newsroom.tiktok.com/en-us/fair-competition-and-transparency-benefits-us-all}{in
the statement}. ``We accept this and embrace the challenge of giving
peace of mind through greater transparency and accountability. We
believe it is essential to show users, advertisers, creators and
regulators that we are responsible and committed members of the American
community that follows U.S. laws.''

He also pointed to Facebook's willingness to launch ``copycat
products,'' like Reels, a TikTok look-alike. Facebook has had a history
of emulating competing products.

``Let's focus our energies on fair and open competition in service of
our consumers, rather than maligning attacks by our competitor ---
namely Facebook --- disguised as patriotism and designed to put an end
to our very presence in the U.S.,'' Mr. Mayer said.

Other tech companies also seized on the hearing to air their thoughts.
Tim Sweeney, chief executive of Epic Games, the Cary, N.C.-based maker
of
\href{https://www.nytimes3xbfgragh.onion/2018/07/25/arts/what-is-fortnite-battle-royale-nyt.html}{the
hit game Fortnite}, lashed out at Apple and Google for price gouging and
unfair policies in what he called their ``app store monopolies.''

``Both stores significantly obstruct competition,'' Mr. Sweeney said in
an interview on Tuesday. He particularly criticized Apple's 30 percent
fee on payments for digital goods, which he said made it difficult for
smaller players to offer artists a better deal.

Apple has said the
\href{https://www.nytimes3xbfgragh.onion/2020/07/28/technology/apple-app-store-airbnb-classpass.html}{30
percent commission it takes from many apps} in its App Store is a
standard fee. Mr. Sweeney called that argument ``silly nonsense.''
Epic's version of an app store charges its developers a 12 percent fee.

Mr. Sweeney, who began programming on an Apple II Plus computer in 1982
and founded Epic nine years later, said he felt a responsibility to
speak out.

``Every tech company that does business in this world is going to have
to live with the power we give these other companies,'' he said.

--- \href{https://www.nytimes3xbfgragh.onion/by/mike-isaac}{Mike Isaac}
and \href{https://www.nytimes3xbfgragh.onion/by/erin-griffith}{Erin
Griffith}

\hypertarget{dont-only-blame-congress-if-this-hearing-goes-off-the-rails}{%
\subsection{\texorpdfstring{\protect\hyperlink{dont-only-blame-congress-if-this-hearing-goes-off-the-rails}{Don't
(only) blame Congress if this hearing goes off the
rails.}}{Don't (only) blame Congress if this hearing goes off the rails.}}\label{dont-only-blame-congress-if-this-hearing-goes-off-the-rails}}

Copied to clipboard.

Members of Congress have been
\href{https://www.thewrap.com/senator-orrin-hatch-facebook-biz-model-zuckerberg/}{mocked}
for asking ridiculous questions in technology hearings like these. That
might happen again today, but it won't be entirely their fault.

These big tech companies intentionally make themselves hard to
understand.

Few people outside these companies can truly examine how Amazon
influences prices of products we buy on its site
or\href{https://www.bloomberg.com/news/articles/2019-08-05/amazon-is-squeezing-sellers-that-offer-better-prices-on-walmart}{at
other retailers}; or assess fears that
Google\href{https://themarkup.org/google-the-giant/2020/07/28/google-search-results-prioritize-google-products-over-competitors}{funnels
people to its own websites},
Apple\href{https://www.nytimes3xbfgragh.onion/interactive/2019/09/09/technology/apple-app-store-competition.html}{steers
people to its own apps} or Facebook peers into what we do online to
\href{https://www.nytimes3xbfgragh.onion/2018/12/05/technology/facebook-emails-privacy-data.html}{squash
its rivals}. All of this is, by design, shrouded in secrecy and mystery.

Big Tech shouldn't want it to stay that way. Even companies like
Facebook and Google are asking for more government guidance and rules
around thorny topics like protecting elections and preventing hate
speech on their sites. That means that the public and the tech companies
have a vested interest in making these fact-finding sessions as
productive as possible.

\href{https://www.nytimes3xbfgragh.onion/2020/07/29/technology/congress-big-tech.html}{Read
more in On Tech}.

\emph{You can}
\href{https://www.nytimes3xbfgragh.onion/newsletters/signup/OT}{\emph{sign
up here}} \emph{for On Tech with Shira Ovide, a newsletter each weekday
about how technology is reshaping our lives and world.}

--- \href{https://www.nytimes3xbfgragh.onion/by/shira-ovide}{Shira
Ovide}

\hypertarget{there-are-many-investigations-into-the-tech-companies-heres-where-they-all-stand}{%
\subsection{\texorpdfstring{\protect\hyperlink{there-are-many-investigations-into-the-tech-companies-heres-where-they-all-stand}{There
are many investigations into the tech companies. Here's where they all
stand.}}{There are many investigations into the tech companies. Here's where they all stand.}}\label{there-are-many-investigations-into-the-tech-companies-heres-where-they-all-stand}}

Copied to clipboard.

\includegraphics{https://static01.graylady3jvrrxbe.onion/images/2020/07/29/business/29tech-hearing-inquiries/merlin_163192332_bc0f35e4-7fc0-481a-bec0-f76d02126a92-articleLarge.jpg?quality=75\&auto=webp\&disable=upscale}

The tech giants are under investigation from numerous federal and state
antitrust officials, as well as by the lawmakers holding today's
hearing.

The Justice Department's investigation of Google appears to be the
furthest along.
\href{https://www.nytimes3xbfgragh.onion/2020/06/25/technology/barr-google-investigation.html}{The
agency is expected to soon announce a case against Google}, focusing on
alleged antitrust violations in online advertising.

The Federal Trade Commission
is\href{https://www.nytimes3xbfgragh.onion/2020/07/17/technology/ftc-facebook-investigation.html}{preparing
to depose}Mark Zuckerberg, the chief executive of Facebook, and other
top executives at the company for its investigation of the social
network. That inquiry appears to focus on whether Facebook illegally
maintained a monopoly in social networking by killing off competition
through its acquisitions of Instagram and WhatsApp. That investigation
may not wrap up before the end of the year.

Other investigations are moving forward, but not as swiftly as the
Google investigation. The Justice Department is also investigating
Apple's power over the app store, along with state attorneys general.
The agency has Facebook under review as well, looking at the company's
position in online advertising. But that investigation appears to be
moving slowly.

State investigators have been
\href{https://www.nytimes3xbfgragh.onion/2020/06/12/technology/state-inquiry-antitrust-amazon.html}{looking
into whether Amazon abuses its power} over sellers on the tech giant's
site. The F.T.C. is also investigating Amazon, but that appears to be
moving slowly.

--- \href{https://www.nytimes3xbfgragh.onion/by/cecilia-kang}{Cecilia
Kang}

\hypertarget{advertisement-2}{%
\subsubsection{Advertisement}\label{advertisement-2}}

\protect\hyperlink{after-dfp-ad-mid3}{Continue reading the main story}

\hypertarget{trump-administration-asks-fcc-to-narrow-protections-for-tech-companies}{%
\subsection{\texorpdfstring{\protect\hyperlink{trump-administration-asks-fcc-to-narrow-protections-for-tech-companies}{Trump
administration asks F.C.C. to narrow protections for tech
companies.}}{Trump administration asks F.C.C. to narrow protections for tech companies.}}\label{trump-administration-asks-fcc-to-narrow-protections-for-tech-companies}}

Copied to clipboard.

The Trump administration asked the Federal Communications Commission
this week to narrow its interpretation of a law that shields internet
platforms like Facebook and YouTube from certain lawsuits over the
content they host.

The request, which stems from an executive order President Trump signed
in May, is part of a growing push by the president and his allies, who
say that tech companies are removing or suppressing conservative
content. Despite evidence that conservative sites and figures perform
well online, the president, along with much of his conservative base,
have repeatedly criticized the platforms over instances in which
conservative content was removed or otherwise moderated for violating a
platform's rules.

In a petition on Monday, the Department of Commerce asked the commission
to clarify that the law, known as Section 230, does not protect a
platform when it moderates or highlights user content based on a
``reasonably discernible viewpoint or message, without having been
prompted to, asked to, or searched for by the user.'' It would also
limit the circumstances under which platforms are protected from
liability over their users' content.

Kayleigh McEnany, the White House spokeswoman, said in a statement on
Wednesday morning that the president wants the F.C.C. ``to clarify that
Section 230 does not permit social media companies that alter or
editorialize users' speech to escape civil liability.''

Mr. Trump weighed in later on Twitter:

\begin{quote}
If Congress doesn't bring fairness to Big Tech, which they should have
done years ago, I will do it myself with Executive Orders. In
Washington, it has been ALL TALK and NO ACTION for years, and the people
of our Country are sick and tired of it!

--- Donald J. Trump (@realDonaldTrump)
\href{https://twitter.com/realDonaldTrump/status/1288506554585505793?ref_src=twsrc\%5Etfw}{July
29, 2020}
\end{quote}

The petition is now in the hands of the F.C.C., an independent agency
currently led by a Republican chairman, Ajit Pai, who was appointed to
the position by Mr. Trump. ``The F.C.C. will carefully review the
petition,'' said Brian Hart, a spokesman for the commission.

--- \href{https://www.nytimes3xbfgragh.onion/by/david-mccabe}{David
McCabe}

\hypertarget{heres-a-tally-of-the-ceos-catchphrases-follow-along-as-we-update-it-live}{%
\subsection{\texorpdfstring{\protect\hyperlink{what-ceos-said}{Here's a
tally of the C.E.O.s' catchphrases. Follow along as we update it
live.}}{Here's a tally of the C.E.O.s' catchphrases. Follow along as we update it live.}}\label{heres-a-tally-of-the-ceos-catchphrases-follow-along-as-we-update-it-live}}

Copied to clipboard.

How often do tech titans repeat themselves? How many times will the
chief executives fall back on buzzwords and catchphrases? And how
frequently will they bring up their rivals (TikTok! Walmart! Each
other!) to downplay their companies' power?

To answer these questions, we're keeping track of how often Jeff Bezos
of Amazon, Sundar Pichai of Google, Tim Cook of Apple and Mark
Zuckerberg of Facebook use certain arguments and phrases throughout the
course of the antitrust hearing. Follow along with us here.

\hypertarget{we-are-not-that-big}{%
\subsubsection{We Are Not That Big}\label{we-are-not-that-big}}

\hypertarget{each-time-a-ceo-argues-that-his-company-is-not-actually-that-powerful-because-its-market-share-is-small-or-its-influence-is-limited}{%
\paragraph{Each time a C.E.O. argues that his company is not actually
that powerful because its market share is small or its influence is
limited.}\label{each-time-a-ceo-argues-that-his-company-is-not-actually-that-powerful-because-its-market-share-is-small-or-its-influence-is-limited}}

\begin{longtable}[]{@{}ll@{}}
\toprule
\textbf{On Repeat} & \textbf{Count}\tabularnewline
\midrule
\endhead
Mark Zuckerberg & 1\tabularnewline
Jeff Bezos & 1\tabularnewline
Tim Cook & 2\tabularnewline
Sundar Pichai & 1\tabularnewline
\bottomrule
\end{longtable}

\hypertarget{we-are-good-for-america}{%
\subsubsection{We Are Good for America}\label{we-are-good-for-america}}

\hypertarget{each-time-a-ceo-boasts-about-how-his-company-has-added-jobs-fueled-economic-growth-accelerated-innovation-or-otherwise-helped-the-country}{%
\paragraph{Each time a C.E.O. boasts about how his company has added
jobs, fueled economic growth, accelerated innovation or otherwise helped
the
country.}\label{each-time-a-ceo-boasts-about-how-his-company-has-added-jobs-fueled-economic-growth-accelerated-innovation-or-otherwise-helped-the-country}}

\begin{longtable}[]{@{}ll@{}}
\toprule
\textbf{On Repeat} & \textbf{Count}\tabularnewline
\midrule
\endhead
Mark Zuckerberg & 4\tabularnewline
Jeff Bezos & 3\tabularnewline
Tim Cook & 3\tabularnewline
Sundar Pichai & 9\tabularnewline
\bottomrule
\end{longtable}

\hypertarget{we-will-get-back-to-you}{%
\subsubsection{We Will Get Back to You}\label{we-will-get-back-to-you}}

\hypertarget{each-time-a-ceo-doesnt-directly-answer-a-question-saying-instead-that-he-will-respond-after-the-company-looks-into-the-matter}{%
\paragraph{Each time a C.E.O. doesn't directly answer a question, saying
instead that he will respond after the company looks into the
matter.}\label{each-time-a-ceo-doesnt-directly-answer-a-question-saying-instead-that-he-will-respond-after-the-company-looks-into-the-matter}}

\begin{longtable}[]{@{}ll@{}}
\toprule
\textbf{On Repeat} & \textbf{Count}\tabularnewline
\midrule
\endhead
Mark Zuckerberg & 0\tabularnewline
Jeff Bezos & 0\tabularnewline
Tim Cook & 0\tabularnewline
Sundar Pichai & 0\tabularnewline
\bottomrule
\end{longtable}

\hypertarget{we-are-not-the-ones-to-worry-about}{%
\subsubsection{We Are Not the Ones to Worry
About}\label{we-are-not-the-ones-to-worry-about}}

\hypertarget{each-time-a-ceo-tries-to-shift-attention-by-citing-a-competitor-or-the-specter-of-how-china-could-dominate-tech-if-their-own-companies-are-curtailed}{%
\paragraph{Each time a C.E.O. tries to shift attention by citing a
competitor or the specter of how China could dominate tech if their own
companies are
curtailed.}\label{each-time-a-ceo-tries-to-shift-attention-by-citing-a-competitor-or-the-specter-of-how-china-could-dominate-tech-if-their-own-companies-are-curtailed}}

\begin{longtable}[]{@{}ll@{}}
\toprule
\textbf{On Repeat} & \textbf{Count}\tabularnewline
\midrule
\endhead
Mark Zuckerberg & 6\tabularnewline
Jeff Bezos & 4\tabularnewline
Tim Cook & 4\tabularnewline
Sundar Pichai & 0\tabularnewline
\bottomrule
\end{longtable}

--- \href{https://www.nytimes3xbfgragh.onion/by/kellen-browning}{Kellen
Browning}

\hypertarget{what-to-expect-from-the-hearing}{%
\subsection{\texorpdfstring{\protect\hyperlink{what-to-expect-from-the-hearing}{What
to expect from the
hearing.}}{What to expect from the hearing.}}\label{what-to-expect-from-the-hearing}}

Copied to clipboard.

\includegraphics{https://static01.graylady3jvrrxbe.onion/images/2020/07/29/business/29tech-hearing-ledeall/29tech-hearing-ledeall-articleLarge.jpg?quality=75\&auto=webp\&disable=upscale}

After lawmakers collected hundreds of hours of interviews and obtained
more than 1.3 million documents about Amazon, Apple, Facebook and
Google, their chief executives will testify before Congress at 1 p.m. on
Wednesday to defend their powerful businesses.

The captains of the New Gilded Age ---
\href{https://www.nytimes3xbfgragh.onion/2020/07/27/business/jeff-bezos-amazon-congress.html}{Jeff
Bezos of Amazon}, Tim Cook of Apple, Mark Zuckerberg of Facebook and
Sundar Pichai of Google --- will appear together before Congress for the
first time to justify their business practices. Members of the House
judiciary's antitrust subcommittee
\href{https://www.nytimes3xbfgragh.onion/2019/06/11/technology/antitrust-hearing.html}{have
investigated the internet giants} for more than a year on accusations
that they have stifled rivals and harmed consumers. The exact contents
of the documents they've collected are unknown, although they are said
to include documents related to some of the companies' acquisitions and
internal communications among top executives.

It is set to be a bizarre spectacle, with four men who run companies
worth nearly \$5 trillion combined --- and who include two of the
world's richest individuals --- primed to argue that their businesses
are not really that powerful after all.

And it will be a first in another way: Mr. Zuckerberg, Mr. Pichai, Mr.
Bezos and Mr. Cook will all be testifying via videoconference, rather
than rising side-by-side for a swearing-in at a witness table in
Washington.

At the hearing, the 15 members of the antitrust subcommittee will have
five minutes for each question. Representative David Cicilline, Democrat
of Rhode Island and the chairman of the subcommittee, will control the
number of rounds of questioning, potentially stretching the hearing into
the evening.

The antitrust issues facing Apple, Facebook,
\href{https://www.nytimes3xbfgragh.onion/2019/06/02/business/google-antitrust-investigation.html}{Google
and Amazon} are complex and vastly different.

Amazon is accused of abusing its role as both a retailer and a platform
hosting third-party sellers on its marketplace. Apple has been accused
of unfairly using its clout over its App Store to block rivals and to
force apps to pay high commissions. Rivals have said Facebook has a
monopoly in social networking. Alphabet, the parent company of Google,
is dealing with multiple antitrust allegations because of Google's
dominance in online advertising, search and smartphone software.

Democrats may also veer off the topic of antitrust and bring up concerns
about misinformation on social media. Some Republicans are expected to
sidetrack discussion with their concerns of liberal bias at the Silicon
Valley companies and accusations that conservative voices are censored.

--- \href{https://www.nytimes3xbfgragh.onion/by/cecilia-kang}{Cecilia
Kang}, \href{https://www.nytimes3xbfgragh.onion/by/jack-nicas}{Jack
Nicas} and
\href{https://www.nytimes3xbfgragh.onion/by/david-mccabe}{David McCabe}

\hypertarget{advertisement-3}{%
\subsubsection{Advertisement}\label{advertisement-3}}

\protect\hyperlink{after-dfp-ad-mid1}{Continue reading the main story}

\hypertarget{todays-hearing-has-echoes-of-bill-gates-22-years-ago}{%
\subsection{\texorpdfstring{\protect\hyperlink{todays-hearing-has-echoes-of-bill-gates-22-years-ago}{Today's
hearing has echoes of Bill Gates, 22 years
ago.}}{Today's hearing has echoes of Bill Gates, 22 years ago.}}\label{todays-hearing-has-echoes-of-bill-gates-22-years-ago}}

Copied to clipboard.

\includegraphics{https://static01.graylady3jvrrxbe.onion/images/2020/08/03/business/03tech-hearing-gates/03tech-hearing-gates-articleLarge.jpg?quality=75\&auto=webp\&disable=upscale}

The tech industry is an engine of innovation, job creation and American
economic prowess. Competition is flourishing, and just a click away.
Sure, we do well, but consumers are the big winners.

That was the gist of
\href{https://archive.nytimes3xbfgragh.onion/www.nytimes3xbfgragh.onion/library/tech/98/03/biztech/articles/04microsoft.html}{Bill
Gates's testimony} before a Senate panel more than two decades ago. And
it's a safe bet the same themes will feature prominently when the
leaders of Amazon, Apple, Facebook and Google testify on Wednesday.

There are differences, but this week's appearance by tech executives is
reminiscent of the congressional grilling Microsoft's chief faced 22
years ago.

In 1998, the spotlight was squarely on Mr. Gates, co-founder of
Microsoft, the tech behemoth of the personal computer era. This time,
the leaders of four big technology companies will be in the dock,
appearing remotely because of a pandemic.

Today, more issues are in play. In the late 1990s, the concern was that
Microsoft would use its dominance in the PC market to stifle internet
upstarts. The sheer market muscle of today's tech giants is a worry, but
so is the role they play broadly in commerce and communication,
influencing public opinion and politics.

When Mr. Gates testified, a formal investigation of Microsoft by federal
regulators and dozens of states was well underway. The same is true now
for Google and Facebook, while Amazon and Apple are also facing
antitrust scrutiny.

There can be gotcha moments. Under pointed questioning, Mr. Gates
rhetorically bobbed and weaved, refusing to use the M-word: monopoly.

But when Jim Barksdale, head of Netscape, the internet company most in
Microsoft's sights, testified that day, he asked the spectators to raise
their hands if they used a PC.

About three-quarters of the room did. Then, how many of them used
Microsoft's Windows operating system? Almost the same number of hands
flew up again.

``That,'' Mr. Barksdale said, ``is a monopoly.''

--- \href{https://www.nytimes3xbfgragh.onion/by/steve-lohr}{Steve Lohr}

\hypertarget{site-index}{%
\subsection{Site Index}\label{site-index}}

\hypertarget{site-information-navigation}{%
\subsection{Site Information
Navigation}\label{site-information-navigation}}

\begin{itemize}
\tightlist
\item
  \href{https://help.nytimes3xbfgragh.onion/hc/en-us/articles/115014792127-Copyright-notice}{©~2020~The
  New York Times Company}
\end{itemize}

\begin{itemize}
\tightlist
\item
  \href{https://www.nytco.com/}{NYTCo}
\item
  \href{https://help.nytimes3xbfgragh.onion/hc/en-us/articles/115015385887-Contact-Us}{Contact
  Us}
\item
  \href{https://www.nytco.com/careers/}{Work with us}
\item
  \href{https://nytmediakit.com/}{Advertise}
\item
  \href{http://www.tbrandstudio.com/}{T Brand Studio}
\item
  \href{https://www.nytimes3xbfgragh.onion/privacy/cookie-policy\#how-do-i-manage-trackers}{Your
  Ad Choices}
\item
  \href{https://www.nytimes3xbfgragh.onion/privacy}{Privacy}
\item
  \href{https://help.nytimes3xbfgragh.onion/hc/en-us/articles/115014893428-Terms-of-service}{Terms
  of Service}
\item
  \href{https://help.nytimes3xbfgragh.onion/hc/en-us/articles/115014893968-Terms-of-sale}{Terms
  of Sale}
\item
  \href{https://spiderbites.nytimes3xbfgragh.onion}{Site Map}
\item
  \href{https://help.nytimes3xbfgragh.onion/hc/en-us}{Help}
\item
  \href{https://www.nytimes3xbfgragh.onion/subscription?campaignId=37WXW}{Subscriptions}
\end{itemize}
