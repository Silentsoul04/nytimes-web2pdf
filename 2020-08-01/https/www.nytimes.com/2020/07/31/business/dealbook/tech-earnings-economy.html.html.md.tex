Sections

SEARCH

\protect\hyperlink{site-content}{Skip to
content}\protect\hyperlink{site-index}{Skip to site index}

\href{https://myaccount.nytimes.com/auth/login?response_type=cookie\&client_id=vi}{}

\href{https://www.nytimes.com/section/todayspaper}{Today's Paper}

\href{/section/business/dealbook}{DealBook}\textbar{}They Made How Much?

\url{https://nyti.ms/3hWMCJA}

\begin{itemize}
\item
\item
\item
\item
\item
\end{itemize}

Advertisement

\protect\hyperlink{after-top}{Continue reading the main story}

Supported by

\protect\hyperlink{after-sponsor}{Continue reading the main story}

DealBook Business and Policy

DealBook Newsletter

\hypertarget{they-made-how-much}{%
\section{They Made How Much?}\label{they-made-how-much}}

\includegraphics{https://static01.nyt.com/images/2020/07/31/business/31db-newsletter-tech/merlin_143699331_c29bfe0a-ad1b-4166-b608-8b284713156f-articleLarge.jpg?quality=75\&auto=webp\&disable=upscale}

July 31, 2020

\begin{itemize}
\item
\item
\item
\item
\item
\end{itemize}

\emph{You won't want to miss our next DealBook Debrief call, where we
will discuss}
\emph{\href{https://timesevents.nytimes.com/dealbookdebrief0806}{\emph{racial
injustice and corporate America}}} \emph{\emph{with special guest Nikole
Hannah-Jones, The Times's domestic correspondent and creator of}}
\emph{\href{https://www.nytimes.com/interactive/2019/08/14/magazine/1619-america-slavery.html}{\emph{The
1619 Project}}}. What more is needed from the business community to
produce lasting change? Join us on Thursday, Aug. 6 at 11 a.m. Eastern
to find out.**
\emph{\href{https://timesevents.nytimes.com/dealbookdebrief0806}{\emph{R.S.V.P.
here}}}. (Want this delivered to your inbox each day?**
\href{http://p.nytimes.com/email/re?location=hdaNaYedr2/IomeWRKt0nffrak8aSGLbQgZx2X6d46QTPfqP1YU5VFs7ogbYiA6geS/dWohdr/Cx4ICfruKKr2xWO3SrRrXz/UnL21ee4RfTsG6Iu6Pi4Vkjv2FNF/lShp8PWKHi4Hr3K9P4/20g1T7hdv9TGo4mWrWdQtjWltpAYrgkCWi4BUsjOsCggHxLuH78WvYURHt22PZDz5BTHU4Eoe8mhdVMdiIkF990xuDQn5eAZPep+/pyFTL/tvm10WIUAo7GiLm9BxSf48VWkcf32z4mKIynkgqkqPLcwrXrPfY6X6/90c1ST1rGGrmJnRt6JgODhB2PXERgSBxx0HiQjPczGK0FdRZfUpCZnJM3hkVVWY5pBphIWP4l17u/+7gnJoDxda0LOMJqbKXNrv72oJVMfG+5UYYab3+1pzaVxdgaCxNityNQ46NiCJL5UelWrB2wIT53X4SLL93iP/AfyDtqFhF2dL2Q47tnvHslU1eZCU05+P4/1mWonIWQ5nZ6aKGnW42FfyL2sEuvM9pqFUwGCW86ZeoCmEOm619h7Iq+egrxxNmh3gt3ZmpVQnOFMTCq2g929XZGwgQ/2dhV26lso+m/\&campaign_id=0\&instance_id=0\&segment_id=0\&user_id=f4577a56bcf773214e9a1d55b5fe9d24\&regi_id=0nl=dealbook\&emc=edit_dk_20180529}{\emph{Sign
up here}}\emph{.)}

\hypertarget{thats-a-lot-of-money}{%
\subsection{That's a lot of money}\label{thats-a-lot-of-money}}

``It's certainly been a busy week, and I'm glad to be here,'' Sundar
Pichai told investors, as he presented Alphabet's latest earnings. The
day after he and his counterparts at Amazon, Apple and Facebook faced a
\href{https://www.nytimes.com/2020/07/30/business/dealbook/tech-hearing-highlights.html}{five-hour
congressional grilling}, the tech giants all unveiled
better-than-expected profits --- much, much better than expected.

\textbf{Combined, Alphabet, Amazon, Apple and Facebook earned more than
\$28 billion.} In after-hours trading, the companies
\href{https://www.bloomberg.com/news/articles/2020-07-30/a-day-after-congress-grilling-big-tech-stocks-add-250-billion}{added
\$250 billion} to their market caps, according to Bloomberg, pushing the
total value of the firms above \$5 trillion, or about a fifth of the
entire S\&P 500.

\textbf{Talk about good timing.} Focusing on
\href{https://www.nytimes.com/2020/07/30/technology/tech-company-earnings-amazon-apple-facebook-google.html?action=click\&module=Top\%20Stories\&pgtype=Homepage}{the
bumper haul} was a relief for the executives, who were on the defensive
in front of lawmakers. It was also probably better to unveil such gaudy
numbers \emph{after} answering pointed questions about their
\href{https://www.nytimes.com/2020/07/30/technology/big-tech-ceos.html}{outsize
power} and the potential for abuse of their dominant positions. The
earnings awkwardly followed a series of grim economic statistics
released earlier in the day --- more on that below --- showing a steep
drop in U.S. economic output and millions of workers struggling to find
jobs.

• \textbf{Alphabet} reported its first-ever decline in quarterly
revenue, hurt by a slowdown in advertising, but its \$7 billion profit
comfortably beat analyst forecasts. Mr. Pichai was the most forthcoming
with investors about the regulatory scrutiny that Big Tech faces. ``We
realize, at our scale, that's appropriate,'' he said.

• \textbf{Amazon} had warned investors to expect profits to be wiped out
by \$4 billion in coronavirus-related costs. Although the firm spent as
much as promised, sales grew so fast that it turned a \$5 billion profit
for the second quarter, double the result of the year before. Its
second-quarter sales of \$89 billion were \$8 billion more than analysts
expected.

• \textbf{Facebook} reported that profits doubled, to \$5 billion, as
its monthly active users rose by 12 percent, to 2.7 billion. A boycott
by big advertisers could hurt its bottom line, the company warned, but
it said that sales so far are holding up in the current quarter.

• \textbf{Apple} said that its profit rose by 12 percent, to more than
\$11 billion. Its service business performed particularly well,
including sales from the App Store, a particular focus of antitrust
interest. The congressional committee investigating the tech giants
released a trove of internal emails this week, including
\href{https://www.theverge.com/2020/7/30/21348130/apple-documents-steve-jobs-email-books-amazon-apps-antitrust-investigation-schiller}{some
from the Apple co-founder Steve Jobs} dating to the early 2010s about
how to prevent Apple users from buying e-books via Amazon's apps.

\textbf{To finish, a quiz:} What do the annual G.D.P. of Qatar, market
cap of Cisco Systems and Apple's second-quarter cash balance all have in
common? They're all around \$195 billion.

\includegraphics{https://static01.nyt.com/images/2020/07/31/business/31db-newsletter-roundup/merlin_175138851_3bf9c439-fda7-4b42-a0d3-7694a349b2a3-articleLarge.jpg?quality=75\&auto=webp\&disable=upscale}

\hypertarget{heres-what-is-happening}{%
\subsection{Here's what is happening}\label{heres-what-is-happening}}

\textbf{President Trump stokes outrage by suggesting a delay to the
election.} On Twitter, he questioned whether the elections on Nov. 3
should proceed as planned,
\href{https://www.nytimes.com/2020/07/30/us/politics/trump-delay-election.html}{drawing
condemnation} even from allies in Congress. Here's why
\href{https://www.nytimes.com/2020/07/30/us/politics/trump-postpone-election.html}{he
has no power} to make that happen.

\textbf{Vaccine makers race to build supply lines.} Drug companies face
pressure not just to test their coronavirus vaccines quickly, but also
to
\href{https://www.wsj.com/articles/drugmakers-race-to-build-covid-19-vaccine-supply-chains-11596101586}{produce
them in huge quantities}. Separately, Gilead said it hopes to produce
enough remdesivir, its Covid-19 treatment, to meet global demand
\href{https://www.ft.com/content/64572a4a-3237-4d62-9928-d90ba49d8c8d}{by
October}.

\textbf{Corporate America can't quit buybacks.} While share repurchases
are expected to drop this year, S\&P 500 companies are
\href{https://www.ft.com/content/1c924be0-5bc0-4eba-a088-b98b13080c04}{still
buying back shares} in the second quarter, in some cases more than in
the first, The Financial Times reports.

\textbf{Basketball is back.} The N.B.A.
\href{https://www.nytimes.com/2020/07/30/sports/basketball/clippers-lakers.html}{resumed
its season} yesterday from its bubble at Walt Disney World with a pair
of nail-biters. (And all the players, coaches and referees
\href{https://sports.yahoo.com/nba-players-referees-all-kneel-as-national-anthem-plays-before-jazz-pelicans-bubble-tipoff-225215284.html}{took
a knee during the national anthems}.) The question now is whether the
league's quarantined approach will spare it the problems bedeviling
Major League Baseball, as
\href{https://www.nytimes.com/2020/07/30/sports/baseball/phillies-blue-jays-postponed-coronavirus.html}{more
games were postponed} after team employees tested positive for the
coronavirus.

\textbf{SpaceX faces its next test.} The two astronauts who traveled to
the International Space Station on the company's rocket are
\href{https://www.businessinsider.com/watch-live-spacex-crew-dragon-returns-to-earth-august-2-2020-7}{scheduled
to return to Earth} this weekend. (Assuming Tropical Storm Isaias
doesn't force a postponement.) Elon Musk has said that the Crew Dragon's
re-entry into the atmosphere is his biggest concern. You can watch the
journey live \href{https://www.youtube.com/watch?v=21X5lGlDOfg}{via
NASA}.

Image

\hypertarget{five-years-of-growth-vanishes-in-months}{%
\subsection{Five years of growth vanishes in
months}\label{five-years-of-growth-vanishes-in-months}}

The U.S. economy recorded one its
\href{https://twitter.com/jimtankersley/status/1288846477448617985}{most
severe declines in history} in the second quarter, as the pandemic
erased years of growth with terrifying speed. Output
\href{https://www.nytimes.com/2020/07/30/business/economy/q2-gdp-coronavirus-economy.html}{fell
by nearly 10 percent} in the three months to June; during the 2008
recession, G.D.P. fell by ``only'' 4 percent over an 18-month span.

\textbf{It could be worse.} Aggressive stimulus programs have cushioned
the blow, as reflected in this extraordinary chart showing how programs
like extra unemployment insurance and payroll support have bolstered
incomes during the downturn, breaking the usual relationship between
income and economic output.

Image

\textbf{It might get worse.} U.S. lawmakers have
\href{https://www.nytimes.com/2020/07/30/us/politics/senate-virus-aid.html}{been
unable to agree} on an extension to supplemental unemployment benefits,
which expire today. More than a million people have submitted new
jobless claims for 19 consecutive weeks, with signs that the recovery in
the labor market is stalling as coronavirus cases rise. Without extra
stimulus,
\href{https://www.washingtonpost.com/business/2020/07/30/economists-favor-big-stimulus/}{economists
warn}, the shock of the pandemic could do permanent damage to the
economy. In other words, the blue line in the chart will follow the red
one downwards, feeding an economic spiral.

\emph{It's not any better in Europe, which today reported its}
\emph{\href{https://www.nytimes.com/live/2020/07/31/business/stock-market-today-coronavirus\#europes-economic-contraction-is-its-worst-on-record}{worst
economic contraction on record}, with eurozone G.D.P. shrinking by
around 12 percent in the second quarter.}

\hypertarget{overheard-on-earnings-calls}{%
\subsection{Overheard on earnings
calls}\label{overheard-on-earnings-calls}}

🍺 ``To put a finer point in the level of demand we're seeing, we
eclipsed July 4 week shipment days in the United States four times
already this year. That's unheard of.'' \emph{--- Gavin Hattersley, the
C.E.O. of Molson Coors}

🇯🇵 ``We would be in Tokyo right now under normal circumstances. So it's
a total bummer for our company that we don't have the Olympics.''
\emph{--- Jeff Shell, the C.E.O. of NBCUniversal}

🥣 ``Special K gained share in quarter two as did Mini-Wheats and Raisin
Bran. We are also excited about the consumer trial and rediscovery we
are seeing from new and lapsed users in cereal.'' --- \emph{Steven
Cahillane, the C.E.O. of Kellogg's}

🧔 ``As people go back to work in offices and outside the home, we'll see
a pickup in the wet shave rate.'' --- \emph{David Taylor, the C.E.O. of
Procter \& Gamble, in response to an analyst question about the rise of}
\href{https://www.nytimes.com/2020/07/30/business/coronabeards-aside-procter-gamble-saw-strong-sales.html}{\emph{``coronabeards''
and mullets}} \emph{during lockdowns}

🍩 ``I love when we really get on our doughnut mojo, but look, we are
leaning into beverages in a big way.'' \emph{--- David Hoffmann, the
C.E.O. of Dunkin' Brands}

Image

The Trump administration appears to trying ``to bring down the Communist
Party'' in China, according to The Times columnist Tom
Friedman.Credit...Kevin Lamarque/Reuters

\hypertarget{innovation-on-steroids}{%
\subsection{`Innovation on steroids'}\label{innovation-on-steroids}}

The pandemic has disrupted life as we knew it, and the Trump
administration has upended American policies along with it. The Times
foreign affairs columnist Tom Friedman joined us for a DealBook Debrief
call with readers to make sense of it all.
(\href{https://youtu.be/4SykcA8jKhQ}{Here's the recording}, if you
missed it.) Some highlights:

\textbf{• ``I think he's out for regime change.''}

President Trump has broken countless norms of political life, both at
home and abroad, in ways that Tom said worried him greatly. One area of
particular concern is China policy, where Mr. Trump's hard-line approach
is both too headstrong and counterproductive. Of Beijing's leaders, Tom
said, ``they know as long as Trump is president, he can never galvanize
the global coalition'' against them.

\textbf{• ``We are going to see innovation on steroids.''}

Not all of the upheaval of recent months has been bad. While
acknowledging the toll of the pandemic on lives and livelihoods, Tom
said that changes brought about in response to the coronavirus have set
us up for ``one of the most massive, amazing, creative and destructive
periods of human history.'' Innovative thinking, aided by advanced and
cheap technology, could bring major changes to how we live and work.

\textbf{• ``It's flatter than ever.''}

Has the pandemic wiped out the idea of globalization? Tom --- whose book
``The World Is Flat'' examined the increasing interconnectedness of the
21st century --- doesn't think so. ``It's flatter than ever because when
it comes to globalization, I am a technological determinist,'' he said,
pointing to the ubiquity of smartphones. ``Technology is not just
interconnecting the world,'' he added. ``It's actually making the world
interdependent.''

Image

\hypertarget{weekend-reading-a-world-without-friction}{%
\subsection{Weekend reading: A world without
friction}\label{weekend-reading-a-world-without-friction}}

One thing that the pandemic has imposed on many businesses is friction.
From sourcing supplies to serving customers, everything has become
harder, for public health-related reasons. Perhaps, then, it's not the
best time to publish a book titled
``\href{https://www.harpercollins.com/products/frictionless-christiane-lemieuxduff-mcdonald}{Frictionless:
Why the Future of Everything Will Be Fast, Fluid, and Made Just for
You}.''

The co-authors, Christiane Lemieux and Duff McDonald, are undeterred.
Ms. Lemieux is a designer, entrepreneur and former creative director of
Wayfair; Mr. McDonald is familiar to DealBook readers as the author of
books about
\href{https://www.nytimes.com/2009/11/01/books/review/Barrett-t.html}{Jamie
Dimon},
\href{https://dealbook.nytimes.com/2013/09/02/in-a-new-book-mckinsey-co-isnt-all-roses/}{McKinsey}
and
\href{https://www.nytimes.com/2017/04/10/business/dealbook/11-andrew-sorkin-harvard-business-school.html}{Harvard
Business School}. So what can a book with case studies about ``companies
that thrive in the world of frictionless commerce'' reveal about the
friction-filled time we now live in? We asked, and they answered:

\begin{quote}
When Covid-19 hit, the concept of frictionlessness was battled-tested in
the biggest possible way. Everyone was at home. Could that website
handle the increased traffic? Was it able to adjust to a disrupted
supply chain? Could it communicate with customers worried about getting
crucial items like toilet paper or food? The companies that have been up
to the task already understood the centrality of the concept before this
national nightmare began.

Why? Because those companies were ready for Covid-19, even if no one
could have predicted it. They were already in the flow, making all the
right and thoughtful moves. They were institutionalizing the concept of
community. They were building diverse --- and dispersed --- workforces.
They made respect for each other a condition of employment, not an
afterthought.
\end{quote}

\hypertarget{the-speed-read}{%
\subsection{The speed read}\label{the-speed-read}}

\textbf{Deals}

• Saudi Arabia's sovereign wealth fund withdrew a \$400 million takeover
bid for the English soccer club Newcastle United.
(\href{https://www.nytimes.com/2020/07/30/sports/soccer/saudi-arabia-newcastle-united.html}{NYT})

• Affirm, the e-commerce lender, has reportedly hired Goldman Sachs to
lead work on an I.P.O. that could value it at up to \$10 billion.
(\href{https://www.wsj.com/articles/affirm-prepares-ipo-that-could-value-fintech-firm-at-up-to-10-billion-11596143292}{WSJ})

• Europe's markets are having a moment, after years of being shunned by
investors.
(\href{https://www.nytimes.com/2020/07/30/business/europes-markets-are-having-a-moment.html}{NYT})

\textbf{Politics and policy}

• The
\href{https://www.nytimes.com/2020/07/30/us/politics/herman-cain-dead.html}{death
of Herman Cain}, the former pizza chain C.E.O. and Republican
presidential candidate, from complications of Covid-19 has
\href{https://www.nytimes.com/2020/07/30/us/politics/herman-cain-gop-coronavirus.html}{set
off speculation} about whether more Republicans will drop their
opposition to face masks. (NYT)

\textbf{Tech}

• A Singaporean's use of LinkedIn to try to recruit targets for Beijing
intelligence highlights China's increasingly sophisticated online
espionage efforts.
(\href{https://www.ft.com/content/0a0e62a9-65ba-494c-a7bb-86f5f66d627f}{FT})

• Google's growing acceptance of remote working in the pandemic could
mean the end of its famously lavish office perks.
(\href{https://www.businessinsider.com/google-reimagining-future-work-environment-culture-2020-7}{Business
Insider})

\textbf{Best of the rest}

• ``Though I am gone, I urge you to answer the highest calling of your
heart and stand up for what you truly believe'': An Op-Ed that the civil
rights leader and congressman John Lewis wrote to be published on the
day of his funeral.
(\href{https://www.nytimes.com/2020/07/30/opinion/john-lewis-civil-rights-america.html}{NYT})

• How Pimco's Cayman Islands-based hedge fund profited from the Fed's
rescue programs.
(\href{https://www.nytimes.com/2020/07/30/business/economy/fed-talf-wall-street.html}{NYT})

• ``Eight Shocking Secrets I Learned While Working on Private Jets''
(\href{https://www.bloomberg.com/news/features/2020-07-30/private-jet-secrets-from-sex-to-pets-to-celebrity-bad-behavior?sref=0w5HLLb3}{Bloomberg
Businessweek})

\emph{We'd love your feedback. Please email thoughts and suggestions to}
\href{mailto:dealbook@nytimes.com?subject=Newsletter\%20Feedback}{\emph{dealbook@nytimes.com}}\emph{.}

Advertisement

\protect\hyperlink{after-bottom}{Continue reading the main story}

\hypertarget{site-index}{%
\subsection{Site Index}\label{site-index}}

\hypertarget{site-information-navigation}{%
\subsection{Site Information
Navigation}\label{site-information-navigation}}

\begin{itemize}
\tightlist
\item
  \href{https://help.nytimes.com/hc/en-us/articles/115014792127-Copyright-notice}{©~2020~The
  New York Times Company}
\end{itemize}

\begin{itemize}
\tightlist
\item
  \href{https://www.nytco.com/}{NYTCo}
\item
  \href{https://help.nytimes.com/hc/en-us/articles/115015385887-Contact-Us}{Contact
  Us}
\item
  \href{https://www.nytco.com/careers/}{Work with us}
\item
  \href{https://nytmediakit.com/}{Advertise}
\item
  \href{http://www.tbrandstudio.com/}{T Brand Studio}
\item
  \href{https://www.nytimes.com/privacy/cookie-policy\#how-do-i-manage-trackers}{Your
  Ad Choices}
\item
  \href{https://www.nytimes.com/privacy}{Privacy}
\item
  \href{https://help.nytimes.com/hc/en-us/articles/115014893428-Terms-of-service}{Terms
  of Service}
\item
  \href{https://help.nytimes.com/hc/en-us/articles/115014893968-Terms-of-sale}{Terms
  of Sale}
\item
  \href{https://spiderbites.nytimes.com}{Site Map}
\item
  \href{https://help.nytimes.com/hc/en-us}{Help}
\item
  \href{https://www.nytimes.com/subscription?campaignId=37WXW}{Subscriptions}
\end{itemize}
