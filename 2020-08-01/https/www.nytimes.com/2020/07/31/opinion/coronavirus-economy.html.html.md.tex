Sections

SEARCH

\protect\hyperlink{site-content}{Skip to
content}\protect\hyperlink{site-index}{Skip to site index}

\href{https://myaccount.nytimes.com/auth/login?response_type=cookie\&client_id=vi}{}

\href{https://www.nytimes.com/section/todayspaper}{Today's Paper}

\href{/section/opinion}{Opinion}\textbar{}3 Things to Make the World
Immediately Better After Covid-19

\url{https://nyti.ms/2Dmtw0p}

\begin{itemize}
\item
\item
\item
\item
\item
\end{itemize}

Advertisement

\protect\hyperlink{after-top}{Continue reading the main story}

\href{/section/opinion}{Opinion}

Supported by

\protect\hyperlink{after-sponsor}{Continue reading the main story}

\hypertarget{3-things-to-make-the-world-immediately-better-after-covid-19}{%
\section{3 Things to Make the World Immediately Better After
Covid-19}\label{3-things-to-make-the-world-immediately-better-after-covid-19}}

The crisis could be a catalyst for overhauling the economic world order.

By Dambisa Moyo

Dr. Moyo is an economist.

\begin{itemize}
\item
  July 31, 2020
\item
  \begin{itemize}
  \item
  \item
  \item
  \item
  \item
  \end{itemize}
\end{itemize}

\includegraphics{https://static01.nyt.com/images/2020/07/31/opinion/31Moyo/31Moyo-articleLarge.jpg?quality=75\&auto=webp\&disable=upscale}

The scale of the coronavirus pandemic and the economic shutdowns it
caused set in motion a series of debates and questions about what the
world may look like once its stranglehold on society loosens: Will we
travel less? Will we work at home more? Will norms in schools and at
large-scale public events be changed for years?

Less noticed, but just as important, is the potential that the
coronavirus could be a catalyst to overhaul the global economic order. A
debate on the failures of the global economy had already started before
the pandemic, born of a sense that capitalism and corporations had
become parasites on the planet.

In the aftermath of the pandemic, the world's rich countries should do
more than just wait for corporations to change. They have to overhaul
their monetary policies, the forms of private investment they
incentivize and the attitudes of their antitrust enforcement.

Until now, monetary policy has rewarded holders of financial assets over
those who have stock in real assets like land, factories and labor.
That's because the world's most powerful central banks have prioritized
controlling inflation over expanding industrial capacity and employment
in what's called the ``real economy.''

This status quo in central banking, which has been dominant for four
decades, has encouraged corporations, especially the largest publicly
traded companies, to focus on short-term financial gains and share
prices at the expense of pursuing longer-term investments that would
reap more broadly shared rewards. Compounding the gains of those who
already own plenty of capital has resulted in the entrenched income
inequality and stagnant wages that citizens in dozens of countries
bemoan.

In the United States, the Federal Reserve is expected to operate under
its dual mandate to promote ``maximum employment'' and stabilize prices
(by limiting inflation). However, while central banks like the Fed have
explicit inflation targets --- typically aiming to keep the rate at 2
percent --- they do not have explicit unemployment targets.

The Fed, could instead put new policies in place that make a very low
unemployment rate --- or more aggressively, underemployment rate --- the
new trigger for whether it decides to stimulate or hit the financial
breaks on the economy. This shift would avert the risk of depressing
wages and be helpful to groups in the work force who are discriminated
against and often ``first fired, last hired.'' And crucially, it would
reward companies for longer-term investments that promote real economic
growth.

How else can the financial markets be encouraged to prioritize real,
productive investment? Governments can begin to issue higher taxes on
dividend payments to large shareholders of big, publicly traded
companies and pair that with tax reductions on long-term investments.

It's not surprising that investors --- who for years looked at a
landscape of sluggish-to-moderate global growth --- have been looking
for quick financial returns rather than productive, but sometimes risky,
long-term investments. Guided by shareholder demands, for the past
decade businesses have focused on delivering returns quickly and
predictably to investors instead of investing in longer-horizon
infrastructures --- like research, plants and machinery that would
ultimately lead to innovation and drive economic growth.

According to a 2019 report,
``\href{https://www.rubio.senate.gov/public/_cache/files/9f25139a-6039-465a-9cf1-feb5567aebb7/4526E9620A9A7DB74267ABEA5881022F.5.15.2019.-final-project-report-american-investment.pdf}{American
Investment in the 21st Century},'' led by Senator Marco Rubio of
Florida, net private domestic investment in fixed assets like equipment,
machinery and property has shrunk in half since the mid-1980s.

Higher taxes on large dividend payments and federal subsidies for
long-term investments could help America reverse course.

We also need to address concentration of corporate power. To overhaul
the prevailing global economic architecture, the globe's leading
governments will need to address the fact that many sectors ---
airlines, banking, technology --- have become oligopolies dominated by
just a few multinational corporations. These Gilded Age style markets
reduce competition and concentrate the pricing power of large,
well-connected corporations.

There have been calls to break up technology companies or to limit their
scale and monopolistic tendencies. However, dozens of national
regulators are pitted against global corporations that can use their
multiple bases to evade rules inconvenient to them. So international
regulatory cooperation will be needed to rein in the increasingly
unfettered power of these multinational behemoths.

At a time when many governments seem steered by nationalism, effective
cross-border cooperation is hard to imagine. However, feats of global
cooperation from the past --- like the post-World War II establishment
of the Bretton Woods system's new world order --- offer examples of
leaders eventually meeting the moment even amid formidable challenges.

The pandemic is not just giving us a chance to rethink how to best live
and work. It is also providing an opportunity to reconsider the way that
the very structures of our world economy operate.

Dr. Moyo is an economist and the author of ``How the West Was Lost:
Fifty Years of Economic Folly --- and the Stark Choices that Lie
Ahead.''

\emph{The Times is committed to publishing}
\href{https://www.nytimes.com/2019/01/31/opinion/letters/letters-to-editor-new-york-times-women.html}{\emph{a
diversity of letters}} \emph{to the editor. We'd like to hear what you
think about this or any of our articles. Here are some}
\href{https://help.nytimes.com/hc/en-us/articles/115014925288-How-to-submit-a-letter-to-the-editor}{\emph{tips}}\emph{.
And here's our email:}
\href{mailto:letters@nytimes.com}{\emph{letters@nytimes.com}}\emph{.}

\emph{Follow The New York Times Opinion section on}
\href{https://www.facebook.com/nytopinion}{\emph{Facebook}}\emph{,}
\href{http://twitter.com/NYTOpinion}{\emph{Twitter (@NYTopinion)}}
\emph{and}
\href{https://www.instagram.com/nytopinion/}{\emph{Instagram}}\emph{.}

Advertisement

\protect\hyperlink{after-bottom}{Continue reading the main story}

\hypertarget{site-index}{%
\subsection{Site Index}\label{site-index}}

\hypertarget{site-information-navigation}{%
\subsection{Site Information
Navigation}\label{site-information-navigation}}

\begin{itemize}
\tightlist
\item
  \href{https://help.nytimes.com/hc/en-us/articles/115014792127-Copyright-notice}{©~2020~The
  New York Times Company}
\end{itemize}

\begin{itemize}
\tightlist
\item
  \href{https://www.nytco.com/}{NYTCo}
\item
  \href{https://help.nytimes.com/hc/en-us/articles/115015385887-Contact-Us}{Contact
  Us}
\item
  \href{https://www.nytco.com/careers/}{Work with us}
\item
  \href{https://nytmediakit.com/}{Advertise}
\item
  \href{http://www.tbrandstudio.com/}{T Brand Studio}
\item
  \href{https://www.nytimes.com/privacy/cookie-policy\#how-do-i-manage-trackers}{Your
  Ad Choices}
\item
  \href{https://www.nytimes.com/privacy}{Privacy}
\item
  \href{https://help.nytimes.com/hc/en-us/articles/115014893428-Terms-of-service}{Terms
  of Service}
\item
  \href{https://help.nytimes.com/hc/en-us/articles/115014893968-Terms-of-sale}{Terms
  of Sale}
\item
  \href{https://spiderbites.nytimes.com}{Site Map}
\item
  \href{https://help.nytimes.com/hc/en-us}{Help}
\item
  \href{https://www.nytimes.com/subscription?campaignId=37WXW}{Subscriptions}
\end{itemize}
