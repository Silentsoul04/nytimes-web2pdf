Sections

SEARCH

\protect\hyperlink{site-content}{Skip to
content}\protect\hyperlink{site-index}{Skip to site index}

\href{https://myaccount.nytimes.com/auth/login?response_type=cookie\&client_id=vi}{}

\href{https://www.nytimes.com/section/todayspaper}{Today's Paper}

\href{/section/opinion}{Opinion}\textbar{}The Less Impossible
Israeli-Palestinian Peace

\url{https://nyti.ms/2Ewd4Lo}

\begin{itemize}
\item
\item
\item
\item
\item
\item
\end{itemize}

Advertisement

\protect\hyperlink{after-top}{Continue reading the main story}

\href{/section/opinion}{Opinion}

Supported by

\protect\hyperlink{after-sponsor}{Continue reading the main story}

\hypertarget{the-less-impossible-israeli-palestinian-peace}{%
\section{The Less Impossible Israeli-Palestinian
Peace}\label{the-less-impossible-israeli-palestinian-peace}}

The two-state idea is comatose but not dead.

\href{https://www.nytimes.com/by/roger-cohen}{\includegraphics{https://static01.nyt.com/images/2014/11/01/opinion/cohen-circular/cohen-circular-thumbLarge-v6.png}}

By \href{https://www.nytimes.com/by/roger-cohen}{Roger Cohen}

Opinion Columnist

\begin{itemize}
\item
  July 31, 2020
\item
  \begin{itemize}
  \item
  \item
  \item
  \item
  \item
  \item
  \end{itemize}
\end{itemize}

\includegraphics{https://static01.nyt.com/images/2020/07/31/opinion/31cohenWeb/merlin_168152001_a0c3adaf-6181-41d2-9c1d-fb79543f7c1e-articleLarge.jpg?quality=75\&auto=webp\&disable=upscale}

Let's play the Israel-Palestine impossibility game. It's timely because
the two-state peace for which I have long argued is now widely deemed
unattainable. The answer, as one of the most thoughtful observers of the
conflict,
\href{https://www.nytimes.com/2020/07/08/opinion/israel-annexation-two-state-solution.html}{Peter
Beinart, has recently argued}, must be one state with equal rights for
Jews and Palestinians, ``a Jewish home that is not a Jewish state.''

Beinart, the editor at large of Jewish Currents and a longtime two-state
advocate, \emph{changed his mind.} Yes, it's still possible. He gave up
a core conviction, based on the evidence. I salute that rare capacity in
an America of declaimed certainties, even as I disagree.

The impossibility game goes like this: You list the reasons that a
two-state outcome is impossible, before listing the reasons that a
one-state solution is impossible, and then you decide which of the two
is \emph{less impossible.} As you do so, set aside the fact that history
is a catalog of ``impossible'' events. Lastly, draw conclusions that
reflect the enigma of personal conviction.

Here we go. A two-state peace is impossible because the 53-year Israeli
occupation of the West Bank has gone too far to be undone; because the
conquest has become so accepted that only a handful of Jewish members of
the Knesset will even use the word ``occupation''; because Messianic
Israeli nationalism, with its claim to all the land between the
Mediterranean and Jordan River, has grown inexorably since the
assassination of Prime Minister Yitzhak Rabin a quarter-century ago;
because President Trump's United States has given carte blanche to that
nationalism through a so-called peace plan contemptuous of viable
Palestinian statehood; because Israeli annexation of parts of the West
Bank is a growing possibility
\href{https://www.nytimes.com/2020/01/28/world/middleeast/israel-west-bank-annex-sovereignty.html}{invoked
regularly by Prime Minister Benjamin Netanyahu}; because some 640,00
Jewish settlers now live in the West Bank and East Jerusalem; because
the Palestinians have never been weaker; because the Palestinians have
never been more abandoned by Arab states; because the Palestinian
Authority, a supposedly interim form of self-government, has become the
corrupt, undemocratic agent of Palestinian emasculation; because the
infrastructure and economy of Israeli dominion are irreversible; because
power on the Israeli side corrupts and powerlessness for Palestinians
leads to the chimera of victimhood; because young Palestinians now
prefer the one-state idea; because Israelis, post-Oslo, have other
things on their minds.

A one-state peace is impossible, on the other hand, because Jews who for
millenniums dreamed of their own homeland will not suddenly concede this
was misguided and believe, the Holocaust notwithstanding, in the
kindness of strangers; because Jews had a ``home'' before in places like
the Netherlands, France and Germany, and concluded from the way their
patriotism led to annihilation that a home was not enough; because a
United States of Jews and Palestinians in the Holy Land could never
agree on a school textbook, or the composition of its army, or indeed
the very name of the state; because Lebanon, Syria and Iraq suggest the
limited chances in the Middle East of harmonious existence among
different national and religious groups; because a country whose birth
would be a day of liberation for half the population and for the other
half a day of catastrophe is unworkable; because Palestinians and Jews
singing ``Kumbaya'' together in the increasingly nationalist and
discriminatory Israel of the Jewish Nation State Law is far-fetched;
because the hatred between Jews and Arabs is more amenable to an
equitable divorce than to forced cohabitation; because many Palestinians
still seek the destruction of Israel and would see one state as the
partial attainment of that goal; because geography and demography
suggest one state would end up as an Arab state in Israel's stead.

It's a tough call. I think a two-state peace is less impossible and more
desirable in part because I am sure the only way to something resembling
one state, a confederation for example, is through the establishment of
two states, Israeli and Palestinian, living beside each other in peace
and security for decades.

When France and Germany, after 1945, opted for reconciliation after
repeated wars, they did not abolish their borders; they knitted together
forms of cooperation that ultimately made the borders invisible and
irrelevant.

The
\href{https://www.nytimes.com/2017/12/07/us/politics/trump-jerusalem-palestinians.html}{two-state
idea is comatose until Trump is gone}, and Netanyahu is gone, and
Mahmoud Abbas, the president of the Palestinian Authority, is gone ---
and then some. But it is no more ``impossible'' than was the fall of the
Berlin Wall or the disappearance of the Soviet Union. ``There is a
distinction between a coma and being dead,'' Jeremy Ben-Ami, the
president of the nonprofit advocacy group J Street, told me.

But the one-state idea is dead on arrival. It does not add up, a
21st-century fantasy built on the pretense that the 20th century did not
exist.

I mentioned personal conviction. People do not reason in a vacuum. I am
not persuaded, as Beinart seems to be, that one state would guarantee
Jews' security, the ultimate raison d'être of Israel. Therefore, I
cannot support it.

Here are my convictions,
\href{https://www.nytimes.com/2014/07/30/opinion/roger-cohen-zionism-and-israels-war-with-hamas-in-gaza.html}{as
I expressed them in a column six years ago}. I have not changed my mind.

\begin{quote}
I am a Zionist because the story of my forebears convinces me that Jews
needed the state voted into existence by United Nations Resolution 181
of 1947, calling for the establishment of two states --- one Jewish, one
Arab --- in Mandate Palestine. I am a Zionist who believes in the words
of Israel's founding charter of 1948 declaring that the nascent state
would be based ``on freedom, justice and peace as envisaged by the
prophets of Israel.''

What I cannot accept, however, is the perversion of Zionism that has
seen the inexorable growth of a Messianic Israeli nationalism claiming
all the land between the Mediterranean and the Jordan River; that has
\ldots{} produced the systematic oppression of another people in the
West Bank; that has led to the steady expansion of Israeli settlements
on the very West Bank land of any Palestinian state; that isolates
moderate Palestinians \ldots{} in the name of divide-and-rule; that
pursues policies that will make it impossible to remain a Jewish and
democratic state; that seeks tactical advantage rather than the
strategic breakthrough of a two-state peace; that blockades Gaza with
1.8 million people locked in its prison and is then surprised by the
periodic eruptions of the inmates; and that responds disproportionately
to attack.
\end{quote}

This, as a Zionist, I cannot accept. The way out of the impasse is the
less impossible path: two states.

\emph{The Times is committed to publishing}
\href{https://www.nytimes.com/2019/01/31/opinion/letters/letters-to-editor-new-york-times-women.html}{\emph{a
diversity of letters}} \emph{to the editor. We'd like to hear what you
think about this or any of our articles. Here are some}
\href{https://help.nytimes.com/hc/en-us/articles/115014925288-How-to-submit-a-letter-to-the-editor}{\emph{tips}}\emph{.
And here's our email:}
\href{mailto:letters@nytimes.com}{\emph{letters@nytimes.com}}\emph{.}

\emph{Follow The New York Times Opinion section on}
\href{https://www.facebook.com/nytopinion}{\emph{Facebook}}\emph{,}
\href{http://twitter.com/NYTOpinion}{\emph{Twitter (@NYTopinion)}}
\emph{and}
\href{https://www.instagram.com/nytopinion/}{\emph{Instagram}}\emph{.}

Advertisement

\protect\hyperlink{after-bottom}{Continue reading the main story}

\hypertarget{site-index}{%
\subsection{Site Index}\label{site-index}}

\hypertarget{site-information-navigation}{%
\subsection{Site Information
Navigation}\label{site-information-navigation}}

\begin{itemize}
\tightlist
\item
  \href{https://help.nytimes.com/hc/en-us/articles/115014792127-Copyright-notice}{©~2020~The
  New York Times Company}
\end{itemize}

\begin{itemize}
\tightlist
\item
  \href{https://www.nytco.com/}{NYTCo}
\item
  \href{https://help.nytimes.com/hc/en-us/articles/115015385887-Contact-Us}{Contact
  Us}
\item
  \href{https://www.nytco.com/careers/}{Work with us}
\item
  \href{https://nytmediakit.com/}{Advertise}
\item
  \href{http://www.tbrandstudio.com/}{T Brand Studio}
\item
  \href{https://www.nytimes.com/privacy/cookie-policy\#how-do-i-manage-trackers}{Your
  Ad Choices}
\item
  \href{https://www.nytimes.com/privacy}{Privacy}
\item
  \href{https://help.nytimes.com/hc/en-us/articles/115014893428-Terms-of-service}{Terms
  of Service}
\item
  \href{https://help.nytimes.com/hc/en-us/articles/115014893968-Terms-of-sale}{Terms
  of Sale}
\item
  \href{https://spiderbites.nytimes.com}{Site Map}
\item
  \href{https://help.nytimes.com/hc/en-us}{Help}
\item
  \href{https://www.nytimes.com/subscription?campaignId=37WXW}{Subscriptions}
\end{itemize}
