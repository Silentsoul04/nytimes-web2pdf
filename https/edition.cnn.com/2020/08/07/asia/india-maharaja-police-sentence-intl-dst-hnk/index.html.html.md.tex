\href{/world}{World}

\begin{itemize}
\tightlist
\item
  \href{/africa}{Africa}
\item
  \href{/americas}{Americas}
\item
  \href{/asia}{Asia}
\item
  \href{/australia}{Australia}
\item
  \href{/china}{China}
\item
  \href{/europe}{Europe}
\item
  \href{/india}{India}
\item
  \href{/middle-east}{Middle East}
\item
  \href{/uk}{United Kingdom}
\end{itemize}

Edition

\begin{itemize}
\tightlist
\item
  \href{//us.cnn.com?hpt=header_edition-picker}{U.S.}
\item
  \href{//edition.cnn.com?hpt=header_edition-picker}{International}
\item
  \href{//arabic.cnn.com?hpt=header_edition-picker}{Arabic}
\item
  \href{//cnnespanol.cnn.com?hpt=header_edition-picker}{Español}
\end{itemize}

Search CNN

Open Menu

\begin{itemize}
\tightlist
\item
  \href{/africa}{Africa}
\item
  \href{/americas}{Americas}
\item
  \href{/asia}{Asia}
\item
  \href{/australia}{Australia}
\item
  \href{/china}{China}
\item
  \href{/europe}{Europe}
\item
  \href{/india}{India}
\item
  \href{/middle-east}{Middle East}
\item
  \href{/uk}{United Kingdom}
\end{itemize}

Search

Edition

\begin{itemize}
\tightlist
\item
  \href{//us.cnn.com?hpt=header_edition-picker}{U.S.}
\item
  \href{//edition.cnn.com?hpt=header_edition-picker}{International}
\item
  \href{//arabic.cnn.com?hpt=header_edition-picker}{Arabic}
\item
  \href{//cnnespanol.cnn.com?hpt=header_edition-picker}{Español}
\end{itemize}

\begin{center}\rule{0.5\linewidth}{\linethickness}\end{center}

\begin{itemize}
\tightlist
\item
  \href{/world}{World}

  \begin{itemize}
  \tightlist
  \item
    \href{/africa}{Africa}
  \item
    \href{/americas}{Americas}
  \item
    \href{/asia}{Asia}
  \item
    \href{/australia}{Australia}
  \item
    \href{/china}{China}
  \item
    \href{/europe}{Europe}
  \item
    \href{/india}{India}
  \item
    \href{/middle-east}{Middle East}
  \item
    \href{/uk}{United Kingdom}
  \end{itemize}
\item
  \href{/politics}{US Politics}

  \begin{itemize}
  \tightlist
  \item
    \href{/election/2020}{2020 Election}
  \item
    \href{/specials/politics/president-donald-trump-45}{Donald Trump}
  \item
    \href{/specials/politics/supreme-court-nine}{Supreme Court}
  \item
    \href{/specials/politics/congress}{Congress}
  \item
    \href{/specials/politics/fact-check-politics}{Facts First}
  \end{itemize}
\item
  \href{/business}{Business}

  \begin{itemize}
  \tightlist
  \item
    \href{https://money.cnn.com/data/markets/}{Markets}
  \item
    \href{/business/tech}{Tech}
  \item
    \href{/business/media}{Media}
  \item
    \href{/business/success}{Success}
  \item
    \href{/business/perspectives}{Perspectives}
  \item
    \href{/business/videos}{Videos}
  \end{itemize}
\item
  \href{/health}{Health}

  \begin{itemize}
  \tightlist
  \item
    \href{/specials/health/food-diet}{Food}
  \item
    \href{/specials/health/fitness-excercise}{Fitness}
  \item
    \href{/specials/health/wellness}{Wellness}
  \item
    \href{/specials/health/parenting}{Parenting}
  \item
    \href{/specials/health/vital-signs}{Vital Signs}
  \end{itemize}
\item
  \href{/entertainment}{Entertainment}

  \begin{itemize}
  \tightlist
  \item
    \href{/entertainment/celebrities}{Stars}
  \item
    \href{/entertainment/movies}{Screen}
  \item
    \href{/entertainment/tv-shows}{Binge}
  \item
    \href{/entertainment/culture}{Culture}
  \item
    \href{/business/media}{Media}
  \end{itemize}
\item
  \href{/business/tech}{Tech}

  \begin{itemize}
  \tightlist
  \item
    \href{/specials/tech/innovate}{Innovate}
  \item
    \href{/specials/tech/gadget}{Gadget}
  \item
    \href{/specials/tech/foreseeable-future}{Foreseeable Future}
  \item
    \href{/specials/tech/mission-ahead}{Mission: Ahead}
  \item
    \href{/specials/tech/upstarts}{Upstarts}
  \item
    \href{/specials/tech/work-transformed}{Work Transformed}
  \item
    \href{/specials/tech/innovative-cities}{Innovative Cities}
  \end{itemize}
\item
  \href{/style}{Style}

  \begin{itemize}
  \tightlist
  \item
    \href{/style/arts}{Arts}
  \item
    \href{/style/design}{Design}
  \item
    \href{/style/fashion}{Fashion}
  \item
    \href{/style/architecture}{Architecture}
  \item
    \href{/style/luxury}{Luxury}
  \item
    \href{/style/beauty}{Beauty}
  \item
    \href{/style/videos}{Video}
  \end{itemize}
\item
  \href{/travel}{Travel}

  \begin{itemize}
  \tightlist
  \item
    \href{/travel/destinations}{Destinations}
  \item
    \href{/travel/food-and-drink}{Food and Drink}
  \item
    \href{/travel/stay}{Stay}
  \item
    \href{/travel/news}{News}
  \item
    \href{/travel/videos}{Videos}
  \end{itemize}
\item
  \href{/sport}{Sports}

  \begin{itemize}
  \tightlist
  \item
    \href{/sport/football}{Football}
  \item
    \href{/sport/tennis}{Tennis}
  \item
    \href{/sport/equestrian}{Equestrian}
  \item
    \href{/sport/golf}{Golf}
  \item
    \href{/sport/skiing}{Skiing}
  \item
    \href{/sport/horse-racing}{Horse Racing}
  \item
    \href{/sport/motorsport}{Motorsport}
  \item
    \href{/specials/sport/formula-e}{Formula E}
  \item
    \href{/specials/esports}{Esports}
  \end{itemize}
\item
  \href{/videos}{Videos}

  \begin{itemize}
  \tightlist
  \item
    \href{//cnn.it/go2}{Live TV}
  \item
    \href{/specials/digital-studios}{Digital Studios}
  \item
    \href{/specials/videos/digital-shorts}{CNN Films}
  \item
    \href{/specials/videos/hln}{HLN}
  \item
    \href{/tv/schedule/cnn}{TV Schedule}
  \item
    \href{/specials/tv/all-shows}{TV Shows A-Z}
  \item
    \href{/vr}{CNNVR}
  \end{itemize}
\item
  \href{/specials}{Features}

  \begin{itemize}
  \tightlist
  \item
    \href{/interactive/call-to-earth}{Call to Earth}
  \item
    \href{/specials/world/freedom-project}{Freedom Project}
  \item
    \href{/specials/impact-your-world}{Impact Your World}
  \item
    \href{/specials/africa/inside-africa}{Inside Africa}
  \item
    \href{/specials/opinions/two-degrees}{2 Degrees}
  \item
    \href{/specials/cnn-heroes}{CNN Heroes}
  \item
    \href{/specials}{All Features}
  \end{itemize}
\item
  \href{/more}{More}

  \begin{itemize}
  \tightlist
  \item
    \href{/specials/photos}{Photos}
  \item
    \href{/specials/cnn-longform}{Longform}
  \item
    \href{/specials/cnn-investigates}{Investigations}
  \item
    \href{/specials/profiles}{CNN Profiles}
  \item
    \href{/specials/more/cnn-leadership}{CNN Leadership}
  \item
    \href{/email/subscription}{CNN Newsletters}
  \item
    \href{https://www.turnerjobs.com/search-jobs?orgIds=1174\&ac=19299}{Work
    for CNN}
  \end{itemize}
\item
  \href{/weather}{Weather}

  \begin{itemize}
  \tightlist
  \item
    \href{/specials/world/cnn-climate}{Climate}
  \item
    \href{/interactive/2020/weather/gonzalo-storm-path-tracker/index.html}{Storm
    Tracker}
  \item
    \href{/specials/weather/weather-video}{Video}
  \end{itemize}
\end{itemize}

\begin{center}\rule{0.5\linewidth}{\linethickness}\end{center}

Follow CNN

\begin{itemize}
\item
\item
\item
\end{itemize}

\hypertarget{an-indian-royal-was-shot-dead-by-police-in-broad-daylight-now-35-years-later-his-killers-have-been-jailed}{%
\section{An Indian royal was shot dead by police in broad daylight. Now,
35 years later, his killers have been
jailed}\label{an-indian-royal-was-shot-dead-by-police-in-broad-daylight-now-35-years-later-his-killers-have-been-jailed}}

By Esha Mitra and \href{/profiles/julia-hollingsworth}{Julia
Hollingsworth}, CNN

Updated 0020 GMT (0820 HKT) August 8, 2020

Chat with us in Facebook Messenger. Find out what's happening in the
world as it unfolds.

\includegraphics{//cdn.cnn.com/cnnnext/dam/assets/200806170548-india-raja-man-singh-tease-large-169.jpg}

New Delhi, India (CNN)On a winter's day in 1985, a one-time Indian
prince was killed by police in a bustling Rajasthan market.

For decades, the police officers on duty and the former royal's
relatives couldn't agree on what had happened that day to a man known
for his fiery nature and political ambition.

Raja Man Singh's family -\/- part of a centuries-old royal lineage -\/-
claimed he had been killed in a premeditated murder plot ordered by the
highest politician in the state.

But police said they opened fire in self-defense, killing a hot-tempered
man who thought himself above the law.

For 35 years, no one was held accountable for Man Singh's death. Then,
last month, after a protracted legal battle, 11 policemen were convicted
of his murder and sentenced to life in prison.

\includegraphics{//cdn.cnn.com/cnnnext/dam/assets/200805171035-mahraja-3-large-169.jpeg}

Dushyant Singh pictured next to a statue of his grandfather, Raja Man
Singh.

Read More

His family say it took them 1,700 court dates over 35 years to get
justice. Because the trial took so long, all of the policemen convicted
are now in their 60s or older, and four policemen who had been accused
died before the verdict.

But the fact there is any result at all is significant in India, where
it is rare for police to be convicted over the killing of a member of
the public -\/- a situation known in India as an "encounter killing."

And, as Man Singh's family points out, there might not have been any
justice at all if it weren't for their royal lineage.

\hypertarget{the-last-of-the-royals}{%
\subsubsection{The last of the royals}\label{the-last-of-the-royals}}

When Man Singh was born in 1921, the Indian subcontinent was still under
British control.

But only about two thirds of the population was directly ruled by the
British Raj -\/- the other third was governed by about
\href{https://www.jstor.org/stable/4414671?read-now=1\&refreqid=excelsior\%3A5a3a0dddb2afd85267868bd5771e0e52\&seq=1\#page_scan_tab_contents}{600
local rulers} who swore allegiance to the British crown.

The "princely states" benefited the British Raj -\/- they reduced the
administrative load as they ruled their own affairs and, by splintering
the population, made it less likely that the Indian subjects would unify
against them.

After India gained independence in 1947, these princely states were
dismantled and the country became the world's biggest democracy. ****
That included the princely state of Bharatpur, then under the rule of
Man Singh's brother, Maharaja Brijendra.

Royal families were allowed to keep their palaces, which many former
rulers converted into magnificent hotels, according to Adnan
Naseemullah, who teaches South Asian politics at King's College London.
And up **** until 1971, the former royal families were paid a privy
purse -\/- compensation from the central government for their loss of
status.

After independence, some former royals -\/- such as the Bharatpur royal
family -\/- moved into politics. Sometimes, they did this to prevent
their property from being transferred to peasants, or to the state,
according to Naseemullah. By becoming involved in politics, they were
able to turn their traditional authority into a modern, legal authority,
said political scientist Vasundhara Sirnate.

\begin{quote}
There's a sense of entitlement with which former royals went into the
political process.

Vasundhara Sirnate, political scientist
\end{quote}

"There's a sense of entitlement with which former royals went into the
political process. They knew that if they lose an election... it hurts
their traditional authority," she said.

In the decades after independence, Man Singh proved himself an adept
political force.

By 1985, he had already won six consecutive **** legislative assembly
elections in Rajasthan. He didn't promote a particular issue -\/-
instead, he won every election by leveraging the maharaja's immense
popularity, campaigning under the slogan
"\href{https://apnews.com/123aa36b7984f59d6409a50a6b3b659b}{Long Live
Giriraj Maharaj}," a reference to the royal family's deity of valor.

In 1985, he was campaigning for his seventh term against a rival from
the then-ruling Indian National Congress Party, which had pushed for
independence from the British.

The seventh campaign would be Man Singh's last.

On February 19, Congress party members went to Man Singh's summer palace
in Deeg, a town in Bharatpur, according to Vijay Singh, Man Singh's
son-in-law. There, they pulled down a flag -\/- it's unclear what kind
of flag it was -\/- and burned it.

The following day, the Chief Minister of Rajasthan, Shiv Charan Mathur,
the highest elected official in the state, held a rally in support of
Man Singh's opponent. ****

Furious, Man Singh showed up **** at the **** rally, according to a
158-page judgment by a special Central Bureau of Investigation (CBI)
court handed down last month.

He drove his military vehicle into the stage, then rammed into the
helicopter that the chief minister had used to fly to the rally. ****
The helicopter's windows were smashed and the chief minister had to
return to Rajasthan's capital, Jaipur, by road.

According to Vijay Singh, police made no attempt to arrest Man Singh
after the incident, although a police report filed that day **** accused
him of attempted murder. Man Singh continued with his electioneering,
and even held a political address near a police station later that
evening.

At around midday the next day -\/- February 21 -\/- Man Singh, his
son-in-law and other party members were on their way to a campaign
meeting, according to the judgment.

They were stopped by around 50 police officers in a crowded market. When
Man Singh attempted to reverse his car, police opened fire, killing him,
**** according to Vijay Singh's account to police.

\hypertarget{self-defence-or-murder}{%
\subsubsection{Self defence or murder?}\label{self-defence-or-murder}}

As police told it, they killed Man Singh in self-defense.

When they got to the market to arrest **** him **** over the incident
the previous day, his party members opened fire using improvised guns
constructed from scrap material, known in India as "country-made guns,"
police said.

When one officer told them to surrender, police reports allege that Man
Singh yelled back: "Kill the scumbags," according to a translation from
the Hindi judgment.

Police claimed they were forced to fire, leaving Man Singh and two of
his party members injured. After the chaos subsided, they took all three
for treatment, according to the original police report.

\includegraphics{//cdn.cnn.com/cnnnext/dam/assets/200805173948-india-maharaja-split-large-169.jpg}

Left: An image of the late Raja Man Singh, who was born into a royal
family in Bharatpur. Right: A young Raja Man Singh.

Lawyers for police pointed to Man Singh's quick temper -\/- during the
1971 elections he rammed his car into his opponent's vehicle, and in
1973 he did the same to a police vehicle, snatching a weapon from an
officer and brawling with police, according to police reports. ****

But Vijay Singh, who was almost hit by a bullet himself in the fatal
shooting, said it wasn't self-defense -\/- it was murder.

He claims that the chief minister of Rajasthan was furious that Man
Singh damaged his helicopter and disrupted his rally. So he came up with
a plan for revenge -\/- he ordered police to kill Man Singh.

\begin{quote}
This was an open daylight murder in the middle of a busy market but they
scared people into not speaking up.

Vijay Singh
\end{quote}

According to Vijay Singh, the first bullet was fired by the Deputy
Superintendent of Police, Kan Singh Bhati, who is now over 82 years old.
Contrary to the police report, Vijay Singh says his father-in-law and
his supporters died on the spot and weren't carrying weapons -\/-
instead police planted evidence to make it look as if there had been a
shoot-out.

"This was an open daylight murder in the middle of a busy market but
they scared people into not speaking up," Vijay Singh told CNN last
month. "Why would a family with tens of licensed guns travel with a
country gun, instead?" ****

In its court ruling last month, the CBI did not deal with Vijay Singh's
claim that the chief minister -\/- who died in 2009 -\/- had ordered the
killing.

But it did side with Vijay Singh's **** version of events. The court
found that the firing began on Bhati's orders. It ruled that Man Singh
and his party members did not have any weapons on them -\/- and that
they had died on the spot.

"The family and the public are both happy for this verdict and we
welcome it," said Krishnendra Kaur, Man Singh's daughter.

She added that she was glad that she and her two sisters were alive to
see the result -\/- Man Singh's wife didn't live to see the outcome.
****

CNN has sought comment from the CBI and Bharatpur police.

\hypertarget{why-the-case-took-so-long}{%
\subsubsection{Why the case took so
long}\label{why-the-case-took-so-long}}

After Man Singh died, many people in Bharatpur were distraught.

India's hundreds of princely states were governed differently -\/- and
in many, there was no love lost between the commoners and their formal
rulers, Naseemullah said. They were seen as "stooges of the British
Empire" who were on the "wrong side of history," he added.

But in Bharatpur, many people loved the royal family. According to Vijay
Singh, Man Singh worked his farms himself and was called a "farmer among
kings and a king among farmers," by his people. There was public
goodwill towards the royal family, who had been kind to their people,
Vijay Singh added.

\includegraphics{//cdn.cnn.com/cnnnext/dam/assets/200729132412-03-india-maharajah-police-sentence-restricted-large-169.jpg}

Durjan Sal Palace, Bharatpur. Engraving from India, 1877, by Louis
Rousselet.

So when Man Singh died, hundreds of people from the town of Deeg
attended his funeral. As they mourned, a curfew was put in place in
Bharatpur to contain protests against the police, according to Vijay
Singh. Three people died in the violence, according to Vijay Singh's
testimony in court.

Soon after Man Singh's death, Vijay Singh took his version of events to
the police.

On February 23, 1985, he filed an incident report, claiming that police
had murdered his father-in-law. Initially, local police refused to
record his complaint, he testified in court. So he complained to the
superintendent of police who told the officers to register his report.
All 18 policemen were charged over the murder in July that year.

"After the incident happened, the atmosphere in the town and in the
district was volatile," Vijay Singh said.

But court proceedings were delayed for decades, according to Narayan
Singh, Man Singh's family lawyer.

The family petitioned to have the case transferred from Rajasthan to
Mathura, in the neighboring state of Uttar Pradesh, fearing the local
government could prevent a successful prosecution. In 1989, the Supreme
Court transferred the case.

But even then, they ran into lengthy delays.

\includegraphics{//cdn.cnn.com/cnnnext/dam/assets/200729132408-01-india-maharajah-police-sentence-restricted-large-169.jpg}

Deeg Palace in Bharatpur district, Rajasthan, India.

Under Indian law, people can be charged collectively for a crime,
meaning that the prosecution doesn't need to prove which one of the
accused fired the fatal bullet, for instance. But if a legal petition
-\/- such as an appeal -\/- is pending in a higher court for one of the
individuals, the case for the collective cannot proceed at the local
court level. A similar approach was used by the four men convicted of
the
\href{https://edition.cnn.com/2020/03/19/asia/india-rape-execution-intl-hnk/index.html}{gang
rape and murder} of a 23-year-old student on a New Delhi bus in 2012.

In Man Singh's case, the petitions caused so many delays that 26
different judges ended up handling the case, Narayan Singh said.

According to Narayan Singh, the court only began hearing evidence from
the 61 prosecution witnesses in 1990, and that process alone took 18
years. It took another four years to question 17 defense witnesses, and
another eight years to hear further arguments and petitions.

"The courts would get adjourned through different applications from the
defense side and taking testimony of one witness could last as long as
four months," Narayan Singh said. "They (the defense) had 100 ways of
delaying the hearings."

According to Vijay Singh, each of the 18 accused would petition the high
court at different times.

"The police definitely knew how to exploit the system to their benefit,"
he said.

\hypertarget{wider-problem}{%
\subsubsection{Wider problem}\label{wider-problem}}

According to lawyer Narayan Singh, it's uncommon for police to be
convicted for killing a member of the public -\/- royal blood or not.

The government doesn't release statistics on the total number of police
convicted in such cases -\/- and even the number of "encounter killings"
are unclear. There are no government statistics released on "fake
encounters" -\/- a term for cases like Man Singh's where the "encounter"
with police was staged.

According to the most recent Crime in India report, released in 2018 by
the National Crime Records Bureau, four "encounter killings" were
registered in 2018. No arrests or convictions were made.

\includegraphics{//cdn.cnn.com/cnnnext/dam/assets/200807173559-maharaja-5-large-169.jpeg}

A shrine in the town of Deeg, Rajasthan, which marks Raja Man Singh's
"place of martyrdom."

However, 164 cases of deaths during police encounters were registered by
the statutory public body National Human Rights Commission (NHRC)
between April 2017 and March 2018.

In total, 46 people died in encounters with police between January and
July this year alone, according to the NHRC. Another 601 cases involving
people dying in encounters with police are currently going through the
courts.

In the past few months, India has seen a few high-profile cases of
deaths allegedly **** at the hands of police.

In June, the
\href{https://edition.cnn.com/2020/07/10/asia/indian-gangster-shot-dead-intl-hnk-scli/index.html}{death
of gangster} Vikas Dubey in a police shoot out sparked debate in India
over extrajudicial killings. On July 22, the Supreme Court ordered an
independent inquiry commission to examine the incident and submit a
report to the court within two months.

Last month, four police officers in Tamil Nadu were arrested for
allegedly murdering a
\href{https://edition.cnn.com/2020/07/01/asia/tamil-nadu-deaths-custody-intl-hnk/index.html}{father
and son}, who were in custody at the time of their deaths.

Those deaths renewed outrage in India over police brutality, with the
men's family members, politicians and human rights activists alleging
officers tortured the pair before they died.

\includegraphics{//cdn.cnn.com/cnnnext/dam/assets/200805171045-maharaja-large-169.jpeg}

Hundreds of people attend a celebration in Bharatpur on July 23, 2020,
the day after 11 policemen were sentenced to life in prison for Raj Man
Singh's murder.

Sirnate, the political scientist, says people in India often think that
these sorts of killings only happen in places where there are
insurgents.

"These are not happening on the peripheries of the country," she said of
encounter killings. "In the Bharatpur case, (it's happened) to a family
that is extremely mainstream."

Vijay Singh believes that the only reason a verdict was delivered in
this case was because of Man Singh's influence -\/- if he hadn't have
been so high profile, the government might not have felt pressure to
continue the case.

Even now, Man Singh's influence can be felt.

Following his death, a shrine was built in the town of Deeg, considered
by supporters to be his "place of martyrdom." Every five years, hundreds
gather for a prayer meeting to remember him, according to Dushyant
Singh, Man Singh's grandson.

On July 23, the day after the 11 policemen were sentenced to prison,
hundreds gathered again -\/- this time around a statue of Man Singh near
the family's palace in Bharatpur -\/- to celebrate the outcome, he said.

And the day that the verdict was announced, around 100 police officials
were stationed outside the Mathura court, to prevent riots if the
verdict didn't side with the family's version of events.

"With the legacy of Raja Man Singh, it is only natural that people
wanted to celebrate the verdict," Vijay Singh said.

Esha Mitra reported from New Delhi, India. Julia Hollingsworth reported
from Hong Kong.

\begin{itemize}
\item
\end{itemize}

\begin{itemize}
\item
\end{itemize}

\begin{itemize}
\item
\end{itemize}

\begin{itemize}
\item
\end{itemize}

\begin{itemize}
\item
\end{itemize}

\begin{itemize}
\item
\end{itemize}

\begin{itemize}
\item
\end{itemize}

Search

\begin{itemize}
\tightlist
\item
  \href{/world}{World}

  \begin{itemize}
  \tightlist
  \item
    \href{/africa}{Africa}
  \item
    \href{/americas}{Americas}
  \item
    \href{/asia}{Asia}
  \item
    \href{/australia}{Australia}
  \item
    \href{/china}{China}
  \item
    \href{/europe}{Europe}
  \item
    \href{/india}{India}
  \item
    \href{/middle-east}{Middle East}
  \item
    \href{/uk}{United Kingdom}
  \end{itemize}
\item
  \href{/politics}{US Politics}

  \begin{itemize}
  \tightlist
  \item
    \href{/election/2020}{2020 Election}
  \item
    \href{/specials/politics/president-donald-trump-45}{Donald Trump}
  \item
    \href{/specials/politics/supreme-court-nine}{Supreme Court}
  \item
    \href{/specials/politics/congress}{Congress}
  \item
    \href{/specials/politics/fact-check-politics}{Facts First}
  \end{itemize}
\item
  \href{/business}{Business}

  \begin{itemize}
  \tightlist
  \item
    \href{https://money.cnn.com/data/markets/}{Markets}
  \item
    \href{/business/tech}{Tech}
  \item
    \href{/business/media}{Media}
  \item
    \href{/business/success}{Success}
  \item
    \href{/business/perspectives}{Perspectives}
  \item
    \href{/business/videos}{Videos}
  \end{itemize}
\item
  \href{/health}{Health}

  \begin{itemize}
  \tightlist
  \item
    \href{/specials/health/food-diet}{Food}
  \item
    \href{/specials/health/fitness-excercise}{Fitness}
  \item
    \href{/specials/health/wellness}{Wellness}
  \item
    \href{/specials/health/parenting}{Parenting}
  \item
    \href{/specials/health/vital-signs}{Vital Signs}
  \end{itemize}
\item
  \href{/entertainment}{Entertainment}

  \begin{itemize}
  \tightlist
  \item
    \href{/entertainment/celebrities}{Stars}
  \item
    \href{/entertainment/movies}{Screen}
  \item
    \href{/entertainment/tv-shows}{Binge}
  \item
    \href{/entertainment/culture}{Culture}
  \item
    \href{/business/media}{Media}
  \end{itemize}
\item
  \href{/business/tech}{Tech}

  \begin{itemize}
  \tightlist
  \item
    \href{/specials/tech/innovate}{Innovate}
  \item
    \href{/specials/tech/gadget}{Gadget}
  \item
    \href{/specials/tech/foreseeable-future}{Foreseeable Future}
  \item
    \href{/specials/tech/mission-ahead}{Mission: Ahead}
  \item
    \href{/specials/tech/upstarts}{Upstarts}
  \item
    \href{/specials/tech/work-transformed}{Work Transformed}
  \item
    \href{/specials/tech/innovative-cities}{Innovative Cities}
  \end{itemize}
\item
  \href{/style}{Style}

  \begin{itemize}
  \tightlist
  \item
    \href{/style/arts}{Arts}
  \item
    \href{/style/design}{Design}
  \item
    \href{/style/fashion}{Fashion}
  \item
    \href{/style/architecture}{Architecture}
  \item
    \href{/style/luxury}{Luxury}
  \item
    \href{/style/beauty}{Beauty}
  \item
    \href{/style/videos}{Video}
  \end{itemize}
\item
  \href{/travel}{Travel}

  \begin{itemize}
  \tightlist
  \item
    \href{/travel/destinations}{Destinations}
  \item
    \href{/travel/food-and-drink}{Food and Drink}
  \item
    \href{/travel/stay}{Stay}
  \item
    \href{/travel/news}{News}
  \item
    \href{/travel/videos}{Videos}
  \end{itemize}
\item
  \href{/sport}{Sports}

  \begin{itemize}
  \tightlist
  \item
    \href{/sport/football}{Football}
  \item
    \href{/sport/tennis}{Tennis}
  \item
    \href{/sport/equestrian}{Equestrian}
  \item
    \href{/sport/golf}{Golf}
  \item
    \href{/sport/skiing}{Skiing}
  \item
    \href{/sport/horse-racing}{Horse Racing}
  \item
    \href{/sport/motorsport}{Motorsport}
  \item
    \href{/specials/sport/formula-e}{Formula E}
  \item
    \href{/specials/esports}{Esports}
  \end{itemize}
\item
  \href{/videos}{Videos}

  \begin{itemize}
  \tightlist
  \item
    \href{//cnn.it/go2}{Live TV}
  \item
    \href{/specials/digital-studios}{Digital Studios}
  \item
    \href{/specials/videos/digital-shorts}{CNN Films}
  \item
    \href{/specials/videos/hln}{HLN}
  \item
    \href{/tv/schedule/cnn}{TV Schedule}
  \item
    \href{/specials/tv/all-shows}{TV Shows A-Z}
  \item
    \href{/vr}{CNNVR}
  \end{itemize}
\item
  \href{/specials}{Features}

  \begin{itemize}
  \tightlist
  \item
    \href{/interactive/call-to-earth}{Call to Earth}
  \item
    \href{/specials/world/freedom-project}{Freedom Project}
  \item
    \href{/specials/impact-your-world}{Impact Your World}
  \item
    \href{/specials/africa/inside-africa}{Inside Africa}
  \item
    \href{/specials/opinions/two-degrees}{2 Degrees}
  \item
    \href{/specials/cnn-heroes}{CNN Heroes}
  \item
    \href{/specials}{All Features}
  \end{itemize}
\item
  \href{/more}{More}

  \begin{itemize}
  \tightlist
  \item
    \href{/specials/photos}{Photos}
  \item
    \href{/specials/cnn-longform}{Longform}
  \item
    \href{/specials/cnn-investigates}{Investigations}
  \item
    \href{/specials/profiles}{CNN Profiles}
  \item
    \href{/specials/more/cnn-leadership}{CNN Leadership}
  \item
    \href{/email/subscription}{CNN Newsletters}
  \item
    \href{https://www.turnerjobs.com/search-jobs?orgIds=1174\&ac=19299}{Work
    for CNN}
  \end{itemize}
\item
  \href{/weather}{Weather}

  \begin{itemize}
  \tightlist
  \item
    \href{/specials/world/cnn-climate}{Climate}
  \item
    \href{/interactive/2020/weather/gonzalo-storm-path-tracker/index.html}{Storm
    Tracker}
  \item
    \href{/specials/weather/weather-video}{Video}
  \end{itemize}
\end{itemize}

\begin{center}\rule{0.5\linewidth}{\linethickness}\end{center}

\href{/world}{World}

Follow CNN

\begin{itemize}
\item
\item
\item
\end{itemize}

\begin{center}\rule{0.5\linewidth}{\linethickness}\end{center}

\begin{itemize}
\tightlist
\item
  \href{/terms}{Terms of Use}
\item
  \href{/privacy}{Privacy Policy}
\item
  \href{/accessibility}{Accessibility \& CC}
\item
  \protect\hyperlink{}{AdChoices}
\item
  \href{/about}{About Us}
\item
  \href{/msa}{Modern Slavery Act Statement}
\item
  \href{https://commercial.cnn.com}{Advertise with us}
\item
  \href{//store.cnn.com}{CNN Store}
\item
  \href{/newsletters}{Newsletters}
\item
  \href{/transcripts}{Transcripts}
\item
  \href{/collection}{License Footage}
\item
  \href{http://cnnnewsource.com}{CNN Newsource}
\item
  \href{https://www.cnn.com/sitemap.html}{Sitemap}
\end{itemize}

© 2020 Cable News Network.\href{//www.turner.com}{Turner Broadcasting
System, Inc.}All Rights Reserved.CNN Sans ™ \& © 2016 Cable News
Network.

\includegraphics{//pixel.quantserve.com/pixel/p-D1yc5zQgjmqr5.gif?labels=noscript:No Labels Set}
