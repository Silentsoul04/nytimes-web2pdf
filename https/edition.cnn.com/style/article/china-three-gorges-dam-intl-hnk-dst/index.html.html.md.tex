\href{/}{}\href{/style}{}

\includegraphics{https://dynaimage.cdn.cnn.com/cnn/e_blur:500,q_auto:low,w_50,c_fill,g_auto,h_50,ar_1:1/http\%3A\%2F\%2Fcdn.cnn.com\%2Fcnnnext\%2Fdam\%2Fassets\%2F200730231600-three-gorges-dam-restricted.jpg}

Feature · architecture

China's Three Gorges Dam is one of the largest ever created. Was it
worth it?

The Three Gorges Dam was designed to tame China's longest river. But
this summer's record rains reveal its limited ability to control floods.

\includegraphics{https://dynaimage.cdn.cnn.com/cnn/e_blur:500,q_auto:low,w_50,c_fill,g_auto,h_50,ar_1:1/http\%3A\%2F\%2Fcdn.cnn.com\%2Fcnnnext\%2Fdam\%2Fassets\%2F200717084450-nector-gan-headshot.jpg}

By \href{/profiles/nectar-gan}{Nectar Gan}, CNN

Published 1st August 2020

Three Gorges Dam is the largest hydropower project ever built.

When construction began in 1994, it was designed not only to generate
electricity to propel China's breakneck economic growth, but also to
tame China's longest river, shield millions of people from fatal floods
and, as a symbol of technological prowess, become a searing point of
national pride. ****

But it hasn't quite worked out that way.

For a start, the whole project
\href{http://politics.people.com.cn/n/2013/0607/c1001-21776413.html}{cost}
200 billion yuan (\$28.6 billion), took nearly two decades to build, and
required uprooting more than a million **** people along the Yangtze
River. And while the government
\href{http://www.npc.gov.cn/wxzl/gongbao/2000-12/14/content_5002697.htm}{promised}
the dam would **** be able to protect communities around its immediate
downstream against a "once in a century flood," its efficacy has
frequently **** been questioned.

Those doubts recently resurfaced, as the Yangtze basin saw its
\href{http://www.xinhuanet.com/english/2020-07/14/c_139209909.htm}{heaviest
average rainfall} in nearly 60 years since June, causing the river and
its many tributaries to overflow.

\href{https://www.mem.gov.cn/xw/bndt/202007/t20200728_354105.html}{More
than 158 people} have died or gone missing, 3.67 million residents have
been displaced and 54.8 million people have been affected, causing a
devastating 144 billion yuan (\$20.5 billion) in economic losses.

Despite the havoc, Chinese authorities claim the Three Gorges Dam has
succeeded in playing a
"\href{http://ccnews.people.com.cn/n1/2020/0722/c141677-31792659.html}{crucial
role}" in intercepting floodwaters. The dam's operator, China Three
Gorges Corporation, **** told China's state news agency
\href{http://www.xinhuanet.com/politics/2020-07/24/c_1126282059.htm}{Xinhua}
that the dam has
\href{http://www.gov.cn/xinwen/2020-07/30/content_5531320.htm}{intercepted}
18.2 billion cubic meters of potential floodwater. A water resources
ministry official
\href{http://ccnews.people.com.cn/n1/2020/0722/c141677-31792659.html}{told}
state-run newspaper China Youth Daily that the dam "effectively reduced
the speed and extent of water level rises" on the middle and lower
reaches of the Yangtze.

But with multiple gauging stations monitoring river flows in the Yangtze
basin seeing
\href{http://www.gov.cn/xinwen/2020-07/13/content_5526182.htm}{record-high
water levels} this summer, some geologists say the limited role of the
Three Gorges Dam in flood control has been laid bare.

\hypertarget{a-tea-cup-for-a-big-tub-of-water}{%
\subsubsection{'A tea cup for a big tub of
water'}\label{a-tea-cup-for-a-big-tub-of-water}}

The Three Gorges Dam is an awe-inspiring structure.

Firstly, it is one of the few man-made structures on Earth that's
visible to the naked eye from space, according to
\href{https://svs.gsfc.nasa.gov/3433}{NASA}. Completed in 2006, the body
of the dam is immense. It is 181 meters (607 feet) tall and spans 2,335
meters (1.45 miles) across the Yangtze just before the deep, narrow
valley gives way to plains.

Then there's its accompanying hydropower plant, which was completed in
2012 and has a generating capacity of 22,500 megawatts, or
\href{https://www.usgs.gov/special-topic/water-science-school/science/three-gorges-dam-worlds-largest-hydroelectric-plant?qt-science_center_objects=0\#qt-science_center_objects}{more
than three times} the capacity of the Grand Coulee Dam, the largest in
the United States.

But according to the Chinese government's 1992
\href{http://www.npc.gov.cn/wxzl/gongbao/2000-12/14/content_5002697.htm}{proposal},
the **** top reason for building the dam **** wasn't power generation,
but to prevent flooding.

\begin{itemize}
\item
  \includegraphics{https://dynaimage.cdn.cnn.com/cnn/e_blur:500,q_auto:low,w_50,c_fit,h_28,ar_16:9/http\%3A\%2F\%2Fcdn.cnn.com\%2Fcnnnext\%2Fdam\%2Fassets\%2F200721003422-01-three-gorges-dam-restricted.jpg}
\item
  \includegraphics{https://dynaimage.cdn.cnn.com/cnn/e_blur:500,q_auto:low,w_50,c_fit,h_28,ar_16:9/http\%3A\%2F\%2Fcdn.cnn.com\%2Fcnnnext\%2Fdam\%2Fassets\%2F200721003645-02-three-gorges-dam-restricted.jpg}
\item
  \includegraphics{https://dynaimage.cdn.cnn.com/cnn/e_blur:500,q_auto:low,w_50,c_fit,h_28,ar_16:9/http\%3A\%2F\%2Fcdn.cnn.com\%2Fcnnnext\%2Fdam\%2Fassets\%2F200721093541-24-three-gorges-dam-1999.jpg}
\item
  \includegraphics{https://dynaimage.cdn.cnn.com/cnn/e_blur:500,q_auto:low,w_50,c_fit,h_28,ar_16:9/http\%3A\%2F\%2Fcdn.cnn.com\%2Fcnnnext\%2Fdam\%2Fassets\%2F200721003913-04-three-gorges-dam-restricted.jpg}
\item
  \includegraphics{https://dynaimage.cdn.cnn.com/cnn/e_blur:500,q_auto:low,w_50,c_fit,h_28,ar_16:9/http\%3A\%2F\%2Fcdn.cnn.com\%2Fcnnnext\%2Fdam\%2Fassets\%2F200721004148-06-three-gorges-dam.jpg}
\item
  \includegraphics{https://dynaimage.cdn.cnn.com/cnn/e_blur:500,q_auto:low,w_50,c_fit,h_28,ar_16:9/http\%3A\%2F\%2Fcdn.cnn.com\%2Fcnnnext\%2Fdam\%2Fassets\%2F200721093834-26-three-gorges-dam-2002.jpg}
\item
  \includegraphics{https://dynaimage.cdn.cnn.com/cnn/e_blur:500,q_auto:low,w_50,c_fit,h_28,ar_16:9/http\%3A\%2F\%2Fcdn.cnn.com\%2Fcnnnext\%2Fdam\%2Fassets\%2F200721094040-27-three-gorges-dam-2002.jpg}
\item
  \includegraphics{https://dynaimage.cdn.cnn.com/cnn/e_blur:500,q_auto:low,w_50,c_fit,h_28,ar_16:9/http\%3A\%2F\%2Fcdn.cnn.com\%2Fcnnnext\%2Fdam\%2Fassets\%2F200721004040-05-three-gorges-dam-restricted.jpg}
\item
  \includegraphics{https://dynaimage.cdn.cnn.com/cnn/e_blur:500,q_auto:low,w_50,c_fit,h_28,ar_16:9/http\%3A\%2F\%2Fcdn.cnn.com\%2Fcnnnext\%2Fdam\%2Fassets\%2F200721004309-07-three-gorges-dam-restricted.jpg}
\item
  \includegraphics{https://dynaimage.cdn.cnn.com/cnn/e_blur:500,q_auto:low,w_50,c_fit,h_28,ar_16:9/http\%3A\%2F\%2Fcdn.cnn.com\%2Fcnnnext\%2Fdam\%2Fassets\%2F200721005128-09-three-gorges-dam.jpg}
\item
  \includegraphics{https://dynaimage.cdn.cnn.com/cnn/e_blur:500,q_auto:low,w_50,c_fit,h_28,ar_16:9/http\%3A\%2F\%2Fcdn.cnn.com\%2Fcnnnext\%2Fdam\%2Fassets\%2F200721005233-10-three-gorges-dam.jpg}
\item
  \includegraphics{https://dynaimage.cdn.cnn.com/cnn/e_blur:500,q_auto:low,w_50,c_fit,h_28,ar_16:9/http\%3A\%2F\%2Fcdn.cnn.com\%2Fcnnnext\%2Fdam\%2Fassets\%2F200721011017-11-three-gorges-dam-restricted.jpg}
\item
  \includegraphics{https://dynaimage.cdn.cnn.com/cnn/e_blur:500,q_auto:low,w_50,c_fit,h_28,ar_16:9/http\%3A\%2F\%2Fcdn.cnn.com\%2Fcnnnext\%2Fdam\%2Fassets\%2F200721094144-28-three-gorges-dam-2004.jpg}
\item
  \includegraphics{https://dynaimage.cdn.cnn.com/cnn/e_blur:500,q_auto:low,w_50,c_fit,h_28,ar_16:9/http\%3A\%2F\%2Fcdn.cnn.com\%2Fcnnnext\%2Fdam\%2Fassets\%2F200721101713-31-three-gorges-dam-2005.jpg}
\item
  \includegraphics{https://dynaimage.cdn.cnn.com/cnn/e_blur:500,q_auto:low,w_50,c_fit,h_28,ar_16:9/http\%3A\%2F\%2Fcdn.cnn.com\%2Fcnnnext\%2Fdam\%2Fassets\%2F200721094505-29-three-gorges-dam-2006.jpg}
\item
  \includegraphics{https://dynaimage.cdn.cnn.com/cnn/e_blur:500,q_auto:low,w_50,c_fit,h_28,ar_16:9/http\%3A\%2F\%2Fcdn.cnn.com\%2Fcnnnext\%2Fdam\%2Fassets\%2F200721021315-15-three-gorges-dam.jpg}
\item
  \includegraphics{https://dynaimage.cdn.cnn.com/cnn/e_blur:500,q_auto:low,w_50,c_fit,h_28,ar_16:9/http\%3A\%2F\%2Fcdn.cnn.com\%2Fcnnnext\%2Fdam\%2Fassets\%2F200721011123-12-three-gorges-dam.jpg}
\item
  \includegraphics{https://dynaimage.cdn.cnn.com/cnn/e_blur:500,q_auto:low,w_50,c_fit,h_28,ar_16:9/http\%3A\%2F\%2Fcdn.cnn.com\%2Fcnnnext\%2Fdam\%2Fassets\%2F200721021406-16-three-gorges-dam-restricted.jpg}
\item
  \includegraphics{https://dynaimage.cdn.cnn.com/cnn/e_blur:500,q_auto:low,w_50,c_fit,h_28,ar_16:9/http\%3A\%2F\%2Fcdn.cnn.com\%2Fcnnnext\%2Fdam\%2Fassets\%2F200721035858-19-three-gorges-dam.jpg}
\item
  \includegraphics{https://dynaimage.cdn.cnn.com/cnn/e_blur:500,q_auto:low,w_50,c_fit,h_28,ar_16:9/http\%3A\%2F\%2Fcdn.cnn.com\%2Fcnnnext\%2Fdam\%2Fassets\%2F200721035306-18-three-gorges-dam-restricted.jpg}
\item
  \includegraphics{https://dynaimage.cdn.cnn.com/cnn/e_blur:500,q_auto:low,w_50,c_fit,h_28,ar_16:9/http\%3A\%2F\%2Fcdn.cnn.com\%2Fcnnnext\%2Fdam\%2Fassets\%2F200721035950-20-three-gorges-dam.jpg}
\item
  \includegraphics{https://dynaimage.cdn.cnn.com/cnn/e_blur:500,q_auto:low,w_50,c_fit,h_28,ar_16:9/http\%3A\%2F\%2Fcdn.cnn.com\%2Fcnnnext\%2Fdam\%2Fassets\%2F200721040049-21-three-gorges-dam.jpg}
\end{itemize}

1/22

Workers hold up a layout plan of the Three Gorges Dam project by the
Yangtze river in Hubei province in September 1995. Scroll through the
gallery for images of the Three Gorges Dam, through the years. Credit:
Chip HIRES/Gamma-Rapho/Getty Images

Here's how it works: the enormous dam is situated on an upstream section
of the Yangtze and helps prevent flooding downstream by trapping
rainwater in a huge reservoir, and then controlling the release of that
water through its sluice gates. The 660 kilometer (410 mile) reservoir
winds upstream through the narrow valleys of the Three Gorges -\/- a
series of steep canyons known for their imposing beauty and once
treacherous currents -\/- to Chongqing, a sprawling municipality of 30.5
million people in western China.

During the dry season, October to May, the reservoir's water level is
kept at a maximum of 175 meters (574 feet) to optimize electricity
generation at the adjoining hydropower plant. Before the summer rains
arrive in June, it's gradually lowered to 145 meters (475 feet) to make
room for the incoming floodwaters.

The lowering of water levels creates 22 billion cubic meters of storage
space -\/- enough to contain nearly 9 million Olympic-size swimming
pools of water. But that's nothing compared with the sheer volume of
floodwater that can flow into the dam **** during bad years, said Fan
Xiao, a Chinese geologist and long-time critic of the dam.

During a "once-a-century flood" more than 244 billion cubic meters of
water -\/- or about twice the volume of the Dead Sea -\/- can pass
through the Three Gorges in two months, according to Fan's calculations.

The storage capacity of the dam's reservoir can handle only about 9\% of
that amount, he added.

"It's like using a small cup to deal with a big tub of water. In terms
of flood control, the cost of the dam has surely outweighed the gain."

Besides, the dam can only hold back the water for so long, as it has to
make room for new rains -\/- and in flood season torrential downpours
can come in quick succession.

Last month, three flood waves have already hit the Three Gorges. ****
The dam has opened its sluice gates multiple times since late June to
release water from its reservoir, drawing criticism on Chinese social
media that this exacerbated the floods downstream.

The company running the dam denied this, telling state-run tabloid the
\href{https://www.globaltimes.cn/content/1195133.shtml}{Global Times}
that it had helped to **** delay and stagger the floodwaters reaching
downstream.

But Poyang Lake, in Jiangxi province, still swelled to its highest level
in history -\/- surpassing the previous record set by catastrophic
floods in 1998, which killed more than 3,000 people. Other places
downstream also **** broke historical records.

\includegraphics{https://dynaimage.cdn.cnn.com/cnn/e_blur:500,q_auto:low,w_50,c_fit/http\%3A\%2F\%2Fcdn.cnn.com\%2Fcnnnext\%2Fdam\%2Fassets\%2F200729213200-poyang-lake-flooding-0715.jpg}

This aerial photo, taken on July 15, 2020, shows a flooded area near
Poyang Lake due to torrential rains in Poyang county, Shangrao city in
China's central Jiangxi province. Credit: STR/AFP/AFP via Getty Images

David Shankman, an emeritus professor of geography at the University of
Alabama, who has studied flooding on the middle Yangtze, said **** the
record-breaking water levels showed that the Three Gorges Dam could not
prevent severe floods. "That's a factual statement," he said. "This dam
is fully operational for many years now, and now we have the highest
water level ever recorded."

\href{http://www.cqvip.com/main/export.aspx?id=3307389\&type=1\&sign=a331b60d8503af649735ace4b2bfd1b7}{Studies}
by Chinese and foreign researchers over the years, Shankman added, have
found that the dam's reservoir is too small to significantly reduce
downstream discharge during severe floods, although it does help
alleviate flooding during normal years.

Miroslav Marence, an associate professor of storage and hydropower at
the IHE Delft Institute for Water Education, said the problem is not the
design of the dam, but the expectation that the dam can solve all the
problems of flooding on the Yangtze, the third largest river by volume
in the world. "It's impossible to do it just with a dam," he said.

For example, while the Three Gorges Dam can reduce the intensity of
floods coming from upstream to a certain extent, it won't be able to
prevent floods caused by intense rainfall on the middle and lower
reaches of the Yangtze or the tributaries in its basin entirely, he
added.

And that is part of the problem: A lot of the flooding in central and
southern China this summer, for instance, was **** caused by rains that
fell **** downstream and didn't ever go through the dam.

\hypertarget{the-dream-of-every-chinese-leader}{%
\subsubsection{The dream of every Chinese
leader}\label{the-dream-of-every-chinese-leader}}

The Chinese have for millennia manipulated waterways for flood control,
irrigation and navigation. For China's imperial rulers, the ability to
harness rivers not only saved lives and brought prosperity, but also
gave legitimacy to their reign, as natural disasters were taken as a
sign that the emperor had lost the mandate of heaven, by which he ruled.

This ambition to control water resources has only grown in modern times,
with the prowess of technology.

Every Chinese leader since Sun Yat-sen, the founding father of modern
China, dreamed of building a massive dam on the Yangtze, which has
repeatedly wreaked havoc on its banks during flood season.

In an industrial blueprint he laid out for the Republic of China in
1919, Sun envisioned damming the Three Gorges to improve navigation and
provide hydropower for the whole country.

The revolutionary leader did not live to see this dream realized. His
successor Chiang Kai-shek carried on with the task in the 1940s,
inviting renowned American engineer John L. Savage -\/- best known for
his work on the Hoover Dam -\/- to survey the valleys and draw up a
design for the Three Gorges Dam. Chiang even sent dozens of Chinese
engineers to the US for training, but the project was abandoned during
the Chinese Civil War.

\includegraphics{https://dynaimage.cdn.cnn.com/cnn/e_blur:500,q_auto:low,w_50,c_fit/http\%3A\%2F\%2Fcdn.cnn.com\%2Fcnnnext\%2Fdam\%2Fassets\%2F200728121027-three-gorges-dam-wuhan-leaders-restricted.jpg}

The faces of Chinese leaders Mao Zedong, Deng Xiaoping and Jiang Zemin
appear on a large mural of the Three Gorges Dam in Wuhan. Credit:
Jacques Langevin/Sygma/Getty Images

After the Chinese Communist Party took power, Chairman Mao Zedong
endorsed the project, writing about "walls of stone" and "a smooth lake
rising in the narrow gorges" in a poem. But his plans were ****
disrupted by the turmoil of the Great Leap Forward and the Cultural
Revolution.

When his successor Deng Xiaoping brought up the idea again in the late
1970s, it was strongly opposed by some leading hydrologists,
intellectuals and environmentalists, who pointed to its human and
environmental costs, from the mass relocation of residents to threats of
geological hazards, environmental damage and loss of archeological
sites.

It was heavily debated throughout the next decade, which was the most
politically relaxed and liberal era in the history of Chinese Communist
rule. But following the Tiananmen Square massacre in 1989, open dissent
was stifled and the political atmosphere turned oppressive. Four months
after the massacre, authorities banned ****
"\href{https://journal.probeinternational.org/three-gorges-probe/yangtze-yangtze/}{Yangtze!
Yangtze!}" -\/- a book highly critical of the project -\/- and jailed
its author, \href{https://www.goldmanprize.org/recipient/dai-qing/}{Dai
Qing}, a journalist and one of China's earliest environmentalists.

Confident that it could now push through the plan, the government put
the dam to a vote before the country's legislature, the National
People's Congress (NPC), in 1992. The dam was approved, but about
\href{http://www.gov.cn/xinwen/2014-12/13/content_2790541.htm}{one-third}
of the delegates refused to endorse the plan -\/- an astonishingly low
approval rate for China's usually compliant rubber-stamp parliament.

\includegraphics{https://dynaimage.cdn.cnn.com/cnn/e_blur:500,q_auto:low,w_50,c_fit/http\%3A\%2F\%2Fcdn.cnn.com\%2Fcnnnext\%2Fdam\%2Fassets\%2F200728115851-li-peng-yang-baibing-national-peoples-congress-1992.jpg}

Chinese Prime Minister Li Peng (left) at the National People's Congress
on March 21, 1992 in Beijing, China. Credit: Mike Fiala/AFP/Getty Images

Some delegates said they were blindsided when the Three Gorges Dam
suddenly appeared on the NPC's agenda, without advance notice or
discussions about the project, according to a 1994 edition of "Yangtze!
Yangtze!"

Yang Xinren, a delegate from Jilin province in northeastern China, was
\href{https://journal.probeinternational.org/1994/05/31/chapter-3-2/}{quoted
by the book as saying}: "The majority of the delegates are not fully
informed of the technical aspects of the project. So no matter how we
vote, we vote in blindness."

\hypertarget{why-is-the-dam-so-controversial}{%
\subsubsection{Why is the dam so
controversial?}\label{why-is-the-dam-so-controversial}}

One of the most controversial aspects of the mega-project was its
enormous cost for villagers who had lived for centuries on the banks of
the river. To make way for the dam's massive reservoir, about
\href{http://www.china.com.cn/economic/txt/2010-01/22/content_19289419.htm}{1.4
million people} were uprooted, their ancestral homes demolished,
communities broken up and farmlands flooded.

Building the Three Gorges Dam displaced
\href{https://online.ucpress.edu/cpcs/article/33/2/223/455/Resettlement-for-China-s-Three-Gorges-Dam-socio}{more
people} than the three largest Chinese dams before it combined. The
reservoir
\href{http://www.gov.cn/ztzl/2006-01/02/content_145309.htm}{submerged}
two cities, 114 towns and 1,680 villages along the river banks.

\includegraphics{https://dynaimage.cdn.cnn.com/cnn/e_blur:500,q_auto:low,w_50,c_fit/http\%3A\%2F\%2Fcdn.cnn.com\%2Fcnnnext\%2Fdam\%2Fassets\%2F200721004148-06-three-gorges-dam.jpg}

Residents of Fengjie, in southwest China's Chongqing, watch the
demolition of buildings in their town on November 4, 2002, to make room
for the Three Gorges Dam's resevoir. Credit: AFP/Getty Images

Displaced residents have
\href{https://www.internationalrivers.org/sites/default/files/attached-files/3gcolor.pdf}{complained
about} inadequate compensation and a lack of farmland and jobs after
relocation. Many have accused local governments of embezzling
resettlement funds and using excessive force to quell protests. In 2013,
the Chinese government
\href{http://www.gov.cn/jrzg/2013-06/07/content_2421831.htm}{acknowledged}
that some of the funds were embezzled or misused.

Many also faced a reduction in living wages. According to Chen Guojie, a
scholar at the government-backed Chinese Academy of Sciences, incomes of
migrant families
\href{http://3gd.ced.berkeley.edu/docs/3GD_Summary.pdf}{dropped by 20\%}
after relocating, as they were forced to abandon their fertile riverside
flatlands to farm on the steep, unsteady slopes.

The dam has also had a serious geological impact. Chinese officials and
experts admitted at a forum in 2007 that the Three Gorges Dam had caused
an array of ecological ills, including more frequent landslides, China's
state news agency Xinhua
\href{https://www.chinadaily.com.cn/china/2007-09/26/content_6137027.htm}{reported}
at the time.

"The huge weight of the water behind the Three Gorges Dam had started to
erode the Yangtze's banks in many places, which, together with frequent
fluctuations in water levels, had triggered a series of landslides," the
Xinhua report said, citing officials and experts at a meeting.

The water in the reservoir saturates and erodes the base of the cliffs,
and the fluctuation in water levels changes the weight of the reservoir
and the pressure on the slopes, destabilizing the shoreline,
\href{https://www.researchgate.net/publication/262973538_Soil_erosion_in_the_Three_Gorges_Reservoir_area}{geologists
say}.

\includegraphics{https://dynaimage.cdn.cnn.com/cnn/e_blur:500,q_auto:low,w_50,c_fit/http\%3A\%2F\%2Fcdn.cnn.com\%2Fcnnnext\%2Fdam\%2Fassets\%2F200721005128-09-three-gorges-dam.jpg}

Water gushes out for the first time through the Three Gorges Dam on June
11, 2003. Credit: AFP/Getty Images

The first disaster came in 2003, shortly after the reservoir started to
fill for the first time. As the water reached 135 meters (115 feet),
landslides began to occur. A few weeks later, on a tributary of the
Three Gorges, a large chunk of a mountain
\href{https://books.google.com.hk/books?id=ATTCtJcrkt4C\&pg=PA210\&lpg=PA210\&dq=Qianjiangping+landslide\&source=bl\&ots=pSZFtOM9s3\&sig=ACfU3U1w2PdniGS4lXmc5Dg_QYpTrzW0XA\&hl=en\&sa=X\&ved=2ahUKEwiJ6qHKr9nqAhUJx4sBHTgaB4g4ChDoATAHegQIBxAB\#v=onepage\&q=Qianjiangping\%20landslide\&f=false}{split
off} and slipped into the river, killing 24 people, destroying 346
houses and capsizing over 20 boats. ****

The dam, which sits near two major fault lines, has also been blamed for
a surge in earthquakes in the region. Scientists argue that the weight
of the large reservoir and the permeation of water into the rocks
underneath can trigger earthquakes in regions already under considerable
tectonic stress.

According to a
\href{https://journal.probeinternational.org/2011/06/01/chinese-study-reveals-three-gorges-dam-triggered-3000-earthquakes-numerous-landslides/}{study}
from the China Earthquake Administration, in the six years after the
reservoir was filled in June 2003, 3,429 earthquakes were recorded along
the reservoir; only 94 earthquakes were recorded from January 2000 to
May 2003.

Another major concern is the blocking of sediments. By cutting the flow
of the Yangtze River, the dam has retained huge amounts of silt, which
not only dampens its flood control capacity by filling the reservoir,
but also causes
\href{https://www.reuters.com/article/environment-china-environment-dam-dc/three-gorges-dam-causes-downstream-erosion-study-idUST31738620070521}{significant
erosion} downstream.

And finally, the discovery of 80 large cracks on the Three Gorges Dam's
concrete face, just **** days after the reservoir was filled for the
first time in 2003, didn't help to alleviate concerns about the dam's
safety. Officials said at the time that the cracks were not a threat to
the dam, but could cause leaking if not fixed, according to
\href{https://www.chinadaily.com.cn/en/doc/2003-06/10/content_242838.htm}{Xinhua.}

For those who remembered the
\href{http://en.people.cn/200510/01/eng20051001_211892.html}{collapse}
of 62 dams in Henan in 1975, amid heavy downpours during a typhoon, it
was of little comfort. That event killed more than 26,000 people by the
official count -\/- though other
\href{https://www.internationalrivers.org/resources/the-forgotten-legacy-of-the-banqiao-dam-collapse-7821}{estimates}
were several times higher.

This year, as the floods worsened, rumors over the Three Gorges Dam's
deformation have resurfaced, drawing
\href{https://www.globaltimes.cn/content/1195239.shtml}{fierce rebuttal}
from state media.

But in 2011, the Chinese government admitted the Three Gorges Dam had
created a range of major problems.

"While the Three Gorges project provides huge comprehensive benefits,
there are urgent problems that need to be addressed, such as stabilizing
and improving living conditions for relocated people, protecting the
environment, and preventing geological disasters," China's cabinet, the
State Council, said in a
\href{http://www.gov.cn/ldhd/2011-05/18/content_1866289.htm}{statement}.

\hypertarget{changing-attitudes}{%
\subsubsection{Changing attitudes}\label{changing-attitudes}}

A month before the Three Gorges Dam broke ground in late 1994, Daniel P.
Beard, the Commissioner of the US Bureau of Reclamation,
\href{https://www.govinfo.gov/content/pkg/CREC-1994-12-20/html/CREC-1994-12-20-pt1-PgE12.htm}{declared}
"the dam building era in the United States" to be over, at an
international conference. The US would be finding alternative ways to
solve water problems.

The costs of such projects exceeded original estimates and many benefits
were never realized, Beard said.

\includegraphics{https://dynaimage.cdn.cnn.com/cnn/e_blur:500,q_auto:low,w_50,c_fit/http\%3A\%2F\%2Fcdn.cnn.com\%2Fcnnnext\%2Fdam\%2Fassets\%2F200723002648-001-world-hydropower-dams.jpg}

Water is released from the Three Gorges Dam to relieve flood pressure in
Yichang, central China's Hubei province on July 19, 2020. Credit:
STR/AFP/Getty Images

Shankman, the geologist at Alabama University, said many dams in the
northwestern coast of the US were actually removed because they blocked
the migration of fish from the ocean up the rivers, causing their
populations to drop. In the southeast of the country, upstream dams in
the mountains created environmental problems, driving fish species to
extinction, causing water pollution, and the recession of coastlines due
to the blocking of sediments.

Marence, the dam expert in the Netherlands, said after the boom in dam
building from the 1950s to the 1980s, more countries and organizations
started to become aware of their environmental impacts.

But China pushed on. By **** 2019, China had
\href{https://www.icold-cigb.org/article/GB/world_register/general_synthesis/number-of-dams-by-country-members}{23,841}
large dams, accounting for
\href{https://www.icold-cigb.org/article/GB/world_register/general_synthesis/general-synthesis}{41\%}
of the world total, with Fan saying most of them were built after 2000.
The US was the runner-up on the list, with 9,263 large dams, according
to the International Commission on Large Dams. The organization defines
a "large dam" as a dam with a height of 15 meters (49 feet) or greater,
or a dam between 5 meters and 15 meters which can contain more than 3
million cubic meters in its reservoir. ****

But dams with hydropower facilities do "produce a lot of cheap energy,
and it's renewable," said Matthijs Kok, a hydraulic engineering
professor at Delft University of Technology.

"However, they have an environmental price, and if we want to build new
dams, we should look carefully at the environmental damage. We have to
find compromise," he said.

\begin{itemize}
\item
  \includegraphics{https://dynaimage.cdn.cnn.com/cnn/e_blur:500,q_auto:low,w_50,c_fit,h_28,ar_16:9/http\%3A\%2F\%2Fcdn.cnn.com\%2Fcnnnext\%2Fdam\%2Fassets\%2F200723000418-01-world-hydropower-dams-restricted.jpg}
\item
  \includegraphics{https://dynaimage.cdn.cnn.com/cnn/e_blur:500,q_auto:low,w_50,c_fit,h_28,ar_16:9/http\%3A\%2F\%2Fcdn.cnn.com\%2Fcnnnext\%2Fdam\%2Fassets\%2F200723000640-02-world-hydropower-dams.jpg}
\item
  \includegraphics{https://dynaimage.cdn.cnn.com/cnn/e_blur:500,q_auto:low,w_50,c_fit,h_28,ar_16:9/http\%3A\%2F\%2Fcdn.cnn.com\%2Fcnnnext\%2Fdam\%2Fassets\%2F200723124810-13-world-hydropower-dams-xiluodu.jpg}
\item
  \includegraphics{https://dynaimage.cdn.cnn.com/cnn/e_blur:500,q_auto:low,w_50,c_fit,h_28,ar_16:9/http\%3A\%2F\%2Fcdn.cnn.com\%2Fcnnnext\%2Fdam\%2Fassets\%2F200723034956-10-world-hydropower-dams.jpg}
\item
  \includegraphics{https://dynaimage.cdn.cnn.com/cnn/e_blur:500,q_auto:low,w_50,c_fit,h_28,ar_16:9/http\%3A\%2F\%2Fcdn.cnn.com\%2Fcnnnext\%2Fdam\%2Fassets\%2F200723000919-04-world-hydropower-dams-restricted.jpg}
\item
  \includegraphics{https://dynaimage.cdn.cnn.com/cnn/e_blur:500,q_auto:low,w_50,c_fit,h_28,ar_16:9/http\%3A\%2F\%2Fcdn.cnn.com\%2Fcnnnext\%2Fdam\%2Fassets\%2F200723001021-05-world-hydropower-dams-restricted.jpg}
\item
  \includegraphics{https://dynaimage.cdn.cnn.com/cnn/e_blur:500,q_auto:low,w_50,c_fit,h_28,ar_16:9/http\%3A\%2F\%2Fcdn.cnn.com\%2Fcnnnext\%2Fdam\%2Fassets\%2F200723001557-07-world-hydropower-dams.jpg}
\item
  \includegraphics{https://dynaimage.cdn.cnn.com/cnn/e_blur:500,q_auto:low,w_50,c_fit,h_28,ar_16:9/http\%3A\%2F\%2Fcdn.cnn.com\%2Fcnnnext\%2Fdam\%2Fassets\%2F200723001449-06-world-hydropower-dams-restricted.jpg}
\item
  \includegraphics{https://dynaimage.cdn.cnn.com/cnn/e_blur:500,q_auto:low,w_50,c_fit,h_28,ar_16:9/http\%3A\%2F\%2Fcdn.cnn.com\%2Fcnnnext\%2Fdam\%2Fassets\%2F200723035323-11-world-hydropower-dams-restricted.jpg}
\item
  \includegraphics{https://dynaimage.cdn.cnn.com/cnn/e_blur:500,q_auto:low,w_50,c_fit,h_28,ar_16:9/http\%3A\%2F\%2Fcdn.cnn.com\%2Fcnnnext\%2Fdam\%2Fassets\%2F200723035458-12-world-hydropower-dams-restricted.jpg}
\end{itemize}

1/10

Here are some of the world's largest hydroelectric dams, ranked by the
installed generation capacity of their power stations.\\
The Three Gorges Dam in China.\\
Installed generation capacity: 22,500 megawatts. Credit: Wang
Gang/Xinhua/Getty images

Some geologists say instead of relying on dams to stop flooding, we
should give rivers space and allow them to expand during the flood
season.

"Large alluvial rivers naturally flood during the wet season. Floodwater
is not a problem, that's simply what rivers do. The problem is when you
have a lot of people living in the areas that are subject to flooding,"
Shankman said.

Along the middle and lower reaches of the Yangtze are some of China's
most densely populated areas. For centuries, people have built levees to
protect their communities and farmlands from flooding. But these
measures, too, are imperfect.

With the climate crisis expected to bring about heavier, more frequent
flooding, some experts say China will be forced to find new solutions
for future generations.

\emph{Graphics by CNN's Jason Kwok.}

Search

\begin{itemize}
\tightlist
\item
  \href{/us}{US}

  \begin{itemize}
  \tightlist
  \item
    \href{/specials/us/crime-and-justice}{Crime + Justice}
  \item
    \href{/specials/us/energy-and-environment}{Energy + Environment}
  \item
    \href{/specials/us/extreme-weather}{Extreme Weather}
  \item
    \href{/specials/space-science}{Space + Science}
  \end{itemize}
\item
  \href{/world}{World}

  \begin{itemize}
  \tightlist
  \item
    \href{/africa}{Africa}
  \item
    \href{/americas}{Americas}
  \item
    \href{/asia}{Asia}
  \item
    \href{/australia}{Australia}
  \item
    \href{/china}{China}
  \item
    \href{/europe}{Europe}
  \item
    \href{/india}{India}
  \item
    \href{/middle-east}{Middle East}
  \item
    \href{/uk}{United Kingdom}
  \end{itemize}
\item
  \href{/politics}{Politics}

  \begin{itemize}
  \tightlist
  \item
    \href{/specials/politics/president-donald-trump-45}{45}
  \item
    \href{/specials/politics/congress-capitol-hill}{Congress}
  \item
    \href{/specials/politics/supreme-court-nine}{SCOTUS}
  \item
    \href{/specials/politics/fact-check-politics}{Facts First}
  \item
    \href{/specials/politics/2020-election-coverage}{2020}
  \item
    \href{/election/2020/candidates}{Candidates}
  \end{itemize}
\item
  \href{/business}{Business}

  \begin{itemize}
  \tightlist
  \item
    \href{https://money.cnn.com/data/markets/}{Markets}
  \item
    \href{/business/tech}{Tech}
  \item
    \href{/business/media}{Media}
  \item
    \href{/business/success}{Success}
  \item
    \href{/business/perspectives}{Perspectives}
  \item
    \href{/business/videos}{Videos}
  \end{itemize}
\item
  \href{/opinions}{Opinion}

  \begin{itemize}
  \tightlist
  \item
    \href{/specials/opinion/opinion-politics}{Political Op-Eds}
  \item
    \href{/specials/opinion/opinion-social-issues}{Social Commentary}
  \end{itemize}
\item
  \href{/health}{Health}

  \begin{itemize}
  \tightlist
  \item
    \href{/specials/health/food-diet}{Food}
  \item
    \href{/specials/health/fitness-excercise}{Fitness}
  \item
    \href{/specials/health/wellness}{Wellness}
  \item
    \href{/specials/health/parenting}{Parenting}
  \item
    \href{/specials/health/vital-signs}{Vital Signs}
  \end{itemize}
\item
  \href{/entertainment}{Entertainment}

  \begin{itemize}
  \tightlist
  \item
    \href{/entertainment/celebrities}{Stars}
  \item
    \href{/entertainment/movies}{Screen}
  \item
    \href{/entertainment/tv-shows}{Binge}
  \item
    \href{/entertainment/culture}{Culture}
  \item
    \href{/business/media}{Media}
  \end{itemize}
\item
  \href{/business/tech}{Tech}

  \begin{itemize}
  \tightlist
  \item
    \href{/specials/tech/innovate}{Innovate}
  \item
    \href{/specials/tech/gadget}{Gadget}
  \item
    \href{/specials/tech/mission-ahead}{Mission: Ahead}
  \item
    \href{/specials/tech/upstarts}{Upstarts}
  \item
    \href{/specials/tech/work-transformed}{Work Transformed}
  \item
    \href{/specials/tech/innovative-cities}{Innovative Cities}
  \end{itemize}
\item
  \href{/style}{Style}

  \begin{itemize}
  \tightlist
  \item
    \href{/style/arts}{Arts}
  \item
    \href{/style/design}{Design}
  \item
    \href{/style/fashion}{Fashion}
  \item
    \href{/style/architecture}{Architecture}
  \item
    \href{/style/luxury}{Luxury}
  \item
    \href{/style/beauty}{Beauty}
  \item
    \href{/style/videos}{Video}
  \end{itemize}
\item
  \href{/travel}{Travel}

  \begin{itemize}
  \tightlist
  \item
    \href{/travel/destinations}{Destinations}
  \item
    \href{/travel/food-and-drink}{Food \& Drink}
  \item
    \href{/travel/news}{News}
  \item
    \href{/travel/stay}{Stay}
  \item
    \href{/travel/videos}{Videos}
  \end{itemize}
\item
  \href{http://bleacherreport.com}{Sports}

  \begin{itemize}
  \tightlist
  \item
    \href{http://bleacherreport.com/nfl}{Pro Football}
  \item
    \href{http://bleacherreport.com/college-football}{College Football}
  \item
    \href{http://bleacherreport.com/nba}{Basketball}
  \item
    \href{http://bleacherreport.com/mlb}{Baseball}
  \item
    \href{http://bleacherreport.com/world-football}{Soccer}
  \item
    \href{/specials/sport/winter-olympics-2018}{Olympics}
  \end{itemize}
\item
  \href{/videos}{Videos}

  \begin{itemize}
  \tightlist
  \item
    \href{//cnn.it/go2}{Live TV}
  \item
    \href{/specials/digital-studios}{Digital Studios}
  \item
    \href{/specials/videos/digital-shorts}{CNN Films}
  \item
    \href{/specials/videos/hln}{HLN}
  \item
    \href{/tv/schedule/cnn}{TV Schedule}
  \item
    \href{/specials/tv/all-shows}{TV Shows A-Z}
  \item
    \href{/vr}{CNNVR}
  \end{itemize}
\item
  \href{//coupons.cnn.com}{Coupons}

  \begin{itemize}
  \tightlist
  \item
    \href{/cnn-underscored/}{CNN Underscored}
  \item
    \href{/specials/cnn-underscored/explore/}{Explore}
  \item
    \href{/specials/cnn-underscored/wellness/}{Wellness}
  \item
    \href{/specials/cnn-underscored/gadgets/}{Gadgets}
  \item
    \href{/specials/cnn-underscored/lifestyle/}{Lifestyle}
  \item
    \href{//store.cnn.com/?utm_source=cnn.com\&utm_medium=referral\&utm_campaign=navbar}{CNN
    Store}
  \end{itemize}
\item
  \href{/more}{More}

  \begin{itemize}
  \tightlist
  \item
    \href{/specials/photos}{Photos}
  \item
    \href{/specials/cnn-longform}{Longform}
  \item
    \href{/specials/cnn-investigates}{Investigations}
  \item
    \href{/specials/profiles}{CNN Profiles}
  \item
    \href{/specials/more/cnn-leadership}{CNN Leadership}
  \item
    \href{/email/subscription}{CNN Newsletters}
  \item
    \href{https://www.turnerjobs.com/search-jobs?orgIds=1174\&ac=19299}{Work
    for CNN}
  \end{itemize}
\end{itemize}

\begin{center}\rule{0.5\linewidth}{\linethickness}\end{center}

Follow CNN

\begin{itemize}
\item
\item
\item
\end{itemize}

\begin{center}\rule{0.5\linewidth}{\linethickness}\end{center}

\begin{itemize}
\tightlist
\item
  \href{/terms}{Terms of Use}
\item
  \href{/privacy}{Privacy Policy}
\item
  \href{/accessibility}{Accessibility \& CC}
\item
  \protect\hyperlink{}{AdChoices}
\item
  \href{/about}{About Us}
\item
  \href{/tour}{CNN Studio Tours}
\item
  \href{/msa}{Modern Slavery Act Statement}
\item
  \href{https://commercial.cnn.com}{Advertise with us}
\item
  \href{//store.cnn.com}{CNN Store}
\item
  \href{/newsletters}{Newsletters}
\item
  \href{/transcripts}{Transcripts}
\item
  \href{/collection}{License Footage}
\item
  \href{http://cnnnewsource.com}{CNN Newsource}
\item
  \href{https://www.cnn.com/sitemap.html}{Sitemap}
\end{itemize}

© 2020 Cable News Network.\href{//www.turner.com}{Turner Broadcasting
System, Inc.}All Rights Reserved.CNN Sans ™ \& © 2016 Cable News
Network.
