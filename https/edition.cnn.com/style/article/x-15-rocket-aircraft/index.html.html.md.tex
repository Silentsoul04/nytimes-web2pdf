\href{/}{}\href{/style}{}

Military WWII

\href{/style/design}{design}

\hypertarget{x-15-the-fastest-manned-rocket-plane-ever}{%
\section{X-15: The fastest manned rocket plane
ever}\label{x-15-the-fastest-manned-rocket-plane-ever}}

Updated 28th July 2020

\includegraphics{https://dynaimage.cdn.cnn.com/cnn/e_blur:500,q_auto:low,w_50,c_fill,g_auto,h_50,ar_1:1/http\%3A\%2F\%2Fcdn.cnn.com\%2Fcnnnext\%2Fdam\%2Fassets\%2F191218144956-ecn-1651-orig-super-tease.jpg}

X-15: The fastest manned rocket plane ever

Written by \href{/profiles/jacopo-prisco}{Jacopo Prisco}, CNN

Bold, black and blazing fast: The North American X-15 was a plane unlike
any other. And although it first flew over 60 years ago, it is still the
quickest manned aircraft ever to fly.

Shaped more like a bullet than a conventional airplane, the
rocket-powered X-15 completed 199 test flights over nine years, starting
in 1959. It could reach the edge of space and then glide back down to
Earth, capturing data that informed the design and engineering of later
American spacecraft, including NASA's space shuttles.

The plane was flown by an elite team of just 12 pilots, including Neil
Armstrong, who would go on to lead the moon landing in 1969.

\includegraphics{https://dynaimage.cdn.cnn.com/cnn/e_blur:500,q_auto:low,w_50,c_fit/http\%3A\%2F\%2Fcdn.cnn.com\%2Fcnnnext\%2Fdam\%2Fassets\%2F200721155028-x15-dv-3.jpg}

The X-15 pictured in the skies above California. Credit: NASA

"One of the X-15 pilots, Bill Dana, once told me that it was the big
ticket -\/- the aircraft to fly," said Christian Gelzer, chief historian
at NASA's Armstrong Flight Research Center, in a phone interview. "It
gave you the greatest speed, the greatest excitement, the greatest
terror. We haven't built anything since that flies within the atmosphere
like the X-15."

\hypertarget{a-big-ask}{%
\subsubsection{'A big ask'}\label{a-big-ask}}

\href{/style/article/doomsday-luxury-bunkers/index.html}{}

\includegraphics{https://dynaimage.cdn.cnn.com/cnn/e_blur:500,q_auto:low,w_50,c_fill,g_auto,h_50,ar_1:1/http\%3A\%2F\%2Fcdn.cnn.com\%2Fcnnnext\%2Fdam\%2Fassets\%2F170320111319-safe-house-3.jpg}

Billionaire bunkers: How the 1\% are preparing for the apocalypse

The
\href{https://www.nasa.gov/centers/armstrong/images/X-Planes/index.html}{"X"
series} consists of more than 60 experimental aircraft produced by US
government agencies, including the Air Force and NASA, since the end of
World War II. They were often extreme machines, designed to push the
envelope, and the X-15 had a particularly ambitious goal.

In 1952, when development of the X-15 started, the official air speed
record for an aircraft was just under 700 mph. The plane's mission was
to reach Mach 5 -\/- five times the speed of sound, or nearly 4,000 mph.

\includegraphics{https://dynaimage.cdn.cnn.com/cnn/e_blur:500,q_auto:low,w_50,c_fit/http\%3A\%2F\%2Fcdn.cnn.com\%2Fcnnnext\%2Fdam\%2Fassets\%2F191218144540-071203-f-9999j-032.jpg}

An X-15 rests on Rogers Dry Lake, California, in September 1961
following a mission. Credit: NASA

"Such an airplane would also have to fly at an altitude of 250,000 feet,
which was well above any aircraft's altitude at that point," said
Gelzer. "It was a very big ask."

The project was led by the US Air Force and the National Advisory
Committee for Aeronautics (NACA), which would become NASA in 1958. "They
were after scientific data and flight dynamics data as well," said
Gelzer. "But in the background was the Cold War, which motivated a lot
of the research."

\hypertarget{flying-start}{%
\subsubsection{Flying start}\label{flying-start}}

The X-15 was essentially a rocket with a cockpit, so unlike other planes
it wasn't designed to take off from a runway. Instead, it had to be
taken to high altitude and released from a mothership, in this case a
specially modified B-52 bomber.

\href{/style/article/grumman-x-29-nasa-darpa-fighter-plane/index.html}{}

\includegraphics{https://dynaimage.cdn.cnn.com/cnn/e_blur:500,q_auto:low,w_50,c_fill,g_auto,h_50,ar_1:1/http\%3A\%2F\%2Fcdn.cnn.com\%2Fcnnnext\%2Fdam\%2Fassets\%2F190705151032-grumman-x-29-151028-f-dw547-002.jpg}

Grumman X-29: The impossible fighter jet with inverted wings

With the 50-foot-long X-15 tucked under its wing, the B-52 would take
off from the Edwards Air Force Base in Southern California and fly
towards Nevada or Utah, before turning back and releasing the aircraft
at an altitude of 45,000 feet and a speed of over 600 mph. Only at that
point would the X-15 pilot ignite the rocket engine and start climbing
out of the Earth's atmosphere and into space.

\includegraphics{https://dynaimage.cdn.cnn.com/cnn/e_blur:500,q_auto:low,w_50,c_fit/http\%3A\%2F\%2Fcdn.cnn.com\%2Fcnnnext\%2Fdam\%2Fassets\%2F200721150416-x15-dv-1.jpg}

The aircraft would reach the edge of space before gliding back down to
Earth. Credit: NASA

The fuel, a combination of ammonia and liquid oxygen, lasted less than
two minutes, and it wasn't a smooth ride.

"It flew aerodynamically like a normal airplane, but it climbed like
nobody's business," said Gelzer. "Milt Thompson, who was one of the
pilots, said that it was the only airplane he ever flew in which he was
glad when the engine quit."

\hypertarget{a-glider}{%
\subsubsection{A glider}\label{a-glider}}

Once the target altitude was reached -\/- the X-15 went as high as
354,200 feet, around 10 times the cruise altitude of a commercial
airliner -\/- the pilots would conduct experiments in this then-unknown
environment, helping experts gather data on hypersonic flight.

Much of the X-15's design was geared towards being able to fly at high
altitudes, where the air is so thin that conventional aerodynamic
appendages no longer work. For that reason, the X-15 was equipped with a
reaction control system, similar to that later used by space shuttles
and the International Space Station. It spewed bursts of hydrogen
peroxide -\/- essentially water oxygenated at very high concentration
-\/- which created small amounts of thrust sufficient to steer the
aircraft in the upper atmosphere's thin air.

\href{/style/article/watches-wonders-2020-luxury-watches/index.html}{}

\includegraphics{https://dynaimage.cdn.cnn.com/cnn/e_blur:500,q_auto:low,w_50,c_fill,g_auto,h_50,ar_1:1/http\%3A\%2F\%2Fcdn.cnn.com\%2Fcnnnext\%2Fdam\%2Fassets\%2F200424120030-watches-wonders-2020-hermes.jpg}

Top 10 timepieces from Watches \& Wonders 2020

Flying at thousands of miles per hour, the outer skin of the X-15 became
very hot due to aerodynamic friction and was therefore made of a special
nickel-chromium alloy called Inconel X. "The aircraft heated up to 1,200
degrees Fahrenheit," Gelzer said. "And the pilots could hear it expand
behind them."

Landing the X-15 wasn't easy. "From the moment it ran out of fuel, or
the pilot turned off the engine, it was a glider. A very heavy, very
fast glider with very small wings -\/- so not even a great glider. At
that point, the pilot only had speed and altitude to exchange for
reaching his destination," said Gelzer.

To make things worse, the front wheel lacked steering and the main
landing gear only had skids (two retractable steel beams that skidded
across the landing surface), so a tarmac runway couldn't be used.
Instead, the aircraft had to land on a dry lake bed.

"By the time they got the aircraft back on the ground it was not the
same airplane that it had been when it left the base. There were holes
burned in from the heat," said Gelzer.

\hypertarget{long-flight}{%
\subsubsection{Long flight}\label{long-flight}}

Most aircraft make their final landing approach at under 200 mph. The
X-15, however, could start its approach at 20,000 feet and at supersonic
speeds in excess of 1,500 mph -\/- radically different conditions than
most pilots experienced. Things did not always end well.

"This was an experimental aircraft, and things went wrong on almost
every single slide. The remarkable thing is that the pilots managed to
bring the aircraft back consistently, despite the problems they had,"
said Gelzer.

Out of nearly 200 flights, only two resulted in crash landings, one of
which killed pilot Michael Adams. On November 15, 1967, Adams went into
a spin during re-entry and could not straighten the aircraft, which
broke up in the air.

\includegraphics{https://dynaimage.cdn.cnn.com/cnn/e_blur:500,q_auto:low,w_50,c_fit/http\%3A\%2F\%2Fcdn.cnn.com\%2Fcnnnext\%2Fdam\%2Fassets\%2F191218144956-ecn-1651-orig.jpg}

Air Force test pilot Maj. Michael J. Adams stands beside X-15 ship
number one. Credit: NASA

The inherent risks of flying this type of aircraft, half-plane and
half-spaceship, is among the reasons why the X-15's records have never
been beaten with modern engineering. It was also a stepping stone
towards the space program, which had grander ambitions than simply
speed.

Nevertheless, the X-15 is consigned to history as one of the most
successful flight research programs ever conducted, and in its nine
years of operation it garnered a wealth of data about high-speed flight,
returning from space and human physiology. And in 1967, pilot Pete
Knight reached the
\href{https://www.nasa.gov/centers/armstrong/news/FactSheets/FS-052-DFRC.html}{record
speed of 4,520 mph}, or Mach 6.7 (6.7 times the speed of sound).

The X-15 also spawned a generation of astronauts, including one of the
greatest: Neil Armstrong. During one of his
\href{https://www.nasa.gov/centers/dryden/multimedia/imagegallery/X-15/E-USAF-Armstrong-X-15.html}{seven
X-15 flights}, Armstrong displayed the legendary problem-solving
abilities that would eventually land him the command of Apollo 11.

\includegraphics{https://dynaimage.cdn.cnn.com/cnn/e_blur:500,q_auto:low,w_50,c_fit/http\%3A\%2F\%2Fcdn.cnn.com\%2Fcnnnext\%2Fdam\%2Fassets\%2F200721154919-x15-dv-2.jpg}

The X-15 still holds the record as history's fastest manned aircraft.
Credit: NASA

\href{/style/article/b-2-spirit-stealth-bomber/index.html}{}

\includegraphics{https://dynaimage.cdn.cnn.com/cnn/e_blur:500,q_auto:low,w_50,c_fill,g_auto,h_50,ar_1:1/http\%3A\%2F\%2Fcdn.cnn.com\%2Fcnnnext\%2Fdam\%2Fassets\%2F191203130722-gettyimages-900999.jpg}

B-2 Spirit: The \$2 billion flying wing

"In 1962, he made a flight that took him to 205,000 feet and Mach 3.8,"
said Gelzer. "On his way back, he ended up bouncing off the top of the
atmosphere at about 90,000 feet and skipped like a rock. By the time he
got the aircraft turned around, he was over a suburb of Los Angeles with
no power. He still managed to bring the aircraft all the way back and
land on Rogers Dry Lake.

"It turned out to be the longest ever X-15 flight."

Search

\begin{itemize}
\tightlist
\item
  \href{/us}{US}

  \begin{itemize}
  \tightlist
  \item
    \href{/specials/us/crime-and-justice}{Crime + Justice}
  \item
    \href{/specials/us/energy-and-environment}{Energy + Environment}
  \item
    \href{/specials/us/extreme-weather}{Extreme Weather}
  \item
    \href{/specials/space-science}{Space + Science}
  \end{itemize}
\item
  \href{/world}{World}

  \begin{itemize}
  \tightlist
  \item
    \href{/africa}{Africa}
  \item
    \href{/americas}{Americas}
  \item
    \href{/asia}{Asia}
  \item
    \href{/australia}{Australia}
  \item
    \href{/china}{China}
  \item
    \href{/europe}{Europe}
  \item
    \href{/india}{India}
  \item
    \href{/middle-east}{Middle East}
  \item
    \href{/uk}{United Kingdom}
  \end{itemize}
\item
  \href{/politics}{Politics}

  \begin{itemize}
  \tightlist
  \item
    \href{/specials/politics/president-donald-trump-45}{45}
  \item
    \href{/specials/politics/congress-capitol-hill}{Congress}
  \item
    \href{/specials/politics/supreme-court-nine}{SCOTUS}
  \item
    \href{/specials/politics/fact-check-politics}{Facts First}
  \item
    \href{/specials/politics/2020-election-coverage}{2020}
  \item
    \href{/election/2020/candidates}{Candidates}
  \end{itemize}
\item
  \href{/business}{Business}

  \begin{itemize}
  \tightlist
  \item
    \href{https://money.cnn.com/data/markets/}{Markets}
  \item
    \href{/business/tech}{Tech}
  \item
    \href{/business/media}{Media}
  \item
    \href{/business/success}{Success}
  \item
    \href{/business/perspectives}{Perspectives}
  \item
    \href{/business/videos}{Videos}
  \end{itemize}
\item
  \href{/opinions}{Opinion}

  \begin{itemize}
  \tightlist
  \item
    \href{/specials/opinion/opinion-politics}{Political Op-Eds}
  \item
    \href{/specials/opinion/opinion-social-issues}{Social Commentary}
  \end{itemize}
\item
  \href{/health}{Health}

  \begin{itemize}
  \tightlist
  \item
    \href{/specials/health/food-diet}{Food}
  \item
    \href{/specials/health/fitness-excercise}{Fitness}
  \item
    \href{/specials/health/wellness}{Wellness}
  \item
    \href{/specials/health/parenting}{Parenting}
  \item
    \href{/specials/health/vital-signs}{Vital Signs}
  \end{itemize}
\item
  \href{/entertainment}{Entertainment}

  \begin{itemize}
  \tightlist
  \item
    \href{/entertainment/celebrities}{Stars}
  \item
    \href{/entertainment/movies}{Screen}
  \item
    \href{/entertainment/tv-shows}{Binge}
  \item
    \href{/entertainment/culture}{Culture}
  \item
    \href{/business/media}{Media}
  \end{itemize}
\item
  \href{/business/tech}{Tech}

  \begin{itemize}
  \tightlist
  \item
    \href{/specials/tech/innovate}{Innovate}
  \item
    \href{/specials/tech/gadget}{Gadget}
  \item
    \href{/specials/tech/mission-ahead}{Mission: Ahead}
  \item
    \href{/specials/tech/upstarts}{Upstarts}
  \item
    \href{/specials/tech/work-transformed}{Work Transformed}
  \item
    \href{/specials/tech/innovative-cities}{Innovative Cities}
  \end{itemize}
\item
  \href{/style}{Style}

  \begin{itemize}
  \tightlist
  \item
    \href{/style/arts}{Arts}
  \item
    \href{/style/design}{Design}
  \item
    \href{/style/fashion}{Fashion}
  \item
    \href{/style/architecture}{Architecture}
  \item
    \href{/style/luxury}{Luxury}
  \item
    \href{/style/beauty}{Beauty}
  \item
    \href{/style/videos}{Video}
  \end{itemize}
\item
  \href{/travel}{Travel}

  \begin{itemize}
  \tightlist
  \item
    \href{/travel/destinations}{Destinations}
  \item
    \href{/travel/food-and-drink}{Food \& Drink}
  \item
    \href{/travel/news}{News}
  \item
    \href{/travel/stay}{Stay}
  \item
    \href{/travel/videos}{Videos}
  \end{itemize}
\item
  \href{http://bleacherreport.com}{Sports}

  \begin{itemize}
  \tightlist
  \item
    \href{http://bleacherreport.com/nfl}{Pro Football}
  \item
    \href{http://bleacherreport.com/college-football}{College Football}
  \item
    \href{http://bleacherreport.com/nba}{Basketball}
  \item
    \href{http://bleacherreport.com/mlb}{Baseball}
  \item
    \href{http://bleacherreport.com/world-football}{Soccer}
  \item
    \href{/specials/sport/winter-olympics-2018}{Olympics}
  \end{itemize}
\item
  \href{/videos}{Videos}

  \begin{itemize}
  \tightlist
  \item
    \href{//cnn.it/go2}{Live TV}
  \item
    \href{/specials/digital-studios}{Digital Studios}
  \item
    \href{/specials/videos/digital-shorts}{CNN Films}
  \item
    \href{/specials/videos/hln}{HLN}
  \item
    \href{/tv/schedule/cnn}{TV Schedule}
  \item
    \href{/specials/tv/all-shows}{TV Shows A-Z}
  \item
    \href{/vr}{CNNVR}
  \end{itemize}
\item
  \href{//coupons.cnn.com}{Coupons}

  \begin{itemize}
  \tightlist
  \item
    \href{/cnn-underscored/}{CNN Underscored}
  \item
    \href{/specials/cnn-underscored/explore/}{Explore}
  \item
    \href{/specials/cnn-underscored/wellness/}{Wellness}
  \item
    \href{/specials/cnn-underscored/gadgets/}{Gadgets}
  \item
    \href{/specials/cnn-underscored/lifestyle/}{Lifestyle}
  \item
    \href{//store.cnn.com/?utm_source=cnn.com\&utm_medium=referral\&utm_campaign=navbar}{CNN
    Store}
  \end{itemize}
\item
  \href{/more}{More}

  \begin{itemize}
  \tightlist
  \item
    \href{/specials/photos}{Photos}
  \item
    \href{/specials/cnn-longform}{Longform}
  \item
    \href{/specials/cnn-investigates}{Investigations}
  \item
    \href{/specials/profiles}{CNN Profiles}
  \item
    \href{/specials/more/cnn-leadership}{CNN Leadership}
  \item
    \href{/email/subscription}{CNN Newsletters}
  \item
    \href{https://www.turnerjobs.com/search-jobs?orgIds=1174\&ac=19299}{Work
    for CNN}
  \end{itemize}
\end{itemize}

\begin{center}\rule{0.5\linewidth}{\linethickness}\end{center}

Follow CNN

\begin{itemize}
\item
\item
\item
\end{itemize}

\begin{center}\rule{0.5\linewidth}{\linethickness}\end{center}

\begin{itemize}
\tightlist
\item
  \href{/terms}{Terms of Use}
\item
  \href{/privacy}{Privacy Policy}
\item
  \href{/accessibility}{Accessibility \& CC}
\item
  \protect\hyperlink{}{AdChoices}
\item
  \href{/about}{About Us}
\item
  \href{/tour}{CNN Studio Tours}
\item
  \href{/msa}{Modern Slavery Act Statement}
\item
  \href{https://commercial.cnn.com}{Advertise with us}
\item
  \href{//store.cnn.com}{CNN Store}
\item
  \href{/newsletters}{Newsletters}
\item
  \href{/transcripts}{Transcripts}
\item
  \href{/collection}{License Footage}
\item
  \href{http://cnnnewsource.com}{CNN Newsource}
\item
  \href{https://www.cnn.com/sitemap.html}{Sitemap}
\end{itemize}

© 2020 Cable News Network.\href{//www.turner.com}{Turner Broadcasting
System, Inc.}All Rights Reserved.CNN Sans ™ \& © 2016 Cable News
Network.
