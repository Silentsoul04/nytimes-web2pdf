Sections

SEARCH

\protect\hyperlink{site-content}{Skip to
content}\protect\hyperlink{site-index}{Skip to site index}

\href{https://myaccount.nytimes.com/auth/login?response_type=cookie\&client_id=vi}{}

\href{https://www.nytimes.com/section/todayspaper}{Today's Paper}

Archives\textbar{}Three Mile Island: Notes From a Nightmare

\url{https://nyti.ms/1Ha6a3R}

\begin{itemize}
\item
\item
\item
\item
\item
\end{itemize}

Advertisement

\protect\hyperlink{after-top}{Continue reading the main story}

Supported by

\protect\hyperlink{after-sponsor}{Continue reading the main story}

\hypertarget{three-mile-island-notes-from-a-nightmare}{%
\section{Three Mile Island: Notes From a
Nightmare}\label{three-mile-island-notes-from-a-nightmare}}

By B. Drummond Ayres Jr.;Special to The New York Times

\begin{itemize}
\item
  April 16, 1979
\item
  \begin{itemize}
  \item
  \item
  \item
  \item
  \item
  \end{itemize}
\end{itemize}

\includegraphics{https://s1.nyt.com/timesmachine/pages/1/1979/04/16/112104648_360W.png?quality=75\&auto=webp\&disable=upscale}

See the article in its original context from\\
April 16, 1979, Section A, Page
1\href{https://store.nytimes.com/collections/new-york-times-page-reprints?utm_source=nytimes\&utm_medium=article-page\&utm_campaign=reprints}{Buy
Reprints}

\href{http://timesmachine.nytimes.com/timesmachine/1979/04/16/112104648.html}{View
on timesmachine}

TimesMachine is an exclusive benefit for home delivery and digital
subscribers.

About the Archive

This is a digitized version of an article from The Times's print
archive, before the start of online publication in 1996. To preserve
these articles as they originally appeared, The Times does not alter,
edit or update them.

Occasionally the digitization process introduces transcription errors or
other problems; we are continuing to work to improve these archived
versions.

MIDDLETOWN, Pa. --- The people living near the 372‐foot‐tall cooling
towers of, the nuclear‐powered electric plant on Three Mile Island had
learned to coexist with the raucous sound of venting steam.

So. when yet another burst let go at 3:53 A.M. on Wednesday. March 28,
1979, the few who were awakened rolled over grumpily and went back to
sleep.

Earl Showalter. a 28‐year‐old engineer who was due for work at the
billion‐dollar plant at 8 A.M., only ??sed once or twice in his sleep.
His wife, Sue, reached over and give him a reassuring pat. ``It's
nothing,'' she whispered into the darkness.

Sue Showalter was wrong. The worst accident in the quarter‐century
history of this country's nuclear power program was beginning to unfold
out on Three Mile Island.

It was an accident destined to threaten not only the lives of thousands,
born and unborn, but also the future of nuclear power itself --- an
accident that would generate a week of doomsday fear, panicky flight,
conflicting statements, noisy demonstrations and intense confusion.

Because of a complex series of human and mechanical errors, signaled by
that harmless blast of steam, a reactor at the Metropolitan Edison
Company plant in the middle of the Susquehanna River began to tear
itself apart., loosing small whiffs of radioactivity into the predawn
drill of central Pennsylvania.

The reactor's cooling system had mal- functioned --- a valve
inexplicably had failed --- and the reactor's nuclear core was rapidly
overheating, raising the possibility of a ``meltdown'' sequence in which
the core would get so hot it would sear its way out of its thick
steel‐and‐concrete cocoon, drop to the open ground and begin to spew
radiation wildly.

Warning Klaxon Sounds

In the plant's control room, an outerspace conglomeration of dials,
lights and switches, a warning Klaxon sounded. Some of the 1,200 lights
on the horseshoeshaped control panel began to blink an ominous red.

But there was no panic. The four ``Met Ed'' operators on duty in the
room had confronted a number of reactor problems and breakdowns in the
three months since the big plant opened. Things had always worked out in
the end, just as they did in ``The China Syndrome,'' the
reactor‐gone‐wild thriller film that was playing at area movie theaters.

But suddenly the plant's computer began tapping out line after line of
question marks. Something was happening that its programmers had never
anticipated.

Seven days were to pass before an emergency assembly of the country's
best nuclear scientists could find an answer to those question marks.
Then the red lights blinked off. Other question marks remain.

For all the fear, no one was seriously hurt. A few plant workers
absorbed unusually high doses of radiation, but most experts do not
think they are in any short‐ or long‐term danger. As for the plant's
neighbors, several experts calculated that they absorbed more harmful
radiation after the last Chinese nuclear bomb test than during the
accident at Three Mile Island. Other health experts spedulated that the
psychological trauma of the crisis would cause more health problems in
the long run than radiation.

Inquiries Are Beginning

That is not the end of Three Mile Island's nuclear nightmare. Now that
meltdown is no longer a threat, nuclear experts, congressional
investigators and a special Presidential panel have ahead of them the
difficult and politically touchy task of finding out precisely what
happened and why, and what lessons can be drawn from it all.

Already they are formulating recommendations and fixing blame, some of
it on machines, some of it on men, some of it on big business and some
on government regulators. They are asking questions as sensitive as they
are complex:

Was the plant properly designed and constructed? Was it rushed to
completion so that the operator could achieve several major tax breaks?
Were its technicians sufficiently trained?

Did Federal regulators license the plant without regard to problems
already manifesting themselves? Did they respond quickly enough to news
of the accident? Are Federal nuclear regulations strict enough?

Will the plant or its customers or its insurance companies pick up the
multimillion‐dollar cost of the accident?

While the search goes on for answers to these and other questions, out
on the island the reactor is being coaxed with painstaking care toward a
``cold shutdown'' and the costly, exceedingly complicated clean‐up is
beginning. It is a job that could last for many months, possibly for
years, because pockets of isolated radiation, so ``hot'' that a
30‐second exposure to any one of them would be fatal, are stigmata of
the reactor's malfunction that morning almost three weeks ago. ating
fluid and pressure levels in the reactor into some sort of balance.
There is never any danger that the reactor might explode like a nuclear
bomb --- the design is far different --- but there is danger that some
of the gases being formed might blow up. And, in fact, there is minor
internal explosion at one point.

At times, the control room operators are forced to wear special masks
and protective clothing because of the radiation. Volunteers dart into
especially ``hot'' areas to adjust valves and draw coolant samples. A
half‐dozen or so receive radiation doses close to the permissible limit.
``Somebody has to do it,'' says Edward Houser, a chemistry foreman.

The struggle with the pumps and valves lasts well Into the night, even
though at 11 A.M. a Met Ed spokesman announces confidently that there is
no danger of a meltdown and even though at 2 P.M., Jack Herbein, a Met
Ed vice president, says, ``I wouldn't call it a very serious accident at
this point.''

In midafternoon, Met Ed officials brief Lieut. Gov. William Scranton 3d.
They insist there is no danger of a meltdown but they acknowledge that
analysis of coolant samples indicate there has been some damage to the
fuel core. And they warn that more radiation might escape.

Mr. Scranton, deeply disturbed, issues a statement: ``The situation is
more complex than the company first led us to believe. Metropolitan
Edison has given you and us conflicting information.''

A few families flee, heading. for the homes of relatives and friends who
live well away from Three Mile Island and the invisible, odorless,
tasteless fallout that has now been detected up to 16 miles away.

The Nuclear Regulatory Commission officials who arrived at the plant at
midmorning are having trouble getting the full story on the accident.
For one thing, radiation levels have climbed off the scale in certain
parts of the reactor. Nevertheless, by suppertime the N.R.C. men figure
that the worst might be over, since some of the pumps and valves have
begun to work again. Asked at a 10 P.M. news conference whether the
reactor is under control, Charles Gallina, an N.R.C. investigator,
replies:

``The reactor is stable. They are now bringing it down to a cold
shutdown condition. It is in a safe condition.''

Charles Gallina, it will turn out, premature in his analysis.

3:53 A.M., Wednesday, March 28

The accident begins as the valve fails and safety devices stop the
electricitygenerating turbine, venting the harmless steam that drives it
and awakening Sue Showalter. Seven seconds later the heatproducing chain
reaction in the Three Mile Island reactor is automatically halted by
control rods that drop into the uranium core.

The failure of the valve has blocked one of the cooling systems that are
needed to keep the core from overheating, even when the control rods are
down. Emergency pumps should cut in, bypassing the valve. But they do
not because several weeks earlier someone closed their flow vents, a
direct violation of safety regulations.

A pressure valve atop the 50‐foot‐tall reactor opens to offset the
abnormal temperatures that quickly rise in the core before the chain
reaction can be stopped. But then that valve fails to close. Thousands
of gallons of water, the reactor's vital coolant fluid, begin to escape,
mostly as steam. The core temperatures shoot even higher. At some point,
portions of the fuel rods are exposed --- the first step of a meltdown.

Special pumps spurt in more coolant. But the open valve offsets their
efforts. Worse, in the control room a gauge indicates that the valve has
closed and the coolant has been replaced. The Met Ed operators relax a
bit and begin cutting off the special pumps.

Fuel Rods Severely Damaged

inside the reactor, the fuel rods crack and bend severely as the heat
builds higher and higher. It is 3:59 A.M. The unthinkable is beginning
to happen in the bowels of one of this country's 72 nuclear reactors and
no one realizes it.

However, a few minutes later the control panel begins to indicate that
all is not well in the reactor. The special pumps are restarted, only to
be turned off once more as gauges again indicate a closed valve and
complete coolant replacement. The heat rises again; the fuel rods start
to deteriorate anew.

At 6:10 A.M., with dawn breaking over Three Mile Island, one of the four
control room operators discovers the stuck valve. It is forced shut. A
major problem has been solved.

But there is a new crisis.

The water that has been spurting out of the stuck valve overflows a
holding tank and spills onto the floor of the circular concrete building
that houses the reactor. It is highly radioactive. But since the
building, whose walls are four feet thick, is sealed, none of the
radioactivity escapes.

A sump pump cuts in, only minutes after the beginning of the accident.
It sucks up the spilled water, now several feet deep, and shunts it out
to a sealed container in another building. But that container overflows.
As the fluid hits the floor of the second building, where it was never
intended to be, radioactive gases well up. The air‐conditioning system
kicks them out into the fresh spring air of the Susquehanna Valley. This
is a development the plant's designers never foresaw.

The gases do not contain fatal doses of radiation. Bart any radiation is
dangerous and now a light breeze is beginning to scatter the fallout
beyond the island.

Until this point, very few people outside the control room know of the
accident. A few Met Ed scientists, techicians and executives have been
hastily summoned by the worried plant operators. Met Ed's 350,000
customers know nothing because when the electric generator shut down,
other electric plants instantly increased output and picked up the
slack.

But now, around 7 A.M., with the radiation wafting toward the nearby
riverside villages of Goldsboro and Royalton, as well as the city of
Middletown, Met Ed officials begin calling civil defense authorities in
surrounding counties and at the capital in Harrisburg, 10 miles up the
Susquehanna. The Federal Nuclear Regulatory Commission is also notified.

Some radiation monitors are already showing readings of 20 millirems an
hour on the island and 7 millirems several miles away. (The average
American absorbs anywhere from 100 to 200 millirems of radiation
annually, some from the sun and various electrical devices, some from
medical X‐rays. A chest X‐ray produces about a 30‐millirem exposure.)

The Met Ed plant officials provide few details as they make their calls.
They say there has been an ``incident'' and small amounts of radiation
have escaped.

Whether the officials are attempting at this point to put an optimistic
face on the accident or whether they are simply telling all they know is
to become a matter of considerable dispute. In coming days, the company
will release few details of the early hours of the crisis. Whatever the
case --- and investigators are to try to get to the bottom of it ---
Pennsylvania's civil defense authorities conclude for the moment that
they have not been given enough information to warrant an immediate
evacuation order.

The Governor, Dick Thornburgh, is informed of the accident at 7:50 A.M.
``I can't make much sense out of what Met Ed is reporting,'' he tells
aides. ``You can't make decisions about people's lives without solid
facts. See if we can't get more information.''

Evacuation Plans Begin

The aides try in vain. In the interim, civil defense officials begin, as
best they can, to work up evacuation plans for the million or so people
living within a 20mile radius of Three Mile Island. Tentatively, they
decide to rely on interstate routes, blocking off incoming lanes and
feeding evacuees out to distant shopping malls, armories and sports
arenas.

There is not a great deal of confidence in the plans. A single wreck can
seal off an escape route. The civil defense officials are dealing with
an emergency that most never contemplated.

Shortly before 9 A.M., the Nuclear Regulatory Commission dispatches a
dozen of its inspectors and technicians to Three Mile Island and, at the
same time, notifies the White House of the crisis there. President
Carter, who worked on nuclear reactors while in the Navy, including a
damaged reactor, instructs his aides to watch the situation.

At 9 A.M. the rest of the world learns of the crisis. Wire service
reporters making routine morning checks with police agencies are told of
the Met Ed emergency calls. Their initial dispatches reflect the Met Ed
report and set off no panic. In one, a state trooper says, ``They say
there's no radiation leak. Whatever it is, it's contained.''

But it is not contained. Radioactive gases continue to seep from the
plant's ventilation system. Inside the control room, Met Ed scientists
and technicians are struggling with pumps and valves in a desperate
effort to bring wildly fluctu-

10 A.M., Thursday, March 29

Metropolitan Edison continues to be optimistic. Jack Herbein tells the
200 reporters who have rushed to Middletown: ``There is presently no
danger to the public health or safety. We didn't injure anybody. We
didn't overexpose anybody. And we certainly didn't kill anybody. The
radiation off‐site was absolutely minuscule.''

But outside experts dispute that assessment. Gloria Beers, 29 years old,
keeps her children out of school and makes them spend the day inside
their Middletown house. ``I'm really getting scared about this,'' she
says.

The battle over nuclear power is joined as never before. Some members of
Congress demand an investigation. ``I'm not sure nuclear power can
survive any more events of this kind,'' declares Representative Morris
K. Udall of Arizona, chairman of a House energy subcommittee. Other
legislators head north for a personal look at the Met Ed plant. Ralph
Nader, the consumer advocate, asserts that the accident will increase
opposition to nuclear power. Energy Secretary James Schlesinger counters
that ``nothing is riskless.'' On Wall Street, nuclear power stocks begin
to tumble.

Antinuclear protests break out in many United States cities, as well as
in cities in Europe and Asia. Demonstrators play dead in front of a
utility office in San Francisco. In Hanover, West Germany, 35,000
protestors chant, ``We all live in Pennsylvania.'' Some supermarkets in
Middle Atlantic cities post signs that say, ``We don't sell Pennsylvania
milk.''

In both Washington and Harrisburg, the authorities are having major
trouble getting information about the accident. One problem is the
jammed phone lines. At times, officials at the Nuclear Regulatory
Commission headquarters in Bethesda, Md., outside Washington, cannot get
through to their men on the island.

At midday, Governor Thornburgh dispatches Lieutenant Governor Scranton
to the island for a quick survey and briefing. ``There is no cause for
alarm,'' he announces when Mr. Scranton returns.

Again the optimism is premature. In mid‐afternoon, Met Ed workers are
forced to dump thousands of gallons of mildly radioactive waste water
into the river to make room for overflow from the accident. The reactor
is not cooling down as it should.

Governor Thornburgh becomes angry when he learns that the water has been
dumped without any warning to towns and cities downstream. He spends the
evening talking to as many state, Met Ed and N.R.C. officials as he can
reach. ``I'm not sure anybody really knows what's going on inside that
reactor,'' he tells aides as he heads for bed.

6:40 A.M., Friday, March 30

An unusually strong burst of radiation rises from Three Mile Island,
caused by technicians' juggling with pumps and valves. Monitors that
have been scattered about the Pennsylvania German countryside
immediately pick it up. Civil defense authorities are warned. ``We don't
know what it is yet or how bad,'' says Joe Comey, a state emergency
official.

What is to be the worst day of the crisis has started ominously.

Governor Thornburgh, desperate, calls the N.R.C. headquarters at
Bethesda and talks with the commission chairman, Joseph M. Hendrie. They
discuss evacuation but reach no solid decision because they lack enough
information.

Mr. Hendrie complains to other commission members that, for two crucial
days, he and the Governor have been forced to operate ``almost totally
in the blind.''

``His information is ambiguous,'' he adds, ``mine is nonexistent and ---
I don't know, it's like a couple of blind men staggering around making
decisions.''

A little later, however, Mr. Thornburg makes a decision. He advises all
persons living within a 10‐mile radius of the Three Mile Island plant to
stay indoors, with windows and doors shut, until it can be dermined
whether the new emissions are serious enough to require an evacuation.

``They are finding more fuel damage at the plant than they anticipated
and this apparently is resulting in the increased radiation discharge,''
Paul Critchlow, the Governor's press secretary, tells reporters. He
urges calm.

A Warning and Traffic Jam

But when a loudspeaker truck cruises through Middletown, broadcasting
the Governor's advisory, instant traffic jams result. Long lines form at
gas stations. The telephone system jams, and most callers get nothing
but a busy signal.

In Harrisburg, a warning siren begins to wail, increasing the tension
almost unbearably. A dozen diners jump up in midmeal and flee the Penn'
Harris restaurant. Prisoners at the county jail cry out that they are
trapped.

At 11:15, President Carter calls the Governor. He has become deeply
concerned about all the confusion and the inability of Federal officials
to get solid information about what is happening. He is dispatching a
top N.R.C. official, Harold R. Denton, to the scene by helicopter.
Special phones will be installed by the Army to eliminate the
communications problem. Antiradiation medicine is being forwarded.
Henceforth, the release of information about the reactor will be
co‐ordinated to reduce contradictions and rumors.

``He thinks we've done the right thing so far,'' the Governor tells Mr.
Critchlow as he hangs up. ``He says it's best to err on the side of
caution and safety.''

It is 11:30. Out on Three Mile Island, another major radiation burst is
released. There is something in the reactor that is thwarting cool‐down
efforts.

The Governor's phone rings again. It is Mr. Hendrie. He says more bursts
may follow and that it might be ``wise'' to urge pregnant women and
preschool children to evacuate if they live within five miles of the
plant. Mr Thornburgh does so at a midday press conference, pointing out
that the unborn and the very young are most susceptible to fallout.

``Current radioactivity readings are no higher than they were
yesterday,'' he adds. ``However, the continued presence of radioactivity
in the area and the possiblity of further emissions has led me to
exercise this utmost caution. There is no reason to panic.''

Met Ed's Jack Herbein, at a news conference, pokes a little ridicule at
the Governor's precautions, saying, ``We have our windows and doors
open.''

At 1 P.M., Harold Denton arrives, accompanied by a dozen or so nuclear
specialists, most of them considerably more expert than the N.R.C.
technicians inidaily sent to the island. His men immediately begin to
pore over the plant.

Some Met Ed officials balk at the Federal invasion. But Mr. Denton, the
director of the N.R.C. office of reactor regulation, is in no mood to
quibble. He already has concluded that the utility does not have the
technical ability to handle the reactor problem. ``They're pretty
thin,'' he tells one aide.

By mid‐afternoon, the N.R.C. team is beginning to get a solid fix on the
situation. It is more complicated than anyone had anticipated. A badly
damaged core is only part of the problem. There is also a hydrogen
bubble in the top of the reactor, a product of the intense heat. It
could grow larger. If it does, it might explode, ripping open the
reactor and possibly the four‐foot‐thick concrete walls of the
surrounding building. Or it might displace coolant in the reactor,
starting a meltdown.

Possibilities Are Detailed

None of this is likely to happen at once. It may never happen. But the
crisis on Three Mile Island is acute.

Mr. Denton calls the President. He briefs Governor Thornburgh. Then he
holds a late‐evening news conference and lays out the facts for the
people of central Pennsylvania. He says that the possibility of a
meltdown is ``very remote,'' that there is no ``imminent'' danger to the
public and that no one need remain indoors or evacuate, except pregnant
women and preschool children. He adds, however, that great care must be
used in bringing the reactor to a cold shutdown. ``I think it will be
days before there's any change.'' he says.

Mr. Denton's qualified reassurance comes too late to halt the headlong
exodus. Perhaps 100,000 people have fled. Entire blocks are empty in
Middletown and the police are under instructions to shoot any looters
--- who never materialize. In Goldsboro, 500 yards across the
Susquehanna from the reactor, only a mongrel wanders Main Street. ``It's
a ghost town,'' says Mayor Kenneth Myers.

Twenty miles away, at an evacuation center in Hershey, 6‐year‐old Abby
Baumbach is confused.'Something's wrong with the air,'' she says. ``My
mommy told me it could kill me.''

Saturday, March 31

Walter Creitz, the Met Ed president, starts the day by announcing that
``there were no surprises thoughout the night.''

On Three Mile Island, work has begun on the solution to the bubble
problem. The narrow roads leading to the island are jammed with trucks
hauling heavy equipment, much of it designed to turn hydrogen into water
by combining it with oxygen. Mr. Denton has called in scores of nuclear
experts from around the country. He has also instructed scientists at a
nuclear laboratory in Idaho to run bubbleremoval experiments in a mock
reactor.

A measure of calm returns to the cities and villages in the area,
although some flight continues. ``I can see things moving in a positive
direction,'' Mr. Denton tells reporters at midafternoon. However, he
sharply disputes Jack Herbein's assertion that ``the crisis is over.''

At 8:30 P.M. comes one of the worst scares of all, a news report that
the bubble is growing and might explode or start a meltdown within 48
hours. Panic follows, with some visiting reporters joining the flight
this time. Mr. Denton holds quick news conference. The report is false.
The reactor is still reasonably stable.

But now President Carter, who has been talking to Mr. Denton at least
twice a day, is concerned about the periodic panic in cent:al
Pennsylvania. He announces shortly before midnight that he will
personally visit Three Mile Island on Sunday.

9:30 A.M., Sunday, April 1

``We're very stable,'' George Troffer of Met Ed says. But in Goldsboro,
at the Church of God, the Reverend Richard Deardoff is not so sure. ``I
think God is saying, `Be careful,' `` he says in his sermon to four
worshipers.

At 12:45 P.M. a green and white military helicopter bearing President
Carter and his wife, Rosalynn, circles over Three Mile Island, then puts
down at Harrisburg Airport. A limousine hurries the President to the
plant, where he puts on radiation badge and protective shoes for a tour
of the control room.

Forty‐five minutes later, he briefs reporters at the Middletown civic
hall, saying that the situation is stabilizing but warning that a
precautionary evacuation is still a possibility. He adds, ``If we make
an error, it should be an error on the side of extra `caution and extra
safety.''

His visit seems to serve its purpose.

``The President of the United States just doesn't walk into a danger
area without knowing what is going on,'' Fred Lynch of Middletown
comments as Mr. Carter heads back to Washington.

All afternoon, the technicians and scientists on Three Mile Island
search for ways to get rid of the bubble. They come up with several
plans. But even as they theorize, the bubble is shrinking. The coolant,
circulating through the reactor again, is gradually absorbing the gas.
As the coolant goes out of the reactor, the hydrogen vents out, much as
fizz leaves an open soft drink bottle.

``The trend is definitely down,'' George Troffer reports after supper.
Joseph Fouchard, an N.R.C. spokesman, agrees that the trend is
encouraging. ``But we don't want to create false optimism,'' he adds,
reflecting the Federal agency's continuing caution.

8 A.M., Monday, April 2

There is no rush hour in Harrisburg. Thousands have fled. Governor
Thornburg has authorized absences from state offices. Many businesses
are closed.

Then, at 9:45, comes the best news since 3:53 A.M. Wednesday. ``The
bubble is gone,'' George Troffer announces. ``The reactor is completely
stable and ready for final cooldown. There are no problems left. We are
not emitting any radioactive gases.''

Federal officials again urge caution. But at 11:15, Harold Denton
confirms most of Mr. Troffer's announcement. ``We are showing a dramatic
decrease in bubble size,'' he says, adding that about 90 percent of the
hydrogen appears to have been bled off. ``I am certain it is cause for
optimism. I didn't expect such a rapid change.''

It will be several more days before Mr. Denton will publicly conclude
that ``time is on our side.'' Another week will pass before the pregnant
women and small children return. But the crisis at Three Mile Island is
over.

\textbf{What Went Wrong}

\textbf{Normally,} water in the reactor is heated \textbf{(1)} by the
radioactive core \textbf{(2)} and pressurized to prevent boiling.
\textbf{(3)} Its heat but not its radioactivity --- is transferred
through coils in the steam generator and the hot water recirculates.
\textbf{(4)} Steam turns the turbine blades, \textbf{(5)} is cooled and
condensed back into water, and recirculated.

\textbf{In the accident,} the condensate pump failed \textbf{(A),}
depriving the steam generator of its ability to draw heat out of the
reactor's water system. As the water from the core overheated, pressure
was relieved \textbf{(B)} by venting the pressurizer and rods were
dropped into the core \textbf{(C)} to control the chain reaction.

\textbf{The Situation worsened} because the vent \textbf{(D)} did not
close, and in the absence of pressure, water in the core boiled. As it
did, \textbf{(E)} steam bubbles in the core deprived the fuel assembly
of necessary coolant and damaged it. Further, the water was broken up
into hydrogen and oxygen, and a large hydrogen bubble formed at the top
of the reactor, preventing water from circulating completely.

Advertisement

\protect\hyperlink{after-bottom}{Continue reading the main story}

\hypertarget{site-index}{%
\subsection{Site Index}\label{site-index}}

\hypertarget{site-information-navigation}{%
\subsection{Site Information
Navigation}\label{site-information-navigation}}

\begin{itemize}
\tightlist
\item
  \href{https://help.nytimes.com/hc/en-us/articles/115014792127-Copyright-notice}{©~2020~The
  New York Times Company}
\end{itemize}

\begin{itemize}
\tightlist
\item
  \href{https://www.nytco.com/}{NYTCo}
\item
  \href{https://help.nytimes.com/hc/en-us/articles/115015385887-Contact-Us}{Contact
  Us}
\item
  \href{https://www.nytco.com/careers/}{Work with us}
\item
  \href{https://nytmediakit.com/}{Advertise}
\item
  \href{http://www.tbrandstudio.com/}{T Brand Studio}
\item
  \href{https://www.nytimes.com/privacy/cookie-policy\#how-do-i-manage-trackers}{Your
  Ad Choices}
\item
  \href{https://www.nytimes.com/privacy}{Privacy}
\item
  \href{https://help.nytimes.com/hc/en-us/articles/115014893428-Terms-of-service}{Terms
  of Service}
\item
  \href{https://help.nytimes.com/hc/en-us/articles/115014893968-Terms-of-sale}{Terms
  of Sale}
\item
  \href{https://spiderbites.nytimes.com}{Site Map}
\item
  \href{https://help.nytimes.com/hc/en-us}{Help}
\item
  \href{https://www.nytimes.com/subscription?campaignId=37WXW}{Subscriptions}
\end{itemize}
