Sections

SEARCH

\protect\hyperlink{site-content}{Skip to
content}\protect\hyperlink{site-index}{Skip to site index}

\href{https://www.nytimes.com/section/business}{Business}

\href{https://myaccount.nytimes.com/auth/login?response_type=cookie\&client_id=vi}{}

\href{https://www.nytimes.com/section/todayspaper}{Today's Paper}

\href{/section/business}{Business}\textbar{}THE EMPIRE AND EGO OF DONALD
TRUMP

\url{https://nyti.ms/29ypzqV}

\begin{itemize}
\item
\item
\item
\item
\item
\end{itemize}

Advertisement

\protect\hyperlink{after-top}{Continue reading the main story}

Supported by

\protect\hyperlink{after-sponsor}{Continue reading the main story}

\hypertarget{the-empire-and-ego-of-donald-trump}{%
\section{THE EMPIRE AND EGO OF DONALD
TRUMP}\label{the-empire-and-ego-of-donald-trump}}

By Marylin Bender

\begin{itemize}
\item
  Aug. 7, 1983
\item
  \begin{itemize}
  \item
  \item
  \item
  \item
  \item
  \end{itemize}
\end{itemize}

\includegraphics{https://s1.nyt.com/timesmachine/pages/1/1983/08/07/086000_360W.png?quality=75\&auto=webp\&disable=upscale}

See the article in its original context from\\
August 7, 1983, Section 3, Page
1\href{https://store.nytimes.com/collections/new-york-times-page-reprints?utm_source=nytimes\&utm_medium=article-page\&utm_campaign=reprints}{Buy
Reprints}

\href{http://timesmachine.nytimes.com/timesmachine/1983/08/07/086000.html}{View
on timesmachine}

TimesMachine is an exclusive benefit for home delivery and digital
subscribers.

About the Archive

This is a digitized version of an article from The Times's print
archive, before the start of online publication in 1996. To preserve
these articles as they originally appeared, The Times does not alter,
edit or update them.

Occasionally the digitization process introduces transcription errors or
other problems; we are continuing to work to improve these archived
versions.

HE made his presence known on the island of Manhattan in the mid 70's, a
brash Adonis from the outer boroughs bent on placing his imprint on the
golden rock. Donald John Trump exhibited a flair for self-promotion,
grandiose schemes - and, perhaps not surprisingly, for provoking fury
along the way.

Senior realty titans scoffed, believing that braggadocio was the sum and
substance of the blond, blue-eyed, six-footer who wore maroon suits and
matching loafers, frequented Elaine's and Regine's in the company of
fashion models, and was not abashed to take his armed
bodyguard-chauffeur into a meeting with an investment banker.

The essence of entrepreneurial capitalism, real estate is a business
with a tradition of high-rolling megalomania, of master builders
striving to erect monuments to their visions. It is also typically
dynastic, with businesses being transmitted from fathers to sons and
grandsons, and carried on by siblings. In New York, the names of
Tishman, Lefrak, Rudin, Fisher, Zeckendorf come to mind.

And now there is Trump, a name that has in the last few years become an
internationally recognized symbol of New York City as mecca for the
world's super rich.

''Not many sons have been able to escape their fathers,'' said Donald
Trump, the president of the Trump Organization, by way of interpreting
his accomplishments. Three of them, built since 1976, stand out amidst
the crowded midtown landscape: the 68- story Trump Tower, with its
six-story Atrium housing some of the world's most elegant stores; the
1,400- room Grand Hyatt Hotel, and Trump Plaza, a \$125 million
cooperative apartment. And more is on the way.

''At 37, no one has done more than I in the last seven years,'' Mr.
Trump asserted.

Fifteen years ago, he joined his father's business, an empire of
middle-class apartment houses in Brooklyn, Queens and Staten Island then
worth roughly \$40 million. Today, the Trump Organization controls
assets worth about \$1 billion.

The largest and most striking properties were developed by the younger
Trump and are owned by him individually or with one non-family partner.
While his father, Fred C. Trump, is the company chairman and oversees
the original holdings, the Trump Organization is unquestionably a Donald
Trump extravaganza.

HE makes that clear. At Trump headquarters on the 26th floor of the
Trump Tower astride Fifth Avenue, he opened the door of a room furnished
with a vast table.

''This was supposed to be a board room but what was the sense when
there's only one member,'' said Donald Trump. ''We changed it to a
conference room.''

Mr. Trump assiduously cultivates a more conservative public image now, a
gentleman of taste in a navy- blue suit with discreetly striped shirts
and blue ties, who weekends with his family in Greenwich, Conn. Last
spring he forsook the Hamptons, his former habitat, to buy an estate in
the conservative community.

His pastor, the Rev. Norman Vincent Peale of New York, avowed that he is
''kindly and courteous in certain business negotiations and has a
profound streak of honest humility.''

But Mr. Trump prides himself on being street smart and boasted that
Brooklyn and Queens, where he was raised, are among ''the toughest,
smartest places in the world.'' Mr. Trump prefers the vocabulary of war
and sports to document his exploits, acknowledging ''I don't like to
lose.'' Nor does he like to receive less than full credit for his
victories.

''He was a pretty rough fellow when he was small,'' recalled his father,
who packed off his obstreperous teen-age son to the New York Military
Academy in Cornwall-on-Hudson for his high school education. According
to some of his peers in the industry, Donald Trump has not really
changed much from those boyhood days.

His alternating skills of charming some individuals and riding roughshod
over others has earned Donald Trump a reputation in some quarters as
someone not to be trusted. He reneged, for example, on a promise to
donate to a museum the Art Deco bas- reliefs on the facade of Bonwit
Teller's - bulldozed to make way for Trump Tower. It was a sin deemed
unforgivable by landmark preservationists. But the only negative
comments about Donald Trump these days are given off the record.

''Donald Trump became a controversial person and it worked for him,''
said the 57-year-old Preston Robert Tisch, president of the Loew's
Corporation, and Mr. Trump's opponent on two bitterly contested real
estate projects in New York - now a good friend.

''He is one hell of a salesman,'' noted Francis L. Bryant, Jr., senior
vice president of Manufacturers Hanover Trust, which extended the
construction loans for the Grand Hyatt Hotel and for Trump Plaza, and
the tenant financing for the cooperative apartment and the Trump Tower
condominium.

The Trump Touch

UNLIKE other real estate czars, notably the septuagenarian Harry
Helmsley, ubiquitous in the city's development of office buildings,
apartment houses and hotels, Mr. Trump does not syndicate his deals.

''I don't have to,'' he stated flatly.

Backed initially by his father, Mr. Trump has operated as a lone wolf in
Manhattan for nearly the last decade. He acquires properties through
Trump Enterprises or Wembly Realty Inc. and has them transferred to
Donald J. Trump so that he can personally take the huge tax write- offs
from real estate projects rather than having them ''wasted,'' as he
called it, on a corporation. He also said he saves corporate and
franchise taxes.

But to protect himself against the great risks in the building trade, he
said, ''I've bought tremendous liability insurance. After \$10 million,
it's cheap. You can get million of dollars of insurance for \$500 in
premiums.''

For major deals, he forges a partnership with a single gilt-edged
financial institution or hotel chain. Holiday Inns, for example, is his
co-venturer in the \$200 million Harrah's hotel casino scheduled to open
in Atlantic City next May, the largest gaming palace in the New Jersey
resort.

''It will be the biggest hit yet,'' he predicted. Benefiting from his
successful track record, Mr. Trump was able to obtain a 50 percent share
of the equity in Harrah's in return for a small investment in land,
which he purchased before the referendum that opened the
down-at-the-heels town to gaming - one example of his good timing, or
good fortune.

The most striking evidence of Mr. Trump's entrepreneurship, however, is
in New York.

There is the Grand Hyatt Hotel, reconstituted with a facade of mirrored
glass on the skeleton of the Commodore Hotel adjoining Grand Central
Terminal. Since it opened in 1980, it has been credited with reversing
the deterioration of East 42d Street.

Then, of course, Trump Tower, a skinny bronze and glass skyscraper at
725 Fifth Avenue, atop the former site of Bonwit Teller at the corner of
56th Street. Its Atrium, a vertical shopping mall rendered in peach
marble and bronze with an 80-foot cascade, is a showcase for 40
purveyors of super luxury wares such as Loewe of Madrid, Asprey's of
London and the jewelers, Cartier, Harry Winston and Buccellati.
Purchasers of condominium apartments - 91 are priced above \$1 million -
will start moving in later this month.

And Trump Plaza, the apartment building at Third Avenue and 61st Street,
is scheduled for occupancy in 1984. Its least expensive unit is a
three-and-one-half room apartment for \$255,000.

Three years ago Mr. Trump bought the Barbizon Plaza Hotel on Central
Park South and an adjacent rent-controlled apartment building at the
corner of the Avenue of the Americas. According to Standard Abstract
Corporation, publishers of daily realty reports, he paid about \$13
million for this prime property to which sources now give a market value
of \$124 million.

Last June he offered to shelter the homeless in some of the vacant
apartments - at least until he succeeded in getting the rest of the
tenants to vacate theirs. He is reportedly about to sell the hotel to
foreign investors.

Other plans are aborning. Mr. Trump is now concluding a deal to develop
another site on the East Side, on the same scale as Trump Tower, in
partnership with another leading financial institution.

Donald Trump thus appears to have followed the classic formula of the
venture capitalist, using leverage and luck, and a third element
peculiar to real estate development, namely location. ''I have the best
diamonds in the city of New York as far as location,'' he boasted.

To this formula, Mr. Trump has also added salesmanship, show business -
and timing, riding the real estate boom of the last few years in
spectacular fashion.

The Trump Organization, an umbrella for more than a dozen entities
engaged in real estate and hotel development and management, consists of
45 key employees. Three are executive vice presidents: Louise M.
Sunshine, who turned a formidable talent for raising millions of dollars
in political campaign contributions into a skill at selling
million-dollar apartments; Donald's 34-year-old brother, Robert, now
supervising the hotel casino project in Atlantic City, N.J., and
Donald's Austrian-Czech wife, Ivana, 30 years old, a former model and
Olympic skier, who is in charge of design.

The secretaries address Mr, Trump by his first name. ''Very little gets
on paper around here. Donald does the work of 50 people in his head,''
said Mrs. Sunshine. He ''never stops envisioning,'' she added.

A promotional slide show for Trump Tower describes it as ''the ultimate
vision of an elegant life seen through a golden eye.'' Flashes of model
room settings appear amid scenes of Manhattan's glamorous restaurants,
museums and theaters, while in the background the voice of Frank Sinatra
belts out ''New York, New York.'' The vision is Donald Trump's, though
the word's are Sinatra's: ''A No. 1 - king of the hill.''

The message is a clarion call to wealthy outsiders - foreigners or
Americans from beyond the Hudson. The doormen's scarlet uniforms and
white pith helmets - or high black fur hats in the winter months - evoke
Buckingham Palace. Ivana Trump had them custom made in London.

In the lobby of the atrium a musician in black tie performs at a pink
piano. ''We try to give people a little show,'' said Mrs. Trump, a
slender blond woman with aquamarine eyes. Her model's figure was
sheathed in a white-and-black polka-dotted dress by Galanos. ''The
atrium is flashy but warm,'' she declared.

The roots of the Trump Organization lie deep in the foundations of New
York politics. Fred Trump, for instance, was closely involved with the
Brooklyn Democratic organization which produced a New York Mayor,
Abraham D. Beame, and a New York Governor, Hugh L. Carey, as well as
lesser officials of strategic influence who were in power when Donald
Trump mounted his invasion of Manhattan in 1974 and 1975, a low point in
the city's economic history.

The tax abatement and other concessions he secured from government
agencies were termed by Trump critics as both ''outrageous,'' and
''sweetheart deals'' - presumably awarded as political favors.

In the mid-70's, when plans were being laid for a New York City
Convention Center, Mr. Trump began lobbying for a site in the West 30's,
the vacant railroad yards of the bankrupt Penn Central on which he had
secured an option in exchange for the promise to develop the site. Mr.
Tisch's group backed a site on West 44th Street.

Municipal and state officials responsible for funding the project
eventually swung to his side, and Mr. Trump collected \$880,000 in
commissions and expenses on the Penn Central's sale of the property to
the city for \$12 million. But it still rankles him that his offer to
build the center at a guaranteed price of \$200 million and to waive his
fee, if it were named after his family, was spurned.

The project, since undertaken by the New York State Urban Development
Corporation, has been plagued by cost overruns of \$125 million and is
two years behind schedule. Mr. Trump's inability to resist saying ''I
told you so'' by offering to take charge of finishing the center without
fee, resulted in a verbal shooting match with the chairman of the
corporation. The Hyatt Deal

DURING this same period, Mr. Trump parlayed an option to buy another
Penn Central property, the nearly defunct Commodore Hotel, into a
renovation project that resulted in the Grand Hyatt Hotel. It was made
financially plausible by a 40-year tax abatement from the city - the
first ever granted to a commercial property. The original option cost
Mr. Trump \$500,000.

As a hotel operator and chairman of the New York Convention \& Visitors
Bureau, Mr. Tisch objected to the abatement on the grounds of unfair
advantage.

In retrospect, Mr. Tisch said, it ''was right in what it did for that
section of the city.'' But, he added, that the \$200,000-a-year rental
the city receives in lieu of taxes from the Grand Hyatt is equivalent to
the tax bill for a motel on Eighth Avenue.

Since that skirmish Mr. Tisch and Mr. Trump have become close friends.
They also buried the hatchet over another issue: whether or not to
permit casino gambling in New York State.

Mr. Trump was gung-ho, having envisioned the lobby of the Grand Hyatt
converted to gaming. Mr. Tisch believes he won him over by proving that
the construction of gambling resorts in the Catskills (already on the
drawing board) would have siphoned convention business from New York
City hotels, and so Mr. Trump joined him as co-chairman of a coalition
against permissive legislation. A more likely reason for the Trump
turnaround is that his political allies, Governor Carey and Attorney
General Robert Abrams, had changed their minds from pro to con on
gambling and the likelihood of getting enabling legislation from Albany
appeared nil.

The complexity of the 42d Street hotel deal and his cool derring-do in
pulling it together before his 30th birthday won him the grudging
respect of adversaries, and more crucial to his future plans, of the
major lenders in New York..

Mr. Trump took his option on the Commodore, for which he would
ultimately pay \$10 million, less \$2 million from the sale of its
furniture and equipment, to line up a partner in the Hyatt Corporation,
which was looking for a New York link for its hotel chain. He would
build it; Hyatt would manage it; they would be equal partners.

He then turned to George Peacock, senior vice president of the Equitable
Life Assurance Society. They had previously not done business, but Mr.
Peacock had once been his guest at a United States Open Tennis
competition.

Concerned about the Grand Central area (several office buildings were in
or on the verge of foreclosure, and the city itself was facing
bankruptcy), the Equitable, along with the Bowery Savings Bank and
several smaller banks, promised him \$70 million in mortgages once the
doors of the renovated hotel opened.

''So I took this commitment, which was a statement with 100
stipulations, to the city,'' Mr. Trump recalled. One of those conditions
was that the financing be predicated on obtaining a tax abatement. ''I
said, 'I will build you this incredible, gorgeous, gleaming hotel. I
will put people to work in the construction trades and save hotel jobs
and the Grand Central area will come around.' So the city made the
deal,'' he commented.

Since there was no statutory basis for tax relief to a private
commercial developer, Mr. Trump offered to sell the hotel for \$1 to the
Urban Development Corporation and lease it back for 99 years at a modest
rental in lieu of taxes which the Commodore could not pay. Meanwhile, he
could use the agency's vast powers of condemnation to get rid of
undesirable retail tenants on the lower levels.

With these pieces in place, he obtained a \$70 million construction loan
from Manufacturers Hanover. ''It was a leap of faith,'' the bank's Mr.
Bryant noted.

Mr. Trump acknowledged that he was at substantial risk. He could have
lost \$3 million in option money, architectural and legal fees. ''It
could have been a disaster,'' he said. His father, he added, had taken
''a neutral position'' and the son had won medals making money on
earlier deals. But still, Mr. Trump said of the potential for failure,
''I would have been embarrassed.''

After construction was underway, in 1979, the city's economy picked up,
hotel rates doubled and he changed his plans. Instead of a moderate,
\$38- a-night hostelry, he would build a super-convention hotel
commanding rates of \$90 a night or more.

''The whole economics of the deal changed,'' he says. ''I got another
\$30 million from the Chase Manhattan Bank, so when the time came for me
to put up equity, the bottom line was so good, I didn't have to put up
money. It was timing. In another year, I wouldn't have gotten the
abatement and no one ever will again.''

The Tower Deal

''THE Hyatt really got us acquainted with Donald and that led us to the
next big one,'' said Mr. Peacock of Equitable. The insurance company
owned the land under Bonwit Teller and would sell only if it could get
an active participation in an exciting project.

In 1974, Donald Trump had hired Mrs. Sunshine, finance director of the
Carey re-election campaign, to help him lobby for the convention center.

Mrs. Sunshine said her political interest has waned, though this doesn't
mean the Trump Organization lacks Democratic political entree - Roy Cohn
of Saxe, Bacon, Bolan \& Manley, and the firm of Shea \& Gould are its
litigators - or that its resources are not available to proper
candidates of either party. Donald Trump supported Ronald Reagan in 1980
and has been to the White House several times.

It was Mrs. Sunshine who introduced Mr. Trump in 1975 to her friend,
Marilyn Evins, wife of David Evins, a major stockholder in Genesco, the
owner of Bonwit's. Through Mr. Evins he ascertained that the cash-hungry
conglomerate might be willing to sell Bonwit's lease, which had 29 years
to run.

''Donald was the only developer who made sense to us,'' Mr. Peacock
said.

The Trump-Equitable Fifth Avenue Company was formed, an equal
partnership. Equitable put in the fee. Mr. Trump contributed the lease,
two small units he acquired on East 57th Street, and the air rights to
Tiffany's on the corner, which he needed for a zoning change to build a
high-rise apartment house. The Trump Organization is sales and managing
agent for the building, and Mr. Trump was able to put the family name
over the four-story portal in colossal bronze letters - and two giant
bronze Ts in the atrium.

Chase Manhattan financed his \$24 million purchases of the various
leases and rights, and the bank also formed a syndicate for the \$150
million construction loan.

Mr. Trump expected that Trump Tower would qualify for a residential tax
abatement. But after construction started, the city denied the
exemption, estimated to be worth \$15 million to \$20 million, claiming
it was intended to encourage low- and middle-income housing - not the
deluxe apartments of the Tower. The city is now appealing a State
Supreme Court ruling in Mr. Trump's favor last June.

''I don't need this one,'' he said, ''but it's wrong to hold out the
carrot and then say, 'Trump is not going to get it.' My psyche can't
take that.''

He can, thus, add another star to the honor badge of the venture
capitalist: for putting up practically none of his own money for an
increasingly valuable equity interest in one of New York's most valuable
pieces of real estate.

Anticipated condominium revenues of \$260 million (85 percent of the 263
apartments have been sold) have effectively paid off the construction
loan, leaving Trump Tower unencumbered by mortgages. The partnership
retains ownership of the retail space and the 13 floors of office space,
not yet rented, that are sandwiched in between. This commercial portion
of the building is projected to yield rentals of \$28 million a year.

Numerous New York merchants and real estate brokers expressed doubts
that the Atrium tenants will be able to meet the lofty rents of \$150 to
\$400 a square foot or to pay such capital expenses as the \$3.5 million
Loewe invested in building its three-level store. To cover the \$1
million rent Charles Jourdan is paying for the first year, the store
must sell \$10 millon in shoes and apparel.

Tenants do not know yet what the common charges or the real estate taxes
will be. Such pass-on charges could possibly equal the basic rents, a
broker said, predicting numerous lawsuits over interpretations of the
10- and 20- year leases. ''The merchants will bear the risk of Trump
Tower,'' he said, ''and Trump will have a chance to weed out the ones he
doesn't want.''

''I only want the best,'' Donald Trump said. Holding out for ''the great
names'' fueled rumors that he was having difficulty filling his Atrium,
which opened last February, half rented. ''I took a chance that they
would sign when they saw the building worked,'' he said. He won his
gamble, though he conceded he gave ''a little better break'' to a few of
the hesitant.

From the triplex atop Trump Tower that he and his wife will occupy in
the fall with their two children (Mrs. Trump is expecting their third
child), he will be able to survey the metropolitan region, including
those areas from which he maneuvered his ''escape'' from his father's
business.

While still at the Wharton School, from which he received a bachelor's
degree in 1968, Donald Trump put into practice what he said he learned
''by osmosis'' from the senior Trump. He began purchasing ''little real
estate pieces in Philadelphia and fixing them up,'' he recalled.

After graduation, he joined his father in Brooklyn and kept on buying -
properties in Virginia, Ohio, Nevada, and land in California. Sometimes,
he built, too. He had an eye for good locations and good financing:
''F.H.A. mortgages 40 years out, and 5 1/2 percent interest'' taken over
from owners desperate to sell ''so we didn't have to put up much cash,''
explained Donald Trump.

He refinanced some of his father's older projects, swapped some in tax-
free exchanges and, recently, has been turning others into cooperatives.
For most of the period since he entered business, real estate and
general inflation were skyrocketing.

After five years of such successes, Donald Trump was poised for escape.
Or as some wags put it: ''To trump his father.''

According to Harry Levinson, a Boston-based business psychologist who
has studied family businesses, ''The core problem of the entrepreneur in
the family business is the unresolved Oedipal problem, trying to beat
the old man.'' This is particularly so where the father has been very
successful.

''The son feels so inadequate and unable to compete with the father that
he works out compensatory behavior,'' Dr. Levinson says. ''He goes to
the opposite and blows himself up to deny his feeling of helplessness.
Particularly with an entrepreneur who has to fight through so many
things, this compensatory self-centeredness serves him well.''

A record of successes ''has made it very easy to do deals,'' said Donald
Trump. ''People want to invest with you.'' Inevitably, he is compared
with the late William Zeckendorf, whose monuments include the United
Nations and the Kips Bay housing complex in New York City, Century City
in Los Angeles and Place Ville Marie in Montreal. Acclaimed a genius, he
was finally forced to file both corporate and personal bankruptcy.

''I used him as a model in a sense,'' Donald Trump acknowledged. ''He
was a great visionary but he wasn't fiscally conservative. Having seen
the way he went down taught me to be overly so.''

Said Mr. Bryant of Manufacturers Hanover Trust: ''Mr. Trump appears to
be a wild man. He is not. Zeckendorf was spread from coast to coast.
Donald stays home. He sticks to what he knows.''

Mr. Trump has been selling the properties he accumulated on his
post-college buying binge outside New York and is co-oping 3,000 units
in Brooklyn and Queens.

''We've built up a lot of cash,'' he said. Cash to use ''not necessarily
in this business - I'm not married to this business.'' Associates say he
likes both the communications and the sports industry, and he admitted
to being fascinated by ''the merger game.'' He does not find the
takeover of a company he considers mismanaged to be daunting.

After the Government legalized private ownership of gold on Jan. 1,
1975, he jumped in and bought heavily. An ounce was then selling for
\$185. ''We sold in the range of \$780, \$790. We did very well. It's
easier than the construction business.'' he said.

Eisner The House

The Trumps Built

The Trump Organization - Umbrella for the Trump Family Assets

The Trump Organization - Umbrella for the Trump Family Assets OWNED
ENTIRELY BY DONALD TRUMP Trump Enterprises Inc. - Purchaser of
investment properties, except the Grand Hyatt Hotel Trump Corporation -
Real estate brokerage service for Trump Tower, Trump Plaza and some
non-Trump projects, including a condominium in St. Moritz, Switzerland
Trump Development Company - Developer of Trump real estate projects
Wembly Realty Inc. - Purchaser of Commodore Hotel on East 42d Street in
Manhattan for conversion to Grand Hyatt Hotel; restoration company for
Grand Central Terminal, and operator of the terminal's tennis courts
Park South Company - Owner of Barbizon Plaza hotel and adjacent building
at 100 Central Park South; sale reportedly imminent Land Corporation of
California - Owner of major California land parcels Gold Company - Buyer
and seller of gold (inactive since January 1980) OWNED IN PARTNERSHIP BY
DONALD TRUMP Regency-Lexington Partners - Half-owner with the Hyatt
Corporation of 1,400-room Grand Hyatt Hotel built on skeleton of
Commodore Hotel Trump-Equitable Fifth Avenue Company - Half-owner with
the Equitable Life Assurance Society of Trump Tower Seashore Corporation
of Atlantic City - Half-owner with Holiday Inn of Harrah's hotel casino,
scheduled to open in 1984 Trump Plaza: The East 61st Street Company -
Limited partnership in 200-unit cooperative apartment, scheduled for
occupancy in 1984; Donald Trump 90 percent, Robert Trump and Louise
Sunshine 5 percent each ,OWNED BY THE TRUMP FAMILY Trump Equities Inc. -
Owner of shopping centers and 25,000 apartment units in Brooklyn, Queens
and Staten Island, and 300-unit senior citizen complex in East Orange,
N.J. Trump Management Inc. - Manager of 25,000 Trump-owned apartment
units in New York area Trump Construction Company - Builder of original
Trump housing units

'HE'S GONEWAY BEYOND ME'

A speck on the horizon is Trump Village in the Coney Island section of
Brooklyn, a middle-income residential complex that is Fred C. Trump's
most visible monument. It was built in the 60's with Mitchell-Lama
financing.

In an office at the rear of 600 Avenue Z, a six-story red brick
apartment house that was formerly the headquarters of the Trump
Organization, sits its chairman, Fred Trump. He has no intention of
moving to Fifth Avenue. ''I don't get involved,'' said the founding
father, tall, reddish-haired and dapper at 77. ''As you know, Donald has
a competitive spirit and I don't want to compete with him.''

Mr. Trump manages the 25,000 units of housing that constitutes the
empire he built, and he boasts about his son, Donald. ''He amazes me.
He's gone way beyond me, absolutely.''

Fred Trump was a prodigy. His mother had to sign his checks when he
started building in 1923 because he was a minor. The impetus for his
large-scale projects came after World War II with Federal financing.

He and his wife, Mary, raised three sons and two daughters in a spacious
house in the Jamaica Estates section of Queens.

The Trump children were indoctrinated in the Protestant work ethic,
loyalty to friends and employees, and in positive thinking, as
promulgated by the family minister, Reverend Norman Vincent Peale. ''The
mind can overcome any obstacle,'' said Donald Trump. ''I never think of
the negative.''

During summers and free time, the boys worked at Trump construction
sites or the rent collection offices. ''Not your normal kid's
vacations,'' noted Robert Trump, executive vice president of the Trump
Organization.

The eldest son, Fred Jr., died a few years ago. Maryanne Trump Barry,
the oldest daughter, is an assistant United States Attorney in Newark
and a candidate for a Federal judgeship. Elizabeth Trump is a secretary
at the Chase Manhattan Bank. ''It's a man's family,'' she said, when
asked why she and Maryanne were not in the real estate business.

Advertisement

\protect\hyperlink{after-bottom}{Continue reading the main story}

\hypertarget{site-index}{%
\subsection{Site Index}\label{site-index}}

\hypertarget{site-information-navigation}{%
\subsection{Site Information
Navigation}\label{site-information-navigation}}

\begin{itemize}
\tightlist
\item
  \href{https://help.nytimes.com/hc/en-us/articles/115014792127-Copyright-notice}{©~2020~The
  New York Times Company}
\end{itemize}

\begin{itemize}
\tightlist
\item
  \href{https://www.nytco.com/}{NYTCo}
\item
  \href{https://help.nytimes.com/hc/en-us/articles/115015385887-Contact-Us}{Contact
  Us}
\item
  \href{https://www.nytco.com/careers/}{Work with us}
\item
  \href{https://nytmediakit.com/}{Advertise}
\item
  \href{http://www.tbrandstudio.com/}{T Brand Studio}
\item
  \href{https://www.nytimes.com/privacy/cookie-policy\#how-do-i-manage-trackers}{Your
  Ad Choices}
\item
  \href{https://www.nytimes.com/privacy}{Privacy}
\item
  \href{https://help.nytimes.com/hc/en-us/articles/115014893428-Terms-of-service}{Terms
  of Service}
\item
  \href{https://help.nytimes.com/hc/en-us/articles/115014893968-Terms-of-sale}{Terms
  of Sale}
\item
  \href{https://spiderbites.nytimes.com}{Site Map}
\item
  \href{https://help.nytimes.com/hc/en-us}{Help}
\item
  \href{https://www.nytimes.com/subscription?campaignId=37WXW}{Subscriptions}
\end{itemize}
