Sections

SEARCH

\protect\hyperlink{site-content}{Skip to
content}\protect\hyperlink{site-index}{Skip to site index}

\href{https://www.nytimes.com/section/business}{Business}

\href{https://myaccount.nytimes.com/auth/login?response_type=cookie\&client_id=vi}{}

\href{https://www.nytimes.com/section/todayspaper}{Today's Paper}

\href{/section/business}{Business}\textbar{}THE STEEL TRADE
NEGOTIATIONS; THE EXPERTS WHO WILL FORGE THE NEW QUOTAS

\url{https://nyti.ms/29xgeMp}

\begin{itemize}
\item
\item
\item
\item
\item
\end{itemize}

Advertisement

\protect\hyperlink{after-top}{Continue reading the main story}

Supported by

\protect\hyperlink{after-sponsor}{Continue reading the main story}

THE STEEL TRADE NEGOTIATIONS

\hypertarget{the-steel-trade-negotiations-the-experts-who-will-forge-the-new-quotas}{%
\section{THE STEEL TRADE NEGOTIATIONS; THE EXPERTS WHO WILL FORGE THE
NEW
QUOTAS}\label{the-steel-trade-negotiations-the-experts-who-will-forge-the-new-quotas}}

Sept. 30, 1984

\begin{itemize}
\item
\item
\item
\item
\item
\end{itemize}

\includegraphics{https://s1.nyt.com/timesmachine/pages/1/1984/09/30/236738_360W.png?quality=75\&auto=webp\&disable=upscale}

See the article in its original context from\\
September 30, 1984, Section 3, Page
6\href{https://store.nytimes.com/collections/new-york-times-page-reprints?utm_source=nytimes\&utm_medium=article-page\&utm_campaign=reprints}{Buy
Reprints}

\href{http://timesmachine.nytimes.com/timesmachine/1984/09/30/236738.html}{View
on timesmachine}

TimesMachine is an exclusive benefit for home delivery and digital
subscribers.

About the Archive

This is a digitized version of an article from The Times's print
archive, before the start of online publication in 1996. To preserve
these articles as they originally appeared, The Times does not alter,
edit or update them.

Occasionally the digitization process introduces transcription errors or
other problems; we are continuing to work to improve these archived
versions.

SUSAN F. RASKY

WASHINGTON PRESIDENT REAGAN and United States trade representative Bill
Brock got most of the attention two weeks ago when the Administration
announced its decision to seek voluntary restraint agreements with
nations that export steel to the United States. As always, however,
there are a number of officials, working behind the scenes, who will
shape and negotiate these agreements, which are aimed at giving American
steel producers a respite from foreign competition.

The roles of the key U.S.T.R. players are likely to overlap and shift as
negotiations progress, but for now the lineup looks like this:

Mr. Brock will oversee the negotiations, although in the next month he
will be busy campaigning for Republican candidates for Congress. Robert
E. Lighthizer, a deputy trade representative who oversees industrial
policy and follows key economic sectors, including steel, agriculture
and services, will be the chief negotiator at most of the sessions.

Michael Smith, also a deputy trade representative and veteran negotiator
with both developing and industrialized countries, will be the
big-picture man. He will guide strategy to insure that an agreement with
one producing nation does not unduly disrupt the other producers.

Charles Blum, an assistant trade representative for industrial and
energy trade policy, is keeping the numbers. His 14-member staff has
already churned out much of the economic, product and market data on
which each agreement will be based.

Free traders all, they have the task, over the next 90 days, to convince
the world's steel-producing countries to reduce their share of the
United States market from 25 percent to 18.5 percent over the next five
years. Each of the exporting countries will have an opportunity to
bargain over the size and composition of its new slice of the smaller
pie. Those that are not persuaded to accept voluntarily a smaller share
risk imposition of higher tariffs.

Some countries with unfair-trade cases currently pending in the United
States already have made a pilgrimage to the trade office for informal
discussions. Over the next 30 days, Mr. Lighthizer plans to meet
informally here with representatives from all the steel-exporting
countries, but most of the formal negotiations are expected to occur in
each country's capital.

Robert E. Lighthizer

It is 18 months since Mr. Lighthizer came to the trade representative's
office from Capitol Hill, and he is still surrounded by the memorabilia
of his career there as staff director and chief counsel to the Senate
Finance Committee. There are innumerable photographs of him with Sen.
Bob Dole, the committee's chairman and Mr. Lighthizer's friend and boss
for five years, as well as one with Rep. Dan Rostenkowski, chairman of
the House Ways and Means Committee, that fondly salutes ''Ambassador
Bob.'' ''There are days when I really miss the Hill,'' he said, propping
his feet up on a coffee table. ''It's a lot easier to get things done up
there.''

Mr. Lighthizer is a veteran of House-Senate conferences on major tax,
budget and trade bills. He is thus used to negotiations, he said, but
not to interminable ones. In Congress, he said, ''ultimately in
committee, in conference and on the floor, there's a show of hands and
the majority wins. That's what forces the compromise. Here, unless you
have a built-in deadline, nobody has the incentive to compromise.''

As he did in the Senate, Mr. Lighthizer said he tries to learn the nuts
and bolts of any issue he negotiates. He likes to do most of the talking
himself rather than delegating negotiations to an aide, he noted,
adding, ''I think it is important for the other side to see a unified
view.''

''I try to be friendly in negotiations,'' he continued. ''I'm not the
theatrical type. The art of persuasion is knowing where the leverage
is.''

During his tenure on the Senate Finance Committee staff, first as chief
minority counsel and later as the youngest Senate staff director, Mr.
Lighthizer built a reputation as a highly intelligent policy aide and a
man who is quick with a comeback.

''Lighthizer always gets the last word in,'' said John J. Salmon, chief
counsel of the House Ways and Means Committee and a close friend. He is
also, he added, a quick study who understands political realities.

Mr. Lighthizer, who is 36 years old, was born in Ashtabula, Ohio,
graduated from Georgetown University and worked at the Washington law
firm of Covington \& Burling before going to the Hill. He has the sort
of freckle-faced looks that are inevitably described as boyish, and that
could be a handicap in a job where much can depend on the appearance and
perception of power. But Mr. Lighthizer shrugs this off.

''As long as they know you've got the portfolio, the age thing doesn't
seem to have any real effect,'' he said. For example, he added, when he
was the chief negotiator for a long-term grain agreement with the Soviet
Union last year, ''I had a sense that, for the first half- hour or so,
they didn't quite accept the fact that I was the person who was going to
conduct the talks for our side. But there came a point where it was
clear they did.''

Michael B. Smith

People who have traveled with Michael Smith on negotiating sessions have
one iron-clad rule: never book the hotel room next to his. ''Picture
yourself lying in bed at three in the morning, and coming through the
wall is this booming voice discussing export credits or citrus quotas
and God knows what else, long-distance to every capital in the world,''
said one colleague who learned the hard way.

''Can I help it if I sparkle at 7 A.M?'' Mr. Smith asked, somewhat
disingenuously. But he admits that he probably is a trade junkie.

Born in Marblehead, Mass., Mr. Smith, 48, an avid sailor, went to
Harvard and joined the State Department. He began his foreign service
career in 1960 as the third secretary at the American Embassy in
Teheran, and he is still officially on loan from State although he has
spent nearly 10 years working for the trade representative's office. In
1979, he was the nation's chief negotiator in the Multi- Fiber
Arrangement talks, and subsequently this country's trade representative
to GATT, the General Agreement on Tariffs and Trade in Geneva. At Mr.
Brock's behest, he has quietly begun trying to lay the groundwork for a
new round of multilateral trade talks, the first in the Reagan
Administration.

Mr. Smith is the United States Trade Office's Japan specialist - he has
just returned from his 47th trip to Tokyo - and has negotiated
semiconductor, beef and citrus agreements, as well as an entire tariff
package earlier this year, with the Japanese. But he has also negotiated
on virtually every type of trade issue and with delegations from most of
the countries that have ever done business with the United States.

''Trade negotiation is a Kabuki operation,'' he said, ''It's posturing
and shadows, wish lists and give lists. I guess I have a reputation for
being the bad cop,'' he added, grinning. ''We go through a negotiating
session, and people think of me as the heavy. Then Brock comes in with
that way that only he has. . .'' While Mr. Brock is reputed to be a
tough negotiator, he can also, by many accounts, lay on Southern charm
that contrasts with Mr. Smith's Yankee severity.

Charles H. Blum

Despite his role as coordinator for the mountain of statistical and
economic data that will be used in the steel negotiations, Charlie Blum
is regarded by friends and colleagues as an idea man.

''I do have oddball ideas,'' he conceded. ''I figure in negotiating, one
of the ways to find out what people really mean, to get past their
rhetoric, is to give them a new idea.'' For example, he asked earnestly,
''What would you think of a group of countries imposing countervailing
duties against another country that subsidizes exports to any one of
them? You could call it Mutual Self Defense Pact for Free Trade and
Steel. It's got problems, of course, but the reason I like this place is
that Brock tolerates oddball ideas.'' He hastened to note that the
notion of mutual self-defense has not been proposed for steel.

The Brooklyn-born Mr. Blum, 39, spent nine years in the Foreign Service,
including a stint as labor attache at the Embassy in San Salvador. Two
of his six children are Salvadoran orphans, adopted, he explained,
because ''there were so many kids out there on the streets, my wife and
I just couldn't bear to leave them.''

''The Foreign Service is awfully good training for a job like this,'' he
said, referring to his current position. ''You learn to understand the
interests of all the parties, to pick up not just what they are saying,
but what they can live with.''

In his last several years at the State Department, Mr. Blum worked
almost exclusively on steel issues and joined the trade office in 1980
as the director of steel trade policy. His responsibilities now include
the full range of industrial products but, at least for the next few
months, steel will occupy most of his time.

''Charlie Blum and Bill Brock are really the ones who devised this new
steel plan and they are committed to making it work,'' said one foreign
steel industry official, who preferred to remain anonymous. ''Charlie
really believes that this new steel program carries the possibility -
not a guarantee, but the possiblity - that the American steel industry
will finally get its act together.''

There is other evidence that Mr. Blum is an optimist: a well-thumbed
copy of the children's book, ''The Little Engine That Could,'' in his
office. ''It's my number one management tool,'' he said. ''I tell
everybody on my staff, whenever they get discouraged, to come in and
read it.''

Advertisement

\protect\hyperlink{after-bottom}{Continue reading the main story}

\hypertarget{site-index}{%
\subsection{Site Index}\label{site-index}}

\hypertarget{site-information-navigation}{%
\subsection{Site Information
Navigation}\label{site-information-navigation}}

\begin{itemize}
\tightlist
\item
  \href{https://help.nytimes.com/hc/en-us/articles/115014792127-Copyright-notice}{©~2020~The
  New York Times Company}
\end{itemize}

\begin{itemize}
\tightlist
\item
  \href{https://www.nytco.com/}{NYTCo}
\item
  \href{https://help.nytimes.com/hc/en-us/articles/115015385887-Contact-Us}{Contact
  Us}
\item
  \href{https://www.nytco.com/careers/}{Work with us}
\item
  \href{https://nytmediakit.com/}{Advertise}
\item
  \href{http://www.tbrandstudio.com/}{T Brand Studio}
\item
  \href{https://www.nytimes.com/privacy/cookie-policy\#how-do-i-manage-trackers}{Your
  Ad Choices}
\item
  \href{https://www.nytimes.com/privacy}{Privacy}
\item
  \href{https://help.nytimes.com/hc/en-us/articles/115014893428-Terms-of-service}{Terms
  of Service}
\item
  \href{https://help.nytimes.com/hc/en-us/articles/115014893968-Terms-of-sale}{Terms
  of Sale}
\item
  \href{https://spiderbites.nytimes.com}{Site Map}
\item
  \href{https://help.nytimes.com/hc/en-us}{Help}
\item
  \href{https://www.nytimes.com/subscription?campaignId=37WXW}{Subscriptions}
\end{itemize}
