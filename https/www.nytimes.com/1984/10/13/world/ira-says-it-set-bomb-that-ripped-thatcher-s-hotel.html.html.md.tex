Sections

SEARCH

\protect\hyperlink{site-content}{Skip to
content}\protect\hyperlink{site-index}{Skip to site index}

\href{https://www.nytimes.com/section/world}{World}

\href{https://myaccount.nytimes.com/auth/login?response_type=cookie\&client_id=vi}{}

\href{https://www.nytimes.com/section/todayspaper}{Today's Paper}

\href{/section/world}{World}\textbar{}I.R.A. SAYS IT SET BOMB THAT
RIPPED THATCHER'S HOTEL

\url{https://nyti.ms/29qaOWP}

\begin{itemize}
\item
\item
\item
\item
\item
\end{itemize}

Advertisement

\protect\hyperlink{after-top}{Continue reading the main story}

Supported by

\protect\hyperlink{after-sponsor}{Continue reading the main story}

\hypertarget{ira-says-it-set-bomb-that-ripped-thatchers-hotel}{%
\section{I.R.A. SAYS IT SET BOMB THAT RIPPED THATCHER'S
HOTEL}\label{ira-says-it-set-bomb-that-ripped-thatchers-hotel}}

By R. W. Apple Jr., Special To the New York Times

\begin{itemize}
\item
  Oct. 13, 1984
\item
  \begin{itemize}
  \item
  \item
  \item
  \item
  \item
  \end{itemize}
\end{itemize}

\includegraphics{https://s1.nyt.com/timesmachine/pages/1/1984/10/13/008333_360W.png?quality=75\&auto=webp\&disable=upscale}

See the article in its original context from\\
October 13, 1984, Section 1, Page
1\href{https://store.nytimes.com/collections/new-york-times-page-reprints?utm_source=nytimes\&utm_medium=article-page\&utm_campaign=reprints}{Buy
Reprints}

\href{http://timesmachine.nytimes.com/timesmachine/1984/10/13/008333.html}{View
on timesmachine}

TimesMachine is an exclusive benefit for home delivery and digital
subscribers.

About the Archive

This is a digitized version of an article from The Times's print
archive, before the start of online publication in 1996. To preserve
these articles as they originally appeared, The Times does not alter,
edit or update them.

Occasionally the digitization process introduces transcription errors or
other problems; we are continuing to work to improve these archived
versions.

The Irish Republican Army took responsibility today for planting a bomb
in a seafront hotel that killed at least two people, wounded at least 34
and came within minutes of killing or maiming Prime Minister Margaret
Thatcher.

Mrs. Thatcher, who escaped without injury, had just left the bathroom of
her second-floor suite to return to work on her speech for the
concluding session of the annual Conservative Party conference when the
explosion ripped through the Grand Hotel early today.

Her husband, Denis, later reported that the bathroom was ''mangled.''

Cabinet Member Trapped

Norman Tebbit, the Minister of Trade and Industry, plunged two stories
when floors collapsed and spent four hours buried beneath rubble before
firemen cut him free. He suffered broken ribs and a gashed thigh but no
internal injuries. Mr. Tebbit is a senior figure in the Government and a
candidate to succeed the Prime Minister.

Sir Anthony Berry, the 59-year-old Tory Member of Parliament for
Enfield, who had been in the House of Commons for two decades, was
officially identified tonight as one of the dead. Sir Anthony's son and
three daughters by his first marriage are first cousins of the Princess
of Wales.

Also believed killed was Roberta Wakeham, the wife of John Wakeham, the
Government Chief Whip, who was himself in intensive care with severe leg
injuries.

Unofficial reports said two other people remained unaccounted for - a
middle-level party organizer and the wife of another. (A fire official
said that a third body, of an unidentified man, was recovered later, The
Associated Press reported, and that workers were searching for a fourth
body.)

All in One Hotel

It appeared that the I.R.A. had tried to wipe out the Cabinet on the
only occasion in the political year when all were staying in one hotel.

In a statement issued through its publicity bureau in Dublin, the I.R.A.
said the bomb had been aimed at Mrs. Thatcher and her ''warmongers.''
British and Irish politicians said it had also been aimed at disrupting
British-Irish discussions of the Northern Ireland problem. (Page 4.)

''Today we were unlucky,'' the I.R.A. statement said, ''but remember we
only have to be lucky once.''

'The Work of Evil Men'

Mrs. Thatcher, shaken, insisted that the conference resume on schedule
this morning and walked in the front door of the Brighton Conference
Center against police advice, describing the bomb attack as ''the work
of evil men.'' Later, in her speech, she termed it ''an attempt to
cripple Her Majesty's democratically elected Government.''

''All attempts to destroy democracy by terrorism will fail,'' the Prime
Minister said, in what many delegates took as a reference not only to
today's explosion but also to violence on picket lines by striking coal
miners.

The police said that whoever planted the explosives must have had
detailed knowledge of both demolitions and the hotel's layout, because
the bomb was placed in such a way that it devastated the white Victorian
building.

It cut a gash through the top four stories and knocked down walls and
partitions, filling hallways and stairwells with dust, rubble and smoke.
Parts of the facade that were left standing were so weakened that they
had to be demolished this afternoon.

Previous I.R.A. Attacks

Last December an I.R.A. bomb planted outside Harrods department store in
London killed six people. Another killed Airey Neave, a close political
associate of the Prime Minister, in 1979. But the explosion today was
the first direct attempt to assassinate Mrs. Thatcher or any other
British prime minister at least since the war.

The bomb went off at 2:50 A.M. (9:50 P.M. Thursday, New York time).

It appeared likely that the devastation this morning would put an end to
the relatively light security, especially in comparison with that in the
United States, that has traditionally surrounded British political
leaders and the royal family. Mrs. Thatcher is accompanied in public by
just a single Scotland Yard plainclothesman.

The police said they had no suspects, although one witness said he had
trailed, and then lost, a man who ran from the building just before the
blast. A separate investigation of security measures was begun at once.

But several Conservatives, including the leader of the House of Lords,
Viscount Whitelaw, defended the police. He said that ''there is no 100
percent security in a democratic country.'' Anger and Condolences

Expressions of anger at the bombing and of condolences for the injured
and the families of the dead came from the Prime Minister Garret
FitzGerald of Ireland, from leaders of all the opposition political
parties, from foreign leaders, from church and business figures, and
from Queen Elizabeth II, who is on vacation near Lexington, Ky.

President Reagan, in Lima, Ohio on a whistle-stop campaign trip, spoke
by telephone with Mrs. Thatcher for six minutes and ''extended
condolences to the families of victims of this morning's incident and
expressed hope that the nations of the world would stand together in an
effort to rid the world of terrorism,'' the White House press office
said in a statement.

Neil Kinnock, the Labor Party leader, told Mrs. Thatcher: ''I hear that
you are carrying on with your normal engagements. That is good. It is
the way that we must respond to such vile acts in this democracy.''

Police experts estimated that the bomb had contained only about 20
pounds of gelignite, although the I.R.A. said it contained 100. The
experts said that it was left in a fifth-floor room on the front of the
hotel and suggested that if more of the hotel's guests had been in their
rooms instead of having a drink in the hotel bars after the conference
ball, many more would have been killed.

At the moment when the shock waves hit, John Selwyn Gummer, the party
chairman, was outside the Prime Minister's door. He said she looked out
and asked, ''Is there anything that I can do to help?'' An Icy Calm

Mrs. Thatcher displayed an icy calm - she later described herself as
''shocked but composed and determined'' - as she and her cabinet
colleagues, many in pajamas, were led by security men down a staircase
to safety. Within minutes, she was at the main Brighton police station,
asking questions, and only six hours later she arrived at the Conference
Center, crisp in blue, exactly on time.

The session opened with a two- minute silence and then prayers. The
first topic to be debated was Northern Ireland, where the I.R.A. has
carried out a bloody campaign against British rule. Speaker after
speaker expressed outrage and pledged that the party would never be
deterred by violence.

During the four hours that Mr. Tebbit, the Trade Secretary, and his wife
were trapped after the explosion, they held hands as they lay moaning
beneath a mass of debris. A mattress had fallen on top of them,
apparently shielding them to some degree. He was dressed in blue
pajamas, his face ashen and contorted with pain, but he was conscious.
The scene was illuminated only by television lights as first the
minister's feet and then his whole body came into view.

Mr. Tebbit told one fireman in jest, ''Get off my bloody feet, Fred.''
Chief Whip Is Rescued

Mr. Wakeham, the Chief Whip, who had been trapped in a vertical
position, was dug out after almost seven hours.

Twelve hours after the explosion, Eric Whittaker, the East Sussex fire
chief, said that eight floors of the hotel, along with furniture, had
collapsed in some places. The inside of the building, he added, was
''all mashed together'' - a hodgepodge of timbers, masonry, furniture,
bedding and plaster.

By then Mrs. Thatcher, who will be 59 on Saturday, was on the rostrum
for her speech, the last event on the conference schedule. She was
welcomed with praise for her ''lionhearted courage and determination''
by Colonel Sir Alastair Graesser, the president of the National Union of
Conservatives.

''The bomb attack at the Grand Hotel earlier this morning,'' Mrs.
Thatcher began, ''was first and foremost an inhuman, undiscriminating
attempt to massacre innocent, unsuspecting men and women staying in
Brighton for our Conservative conference.'' She pledged that the attack
would change nothing and then, with the comment, ''Now, it must be
business as usual,'' she launched into a wide-ranging defense of her
policies.

Advertisement

\protect\hyperlink{after-bottom}{Continue reading the main story}

\hypertarget{site-index}{%
\subsection{Site Index}\label{site-index}}

\hypertarget{site-information-navigation}{%
\subsection{Site Information
Navigation}\label{site-information-navigation}}

\begin{itemize}
\tightlist
\item
  \href{https://help.nytimes.com/hc/en-us/articles/115014792127-Copyright-notice}{©~2020~The
  New York Times Company}
\end{itemize}

\begin{itemize}
\tightlist
\item
  \href{https://www.nytco.com/}{NYTCo}
\item
  \href{https://help.nytimes.com/hc/en-us/articles/115015385887-Contact-Us}{Contact
  Us}
\item
  \href{https://www.nytco.com/careers/}{Work with us}
\item
  \href{https://nytmediakit.com/}{Advertise}
\item
  \href{http://www.tbrandstudio.com/}{T Brand Studio}
\item
  \href{https://www.nytimes.com/privacy/cookie-policy\#how-do-i-manage-trackers}{Your
  Ad Choices}
\item
  \href{https://www.nytimes.com/privacy}{Privacy}
\item
  \href{https://help.nytimes.com/hc/en-us/articles/115014893428-Terms-of-service}{Terms
  of Service}
\item
  \href{https://help.nytimes.com/hc/en-us/articles/115014893968-Terms-of-sale}{Terms
  of Sale}
\item
  \href{https://spiderbites.nytimes.com}{Site Map}
\item
  \href{https://help.nytimes.com/hc/en-us}{Help}
\item
  \href{https://www.nytimes.com/subscription?campaignId=37WXW}{Subscriptions}
\end{itemize}
