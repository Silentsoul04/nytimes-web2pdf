Sections

SEARCH

\protect\hyperlink{site-content}{Skip to
content}\protect\hyperlink{site-index}{Skip to site index}

\href{https://www.nytimes.com/section/us}{U.S.}

\href{https://myaccount.nytimes.com/auth/login?response_type=cookie\&client_id=vi}{}

\href{https://www.nytimes.com/section/todayspaper}{Today's Paper}

\href{/section/us}{U.S.}\textbar{}SPEAKING FOR THE TREES

\href{https://nyti.ms/29CltL9}{https://nyti.ms/29CltL9}

\begin{itemize}
\item
\item
\item
\item
\item
\end{itemize}

Advertisement

\protect\hyperlink{after-top}{Continue reading the main story}

Supported by

\protect\hyperlink{after-sponsor}{Continue reading the main story}

\hypertarget{speaking-for-the-trees}{%
\section{SPEAKING FOR THE TREES}\label{speaking-for-the-trees}}

By Philip Shabecoff

\begin{itemize}
\item
  Dec. 10, 1984
\item
  \begin{itemize}
  \item
  \item
  \item
  \item
  \item
  \end{itemize}
\end{itemize}

\includegraphics{https://s1.nyt.com/timesmachine/pages/1/1984/12/10/139916_360W.png?quality=75\&auto=webp\&disable=upscale}

See the article in its original context from\\
December 10, 1984, Section B, Page
10\href{https://store.nytimes.com/collections/new-york-times-page-reprints?utm_source=nytimes\&utm_medium=article-page\&utm_campaign=reprints}{Buy
Reprints}

\href{http://timesmachine.nytimes.com/timesmachine/1984/12/10/139916.html}{View
on timesmachine}

TimesMachine is an exclusive benefit for home delivery and digital
subscribers.

About the Archive

This is a digitized version of an article from The Times's print
archive, before the start of online publication in 1996. To preserve
these articles as they originally appeared, The Times does not alter,
edit or update them.

Occasionally the digitization process introduces transcription errors or
other problems; we are continuing to work to improve these archived
versions.

''He was shortish. And oldish. And brownish. And mossy. And he spoke
with a voice that was sharpish and bossy.''

He is the Lorax, Dr. Seuss's feisty little defender of Truffula trees
and other living things, and he has been adopted as a symbol by a
Washington-based environment group and by the United Nations
Environmental Program.

The Washington group, the Global Tomorrow Coalition, an alliance of more
than 70 organizations concerned with the United States role in helping
solve long-range environmental and resource problems, has begun
presenting semiannual ''Lorax Awards'' for notable service in helping
people understand broad global environmental issues. The first awards
were made in September to Senator Claiborne Pell, Democrat of Rhode
Island, and Representative Clarence D. Long, Democrat of Maryland.

The United Nations Environmental Program is trying to reinforce its
message by distributing versions of the Lorax book in Portuguese,
Swahili, Hindi, Arabic and Chinese. 'I Speak for the Trees'

Why the Lorax? The articlulate bean-shaped critter with the walrus
mustache can answer for himself, as he does in Dr. Seuss's book about
him, an environmental tale of Once-lers that chopped down the Truffula
trees to make thneed to sell so they could grow rich, only to find that
when the last tree was gone there was no more thneed:

'' 'Mister,' he said with a sawdusty sneeze. 'I am the Lorax. I speak
for the trees. I speak for trees, for the trees have no tongues.' ''

Don Lesh, executive director of Global Tomorrow, says the Lorax was
adopted as a mascot because ''he is a neat way of expressing ideas of
concern for the environment, that someone must be brave enough and
foresighted enough to speak for the trees because the trees do not speak
for themselves.''

The creator of the Lorax, a man who in real life is not named Seuss but
instead is Theodore Geisel, has readily given his approval for use of
the creature in the environmental struggle.

Looks as if the Once-lers are going to get their comeuppance all around.

Advertisement

\protect\hyperlink{after-bottom}{Continue reading the main story}

\hypertarget{site-index}{%
\subsection{Site Index}\label{site-index}}

\hypertarget{site-information-navigation}{%
\subsection{Site Information
Navigation}\label{site-information-navigation}}

\begin{itemize}
\tightlist
\item
  \href{https://help.nytimes.com/hc/en-us/articles/115014792127-Copyright-notice}{©~2020~The
  New York Times Company}
\end{itemize}

\begin{itemize}
\tightlist
\item
  \href{https://www.nytco.com/}{NYTCo}
\item
  \href{https://help.nytimes.com/hc/en-us/articles/115015385887-Contact-Us}{Contact
  Us}
\item
  \href{https://www.nytco.com/careers/}{Work with us}
\item
  \href{https://nytmediakit.com/}{Advertise}
\item
  \href{http://www.tbrandstudio.com/}{T Brand Studio}
\item
  \href{https://www.nytimes.com/privacy/cookie-policy\#how-do-i-manage-trackers}{Your
  Ad Choices}
\item
  \href{https://www.nytimes.com/privacy}{Privacy}
\item
  \href{https://help.nytimes.com/hc/en-us/articles/115014893428-Terms-of-service}{Terms
  of Service}
\item
  \href{https://help.nytimes.com/hc/en-us/articles/115014893968-Terms-of-sale}{Terms
  of Sale}
\item
  \href{https://spiderbites.nytimes.com}{Site Map}
\item
  \href{https://help.nytimes.com/hc/en-us}{Help}
\item
  \href{https://www.nytimes.com/subscription?campaignId=37WXW}{Subscriptions}
\end{itemize}
