Sections

SEARCH

\protect\hyperlink{site-content}{Skip to
content}\protect\hyperlink{site-index}{Skip to site index}

\href{https://www.nytimes.com/pages/weekinreview/index.html}{Week in
Review}

\href{https://myaccount.nytimes.com/auth/login?response_type=cookie\&client_id=vi}{}

\href{https://www.nytimes.com/section/todayspaper}{Today's Paper}

\href{/pages/weekinreview/index.html}{Week in
Review}\textbar{}ANGLO-IRISH AGREEMENT PITS BOTH ENDS AGAINST THE MIDDLE

\url{https://nyti.ms/29wM37P}

\begin{itemize}
\item
\item
\item
\item
\item
\end{itemize}

Advertisement

\protect\hyperlink{after-top}{Continue reading the main story}

Supported by

\protect\hyperlink{after-sponsor}{Continue reading the main story}

\hypertarget{anglo-irish-agreement-pits-both-ends-against-the-middle}{%
\section{ANGLO-IRISH AGREEMENT PITS BOTH ENDS AGAINST THE
MIDDLE}\label{anglo-irish-agreement-pits-both-ends-against-the-middle}}

By Jo Thomas

\begin{itemize}
\item
  Nov. 24, 1985
\item
  \begin{itemize}
  \item
  \item
  \item
  \item
  \item
  \end{itemize}
\end{itemize}

\includegraphics{https://s1.nyt.com/timesmachine/pages/1/1985/11/24/178578_360W.png?quality=75\&auto=webp\&disable=upscale}

See the article in its original context from\\
November 24, 1985, Section 4, Page
3\href{https://store.nytimes.com/collections/new-york-times-page-reprints?utm_source=nytimes\&utm_medium=article-page\&utm_campaign=reprints}{Buy
Reprints}

\href{http://timesmachine.nytimes.com/timesmachine/1985/11/24/178578.html}{View
on timesmachine}

TimesMachine is an exclusive benefit for home delivery and digital
subscribers.

About the Archive

This is a digitized version of an article from The Times's print
archive, before the start of online publication in 1996. To preserve
these articles as they originally appeared, The Times does not alter,
edit or update them.

Occasionally the digitization process introduces transcription errors or
other problems; we are continuing to work to improve these archived
versions.

The agreement Britain signed with the Irish Republic Nov. 15 is supposed
to bring peace and stability to Northern Ireland, but last week it drew
determined opposition from the leaders of the strongest political
parties on both sides of the border.

The agreement sets up an Intergovernmental Conference in which Ireland
will have a consultative voice on a broad range of policy matters in the
six northern counties, whose status as part of Britain would change only
with the consent of the population.

Far from being reassured, the Rev. Ian Paisley and James Molyneaux,
members of the British Parliament and leaders of the two unionist
parties that represent the Protestant majority in Northern Ireland, saw
Dublin's assumption of a say in policy as itself a change in status.
They accused Prime Minister Margaret Thatcher of betrayal. The unionist
leaders contended that the agreement gave Dublin and London ''covert
joint authority'' over Northern Ireland, which they called the first
step toward a united Ireland.

A group of about 30 unionists kicked and punched Tom King, the Secretary
for Northern Ireland, as he arrived for lunch last week at Belfast City
Hall. Afterward Mr. Paisley told Mr. King, who will be Britain's main
representative on the joint body, to ''stay off the streets of Northern
Ireland.'' Yesterday in Belfast, protesting unionists staged one of
their biggest demonstrations yet.

In the Irish Republic, Charles Haughey, leader of Fianna Fail, the
largest political party, attacked the accord for the opposite reasons.
He said that by recognizing British sovereignty in Northern Ireland, the
accord had severely set back Irish unity. He also warned the coalition
Government of Prime Minister Garret FitzGerald that, by agreeing to a
consultative role under continuing British rule, it would have
responsibility without power.

Despite this opposition, the Irish Parliament approved the agreement
last week by a comfortable 13 votes after three days of debate that
became more muted as deputies sensed broad support for at least giving
the agreement a try. It is also expected to pass this week in the
British House of Commons by a large margin. A revolt by a small number
of Tories is possible, but the agreement has won the support of
opposition leaders. Afterward, all 15 unionist members are expected to
resign as a way of forcing by-elections Dec. 19.

The British Government has rejected unionist demands for a referendum.
By-elections will be another way of making known the opinion of the
majority, depending on the turnout. Such a vote will not be without
risk. Four of the constituencies the unionists hold could go to one of
the nationalist parties, the Social Democratic and Labor Party or Sinn
Fein, the political arm of the Irish Republican Army.

Gerry Adams, the president of Sinn Fein and a member of Parliament for
West Belfast, asked the members of the Social Democratic and Labor Party
to cooperate with his party to avoid splitting the nationalist vote and
re-electing the unionists. But they emphatically refused. Their leader,
John Hume, was the driving force behind the agreement and abhors the
I.R.A.

Whatever the election results, Mrs. Thatcher appears determined to go
ahead with the agreement. The conference of ministers it will set up is
likely to meet before Christmas. At the top of its agenda is security.

Those who back the agreement say it will stand or fall on this issue,
although their rationales differ. The British and Irish Governments,
which are looking to cooperate in the defeat of the I.R.A., believe they
can starve the I.R.A. of support by improving the administration of
justice in Northern Ireland. Many Catholic nationalists there view the
police and the locally recruited Ulster Defense Regiment, both
overwhelmingly Protestant, as sectarian forces and complain of abuses.
Mr. Hume says the acid test of the agreement will be the difference it
makes in the lives of ordinary Catholics.

Peter Barry, the Irish Foreign Minister, who will be Dublin's
representative in Northern Ireland, has been critical of the security
forces. In Parliament last week, he also called on the I.R.A. to ''put
away the bomb and the gun,'' declaring: ''The people do not want another
50 years of horror, shame and despair, whatever Mr. Adams and his
lieutenants may say. The people want peace, stability and some measure
of prosperity.''

ENGLISH AND IRISH SIGNIFICANT DATES 1541 King Henry VIII declares
himself king of Ireland. 1603: Queen Elizabeth I crushes Irish
rebellion; English government controls all of Ireland. 1782: Irish
Parliament granted independence, though Dublin Administration is still
appointed by the King. 1916: Easter Uprising in Dublin against British
rule is crushed. 1918: Sinn Fein, nationalist party, sweeps election and
forms first independent Irish Parliament in Dublin. British attempt to
destroy Sinn Fein leads to Anglo-Irish war of 1919-21. 1920-21: Six
Ulster counties are given parliament in Belfast; dominion status granted
to remaining 26. 1968: Civil rights marches in North are violently
attacked. 1969: British troops arrive in Londonderry. Present phase of
violence begins 1972: Britain imposes direct rule in North. 1981: Bobby
Sands, I.R.A. prisoner, dies in hunger strike, prompting riots. Nine
other prisoners die before strike is called off.

Advertisement

\protect\hyperlink{after-bottom}{Continue reading the main story}

\hypertarget{site-index}{%
\subsection{Site Index}\label{site-index}}

\hypertarget{site-information-navigation}{%
\subsection{Site Information
Navigation}\label{site-information-navigation}}

\begin{itemize}
\tightlist
\item
  \href{https://help.nytimes.com/hc/en-us/articles/115014792127-Copyright-notice}{©~2020~The
  New York Times Company}
\end{itemize}

\begin{itemize}
\tightlist
\item
  \href{https://www.nytco.com/}{NYTCo}
\item
  \href{https://help.nytimes.com/hc/en-us/articles/115015385887-Contact-Us}{Contact
  Us}
\item
  \href{https://www.nytco.com/careers/}{Work with us}
\item
  \href{https://nytmediakit.com/}{Advertise}
\item
  \href{http://www.tbrandstudio.com/}{T Brand Studio}
\item
  \href{https://www.nytimes.com/privacy/cookie-policy\#how-do-i-manage-trackers}{Your
  Ad Choices}
\item
  \href{https://www.nytimes.com/privacy}{Privacy}
\item
  \href{https://help.nytimes.com/hc/en-us/articles/115014893428-Terms-of-service}{Terms
  of Service}
\item
  \href{https://help.nytimes.com/hc/en-us/articles/115014893968-Terms-of-sale}{Terms
  of Sale}
\item
  \href{https://spiderbites.nytimes.com}{Site Map}
\item
  \href{https://help.nytimes.com/hc/en-us}{Help}
\item
  \href{https://www.nytimes.com/subscription?campaignId=37WXW}{Subscriptions}
\end{itemize}
