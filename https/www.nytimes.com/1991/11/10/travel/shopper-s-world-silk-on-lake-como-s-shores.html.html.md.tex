Sections

SEARCH

\protect\hyperlink{site-content}{Skip to
content}\protect\hyperlink{site-index}{Skip to site index}

\href{https://www.nytimes.com/section/travel}{Travel}

\href{https://myaccount.nytimes.com/auth/login?response_type=cookie\&client_id=vi}{}

\href{https://www.nytimes.com/section/todayspaper}{Today's Paper}

\href{/section/travel}{Travel}\textbar{}SHOPPER'S WORLD; Silk on Lake
Como's Shores

\href{https://nyti.ms/29nAoZ7}{https://nyti.ms/29nAoZ7}

\begin{itemize}
\item
\item
\item
\item
\item
\end{itemize}

Advertisement

\protect\hyperlink{after-top}{Continue reading the main story}

Supported by

\protect\hyperlink{after-sponsor}{Continue reading the main story}

SHOPPER'S WORLD

\hypertarget{shoppers-world-silk-on-lake-comos-shores}{%
\section{SHOPPER'S WORLD; Silk on Lake Como's
Shores}\label{shoppers-world-silk-on-lake-comos-shores}}

By Deborah Blumenthal

\begin{itemize}
\item
  Nov. 10, 1991
\item
  \begin{itemize}
  \item
  \item
  \item
  \item
  \item
  \end{itemize}
\end{itemize}

\includegraphics{https://s1.nyt.com/timesmachine/pages/1/1991/11/10/531491_360W.png?quality=75\&auto=webp\&disable=upscale}

See the article in its original context from\\
November 10, 1991, Section 5, Page
6\href{https://store.nytimes.com/collections/new-york-times-page-reprints?utm_source=nytimes\&utm_medium=article-page\&utm_campaign=reprints}{Buy
Reprints}

\href{http://timesmachine.nytimes.com/timesmachine/1991/11/10/531491.html}{View
on timesmachine}

TimesMachine is an exclusive benefit for home delivery and digital
subscribers.

About the Archive

This is a digitized version of an article from The Times's print
archive, before the start of online publication in 1996. To preserve
these articles as they originally appeared, The Times does not alter,
edit or update them.

Occasionally the digitization process introduces transcription errors or
other problems; we are continuing to work to improve these archived
versions.

A SEDUCTIVE landscape is what draws most visitors to Lake Como, but
little-known to the those outside the fashion industry, Como, rather
than Milan, is the best place to shop for silk.

While visitors to other major Italian cities these days complain of a
dearth of bargains visitors to Como and its environs can find fine goods
-\/- some with designer names and some without -\/- at very favorable
prices if they shop with a sharp eye. Designer accessories like ties and
scarves are particularly worth considering.

Since the turn of the century, Como, a 2,000-year-old Roman town on the
southwestern shore of the lake of the same name, has been the center of
Italy's silk industry.

Despite the current weakness of the dollar, top-quality silk clothing
and accessories can be bought for 25 to 75 percent less than in the
United States.

At factory shops, the best place for bargains, ties with sophisticated
patterns, reminiscent of Hermes prints, cost as little as \$14
(calculated at 1,270 lire to the dollar). Signature scarves by Yves St.
Laurent, Karl Lagerfeld and Ungaro start at \$40. Blouses made by the
silk manufacturer that is the source for Celine and Escada sell for
\$236.

The styles and technological innovations made by Como's silk
manufacturers may be new, but the raw material has remained constant for
more than 4,000 years. Filaments of 300 to 1,600 yards extruded by
silkworms and used to form their cocoons have been the basis of coveted
fabrics.

In the sixth century silkworms were smuggled out of China in bamboo
canes and brought to the eastern Mediterranean by two Persian merchants
disguised as priests. From there the labor-intensive business of
breeding silkworms is believed to have spread to Sicily in the 12th
century and then north. Master weavers were already working in Florence
in the 13th century, and in the 15th century Venice became a silk
processing center. In the 16th and 17th centuries Milan assumed
prominence as both the Italian and European silk capital, and toward the
turn of the century Como became the country's largest producer.

Como now produces 85 percent of all the silk made in Italy. But the
costly business of raising silkworms was discontinued in Italy after
World War II, and today the fibers are imported from China ready to be
woven, dyed and finished. One cannot help but appreciate what goes into
the production of silk with these figures in mind: It takes 100 cocoons
to weave just one tie, and 630 cocoons to make a blouse.

Designers from virtually every fine house -\/- Armani, Chanel, Hermes,
Ferre, Ungaro, Valentino and Versace, to name just a few -\/- rely on
silk from Como. A good deal of the credit for a designer's success
should go to the silk houses. While the designers may come to the
manufacturers with guidelines and inspiration for the types of fabric
designs they envison, it is the manufacturers' artists in Como who
actually execute the designs.

Some manufacturers have licensing agreements with designers under which
they not only make but also market their goods. In other cases the
manufacturers make the goods but then turn them over to the companies
that market them. Mantero, for example, one of Como's most prestigious
silk manufacturers, is the exclusive maker of Chanel scarves. But to the
chagrin of bargain hunters, none of these are sold at the Mantero shop
in Como. The shop was closed the day I visited but I was told it sells
scarves and ties bearing other designer names such as Ungaro, Ferre,
Ricci and Yves St. Laurent.

One of the most scenic spots for shopping is an upstairs boutique in the
19th-century ivy-covered Villa Sucota, at 19 Via Cernobbio (telephone
23.31.11; the area code for Italy is 39, the city code for Como is 31),
facing Lake Como. Here one finds goods of first and second quality made
by the Ratti company, one of the largest and most respected
manufacturers. Just down the road in Cernobbio is the elegant Villa
d'Este hotel.

The two-room Ratti factory outlet shop is stocked with pillows, ties,
fabric, shelves of scarves, and handbags, and racks of blouses and
dressing gowns. The ties, displayed on three different racks, were
priced at \$14, \$19 and \$23. The \$14 rack held ties with some printed
designs similar to those sold by Hermes. The more expensive ties were
mostly subdued paisleys, designs close to the heart of Antonio Ratti,
the company founder and president. (Not surprisingly, Ralph Lauren is a
Ratti customer.)

Some of the goods are imperfect, but the minor imperfections are not
marked, so every item should be inspected carefully.

Dressing gowns in rich paisleys in green and red tones, made of the
company's soft suede-like Daino silk, sold for \$236. Blouses with
equestrian motifs were also \$236. Although the showroom featured
pictures of models wearing Escada blouses, none were for sale. Most
plentiful were scarves, in prices ranging from \$40 (for 90 by 90
centimeters, nearly a yard square) with signatures by Yves St. Laurent,
Ungaro, Karl Lagerfeld, Celine, Louis Feraud, Jean Patou, Ferre,
Valentino, Balenciaga and Carlos Falchi. Scarves of the same size
without prestigious designer names were \$27. Designer squares measuring
some 55 inches on each side were \$77. Among the most eye-catching items
were large rectangular scarves of black chiffon decorated with
shimmering velvet rectangles in brown, gold, black and blue; they were
priced at \$197.

In addition to pure silk versions, the shop sold scarves of silk and
cashmere, and pure alpaca. A second Ratti shop, on the grounds of the
factory at Guanzate, 30 Via Madonna in Campagna, a 20-minute drive from
the Villa Sucota, had similar items but a little less stock. These
included dashing white Valentino silk rectangles, the perfect New Year's
Eve accessory, for \$47. (The factory can be reached from Como by taking
the autostrada to Milan, exiting at Lomazzo, turning left and then
following the yellow Ratti signs to Guanzate.)

Visitors should be aware that just about every shop in Italy closes for
lunch, generally from 12:15 or 12:30 until anywhere from 2:30 to 3:30.
Most are closed Sunday and occasionally Monday. Phone before visiting.

On the return from Guanzate, we found some other outlet-type shops on
the road back to Como. The Seta Shop, with two locations, Montano
Lucino, 12 Via Manzoni, (47.43.93) and Villaguardia, 7 Via Varesina
(48.19.30), sell a wide variety of clothing and accessories made of silk
-\/- none with designer names -\/- including picture frames, cosmetic
kits, jewelry boxes, shorts, jackets, shirts and pajamas. The most
interesting items at the Montano Lucino shop were men's black silk
pajamas for \$150 and paisley silk shoulder bags trimmed in leather for
\$127.

A far larger selection of silk clothing and accessories as well as
leather goods made by Frey \& C. are sold at Emporio della Seta at
Albate, 190 Via Canturina (59.14.20). This spacious shop is made up of
three cavernous rooms, one selling bolts and remnants of fabric, another
devoted to menswear and a third to women's clothing. The women's shops
offered exquisite silk pareos in floral patterns by Ungaro and Laura
Biagotti (about 55 inches square) for \$79, yard-square scarves by
Charles Jourdan, Montana and Erreuno for \$51, and stylish silk suits,
some reminiscent of alluring Versace designs, for \$340. In the men's
area, Krizia Uomo ties were \$30, paisley silk ties without designer
names \$17 to \$20, and brown paisley men's silk umbrellas \$59.

Mantero, which has a yearly output of between eight and ten million
meters, boasts a client list that includes Chanel, Vuitton, Cartier,
Ricci and Ungaro. Just a smattering of the goods it produces can be
bought at its shop at 68 Via Volta (27.98.61). Among the items on sale
are ties and scarves by Ferre, Ricci, Ungaro and Yves St. Laurent.
Prices here are considerably higher than at outlet shops. Ties range
from about \$39 to \$51, scarves from \$47 to \$94.

Also in the city's center is an old-established shop catering to a
conservative clientele. Moretti, at 69 Via Garibaldi (27.30.49) sells
high-quality silks, including a stellar collection of fabrics for men's
shirts, and some simple, but rather uninspiring ties ranging from \$15
to \$29. A better choice are the opulent alpaca scarves in solids made
by Agnona, some 55 inches square, for \$149.

Travelers planning to spend time visiting Bellagio, an ancient hill town
on a promontory on Lake Como, can also find silks at very attractive
prices. Outdoor display racks in front of boutiques facing the lake
carry fine ties selling for as little as \$12 and \$14. Rolando, at 13
Salita Serbelloni, sells Ferragamo ties, shoes and bags at prices at
least 25 percent less than in the States, and the Azalea boutique at 31
Salita Serbelloni, carries scarves and ties by Missoni.

Advertisement

\protect\hyperlink{after-bottom}{Continue reading the main story}

\hypertarget{site-index}{%
\subsection{Site Index}\label{site-index}}

\hypertarget{site-information-navigation}{%
\subsection{Site Information
Navigation}\label{site-information-navigation}}

\begin{itemize}
\tightlist
\item
  \href{https://help.nytimes.com/hc/en-us/articles/115014792127-Copyright-notice}{©~2020~The
  New York Times Company}
\end{itemize}

\begin{itemize}
\tightlist
\item
  \href{https://www.nytco.com/}{NYTCo}
\item
  \href{https://help.nytimes.com/hc/en-us/articles/115015385887-Contact-Us}{Contact
  Us}
\item
  \href{https://www.nytco.com/careers/}{Work with us}
\item
  \href{https://nytmediakit.com/}{Advertise}
\item
  \href{http://www.tbrandstudio.com/}{T Brand Studio}
\item
  \href{https://www.nytimes.com/privacy/cookie-policy\#how-do-i-manage-trackers}{Your
  Ad Choices}
\item
  \href{https://www.nytimes.com/privacy}{Privacy}
\item
  \href{https://help.nytimes.com/hc/en-us/articles/115014893428-Terms-of-service}{Terms
  of Service}
\item
  \href{https://help.nytimes.com/hc/en-us/articles/115014893968-Terms-of-sale}{Terms
  of Sale}
\item
  \href{https://spiderbites.nytimes.com}{Site Map}
\item
  \href{https://help.nytimes.com/hc/en-us}{Help}
\item
  \href{https://www.nytimes.com/subscription?campaignId=37WXW}{Subscriptions}
\end{itemize}
