Sections

SEARCH

\protect\hyperlink{site-content}{Skip to
content}\protect\hyperlink{site-index}{Skip to site index}

\href{https://www.nytimes.com/section/movies}{Movies}

\href{https://myaccount.nytimes.com/auth/login?response_type=cookie\&client_id=vi}{}

\href{https://www.nytimes.com/section/todayspaper}{Today's Paper}

\href{/section/movies}{Movies}\textbar{}Review/Film; Indian Immigrants
in a Black-and-White Milieu

\href{https://nyti.ms/29gOXQK}{https://nyti.ms/29gOXQK}

\begin{itemize}
\item
\item
\item
\item
\item
\end{itemize}

Advertisement

\protect\hyperlink{after-top}{Continue reading the main story}

Supported by

\protect\hyperlink{after-sponsor}{Continue reading the main story}

\hypertarget{reviewfilm-indian-immigrants-in-a-black-and-white-milieu}{%
\section{Review/Film; Indian Immigrants in a Black-and-White
Milieu}\label{reviewfilm-indian-immigrants-in-a-black-and-white-milieu}}

\begin{itemize}
\tightlist
\item
  Mississippi Masala\\
  Directed by Mira Nair Drama, Romance R 1h 58m
\end{itemize}

By Vincent Canby

\begin{itemize}
\item
  Feb. 5, 1992
\item
  \begin{itemize}
  \item
  \item
  \item
  \item
  \item
  \end{itemize}
\end{itemize}

\includegraphics{https://s1.nyt.com/timesmachine/pages/1/1992/02/05/621292_360W.png?quality=75\&auto=webp\&disable=upscale}

See the article in its original context from\\
February 5, 1992, Section C, Page
15\href{https://store.nytimes.com/collections/new-york-times-page-reprints?utm_source=nytimes\&utm_medium=article-page\&utm_campaign=reprints}{Buy
Reprints}

\href{http://timesmachine.nytimes.com/timesmachine/1992/02/05/621292.html}{View
on timesmachine}

TimesMachine is an exclusive benefit for home delivery and digital
subscribers.

About the Archive

This is a digitized version of an article from The Times's print
archive, before the start of online publication in 1996. To preserve
these articles as they originally appeared, The Times does not alter,
edit or update them.

Occasionally the digitization process introduces transcription errors or
other problems; we are continuing to work to improve these archived
versions.

Near the beginning of Mira Nair's sweetly pungent new comedy,
"Mississippi Masala," Mina (Sarita Choudhury) is driving a large,
borrowed American automobile down a highway near Greenwood, Miss.,
arguing with her mother, who sits imperially in the back. Mina drives
with the hapless self-assurance of someone who doesn't often get behind
a wheel.

With her head turned around to answer her mother, Mina slams into the
rear of the stopped van owned by Demetrius (Denzel Washington), who owns
a rug-cleaning service. The van is slightly damaged, but no one is hurt.
Names and addresses are exchanged. The incident is handled with
comparative amiability, considering the nature of most such encounters.

It is also the first of a series of collisions by which "Mississippi
Masala" vividly dramatizes the uncertain, frequently comic progress of
the love affair of Mina, a spirited young Indian who has never seen
India, and Demetrius, a conscientious, upwardly mobile black American
who has never seen Africa.

The landscape of "Mississippi Masala" is brown and black and white. The
blacks and whites have been in Greenwood for generations. The browns are
newcomers. They are the Indian immigrants who have somehow found their
way to Greenwood and, for reasons not entirely clear, have wound up
owning most of the motels.

The Indian innkeepers are fastidious about their own manners and morals,
but they are equally willing to rent rooms by the night, day or hour.
It's recognized as a respectable business. Yet the so-called New South
remains a network of social and cultural taboos that almost wreck the
lives of Mina and Demetrius.

"Mississippi Masala" appears to have been produced on a modest (by
Hollywood standards) budget, but it is a big movie in terms of talent,
geography and concerns. Racism isn't the major issue, at least on the
surface. Mina and Demetrius must fight the sense of cultural dislocation
that, for different reasons, has become a part of the heritage of each.

"Mississippi Masala" demonstrates that the success of "Salaam Bombay!"
(1988), the first collaboration of Ms. Nair, the director, and Sooni
Taraporevala, her screenwriter, was not an accident. The new film has
its own engagingly idiosyncratic pace. It hurries up, dawdles and then
moves on. It is full of odd characters who are not neatly explained. It
is melancholy without tears.

By way of background for the contemporary story, "Mississippi Masala"
opens with an extensive pre-credit sequence set in Uganda in 1972, a
time of tumult, rude awakenings and "Africa for the Africans." Gen. Idi
Amin has just ordered the expulsion of all Asians from his country.

Mina's father, Jay (Roshan Seth), a successful journalist, is seen being
sent into exile with his wife Kinnu (Sharmila Tagore) and small
daughter. Jay, whose family had lived in Africa for three generations,
always considered himself Ugandan first, Indian second. Career, home,
friends are abruptly abandoned. It is something he never quite accepts,
even long after he has been working at the Monte Cristo, the Greenwood
motel owned by one of his many relatives.

"Mississippi Masala" (masala being the name of a mixture of Indian
spices) displays a generous kind of interest in all of the people who
make up the separate worlds of Mina and Demetrius. Among others there
are Demetrius's layabout brother (Tico Wells) and his father (Joe
Seneca), who has survived by being mannerly in a white society.

There's also Demetrius's former girlfriend (Natalie Oliver), who is on
her way up in the music world and is no longer interested in his
business success. Poor Demetrius. When he finally has a word alone with
her, all that he can say is: "We got some new machines. We're doing work
with deep shags."

Ms. Nair is slightly more caustic about her Indian characters. They
worry and bicker among themselves and work various scams, but they are
energized into collective Indian outrage by the scandalous behavior of
Mina and Demetrius.

Mr. Washington and Ms. Choudhury, whose first film this is, work well
together. He has a screen heft that gives the film its dramatic point.
Her voluptuous presence defines the urgency of the love affair. In terms
of wit and plain old good humor, they are each other's equals.

Mr. Seth ("Ghandi," "My Beautiful Laundrette") and the other members of
the huge cast also count a lot in creating the sense of community, or
lack of same, which is the heart of a film about displaced persons and
reassuring emotional continuity against all odds.

"Mississippi Masala," which has been rated R (under 17 requires
accompanying parent or adult guardian), has some sex, partial nudity and
vulgar language. Mississippi Masala Directed by Mira Nair; written by
Sooni Taraporevala; director of photography, Ed Lachman; edited by
Roberto Silvi; music by L. Subramaniam; production designer, Mitch
Epstein; produced by Michael Nozik and Ms. Nair; released by the Samuel
Goldwyn Company. Running time: 118 minutes. This film is rated R.
Demetrius . . . Denzel Washington Mina . . . Sarita Choudhury Jay . . .
Roshan Seth Kinnu . . . Sharmila Tagore Tyrone . . . Charles S. Dutton
Williben . . . Joe Seneca Anil . . . Ranjit Chowdhry Dexter . . . Tico
Wells Alicia LeShay . . . Natalie Oliver

Advertisement

\protect\hyperlink{after-bottom}{Continue reading the main story}

\hypertarget{site-index}{%
\subsection{Site Index}\label{site-index}}

\hypertarget{site-information-navigation}{%
\subsection{Site Information
Navigation}\label{site-information-navigation}}

\begin{itemize}
\tightlist
\item
  \href{https://help.nytimes.com/hc/en-us/articles/115014792127-Copyright-notice}{©~2020~The
  New York Times Company}
\end{itemize}

\begin{itemize}
\tightlist
\item
  \href{https://www.nytco.com/}{NYTCo}
\item
  \href{https://help.nytimes.com/hc/en-us/articles/115015385887-Contact-Us}{Contact
  Us}
\item
  \href{https://www.nytco.com/careers/}{Work with us}
\item
  \href{https://nytmediakit.com/}{Advertise}
\item
  \href{http://www.tbrandstudio.com/}{T Brand Studio}
\item
  \href{https://www.nytimes.com/privacy/cookie-policy\#how-do-i-manage-trackers}{Your
  Ad Choices}
\item
  \href{https://www.nytimes.com/privacy}{Privacy}
\item
  \href{https://help.nytimes.com/hc/en-us/articles/115014893428-Terms-of-service}{Terms
  of Service}
\item
  \href{https://help.nytimes.com/hc/en-us/articles/115014893968-Terms-of-sale}{Terms
  of Sale}
\item
  \href{https://spiderbites.nytimes.com}{Site Map}
\item
  \href{https://help.nytimes.com/hc/en-us}{Help}
\item
  \href{https://www.nytimes.com/subscription?campaignId=37WXW}{Subscriptions}
\end{itemize}
