Sections

SEARCH

\protect\hyperlink{site-content}{Skip to
content}\protect\hyperlink{site-index}{Skip to site index}

\href{https://www.nytimes.com/section/world}{World}

\href{https://myaccount.nytimes.com/auth/login?response_type=cookie\&client_id=vi}{}

\href{https://www.nytimes.com/section/todayspaper}{Today's Paper}

\href{/section/world}{World}\textbar{}1 Dead, 40 Hurt As a Blast Rips
Central London

\url{https://nyti.ms/298pFl5}

\begin{itemize}
\item
\item
\item
\item
\item
\end{itemize}

Advertisement

\protect\hyperlink{after-top}{Continue reading the main story}

Supported by

\protect\hyperlink{after-sponsor}{Continue reading the main story}

\hypertarget{1-dead-40-hurt-as-a-blast-rips-central-london}{%
\section{1 Dead, 40 Hurt As a Blast Rips Central
London}\label{1-dead-40-hurt-as-a-blast-rips-central-london}}

By William E. Schmidt

\begin{itemize}
\item
  April 25, 1993
\item
  \begin{itemize}
  \item
  \item
  \item
  \item
  \item
  \end{itemize}
\end{itemize}

\includegraphics{https://s1.nyt.com/timesmachine/pages/1/1993/04/25/578993_360W.png?quality=75\&auto=webp\&disable=upscale}

See the article in its original context from\\
April 25, 1993, Section 1, Page
18\href{https://store.nytimes.com/collections/new-york-times-page-reprints?utm_source=nytimes\&utm_medium=article-page\&utm_campaign=reprints}{Buy
Reprints}

\href{http://timesmachine.nytimes.com/timesmachine/1993/04/25/578993.html}{View
on timesmachine}

TimesMachine is an exclusive benefit for home delivery and digital
subscribers.

About the Archive

This is a digitized version of an article from The Times's print
archive, before the start of online publication in 1996. To preserve
these articles as they originally appeared, The Times does not alter,
edit or update them.

Occasionally the digitization process introduces transcription errors or
other problems; we are continuing to work to improve these archived
versions.

A huge bomb hidden in a parked construction truck shattered the heart of
London's financial district this morning, killing one man, wounding more
than 40 people and raising a cloud of smoke that was visible across much
of the capital.

Detectives at Scotland Yard immediately blamed the Irish Republican
Army, which set off a similar bomb in nearly the same neighborhood just
over a year ago, killing three people and causing \$1.25 billion in
damage.

The body of the dead man was discovered nearly six hours after the
explosion by police searching buildings on Bishopsgate, the busy city
thoroughfare on which the truck believed to contain the bomb was parked.

Because it was a weekend morning, only a handful of office workers and
building security personnel were in the City of London, as the financial
district is officially known. The City is also well-traveled by tourists
and tour buses, drawn by its narrow streets and many old churches.

Late tonight, two car bombs exploded in the city, damaging buildings but
causing no casualties, officials said.

Witnesses said the force of this morning's explosion, which came after
several telephone warnings, sent people crying and screaming through the
streets as glass from the windows of banks and skyscrapers rained down
over an area of several square blocks. Streets were carpeted with broken
shards, and the interiors of some buildings resembled stage sets, the
furniture inside exposed to the open.

Most of the windows on the eastern side of the 52-story National
Westminister Tower were gone, and window blinds fluttered in the spring
wind. The explosion gouged a 15-foot wide crater in the street near the
Hong Kong and Shanghai Banking Corporation building and blew in the
building's lobby.

"It's just damage everywhere," said Nigel Tree, working in the bank.
"All the doors of the lift shafts have been blown in. There's very
little standing apart from the core wall."

One insurance executive said today's damage might exceed losses from the
April 10 explosion of last year. Nicholas Balcombe, the chief executive
of a London insurance broker, estimated that damage from this bombing
would exceed \$1.5 billion. Government to Meet Costs

After last year's explosion, many insurance companies refused to cover
any further losses resulting from terrorist attacks in the City of
London, and the Government moved to create a special fund to help
underwrite such losses. While the measure has not yet been passed into
law by Parliament, the Government said today that it would help meet
costs.

Police this morning cordoned off nearly a square mile of the City of
London, while they searched through damaged buildings. Most of those
wounded were treated for shock or cuts received as a result of flying
glass.

Among the wounded was a policeman who was helping to evacuate the area
at the time the bomb detonated, about 10:25 A.M. The first telephoned
warning, which carried a prearranged I.R.A. code word to indicate it is
not a crank call, was received by the police about an hour earlier.

As a result, the police evacuated worshipers at the city's oldest
synagogue, where the blast later blew out three windows, and telephoned
building security guards in the area to get them to herd occupants away
from windows.

But many had no warning. In the Moorgate Underground station, not far
from the blast site, the concussion from the blast caused panic and
commotion among passengers on a train that had just arrived in the
station. For others, the warnings came too late.

Raymond Fayer, a security guard at the Hong Kong and Shanghai Bank, was
knocked unconscious by the blast, not long after taking a telephone call
warning about the bomb. "It all went black and then the next thing I
looking up and seeing the ceiling down on me," he said from a London
hospital.

Advertisement

\protect\hyperlink{after-bottom}{Continue reading the main story}

\hypertarget{site-index}{%
\subsection{Site Index}\label{site-index}}

\hypertarget{site-information-navigation}{%
\subsection{Site Information
Navigation}\label{site-information-navigation}}

\begin{itemize}
\tightlist
\item
  \href{https://help.nytimes.com/hc/en-us/articles/115014792127-Copyright-notice}{©~2020~The
  New York Times Company}
\end{itemize}

\begin{itemize}
\tightlist
\item
  \href{https://www.nytco.com/}{NYTCo}
\item
  \href{https://help.nytimes.com/hc/en-us/articles/115015385887-Contact-Us}{Contact
  Us}
\item
  \href{https://www.nytco.com/careers/}{Work with us}
\item
  \href{https://nytmediakit.com/}{Advertise}
\item
  \href{http://www.tbrandstudio.com/}{T Brand Studio}
\item
  \href{https://www.nytimes.com/privacy/cookie-policy\#how-do-i-manage-trackers}{Your
  Ad Choices}
\item
  \href{https://www.nytimes.com/privacy}{Privacy}
\item
  \href{https://help.nytimes.com/hc/en-us/articles/115014893428-Terms-of-service}{Terms
  of Service}
\item
  \href{https://help.nytimes.com/hc/en-us/articles/115014893968-Terms-of-sale}{Terms
  of Sale}
\item
  \href{https://spiderbites.nytimes.com}{Site Map}
\item
  \href{https://help.nytimes.com/hc/en-us}{Help}
\item
  \href{https://www.nytimes.com/subscription?campaignId=37WXW}{Subscriptions}
\end{itemize}
