Sections

SEARCH

\protect\hyperlink{site-content}{Skip to
content}\protect\hyperlink{site-index}{Skip to site index}

\href{https://www.nytimes.com/pages/garden/index.html}{Home \& Garden}

\href{https://myaccount.nytimes.com/auth/login?response_type=cookie\&client_id=vi}{}

\href{https://www.nytimes.com/section/todayspaper}{Today's Paper}

\href{/pages/garden/index.html}{Home \& Garden}\textbar{}Pssst!
Serendipity's Chocolate Secret

\href{https://nyti.ms/29dJhXZ}{https://nyti.ms/29dJhXZ}

\begin{itemize}
\item
\item
\item
\item
\item
\end{itemize}

Advertisement

\protect\hyperlink{after-top}{Continue reading the main story}

Supported by

\protect\hyperlink{after-sponsor}{Continue reading the main story}

\hypertarget{pssst-serendipitys-chocolate-secret}{%
\section{Pssst! Serendipity's Chocolate
Secret}\label{pssst-serendipitys-chocolate-secret}}

By \href{https://www.nytimes.com/by/james-barron}{James Barron}

\begin{itemize}
\item
  Dec. 14, 1994
\item
  \begin{itemize}
  \item
  \item
  \item
  \item
  \item
  \end{itemize}
\end{itemize}

\includegraphics{https://s1.nyt.com/timesmachine/pages/1/1994/12/14/105880_360W.png?quality=75\&auto=webp\&disable=upscale}

See the article in its original context from\\
December 14, 1994, Section C, Page
3\href{https://store.nytimes.com/collections/new-york-times-page-reprints?utm_source=nytimes\&utm_medium=article-page\&utm_campaign=reprints}{Buy
Reprints}

\href{http://timesmachine.nytimes.com/timesmachine/1994/12/14/105880.html}{View
on timesmachine}

TimesMachine is an exclusive benefit for home delivery and digital
subscribers.

About the Archive

This is a digitized version of an article from The Times's print
archive, before the start of online publication in 1996. To preserve
these articles as they originally appeared, The Times does not alter,
edit or update them.

Occasionally the digitization process introduces transcription errors or
other problems; we are continuing to work to improve these archived
versions.

RIGHT there on page 152 of the just-released paperback version of "The
Serendipity Cookbook" is the secret recipe. The recipe that was censored
in the hardcover version, which came out four years ago. The recipe the
White House asked for more than 30 years ago but never got.

The recipe is for a chocoholic's fantasy, something called the frozen
hot chocolate blend. It is the essential element of one of the
diet-destroying confections that the whimsical restaurant Serendipity
has been serving beneath its bought-at-a-junkyard Tiffany lamps on the
Upper East Side for 40 years. The secret blend is then ladled over a
dozen imported chocolates, milk and crushed ice to make frozen hot
chocolate.

Serendipity's owner, Stephen Bruce, has a copyright on the recipe and a
trademark on the name of the frozen hot-chocolate blend, and has served
legal papers on restaurateurs who served imitations. So what is the
recipe doing in a cookbook? And a Serendipity cookbook, with Mr. Bruce
as an author?

"We struggled with that," acknowledged Debra Christie, the general
manager of Serendipity for 15 years.

She said it was Mr. Bruce who had suggested the solution that lasted for
5,000 copies in the hardcover version, published by Wynwood Press: "
'Let's put it in, but let's censor it,' " she recalled him saying. "We
all said, 'Wonderful.' It would not be a complete book without it, but
we maintained the mystique."

Never mind that the recipe looked like a declassified F.B.I. dossier,
with CENSORED printed over the ingredients.

But this year, in preparing the paperback, the decision was made by the
new publisher, Citadel Press: no secrets. "We felt that to not list the
ingredients would be doing a disservice to buyers of the book," said Ben
Petrone, a spokesman for Citadel. "That was one of the conditions we
made for doing the book."

Mr. Bruce was resistant at first, Mr. Petrone said, but he added, "As
soon as we convinced him we could sell more books with the ingredients,
I believe he came around."

But even in its complete and unexpurgated form, Mr. Bruce said, the real
frozen hot chocolate could not be duplicated. "It will never come close
to what we serve," he said. "You can't get the same ingredients."

Maybe you don't want to. Serendipity's recipes are almost anachronisms
-\/- high-calorie dishes like dark double-devil mousse and spiced
chicken flambe (in heavy cream sauce) in a low-fat age. The recipe for
the frozen hot chocolate blend calls for 11 kinds of chocolate,
including half-ounce portions of Belgian, French, Danish, Norwegian,
Swedish and German varieties. Many of the types are available at
speciality stores, but a cook would have to make several stops to get
them. Some are unknown even to the city's chocolate specialists.

Mr. Bruce said he had not been hurt by the trend toward lighter eating.
"People come here for an indulgent experience," he said. "We added
certain things to keep fresh with the menu: pasta, lemon chicken,
vegetarian dishes. But it's amazing how a formula created 40 years ago
remains."

The formula started when three young roommates decided to go commercial
with dishes they had been serving their friends at parties: Mr. Bruce,
Calvin L. Holt and Preston (Patch) Caradine. Mr. Bruce, who refuses to
say how old he is, is the sole survivor. Mr. Holt died in 1991 at 66.
Mr. Caradine, who left Serendipity after 15 years, died in July at 69.

Pooling \$500 in 1954, they found a tiny storefront on East 58th Street
-\/- six tables, 15 seats, \$550 a month -\/- and stumbled onto a look
that became so popular that Andy Warhol hired them to decorate his
apartment.

Other people's castoffs became their quirky icons, and quickly became
valuable. Mr. Bruce said they were stuck for lighting until they
discovered a trove of Tiffany lamps in upstate New York. "People
considered them junk," he recalled. "We could consider them at all
because they were only \$10 or \$15 apiece." Now the tinted-glass-mosaic
lamps are valued at six-figure prices. The owners also brought in an
outsize clock from a meat market on Third Avenue and an espresso machine
from Little Italy.

Mr. Caradine came up with the restaurant's name from a crossword-puzzle
clue that referred to Horace Walpole's "Three Princes of Serendip,"
about a threesome with a knack for the unusual. The fledgling
restaurateurs got advice from the chef James Beard. And soon Mr. Bruce
was getting on-the-job training in celebrity sighting: Truman Capote.
Tennessee Williams. Gloria Vanderbilt once ordered 24 pink flower
arrangements that Mr. Bruce had to stay up all night to finish. "Very
pretty, very elegant, very G. V.," he said. There was the time that
Marilyn Monroe ("she always had chicken salad") came in wearing a man's
trench coat. "Suddenly my eyes popped out," Mr. Bruce said. "When she
crossed her legs, there was no skirt." And there was Andy Warhol.
Serendipity was Warhol's first art gallery, selling drawings for
department-store ads that the stores had rejected.

Celebrities still stop by the restaurant and shop, which moved to 225
East 60th Street in 1959 -\/- Cher, Candace Bergen, Tom Cruise. But
somehow, what stands out are frozen hot chocolates that date from a
generation ago. There was the time that Jacqueline Kennedy asked for the
recipe to serve at a White House gala. "We said we'd only do it if we
could make it," Mr. Bruce recalled. "We were declined."

And Mr. Bruce recalls the frozen hot chocolate that he served to Grace
Kelly and Cary Grant after they had finished "To Catch a Thief." Almost
before Mr. Bruce had realized it, the unthinkable happened.

"He walked out without paying the bill," he said. "And at that
particular time, I needed every dollar I could get."

Advertisement

\protect\hyperlink{after-bottom}{Continue reading the main story}

\hypertarget{site-index}{%
\subsection{Site Index}\label{site-index}}

\hypertarget{site-information-navigation}{%
\subsection{Site Information
Navigation}\label{site-information-navigation}}

\begin{itemize}
\tightlist
\item
  \href{https://help.nytimes.com/hc/en-us/articles/115014792127-Copyright-notice}{©~2020~The
  New York Times Company}
\end{itemize}

\begin{itemize}
\tightlist
\item
  \href{https://www.nytco.com/}{NYTCo}
\item
  \href{https://help.nytimes.com/hc/en-us/articles/115015385887-Contact-Us}{Contact
  Us}
\item
  \href{https://www.nytco.com/careers/}{Work with us}
\item
  \href{https://nytmediakit.com/}{Advertise}
\item
  \href{http://www.tbrandstudio.com/}{T Brand Studio}
\item
  \href{https://www.nytimes.com/privacy/cookie-policy\#how-do-i-manage-trackers}{Your
  Ad Choices}
\item
  \href{https://www.nytimes.com/privacy}{Privacy}
\item
  \href{https://help.nytimes.com/hc/en-us/articles/115014893428-Terms-of-service}{Terms
  of Service}
\item
  \href{https://help.nytimes.com/hc/en-us/articles/115014893968-Terms-of-sale}{Terms
  of Sale}
\item
  \href{https://spiderbites.nytimes.com}{Site Map}
\item
  \href{https://help.nytimes.com/hc/en-us}{Help}
\item
  \href{https://www.nytimes.com/subscription?campaignId=37WXW}{Subscriptions}
\end{itemize}
