Sections

SEARCH

\protect\hyperlink{site-content}{Skip to
content}\protect\hyperlink{site-index}{Skip to site index}

\href{https://www.nytimes.com/section/world}{World}

\href{https://myaccount.nytimes.com/auth/login?response_type=cookie\&client_id=vi}{}

\href{https://www.nytimes.com/section/todayspaper}{Today's Paper}

\href{/section/world}{World}\textbar{}Bomb Wounds 100 in London As
I.R.A. Truce Is Said to End

\begin{itemize}
\item
\item
\item
\item
\item
\end{itemize}

Advertisement

\protect\hyperlink{after-top}{Continue reading the main story}

Supported by

\protect\hyperlink{after-sponsor}{Continue reading the main story}

\hypertarget{bomb-wounds-100-in-london-as-ira-truce-is-said-to-end}{%
\section{Bomb Wounds 100 in London As I.R.A. Truce Is Said to
End}\label{bomb-wounds-100-in-london-as-ira-truce-is-said-to-end}}

By \href{https://www.nytimes.com/by/richard-w-stevenson}{Richard W.
Stevenson}

\begin{itemize}
\item
  Feb. 10, 1996
\item
  \begin{itemize}
  \item
  \item
  \item
  \item
  \item
  \end{itemize}
\end{itemize}

See the article in its original context from\\
February 10, 1996, Section 1, Page
1\href{https://store.nytimes.com/collections/new-york-times-page-reprints?utm_source=nytimes\&utm_medium=article-page\&utm_campaign=reprints}{Buy
Reprints}

\href{http://timesmachine.nytimes.com/timesmachine/1996/02/10/033677.html}{View
on timesmachine}

TimesMachine is an exclusive benefit for home delivery and digital
subscribers.

About the Archive

This is a digitized version of an article from The Times's print
archive, before the start of online publication in 1996. To preserve
these articles as they originally appeared, The Times does not alter,
edit or update them.

Occasionally the digitization process introduces transcription errors or
other problems; we are continuing to work to improve these archived
versions.

Just an hour after reports that the Irish Republican Army had abandoned
its 17-month-old cease-fire, a powerful bomb exploded in East London
tonight, wounding about 100 people, shaking skyscrapers and leaving the
Northern Ireland peace effort in tatters.

The blast occurred at 7:01 P.M. in a parking lot beneath a train station
in the Docklands, near the Canary Wharf office complex, badly damaging a
six-story building and sending glass from other buildings flying onto
passers-by.

The police said that 33 people had been treated in local hospitals and
that two were in critical condition. Most of the rest of the injuries
were minor cuts and bruises.

The police said that they had received a warning of a bomb in the area
an hour earlier, and that they were evacuating the area when the bomb
went off.

Earlier a Dublin radio station received a statement, apparently from the
I.R.A., announcing that "the complete cessation of military operations
will end at 6 P.M." because "selfish party political and sectional
interests in the London Parliament have been placed before the rights of
the people of Ireland."

It was not immediately clear whether the statement represented the views
of the I.R.A. leadership. Political analysts in Britain and Ireland said
it was possible that the statement and the bomb were the work of a
hard-line splinter group that was unhappy with the I.R.A.'s role in the
peace effort.

Political leaders in Britain and Ireland were reluctant to pronounce the
peace moves dead. But leaders of the Protestant Unionist parties made
clear that it would be hard for them ever to come to the negotiating
table with Sinn Fein, the I.R.A.'s political arm.

And the bombing appeared likely to strain the already tense relationship
between Britain, which has taken a cautious approach in dealing with the
I.R.A., and Ireland, which had urged starting peace talks with the
organization despite its unwillingness to begin giving up its explosives
and other weapons.

Attempts to negotiate a permanent solution to Northern Ireland's
sectarian conflict had been stalled in recent weeks, and the British
Government by some reports had received intelligence warnings of a
possible return to terrorist violence.

Nonetheless, tonight's rapid-fire events clearly caught the British and
Irish Governments by surprise and shocked residents of London, who had
hoped that the long terrorist campaign in their city was over for good.

The bombing was condemned by Prime Minister John Major of Britain and
his counterpart in Ireland, John Bruton.

President Clinton, who visited Northern Ireland in November and has made
the province's future a foreign policy priority for his Administration,
said he was "deeply concerned" by the situation. A White House spokesman
said the United States was contacting all parties to urge that they
"continue their search for peace."

Gerry Adams, the leader of Sinn Fein, said his reaction was one of
"sadness," but he did not condemn the attack. He said he regretted that
"an unprecedented opportunity for peace had foundered on the refusal of
the British Government and the unionist leaders to enter into dialogue
on substantive negotiation."

But Mr. Adams added, "Sinn Fein's peace strategy remains as the main
function of our party."

In Washington, a White House official said Mr. Adams had phoned Mr.
Clinton's national security adviser, Anthony Lake, before the bomb went
off and simply said he was hearing "disturbing rumors." Mr. Adams did
not explicitly say he knew a bomb would go off, the official said.

The official said the Administration had heard rumors over the last
several months that the peace effort was fraying at the edges, and that
there were internal pressures inside the I.R.A. But the official said
Mr. Adams had given no indication that a bombing or an end to the
cease-fire was imminent when he came to the White House last week.

Mr. Clinton said tonight that "the terrorists who perpetrated today's
attack cannot be allowed to derail the effort to bring peace to the
people of Northern Ireland -\/- a peace they overwhelmingly support."

"For a year and a half, the people of the United Kingdom and Ireland
have enjoyed living in peace, free to go about their daily lives without
the threat of the bomb and the bullet," the President said in a written
statement. "The people want peace. No one and no organization has the
right to deny them that wish."

Members of Congress called for new efforts to convene talks. Senator
Christopher J. Dodd, a Connecticut Democrat who is a member of the
Foreign Relations Committee, said, "This horrible incident should remind
everyone of the critically important need for all parties to come to the
table in good faith to negotiate a lasting peace."

Senator Edward M. Kennedy, a Massachusetts Democrat, called the reported
end of the cease-fire "an extremely ominous development" that "makes
talks more imperative than ever."

Leaders of the Protestant Unionist parties in Northern Ireland said the
reported I.R.A. statement and the bombing proved that they had been
justified in not trusting the I.R.A.'s avowed peaceful intentions.

In September 1994, the I.R.A. called a unilateral halt to its violent
campaign to force Britain from Northern Ireland and unite the province
with the overwhelmingly Catholic Republic of Ireland. The loyalist
paramilitaries that had engaged in tit-for-tat bombings and shootings
with the I.R.A. called a cease-fire of their own soon after.

Since then the British and Irish Governments have been seeking a formula
to allow all parties to the conflict to begin full-scale negotiations on
the province's future.

But progress has been blocked by one main issue: the demand by the
British Government that the I.R.A. begin to turn in its arms before Sinn
Fein would get a place at the negotiating table, and the I.R.A.'s demand
that talks begin before it would consider handing over any weapons.

There has been a growing gulf over the issue between the British
Government and the Irish Government, which has argued that it is
unrealistic to expect the Irish Republican Army in effect to surrender
before full-scale peace negotiations even begin.

The impasse has grown in recent weeks, since an international
commission, appointed by the British and Irish Governments and headed by
former Senator George J. Mitchell, a Maine Democrat, recommended that
Britain ease its position.

Mr. Major in effect rejected the commission's recommendation by
countering with a proposal for elections to an assembly in Northern
Ireland that would serve as a negotiating forum in which elected
representatives of Sinn Fein could take part.

Mr. Major's proposal was rejected by Sinn Fein, which saw the elections
-\/- a step long favored by the Unionist parties -\/- as a sign that Mr.
Major was bowing to pressure from the Protestant majority.

Mr. Major's governing Conservatives have only a slim majority in the
House of Commons, and could end up relying on the Unionist parties to
remain in power. Sinn Fein has long complained that Mr. Major's position
on Northern Ireland was being driven by electoral politics.

Political analysts in both Britain and Ireland said Mr. Major's decision
not to endorse the Mitchell commission's recommendations might have
pushed the I.R.A., or at least hard-liners in the organization, to
decide on a return to violence.

Terrorism experts said I.R.A. members around Belfast appear to remain
generally in favor of the strategy of peaceful negotiation supported by
Mr. Adams. They said there were signs, however, that a more militant
group in the organization, largely based on the Northern Ireland-Ireland
border or in Ireland, had run out of patience with the peace effort.

Mr. Mitchell said on Sunday that he feared that a breakaway faction of
the I.R.A. might return to the tactics of terror.

"I think there is a danger of a fracture within that organization," he
told the British Broadcasting Corporation. "It seems clear not all on
the republican side favor the cease-fire, and the potential for some
elements to take direct and violent action, I think, does remain."

Mr. Adams, however, told American lawmakers a few days ago that "the
I.R.A. was not breaking up," and that there was no internal split over
the use of violent tactics, said Representative Joseph P. Kennedy 2d,
the Massachusetts Democrat.

William O'Donnell, the president of Boston Ireland Ventures, a nonprofit
group in Boston promoting Irish-American trade, said the news was
heartbreaking.

"People in Ireland are really weary of 25 years of going to funerals,"
Mr. O'Donnell said. "You always want to say the glass is half full, but
there's not much in that glass today."

London has always been a favorite target of the I.R.A. The police foiled
another attempt to bomb Canary Wharf more than three years ago, and the
I.R.A. had successfully detonated large car and truck bombs in London's
financial district in the three years before the cease-fires.

Tonight the police were out in force around London, and in Northern
Ireland the police and the military were stepping up security measures
that had been gradually reduced after the cease-fires. There was no
indication whether the Protestant paramilitary organizations would call
off their cease-fire.

"Steps have now been taken to reintroduce such security measures as we
consider prudent, and where necessary this will involve military
support," said Deputy Chief Constable Blair Wallace of the Royal Ulster
Constabulary.

Officials said the steps would probably include the use of armored
vehicles to patrol in Belfast and the use of British troops to protect
police stations.

Most of those injured in the bombing suffered cuts and bruises, and were
released after treatment at hospitals.

The bomb went off in an area just south of the Canary Wharf complex,
where tens of thousands of people work. The area around the blast has
smaller office buildings, pubs and other businesses, as well as
apartment complexes.

The police had cleared a nearby train station before the blast. The area
remained sealed off afterward as ambulances ferried the wounded to
hospitals and police and anti-bomb squads searched for evidence.

Advertisement

\protect\hyperlink{after-bottom}{Continue reading the main story}

\hypertarget{site-index}{%
\subsection{Site Index}\label{site-index}}

\hypertarget{site-information-navigation}{%
\subsection{Site Information
Navigation}\label{site-information-navigation}}

\begin{itemize}
\tightlist
\item
  \href{https://help.nytimes.com/hc/en-us/articles/115014792127-Copyright-notice}{©~2020~The
  New York Times Company}
\end{itemize}

\begin{itemize}
\tightlist
\item
  \href{https://www.nytco.com/}{NYTCo}
\item
  \href{https://help.nytimes.com/hc/en-us/articles/115015385887-Contact-Us}{Contact
  Us}
\item
  \href{https://www.nytco.com/careers/}{Work with us}
\item
  \href{https://nytmediakit.com/}{Advertise}
\item
  \href{http://www.tbrandstudio.com/}{T Brand Studio}
\item
  \href{https://www.nytimes.com/privacy/cookie-policy\#how-do-i-manage-trackers}{Your
  Ad Choices}
\item
  \href{https://www.nytimes.com/privacy}{Privacy}
\item
  \href{https://help.nytimes.com/hc/en-us/articles/115014893428-Terms-of-service}{Terms
  of Service}
\item
  \href{https://help.nytimes.com/hc/en-us/articles/115014893968-Terms-of-sale}{Terms
  of Sale}
\item
  \href{https://spiderbites.nytimes.com}{Site Map}
\item
  \href{https://help.nytimes.com/hc/en-us}{Help}
\item
  \href{https://www.nytimes.com/subscription?campaignId=37WXW}{Subscriptions}
\end{itemize}
