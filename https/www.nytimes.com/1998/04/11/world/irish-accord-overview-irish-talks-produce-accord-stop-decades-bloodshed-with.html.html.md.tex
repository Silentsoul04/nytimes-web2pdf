Sections

SEARCH

\protect\hyperlink{site-content}{Skip to
content}\protect\hyperlink{site-index}{Skip to site index}

\href{https://www.nytimes.com/section/world}{World}

\href{https://myaccount.nytimes.com/auth/login?response_type=cookie\&client_id=vi}{}

\href{https://www.nytimes.com/section/todayspaper}{Today's Paper}

\href{/section/world}{World}\textbar{}AN IRISH ACCORD: THE OVERVIEW;
IRISH TALKS PRODUCE AN ACCORD TO STOP DECADES OF BLOODSHED WITH SHARING
OF ULSTER POWER

\begin{itemize}
\item
\item
\item
\item
\item
\end{itemize}

Advertisement

\protect\hyperlink{after-top}{Continue reading the main story}

Supported by

\protect\hyperlink{after-sponsor}{Continue reading the main story}

AN IRISH ACCORD: THE OVERVIEW

\hypertarget{an-irish-accord-the-overview-irish-talks-produce-an-accord-to-stop-decades-of-bloodshed-with-sharing-of-ulster-power}{%
\section{AN IRISH ACCORD: THE OVERVIEW; IRISH TALKS PRODUCE AN ACCORD TO
STOP DECADES OF BLOODSHED WITH SHARING OF ULSTER
POWER}\label{an-irish-accord-the-overview-irish-talks-produce-an-accord-to-stop-decades-of-bloodshed-with-sharing-of-ulster-power}}

By \href{https://www.nytimes.com/by/warren-hoge}{Warren Hoge}

\begin{itemize}
\item
  April 11, 1998
\item
  \begin{itemize}
  \item
  \item
  \item
  \item
  \item
  \end{itemize}
\end{itemize}

See the article in its original context from\\
April 11, 1998, Section A, Page
1\href{https://store.nytimes.com/collections/new-york-times-page-reprints?utm_source=nytimes\&utm_medium=article-page\&utm_campaign=reprints}{Buy
Reprints}

\href{http://timesmachine.nytimes.com/timesmachine/1998/04/11/454966.html}{View
on timesmachine}

TimesMachine is an exclusive benefit for home delivery and digital
subscribers.

The Northern Ireland peace talks produced a landmark settlement today
that forged concessions from fiercely antagonistic Catholic and
Protestant figures in an effort to settle one of the century's most
enduring conflicts.

A marathon negotiating struggle went 17 hours past its midnight Thursday
deadline and required the personal intervention of Prime Minister Tony
Blair of Britain and Prime Minister Bertie Ahern of Ireland, as well as
last-minute telephone calls from President Clinton.

In an effort that faltered several times during the long last night and
day of dealing and drafting, representatives of eight political parties
finally agreed to a fundamental reshaping of the political institutions
of this tormented province.

Under the agreement, a 67-page framework for power-sharing that sought
to address the fears and demands of the province's two main religious
groups, Protestants and Catholics in Northern Ireland will govern
jointly in a 108-seat National Assembly, which will in turn cooperate
with the Irish Republic in a new North-South Council of lawmakers.

The settlement will now be put to referendums in the Irish Republic and
Northern Ireland on May 22.

It was with as much a spirit of deliverance as one of jubilation that
the delegates and a crush of aides greeted a declaration by the chairman
of the talks, the former United States Senator George J. Mitchell: ''I
am pleased to announce that the two Governments and the political
parties of Northern Ireland have reached agreement.'' {[}Excerpts from
the agreement, page A5.{]}

''It doesn't take courage to shoot a policeman in the back of the head,
or to murder an unarmed taxi driver,'' he said. ''What takes courage is
to compete in the arena of democracy as these men and women are
tonight.''

Mr. Blair said he hoped that the agreement would lift the ''burden'' of
Northern Ireland's tortuous past, and Mr. Ahern voiced the hope that it
would ''exorcise the demons of history.''

Mr. Ahern, who took time off during the week to attend the funeral of
his mother, a known champion of Irish unity, said: ''Today's agreement
is a victory for peace and democratic politics. We have seized the
initiative from the men of violence. Let's not relinquish it, now or
ever.''

Looking ahead to the referendums and the effort required to put today's
plan into place, Mr. Blair said, ''I stress that this is the beginning
of a process of change where people can work together in ways that they
haven't been able to before.'' He said the agreement had ended the
tradition in Northern Ireland of there being no winners without losers.
''We can all win,'' he said. ''Put this agreement into practice, and we
all win.''

Gerry Adams, president of Sinn Fein, the political wing of the Irish
Republican Army, called the announcement ''part of our collective
journey from the failures of the past'' and said he was still committed
to the long-range goal of a united Ireland. He added that he still
worried about the ''huge gap of distrust'' between Northern Ireland's
two religious communities.

David Trimble, leader of the Ulster Unionists, the largest Protestant
party, who has refused to speak directly to Mr. Adams during the months
of talks, said he would continue to snub him until ''he stops that dirty
squalid little terrorist war.''

It was a late objection from Mr. Trimble and his party that nearly
scuttled the negotiations this afternoon and brought a phone call of
reassurance from Mr. Clinton, requested by Mr. Blair, about guarantees
being sought by the Ulster Unionists over the ultimate disarmament of
paramilitary groups.

During the early hours of the morning it had been the representatives of
Sinn Fein who said they were unhappy with so many provisions under
discussion that they were not disposed to sign the agreement.

The settlement, after 22 months of talks, represents a framework for
sharing power that is intended to satisfy Protestant demands for a
reaffirmation of their national identity as British. At the same time it
addresses Catholic desires for a closer relationship with the Republic
of Ireland and Britain's wish to return to Northern Ireland the powers
London assumed in 1972 when the local Stormont legislature was
disbanded.

Under the agreement there will be a new democratically elected
legislature in Belfast and a new ministerial council, the North-South
Council, giving the Governments of Northern Ireland and Ireland joint
responsibilities in areas like tourism, transportation and the
environment. A new consultative council, the Council of the Isles, will
also meet twice a year to bring together ministers from the British and
Irish Parliaments and the three assemblies being created here and in
Scotland and Wales.

The Irish Government will alter the articles in its Constitution that
lay claim to the territory of the North, to spell out that no change can
take place without the consent of the people of Ulster.

The critical issues of police and judicial system reform, the release of
paramilitary prisoners and the dismantling of the vast underground
arsenals of weaponry in the province will be submitted to new
commissions for study with deadlines for recommendations. The time frame
for disarming paramilitary groups is two years, with the first steps to
begin in two months.

The National Assembly will have a prime minister, called a first
minister, and it is likely to be Mr. Trimble, whose party is the
province's biggest vote getter. The deputy first minister will almost
certainly be John Hume, head of the mainstream Catholic Social and
Democratic Labor Party and the original architect of the process that
produced the agreement.

The accord represented the most significant and comprehensive step ever
taken to try to end religious hatreds going back 300 years.

For the last 30 years, Northern Ireland has known almost perpetual
violence, interrupted occasionally by cease-fires greeted with relief
and outbursts of joy, only to end abruptly in despair, grief and
recriminations. It has become so wearyingly familiar to residents of
this conflicted place that it is referred to simply as ''The Troubles.''

Protestant leaders must now try to assure their followers that the
agreement does not represent the beginning of the integration of the
British province into Ireland, and that the new cross-border council is
not the embryo of an all-Ireland Government. Catholic leaders must
persuade their constituents that the pact does not represent abandonment
of the long-term goal of union with the Republic, and does not
constitute ratification of the island's permanent partitioning.

Catholics date their resentment of the dominant Protestants to the
British development of farming estates here in the 17th century that
deprived the Irish of some of their best land and drove them into
subservience. When Protestant power came under serious challenge three
decades ago, loyalist paramilitary groups arose to combat the I.R.A.,
and the province entered its bloodiest period. Since 1969, 3,248 people,
by official count, have died in bombings, shootings and massacres.

Territorial tensions date from the 1922 division of Ireland into what is
now the 26 counties of the predominantly Catholic Republic and the six
counties of predominantly Protestant Northern Ireland, which is a
province of Britain.

Taking part in the talks were 8 of Northern Ireland's 10 political
parties, some of them admitted only after paramilitary forces that they
represented adopted cease-fires. Two groups, the Ulster Democratic
Party, which represents a Protestant force, and Sinn Fein, were
suspended from the talks for weeks because of evidence that some of
their armed followers had been involved in sectarian killings after
Christmas. Both parties were later readmitted.

Among the men around the table were murderers and bombers who had
emerged from prison with a commitment to peace. And sentiment aside, the
paramilitary groups had also made the tactical decision that violence
would never secure their goals, a shared conviction that gave these
talks a chance for success that fitful past attempts had lacked.

The talks began 22 months ago and moved in a desultory and halting
manner until Mr. Blair became Prime Minister last May with a Labor
majority of 179 seats. The huge victory gave him independence and
flexibility that a slim majority had denied his predecessor, John Major.
Mr. Blair decided to apply the strength of his position to Northern
Ireland, Britain's longest running and most intractable problem.

His first trip out of London as Prime Minister, during only his second
week in office, was to Belfast, and his first major speech was a warning
to Sinn Fein that if the I.R.A. did not resume its cease-fire the peace
talks would move forward without it.

The I.R.A. declared its cease-fire in July, and by September Mr. Adams
and his chief negotiator, Martin McGuinness, led a Sinn Fein delegation
into the Stormont Castle Buildings to take their seats at the table.
This winter they were invited to 10 Downing Street, becoming the first
Irish republicans to cross that threshold since Michael Collins went
there to see Lloyd George in 1922.

Mr. Blair also gave Mr. Trimble unprecedented access, meeting with him
continually at the Prime Minister's official residence to shore up the
Unionist's standing in his community. The two parties that did not
accept the invitation to join the talks are both hard-line Protestant
groups resentful of Mr. Trimble's position. One of the leaders, the Rev.
Ian Paisley, called the settlement a ''time bomb;'' the other, Robert
McCartney, said it would cause civil war.

Advertisement

\protect\hyperlink{after-bottom}{Continue reading the main story}

\hypertarget{site-index}{%
\subsection{Site Index}\label{site-index}}

\hypertarget{site-information-navigation}{%
\subsection{Site Information
Navigation}\label{site-information-navigation}}

\begin{itemize}
\tightlist
\item
  \href{https://help.nytimes.com/hc/en-us/articles/115014792127-Copyright-notice}{©~2020~The
  New York Times Company}
\end{itemize}

\begin{itemize}
\tightlist
\item
  \href{https://www.nytco.com/}{NYTCo}
\item
  \href{https://help.nytimes.com/hc/en-us/articles/115015385887-Contact-Us}{Contact
  Us}
\item
  \href{https://www.nytco.com/careers/}{Work with us}
\item
  \href{https://nytmediakit.com/}{Advertise}
\item
  \href{http://www.tbrandstudio.com/}{T Brand Studio}
\item
  \href{https://www.nytimes.com/privacy/cookie-policy\#how-do-i-manage-trackers}{Your
  Ad Choices}
\item
  \href{https://www.nytimes.com/privacy}{Privacy}
\item
  \href{https://help.nytimes.com/hc/en-us/articles/115014893428-Terms-of-service}{Terms
  of Service}
\item
  \href{https://help.nytimes.com/hc/en-us/articles/115014893968-Terms-of-sale}{Terms
  of Sale}
\item
  \href{https://spiderbites.nytimes.com}{Site Map}
\item
  \href{https://help.nytimes.com/hc/en-us}{Help}
\item
  \href{https://www.nytimes.com/subscription?campaignId=37WXW}{Subscriptions}
\end{itemize}
