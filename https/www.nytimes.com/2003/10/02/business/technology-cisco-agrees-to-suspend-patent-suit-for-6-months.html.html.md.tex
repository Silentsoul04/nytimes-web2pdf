Sections

SEARCH

\protect\hyperlink{site-content}{Skip to
content}\protect\hyperlink{site-index}{Skip to site index}

\href{https://www.nytimes.com/section/business}{Business}

\href{https://myaccount.nytimes.com/auth/login?response_type=cookie\&client_id=vi}{}

\href{https://www.nytimes.com/section/todayspaper}{Today's Paper}

\href{/section/business}{Business}\textbar{}TECHNOLOGY; Cisco Agrees To
Suspend Patent Suit For 6 Months

\begin{itemize}
\item
\item
\item
\item
\item
\end{itemize}

Advertisement

\protect\hyperlink{after-top}{Continue reading the main story}

Supported by

\protect\hyperlink{after-sponsor}{Continue reading the main story}

TECHNOLOGY

\hypertarget{technology-cisco-agrees-to-suspend-patent-suit-for-6-months}{%
\section{TECHNOLOGY; Cisco Agrees To Suspend Patent Suit For 6
Months}\label{technology-cisco-agrees-to-suspend-patent-suit-for-6-months}}

By Laurie J. Flynn

\begin{itemize}
\item
  Oct. 2, 2003
\item
  \begin{itemize}
  \item
  \item
  \item
  \item
  \item
  \end{itemize}
\end{itemize}

Cisco Systems Inc. said on Wednesday that it had agreed to suspend its
patent infringement lawsuit against Huawei Technologies, a Chinese maker
of telecommunications and network equipment.

Cisco sued Huawei in Federal District Court in Texas in January claiming
that the Chinese company infringed numerous patents held by Cisco and
illegally copied its software source code and documentation. On
Wednesday Cisco and Huawei announced an agreement to suspend the lawsuit
for six months.

As part of the stay, Huawei said it would stop selling the disputed
products and agreed to make modifications to its router and switch
products that Cisco asserts violates its patents.

Huawei's modifications will be reviewed by independent experts over that
period. Both companies said they expected the agreement to lead to an
end to the legal dispute.

The agreement could also significantly benefit 3Com, the Cisco
competitor that entered into a joint venture with Huawei earlier this
year. 3Com intervened in the lawsuit and has also agreed to the stay.

''As part of the agreement, Huawei has stopped selling the products at
issue in Cisco's lawsuit and will only offer for sale new modified
products on a worldwide basis,'' a Cisco spokeswoman, Penny Bruce, said.

Ms. Bruce said the suit had focused on four areas in Huawei's products
-\/- the software interface that controls the networking equipment, user
manuals, help screens and some source code -\/- and that Huawei has made
modifications in each of those areas. Other terms of the agreement, like
any damages Huawei will pay Cisco, will not be made public, the
companies said.

For Cisco, the lawsuit against Huawei has already had its intended
result: slowing Huawei's move into the American and European network
equipment markets, said Hasan Imam, an analyst at Thomas Weisel
Partners. And because the stay permits Cisco to press forward with the
lawsuit if necessary, Huawei will surely tread carefully in Cisco's
chief markets in the coming year, he added.

3Com stands to gain at least as much as Cisco and Huawei from a legal
settlement. In March, 3Com and Huawei formed a joint venture to develop
and sell telecommunications products under the 3Com brand in all markets
outside of Japan and China, while selling products in those two
countries under the 3Com-Huawei brand. 3Com, which currently owns 49
percent of the venture, has contributed \$160 million, in an effort to
lower its costs and offer an expanded product line.

Removing the uncertainty created by the Cisco lawsuit could give 3Com a
much needed lift. In its request to intervene in the Cisco suit, 3Com
asked the court to clarify that any new products that 3Com will ship
from its joint venture with Huawei would not be subject to Cisco's
intellectual property claims. Shares of 3Com climbed 54 cents, or 9
percent Wednesday to \$6.45. Cisco shares rose 61 cents, or 3 percent,
to \$20.20.

Bruce Claflin, chief executive of 3Com, said the companies were awaiting
regulatory approval for the joint venture, which they expect to receive
by the end of November. In the meantime, 3Com continues to sell Huawei
products under the 3Com brand through an original equipment manufacturer
arrangement.

3Com, now based in Marlborough, Mass., was once a giant in networking
technology and a huge beneficiary of the networking equipment boom of
the 1990's. Since then, however, Cisco's growth has outpaced 3Com's.
3Com is a fraction of the size it once was, employing 3,400 people, down
from 10,000 only three years ago.

Advertisement

\protect\hyperlink{after-bottom}{Continue reading the main story}

\hypertarget{site-index}{%
\subsection{Site Index}\label{site-index}}

\hypertarget{site-information-navigation}{%
\subsection{Site Information
Navigation}\label{site-information-navigation}}

\begin{itemize}
\tightlist
\item
  \href{https://help.nytimes.com/hc/en-us/articles/115014792127-Copyright-notice}{©~2020~The
  New York Times Company}
\end{itemize}

\begin{itemize}
\tightlist
\item
  \href{https://www.nytco.com/}{NYTCo}
\item
  \href{https://help.nytimes.com/hc/en-us/articles/115015385887-Contact-Us}{Contact
  Us}
\item
  \href{https://www.nytco.com/careers/}{Work with us}
\item
  \href{https://nytmediakit.com/}{Advertise}
\item
  \href{http://www.tbrandstudio.com/}{T Brand Studio}
\item
  \href{https://www.nytimes.com/privacy/cookie-policy\#how-do-i-manage-trackers}{Your
  Ad Choices}
\item
  \href{https://www.nytimes.com/privacy}{Privacy}
\item
  \href{https://help.nytimes.com/hc/en-us/articles/115014893428-Terms-of-service}{Terms
  of Service}
\item
  \href{https://help.nytimes.com/hc/en-us/articles/115014893968-Terms-of-sale}{Terms
  of Sale}
\item
  \href{https://spiderbites.nytimes.com}{Site Map}
\item
  \href{https://help.nytimes.com/hc/en-us}{Help}
\item
  \href{https://www.nytimes.com/subscription?campaignId=37WXW}{Subscriptions}
\end{itemize}
