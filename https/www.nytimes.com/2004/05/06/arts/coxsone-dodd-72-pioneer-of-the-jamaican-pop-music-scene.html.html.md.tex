Sections

SEARCH

\protect\hyperlink{site-content}{Skip to
content}\protect\hyperlink{site-index}{Skip to site index}

\href{https://www.nytimes.com/section/arts}{Arts}

\href{https://myaccount.nytimes.com/auth/login?response_type=cookie\&client_id=vi}{}

\href{https://www.nytimes.com/section/todayspaper}{Today's Paper}

\href{/section/arts}{Arts}\textbar{}Coxsone Dodd, 72, Pioneer of the
Jamaican Pop Music Scene

\begin{itemize}
\item
\item
\item
\item
\item
\end{itemize}

Advertisement

\protect\hyperlink{after-top}{Continue reading the main story}

Supported by

\protect\hyperlink{after-sponsor}{Continue reading the main story}

\hypertarget{coxsone-dodd-72-pioneer-of-the-jamaican-pop-music-scene}{%
\section{Coxsone Dodd, 72, Pioneer of the Jamaican Pop Music
Scene}\label{coxsone-dodd-72-pioneer-of-the-jamaican-pop-music-scene}}

By \href{https://www.nytimes.com/by/kelefa-sanneh}{Kelefa Sanneh}

\begin{itemize}
\item
  May 6, 2004
\item
  \begin{itemize}
  \item
  \item
  \item
  \item
  \item
  \end{itemize}
\end{itemize}

Coxsone Dodd, the record producer and entrepreneur who helped invent the
Jamaican music industry, died on Tuesday night at his studio in
Kingston. He was 72.

The cause was a heart attack, said his daughter Carol Dodd.

Mr. Dodd was best known as the force behind Studio One, a record label
he started in 1963; in the years that followed, Studio One released some
of the most influential and enduring Jamaican records of all time. His
popular tracks were endlessly recycled and rerecorded, often without his
knowledge or permission, in a musical tradition built on borrowing and
collaborating.

Mr. Dodd ran a record shop on Fulton Street in Brooklyn, Coxsone's Music
City. But he maintained his studio in Kingston, on a street known until
recently as Brentford Road. During a ceremony held last Friday,
Brentford Road was renamed Studio One Boulevard.

Two days before the ceremony, Mr. Dodd told The Jamaica Observer, ''It
is a wonderful tribute to my contribution to the industry and my years
in the business and it shows that my work is highly appreciated.''

Clement Dodd was born in Kingston and he began his career in the
mid-1950's when he set up his own sound system. Sir Coxsone's Downbeat,
as it was called, was his entry in the competition among other Jamaican
sound systems to see who had the loudest speakers, who could get the
best records and who could attract the most revelers.

He was among the first to realize that instead of importing American R
\& B records, it might be more profitable to produce some Jamaican
originals; soon, Jamaican records were outselling American imports.

At one point, Mr. Dodd was running no fewer than five record labels,
including Studio One, and he assembled a remarkable roster of talent
that included the Wailers, Bob Marley's first group, who released their
hit ''Simmer Down'' on Studio One in 1963.

Soon ska, the sweet and up-tempo Jamaican style that dominated the early
1960's, gave way to the styles called rocksteady and then reggae, each
slower and tougher than its predecessor.

Mr. Dodd kept pace, thanks in large part to session musicians like the
keyboardist Jackie Mittoo and the bassist Leroy Sibbles.

In the late 1960's, Studio One created a series of rhythm tracks, or
''riddims,'' that would serve as the foundations of songs for decades to
come. A 1967 instrumental track called ''Real Rock,'' for example,
quickly came to seem like part of reggae's DNA, as successive
generations of singers and producers reworked the track.

Mr. Dodd's daughter Carol remembers that the ubiquity of Studio One
tracks like ''Real Rock'' was a mixed blessing for her father, who
wasn't always compensated, or even acknowledged.

Even as it made him proud, she said, he was concerned that he wasn't
given credit.

In addition to his daughter, Mr. Dodd is survived by six other children
and by his wife, Norma Dodd.

The reggae historian Rob Kenner, editor at large for Vibe magazine,
compared Studio One to pioneering American labels like Stax Records.
''The Studio One sound is kind of like Stax,'' he said. ''It never gets
exhausted.''

Mr. Dodd never fully embraced dance-hall reggae, the computerized,
heavily percussive, sometimes-foul-mouthed style that has ruled reggae
since the early 1980's.

But he kept working, dividing his time between Kingston and Brooklyn
while working on the Sisyphean task of figuring out exactly who owned
the rights to which records. Just as Mr. Dodd claimed that lots of
latter-day producers used his music without permission, some of the
musicians who worked for him claimed that they had not been fairly
compensated.

But Chris Wilson of Heartbeat Records, who collaborated with Mr. Dodd on
a series of releases and reissues, notes that Mr. Dodd was, above all,
pleased to see that his music had stayed so fresh.

Mr. Wilson said, ''He was kind of amused by the fact that some of his
songs are 25 or 30 years old and people were still, for the umpteenth
time, rerecording them.''

Advertisement

\protect\hyperlink{after-bottom}{Continue reading the main story}

\hypertarget{site-index}{%
\subsection{Site Index}\label{site-index}}

\hypertarget{site-information-navigation}{%
\subsection{Site Information
Navigation}\label{site-information-navigation}}

\begin{itemize}
\tightlist
\item
  \href{https://help.nytimes.com/hc/en-us/articles/115014792127-Copyright-notice}{©~2020~The
  New York Times Company}
\end{itemize}

\begin{itemize}
\tightlist
\item
  \href{https://www.nytco.com/}{NYTCo}
\item
  \href{https://help.nytimes.com/hc/en-us/articles/115015385887-Contact-Us}{Contact
  Us}
\item
  \href{https://www.nytco.com/careers/}{Work with us}
\item
  \href{https://nytmediakit.com/}{Advertise}
\item
  \href{http://www.tbrandstudio.com/}{T Brand Studio}
\item
  \href{https://www.nytimes.com/privacy/cookie-policy\#how-do-i-manage-trackers}{Your
  Ad Choices}
\item
  \href{https://www.nytimes.com/privacy}{Privacy}
\item
  \href{https://help.nytimes.com/hc/en-us/articles/115014893428-Terms-of-service}{Terms
  of Service}
\item
  \href{https://help.nytimes.com/hc/en-us/articles/115014893968-Terms-of-sale}{Terms
  of Sale}
\item
  \href{https://spiderbites.nytimes.com}{Site Map}
\item
  \href{https://help.nytimes.com/hc/en-us}{Help}
\item
  \href{https://www.nytimes.com/subscription?campaignId=37WXW}{Subscriptions}
\end{itemize}
