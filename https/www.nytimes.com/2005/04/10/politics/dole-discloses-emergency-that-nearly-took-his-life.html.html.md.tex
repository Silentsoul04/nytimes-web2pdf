Sections

SEARCH

\protect\hyperlink{site-content}{Skip to
content}\protect\hyperlink{site-index}{Skip to site index}

\href{https://www.nytimes.com/section/books}{Books}

\href{https://myaccount.nytimes.com/auth/login?response_type=cookie\&client_id=vi}{}

\href{https://www.nytimes.com/section/todayspaper}{Today's Paper}

\href{/section/books}{Books}\textbar{}Dole Discloses Emergency That
Nearly Took His Life

\begin{itemize}
\item
\item
\item
\item
\item
\end{itemize}

Advertisement

\protect\hyperlink{after-top}{Continue reading the main story}

Supported by

\protect\hyperlink{after-sponsor}{Continue reading the main story}

\hypertarget{dole-discloses-emergency-that-nearly-took-his-life}{%
\section{Dole Discloses Emergency That Nearly Took His
Life}\label{dole-discloses-emergency-that-nearly-took-his-life}}

By \href{https://www.nytimes.com/by/katharine-q-seelye}{Katharine Q.
Seelye}

\begin{itemize}
\item
  April 10, 2005
\item
  \begin{itemize}
  \item
  \item
  \item
  \item
  \item
  \end{itemize}
\end{itemize}

Former Senator Bob Dole, whose life was shaped by a devastating war
injury, suffered a medical emergency earlier this year that doctors told
him nearly claimed his life.

Mr. Dole disclosed details of the episode for the first time in an
epilogue to his new memoir, "One Soldier's Story," which was published
by HarperCollins and is to be released this week.

"I was on the bedroom floor, with blood streaming from my left arm and
right eye, and a sharp pain engulfing my left arm -\/- the better one,"
Mr. Dole writes. Mr. Dole, 81, the former Senate majority leader and the
Republican presidential nominee in 1996, said that the accident occurred
on Jan. 11 at his Watergate apartment in Washington as he was picking up
a suitcase and lost his balance and fell. He said that he had undergone
hip replacement surgery in New York the month before.

His doctors told him that the blood thinner he was taking after the hip
surgery had caused internal bleeding and apparently led to his fall.

Mr. Dole writes that his housekeeper and doorman had to lift him off the
floor of his apartment. His driver then took him to Walter Reed Army
Medical Center in Washington, where he underwent surgery and received
stitches on his right eye.

He returned home but later that night he did not feel well, and an Army
ambulance was dispatched to take him back to the hospital. Doctors found
multiple problems, including internal bleeding in his head.

Mr. Dole said that he was so ill that apart from everything else his
doctors did to save his life, they told him that "a higher power
intervened to halt the bleeding and avert fatal complications."

He spent 40 days at Walter Reed, which he said conjured memories of his
recuperation decades ago at other Army hospitals after his war injuries.

"I'm back where I started, 60 years after Hill 913," he writes in his
book, referring to the spot where a mortar round or an exploding shell
or machine-gun fire -\/- he never knew which -\/- had felled him in
1945.

Mr. Dole nearly died during a battle in Italy on April 14, 1945, when he
sustained injuries to his spinal cord and collarbone on that hill. He
ended up in a body cast from his neck to his hip, unable to feed or care
for himself. His right arm and hand were paralyzed.

He spent 39 months recuperating, a grueling process that he has said
altered his life and became a touchstone for him as a politician. Every
April 14, Mr. Dole speaks on behalf of the disabled. This year, he plans
to appear at the Smithsonian Institution to discuss his experiences and
his memoir.

Mr. Dole wrote in his memoir that he found striking similarities between
his debilitation after World War II and his hospitalization in January.

"I'm not paralyzed, but my left arm and hand are useless," he says of
the episode in January. "With both arms out of commission, I must rely
on others to feed me. I can't scratch my nose or go to the bathroom
without assistance. It is humiliating."

An aide, Mike Marshall, said that Mr. Dole still goes to Walter Reed for
therapy for his left shoulder several times a week, and his doctors have
said that he may not regain the full use of his left arm.

"He's still recovering," Mr. Marshall said. "He is not 100 percent, but
he is on the mend." He said Mr. Dole shows up every day at Alston \&
Bird, the Washington lobbying firm where he is special counsel.

While the story of Mr. Dole's war wounds and recovery has been written
elsewhere, he was prompted to write this memoir after one of his sisters
found nearly 300 letters between him and his family from World War II.
He has also written memoirs about his political career and his marriage
to Elizabeth Dole. He writes in this memoir that he hopes it will help
others understand the pain and trauma that soldiers go through after
being wounded.

Mr. Dole begins a book tour this week, with an appearance on "Meet the
Press" today and a book signing in New York on Wednesday.

Advertisement

\protect\hyperlink{after-bottom}{Continue reading the main story}

\hypertarget{site-index}{%
\subsection{Site Index}\label{site-index}}

\hypertarget{site-information-navigation}{%
\subsection{Site Information
Navigation}\label{site-information-navigation}}

\begin{itemize}
\tightlist
\item
  \href{https://help.nytimes.com/hc/en-us/articles/115014792127-Copyright-notice}{©~2020~The
  New York Times Company}
\end{itemize}

\begin{itemize}
\tightlist
\item
  \href{https://www.nytco.com/}{NYTCo}
\item
  \href{https://help.nytimes.com/hc/en-us/articles/115015385887-Contact-Us}{Contact
  Us}
\item
  \href{https://www.nytco.com/careers/}{Work with us}
\item
  \href{https://nytmediakit.com/}{Advertise}
\item
  \href{http://www.tbrandstudio.com/}{T Brand Studio}
\item
  \href{https://www.nytimes.com/privacy/cookie-policy\#how-do-i-manage-trackers}{Your
  Ad Choices}
\item
  \href{https://www.nytimes.com/privacy}{Privacy}
\item
  \href{https://help.nytimes.com/hc/en-us/articles/115014893428-Terms-of-service}{Terms
  of Service}
\item
  \href{https://help.nytimes.com/hc/en-us/articles/115014893968-Terms-of-sale}{Terms
  of Sale}
\item
  \href{https://spiderbites.nytimes.com}{Site Map}
\item
  \href{https://help.nytimes.com/hc/en-us}{Help}
\item
  \href{https://www.nytimes.com/subscription?campaignId=37WXW}{Subscriptions}
\end{itemize}
