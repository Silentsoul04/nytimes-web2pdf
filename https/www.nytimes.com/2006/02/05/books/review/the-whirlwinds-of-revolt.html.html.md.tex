Sections

SEARCH

\protect\hyperlink{site-content}{Skip to
content}\protect\hyperlink{site-index}{Skip to site index}

\href{https://www.nytimes.com/section/books}{Books}

\href{https://myaccount.nytimes.com/auth/login?response_type=cookie\&client_id=vi}{}

\href{https://www.nytimes.com/section/todayspaper}{Today's Paper}

\href{/section/books}{Books}\textbar{}The Whirlwinds of Revolt

\begin{itemize}
\item
\item
\item
\item
\item
\end{itemize}

Advertisement

\protect\hyperlink{after-top}{Continue reading the main story}

Supported by

\protect\hyperlink{after-sponsor}{Continue reading the main story}

'At Canaan's Edge: America in the King Years, 1965-68,' By Taylor Branch

\hypertarget{the-whirlwinds-of-revolt}{%
\section{The Whirlwinds of Revolt}\label{the-whirlwinds-of-revolt}}

By \href{https://www.nytimes.com/by/anthony-lewis}{Anthony Lewis}

\begin{itemize}
\item
  Feb. 5, 2006
\item
  \begin{itemize}
  \item
  \item
  \item
  \item
  \item
  \end{itemize}
\end{itemize}

AT CANAAN'S EDGE

America in the King

Years, 1965-68.

By Taylor Branch.

Illustrated. 1,039 pp.

Simon \& Schuster. \$35.

We have had nothing like it in this country in living memory: a
commanding moral voice, attached to no political party or public office,
that moved governments and changed social institutions. That was Martin
Luther King Jr.

He was despised by many. His ideas were sometimes rejected. He failed as
well as succeeded. But he would not retreat from attacking what he came
to believe were the three great afflictions of mankind: racism, war and
poverty. In little more than a dozen years -\/- from Dec. 5, 1955, when
he set the Montgomery bus boycott on its way, to April 4, 1968, when he
was murdered -\/- he changed the face of America.

This is the last of three volumes in which Taylor Branch chronicles
those years. It is a thrilling book, marvelous in both its breadth and
its detail. There is drama in every paragraph. Every factual statement
is backed up in 200 pages of endnotes.

"America in the King Years," Branch's running title for the trilogy, is
not a mere conceit, a fancy way of describing a biography. It is not a
biography of Dr. King. It is a picture of the country and the times as
he intersected with them.

What a different country it was. I lived through those times, but "At
Canaan's Edge" made me realize that I did not remember how different. It
was before the revolution in women's roles, for example, as Branch tells
us in a couple of quick sketches. Southerners had added a ban on sex
discrimination to the Civil Rights Act of 1964 as a way to mock the
bill, and at first it was widely treated as a joke. A Page 1 article in
The New York Times in 1965 raised the question whether executives must
let a "dizzy blonde" drive a tugboat or pitch for the Mets. In 1966 the
Equal Employment Opportunity Commission wondered, in a newsletter,
whether an employer could be penalized for refusing to hire "a woman as
a dog warden."

But of course it is the virulence of Southern racism at that time that
is most striking. This was only 40 years ago, after the passage of the
1964 act, but racist violence and murder were still widespread in the
Deep South. Everyone knew who the killers were, but juries would not
convict -\/- all-white juries. The openness of the violence was
staggering. When Viola Liuzzo, a white woman, came down from Michigan to
Selma, Ala., to help in the protest movement, a Ku Klux Klan gang pulled
up alongside the car she was driving and shot her dead.

Branch has been working on these books for more than 20 years, exploring
endless materials: newspapers, audiotapes, reports, books, personal
memories. He has an incredible command of it all, bringing history to
life with a few sentences here, extended chapters there on something
like the march from Selma to Montgomery. I can pick out only a few
themes to indicate the scope of his work.

Selma was about a basic right explicitly guaranteed by the Constitution,
the right to vote without discrimination. In Alabama, Mississippi and
large parts of other states in the Deep South, the right was a myth for
blacks. They were threatened, abused, even murdered if they tried to
register or vote; they often lost their homes or their jobs. Armed white
mobs menaced them.

It was in the face of those tactics that King decided to lead a march
from Selma to Montgomery as a protest for the vote. At the first attempt
marchers were brutalized, the march turned back. But they persisted.
Branch, usually given to understatement, lets himself go and speaks of
"yearnings and exertions toward freedom seldom matched since Valley
Forge."

Before a second attempt could be made to march to Montgomery, a
difficulty intervened. Judge Frank M. Johnson enjoined the march because
of likely violence. Johnson was a highly respected federal judge who had
made many decisions in favor of civil rights. Justice Department
officials pleaded with King not to violate the order lest he sacrifice
the movement's reliance on law and the Constitution. But the protesters,
many of them, did not want to give way. King did not say what he would
do. The march began. He led it onto the Pettus Bridge at the edge of
Selma, faced 500 state troopers -\/- and suddenly turned and led the
marchers back into Selma. He had made the point and desisted, obeying
the law.

There followed a remarkable episode. Judge Johnson was now asked to let
the march go forward and enjoin interference with it. But in a telephone
conversation with the United States attorney general, Nicholas deB.
Katzenbach, he said he would not do so unless the federal government
undertook to protect the marchers. And he wanted that assurance from the
president, he said. Katzenbach gave him the assurance. Lyndon B. Johnson
called the Alabama National Guard into federal service and sent regular
Army detachments. On their third try, the marchers made it to
Montgomery.

King believed that if Americans outside the South were aware of its
brutal racism -\/- as few then were -\/- they would want to end it. The
violent response to nonviolent protest made the brutality plain. What
Americans read in newspapers and saw on television shocked them, and
jump-started the political process. Meaningful civil rights legislation
made it past Senate filibusters at last.

It was a crucial part of King's thinking to engage the president. As
Robert Caro has demonstrated in his biography, Lyndon Johnson had shown
streaks of racism in his life. But fundamentally he was for equal
rights, and he seized the opportunity presented by the King campaign and
the ugly Southern response. In a speech to the nation on March 15, 1965,
he memorably adopted the words of the civil rights movement: "It's all
of us who must overcome the crippling legacy of bigotry and injustice.
And -\/- we -\/- shall -\/- overcome."

L.B.J. is a second object of Branch's penetrating gaze in this volume:
not just what he did on civil rights but his whole whirlwind of
activity. Here he is on the telephone with Attorney General Katzenbach
in Alabama, warning him not to smoke too much during late-night vigils.
On one day in 1965 he takes a phone call from Drew Pearson, the
columnist, and lectures him for 15 minutes about Vietnam. He receives
the British foreign secretary, Michael Stewart, and a delegation,
talking long past the scheduled time and telling them -\/- to their
confusion -\/- "Sometimes I just get all hunkered up like a jackass in a
hailstorm." He has a conference call with House leaders about the
legislation to establish Medicare. He gets a telephone report from
Selma.

FOR Johnson, race and Vietnam were preoccupations in tandem. In the same
month as the march from Selma to Montgomery, March 1965, the first
American combat units went ashore at Da Nang. King had had a good
relationship with the president, but it broke down over the issue that
Johnson rightly feared would overwhelm his reputation on social justice.

Branch's picture of Dr. King on Vietnam is of a man coming slowly,
reluctantly, but irresistibly to embrace the issue -\/- against the
advice of many supporters. Finally, at Riverside Church in New York on
April 4, 1967, he called for the United States to "set a date that we
will remove all foreign troops from Vietnam in accordance with the 1954
Geneva Agreement."

The Riverside speech drew heavy criticism. John Roche, a Brandeis
University professor who was then on the White House staff, said King
had "thrown in with the Commies." He told the president that King was
"inordinately ambitious and quite stupid (a bad combination)." A
Washington Post editorial said, "Many who have listened to him with
respect will never again accord him the same confidence." But King did
not give way. He told a church audience that the press had been "so
noble in its praise" when he preached nonviolence toward white
oppressors but inconsistently "will curse you and damn you when you say
be nonviolent toward little brown Vietnamese children."

Racism in America was not -\/- and is not -\/- confined to the South.
Branch reminds us of that in small ways and large. In 1965, he notes,
Mary Travers of the trio Peter, Paul and Mary kissed Harry Belafonte on
the cheek at a rally. CBS television, which was showing the rally, was
besieged by protesting callers, and took the rally off the air for 90
minutes. In the border state of Kentucky, the famous basketball coach
Adolph Rupp kept his University of Kentucky team all white. He
complained of calls from the university president, "That son of a bitch
wants me to get some niggers in here." A little-noted team from Texas
Western, with five black players starting, upset Kentucky in the 1966
championship game -\/- a story told just now in the movie "Glory Road."
Only slowly, after that, did the bar on black athletes break down in the
South. Many people watching college sports on television today would not
have dreamed that such a policy ever existed.

Chicago dramatized the reality of antiblack feelings in the North.
Marches organized by King to protest segregated housing and unequal
government benefits were met with mob taunts and rocks. "Burn them like
Jews!" one white group shouted at the marchers. Branch concludes that
"the violence against Northern demonstrations cracked a beguiling,
cultivated conceit that bigotry was the province of backward
Southerners."

The most chilling passages in this book, for me, are about J. Edgar
Hoover, the F.B.I. director. His hatred of King was not a secret. But
Branch shows how far it went -\/- beyond extremity to morbid depravity.

Hoover instructed all in the bureau not to warn King of death threats.
He told President Johnson that any requests for federal protection of
King would come from subversives, and that King was "an instrument in
the hands of subversive forces seeking to undermine our Nation." He
listed King as a prominent target in an order to all F.B.I. offices "to
expose, disrupt, misdirect, discredit or otherwise neutralize the
activities of black nationalist hate-type organizations." There was no
basis in fact for the calumnies. The charge of subversion hung on the
dubious thread of an allegation that Stanley Levison, an adviser to
King, was a Communist agent -\/- an allegation never shown to have any
convincing support.

The low point in the Hoover story may have been his performance on the
killing of Viola Liuzzo. He tried to conceal the fact that one of the
Klansmen who shot at her was an F.B.I. informant, Gary Thomas Rowe -\/-
and lied to President Johnson about it. He urged the president not to
speak with the Liuzzo family, telling Johnson that "the woman had
indications of needle marks in her arms where she had been taking dope;
that she was sitting very, very close to the Negro in the car; that it
had the appearance of a necking party." (Liuzzo's arm was cut by a shard
of glass from the shattered car window.) Branch calls Hoover's comments
"slanderous Klan fantasy dressed as evidence."

J. Edgar Hoover was either a profoundly disturbed man by this time or
that rarity, actual evil. The question that Branch leaves unaddressed is
why President Johnson didn't fire him. The familiar explanation is fear
of the poison that Hoover would spew out in response. But Lyndon Johnson
could have handled that.

Under provocation that hardly any other human being could have resisted,
King never gave up on nonviolence. The rise of black-power advocates
like Stokely Carmichael did not move him. "I am not going to allow
anybody to pull me so low as to use the very methods that perpetuated
evil throughout our civilization," he told a meeting in 1966. "I'm sick
and tired of violence. I'm tired of the war in Vietnam. I'm tired of war
and conflict in the world. I'm tired of shooting. I'm tired of hatred.
I'm tired of selfishness. I'm tired of evil. I'm not going to use
violence no matter who says it!"

One cannot read this amazing book without thinking about what King would
be saying if he were with us today. He would surely be pointing to the
vast racial injustice that remains in this country, and to the growing
gap between rich and poor. I think there can be no doubt that he would
also be speaking strongly against the war in Iraq, warning that it was
killing Americans and Iraqis, nurturing terrorism, eroding the world's
regard for America.

This third volume of Branch's trilogy deepens a feeling many have had
about Dr. King, a mystery. He moved sometimes as if propelled by a force
that others could not see. He rose to make a speech, and extemporaneous
biblical eloquence would pour forth. His friends and supporters were
often uncertain what he would do. But on the great issues he was right,
and brave.

"To the end," Taylor Branch concludes, "he resisted incitements to
violence, cynicism and tribal retreat. He grasped freedom seen and
unseen, rooted in ecumenical faith, sustaining patriotism to brighten
the heritage of his country for all people. These treasures abide with
lasting promise from America in the King years."

'At Canaan's Edge: America in the King Years, 1965-68,' By Taylor Branch
Anthony Lewis is a former columnist for The Times.

Advertisement

\protect\hyperlink{after-bottom}{Continue reading the main story}

\hypertarget{site-index}{%
\subsection{Site Index}\label{site-index}}

\hypertarget{site-information-navigation}{%
\subsection{Site Information
Navigation}\label{site-information-navigation}}

\begin{itemize}
\tightlist
\item
  \href{https://help.nytimes.com/hc/en-us/articles/115014792127-Copyright-notice}{©~2020~The
  New York Times Company}
\end{itemize}

\begin{itemize}
\tightlist
\item
  \href{https://www.nytco.com/}{NYTCo}
\item
  \href{https://help.nytimes.com/hc/en-us/articles/115015385887-Contact-Us}{Contact
  Us}
\item
  \href{https://www.nytco.com/careers/}{Work with us}
\item
  \href{https://nytmediakit.com/}{Advertise}
\item
  \href{http://www.tbrandstudio.com/}{T Brand Studio}
\item
  \href{https://www.nytimes.com/privacy/cookie-policy\#how-do-i-manage-trackers}{Your
  Ad Choices}
\item
  \href{https://www.nytimes.com/privacy}{Privacy}
\item
  \href{https://help.nytimes.com/hc/en-us/articles/115014893428-Terms-of-service}{Terms
  of Service}
\item
  \href{https://help.nytimes.com/hc/en-us/articles/115014893968-Terms-of-sale}{Terms
  of Sale}
\item
  \href{https://spiderbites.nytimes.com}{Site Map}
\item
  \href{https://help.nytimes.com/hc/en-us}{Help}
\item
  \href{https://www.nytimes.com/subscription?campaignId=37WXW}{Subscriptions}
\end{itemize}
