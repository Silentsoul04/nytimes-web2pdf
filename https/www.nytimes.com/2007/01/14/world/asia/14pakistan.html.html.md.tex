Sections

SEARCH

\protect\hyperlink{site-content}{Skip to
content}\protect\hyperlink{site-index}{Skip to site index}

\href{https://www.nytimes.com/section/world/asia}{Asia Pacific}

\href{https://myaccount.nytimes.com/auth/login?response_type=cookie\&client_id=vi}{}

\href{https://www.nytimes.com/section/todayspaper}{Today's Paper}

\href{/section/world/asia}{Asia Pacific}\textbar{}Kin and Rights Groups
Search for Pakistan's Missing

\begin{itemize}
\item
\item
\item
\item
\item
\end{itemize}

Advertisement

\protect\hyperlink{after-top}{Continue reading the main story}

Supported by

\protect\hyperlink{after-sponsor}{Continue reading the main story}

\hypertarget{kin-and-rights-groups-search-for-pakistans-missing}{%
\section{Kin and Rights Groups Search for Pakistan's
Missing}\label{kin-and-rights-groups-search-for-pakistans-missing}}

\includegraphics{https://static01.nyt.com/images/2007/01/14/world/14pakistan.xlarge1.jpg?quality=75\&auto=webp\&disable=upscale}

By \href{https://www.nytimes.com/by/salman-masood}{Salman Masood}

\begin{itemize}
\item
  Jan. 14, 2007
\item
  \begin{itemize}
  \item
  \item
  \item
  \item
  \item
  \end{itemize}
\end{itemize}

RAWALPINDI, Pakistan, Jan. 9 --- Amina Masood Janjua has been fighting
for some word on the fate of her husband since he vanished from a bus
station here in July 2005. In recent months, she and her two teenage
sons and 11-year-old daughter have begun a campaign of court petitions,
protests and press releases.

More than 30 families of other missing men have joined her, all seeking
to locate what they and human rights groups say are hundreds of people
who have disappeared into the hands of the country's feared intelligence
agencies in the last few years.

The Human Rights Commission of Pakistan, an independent group, estimates
that 400 citizens have been abducted and detained across the country
since 2001. Amnesty International says many have been swept up in a
campaign against people suspected of being extremists and terrorists.
But some here also charge that the government is using the pretext of
the war on terror to crack down on opponents.

In addition to some with ties to extremist groups, those missing include
critics of the government, nationalists, journalists, scientists,
researchers and social and political workers, the groups say. Mrs.
Janjua says she has compiled a list of 115 missing persons, and says the
list could grow as more families gain the courage to come out in the
open.

Pakistani officials deny any involvement in extrajudicial detentions or
any knowledge of the men's whereabouts.

This week a Supreme Court judge nonetheless ordered the government to
speed up the process of finding 41 men listed as missing by Mrs. Janjua
and her supporters after the court took up their cases in an
unprecedented decision in October.

At the court hearing on Monday, the government acknowledged that it had
located 25 of the 41 men listed by Mrs. Janjua, ``who are now free,''
according to Nasir Saeed Sheikh, the deputy attorney general, though it
refused to say from where they had been released.

Mrs. Janjua and others said the men were held in detention centers and
safe houses of military intelligence, though most of those freed were
reluctant to talk about their experiences. Mrs. Janjua maintained that
only 18 persons had actually been freed.

Her husband, Masood Ahmed Janjua, 45, an educator and businessman, was
not among them. Mr. Sheikh told the court that, according to a report by
the Interior Ministry, all intelligence agencies had denied detaining
Mr. Janjua.

Mr. Janjua left his home around 9:30 a.m. on July 30, 2005. He was
heading to Peshawar in the northwest to attend a religious gathering
with a friend, Faisal Fraz, 26, a mechanical engineer from the eastern
city of Lahore.

Both had reservations on a 10 o'clock bus bound for Peshawar, but never
made it to their destination, according to the families. ``Before even
reaching the bus stop, somewhere on the way, they were picked up,'' Mrs.
Janjua says.

Relatives of missing persons and rights advocates here say Mr. Janjua
and the others are among the many ``forced disappearances'' or ``illegal
detentions'' that were rare before 2001. In many cases, family members
have received no news of the presumed detainees for months and even
years.

``Hundreds of people suspected of links to Al Qaeda or the Taliban have
been arbitrarily arrested and detained,'' a report by Amnesty
International issued in September said. ``Scores have become victims of
enforced disappearances; some of these have been unlawfully transferred
(sometimes in return for money) to the custody of other countries,
notably the U.S.A.,'' the report said.

``The clandestine nature of the arrest and detention of terror suspects
make it impossible to ascertain exactly how many people have been
subjected to arbitrary detention or enforced disappearance,'' it added.

I. A. Rahman, director of the Human Rights Commission of Pakistan, said
the government was using the cover of a war on terrorism to flout the
law. ``Unstable states like Pakistan are taking full advantage of `war
on terror,' '' Mr. Rahman said. He said the government was using the
antiterror campaign to crack down on its opponents and critics,
especially in Baluchistan, where government forces are fighting a
nationalist insurrection.

``It is correct that many of those arrested or detained were connected
with Al Qaeda or extremist organizations,'' he said. ``But a number of
people have been taken into custody whose only crime seems to be that
they are nationalists in Baluchistan or Sindh. In Baluchistan, there is
no Al Qaeda activity,'' he said.

In cases that are brought before a court, he noted, a government denial
of detention basically closes the case on a habeas corpus petition. ``It
was only in the end of 2006 that the Supreme Court said the government
must find out where are these people,'' he said.

Image

Mrs. Janjua's son Muhammad, 17, was beaten as police officers broke up
the march. They lowered his trousers as a means of humiliating
him.Credit...Faisal Mahmood/Reuters

While many of those missing persons were suspected of having links to
extremist or terrorist activities or have been involved in them, many
among them were innocent, the relatives maintained.

Majid Khan, 26, a computer engineer, disappeared from of Karachi, a
southern port city, four years ago and is now in Guantánamo Bay, Cuba,
said his wife, Rabiya Majid. ``We don't why he was arrested,'' she said.

Mrs. Janjua, too, says she has no clue as to why her husband
disappeared. The Janjua family lives in Rawalpindi, in the neighborhood
of Westridge, a relatively well-off enclave inhabited mostly by active
and retired military officers.

Before his disappearance, Mr. Janjua, who holds a bachelor's degree in
marine engineering, was working as managing director of a private
institute here, the College of Information and Technology. He was also
running a travel agency and involved in charity work, his wife said.

``He had no links with any extremist organization,'' Mrs. Janjua said,
though she acknowledged that he worked ``off and on'' with Tablighi
Jamaat. The group characterizes itself as a nonpolitical, nonviolent
movement that seeks to spread Orthodox Islam by proselytizing, but it
has also come under suspicion by authorities as a potential recruiting
ground for extremists.

Since her husband's disappearance, Mrs. Janjua has taken over his
business and his work at the college in addition to leading the drive,
with the other families, to find the missing. Together they have formed
a group called Defense of Human Rights.

In the last week of December, wives, daughters and sisters of dozens of
missing men, led by Mrs. Janjua, gathered in Rawalpindi, holding up
posters and portraits of the missing men and shouting, ``Give our loved
ones back.''

But their protest was quickly thwarted by the police. The photographs of
the missing men were snatched. The posters were confiscated.

Mrs. Janjua's eldest son, Muhammad, 17, was beaten by the police, who
removed his pants to humiliate him before they whisked him away in a
police van. He was freed that evening but the next morning the image of
Muhammad with his baggy trousers pulled down by the police appeared in
newspapers across the country. Op-ed columnists and editorials expressed
outrage at police ``brutality'' and sympathy for the missing people's
families surged.

Some of those released, like Muhammad Tariq, 35, have returned home. He
is one of the few willing to talk. Mr. Tariq acknowledges that he
formerly belonged to Jaish-e-Muhammad, a banned extremist group, but
says he just gave the group money and was not an active member.

Mr. Tariq, a business owner from Gujranwala in the east who sells iron
pipe, was ``picked up in broad daylight on June 14, 2004, by around a
dozen plainclothesmen and elite police commandos,'' his father, Nizamud
Din, said in an interview.

Mr. Din said he had been unsuccessful in locating his son through the
courts, police officials and even the Senate's Standing Committee on
Human Rights. ``He was portrayed as a big catch --- a big terrorist,''
Mr. Din said.

President Pervez Musharraf even alluded to the case, without mentioning
Mr. Tariq by name, in his book ``In the Line of Fire'' in connection
with a failed assassination attempt in December 2003, Mr. Din said.

General Musharraf wrote in his memoirs that a person from Gujranwala
gave refuge to Abu Faraj al-Libbi, the No. 3 Qaeda leader. He was
arrested in Pakistan in May 2005 and accused of organizing the failed
assassination.

``It is all nonsense,'' Mr. Tariq said. ``I have no link. I don't even
know Libbi.''

Mr. Tariq says he was singled out because in 2003 he briefly put up a
family, introduced to him through a friend, of an Arab man who had been
arrested in Quetta.

Mr. Tariq, a father of five, stammers while recounting his time in
detention.

``For two years, I did not see the sky, the sun or the moon,'' he said.
He said he was kept in a 4 foot by 7 foot cell in this city, was
interrogated by Pakistani military officers, mostly about Mr. Libbi, and
endured ``all kinds of imaginable torture.''

He was released Nov. 27 and pushed from a vehicle at night at an
intersection near Islamabad. He said he had never been brought before a
court. Mr. Din and Mr. Tariq said they believed the release was a result
of the pressure from Mrs. Janjua's group and the Supreme Court case.

Mrs. Janjua hopes her husband will return the same way, some day soon.
``At every doorbell,'' she said, ``I think he is back.''

Advertisement

\protect\hyperlink{after-bottom}{Continue reading the main story}

\hypertarget{site-index}{%
\subsection{Site Index}\label{site-index}}

\hypertarget{site-information-navigation}{%
\subsection{Site Information
Navigation}\label{site-information-navigation}}

\begin{itemize}
\tightlist
\item
  \href{https://help.nytimes.com/hc/en-us/articles/115014792127-Copyright-notice}{©~2020~The
  New York Times Company}
\end{itemize}

\begin{itemize}
\tightlist
\item
  \href{https://www.nytco.com/}{NYTCo}
\item
  \href{https://help.nytimes.com/hc/en-us/articles/115015385887-Contact-Us}{Contact
  Us}
\item
  \href{https://www.nytco.com/careers/}{Work with us}
\item
  \href{https://nytmediakit.com/}{Advertise}
\item
  \href{http://www.tbrandstudio.com/}{T Brand Studio}
\item
  \href{https://www.nytimes.com/privacy/cookie-policy\#how-do-i-manage-trackers}{Your
  Ad Choices}
\item
  \href{https://www.nytimes.com/privacy}{Privacy}
\item
  \href{https://help.nytimes.com/hc/en-us/articles/115014893428-Terms-of-service}{Terms
  of Service}
\item
  \href{https://help.nytimes.com/hc/en-us/articles/115014893968-Terms-of-sale}{Terms
  of Sale}
\item
  \href{https://spiderbites.nytimes.com}{Site Map}
\item
  \href{https://help.nytimes.com/hc/en-us}{Help}
\item
  \href{https://www.nytimes.com/subscription?campaignId=37WXW}{Subscriptions}
\end{itemize}
