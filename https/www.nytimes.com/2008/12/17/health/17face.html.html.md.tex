Sections

SEARCH

\protect\hyperlink{site-content}{Skip to
content}\protect\hyperlink{site-index}{Skip to site index}

\href{https://www.nytimes.com/section/health}{Health}

\href{https://myaccount.nytimes.com/auth/login?response_type=cookie\&client_id=vi}{}

\href{https://www.nytimes.com/section/todayspaper}{Today's Paper}

\href{/section/health}{Health}\textbar{}First Face Transplant Performed
in the U.S.

\begin{itemize}
\item
\item
\item
\item
\item
\end{itemize}

Advertisement

\protect\hyperlink{after-top}{Continue reading the main story}

Supported by

\protect\hyperlink{after-sponsor}{Continue reading the main story}

\hypertarget{first-face-transplant-performed-in-the-us}{%
\section{First Face Transplant Performed in the
U.S.}\label{first-face-transplant-performed-in-the-us}}

By \href{https://www.nytimes.com/by/lawrence-k-altman}{Lawrence K.
Altman}

\begin{itemize}
\item
  Dec. 16, 2008
\item
  \begin{itemize}
  \item
  \item
  \item
  \item
  \item
  \end{itemize}
\end{itemize}

Cleveland Clinic surgeons have performed the nation's first near total
face transplant, officials said on Tuesday. The patient is a woman who
was not identified.

Three partial face transplants have been performed since 2005, two in
France and one in China. All have involved using facial tissue from a
dead donor with permission from their families.

The Cleveland surgical team, led by Dr. Maria Siemionow, said it had
replaced about 80 percent of the patient's face with that of a dead
woman in the last two weeks. The doctors offered no details on the
patient, but said they would discuss her surgery at a news conference on
Wednesday.

Recent improvements in managing the care of transplant surgical
patients, including the use of better anti-rejection drugs, have allowed
doctors to forge into new areas of tissue transplants, including the
hands and face.

Such transplants are experimental and highly controversial.

A main area of concern, critics contend, is that the recipients must
take anti-rejection drugs for the rest of their lives. An adverse
reaction can come at any time, but can often be managed by adjusting the
dose of the drugs. But such fine-tuning involves a balancing act ---
giving sufficient amounts of the drugs to prevent rejection of the
tissue but not enough to lead to infection. What can make a face
transplant particularly risky is that, if the drugs fail, surgeons may
have little to offer the recipient.

Critics have also raised ethical concerns, including protecting the
donor's identity. Plans for face transplants at a number of medical
centers in this country and Europe have been slowed by difficulty in
finding donors.

But transplant pioneers say that the psychological effects of facial
damage from injuries, birth defects, burns and a number of diseases can
be psychologically devastating. Though reconstructive surgery is
possible in many cases, proponents say that in other cases, an
experimental face transplant could be worth the risks if patients and
donors and their families understand them.

Transplant surgery pioneers also point to the apparent success of the
three earlier face transplants and a number of hand transplants. Some of
these operations --- so-called composite transplants --- have involved
transplanting not only the skin, but also underlying soft tissues.

In November 2005, a team in Amiens, France, performed the first partial
face transplant. The recipient, Isabelle Dinoire, then 38, was seriously
disfigured when her Labrador retriever mauled her. The surgeons grafted
a nose, lips and chin from a donor who had been declared brain dead.

In a published report in December 2007, Ms. Dinoire's doctors said she
was satisfied with the aesthetic result. She has spoken in a news
conference.

In 2006, Chinese doctors did a partial face transplant on a farmer who
lost much of the right side of his face in a bear attack.

In 2007, a French team performed the third partial facial transplant, on
a 29-year-old man. His face had been disfigured by neurofibromatosis, a
genetic disorder of the nervous system that causes tumors to grow in
tissues around nerves.

Advertisement

\protect\hyperlink{after-bottom}{Continue reading the main story}

\hypertarget{site-index}{%
\subsection{Site Index}\label{site-index}}

\hypertarget{site-information-navigation}{%
\subsection{Site Information
Navigation}\label{site-information-navigation}}

\begin{itemize}
\tightlist
\item
  \href{https://help.nytimes.com/hc/en-us/articles/115014792127-Copyright-notice}{©~2020~The
  New York Times Company}
\end{itemize}

\begin{itemize}
\tightlist
\item
  \href{https://www.nytco.com/}{NYTCo}
\item
  \href{https://help.nytimes.com/hc/en-us/articles/115015385887-Contact-Us}{Contact
  Us}
\item
  \href{https://www.nytco.com/careers/}{Work with us}
\item
  \href{https://nytmediakit.com/}{Advertise}
\item
  \href{http://www.tbrandstudio.com/}{T Brand Studio}
\item
  \href{https://www.nytimes.com/privacy/cookie-policy\#how-do-i-manage-trackers}{Your
  Ad Choices}
\item
  \href{https://www.nytimes.com/privacy}{Privacy}
\item
  \href{https://help.nytimes.com/hc/en-us/articles/115014893428-Terms-of-service}{Terms
  of Service}
\item
  \href{https://help.nytimes.com/hc/en-us/articles/115014893968-Terms-of-sale}{Terms
  of Sale}
\item
  \href{https://spiderbites.nytimes.com}{Site Map}
\item
  \href{https://help.nytimes.com/hc/en-us}{Help}
\item
  \href{https://www.nytimes.com/subscription?campaignId=37WXW}{Subscriptions}
\end{itemize}
