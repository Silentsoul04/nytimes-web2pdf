Sections

SEARCH

\protect\hyperlink{site-content}{Skip to
content}\protect\hyperlink{site-index}{Skip to site index}

\href{https://www.nytimes.com/section/us}{U.S.}

\href{https://myaccount.nytimes.com/auth/login?response_type=cookie\&client_id=vi}{}

\href{https://www.nytimes.com/section/todayspaper}{Today's Paper}

\href{/section/us}{U.S.}\textbar{}In Budget Deal, California Shuts \$41
Billion Gap

\begin{itemize}
\item
\item
\item
\item
\item
\item
\end{itemize}

Advertisement

\protect\hyperlink{after-top}{Continue reading the main story}

Supported by

\protect\hyperlink{after-sponsor}{Continue reading the main story}

\hypertarget{in-budget-deal-california-shuts-41-billion-gap}{%
\section{In Budget Deal, California Shuts \$41 Billion
Gap}\label{in-budget-deal-california-shuts-41-billion-gap}}

\includegraphics{https://static01.nyt.com/images/2009/02/19/us/20california2_600.JPG?quality=75\&auto=webp\&disable=upscale}

By \href{https://www.nytimes.com/by/jennifer-steinhauer}{Jennifer
Steinhauer}

\begin{itemize}
\item
  Feb. 19, 2009
\item
  \begin{itemize}
  \item
  \item
  \item
  \item
  \item
  \item
  \end{itemize}
\end{itemize}

LOS ANGELES --- Take-home pay for Californians is about to shrink.
Jeans, hammers, burgers and fries will cost more. Public school children
will make do with old textbooks and find more classmates sitting next to
them. Parents will receive fewer tax benefits, and state university
students will pay 9 percent more in tuition.

As the sun rose in Sacramento on Thursday, state lawmakers ended months
of political gridlock and agreed on a series of budget measures that
included something for most everyone in California to despise. The \$143
billion budget closes a \$41 billion deficit through 2010 with tax
increases, deep cuts in services and extensive borrowing.

Although California's budget process is unusual and its fiscal problems
outsized --- the state's deficit is larger than the expenditures of all
but 10 other states --- economists say this budget foreshadows the
difficult choices that other state legislatures will soon face as the
national economy worsens.

Republican lawmakers voted for tax increases at the possible expense of
losing the next election; Democrats agreed to spending cuts unheard of
in other downturns; and most everyone in Sacramento averted even greater
compromises by looking to the federal stimulus money to bail them out.

California wrestled with its budget shortfall earlier than other states
essentially because of a trick of timing. The state's current budget was
passed months late last fall and was immediately shot through with holes
because of the economic downturn. In a lengthy emergency session that
ended with the vote on Thursday, legislators closed the current gap as
well as the projected gap for the next fiscal year.

Image

Most other states are only beginning to address their shortfalls. But
with 40 states operating in the red, similar days of reckoning will soon
be coming to state capitals from Florida to Arizona, state budget
officials say.

``California is an example of what you will see'' across the country,
said Susan Urahn, the managing director of the Pew Center on the States
and a budget expert. The size of budget deficits in other states will
lead to similarly hard-fought debates on how to close the gaps, Ms.
Urahn said.

What is more, California might have set the template as other states
ponder how to apply the more fungible outlays of the federal stimulus
money. ``My guess is states will use what they can to reduce cuts to the
bone in education and health care,'' Ms. Urahn said.

California's budget agreement came after a record-long floor session of
nearly 46 hours. Democrats, who control both houses of the Legislature,
reluctantly agreed to a series of concessions to win the support of a
single Republican senator, whose vote was necessary to reach the
two-thirds majority required under state law for budget bills.

All 75 Democrats in the Legislature voted for the budget agreement,
along with 6 of the 44 Republicans in the two houses. Gov. Arnold
Schwarzenegger, a Republican, said he would sign the agreement on
Friday.

The pact contains \$14.8 billion in spending cuts, including to public
transit, health care, schools and the courts; \$12.5 billion in tax
increases; \$5.4 billion in new borrowing; and the creation of a \$1
billion reserve fund.

Image

Gov. Arnold Schwarzenegger updated a "deficit clock" on Thursday after a
budget was approved following months of gridlock.Credit...Rich
Pedroncelli/Associated Press

Personal income tax rates will rise by one-quarter of a percent, and the
state sales tax will climb by 1 percentage point, to 6 percent, though
each county will have different rates and the average will be 8.9
percent. The state's vehicle license fee --- which Mr. Schwarzenegger
abolished with great theatrics when he took office in 2003 --- will
nearly double, to 1.15 percent of the value of the vehicle.

Left on the cutting-room floor was a proposed 12-cent increase in the
gasoline tax; lawmakers filled the gap instead with \$600 million in
cuts and an infusion of federal stimulus money.

According to the budget documents, if the state receives what it
predicts from the federal stimulus package --- more than \$9 billion ---
there would be other benefits to the budget: borrowing would be reduced
by roughly half, \$950 million in cuts would be restored and the tax
increases would be reduced.

After negotiating almost without sleep since Saturday, a deal was struck
in the early hours on Thursday when Democratic lawmakers agreed to the
lion's share of the demands made by the holdout Republican, Abel
Maldonado of Santa Maria, who wanted state constitutional amendments
banning legislative pay increases during deficit years and creating more
competitive elections by establishing open primaries.

It was a hard-fought but costly victory for Mr. Schwarzenegger, who
became governor during another budget crisis in 2003 in part by vowing
never to raise taxes. He will soon have the distinction of being the
first California governor to sign off on a major tax increase since Pete
Wilson, a Republican, negotiated a \$7 billion broad-based increase in
1991.

Mr. Schwarzenegger rode into office as a reformer but has morphed into a
centrist often at odds with his party.

Image

California State Senate President Pro Tem Darrell Steinberg, left, with
Minority Leader Dennis Hollingsworth at the Capitol in Sacramento on
Thursday morning before the vote on the state budget plan.Credit...Rich
Pedroncelli/Associated Press

He said the state's election processes, including its many ballot
initiatives and the drawing of uncompetitive political districts, were
to blame for the three-month budget stalemate, during which
infrastructure projects were shut down, workers were furloughed,
payments to counties were withheld and tax refunds were left in state
coffers.

``We've got to bring people to the center,'' Mr. Schwarzenegger said at
a news conference in Sacramento, during which he vowed to campaign for
open primaries, which would require voter approval. ``We need to change
the system itself.''

The governor worked behind the scenes to win Mr. Maldonado's vote and
backed his demand for legislative approval for the amendment to make
California political primaries open and nonpartisan. A similar measure
was recently enacted in Washington State.

Proponents of open primaries, which weaken traditional parties, cite the
gerrymandered districts here that have typically resulted in the
election of Republicans who are more conservative than the general
population and Democrats who are more liberal.

In the past, Democrats have been able to count on Mr. Maldonado, a
moderate Republican from the Central Coast, to break with his party
without making such demands. The only other Republican who seemed
prepared to break the deadlock in the Senate, Dave Cox, indicated mild
support in exchange for dismantling environmental legislation near to
the hearts of Democrats and Mr. Schwarzenegger.

The new taxes in the budget agreement are set to last for two years but
could be extended another two years if voters approve a permanent state
spending cap in a referendum on May 19. The open-primary measure,
intended to apply to legislative and Congressional races, is set to be
on the ballot in June 2010.

It was not immediately clear on Thursday when the \$3.3 billion that was
not paid during the stalemate to taxpayers, contractors and local
governments would be forthcoming from the state. The state controller
and the Finance Department will work this week to set a pay schedule.

Advertisement

\protect\hyperlink{after-bottom}{Continue reading the main story}

\hypertarget{site-index}{%
\subsection{Site Index}\label{site-index}}

\hypertarget{site-information-navigation}{%
\subsection{Site Information
Navigation}\label{site-information-navigation}}

\begin{itemize}
\tightlist
\item
  \href{https://help.nytimes.com/hc/en-us/articles/115014792127-Copyright-notice}{©~2020~The
  New York Times Company}
\end{itemize}

\begin{itemize}
\tightlist
\item
  \href{https://www.nytco.com/}{NYTCo}
\item
  \href{https://help.nytimes.com/hc/en-us/articles/115015385887-Contact-Us}{Contact
  Us}
\item
  \href{https://www.nytco.com/careers/}{Work with us}
\item
  \href{https://nytmediakit.com/}{Advertise}
\item
  \href{http://www.tbrandstudio.com/}{T Brand Studio}
\item
  \href{https://www.nytimes.com/privacy/cookie-policy\#how-do-i-manage-trackers}{Your
  Ad Choices}
\item
  \href{https://www.nytimes.com/privacy}{Privacy}
\item
  \href{https://help.nytimes.com/hc/en-us/articles/115014893428-Terms-of-service}{Terms
  of Service}
\item
  \href{https://help.nytimes.com/hc/en-us/articles/115014893968-Terms-of-sale}{Terms
  of Sale}
\item
  \href{https://spiderbites.nytimes.com}{Site Map}
\item
  \href{https://help.nytimes.com/hc/en-us}{Help}
\item
  \href{https://www.nytimes.com/subscription?campaignId=37WXW}{Subscriptions}
\end{itemize}
