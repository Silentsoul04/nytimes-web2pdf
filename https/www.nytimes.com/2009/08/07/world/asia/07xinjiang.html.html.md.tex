Sections

SEARCH

\protect\hyperlink{site-content}{Skip to
content}\protect\hyperlink{site-index}{Skip to site index}

\href{https://www.nytimes.com/section/world/asia}{Asia Pacific}

\href{https://myaccount.nytimes.com/auth/login?response_type=cookie\&client_id=vi}{}

\href{https://www.nytimes.com/section/todayspaper}{Today's Paper}

\href{/section/world/asia}{Asia Pacific}\textbar{}Migrants to China's
West Bask in Prosperity

\begin{itemize}
\item
\item
\item
\item
\item
\end{itemize}

Advertisement

\protect\hyperlink{after-top}{Continue reading the main story}

Supported by

\protect\hyperlink{after-sponsor}{Continue reading the main story}

\hypertarget{migrants-to-chinas-west-bask-in-prosperity}{%
\section{Migrants to China's West Bask in
Prosperity}\label{migrants-to-chinas-west-bask-in-prosperity}}

\includegraphics{https://static01.nyt.com/images/2009/08/06/world/asia/07xinjiang_600.jpg?quality=75\&auto=webp\&disable=upscale}

By \href{https://www.nytimes.com/by/andrew-jacobs}{Andrew Jacobs}

\begin{itemize}
\item
  Aug. 6, 2009
\item
  \begin{itemize}
  \item
  \item
  \item
  \item
  \item
  \end{itemize}
\end{itemize}

SHIHEZI, China --- They marched through the streets of Beijing, Shanghai
and countless small towns propelled by patriotic cheers and thumping
drums. It was 1956, and Mao Zedong was calling on China's youth to
``open up the west,'' the vast borderland known as Xinjiang that for
centuries had defied subjugation.

After a monthlong journey by train and open-air truck, thousands arrived
at this Gobi Desert army outpost to find that the factory jobs, hot
baths and telephones in every house were nothing but empty promises to
lure them to a faraway land.

``We lived in holes in the ground, and all we did night and day was hard
labor,'' recalled Han Zuxue, a sun-creased 72-year-old who was a
teenager when he left his home in eastern Henan Province. ``At first we
cried every day but over time we forgot our sadness.''

More than five decades of toil later, men and women like Mr. Han have
helped transform
\href{http://www.china.org.cn/english/2004/Oct/108620.htm}{Shihezi} into
a tree-shaded, bustling oasis whose canned tomatoes, fiery grain alcohol
and enormous cotton yields are famous throughout China.

This city of 650,000 is a showcase of the Xinjiang Production and
Construction Corps, a uniquely Chinese conglomerate of farms and
factories that were created by decommissioned Red Army soldiers at the
end of the civil war.

Image

A statue of Mao Zedong at the museum in Shihezi. In 1956, Mao called on
China's young people to ``open up the west,'' the border region known as
Xinjiang.Credit...Katharina Hesse for The New York Times

``Put your weapons aside and pick up the tools of construction,'' one
popular slogan went. ``Develop Xinjiang, defend the nation's borders and
protect social stability.''

With a total population of 2.6 million, 95 percent of it ethnic Han
Chinese, Shihezi and a string of other settlements created by the
military are stable strongholds in a region whose majority non-Han
populace has often been unhappy under Beijing's rule. Last month, that
discontent showed itself during vicious ethnic rioting that claimed 197
lives in Urumqi, the regional capital, which is a two-hour drive away.

The government says that most of the dead were Han Chinese bludgeoned by
mobs of
\href{http://www.uhrp.org/articles/855/1/Retaining-the-Loyalty-of-Xinjiangs-Hans-/index.html}{Uighurs},
Muslims of Turkish ancestry whose presence in Xinjiang has been steadily
diluted by migration from China's densely populated east.

``Ever since we arrived they've resented us and had no appreciation for
how we've improved this place,'' said He Zhenjie, 76, who has spent his
adult life leveling sand dunes, planting trees and digging irrigation
ditches. ``But we're here to stay. The Uighurs will never wrest Xinjiang
away.''

Even if many Uighurs view the settlers as nothing more than Chinese
colonists, many Chinese consider the bingtuan, meaning soldier corps, a
major success. In one fell swoop Mao deployed 200,000 idle soldiers to
help develop and occupy a resource-rich, politically strategic region
bordering India, Mongolia and the Soviet Union, a onetime ally turned
menace.

Shihezi and other bingtuan settlements quickly became self-sufficient, a
relief to a government lacking resources, and its ``reclamation
warriors'' worked without pay those first few years, steadily turning
thousands of acres of inhospitable scrubland into some of the country's
most fertile terrain.

Image

Yue Caiying, who moved to the region in 1963, had to put aside her dream
of being a nurse.Credit...Katharina Hesse for The New York Times

With an annual output of goods and services of \$7 billion, the
settlements run by the bingtuan include five cities, 180 farming
communities and 1,000 companies. They also report directly to Beijing
and run their own courts, colleges and newspapers.

``During peaceful times, they are a force for development, but if
anything urgent happens, they will step out and maintain social
stability and combat the separatists,'' said Li Sheng, a researcher at
the Chinese Academy of Social Sciences and a former bingtuan member who
writes about the region's history.

In those early years, the ranks of the bingtuan were fortified by petty
criminals, former prisoners of war, prostitutes and intellectuals, all
sent west for ``re-education.'' During the mid-1950s, 40,000 young women
were lured to Xinjiang with promises of the good life: they arrived to
discover their main purpose was to relieve the loneliness of the male
pioneers and cement the region's Han presence through their progeny.

Demographics have always been a tactical element of the campaign to
pacify the region. In 1949, when the Communists declared the
establishment of the People's Republic of China, there were just 300,000
Han Chinese in Xinjiang. Today, the number of Han has grown to 7.5
million, just over 40 percent of the region's population. The percentage
of Uighurs has fallen to 45 percent, or about 8.3 million.

Their grievances have multiplied even as Xinjiang has grown more
prosperous, thanks in part to its huge reserves of natural gas, oil and
minerals. Many Uighurs complain about the repression of their Islamic
faith, official policies that marginalize their language and a lack of
job opportunities, especially at government bureaus and inside the
bingtuan.

During a recent visit to Shihezi, armed paramilitary policemen stopped
every car and bus entering the city. But only Uighurs were made to step
out of vehicles for identification checks and searches.

Image

Years of toil have turned Shihezi into a bustling oasis.

Neatly laid out on a grid, its sidewalks graced by apple trees and elms,
the city is populated by the sturdy and defiantly proud who think of
Xinjiang as China's version of Manifest Destiny, the doctrine
undergirding the westward expansion of the United States in the 19th
century. But just beneath the self-satisfaction runs a deep vein of
bitterness, especially among those who arrived in the 1950s and 1960s.

``I thought I was going to be a nurse, but I ended up sweeping the
streets and cleaning toilets,'' said Yue Caiying, who moved here in
1963, and, like many of those with an education, was forced to set aside
personal ambition.

Lu Yiping, an author who spent five years interviewing women trucked
into Xinjiang from Hunan Province, tells of girls lured with promises of
Russian-language classes and textile-mill jobs. In an interview
published online, he told the story of arriving women greeted by Wang
Zhen, the famously hard-line general who helped tame the region.
``Comrades, you must prepare to bury your bones in Xinjiang,'' he quoted
Mr. Wang as telling the women.

Still, for many early settlers, Xinjiang offered an escape from the
deprivation that stalked many rural areas between 1959 and 1962, when
Mao's disastrous attempt to start up China's industrialization led to
famine that killed millions.

Early settlers like Ma Xianwu, who arrived here in 1951 and helped dig
the first thatch-covered pits that served as shelter, offer a typical
mix of conflicted emotions. He expressed wonder at the city he had
helped create, but also sorrow over the hardship he and others had
endured.

``People would lose ears and toes to frostbite,'' said Mr. Ma, who is 94
and nearly toothless.

But any sense of bitterness has faded. ``We were serving the
motherland,'' he said, waving off the adulation of a visitor. ``The
glory belongs to the party. I'm just one drop of water in the ocean.''

Advertisement

\protect\hyperlink{after-bottom}{Continue reading the main story}

\hypertarget{site-index}{%
\subsection{Site Index}\label{site-index}}

\hypertarget{site-information-navigation}{%
\subsection{Site Information
Navigation}\label{site-information-navigation}}

\begin{itemize}
\tightlist
\item
  \href{https://help.nytimes.com/hc/en-us/articles/115014792127-Copyright-notice}{©~2020~The
  New York Times Company}
\end{itemize}

\begin{itemize}
\tightlist
\item
  \href{https://www.nytco.com/}{NYTCo}
\item
  \href{https://help.nytimes.com/hc/en-us/articles/115015385887-Contact-Us}{Contact
  Us}
\item
  \href{https://www.nytco.com/careers/}{Work with us}
\item
  \href{https://nytmediakit.com/}{Advertise}
\item
  \href{http://www.tbrandstudio.com/}{T Brand Studio}
\item
  \href{https://www.nytimes.com/privacy/cookie-policy\#how-do-i-manage-trackers}{Your
  Ad Choices}
\item
  \href{https://www.nytimes.com/privacy}{Privacy}
\item
  \href{https://help.nytimes.com/hc/en-us/articles/115014893428-Terms-of-service}{Terms
  of Service}
\item
  \href{https://help.nytimes.com/hc/en-us/articles/115014893968-Terms-of-sale}{Terms
  of Sale}
\item
  \href{https://spiderbites.nytimes.com}{Site Map}
\item
  \href{https://help.nytimes.com/hc/en-us}{Help}
\item
  \href{https://www.nytimes.com/subscription?campaignId=37WXW}{Subscriptions}
\end{itemize}
