Sections

SEARCH

\protect\hyperlink{site-content}{Skip to
content}\protect\hyperlink{site-index}{Skip to site index}

\href{https://www.nytimes.com/section/politics}{Politics}

\href{https://myaccount.nytimes.com/auth/login?response_type=cookie\&client_id=vi}{}

\href{https://www.nytimes.com/section/todayspaper}{Today's Paper}

\href{/section/politics}{Politics}\textbar{}With Just Three 9s, Cain
Refigured Math for Taxes

\begin{itemize}
\item
\item
\item
\item
\item
\item
\end{itemize}

Advertisement

\protect\hyperlink{after-top}{Continue reading the main story}

Supported by

\protect\hyperlink{after-sponsor}{Continue reading the main story}

\hypertarget{with-just-three-9s-cain-refigured-math-for-taxes}{%
\section{With Just Three 9s, Cain Refigured Math for
Taxes}\label{with-just-three-9s-cain-refigured-math-for-taxes}}

By \href{https://www.nytimes.com/by/trip-gabriel}{Trip Gabriel} and
\href{https://www.nytimes.com/by/susan-saulny}{Susan Saulny}

\begin{itemize}
\item
  Oct. 12, 2011
\item
  \begin{itemize}
  \item
  \item
  \item
  \item
  \item
  \item
  \end{itemize}
\end{itemize}

It was a strategy session at 28,000 feet.
\href{http://elections.nytimes.com/2012/primaries/candidates/herman-cain}{Herman
Cain}, the Republican presidential candidate, and his advisers were on a
campaign flight this summer between Atlanta and New Hampshire, tossing
around policy ideas.

Mr. Cain, a former pizza chain chief executive, wanted a proposal to
jolt the economy and give his candidacy some definition. ``I said, `The
first fundamental, guys, is we have to throw out the tax code,'~'' Mr.
Cain said Wednesday in an interview.

``How do we come up with a bolder plan?'' he pressed two of his close
advisers.

From that exchange emerged the plan that Mr. Cain calls 9-9-9: a flat 9
percent individual income tax rate, a 9 percent corporate tax rate and a
9 percent national sales tax.

He has uttered the triple digits repeatedly, metronome-like, in speeches
and debates, until they have acquired the catchy power of a brand.

Although Mr. Cain's rivals have tried to use the plan's simplicity
against him --- responding that it sounds like the price of a pie with
pepperoni, for example --- he has stuck to his message.

Now both he and his proposal are getting intensive new scrutiny as
Republicans continue to flirt with their candidates less than three
months before casting the first votes of the primary season. He
continues to surge in national polls.

A poll released Wednesday by NBC News and The Wall Street Journal found
that Mr. Cain was effectively tied with Mitt Romney; on Tuesday night,
Mr. Cain and his tax plan were at the center of the candidates' debate.

But although the specifics of the 9-9-9 plan were developed only in the
last few months, it is only the latest incarnation of two ideas popular
among some supply-side conservatives for decades.

Mr. Cain was a co-chairman in 1996 of the presidential campaign of Steve
Forbes, who advocated a flat tax --- a single rate on income for all
payers. Mr. Cain later supported a ``fair tax,'' one that would replace
all other taxes with a national sales tax.

The 9-9-9 plan combines elements of both ideas. But it is little more
than a sketch of what would be a radical and complex overhaul of the tax
system. In developing it, Mr. Cain relied heavily on Rich Lowrie, whom
he calls his lead economist. Mr. Lowrie is an investment adviser at a
Wells Fargo office in Pepper Pike, Ohio. Although he is an unpaid member
of an advisory board of the \href{http://www.conservative.org/}{American
Conservative Union}, he has never worked for a policy research group or
an academic institution, or made a name through economic analysis.

\includegraphics{https://static01.nyt.com/images/2011/10/13/us/jp-CAIN/jp-CAIN-jumbo.jpg?quality=75\&auto=webp\&disable=upscale}

In an interview, Mr. Lowrie said he had a bachelor of science degree in
accountancy from Case Western Reserve University. On his Facebook page,
he describes his political views as ``free markets.'' Mr. Lowrie said he
had been inspired by two well-known proponents of supply-side thinking:
Arthur Laffer, often considered the father of the concept that lower tax
rates help pay for themselves by generating additional economic growth,
and Jude Wanniski, who promoted the idea among politicians. Mr. Lowrie
became involved with the Ohio chapter of Americans for Prosperity, the
conservative organization supported by the billionaire Koch brothers.

The plan could have major economic and political challenges: It might
result in a substantial revenue loss for the government and shift the
tax burden toward lower- and middle-income people.

In an interview, Mr. Cain, a math major in college, said he had asked
Mr. Lowrie to do a ``regression analysis'' that would allow the
government to eliminate all existing taxes, including those on capital
gains and estates, and collect the same revenue from just three streams.
``The number came up to be 9 percent,'' Mr. Cain said. ``And that's how
we came up with 9-9-9.''

Mr. Lowrie, who met Mr. Cain at a conference sponsored by the
conservative Club for Growth, dismissed the notion that his own
understanding of economics was limited by lack of a Ph.D. ``I don't list
myself as an economist,'' he said. ``I have an accounting degree, and
I'm an investment adviser. I've never hung out in a faculty lounge.''

A former staff member for Mr. Cain in Iowa described his and Mr.
Lowrie's relationship as ``buddy-buddy,'' adding, ``They were just like
two executives palling around together.''

Their plan has drawn fire from both right and left. Conservatives are
wary of a national sales tax, concerned that it would create another,
easily increased method of taxation. Among the critics are The Wall
Street Journal editorial page and Bruce Bartlett, an official in the
Reagan and first Bush administrations, who contributes to the Economix
blog for The New York Times.

Critics, especially liberals, say the plan offers a huge tax break for
the wealthy while imposing a steep, regressive new sales tax on the
middle-class and working poor, with everyday items like milk and bread
being subject to a 9 percent tax. In Tuesday's debate, defending his own
59-point economic plan, Mr. Romney took aim at Mr. Cain's: ``Simple
answers are always very helpful, but oftentimes inadequate.''

Mr. Cain said an independent analyst had examined the plan and found
that it would raise the same revenue as the existing tax structure. The
analyst, Gary Robbins, a consultant in Arlington, Va., said the Cain
plan was ``revenue neutral.'' That is, if it had been in place in 2008,
the last year taxes were not affected by the recession, it would have
raised \$2.3 trillion in revenue, the amount the federal government
collected from all sources other than excise taxes.

``It's not the plan I particularly would do, but it's a sound plan,''
said Mr. Robbins, who worked for Mr. Forbes's campaigns and recalled
that his flat tax was ``designed on my dining room table.''

Actually, Mr. Robbins's math determined that the across-the-board rates
necessary to raise the same money as existing federal taxes should be
9.1 percent.

Somehow, 9.1-9.1-9.1 does not trip off the tongue.

Advertisement

\protect\hyperlink{after-bottom}{Continue reading the main story}

\hypertarget{site-index}{%
\subsection{Site Index}\label{site-index}}

\hypertarget{site-information-navigation}{%
\subsection{Site Information
Navigation}\label{site-information-navigation}}

\begin{itemize}
\tightlist
\item
  \href{https://help.nytimes.com/hc/en-us/articles/115014792127-Copyright-notice}{©~2020~The
  New York Times Company}
\end{itemize}

\begin{itemize}
\tightlist
\item
  \href{https://www.nytco.com/}{NYTCo}
\item
  \href{https://help.nytimes.com/hc/en-us/articles/115015385887-Contact-Us}{Contact
  Us}
\item
  \href{https://www.nytco.com/careers/}{Work with us}
\item
  \href{https://nytmediakit.com/}{Advertise}
\item
  \href{http://www.tbrandstudio.com/}{T Brand Studio}
\item
  \href{https://www.nytimes.com/privacy/cookie-policy\#how-do-i-manage-trackers}{Your
  Ad Choices}
\item
  \href{https://www.nytimes.com/privacy}{Privacy}
\item
  \href{https://help.nytimes.com/hc/en-us/articles/115014893428-Terms-of-service}{Terms
  of Service}
\item
  \href{https://help.nytimes.com/hc/en-us/articles/115014893968-Terms-of-sale}{Terms
  of Sale}
\item
  \href{https://spiderbites.nytimes.com}{Site Map}
\item
  \href{https://help.nytimes.com/hc/en-us}{Help}
\item
  \href{https://www.nytimes.com/subscription?campaignId=37WXW}{Subscriptions}
\end{itemize}
