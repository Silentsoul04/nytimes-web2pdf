Sections

SEARCH

\protect\hyperlink{site-content}{Skip to
content}\protect\hyperlink{site-index}{Skip to site index}

\href{https://www.nytimes.com/section/world/middleeast}{Middle East}

\href{https://myaccount.nytimes.com/auth/login?response_type=cookie\&client_id=vi}{}

\href{https://www.nytimes.com/section/todayspaper}{Today's Paper}

\href{/section/world/middleeast}{Middle East}\textbar{}From Unyielding
Cameraman, an Acclaimed Film

\begin{itemize}
\item
\item
\item
\item
\item
\end{itemize}

Advertisement

\protect\hyperlink{after-top}{Continue reading the main story}

Supported by

\protect\hyperlink{after-sponsor}{Continue reading the main story}

Bilin Journal

\hypertarget{from-unyielding-cameraman-an-acclaimed-film}{%
\section{From Unyielding Cameraman, an Acclaimed
Film}\label{from-unyielding-cameraman-an-acclaimed-film}}

\includegraphics{https://static01.nyt.com/images/2012/01/23/world/WEST-BANK/WEST-BANK-articleLarge.jpg?quality=75\&auto=webp\&disable=upscale}

By \href{https://www.nytimes.com/by/ethan-bronner}{Ethan Bronner}

\begin{itemize}
\item
  Jan. 22, 2012
\item
  \begin{itemize}
  \item
  \item
  \item
  \item
  \item
  \end{itemize}
\end{itemize}

BILIN, West Bank --- Emad Burnat was born to the land and, like
generations of his family in this hilltop West Bank village, he has eked
out a modest living from its rocky soil. But six years ago, at the birth
of a son, he was given a video camera and turned unexpectedly into the
village chronicler.

There was a great deal to record. Israel was building a separation
barrier on village land that included some of his family's own. The
rationale behind it was to stop suicide bombers, but the move
confiscated most of the village's arable land and allowed for the
expansion of an enormous Israeli settlement.

Bulldozers uprooted centuries-old olive trees while settlers drove up
with furniture and mobile homes. Villagers stood in the way; soldiers
arrested them. Mr. Burnat was there, day in, day out, filming with his
new camera.

Now, working with an Israeli filmmaker, Guy Davidi, Mr. Burnat has taken
his years of video and turned them into a compelling personal tale.
\href{http://www.idfa.nl/industry/tags/project.aspx?ID=3c05550a-e5e2-4ef4-83ce-78e101bd811e}{The
film}, ``Five Broken Cameras,'' won two awards in November at the
\href{http://www.idfa.nl/nl.aspx}{International Documentary Film
Festival Amsterdam}, including the Audience Award, and is one of about a
dozen films competing at Sundance this week in the World Documentary
category.

As Bilin became the center for Palestinian popular resistance --- weekly
demonstrations joined by Israeli and foreign activists, and
\href{http://www.nytimes.com/2011/06/25/world/middleeast/25palestinians.html}{a
partial Supreme Court victory forcing the barrier to be moved} and some
land returned --- Mr. Burnat's images became crucial. They were used not
only by journalists but by those fighting charges in Israeli military
courts. Accusations of assault were sometimes countered with a common
refrain: let's go to Emad's videotape.

The new documentary intersperses scenes of villagers fighting the
barrier with Mr. Burnat's son Gibreel's first words (``cartridge,''
``army''), undercover Israeli agents taking away friends and relatives,
and Mr. Burnat's wife, Soraya, begging him to turn his attention away
from politics and be with his family. Over six years, Mr. Burnat went
through five cameras, each broken in the course of filming --- among
other things, by soldiers' bullets and an angry settler. At the start of
the film, Mr. Burnat lines up the cameras on a table. They form the
movie's chapters and create a motif for the unfolding drama --- the
power of bearing witness. Mr. Burnat never puts his camera down and it
drives his opponents mad.

``Tell him if he keeps filming I will break his bones!'' a settler
declares to a soldier. Mr. Burnat keeps filming. The settler approaches
him and, as the camera rolls, throws it to the ground, breaking it. The
screen goes blank.

``When I film, I feel like the camera protects me,'' Mr. Burnat says in
his soft-voiced narration of the movie, making a point familiar to all
journalists. ``But it is an illusion.''

In one scene, soldiers come to Mr. Burnat's house (``Now it's my turn,''
he says into the camera) to arrest him on charges of throwing stones and
assaulting a soldier --- charges he denied and of which he was later
exonerated, according to an army spokesman. He films the soldiers' entry
into his house and their surreal assertion that he must turn off his
camera because he is in a ``closed military area.'' ``I am in my own
home,'' he replies. He spends three weeks in prison and six weeks under
house arrest. It takes three years for the case to be dismissed.

A subtheme of the film is the activism of Mr. Burnat's two close
friends, Adeeb and Bassem Abu Rahma, who were cousins. Bassem was
nicknamed ``Phil,'' the Arabic word for elephant. Both were playful,
big-hearted guys at the front of the demonstrations. Phil was killed at
a demonstration in 2009, and Mr. Burnat originally thought of making the
movie about him.

Image

Bilin has become the center of popular resistance.Credit...The New York
Times

But Mr. Davidi and an Israeli organization called Greenhouse, which
pairs regional filmmakers with European mentors and is financed by the
European Union, persuaded Mr. Burnat to place himself at the center of
his story. It was a crucial move that gave the film its power and
intimacy. But it did not come naturally.

``It was a very difficult decision to make such a personal film,'' Mr.
Burnat, 40, said as he sat in the garden of his home. Gibreel, now 6,
and his older sons were wandering in and out, and the high-rises of the
Modiin Illit settlement could be seen in the distance. ``I was
uncomfortable about showing footage of my wife. This may be normal in
Europe, but here in Palestine you have to answer many questions. I have
so far avoided showing the film here.''

Mr. Davidi, the 33-year-old Israeli co-director, first came to Bilin in
2005 to shoot a documentary on Palestinian workers who take construction
jobs in the settlement, and he met Mr. Burnat then. ``We wanted our film
to be an understatement, not to be provocative or combative,'' Mr.
Davidi said.

The movie's personal style is not the only issue bringing Mr. Burnat
heat. Working with an Israeli filmmaker and taking help from Greenhouse
have been controversial. The Palestinian movement increasingly promotes
a boycott of all things Israeli on the theory that contact serves to
``normalize'' relations that should be frozen until progress is made on
ending the occupation.

``When we showed the film in Amsterdam, Palestinians and other Arabs
came up to me and asked how I could work with Israelis,'' Mr. Burnat
said. ``But from the start, the struggle for Bilin involved Israeli
activists.''

Greenhouse, which has sponsored 15 films and has brought together 100
moviemakers from Israel, Lebanon, Turkey, Egypt, Algeria, Jordan,
Morocco, Tunisia and the Palestinian territories, knows that many in the
Middle East object to what it does.

Sigal Yehuda, Greenhouse's managing director, says the issue is
discussed openly at the seminars the group sponsors around the region.

``Coexistence is one of the most important fruits of this activity,''
she said. ``These are intelligent and influential people who have lived
with, eaten with and even danced with Israelis.''

For Mr. Burnat, the coexistence question is especially delicate. In late
2008, he accidently drove a truck into the separation barrier and was
badly injured. A Palestinian ambulance arrived at the same time as
Israeli soldiers, who saw what bad shape he was in and took him to an
Israeli hospital.

``If I had been taken to a Palestinian hospital,'' Mr. Burnat said, ``I
probably wouldn't have survived.'' He was unconscious for 20 days. Three
months later he was back filming, little Gibreel trailing behind.

``The only protection I can offer him,'' Mr. Burnat says of Gibreel at
that point in the film, speaking for chroniclers everywhere, ``is
allowing him to see everything with his own eyes so he can confront just
how vulnerable life is.''

Weeks later, an Israeli tear gas canister hit his friend Phil in the
chest and killed him.

Advertisement

\protect\hyperlink{after-bottom}{Continue reading the main story}

\hypertarget{site-index}{%
\subsection{Site Index}\label{site-index}}

\hypertarget{site-information-navigation}{%
\subsection{Site Information
Navigation}\label{site-information-navigation}}

\begin{itemize}
\tightlist
\item
  \href{https://help.nytimes.com/hc/en-us/articles/115014792127-Copyright-notice}{©~2020~The
  New York Times Company}
\end{itemize}

\begin{itemize}
\tightlist
\item
  \href{https://www.nytco.com/}{NYTCo}
\item
  \href{https://help.nytimes.com/hc/en-us/articles/115015385887-Contact-Us}{Contact
  Us}
\item
  \href{https://www.nytco.com/careers/}{Work with us}
\item
  \href{https://nytmediakit.com/}{Advertise}
\item
  \href{http://www.tbrandstudio.com/}{T Brand Studio}
\item
  \href{https://www.nytimes.com/privacy/cookie-policy\#how-do-i-manage-trackers}{Your
  Ad Choices}
\item
  \href{https://www.nytimes.com/privacy}{Privacy}
\item
  \href{https://help.nytimes.com/hc/en-us/articles/115014893428-Terms-of-service}{Terms
  of Service}
\item
  \href{https://help.nytimes.com/hc/en-us/articles/115014893968-Terms-of-sale}{Terms
  of Sale}
\item
  \href{https://spiderbites.nytimes.com}{Site Map}
\item
  \href{https://help.nytimes.com/hc/en-us}{Help}
\item
  \href{https://www.nytimes.com/subscription?campaignId=37WXW}{Subscriptions}
\end{itemize}
