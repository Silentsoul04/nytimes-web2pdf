Sections

SEARCH

\protect\hyperlink{site-content}{Skip to
content}\protect\hyperlink{site-index}{Skip to site index}

\href{https://www.nytimes.com/section/movies}{Movies}

\href{https://myaccount.nytimes.com/auth/login?response_type=cookie\&client_id=vi}{}

\href{https://www.nytimes.com/section/todayspaper}{Today's Paper}

\href{/section/movies}{Movies}\textbar{}A Palestinian Whose Cameras Are
Witnesses and Casualties of Conflict

\href{https://nyti.ms/KPOi3B}{https://nyti.ms/KPOi3B}

\begin{itemize}
\item
\item
\item
\item
\item
\end{itemize}

Advertisement

\protect\hyperlink{after-top}{Continue reading the main story}

Supported by

\protect\hyperlink{after-sponsor}{Continue reading the main story}

Movie Review

\hypertarget{a-palestinian-whose-cameras-are-witnesses-and-casualties-of-conflict}{%
\section{A Palestinian Whose Cameras Are Witnesses and Casualties of
Conflict}\label{a-palestinian-whose-cameras-are-witnesses-and-casualties-of-conflict}}

\includegraphics{https://static01.nyt.com/images/2012/05/30/arts/FIVE1/FIVE1-articleLarge.jpg?quality=75\&auto=webp\&disable=upscale}

\begin{itemize}
\tightlist
\item
  5 Broken Cameras\\
  **NYT Critic's Pick Directed by Emad Burnat, Guy Davidi Documentary,
  War Not Rated 1h 34m
\end{itemize}

By \href{https://www.nytimes.com/by/a-o--scott}{A.O. Scott}

\begin{itemize}
\item
  May 29, 2012
\item
  \begin{itemize}
  \item
  \item
  \item
  \item
  \item
  \end{itemize}
\end{itemize}

\href{http://vimeo.com/15843191}{``5 Broken Cameras''} provides a grim
reminder --- just in case you needed one --- of the bitter
intractability of the Israeli-Palestinian conflict. A chronicle of
protest and endurance, punctuated by violence and tiny glimmers of hope,
this documentary is unlikely to persuade anyone with a hardened view of
the issue to think again. For anyone who retains an interest in the
human contours of the situation, however, the movie is necessary, if
difficult, viewing.

For while it is hardly neutral --- presenting an extended, highly
personal view of life in a West Bank village adjacent to Israel's
controversial security fence ---
\href{http://www.nytimes.com/2012/01/23/world/middleeast/documentary-from-emad-burnats-camera-competes-at-sundance.html}{``5
Broken Cameras''} is much more than yet another polemical bulletin from
an embattled region. It may seem perverse to praise an eyewitness
account of political trauma for its formal accomplishments, but for a
project like this to claim the attention of an audience it has to
justify itself as cinema. There is no shortage of information and
opinion about the Middle East, and this film, made collaboratively by
Emad Burnat and Guy Davidi, is partly a piece of advocacy journalism.
But it is also a visual essay in autobiography and, as such, a modest,
rigorous and moving work of art.

Mr. Burnat, a Palestinian farmer in the tiny town of Bilin, was given
his first camera in 2005, when his youngest son, Gibreel, was born.
Almost simultaneously, the Israeli Army began building a barrier between
Bilin and a nearby Jewish settlement.

The residents of Bilin, outraged as their olive groves were bulldozed by
the military and burned by settlers,
\href{http://www.nytimes.com/2009/08/28/world/middleeast/28bilin.html?ref=middleeast}{organized
weekly protests}, attended by left-wing Israelis and sympathizers from
other countries. In 2007 the Israeli Supreme Court
\href{http://www.nytimes.com/2007/09/05/world/middleeast/05mideast.html}{ordered
the barrier rerouted}, and four years later, after village access to
some of the land was restored,
\href{http://www.nytimes.com/2011/06/25/world/middleeast/25palestinians.html}{the
demonstrations were called off}. Mr. Burnat's was not the only camera
present at these protests, but the footage he shot, which is accompanied
by after-the-fact voice-over narration and part of a video diary of his
daily life, is especially poignant and intimate.

\includegraphics{https://static01.nyt.com/images/2012/05/30/arts/FIVE2/FIVE2-articleLarge.jpg?quality=75\&auto=webp\&disable=upscale}

He and Mr. Davidi, a Jewish Israeli filmmaker, combed through hundreds
of hours of images gathered over more than five years. In the course of
the condensed narrative that results from their editing, we meet Mr.
Burnat's family and his neighbors, sometimes captured in candid moments
and sometimes, it appears, acting out such moments for the camera. Or
cameras, rather, since the soldiers and settlers are not always happy to
be filmed, and it is hard to protect a delicate piece of electronic
equipment when rocks, rubber bullets and tear gas grenades are flying.

The encounters between the soldiers and the demonstrators have a
ritualistic quality, but the consequences could hardly be more serious.
There are injuries --- including one sustained by Mr. Burnat himself ---
and several deaths. Many of these incidents are captured in real time,
at close range (in some cases by other people's cameras), and the
cumulative effect on the viewer is an intense, despairing sense of
frustration.

Mr. Burnat, however, is more philosophical, even when his pain and
indignation are at their highest pitch. He notes the intersection of his
family's life with the ebb and flow of Palestinian and Israeli politics,
from the relative optimism of the post-Oslo years in the early 1990s
(when his first son was born) to the current era of diplomatic stasis
and ideological retrenchment. He lives through periods of anxiety and
horror and yet remains attuned to the fine grain of everyday experience,
as his children grow up, his hair turns gray, and he has to find a new
camera.

In other circumstances Mr. Burnat might fit comfortably into the ranks
of artists who use the medium of digital video for private reflection
and ruminations on the small epiphanies of daily life. And ``5 Broken
Cameras'' deserves to be appreciated for the lyrical delicacy of his
voice and the precision of his eye. That it is almost possible to look
at the film this way --- to foresee a time when it might be understood,
above all, as a film --- may be the only concrete hope Mr. Burnat and
Mr. Davidi have to offer.

This is not to say that the political crisis that unites and separates
them is likely to be resolved any time soon, but rather that, even in
the midst of that crisis, it is more than just politics that needs to be
seen and understood.

Advertisement

\protect\hyperlink{after-bottom}{Continue reading the main story}

\hypertarget{site-index}{%
\subsection{Site Index}\label{site-index}}

\hypertarget{site-information-navigation}{%
\subsection{Site Information
Navigation}\label{site-information-navigation}}

\begin{itemize}
\tightlist
\item
  \href{https://help.nytimes.com/hc/en-us/articles/115014792127-Copyright-notice}{©~2020~The
  New York Times Company}
\end{itemize}

\begin{itemize}
\tightlist
\item
  \href{https://www.nytco.com/}{NYTCo}
\item
  \href{https://help.nytimes.com/hc/en-us/articles/115015385887-Contact-Us}{Contact
  Us}
\item
  \href{https://www.nytco.com/careers/}{Work with us}
\item
  \href{https://nytmediakit.com/}{Advertise}
\item
  \href{http://www.tbrandstudio.com/}{T Brand Studio}
\item
  \href{https://www.nytimes.com/privacy/cookie-policy\#how-do-i-manage-trackers}{Your
  Ad Choices}
\item
  \href{https://www.nytimes.com/privacy}{Privacy}
\item
  \href{https://help.nytimes.com/hc/en-us/articles/115014893428-Terms-of-service}{Terms
  of Service}
\item
  \href{https://help.nytimes.com/hc/en-us/articles/115014893968-Terms-of-sale}{Terms
  of Sale}
\item
  \href{https://spiderbites.nytimes.com}{Site Map}
\item
  \href{https://help.nytimes.com/hc/en-us}{Help}
\item
  \href{https://www.nytimes.com/subscription?campaignId=37WXW}{Subscriptions}
\end{itemize}
