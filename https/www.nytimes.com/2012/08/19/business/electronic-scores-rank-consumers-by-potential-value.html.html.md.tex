Sections

SEARCH

\protect\hyperlink{site-content}{Skip to
content}\protect\hyperlink{site-index}{Skip to site index}

\href{https://www.nytimes.com/section/business}{Business}

\href{https://myaccount.nytimes.com/auth/login?response_type=cookie\&client_id=vi}{}

\href{https://www.nytimes.com/section/todayspaper}{Today's Paper}

\href{/section/business}{Business}\textbar{}Secret E-Scores Chart
Consumers' Buying Power

\url{https://nyti.ms/S5HBzk}

\begin{itemize}
\item
\item
\item
\item
\item
\end{itemize}

Advertisement

\protect\hyperlink{after-top}{Continue reading the main story}

Supported by

\protect\hyperlink{after-sponsor}{Continue reading the main story}

You for Sale

\hypertarget{secret-e-scores-chart-consumers-buying-power}{%
\section{Secret E-Scores Chart Consumers' Buying
Power}\label{secret-e-scores-chart-consumers-buying-power}}

\includegraphics{https://static01.nyt.com/images/2012/08/19/business/19-SCORE/19-SCORE-articleLarge.jpg?quality=75\&auto=webp\&disable=upscale}

By \href{https://www.nytimes.com/by/natasha-singer}{Natasha Singer}

\begin{itemize}
\item
  Aug. 18, 2012
\item
  \begin{itemize}
  \item
  \item
  \item
  \item
  \item
  \end{itemize}
\end{itemize}

ST. CLOUD, Minn.

AMERICANS are obsessed with their scores. Credit scores, G.P.A.'s,
SAT's, blood pressure and cholesterol levels --- you name it.

So here's a new score to obsess about: the e-score, an online
calculation that is assuming an increasingly important, and
controversial, role in e-commerce.

These digital scores, known broadly as consumer valuation or
buying-power scores, measure our potential value as customers. What's
your e-score? You'll probably never know. That's because they are
largely invisible to the public. But they are highly valuable to
companies that want --- or in some cases, don't want --- to have you as
their customer.

Online consumer scores are calculated by a handful of start-ups, as well
as a few financial services stalwarts, that specialize in the
flourishing field of predictive consumer analytics. It is a Google-esque
business, one fueled by almost unimaginable amounts of data and powered
by complex computer algorithms. The result is a private, digital ranking
of American society unlike anything that has come before.

It's true that credit scores, based on personal credit reports, have
been around for decades. And direct marketing companies have long ranked
consumers by their socioeconomic status. But e-scores go further. They
can take into account facts like occupation, salary and home value to
spending on luxury goods or pet food, and do it all with algorithms that
their creators say accurately predict spending.

A growing number of companies, including banks, credit and debit card
providers, insurers and online educational institutions are using these
scores to choose whom to woo on the Web. These scores can determine
whether someone is pitched a platinum credit card or a plain one, a
full-service cable plan or none at all. They can determine whether a
customer is routed promptly to an attentive service agent or relegated
to an overflow call center.

Federal regulators and consumer advocates worry that these scores could
eventually put some consumers at a disadvantage, particularly those
under financial stress. In effect, they say, the scores could create a
new subprime class: people who are bypassed by companies online without
even knowing it. Financial institutions, in particular, might avoid
people with low scores, reducing those people's access to home loans,
credit cards and insurance.

It might seem strange that one innovator in this sphere has blossomed
here in St. Cloud, a world away from the hothouse of Silicon Valley.
\href{http://www.ebureau.com/}{It is called eBureau,} and it develops
eScores --- its name for custom scoring algorithms --- to predict
whether someone is likely to become a customer or a money-loser. Gordy
Meyer, the founder and chief executive, says his system needs less than
a second to size up a consumer and to transmit his or her score to an
eBureau client.

``It's like gambling,'' Mr. Meyer says. ``It's a game of odds, when to
double down and when to pass.''

Every month, eBureau scores about 20 million American adults on behalf
of clients like banks, payday lenders and insurers, looking to buy the
names of prospective customers. \href{http://www.tru-signal.com/}{An
eBureau spinoff called TruSignal}, also located here, scores about 110
million consumers monthly for advertisers seeking select audiences for
online ads. Mr. Meyer says eBureau's clients use the scores to answer
basic business questions about their potential audience.

``Are they legitimate?'' Mr. Meyer asks. ``Are they worth pursuing? Are
they worth spending money on?'' The scores, he adds, are generated
without using federally regulated consumer data and are not used to make
credit decisions about consumers. (Using regulated credit data for
marketing purposes could run afoul of federal law.)

Such assurances aside, consumer value scores have begun to trouble some
federal regulators. One of their worries is that these scores, which
have spread quietly through American business, measure individuals
against one another, using yardsticks that are essentially secret.
Another is that the scores could pigeonhole people, limit their
financial choices and channel some into predatory loans, they say.

``The scoring is a tool to enable financial institutions to make
decisions about financing based on unconventional methods,'' says David
Vladeck, the director of the
\href{http://www.ftc.gov/bcp/index.shtml}{bureau of consumer protection
at the Federal Trade Commission.} ``We are troubled by these
practices.''

Federal law governs the use of old-fashioned credit scores. Companies
must have a legally permissible purpose before checking consumers'
credit reports and must alert them if they are denied credit or
insurance based on information in those reports. But the law does not
extend to the new valuation scores because they are derived from
nontraditional data and promoted for marketing.

Ed Mierzwinski, consumer program director at the
\href{http://www.uspirg.org/}{United States Public Interest Research
Group} in Washington, worries that federal laws haven't kept pace with
change in the digital age.

``There's a nontransparent, opaque scoring system that collects
information about you to generate a score --- and what your score is
results in the offers you get on the Internet,'' he says. ``In most
cases, you don't know who is collecting the information, you don't know
what predictions they have made about you, or the potential for being
denied choice or paying too much.''

ON the ground floor of eBureau's headquarters are the company's prized
assets: several hundred computer processors that analyze billions of
details about consumers every month. EBureau has built a glass enclosure
on a raised platform to showcase the machines. From the dimly lit
viewing hall, tiny green and blue lights flicker behind glass.

Like many facets of eBureau, the idea of putting the processors on a
pedestal came from Mr. Meyer, 51, whose relaxed uniform of jeans and
cotton shirts belies the methodological decider underneath.

\includegraphics{https://static01.nyt.com/images/2012/08/19/business/19-SCORE-JP1/19-SCORE-JP1-jumbo.jpg?quality=75\&auto=webp\&disable=upscale}

``In this business, it's intangible. It's data through wires,'' Mr.
Meyer says, walking the perimeter of the enclosure. To help clients
relate to his business, he says, he tried to make his data center appear
technological, reliable and safe. ``We wanted it to feel like it's a
bunker.''

It is actually no coincidence that one of the country's leading
consumer-scoring companies is located here, in this former
granite-mining town about a 90-minute drive northwest of Minneapolis.
Mr. Meyer, a Minnesota native, learned the scoring principles that
underlie eBureau decades ago by working at another local company.

Nearly 30 years ago, he says he took a job as a ``lowly number
cruncher'' at Fingerhut, a general merchandise catalog company based
near Minneapolis. A pioneer in customer analytics, Fingerhut specialized
in marketing to mid- and low-income customers, offering consumer
electronics and other items for sale on monthly installment plans. At
the time Mr. Meyer worked there, he says, many Fingerhut customers had
little to no credit history.

``Traditional ways to evaluate credit didn't exist on half of them,'' he
recalls. ``So Fingerhut had to come up with a way to decide who they
mailed catalogs to and who they ultimately approved orders to.''

Back then, he says, Fingerhut evaluated creditworthiness based in part
on how people filled out order forms. Those who used pens were seen as
safer bets than those who used pencils. People who used a middle initial
were considered better credit risks than those who didn't. After an
analysis by Mr. Meyer, he says, the company also began scoring
first-time customers based on whether their phones were connected and
their phone numbers legitimate. (Those whose phones did not work were
considered at high risk of defaulting on payments.)

Using these different scoring techniques, Mr. Meyer says, Fingerhut
could efficiently tailor its catalogs and offers to different customers;
decide whether to approve or decline certain product orders; or choose
which customer debts to collect on or write off.

``Without Fingerhut,'' Mr. Meyer says, ``I would never be in this
business.''(\href{http://www.fingerhut.com/}{Fingerhut is now an online
and catalog retailer}.)

In the 1990s, Mr. Meyer decided to use his expertise in spotting
patterns of fraud to start RiskWise, an analytics enterprise of his own.
After selling it, and two other companies, to
\href{http://www.lexisnexis.com/en-us/home.page}{LexisNexis} in 2000 for
about \$89 million, he founded another start-up: a predictive analytics
company that would become eBureau.

EVERY business needs customers. But how do you find them, and how do you
know they will be good ones? In 2006, Mr. Meyer began to answer that
question by carving a niche for himself in a nascent online industry
called ``lead generation.''

Lead generators are companies that set up consumer-friendly Web sites
with the goal of funneling potential customers to businesses ranging
from financial institutions to wedding photographers. It is a
multibillion-dollar industry in the United States, says Jay Weintraub,
chief executive
\href{http://www.leadscon.com/?gclid=CK-3gtXB57ECFcHd4AodDXYAng}{of
LeadsCon, a conference} for Web sites that specializes in online
customer acquisition.

Lead-generation sites like Bankrate.com, for example, offer rate
calculators and other tools that prompt people to fill out forms with
their names and contact information. The sites then transmit those
consumers' information to mortgage brokers, credit card issuers, car
insurers and the like, offering access to these prospective customers,
or leads, in return for a finder's fee. The price varies. Lead
generators may charge \$8 for an insurance prospect; \$35 for a finance
lead; or \$75 for a mortgage prospect, Mr. Meyer says.

But, he says, some companies were buying more than 100,000 leads a month
without being able to distinguish one from another. They couldn't sort
potentially profitable customers from window-shoppers and fakes.

``Are people who are filling out the forms telling the truth? Because
Yogi Bear and Fred Flintstone don't buy a lot of stuff,'' Mr. Meyer
says. ``Companies needed to figure out whether these leads were quality
or not.''

Big national and international brands, Mr. Meyer knew, already employed
data analytics to rate consumers. To distinguish his firm, he developed
eBureau to offer customized scoring systems to midsize companies.

Here's how eScores work:

A client submits a data set containing names of tens of thousands of
sales leads it has already bought, along with the names of leads who
went on to become customers. EBureau then adds several thousand details
--- like age, income, occupation, property value, length of residence
and retail history --- from its databases to each customer profile. From
those raw data points, the system extrapolates up to 50,000 additional
variables per person. Then it scours all that data for the rare common
factors among the existing customer base. The resulting algorithm scores
prospective customers based on their resemblance to previous customers.

EScores might range from 0 to 99, with 99 indicating a consumer who is a
likely return on investment and 0 indicating an unprofitable one. But in
some industries, ``knowing the bottom is more important than knowing the
top,'' Mr. Meyer says. In online education, for instance, scores help
schools winnow prospective students who are not worth the investment of
expensive course catalogs or attentive follow-up calls --- like people
who use fake names or adopt the identities of relatives.

``If we can find 25 percent who have zero chance of enrolling, we can
say `don't waste your money on them,'~'' he says.

EBureau charges clients 3 to 75 cents a score, depending on the industry
and the volume of leads.

Such scores increase the accuracy and speed with which companies can
identify potential customers, says Mr. Weintraub of the LeadsCon
conference.

Image

At the eBureau headquarters in St. Cloud, Minn., computer processors
analyze billions of details about American customers
monthly.Credit...Tim Gruber for The New York Times

``Scores tell you `this person might actually qualify, so let's focus on
them,'~'' he says. ``This way you are not focusing on people who really
can't qualify.''

MOST people never see their value scores. But some services openly
discuss how their measurements work.
\href{http://www.ebureau.com/sites/default/files/file/ebureau_prepaid_debtcard_provider.pdf}{A
case study on the eBureau site}, for example, describes how the company
ranked prospective customers for a national prepaid debit card issuer,
assigning each a score of 0 to 998. People who scored above 950 were
considered likely to become highly profitable customers, generating
revenue over six months of an estimated \$213 per card. Those who scored
less than 550 were predicted to be unprofitable clients, with estimated
revenue of \$74 or less. With e-Bureau's system, the card issuer could
identify and court the high scorers while avoiding low scorers.

\href{http://www.targusinfo.com/}{TargusInfo}, a subsidiary of Neustar
that is an eBureau competitor, is even more explicit about how a
multinational credit card issuer used its scores.

\href{http://www.targusinfo.com/files/PDF/case_studies/CreditCardIssuer.pdf}{According
to a case study on its site}, TargusInfo instantly scores prospective
customers who call the card company's call centers, selecting the kind
of card to offer even before an agent picks up the phone. The scores
also alert agents to high-value prospects, people ``who are more likely
to apply, be approved, request supplemental cards or spend more in their
first year,'' the case study says. While high-value callers are
immediately routed to dedicated agents, it says, ``less-qualified
callers no longer waste the valuable time of the card issuer's dedicated
agents and are routed to an outsourced overflow call center.''

Becky Burr,
\href{http://www.targusinfo.com/solutions/adadvisor/how-it-works/privacy-by-design/}{chief
privacy officer of Neustar}, sees TargusInfo's scoring system as a
modern incarnation of marketing services to help companies find~and
communicate with their audiences.

``They want to allocate their marketing money efficiently, and consumers
want messages that are relevant,'' she says. The scores, she adds,
should be seen as predictions about groups of consumers, not judgments
on individuals.

For companies, this kind of scoring clearly increases the speed and
reduces the cost of acquiring customers. But consumers are paying a
heavy price for that increased corporate efficiency, public interests
advocates say.

The digital scores create a two-tiered system that invisibly prioritizes
some online users for credit and insurance offers while denying the same
opportunities to others, says Mr. Mierzwinski of the Public Interest
Research Group. The decades-old federal law that protects consumers from
unfair credit practices, he says, has not kept pace with online
innovation.

The Fair Credit Reporting Act requires that consumer reporting agencies,
the companies that compile credit data, show people their credit reports
and allow them to correct errors. Companies that use the reports must
notify consumers if they take adverse action based on information in
those reports. But digital marketers, Mr. Mierzwinski says, are able to
work around the rules by using alternative financial data to calculate
consumer scores. In an article scheduled to be published next spring in
the
\href{http://www.law.suffolk.edu/highlights/stuorgs/lawreview/index.cfm}{Suffolk
University Law Review}, Mr. Mierzwinski and a co-author argue that new
digital techniques like scoring let sales agents rapidly convert online
prospects to customers, blurring the line between marketing and actual
credit offers.

``The relationship between marketing and making a distinct offer of
credit to a consumer is becoming blurred given contemporary digital
marketing practices,'' Mr. Mierzwinski and his co-author, Jeffrey
Chester of the \href{http://www.democraticmedia.org/}{Center for Digital
Democracy}, write in the article. Federal regulators, they add, ``should
ensure consumers know whether and how they have been secretly scored or
rated by the digital financial marketers, especially those labeled as
less profitable or desirable.''

Mr. Meyer and other eBureau executives disagree, saying the concerns are
misplaced.

EBureau, Mr. Meyer says, went to great lengths to build a system with
both regulatory requirements and consumer privacy in mind. The company,
he says, has put firewalls in place to separate databases containing
federally regulated data, like credit or debt information used for
purposes like risk management, from databases about consumers used to
generate scores for marketing purposes.

He adds that eBureau's clients use the scores only to narrow their field
of prospective customers --- not for the purposes of approving people
for credit, loans or insurance. Moreover, he says, the company does not
sell consumer data to others, nor does it retain the scores it transmits
to clients.

``We are an evaluator,'' Mr. Meyer says. ``We are trying to stay away
from being intrusive to the consumer.''

AT a LeadsCon conference in Midtown Manhattan last month, eBureau was
among those making its sales pitch. Its exhibition booth depicted a
multiethnic group of fictional consumers and their hypothetical scores.

Score boxes superimposed over a young African-American male read
variously: ``eScore: 811, high lifetime value potential'' and ``eScore:
524, underbanked, but safe credit risk.'' Another caption floating over
the crowd read: ``eScore: 906, route to best call center agent NOW!''

It's just another sign of the rise of what might be called the Scored
Society. Google ranks our search results by our location and search
history. Facebook scores us based on our online activities.
\href{http://klout.com/understand/score}{Klout scores us} by how many
followers we have on Twitter, among other things.

And now e-scores rank our potential value to companies.

But the spread of consumer rankings raises deep questions of fairness,
says
\href{http://law.shu.edu/Faculty/display-profile.cfm?customel_datapageid_4018=22642}{Frank
Pasquale, a professor at Seton Hall University School of Law}, who is
writing a book about scoring technologies. The scores may help
companies, he says. But over time, they may send some consumers into a
downward spiral, locking them into a world of digital disadvantage.

``I'm troubled by the idea that some people will essentially be seeing
ads for subprime loans, vocational schools and payday loans,'' Professor
Pasquale says, ``while others might be seeing ads for regular banks and
colleges, and not know why.''

Advertisement

\protect\hyperlink{after-bottom}{Continue reading the main story}

\hypertarget{site-index}{%
\subsection{Site Index}\label{site-index}}

\hypertarget{site-information-navigation}{%
\subsection{Site Information
Navigation}\label{site-information-navigation}}

\begin{itemize}
\tightlist
\item
  \href{https://help.nytimes.com/hc/en-us/articles/115014792127-Copyright-notice}{©~2020~The
  New York Times Company}
\end{itemize}

\begin{itemize}
\tightlist
\item
  \href{https://www.nytco.com/}{NYTCo}
\item
  \href{https://help.nytimes.com/hc/en-us/articles/115015385887-Contact-Us}{Contact
  Us}
\item
  \href{https://www.nytco.com/careers/}{Work with us}
\item
  \href{https://nytmediakit.com/}{Advertise}
\item
  \href{http://www.tbrandstudio.com/}{T Brand Studio}
\item
  \href{https://www.nytimes.com/privacy/cookie-policy\#how-do-i-manage-trackers}{Your
  Ad Choices}
\item
  \href{https://www.nytimes.com/privacy}{Privacy}
\item
  \href{https://help.nytimes.com/hc/en-us/articles/115014893428-Terms-of-service}{Terms
  of Service}
\item
  \href{https://help.nytimes.com/hc/en-us/articles/115014893968-Terms-of-sale}{Terms
  of Sale}
\item
  \href{https://spiderbites.nytimes.com}{Site Map}
\item
  \href{https://help.nytimes.com/hc/en-us}{Help}
\item
  \href{https://www.nytimes.com/subscription?campaignId=37WXW}{Subscriptions}
\end{itemize}
