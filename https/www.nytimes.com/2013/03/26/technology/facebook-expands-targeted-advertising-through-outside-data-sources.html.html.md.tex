Sections

SEARCH

\protect\hyperlink{site-content}{Skip to
content}\protect\hyperlink{site-index}{Skip to site index}

\href{https://www.nytimes.com/section/technology}{Technology}

\href{https://myaccount.nytimes.com/auth/login?response_type=cookie\&client_id=vi}{}

\href{https://www.nytimes.com/section/todayspaper}{Today's Paper}

\href{/section/technology}{Technology}\textbar{}What You Didn't Post,
Facebook May Still Know

\url{https://nyti.ms/14jUjAw}

\begin{itemize}
\item
\item
\item
\item
\item
\end{itemize}

Advertisement

\protect\hyperlink{after-top}{Continue reading the main story}

Supported by

\protect\hyperlink{after-sponsor}{Continue reading the main story}

\hypertarget{what-you-didnt-post-facebook-may-still-know}{%
\section{What You Didn't Post, Facebook May Still
Know}\label{what-you-didnt-post-facebook-may-still-know}}

By \href{https://www.nytimes.com/by/somini-sengupta}{Somini Sengupta}

\begin{itemize}
\item
  March 25, 2013
\item
  \begin{itemize}
  \item
  \item
  \item
  \item
  \item
  \end{itemize}
\end{itemize}

SAN FRANCISCO --- Debra Aho Williamson, an advertising industry analyst
and devoted coffee drinker, was intrigued by a promotion that popped up
on her Facebook page recently. Sign up for a Starbucks loyalty card, it
said, and get \$5 off.

``When I saw that, I thought, I'm already a member of their loyalty
club,'' she said. ``Why don't they know that?''

Despite the streams of data Facebook has collected about people like Ms.
Williamson, the social network needs to know its users much better if it
is going to become, as the company hopes, the Web's most effective
advertising platform. And Facebook is scrambling to do just that.

In shaping its targeted advertising strategy, it is no longer relying
solely on what Facebook users reveal about themselves. Instead, it is
tapping into outside sources of data to learn even more about them ---
and to sell ads that are more finely targeted to them. Facebook says
that this way, marketers will be able to reach the right audience for
the right products, and consumers will see advertisements that are, as
the company calls it, ``relevant'' to them.

In late February, Facebook announced partnerships with four companies
that collect lucrative behavioral data, from store loyalty card
transactions and customer e-mail lists to divorce and Web browsing
records.

They include Acxiom, which aggregates data from a variety of sources,
including financial services companies, court records and federal
government documents; Datalogix, which claims to have a database on the
spending habits of more than 100 million Americans in categories like
fine jewelry, cough medicine and college tuition; and Epsilon, which
also collects transaction data from retailers.

Acxiom and Datalogix are
\href{http://www.ftc.gov/opa/2012/12/databrokers.shtm}{among nine
companies} that the Federal Trade Commission
\href{http://www.nytimes.com/2012/07/25/technology/congress-opens-inquiry-into-data-brokers.html}{is
investigating} to see how they collect and use consumer data.

Facebook's fourth partner is BlueKai, based in Cupertino, Calif., which
creates tracking cookies for brands to monitor customers who visit their
Web sites. That data can be used to show an advertisement when those
users log on to Facebook.

Image

Facebook is tapping into outside data sources to learn even more about
its users' habits and preferences.

``Our goal is to improve the relevance of ads people see on Facebook and
the efficacy of marketing campaigns,'' Gokul Rajaram, product director
for ads at Facebook, said in an interview on Friday.

In
\href{http://www.facebook-studio.com/news/item/new-ways-to-reach-the-right-audience}{announcing
the partnerships}, Facebook said it would allow, for instance, a
carmaker to customize an advertisement to users interested in a new car.

The push to refine targeted advertising reflects the company's need to
increase its revenue. Its shares are worth far less than its ambitious
initial public offering price of \$38 a share last May, and Wall Street
wants to see it take concrete steps to prove to advertisers that it can
show the right promotions to the right users and turn them into
customers.

The partnerships are part of a continuum of efforts by Facebook to hone
targeted advertising. Last fall, it invited potential advertisers to
provide the e-mail addresses of their customers; Facebook then found
those customers among its users and showed them ads on behalf of the
brands.

JackThreads, a members-only online men's retailer, tried this tactic
recently. Of the two million customer e-mails it had on file, Facebook
found more than two-thirds of them on the social network, aided in part
by the fact that JackThreads allows members to sign in using Facebook
login credentials. Facebook then showed those customers ads for the
items they had once eyed on the JackThreads site.

The nudge seemed to get people to open up their pocketbooks. Sales
increased 26 percent at JackThreads, according to AdParlor, an agency
that buys the company's advertisements on Facebook.

Targeted advertising bears important implications for consumers. It
could mean seeing advertisements based not just on what they ``like'' on
Facebook, but on what they eat for breakfast, whether they buy khakis or
jeans and whether they are more likely to give their wives roses or
tulips on their wedding anniversary. It means that even things people
don't reveal on Facebook may be discovered from their online and offline
proclivities.

Facebook says that in devising targeted ads, no identifying information
about users is shared with advertisers. E-mail addresses and Facebook
user names are encrypted and then matched. Users can
\href{https://www.facebook.com/notes/facebook-and-privacy/ad-targeting-and-your-privacy-keeping-you-informed-on-ad-targeting-updates/517330291650191)}{opt
out} of seeing specific brand advertisements on their page, and they can
opt out of receiving any targeted messages by visiting each third-party
data partner's Web site.

That is a somewhat complicated process, though, which has prompted the
Electronic Frontier Foundation to issue step-by-step
\href{https://www.eff.org/deeplinks/2013/02/howto-opt-out-databrokers-showing-your-targeted-advertisements-facebook}{instructions}.
The foundation suggests that consumers who want to avoid ubiquitous
tracking install tools to block Web trackers and be mindful about
sharing their e-mail addresses with marketers. Facebook declined to
provide data on how often users opt out of seeing ads.

``It's ultimately good for the users,'' Mr. Rajaram said. ``They get to
see better, more relevant ads from brands and businesses they care about
and that they have a prior relationship with.''

Image

An example of a targeted ad that has appeared on Facebook, which last
month partnered with four companies that collect behavioral data from
store loyalty card transactions and other sources.

He added, ``There is no information on users that's being shared that
they haven't shared already.''

Whether Facebook users will enjoy seeing ``relevant'' ads or be
alienated by more intensive tracking remains to be seen.

At the very least, said Ms. Williamson, an analyst with the research
firm eMarketer, consumers will be ``forced to become more aware of the
data trail they leave behind them and how companies are putting all that
data together in new ways to reach them.'' She knows, for instance, that
if she uses her supermarket loyalty card to buy cornflakes, she can
expect to see a cornflakes advertisement when she logs in to Facebook.

After all, she said, ``data is the new currency of marketing.''

These efforts speak volumes about the data trail that consumers leave
every day, online and off --- a trail that can follow them back to
Facebook or to any other advertising platform on the Web. They offer
lucrative information every time they provide their e-mail address to a
dressmaker or a doctor, and even when they give their ZIP code at the
checkout counter. They use loyalty cards to buy snorkeling gear or
antidepressants. They browse a retail Web site, leaving a detailed
portrait of whether they are interested in ergonomic work chairs or
nursery furniture.

Facebook said it was too early to reveal details about how the data
collected through its new partnerships would be put to use by marketers.

1-800-Flowers, the online florist, said it had been experimenting with
targeted ads on Facebook. What the company was most looking forward to
was a new advertising conceit, which Facebook calls Lookalike, that
would allow 1-800-Flowers to show its ads to other Facebook users who
are similar to the company's known customers.

Christopher G. McCann, president of 1-800-Flowers, said he had no idea
how Facebook planned to identify ``look-alikes,'' only that it had
promised to find potential new customers through a proprietary algorithm
that matches demographic traits.

Last year, Facebook also introduced a so-called retargeting campaign. A
travel Web site could track what its customers were looking at ---
hotels in New York, for instance --- and show those customers an ad once
they logged on to Facebook. The tracking is done by a piece of code
embedded in the travel company's site.

For marketers, more data could mean getting closer to the ultimate goal
of advertising: sending the right message to the right consumer at the
right time.

When Facebook announced its targeted ad offerings, Justin Bazan, an
optometrist in Park Slope, Brooklyn, immediately saw an opportunity for
his business. He combed through his office records for the e-mail
addresses of patients who were overdue for an annual exam. Facebook
matched most of those e-mails to Facebook user names, and Dr. Bazan paid
\$50 to show those users an advertisement. ``You're overdue,'' the ad
read. ``Click here to make an appointment.''

Within a week, more than 50 people had clicked on his ad, he said.

Dr. Bazan dismissed concerns about federal confidentiality laws that
protect health information. Facebook, he said, encrypts the e-mail
addresses furnished by any advertiser, including doctors.

Advertisement

\protect\hyperlink{after-bottom}{Continue reading the main story}

\hypertarget{site-index}{%
\subsection{Site Index}\label{site-index}}

\hypertarget{site-information-navigation}{%
\subsection{Site Information
Navigation}\label{site-information-navigation}}

\begin{itemize}
\tightlist
\item
  \href{https://help.nytimes.com/hc/en-us/articles/115014792127-Copyright-notice}{©~2020~The
  New York Times Company}
\end{itemize}

\begin{itemize}
\tightlist
\item
  \href{https://www.nytco.com/}{NYTCo}
\item
  \href{https://help.nytimes.com/hc/en-us/articles/115015385887-Contact-Us}{Contact
  Us}
\item
  \href{https://www.nytco.com/careers/}{Work with us}
\item
  \href{https://nytmediakit.com/}{Advertise}
\item
  \href{http://www.tbrandstudio.com/}{T Brand Studio}
\item
  \href{https://www.nytimes.com/privacy/cookie-policy\#how-do-i-manage-trackers}{Your
  Ad Choices}
\item
  \href{https://www.nytimes.com/privacy}{Privacy}
\item
  \href{https://help.nytimes.com/hc/en-us/articles/115014893428-Terms-of-service}{Terms
  of Service}
\item
  \href{https://help.nytimes.com/hc/en-us/articles/115014893968-Terms-of-sale}{Terms
  of Sale}
\item
  \href{https://spiderbites.nytimes.com}{Site Map}
\item
  \href{https://help.nytimes.com/hc/en-us}{Help}
\item
  \href{https://www.nytimes.com/subscription?campaignId=37WXW}{Subscriptions}
\end{itemize}
