Sections

SEARCH

\protect\hyperlink{site-content}{Skip to
content}\protect\hyperlink{site-index}{Skip to site index}

\href{https://www.nytimes.com/section/world/asia}{Asia Pacific}

\href{https://myaccount.nytimes.com/auth/login?response_type=cookie\&client_id=vi}{}

\href{https://www.nytimes.com/section/todayspaper}{Today's Paper}

\href{/section/world/asia}{Asia Pacific}\textbar{}China Takes Aim at
Western Ideas

\url{https://nyti.ms/170iBzR}

\begin{itemize}
\item
\item
\item
\item
\item
\item
\end{itemize}

Advertisement

\protect\hyperlink{after-top}{Continue reading the main story}

Supported by

\protect\hyperlink{after-sponsor}{Continue reading the main story}

\hypertarget{china-takes-aim-at-western-ideas}{%
\section{China Takes Aim at Western
Ideas}\label{china-takes-aim-at-western-ideas}}

\includegraphics{https://static01.nyt.com/images/2013/08/20/world/JP-DOCUMENT/JP-DOCUMENT-articleLarge.jpg?quality=75\&auto=webp\&disable=upscale}

By \href{https://www.nytimes.com/by/chris-buckley}{Chris Buckley}

\begin{itemize}
\item
  Aug. 19, 2013
\item
  \begin{itemize}
  \item
  \item
  \item
  \item
  \item
  \item
  \end{itemize}
\end{itemize}

HONG KONG --- Communist Party cadres have filled meeting halls around
China to hear a somber, secretive warning issued by senior leaders.
Power could escape their grip, they have been told, unless the party
eradicates seven subversive currents coursing through Chinese society.

These seven perils were enumerated in a memo, referred to as Document
No. 9, that bears the unmistakable imprimatur of Xi Jinping, China's new
top leader. The first was ``Western constitutional democracy''; others
included promoting ``universal values'' of human rights,
Western-inspired notions of media independence and civic participation,
ardently pro-market ``neo-liberalism,'' and ``nihilist'' criticisms of
the party's traumatic past.

Even as Mr. Xi has sought to prepare some reforms to expose China's
economy to stronger market forces, he has undertaken a ``mass line''
campaign to enforce party authority that goes beyond the party's
periodic calls for discipline. The internal warnings to cadres show that
Mr. Xi's confident public face has been accompanied by fears that the
party is vulnerable to an economic slowdown, public anger about
corruption and challenges from liberals impatient for political change.

``Western forces hostile to China and dissidents within the country are
still constantly infiltrating the ideological sphere,'' says Document
No. 9, the number given to it by the central party office that issued it
in April. It has not been openly published, but a version was shown to
The New York Times and was verified by four sources close to senior
officials, including an editor with a party newspaper.

Opponents of one-party rule, it says, ``have stirred up trouble about
disclosing officials' assets, using the Internet to fight corruption,
media controls and other sensitive topics, to provoke discontent with
the party and government.''

The warnings were not idle. Since the circular was issued, party-run
publications and Web sites have vehemently denounced constitutionalism
and civil society, notions that were not considered off limits in recent
years. Officials have intensified efforts to block access to critical
views on the Internet. Two prominent rights advocates have been detained
in the past few weeks, in what their supporters have called a blow to
the ``rights defense movement,'' which was already beleaguered under Mr.
Xi's predecessor, Hu Jintao.

Mr. Xi's hard line has disappointed Chinese liberals, some of whom once
hailed his rise to power as an opportunity to push for political change
after a long period of stagnation. Instead, Mr. Xi has signaled a shift
to a more conservative, traditional leftist stance with his
``rectification'' campaign to ensure discipline and conspicuous attempts
to defend the legacy of Mao Zedong. That has included a visit to a
historic site where Mao undertook one of his own attempts to remake the
ruling party in the 1950s.

Mr. Xi's edicts have been disseminated in a series of compulsory study
sessions across the country, like one in the southern province of Hunan
that was recounted on a local government Web site.

``Promotion of Western constitutional democracy is an attempt to negate
the party's leadership,'' Cheng Xinping, a deputy head of propaganda for
Hengyang, a city in Hunan, told a gathering of mining industry
officials. Human rights advocates, he continued, want ``ultimately to
form a force for political confrontation.''

The campaign carries some risks for Mr. Xi, who has indicated that the
slowing economy needs new, more market-driven momentum that can come
only from a relaxation of state influence.

In China's tight but often contentious political circles, proponents of
deeper Western-style economic changes are often allied with those
pushing for rule of law and a more open political system, while
traditionalists favor greater state control of both economic and
political life. Mr. Xi's cherry picking of approaches from each of the
rival camps, analysts say, could end up miring his own agenda in
intraparty squabbling.

Condemnations of constitutional government have prompted dismayed
opposition from liberal intellectuals and even some moderate-minded
former officials. The campaign has also exhilarated leftist defenders of
party orthodoxy, many of whom pointedly oppose the sort of market
reforms that Mr. Xi and Prime Minister Li Keqiang have said are needed.

The consequent rifts are unusually open, and they could widen and bog
down Mr. Xi, said Xiao Gongqin, a professor of history at Shanghai
Normal University who is also a prominent proponent of gradual,
party-guided reform.

``Now the leftists feel very excited and elated, while the liberals feel
very discouraged and discontented,'' said Professor Xiao, who said he
was generally sympathetic to Mr. Xi's aims. ``The ramifications are very
serious, because this seriously hurts the broad middle class and
moderate reformers --- entrepreneurs and intellectuals. It's possible
that this situation will get out of control, and that won't help the
political stability that the central leadership stresses.''

The pressures that prompted the party's ideological counteroffensive
spilled onto the streets of Guangzhou, a city in southern China, early
this year. Staff members at the Southern Weekend newspaper there
\href{http://www.nytimes.com/2013/01/07/world/asia/chinese-newspaper-challenges-the-censors.html}{protested}
after a propaganda official rewrote an editorial celebrating
constitutionalism --- the idea that state and party power should be
subject to a supreme law that prevents abuses and protects citizens'
rights.

The confrontation at the newspaper and campaign demanding that officials
disclose their wealth alarmed leaders and helped galvanize them into
issuing Document No. 9, said Professor Xiao, the historian. Indeed,
senior central propaganda officials met to discuss the newspaper
protest, among other issues, and called it a plot to subvert the party,
according to a speech on a party Web site of Lianyungang, a port city in
eastern China.

``Western anti-China forces led by the United States have joined in one
after the other, and colluded with dissidents within the country to make
slanderous attacks on us in the name of so-called press freedom and
constitutional democracy,'' said Zhang Guangdong, a propaganda official
in Lianyungang, citing the conclusions from the meeting of central
propaganda officials. ``They are trying to break through our political
system, and this was a classic example,'' he said of the newspaper
protest.

But Mr. Xi and his colleagues were victims of expectations that they
themselves encouraged, rather than a foreign conspiracy, analysts said.
The citizen-activists demanding that party officials reveal their family
wealth cited Mr. Xi's own vows to end official corruption and deliver
more candid government. Likewise, scholars and lawyers who have
campaigned for limiting party power under the rule of law have also
invoked Mr. Xi's promise to honor China's Constitution.

Even these relatively measured campaigns proved too much for party
leaders, who are wary of any challenges that could swell into outright
opposition. Document No. 9 was issued by the Central Committee General
Office, the administrative engine room of the central leadership, and
required the approval of Mr. Xi and other top leaders, said Li Weidong,
a political commentator and former magazine editor in Beijing.

``There's no doubt then it had direct endorsement from Xi Jinping,'' Mr.
Li said. ``It's certainly had his approval and reflects his general
views.''

Since the document was issued, the campaign for ideological orthodoxy
has prompted a torrent of commentary and articles in party-run
periodicals. Many of them have invoked Maoist talk of class war rarely
seen in official publications in recent years. Some have said that
constitutionalism and similar ideas were tools of Western subversion
that helped topple the former Soviet Union --- and that a similar threat
faces China.

``Constitutionalism belongs only to capitalism,'' said one commentary in
the overseas edition of the People's Daily. Constitutionalism ``is a
weapon for information and psychological warfare used by the magnates of
American monopoly capitalism and their proxies in China to subvert
China's socialist system,'' said another commentary in the paper.

But leftists, feeling emboldened, could create trouble for Mr. Xi's
government, some analysts said. Mr. Xi has indicated that he wants a
party meeting in the fall to endorse policies that would give market
competition and private businesses a bigger role in the economy --- and
Marxist stalwarts in the party are deeply wary of such proposals.

Relatively liberal officials and intellectuals hoped the ousting last
year of Bo Xilai, a charismatic politician who favored leftist policies,
would help their cause. But they have been disappointed. Mr. Bo goes on
trial on Thursday.

Hu Deping, a reform-minded former government official who has met Mr.
Xi, recently issued a public warning about the leftward drift. ``Just
what is the bottom line for reform?'' Mr. Hu said on a Web site run by
his family to commemorate his father, Hu Yaobang, a leader of political
and economic relaxation in the 1980s.

Mr. Xi will face another ideological test later in the year when the
Communist Party celebrates the 120th anniversary of Mao's birth. The
scale of those celebrations has not been announced. But Xiangtan, the
area in Hunan Province that encompasses Mao's hometown, is spending \$1
billion to spruce up commemorative sites and facilities for the
occasion, according to the Xiangtan government Web site.

``You have to commemorate him, and because he's already passed away, you
can only speak well of him, not ill,'' Professor Xiao, the historian,
said of Mao's anniversary. ``That's like pouring petrol on the leftists'
fire.''

Advertisement

\protect\hyperlink{after-bottom}{Continue reading the main story}

\hypertarget{site-index}{%
\subsection{Site Index}\label{site-index}}

\hypertarget{site-information-navigation}{%
\subsection{Site Information
Navigation}\label{site-information-navigation}}

\begin{itemize}
\tightlist
\item
  \href{https://help.nytimes.com/hc/en-us/articles/115014792127-Copyright-notice}{©~2020~The
  New York Times Company}
\end{itemize}

\begin{itemize}
\tightlist
\item
  \href{https://www.nytco.com/}{NYTCo}
\item
  \href{https://help.nytimes.com/hc/en-us/articles/115015385887-Contact-Us}{Contact
  Us}
\item
  \href{https://www.nytco.com/careers/}{Work with us}
\item
  \href{https://nytmediakit.com/}{Advertise}
\item
  \href{http://www.tbrandstudio.com/}{T Brand Studio}
\item
  \href{https://www.nytimes.com/privacy/cookie-policy\#how-do-i-manage-trackers}{Your
  Ad Choices}
\item
  \href{https://www.nytimes.com/privacy}{Privacy}
\item
  \href{https://help.nytimes.com/hc/en-us/articles/115014893428-Terms-of-service}{Terms
  of Service}
\item
  \href{https://help.nytimes.com/hc/en-us/articles/115014893968-Terms-of-sale}{Terms
  of Sale}
\item
  \href{https://spiderbites.nytimes.com}{Site Map}
\item
  \href{https://help.nytimes.com/hc/en-us}{Help}
\item
  \href{https://www.nytimes.com/subscription?campaignId=37WXW}{Subscriptions}
\end{itemize}
