Sections

SEARCH

\protect\hyperlink{site-content}{Skip to
content}\protect\hyperlink{site-index}{Skip to site index}

\href{https://www.nytimes.com/section/sports/football}{Pro Football}

\href{https://myaccount.nytimes.com/auth/login?response_type=cookie\&client_id=vi}{}

\href{https://www.nytimes.com/section/todayspaper}{Today's Paper}

\href{/section/sports/football}{Pro Football}\textbar{}Not Quite 75 and
Sunny, but a Mild Day Dispels the Worries About the Weather

\url{https://nyti.ms/1k1CjD3}

\begin{itemize}
\item
\item
\item
\item
\item
\end{itemize}

Advertisement

\protect\hyperlink{after-top}{Continue reading the main story}

Supported by

\protect\hyperlink{after-sponsor}{Continue reading the main story}

\hypertarget{not-quite-75-and-sunny-but-a-mild-day-dispels-the-worries-about-the-weather}{%
\section{Not Quite 75 and Sunny, but a Mild Day Dispels the Worries
About the
Weather}\label{not-quite-75-and-sunny-but-a-mild-day-dispels-the-worries-about-the-weather}}

\includegraphics{https://static01.nyt.com/images/2014/02/03/sports/WEATHER/WEATHER-articleLarge.jpg?quality=75\&auto=webp\&disable=upscale}

By \href{http://www.nytimes.com/by/ken-belson}{Ken Belson}

\begin{itemize}
\item
  Feb. 2, 2014
\item
  \begin{itemize}
  \item
  \item
  \item
  \item
  \item
  \end{itemize}
\end{itemize}

EAST RUTHERFORD, N.J. --- John Dalena and his son Mike were prepared for
the worst. In the back of their car parked near MetLife Stadium, they
had snow pants, extra sweatshirts and other cold-weather gear to keep
them warm after kickoff at the Super Bowl still three hours away.

But a funny thing happened as they spoke: The sun peaked through the
gray clouds.

``Of course, we were prepared to sit in a snowstorm, but this is
great,'' said John, who lives in Madison, N.J., and wore a Seahawks
jersey. ``I don't think I ever checked the weather reports as much as
for this game.''

After months of predictions of snow, howling wind and a frigid
temperature, the weather on Super Bowl Sunday felt like a breath of
fresh air. No polar vortex. No ice or sleet. No record-setting,
bone-chilling temperatures, just a game-time temperature of 49 degrees
and overcast, with a chance of light rain.

The temperature at kickoff was actually the same as it was before the
first pitch of the sixth and clinching game of the World Series in
Boston last October.

``I took my jacket off because it's too warm,'' said Paul Melkers, a
Seahawks fan from Gig Harbor, Wash.

His friend Garry Horvitz added, ``It's a nice Seattle day.'' From the
moment the N.F.L. chose to hold its first outdoor Super Bowl in a cold
weather region nearly four years ago, weather has been a topic of
continual discussion. It also raised the chance that other cold-weather
cities with outdoor stadiums could be added to the rotation of potential
hosts.

Commissioner Roger Goodell, an artful marketer, reminded fans that some
of the N.F.L.'s classic contests were played in cold weather. He even
poked fun at the worrywarts, with fake snowflakes falling from the
ceiling during his annual state of the league address Friday.

``Of course, we cannot control the weather,'' he said. ``I told you we
were going to embrace the weather; here we go.''

Police and traffic officials assured fans that they were ready to keep
the trains running and the roads clear, if the weather were to turn
wintry. The players shrugged. They came from Denver and Seattle, not
Miami and San Diego.

Instead of chasing tornadoes and tracking hurricanes, weather
forecasters flexed their meteorological muscles when, in fact, a clear
picture of the weather on a single day is rarely known more than a few
days in advance.

``About 10 days ago, the temperature was going to be somewhere between
the teens and the 50s,'' said Elliot Abrams, senior vice president at
Accuweather.com. ``It's good for interest value or entertainment value.
You have to keep your sense of humor.''

The fixation on the weather at the Super Bowl managed to crowd out one
of the year's more prominent weather-related holidays, Groundhog Day.

``It sort of did steal his thunder,'' Abrams said. (For the record,
Punxsutawney Phil predicted six more weeks of winter after seeing his
shadow.)

The dire forecasts leading to the game had a silver lining: falling
ticket prices. Fans who waited until the weekend to buy their tickets
found relative bargains.

``The bad weather they expected was good for me,'' said Ruben Menendez,
who visited from Dallas. ``If everyone knew it was going to be 50
degrees, prices would have been higher.''

The larger question is whether the N.F.L. will commit to another Super
Bowl outdoors in a cold-weather climate. The league granted New York and
New Jersey a waiver from the requirement that the average temperature
during the first week of February be 50 degrees or warmer, said Mark
Lamping, who as the chief executive ~of New Meadowlands Stadium Company
oversaw the construction of MetLife Stadium when the bid to host this
year's Super Bowl was put together. Lamping is now the president of the
Jacksonville Jaguars.

``It was obviously the first thing most people thought of, but the
second thing was that some of the most historic games in the N.F.L. were
played in less-than-ideal weather conditions,'' Lamping said.

Goodell said Friday that this year's game was as much about it being in
the New York metropolitan area than about breaking a weather taboo. It
was a hint that cities like Denver and Seattle, which have outdoor
stadiums and have expressed interest in hosting the game, will need to
do more than just get in line if they hope to convince the league.

But by dodging a weather bullet this week, owners may now be more
receptive to giving another cold-weather city a chance, said Clark Hunt,
the chairman of the Kansas City Chiefs.

``When it comes time to vote, owners will remember how well things went
this week,'' Hunt said Sunday, adding that he envisioned a cold-weather
city with an outdoor stadium being chosen about once every 10 years.

``I think it is really part of football,'' Shad Khan, the owner of the
Jaguars, said about a cold-weather Super Bowl. ``It's going to be part
of the future of this game.''

Warm-weather cities, fearful of being pushed aside as Super Bowl hosts,
have fought back. Steve Ross, the owner of the Dolphins, has tried to
win subsidies from Miami-Dade County to help improve his stadium to
raise the odds that a Super Bowl returns to South Florida.

On Wednesday, the host committee for next year's Super Bowl in Glendale,
Ariz., held a cocktail party where the mantra, shouted loud and often,
was ``75 and sunny,'' a reference to the weather in the Phoenix area.

But the N.F.L. has been dealing with the weather almost since the Super
Bowl began. The coldest Super Bowl was played in 1972 in New Orleans,
before the Superdome was completed, where it was 39 degrees at kickoff.
Atlanta and Dallas, which hosted games in domed stadiums, were hit by
ice storms days before the game, snarling traffic and dampening the
festival that precedes the game.

After a week of brutal cold, Central Park finally had that festival feel
on Sunday afternoon, where it was a balmy 55 degrees. Jawn Chasteen and
his 10-year-old son Alexander were splattered in mud as they threw a
football to each other near the Sheep Meadow.

``We're getting pumped for the game,'' Chasteen said between tosses.
``It's such a random shot to get a nice day in February. We always get
blizzards and stuff. And then it's going to snow again tonight. It's a
perfect little window for the Super Bowl.''

Advertisement

\protect\hyperlink{after-bottom}{Continue reading the main story}

\hypertarget{site-index}{%
\subsection{Site Index}\label{site-index}}

\hypertarget{site-information-navigation}{%
\subsection{Site Information
Navigation}\label{site-information-navigation}}

\begin{itemize}
\tightlist
\item
  \href{https://help.nytimes.com/hc/en-us/articles/115014792127-Copyright-notice}{©~2020~The
  New York Times Company}
\end{itemize}

\begin{itemize}
\tightlist
\item
  \href{https://www.nytco.com/}{NYTCo}
\item
  \href{https://help.nytimes.com/hc/en-us/articles/115015385887-Contact-Us}{Contact
  Us}
\item
  \href{https://www.nytco.com/careers/}{Work with us}
\item
  \href{https://nytmediakit.com/}{Advertise}
\item
  \href{http://www.tbrandstudio.com/}{T Brand Studio}
\item
  \href{https://www.nytimes.com/privacy/cookie-policy\#how-do-i-manage-trackers}{Your
  Ad Choices}
\item
  \href{https://www.nytimes.com/privacy}{Privacy}
\item
  \href{https://help.nytimes.com/hc/en-us/articles/115014893428-Terms-of-service}{Terms
  of Service}
\item
  \href{https://help.nytimes.com/hc/en-us/articles/115014893968-Terms-of-sale}{Terms
  of Sale}
\item
  \href{https://spiderbites.nytimes.com}{Site Map}
\item
  \href{https://help.nytimes.com/hc/en-us}{Help}
\item
  \href{https://www.nytimes.com/subscription?campaignId=37WXW}{Subscriptions}
\end{itemize}
