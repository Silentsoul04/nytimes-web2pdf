Sections

SEARCH

\protect\hyperlink{site-content}{Skip to
content}\protect\hyperlink{site-index}{Skip to site index}

\href{https://www.nytimes.com/section/business}{Business}

\href{https://myaccount.nytimes.com/auth/login?response_type=cookie\&client_id=vi}{}

\href{https://www.nytimes.com/section/todayspaper}{Today's Paper}

\href{/section/business}{Business}\textbar{}Plan to Limit Some Drugs in
Medicare Is Criticized

\url{https://nyti.ms/1bukLg8}

\begin{itemize}
\item
\item
\item
\item
\item
\end{itemize}

Advertisement

\protect\hyperlink{after-top}{Continue reading the main story}

Supported by

\protect\hyperlink{after-sponsor}{Continue reading the main story}

\hypertarget{plan-to-limit-some-drugs-in-medicare-is-criticized}{%
\section{Plan to Limit Some Drugs in Medicare Is
Criticized}\label{plan-to-limit-some-drugs-in-medicare-is-criticized}}

\includegraphics{https://static01.nyt.com/images/2014/02/22/business/22drug/22drug-articleLarge.jpg?quality=75\&auto=webp\&disable=upscale}

By \href{http://www.nytimes.com/by/katie-thomas}{Katie Thomas} and
\href{https://www.nytimes.com/by/robert-pear}{Robert Pear}

\begin{itemize}
\item
  Feb. 21, 2014
\item
  \begin{itemize}
  \item
  \item
  \item
  \item
  \item
  \end{itemize}
\end{itemize}

An alliance of drug companies and patient advocates, joined by Democrats
and Republicans in Congress, is fiercely opposing an Obama
administration proposal that would allow insurers to limit
\href{http://topics.nytimes.com/top/news/health/diseasesconditionsandhealthtopics/medicare/index.html?inline=nyt-classifier}{Medicare}
coverage for certain classes of drugs, including those used to treat
\href{http://health.nytimes.com/health/guides/symptoms/depression/overview.html?inline=nyt-classifier}{depression}
and
\href{http://health.nytimes.com/health/guides/disease/schizophrenia-disorganized-type/overview.html?inline=nyt-classifier}{schizophrenia}.

Opponents warn that the proposal, if enacted, could harm patients.
Federal officials say it would lower costs and reduce overuse of the
drugs.

The
\href{http://www.gpo.gov/fdsys/pkg/FR-2014-01-10/pdf/2013-31497.pdf}{proposed
rule}, which would lift a requirement that insurers cover ``all or
substantially all'' drugs in certain treatment areas, is just one of a
series of changes to the drug program that are being opposed by the
unlikely alliance. Even insurers and drug benefit managers, who have
previously supported added limits on drug coverage, oppose the rule.
They object to provisions including changes to so-called preferred
pharmacy networks, where consumers are steered toward a limited network
of pharmacies, and to reducing the number of plans that insurers can
offer in any one region.

A House subcommittee plans to hold a hearing on the proposal next week,
and the rule is open for public comment until March 7.

``We've been scratching our heads over this,'' said John J. Castellani,
the chief executive of the Pharmaceutical Research and Manufacturers of
America, the drug-industry trade group. Medicare Part D, he noted, is
the rare government program that not only gets high marks from consumers
but also has cost taxpayers billions of dollars less than originally
expected. ``Why is the administration trying to make such extensive
changes to a program that isn't broken?''

Mr. Castellani's organization was one of more than 200 groups that
\href{http://www.hlc.org/blog/wp-content/uploads/2014/02/Comment-Ltr-as-of-2-19.pdf}{signed
a letter} this week asking that the rule be withdrawn. Earlier this
month, Republican and Democratic members of the Senate Finance Committee
warned that the proposal could ``diminish access to needed medication''
without saving much money.

The administration's proposal would remove the protected status from
three classes of drugs that has been in place since the program's
inception in 2006: immunosuppressant drugs used in transplant patients,
\href{http://topics.nytimes.com/top/news/health/diseasesconditionsandhealthtopics/antidepressants/index.html?inline=nyt-classifier}{antidepressants}
and antipsychotic medicines. They include many well-known drugs, such as
Wellbutrin,
\href{http://topics.nytimes.com/top/news/health/diseasesconditionsandhealthtopics/paxil_drug/index.html?inline=nyt-classifier}{Paxil}
and
\href{http://topics.nytimes.com/top/news/health/diseasesconditionsandhealthtopics/prozac_drug/index.html?inline=nyt-classifier}{Prozac}
to treat depression, and Abilify and Seroquel to treat schizophrenia.
Three other categories ---
\href{http://health.nytimes.com/health/guides/disease/cancer/overview.html?inline=nyt-classifier}{cancer},
\href{http://health.nytimes.com/health/guides/disease/aids/overview.html?inline=nyt-classifier}{H.I.V.}
and
anti-\href{http://health.nytimes.com/health/guides/symptoms/seizures/overview.html?inline=nyt-classifier}{seizure}
drugs --- would retain their status as protected classes and insurance
companies would be required to continue covering nearly all drugs in
those treatment areas. Medicare has traditionally required the broad
coverage because patients with these conditions must often try several
drugs before finding one that works.

In proposing the change last month, the administration said that the
policy was envisioned as a temporary measure to help ease patients'
transition to the new Medicare drug program, and that since then,
insurers had lost their leverage in negotiating with drug companies
because the drug companies knew the insurers were required to cover
their drug costs and were therefore less willing to offer lower prices.

In its proposal, the Obama administration
\href{http://amcp.org/WorkArea/DownloadAsset.aspx?id=9279}{cited a 2008
study} by the actuarial and consulting firm Milliman that showed that
the six protected classes accounted for anywhere from 17 to 33 percent
of total outpatient drug spending under Part D of Medicare. In addition,
it said that the costs of those drugs were on average 10 percent higher
than they would be without the requirement to cover substantially all
drugs in these classes.

The administration predicted savings for both beneficiaries and the
Medicare program if prescription drug plans could remove some currently
covered drugs from their formularies. It could also give insurers
additional tools to limit overuse of certain drugs, such as the
prescribing of antipsychotic drugs to nursing-home patients with
\href{http://health.nytimes.com/health/guides/disease/dementia/overview.html?inline=nyt-classifier}{dementia},
a common practice that is widely viewed as inappropriate.

``We believe the Part D program has been a phenomenal success,'' said
Jonathan Blum, principal deputy administrator of the Center for Medicare
and Medicaid Services, which oversees the Part D program. But, he added,
``We also see vulnerabilities in the program, and we have proposed for
public input into ways to improve it.''

Leaders of numerous patient advocacy groups, many of whom met last week
with White House officials to express concern about the proposed rule,
said they were worried that patients could be harmed if the policy
changed.

``The proposal undermines a key protection for some of the sickest, most
vulnerable Medicare beneficiaries,'' said Andrew Sperling, a lobbyist at
the National Alliance on Mental Illness.

Under the proposal, Mr. Sperling said, a Medicare drug plan could have a
list of preferred drugs with just two medications to treat
schizophrenia. That is inadequate, he said, because antipsychotic drugs
work in different ways in the body, and have different side effects.
``You get much better outcomes when a doctor can work with patients to
figure out which medications will work best for them,'' he said.

\href{http://naminc.org/wp-content/uploads/2014/01/SFC-Letter-to-CMS_Protected-Classes.pdf}{In
a letter} written by members of the Senate Finance Committee, the
senators suggested that the change could raise costs in other areas.
``If beneficiaries do not have access to needed medication,'' the letter
said, ``costs will be incurred as a result of unnecessary and avoidable
hospitalizations, physician visits and other medical interventions.''

The new
\href{http://topics.nytimes.com/top/news/health/diseasesconditionsandhealthtopics/health_insurance_and_managed_care/health_care_reform/index.html?inline=nyt-classifier}{federal
health care law} requires that Medicare drug plans include all drugs in
certain categories and classes ``of clinical concern,'' and it
authorized the secretary of health and human services to identify those
categories.

Mr. Sperling said lawmakers had assumed that Medicare officials would
keep the original six protected classes and add to them, not cut them.
The administration proposal sets a high standard for designating
protected classes, saying the drugs must be needed to prevent
``hospitalization, persistent or significant disability or incapacity,
or death'' that would otherwise occur within a week.

Emily Shetty, a lobbyist for the Leukemia and Lymphoma Society, said
Medicare beneficiaries, who include older and disabled Americans, should
be treated with special care. ``They are a more vulnerable patient
population as a whole, and having access to a full range of therapies is
crucial to ensure that they are able to get the care that they need,''
she said.

The Medicare Part D program is unusual in that it requires broad
coverage of drugs in these categories. Commercial insurance plans,
including those in the new marketplaces operating under the federal
health care law, have more flexibility. Some drugs are simply not
covered, and some plans require that patients and doctors go through
additional steps --- such as trying other drugs first, or getting
approval from the insurer --- before a drug will be paid for.

Insurers and the companies that manage their drug benefits argue that
this arrangement has worked well for consumers, ensuring that drugs are
being used properly and helping to keep prices low. But others have
identified what they describe as a worrying trend toward more limited
drug coverage, and higher out-of-pocket costs for the most expensive
drugs.

The rule has some supporters, and many groups back some aspects of the
proposal while opposing others.

``Just because a program is popular doesn't mean that it's being run the
most efficiently, and at the best value for taxpayers and patients,''
said B. Douglas Hoey, chief executive of the National Community
Pharmacists Association, which supports many aspects of the rule.

Advertisement

\protect\hyperlink{after-bottom}{Continue reading the main story}

\hypertarget{site-index}{%
\subsection{Site Index}\label{site-index}}

\hypertarget{site-information-navigation}{%
\subsection{Site Information
Navigation}\label{site-information-navigation}}

\begin{itemize}
\tightlist
\item
  \href{https://help.nytimes.com/hc/en-us/articles/115014792127-Copyright-notice}{©~2020~The
  New York Times Company}
\end{itemize}

\begin{itemize}
\tightlist
\item
  \href{https://www.nytco.com/}{NYTCo}
\item
  \href{https://help.nytimes.com/hc/en-us/articles/115015385887-Contact-Us}{Contact
  Us}
\item
  \href{https://www.nytco.com/careers/}{Work with us}
\item
  \href{https://nytmediakit.com/}{Advertise}
\item
  \href{http://www.tbrandstudio.com/}{T Brand Studio}
\item
  \href{https://www.nytimes.com/privacy/cookie-policy\#how-do-i-manage-trackers}{Your
  Ad Choices}
\item
  \href{https://www.nytimes.com/privacy}{Privacy}
\item
  \href{https://help.nytimes.com/hc/en-us/articles/115014893428-Terms-of-service}{Terms
  of Service}
\item
  \href{https://help.nytimes.com/hc/en-us/articles/115014893968-Terms-of-sale}{Terms
  of Sale}
\item
  \href{https://spiderbites.nytimes.com}{Site Map}
\item
  \href{https://help.nytimes.com/hc/en-us}{Help}
\item
  \href{https://www.nytimes.com/subscription?campaignId=37WXW}{Subscriptions}
\end{itemize}
