Sections

SEARCH

\protect\hyperlink{site-content}{Skip to
content}\protect\hyperlink{site-index}{Skip to site index}

\href{https://www.nytimes.com/section/politics}{Politics}

\href{https://myaccount.nytimes.com/auth/login?response_type=cookie\&client_id=vi}{}

\href{https://www.nytimes.com/section/todayspaper}{Today's Paper}

\href{/section/politics}{Politics}\textbar{}A Republican's Tax Overhaul
Envisions Big Changes

\url{https://nyti.ms/1hpKtBZ}

\begin{itemize}
\item
\item
\item
\item
\item
\end{itemize}

Advertisement

\protect\hyperlink{after-top}{Continue reading the main story}

Supported by

\protect\hyperlink{after-sponsor}{Continue reading the main story}

\hypertarget{a-republicans-tax-overhaul-envisions-big-changes}{%
\section{A Republican's Tax Overhaul Envisions Big
Changes}\label{a-republicans-tax-overhaul-envisions-big-changes}}

\includegraphics{https://static01.nyt.com/images/2014/02/27/us/CONG/CONG-articleLarge.jpg?quality=75\&auto=webp\&disable=upscale}

By \href{http://www.nytimes.com/by/jonathan-weisman}{Jonathan Weisman}

\begin{itemize}
\item
  Feb. 26, 2014
\item
  \begin{itemize}
  \item
  \item
  \item
  \item
  \item
  \end{itemize}
\end{itemize}

WASHINGTON --- Big banks would face a new tax on lending. Taxes paid to
state and local governments would no longer be deductible. The earned
income credit for low-wage workers would be converted to a more limited
deduction on payroll taxes. The mortgage deduction and retirement
savings breaks would be curtailed.

Representative Dave Camp of Michigan, a Republican who is chairman of
the House Ways and Means Committee, unveiled a sweeping overhaul of the
70,000-page federal tax code on Wednesday that would collapse seven
personal income tax brackets to two and lower the corporate rate to 25
percent from 35 percent.

But the seeds of the plan's destruction might be found in the fine
print. When asked about the proposal's details on Wednesday, House
Speaker John A. Boehner replied, ``Blah, blah, blah, blah.''

Faced with that cool reaction from the highest echelons of his party's
leadership, Mr. Camp pleaded: ``You are going to hear a lot about one
provision or another provision, or even another provision. But the truth
is people want a simpler, fairer and flatter code.''

The proposal --- deemed a ``discussion draft'' --- is sweeping, touching
on almost every facet of the tax code. Mr. Camp's goal was to keep all
taxpayers in, at most, a 25 percent bracket. In fact, married households
with incomes over \$450,000 (\$425,000 for singles) would also be
subject to a 10 percent surtax, effectively raising their income tax
rate to 35 percent --- but still lower than the current 39.6 percent.

To get those rate reductions, Mr. Camp would leave behind what Chris
Krueger, a policy analyst at Guggenheim Partners, called a ``veritable
killing field of sacred cows.''

Both rich and poor taxpayers would face big changes. The earned income
credit for low-wage workers --- which, for many, is essentially a check
from the Internal Revenue Service --- would be converted to an exemption
of up to \$4,000 on Social Security and Medicare payroll taxes. That
would raise taxes for some workers, but, Mr. Camp argued, it would
eliminate about \$133 billion in fraud over 10 years.

The rich would be hit as well. The deduction for state and local taxes
costs the Treasury \$402 billion over five years, but much of that comes
from states like New York and California, where the most affluent
taxpayers live.

Senator Charles E. Schumer of New York, the chamber's No. 3 Democrat,
said any proposal that eliminates that deduction ``doesn't have a
snowball's chance of passing.''

Private equity and hedge fund managers would also be prevented from
classifying much of their income as lower-taxed capital gains for tax
purposes, closing what Democrats have criticized as the ``carried
interest loophole.''

Taxpayers now can save up to \$17,500 a year in 401(k) plans, plus more
in traditional Individual Retirement Accounts --- both taxed when the
money is withdrawn. In Roth I.R.A.'s the deposits are already subject to
tax. Under the Camp plan, the cap on I.R.A.'s would be eliminated; any
savings over \$8,750 would have to go into a Roth account. Higher
contributions to 401(k) plans would also be taxed before they were
deposited.

Income from capital gains, which is now taxed at 20 percent, would be
treated as ordinary income, but the first 40 percent of annual capital
gains would face no taxation.

Deductions of mortgage interest in the future would be capped at loans
of \$500,000, half the current cap.

Mr. Camp even embraced an idea that has taken hold in the most
progressive parts of the Democratic Party, a tax on assets of the
biggest banks.

Wall Street objected to that part of the plan. ``We strongly urge policy
makers to reject this arbitrary lending tax and instead place their
focus on achieving pro-growth tax reform with policies that increase
economic growth and expand access to sound lending and credit,'' said
Robert S. Nichols, president of the Financial Services Forum, which
represents financial interests in Washington.

Even the National Federation of Independent Business, the small-business
lobby that has been a stalwart Republican ally and a cheerleader for a
tax overhaul, pronounced itself ``very concerned.''

On the flip side, Representative Lloyd Doggett of Texas, a liberal
Democrat and a senior member of the Ways and Means Committee, complained
that the overhaul would do too little to discourage corporations from
moving operations overseas. Currently, profits earned by United States
corporations overseas are subject to taxes if they are brought back to
the country. The Camp plan would exempt 95 percent of income from
foreign subsidiaries from taxes.

Senator Ron Wyden, Democrat of Oregon and the newly seated chairman of
the Senate Finance Committee, paraphrased former Senator Bill Bradley of
New Jersey, saying that any tax overhaul is completely impossible until
15 minutes before it happens. Mr. Bradley shepherded the last major tax
overhaul through Congress in 1986.

But, Mr. Wyden said, the Camp proposal has one disadvantage over Mr.
Bradley's plan: It was not drafted with bipartisan support.

Advertisement

\protect\hyperlink{after-bottom}{Continue reading the main story}

\hypertarget{site-index}{%
\subsection{Site Index}\label{site-index}}

\hypertarget{site-information-navigation}{%
\subsection{Site Information
Navigation}\label{site-information-navigation}}

\begin{itemize}
\tightlist
\item
  \href{https://help.nytimes.com/hc/en-us/articles/115014792127-Copyright-notice}{©~2020~The
  New York Times Company}
\end{itemize}

\begin{itemize}
\tightlist
\item
  \href{https://www.nytco.com/}{NYTCo}
\item
  \href{https://help.nytimes.com/hc/en-us/articles/115015385887-Contact-Us}{Contact
  Us}
\item
  \href{https://www.nytco.com/careers/}{Work with us}
\item
  \href{https://nytmediakit.com/}{Advertise}
\item
  \href{http://www.tbrandstudio.com/}{T Brand Studio}
\item
  \href{https://www.nytimes.com/privacy/cookie-policy\#how-do-i-manage-trackers}{Your
  Ad Choices}
\item
  \href{https://www.nytimes.com/privacy}{Privacy}
\item
  \href{https://help.nytimes.com/hc/en-us/articles/115014893428-Terms-of-service}{Terms
  of Service}
\item
  \href{https://help.nytimes.com/hc/en-us/articles/115014893968-Terms-of-sale}{Terms
  of Sale}
\item
  \href{https://spiderbites.nytimes.com}{Site Map}
\item
  \href{https://help.nytimes.com/hc/en-us}{Help}
\item
  \href{https://www.nytimes.com/subscription?campaignId=37WXW}{Subscriptions}
\end{itemize}
