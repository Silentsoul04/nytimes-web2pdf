Sections

SEARCH

\protect\hyperlink{site-content}{Skip to
content}\protect\hyperlink{site-index}{Skip to site index}

\href{https://www.nytimes.com/section/world/europe}{Europe}

\href{https://myaccount.nytimes.com/auth/login?response_type=cookie\&client_id=vi}{}

\href{https://www.nytimes.com/section/todayspaper}{Today's Paper}

\href{/section/world/europe}{Europe}\textbar{}Putin Reclaims Crimea for
Russia and Bitterly Denounces the West

\url{https://nyti.ms/1lK9kBQ}

\begin{itemize}
\item
\item
\item
\item
\item
\item
\end{itemize}

Advertisement

\protect\hyperlink{after-top}{Continue reading the main story}

Supported by

\protect\hyperlink{after-sponsor}{Continue reading the main story}

\hypertarget{putin-reclaims-crimea-for-russia-and-bitterly-denounces-the-west}{%
\section{Putin Reclaims Crimea for Russia and Bitterly Denounces the
West}\label{putin-reclaims-crimea-for-russia-and-bitterly-denounces-the-west}}

\includegraphics{https://static01.nyt.com/images/2014/03/19/world/19ukraine2/19ukraine2-articleLarge-v3.jpg?quality=75\&auto=webp\&disable=upscale}

By \href{http://www.nytimes.com/by/steven-lee-myers}{Steven Lee Myers}
and \href{https://www.nytimes.com/by/ellen-barry}{Ellen Barry}

\begin{itemize}
\item
  March 18, 2014
\item
  \begin{itemize}
  \item
  \item
  \item
  \item
  \item
  \item
  \end{itemize}
\end{itemize}

MOSCOW --- President Vladimir V. Putin reclaimed Crimea as a part of
Russia on Tuesday, reversing what he described as a historic injustice
inflicted by the Soviet Union 60 years ago and brushing aside
international condemnation that could leave Russia isolated for years to
come.

In an emotional address steeped in years of resentment and bitterness at
perceived slights from the West, Mr. Putin made it clear that Russia's
patience for post-Cold War accommodation, much diminished of late, had
finally been exhausted. Speaking to the country's political elite in the
Grand Kremlin Palace, he said he did not seek to divide Ukraine any
further, but he vowed to protect Russia's interests there from what he
described as Western actions that had left Russia feeling cornered.

``Crimea has always been an integral part of Russia in the hearts and
minds of people,'' Mr. Putin declared in his address, delivered in the
chandeliered St. George's Hall before hundreds of members of Parliament,
governors and others. His remarks, which lasted 47 minutes, were
interrupted repeatedly by thunderous applause, standing ovations and at
the end chants of ``Russia, Russia.'' Some in the audience wiped tears
from their eyes.

A theme coursing throughout his remarks was the restoration of Russia
after a period of humiliation following the Soviet collapse, which he
has famously called ``the greatest geopolitical catastrophe of the 20th
century.''

He denounced what he called the global domination of one superpower and
its allies that emerged. ``They cheated us again and again, made
decisions behind our back, presenting us with completed facts,'' he
said. ``That's the way it was with the expansion of NATO in the East,
with the deployment of military infrastructure at our borders. They
always told us the same thing: `Well, this doesn't involve you.'~''

\includegraphics{https://static01.nyt.com/images/2014/03/19/world/UKRAINE/UKRAINE-articleLarge.jpg?quality=75\&auto=webp\&disable=upscale}

The speed of Mr. Putin's annexation of Crimea, redrawing an
international border that has been recognized as part of an independent
Ukraine for 23 years, has been breathtaking and so far apparently
unstoppable.

While his actions, which the United States, Europe and Ukraine do not
recognize, provoked renewed denunciations and threats of tougher
sanctions and diplomatic isolation, it remained unclear how far the West
was willing to go to punish Mr. Putin. The leaders of what had been the
Group of 8 nations announced they would meet next week as the Group of
7, excluding Russia from a club Russia once desperately craved to join.

Certainly the sanctions imposed on Russia ahead of Tuesday's steps did
nothing to dissuade Mr. Putin, as he rushed to make a claim to Crimea
that he argued conformed to international law and precedent. In his
remarks he made clear that Russia was prepared to withstand worse
punishment in the name of restoring a lost part of the country's
historic empire, effectively daring world leaders to sever political or
economic ties and risk the consequences to their own economies.

Mr. Putin, the country's paramount leader for more than 14 years,
appeared to be gambling that the outrage would eventually pass, as it
did after Russia's war with Georgia in 2008, because a newly assertive
Russia would be simply too important to ignore on the world stage. As
with any gamble, though, the annexation of Crimea carries potentially
grave risks.

Only hours after Mr. Putin declared that ``not a single shot'' had been
fired in the military intervention in Crimea, a group of soldiers opened
fire as they stormed a Ukrainian military mapping office near
Simferopol, killing a Ukrainian soldier and wounding another, according
to a Ukrainian officer inside the base and a statement by Ukraine's
Defense Ministry.

The base appeared to be under the control of the attacking soldiers, who
like most of the Russians in Crimea wore no insignia, and the ministry
said that Ukrainian forces in Crimea were now authorized to use force to
defend themselves.

Image

Near Simferopol, Crimea, women watched at lunchtime as Mr. Putin
addressed legislators and other officials in Moscow on
Tuesday.Credit...Yuri Kochetkov/European Pressphoto Agency

The episode underscored the fact that the fate of hundreds of Ukrainian
soldiers, as well military bases and ships, remains dangerously
unresolved.

In the capital, Kiev, Ukraine's new prime minister, Arseniy P.
Yatsenyuk, declared that the conflict had moved from ``a political to a
military phase'' and laid the blame squarely on Russia.

Mr. Putin's determined response to the ouster of Ukraine's president,
Viktor F. Yanukovych, last month has left American and European leaders
scrambling to find an adequate response after initially clinging to the
hope that Mr. Putin was prepared to find a political solution --- or
``off ramp'' --- to an escalating crisis that began with the collapse of
Mr. Yanukovych's government on the night of Feb. 21.

Within a week, Russian special operations troops had seized control of
strategic locations across Crimea, while the regional authorities moved
to declare independence and schedule a referendum on joining Russia that
was held on Sunday.

Even as others criticized the vote as a fraud, Mr. Putin moved quickly
on Monday to recognize its result, which he called ``more than
convincing'' with nearly 97 percent of voters in favor of seceding from
Ukraine. By Tuesday he signed a treaty of accession with the region's
new leaders to make Crimea and the city of Sevastopol the 84th and 85th
regions of the Russian Federation.

The treaty requires legislative approval, but that is a mere formality
given Mr. Putin's unchallenged political authority and the wild
popularity of his actions, which have raised his approval ratings and
unleashed a nationalistic fervor that has drowned out the few voices of
opposition or even caution about the potential costs to Russia.

\includegraphics{https://static01.nyt.com/images/2014/03/18/multimedia/putin-grieviences/putin-grieviences-videoSixteenByNine1050.jpg}

Mr. Putin appeared Tuesday evening at a rally and concert on Red Square
to celebrate an event charged with emotional and historical significance
for many Russians. Among the music played was a sentimental Soviet song
called ``Sevastopol Waltz.''

``After a long, hard and exhaustive journey at sea, Crimea and
Sevastopol are returning to their home harbor, to the native shores, to
the home port, to Russia!'' Mr. Putin told the crowd. When he finished
speaking, he joined a military chorus in singing the national anthem.

He recited a list of grievances --- from the Soviet Union's transfer of
Crimea to the Ukrainian republic in 1954, to NATO's expansion to
Russia's borders, to its war in Kosovo in 1999, when he was a
little-known aide to President Boris N. Yeltsin, to the conflict in
Libya that toppled Col. Muammar el-Qaddafi in 2011 on what he called the
false pretense of a humanitarian intervention.

Since Russia's stealthy takeover of Crimea began, Mr. Putin has said
very little in public about his ultimate goals. His only extensive
remarks came in
\href{http://www.nytimes.com/2014/03/05/world/europe/putin-flashing-disdain-defends-action-in-crimea.html}{a
news conference} with a pool of Kremlin journalists in which he appeared
uncomfortable, uncertain and angry at times. In the grandeur of the
Kremlin's walls on Tuesday, Mr. Putin sounded utterly confident and
defiant.

Reaching deep into Russian and Soviet history, he cast himself as the
guardian of the Russian people, even those beyond its post-Soviet
borders, restoring a part of an empire that the collapse of the Soviet
Union had left abandoned to the cruel fates of what he described as a
procession of hapless democratic leaders in Ukraine.

``Millions of Russians went to bed in one country and woke up abroad,''
he said. ``Overnight, they were minorities in the former Soviet
republics, and the Russian people became one of the biggest --- if not
the biggest --- divided nations in the world.''

Image

Russian forces arresting Ukrainian Army officers in Simferopol, Crimea,
on Tuesday. The United States and European nations have strongly opposed
the Russian takeover of Crimea from Ukraine.Credit...Alisa
Borovikova/Agence France-Presse --- Getty Images

He cited the 10th-century baptism of Prince Vladimir, whose conversion
to Orthodox Christianity transformed the kingdom then known as Rus into
the foundation of the empire that became Russia. He called Kiev ``the
mother of Russian cities,'' making clear that he considered Ukraine,
along with Belarus, to be countries where Russia's own interests would
remain at stake regardless of the fallout from Crimea's annexation.

He listed the cities and battlefields of Crimea --- from the
19th-century war with Britain, France and the Turks to the Nazi sieges
of World War II --- as places ``dear to our hearts, symbolizing Russian
military glory and outstanding valor.''

He said that the United States and Europe had crossed ``a red line'' on
Ukraine by throwing support to the new government that quickly emerged
after Mr. Yanukovych fled the capital following months of protests and
two violent days of clashes that left scores dead.

Mr. Putin, as he has before, denounced the uprising as a coup carried
out by ``Russophobes and neo-Nazis'' and abetted by foreigners, saying
it justified Russia's efforts to protect Crimea's population.

``If you press a spring too hard,'' he said, ``it will recoil.''

He justified the annexation using the same arguments that the United
States and Europe cited to justify the independence of Kosovo from
Serbia and even quoted from the American submission to the United
Nations International Court when it reviewed the matter in 2009.

Mr. Putin did not declare a new Cold War, but he bluntly challenged the
post-Soviet order that had more or less held for nearly a
quarter-century, and made it clear that Russia was prepared to defend
itself from any further encroachment or interference in areas it
considers part of its core security, including Russia itself.

He linked the uprisings in Ukraine and the Arab world and ominously
warned that there were efforts to agitate inside Russia. He suggested
that dissenters at home would be considered traitors, a theme that has
reverberated through society with propagandistic documentaries on state
television and moves to mute or close opposition news organizations and
websites.

``Some Western politicians already threaten us not only with sanctions,
but also with the potential for domestic problems,'' he said. ``I would
like to know what they are implying --- the actions of a certain fifth
column, of various national traitors? Or should we expect that they will
worsen the social and economic situation, and therefore provoke people's
discontent?''

Advertisement

\protect\hyperlink{after-bottom}{Continue reading the main story}

\hypertarget{site-index}{%
\subsection{Site Index}\label{site-index}}

\hypertarget{site-information-navigation}{%
\subsection{Site Information
Navigation}\label{site-information-navigation}}

\begin{itemize}
\tightlist
\item
  \href{https://help.nytimes.com/hc/en-us/articles/115014792127-Copyright-notice}{©~2020~The
  New York Times Company}
\end{itemize}

\begin{itemize}
\tightlist
\item
  \href{https://www.nytco.com/}{NYTCo}
\item
  \href{https://help.nytimes.com/hc/en-us/articles/115015385887-Contact-Us}{Contact
  Us}
\item
  \href{https://www.nytco.com/careers/}{Work with us}
\item
  \href{https://nytmediakit.com/}{Advertise}
\item
  \href{http://www.tbrandstudio.com/}{T Brand Studio}
\item
  \href{https://www.nytimes.com/privacy/cookie-policy\#how-do-i-manage-trackers}{Your
  Ad Choices}
\item
  \href{https://www.nytimes.com/privacy}{Privacy}
\item
  \href{https://help.nytimes.com/hc/en-us/articles/115014893428-Terms-of-service}{Terms
  of Service}
\item
  \href{https://help.nytimes.com/hc/en-us/articles/115014893968-Terms-of-sale}{Terms
  of Sale}
\item
  \href{https://spiderbites.nytimes.com}{Site Map}
\item
  \href{https://help.nytimes.com/hc/en-us}{Help}
\item
  \href{https://www.nytimes.com/subscription?campaignId=37WXW}{Subscriptions}
\end{itemize}
