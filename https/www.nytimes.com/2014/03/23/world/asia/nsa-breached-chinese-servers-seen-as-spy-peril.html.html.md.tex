Sections

SEARCH

\protect\hyperlink{site-content}{Skip to
content}\protect\hyperlink{site-index}{Skip to site index}

\href{https://www.nytimes.com/section/world/asia}{Asia Pacific}

\href{https://myaccount.nytimes.com/auth/login?response_type=cookie\&client_id=vi}{}

\href{https://www.nytimes.com/section/todayspaper}{Today's Paper}

\href{/section/world/asia}{Asia Pacific}\textbar{}N.S.A. Breached
Chinese Servers Seen as Security Threat

\url{https://nyti.ms/1gc96TE}

\begin{itemize}
\item
\item
\item
\item
\item
\item
\end{itemize}

Advertisement

\protect\hyperlink{after-top}{Continue reading the main story}

Supported by

\protect\hyperlink{after-sponsor}{Continue reading the main story}

\hypertarget{nsa-breached-chinese-servers-seen-as-security-threat}{%
\section{N.S.A. Breached Chinese Servers Seen as Security
Threat}\label{nsa-breached-chinese-servers-seen-as-security-threat}}

\includegraphics{https://static01.nyt.com/images/2014/03/23/world/SUB-JP-NSA-1/SUB-JP-NSA-1-articleLarge.jpg?quality=75\&auto=webp\&disable=upscale}

By \href{http://www.nytimes.com/by/david-e-sanger}{David E. Sanger} and
\href{http://www.nytimes.com/by/nicole-perlroth}{Nicole Perlroth}

\begin{itemize}
\item
  March 22, 2014
\item
  \begin{itemize}
  \item
  \item
  \item
  \item
  \item
  \item
  \end{itemize}
\end{itemize}

WASHINGTON --- American officials have long considered Huawei, the
Chinese telecommunications giant, a security threat, blocking it from
business deals in the United States for fear that the company would
create ``back doors'' in its equipment that could allow the Chinese
military or Beijing-backed hackers to steal corporate and government
secrets.

But even as the United States made a public case about the dangers of
buying from Huawei, classified documents show that the National Security
Agency was creating its own back doors --- directly into Huawei's
networks.

The agency pried its way into the servers in Huawei's sealed
headquarters in Shenzhen, China's industrial heart, according to N.S.A.
documents provided by the former contractor Edward J. Snowden. It
obtained information about the workings of the giant routers and complex
digital switches that Huawei boasts connect a third of the world's
population, and monitored communications of the company's top
executives.

One of the goals of the operation, code-named ``Shotgiant,'' was to find
any links between Huawei and the People's Liberation Army, one 2010
document made clear. But the plans went further: to exploit Huawei's
technology so that when the company sold equipment to other countries
--- including both allies and nations that avoid buying American
products --- the N.S.A. could roam through their computer and telephone
networks to conduct surveillance and, if ordered by the president,
offensive cyberoperations.

``Many of our targets communicate over Huawei-produced products,'' the
N.S.A. document said. ``We want to make sure that we know how to exploit
these products,'' it added, to ``gain access to networks of interest''
around the world.

The documents were disclosed by The New York Times and
\href{http://www.spiegel.de/international/world/nsa-spied-on-chinese-government-and-networking-firm-huawei-a-960199.html}{Der
Spiegel}, and are also part of a book by Der Spiegel, ``The N.S.A.
Complex.'' The documents, as well as interviews with intelligence
officials, offer new insights into the United States' escalating digital
cold war with Beijing. While President Obama and China's president, Xi
Jinping, have begun talks about limiting the cyber conflict, it appears
to be intensifying.

The N.S.A., for example, is tracking more than 20 Chinese hacking groups
--- more than half of them Chinese Army and Navy units --- as they break
into the networks of the United States government, companies including
Google, and drone and nuclear-weapon part makers, according to a
half-dozen current and former American officials.

Image

Ren Zhengfei, founder of Huawei, is seen as a Chinese version of Steve
Jobs.Credit...Dmitry Lovetsky/Associated Press

If anything, they said, the pace has increased since the revelation last
year that some of the most aggressive Chinese hacking originated at a
People's Liberation Army facility,
\href{http://www.nytimes.com/2013/02/19/technology/chinas-army-is-seen-as-tied-to-hacking-against-us.html?_r=0\&gwh=1F8882CC86CB87898C237E9DC1989D5D\&gwt=regi}{Unit
61398}, in Shanghai.

The Obama administration distinguishes between the hacking and corporate
theft that the Chinese conduct against American companies to buttress
their own state-run businesses, and the intelligence operations that the
United States conducts against Chinese and other targets.

American officials have repeatedly said that the N.S.A. breaks into
foreign networks only for legitimate national security purposes.

A White House spokeswoman, Caitlin M. Hayden, said: ``We do not give
intelligence we collect to U.S. companies to enhance their international
competitiveness or increase their bottom line. Many countries cannot say
the same.''

But that does not mean the American government does not conduct its own
form of corporate espionage with a different set of goals. Those
concerning Huawei were described in the 2010 document.

``If we can determine the company's plans and intentions,'' an analyst
wrote, ``we hope that this will lead us back to the plans and intentions
of the PRC,'' referring to the People's Republic of China. The N.S.A.
saw an additional opportunity: As Huawei invested in new technology and
laid undersea cables to connect its \$40 billion-a-year networking
empire, the agency was interested in tunneling into key Chinese
customers, including ``high priority targets --- Iran, Afghanistan,
Pakistan, Kenya, Cuba.''

The documents offer no answer to a central question: Is Huawei an
independent company, as its leaders contend, or a front for the People's
Liberation Army, as American officials suggest but have never publicly
proved?

Two years after Shotgiant became a major program, the House Intelligence
Committee delivered an unclassified report on Huawei and another Chinese
company, ZTE, that cited no evidence confirming the suspicions about
Chinese government ties. Still, the October 2012 report concluded that
the companies must be blocked from ``acquisitions, takeover or mergers''
in the United States, and ``cannot be trusted to be free of foreign
state influence.''

\href{https://www.nytimes.com/interactive/2014/03/23/world/asia/23nsa-docs.html}{}

\includegraphics{https://static01.nyt.com/images/2014/03/23/us/23nsa-docs-promo/23nsa-docs-promo-thumbLarge.png}

\hypertarget{slides-describe-mission-involving-huawei}{%
\subsection{Slides Describe Mission Involving
Huawei}\label{slides-describe-mission-involving-huawei}}

Powerpoint slides from 2010 lay out the National Security Agency's goals
in an effort to break into networks of Huawei, the telecommunications
giant.

Huawei, which has all but given up its hopes of entering the American
market, complains that it is the victim of protectionism, swathed in
trumped-up national security concerns. Company officials insist that it
has no connection to the People's Liberation Army.

William Plummer, a senior Huawei executive in the United States, said
the company had no idea it was an N.S.A. target, adding that in his
personal opinion, ``The irony is that exactly what they are doing to us
is what they have always charged that the Chinese are doing through
us.''

``If such espionage has been truly conducted,'' Mr. Plummer added,
``then it is known that the company is independent and has no unusual
ties to any government, and that knowledge should be relayed publicly to
put an end to an era of mis- and disinformation.''

\textbf{Blocked at Every Turn}

Washington's concerns about Huawei date back nearly a decade, since the
RAND Corporation, the research organization, evaluated the potential
threat of China for the American military. RAND concluded that ``private
Chinese companies such as Huawei'' were part of a new ``digital
triangle'' of companies, institutes and government agencies that worked
together secretly.

Huawei is a global giant: it manufactures equipment that makes up the
backbone of the Internet, lays submarine cables from Asia to Africa and
has become the world's third largest smartphone maker after Samsung and
Apple.

The man behind its strategy is Ren Zhengfei, the company's elusive
founder, who was a P.L.A. engineer in the 1970s. To the Chinese, he is
something akin to Steve Jobs --- an entrepreneur who started a digital
empire with little more than \$3,000 in the mid-1980s, and took on both
state-owned companies and foreign competitors. But to American
officials, he is a link to the People's Liberation Army.

They have blocked his company at every turn: pressing Sprint to kill a
\$3 billion deal to buy Huawei's fourth generation, or 4G, network
technology; scuttling a planned purchase of 3Com for fear that Huawei
would alter computer code sold to the United States military; and
pushing allies, like Australia, to back off from major projects.

As long ago as 2007, the N.S.A. began a covert program against Huawei,
the documents show. By 2010, the agency's Tailored Access Operations
unit --- which breaks into hard-to-access networks --- found a way into
Huawei's headquarters. The agency collected Mr. Ren's communications,
one document noted, though analysts feared they might be missing many of
them.

N.S.A. analysts made clear that they were looking for more than just
``signals intelligence'' about the company and its connections to
Chinese leaders; they wanted to learn how to pierce its systems so that
when adversaries and allies bought Huawei equipment, the United States
would be plugged into those networks. (The Times withheld technical
details of the operation at the request of the Obama administration,
which cited national security concerns.)

The N.S.A.'s operations against China do not stop at Huawei. Last year,
the agency cracked two of China's biggest cellphone networks, allowing
it to track strategically important Chinese military units, according to
an April 2013 document leaked by Mr. Snowden. Other major targets, the
document said, are the locations where the Chinese leadership works. The
country's leaders, like everyone else, are constantly upgrading to
better, faster Wi-Fi --- and the N.S.A. is constantly finding new ways
in.

\textbf{Hack Attacks Accelerate}

Chinese state attacks have only accelerated in recent years, according
to the current and former intelligence officials, who spoke on condition
of anonymity about classified information.

A dozen P.L.A. military units --- aside from Unit 61398 --- do their
hacking from eavesdropping posts around China, and though their targets
were initially government agencies and foreign ministries around the
world, they have since expanded into the private sector. For example,
officials point to the First Bureau of the army's Third Department,
which the N.S.A. began tracking in 2004 after it hacked into the
Pentagon's networks. The unit's targets have grown to include telecom
and technology companies that specialize in networking and encryption
equipment --- including some Huawei competitors.

For some of its most audacious attacks, China relies on hackers at
state-funded universities and privately owned Chinese technology
companies, apparently as much for their skills as for the plausible
deniability it offers the state if it gets caught. The N.S.A. is
tracking more than half a dozen such groups suspected of operating at
the behest of the Chinese Ministry of State Security, China's civilian
spy agency, the officials said.

Their targets, they noted, closely align with China's stated economic
and strategic directives. As China strove to develop drones and
next-generation ballistic and submarine-launched missiles in recent
years, the N.S.A. and its partners watched as one group of privately
employed engineers based in Guangzhou in southern China pilfered the
blueprints to missile, satellite, space, and nuclear propulsion
technology from businesses in the United States, Canada, Europe, Russia
and Africa.

And as China strove to make its own inroads on the web, officials said
another group of private hackers infiltrated Google, Adobe and dozens of
other global technology companies in 2010. Lately, the officials said,
that group and its counterparts are also going after security firms,
banks, chemical companies, automakers and even nongovernment
organizations.

``China does more in terms of cyberespionage than all other countries
put together,'' said James A. Lewis, a computer security expert at the
Center for Strategic and International Studies in Washington.

``The question is no longer which industries China is hacking into,'' he
added. ``It's which industries they aren't hacking into.''

Advertisement

\protect\hyperlink{after-bottom}{Continue reading the main story}

\hypertarget{site-index}{%
\subsection{Site Index}\label{site-index}}

\hypertarget{site-information-navigation}{%
\subsection{Site Information
Navigation}\label{site-information-navigation}}

\begin{itemize}
\tightlist
\item
  \href{https://help.nytimes.com/hc/en-us/articles/115014792127-Copyright-notice}{©~2020~The
  New York Times Company}
\end{itemize}

\begin{itemize}
\tightlist
\item
  \href{https://www.nytco.com/}{NYTCo}
\item
  \href{https://help.nytimes.com/hc/en-us/articles/115015385887-Contact-Us}{Contact
  Us}
\item
  \href{https://www.nytco.com/careers/}{Work with us}
\item
  \href{https://nytmediakit.com/}{Advertise}
\item
  \href{http://www.tbrandstudio.com/}{T Brand Studio}
\item
  \href{https://www.nytimes.com/privacy/cookie-policy\#how-do-i-manage-trackers}{Your
  Ad Choices}
\item
  \href{https://www.nytimes.com/privacy}{Privacy}
\item
  \href{https://help.nytimes.com/hc/en-us/articles/115014893428-Terms-of-service}{Terms
  of Service}
\item
  \href{https://help.nytimes.com/hc/en-us/articles/115014893968-Terms-of-sale}{Terms
  of Sale}
\item
  \href{https://spiderbites.nytimes.com}{Site Map}
\item
  \href{https://help.nytimes.com/hc/en-us}{Help}
\item
  \href{https://www.nytimes.com/subscription?campaignId=37WXW}{Subscriptions}
\end{itemize}
