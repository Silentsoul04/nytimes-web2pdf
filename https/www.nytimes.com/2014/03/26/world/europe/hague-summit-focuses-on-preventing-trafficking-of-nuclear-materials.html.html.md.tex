Sections

SEARCH

\protect\hyperlink{site-content}{Skip to
content}\protect\hyperlink{site-index}{Skip to site index}

\href{https://www.nytimes.com/section/world/europe}{Europe}

\href{https://myaccount.nytimes.com/auth/login?response_type=cookie\&client_id=vi}{}

\href{https://www.nytimes.com/section/todayspaper}{Today's Paper}

\href{/section/world/europe}{Europe}\textbar{}Obama Answers Critics,
Dismissing Russia as a `Regional Power'

\url{https://nyti.ms/1jqFpl3}

\begin{itemize}
\item
\item
\item
\item
\item
\item
\end{itemize}

Advertisement

\protect\hyperlink{after-top}{Continue reading the main story}

Supported by

\protect\hyperlink{after-sponsor}{Continue reading the main story}

\hypertarget{obama-answers-critics-dismissing-russia-as-a-regional-power}{%
\section{Obama Answers Critics, Dismissing Russia as a `Regional
Power'}\label{obama-answers-critics-dismissing-russia-as-a-regional-power}}

\includegraphics{https://static01.nyt.com/images/2014/03/25/multimedia/obama-ukraine-hague/obama-ukraine-hague-videoSixteenByNine1050-v2.jpg}

By \href{http://www.nytimes.com/by/michael-d-shear}{Michael D. Shear}
and \href{http://www.nytimes.com/by/peter-baker}{Peter Baker}

\begin{itemize}
\item
  March 25, 2014
\item
  \begin{itemize}
  \item
  \item
  \item
  \item
  \item
  \item
  \end{itemize}
\end{itemize}

THE HAGUE --- Amid the chest-thumping between President Obama and
President Vladimir V. Putin of Russia in recent weeks, one question has
lingered: How big a threat is Russia, anyway?

Mitt Romney, Mr. Obama's 2012 presidential challenger, made clear his
own assessment during the campaign, saying repeatedly that Russia was
America's ``No. 1 geopolitical foe'' and arguing that Mr. Putin's
aggressive stance demanded a similar response from the American
president.

On Tuesday, Mr. Obama offered his answer, saying that Mr. Putin leads a
``regional power'' whose real threat extends largely to its bordering
nations. In language that seemed to be aimed at the highest ranks inside
the Kremlin, Mr. Obama dismissed Russia as a country that is lashing out
at its neighbors ``not out of strength, but out of weakness.''

Mr. Obama's decision to engage a reporter's question about Mr. Romney
during a foreign trip suggests that the president was eager to deflect
criticism at home that he has been naïve about his approach to Mr.
Putin. In Mr. Obama's first term, he
\href{http://www.nytimes.com/2009/07/08/world/europe/08prexy.html?action=click\&module=Search\&region=searchResults\%230\&version=\&url=http\%3A\%2F\%2Fquery.nytimes.com\%2Fsearch\%2Fsitesearch\%2F\%3Faction\%3Dclick\%26region\%3DMasthead\%26pgtype\%3DHomepage\%26module\%3DSearchSubmit\%26contentCollection\%3DHomepage\%26t\%3Dqry436\%23\%2Fobama\%2520reset\%2520russia}{pursued
a ``reset'' in relations with Russia}, and during the campaign, he
mocked Mr. Romney, saying during a televised debate that
``\href{http://www.nytimes.com/2012/10/23/us/politics/obama-and-romney-meet-in-foreign-policy-debate.html?pagewanted=all\&action=click\&module=Search\&region=searchResults\%230\&version=\&url=http\%3A\%2F\%2Fquery.nytimes.com\%2Fsearch\%2Fsitesearch\%2F\%23\%2Fthe\%2B\%2B1980s\%2Bare\%2Bcalling\%2C\%2Bthey\%2Bwant\%2Btheir\%2B\%2Bforeign\%2Bpolicy\%2Bback.\%2F}{the
1980s, they're now calling to ask for their foreign policy back}.''

In recent weeks, as Mr. Putin's forces rolled through Crimea with little
regard to warnings by Mr. Obama, Republicans have said Mr. Romney has
been vindicated, and Mr. Obama proved wrong. In February, Senator John
McCain of Arizona, Mr. Obama's 2008 rival, called him
``\href{http://thehill.com/blogs/blog-briefing-room/news/198843-sen-mccain-obama-the-most-naive-president-in-history}{the
most naïve president in history}.''

After Russian troops began taking control of Crimea, Sarah Palin, the
Republican Party's 2008 vice-presidential nominee, took credit for
predicting it. ``Yes, I could see this one from Alaska,'' she
\href{https://www.facebook.com/sarahpalin/posts/10201573917093799}{wrote
on her Facebook page}. ``I'm usually not one to Told-Ya-So, but I did,
despite my accurate prediction being derided as `an extremely
far-fetched scenario' by the `high-brow' Foreign Policy magazine.''

And Sunday, on CBS's ``Face the Nation,''
\href{http://www.cbsnews.com/news/mitt-romney-slams-obama-for-naivete-in-russian-relations/}{Mr.
Romney echoed Mr. McCain's assertion} that the president had been naïve
about Russia.

``His faulty judgment about Russia's intentions and objectives has led
to a number of foreign policy challenges that we face,'' Mr. Romney
said. ``This is not fantasy land. They are not our enemy but an
adversary on the world stage.''

Seizing his news conference with Prime Minister Mark Rutte of the
Netherlands as a platform to respond, the president on Tuesday explained
his thinking about Mr. Putin and the country he governs, saying that the
influence of Russia on the world stage has languished since the breakup
of the Soviet Union. He said the situation in Ukraine in recent weeks
proved that he is right.

``The fact that Russia felt compelled to go in militarily and lay bare
these violations of international law indicates less influence, not
more,'' Mr. Obama said.

\includegraphics{https://static01.nyt.com/images/2014/03/26/world/26PREXY2/26PREXY2-articleLarge.jpg?quality=75\&auto=webp\&disable=upscale}

Internally, that is the blunt assessment by the United States of Mr.
Putin: He is, according to the president's national security team,
someone whose bluster about his closest neighbors can be backed up with
action. And they view Russia's actions --- or inaction --- as critical
to resolving some of the world's most enduring conflicts far beyond
Russia's borders.

But despite Russia's vast Cold War arsenal, the administration does not
view the country as an existential threat to the American homeland. On
Tuesday, Mr. Obama again rejected Mr. Romney's assertion, though he
misquoted the former Republican governor slightly, saying that Russia
does not ``pose the No. 1 national security threat to the United
States.''

``My response then continues to be what I believe today,'' he said,
referring to his answer to Mr. Romney in 2012. ``Which is: Russia's
actions are a problem. They don't pose the No. 1 national security
threat to the United States. I continue to be much more concerned when
it comes to our security with the prospect of a nuclear weapon going off
in Manhattan.''

That comment might have been of particular concern to the residents of
New York City. But it matches the overall assessment of most Americans
when it comes to Russia, according to a
\href{http://www.people-press.org/2014/03/25/concerns-about-russia-rise-but-just-a-quarter-call-moscow-an-adversary/}{survey
by the Pew Research Center}.

In that poll, conducted Thursday through Sunday, concern about Russia
has increased recently, but only about a quarter of those surveyed said
they viewed Russia as an adversary of the United States. About four in
10 said Russia is a serious problem, but about half said it is important
for the United States not to get involved in the situation between
Russia and Ukraine.

In the weeks ahead, Mr. Obama may face more criticism as the
confrontation between Mr. Putin and the Western nations continues with
no end in sight. But Mr. Obama's aides have made clear that they have no
intention of letting Mr. Romney or Mr. McCain succeed in painting the
president as doe-eyed in the face of a harsh reality.

Benjamin J. Rhodes, a deputy national security adviser to the president,
said, ``Well, look, we've been very cleareyed about our Russia policy
from when we came into office, which is that we will cooperate when we
have common interests and we can form common positions, but we'll be
very clear when we have differences.''

Mr. Rhodes and other advisers argue that the cooperation that Mr. Obama
sought with Russia at the beginning of the administration --- at a time
when Mr. Putin was prime minister and was not in direct control of the
country --- helped the United States succeed in several initiatives,
including supplying troops in Afghanistan, putting sanctions on Iran and
passing a new Start treaty.

And they insist that Mr. Obama's decision against using force in Syria
had nothing to do with Mr. Putin's calculations in Ukraine.

``When George Bush was president, we went to war in Iraq, we went to war
in Afghanistan; that did not in any way deter Russia from going into
Georgia in 2008,'' Mr. Rhodes told reporters in The Hague. ``Frankly, in
terms of the steps that we've outlined and the steps that we're taking,
they go far beyond any previous steps that have been taken in response
to Russian aggression.''

On Twitter on Tuesday, one of Mr. Obama's top aides was more blunt. Dan
Pfeiffer, the senior adviser and chief communications strategist, said
the president absolutely rejected Mr. Romney's criticisms. Mr. Obama,
\href{https://twitter.com/pfeiffer44/with_replies}{Mr. Pfeiffer posted
on Twitter}, ``actually said he was wrong cuz he was.''

Advertisement

\protect\hyperlink{after-bottom}{Continue reading the main story}

\hypertarget{site-index}{%
\subsection{Site Index}\label{site-index}}

\hypertarget{site-information-navigation}{%
\subsection{Site Information
Navigation}\label{site-information-navigation}}

\begin{itemize}
\tightlist
\item
  \href{https://help.nytimes.com/hc/en-us/articles/115014792127-Copyright-notice}{©~2020~The
  New York Times Company}
\end{itemize}

\begin{itemize}
\tightlist
\item
  \href{https://www.nytco.com/}{NYTCo}
\item
  \href{https://help.nytimes.com/hc/en-us/articles/115015385887-Contact-Us}{Contact
  Us}
\item
  \href{https://www.nytco.com/careers/}{Work with us}
\item
  \href{https://nytmediakit.com/}{Advertise}
\item
  \href{http://www.tbrandstudio.com/}{T Brand Studio}
\item
  \href{https://www.nytimes.com/privacy/cookie-policy\#how-do-i-manage-trackers}{Your
  Ad Choices}
\item
  \href{https://www.nytimes.com/privacy}{Privacy}
\item
  \href{https://help.nytimes.com/hc/en-us/articles/115014893428-Terms-of-service}{Terms
  of Service}
\item
  \href{https://help.nytimes.com/hc/en-us/articles/115014893968-Terms-of-sale}{Terms
  of Sale}
\item
  \href{https://spiderbites.nytimes.com}{Site Map}
\item
  \href{https://help.nytimes.com/hc/en-us}{Help}
\item
  \href{https://www.nytimes.com/subscription?campaignId=37WXW}{Subscriptions}
\end{itemize}
