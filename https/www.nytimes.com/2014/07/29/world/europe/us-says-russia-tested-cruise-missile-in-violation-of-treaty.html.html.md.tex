Sections

SEARCH

\protect\hyperlink{site-content}{Skip to
content}\protect\hyperlink{site-index}{Skip to site index}

\href{https://www.nytimes.com/section/world/europe}{Europe}

\href{https://myaccount.nytimes.com/auth/login?response_type=cookie\&client_id=vi}{}

\href{https://www.nytimes.com/section/todayspaper}{Today's Paper}

\href{/section/world/europe}{Europe}\textbar{}U.S. Says Russia Tested
Cruise Missile, Violating Treaty

\url{https://nyti.ms/1uBhuEK}

\begin{itemize}
\item
\item
\item
\item
\item
\item
\end{itemize}

Advertisement

\protect\hyperlink{after-top}{Continue reading the main story}

Supported by

\protect\hyperlink{after-sponsor}{Continue reading the main story}

\hypertarget{us-says-russia-tested-cruise-missile-violating-treaty}{%
\section{U.S. Says Russia Tested Cruise Missile, Violating
Treaty}\label{us-says-russia-tested-cruise-missile-violating-treaty}}

By \href{http://www.nytimes.com/by/michael-r-gordon}{Michael R. Gordon}

\begin{itemize}
\item
  July 28, 2014
\item
  \begin{itemize}
  \item
  \item
  \item
  \item
  \item
  \item
  \end{itemize}
\end{itemize}

WASHINGTON --- The United States has concluded that Russia violated a
landmark arms control treaty by testing a prohibited ground-launched
cruise missile, according to senior American officials, a finding that
was conveyed by President Obama to President Vladimir V. Putin of Russia
in a letter on Monday.

It is the most serious allegation of an arms control treaty violation
that the Obama administration has leveled against Russia and adds
another dispute to a relationship already burdened by tensions over the
Kremlin's support for separatists in Ukraine and its decision to grant
asylum to Edward J. Snowden, the former National Security Agency
contractor.

At the heart of the issue is the 1987 treaty that bans American and
Russian ground-launched ballistic or cruise missiles capable of flying
300 to 3,400 miles. That
\href{http://www.state.gov/www/global/arms/treaties/inf1.html}{accord},
which was signed by President Ronald Reagan and Mikhail S. Gorbachev,
the Soviet leader, helped seal the end of the Cold War and has been
regarded as a cornerstone of American-Russian arms control efforts.

Russia first began testing the cruise missiles as early as 2008,
according to American officials, and the Obama administration concluded
by the end of 2011 that they were a compliance concern. In May 2013,
Rose Gottemoeller, the State Department's senior arms control official,
first raised the possibility of a violation with Russian officials.

The New York Times
\href{http://www.nytimes.com/2014/01/30/world/europe/us-says-russia-tested-missile-despite-treaty.html?_r=0}{reported}
in January that American officials had informed the NATO allies that
Russia had tested a ground-launched cruise missile, raising serious
concerns about Russia's compliance with the Intermediate-range Nuclear
Forces Treaty, or I.N.F. Treaty as it is commonly called. The State
Department said at the time that the issue was under review and that the
Obama administration was not yet ready to formally declare it to be a
treaty violation.

In recent months, however, the issue has been taken up by top-level
officials, including a meeting early this month of the Principals'
Committee, a cabinet-level body that includes Mr. Obama's national
security adviser, the defense secretary, the chairman of the Joint
Chiefs of Staff, the secretary of state and the director of the Central
Intelligence Agency. Senior officials said the president's most senior
advisers unanimously agreed that the test was a serious violation, and
the allegation will be made public soon in the State Department's annual
report on international compliance with arms control agreements.

``The United States has determined that the Russian Federation is in
violation of its obligations under the I.N.F. treaty not to possess,
produce or flight test a ground-launched cruise missile (GLCM) with a
range capability of 500 kilometers to 5,500 kilometers or to possess or
produce launchers of such missiles,'' that report will say.

In his letter to Mr. Putin, delivered by the American Embassy, Mr. Obama
underscored his interest in a high-level dialogue with Moscow with the
aim of preserving the 1987 treaty and discussing steps the Kremlin might
take to come back into compliance. Secretary of State John Kerry
delivered a similar message in a Sunday phone call to Sergey V. Lavrov,
the Russian foreign minister.

Because the treaty proscribes testing ground-launched cruise missiles of
medium-range, the Kremlin cannot undo the violation. But administration
officials do not believe the cruise missile has been deployed and say
there are measures the Russians can take to ameliorate the problem.

Image

President Vladimir V. Putin of Russia on Sunday aboard a ship in
Severomorsk, Russia.Credit...Pool photo by Mikhail Klimentyev

Administration officials declined to say what such steps might be, but
arms control experts say they could include a promise not to deploy the
system and inspections to demonstrate that the cruise missiles and their
launchers have been destroyed. Because the missiles are small and easily
concealed, obtaining complete confidence that the weapons have been
eliminated might be difficult.

NATO's top commander, Gen. Philip M. Breedlove, has said that the
violation requires a response if it cannot be resolved.

``A weapon capability that violates the I.N.F., that is introduced into
the greater European land mass, is absolutely a tool that will have to
be dealt with,'' he said in an
\href{http://www.nytimes.com/2014/04/03/world/europe/nato-general-says-russian-force-poised-to-invade-ukraine.html}{interview}
in April. ``It can't go unanswered.'' Mr. Obama has determined that the
United States will not retaliate against the Russians by violating the
treaty and deploying its own prohibited medium-range system, officials
said. So the responses might include deploying sea- and air-launched
cruise missiles, which would be allowable under the accord.

Republican lawmakers have repeatedly criticized the administration for
dragging its feet on the issue. Ms. Gottemoeller, the State Department
official, has had no discussions with her Russian counterparts on the
subject since February. And Mr. Kerry's call on Sunday was the first
time he had directly raised the violation with Mr. Lavrov, State
Department officials said. Administration officials said the upheaval in
Ukraine pushed the issue to the back burner and that the downturn in
American-Russian relations has led to an interruption of regular
arms-control meetings.

The prospects for resolving the violation were also uncertain at best.
After Ms. Gottemoeller first raised the matter in 2013, Russian
officials said that they had looked into the matter and consider the
issue to be closed.

The Russians have also raised their own allegations, a move that
American officials believe is intended to muddy the issue and perhaps
give them leverage in any negotiations over compliance. One month after
Ms. Gottemoeller raised the American concerns, the Russians responded by
pointing to the United States plans to base the Aegis missile system in
Romania.

The Aegis system, which is commonly used on warships, would be used to
protect American and NATO forces from missile attacks. But the Russians
have alleged that it could be used to fire prohibited cruise missiles.

When Mr. Kerry spoke with Mr. Lavrov on Sunday, the Russian foreign
minister cited Russia's concerns over ``decoys.'' That may have been a
reference to Russian charges that the targets that the United States
uses in antimissile tests are an I.N.F. treaty violation. American
officials regard that allegation, about the issue of the Aegis system
and complaints about the use of targets, to be spurious.

An underlying concern of the Obama administration in dealing with the
Russians is that the Kremlin may not be wedded to the I.N.F. agreement.
During the George W. Bush administration, some Russians officials argued
that the treaty should be dropped so that the Kremlin could augment its
military abilities to deal with threats on its periphery, including
China and Pakistan.

In a June 2013 meeting with Russian defense industry officials, Mr.
Putin \href{http://eng.news.kremlin.ru/news/5615}{described} Mr.
Gorbachev's decision to sign the accord as ``debatable to say the
least,'' but asserted that Russia would uphold the agreement. Even some
American conservative analysts say that in pursuing the compliance
concern, the United States should not provide the Kremlin with an
opportunity to back out of the agreement.

``For the United States to declare that we are pulling out of the treaty
in response to what Russia has done would actually be welcome in Moscow
because they are wrestling with the question of how they terminate,''
Stephen Rademaker, a former Bush administration official, told the House
Armed Services Committee this month. ``We shouldn't make it any easier
for them. We should force them to take the onus of that.''

Advertisement

\protect\hyperlink{after-bottom}{Continue reading the main story}

\hypertarget{site-index}{%
\subsection{Site Index}\label{site-index}}

\hypertarget{site-information-navigation}{%
\subsection{Site Information
Navigation}\label{site-information-navigation}}

\begin{itemize}
\tightlist
\item
  \href{https://help.nytimes.com/hc/en-us/articles/115014792127-Copyright-notice}{©~2020~The
  New York Times Company}
\end{itemize}

\begin{itemize}
\tightlist
\item
  \href{https://www.nytco.com/}{NYTCo}
\item
  \href{https://help.nytimes.com/hc/en-us/articles/115015385887-Contact-Us}{Contact
  Us}
\item
  \href{https://www.nytco.com/careers/}{Work with us}
\item
  \href{https://nytmediakit.com/}{Advertise}
\item
  \href{http://www.tbrandstudio.com/}{T Brand Studio}
\item
  \href{https://www.nytimes.com/privacy/cookie-policy\#how-do-i-manage-trackers}{Your
  Ad Choices}
\item
  \href{https://www.nytimes.com/privacy}{Privacy}
\item
  \href{https://help.nytimes.com/hc/en-us/articles/115014893428-Terms-of-service}{Terms
  of Service}
\item
  \href{https://help.nytimes.com/hc/en-us/articles/115014893968-Terms-of-sale}{Terms
  of Sale}
\item
  \href{https://spiderbites.nytimes.com}{Site Map}
\item
  \href{https://help.nytimes.com/hc/en-us}{Help}
\item
  \href{https://www.nytimes.com/subscription?campaignId=37WXW}{Subscriptions}
\end{itemize}
