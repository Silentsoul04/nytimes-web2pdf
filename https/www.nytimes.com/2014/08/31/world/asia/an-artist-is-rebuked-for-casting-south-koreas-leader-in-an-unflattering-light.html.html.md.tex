Sections

SEARCH

\protect\hyperlink{site-content}{Skip to
content}\protect\hyperlink{site-index}{Skip to site index}

\href{https://www.nytimes.com/section/world/asia}{Asia Pacific}

\href{https://myaccount.nytimes.com/auth/login?response_type=cookie\&client_id=vi}{}

\href{https://www.nytimes.com/section/todayspaper}{Today's Paper}

\href{/section/world/asia}{Asia Pacific}\textbar{}An Artist Is Rebuked
for Casting South Korea's Leader in an Unflattering Light

\url{https://nyti.ms/1tkxQQW}

\begin{itemize}
\item
\item
\item
\item
\item
\end{itemize}

Advertisement

\protect\hyperlink{after-top}{Continue reading the main story}

Supported by

\protect\hyperlink{after-sponsor}{Continue reading the main story}

\hypertarget{an-artist-is-rebuked-for-casting-south-koreas-leader-in-an-unflattering-light}{%
\section{An Artist Is Rebuked for Casting South Korea's Leader in an
Unflattering
Light}\label{an-artist-is-rebuked-for-casting-south-koreas-leader-in-an-unflattering-light}}

\includegraphics{https://static01.nyt.com/images/2014/08/31/world/KOREAPAINTER-2/KOREAPAINTER-2-articleLarge.jpg?quality=75\&auto=webp\&disable=upscale}

By \href{http://www.nytimes.com/by/choe-sang-hun}{Choe Sang-Hun}

\begin{itemize}
\item
  Aug. 30, 2014
\item
  \begin{itemize}
  \item
  \item
  \item
  \item
  \item
  \end{itemize}
\end{itemize}

GWANGJU, South Korea --- After 250 South Korean high school students
died in the
\href{http://www.nytimes.com/2014/04/17/world/asia/south-korean-ferry-accident.html}{sinking
of the Sewol ferry} in April, the artist Hong Sung-dam lashed out at a
political and business elite he considers responsible for the disaster,
doing so in the way he knows best. He painted, pouring his protest onto
canvas just as he did during the country's long years of military
dictatorship.

He was not imprisoned this time, as he was in the waning years of
military-backed rule. But Mr. Hong's 34 foot by 8 foot canvas, which
includes a caricature of
\href{http://topics.nytimes.com/top/reference/timestopics/people/p/park_geunhye/index.html}{President
Park Geun-hye}, was pulled from South Korea's best-known international
art festival in a type of censorship usually reserved for those accused
of supporting communist North Korea.

``This is a ridiculous insult to an artist,'' Mr. Hong said of the
treatment of the painting, in which Ms. Park is depicted as a puppet
controlled by her late father, who led the country for nearly two
decades after engineering a coup. ``What they did was proof of what I
tried to say in the painting. Under Park Geun-hye, the country is
reverting to the old practices of her father's era, repressing freedom
of expression.''

Ms. Park's administration has come under
\href{http://www.nytimes.com/2014/04/30/world/asia/south-korea-ferry-disaster.html}{withering
criticism} since the disaster, first for a botched rescue effort, then
for resisting the kind of broad independent investigation the victims'
families have demanded into the muddled emergency response and the lax
government regulatory system many say helped lead to the sinking.

The painting, which Mr. Hong painted with other artists he invited to
participate, shows the doomed ship at its center, upside down. Two
enormous figures have lifted it out of the water, and --- in an imagined
happier ending --- the passengers are emerging from the boat, smiling
and waving. Surrounding that scene is a phantasmagoria of politically
charged images from South Korean history, some dating from the country's
years of military rule. A prisoner is tortured under interrogation, and
sinister figures lurk, wearing sunglasses and army uniforms.

Gwangju's leaders defended their initial refusal to allow the painting
in the festival, the \href{http://www.gwangjubiennale.org/eng/}{Gwangju
Biennale}, an act that was unexpected in a city with a long history of
resistance to conservative political power.

``We demanded the exclusion of Mr. Hong's painting because of its
explicit political intention, such as the parodying of the president,''
Oh Hyeong-guk, a vice mayor of Gwangju, told reporters this month,
adding that the city could not tolerate such a work in an art exhibition
it helped finance. But as criticism mounted, the city later backtracked
a bit, leaving the final decision to the festival's authorities, who
banned the painting.

Some artists pulled out of the biennale in protest, and a few of its top
officials resigned, saying they were torn between defending Mr. Hong's
freedom of expression and respecting the wishes of the city, one of the
event's main financial backers.

The controversy, which Ms. Park's office has not commented on, has
renewed longstanding questions about the limits to artistic expression
in South Korea.

South Korean artists are vastly freer than they were under military
rule, when a brand of crayon called Picasso was once banned because of
the artist's Communist associations. But artists who venture into
political satire --- like other
\href{http://www.nytimes.com/2012/08/13/world/asia/critics-see-south-korea-internet-curbs-as-censorship.html}{government
critics} --- often say they feel ostracized and harassed, and
unflattering depictions of political leaders can lead to lawsuits and
even criminal defamation charges.

For Mr. Hong, a 59-year-old painter who decades ago was jailed and
tortured for his political expression, the fight over the painting is
the latest skirmish in a long battle with repressive forces he believes
are still at work.

``He is the last standing artist of resistance from the days of
dictatorship,'' said Gim Jong-gil, an art critic.

\includegraphics{https://static01.nyt.com/images/2014/08/31/world/KOREAPAINTER-1/KOREAPAINTER-1-articleLarge.jpg?quality=75\&auto=webp\&disable=upscale}

As a young man, Mr. Hong studied art in Gwangju, which in the 1970s was
a center of activism against the dictatorship of Park Chung-hee, Ms.
Park's father. By the time Mr. Park's rule ended with his assassination
in late 1979, Mr. Hong was an award-winning painter and a participant in
the city's underground pro-democracy movement.

Hopes that Mr. Park's death would lead to democratization were soon
crushed as power was seized by Gen. Chun Doo-hwan, one of Mr. Park's
protégés who was the head of military intelligence during the last year
of his rule. Gwangju erupted in protests, and the regime sent tanks and
paratroopers into the city in May 1980, killing hundreds.

``I saw with my own eyes so many of my friends and colleagues killed,''
Mr. Hong said. ``I decided then and there I would make it my lifetime
duty to record and indict state brutality. Painting is my language, my
picket protest, my placard.''

Under Mr. Chun's rule, Mr. Hong, who was often on the run from the
authorities, produced lithographs depicting scenes from the
\href{http://www.nytimes.com/1996/08/29/world/the-people-of-kwangju-recall-1980-massacre.html}{Gwangju
massacre}. But he is best known for his large canvases, which have often
focused on South Koreans who suffered at the hands of their leaders.

Mr. Hong's works, along with pieces by other activist artists, were put
to use by the student-led democracy movement of the 1980s. Police
officers using tear gas raided university campuses and tore their
paintings down.

Although Mr. Hong escaped imprisonment during the darkest years of
dictatorship, he was arrested in 1989 as the country was moving toward
democracy. Mr. Hong was arrested after sending slides of some of his
work, including a painting that depicted the Gwangju uprising, to
Korean-Americans who were headed to a youth festival in Pyongyang, North
Korea. He was deemed to have violated the National Security Law, still
in effect today, which prohibits any act judged to be ``helping the
enemy'' in North Korea. He was interrogated under torture and spent
three years in prison.

After South Korea's transition to democracy in the 1990s, most artists
who had been active in the resistance to military rule moved on to other
themes. But Mr. Hong continued to produce politically oriented work. In
2012, when the conservative Ms. Park ran for president, he made her a
target. One scathing painting showed Ms. Park doing the now-famous
``Gangnam Style'' dance created by a South Korean performer. She was
dancing below a noose, an allusion to the hanging of dissidents under
her father's regime. Another painting put Ms. Park
\href{http://www.mediatoday.co.kr/news/articleView.html}{in a delivery
room}, having just given birth to a baby who resembles her late father.

Furious conservative politicians have compared Mr. Hong to Joseph
Goebbels. The National Election Committee accused him of violating South
Korean election law, which prohibits defaming candidates with the intent
of preventing their election. But he was not charged. (Another artist,
Lee Ha, was less fortunate; he was indicted after
\href{http://blogs.wsj.com/korearealtime/2013/05/30/pop-artist-challenges-censorship/}{depicting
Ms. Park as Snow White}, holding a rotten apple with her father's image
engraved in its skin. He was acquitted two years later.)

``Sewol Owol,'' Mr. Hong's painting about the ferry sinking, alludes
both to the disaster of the ferry, the Sewol, and the Gwangju killings.
(``Owol'' means May, the month when the massacre occurred.) Both events
hit especially close to home for Mr. Hong, who not only witnessed the
Gwangju murders, but has lived for years in Ansan, the city where the
high school students who died aboard the Sewol were from.

One of those students, a girl who was in her junior year and who he said
came from a poor family, worked part-time at his studio to help earn
money and pick up skills to pursue her dream of being a painter.
``Thirty-four years after the Gwangju massacre, in the Sewol disaster, I
saw another massacre perpetrated by a cartel of crude capitalist
businesses, corrupt bureaucrats and an irresponsible and feckless
government,'' Mr. Hong said, referring to the corporate greed and
government corruption that investigators say
\href{http://www.nytimes.com/2014/07/27/world/asia/in-ferry-deaths-a-south-korean-tycoons-downfall.html}{contributed
to the disaster}. ``This was state brutality.''

After the city's original rejection of his work, he retooled the
painting, slightly. He replaced the caricature of the president with a
chicken, a reference to a nickname used by critics: Geun-hye, the
chicken. Startled city officials rejected that version as well.

Mr. Hong sees the reaction to the Sewol painting as symptomatic of a
dysfunctional society that cannot discuss its problems openly. Such a
society, he said, is prone to disasters.

``Satirizing political power should not be a crime,'' he said.

Advertisement

\protect\hyperlink{after-bottom}{Continue reading the main story}

\hypertarget{site-index}{%
\subsection{Site Index}\label{site-index}}

\hypertarget{site-information-navigation}{%
\subsection{Site Information
Navigation}\label{site-information-navigation}}

\begin{itemize}
\tightlist
\item
  \href{https://help.nytimes.com/hc/en-us/articles/115014792127-Copyright-notice}{©~2020~The
  New York Times Company}
\end{itemize}

\begin{itemize}
\tightlist
\item
  \href{https://www.nytco.com/}{NYTCo}
\item
  \href{https://help.nytimes.com/hc/en-us/articles/115015385887-Contact-Us}{Contact
  Us}
\item
  \href{https://www.nytco.com/careers/}{Work with us}
\item
  \href{https://nytmediakit.com/}{Advertise}
\item
  \href{http://www.tbrandstudio.com/}{T Brand Studio}
\item
  \href{https://www.nytimes.com/privacy/cookie-policy\#how-do-i-manage-trackers}{Your
  Ad Choices}
\item
  \href{https://www.nytimes.com/privacy}{Privacy}
\item
  \href{https://help.nytimes.com/hc/en-us/articles/115014893428-Terms-of-service}{Terms
  of Service}
\item
  \href{https://help.nytimes.com/hc/en-us/articles/115014893968-Terms-of-sale}{Terms
  of Sale}
\item
  \href{https://spiderbites.nytimes.com}{Site Map}
\item
  \href{https://help.nytimes.com/hc/en-us}{Help}
\item
  \href{https://www.nytimes.com/subscription?campaignId=37WXW}{Subscriptions}
\end{itemize}
