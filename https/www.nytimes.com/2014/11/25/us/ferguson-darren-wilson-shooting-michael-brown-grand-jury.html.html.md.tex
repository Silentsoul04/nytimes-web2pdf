Sections

SEARCH

\protect\hyperlink{site-content}{Skip to
content}\protect\hyperlink{site-index}{Skip to site index}

\href{https://www.nytimes.com/section/us}{U.S.}

\href{https://myaccount.nytimes.com/auth/login?response_type=cookie\&client_id=vi}{}

\href{https://www.nytimes.com/section/todayspaper}{Today's Paper}

\href{/section/us}{U.S.}\textbar{}Protests Flare After Ferguson Police
Officer Is Not Indicted

\url{https://nyti.ms/1yNsywu}

\begin{itemize}
\item
\item
\item
\item
\item
\item
\end{itemize}

Advertisement

\protect\hyperlink{after-top}{Continue reading the main story}

Supported by

\protect\hyperlink{after-sponsor}{Continue reading the main story}

\hypertarget{protests-flare-after-ferguson-police-officer-is-not-indicted}{%
\section{Protests Flare After Ferguson Police Officer Is Not
Indicted}\label{protests-flare-after-ferguson-police-officer-is-not-indicted}}

Slide 1 of 16

1/16

A protester confronted police vehicles in Ferguson, Mo., on Monday night
after the grand jury announcement of no indictment.

Credit...Justin Sullivan/Getty Images

\begin{itemize}
\item
  \includegraphics{https://static01.nyt.com/images/2014/11/25/us/subFERGUSON/subFERGUSON-superJumbo.jpg}
\item
  \includegraphics{https://static01.nyt.com/images/2014/11/25/us/25ferguson-hp-slide-OHE6/25ferguson-hp-slide-OHE6-superJumbo.jpg}
\item
  \includegraphics{https://static01.nyt.com/images/2014/11/25/us/25ferguson-hp-slide-LH9B/25ferguson-hp-slide-LH9B-superJumbo.jpg}
\item
  \includegraphics{https://static01.nyt.com/images/2014/11/25/us/25ferguson-hp-slide-7BJ6/25ferguson-hp-slide-7BJ6-superJumbo.jpg}
\item
  \includegraphics{https://static01.nyt.com/images/2014/11/25/us/25ferguson-hp-slide-6F6L/25ferguson-hp-slide-6F6L-superJumbo.jpg}
\item
  \includegraphics{https://static01.nyt.com/images/2014/11/25/us/25ferguson-hp-slide-XYQO/25ferguson-hp-slide-XYQO-superJumbo.jpg}
\item
  \includegraphics{https://static01.nyt.com/images/2014/11/25/us/25ferguson-hp-slide-HPDR/25ferguson-hp-slide-HPDR-superJumbo.jpg}
\item
  \includegraphics{https://static01.nyt.com/images/2014/11/25/us/25ferguson-hp-slide-HODZ/25ferguson-hp-slide-HODZ-superJumbo.jpg}
\item
  \includegraphics{https://static01.nyt.com/images/2014/11/25/us/25ferguson-hp-slide-NB3O/25ferguson-hp-slide-NB3O-superJumbo.jpg}
\item
  \includegraphics{https://static01.nyt.com/images/2014/11/25/us/25ferguson-hp-slide-H680/25ferguson-hp-slide-H680-superJumbo-v2.jpg}
\item
  \includegraphics{https://static01.nyt.com/images/2014/11/25/us/25ferguson-hp-slide-1JLF/25ferguson-hp-slide-1JLF-superJumbo.jpg}
\item
  \includegraphics{https://static01.nyt.com/images/2014/11/25/us/25ferguson-hp-slide-187A/25ferguson-hp-slide-187A-superJumbo.jpg}
\item
  \includegraphics{https://static01.nyt.com/images/2014/11/25/us/25ferguson-hp-slide-W3N1/25ferguson-hp-slide-W3N1-superJumbo.jpg}
\item
  \includegraphics{https://static01.nyt.com/images/2014/11/25/us/25ferguson-hp-slide-JGN6/25ferguson-hp-slide-JGN6-superJumbo.jpg}
\item
  \includegraphics{https://static01.nyt.com/images/2014/11/25/us/25ferguson-hp-slide-ABVU/25ferguson-hp-slide-ABVU-superJumbo.jpg}
\item
  \includegraphics{https://static01.nyt.com/images/2014/11/25/us/25ferguson-hp-slide-MT9N/25ferguson-hp-slide-MT9N-superJumbo-v2.jpg}
\end{itemize}

By \href{http://www.nytimes.com/by/monica-davey}{Monica Davey} and
\href{http://www.nytimes.com/by/julie-bosman}{Julie Bosman}

\begin{itemize}
\item
  Nov. 24, 2014
\item
  \begin{itemize}
  \item
  \item
  \item
  \item
  \item
  \item
  \end{itemize}
\end{itemize}

CLAYTON, Mo. --- A St. Louis County grand jury has brought no criminal
charges against Darren Wilson, a white police officer who fatally shot
Michael Brown, an unarmed African-American teenager, more than three
months ago in nearby Ferguson.

The decision by the grand jury of nine whites and three blacks was
announced Monday night by the St. Louis County prosecutor, Robert P.
McCulloch, at a news conference packed with reporters from around the
world. The killing, on a residential street in Ferguson, set off weeks
of civil unrest --- and a national debate --- fueled by protesters'
outrage over what they called a pattern of police brutality against
young black men. Mr. McCulloch said Officer Wilson had faced charges
ranging from first-degree murder to involuntary manslaughter.

Word of the decision set off a new wave of anger among hundreds who had
gathered outside the Ferguson Police Department. Police officers in riot
gear stood in a line as demonstrators chanted and threw signs and other
objects toward them as the news spread. ``The system failed us again,''
one woman said. In downtown Ferguson, the sound of breaking glass could
be heard as crowds ran through the streets.

As the night went on, the situation grew more intense and chaotic in
several locations around the region. Bottles and rocks were thrown at
officers, and windows of businesses were smashed. Several police cars
were burned; buildings, including a Walgreens, a meat market and a
storage facility, were on fire, and looting was reported in several
businesses. Gunshots could be heard along the streets of Ferguson, and
law enforcement authorities deployed smoke and gas to control the
crowds. In St. Louis, protesters swarmed Interstate 44 and blocked all
traffic near the neighborhood where another man was shot by police this
fall.

Before midnight, St. Louis County police officers reported heavy
automatic gunfire in the area where some of the largest protests were
taking place. Flights to Lambert-St. Louis International Airport were
not permitted to land late Monday as a safety precaution, officials
said.

Mayor James Knowles III of Ferguson, reached on his cellphone late
Monday, said he was there and wanted to see National Guard troops, some
of whom were stationed at a police command center, move to protect his
city. ``They're here in the area,'' he said. ``I don't know why they're
not deploying.''

Just after 1 a.m., Gov. Jay Nixon called up additional members of the
National Guard to Ferguson, where they will provide security for the
police headquarters.

At a news conference around 1:30 a.m., Jon Belmar, the St. Louis County
police chief, said at least a dozen buildings had been set on fire.

\href{https://www.nytimes.com/interactive/2014/11/25/us/evidence-released-in-michael-brown-case.html}{}

\includegraphics{https://static01.nyt.com/images/2014/11/25/us/evidence-released-in-michael-brown-case-1416904064502/evidence-released-in-michael-brown-case-1416904064502-articleLarge.jpg}

\hypertarget{documents-released-in-the-ferguson-case}{%
\subsection{Documents Released in the Ferguson
Case}\label{documents-released-in-the-ferguson-case}}

Documents and evidence presented to the grand jury that was deciding
whether to indict Officer Darren Wilson in the shooting of Michael
Brown.

``As soon as Mr. McCulloch announced the verdict, the officers started
taking rocks and batteries,'' said Chief Belmar, who said he personally
heard about 150 shots fired. He said the police did not fire a shot.

He added that 29 people were arrested.

``I didn't foresee an evening like this,'' Chief Belmar said. The
night's damage had been far worse than any of the nights of unrest that
had followed the shooting in August, he said.

Mr. Brown's family issued a statement expressing sadness, but calling
for peaceful protest and a campaign to require body cameras on police
officers nationwide. ``We are profoundly disappointed that the killer of
our child will not face the consequence of his actions,'' the statement
said. ``While we understand that many others share our pain, we ask that
you channel your frustration in ways that will make a positive change.
We need to work together to fix the system that allowed this to
happen.''

But outside the police station, Lesley McSpadden, Mr. Brown's mother,
voiced frustration with the decision. ``They wrong!'' she yelled,
pointing toward the police officers standing outside of the station.
``Y'all know y'all wrong!''

At the White House, President Obama appealed for peaceful protest and
``care and restraint'' from law enforcement after the grand jury's
decision not to indict Officer Wilson, even as he said the situation
spoke to broader racial challenges in America.

``We have made enormous progress in race relations over the course of
the past several decades,'' Mr. Obama said in the briefing room, where
he made an unusual late-night appearance to respond to the decision.
``But what is also true is that there are still problems, and
communities of color aren't just making these problems up.''

Protests, often well organized and orderly, also occurred in cities
across the country, including Los Angeles, Seattle, Philadelphia and
Chicago, where about 200 mostly young and mostly white protesters
gathered at police headquarters, despite frigid temperatures and light
snow.

In a lengthy news conference, Mr. McCulloch described the series of
events, step by step, that had led to the shooting, and the enormous
array of evidence and witnesses brought before the grand jury. He
described an altercation inside Officer Wilson's vehicle, after which
Officer Wilson had Mr. Brown's blood on his weapon, shirt and pants, the
prosecutor said, as well as swelling and redness on his face.

\href{https://www.nytimes.com/interactive/2014/08/13/us/ferguson-missouri-town-under-siege-after-police-shooting.html}{}

\includegraphics{https://static01.nyt.com/images/2014/08/13/us/ferguson-missouri-town-under-siege-after-police-shooting-1415998664223/ferguson-missouri-town-under-siege-after-police-shooting-1415998664223-videoLarge-v2.png}

\hypertarget{what-happened-in-ferguson}{%
\subsection{What Happened in
Ferguson?}\label{what-happened-in-ferguson}}

Here's what you need to know about events in Ferguson, Mo.

``Physical evidence does not look away as events unfold,'' he said.

Mr. McCulloch also pointed to inconsistent and changing statements from
witnesses, including observations about the position of Mr. Brown's
hands. Some witnesses have said he had his hands up as the final shots
were fired. The prosecutor, who had faced widespread calls to recuse
himself after opponents cited what they called flawed investigations,
took the unusual step of directing his staff to present ``absolutely
everything'' --- rather than a witness or two --- to the grand jury.

Even before the decision was announced, National Guard troops were sent
to a police command post; political leaders, including Governor Nixon,
flew in to hold last-minute meetings with community members; schools
closed for the week; and businesses and residents, including parents of
schoolchildren, braced for what might come next.

Mr. Nixon, who had declared a state of emergency and called up the
Missouri National Guard last week, called for peace and calm in a news
conference several hours before the decision was announced. ``Our shared
hope and expectation is that regardless of the decision, people on all
sides show tolerance, mutual respect and restraint,'' he said.

Yet many here questioned why the authorities would announce the decision
in the evening, rather than waiting for daylight hours. Furious,
sometimes violent, demonstrations and tense clashes with the police took
place late into the night for several weeks in August, and some law
enforcement officers had urged a daytime announcement. Over a period of
weeks, many leaders here had suggested that a Sunday morning
announcement would be best, but the grand jury, which had been meeting
on the case since Aug. 20, finished its work on Monday. Asked about the
timing, Mr. Nixon said it had been the choice of Mr. McCulloch.

\href{https://www.nytimes.com/interactive/2014/11/09/us/10ferguson-michael-brown-shooting-grand-jury-darren-wilson.html}{}

\includegraphics{https://static01.nyt.com/images/2014/08/12/us/JP-STLOUIS3/JP-STLOUIS3-videoLarge-v2.jpg}

\hypertarget{tracking-the-events-in-the-wake-of-michael-browns-shooting}{%
\subsection{Tracking the Events in the Wake of Michael Brown's
Shooting}\label{tracking-the-events-in-the-wake-of-michael-browns-shooting}}

Updates on the events in Ferguson, Mo., following the shooting of
Michael Brown, an unarmed teenager, by a police officer on Aug. 9.

Many of the elaborate plans for how the grand jury's decision would be
released --- including 48-hour notice for the police after the decision
--- appeared to have been scrapped. The family of Mr. Brown, 18, who was
killed by Officer Wilson on Aug. 9, was notified by prosecutors in the
afternoon, after some reports had already appeared on television and
online. A lawyer for the family expressed frustration that they had not
been told sooner.

The lawyer, Benjamin Crump, added that the family would be exploring
their legal options now that the grand jury has failed to indict Officer
Wilson. ``They don't trust this prosecutor; they never did from the
beginning,'' Mr. Crump said. ``And they are going to try to see if they
can do something to get some positive change out of this because they
understand this system needs to be changed.''

Since August, Officer Wilson has stayed close to St. Louis, reading news
articles and following television coverage of the case, those close to
him said. He has made no public statements or appearances. In a private
ceremony in October, he married his fiancée, Barbara Spradling, also a
Ferguson police officer, court records show. Officer Wilson, who
testified before the grand jury for more than four hours, saying he was
convinced that his life was in danger, remains on paid administrative
leave from the police department, but local officials said they expected
that he would resign in the coming days, regardless of the grand jury's
decision.

The Brown family has, by contrast, traveled widely to speak out,
including appearing at the BET Hip Hop Awards, meeting with United
Nations officials in Geneva and talking with protesters near the spot
where Mr. Brown was killed.

Mr. Brown's father, Michael Brown Sr., handed out turkeys to needy
families over the weekend, and he filmed a
\href{https://www.youtube.com/watch?v=x4LkX7PZCoo}{public service
announcement} urging calm once the grand jury decision was announced.
The parents have been pushing for what supporters have called the
Michael Brown Law, which would require officers to wear body cameras.

As the news of the decision spread, school officials were deciding
whether to open schools on Tuesday. At least one district canceled
after-school and evening activities, and at least four announced they
would not hold classes on Tuesday.

All around, there were signs of businesses closing at the prospect of
trouble. At least two area malls, including the St. Louis Galleria and
the Plaza Frontenac, closed early on Monday evening.

Another investigation, a federal civil rights inquiry into the case,
continues, though federal officials have said that the evidence so far
does not support such a case against Officer Wilson. A second federal
investigation is examining whether the Ferguson police have engaged in a
pattern of civil rights violations.

Advertisement

\protect\hyperlink{after-bottom}{Continue reading the main story}

\hypertarget{site-index}{%
\subsection{Site Index}\label{site-index}}

\hypertarget{site-information-navigation}{%
\subsection{Site Information
Navigation}\label{site-information-navigation}}

\begin{itemize}
\tightlist
\item
  \href{https://help.nytimes.com/hc/en-us/articles/115014792127-Copyright-notice}{©~2020~The
  New York Times Company}
\end{itemize}

\begin{itemize}
\tightlist
\item
  \href{https://www.nytco.com/}{NYTCo}
\item
  \href{https://help.nytimes.com/hc/en-us/articles/115015385887-Contact-Us}{Contact
  Us}
\item
  \href{https://www.nytco.com/careers/}{Work with us}
\item
  \href{https://nytmediakit.com/}{Advertise}
\item
  \href{http://www.tbrandstudio.com/}{T Brand Studio}
\item
  \href{https://www.nytimes.com/privacy/cookie-policy\#how-do-i-manage-trackers}{Your
  Ad Choices}
\item
  \href{https://www.nytimes.com/privacy}{Privacy}
\item
  \href{https://help.nytimes.com/hc/en-us/articles/115014893428-Terms-of-service}{Terms
  of Service}
\item
  \href{https://help.nytimes.com/hc/en-us/articles/115014893968-Terms-of-sale}{Terms
  of Sale}
\item
  \href{https://spiderbites.nytimes.com}{Site Map}
\item
  \href{https://help.nytimes.com/hc/en-us}{Help}
\item
  \href{https://www.nytimes.com/subscription?campaignId=37WXW}{Subscriptions}
\end{itemize}
