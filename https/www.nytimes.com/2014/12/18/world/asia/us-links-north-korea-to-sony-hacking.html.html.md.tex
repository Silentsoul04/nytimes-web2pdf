Sections

SEARCH

\protect\hyperlink{site-content}{Skip to
content}\protect\hyperlink{site-index}{Skip to site index}

\href{https://www.nytimes.com/section/world/asia}{Asia Pacific}

\href{https://myaccount.nytimes.com/auth/login?response_type=cookie\&client_id=vi}{}

\href{https://www.nytimes.com/section/todayspaper}{Today's Paper}

\href{/section/world/asia}{Asia Pacific}\textbar{}U.S. Said to Find
North Korea Ordered Cyberattack on Sony

\url{https://nyti.ms/1z5CZPF}

\begin{itemize}
\item
\item
\item
\item
\item
\end{itemize}

Advertisement

\protect\hyperlink{after-top}{Continue reading the main story}

Supported by

\protect\hyperlink{after-sponsor}{Continue reading the main story}

\hypertarget{us-said-to-find-north-korea-ordered-cyberattack-on-sony}{%
\section{U.S. Said to Find North Korea Ordered Cyberattack on
Sony}\label{us-said-to-find-north-korea-ordered-cyberattack-on-sony}}

\includegraphics{https://static01.nyt.com/images/2014/12/18/world/asia/CYBER/CYBER-articleLarge.jpg?quality=75\&auto=webp\&disable=upscale}

By \href{http://www.nytimes.com/by/david-e-sanger}{David E. Sanger} and
\href{http://www.nytimes.com/by/nicole-perlroth}{Nicole Perlroth}

\begin{itemize}
\item
  Dec. 17, 2014
\item
  \begin{itemize}
  \item
  \item
  \item
  \item
  \item
  \end{itemize}
\end{itemize}

WASHINGTON --- American officials have concluded that North Korea was
``centrally involved'' in the hacking of Sony Pictures computers, even
as the studio
\href{http://www.nytimes.com/2014/12/18/business/sony-the-interview-threats.html?_r=0}{canceled
the release} of a far-fetched comedy about the assassination of the
North's leader that is believed to have led to the cyberattack.

Senior administration officials, who would not speak on the record about
the intelligence findings, said the White House was debating whether to
publicly accuse
\href{http://topics.nytimes.com/top/news/international/countriesandterritories/northkorea/index.html?inline=nyt-geo}{North
Korea} of what amounts to a cyberterrorism attack. Sony capitulated
after the hackers
\href{http://www.nytimes.com/2014/12/17/business/media/sony-weighs-terrorism-threat-against-opening-of-the-interview.html}{threatened
additional attacks}, perhaps on theaters themselves, if the movie, ``The
Interview,'' was released.

Officials said it was not clear how the White House would respond. Some
within the Obama administration argue that the government of Kim Jong-un
must be confronted directly. But that raises questions of what actions
the administration could credibly threaten, or how much evidence to make
public without revealing details of how it determined North Korea's
culpability, including the possible penetration of the North's computer
networks.

Other administration officials said a direct confrontation with the
North would provide North Korea with the kind of dispute it covets.
Japan, where Sony is an iconic corporate name, has argued that a public
accusation could interfere with
\href{http://www.nytimes.com/2014/05/30/world/asia/north-korea-agrees-to-investigate-fate-of-japanese-abducted-decades-ago.html?module=Search\&mabReward=relbias\%3As\%2C\%7B\%221\%22\%3A\%22RI\%3A8\%22\%7D}{delicate
diplomatic negotiations} for the return of Japanese citizens kidnapped
years ago.

The government is ``considering a range of options in weighing a
potential response,'' said Bernadette Meehan, a spokeswoman for the
National Security Council.

The administration's sudden urgency came after a new threat was
delivered this week to desktop computers at Sony's offices, warning that
if ``The Interview'' was released on Dec. 25, ``the world will be full
of fear.''

``Remember the 11th of September 2001,'' it said. ``We recommend you to
keep yourself distant from the places at that time.''

Hours before Sony canceled the movie, the four largest theater chains in
the United States ---
\href{http://topics.nytimes.com/top/news/business/companies/regal-entertainment-group/index.html?module=Search\&mabReward=relbias\%3Aw\%2C\%7B\%221\%22\%3A\%22RI\%3A8\%22\%7D}{Regal
Entertainment}, AMC Entertainment,
\href{http://topics.nytimes.com/top/news/business/companies/cinemark-holdings-inc/index.html}{Cinemark}
and
\href{http://topics.nytimes.com/top/news/business/companies/carmike-cinemas-inc/index.html}{Carmike
Cinemas} --- and several smaller chains said they would not show ``The
Interview'' as a result of the threat. The cancellations virtually
killed the movie as a theatrical enterprise, at least in the near term,
one of the first known instances of a threat from another nation
pre-empting the release of a movie.

While intelligence officials have concluded that the
\href{http://www.nytimes.com/2014/12/03/business/media/sony-is-again-target-of-hackers.html}{cyberattack}
was both state-sponsored and far more destructive than any seen before
on American soil, there are still differences of opinion over whether
North Korea was aided by Sony insiders with knowledge of the company's
computer systems, senior administration officials said.

``This is of a different nature than past attacks,'' one official said.

An attack that began by wiping out data on corporate computers ---
something that had been previously seen in
\href{http://www.nytimes.com/2013/03/21/world/asia/south-korea-computer-network-crashes.html?pagewanted=all\&module=Search\&mabReward=relbias\%3Ar\%2C\%7B\%221\%22\%3A\%22RI\%3A8\%22\%7D}{South
Korea} and
\href{http://www.nytimes.com/2012/10/14/world/middleeast/us-suspects-iranians-were-behind-a-wave-of-cyberattacks.html?pagewanted=all\&module=Search\&mabReward=relbias\%3Ar\%2C\%7B\%221\%22\%3A\%22RI\%3A8\%22\%7D}{Saudi
Arabia} --- had turned ``into a threat to the safety of Americans,'' the
official said. But echoing a statement from the Department of Homeland
Security, the official said there was no specific information that an
attack was likely.

It is not clear how the United States determined that Mr. Kim's
government had played a central role in the Sony attacks. North Korea's
computer network has been notoriously difficult to infiltrate. But the
National Security Agency began a major effort four years ago to
penetrate the country's computer operations, including its elite
cyberteam, and to establish ``implants'' in the country's networks that,
like a radar system, would monitor the development of malware
transmitted from the country.

It is hardly a foolproof system. Much of North Korea's hacking is done
from China. And while the attack on Sony used some commonly available
cybertools, one intelligence official said, ``this was of a
sophistication that a year ago we would have said was beyond the North's
capabilities.''

It is rare for the United States to publicly accuse countries suspected
of involvement in cyberintrusions. The administration never publicly
said who attacked White House and State Department computers over the
past two months, or
\href{http://dealbook.nytimes.com/2014/10/02/jpmorgan-discovers-further-cyber-security-issues/?module=Search\&mabReward=relbias\%3Ar\%2C\%7B\%221\%22\%3A\%22RI\%3A8\%22\%7D}{JPMorgan
Chase's systems} last summer. Russia is suspected in the first two
cases, but there is conflicting evidence in the JPMorgan case.

But there is a long forensic trail involving the Sony hacking, several
security researchers said. The attackers used readily available
commercial tools to wipe data off Sony's machines. They also borrowed
tools and techniques that had been used in at least two previous
attacks,
\href{http://www.nytimes.com/2012/10/14/world/middleeast/us-suspects-iranians-were-behind-a-wave-of-cyberattacks.html?pagewanted=all\&module=Search\&mabReward=relbias\%3Ar\%2C\%7B\%221\%22\%3A\%22RI\%3A8\%22\%7D}{one
in Saudi Arabia} two years ago --- widely attributed to Iran --- and
another
\href{http://www.nytimes.com/2013/03/21/world/asia/south-korea-computer-network-crashes.html?pagewanted=all\&module=Search\&mabReward=relbias\%3Ar\%2C\%7B\%221\%22\%3A\%22RI\%3A8\%22\%7D}{last
year in South Korea} aimed at banks and media companies.

The Sony attacks were routed from command-and-control centers across the
world, including a convention center in Singapore and Thammasat
University in Thailand, the researchers said. But one of those servers,
in Bolivia, had been used in limited cyberattacks on South Korean
targets two years ago. That suggested that the same group or individuals
might have been behind the Sony attack.

The Sony malware shares remarkable similarities with that used in
attacks on South Korean banks and broadcasters last year. Those
intrusions, which also destroyed data belonging to their victims, are
believed to have been the work of a cybercriminal gang known as Dark
Seoul. Some experts say they cannot rule out the possibility that the
Sony attack was the work of a Dark Seoul copycat, the security
researchers said.

The Sony attack also borrowed a wiping tool from an attack two years ago
at Saudi Aramco, the national oil company, where hackers wiped off data
on
\href{http://bits.blogs.nytimes.com/2012/08/23/hackers-lay-claim-to-saudi-aramco-cyberattack/?module=Search\&mabReward=relbias\%3Ar\%2C\%7B\%221\%22\%3A\%22RI\%3A8\%22\%7D}{30,000
of the company's computers}, replacing it with
\href{http://bits.blogs.nytimes.com/2012/08/27/connecting-the-dots-after-cyberattack-on-saudi-aramco/?module=Search\&mabReward=relbias\%3Ar\%2C\%7B\%221\%22\%3A\%22RI\%3A8\%22\%7D}{an
image of a burning American flag}.

Security experts were never able to track down those hackers, though
United States officials have long said they believed the attacks
emanated from Iran, using tools that are now on the black market.

At Sony, investigators are looking into the possibility that the
attackers had inside help. Embedded in the malicious code were the names
of Sony servers and administrative credentials that allowed the malware
to spread across Sony's network.

``It's clear that they already had access to Sony's network before the
attack,'' said Jaime Blasco, a researcher at AlienVault, a cybersecurity
consulting firm.

What is remarkable in this case is that after three weeks of pressure,
the attack forced one of Hollywood's largest studios and Japan's most
famous companies to surrender.

Many attacks have been aimed at stealing credit card data, like the
intrusions on the
\href{http://bits.blogs.nytimes.com/2014/09/08/home-depot-confirms-that-it-was-hacked/?module=Search\&mabReward=relbias\%3Ar\%2C\%7B\%221\%22\%3A\%22RI\%3A8\%22\%7D}{Home
Depot} and
\href{http://www.nytimes.com/2014/03/14/business/target-missed-signs-of-a-data-breach.html?module=Search\&mabReward=relbias\%3As\%2C\%7B\%221\%22\%3A\%22RI\%3A8\%22\%7D}{Target}
networks --- and others at disrupting ATMs. An American and Israeli
attack known as Olympic Games that
\href{http://www.nytimes.com/2012/06/01/world/middleeast/obama-ordered-wave-of-cyberattacks-against-iran.html?pagewanted=all\&module=Search\&mabReward=relbias\%3Aw\%2C\%7B\%221\%22\%3A\%22RI\%3A8\%22\%7D}{targeted
Iran's nuclear program} was a rare attack on infrastructure.

Sony has tried to put the best face on the situation, saying it
understood that movie theaters had to be worried about the safety of
their customers.

But the precedent set Wednesday could be damaging. Other countries or
hacking groups could try similar tactics over movies, books or
television broadcasts that they find offensive.

The cost of the assault was small: The attackers used readily available
tools to steal data and then wipe it off Sony's machines. Representative
Mike Rogers, the Michigan Republican who leads the House Intelligence
Committee, said the hackers had ``created a backdoor to Sony's systems''
that they repeatedly re-entered to send threatening messages to Sony
employees.

The North Koreans have half-denied involvement, but have left open the
possibility that the attacks were the ``righteous deed of supporters and
sympathizers.''

But that leaves open the question of what to do about the Sony attack.
The North is under some of the heaviest economic sanctions ever applied.
A large-scale American cyberattack would require a presidential order,
and Mr. Obama has been hesitant to use the country's cyberarsenal for
fear of retaliation.

Advertisement

\protect\hyperlink{after-bottom}{Continue reading the main story}

\hypertarget{site-index}{%
\subsection{Site Index}\label{site-index}}

\hypertarget{site-information-navigation}{%
\subsection{Site Information
Navigation}\label{site-information-navigation}}

\begin{itemize}
\tightlist
\item
  \href{https://help.nytimes.com/hc/en-us/articles/115014792127-Copyright-notice}{©~2020~The
  New York Times Company}
\end{itemize}

\begin{itemize}
\tightlist
\item
  \href{https://www.nytco.com/}{NYTCo}
\item
  \href{https://help.nytimes.com/hc/en-us/articles/115015385887-Contact-Us}{Contact
  Us}
\item
  \href{https://www.nytco.com/careers/}{Work with us}
\item
  \href{https://nytmediakit.com/}{Advertise}
\item
  \href{http://www.tbrandstudio.com/}{T Brand Studio}
\item
  \href{https://www.nytimes.com/privacy/cookie-policy\#how-do-i-manage-trackers}{Your
  Ad Choices}
\item
  \href{https://www.nytimes.com/privacy}{Privacy}
\item
  \href{https://help.nytimes.com/hc/en-us/articles/115014893428-Terms-of-service}{Terms
  of Service}
\item
  \href{https://help.nytimes.com/hc/en-us/articles/115014893968-Terms-of-sale}{Terms
  of Sale}
\item
  \href{https://spiderbites.nytimes.com}{Site Map}
\item
  \href{https://help.nytimes.com/hc/en-us}{Help}
\item
  \href{https://www.nytimes.com/subscription?campaignId=37WXW}{Subscriptions}
\end{itemize}
