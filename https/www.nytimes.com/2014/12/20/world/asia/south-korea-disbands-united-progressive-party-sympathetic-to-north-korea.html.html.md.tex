Sections

SEARCH

\protect\hyperlink{site-content}{Skip to
content}\protect\hyperlink{site-index}{Skip to site index}

\href{https://www.nytimes.com/section/world/asia}{Asia Pacific}

\href{https://myaccount.nytimes.com/auth/login?response_type=cookie\&client_id=vi}{}

\href{https://www.nytimes.com/section/todayspaper}{Today's Paper}

\href{/section/world/asia}{Asia Pacific}\textbar{}South Korea Disbands
Party Sympathetic to North

\url{https://nyti.ms/13jL2uo}

\begin{itemize}
\item
\item
\item
\item
\item
\end{itemize}

Advertisement

\protect\hyperlink{after-top}{Continue reading the main story}

Supported by

\protect\hyperlink{after-sponsor}{Continue reading the main story}

\hypertarget{south-korea-disbands-party-sympathetic-to-north}{%
\section{South Korea Disbands Party Sympathetic to
North}\label{south-korea-disbands-party-sympathetic-to-north}}

By \href{http://www.nytimes.com/by/choe-sang-hun}{Choe Sang-Hun}

\begin{itemize}
\item
  Dec. 19, 2014
\item
  \begin{itemize}
  \item
  \item
  \item
  \item
  \item
  \end{itemize}
\end{itemize}

In the first verdict of its kind, the Constitutional Court of South
Korea on Friday ordered the dissolution of a small leftist party accused
of supporting North Korea at the cost of the South's national security
and in violation of its Constitution.

The court also ordered all of the party's five lawmakers stripped of
their parliamentary seats.

The three-year-old United Progressive Party ``aimed at using violent
means to overthrow our free democratic system'' and ``ultimately
establishing a North Korean-style socialist system,'' the nine-member
court said in its nationally televised ruling.

The ruling marked a political victory for President
\href{http://topics.nytimes.com/top/reference/timestopics/people/p/park_geunhye/index.html}{Park
Geun-hye} and her National Intelligence Service and Justice Ministry,
which
\href{http://www.nytimes.com/2013/11/06/world/asia/south-korean-government-seeks-ban-of-small-leftist-party.html}{filed
a lawsuit} in November 2013 asking the Constitutional Court to disband
the party. The party's estimated 100,000 members have been among the
most vocal critics of Ms. Park, often calling her a reincarnation of her
father, the military dictator Park Chung-hee, who ruled the country with
an iron fist from 1961 to 1979.

With only five seats, the party is a minor force in the 300-member
National Assembly. But the move to dissolve it has incited an intensely
watched yearlong legal battle over the limits to the freedom of
political activities in
\href{http://topics.nytimes.com/top/news/international/countriesandterritories/southkorea/index.html}{South
Korea}.

No political party in democratized South Korea had been shut down by the
government or a court decision since Syngman Rhee, South Korea's
dictatorial founding president, forced the closure of a leftist party in
1958. By law, the Constitutional Court can disband a political party if
six or more of its nine justices agree that the party ``violated the
basic democratic order.'' In its ruling on Friday, the court said that
all but one of its justices agreed on the dissolution of the party.

The verdict took effect immediately. By law, all the assets of the party
will be confiscated by the government. The country's national election
commission said that new elections would take place in April to fill
seats vacated by the ruling on Friday.

Ms. Park's office had no immediate comment on the verdict.

Outside the courthouse, conservative activists cheered at the news,
waving national flags. Park Dae-chool, a spokesman for Ms. Park's
governing Saenuri Party, hailed the verdict as ``a stern judgment
against those who deny the Republic of Korea'' and ``a victory for free
democracy.'' The Republic of Korea is the official name of South Korea.

``Today is the day when democracy in South Korea is pronounced dead,''
Lee Jung-hee, head of the United Progressive Party, said during a rally
of party members outside the courthouse. Ms. Lee accused the
Constitutional Court of ``opening the door for totalitarianism.'' Some
party members wept.

The main opposition party, the New Politics Alliance for Democracy,
issued a statement saying that although it opposed the United
Progressive Party's policies, it also opposed the fate of a political
party being decided by a court ruling, rather than through elections.

Ms. Park took office in February last year amid heightened tensions with
North Korea. The North launched a long-range rocket in December 2012 and
conducted its third nuclear test two weeks before her inauguration.

While promising strong retaliation against any further North Korean
provocation, her government moved against domestic politicians accused
of ``following North Korea.'' It
\href{http://www.nytimes.com/2013/09/05/world/asia/south-korean-lawmakers-back-arrest-of-colleague-for-treason.html}{arrested
the lawmaker Lee Seok-ki} and a few other key members of the United
Progressive Party on highly unusual charges of treason in September last
year. It then filed a lawsuit to disband their party.

Mr. Lee and the others were sentenced to two to nine years in prison by
an appeals court in August on charges of bringing together 130 followers
in May last year and calling for an armed rebellion against the South
Korean government in the event of war on the divided Korean Peninsula.
Mr. Lee's lawyers have denied the charges. The Supreme Court is
scheduled to rule on his case in January.

South Korea remained technically at war with North Korea after the
three-year Korean War ended in 1953 in a truce, not with a peace treaty.
It blocks access to North Korean websites and people are still arrested
for resending Twitter posts of North Korean propaganda materials.

The United Progressive Party's platform calls for ``rectifying our
nation's shameful history, tainted by imperialist invasions, the
national divide, military dictatorship, the tyranny and plunder of
transnational monopoly capital'' and giant family-controlled business
conglomerates. The party wants to end the American military presence,
dismantle South Korea's ``subordinate alliance with the United States''
and unify the North and the South.

Advertisement

\protect\hyperlink{after-bottom}{Continue reading the main story}

\hypertarget{site-index}{%
\subsection{Site Index}\label{site-index}}

\hypertarget{site-information-navigation}{%
\subsection{Site Information
Navigation}\label{site-information-navigation}}

\begin{itemize}
\tightlist
\item
  \href{https://help.nytimes.com/hc/en-us/articles/115014792127-Copyright-notice}{©~2020~The
  New York Times Company}
\end{itemize}

\begin{itemize}
\tightlist
\item
  \href{https://www.nytco.com/}{NYTCo}
\item
  \href{https://help.nytimes.com/hc/en-us/articles/115015385887-Contact-Us}{Contact
  Us}
\item
  \href{https://www.nytco.com/careers/}{Work with us}
\item
  \href{https://nytmediakit.com/}{Advertise}
\item
  \href{http://www.tbrandstudio.com/}{T Brand Studio}
\item
  \href{https://www.nytimes.com/privacy/cookie-policy\#how-do-i-manage-trackers}{Your
  Ad Choices}
\item
  \href{https://www.nytimes.com/privacy}{Privacy}
\item
  \href{https://help.nytimes.com/hc/en-us/articles/115014893428-Terms-of-service}{Terms
  of Service}
\item
  \href{https://help.nytimes.com/hc/en-us/articles/115014893968-Terms-of-sale}{Terms
  of Sale}
\item
  \href{https://spiderbites.nytimes.com}{Site Map}
\item
  \href{https://help.nytimes.com/hc/en-us}{Help}
\item
  \href{https://www.nytimes.com/subscription?campaignId=37WXW}{Subscriptions}
\end{itemize}
