Sections

SEARCH

\protect\hyperlink{site-content}{Skip to
content}\protect\hyperlink{site-index}{Skip to site index}

\href{https://www.nytimes.com/section/technology/personaltech}{Personal
Tech}

\href{https://myaccount.nytimes.com/auth/login?response_type=cookie\&client_id=vi}{}

\href{https://www.nytimes.com/section/todayspaper}{Today's Paper}

\href{/section/technology/personaltech}{Personal Tech}\textbar{}With
Galaxy S6 and S6 Edge, Samsung Tries to Regain Its Footing

\url{https://nyti.ms/1MAd4qs}

\begin{itemize}
\item
\item
\item
\item
\item
\end{itemize}

Advertisement

\protect\hyperlink{after-top}{Continue reading the main story}

Supported by

\protect\hyperlink{after-sponsor}{Continue reading the main story}

\href{/column/state-of-the-art}{State of the Art}

\hypertarget{with-galaxy-s6-and-s6-edge-samsung-tries-to-regain-its-footing}{%
\section{With Galaxy S6 and S6 Edge, Samsung Tries to Regain Its
Footing}\label{with-galaxy-s6-and-s6-edge-samsung-tries-to-regain-its-footing}}

\includegraphics{https://static01.nyt.com/images/2015/04/01/business/02state-web1/02state-web1-articleLarge.jpg?quality=75\&auto=webp\&disable=upscale}

By \href{http://www.nytimes.com/by/farhad-manjoo}{Farhad Manjoo}

\begin{itemize}
\item
  April 1, 2015
\item
  \begin{itemize}
  \item
  \item
  \item
  \item
  \item
  \end{itemize}
\end{itemize}

Samsung's internal code name for its latest top-of-the-line smartphones,
the Galaxy S6 and S6 Edge, is ``Project Zero,'' signaling what Samsung
calls ``a return to fundamentals.''

The code name also suggests that Samsung finally seems to understand the
many criticisms that have long been leveled at its phones: the plastic
hardware looked cheap, the most promoted features were mostly useless
and the software was too complicated.

Samsung, according to Samsung, has realized the errors of it ways.

The realization was born out of necessity. Samsung's market share and
profits in the smartphone business have plummeted over the last year.
The company, which is based in South Korea, is in the unenviable
position of getting squeezed from the bottom by the affordable phones
made by Chinese upstarts like Xiaomi and at the top by Apple's
powerhouse line of iPhones.

The elegant new Galaxy phones, which went on sale in the United States
last week, are aiming to pull Samsung out of that pickle. But while the
phones are magnificent to look at, they are most likely not quite enough
to fix what ails the company.

Despite improved hardware, the S6 and S6 Edge still lack compelling
software.

Unlike Apple, Samsung has never managed to create a built-in suite of
software and services to keep people hooked to its own phones. And there
are few obvious ways for Samsung to address this glaring flaw.

\includegraphics{https://static01.nyt.com/images/2015/04/02/business/STATE-1/STATE-1-articleLarge.jpg?quality=75\&auto=webp\&disable=upscale}

``You can argue that they're in phase one of fixing their software,
which is getting rid of a lot of the junk,'' said
\href{http://www.beyonddevic.es/}{Jan Dawson}, an independent technology
analyst who
\href{https://techpinions.com/is-samsungs-exceptionalism-coming-to-an-end/32325}{anticipated
Samsung's recent troubles}. ``But we haven't really seen phase two,
which would be building its own stuff. We haven't really seen much of
that so far.''

The question of what Samsung can do to differentiate its phones is
urgent. Samsung became the most popular smartphone maker in the world by
producing alternatives to the iPhone at attractive prices, and by
\href{http://www.businessinsider.com/chart-of-the-day-samsung-marketing-budget-2013-6}{outspending
all of its rivals on marketing}. More than any other company, Samsung
developed phones with big screens, a surprising hit with consumers.

But last year, Apple produced its own big phones. They were also a hit,
and Samsung's spiral accelerated.

The holidays were particularly brutal. Samsung's smartphone sales in the
last quarter of 2014 declined from the year before, while the overall
market grew, according to the research firm Gartner. By some estimates
Apple claimed
\href{http://blogs.barrons.com/techtraderdaily/2015/02/09/apple-has-93-of-mobile-profits-650m-users-by-2018-says-canaccord/\#9to5mac}{more
than 90 percent of the profit} in the smartphone industry during the
holidays.

Samsung is still the largest smartphone maker in the world, but its
share fell from about 31 percent to less than 25 percent between 2013
and 2014,\href{http://www.gartner.com/newsroom/id/2996817}{Gartner
reported}. And in China, widely considered the big growth market for
phones, Samsung was ranked fifth behind Xiaomi, Apple, Huawei and Lenovo
during the last quarter,
\href{http://www.idc.com/getdoc.jsp?containerId=prHK25437515}{according
to the research firm IDC}.

Image

Despite improved hardware, the Galaxy phones lack compelling
software.Credit...Samsung

The new S6 and S6 Edge --- which are nearly identical to one another
except that the Edge's screen curves intriguingly, though mostly
uselessly, on its left and right side --- are at least an answer to
critics who say Samsung's devices look cheap.

The S6 phones are made out of aluminum and glass rather than the plastic
in Samsung's older phones. Both the S6 and S6 Edge strongly resemble
Apple's iPhone. The S6 in particular looks like Apple's brother from
another mother. Samsung has also co-opted many of the design ideas for
which its fans have long criticized Apple. The new Galaxys no longer
offer a removable battery, for example, or a slot for add-on storage
cards, and unlike the Galaxy S5, the S6es aren't waterproof.

Samsung goes far in checking off every other hardware box: The S6 and S6
Edge are blazingly fast, their cameras are excellent, their fingerprint
sensors work very well, and --- with an add-on charging pad --- they can
be recharged wirelessly.

But if the new phones are beautiful and functional, they are still
something of a pain to use. The S6 and S6 Edge run Samsung's modified
version of Google's Android operating system. Despite Samsung's
engineers' efforts to clean up the software, the phone's interface is a
hodgepodge of odd design decisions and overly complicated functions.

The situation is made worse by the many companies competing for space on
your phone. Open a new Galaxy and you'll find a host of duplicative apps
preloaded by Samsung, Google, your carrier and even Microsoft, an
ostensible competitor of both Samsung and Google. The crush of apps
would be funny if it weren't so annoying. Why does a brand-new phone
have two web browsers, two email apps, two app stores, a handful of
music and video services and four different messaging apps?

Image

The Galaxy S6. Samsung has been steadily losing ground to rivals,
especially in China.Credit...Samsung

The new phones also do little to help Samsung compete with lower-priced
alternatives in Asia.

In the international market for phones, Samsung's Galaxys are relatively
expensive. They sell for about the same price as Apple's latest devices,
\$199 and up with a two-year contract, or more than \$650 without a
contract. But powerful phones made by low-priced Chinese sellers, like
the
\href{http://www.nytimes.com/2014/10/09/technology/personaltech/oneplus-one-review-high-hopes-for-low-price-phone.html?_r=0}{OnePlus
One,} often sell for less than half the price of high-end Samsung and
Apple devices.

If you pay the premium price to Apple, you get a phone with a
well-designed operating system, no overlapping preloaded apps, and a
host of services that often work very well, like iMessage, Apple Pay and
expanding compatibilities with Apple's personal computers and devices
like the Apple TV and, soon, the Apple Watch. You can criticize Apple's
\href{http://www.nytimes.com/2014/10/23/technology/personaltech/devices-with-yosemite-and-ios-8-operating-systems-seamlessly-connect-in-apples-ecosystem.html}{sticky
ecosystem} as a form of consumer lock-in, but Apple sure has built a
luxurious prison, and customers are willing to pay extra for it.

If you pay that premium to Samsung, you don't get a whole lot more than
you can get on, say, a phone made by Xiaomi, OnePlus or any of a dozen
smaller players.

Samsung appears to understand the dilemma. Minhyouk Lee, the head of
Samsung's mobile design team, said in an email that the company's new
``user experience flow is simpler and easier, with features and settings
that are displayed in a more natural and intuitive way.'' Samsung has
also been working on better services like Samsung Pay, a wireless
payment service that will allow you to use your new Galaxy phones to pay
for items at a wide range of stores --- more stores than can accept
Apple Pay. But you'll have to wait until this summer to use it, when it
goes online in a software update.

Still, Samsung's long history of subpar software might not inspire
droves of customers to buy into its world. What's more, since Samsung's
phones are based on Google's operating system, customers are better off
buying into that company's services because they're usually better
designed and will work on most other Android phones.

``The reality is that Samsung doesn't have anything that's better than
Google's services in most categories, so from the consumer's perspective
it's not clear that there's any benefit for Samsung to make its own
stuff,'' Mr. Dawson said.

Hence the catastrophic question for Samsung: If lots of other, cheaper,
almost-as-good phones run Android, why pay extra for a Samsung?

Advertisement

\protect\hyperlink{after-bottom}{Continue reading the main story}

\hypertarget{site-index}{%
\subsection{Site Index}\label{site-index}}

\hypertarget{site-information-navigation}{%
\subsection{Site Information
Navigation}\label{site-information-navigation}}

\begin{itemize}
\tightlist
\item
  \href{https://help.nytimes.com/hc/en-us/articles/115014792127-Copyright-notice}{©~2020~The
  New York Times Company}
\end{itemize}

\begin{itemize}
\tightlist
\item
  \href{https://www.nytco.com/}{NYTCo}
\item
  \href{https://help.nytimes.com/hc/en-us/articles/115015385887-Contact-Us}{Contact
  Us}
\item
  \href{https://www.nytco.com/careers/}{Work with us}
\item
  \href{https://nytmediakit.com/}{Advertise}
\item
  \href{http://www.tbrandstudio.com/}{T Brand Studio}
\item
  \href{https://www.nytimes.com/privacy/cookie-policy\#how-do-i-manage-trackers}{Your
  Ad Choices}
\item
  \href{https://www.nytimes.com/privacy}{Privacy}
\item
  \href{https://help.nytimes.com/hc/en-us/articles/115014893428-Terms-of-service}{Terms
  of Service}
\item
  \href{https://help.nytimes.com/hc/en-us/articles/115014893968-Terms-of-sale}{Terms
  of Sale}
\item
  \href{https://spiderbites.nytimes.com}{Site Map}
\item
  \href{https://help.nytimes.com/hc/en-us}{Help}
\item
  \href{https://www.nytimes.com/subscription?campaignId=37WXW}{Subscriptions}
\end{itemize}
