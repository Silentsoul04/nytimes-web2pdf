Sections

SEARCH

\protect\hyperlink{site-content}{Skip to
content}\protect\hyperlink{site-index}{Skip to site index}

\href{https://www.nytimes.com/section/us}{U.S.}

\href{https://myaccount.nytimes.com/auth/login?response_type=cookie\&client_id=vi}{}

\href{https://www.nytimes.com/section/todayspaper}{Today's Paper}

\href{/section/us}{U.S.}\textbar{}Cash Flowed to Clinton Foundation Amid
Russian Uranium Deal

\url{https://nyti.ms/1DkztP8}

\begin{itemize}
\item
\item
\item
\item
\item
\item
\end{itemize}

Advertisement

\protect\hyperlink{after-top}{Continue reading the main story}

Supported by

\protect\hyperlink{after-sponsor}{Continue reading the main story}

\hypertarget{cash-flowed-to-clinton-foundation-amid-russian-uranium-deal}{%
\section{Cash Flowed to Clinton Foundation Amid Russian Uranium
Deal}\label{cash-flowed-to-clinton-foundation-amid-russian-uranium-deal}}

\includegraphics{https://static01.nyt.com/images/2015/04/24/us/24URANIUM1/24URANIUM1-articleLarge.jpg?quality=75\&auto=webp\&disable=upscale}

By \href{http://www.nytimes.com/by/jo-becker}{Jo Becker} and
\href{http://www.nytimes.com/by/mike-mcintire}{Mike McIntire}

\begin{itemize}
\item
  April 23, 2015
\item
  \begin{itemize}
  \item
  \item
  \item
  \item
  \item
  \item
  \end{itemize}
\end{itemize}

The headline on the website Pravda trumpeted President Vladimir V.
Putin's latest coup, its nationalistic fervor recalling an era when its
precursor served as the official mouthpiece of the Kremlin: ``Russian
Nuclear Energy Conquers the World.''

The article, in January 2013, detailed how the Russian atomic energy
agency, Rosatom, had taken over a Canadian company with uranium-mining
stakes stretching from Central Asia to the American West. The deal made
Rosatom one of the world's largest uranium producers and brought Mr.
Putin closer to his goal of controlling much of the global uranium
supply chain.

But the untold story behind that story is one that involves not just the
Russian president, but also a former American president and a woman who
would like to be the next one.

At the heart of the tale are several men, leaders of the Canadian mining
industry, who have been major donors to the charitable endeavors of
former President Bill Clinton and his family. Members of that group
built, financed and eventually sold off to the Russians a company that
would become known as Uranium One.

Beyond mines in Kazakhstan that are among the most lucrative in the
world, the sale gave the Russians control of one-fifth of all uranium
production capacity in the United States. Since uranium is considered a
strategic asset, with implications for national security, the deal had
to be approved by a committee composed of representatives from a number
of United States government agencies. Among the agencies that eventually
signed off was the State Department, then headed by Mr. Clinton's wife,
Hillary Rodham Clinton.

As the Russians gradually assumed control of Uranium One in three
separate transactions from 2009 to 2013, Canadian records show, a flow
of cash made its way to the Clinton Foundation. Uranium One's chairman
used his family foundation to make four donations totaling \$2.35
million. Those contributions were not publicly disclosed by the
Clintons, despite an agreement Mrs. Clinton had struck with the Obama
White House to publicly identify all donors. Other people with ties to
the company made donations as well.

And shortly after the Russians announced their intention to acquire a
majority stake in Uranium One, Mr. Clinton received \$500,000 for a
Moscow speech from a Russian investment bank with links to the Kremlin
that was promoting Uranium One stock.

Image

Frank Giustra, right, a mining financier, has donated \$31.3 million to
the foundation run by former President Bill Clinton,
left.Credit...Joaquin Sarmiento/Agence France-Presse --- Getty Images

At the time, both Rosatom and the United States government made promises
intended to ease concerns about ceding control of the company's assets
to the Russians. Those promises have been repeatedly broken, records
show.

The New York Times's examination of the Uranium One deal is based on
dozens of interviews, as well as a review of public records and
securities filings in Canada, Russia and the United States. Some of the
connections between Uranium One and the Clinton Foundation were
unearthed by Peter Schweizer, a former fellow at the right-leaning
Hoover Institution and author of
\href{http://www.nytimes.com/2015/04/20/us/politics/new-book-clinton-cash-questions-foreign-donations-to-foundation.html}{the
forthcoming book ``Clinton Cash.''} Mr. Schweizer provided a preview of
material in the book to The Times, which scrutinized his information and
built upon it with its own reporting.

Whether the donations played any role in the approval of the uranium
deal is unknown. But the episode underscores the special ethical
challenges presented by the Clinton Foundation, headed by a former
president who relied heavily on foreign cash to accumulate \$250 million
in assets even as his wife helped steer American foreign policy as
secretary of state, presiding over decisions with the potential to
benefit the foundation's donors.

In a statement, Brian Fallon, a spokesman for Mrs. Clinton's
presidential campaign, said no one ``has ever produced a shred of
evidence supporting the theory that Hillary Clinton ever took action as
secretary of state to support the interests of donors to the Clinton
Foundation.'' He emphasized that multiple United States agencies, as
well as the Canadian government, had signed off on the deal and that, in
general, such matters were handled at a level below the secretary. ``To
suggest the State Department, under then-Secretary Clinton, exerted
undue influence in the U.S. government's review of the sale of Uranium
One is utterly baseless,'' he added.

American political campaigns are barred from accepting foreign
donations. But foreigners may give to foundations in the United States.
In the days since
\href{http://www.nytimes.com/2015/04/13/us/politics/hillary-clinton-2016-presidential-campaign.html}{Mrs.
Clinton announced her candidacy} for president, the Clinton Foundation
has announced changes meant to quell longstanding concerns about
potential conflicts of interest in such donations; it has limited
donations from foreign governments, with many, like Russia's, barred
from giving to all but its health care initiatives. That policy stops
short of a more stringent agreement between Mrs. Clinton and the Obama
administration that was in effect while she was secretary of state.

Either way, the Uranium One deal highlights the limits of such
prohibitions. The foundation will continue to accept contributions from
foreign sources whose interests, like Uranium One's, may overlap with
those of foreign governments, some of which may be at odds with the
United States.

When the Uranium One deal was approved, the geopolitical backdrop was
far different from today's. The Obama administration was seeking to
``reset'' strained relations with Russia. The deal was strategically
important to Mr. Putin, who shortly after the Americans gave their
blessing sat down for a staged interview with Rosatom's chief executive,
Sergei Kiriyenko. ``Few could have imagined in the past that we would
own 20 percent of U.S. reserves,'' Mr. Kiriyenko told Mr. Putin.

\href{https://www.nytimes.com/interactive/2015/04/23/us/clinton-foundation-donations-uranium-investors.html}{}

\includegraphics{https://static01.nyt.com/images/2015/04/23/us/clinton-foundation-donations-uranium-investors-1429749669022/clinton-foundation-donations-uranium-investors-1429749669022-videoLarge.png}

\hypertarget{donations-to-the-clinton-foundation-and-a-russian-uranium-takeover}{%
\subsection{Donations to the Clinton Foundation, and a Russian Uranium
Takeover}\label{donations-to-the-clinton-foundation-and-a-russian-uranium-takeover}}

Uranium investors gave millions to the Clinton Foundation while
Secretary of State Hillary Rodham Clinton's office was involved in
approving a Russian bid for mining assets in Kazakhstan and the United
States.

Now, after Russia's annexation of Crimea and aggression in Ukraine, the
Moscow-Washington relationship is devolving toward Cold War levels, a
point several experts made in evaluating a deal so beneficial to Mr.
Putin, a man known to use energy resources to project power around the
world.

``Should we be concerned? Absolutely,'' said Michael McFaul, who served
under Mrs. Clinton as the American ambassador to Russia but said he had
been unaware of the Uranium One deal until asked about it. ``Do we want
Putin to have a monopoly on this? Of course we don't. We don't want to
be dependent on Putin for anything in this climate.''

\textbf{A Seat at the Table}

The path to a Russian acquisition of American uranium deposits began in
2005 in Kazakhstan, where the Canadian mining financier Frank Giustra
orchestrated his first big uranium deal, with Mr. Clinton at his side.

The two men had flown aboard Mr. Giustra's private jet to Almaty,
Kazakhstan, where they dined with the authoritarian president, Nursultan
A. Nazarbayev. Mr. Clinton handed the Kazakh president a propaganda coup
when he expressed support for Mr. Nazarbayev's bid to head an
international elections monitoring group, undercutting American foreign
policy and criticism of Kazakhstan's poor human rights record by, among
others, his wife, then a senator.

Within days of the visit, Mr. Giustra's fledgling company, UrAsia Energy
Ltd., signed a preliminary deal giving it stakes in three uranium mines
controlled by the state-run uranium agency Kazatomprom.

If the Kazakh deal was a major victory, UrAsia did not wait long before
resuming the hunt. In 2007, it merged with Uranium One, a South African
company with assets in Africa and Australia, in what was described as a
\$3.5 billion transaction. The new company, which kept the Uranium One
name, was controlled by UrAsia investors including Ian Telfer, a
Canadian who became chairman. Through a spokeswoman, Mr. Giustra, whose
personal stake in the deal was estimated at about \$45 million, said he
sold his stake in 2007.

Soon, Uranium One began to snap up companies with assets in the United
States. In April 2007, it announced the purchase of a uranium mill in
Utah and more than 38,000 acres of uranium exploration properties in
four Western states, followed quickly by the acquisition of the Energy
Metals Corporation and its uranium holdings in Wyoming, Texas and Utah.
That deal made clear that Uranium One was intent on becoming ``a
powerhouse in the United States uranium sector with the potential to
become the domestic supplier of choice for U.S. utilities,'' the company
declared.

Image

Ian Telfer was chairman of Uranium One and made large donations to the
Clinton Foundation.Credit...Galit Rodan/Bloomberg, via Getty Images

Still, the company's story was hardly front-page news in the United
States --- until early 2008, in the midst of Mrs. Clinton's failed
presidential campaign, when The Times published an article revealing the
2005 trip's link to Mr. Giustra's Kazakhstan mining deal. It also
reported that several months later, Mr. Giustra had
\href{http://www.nytimes.com/2008/01/31/us/politics/31donor.html}{donated
\$31.3 million} to Mr. Clinton's foundation.

(In a statement issued after this article appeared online, Mr. Giustra
said he was ``extremely proud'' of his charitable work with Mr. Clinton,
and he urged the media to focus on poverty, health care and ``the real
challenges of the world.'')

Though the 2008 article quoted the former head of Kazatomprom, Moukhtar
Dzhakishev, as saying that the deal required government approval and was
discussed at a dinner with the president, Mr. Giustra insisted that it
was a private transaction, with no need for Mr. Clinton's influence with
Kazakh officials. He described his relationship with Mr. Clinton as
motivated solely by a shared interest in philanthropy.

As if to underscore the point, five months later Mr. Giustra held a
fund-raiser for the Clinton Giustra Sustainable Growth Initiative, a
project aimed at fostering progressive environmental and labor practices
in the natural resources industry, to which he had pledged \$100
million. The star-studded gala, at a conference center in Toronto,
featured performances by Elton John and Shakira and celebrities like Tom
Cruise, John Travolta and Robin Williams encouraging contributions from
the many so-called F.O.F.s --- Friends of Frank --- in attendance, among
them Mr. Telfer. In all, the evening generated \$16 million in pledges,
according to an article in The Globe and Mail.

``None of this would have been possible if Frank Giustra didn't have a
remarkable combination of caring and modesty, of vision and energy and
iron determination,'' Mr. Clinton told those gathered, adding: ``I love
this guy, and you should, too.''

But what had been a string of successes was about to hit a speed bump.

\textbf{Arrest and Progress}

By June 2009, a little over a year after the star-studded evening in
Toronto, Uranium One's stock was in free-fall, down 40 percent. Mr.
Dzhakishev, the head of Kazatomprom, had just been arrested on charges
that he illegally sold uranium deposits to foreign companies, including
at least some of those won by Mr. Giustra's UrAsia and now owned by
Uranium One.

Publicly, the company tried to reassure shareholders. Its chief
executive, Jean Nortier, issued a confident statement calling the
situation a ``complete misunderstanding.'' He also contradicted Mr.
Giustra's contention that the uranium deal had not required government
blessing. ``When you do a transaction in Kazakhstan, you need the
government's approval,'' he said, adding that UrAsia had indeed received
that approval.

\includegraphics{https://static01.nyt.com/images/2015/04/24/us/JPURANIUM4/JPURANIUM4-articleLarge-v2.jpg?quality=75\&auto=webp\&disable=upscale}

But privately, Uranium One officials were worried they could lose their
joint mining ventures. American diplomatic cables made public by
WikiLeaks also reflect concerns that Mr. Dzhakishev's arrest was part of
a Russian power play for control of Kazakh uranium assets.

At the time, Russia was already eying a stake in Uranium One, Rosatom
company documents show. Rosatom officials say they were seeking to
acquire mines around the world because Russia lacks sufficient domestic
reserves to meet its own industry needs.

It was against this backdrop that the Vancouver-based Uranium One
pressed the American Embassy in Kazakhstan, as well as Canadian
diplomats, to take up its cause with Kazakh officials, according to the
American cables.

``We want more than a statement to the press,'' Paul Clarke, a Uranium
One executive vice president, told the embassy's energy officer on June
10, the officer reported in a cable. ``That is simply chitchat.'' What
the company needed, Mr. Clarke said, was official written confirmation
that the licenses were valid.

The American Embassy ultimately reported to the secretary of state, Mrs.
Clinton. Though the Clarke cable was copied to her, it was given wide
circulation, and it is unclear if she would have read it; the Clinton
campaign did not address questions about the cable.

What is clear is that the embassy acted, with the cables showing that
the energy officer met with Kazakh officials to discuss the issue on
June 10 and 11.

Three days later, a wholly owned subsidiary of Rosatom completed a deal
for 17 percent of Uranium One. And within a year, the Russian government
substantially upped the ante, with a generous offer to shareholders that
would give it a 51 percent controlling stake. But first, Uranium One had
to get the American government to sign off on the deal.

\textbf{The Power to Say No}

When a company controlled by the Chinese government sought a 51 percent
stake in a tiny Nevada gold mining operation in 2009, it set off a
secretive review process in Washington, where officials raised concerns
primarily about the mine's proximity to a military installation, but
also about the potential for minerals at the site, including uranium, to
come under Chinese control. The officials killed the deal.

Such is the power of the Committee on Foreign Investment in the United
States. The committee comprises some of the most powerful members of the
cabinet, including the attorney general, the secretaries of the
Treasury, Defense, Homeland Security, Commerce and Energy, and the
secretary of state. They are charged with reviewing any deal that could
result in foreign control of an American business or asset deemed
important to national security.

The national security issue at stake in the Uranium One deal was not
primarily about nuclear weapons proliferation; the United States and
Russia had for years cooperated on that front, with Russia sending
enriched fuel from decommissioned warheads to be used in American
nuclear power plants in return for raw uranium.

Instead, it concerned American dependence on foreign uranium sources.
While the United States gets one-fifth of its electrical power from
nuclear plants, it produces only around 20 percent of the uranium it
needs, and most plants have only 18 to 36 months of reserves, according
to Marin Katusa, author of ``The Colder War: How the Global Energy Trade
Slipped From America's Grasp.''

``The Russians are easily winning the uranium war, and nobody's talking
about it,'' said Mr. Katusa, who explores the implications of the
Uranium One deal in his book. ``It's not just a domestic issue but a
foreign policy issue, too.''

When ARMZ, an arm of Rosatom, took its first 17 percent stake in Uranium
One in 2009, the two parties signed an agreement, found in securities
filings, to seek the foreign investment committee's review. But it was
the 2010 deal, giving the Russians a controlling 51 percent stake, that
set off alarm bells. Four members of the House of Representatives signed
a letter expressing concern. Two more began pushing legislation to kill
the deal.

Senator John Barrasso, a Republican from Wyoming, where Uranium One's
largest American operation was, wrote to President Obama, saying the
deal ``would give the Russian government control over a sizable portion
of America's uranium production capacity.''

Image

President Putin during a meeting with Rosatom's chief executive, Sergei
Kiriyenko, in December 2007.Credit...Dmitry Astakhov/Ria Novosti, via
Agence France-Presse --- Getty Images

``Equally alarming,'' Mr. Barrasso added, ``this sale gives ARMZ a
significant stake in uranium mines in Kazakhstan.''

Uranium One's shareholders were also alarmed, and were ``afraid of
Rosatom as a Russian state giant,'' Sergei Novikov, a company spokesman,
recalled in an interview. He said Rosatom's chief, Mr. Kiriyenko, sought
to reassure Uranium One investors, promising that Rosatom would not
break up the company and would keep the same management, including Mr.
Telfer, the chairman. Another Rosatom official said publicly that it did
not intend to increase its investment beyond 51 percent, and that it
envisioned keeping Uranium One a public company

American nuclear officials, too, seemed eager to assuage fears. The
Nuclear Regulatory Commission wrote to Mr. Barrasso assuring him that
American uranium would be preserved for domestic use, regardless of who
owned it.

``In order to export uranium from the United States, Uranium One Inc. or
ARMZ would need to apply for and obtain a specific NRC license
authorizing the export of uranium for use as reactor fuel,'' the letter
said.

Still, the ultimate authority to approve or reject the Russian
acquisition rested with the cabinet officials on the foreign investment
committee, including Mrs. Clinton --- whose husband was collecting
millions in donations from people associated with Uranium One.

\textbf{Undisclosed Donations}

Before Mrs. Clinton could assume her post as secretary of state, the
White House demanded that she sign a memorandum of understanding placing
limits on the activities of her husband's foundation. To avoid the
perception of conflicts of interest, beyond the ban on foreign
government donations, the foundation was required to publicly disclose
all contributors.

To judge from those disclosures --- which list the contributions in
ranges rather than precise amounts --- the only Uranium One official to
give to the Clinton Foundation was Mr. Telfer, the chairman, and the
amount was relatively small: no more than \$250,000, and that was in
2007, before talk of a Rosatom deal began percolating.

Image

Uranium One's Russian takeover was approved by the United States while
Hillary Rodham Clinton was secretary of state.Credit...Doug Mills/The
New York Times

But a review of tax records in Canada, where Mr. Telfer has a family
charity called the Fernwood Foundation, shows that he donated millions
of dollars more, during and after the critical time when the foreign
investment committee was reviewing his deal with the Russians. With the
Russians offering a special dividend, shareholders like Mr. Telfer stood
to profit.

His donations through the Fernwood Foundation included \$1 million
reported in 2009, the year his company appealed to the American Embassy
to help it keep its mines in Kazakhstan; \$250,000 in 2010, the year the
Russians sought majority control; as well as \$600,000 in 2011 and
\$500,000 in 2012. Mr. Telfer said that his donations had nothing to do
with his business dealings, and that he had never discussed Uranium One
with Mr. or Mrs. Clinton. He said he had given the money because he
wanted to support Mr. Giustra's charitable endeavors with Mr. Clinton.
``Frank and I have been friends and business partners for almost 20
years,'' he said.

The Clinton campaign left it to the foundation to reply to questions
about the Fernwood donations; the foundation did not provide a response.

Mr. Telfer's undisclosed donations came in addition to between \$1.3
million and \$5.6 million in contributions, which were reported, from a
constellation of people with ties to Uranium One or UrAsia, the company
that originally acquired Uranium One's most valuable asset: the Kazakh
mines. Without those assets, the Russians would have had no interest in
the deal: ``It wasn't the goal to buy the Wyoming mines. The goal was to
acquire the Kazakh assets, which are very good,'' Mr. Novikov, the
Rosatom spokesman, said in an interview.

Amid this influx of Uranium One-connected money, Mr. Clinton was invited
to speak in Moscow in June 2010, the same month Rosatom struck its deal
for a majority stake in Uranium One.

The \$500,000 fee --- among Mr. Clinton's highest --- was paid by
Renaissance Capital, a Russian investment bank with ties to the Kremlin
that has invited world leaders, including Tony Blair, the former British
prime minister, to speak at its investor conferences.

Renaissance Capital analysts talked up Uranium One's stock, assigning it
a ``buy'' rating and saying in a July 2010 research report that it was
``the best play'' in the uranium markets. In addition, Renaissance
Capital turned up that same year as a major donor, along with Mr.
Giustra and several companies linked to Uranium One or UrAsia, to a
small medical charity in Colorado run by a friend of Mr. Giustra's. In a
newsletter to supporters, the friend credited Mr. Giustra with helping
get donations from ``businesses around the world.''

Image

John Christensen sold the mining rights on his ranch in Wyoming to
Uranium One.Credit...Matthew Staver for The New York Times

Renaissance Capital would not comment on the genesis of Mr. Clinton's
speech to an audience that included leading Russian officials, or on
whether it was connected to the Rosatom deal. According to a Russian
government news service, Mr. Putin personally thanked Mr. Clinton for
speaking.

A person with knowledge of the Clinton Foundation's fund-raising
operation, who requested anonymity to speak candidly about it, said that
for many people, the hope is that money will in fact buy influence:
``Why do you think they are doing it --- because they love them?'' But
whether it actually does is another question. And in this case, there
were broader geopolitical pressures that likely came into play as the
United States considered whether to approve the Rosatom-Uranium One
deal.

\textbf{Diplomatic Considerations}

If doing business with Rosatom was good for those in the Uranium One
deal, engaging with Russia was also a priority of the incoming Obama
administration, which was hoping for a new era of cooperation as Mr.
Putin relinquished the presidency --- if only for a term --- to Dmitri
A. Medvedev.

``The assumption was we could engage Russia to further core U.S.
national security interests,'' said Mr. McFaul, the former ambassador.

It started out well. The two countries made progress on nuclear
proliferation issues, and expanded use of Russian territory to resupply
American forces in Afghanistan. Keeping Iran from obtaining a nuclear
weapon was among the United States' top priorities, and in June 2010
Russia signed off on a United Nations resolution imposing tough new
sanctions on that country.

Two months later, the deal giving ARMZ a controlling stake in Uranium
One was submitted to the Committee on Foreign Investment in the United
States for review. Because of the secrecy surrounding the process, it is
hard to know whether the participants weighed the desire to improve
bilateral relations against the potential risks of allowing the Russian
government control over the biggest uranium producer in the United
States. The deal was ultimately approved in October, following what two
people involved in securing the approval said had been a relatively
smooth process.

Not all of the committee's decisions are personally debated by the
agency heads themselves; in less controversial cases, deputy or
assistant secretaries may sign off. But experts and former committee
members say Russia's interest in Uranium One and its American uranium
reserves seemed to warrant attention at the highest levels.

Image

Moukhtar Dzhakishev was arrested in 2009 while the chief of
Kazatomprom.Credit...Daniel Acker/Bloomberg, via Getty Images

``This deal had generated press, it had captured the attention of
Congress and it was strategically important,'' said Richard Russell, who
served on the committee during the George W. Bush administration. ``When
I was there invariably any one of those conditions would cause this to
get pushed way up the chain, and here you had all three.''

And Mrs. Clinton brought a reputation for hawkishness to the process; as
a senator, she was a vocal critic of the committee's approval of a deal
that would have transferred the management of major American seaports to
a company based in the United Arab Emirates, and as a presidential
candidate she had advocated legislation to strengthen the process.

The Clinton campaign spokesman, Mr. Fallon, said that in general, these
matters did not rise to the secretary's level. He would not comment on
whether Mrs. Clinton had been briefed on the matter, but he gave The
Times a statement from the former assistant secretary assigned to the
foreign investment committee at the time, Jose Fernandez. While not
addressing the specifics of the Uranium One deal, Mr. Fernandez said,
``Mrs. Clinton never intervened with me on any C.F.I.U.S. matter.''

Mr. Fallon also noted that if any agency had raised national security
concerns about the Uranium One deal, it could have taken them directly
to the president.

Anne-Marie Slaughter, the State Department's director of policy planning
at the time, said she was unaware of the transaction --- or the extent
to which it made Russia a dominant uranium supplier. But speaking
generally, she urged caution in evaluating its wisdom in hindsight.

``Russia was not a country we took lightly at the time or thought was
cuddly,'' she said. ``But it wasn't the adversary it is today.''

That renewed adversarial relationship has raised concerns about European
dependency on Russian energy resources, including nuclear fuel. The
unease reaches beyond diplomatic circles. In Wyoming, where Uranium One
equipment is scattered across his 35,000-acre ranch, John Christensen is
frustrated that repeated changes in corporate ownership over the years
led to French, South African, Canadian and, finally, Russian control
over mining rights on his property.

``I hate to see a foreign government own mining rights here in the
United States,'' he said. ``I don't think that should happen.''

Mr. Christensen, 65, noted that despite assurances by the Nuclear
Regulatory Commission that uranium could not leave the country without
Uranium One or ARMZ obtaining an export license --- which they do not
have --- yellowcake from his property was routinely packed into drums
and trucked off to a processing plant in Canada.

Asked about that, the commission confirmed that Uranium One has, in
fact, shipped yellowcake to Canada even though it does not have an
export license. Instead, the transport company doing the shipping, RSB
Logistic Services, has the license. A commission spokesman said that
``to the best of our knowledge'' most of the uranium sent to Canada for
processing was returned for use in the United States. A Uranium One
spokeswoman, Donna Wichers, said 25 percent had gone to Western Europe
and Japan. At the moment, with the uranium market in a downturn, nothing
is being shipped from the Wyoming mines.

The ``no export'' assurance given at the time of the Rosatom deal is not
the only one that turned out to be less than it seemed. Despite pledges
to the contrary, Uranium One was delisted from the Toronto Stock
Exchange and taken private. As of 2013, Rosatom's subsidiary, ARMZ,
owned 100 percent of it.

Advertisement

\protect\hyperlink{after-bottom}{Continue reading the main story}

\hypertarget{site-index}{%
\subsection{Site Index}\label{site-index}}

\hypertarget{site-information-navigation}{%
\subsection{Site Information
Navigation}\label{site-information-navigation}}

\begin{itemize}
\tightlist
\item
  \href{https://help.nytimes.com/hc/en-us/articles/115014792127-Copyright-notice}{©~2020~The
  New York Times Company}
\end{itemize}

\begin{itemize}
\tightlist
\item
  \href{https://www.nytco.com/}{NYTCo}
\item
  \href{https://help.nytimes.com/hc/en-us/articles/115015385887-Contact-Us}{Contact
  Us}
\item
  \href{https://www.nytco.com/careers/}{Work with us}
\item
  \href{https://nytmediakit.com/}{Advertise}
\item
  \href{http://www.tbrandstudio.com/}{T Brand Studio}
\item
  \href{https://www.nytimes.com/privacy/cookie-policy\#how-do-i-manage-trackers}{Your
  Ad Choices}
\item
  \href{https://www.nytimes.com/privacy}{Privacy}
\item
  \href{https://help.nytimes.com/hc/en-us/articles/115014893428-Terms-of-service}{Terms
  of Service}
\item
  \href{https://help.nytimes.com/hc/en-us/articles/115014893968-Terms-of-sale}{Terms
  of Sale}
\item
  \href{https://spiderbites.nytimes.com}{Site Map}
\item
  \href{https://help.nytimes.com/hc/en-us}{Help}
\item
  \href{https://www.nytimes.com/subscription?campaignId=37WXW}{Subscriptions}
\end{itemize}
