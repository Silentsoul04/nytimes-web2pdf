Sections

SEARCH

\protect\hyperlink{site-content}{Skip to
content}\protect\hyperlink{site-index}{Skip to site index}

\href{https://myaccount.nytimes.com/auth/login?response_type=cookie\&client_id=vi}{}

\href{https://www.nytimes.com/section/todayspaper}{Today's Paper}

\href{/section/business/dealbook}{DealBook}\textbar{}An Activist
Investor Takes Aim at Bid for Samsung

\url{https://nyti.ms/1RKStPo}

\begin{itemize}
\item
\item
\item
\item
\item
\end{itemize}

Advertisement

\protect\hyperlink{after-top}{Continue reading the main story}

Supported by

\protect\hyperlink{after-sponsor}{Continue reading the main story}

DealBook Business and Policy

\hypertarget{an-activist-investor-takes-aim-at-bid-for-samsung}{%
\section{An Activist Investor Takes Aim at Bid for
Samsung}\label{an-activist-investor-takes-aim-at-bid-for-samsung}}

\includegraphics{https://static01.nyt.com/images/2015/06/05/business/05db-samsung/05db-samsung-articleLarge.jpg?quality=75\&auto=webp\&disable=upscale}

By \href{http://www.nytimes.com/by/neil-gough}{Neil Gough} and
\href{http://www.nytimes.com/by/choe-sang-hun}{Choe Sang-Hun}

\begin{itemize}
\item
  June 3, 2015
\item
  \begin{itemize}
  \item
  \item
  \item
  \item
  \item
  \end{itemize}
\end{itemize}

HONG KONG --- One of America's biggest activist hedge funds is making a
rare foray into Asia, betting that it can alter the restructuring plans
of the heavyweight Samsung.

The move by Elliott Management, run by Paul E. Singer, essentially
threatens efforts by the family that controls the South Korean tech
giant to consolidate its hold over a sprawling corporate empire. The
fund, which bought a 7.1 percent stake in a Samsung Group company that
Samsung had planned to sell to another unit, objected on Thursday to
what it said was a low sale price.

Elliott's move is an unusual glimpse of investor activism in a region
where such campaigns are seldom seen and where they have met mixed
results. Family or state shareholders wield control of some of the
biggest public companies in Asia and often bristle at outsiders' telling
them how to run their businesses.

The Samsung patriarch, Lee Kun-hee, 73, has been ailing since May 2014,
when he was hospitalized after a heart attack. His son and heir
apparent, Jay Y. Lee, is vice chairman of Samsung Electronics and holds
a 23 percent stake in what is now effectively the main holding company
for the Lee family's interests across Samsung Group: Cheil Industries.

Last month, Cheil began an all-shares merger bid, worth a minimum of \$8
billion, for another group company, Samsung C\&T, a construction and
industrial investment business that also owns a valuable stake in
Samsung Electronics.

Elliott said it had paid about \$630 million for its stake in Samsung
C\&T. In a statement, the hedge fund said that it objected to Cheil's
bid because it ``significantly undervalues Samsung C\&T.''

``The terms are neither fair to nor in the best interests of Samsung
C\&T's shareholders,'' the statement added.

In a separate regulatory filing in Seoul, Elliott, which has \$26
billion in assets under management, said it bought the shares ``for the
purpose of participating in management'' of Samsung C\&T.

Analysts said Elliott was effectively challenging the Lee family's plans
to consolidate its hold over the dozens of companies that make up the
Samsung Group.

``The company to be created through the merger will be at the peak of
the system of management control over the entire Samsung Group, even if
we cannot call it a holding company for the conglomerate,'' said Cho
Yoon-ho, an analyst at Dongbu Securities in Seoul.

Mr. Cho and others said Elliott was unlikely to be able to block the
deal. But he said that complicating the restructuring process could lead
to ``capital gains'' if the share price rose further, increasing the
pressure on Cheil.

Samsung C\&T on Thursday rejected Elliott's claim that the terms of the
merger plan were poor. In a news release, the company said that it had
pushed for the deal ``to strengthen the future value of the company and
eventually the values of stock owners.''

The statement added that Samsung C\&T would ``communicate with various
shareholders and try to boost the company's value.'' Representatives of
Cheil could not be reached for comment.

Last year, Elliot built up a small stake in the Bank of East Asia, one
of Hong Kong's biggest local banks, which is controlled by the family of
David Li. But its stake was diluted after the bank sold new shares to
the Sumitomo Mitsui Financial Group of Japan. Elliott sought to
challenge the decision, which some analysts described as a defensive
move by the Li family. The matter is working its way through the courts
in Hong Kong.

Other activist hedge funds have also met resistance in Asia. Two years
ago, another Wall Street firm, the billionaire Daniel S. Loeb's Third
Point, built a substantial stake in Sony and began pressing the company
for a board seat. It also sought to have the company spin off part of
its entertainment arm, which includes a Hollywood film studio and a
music label.

Mr. Loeb was largely rebuffed and eventually sold his Sony stake. More
recently, he has found his new investment in Fanuc, the Japanese robot
maker, better received. After a recent meeting with the company's
management, Fanuc agreed to double its dividend.

``Nobody thought this could be done,'' Mr. Loeb told the audience at an
investment conference last month.

Elliott will be hoping for a similarly warm response to its criticism of
Cheil's offer for Samsung C\&T. Under the terms of the deal,
shareholders will receive 0.35 share in Cheil for each share they hold
in Samsung C\&T. The timing of the deal looks favorable to Cheil, whose
businesses include fashion, construction and theme parks and whose own
stock has soared since its initial public offering in December. Samsung
C\&T's shares have underperformed in the same period.

Shareholders of Samsung C\&T are set to vote on the offer in July, and
the merger needs a two-thirds majority to pass.

``Elliott's 7 percent alone cannot block the merger,'' said Park
Hyung-ryul, an analyst at Daewoo Securities in Seoul.

Other Samsung subsidiaries control a combined 19 percent of Samsung
C\&T. But the biggest single shareholder is the National Pension
Service, with a 10 percent stake. The pension service effectively
blocked a planned \$2.4 billion merger between Samsung Heavy Industries
and Samsung Engineering late last year, when it exercised an option
under South Korean securities law to compel the companies to buy back
its shares.

The companies scrapped the deal because the buyback would have been too
costly.

Likewise, shareholders in Cheil and Samsung C\&T who oppose the merger
can also force the companies to buy back their shares. But in this case,
the price for the option is well below where the shares in both
companies are trading now, so exercising it would effectively mean
selling at a loss.

``That also makes any opposition of the merger unlikely,'' Mr. Park
added.

Advertisement

\protect\hyperlink{after-bottom}{Continue reading the main story}

\hypertarget{site-index}{%
\subsection{Site Index}\label{site-index}}

\hypertarget{site-information-navigation}{%
\subsection{Site Information
Navigation}\label{site-information-navigation}}

\begin{itemize}
\tightlist
\item
  \href{https://help.nytimes.com/hc/en-us/articles/115014792127-Copyright-notice}{©~2020~The
  New York Times Company}
\end{itemize}

\begin{itemize}
\tightlist
\item
  \href{https://www.nytco.com/}{NYTCo}
\item
  \href{https://help.nytimes.com/hc/en-us/articles/115015385887-Contact-Us}{Contact
  Us}
\item
  \href{https://www.nytco.com/careers/}{Work with us}
\item
  \href{https://nytmediakit.com/}{Advertise}
\item
  \href{http://www.tbrandstudio.com/}{T Brand Studio}
\item
  \href{https://www.nytimes.com/privacy/cookie-policy\#how-do-i-manage-trackers}{Your
  Ad Choices}
\item
  \href{https://www.nytimes.com/privacy}{Privacy}
\item
  \href{https://help.nytimes.com/hc/en-us/articles/115014893428-Terms-of-service}{Terms
  of Service}
\item
  \href{https://help.nytimes.com/hc/en-us/articles/115014893968-Terms-of-sale}{Terms
  of Sale}
\item
  \href{https://spiderbites.nytimes.com}{Site Map}
\item
  \href{https://help.nytimes.com/hc/en-us}{Help}
\item
  \href{https://www.nytimes.com/subscription?campaignId=37WXW}{Subscriptions}
\end{itemize}
