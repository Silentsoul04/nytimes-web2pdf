Sections

SEARCH

\protect\hyperlink{site-content}{Skip to
content}\protect\hyperlink{site-index}{Skip to site index}

\href{https://www.nytimes.com/section/us}{U.S.}

\href{https://myaccount.nytimes.com/auth/login?response_type=cookie\&client_id=vi}{}

\href{https://www.nytimes.com/section/todayspaper}{Today's Paper}

\href{/section/us}{U.S.}\textbar{}Hacking Linked to China Exposes
Millions of U.S. Workers

\href{https://nyti.ms/1KHE6r1}{https://nyti.ms/1KHE6r1}

\begin{itemize}
\item
\item
\item
\item
\item
\item
\end{itemize}

Advertisement

\protect\hyperlink{after-top}{Continue reading the main story}

Supported by

\protect\hyperlink{after-sponsor}{Continue reading the main story}

\hypertarget{hacking-linked-to-china-exposes-millions-of-us-workers}{%
\section{Hacking Linked to China Exposes Millions of U.S.
Workers}\label{hacking-linked-to-china-exposes-millions-of-us-workers}}

By \href{http://www.nytimes.com/by/david-e-sanger}{David E. Sanger} and
\href{https://www.nytimes.com/by/julie-hirschfeld-davis}{Julie
Hirschfeld Davis}

\begin{itemize}
\item
  June 4, 2015
\item
  \begin{itemize}
  \item
  \item
  \item
  \item
  \item
  \item
  \end{itemize}
\end{itemize}

WASHINGTON --- The Obama administration on Thursday announced what
appeared to be one of the largest breaches of federal employees' data,
involving at least four million current and former government workers in
an intrusion that officials said apparently originated in China.

The compromised data was held by the Office of Personnel Management,
which handles government security clearances and federal employee
records. The breach was first detected in April, the office said, but it
appears to have begun at least late last year.

The target appeared to be Social Security numbers and other ``personal
identifying information,'' but it was unclear whether the attack was
related to commercial gain or espionage. The announcement of the
intrusion came on the same day The New York Times reported that the
National Security Agency had expanded warrantless surveillance of
foreign hackers, an effort that could sweep up the information of
innocent Americans.

There seemed to be little doubt among federal officials that the attack
was launched from China, but it was unclear whether it might have been
state sponsored. The administration did not publicly identify Chinese
hackers as the culprits because it is difficult to definitively
attribute the source of cyberattacks and to back up such an attribution
without divulging classified data.

The breach is the third major foreign intrusion into an important
federal computer system in the past year. Last year, the White House and
the State Department found that their email systems had been compromised
in an attack that was attributed to Russian hackers.
\href{http://www.nytimes.com/2015/04/26/us/russian-hackers-read-obamas-unclassified-emails-officials-say.html?_r=0}{In
that case,} some of President Obama's unclassified emails were
apparently obtained by the intruders.

And last summer, the personnel office announced
\href{http://www.nytimes.com/2014/07/10/world/asia/chinese-hackers-pursue-key-data-on-us-workers.html}{an
intrusion} in which hackers appeared to have targeted the files of tens
of thousands of workers who had applied for top-secret security
clearances.

In that case, the objective seemed clear: The information on security
clearances could help identify covert agents, scientists and others with
data of great interest to foreign governments. That breach also appeared
to have involved Chinese hackers.

But because the breadth of the new attack was so much greater, the
objective seemed less clear.

The intrusion came before the personnel office fully put into place a
series of new security procedures that restricted remote access for
administrators of the network and reviewed all connections to the
outside world through the Internet. In acting too late, the personnel
agency was not alone: The N.S.A. was also beginning to put in place new
network precautions after its most delicate information was taken by
Edward J. Snowden.

The Department of Homeland Security's emergency cyberteam used an
antihacking system called Einstein that alerted the agency to the
potential compromise of federal employee data, S. Y. Lee, a spokesman,
said in a statement.

The F.B.I. said it was working with other agencies to investigate the
matter. ``We take all potential threats to public and private sector
systems seriously, and will continue to investigate and hold accountable
those who pose a threat in cyberspace,'' Joshua Campbell, a spokesman,
said in a statement.

\includegraphics{https://static01.nyt.com/images/2015/06/05/us/05hack-web01/05hack-web01-articleLarge.jpg?quality=75\&auto=webp\&disable=upscale}

The personnel office told current and former federal employees that they
could request 18 months of free credit monitoring to make sure that
their identities had not been stolen, and it said it was working with
cybersecurity specialists to assess the effects of the breach. It was
clear, however, that the scope was sweeping, potentially affecting a
vast majority of the federal work force. J. David Cox Sr., the president
of the American Federation of Government Employees, said he had been
told that the breach might have affected ``all 2.1 million current
federal employees and an additional two million federal retirees and
former employees.''

Katherine Archuleta, the personnel agency's director, said in a
statement, ``Protecting our federal employee data from malicious
cyberincidents is of the highest priority at O.P.M.''

``We take very seriously our responsibility to secure the information
stored in our systems, and in coordination with our agency partners, our
experienced team is constantly identifying opportunities to further
protect the data with which we are entrusted,'' she added.

Administration officials said they made the breach public only after
confirming last month that the data had been compromised and after
taking additional steps to insulate other government agencies from the
intrusion. Mr. Obama has been briefed on the case, officials said.

The attack drew calls for legislation to bolster the nation's
cyberdefenses. \href{https://twitter.com/repadamschiff}{In a series of
Twitter posts}, Representative Adam B. Schiff of California, the senior
Democrat on the Intelligence Committee, called the intrusion ``shocking
because Americans may expect that federal computer networks are
maintained with state of the art defenses.''

He said enactment of new cybersecurity measures was ``perilously
overdue.''

While determining the source of cyberattacks is notoriously difficult,
federal officials say they have become far more skilled at it in recent
years, largely because of increased monitoring of malicious software
entering the United States over international networks. But the most
sophisticated attacks often look as if they were initiated inside the
United States, and tracking their true origin can lead down many blind
paths.

Most Chinese cyberintrusions into the United States, at least until
recently, were aimed at the theft of intellectual property, rather than
at sweeping up vast amounts of personal data.

One senior federal official said it was not clear what the Chinese
government would want from personnel databases. But if the attribution
to China holds, it poses an additional challenge to the Obama
administration. For the past three years, Mr. Obama has been trying to
move the subject of cyberattacks to the center of the American-Chinese
relationship. He has spent hours discussing the subject with Xi Jinping,
the Chinese president.

A year ago, the Justice Department indicted five members of Unit 61398,
a hacking unit of the Chinese People's Liberation Army, accusing them of
stealing data from American firms to benefit state-owned Chinese
companies.

But rather than change Chinese behavior, the indictments shut down many
of the formal and informal discussions between the United States and
China. Chinese officials have often said that they, too, are the victims
of hackers.

An annual ``Strategic and Economic Dialogue'' with Chinese officials is
scheduled to take place this month, and cyberissues will again be in the
forefront.

Advertisement

\protect\hyperlink{after-bottom}{Continue reading the main story}

\hypertarget{site-index}{%
\subsection{Site Index}\label{site-index}}

\hypertarget{site-information-navigation}{%
\subsection{Site Information
Navigation}\label{site-information-navigation}}

\begin{itemize}
\tightlist
\item
  \href{https://help.nytimes.com/hc/en-us/articles/115014792127-Copyright-notice}{©~2020~The
  New York Times Company}
\end{itemize}

\begin{itemize}
\tightlist
\item
  \href{https://www.nytco.com/}{NYTCo}
\item
  \href{https://help.nytimes.com/hc/en-us/articles/115015385887-Contact-Us}{Contact
  Us}
\item
  \href{https://www.nytco.com/careers/}{Work with us}
\item
  \href{https://nytmediakit.com/}{Advertise}
\item
  \href{http://www.tbrandstudio.com/}{T Brand Studio}
\item
  \href{https://www.nytimes.com/privacy/cookie-policy\#how-do-i-manage-trackers}{Your
  Ad Choices}
\item
  \href{https://www.nytimes.com/privacy}{Privacy}
\item
  \href{https://help.nytimes.com/hc/en-us/articles/115014893428-Terms-of-service}{Terms
  of Service}
\item
  \href{https://help.nytimes.com/hc/en-us/articles/115014893968-Terms-of-sale}{Terms
  of Sale}
\item
  \href{https://spiderbites.nytimes.com}{Site Map}
\item
  \href{https://help.nytimes.com/hc/en-us}{Help}
\item
  \href{https://www.nytimes.com/subscription?campaignId=37WXW}{Subscriptions}
\end{itemize}
