Sections

SEARCH

\protect\hyperlink{site-content}{Skip to
content}\protect\hyperlink{site-index}{Skip to site index}

\href{https://www.nytimes.com/section/politics}{Politics}

\href{https://myaccount.nytimes.com/auth/login?response_type=cookie\&client_id=vi}{}

\href{https://www.nytimes.com/section/todayspaper}{Today's Paper}

\href{/section/politics}{Politics}\textbar{}South Carolina Governor
Points to Personal Reasons, Not Politics, for Shift on Confederate Flag

\url{https://nyti.ms/1BLbZI4}

\begin{itemize}
\item
\item
\item
\item
\item
\end{itemize}

Advertisement

\protect\hyperlink{after-top}{Continue reading the main story}

Supported by

\protect\hyperlink{after-sponsor}{Continue reading the main story}

\hypertarget{south-carolina-governor-points-to-personal-reasons-not-politics-for-shift-on-confederate-flag}{%
\section{South Carolina Governor Points to Personal Reasons, Not
Politics, for Shift on Confederate
Flag}\label{south-carolina-governor-points-to-personal-reasons-not-politics-for-shift-on-confederate-flag}}

\includegraphics{https://static01.nyt.com/images/2015/06/24/us/24HALEY/24HALEY-articleLarge.jpg?quality=75\&auto=webp\&disable=upscale}

By \href{https://www.nytimes.com/by/richard-fausset}{Richard Fausset}

\begin{itemize}
\item
  June 23, 2015
\item
  \begin{itemize}
  \item
  \item
  \item
  \item
  \item
  \end{itemize}
\end{itemize}

CHARLESTON, S.C. --- Gov. Nikki R. Haley was at home in Columbia last
Wednesday night, when the gunman began shooting and killing inside the
historic black church here.

At first, she had only partial information, but eventually learned it
had happened at Emanuel African Methodist Episcopal Church, where State
Senator Clementa C. Pinckney, a Democrat she considered a friendly
adversary, presided as pastor.

She called and left Mr. Pinckney a voice mail message, unaware that he
was one of the nine victims of a gunman who witnesses said was motivated
by racial hatred.

By Saturday morning,
\href{http://www.nytimes.com/2015/06/23/us/south-carolina-confederate-flag-dylann-roof.html}{calls
were mounting} for the Legislature to remove what many consider the
state's most visible symbol of racial animus: the Confederate battle
flag, which has flown on the grounds of the State House since 1962. Ms.
Haley, a Republican who is the first ethnic minority and first woman to
serve as governor of South Carolina, decided to reverse her previous
position and tell lawmakers they needed to
\href{http://www.nytimes.com/interactive/2015/06/22/us/Transcript-Gov-Nikki-R-Haley-of-South-Carolina-Addresses-Removing-the-Confederate-Battle-Flag.html}{remove
the flag} once and for all.

``It came down to one simple thing,'' Ms. Haley said in a phone
interview Tuesday. ``I couldn't look my son or daughter in the face and
justify that flag flying anymore.''

The incendiary issue of the flag was one that Ms. Haley, 43, had
sidestepped in her five years in office. At times, she had defended the
flag's presence. During her initial run for governor, she said the flag
was ``not something that is racist,'' but rather, ``a tradition that
people feel proud of.''

\includegraphics{https://static01.nyt.com/images/2015/06/22/multimedia/haley-confederate-flag/haley-confederate-flag-videoSixteenByNine3000-v2.jpg}

Ms. Haley was elected in 2010 on a cresting wave of
\href{http://topics.nytimes.com/top/reference/timestopics/subjects/t/tea_party_movement/index.html?inline=nyt-classifier}{Tea
Party} sentiment, and despite all of her firsts, she is more a product
of American conservatism than the civil rights movement. Her focus as
governor has been on aggressively promoting economic development more
than addressing historical racial wrongs --- although her supporters
argue that her job-creation efforts have helped all people.

But last week's killings have, at least for now, upended the political
status quo in South Carolina. They have thrust Ms. Haley into the
national spotlight, and into uncharted political territory here at home,
where she now finds herself leading her party to align, on the matter of
the flag, with liberals and blacks with whom she has clashed over many
policy issues.

Ms. Haley said her decision to take a stand on the flag was an
``emotional'' one that she hoped would help heal her state. She spoke of
her pride in the way South Carolinians have rallied across the racial
divide.

She acknowledged the hurtful racial attacks that have been lobbed at her
over the years, but insisted that her career was proof that race
relations in this Southern state were, in fact, improving.

``The South Carolina I am privileged to lead today is not the South
Carolina I grew up in,'' she said. ``It has continued to change and it
has continued to evolve. It is the people who have done that.''

In the hours before the interview, the Legislature passed a procedural
measure that will allow it to consider removing the flag from the State
House grounds while lawmakers meet in special session this week.

\href{https://www.nytimes.com/interactive/2015/06/23/us/Calls-to-Cut-Ties-to-Symbols-of-the-South.html}{}

\includegraphics{https://static01.nyt.com/images/2015/06/24/us/24charlestonlisty5/24charlestonlisty5-videoLarge.jpg}

\hypertarget{controversial-confederate-symbols}{%
\subsection{Controversial Confederate
Symbols}\label{controversial-confederate-symbols}}

The Confederate battle flag and other symbols of the Confederacy have
come under renewed criticism. Here are recent protests and calls for the
removal of Confederate symbols from public places.

But victory is not assured. In her news conference Monday, Ms. Haley,
who was elected to a second term last year, said the flag was a ``deeply
offensive symbol'' for some. She was also careful to acknowledge the
many whites here who view the flag as means of honoring the Confederate
dead. ``That is not hate,'' she said, ``Nor is it racism.''

Her language demonstrated the delicate political balance that she and
other Republicans navigate in a state where the flag is still revered in
some circles.

Yet as the drama unfolds on shifting political terrain, Ms. Haley finds
herself in a familiar place: somewhere between the Southern binary of
black and white.

She was born Nimrata Nikki Randhawa in Bamberg, S.C., a small city
between Columbia, the capital, and Charleston. Her parents were
immigrants from India's Punjab State. She has said the locals often did
not know what to make of the family's Indian heritage. When she was
about 5, Ms. Haley and her sister entered a ``Little Miss Bamberg''
pageant, where, traditionally, a black queen and a white queen were
crowned. The judges decided the sisters fit neither category, so they
disqualified them.

Ms. Haley began working for the family clothing business at an early
age, and eventually received an accounting degree from Clemson
University. She was elected to the state House in 2004, defeating Larry
Koon, a longtime incumbent and fellow Republican. Ms. Haley was the
subject of racial attacks during the campaign. Even her name became an
issue, with Mr. Koon pointing out that she was enrolled to vote as
Nimrata Randhawa.

In the House, she earned a reputation as a fiscal hawk. In 2009, she
declared she was running for governor, winning an endorsement from Sarah
Palin, the former Republican vice-presidential candidate and Alaska
governor. In the midst of the primary, two Republicans operatives
emerged, making separate and unproven accusations that they had had
sexual encounters with her. Ms. Haley, who was by that time married,
strongly denied the assertions.

\href{https://www.nytimes.com/interactive/2015/06/22/us/Divisive-Symbolism-of-a-Southern-Flag.html}{}

\includegraphics{https://static01.nyt.com/images/2015/06/22/us/Flag-Listy-1/Flag-Listy-1-videoLarge.jpg}

\hypertarget{divisive-symbolism-of-a-southern-flag}{%
\subsection{Divisive Symbolism of a Southern
Flag}\label{divisive-symbolism-of-a-southern-flag}}

Supporters of the flag have said they view the shooting and the flag as
unrelated. But Cornell William Brooks, the national president of the
N.A.A.C.P., has called it an emblem of hate.

A Republican state senator, Jake Knotts, also went on a radio show and
called her a ``raghead.'' The governor said Democrats, too, dealt
insensitively with her, noting that State Representative Gilda
Cobb-Hunter once said that voters did not think of her as a minority,
but as ``a nice conservative woman with a tan.''

Ms. Haley handily won the primary and went on to defeat her Democratic
challenger, State Senator Vincent Sheheen, 51 percent to 47 percent.

In office, Ms. Haley focused squarely on efforts to recruit companies to
South Carolina, luring them with a combination of tax incentives, the
state's anti-union environment, and a willingness, her admirers say, to
work the phones like a telemarketer and sell South Carolina to
out-of-state chief executives.

But some of Ms. Haley's positions angered many African-American leaders,
including her support of a law requiring voters to show identification
cards at the polls, and her refusal to expand
\href{http://topics.nytimes.com/top/news/health/diseasesconditionsandhealthtopics/medicaid/index.html?inline=nyt-classifier}{Medicaid}
under President Obama's
\href{http://topics.nytimes.com/top/news/health/diseasesconditionsandhealthtopics/health_insurance_and_managed_care/health_care_reform/index.html?inline=nyt-classifier}{health
care law}. In 2013, Ms. Haley removed a member of her re-election
campaign's advisory committee after it was revealed that the member had
ties to the Council of Conservative Citizens, a group that opposes ``all
efforts to mix the races,'' and whose writings resonate in an
\href{http://www.nytimes.com/2015/06/21/us/dylann-storm-roof-photos-website-charleston-church-shooting.html}{online
manifesto} apparently written by Dylann Roof, who has been charged in
the killings at Emanuel A.M.E. Church.

During her 2014 re-election bid, her team said that more than 56,000
jobs had been created in, relocated to, or promised for South Carolina
since she took office. When asked about the Confederate flag at the
time, she said that the chief executives she spoke with did not think it
was really an issue.

Ms. Haley's new stand on the flag may raise her national stature,
allowing her to distance herself from a symbol that many are finding
increasingly indefensible. Bakari Sellers, a former Democratic state
representative who is African-American, said the governor's national
profile had waned somewhat after her election in 2010, but that her
stand on the flag had ``catapulted her back into a national
discussion.''

The governor is often mentioned as a potential presidential running mate
or future cabinet member. But Mr. Sellers said he did not think politics
motivated her on Monday.

``Knowing Nikki, I know that's not why she did it,'' he said.

``She could have cowered, like many of the people running for president,
and punted for a later moment, but she didn't,'' he added. ``She took it
head on.''

Advertisement

\protect\hyperlink{after-bottom}{Continue reading the main story}

\hypertarget{site-index}{%
\subsection{Site Index}\label{site-index}}

\hypertarget{site-information-navigation}{%
\subsection{Site Information
Navigation}\label{site-information-navigation}}

\begin{itemize}
\tightlist
\item
  \href{https://help.nytimes.com/hc/en-us/articles/115014792127-Copyright-notice}{©~2020~The
  New York Times Company}
\end{itemize}

\begin{itemize}
\tightlist
\item
  \href{https://www.nytco.com/}{NYTCo}
\item
  \href{https://help.nytimes.com/hc/en-us/articles/115015385887-Contact-Us}{Contact
  Us}
\item
  \href{https://www.nytco.com/careers/}{Work with us}
\item
  \href{https://nytmediakit.com/}{Advertise}
\item
  \href{http://www.tbrandstudio.com/}{T Brand Studio}
\item
  \href{https://www.nytimes.com/privacy/cookie-policy\#how-do-i-manage-trackers}{Your
  Ad Choices}
\item
  \href{https://www.nytimes.com/privacy}{Privacy}
\item
  \href{https://help.nytimes.com/hc/en-us/articles/115014893428-Terms-of-service}{Terms
  of Service}
\item
  \href{https://help.nytimes.com/hc/en-us/articles/115014893968-Terms-of-sale}{Terms
  of Sale}
\item
  \href{https://spiderbites.nytimes.com}{Site Map}
\item
  \href{https://help.nytimes.com/hc/en-us}{Help}
\item
  \href{https://www.nytimes.com/subscription?campaignId=37WXW}{Subscriptions}
\end{itemize}
