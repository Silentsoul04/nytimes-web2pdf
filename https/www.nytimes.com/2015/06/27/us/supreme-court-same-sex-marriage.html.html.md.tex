Sections

SEARCH

\protect\hyperlink{site-content}{Skip to
content}\protect\hyperlink{site-index}{Skip to site index}

\href{https://www.nytimes.com/section/us}{U.S.}

\href{https://myaccount.nytimes.com/auth/login?response_type=cookie\&client_id=vi}{}

\href{https://www.nytimes.com/section/todayspaper}{Today's Paper}

\href{/section/us}{U.S.}\textbar{}Supreme Court Ruling Makes Same-Sex
Marriage a Right Nationwide

\url{https://nyti.ms/1GNIDHS}

\begin{itemize}
\item
\item
\item
\item
\item
\item
\end{itemize}

Advertisement

\protect\hyperlink{after-top}{Continue reading the main story}

Supported by

\protect\hyperlink{after-sponsor}{Continue reading the main story}

\hypertarget{supreme-court-ruling-makes-same-sex-marriage-a-right-nationwide}{%
\section{Supreme Court Ruling Makes Same-Sex Marriage a Right
Nationwide}\label{supreme-court-ruling-makes-same-sex-marriage-a-right-nationwide}}

Slide 1 of 14

1/14

Pooja Mandagere, left, and Natalie Thompson outside the Supreme Court on
Friday after it ruled in favor of same-sex marriage.

Credit...Doug Mills/The New York Times

\begin{itemize}
\item
  \includegraphics{https://static01.nyt.com/images/2015/06/26/us/20150627_marriage_ss-slide-2NMH/20150627_marriage_ss-slide-2NMH-superJumbo.jpg}
\item
  \includegraphics{https://static01.nyt.com/images/2015/06/26/us/20150627_marriage_ss-slide-K7EQ/20150627_marriage_ss-slide-K7EQ-superJumbo.jpg}
\item
  \includegraphics{https://static01.nyt.com/images/2015/06/26/us/20150627_marriage_ss-slide-IJ1G/20150627_marriage_ss-slide-IJ1G-superJumbo.jpg}
\item
  \includegraphics{https://static01.nyt.com/images/2015/06/27/us/27SCOTUS-SS1/27SCOTUS-SS1-superJumbo.jpg}
\item
  \includegraphics{https://static01.nyt.com/images/2015/06/27/us/27SCOTUS-SS2/27SCOTUS-SS2-superJumbo.jpg}
\item
  \includegraphics{https://static01.nyt.com/images/2015/06/27/us/27SCOTUS-SS3/27SCOTUS-SS3-superJumbo.jpg}
\item
  \includegraphics{https://static01.nyt.com/images/2015/06/27/us/27SCOTUS-SS4/27SCOTUS-SS4-superJumbo.jpg}
\item
  \includegraphics{https://static01.nyt.com/images/2015/06/27/us/27SCOTUS-SS7/27SCOTUS-SS7-superJumbo.jpg}
\item
  \includegraphics{https://static01.nyt.com/images/2015/06/27/us/27SCOTUS-SS6/27SCOTUS-SS6-superJumbo.jpg}
\item
  \includegraphics{https://static01.nyt.com/images/2015/06/26/us/20150627_SCOTUS-SS-slide-B0JK/20150627_SCOTUS-SS-slide-B0JK-superJumbo.jpg}
\item
  \includegraphics{https://static01.nyt.com/images/2015/06/26/us/20150627_marriage_ss-slide-B7F7/20150627_marriage_ss-slide-B7F7-superJumbo.jpg}
\item
  \includegraphics{https://static01.nyt.com/images/2015/06/26/us/20150627_marriage_ss-slide-YTS8/20150627_marriage_ss-slide-YTS8-superJumbo.jpg}
\item
  \includegraphics{https://static01.nyt.com/images/2015/06/26/us/20150627_marriage_ss-slide-MEE3/20150627_marriage_ss-slide-MEE3-superJumbo.jpg}
\item
  \includegraphics{https://static01.nyt.com/images/2015/06/26/us/20150627_SCOTUS-SS-slide-6K9G/20150627_SCOTUS-SS-slide-6K9G-superJumbo.jpg}
\end{itemize}

By \href{http://www.nytimes.com/by/adam-liptak}{Adam Liptak}

\begin{itemize}
\item
  June 26, 2015
\item
  \begin{itemize}
  \item
  \item
  \item
  \item
  \item
  \item
  \end{itemize}
\end{itemize}

WASHINGTON --- In a long-sought victory for the gay rights movement, the
\href{http://topics.nytimes.com/top/reference/timestopics/organizations/s/supreme_court/index.html?inline=nyt-org}{Supreme
Court}
\href{http://www.supremecourt.gov/opinions/14pdf/14-556_3204.pdf}{ruled
by a 5-to-4 vote on Friday} that the Constitution guarantees a right to
\href{http://topics.nytimes.com/top/reference/timestopics/subjects/s/same_sex_marriage/index.html?inline=nyt-classifier}{same-sex
marriage}.

``No longer may this liberty be denied,'' Justice Anthony M. Kennedy
wrote for the majority in the historic decision. ``No union is more
profound than marriage, for it embodies the highest ideals of love,
fidelity, devotion, sacrifice and family. In forming a marital union,
two people become something greater than once they were.''

Marriage is a ``keystone of our social order,'' Justice Kennedy said,
adding that the plaintiffs in the case were seeking ``equal dignity in
the eyes of the law.''

The decision, which was the culmination of decades of litigation and
activism,
\href{http://www.nytimes.com/live/supreme-court-rulings/?hp\&action=click\&pgtype=Homepage\&module=a-lede-package-region\&region=top-news\&WT.nav=top-news}{set
off jubilation and tearful embraces across the country}, the first
same-sex marriages in several states, and resistance --- or at least
stalling --- in others. It came against the backdrop of fast-moving
changes in public opinion, with polls indicating that most Americans now
approve of the unions.

The court's four more liberal justices joined Justice Kennedy's majority
opinion. Each member of the court's conservative wing filed a separate
dissent, in tones ranging from resigned dismay to bitter scorn.

In dissent, Chief Justice John G. Roberts Jr. said the Constitution had
nothing to say on the subject of same-sex marriage.

\includegraphics{https://static01.nyt.com/images/2015/06/27/us/27SCOTUSJP/27SCOTUSJP-articleLarge.jpg?quality=75\&auto=webp\&disable=upscale}

``If you are among the many Americans --- of whatever sexual orientation
--- who favor expanding same-sex marriage, by all means celebrate
today's decision,'' Chief Justice Roberts wrote. ``Celebrate the
achievement of a desired goal. Celebrate the opportunity for a new
expression of commitment to a partner. Celebrate the availability of new
benefits. But do not celebrate the Constitution. It had nothing to do
with it.''

In a second dissent, Justice Antonin Scalia mocked the soaring language
of Justice Kennedy, who has become the nation's most important judicial
champion of gay rights.

``The opinion is couched in a style that is as pretentious as its
content is egotistic,'' Justice Scalia wrote of his colleague's work.
``Of course the opinion's showy profundities are often profoundly
incoherent.''

As Justice Kennedy finished announcing his opinion from the bench on
Friday, several lawyers seated in the bar section of the court's gallery
wiped away tears, while others grinned and exchanged embraces.

Justice John Paul Stevens, who retired in 2010, was on hand for the
decision, and many of the justices' clerks took seats in the chamber,
which was nearly full as the ruling was announced. The decision made
same-sex marriage a reality in the 13 states that had continued to ban
it.

Outside the Supreme Court, the police allowed hundreds of people waving
rainbow flags and holding signs to advance onto the court plaza as those
present for the decision streamed down the steps. ``Love has won,'' the
crowd chanted as courtroom witnesses threw up their arms in victory.

In remarks in the Rose Garden,
\href{http://www.nytimes.com/video/us/politics/100000003766147/obama-on-same-sex-marriage-ruling.html?hp\&action=click\&pgtype=Homepage\&module=a-lede-package-region\&region=top-news\&WT.nav=top-news}{President
Obama welcomed the decision}, saying it ``affirms what millions of
Americans already believe in their hearts.''

\includegraphics{https://static01.nyt.com/images/2015/06/29/us/27marriage-plaintiffs/27marriage-plaintiffs-videoSixteenByNine1050.jpg}

``Today,'' he said, ``we can say, in no uncertain terms, that we have
made our union a little more perfect.''

Justice Kennedy was the author of all three of the Supreme Court's
previous gay rights landmarks. The latest decision came exactly two
years after his majority opinion in
\href{https://www.law.cornell.edu/supremecourt/text/12-307}{United
States v. Windsor}, which struck down a federal law denying benefits to
married same-sex couples, and exactly 12 years after his majority
opinion in
\href{https://www.law.cornell.edu/supct/html/02-102.ZO.html}{Lawrence v.
Texas}, which struck down laws making gay sex a crime.

In all of those decisions, Justice Kennedy embraced a vision of a living
Constitution, one that evolves with societal changes.

``The nature of injustice is that we may not always see it in our own
times,'' he wrote on Friday. ``The generations that wrote and ratified
the Bill of Rights and the Fourteenth Amendment did not presume to know
the extent of freedom in all of its dimensions, and so they entrusted to
future generations a charter protecting the right of all persons to
enjoy liberty as we learn its meaning.''

This drew a withering response from Justice Scalia, a proponent of
reading the Constitution according to the original understanding of
those who adopted it. His dissent was joined by Justice Clarence Thomas.

``They have discovered in the Fourteenth Amendment,'' Justice Scalia
wrote of the majority, ``a `fundamental right' overlooked by every
person alive at the time of ratification, and almost everyone else in
the time since.''

``These justices know,'' Justice Scalia said, ``that limiting marriage
to one man and one woman is contrary to reason; they know that an
institution as old as government itself, and accepted by every nation in
history until 15 years ago, cannot possibly be supported by anything
other than ignorance or bigotry.''

Image

Supporters of same-sex marriage gathered outside the Supreme Court on
Friday.Credit...Doug Mills/The New York Times

Justice Kennedy rooted the ruling in a fundamental right to marriage. Of
special importance to couples, he said, is raising children.

``Without the recognition, stability and predictability marriage
offers,'' he wrote, ``their children suffer the stigma of knowing their
families are somehow lesser. They also suffer the significant material
costs of being raised by unmarried parents, relegated through no fault
of their own to a more difficult and uncertain family life. The marriage
laws at issue here thus harm and humiliate the children of same-sex
couples.''

Justices Ruth Bader Ginsburg, Stephen G. Breyer, Sonia Sotomayor and
Elena Kagan joined the majority opinion.

In dissent, Chief Justice Roberts said the majority opinion was ``an act
of will, not legal judgment.''

``The court invalidates the marriage laws of more than half the states
and orders the transformation of a social institution that has formed
the basis of human society for millennia, for the Kalahari Bushmen and
the Han Chinese, the Carthaginians and the Aztecs,'' he wrote. ``Just
who do we think we are?''

The majority and dissenting opinions took differing views about whether
the decision would harm religious liberty. Justice Kennedy said the
First Amendment ``ensures that religious organizations and persons are
given proper protection as they seek to teach the principles that are so
fulfilling and so central to their lives and faiths.'' He said both
sides should engage in ``an open and searching debate.''

Chief Justice Roberts responded that ``people of faith can take no
comfort in the treatment they receive from the majority today.''

Justice Samuel A. Alito Jr., in his dissent, saw a broader threat from
the majority opinion. ``It will be used to vilify Americans who are
unwilling to assent to the new orthodoxy,'' Justice Alito wrote. ``In
the course of its opinion, the majority compares traditional marriage
laws to laws that denied equal treatment for African-Americans and
women. The implications of this analogy will be exploited by those who
are determined to stamp out every vestige of dissent.''

Gay rights advocates had constructed a careful litigation and public
relations strategy to build momentum and bring the issue to the Supreme
Court when it appeared ready to rule in their favor. As in earlier civil
rights cases, the court had responded cautiously and methodically,
laying judicial groundwork for a transformative decision.

It waited for scores of lower courts to strike down bans on same-sex
marriages before addressing the issue, and Justice Kennedy took the
unusual step of listing those decisions in an appendix to his opinion.

Chief Justice Roberts said that only 11 states and the District of
Columbia had embraced the right to same-sex marriage democratically, at
voting booths and in legislatures. The rest of the 37 states that allow
such unions did so because of court rulings. Gay rights advocates, the
chief justice wrote, would have been better off with a victory achieved
through the political process, particularly ``when the winds of change
were freshening at their backs.''

In his own dissent, Justice Scalia took a similar view, saying that the
majority's assertiveness represented a ``threat to American democracy.''

But Justice Kennedy rejected that idea. ``It is of no moment whether
advocates of same-sex marriage now enjoy or lack momentum in the
democratic process,'' he wrote. ``The issue before the court here is the
legal question whether the Constitution protects the right of same-sex
couples to marry.''

Later in the opinion, Justice Kennedy answered the question. ``The
Constitution,'' he wrote, ``grants them that right.''

Advertisement

\protect\hyperlink{after-bottom}{Continue reading the main story}

\hypertarget{site-index}{%
\subsection{Site Index}\label{site-index}}

\hypertarget{site-information-navigation}{%
\subsection{Site Information
Navigation}\label{site-information-navigation}}

\begin{itemize}
\tightlist
\item
  \href{https://help.nytimes.com/hc/en-us/articles/115014792127-Copyright-notice}{©~2020~The
  New York Times Company}
\end{itemize}

\begin{itemize}
\tightlist
\item
  \href{https://www.nytco.com/}{NYTCo}
\item
  \href{https://help.nytimes.com/hc/en-us/articles/115015385887-Contact-Us}{Contact
  Us}
\item
  \href{https://www.nytco.com/careers/}{Work with us}
\item
  \href{https://nytmediakit.com/}{Advertise}
\item
  \href{http://www.tbrandstudio.com/}{T Brand Studio}
\item
  \href{https://www.nytimes.com/privacy/cookie-policy\#how-do-i-manage-trackers}{Your
  Ad Choices}
\item
  \href{https://www.nytimes.com/privacy}{Privacy}
\item
  \href{https://help.nytimes.com/hc/en-us/articles/115014893428-Terms-of-service}{Terms
  of Service}
\item
  \href{https://help.nytimes.com/hc/en-us/articles/115014893968-Terms-of-sale}{Terms
  of Sale}
\item
  \href{https://spiderbites.nytimes.com}{Site Map}
\item
  \href{https://help.nytimes.com/hc/en-us}{Help}
\item
  \href{https://www.nytimes.com/subscription?campaignId=37WXW}{Subscriptions}
\end{itemize}
