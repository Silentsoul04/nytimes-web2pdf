Sections

SEARCH

\protect\hyperlink{site-content}{Skip to
content}\protect\hyperlink{site-index}{Skip to site index}

\href{https://www.nytimes.com/section/business}{Business}

\href{https://myaccount.nytimes.com/auth/login?response_type=cookie\&client_id=vi}{}

\href{https://www.nytimes.com/section/todayspaper}{Today's Paper}

\href{/section/business}{Business}\textbar{}Samsung Merger Plan Called
Unfair to Some Investors

\url{https://nyti.ms/1LODm7a}

\begin{itemize}
\item
\item
\item
\item
\item
\end{itemize}

Advertisement

\protect\hyperlink{after-top}{Continue reading the main story}

Supported by

\protect\hyperlink{after-sponsor}{Continue reading the main story}

\href{/column/fair-game}{Fair Game}

\hypertarget{samsung-merger-plan-called-unfair-to-some-investors}{%
\section{Samsung Merger Plan Called Unfair to Some
Investors}\label{samsung-merger-plan-called-unfair-to-some-investors}}

\includegraphics{https://static01.nyt.com/images/2015/07/05/business/05Gret-web/05Gret-web-articleLarge.jpg?quality=75\&auto=webp\&disable=upscale}

By \href{http://www.nytimes.com/by/gretchen-morgenson}{Gretchen
Morgenson}

\begin{itemize}
\item
  July 4, 2015
\item
  \begin{itemize}
  \item
  \item
  \item
  \item
  \item
  \end{itemize}
\end{itemize}

Samsung has built a name for itself among consumers around the world as
a leading manufacturer of smartphones, televisions and home appliances.
For investors, though, its reputation is not so stellar.

The latest issue for Samsung shareholders is a proposed \$8 billion
merger of two company affiliates: Samsung C\&T, a construction and
engineering company, and Cheil Industries, a holding company that issued
shares to the public for the first time in December and that has
interests in amusement parks, retail fashion, biopharmaceuticals and
life insurance.

The deal was proposed in late May and is scheduled to go to a
shareholder vote on July 17. But some outside investors have criticized
the transaction, saying that it is being struck at a price that is
unfair to Samsung C\&T shareholders. And, they say, Samsung has made
matters worse by shifting a 5.76 percent stake of C\&T stock held in the
company's treasury into friendly hands to help ensure the deal will go
through.

It's not news to investors in any of the large South Korean companies,
with their hydra-headed holdings, that they must be wary of transactions
between its affiliates. Deals that are not conducted at arms-length can
wind up benefiting insiders at the expense of outside investors.

Such transactions are common among the large, family-controlled
companies in South Korea known as chaebol.
\href{http://www.samsung.com/us/aboutsamsung/investor_relations/financial_information/downloads/2015/2015_business_quarter01.pdf}{Samsung
Electronics} is one of the largest of these vast and complex
conglomerates with strong ties to the South Korean government. Samsung
has 67 domestic affiliates, 18 of which have listed shares trading on
the Korea stock exchange.

Members of the founding family of Samsung own 42 percent of Cheil
Industries, according to Forbes magazine. And that has led some
investors to speculate that the merger between C\&T and Cheil may be
motivated by family concerns.

\href{http://www.forbes.com/profile/lee-kun-hee/}{Lee Kun-hee}, the
chairman of Samsung and the family patriarch, had a heart attack last
year, and his children have been working to consolidate company holdings
for a smooth transition. Acquiring C\&T, which owns a 4.1 percent stake
in Samsung Electronics and a 17 percent stake in another affiliate,
Samsung SDS, which provides information technology services, may be a
way for them to do that.

In addition, there's the specter of a \$5 billion or \$6 billion
inheritance tax owed by his children upon his death. Paying such a bill
will require that they sell some of their company shares.

Chief among the critics of the Cheil-C\&T merger is Elliott Management,
the \$26 billion investment firm overseen by Paul Singer. Elliott, which
owns a 7 percent stake in Samsung C\&T, filed a lawsuit in South Korea
asking the Seoul central district court to stop the transaction. On
Wednesday, a judge declined that request.

But he has yet to rule on a second suit filed
\href{http://www.fairdealforsct.com/present/}{by Elliott}. That suit
asks the court to block the sale of the 5.76 percent C\&T stake to KCC,
a company that holds a stake in Cheil Industries and therefore has an
interest in seeing the deal go through.

Samsung contends that the terms of the C\&T deal comply with South
Korean law, which mandates that the computation of a merger ratio be
based upon recent stock prices. Management argues that the merger is
necessary if C\&T is to turn itself around and accuses Elliott of being
interested in short-term gains.

As for the sale of C\&T shares, Samsung called the transaction part of a
``promotion of approval of the merger for procurement of the growth of
Samsung C\&T.''

Officials at Elliott declined to comment beyond their court filings and
public releases. But other investors are also upset about the merger and
say they plan to vote against it.

``We are not convinced that it is a good deal for Samsung C\&T
shareholders, particularly at the proposed merger ratio,'' said Hugh
Young, managing director in Asia for Aberdeen Asset Management, a \$491
billion financial firm.

It is not enough that the company simply followed the letter of the law
when calculating the merger ratio. ``For the board to fulfill their
fiduciary duties, it must be clear that the price is fair to all
shareholders of Samsung C\&T and fully recognizes the value of the
company,'' Mr. Young said.

According to an analysis by Elliott, C\&T was trading at severely
depressed levels before the proposed takeover came about. A revenue
slowdown in the first quarter of 2015 may have contributed.

The day before the deal was announced, C\&T's share price was 40 percent
below its aggregate net asset value, Elliott says. That reflected just
63 percent of C\&T's stakes in publicly traded affiliates such as
Samsung Electronics, Samsung SDS and Samsung Engineering.

At the same time C\&T's shares were falling, Cheil Industries' stock was
racing ahead. The day of the merger, it was trading at 131 times
estimated earnings, an enormous premium to its peers and to the South
Korean market index, which trades at around 11 times forward earnings.

C\&T stockholders will receive just 0.35 Cheil shares for each of their
shares if the deal goes through. By Elliott's reckoning, that represents
22 percent of the fair value of C\&T shares and will result in a
transfer of \$7 billion in asset value from C\&T shareholders to Cheil.

How the merger plays out probably rests in the hands of NPS, the South
Korean pension system. It owns a 10 percent stake in C\&T; it has not
said how it will vote.

C\&T investors who are siding with Elliott say this transaction
highlights the anti-investor practices that are common among the
chaebol.

These practices, they say, are one of the reasons stocks of South Korean
companies trade at such a discount to their rivals in other developed
countries.

It's time for Samsung and other large Korean companies to take a more
enlightened approach to their dealings with minority shareholders, said
Yoo-Kyung Park, director of sustainability and governance for Asia at
\href{http://www.apg.nl/en}{APG Asset Management}, a \$486 billion
pension services manager for government and education employees in the
Netherlands.

``We hope that the Samsung group companies immediately start conducting
a thorough review of the quality of their respective governance
structure and actively communicate with investors on a newly established
vision to which corporate governance is central,'' Ms. Park said.

``We also wish,'' she added, ``that the lessons learned by the Samsung
companies through this merger process will be taken aboard by other
major groups in Korea, as they face similar structural corporate
governance issues.''

Will Samsung be forced to forge a new path with investors? Let's hope
so, but don't hold your breath.

Advertisement

\protect\hyperlink{after-bottom}{Continue reading the main story}

\hypertarget{site-index}{%
\subsection{Site Index}\label{site-index}}

\hypertarget{site-information-navigation}{%
\subsection{Site Information
Navigation}\label{site-information-navigation}}

\begin{itemize}
\tightlist
\item
  \href{https://help.nytimes.com/hc/en-us/articles/115014792127-Copyright-notice}{©~2020~The
  New York Times Company}
\end{itemize}

\begin{itemize}
\tightlist
\item
  \href{https://www.nytco.com/}{NYTCo}
\item
  \href{https://help.nytimes.com/hc/en-us/articles/115015385887-Contact-Us}{Contact
  Us}
\item
  \href{https://www.nytco.com/careers/}{Work with us}
\item
  \href{https://nytmediakit.com/}{Advertise}
\item
  \href{http://www.tbrandstudio.com/}{T Brand Studio}
\item
  \href{https://www.nytimes.com/privacy/cookie-policy\#how-do-i-manage-trackers}{Your
  Ad Choices}
\item
  \href{https://www.nytimes.com/privacy}{Privacy}
\item
  \href{https://help.nytimes.com/hc/en-us/articles/115014893428-Terms-of-service}{Terms
  of Service}
\item
  \href{https://help.nytimes.com/hc/en-us/articles/115014893968-Terms-of-sale}{Terms
  of Sale}
\item
  \href{https://spiderbites.nytimes.com}{Site Map}
\item
  \href{https://help.nytimes.com/hc/en-us}{Help}
\item
  \href{https://www.nytimes.com/subscription?campaignId=37WXW}{Subscriptions}
\end{itemize}
