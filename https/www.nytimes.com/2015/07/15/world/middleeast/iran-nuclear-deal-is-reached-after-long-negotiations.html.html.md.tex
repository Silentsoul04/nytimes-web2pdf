Sections

SEARCH

\protect\hyperlink{site-content}{Skip to
content}\protect\hyperlink{site-index}{Skip to site index}

\href{https://www.nytimes.com/section/world/middleeast}{Middle East}

\href{https://myaccount.nytimes.com/auth/login?response_type=cookie\&client_id=vi}{}

\href{https://www.nytimes.com/section/todayspaper}{Today's Paper}

\href{/section/world/middleeast}{Middle East}\textbar{}Deal Reached on
Iran Nuclear Program; Limits on Fuel Would Lessen With Time

\url{https://nyti.ms/1MqGGm1}

\begin{itemize}
\item
\item
\item
\item
\item
\item
\end{itemize}

Advertisement

\protect\hyperlink{after-top}{Continue reading the main story}

Supported by

\protect\hyperlink{after-sponsor}{Continue reading the main story}

\hypertarget{deal-reached-on-iran-nuclear-program-limits-on-fuel-would-lessen-with-time}{%
\section{Deal Reached on Iran Nuclear Program; Limits on Fuel Would
Lessen With
Time}\label{deal-reached-on-iran-nuclear-program-limits-on-fuel-would-lessen-with-time}}

\includegraphics{https://static01.nyt.com/images/2015/07/14/multimedia/obama-iran-nuke-deal/obama-iran-nuke-deal-videoSixteenByNine1050-v2.jpg}

By \href{https://www.nytimes.com/by/michael-r-gordon}{Michael R. Gordon}
and \href{https://www.nytimes.com/by/david-e-sanger}{David E. Sanger}

\begin{itemize}
\item
  July 14, 2015
\item
  \begin{itemize}
  \item
  \item
  \item
  \item
  \item
  \item
  \end{itemize}
\end{itemize}

VIENNA ---
\href{http://topics.nytimes.com/top/news/international/countriesandterritories/iran/index.html?inline=nyt-geo}{Iran}
and a group of six nations led by the United States reached a historic
accord on Tuesday to significantly limit Tehran's nuclear ability for
more than a decade in return for lifting international oil and financial
sanctions.

The deal culminates 20 months of negotiations on an agreement that
President Obama had long sought as the biggest diplomatic achievement of
his presidency. Whether it portends a new relationship between the
United States and Iran --- after decades of coups, hostage-taking,
terrorism and sanctions --- remains a bigger question.

Mr. Obama, in an early morning appearance at the White House that was
broadcast live in Iran, began what promised to be an arduous effort to
sell the deal to Congress and the American public, saying the agreement
is ``not built on trust --- it is built on verification.''

\href{https://www.nytimes.com/interactive/2015/03/31/world/middleeast/simple-guide-nuclear-talks-iran-us.html}{}

\includegraphics{https://static01.nyt.com/images/2015/03/31/world/middleeast/simple-guide-nuclear-talks-iran-us-1427818203771/simple-guide-nuclear-talks-iran-us-1427818203771-videoLarge.jpg}

\hypertarget{the-iran-nuclear-deal--a-simple-guide}{%
\subsection{The Iran Nuclear Deal -- A Simple
Guide}\label{the-iran-nuclear-deal--a-simple-guide}}

A guide to help you navigate the deal between global powers and Tehran.

He made it abundantly clear he would fight to preserve the deal from
critics in Congress who are beginning a 60-day review, declaring, ``I
will veto any legislation that prevents the successful implementation of
this deal.''

Almost as soon as the agreement was announced, to cheers in Vienna and
on the streets of Tehran, its harshest critics said it would ultimately
empower Iran rather than limit its capability. Israel's prime minister,
Benjamin Netanyahu, called it a ``historic mistake'' that would create a
``terrorist nuclear superpower.''

A review of the 109-page text of the agreement, which includes five
annexes, showed that the United States preserved --- and in some cases
extended --- the nuclear restrictions it sketched out with Iran in early
April in Lausanne, Switzerland.

Yet, it left open areas that are sure to raise fierce objections in
Congress. It preserves Iran's ability to produce as much nuclear fuel as
it wishes after year 15 of the agreement, and allows it to conduct
research on advanced centrifuges after the eighth year. Moreover, the
Iranians won the eventual lifting of an embargo on the import and export
of conventional arms and ballistic missiles --- a step the departing
chairman of the Joint Chiefs of Staff, Gen. Martin E. Dempsey,
\href{http://www.nytimes.com/2015/07/11/world/middleeast/un-arms-ban-on-iran-remains-a-hurdle-to-nuclear-deal.html}{warned
about} just last week.

\includegraphics{https://static01.nyt.com/images/2015/07/15/world/JP-IRAN/JP-IRAN-articleLarge.jpg?quality=75\&auto=webp\&disable=upscale}

American officials said the core of the agreement, secured in 18
consecutive days of talks here, lies in the restrictions on the amount
of nuclear fuel that Iran can keep for the next 15 years. The current
stockpile of low enriched uranium will be reduced by 98 percent, most
likely by shipping much of it to Russia.

That limit, combined with a two-thirds reduction in the number of its
centrifuges, would extend to a year the amount of time it would take
Iran to make enough material for a single bomb should it abandon the
accord and race for a weapon --- what officials call ``breakout time.''
By comparison, analysts say Iran now has a breakout time of two to three
months.

But American officials also acknowledged that after the first decade,
the breakout time would begin to shrink. It was unclear how rapidly,
because Iran's longer-term plans to expand its enrichment capability
will be kept confidential.

The concern that Iran's breakout time could shrink sharply in the waning
years of the restrictions has already been a contentious issue in
Congress. Mr. Obama contributed to that in an interview with National
Public Radio in April, when he said that in ``year 13, 14, 15'' of the
agreement, the breakout time might shrink ``almost down to zero,'' as
Iran is expected to develop and use advanced centrifuges then.

\href{https://www.nytimes.com/interactive/2015/07/14/world/middleeast/iran-nuclear-deal-who-got-what-they-wanted.html}{}

\includegraphics{https://static01.nyt.com/images/2015/07/24/world/24iran-1/24iran-1-videoLarge.jpg}

\hypertarget{who-got-what-they-wanted-in-the-iran-nuclear-deal}{%
\subsection{Who Got What They Wanted in the Iran Nuclear
Deal}\label{who-got-what-they-wanted-in-the-iran-nuclear-deal}}

Here is a look at what Iran and the United States wanted, and what they
got.

Pressed on that point, an American official who briefed reporters on
Tuesday said that Iran's long-term plans to expand its enrichment
capability would be shared with the International Atomic Energy Agency
and other parties to the accord.

``It is going to be a gradual decline,'' the official said. ``At the end
of, say, 15 years, we are not going to know what that is.'' But clearly
there are intelligence agency estimates, and one diplomat involved in
the talks said that internal estimates suggested Iran's breakout time
could shrink to about five months in year 14 of the plan.

Secretary of State
\href{http://topics.nytimes.com/top/reference/timestopics/people/k/john_kerry/index.html?inline=nyt-per}{John
Kerry}, who led the negotiations for the United States in the final
rounds, sought in his remarks Tuesday to blunt criticism on this point.
``Iran will not produce or acquire highly enriched uranium'' or
plutonium for at least 15 years, he said. Verification measures, he
added, will ``stay in place permanently.''

He stressed that Tehran and the International Atomic Energy Agency had
``entered into an agreement to address all questions'' about Iran's past
actions within three months, and that completing this task was
``fundamental for sanctions relief.''

\href{https://www.nytimes.com/interactive/2015/07/14/world/middleeast/reactions-to-iran-nuclear-deal.html}{}

\includegraphics{https://static01.nyt.com/images/2015/07/14/world/middleeast/reactions-to-iran-nuclear-deal-1436886907449/reactions-to-iran-nuclear-deal-1436886907449-videoLarge.png}

\hypertarget{what-key-players-are-saying-about-the-iran-nuclear-deal}{%
\subsection{What Key Players Are Saying About the Iran Nuclear
Deal}\label{what-key-players-are-saying-about-the-iran-nuclear-deal}}

A guide to international reaction to the historic accord.

Compared with many past efforts to slow a nation's
\href{http://topics.nytimes.com/top/news/international/countriesandterritories/iran/nuclear_program/index.html?inline=nyt-classifier}{nuclear
program} --- including a deal struck with North Korea 20 years ago ---
this agreement is remarkably specific. Nevertheless, some mysteries
remain. For example, it is not clear whether the inspectors would be
able to interview the scientists and engineers who were believed to have
been at the center of an effort by the Islamic Revolutionary Guard Corps
to design a weapon that Iran could manufacture in short order.

In building his argument for the deal, Mr. Obama stressed that the
accord was vastly preferable to the alternate scenario: no agreement and
an unbridled nuclear arms race in the Middle East. ``Put simply, no deal
means a greater chance of more war in the Middle East,'' he said. He
said his successors in the White House ``will be in a far stronger
position'' to restrain Iran for decades to come than they would be
without the pact.

In
\href{http://www.nytimes.com/2015/07/15/opinion/thomas-friedman-obama-makes-his-case-on-iran-nuclear-deal.html?hp\&action=click\&pgtype=Homepage\&module=b-lede-package-region\&region=top-news\&WT.nav=top-news}{an
interview Tuesday} with Thomas L. Friedman, an Op-Ed columnist with The
New York Times, Mr. Obama also answered Mr. Netanyahu and other critics
who, he said, would prefer that the Iranians ``don't even have any
nuclear capacity.'' Mr. Obama said, ``But really, what that involves is
eliminating the presence of knowledge inside of Iran.'' Since that is
not realistic, the president added, ``The question is, Do we have the
kind of inspection regime and safeguards and international consensus
whereby it's not worth it for them to do it? We have accomplished
that.''

As news of a nuclear deal spread, Iranians reacted with a mix of
jubilation, cautious optimism and disbelief that decades of a seemingly
intractable conflict could be coming to an end.

\includegraphics{https://static01.nyt.com/images/2015/07/15/world/middleeast/15Iran8-web/15Iran8-web-videoSixteenByNine1050.jpg}

``Have they really reached a deal?'' asked Masoud Derakhshani, a
93-year-old widower who had come down to the lobby of his apartment
building for his daily newspaper. Mr. Derakhshani remained cautious,
even incredulous. ``I can't believe it,'' he said. ``They will most
probably hit some last-minute snag.''

Across Tehran, many Iranians expressed hope for better economic times
after years in which crippling sanctions have severely depressed the
value of the national currency, the rial. That in turn caused inflation
and shortages of goods, including vital medicines, and forced Iranians
to carry fat wads of bank notes to pay for everyday items such as meat,
rice and beans.

``I am desperate to feed my three sons,'' said Ali, 53, a cleaner.
``This deal should bring investment for jobs so they can start working
for a living.''

National dignity, a major demand of Iran's leader, did not matter to
him, he said. ``I really do not care if this is a victory for us or
not,'' he said. ``I want relations with the West. If we compromised, so
be it.''

Image

Delegates from Iran and a group of six nations led by the United States
in Vienna on Tuesday after agreeing to an accord to significantly limit
Tehran's nuclear ability.Credit...Pool photo by Carlos Barria

Iran's president, Hassan Rouhani, who was elected in 2013 on a platform
of ridding the country of the sanctions, said that the Iranian people's
``prayers have come true.''

One of the last, and most contentious, issues was the question of
whether and how fast an arms embargo on conventional weapons and
missiles, imposed starting in 2006, would be lifted.

After days of haggling, Secretary of State Kerry and his Iranian
counterpart, Mohammad Javad Zarif, agreed that the missile restrictions
would remain for eight years and that a similar ban on the purchase and
sale of conventional weapons would be removed in five years.

Those bans would be removed even sooner if the International Atomic
Energy Agency reached a definitive conclusion that the Iranian nuclear
program is entirely peaceful, and that there was no evidence of cheating
on the accord or any activity to obtain weapons covertly.

The provisions on the arms embargo are expected to dominate the coming
debate in Congress on the accord.

Even before the deal was announced, critics expressed fears that Iran
would use some of the billions of dollars it will receive after
sanctions relief to build up its military power. Iranian officials,
however, have said that Iran should be treated like any other nation,
and not be subjected to an arms embargo if it meets the terms of a
nuclear deal.

Defending the outcome, Mr. Kerry told reporters here that China and
Russia had favored lifting the entire arms embargo immediately,
suggesting he had no choice but to try to strike a middle ground.

Mr. Kerry appeared to secure another commitment that was not part of
\href{http://www.nytimes.com/2015/04/04/world/middleeast/an-iran-nuclear-deal-built-on-coffee-all-nighters-and-compromise.html}{a
preliminary agreement} negotiated in Lausanne. Iranian officials agreed
here on a multiyear ban on designing warheads and conducting tests,
including with detonators and nuclear triggers, that would contribute to
the design and manufacture of a nuclear weapon. Accusations that Tehran
conducted that kind of research in the past led to a standoff with
inspectors.

Diplomats also came up with unusual procedure to ``snap back'' the
sanctions against Iran if an eight-member panel determines that Tehran
is violating the nuclear provisions. The members of the panel are
Britain, China, France, Germany, Russia, the United States, the European
Union and Iran itself. A majority vote is required, meaning that Russia,
China and Iran could not collectively block action.

With the announcement of the accord, Mr. Obama has now made major
strides toward fundamentally changing the American diplomatic
relationships with three nations:
\href{http://www.nytimes.com/2014/12/18/world/americas/us-cuba-relations.html}{Cuba},
Iran and
\href{http://www.nytimes.com/2012/01/14/world/asia/united-states-resumes-diplomatic-relations-with-myanmar.html}{Myanmar}.
Of the three, Iran is the most strategically important, the only one
with a nuclear program, and it is still on the State Department's
\href{http://www.state.gov/j/ct/list/c14151.htm}{list of state sponsors
of terrorism}.

While the agreement faces heavy opposition from Republicans in Congress,
and even some Democrats, Mr. Obama's chances of prevailing are
considered high. Even if the accord is voted down by one or both houses,
he could veto that action, and he is likely to have the votes he would
need to override the veto. But he has told aides that for an accord as
important as this one --- which he hopes will usher in a virtual truce
with a country that has been a major American adversary for 35 years ---
he wants a congressional endorsement.

{[}\emph{\href{https://www.nytimes.com/2019/06/17/world/middleeast/iran-nuclear-deal-compliance.html}{On
June 17, 2019. Iran announced that it would soon exceed the limits on
the nuclear fuel it is permitted to possess under the landmark 2015
nuclear deal.}}{]}

Mr. Obama will also have to manage the breach with Mr. Netanyahu and the
leaders of Saudi Arabia and other Arab states who have warned against
the deal, saying the relief of sanctions will ultimately empower the
Iranians throughout the Middle East.

Advertisement

\protect\hyperlink{after-bottom}{Continue reading the main story}

\hypertarget{site-index}{%
\subsection{Site Index}\label{site-index}}

\hypertarget{site-information-navigation}{%
\subsection{Site Information
Navigation}\label{site-information-navigation}}

\begin{itemize}
\tightlist
\item
  \href{https://help.nytimes.com/hc/en-us/articles/115014792127-Copyright-notice}{©~2020~The
  New York Times Company}
\end{itemize}

\begin{itemize}
\tightlist
\item
  \href{https://www.nytco.com/}{NYTCo}
\item
  \href{https://help.nytimes.com/hc/en-us/articles/115015385887-Contact-Us}{Contact
  Us}
\item
  \href{https://www.nytco.com/careers/}{Work with us}
\item
  \href{https://nytmediakit.com/}{Advertise}
\item
  \href{http://www.tbrandstudio.com/}{T Brand Studio}
\item
  \href{https://www.nytimes.com/privacy/cookie-policy\#how-do-i-manage-trackers}{Your
  Ad Choices}
\item
  \href{https://www.nytimes.com/privacy}{Privacy}
\item
  \href{https://help.nytimes.com/hc/en-us/articles/115014893428-Terms-of-service}{Terms
  of Service}
\item
  \href{https://help.nytimes.com/hc/en-us/articles/115014893968-Terms-of-sale}{Terms
  of Sale}
\item
  \href{https://spiderbites.nytimes.com}{Site Map}
\item
  \href{https://help.nytimes.com/hc/en-us}{Help}
\item
  \href{https://www.nytimes.com/subscription?campaignId=37WXW}{Subscriptions}
\end{itemize}
