Sections

SEARCH

\protect\hyperlink{site-content}{Skip to
content}\protect\hyperlink{site-index}{Skip to site index}

\href{https://myaccount.nytimes.com/auth/login?response_type=cookie\&client_id=vi}{}

\href{https://www.nytimes.com/section/todayspaper}{Today's Paper}

\href{/section/business/dealbook}{DealBook}\textbar{}Regulators Approve
Merger of CIT and OneWest in \$3.4 Billion Deal

\url{https://nyti.ms/1DtzIaS}

\begin{itemize}
\item
\item
\item
\item
\item
\end{itemize}

Advertisement

\protect\hyperlink{after-top}{Continue reading the main story}

Supported by

\protect\hyperlink{after-sponsor}{Continue reading the main story}

DealBook Business and Policy

\hypertarget{regulators-approve-merger-of-cit-and-onewest-in-34-billion-deal}{%
\section{Regulators Approve Merger of CIT and OneWest in \$3.4 Billion
Deal}\label{regulators-approve-merger-of-cit-and-onewest-in-34-billion-deal}}

\includegraphics{https://static01.nyt.com/images/2015/07/23/business/db-onewest-web1/db-onewest-web1-articleLarge.jpg?quality=75\&auto=webp\&disable=upscale}

By \href{http://www.nytimes.com/by/michael-corkery}{Michael Corkery}

\begin{itemize}
\item
  July 22, 2015
\item
  \begin{itemize}
  \item
  \item
  \item
  \item
  \item
  \end{itemize}
\end{itemize}

Federal banking regulators have approved CIT Group's acquisition of
OneWest bank, completing one of the largest deals in the financial
services industry since the mortgage crisis.

After nearly a year of scrutiny from community groups and consumer
advocates, the Federal Reserve, the Office of Comptroller of the
Currency and various state banking regulators on Tuesday approved the
merger between the commercial lender CIT and OneWest, a California
retail bank that was formerly IndyMac, a failed mortgage lender.

The deal, which was announced in July 2014, is a big win for CIT's chief
executive, John A. Thain, who ran Merrill Lynch when it was sold to Bank
of America in the depth of the 2008 financial crisis. Mr. Thain took
over CIT not long after it emerged from bankruptcy in 2009.

The \$3.4 billion acquisition of OneWest will nearly double the size of
CIT.

It is a boon to the private equity investors who bought up the soured
assets of IndyMac with help from the Federal Deposit Insurance
Corporation, and have collected billions from their investment in the
renamed OneWest.

The combined company, which will be called the CIT Group, will have more
than \$65 billion in assets and more than \$30 billion of deposits.

``We welcome OneWest employees to CIT and we look forward to working
with them to meet the needs of small and middle market businesses, the
transportation industry, real estate sector and our retail bank
customers,'' Mr. Thain said in a statement.

Consumer advocates opposed the merger, saying that OneWest had a poor
track record in fulfilling its obligations to the Community Reinvestment
Act, a law designed to ensure that banks serve lower income and minority
neighborhoods.

Some also took issue with OneWest's foreclosures practices, particularly
in its reverse mortgage unit. This winter, federal banking regulators
hosted a rare public hearing in California to allow the public to air
their feelings about the merger.

As a condition of approving the merger, the Office of Comptroller of the
Currency said that CIT must satisfy several requirements related to the
Community Reinvestment Act, including that it create a special committee
dedicated to developing banking products for low and moderate-income
communities and report on its progress regularly to banking regulators
and the public.

The deal is likely to ease concerns among analysts and deal makers that
bank regulators are hostile toward mergers in the industry. Since the
financial crisis, there have been several deals that have been held up
for months -- even years -- amid intense scrutiny.

Advertisement

\protect\hyperlink{after-bottom}{Continue reading the main story}

\hypertarget{site-index}{%
\subsection{Site Index}\label{site-index}}

\hypertarget{site-information-navigation}{%
\subsection{Site Information
Navigation}\label{site-information-navigation}}

\begin{itemize}
\tightlist
\item
  \href{https://help.nytimes.com/hc/en-us/articles/115014792127-Copyright-notice}{©~2020~The
  New York Times Company}
\end{itemize}

\begin{itemize}
\tightlist
\item
  \href{https://www.nytco.com/}{NYTCo}
\item
  \href{https://help.nytimes.com/hc/en-us/articles/115015385887-Contact-Us}{Contact
  Us}
\item
  \href{https://www.nytco.com/careers/}{Work with us}
\item
  \href{https://nytmediakit.com/}{Advertise}
\item
  \href{http://www.tbrandstudio.com/}{T Brand Studio}
\item
  \href{https://www.nytimes.com/privacy/cookie-policy\#how-do-i-manage-trackers}{Your
  Ad Choices}
\item
  \href{https://www.nytimes.com/privacy}{Privacy}
\item
  \href{https://help.nytimes.com/hc/en-us/articles/115014893428-Terms-of-service}{Terms
  of Service}
\item
  \href{https://help.nytimes.com/hc/en-us/articles/115014893968-Terms-of-sale}{Terms
  of Sale}
\item
  \href{https://spiderbites.nytimes.com}{Site Map}
\item
  \href{https://help.nytimes.com/hc/en-us}{Help}
\item
  \href{https://www.nytimes.com/subscription?campaignId=37WXW}{Subscriptions}
\end{itemize}
