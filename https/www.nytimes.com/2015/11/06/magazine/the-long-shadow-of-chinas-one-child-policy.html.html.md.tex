Sections

SEARCH

\protect\hyperlink{site-content}{Skip to
content}\protect\hyperlink{site-index}{Skip to site index}

\href{https://myaccount.nytimes.com/auth/login?response_type=cookie\&client_id=vi}{}

\href{https://www.nytimes.com/section/todayspaper}{Today's Paper}

The Long Shadow of China's One-Child Policy

\url{https://nyti.ms/1kyITDC}

\begin{itemize}
\item
\item
\item
\item
\item
\item
\end{itemize}

Advertisement

\protect\hyperlink{after-top}{Continue reading the main story}

Supported by

\protect\hyperlink{after-sponsor}{Continue reading the main story}

Notebook

\hypertarget{the-long-shadow-of-chinas-one-child-policy}{%
\section{The Long Shadow of China's One-Child
Policy}\label{the-long-shadow-of-chinas-one-child-policy}}

\includegraphics{https://static01.nyt.com/images/2015/11/06/magazine/06mag-china-1/06mag-china-1-articleLarge.jpg?quality=75\&auto=webp\&disable=upscale}

By Brook Larmer

\begin{itemize}
\item
  Nov. 6, 2015
\item
  \begin{itemize}
  \item
  \item
  \item
  \item
  \item
  \item
  \end{itemize}
\end{itemize}

If you get stuck in a crowd in China --- it's not hard to do in a
country of nearly 1.4 billion --- you may hear someone mutter,
``\emph{Ren tai duo}!'': ``Too many people!'' It's a common but
misleading complaint. The real demographic crisis that prompted the
Chinese government's decision last week to
\href{http://www.nytimes.com/2015/10/30/world/asia/china-end-one-child-policy.html}{end
its one-child policy} is more palpable on the quiet Shanghai lane where
I live with my family: There is a dearth of young people.

Our neighbors consist mainly of aging pensioners and young Chinese
families with a single child, or no children at all. After 35 years of
one of the world's most radical experiments in social engineering,
Shanghai's fertility rates have plunged to perilously low levels: just
0.7 children per couple, less than half the national average and a third
of the 2.1 replacement rate. (The United States' replacement rate is
about 1.9.)

When we go out together on the streets of Shanghai, our two sons draw
double takes (along with the inevitable question: ``They're twins,
right?''). The confusion provoked by the sight of two boys in a single
family may soon dissipate, even if the social complications triggered by
the one-child policy will continue to shape China for decades to come.
By promising to allow families to have two children --- but no more ---
the government hopes to avert a demographic time bomb that is the
precise opposite of the one it faced 35 years ago. Back then, in the
aftermath of Mao Zedong's patriotic campaign to produce more children to
``make the nation stronger,'' Deng Xiaoping instituted the one-child
policy to reduce the number of mouths to feed, stimulating economic
growth and prosperity.

The debate over whether the one-child policy has been essential to
China's rise, or whether that would have been achieved naturally without
such an intrusive campaign, will rage for years to come. But even the
government has come to recognize, belatedly, its dangerous social and
economic consequences.

Chinese officials still seem impervious to the needless human suffering
the policy has inflicted: the forced abortions and sterilizations, the
undocumented children born and raised in the shadows, the persecution
and even imprisonment of those (like the blind lawyer
\href{http://www.nytimes.com/2012/05/20/world/asia/china-dissident-chen-guangcheng-united-states.html}{Chen
Guancheng}) who tried to expose its abuses. But Beijing's reversal is an
attempt to mitigate the massive social imbalances that will most likely
reverberate for generations: the shrinking work force that is hurting
China's competitiveness; a rapidly aging population with too few young
people to shoulder the burden; and a sex ratio so skewed that there is
now a bubble of 25 million extra males of marrying age, ``bare
branches'' on the family tree with few prospects of ever finding a wife.

The gender gap can be seen in every corner of China, from our Shanghai
lane to the tiny village of Yihe in the mountains of Hebei province.
When I visited Yihe a couple years ago, I met a father who had recently
sold his flock of sheep --- his life savings --- to pay for the
foundation of his son's so-called ``wife-attracting house'' in a town
nearby. His 26-year-old son, Liu Wushu, had returned from a stint in the
Army to find that he and his male cousins were the only young adults
left in Yihe. The rest were old people. The young women Liu knew had all
married men of higher social status and moved out. Left behind, the
bachelors were now begging families for loans so they could build houses
as part of a ``bride price'' to lure a wife. In a region where the sex
ratio is even higher than the national average of 116 boys for every 100
girls, Liu admitted: ``This might be my only chance.''

The emergence of ``bachelor villages'' like Yihe was not unforeseen. A
cultural preference for boys as family heirs meant that many parents
tried to avoid having a daughter through selective abortion, adoption,
even infanticide. One architect of the original one-child policy, Tian
Xueyuan,
\href{http://www.reuters.com/article/2013/01/22/us-china-population-idUSBRE90K0UV20130122}{told
Reuters} he warned top officials years ago that ``a substantial portion
of China's men will not be able to find a match \ldots{} and that will
be a major factor of social instability.''

With an estimated 30 million to 35 million unmarried men by 2020 --- a
huge mass of unchanneled testosterone --- critics worry that China could
face rising crime rates, social protests or a larger, more aggressive
military. Already, the shortage of brides has fueled an
\href{http://www.globaltimes.cn/content/814283.shtml}{underground
wife-smuggling trade} from the poorest Chinese provinces and neighboring
countries like Vietnam and Myanmar. Two weeks ago, the Chinese economist
Xie Zuoshi
\href{http://www.bbc.com/news/world-asia-china-34612919}{sparked
outrage} with a modest proposal for poor bachelors: wife-sharing.
(Polyandry is practiced in a few pockets of rural China. I have spent
time with two brothers who share a wife --- and, confusingly, a single
child --- in a Tibetan part of Yunnan province.) Despite a barrage of
criticism, Xie insisted he was only being practical. If China were to
let ``30 million bachelors have no women and no hope, they will hate
society,'' he said. ``We would have a serious social problem.''

When I returned to Yihe recently, Liu's house was nearly finished. A
matchmaker had introduced him to a few women from other villages, but
none had worked out. Over a bowl of noodles, Liu confessed something he
hadn't yet told his father: He was giving up. In a few days, he would be
leaving Yihe to join some of his Army buddies working in the oil fields.
What about the house for which his father had sold his last sheep?
``That was my father's dream, not mine,'' he said, tears pooling in his
eyes. The thought of spending his life in this desolate valley,
unmarried, was too much. ``I can't live my father's life,'' he said.

That afternoon, Liu's father gave me a tour of the new house, eagerly
showing off its smooth concrete floors, indoor bathroom and electrical
outlets, so different from the dirt-floor home he shares with his wife.
``What girl wouldn't want this?'' he said. Liu didn't say a word. When
his plan to leave finally came up, his father couldn't look at him.
``And where would he go?'' he asked me, before retreating to another
room to stew in silence. Liu left a few days later, leaving the new
house empty and abandoned. It stands as a symbol of two implacable
legacies of the one-child policy: a ``bare branch'' unable to find a
wife and an aging father unable to keep his only son close to take of
him and his wife in their old age.

Last week, back in Shanghai, I looked out my window and saw a young
girl, one of the single children living on our lane, being carried home
in the arms of her father as her mother and grandparents looked on. It
was a happy scene, but it still made me reflect on the all-pervasive
influence of the one-child policy's social engineering. China has gone
through periods of historic suffering (the Cultural Revolution, the
Great Leap Forward) that have been all but erased from the collective
memory. But the one-child policy may endure far longer, for it is a
policy that has been written in the lives --- and absences --- of nearly
every Chinese family.

Advertisement

\protect\hyperlink{after-bottom}{Continue reading the main story}

\hypertarget{site-index}{%
\subsection{Site Index}\label{site-index}}

\hypertarget{site-information-navigation}{%
\subsection{Site Information
Navigation}\label{site-information-navigation}}

\begin{itemize}
\tightlist
\item
  \href{https://help.nytimes.com/hc/en-us/articles/115014792127-Copyright-notice}{©~2020~The
  New York Times Company}
\end{itemize}

\begin{itemize}
\tightlist
\item
  \href{https://www.nytco.com/}{NYTCo}
\item
  \href{https://help.nytimes.com/hc/en-us/articles/115015385887-Contact-Us}{Contact
  Us}
\item
  \href{https://www.nytco.com/careers/}{Work with us}
\item
  \href{https://nytmediakit.com/}{Advertise}
\item
  \href{http://www.tbrandstudio.com/}{T Brand Studio}
\item
  \href{https://www.nytimes.com/privacy/cookie-policy\#how-do-i-manage-trackers}{Your
  Ad Choices}
\item
  \href{https://www.nytimes.com/privacy}{Privacy}
\item
  \href{https://help.nytimes.com/hc/en-us/articles/115014893428-Terms-of-service}{Terms
  of Service}
\item
  \href{https://help.nytimes.com/hc/en-us/articles/115014893968-Terms-of-sale}{Terms
  of Sale}
\item
  \href{https://spiderbites.nytimes.com}{Site Map}
\item
  \href{https://help.nytimes.com/hc/en-us}{Help}
\item
  \href{https://www.nytimes.com/subscription?campaignId=37WXW}{Subscriptions}
\end{itemize}
