Sections

SEARCH

\protect\hyperlink{site-content}{Skip to
content}\protect\hyperlink{site-index}{Skip to site index}

\href{https://www.nytimes.com/section/us}{U.S.}

\href{https://myaccount.nytimes.com/auth/login?response_type=cookie\&client_id=vi}{}

\href{https://www.nytimes.com/section/todayspaper}{Today's Paper}

\href{/section/us}{U.S.}\textbar{}Citing Climate Change, Obama Rejects
Construction of Keystone XL Oil Pipeline

\url{https://nyti.ms/1WCilTy}

\begin{itemize}
\item
\item
\item
\item
\item
\item
\end{itemize}

Advertisement

\protect\hyperlink{after-top}{Continue reading the main story}

Supported by

\protect\hyperlink{after-sponsor}{Continue reading the main story}

\hypertarget{citing-climate-change-obama-rejects-construction-of-keystone-xl-oil-pipeline}{%
\section{Citing Climate Change, Obama Rejects Construction of Keystone
XL Oil
Pipeline}\label{citing-climate-change-obama-rejects-construction-of-keystone-xl-oil-pipeline}}

\includegraphics{https://static01.nyt.com/images/2015/11/07/us/07keystone-hp2/07keystone-hp2-videoSixteenByNine1050-v3.jpg}

By \href{https://www.nytimes.com/by/coral-davenport}{Coral Davenport}

\begin{itemize}
\item
  Nov. 6, 2015
\item
  \begin{itemize}
  \item
  \item
  \item
  \item
  \item
  \item
  \end{itemize}
\end{itemize}

WASHINGTON --- President Obama announced on Friday that he had rejected
the request from a Canadian company to build the Keystone XL oil
pipeline, ending a seven-year review that had become a symbol of the
debate over his climate policies.

Mr. Obama's denial of the proposed 1,179-mile pipeline, which would have
carried 800,000 barrels a day of carbon-heavy petroleum from the
Canadian oil sands to the Gulf Coast, comes as he seeks to build an
ambitious legacy on climate change.

``America is now a global leader when it comes to taking serious action
to fight climate change,'' Mr. Obama said in remarks from the White
House. ``And, frankly, approving this project would have undercut that
global leadership.''

The move was made ahead of a major
\href{http://www.nytimes.com/news-event/un-climate-change-conference}{United
Nations summit meeting on climate change} to be held in Paris in
December, when Mr. Obama hopes to help broker a historic agreement
committing the world's nations to enacting new policies to counter
global warming. While the rejection of the pipeline is largely symbolic,
Mr. Obama has sought to telegraph to other world leaders that the United
States is serious about acting on climate change.

The once-obscure Keystone project became a political symbol amid broader
clashes over energy, climate change and the economy. The rejection of a
single oil infrastructure project will have little impact on efforts to
reduce greenhouse gas pollution, but the pipeline plan gained an outsize
profile after environmental activists spent four years marching and
rallying against it in front of the White House and across the country.

Mr. Obama said that the pipeline has occupied what he called ``an
overinflated role in our political discourse.''

``It has become a symbol too often used as a campaign cudgel by both
parties rather than a serious policy matter,'' he said. ``And all of
this obscured the fact that this pipeline would neither be a silver
bullet for the economy, as was promised by some, nor the express lane to
climate disaster proclaimed by others.''

Republicans and the oil industry had demanded that the president approve
the pipeline, which they said would create jobs and stimulate economic
growth. Many Democrats, particularly those in oil-producing states such
as North Dakota, also supported the project. In February, congressional
Democrats joined with Republicans in sending Mr. Obama a bill to speed
approval of the project,
\href{http://www.nytimes.com/2015/02/25/us/politics/as-expected-obama-vetoes-keystone-xl-pipeline-bill.html}{but
the president vetoed the measure}.

The rejection of the pipeline is one of several actions Mr. Obama has
taken as he intensifies his push on climate change in his last year in
office. In August, he announced his most significant climate policy, a
set of aggressive new regulations to cut emissions of planet-warming
carbon pollution from the nation's power plants.

Both sides of the debate saw the Keystone rejection as a major symbolic
step, a sign that the president was willing to risk angering a
bipartisan majority of lawmakers in the pursuit of his environmental
agenda. And both supporters and critics of Mr. Obama saw the
surprisingly powerful influence of environmental activists in the
decision.

``Once the grass-roots movement on the Keystone pipeline mobilized, it
changed what it meant to the president,'' said Douglas G. Brinkley, a
historian at Rice University who writes about presidential environmental
legacies. ``It went from a routine infrastructure project to the symbol
of an era.''

Environmental activists cheered the decision as a vindication of their
influence.

``President Obama is the first world leader to reject a project because
of its effect on the climate,'' said Bill McKibben, founder of the
activist group 350.org, which led the campaign against the pipeline.
``That gives him new stature as an environmental leader, and it
eloquently confirms the five years and millions of hours of work that
people of every kind put into this fight.''

Environmentalists had sought to block construction of the pipeline
because it would have provided a conduit for petroleum extracted from
the Canadian oil sands. The process of extracting that oil produces
about 17 percent more planet-warming greenhouse gases than the process
of extracting conventional oil.

\includegraphics{https://static01.nyt.com/images/2015/11/07/us/07keystone-jp2/07keystone-jp2-articleLarge.jpg?quality=75\&auto=webp\&disable=upscale}

But numerous State Department reviews concluded that construction of the
pipeline would have little impact on whether that type of oil was
burned, because it was already being extracted and moving to market via
rail and existing pipelines. In citing his reason for the decision, Mr.
Obama noted the State Department findings that construction of the
pipeline would not have created a significant number of new jobs,
lowered oil or gasoline prices or significantly reduced American
dependence on foreign oil.

``From a market perspective, the industry can find a different way to
move that oil,'' said Christine Tezak, an energy market analyst at
ClearView Energy Partners, a Washington firm. ``How long it takes is
just a result of oil prices. If prices go up, companies will get the oil
out.''

However, a State Department review also found that demand for the oil
sands fuel would drop if oil prices fell below \$65 a barrel, since
moving oil by rail is more expensive than using a pipeline. An
Environmental Protection Agency review of the project this year noted
that under such circumstances, construction of the pipeline could be
seen as contributing to emissions, since companies might be less likely
to move the oil via expensive rail when oil prices are low --- but would
be more likely to move it cheaply via the pipeline. The price of oil has
plummeted this year, hovering at less than \$50 a barrel.

The recent election of a new Canadian prime minister, Justin Trudeau,
may also have influenced Mr. Obama's decision. Mr. Trudeau's
predecessor, Stephen Harper, had pushed the issue as a top priority in
the relationship between the United States and Canada, personally urging
Mr. Obama to approve the project. Blocking the project during the Harper
administration would have bruised ties with a crucial ally.

While Mr. Trudeau also supports construction of the Keystone pipeline,
he has not made the issue central to Canada's relationship with the
United States, and has criticized Mr. Harper for presenting Canada's
position as an ultimatum, while not taking substantial action on climate
change related to the oil sands.

Mr. Trudeau did not raise the issue during his first post-election
conversation with Mr. Obama.

The construction would have had little impact on the nation's economy. A
State Department analysis concluded that building the pipeline would
have created jobs, but the total number represented less than one-tenth
of 1 percent of the nation's total employment. The analysis estimated
that Keystone would support 42,000 temporary jobs over its two-year
construction period --- about 3,900 of them in construction and the rest
in indirect support jobs, such as food service. The department estimated
that the project would create about 35 permanent jobs.

Republicans and the oil industry criticized Mr. Obama for what they have
long said was his acquiescence to the pressure of activists and
environmentally minded political donors.

\includegraphics{https://static01.nyt.com/images/2015/02/12/us/12KEYSTONE2/12KEYSTONE2-videoSixteenByNine1050.jpg}

``A decision this poorly made is not symbolic, but deeply cynical,''
said Senator Lisa Murkowski, the Alaska Republican who leads the Senate
Energy and Natural Resources Committee. ``It does not rest on the facts
--- it continues to distort them.''

Jack Gerard, the head of the American Petroleum Institute, which lobbies
for oil companies, said in a statement, ``Unfortunately for the majority
of Americans who have said they want the jobs and economic benefits
Keystone XL represents, the White House has placed political
calculations above sound science.''

Russ Girling, the president and chief executive of TransCanada, said in
a statement that the president's decision was not consistent with the
State Department's review. ``Today, misplaced symbolism was chosen over
merit and science,'' said Mr. Girling, whose company is based in
Calgary, Alberta. ``Rhetoric won out over reason.''

The statement said that the company was reviewing the decision but
offered no indication if it planned to submit a new application. If a
Republican wins the 2016 presidential election, a new submission of the
pipeline permit application could yield a different outcome.

``President Obama's rejection of the Keystone XL pipeline is a huge
mistake, and is the latest reminder that this administration continues
to prioritize the demands of radical environmentalists over America's
energy security,'' said Senator Marco Rubio of Florida, who is seeking
the Republican nomination for president. ``When I'm president, Keystone
will be approved, and President Obama's backward energy policies will
come to an end.''

As Mr. Obama seeks to carve out a substantial environmental legacy, his
decision on the pipeline pales in import compared with his use of
Environmental Protection Agency regulations. The power plant rules he
announced in August have met with legal challenges, but if they are put
in place, they could lead to a transformation of the nation's energy
economy, shuttering fossil fuel plants and rapidly increasing production
of wind and solar.

Those rules are at the heart of Mr. Obama's push for a global agreement.

But advocates of the agreement said that the Keystone decision, even
though it is largely symbolic, could show other countries that Mr. Obama
is willing to make tough choices about climate change.

``The rejection of the Keystone permit was key for the president to keep
his climate chops at home and with the rest of the world,'' said Durwood
Zaelke, the president of the Institute for Governance and Sustainable
Development, a Washington research organization.

Advertisement

\protect\hyperlink{after-bottom}{Continue reading the main story}

\hypertarget{site-index}{%
\subsection{Site Index}\label{site-index}}

\hypertarget{site-information-navigation}{%
\subsection{Site Information
Navigation}\label{site-information-navigation}}

\begin{itemize}
\tightlist
\item
  \href{https://help.nytimes.com/hc/en-us/articles/115014792127-Copyright-notice}{©~2020~The
  New York Times Company}
\end{itemize}

\begin{itemize}
\tightlist
\item
  \href{https://www.nytco.com/}{NYTCo}
\item
  \href{https://help.nytimes.com/hc/en-us/articles/115015385887-Contact-Us}{Contact
  Us}
\item
  \href{https://www.nytco.com/careers/}{Work with us}
\item
  \href{https://nytmediakit.com/}{Advertise}
\item
  \href{http://www.tbrandstudio.com/}{T Brand Studio}
\item
  \href{https://www.nytimes.com/privacy/cookie-policy\#how-do-i-manage-trackers}{Your
  Ad Choices}
\item
  \href{https://www.nytimes.com/privacy}{Privacy}
\item
  \href{https://help.nytimes.com/hc/en-us/articles/115014893428-Terms-of-service}{Terms
  of Service}
\item
  \href{https://help.nytimes.com/hc/en-us/articles/115014893968-Terms-of-sale}{Terms
  of Sale}
\item
  \href{https://spiderbites.nytimes.com}{Site Map}
\item
  \href{https://help.nytimes.com/hc/en-us}{Help}
\item
  \href{https://www.nytimes.com/subscription?campaignId=37WXW}{Subscriptions}
\end{itemize}
