Sections

SEARCH

\protect\hyperlink{site-content}{Skip to
content}\protect\hyperlink{site-index}{Skip to site index}

\href{https://www.nytimes.com/section/politics}{Politics}

\href{https://myaccount.nytimes.com/auth/login?response_type=cookie\&client_id=vi}{}

\href{https://www.nytimes.com/section/todayspaper}{Today's Paper}

\href{/section/politics}{Politics}\textbar{}President Obama Signs Into
Law a Rewrite of No Child Left Behind

\url{https://nyti.ms/1lSY23t}

\begin{itemize}
\item
\item
\item
\item
\item
\end{itemize}

Advertisement

\protect\hyperlink{after-top}{Continue reading the main story}

Supported by

\protect\hyperlink{after-sponsor}{Continue reading the main story}

\hypertarget{president-obama-signs-into-law-a-rewrite-of-no-child-left-behind}{%
\section{President Obama Signs Into Law a Rewrite of No Child Left
Behind}\label{president-obama-signs-into-law-a-rewrite-of-no-child-left-behind}}

\includegraphics{https://static01.nyt.com/images/2015/12/11/us/11obama-web/11obama-web-videoSixteenByNine1050.jpg}

By \href{https://www.nytimes.com/by/julie-hirschfeld-davis}{Julie
Hirschfeld Davis}

\begin{itemize}
\item
  Dec. 10, 2015
\item
  \begin{itemize}
  \item
  \item
  \item
  \item
  \item
  \end{itemize}
\end{itemize}

WASHINGTON --- Putting an end to more than a decade of strict federal
control of public education,
\href{http://topics.nytimes.com/top/reference/timestopics/people/o/barack_obama/index.html?inline=nyt-per}{President
Obama} on Thursday signed a sweeping rewrite of the
\href{http://topics.nytimes.com/top/reference/timestopics/subjects/n/no_child_left_behind_act/index.html?inline=nyt-classifier}{No
Child Left Behind act} that returns power to states and local districts
to determine how to improve troubled schools.

The bill is a bipartisan measure that preserves federally mandated
standardized testing but eliminates the punitive consequences for states
and districts that perform poorly. The new version, renamed the Every
Student Succeeds Act, also bars the government from imposing academic
requirements like the Common Core.

``This bill makes long-overdue fixes to the last education law,
replacing the one-size-fits-all approach to reform with a commitment to
provide every student with a well-rounded education,'' Mr. Obama said at
a White House signing ceremony for the law. ``With this bill, we
reaffirm that fundamental American ideal that every child --- regardless
of race, income, background, the ZIP code where they live --- deserves
the chance to make out of their lives what they will.''

Embraced by an unusual coalition of Republican, Democrats, business
groups and teachers' unions, the law was a curiosity in a capital more
often gripped lately by partisan gridlock. Mr. Obama referred to the
bipartisan bill-signing as ``a Christmas miracle.''

The law is the latest revision of the 1965 Elementary and Secondary
Education Act, which sets out the government's role in schooling from
kindergarten to 12th grade. It was the product of lengthy negotiations
between Democrats and Republicans and a shared opposition to the
strictures in the No Child Left Behind law signed by George W. Bush 14
years ago --- and to the Obama administration's efforts to institute its
own performance incentives tied to teachers' pay.

``The backlash to Washington trying to tell 100,000 schools too much
about what they should be doing,'' said Senator Lamar Alexander,
Republican of Tennessee, an architect of the law, ``caused people on
both the left and the right to remember that the path to higher
standards and better teaching and real accountability is community by
community, classroom by classroom, state by state, and not through the
federal government dictating the solution.''

``What we've learned from this is that a national school board doesn't
work in the United States of America --- we're just too big and complex
a country,'' added Mr. Alexander, who served as education secretary from
1991 to 1993. ``We've settled the question of where the responsibility
is going to be, probably for the next 10 to 20 years.''

The new measure will maintain the mandatory standardized testing in
reading and math established by the Bush-era law, but leave it up to
state and local officials to set their own performance goals, rate
schools and determine how to fix those that fail to meet their
objectives.

It will remove a requirement that all children become proficient in
reading and math by a certain date.

States will still face some federal requirements for the schools that
struggle most, including the lowest-performing 5 percent of schools and
those where more than a third of high school students do not graduate on
time. Those schools will also be required to take steps to close gaps in
achievement and in graduation rates between poor and minority students
and other groups. But the federal government will not dictate how they
must do so.

The law explicitly prohibits the government from imposing academic
standards on states and from issuing waivers that are not authorized by
law.
\href{http://topics.nytimes.com/top/reference/timestopics/people/d/arne_duncan/index.html?inline=nyt-per}{Arne
Duncan}, Mr. Obama's outgoing education secretary, used such waivers to
exempt states from the most burdensome federal mandates if the states
agreed to establish their own rigorous academic standards.

Randi Weingarten, the president of the American Federation of Teachers,
said the law marked ``a new day in public education'' that would bring
about ``the most sweeping, positive changes to public education we've
seen in two decades.''

``It ensures that the federal government can no longer require these
tests as part of teacher evaluation,'' she said. ``And it makes public
education a joint responsibility.''

For some, the pendulum may have swung too far from a robust federal
role.

Civil rights groups are concerned that by restricting federal authority
to intervene in states and districts, Mr. Obama is surrendering what has
been a potent tool in decades past to rectify racial discrimination in
the nation's schools.

``The whole purpose behind the original bill was to ensure that there
were consistent standards and federal oversight to make sure that states
and localities were doing the right thing by poor children, by children
who needed that assistance the most, and reducing that and granting so
much discretion to states is just worrisome,'' said Leslie Proll, the
director of policy at the NAACP Legal Defense Fund. ``Some states will
do the right thing, and that's great, others may not, and therein lies
the problem.''

She said her group and other civil rights organizations would press Mr.
Obama to act swiftly during his final year in office to use whatever
federal authority was left to set education parameters for states.

Margaret Spellings, who served as Mr. Bush's education secretary from
2005 to 2009, said she worried that in removing the consequences for
failing to meet a federal educational standard, the law would take the
pressure off states and districts to perform, especially for poor and
minority students.

``I'm disappointed that the law doesn't have consequences except for the
bottom 5 percent,'' Ms. Spellings said. ``We are now in the era of local
control once again, and with that comes a lot of responsibility to work
with states and school districts to make sure that we close the
achievement gap.''

``I'm a little bit skeptical,'' she added. ``We've tried the local
control approach before, and we saw pretty pitiful results.''

The law falls short of some of Mr. Obama's top priorities. It does not
include a major expansion of early childhood education, as the president
wanted. And it does not go as far as some Republicans wanted in
providing flexibility for states from federal oversight.

But it represents a compromise that neither side thought was
particularly likely just a year ago.

Mr. Alexander said he had recently told Mr. Duncan that when he
discussed the prospects with a Democratic colleague at the start of the
year, they thought the odds against enactment of an education rewrite
were 5 to 1. ``Arne said, `I would have put it at 10 to 1,'~'' Mr.
Alexander said.

Advertisement

\protect\hyperlink{after-bottom}{Continue reading the main story}

\hypertarget{site-index}{%
\subsection{Site Index}\label{site-index}}

\hypertarget{site-information-navigation}{%
\subsection{Site Information
Navigation}\label{site-information-navigation}}

\begin{itemize}
\tightlist
\item
  \href{https://help.nytimes.com/hc/en-us/articles/115014792127-Copyright-notice}{©~2020~The
  New York Times Company}
\end{itemize}

\begin{itemize}
\tightlist
\item
  \href{https://www.nytco.com/}{NYTCo}
\item
  \href{https://help.nytimes.com/hc/en-us/articles/115015385887-Contact-Us}{Contact
  Us}
\item
  \href{https://www.nytco.com/careers/}{Work with us}
\item
  \href{https://nytmediakit.com/}{Advertise}
\item
  \href{http://www.tbrandstudio.com/}{T Brand Studio}
\item
  \href{https://www.nytimes.com/privacy/cookie-policy\#how-do-i-manage-trackers}{Your
  Ad Choices}
\item
  \href{https://www.nytimes.com/privacy}{Privacy}
\item
  \href{https://help.nytimes.com/hc/en-us/articles/115014893428-Terms-of-service}{Terms
  of Service}
\item
  \href{https://help.nytimes.com/hc/en-us/articles/115014893968-Terms-of-sale}{Terms
  of Sale}
\item
  \href{https://spiderbites.nytimes.com}{Site Map}
\item
  \href{https://help.nytimes.com/hc/en-us}{Help}
\item
  \href{https://www.nytimes.com/subscription?campaignId=37WXW}{Subscriptions}
\end{itemize}
