Sections

SEARCH

\protect\hyperlink{site-content}{Skip to
content}\protect\hyperlink{site-index}{Skip to site index}

\href{https://www.nytimes.com/section/world/europe}{Europe}

\href{https://myaccount.nytimes.com/auth/login?response_type=cookie\&client_id=vi}{}

\href{https://www.nytimes.com/section/todayspaper}{Today's Paper}

\href{/section/world/europe}{Europe}\textbar{}Britain Considers
Deflating Power of House of Lords

\url{https://nyti.ms/1k5KUqz}

\begin{itemize}
\item
\item
\item
\item
\item
\end{itemize}

Advertisement

\protect\hyperlink{after-top}{Continue reading the main story}

Supported by

\protect\hyperlink{after-sponsor}{Continue reading the main story}

\hypertarget{britain-considers-deflating-power-of-house-of-lords}{%
\section{Britain Considers Deflating Power of House of
Lords}\label{britain-considers-deflating-power-of-house-of-lords}}

By \href{http://www.nytimes.com/by/stephen-castle}{Stephen Castle}

\begin{itemize}
\item
  Dec. 17, 2015
\item
  \begin{itemize}
  \item
  \item
  \item
  \item
  \item
  \end{itemize}
\end{itemize}

LONDON --- Few dispute the need for changes to the House of Lords,
Britain's unelected and scandal-hit upper parliamentary chamber, which
is the world's largest legislative assembly outside China. Most critics
complain about the chamber's increase in number in the past 15 years,
now with close to 800 voting members.

But on Thursday, recommendations were published that would reduce its
powers rather than its size.

The initiative follows a political tussle between ministers and the
chamber: the House of Lords rejected government proposals on curbing
social welfare payments in October.

That left a hole of about 4.4 billion pounds, or \$6.6 billion, in the
government's plans to balance the budget and prompted the chancellor of
the Exchequer, George Osborne, to make a humiliating
\href{http://www.nytimes.com/2015/11/26/world/europe/britain-chancellor-of-the-exchequer-george-osborne-welfare.html}{retreat}
over the issue.

In flexing its muscles, the Lords revived a debate that has stirred,
intermittently, for more than a century over the role of the upper
chamber, which generally accepts its junior status in Parliament to
elected lawmakers in the House of Commons.

This past summer, the House of Lords also made embarrassing headlines
when John Sewel, the chairman of its committee responsible for upholding
standards among members,
\href{http://www.nytimes.com/2015/07/29/world/europe/john-sewel-resigns-from-house-of-lords-in-britain-amid-drugs-prostitutes-scandal.html}{stepped
down} after the newspaper The Sun on Sunday accused him of snorting
cocaine off the chest of a prostitute and
\href{http://www.thesun.co.uk/sol/homepage/news/6560352/Baron-John-Sewel-drug-binges-with-prostitutes.html}{released
a video} that showed him consuming a white powder.

Tensions over the powers of the House of Lords are acute now because the
Conservative government, in which Mr. Osborne serves, has an outright
majority in the House of Commons.

But it does not have the same numerical advantage as in the House of
Lords, where it can sometimes be outvoted if the opposition Labour Party
and the center-left Liberal Democrats join forces, leading to
legislation being amended or delayed.

After the welfare vote in October, the government asked a senior
Conservative politician, Thomas Galbraith, known as Lord Strathclyde, to
investigate some of the powers of the Lords (though not other issues,
such as the size of the assembly).

\includegraphics{https://static01.nyt.com/images/2015/12/18/world/18lords_web1/18lords_web1-articleLarge.jpg?quality=75\&auto=webp\&disable=upscale}

On Thursday, he outlined three possible ways to reduce the power of the
Lords over statutory instruments, a form of secondary legislation. Under
Lord Strathclyde's preferred option, the House of Lords would be able to
ask the House of Commons to reconsider secondary legislation when
disputes exist, but the elected chamber would have the final say.

In a statement, Prime Minister David Cameron thanked Lord Strathclyde
and said he would ``consider his recommendations carefully before
responding in the new year.''

Baroness Angela Smith, who leads the Labour members of the House of
Lords, said that if the proposals were a reaction to the welfare vote in
October, it ``would be a massive overreaction'' but that in fact it was
``far more serious than that.''

Meg Russell, director of the Constitution Unit at University College in
London, said that the proposed changes were ``potentially very
significant'' but that there was ``some key detail missing,'' which made
it hard to assess the likely impact.

But she added that there was an opportunity to negotiate a bigger
package of changes and to agree to limits on the size of the chamber as
a quid pro quo for the reduction of powers that the government wanted.

``If you ask most people what is the problem with the House of Lords ---
if they have a view --- they will point to the increase in its size in
recent years,'' Prof. Russell said. ``I think the government ought to be
prepared to do a deal on limiting numbers in exchange to changes to
powers.''

Many of the 790 voting members of the Lords are retired politicians or
other party loyalists. A large number exists because the government can
nominate supporters to the House of Lords, who are then appointed by the
monarch.

The Lords do not receive a salary, but members can claim up to £300 as a
daily allowance for attending sessions, and do not have to give up any
other job. A seat there also brings prestige, a public platform and an
opportunity to shape laws, without having to fight any election.

Despite its critics, the House of Lords has survived largely because of
a lack of an agreement on a workable alternative and the fear that, were
it converted into an elected chamber, it would produce the sort of
deadlock seen in some bicameral legislatures, such as that of the United
States.

Advertisement

\protect\hyperlink{after-bottom}{Continue reading the main story}

\hypertarget{site-index}{%
\subsection{Site Index}\label{site-index}}

\hypertarget{site-information-navigation}{%
\subsection{Site Information
Navigation}\label{site-information-navigation}}

\begin{itemize}
\tightlist
\item
  \href{https://help.nytimes.com/hc/en-us/articles/115014792127-Copyright-notice}{©~2020~The
  New York Times Company}
\end{itemize}

\begin{itemize}
\tightlist
\item
  \href{https://www.nytco.com/}{NYTCo}
\item
  \href{https://help.nytimes.com/hc/en-us/articles/115015385887-Contact-Us}{Contact
  Us}
\item
  \href{https://www.nytco.com/careers/}{Work with us}
\item
  \href{https://nytmediakit.com/}{Advertise}
\item
  \href{http://www.tbrandstudio.com/}{T Brand Studio}
\item
  \href{https://www.nytimes.com/privacy/cookie-policy\#how-do-i-manage-trackers}{Your
  Ad Choices}
\item
  \href{https://www.nytimes.com/privacy}{Privacy}
\item
  \href{https://help.nytimes.com/hc/en-us/articles/115014893428-Terms-of-service}{Terms
  of Service}
\item
  \href{https://help.nytimes.com/hc/en-us/articles/115014893968-Terms-of-sale}{Terms
  of Sale}
\item
  \href{https://spiderbites.nytimes.com}{Site Map}
\item
  \href{https://help.nytimes.com/hc/en-us}{Help}
\item
  \href{https://www.nytimes.com/subscription?campaignId=37WXW}{Subscriptions}
\end{itemize}
