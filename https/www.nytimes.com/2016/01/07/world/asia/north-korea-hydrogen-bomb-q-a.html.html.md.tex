Sections

SEARCH

\protect\hyperlink{site-content}{Skip to
content}\protect\hyperlink{site-index}{Skip to site index}

\href{https://www.nytimes.com/section/world/asia}{Asia Pacific}

\href{https://myaccount.nytimes.com/auth/login?response_type=cookie\&client_id=vi}{}

\href{https://www.nytimes.com/section/todayspaper}{Today's Paper}

\href{/section/world/asia}{Asia Pacific}\textbar{}Did North Korea
Detonate a Hydrogen Bomb? Here's What We Know

\url{https://nyti.ms/1JtIkGW}

\begin{itemize}
\item
\item
\item
\item
\item
\end{itemize}

Advertisement

\protect\hyperlink{after-top}{Continue reading the main story}

Supported by

\protect\hyperlink{after-sponsor}{Continue reading the main story}

Q. and A.

\hypertarget{did-north-korea-detonate-a-hydrogen-bomb-heres-what-we-know}{%
\section{Did North Korea Detonate a Hydrogen Bomb? Here's What We
Know}\label{did-north-korea-detonate-a-hydrogen-bomb-heres-what-we-know}}

\includegraphics{https://static01.nyt.com/images/2016/01/07/world/07northkoreaQA_bomb/07northkoreaQA_bomb-articleLarge.jpg?quality=75\&auto=webp\&disable=upscale}

By The New York Times

\begin{itemize}
\item
  Jan. 6, 2016
\item
  \begin{itemize}
  \item
  \item
  \item
  \item
  \item
  \end{itemize}
\end{itemize}

\href{http://www.nytimes.com/2016/01/07/world/asia/north-korea-hydrogen-bomb-claim-reactions.html}{North
Korea's claim} that it set off a hydrogen bomb on Wednesday --- in what
would be the fourth time it has tested a nuclear weapon since 2006 ---
has stirred concerns among governments around the world. Below is a
brief primer on some of the central issues at stake.

\textbf{Q.} \emph{How is a hydrogen bomb different from an atomic bomb?}

\textbf{A.} A hydrogen bomb, also known as a thermonuclear bomb,
combines hydrogen isotopes under extremely high temperatures to form
helium, in a process known as nuclear fusion. It is
\href{http://www.ucsusa.org/nuclear-weapons/us-nuclear-weapons-policy/how-nuclear-weapons-work\#.Vo0mr5OLRQN}{more
powerful} than a conventional atomic weapon: It uses the energy released
from the combination of two light atomic nuclei, while an atomic bomb
uses the energy released when a heavy atomic nucleus splits, a process
known as nuclear fission. American scientists developed the hydrogen
bomb, which was first tested in 1952.

Britain, China, France, Russia and the United States are known to
possess thermonuclear weapons. Whether the four other countries known to
have atomic bombs --- India, Israel, North Korea and Pakistan --- also
have hydrogen bombs is not certain.

\textbf{Q.} \emph{What, precisely, did North Korea announce?}

\textbf{A.} The North's government said that it had
\href{http://www.nytimes.com/2016/01/06/world/asia/north-korea-hydrogen-bomb-test.html}{detonated
a hydrogen bomb} --- its first --- at 10 a.m. on Wednesday.

``This test is a measure for self-defense the D.P.R.K. has taken to
firmly protect the sovereignty of the country and the vital right of the
nation from the ever-growing nuclear threat and blackmail by the
U.S.-led hostile forces and to reliably safeguard the peace on the
Korean Peninsula and regional security,'' it said, referring to the
North's formal name, the Democratic People's Republic of Korea.

\includegraphics{https://static01.nyt.com/images/2016/01/07/world/07northkoreaQA_jongun/07northkoreaQA_jongun-articleLarge.jpg?quality=75\&auto=webp\&disable=upscale}

\textbf{Q.} \emph{How can North Korea's claim be verified?}

\textbf{A.} The United States, South Korea and scientists doubt that the
claim is true.

The White House said that initial data from its monitoring stations in
Asia were ``not consistent'' with a test of a hydrogen bomb. When the
United States
\href{http://www.nytimes.com/2016/01/07/science/comparisons-dont-support-north-koreas-claims-of-a-hydrogen-bomb-experts-say.html}{detonated
a hydrogen bomb} deep beneath the Alaskan tundra in 1971, it set off a
colossal upheaval of rock and earth, a magnitude 6.8 seismic event. In
contrast, South Korea estimated that the bomb detonated on Wednesday was
a magnitude 4.8 event --- smaller even than the magnitude 4.9 reported
after the North's last nuclear test, in 2013.

\textbf{Q.} \emph{What if it was not a hydrogen bomb?}

\textbf{A.} When Kim Jong-un, the North's leader, announced in December
that his country had finally developed the technology to build a
thermonuclear weapon, experts were skeptical. Some said that North Korea
might be preparing to test a ``boosted-fission weapon,'' more powerful
than a traditional atomic bomb. Designers can easily increase the
destructive power of an atomic bomb by putting at its core a small
amount of tritium, a radioactive form of hydrogen. The Yonhap News
Agency of South Korea reported that the government in Seoul was leaning
toward the theory of a boosted-fission weapon, ``one level away from a
hydrogen bomb.''

\textbf{Q.} \emph{How many times has North Korea detonated a nuclear
weapon?}

\textbf{A.} This appears to be the fourth time. North Korea conducted
underground nuclear tests on
\href{http://www.nytimes.com/2006/10/09/world/asia/09korea.html}{Oct. 9,
2006};
\href{http://www.nytimes.com/2009/05/25/world/asia/25nuke.html}{May 25,
2009}; and
\href{http://www.nytimes.com/2013/02/12/world/asia/north-korea-nuclear-test.html}{Feb.
12, 2013}.

\textbf{Q.} \emph{What might North Korea be trying to accomplish with
its threats?}

\textbf{A.} In the past, United States administrations and South Korean
governments managed to tamp down periodic heightened tensions with North
Korea by offering concessions, including much-needed aid, in return for
the North's promising to end its nuclear weapons programs. Many analysts
believe that North Korea is again seeking aid and other concessions,
while some suggest that it merely wants to be recognized as a nuclear
state, like Pakistan.

Image

Yohei Hasegawa,~an officer at Japan's meteorological agency, displayed a
chart showing seismic activity in Tokyo on Wednesday.Credit...Yoshikazu
Tsuno/Agence France-Presse --- Getty Images

Still others suggest that the North genuinely fears an attack by the
United States or South Korea and views the warnings as deterrence.
Highlighting a perceived threat from abroad is also a favorite tool the
North Korean government uses to ensure internal cohesion in an
impoverished country that has experienced enormous privation, including
devastating famine and continuing pervasive hunger.

\textbf{Q.} \emph{Could North Korea attack the United States?}

\textbf{A.} Maybe. In 2012, North Korea launched a rocket that put its
first satellite into orbit --- raising the possibility of
intercontinental ballistic missiles that could reach North America. The
United Nations Security Council condemned the launching as a violation
of several Security Council resolutions and tightened sanctions against
it.

\textbf{Q.} \emph{How might the United States, China, Japan and South
Korea respond to a missile test or an attack?}

\textbf{A.} If a missile attack went into the water, even if it passed
over Japan, the two countries could ignore it. But if it headed for
land, the United States would probably use its missile interception
technology, including on Aegis-equipped ships off the Korean coast. If
there were to be a more direct attack, like the torpedo that sank a
South Korean warship in 2010, it is likely that both the United States
and South Korea would respond. China would be less likely to take
action.

\textbf{Q.} \emph{What was the global response to previous North Korean
rocket launchings?}

\textbf{A.} As the North's missile technology has become more
sophisticated, the launching of longer-range missiles has evoked more
international concern.

In 1998, when the North launched a Taepodong that flew over Japan, Japan
temporarily cut off its contribution toward a North Korean energy
project. But in July 2006, when the North launched another long-range
missile, various countries began imposing sanctions, while the Security
Council began adding to economic sanctions.

Image

President Obama looked toward North Korea during a visit to the
demilitarized zone between North and South Korea in 2012.Credit...Saul
Loeb/Agence France-Presse --- Getty Images

In April 2009, when the North's efforts to launch a three-stage Unha-2
rocket failed, the Security Council said it would strengthen punitive
measures. It did so after the North conducted a nuclear test the next
month.

In April 2012, the United States canceled planned food aid when the
North tried to launch a more advanced missile, the Unha-3. That
launching failed, but another in December succeeded in lifting a small
satellite into orbit. The Security Council tightened sanctions yet
again. After the North's nuclear test in February 2013, China, the
North's longtime protector, participated in writing painful new
sanctions aimed at North Korean banking, trade and travel.

\textbf{Q.} \emph{What is the Obama administration's policy on North
Korea?}

\textbf{A.} The Obama administration adopted a policy of ``strategic
patience'' in 2009, under which direct negotiations or offers of aid to
Pyongyang are withheld unless the North Korea leadership shows
``positive, constructive behavior'' and willingness to negotiate over
the dismantling of its nuclear weapons program.

The policy is a response to the American belief that the United States
had unwisely offered aid, often in the wake of Pyongyang's provocations,
or struck agreements with the North on which the North later reneged.
Strategic patience, in the words of Robert M. Gates, the former defense
secretary, grew out of a desire not ``to buy the same horse twice.''

Critics say that while the policy has allowed the United States to
weather multiple rounds of belligerence by Kim Jong-il and his son, Kim
Jong-un, without making concessions, it has done little to curb the
development of North Korea's nuclear weapons program.

Image

Kim Il-sung, left, the founder of North Korea, with his son, Kim
Jong-il, in Pyongyang in 1983~at a mass rally celebrating the country's
beginnings.Credit...Reuters

\textbf{Q.} \emph{What sanctions are currently in place?}

\textbf{A.} The Security Council has passed four resolutions since 2006
aimed at penalizing North Korea for its nuclear weapons program. In
addition, the United States, which remains in a technical state of war
with North Korea, has imposed its own regimen of strict economic
sanctions. The combined effects have severely squeezed, but not
crippled, North Korea's economy. The United Nations has prohibited the
North from conducting nuclear tests or launching ballistic missiles,
requested that it abandon all future efforts to pursue nuclear weapons
and urged it to return to negotiations with China, Japan, Russia, South
Korea and the United States, the so-called six-party talks.

The resolutions have also imposed embargoes on large-scale arms,
weapons-related research and development materials, and luxury goods;
banned many types of financial transactions including transfers of cash;
placed new restrictions on diplomats; and created monitoring mechanisms
for enforcement.

The
\href{http://www.treasury.gov/resource-center/sanctions/Programs/Documents/nkorea.txt}{American
sanctions} freeze all North Korean property interests in the United
States, ban most imports of goods and services from the North, and
prohibit American dealings with any names on a blacklist of North Korean
businesses and individuals suspected of illicit activities including
money laundering, counterfeiting, currency smuggling and narcotics
trafficking.

Nothing in the American sanctions prohibits American travel to North
Korea or the export of food and other types of humanitarian aid,
although there are some restrictions.

The sanctions leave room for considerable trade in many types of goods
and services. China, which supplies much of North Korea's basic needs,
is not in any violation of the United Nations resolutions.

Image

President Park Geun-hye of South Korea, center, during an urgent meeting
of her top national security aides in Seoul, South Korea, on
Wednesday.Credit...Jeon Jin-Hwan/Newsis, via Associated Press

\textbf{Q.} \emph{How is the South Korean government responding to the
North's threats?}

\textbf{A.} President Park Geun-hye, who took office in 2013, is the
daughter of a former president who ran South Korea as a dictator during
the Cold War. She once promised that if the North mounted a nuclear
attack, its government would be ``erased from the earth.'' She has
largely held a firm line on North Korea, after a more conciliatory
stance in the 1990s.

From 1998 to 2008, the South pursued a ``sunshine policy'' of
reconciliation and economic cooperation that sent billions of dollars in
business investments, goods and humanitarian aid to the North. Ms.
Park's immediate predecessor, Lee Myung-bak, said the North would need
to give up its nuclear weapons to receive any more aid. But he was
criticized for what many saw as a weak response after the North shelled
a South Korean island in 2010, killing four people.

\textbf{Q.} \emph{Why hasn't China stopped North Korea from its campaign
of threats? Is there any other country that has enough influence on
North Korea to stop it?}

\textbf{A.} China, the North's patron, has long feared that a collapse
of the North Korean government could lead to a unified Korea allied with
the United States.

China helped write and backed the most recent round of United Nations
sanctions, but it has been loath to push the North too hard. Its
patience with the North may be running out, but even China may have only
limited information about the machinations within the Pyongyang
government.

\textbf{Q.} \emph{Why are relations so bad between North and South
Korea?}

\textbf{A.} After the United States and the Soviet Union divided the
Korean Peninsula at the end of World War II in 1945, they helped install
rival governments in Seoul and Pyongyang. Each asserted claims to the
whole of Korea. The two fought the 1950-53 Korean War, which ended not
in a peace treaty but a truce. Mutual mistrust runs deep, although there
have been intermittent attempts at political reconciliation and economic
cooperation.

\textbf{Q.} \emph{How did the North get nuclear weapons?}

\textbf{A.} The project started under Kim Il-sung, the country's founder
and the grandfather of the current leader. Mr. Kim knew that Gen.
Douglas MacArthur wanted Washington to allow the use of nuclear weapons
against Chinese and North Korean troops during the Korean War.

By the 1980s, American intelligence satellites were watching the nuclear
complex at Yongbyon come together. Relations between the United States
and the North grew especially tense over the issue in 1994, and some in
the White House feared a war could break out. A pact was eventually
hammered out that year, the Agreed Framework, but it fell apart in 2002,
during the George W. Bush administration, partly over allegations the
North was cheating on its agreements and developing another path to a
bomb. In 2006, the North conducted its first nuclear test, a partial
fizzle. But the subsequent tests were more successful.

Advertisement

\protect\hyperlink{after-bottom}{Continue reading the main story}

\hypertarget{site-index}{%
\subsection{Site Index}\label{site-index}}

\hypertarget{site-information-navigation}{%
\subsection{Site Information
Navigation}\label{site-information-navigation}}

\begin{itemize}
\tightlist
\item
  \href{https://help.nytimes.com/hc/en-us/articles/115014792127-Copyright-notice}{©~2020~The
  New York Times Company}
\end{itemize}

\begin{itemize}
\tightlist
\item
  \href{https://www.nytco.com/}{NYTCo}
\item
  \href{https://help.nytimes.com/hc/en-us/articles/115015385887-Contact-Us}{Contact
  Us}
\item
  \href{https://www.nytco.com/careers/}{Work with us}
\item
  \href{https://nytmediakit.com/}{Advertise}
\item
  \href{http://www.tbrandstudio.com/}{T Brand Studio}
\item
  \href{https://www.nytimes.com/privacy/cookie-policy\#how-do-i-manage-trackers}{Your
  Ad Choices}
\item
  \href{https://www.nytimes.com/privacy}{Privacy}
\item
  \href{https://help.nytimes.com/hc/en-us/articles/115014893428-Terms-of-service}{Terms
  of Service}
\item
  \href{https://help.nytimes.com/hc/en-us/articles/115014893968-Terms-of-sale}{Terms
  of Sale}
\item
  \href{https://spiderbites.nytimes.com}{Site Map}
\item
  \href{https://help.nytimes.com/hc/en-us}{Help}
\item
  \href{https://www.nytimes.com/subscription?campaignId=37WXW}{Subscriptions}
\end{itemize}
