Sections

SEARCH

\protect\hyperlink{site-content}{Skip to
content}\protect\hyperlink{site-index}{Skip to site index}

\href{https://www.nytimes.com/section/world/europe}{Europe}

\href{https://myaccount.nytimes.com/auth/login?response_type=cookie\&client_id=vi}{}

\href{https://www.nytimes.com/section/todayspaper}{Today's Paper}

\href{/section/world/europe}{Europe}\textbar{}British Labour Leader
Offers Compromise on Trident Program

\url{https://nyti.ms/1SXf1P6}

\begin{itemize}
\item
\item
\item
\item
\item
\end{itemize}

Advertisement

\protect\hyperlink{after-top}{Continue reading the main story}

Supported by

\protect\hyperlink{after-sponsor}{Continue reading the main story}

\hypertarget{british-labour-leader-offers-compromise-on-trident-program}{%
\section{British Labour Leader Offers Compromise on Trident
Program}\label{british-labour-leader-offers-compromise-on-trident-program}}

\includegraphics{https://static01.nyt.com/images/2016/01/18/world/SUB-BRITAIN/SUB-BRITAIN-articleLarge.jpg?quality=75\&auto=webp\&disable=upscale}

By \href{http://www.nytimes.com/by/stephen-castle}{Stephen Castle}

\begin{itemize}
\item
  Jan. 17, 2016
\item
  \begin{itemize}
  \item
  \item
  \item
  \item
  \item
  \end{itemize}
\end{itemize}

LONDON --- Stirring a divisive internal debate over defense, Jeremy
Corbyn, the leader of Britain's opposition Labour Party, suggested on
Sunday that he might support the continued existence of the country's
Trident submarine fleet if it were sent on patrol without carrying
nuclear warheads.

Mr. Corbyn, who was
\href{http://www.nytimes.com/2015/09/13/world/europe/labour-party-election-jeremy-corbyn.html}{elected
as the party's leader} last year, is trying to shift Labour leftward on
a range of economic issues, such as opposition to inequality and
government spending cuts, but defense has become central to his efforts
to reshape the party.

As a lifelong opponent of
\href{http://topics.nytimes.com/top/news/science/topics/atomic_weapons/index.html?inline=nyt-classifier}{nuclear
weapons}, Mr. Corbyn has opposed Labour's support for the Trident
submarine system, and last year he said that, if elected prime minister,
he could never order the use of such weapons.

While many of the party members who elected him their leader back that
stance, Mr. Corbyn faces fierce opposition from some Labour lawmakers,
and tensions have surfaced ahead of a parliamentary vote, likely in the
spring, on the Conservative government's plan to renew the Trident
program.

\href{http://www.bbc.co.uk/programmes/p03fr07z}{Speaking on the BBC},
Mr. Corbyn argued that even Prime Minister David Cameron would be
unlikely to order the use of Trident missiles, and when asked about the
point of keeping submarines on patrol, Mr. Corbyn replied, ``They don't
have to have warheads on them.''

He also described the protection of employment in the defense sector as
a priority, suggesting that his position was designed at least partly to
allay concerns among union leaders who argue that the cancellation of
Trident would cost many jobs.

Trident is a highly sensitive issue for Labour. This month, Mr. Corbyn
\href{http://www.nytimes.com/2016/01/07/world/europe/jeremy-corbyn-british-labour-leader-finishes-shadow-cabinet-reorganization.html}{removed
his shadow cabinet's defense secretary}, Maria Eagle, a supporter of
nuclear deterrent. Her successor as spokeswoman on defense issues, Emily
Thornberry, confirmed on Sunday that one option being considered was to
have the capability to deploy nuclear weapons without routinely doing
so, a stance she likened to that of Japan.

For Mr. Corbyn's internal opponents, the issue is totemic because, while
out of power in the 1980s, Labour shifted away from a unilateralist
position on nuclear disarmament as part of a change championed first by
Neil Kinnock and later by Tony Blair.

In recent years the debate has evolved somewhat, across the political
spectrum. The Scottish National Party
\href{http://www.nytimes.com/2015/05/09/world/europe/david-cameron-and-conservatives-emerge-victorious-in-british-election.html}{won
a landslide in Scotland} in last year's general elections on a platform
that included scrapping the Trident program. Some military figures have
also argued that, in an era of strained budgets, Britain could be better
off spending its scarce resources on conventional capabilities.

In response to Mr. Corbyn's comments, Michael Fallon, the defense
secretary, described the Labour Party as a ``threat to our national
security.'' But Bill Kidd, a member of the Scottish Parliament from the
Scottish National Party, argued that ``keeping the capability to launch
nuclear weapons, and therefore the ability to cause catastrophic and
unimaginable destruction, is not a suitable solution, and Trident should
be scrapped altogether.''

In his BBC interview, Mr. Corbyn also said a channel of communication to
Islamic State militants should be created, and cited the secret contacts
between the British government and the Irish Republican Army during the
decadeslong conflict in Northern Ireland.

``There has to be a route through somewhere,'' he said, adding that some
senior Islamic State commanders were former officers in the Iraqi Army,
and that ``there has to be some understanding of where their strong
points are, where their weak points are.''

Mr. Corbyn also said there should be a ``discussion'' with Argentina
about the future of the Falkland Islands, and on domestic issues,
suggested a repeal of laws outlawing labor action by trade unions in
sympathy with other workers.

Advertisement

\protect\hyperlink{after-bottom}{Continue reading the main story}

\hypertarget{site-index}{%
\subsection{Site Index}\label{site-index}}

\hypertarget{site-information-navigation}{%
\subsection{Site Information
Navigation}\label{site-information-navigation}}

\begin{itemize}
\tightlist
\item
  \href{https://help.nytimes.com/hc/en-us/articles/115014792127-Copyright-notice}{©~2020~The
  New York Times Company}
\end{itemize}

\begin{itemize}
\tightlist
\item
  \href{https://www.nytco.com/}{NYTCo}
\item
  \href{https://help.nytimes.com/hc/en-us/articles/115015385887-Contact-Us}{Contact
  Us}
\item
  \href{https://www.nytco.com/careers/}{Work with us}
\item
  \href{https://nytmediakit.com/}{Advertise}
\item
  \href{http://www.tbrandstudio.com/}{T Brand Studio}
\item
  \href{https://www.nytimes.com/privacy/cookie-policy\#how-do-i-manage-trackers}{Your
  Ad Choices}
\item
  \href{https://www.nytimes.com/privacy}{Privacy}
\item
  \href{https://help.nytimes.com/hc/en-us/articles/115014893428-Terms-of-service}{Terms
  of Service}
\item
  \href{https://help.nytimes.com/hc/en-us/articles/115014893968-Terms-of-sale}{Terms
  of Sale}
\item
  \href{https://spiderbites.nytimes.com}{Site Map}
\item
  \href{https://help.nytimes.com/hc/en-us}{Help}
\item
  \href{https://www.nytimes.com/subscription?campaignId=37WXW}{Subscriptions}
\end{itemize}
