Sections

SEARCH

\protect\hyperlink{site-content}{Skip to
content}\protect\hyperlink{site-index}{Skip to site index}

\href{https://www.nytimes.com/section/world/asia}{Asia Pacific}

\href{https://myaccount.nytimes.com/auth/login?response_type=cookie\&client_id=vi}{}

\href{https://www.nytimes.com/section/todayspaper}{Today's Paper}

\href{/section/world/asia}{Asia Pacific}\textbar{}Reporting on Life,
Death and Corruption in Southeast Asia

\url{https://nyti.ms/1oVP5rQ}

\begin{itemize}
\item
\item
\item
\item
\item
\end{itemize}

Advertisement

\protect\hyperlink{after-top}{Continue reading the main story}

Supported by

\protect\hyperlink{after-sponsor}{Continue reading the main story}

Reporter's Notebook

\hypertarget{reporting-on-life-death-and-corruption-in-southeast-asia}{%
\section{Reporting on Life, Death and Corruption in Southeast
Asia}\label{reporting-on-life-death-and-corruption-in-southeast-asia}}

\includegraphics{https://static01.nyt.com/images/2016/02/21/world/alt-fuller/alt-fuller-articleLarge.jpg?quality=75\&auto=webp\&disable=upscale}

By \href{https://www.nytimes.com/by/thomas-fuller}{Thomas Fuller}

\begin{itemize}
\item
  Feb. 21, 2016
\item
  \begin{itemize}
  \item
  \item
  \item
  \item
  \item
  \end{itemize}
\end{itemize}

BANGKOK --- The protesters built what looked like medieval ramparts
topped with sharpened wooden stakes in the heart of Bangkok. The
military was preparing to sweep them out.

As the sun was setting, I spotted Maj. Gen. Khattiya Sawatdiphol, a
renegade who had defected to the protesters, and asked him what he would
do next.

His ``people's army'' would not back down, he said. ``The military
cannot get in here.''

Then came a loud crack, the sound of a sniper's bullet breaking the
sound barrier. General Khattiya
\href{http://www.nytimes.com/2010/05/14/world/asia/14thai.html}{collapsed
at my feet}.

One blink earlier he was answering my questions. Now he was slumped on
the ground, his vacant eyes still open, as blood spilled onto his
camouflage uniform. The world around me went into slow motion as I
watched the general being dragged away by his supporters.

I have covered life and death in Southeast Asia for the past decade, a
job that has entailed puzzling over a missing Malaysian plane one day
(two years later, it's still missing) and interviewing former C.I.A.
mercenaries who were being hunted by the government in the jungles of
\href{http://topics.nytimes.com/top/news/international/countriesandterritories/laos/index.html?inline=nyt-geo}{Laos}
another. I seemed to spend almost as much time dodging the authorities
as interviewing them.

\includegraphics{https://static01.nyt.com/images/2016/02/13/world/fuller-web2/fuller-web2-articleLarge.jpg?quality=75\&auto=webp\&disable=upscale}

The bullet that felled General Khattiya in 2010 missed my head by
inches.

It is hard to speak collectively about a region of so many different
languages, ethnicities, religions and political traditions. But as I
start a new assignment in a part of the world that may as well be a
different cosmos --- Northern California --- I have been trying to make
sense of what I have seen in Southeast Asia.

I come back to one theme again and again: impunity.

In the killing of General Khattiya, who never regained consciousness and
died several days later, a report by an independent body concluded that
the assassin had probably fired from a building controlled by the
military.

Yet no one has ever been charged. The general who helped lead the deadly
military crackdown that ensued, killing 58 civilians, is now
\href{http://topics.nytimes.com/top/news/international/countriesandterritories/thailand/index.html?inline=nyt-geo}{Thailand}'s
prime minister.

``Unfortunately, some people died,'' said the prime minister at the
time, Abhisit Vejjajiva. A murder case against him was dismissed.

It is often no secret who is committing abuses in Southeast Asia,
whether they are illegally cutting down forests, trafficking drugs,
skimming a percentage from government transactions or shooting
protesters.

Unusual wealth, the euphemism for suspected graft, is everywhere.

The general now running Thailand, Prayuth Chan-ocha, is a career soldier
from a modest background. Yet he
\href{http://www.nytimes.com/2015/02/10/world/asia/thailand-junta-drowning-the-opposition-in-paperwork.html}{declared
a net worth of \$4 million}, nearly half of it in cash, soon after
seizing power in a coup two years ago. (In an odd remnant of the
country's democratic past, the members of the junta were required to
declare their assets.)

Image

A Thai antigovernment protester taking cover in Bangkok in 2010. The
general who helped lead the crackdown is now the prime
minister.Credit...Agnes Dherbeys for The New York Times

He has never explained how he amassed this tidy sum on his annual army
salary of \$40,000. ``Do not judge people based on your perceptions,''
he said in a television address after he and other top-ranking army
officers and police officers revealed their fortunes.

Even in countries with tight controls on the news media, like Vietnam or
\href{http://topics.nytimes.com/top/news/international/countriesandterritories/malaysia/index.html?inline=nyt-geo}{Malaysia},
there are brave journalists and armies of bloggers and Facebook
commenters who try to expose wrongdoing. But the problem in Southeast
Asia seems not so much exposing the truth as doing anything about it.

Watching the rise of Asia during my time here, I have wondered whether
there can be continued prosperity without justice. Can societies so
thoroughly riddled with corruption carry through with the remarkable
economic advances made over recent decades?

To see wrongdoing here, sometimes all you have to do is knock. Across
the Mekong River, in Laos, at the edge of a forest, I found the walled
compound of Vixay Keosavang, a Laotian businessman who has been
described as the Pablo Escobar of wildlife trafficking.

After I banged on the compound's heavy metal gate, a security guard
rolled it open. Yes, the guard said, there were live tigers, bears and
many other endangered species inside. Neighbors said trucks regularly
left Mr. Vixay's compound loaded with lizards and pangolins, an
anteater-like animal that is rapidly disappearing because it is eaten
for supposed medicinal qualities.

Mr. Vixay had been so nonchalant in his trafficking business that he
used commercial courier services to send rhino horns and ivory tusks
directly to his company's office in Laos.

Image

General Khattiya, a renegade officer who had allied himself with
protesters, was shot minutes after he was photographed talking with
supporters in Bangkok.Credit...Thomas Fuller/The New York Times

Prompted by
\href{http://www.nytimes.com/2013/03/04/world/asia/notorious-figure-in-animal-smuggling-beyond-reach-in-laos.html}{my
article}, the United States State Department offered a reward of \$1
million for information leading to the dismantling of Mr. Vixay's
business, the first such reward of its kind.

No one has come forward to claim it. Mr. Vixay has never been charged.
The Laotian authorities say they have no evidence against him.

After telling me about the animals inside, the guard called Mr. Vixay on
a cellphone and handed it to my interpreter.

``There's nothing there,'' Mr. Vixay said. ``Who told you about it?''

Laos, ruled by an authoritarian Communist party, has also constructed a
wall of silence over
\href{http://www.nytimes.com/2013/01/11/world/asia/with-laos-disappearance-signs-of-a-liberalization-in-backslide.html}{the
disappearance of Sombath Somphone}, a civic leader who had called for
more public participation and decision-making in society. Security
cameras showed him being stopped at a police checkpoint and led away in
December 2012. Yet the government has repeatedly said it has no
information on his whereabouts.

The authorities in Southeast Asia have access to many of the same tools
as their counterparts in wealthier countries. What seems to be lacking
is not technology but political will to investigate powerfully connected
people. Tony Pua, an opposition leader in Malaysia, calls it a culture
of ``forget it and move on.''

When a boat filled with refugees from
\href{http://topics.nytimes.com/top/news/international/countriesandterritories/myanmar/index.html?inline=nyt-geo}{Myanmar}
was abandoned by its crew, adrift in the Andaman Sea without adequate
food or fuel last May, I obtained the number of someone on board and
asked the phone company to track the phone's location.

Image

Rohingya migrants passing food supplies dropped by a Thai Army
helicopter to others on a boat drifting in Thai waters last May off the
southern island of Koh Lipe in the Andaman Sea.Credit...Christophe
Archambault/Agence France-Presse --- Getty Images

The phone company balked, so I contacted a friendly naval officer, Lt.
Cmdr. Veerapong Nakprasit, who persuaded the company to give me the
phone's location on humanitarian grounds. The navy, aware that the
refugees could die without help, presumably could have made the request
and found the boat on its own.

We rented a speedboat and followed the coordinates until
\href{http://www.nytimes.com/2015/05/15/world/asia/burmese-rohingya-bangladeshi-migrants-andaman-sea.html}{we
found the stranded boat}. Upon seeing us, several hundred rail-thin
refugees, many of them women and children, called out for help. I
dictated a story by phone to the newsroom in Hong Kong, and soon readers
around the world were aware of the refugees' plight. We had brought
bottles of water, and we tossed them to the grateful passengers.

That evening, out of sight of journalists, the Thai Navy pushed the boat
back out into the open sea.

The refugee crisis in Southeast Asia last year spiraled into a regional
embarrassment that forced governments to admit that their own officials
were complicit in trafficking desperate migrants from Myanmar. Yet in
Thailand, amid a supposed crackdown on trafficking by the military
junta, the head of the investigation fled to Australia and applied for
political asylum, saying he had been threatened by powerful people.

The Thai junta has not set a firm timetable for leaving power, but its
members are taking no chances.

Soon after the May 2014 coup, they issued a decree that put them above
the law for ``all acts,'' including the seizure of power and any
``punishments'' they meted out.

The last words of the Constitution they wrote for themselves call for
blanket immunity. The junta members are ``entirely discharged'' for
their acts.

Lawyers representing the victims of the crackdown in 2010 say they see
little hope for justice now that the military is in power.

One of the key witnesses in the crackdown, Nattatida Meewangpla, is a
paramedic who says she saw six people shot by soldiers.

For the past year, she has been held in detention on the orders of a
military court, charged with participating in a social media chat group
that opposed the military takeover. Her lawyers say the military is
trying to silence her.

``People were chased and killed,'' she wrote to me from prison last
month. ``I am the only witness still breathing.''

My decade here has been a time of intense ambivalence. I was enchanted
by people's warmth, congeniality and politeness. When I interviewed
protesters on torrid summer days, they would often fan my face as we
spoke. I learned from my Thai friends how to laugh away life's
disappointments and annoyances. I relished the food and marveled at the
hospitality.

But I despaired at the venality of the elites and the corruption that
engulfed the lives of so many people I interviewed. I came to see
Southeast Asia as a land of great people and bad governments, of
remarkable graciousness but distressing levels of impunity.

Advertisement

\protect\hyperlink{after-bottom}{Continue reading the main story}

\hypertarget{site-index}{%
\subsection{Site Index}\label{site-index}}

\hypertarget{site-information-navigation}{%
\subsection{Site Information
Navigation}\label{site-information-navigation}}

\begin{itemize}
\tightlist
\item
  \href{https://help.nytimes.com/hc/en-us/articles/115014792127-Copyright-notice}{©~2020~The
  New York Times Company}
\end{itemize}

\begin{itemize}
\tightlist
\item
  \href{https://www.nytco.com/}{NYTCo}
\item
  \href{https://help.nytimes.com/hc/en-us/articles/115015385887-Contact-Us}{Contact
  Us}
\item
  \href{https://www.nytco.com/careers/}{Work with us}
\item
  \href{https://nytmediakit.com/}{Advertise}
\item
  \href{http://www.tbrandstudio.com/}{T Brand Studio}
\item
  \href{https://www.nytimes.com/privacy/cookie-policy\#how-do-i-manage-trackers}{Your
  Ad Choices}
\item
  \href{https://www.nytimes.com/privacy}{Privacy}
\item
  \href{https://help.nytimes.com/hc/en-us/articles/115014893428-Terms-of-service}{Terms
  of Service}
\item
  \href{https://help.nytimes.com/hc/en-us/articles/115014893968-Terms-of-sale}{Terms
  of Sale}
\item
  \href{https://spiderbites.nytimes.com}{Site Map}
\item
  \href{https://help.nytimes.com/hc/en-us}{Help}
\item
  \href{https://www.nytimes.com/subscription?campaignId=37WXW}{Subscriptions}
\end{itemize}
