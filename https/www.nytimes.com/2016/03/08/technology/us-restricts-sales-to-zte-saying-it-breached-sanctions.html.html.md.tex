Sections

SEARCH

\protect\hyperlink{site-content}{Skip to
content}\protect\hyperlink{site-index}{Skip to site index}

\href{https://www.nytimes.com/section/technology}{Technology}

\href{https://myaccount.nytimes.com/auth/login?response_type=cookie\&client_id=vi}{}

\href{https://www.nytimes.com/section/todayspaper}{Today's Paper}

\href{/section/technology}{Technology}\textbar{}U.S. Restricts Sales to
ZTE, Saying It Breached Sanctions

\url{https://nyti.ms/1R1unx9}

\begin{itemize}
\item
\item
\item
\item
\item
\end{itemize}

Advertisement

\protect\hyperlink{after-top}{Continue reading the main story}

Supported by

\protect\hyperlink{after-sponsor}{Continue reading the main story}

\hypertarget{us-restricts-sales-to-zte-saying-it-breached-sanctions}{%
\section{U.S. Restricts Sales to ZTE, Saying It Breached
Sanctions}\label{us-restricts-sales-to-zte-saying-it-breached-sanctions}}

\includegraphics{https://static01.nyt.com/images/2016/03/08/business/08zte/08zte-articleLarge.jpg?quality=75\&auto=webp\&disable=upscale}

By \href{https://www.nytimes.com/by/paul-mozur}{Paul Mozur}

\begin{itemize}
\item
  March 7, 2016
\item
  \begin{itemize}
  \item
  \item
  \item
  \item
  \item
  \end{itemize}
\end{itemize}

HONG KONG --- ZTE is one of China's few truly international electronics
firms. Yet American companies will now need special permission to sell
to it.

The company, which makes smartphones, was found to have violated
American sanctions against Iran by selling United States-made goods to
the country, according to a
\href{https://s3.amazonaws.com/public-inspection.federalregister.gov/2016-05104.pdf}{Commerce
Department statement} on Monday. As a result, ZTE will be blocked from
buying any technology from American companies without a special license.

ZTE planned to ``illicitly re-export controlled items to Iran in
violation of U.S export laws,'' the Commerce Department said. The
sanctions against Iran, many of which were
\href{http://www.nytimes.com/2016/03/08/world/middleeast/embargo-lifted-iranian-oil-reaches-europe.html}{recently
lifted}, were intended to restrict Iran's nuclear work.

The export controls against ZTE are unusual because such actions are
rarely taken against such large companies. The action underscores how
important the push is by the United States to gain China's cooperation
in embargoes intended to combat nuclear proliferation.

The export controls are also risky because they could easily prompt a
backlash from Beijing. Technology has become a sticking point in
Chinese-American relations. Washington has accused Chinese
government-sponsored hackers of stealing American trade secrets.

On Tuesday, the Commerce Ministry of China criticized the American
restrictions. In a
\href{http://www.mofcom.gov.cn/article/ae/ag/201603/20160301270246.shtml}{statement
posted on its website}, the ministry said: ``The U.S. move will severely
affect normal operations of Chinese companies. China will continue
negotiating with the U.S. side on this issue.''

Recent scrutiny in the United States has also scuttled some Chinese
investments in American tech companies. ZTE's much larger domestic
competitor, Huawei, is effectively banned from selling its telecom
network equipment in the United States.

Beijing has fought back by increasing scrutiny of American companies'
operations in China. It
\href{http://www.nytimes.com/2015/02/10/business/international/qualcomm-fine-china-antitrust-investigation.html}{fined
Qualcomm} for antitrust violations and raided Microsoft's offices as
part of a continuing investigation.

Chinese state news media has complained bitterly about revelations from
the former National Security Agency contractor Edward J. Snowden about
American spying and has called for a domestic purge of United States
technology.

Although analysts said that the export controls against ZTE were most
likely aimed at nuclear proliferation rather than being a new jab in
heightened technology trade tensions, China's interpretation of the
action was an open question.

``Depending on how both sides read it, this could be a specific case, or
it could get overheated and extended,'' said Scott Kennedy, a scholar at
the Center for Strategic and International Studies, a nonprofit research
group.

Trading of ZTE's shares was suspended Monday before the announcement.
News of the export controls was first reported by Reuters.

ZTE
\href{http://wwwen.zte.com.cn/en/press_center/news/201603/t20160308_448866.html}{said
in a statement} Tuesday morning that it was ``fully committed to
compliance with the laws and regulations in the jurisdictions in which
it operates. ZTE has been cooperating, will continue to cooperate and
communicate with all U.S. agencies as required.''

ZTE's status within China is likely to make the export controls big news
there. Though not well known in the United States, ZTE is an
international champion of the Chinese high-tech industry, with a market
capitalization of around \$10 billion. After China's first lady, Peng
Liyuan, aroused online criticism by using an Apple iPhone during a 2013
trip to Mexico, she switched to a ZTE phone for a public trip in 2014.

In Monday's statement, the Commerce Department
\href{http://www.bis.doc.gov/index.php/about-bis/newsroom}{provided} two
internal ZTE documents to back up the claims that the company was
violating sanctions. One, from 2011, signed by several senior ZTE
executives, discussed the risks of United States export controls and
noted that ZTE had ``ongoing projects in all five major embargoed
countries --- Iran, Sudan, North Korea, Syria and Cuba.'' It also said
that the Iran project was the ``biggest risk.''

In the other document, ZTE mapped out the way it could circumvent
American export controls in a complex flow chart, including using a
``shell'' company structure.

The new export controls are likely to make business difficult for ZTE.
Though the company sells its own branded smartphones and telecom
infrastructure equipment, it buys components from American tech
companies, using, for example,
\href{https://www.qualcomm.com/products/snapdragon/zte}{Qualcomm chips}
in some of its phones. Given the complexity of the electronics supply
chain and the mass production of specific devices, it will probably
prove costly for ZTE to shuffle the design and sourcing for its
products.

Daniel H. Rosen, a partner at the research firm Rhodium Group, said that
given ZTE's behavior, it would have ``required an extraordinary degree
of confidence building'' between the United States and China to avoid
the current situation.

``That does not appear to have taken place,'' he said.

Advertisement

\protect\hyperlink{after-bottom}{Continue reading the main story}

\hypertarget{site-index}{%
\subsection{Site Index}\label{site-index}}

\hypertarget{site-information-navigation}{%
\subsection{Site Information
Navigation}\label{site-information-navigation}}

\begin{itemize}
\tightlist
\item
  \href{https://help.nytimes.com/hc/en-us/articles/115014792127-Copyright-notice}{©~2020~The
  New York Times Company}
\end{itemize}

\begin{itemize}
\tightlist
\item
  \href{https://www.nytco.com/}{NYTCo}
\item
  \href{https://help.nytimes.com/hc/en-us/articles/115015385887-Contact-Us}{Contact
  Us}
\item
  \href{https://www.nytco.com/careers/}{Work with us}
\item
  \href{https://nytmediakit.com/}{Advertise}
\item
  \href{http://www.tbrandstudio.com/}{T Brand Studio}
\item
  \href{https://www.nytimes.com/privacy/cookie-policy\#how-do-i-manage-trackers}{Your
  Ad Choices}
\item
  \href{https://www.nytimes.com/privacy}{Privacy}
\item
  \href{https://help.nytimes.com/hc/en-us/articles/115014893428-Terms-of-service}{Terms
  of Service}
\item
  \href{https://help.nytimes.com/hc/en-us/articles/115014893968-Terms-of-sale}{Terms
  of Sale}
\item
  \href{https://spiderbites.nytimes.com}{Site Map}
\item
  \href{https://help.nytimes.com/hc/en-us}{Help}
\item
  \href{https://www.nytimes.com/subscription?campaignId=37WXW}{Subscriptions}
\end{itemize}
