Sections

SEARCH

\protect\hyperlink{site-content}{Skip to
content}\protect\hyperlink{site-index}{Skip to site index}

\href{https://www.nytimes.com/section/movies}{Movies}

\href{https://myaccount.nytimes.com/auth/login?response_type=cookie\&client_id=vi}{}

\href{https://www.nytimes.com/section/todayspaper}{Today's Paper}

\href{/section/movies}{Movies}\textbar{}Review: `Hello, My Name Is
Doris,' About an Older Woman's Love for a Much Younger Man

\href{https://nyti.ms/226bTDw}{https://nyti.ms/226bTDw}

\begin{itemize}
\item
\item
\item
\item
\item
\end{itemize}

Advertisement

\protect\hyperlink{after-top}{Continue reading the main story}

Supported by

\protect\hyperlink{after-sponsor}{Continue reading the main story}

\hypertarget{review-hello-my-name-is-doris-about-an-older-womans-love-for-a-much-younger-man}{%
\section{Review: `Hello, My Name Is Doris,' About an Older Woman's Love
for a Much Younger
Man}\label{review-hello-my-name-is-doris-about-an-older-womans-love-for-a-much-younger-man}}

\includegraphics{https://static01.nyt.com/images/2016/03/11/movies/11HELLO/11HELLO-articleLarge.jpg?quality=75\&auto=webp\&disable=upscale}

\begin{itemize}
\tightlist
\item
  Hello, My Name Is Doris\\
  Directed by Michael Showalter Comedy, Drama, Romance R 1h 30m
\end{itemize}

By \href{https://www.nytimes.com/by/manohla-dargis}{Manohla Dargis}

\begin{itemize}
\item
  March 10, 2016
\item
  \begin{itemize}
  \item
  \item
  \item
  \item
  \item
  \end{itemize}
\end{itemize}

The first time that the heroine in the disarming comedy ``Hello, My Name
Is Doris'' sees the kid, they're in a crowded office elevator. He's not
a child at all, but somewhere in his mid-30s, which can seem light-years
away for a woman who has been
\href{http://www.nytimes.com/1995/07/02/magazine/in-language-a-woman-of-a-certain-age.html}{of
a certain age} for decades. So when he jostles Doris (Sally Field), she
braces for the usual morning-elevator scrum. Instead, he straightens her
lopsided eyeglasses. With this one small, human kindness he does
something that astonishes Doris, something that doesn't often happen to
the world's invisible women: He sees her.

Not that Doris, who's in her 60s, tries to be invisible, exactly. From
her cat-eye glasses to the headscarves that make her look hastily
regifted, she seems like someone yearning to be seen. But wrinkles have
a way of making women disappear one crease at a time, and Doris, who's
in mourning when the movie opens, has done her part to vanish. When this
kid --- he turns out to be a new co-worker, John (Max Greenfield, an
effortless charmer) --- notices Doris, it changes everything. Doris is
more than just surprised by his attention, she is also transformed. He
makes her visible, most importantly to herself, a revelation that turns
Doris into a woman who desires and is desired in turn. It's a ferocious
awakening.

It's also a fairly slow, gaudy bloom. Like his heroine, the director
Michael Showalter eagerly oversells the goods. Right off, he throws in a
lot that's hard on the eyes and ears. There's Doris's dowager-dumpster
wardrobe and topsy-turvy Staten Island house, along with her mutterings
and facial contortions, which seem one tic away from a medical
diagnosis. As he puts these messy parts into play, I kept thinking no,
no, no, no. It's all much too much (those shoes, that hair!) and
together they announce that you're in for an ingratiatingly cutesy slog
about a lovable kook --- except that the movie and Doris aren't easy to
love, which is partly why they work.

The movie starts big and broad at the funeral for Doris's mother, where
everyone is shedding stage tears. Seated pointedly alone, Doris looks
ashen enough that she wouldn't be out of place next to Mom in the open
coffin. The whole thing is as dire as the priest's eulogy, but a few
beats later, when Doris's brother, Todd (the great character actor
Stephen Root), speaks to her with unforced feeling, the movie shifts
into something more complex. Mr. Showalter continues to play with comic
tone and mood --- he folds in some slapstick, enables the mugging and
stages several cringing fantasy sequences that encourage you to laugh at
Doris. It feels cruel, specifically because she isn't an ordinary
American movie protagonist: She's an older, frumpy, lower-middle-class
woman who works in a cubicle.

Indirection can be a beautiful tool in comedy and so it is in ``Hello,
My Name Is Doris,'' which uses this funny, outwardly ridiculous
character to tell a simple story about a love that rarely speaks its
name, including in movies: that of an older woman for a much younger
man. These kinds of screen stories have always been few and far between,
and it's instructive, given the prevailing cultural horror of aging,
that some of the more memorable ones turn on her-or-his pathology,
whether it's the deranged actress in
``\href{http://www.tcm.com/this-month/article/357054\%7C373028/Sunset-Blvd-.html}{Sunset
Boulevard}'' or the traumatized boy-man in the cult film
``\href{https://www.youtube.com/watch?v=5mz3TkxJhPc}{Harold and
Maude}.'' Doris has issues, mostly grief and social isolation, which Ms.
Field makes movingly real with a performance that reveals its truth
incrementally.

Less interestingly, Doris has also become a hoarder. The home that she
shared with her long-term invalid mother has become --- with its
bric-a-brac, bounty of old shampoo bottles and stacks of magazines --- a
showcase for their obsessive, compulsive behavior and an overly obvious
manifestation of Doris's struggles. Mr. Showalter doesn't hang out in
the house much, but the pack-rat motif allows him and his co-writer,
Laura Terruso, to pad the story with some family storming and stressing.
This mostly concerns Todd and his wife, Cindy (Wendi McLendon-Covey),
urging Doris to declutter and insisting that she visit a therapist
(Elizabeth Reaser).

Doris's burgeoning friendship with John, meanwhile, leads her down
alternately goofy and sweet avenues involving online stalking,
late-night clubbing and new alliances, including with his girlfriend,
Brooklyn (Beth Behrs). Doris's female attachments, including with her
best friend, Roz (Tyne Daly), are particularly appealing because --- as
with John's first kindness --- they're reminders that you can tell a lot
about people from how they are loved. To that end, in one funny,
meaningful passage, Doris ends up at a concert with John, wearing an
eye-popping yellow outfit, a look that draws compliments and attention
from various young revelers, who, treating her as a kindred spirit, are
understandably taken with her. Doris turns out to be an excellent
mirror, including for your own chauvinism.

\emph{``Hello, My Name Is Doris'' is rated R (under 17 requires
accompanying parent or adult guardian) for some language, none of which
should shock anybody. Running time: 1 hour 30 minutes.}

Advertisement

\protect\hyperlink{after-bottom}{Continue reading the main story}

\hypertarget{site-index}{%
\subsection{Site Index}\label{site-index}}

\hypertarget{site-information-navigation}{%
\subsection{Site Information
Navigation}\label{site-information-navigation}}

\begin{itemize}
\tightlist
\item
  \href{https://help.nytimes.com/hc/en-us/articles/115014792127-Copyright-notice}{©~2020~The
  New York Times Company}
\end{itemize}

\begin{itemize}
\tightlist
\item
  \href{https://www.nytco.com/}{NYTCo}
\item
  \href{https://help.nytimes.com/hc/en-us/articles/115015385887-Contact-Us}{Contact
  Us}
\item
  \href{https://www.nytco.com/careers/}{Work with us}
\item
  \href{https://nytmediakit.com/}{Advertise}
\item
  \href{http://www.tbrandstudio.com/}{T Brand Studio}
\item
  \href{https://www.nytimes.com/privacy/cookie-policy\#how-do-i-manage-trackers}{Your
  Ad Choices}
\item
  \href{https://www.nytimes.com/privacy}{Privacy}
\item
  \href{https://help.nytimes.com/hc/en-us/articles/115014893428-Terms-of-service}{Terms
  of Service}
\item
  \href{https://help.nytimes.com/hc/en-us/articles/115014893968-Terms-of-sale}{Terms
  of Sale}
\item
  \href{https://spiderbites.nytimes.com}{Site Map}
\item
  \href{https://help.nytimes.com/hc/en-us}{Help}
\item
  \href{https://www.nytimes.com/subscription?campaignId=37WXW}{Subscriptions}
\end{itemize}
