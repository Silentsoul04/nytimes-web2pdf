Sections

SEARCH

\protect\hyperlink{site-content}{Skip to
content}\protect\hyperlink{site-index}{Skip to site index}

\href{https://www.nytimes.com/section/technology}{Technology}

\href{https://myaccount.nytimes.com/auth/login?response_type=cookie\&client_id=vi}{}

\href{https://www.nytimes.com/section/todayspaper}{Today's Paper}

\href{/section/technology}{Technology}\textbar{}ZTE Document Raises
Questions About Huawei and Sanctions

\url{https://nyti.ms/22rkpgI}

\begin{itemize}
\item
\item
\item
\item
\item
\end{itemize}

Advertisement

\protect\hyperlink{after-top}{Continue reading the main story}

Supported by

\protect\hyperlink{after-sponsor}{Continue reading the main story}

\hypertarget{zte-document-raises-questions-about-huawei-and-sanctions}{%
\section{ZTE Document Raises Questions About Huawei and
Sanctions}\label{zte-document-raises-questions-about-huawei-and-sanctions}}

\includegraphics{https://static01.nyt.com/images/2016/03/19/business/19chinatech-web1/19chinatech-web1-articleLarge.jpg?quality=75\&auto=webp\&disable=upscale}

By \href{https://www.nytimes.com/by/paul-mozur}{Paul Mozur}

\begin{itemize}
\item
  March 18, 2016
\item
  \begin{itemize}
  \item
  \item
  \item
  \item
  \item
  \end{itemize}
\end{itemize}

HONG KONG --- When the United States government punished ZTE of China
this month, saying it had done business with Iran, it released
\href{http://www.bis.doc.gov/index.php/about-bis/newsroom}{internal
company documents} that it said detailed how the electronic equipment
maker had done it --- and that also suggested the problem might not be
limited to one Chinese company.

One document described how ZTE would set up seemingly independent
companies --- called ``cut-off companies'' --- that would sign the deals
in other countries. That could enable it to continue to do business in
Iran, North Korea and other countries placed under American
restrictions.

In describing the effort, the document cited as a model --- and at times
a cautionary tale --- a rival company it called F7. ZTE said F7 had done
something similar, though its business in restricted companies ended up
hurting its American ambitions.

The document does not give F7's real name. But the description offered
by ZTE matches a company far larger and more politically sensitive:
Huawei Technologies, its chief rival and a major force in the technology
world.

The ZTE document, dated August 2011, suggests that other Chinese
companies could have potential exposure to American export limits. Given
\href{http://www.nytimes.com/2016/03/08/technology/us-restricts-sales-to-zte-saying-it-breached-sanctions.html}{the
recent sanctions against ZTE}, it also suggests that the issue could be
a continuing one between Chinese and American government officials.

ZTE on Thursday said that it had
\href{http://www.bbc.com/news/business-35828741}{delayed the release} of
its annual financial results because of the sanctions, which limit the
ability of American companies to sell equipment to it.

ZTE officials declined to comment on the identity of F7, and Huawei
declined to comment. ZTE has said it is cooperating with investigators
and is committed to complying with the law.

The United States Commerce Department, which last week restricted sales
of American telecommunications equipment to ZTE, accusing it of
violating embargoes, did not respond to requests for comment.

It is rare for the Commerce Department to publicly provide evidence for
an addition to its blacklist of restricted companies, especially full
disclosure of internal documents.

It is not clear how accurate ZTE's version of the events might be. The
document says some information about F7 was gathered by ZTE's legal
department, without offering details.

F7, the document says, tried in 2010 to buy an American company called
3Leaf but met with opposition from American officials. That same year,
Huawei
\href{http://dealbook.nytimes.com/2011/02/18/the-big-chill-huawei-imbroglio-puts-countries-at-odds/}{agreed
to buy} major assets from 3Leaf, but it dropped the bid in February 2011
because of opposition from American officials.

F7 also has a joint venture with the American digital security company
Symantec, the 2011 document says. Huawei
\href{http://www.nytimes.com/2012/03/27/technology/symantec-dissolves-alliance-with-huawei-of-china.html}{had
a joint venture with Symantec} before the American company dissolved it
in 2012.

Like ZTE, Huawei makes telecommunications equipment for corporate
networks and for big telecommunications systems such as phone companies.
American officials have long suspected it has Chinese government ties,
and United States intelligence officials
\href{http://www.nytimes.com/2014/03/23/world/asia/nsa-breached-chinese-servers-seen-as-spy-peril.html}{have
tried} to tap into the company's network. Both companies
\href{http://www.nytimes.com/2012/10/09/us/us-panel-calls-huawei-and-zte-national-security-threat.html}{are
effectively barred} from selling equipment for American networks.

Huawei says that it is privately owned and that accusations of
government ties are an excuse to hurt the company for protectionist
purposes.

Huawei is much larger than ZTE. In 2014, it reported revenue of about
\$60 billion, about four times that of ZTE. Depending on the measure, it
ranks with Sweden's Ericsson as the world's largest supplier of the base
stations and other equipment that make mobile telecom networks run.
Huawei equipment supports networks in countries across the world,
including many European markets.

While both Huawei and ZTE are given privileged status as high-tech
innovators by China's leadership, Huawei is more prominent.

Huawei has also had greater success selling its smartphones in America,
and indeed across the world. The company was the third-largest
smartphone vendor by units sold in the fourth quarter of 2015 according
to IDC, with an 8.1 percent share of the global market, compared with
the 21.4 percent share of Samsung, the company in first place.

Despite the trouble in the United States, Huawei has not shied away from
potentially controversial deals. In September, Huawei
\href{http://www.telecompaper.com/news/syria-huawei-ink-ict-deal--1104884}{signed
a deal} with Syria's Communications and Technology Ministry to help the
country develop its communications networks.

The ZTE document details how F7 recruited compliance experts and placed
them in its joint ventures as part of efforts to mitigate its risks. It
says that the company recruited one ``senior export control compliance
specialist from Texas Instruments'' and a ``Chinese-American attorney
who is familiar with the related laws in the U.S.''

It also describes how F7 found partners that it could say were
independent companies and that could work on its behalf in countries
under embargo. F7, it said, found a big information technology company
that was ``serving as its agent to sign contracts for projects in
embargoed countries.''

``This cut-off company's capital credit and capability are relatively
strong compared to our company; it can cut off risks more effectively,''
the document said.

But ZTE came to believe that F7's activities in embargoed countries hurt
its American expansion efforts.

It said it believed that F7's efforts to acquire companies in the United
States were in part blocked because of its ``ongoing projects in
embargoed countries.''

Advertisement

\protect\hyperlink{after-bottom}{Continue reading the main story}

\hypertarget{site-index}{%
\subsection{Site Index}\label{site-index}}

\hypertarget{site-information-navigation}{%
\subsection{Site Information
Navigation}\label{site-information-navigation}}

\begin{itemize}
\tightlist
\item
  \href{https://help.nytimes.com/hc/en-us/articles/115014792127-Copyright-notice}{©~2020~The
  New York Times Company}
\end{itemize}

\begin{itemize}
\tightlist
\item
  \href{https://www.nytco.com/}{NYTCo}
\item
  \href{https://help.nytimes.com/hc/en-us/articles/115015385887-Contact-Us}{Contact
  Us}
\item
  \href{https://www.nytco.com/careers/}{Work with us}
\item
  \href{https://nytmediakit.com/}{Advertise}
\item
  \href{http://www.tbrandstudio.com/}{T Brand Studio}
\item
  \href{https://www.nytimes.com/privacy/cookie-policy\#how-do-i-manage-trackers}{Your
  Ad Choices}
\item
  \href{https://www.nytimes.com/privacy}{Privacy}
\item
  \href{https://help.nytimes.com/hc/en-us/articles/115014893428-Terms-of-service}{Terms
  of Service}
\item
  \href{https://help.nytimes.com/hc/en-us/articles/115014893968-Terms-of-sale}{Terms
  of Sale}
\item
  \href{https://spiderbites.nytimes.com}{Site Map}
\item
  \href{https://help.nytimes.com/hc/en-us}{Help}
\item
  \href{https://www.nytimes.com/subscription?campaignId=37WXW}{Subscriptions}
\end{itemize}
