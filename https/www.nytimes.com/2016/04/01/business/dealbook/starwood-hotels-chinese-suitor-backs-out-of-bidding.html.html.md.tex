Sections

SEARCH

\protect\hyperlink{site-content}{Skip to
content}\protect\hyperlink{site-index}{Skip to site index}

\href{https://myaccount.nytimes.com/auth/login?response_type=cookie\&client_id=vi}{}

\href{https://www.nytimes.com/section/todayspaper}{Today's Paper}

\href{/section/business/dealbook}{DealBook}\textbar{}Starwood Bidding
War Ends Abruptly, Yielding a Merger and a Puzzle

\url{https://nyti.ms/1TlSNr5}

\begin{itemize}
\item
\item
\item
\item
\item
\end{itemize}

Advertisement

\protect\hyperlink{after-top}{Continue reading the main story}

Supported by

\protect\hyperlink{after-sponsor}{Continue reading the main story}

DealBook Business and Policy

\hypertarget{starwood-bidding-war-ends-abruptly-yielding-a-merger-and-a-puzzle}{%
\section{Starwood Bidding War Ends Abruptly, Yielding a Merger and a
Puzzle}\label{starwood-bidding-war-ends-abruptly-yielding-a-merger-and-a-puzzle}}

\includegraphics{https://static01.nyt.com/images/2016/04/01/business/cnbc-starwood1/cnbc-starwood1-videoSixteenByNine1050.png}

By \href{http://www.nytimes.com/by/michael-j-de-la-merced}{Michael J. de
la Merced} and Leslie Picker

\begin{itemize}
\item
  March 31, 2016
\item
  \begin{itemize}
  \item
  \item
  \item
  \item
  \item
  \end{itemize}
\end{itemize}

Ever since a group led by the acquisitive --- and secretive --- Chinese
firm Anbang Insurance Group
\href{http://www.nytimes.com/2016/03/29/business/dealbook/starwood-bidding-war-increases-with-higher-offer.html}{raised
its bid} for Starwood Hotels and Resorts, advisers to the American hotel
company were a little wary that its new suitor might not be able to
follow through.

And then early Thursday morning, Starwood and its advisers began to
learn that Anbang was likely to walk away, just weeks after first
emerging to challenge Marriott International in a highly visible merger
contest.

By Thursday afternoon, Anbang and its partners
\href{http://www.prnewswire.com/news-releases/anbang-jc-flowers-primavera-consortium-determines-not-to-proceed-with-its-proposal-to-acquire-starwood-300244405.html}{formally
withdrew} their \$14 billion takeover offer for Starwood, ceding the
operator of the Westin and Sheraton chains to Marriott in a puzzling
turn of events.

So ends what had been poised to become one of the big merger battles of
2016, as the century-old Marriott faced losing to a consortium whose
leader
\href{http://www.nytimes.com/2016/03/30/business/dealbook/china-anbang-starwood-wu-xiaohui.html}{boasted
close ties} to the Chinese government.

But after being topped twice in bidding by the group --- which included
Anbang, the American private equity firm J.C. Flowers \& Company and
Primavera Capital, an investment firm led by a former chairman of
Goldman Sachs for Asia --- Marriott decided earlier this week not to
immediately raise its latest offer beyond roughly \$13.25 billion,
betting that something would befall the consortium's efforts.

That wager paid off. Combining Starwood with Marriott will create the
biggest hotel company in the world, with more than 5,500 owned or
franchised hotels and 1.1 million rooms.

\includegraphics{https://static01.nyt.com/images/2016/04/01/business/01db-starwood/01db-starwood-articleLarge.jpg?quality=75\&auto=webp\&disable=upscale}

Starwood
\href{http://www.businesswire.com/news/home/20160331006518/en/Starwood-Hotels-Resorts-Announces-Anbang-Consortium-Withdrawn}{said
in a statement} on Thursday that it had still managed to extract more
money for the company's investors and looked forward to combining with
Marriott.

``Throughout this process, we have been focused on maximizing
stockholder value now and in the future,'' Bruce Duncan, Starwood's
chairman, said. ``Our board is confident this transaction offers
superior value for Starwood's stockholders, can close quickly and
provides value-creation potential that will enable both sets of
stockholders to benefit from future financial performance.''

Shares in Starwood fell more than 4 percent in after-hours trading after
Anbang's disclosure, to \$79.92. Shares in Marriott fell 5 percent, to
\$67.61.

What happened to Anbang's takeover effort is unclear. In a statement on
Thursday, the insurer's consortium blamed unspecified ``various market
considerations'' for its need to withdraw.

The abrupt withdrawal of the offer raised new questions, including
whether the Chinese government, which has close ties with Anbang, had
blocked the proposed transaction, or whether the insurer and its fellow
bidders had run into issues with the financing for the deal.

It is a mysterious end to the pursuit by Anbang, a huge insurer that has
risen to prominence in recent years, in part through audacious
deal-making. The Chinese firm, which has assets of more than \$291
billion, became a force in the luxury hotel business in less than two
years after buying the likes of the Waldorf Astoria and the JW Marriott
Essex House.

Its chairman, Wu Xiaohui, had begun to gain a reputation as a Chinese
counterpart to Warren E. Buffett, his wealth compounding rapidly since
founding the insurer in 2004.

Still, the company he oversees has been criticized for its unusually
opaque corporate structure: Thirty-seven interlocking holding companies
control over 93 percent of Anbang's shares, while two government-owned
companies own the remainder.

Had Anbang won Starwood, its deal would have been the biggest takeover
of an American target by a Chinese buyer, according to data from
Dealogic.

Yet from the time Anbang publicly bid for the hotel chain, investors and
analysts questioned whether the Chinese-led group could actually close
on its offer.

Anbang sought to break up Starwood's first deal with Marriott
\href{http://www.nytimes.com/2016/03/15/business/dealbook/starwood-receives-unsolicited-bid-putting-marriott-plan-in-doubt.html}{by
offering \$76} a share in
cash\href{http://www.nytimes.com/2016/03/19/business/dealbook/starwood-says-rivals-counteroffer-tops-bid-from-marriott.html}{,
going up to \$78} a share, a proposal that people briefed on the
discussions said was fully documented, meaning the financing was in
place.

Marriott countered with a
\href{http://www.nytimes.com/2016/03/22/business/dealbook/marriott-raises-bid-for-starwood.html}{new
cash-and-stock} proposal on March 21 valued then at \$79.53 a share,
raising the prospect that the Chinese-led group would come back with an
even higher bid.

Anbang and its
partners\href{http://www.nytimes.com/2016/03/29/business/dealbook/starwood-bidding-war-increases-with-higher-offer.html}{indeed
responded}, by offering \$81 a share and then \$82.75 a share in cash.
But the latest offer, which came last weekend, seemed on shakier grounds
with its financing, according to people briefed on the matter.

Analysts said that Marriott would be hard-pressed to beat that price,
since doing so could hurt its earnings per share.

But Marriott publicly questioned whether Anbang and its partners truly
had the financing needed to close their offer, as well as how long
American government regulators would take to bless the deal.

Another particular concern was whether a government panel focused on the
national security aspects of mergers would require selling off Starwood
properties near sensitive locations. The St. Regis Washington D.C., for
example, is only blocks away from the White House, while a W hotel is
near the Treasury Department.

Starwood and its advisers pressed the consortium for more information
about financing and whether the Chinese government would bless the new
proposal, these people said, requesting anonymity to discuss
confidential negotiations.

People involved in the transaction spoke on the condition of anonymity.

Publicly, Starwood
\href{http://www.businesswire.com/news/home/20160328005392/en/Starwood-Hotels-Resorts-Board-Directors-Determines-Revised}{noted
on March 28} that while its board had determined that Anbang's second
bid was ``reasonably likely to lead to a `superior proposal,''' the new
proposal was nonbinding and that the two sides needed to hammer out
``nonprice terms.''

The hotelier had been waiting several days for a response from the
Chinese-led group when word first began to filter in early Thursday that
Anbang and its partners were preparing to walk away, according to the
people briefed on the matter.

Then around midafternoon on Thursday, Anbang's consortium sent
Starwood's board a letter thanking them for their work but stating that
it needed to walk away for unspecified market reasons.

No further reason was given.

Advertisement

\protect\hyperlink{after-bottom}{Continue reading the main story}

\hypertarget{site-index}{%
\subsection{Site Index}\label{site-index}}

\hypertarget{site-information-navigation}{%
\subsection{Site Information
Navigation}\label{site-information-navigation}}

\begin{itemize}
\tightlist
\item
  \href{https://help.nytimes.com/hc/en-us/articles/115014792127-Copyright-notice}{©~2020~The
  New York Times Company}
\end{itemize}

\begin{itemize}
\tightlist
\item
  \href{https://www.nytco.com/}{NYTCo}
\item
  \href{https://help.nytimes.com/hc/en-us/articles/115015385887-Contact-Us}{Contact
  Us}
\item
  \href{https://www.nytco.com/careers/}{Work with us}
\item
  \href{https://nytmediakit.com/}{Advertise}
\item
  \href{http://www.tbrandstudio.com/}{T Brand Studio}
\item
  \href{https://www.nytimes.com/privacy/cookie-policy\#how-do-i-manage-trackers}{Your
  Ad Choices}
\item
  \href{https://www.nytimes.com/privacy}{Privacy}
\item
  \href{https://help.nytimes.com/hc/en-us/articles/115014893428-Terms-of-service}{Terms
  of Service}
\item
  \href{https://help.nytimes.com/hc/en-us/articles/115014893968-Terms-of-sale}{Terms
  of Sale}
\item
  \href{https://spiderbites.nytimes.com}{Site Map}
\item
  \href{https://help.nytimes.com/hc/en-us}{Help}
\item
  \href{https://www.nytimes.com/subscription?campaignId=37WXW}{Subscriptions}
\end{itemize}
