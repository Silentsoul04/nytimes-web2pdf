Sections

SEARCH

\protect\hyperlink{site-content}{Skip to
content}\protect\hyperlink{site-index}{Skip to site index}

\href{https://www.nytimes.com/section/business}{Business}

\href{https://myaccount.nytimes.com/auth/login?response_type=cookie\&client_id=vi}{}

\href{https://www.nytimes.com/section/todayspaper}{Today's Paper}

\href{/section/business}{Business}\textbar{}Despite Fears, Affordable
Care Act Has Not Uprooted Employer Coverage

\url{https://nyti.ms/1RYlIlT}

\begin{itemize}
\item
\item
\item
\item
\item
\item
\end{itemize}

Advertisement

\protect\hyperlink{after-top}{Continue reading the main story}

Supported by

\protect\hyperlink{after-sponsor}{Continue reading the main story}

\hypertarget{despite-fears-affordable-care-act-has-not-uprooted-employer-coverage}{%
\section{Despite Fears, Affordable Care Act Has Not Uprooted Employer
Coverage}\label{despite-fears-affordable-care-act-has-not-uprooted-employer-coverage}}

\includegraphics{https://static01.nyt.com/images/2016/03/31/business/31employers1/31employers1-articleLarge.jpg?quality=75\&auto=webp\&disable=upscale}

By \href{http://www.nytimes.com/by/reed-abelson}{Reed Abelson}

\begin{itemize}
\item
  April 4, 2016
\item
  \begin{itemize}
  \item
  \item
  \item
  \item
  \item
  \item
  \end{itemize}
\end{itemize}

The Affordable Care Act was aimed mainly at giving people better options
for buying health insurance on their own. There were widespread
predictions that employers would leap at the chance to drop coverage and
send workers to fend for themselves.

But those predictions were largely wrong. Most companies, and
particularly large employers, that offered coverage before the law have
stayed committed to providing health insurance.

As it turns out, health care remains an important recruitment and
retention tool as the labor market has tightened in recent years.
Desirable employees still expect health benefits, and companies are
responding, new analyses of federal data show.

``We're more confident than ever that we'll offer benefits,'' said
Robert Ihrie Jr., a senior vice president for Lowe's Companies, the home
improvement retailer.

Companies get a sizable federal tax break from providing the insurance.
And if they dropped the coverage, many workers would expect the money in
their paycheck to increase enough to pay for outside insurance --- or
would look for a new job.

The reversal in thinking about employer benefits is so stark that even
government budget officials are singing an optimistic tune. They lowered
the number of people they think will lose coverage because of the health
law and now predict employers will remain the source of coverage for a
majority of working Americans for the next decade.

The surprise turnaround adds to an emerging consensus about the
contentious health law: It has not upturned the core of the country's
health insurance system, even while insuring millions of low-income
people.

``The employer-based system is alive and well,'' said Jeff Alter, the
chief executive of the commercial insurance business for
UnitedHealthcare, one of the nation's largest health insurance
companies. Even among critics of the law, including the Republican
presidential candidates, there has been virtually no debate about
employer coverage.

About 155 million Americans have employer-based health insurance
coverage in 2016, according to
\href{https://www.cbo.gov/sites/default/files/114th-congress-2015-2016/reports/51385-HealthInsuranceBaseline_OneCol.pdf}{an
analysis} released by the Congressional Budget Office last month. The
number will fall to 152 million people in 2019, the C.B.O. estimates,
but will remain stable through 2026. Slightly more than half of people
under 65 will be enrolled in employment-based coverage.

Employers seem to be staying the course even more strongly than they did
before the law. The percentage of adults under 65 with employer-based
insurance held firm for the last five years after steadily declining
since 1999, according to
\href{http://kff.org/private-insurance/issue-brief/trends-in-employer-sponsored-insurance-offer-and-coverage-rates-1999-2014/}{an
analysis} of federal data released last month by the Kaiser Family
Foundation, which closely tracks the health insurance market.

The health plans employees get to choose from also look much the same as
before the law went into effect. The industry remains dominated by
familiar names, like nonprofit Blue Cross plans or for-profit companies
like UnitedHealthcare and Anthem.

``Employer coverage is much more stable than anyone anticipated,'' said
Larry Levitt, a senior executive at Kaiser.

Companies say they are responding to the realities of the labor market,
but they also say the online marketplaces where individuals can more
easily buy plans, a creation of the health care law, have not been an
enticing alternative for workers.

Employers may feel differently if the economy turns down and the labor
market is less robust or if there is a sudden spike in health care
costs. Because workers can no longer be denied an insurance policy
because of poor health, companies may be willing to drop coverage under
the right circumstances, knowing that insurance is more available to
everyone.

\includegraphics{https://static01.nyt.com/images/2016/03/31/business/31employers2/31employers2-articleLarge.jpg?quality=75\&auto=webp\&disable=upscale}

But there are no plans for a mass exodus.

``The demise of employer-based coverage was definitely overstated,''
said Michael Thompson, the chief executive of the National Business
Coalition on Health, which represents employers and other buyers of
insurance.

The steepest declines in coverage have been in small businesses, which
had been steadily dropping coverage before the law. The percentage of
small employers offering health benefits decreased from 68 percent in
2010 to 56 percent in 2015, according to the annual
\href{http://kff.org/report-section/ehbs-2015-section-two-health-benefits-offer-rates/}{Kaiser
Family Foundation survey}.

But those companies now seem less likely to exit than just a few years
ago. In 2013, as many as a fifth of employers with fewer than 500
workers said they were likely to drop coverage in the next five years,
compared with 7 percent today, according to a survey from Mercer, the
benefits consultant.

Tracy Watts, a senior partner at Mercer, said the stabilization at small
businesses was mostly a product of their health care costs staying the
same or rising only modestly.

``That will keep you in the game,'' Ms. Watts said.

The early tumult in the insurance marketplaces, including the troubled
introduction of HealthCare.gov, the federally run insurance marketplace
for the Affordable Care Act, also made dropping coverage less tenable,
analysts said.

The law has resulted in more coverage for low-income people, as
expected. But the unexpected exit by some of the start-up insurers has
limited options on the marketplaces. And the plans on the exchanges
remain less generous than those offered by many employers, with
significantly higher deductibles and a significantly narrower choice of
hospitals and networks.

Lowe's, by comparison, said it has tried to keep employee costs low by
contributing about 70 percent of the cost of the annual premiums. It
offers plans with deductibles as low as \$1,000, versus several thousand
dollars for many of the exchange plans.

``The exchanges have been less of a disrupter than I expected,'' said
Thomas Buchmueller, a business professor at the University of Michigan.

Employers say there is less financial advantage to dropping coverage
than first thought. The law penalizes large employers, about \$2,000 per
worker, when they do not offer health insurance. That is far less than
the average cost of family coverage, now \$12,600 a year, according to
the Kaiser Family Foundation.

But those calculations do not figure in the sizable tax break that comes
with providing coverage. In addition, if the employers do not provide
insurance, they would almost certainly be pressured --- especially in a
strong labor market --- to add enough money to workers' paychecks to
cover the cost of buying insurance on the marketplace.

Some employers, particularly the smallest businesses and those who
employ low-income workers eligible for a subsidy, clearly favor moving
employees to the exchanges. Other employers do not.

``The math really worked in favor of providing coverage,'' Mr. Thompson
said.

While the C.B.O. predicts the employer market will be stable until 2026,
10 years is a long time and federal officials could be wrong .

``They are taking their best guess, and I think it is a reasonable best
guess,'' said Loren Adler, a health policy analyst at the Brookings
Institution in Washington. ``I wouldn't take it as gospel.''

If the cost of providing coverage spikes, as it did through much of the
late 1990s and early 2000s, employers might start talking about dropping
coverage again, Mr. Adler said. Congress could also take steps to
diminish the tax preference for employer-provided insurance.

But so far, employers like Mr. Ihrie of Lowe's say they do not feel
compelled to change.

``People have concluded,'' he said, ``that it's better to stay where we
are.''

Advertisement

\protect\hyperlink{after-bottom}{Continue reading the main story}

\hypertarget{site-index}{%
\subsection{Site Index}\label{site-index}}

\hypertarget{site-information-navigation}{%
\subsection{Site Information
Navigation}\label{site-information-navigation}}

\begin{itemize}
\tightlist
\item
  \href{https://help.nytimes.com/hc/en-us/articles/115014792127-Copyright-notice}{©~2020~The
  New York Times Company}
\end{itemize}

\begin{itemize}
\tightlist
\item
  \href{https://www.nytco.com/}{NYTCo}
\item
  \href{https://help.nytimes.com/hc/en-us/articles/115015385887-Contact-Us}{Contact
  Us}
\item
  \href{https://www.nytco.com/careers/}{Work with us}
\item
  \href{https://nytmediakit.com/}{Advertise}
\item
  \href{http://www.tbrandstudio.com/}{T Brand Studio}
\item
  \href{https://www.nytimes.com/privacy/cookie-policy\#how-do-i-manage-trackers}{Your
  Ad Choices}
\item
  \href{https://www.nytimes.com/privacy}{Privacy}
\item
  \href{https://help.nytimes.com/hc/en-us/articles/115014893428-Terms-of-service}{Terms
  of Service}
\item
  \href{https://help.nytimes.com/hc/en-us/articles/115014893968-Terms-of-sale}{Terms
  of Sale}
\item
  \href{https://spiderbites.nytimes.com}{Site Map}
\item
  \href{https://help.nytimes.com/hc/en-us}{Help}
\item
  \href{https://www.nytimes.com/subscription?campaignId=37WXW}{Subscriptions}
\end{itemize}
