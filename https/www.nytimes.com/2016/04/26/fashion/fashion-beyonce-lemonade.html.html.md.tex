Sections

SEARCH

\protect\hyperlink{site-content}{Skip to
content}\protect\hyperlink{site-index}{Skip to site index}

\href{https://www.nytimes.com/section/fashion}{Fashion}

\href{https://myaccount.nytimes.com/auth/login?response_type=cookie\&client_id=vi}{}

\href{https://www.nytimes.com/section/todayspaper}{Today's Paper}

\href{/section/fashion}{Fashion}\textbar{}For Fashion, Beyoncé's
`Lemonade' Is the Anti-Coachella

\url{https://nyti.ms/1T94Pj3}

\begin{itemize}
\item
\item
\item
\item
\item
\end{itemize}

Advertisement

\protect\hyperlink{after-top}{Continue reading the main story}

Supported by

\protect\hyperlink{after-sponsor}{Continue reading the main story}

\href{/column/on-the-runway}{On the Runway}

\hypertarget{for-fashion-beyoncuxe9s-lemonade-is-the-anti-coachella}{%
\section{For Fashion, Beyoncé's `Lemonade' Is the
Anti-Coachella}\label{for-fashion-beyoncuxe9s-lemonade-is-the-anti-coachella}}

\includegraphics{https://static01.nyt.com/images/2016/04/25/fashion/25OTR-BEYONCE/25OTR-BEYONCE-articleLarge.jpg?quality=75\&auto=webp\&disable=upscale}

By \href{https://www.nytimes.com/by/vanessa-friedman}{Vanessa Friedman}

\begin{itemize}
\item
  April 25, 2016
\item
  \begin{itemize}
  \item
  \item
  \item
  \item
  \item
  \end{itemize}
\end{itemize}

Beyoncé's new visual album
``\href{http://www.nytimes.com/2016/04/25/arts/music/beyonce-unearths-pain-and-lets-it-flow-in-lemonade.html}{Lemonade},''
released on HBO late on Saturday and causing paroxysms of rapture across
the Internet ever since, has been called many things, among them: an ode
to ``female solidarity,'' a portrait of ``Southern gothic empowerment,''
``an emotional odyssey,'' ``a love letter to black women'' and a series
of ``fashion statements.''

Which of these things is not like the others?

If you guessed the last, you are correct. Though Fashion (capital F) has
been trying to claim Beyoncé since, well, forever, she is not its
creature, and this album proves it. It is the anti-Coachella: the
opposite of a musical event leeched of meaning by branded commercial
enterprises. Indeed, it crushes branded commercial enterprises under its
powerful feet.

True, its 12 videos are full of fashion names, including Peter Dundas
for Roberto Cavalli, Saint Laurent (the dress and shoes in ``Denial'');
Hood by Air and Yeezy (coat and top in ``Don't Hurt Yourself''); Rosie
Assoulin (off-the-shoulder peach top); and Phelan (worn by Zendaya).

But it is also full of names most people have never heard of: the Kuwait
designer Yousef al-Jasmi (the crystal bodysuit in ``Sorry''); the New
York-based Zana Bayne (the leather cone bra worn à la Nefertiti); the
Lebanese designer Nicolas Jebran (a high-necked orange ball gown with
jet embroidered geometry). It is packed with historical references and
vintage pieces; white (lots of white) wedding dresses (including her
own) and slip dresses and lace high-necked Victorian body suits; a
courtlike dress remade in African print jacquard and chopped short at
the front; a tulle shroud; tribal markings --- you get the idea.

All together these create a world of intense visual richness and power
that is unidentifiable by logo or look. And all of it has the effect of
elevating the whole out of a world ruled by designers and recognizable
global names and into a personal aesthetic statement.

Recently, it has become trendy once again for musicians to team up with
brands for their tour wardrobes --- this year Adele is being outfitted
by Burberry; Florence Welch by Gucci --- and it would not have been
surprising if, for a project such as ``Lemonade,'' Beyoncé had chosen to
work with a single designer. (At the very least, there's probably a
compelling financial reason for such collaboration, and it is more
efficient.)

But if that had happened, said designer would have been stamped with
Beyoncé's endorsement, and her name would have been forever linked with
that designer. Some of her influence would have rubbed off, and some of
the focus on her work would have, too.

Many of the stories that appeared about the most recent Coachella
festival focused more on the fashion than on the music, to the extent
that the fashion messaging overwhelmed the music messaging, and it began
to feel a bit like an extended ad campaign.

By contrast, on ``Lemonade,'' the clothes support the point, or points;
they are not the point. Which is as it should be --- and is totally in
line with Beyoncé's past
\href{http://www.nytimes.com/2014/07/31/fashion/beyonce-discounts-the-fashion-icon.html}{approach}
to fashion. In their breadth and diversity and unpredictability, the
costumes emphasize the idea, embedded in the lyrics as well as the
album's narrative chapter structure (Intuition to Denial to Anger to,
ultimately, Hope and Redemption), that the power in this world belongs
to Beyoncé and to her alone. That includes the power to decide, to
declare her feelings and needs and pain, to decide to go back on her
promise to leave and to stay, to --- yes, even this --- choose what she
wants to wear that expresses all of the above.

(O.K., maybe after some discussion with her stylist, Marni Senofonte.)

The clothes serve the woman. Which, in fact, is what all fashion should
do. As a result, although it did not take its cues from the catwalk, it
is possible the album may have a knock-on effect on the catwalk --- that
next season we will suddenly see a lot of high-necked shredded
Victoriana that likewise takes the uniform of hothouse fragility and
inverts it so that it becomes a statement about strength.

In the meantime, however, it is worth reminding ourselves of that truth
every once in a while. If we forget, Beyoncé is here to remind us.

Advertisement

\protect\hyperlink{after-bottom}{Continue reading the main story}

\hypertarget{site-index}{%
\subsection{Site Index}\label{site-index}}

\hypertarget{site-information-navigation}{%
\subsection{Site Information
Navigation}\label{site-information-navigation}}

\begin{itemize}
\tightlist
\item
  \href{https://help.nytimes.com/hc/en-us/articles/115014792127-Copyright-notice}{©~2020~The
  New York Times Company}
\end{itemize}

\begin{itemize}
\tightlist
\item
  \href{https://www.nytco.com/}{NYTCo}
\item
  \href{https://help.nytimes.com/hc/en-us/articles/115015385887-Contact-Us}{Contact
  Us}
\item
  \href{https://www.nytco.com/careers/}{Work with us}
\item
  \href{https://nytmediakit.com/}{Advertise}
\item
  \href{http://www.tbrandstudio.com/}{T Brand Studio}
\item
  \href{https://www.nytimes.com/privacy/cookie-policy\#how-do-i-manage-trackers}{Your
  Ad Choices}
\item
  \href{https://www.nytimes.com/privacy}{Privacy}
\item
  \href{https://help.nytimes.com/hc/en-us/articles/115014893428-Terms-of-service}{Terms
  of Service}
\item
  \href{https://help.nytimes.com/hc/en-us/articles/115014893968-Terms-of-sale}{Terms
  of Sale}
\item
  \href{https://spiderbites.nytimes.com}{Site Map}
\item
  \href{https://help.nytimes.com/hc/en-us}{Help}
\item
  \href{https://www.nytimes.com/subscription?campaignId=37WXW}{Subscriptions}
\end{itemize}
