Sections

SEARCH

\protect\hyperlink{site-content}{Skip to
content}\protect\hyperlink{site-index}{Skip to site index}

\href{https://www.nytimes.com/section/world/australia}{Australia}

\href{https://myaccount.nytimes.com/auth/login?response_type=cookie\&client_id=vi}{}

\href{https://www.nytimes.com/section/todayspaper}{Today's Paper}

\href{/section/world/australia}{Australia}\textbar{}Papua New Guinea
Finds Australian Offshore Detention Center Illegal

\url{https://nyti.ms/1MVwwzd}

\begin{itemize}
\item
\item
\item
\item
\item
\end{itemize}

Advertisement

\protect\hyperlink{after-top}{Continue reading the main story}

Supported by

\protect\hyperlink{after-sponsor}{Continue reading the main story}

\hypertarget{papua-new-guinea-finds-australian-offshore-detention-center-illegal}{%
\section{Papua New Guinea Finds Australian Offshore Detention Center
Illegal}\label{papua-new-guinea-finds-australian-offshore-detention-center-illegal}}

\includegraphics{https://static01.nyt.com/images/2016/04/27/world/27AUSTRALIA-web1/27AUSTRALIA-web1-videoSixteenByNineJumbo1600.jpg}

By Austin Ramzy

\begin{itemize}
\item
  April 26, 2016
\item
  \begin{itemize}
  \item
  \item
  \item
  \item
  \item
  \end{itemize}
\end{itemize}

The Supreme Court of Papua New Guinea ruled on Tuesday that the Pacific
island nation's detention of people seeking asylum in Australia was
illegal, but an Australian official said the decision would not change
his country's tough stand on seaborne migrants.

More than 800 men who tried to reach Australia by boat in recent years
are being held in an Australian-funded detention center on Manus Island
in northern Papua New Guinea.

In its ruling, the five-judge court ordered the governments of Papua New
Guinea and Australia to end those detentions. The ruling gave no
timetable for the release of the detainees, and it was unclear what
would happen to them.

The court declared that because the asylum seekers had not entered Papua
New Guinea of their own accord, they were not guilty of immigration
violations, and that holding them ignored constitutional protections of
personal liberty.

Peter Dutton, Australia's immigration minister, said the ruling would
not mean that the asylum seekers would be allowed to come to Australia.

``It does not alter Australia's border protection policies --- they
remain unchanged,'' he said in a
\href{http://www.minister.border.gov.au/peterdutton/2016/Pages/png-supreme-court-judgement.aspx}{statement}.
``No one who attempts to travel to Australia illegally by boat will
settle in Australia.''

Human rights advocates have criticized Australia's offshore detention of
asylum seekers who seek to arrive by boat. In 2014, an
\href{http://www.nytimes.com/2014/12/12/world/asia/australia-rioting-papua-new-guinea-manus-island.html}{Iranian
detainee was killed} during two days of rioting at the Manus Island
center.

Opponents of the offshore detentions said that the ruling on Tuesday
suggested that the policy was unsustainable.

``Time to bring those left there to Australia to be cared for,'' Sarah
Hanson-Young, a Greens party senator for South Australia,
\href{https://twitter.com/sarahinthesen8/status/724835760273297409}{wrote}
on Twitter.

``It is well time to close these awful detention camps on Manus \& Nauru
and start treating people like human beings,''
\href{https://twitter.com/sarahinthesen8/status/724849198223470592}{she
added}, referring also to an Australian-backed processing center on
another Pacific island nation. ``Anything less is senseless.''

Besides the detainees on Manus, 468 men, women and children were being
\href{https://www.border.gov.au/ReportsandPublications/Documents/statistics/immigration-detention-statistics-31-mar-2016.pdf}{held
in the processing center} on Nauru as of March 31.

Australian leaders have said the offshore detentions have led to a
steady decrease in the number of attempted arrivals in their country by
sea, and those officials have shown few signs of wanting to change the
policy.

Last year, Mr. Dutton announced that
\href{http://www.nytimes.com/2015/10/24/world/australia/papua-new-guinea-to-resettle-refugees-from-australian-detention-center.html}{Papua
New Guinea would begin resettling refugees} held at the Manus camp. But
Papua New Guinea officials cautioned that the movement of refugees to
cities would have to be carefully controlled to prevent the new arrivals
from competing with residents for jobs.

So far, the effort has had little success. Although about half of the
Manus detainees have been designated as refugees, only eight have been
resettled, and at least three have tried to return to the detention
center, Fairfax Media in Australia
\href{http://www.smh.com.au/federal-politics/political-news/papua-new-guinea-court-finds-australias-detention-of-asylum-seekers-on-manus-island-is-illegal-20160426-gofaaj.html}{reported}.

Advertisement

\protect\hyperlink{after-bottom}{Continue reading the main story}

\hypertarget{site-index}{%
\subsection{Site Index}\label{site-index}}

\hypertarget{site-information-navigation}{%
\subsection{Site Information
Navigation}\label{site-information-navigation}}

\begin{itemize}
\tightlist
\item
  \href{https://help.nytimes.com/hc/en-us/articles/115014792127-Copyright-notice}{©~2020~The
  New York Times Company}
\end{itemize}

\begin{itemize}
\tightlist
\item
  \href{https://www.nytco.com/}{NYTCo}
\item
  \href{https://help.nytimes.com/hc/en-us/articles/115015385887-Contact-Us}{Contact
  Us}
\item
  \href{https://www.nytco.com/careers/}{Work with us}
\item
  \href{https://nytmediakit.com/}{Advertise}
\item
  \href{http://www.tbrandstudio.com/}{T Brand Studio}
\item
  \href{https://www.nytimes.com/privacy/cookie-policy\#how-do-i-manage-trackers}{Your
  Ad Choices}
\item
  \href{https://www.nytimes.com/privacy}{Privacy}
\item
  \href{https://help.nytimes.com/hc/en-us/articles/115014893428-Terms-of-service}{Terms
  of Service}
\item
  \href{https://help.nytimes.com/hc/en-us/articles/115014893968-Terms-of-sale}{Terms
  of Sale}
\item
  \href{https://spiderbites.nytimes.com}{Site Map}
\item
  \href{https://help.nytimes.com/hc/en-us}{Help}
\item
  \href{https://www.nytimes.com/subscription?campaignId=37WXW}{Subscriptions}
\end{itemize}
