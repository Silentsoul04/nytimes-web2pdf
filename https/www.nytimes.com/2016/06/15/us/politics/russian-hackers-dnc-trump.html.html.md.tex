Sections

SEARCH

\protect\hyperlink{site-content}{Skip to
content}\protect\hyperlink{site-index}{Skip to site index}

\href{https://www.nytimes.com/section/politics}{Politics}

\href{https://myaccount.nytimes.com/auth/login?response_type=cookie\&client_id=vi}{}

\href{https://www.nytimes.com/section/todayspaper}{Today's Paper}

\href{/section/politics}{Politics}\textbar{}D.N.C. Says Russian Hackers
Penetrated Its Files, Including Dossier on Donald Trump

\url{https://nyti.ms/1UsN9Ub}

\begin{itemize}
\item
\item
\item
\item
\item
\end{itemize}

Advertisement

\protect\hyperlink{after-top}{Continue reading the main story}

Supported by

\protect\hyperlink{after-sponsor}{Continue reading the main story}

\hypertarget{dnc-says-russian-hackers-penetrated-its-files-including-dossier-on-donald-trump}{%
\section{D.N.C. Says Russian Hackers Penetrated Its Files, Including
Dossier on Donald
Trump}\label{dnc-says-russian-hackers-penetrated-its-files-including-dossier-on-donald-trump}}

\includegraphics{https://static01.nyt.com/images/2016/06/15/us/15dnc/15dnc-articleLarge.jpg?quality=75\&auto=webp\&disable=upscale}

By \href{http://www.nytimes.com/by/david-e-sanger}{David E. Sanger} and
\href{http://www.nytimes.com/by/nick-corasaniti}{Nick Corasaniti}

\begin{itemize}
\item
  June 14, 2016
\item
  \begin{itemize}
  \item
  \item
  \item
  \item
  \item
  \end{itemize}
\end{itemize}

WASHINGTON --- Two groups of Russian hackers, working for competing
government intelligence agencies, penetrated computer systems of the
\href{http://topics.nytimes.com/top/reference/timestopics/organizations/d/democratic_national_committee/index.html?inline=nyt-org}{Democratic
National Committee} and gained access to emails, chats and a trove of
opposition research against
\href{http://www.nytimes.com/interactive/2016/us/elections/donald-trump-on-the-issues.html?inline=nyt-per}{Donald
J. Trump}, according to the party and a cybersecurity firm.

One group placed espionage software on the committee's computer servers
last summer, giving it unimpeded access to communications for about a
year. The committee called in CrowdStrike, a cybersecurity firm, early
last month after the Democratic Party began to suspect an intrusion.

A senior government official said Hillary Clinton's presidential
campaign, based in Brooklyn, also appeared to have been targeted, but it
was not clear whether it lost any data. The breach at the Democratic
committee was first
\href{https://www.washingtonpost.com/world/national-security/russian-government-hackers-penetrated-dnc-stole-opposition-research-on-trump/2016/06/14/cf006cb4-316e-11e6-8ff7-7b6c1998b7a0_story.html?hpid=hp_hp-banner-main_dnc-hackers-1145a-banner\%3Ahomepage\%2Fstory}{reported
on Tuesday by The Washington Post}.

The committee's systems appeared to have had standard cyberprotections,
which are no challenge for determined state-sponsored hacking groups.
The attackers were expelled last weekend with CrowdStrike's help, the
committee said. It did not provide a detailed account of what had been
copied from the systems, and it may never know.

The connection to Russia may be explained simply by the global
fascination with the presidential campaign and the mystery surrounding
Mr. Trump, who has not been a major subject of foreign intelligence
collection. But it also recalls a subplot to the race: Paul Manafort,
Mr. Trump's campaign chairman, previously advised pro-Russian
politicians in Ukraine and other parts of Eastern Europe, including
former President Viktor F. Yanukovych of Ukraine.

Opposition research itself is not all that valuable to a foreign
government, but it can point to a candidate's vulnerabilities. To a
foreign government fascinated by an American election, any intelligence
a campaign develops on an opponent could be exploited.

Dmitri Alperovitch, a co-founder of CrowdStrike, said he believed that
the group that first hacked the committee's servers --- a group his firm
had named Cozy Bear long before the breach --- appeared to be the same
that downloaded communications in recent years from unclassified email
systems used by the State Department and the White House.

In 2014 and 2015, the effort to clean the State Department systems after
those intrusions resulted in several shutdowns, some in the midst of
delicate negotiations with Iran. The administration has never confirmed
that the Russian government was behind those intrusions, but it has
briefed officials on the details in classified sessions.

``These are incredibly sophisticated groups,'' Mr. Alperovitch said.
``They covered their tracks well. It wasn't until the second group came
in,'' stealing the opposition research on Mr. Trump, ``that their
presence was detected.''

The second group, named Fancy Bear, which appeared to have attacked in
April, is believed to be operated by the G.R.U., the military
intelligence service. Its past targets have included military and
aerospace organizations from the United States, Europe, Canada, Japan
and South Korea.

CrowdStrike concluded that neither Russian group knew the other was
attacking the same organization. ``One would steal a password, and the
next day the other group would steal the same password,'' Mr.
Alperovitch said.

Mrs. Clinton said on Telemundo that she had learned of the breach only
after news outlets reported it. She called it ``troubling,'' but said
she was unsure about the hackers' goals.

``Now, why?'' she asked. ``We don't know yet. So far as we know, my
campaign has not been hacked into, and we're obviously looking hard at
that. But cybersecurity will be an issue that I will be absolutely
focused on as president. Because whether it's Russia, or China, Iran or
North Korea, more and more countries are using hacking to steal our
information, to use it to their advantage, and we can't let that go
on.''

The Office of Personnel Management, whose files on about 22 million
Americans with security clearances or applications for them were
breached by Chinese hackers, is still trying to assess the damage first
detected last year.

The Democratic committee avoided any discussion of its vulnerabilities.

``The security of our system is critical to our operation and to the
confidence of the campaigns and state parties we work with,'' said
Representative Debbie Wasserman Schultz of Florida, the Democratic
national chairwoman. ``When we discovered the intrusion, we treated this
like the serious incident it is and reached out to CrowdStrike
immediately. Our team moved as quickly as possible to kick out the
intruders and secure our network.''

The party did not say how it came to suspect the intrusion.

Cyberattacks by foreign governments are a constant threat to political
campaigns. Because campaign operations are temporary, they often do not
invest heavily in the kind of security that financial institutions,
large companies and government agencies spend millions or billions of
dollars on each year.

And because campaigns are so far-flung, with volunteers connecting
through laptops and cellphones, they are particularly vulnerable. In
2008, hackers traced to the Chinese government
\href{http://investigations.nbcnews.com/_news/2013/06/06/18807056-chinese-hacked-obama-mccain-campaigns-took-internal-documents-officials-say}{infiltrated
the campaigns of both Barack Obama and John McCain.}

``It should come as no surprise to anyone that political parties are
high-profile targets for foreign intelligence gathering,''
Representative Jim Langevin, a Rhode Island Democrat who has been deeply
involved in cyberissues, said in a statement. ``Nonetheless, it is
disconcerting that two independent operations were able to penetrate the
D.N.C., one of which was able to stay embedded for nearly a year.''

Advertisement

\protect\hyperlink{after-bottom}{Continue reading the main story}

\hypertarget{site-index}{%
\subsection{Site Index}\label{site-index}}

\hypertarget{site-information-navigation}{%
\subsection{Site Information
Navigation}\label{site-information-navigation}}

\begin{itemize}
\tightlist
\item
  \href{https://help.nytimes.com/hc/en-us/articles/115014792127-Copyright-notice}{©~2020~The
  New York Times Company}
\end{itemize}

\begin{itemize}
\tightlist
\item
  \href{https://www.nytco.com/}{NYTCo}
\item
  \href{https://help.nytimes.com/hc/en-us/articles/115015385887-Contact-Us}{Contact
  Us}
\item
  \href{https://www.nytco.com/careers/}{Work with us}
\item
  \href{https://nytmediakit.com/}{Advertise}
\item
  \href{http://www.tbrandstudio.com/}{T Brand Studio}
\item
  \href{https://www.nytimes.com/privacy/cookie-policy\#how-do-i-manage-trackers}{Your
  Ad Choices}
\item
  \href{https://www.nytimes.com/privacy}{Privacy}
\item
  \href{https://help.nytimes.com/hc/en-us/articles/115014893428-Terms-of-service}{Terms
  of Service}
\item
  \href{https://help.nytimes.com/hc/en-us/articles/115014893968-Terms-of-sale}{Terms
  of Sale}
\item
  \href{https://spiderbites.nytimes.com}{Site Map}
\item
  \href{https://help.nytimes.com/hc/en-us}{Help}
\item
  \href{https://www.nytimes.com/subscription?campaignId=37WXW}{Subscriptions}
\end{itemize}
