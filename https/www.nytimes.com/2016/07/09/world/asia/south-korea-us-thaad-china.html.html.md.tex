Sections

SEARCH

\protect\hyperlink{site-content}{Skip to
content}\protect\hyperlink{site-index}{Skip to site index}

\href{https://www.nytimes.com/section/world/asia}{Asia Pacific}

\href{https://myaccount.nytimes.com/auth/login?response_type=cookie\&client_id=vi}{}

\href{https://www.nytimes.com/section/todayspaper}{Today's Paper}

\href{/section/world/asia}{Asia Pacific}\textbar{}For China, a Missile
Defense System in South Korea Spells a Failed Courtship

\url{https://nyti.ms/29DDIzW}

\begin{itemize}
\item
\item
\item
\item
\item
\end{itemize}

Advertisement

\protect\hyperlink{after-top}{Continue reading the main story}

Supported by

\protect\hyperlink{after-sponsor}{Continue reading the main story}

\hypertarget{for-china-a-missile-defense-system-in-south-korea-spells-a-failed-courtship}{%
\section{For China, a Missile Defense System in South Korea Spells a
Failed
Courtship}\label{for-china-a-missile-defense-system-in-south-korea-spells-a-failed-courtship}}

\includegraphics{https://static01.nyt.com/images/2016/07/09/world/09KOREA-web1/09KOREA-web1-articleLarge.jpg?quality=75\&auto=webp\&disable=upscale}

By \href{http://www.nytimes.com/by/jane-perlez}{Jane Perlez}

\begin{itemize}
\item
  July 8, 2016
\item
  \begin{itemize}
  \item
  \item
  \item
  \item
  \item
  \end{itemize}
\end{itemize}

BEIJING --- However isolated North Korea may be, it has long had one
major ally: China. But for two years, China's leader, President Xi
Jinping, seemed to be favoring Pyongyang's neighbor and nemesis to the
south.

He spent much political capital wooing South Korea's president, Park
Geun-hye, in hopes of drawing the country away from its longtime ally,
the United States. He made an elaborate state visit to Seoul while
shunning North Korea and its young leader, Kim Jong-un, whom he has yet
to meet. Ms. Park returned the favor last year, coming to Beijing for a
major military parade at Tiananmen Square, the only leader of an
American ally to attend.

But on Friday, it became clear that Mr. Xi's efforts had fallen short.
In
\href{http://www.nytimes.com/2016/07/08/world/asia/south-korea-and-us-agree-to-deploy-missile-defense-system.html}{announcing
plans to deploy an advanced American missile defense system} in South
Korean, Ms. Park's government showed that it was embracing its alliance
with Washington more than ever, and that it would rely less on China to
keep North Korea and its nuclear arsenal at bay.

In Beijing, the decision was seen as a major setback, one that went
beyond its interests on the Korean Peninsula to the larger strategic
question of an arms race in Northeast Asia that could impel China ---
and Russia --- to develop more sophisticated weapons.

Analysts said the deployment of the so-called Terminal High-Altitude
Area Defense system, or Thaad, would reinforce the already high level of
mistrust in United States-China relations as the Obama administration
nears its end, adding to the raw nerves over disputes in the South China
Sea and differences over American business access to the Chinese market.

And North Korea, an issue on which there had been some common ground
between the two powers --- at least when it came to the latest round of
United Nations sanctions --- is likely to become a greater source of
irritation, as China loses an incentive to be tougher on the regime.

On Saturday, North Korea test-fired a submarine-launched ballistic
missile off its east coast at 11:30 a.m., the South Korean military
said. The missile was successfully ejected from the submarine, it said,
but failed in the first stage of flight. The North also tested a
submarine-launched ballistic missile in April.

In announcing the American missile defense system, which has been under
discussion for years, the top commander of the United States military in
South Korea, Gen. Vincent K. Brooks, said Friday that it was needed to
protect South Korea from the North's nuclear weapons.

But Chinese officials have repeatedly said that they do not believe the
North Korean threat is the true reason for the American-initiated
deployment. Rather, they say, the purpose of the Thaad system, which
\href{http://blogs.cfr.org/davidson/2015/03/26/korea-not-a-shrimp-anymore/}{detects
and intercepts incoming missiles} at high altitudes, is to track
missiles launched from China.

Now that the system's implementation has been confirmed, China will
almost certainly consider developing more advanced missiles as a
countermeasure, said Cheng Xiaohe, an associate professor at Renmin
University in Beijing and a North Korea expert.

\includegraphics{https://static01.nyt.com/images/2016/07/09/world/09KOREA-web2/09KOREA-web2-articleLarge.jpg?quality=75\&auto=webp\&disable=upscale}

``A way to deal with Thaad --- a shield --- is to sharpen your spear,''
Mr. Cheng said.

The possibility of the Thaad deployment has bedeviled relations between
Washington and Beijing for more than a year.

Last month, Mr. Xi and President Vladimir V. Putin of Russia made a
point of
\href{http://news.xinhuanet.com/english/2016-06/26/c_135466187.htm}{denouncing
the Thaad system} during Mr. Putin's visit to Beijing, equating it with
the American-built Aegis Ashore ballistic missile defense system
deployed in some NATO countries. The implicit message was that the
United States was trying to encircle China in the same way that,
according to Mr. Putin, it was trying to contain Russia.

Before Mr. Putin's visit, China's foreign minister, Wang Yi, expressed
the Chinese view that the Thaad system is a strategic game-changer in
Northeast Asia.

``The Thaad system has far exceeded the need for defense in the Korean
Peninsula and will undermine the security interests of China and Russia,
shatter the regional strategic balance and trigger an arms race,'' Mr.
Wang said. China understands South Korea's ``rational need'' for
defense, he said, ``but we can't understand and we will not accept why
they made a deployment exceeding the need.''

Chinese analysts have said that they expect Japan to eventually deploy
Thaad as well, in what they say would be an American attempt to draw it
closer into a three-way alliance with South Korea. So far, Japan has
shown little interest in the Thaad system, but Washington and Tokyo are
jointly working on a
\href{http://spacenews.com/new-sm-3-variant-faces-two-intercept-tests-this-year/}{new
missile interceptor} that is expected to start production in 2017.

Talks between Seoul and Washington on the Thaad deployment picked up
speed after
\href{http://www.nytimes.com/2016/01/07/world/asia/north-korea-hydrogen-bomb-q-a.html}{North
Korea conducted its fourth nuclear test} in January. After that test,
which Pyongyang claimed was of a hydrogen bomb, Ms. Park tried but
failed to reach Mr. Xi by telephone, according to South Korean news
reports that were later confirmed by Chinese officials.

The nuclear test left Ms. Park convinced that Mr. Xi could not rein in
North Korea's nuclear ambitions, and that China was uninterested in her
\href{http://www.nytimes.com/2013/05/08/world/asia/obama-backs-policy-of-south-koreas-president-on-north.html}{``trustpolitik''
strategy} of finding ways to engage with the North while responding
strongly to provocations, South Korean officials said.

In March, South Korea and the United States began formal talks on the
Thaad deployment. China tried to persuade Ms. Park to accommodate
Beijing's interests by asking for technical adjustments to the system,
under which its radar would penetrate less deeply into China, according
to Wu Xinbo, the director of the Center for American Studies at Fudan
University in Shanghai. But those adjustments were not made, he said.

Some in South Korea have expressed concern that China, the country's top
trading partner, might engage in economic retaliation for the Thaad
deployment. Cheong Seong-chang, a senior analyst with the Sejong
Institute in Seongnam, south of Seoul, said China could reduce the
number of tourists it allows into the country or boycott some South
Korean goods.

Mr. Wu said Beijing was unlikely to take such measures in this period of
slower economic growth. But he said the debate over North Korea among
senior Chinese leaders would almost certainly be reshaped, with
officials who favor better relations with Pyongyang gaining more
influence, after two years of Mr. Xi keeping its isolated neighbor at a
distance.

``The school in favor of a more balanced approach to North Korea will
get more sway,'' Mr. Wu said.

Advertisement

\protect\hyperlink{after-bottom}{Continue reading the main story}

\hypertarget{site-index}{%
\subsection{Site Index}\label{site-index}}

\hypertarget{site-information-navigation}{%
\subsection{Site Information
Navigation}\label{site-information-navigation}}

\begin{itemize}
\tightlist
\item
  \href{https://help.nytimes.com/hc/en-us/articles/115014792127-Copyright-notice}{©~2020~The
  New York Times Company}
\end{itemize}

\begin{itemize}
\tightlist
\item
  \href{https://www.nytco.com/}{NYTCo}
\item
  \href{https://help.nytimes.com/hc/en-us/articles/115015385887-Contact-Us}{Contact
  Us}
\item
  \href{https://www.nytco.com/careers/}{Work with us}
\item
  \href{https://nytmediakit.com/}{Advertise}
\item
  \href{http://www.tbrandstudio.com/}{T Brand Studio}
\item
  \href{https://www.nytimes.com/privacy/cookie-policy\#how-do-i-manage-trackers}{Your
  Ad Choices}
\item
  \href{https://www.nytimes.com/privacy}{Privacy}
\item
  \href{https://help.nytimes.com/hc/en-us/articles/115014893428-Terms-of-service}{Terms
  of Service}
\item
  \href{https://help.nytimes.com/hc/en-us/articles/115014893968-Terms-of-sale}{Terms
  of Sale}
\item
  \href{https://spiderbites.nytimes.com}{Site Map}
\item
  \href{https://help.nytimes.com/hc/en-us}{Help}
\item
  \href{https://www.nytimes.com/subscription?campaignId=37WXW}{Subscriptions}
\end{itemize}
