Sections

SEARCH

\protect\hyperlink{site-content}{Skip to
content}\protect\hyperlink{site-index}{Skip to site index}

\href{https://www.nytimes.com/section/world/asia}{Asia Pacific}

\href{https://myaccount.nytimes.com/auth/login?response_type=cookie\&client_id=vi}{}

\href{https://www.nytimes.com/section/todayspaper}{Today's Paper}

\href{/section/world/asia}{Asia Pacific}\textbar{}South Korean Villagers
Protest Plans for U.S. Missile Defense System

\url{https://nyti.ms/29WS2pK}

\begin{itemize}
\item
\item
\item
\item
\item
\end{itemize}

Advertisement

\protect\hyperlink{after-top}{Continue reading the main story}

Supported by

\protect\hyperlink{after-sponsor}{Continue reading the main story}

\hypertarget{south-korean-villagers-protest-plans-for-us-missile-defense-system}{%
\section{South Korean Villagers Protest Plans for U.S. Missile Defense
System}\label{south-korean-villagers-protest-plans-for-us-missile-defense-system}}

\includegraphics{https://static01.nyt.com/images/2016/07/14/world/14KOREA-web1/14KOREA-web1-articleLarge.jpg?quality=75\&auto=webp\&disable=upscale}

By \href{http://www.nytimes.com/by/choe-sang-hun}{Choe Sang-Hun}

\begin{itemize}
\item
  July 13, 2016
\item
  \begin{itemize}
  \item
  \item
  \item
  \item
  \item
  \end{itemize}
\end{itemize}

SEOUL, South Korea --- South Korea announced on Wednesday that a rural
southern county would be the site of an advanced American missile
defense battery, the planned deployment of which has angered China and
North Korea --- and, now, thousands of local residents, who demonstrated
against the plan.

Villagers rallied under a sweltering sun to condemn the choice of their
county, Seongju, which is about 135 miles southeast of Seoul, the
capital, for
\href{http://www.nytimes.com/2016/07/08/world/asia/south-korea-and-us-agree-to-deploy-missile-defense-system.html}{the
so-called Terminal High-Altitude Area Defense system}, known as Thaad.
South Korea and the United States say the powerful missile and radar
system is needed to defend the country, and American forces stationed
here, against North Korean missiles, but residents fear it will threaten
their health and ruin their agricultural economy.

``We oppose Thaad with our lives!'' the residents chanted, holding
banners that bore the same slogan. Local political leaders, wearing red
headbands,
\href{http://www.yonhapnews.co.kr/society/2016/07/13/0701000000AKR20160713113900053.HTML?template=2087}{wrote
the same vow in blood} after cutting their fingers, a dramatic form of
protest that has a long history in South Korea. Some of the politicians
and protest leaders also began a hunger strike.

``If we lose our precious land to Thaad,
\href{http://www.yonhapnews.co.kr/society/2016/07/13/0701000000AKR20160713113900053.HTML?template=2087}{we
will be ashamed before our ancestors and posterity,}'' Kim Hang-gon, who
oversees the Seongju county government, told the crowd, many of them
aging melon farmers, according to the news agency Yonhap. The county,
which has a population of about 50,000, provides 60 percent of all
melons sold in South Korea.

The opposition could bode ill for the American and South Korean
militaries, which hope to install the Thaad battery by late 2017. In the
past, villagers have joined forces with environmental and political
activists to initiate
\href{http://www.nytimes.com/2006/05/04/world/asia/04iht-korea.html}{prolonged
and often violent campaigns} against new United States military bases.

Most South Koreans support the country's military alliance with the
United States, citing the need to deter the North. But many also fear
that any expansion of the American military presence could worsen
tensions with the North and with China, and in some cases could damage
local ways of life.

After South Korea and the United States announced
\href{http://www.nytimes.com/2016/07/08/world/asia/south-korea-and-us-agree-to-deploy-missile-defense-system.html}{the
agreement} to deploy Thaad on Friday, local news reports mentioned
Seongju and several other towns as possible sites. Protests against
Thaad have since been held in those communities. Some demonstrators
expressed concern that hosting the system could make their towns
high-priority targets for North Korea in the event of war.

South Korea's Defense Ministry said on Wednesday that the Thaad battery
would be installed at an existing South Korean Air Force radar and
missile base on a mountain in Seongju. The South Korean unit will be
moved elsewhere, it said.

The deployment in Seongju will allow the Thaad system's interceptor
missiles to protect from half to two-thirds of the country from North
Korean missiles, the ministry said. It said the radar system would be
positioned in such a way that its powerful signals would pose no threat
to human health, an assurance that villagers in Seongju did not accept.

South Koreans are divided over the Thaad system, whose deployment has
been sought for years by the United States but
\href{http://www.nytimes.com/2016/07/09/world/asia/south-korea-us-thaad-china.html}{angrily
opposed by China}, South Korea's top trade partner. China asserts that
it, not the North, is the system's true target, and Russia has joined
Beijing in contending that its deployment would compromise their
security and worsen tensions in the region, making it even more
difficult to persuade North Korea to end its pursuit of nuclear weapons.

On Monday, President Park Geun-hye said that the deployment ``neither
targets third countries nor undermines their security interests.'' But
critics of the government, including many opposition lawmakers, worry
that China will engage in economic retaliation against South Korea and
cooperate less on reining in the North's nuclear ambitions.

``It will do more harm than good to our national interest,'' a prominent
opposition leader, Moon Jae-in, said in a statement on Wednesday. He
also called on the government to submit the deployment for parliamentary
approval.

Under its deal with Washington, South Korea will provide land and build
the base for the Thaad battery, but the United States will pay for the
missile system, to be built by Lockheed Martin, as well as its
operational costs.

Some critics in South Korea found fault with the government's choice of
Seongju as the Thaad site, noting that Seoul, with its 10 million
people, will lie outside the coverage of its intercept missiles, which
have a range of just under 125 miles. The Defense Ministry said it would
operate low-altitude Patriot missile defense systems together with Thaad
to help defend the capital.

On Monday, North Korea
\href{http://www.nytimes.com/2016/07/12/world/asia/north-korea-missile-defense-thaad.html}{threatened}
an unspecified ``physical counteraction'' against the Thaad deployment,
which it said was part of an American plan to build ``an Asian version
of NATO'' to secure military hegemony in the region.

Advertisement

\protect\hyperlink{after-bottom}{Continue reading the main story}

\hypertarget{site-index}{%
\subsection{Site Index}\label{site-index}}

\hypertarget{site-information-navigation}{%
\subsection{Site Information
Navigation}\label{site-information-navigation}}

\begin{itemize}
\tightlist
\item
  \href{https://help.nytimes.com/hc/en-us/articles/115014792127-Copyright-notice}{©~2020~The
  New York Times Company}
\end{itemize}

\begin{itemize}
\tightlist
\item
  \href{https://www.nytco.com/}{NYTCo}
\item
  \href{https://help.nytimes.com/hc/en-us/articles/115015385887-Contact-Us}{Contact
  Us}
\item
  \href{https://www.nytco.com/careers/}{Work with us}
\item
  \href{https://nytmediakit.com/}{Advertise}
\item
  \href{http://www.tbrandstudio.com/}{T Brand Studio}
\item
  \href{https://www.nytimes.com/privacy/cookie-policy\#how-do-i-manage-trackers}{Your
  Ad Choices}
\item
  \href{https://www.nytimes.com/privacy}{Privacy}
\item
  \href{https://help.nytimes.com/hc/en-us/articles/115014893428-Terms-of-service}{Terms
  of Service}
\item
  \href{https://help.nytimes.com/hc/en-us/articles/115014893968-Terms-of-sale}{Terms
  of Sale}
\item
  \href{https://spiderbites.nytimes.com}{Site Map}
\item
  \href{https://help.nytimes.com/hc/en-us}{Help}
\item
  \href{https://www.nytimes.com/subscription?campaignId=37WXW}{Subscriptions}
\end{itemize}
