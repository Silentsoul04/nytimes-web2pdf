Sections

SEARCH

\protect\hyperlink{site-content}{Skip to
content}\protect\hyperlink{site-index}{Skip to site index}

\href{https://www.nytimes.com/section/world/asia}{Asia Pacific}

\href{https://myaccount.nytimes.com/auth/login?response_type=cookie\&client_id=vi}{}

\href{https://www.nytimes.com/section/todayspaper}{Today's Paper}

\href{/section/world/asia}{Asia Pacific}\textbar{}How Killing of
Prominent Separatist Set Off Turmoil in Kashmir

\url{https://nyti.ms/2a4Ajwt}

\begin{itemize}
\item
\item
\item
\item
\item
\end{itemize}

Advertisement

\protect\hyperlink{after-top}{Continue reading the main story}

Supported by

\protect\hyperlink{after-sponsor}{Continue reading the main story}

\hypertarget{how-killing-of-prominent-separatist-set-off-turmoil-in-kashmir}{%
\section{How Killing of Prominent Separatist Set Off Turmoil in
Kashmir}\label{how-killing-of-prominent-separatist-set-off-turmoil-in-kashmir}}

\includegraphics{https://static01.nyt.com/images/2016/07/17/world/17KASHMIR-web1/17KASHMIR-web1-articleLarge.jpg?quality=75\&auto=webp\&disable=upscale}

By Nida Najar

\begin{itemize}
\item
  July 15, 2016
\item
  \begin{itemize}
  \item
  \item
  \item
  \item
  \item
  \end{itemize}
\end{itemize}

NEW DELHI --- Thousands of protesters thronged streets in towns across
the Kashmir valley on July 9, the first of days of clashes with security
officers. The violence was among the worst in the restive region in
years, leaving more than 30 people dead, including a police officer, and
thousands injured. Most of the deaths were protesters shot by security
forces, and hospitals were crowded with wounded civilians.
\href{http://www.nytimes.com/2016/07/11/world/asia/death-toll-kashmir-protests.html?_r=0}{Protesters
attacked} police vehicles, security posts and other government property.

Kashmir is divided between India and Pakistan, and Indian-administered
Kashmir has long been troubled, plagued by the aftershocks of an armed
insurgency born in the late 1980s, which was aided by Pakistan. A heavy
Indian military presence in the Kashmir valley largely vanquished the
insurgency, but calls for self-rule persist, and civilians bristle
against the security forces, which human rights groups accuse of abuses.

\hypertarget{what-set-the-protesters-off}{%
\subsection{What set the protesters
off?}\label{what-set-the-protesters-off}}

The protests began the day after Burhan Muzaffar Wani, a commander for
the Hizbul Mujahedeen, a Kashmiri militant group, was killed by security
forces in a gun battle on July 8. The next day,
\href{http://timesofindia.indiatimes.com/india/Violence-erupts-in-Kashmir-after-death-of-Hizbul-Mujahideen-terrorist-Burhan-Wani-3-killed/articleshow/53130289.cms}{thousands}
attended his funeral in his home village of Tral, in south Kashmir.

\hypertarget{who-was-burhan-muzaffar-wani}{%
\subsection{Who was Burhan Muzaffar
Wani?}\label{who-was-burhan-muzaffar-wani}}

Mr. Wani, in his early 20s, had become a prominent face of separatist
sentiment in the Kashmir valley.

He built up a following on social media, posting pictures of himself and
his associates in combat fatigues, often carrying arms. Though the
numbers of militants in the region has declined sharply since the 1990s,
he became the face of a small, new homegrown militancy based in south
Kashmir, his appeal apparently heightened by his educated, middle-class
roots. His father, Muzaffar Ahmad Wani, the head of a secondary school
in Kashmir,
\href{http://indianexpress.com/article/india/india-others/i-feared-seeing-burhan-dead-never-thought-it-would-be-my-other-son-tral-victims-father/}{told
The Indian Express} that his son's inspiration to join the militancy had
sprung from a beating he and his brother received at the hands of the
security forces in 2010.

His appeal, observers say, spread across social media, reaching a large
number of young people who had known only conflict in Kashmir, and his
death, some fear, will only magnify that appeal.

``Mark my words --- Burhan's ability to recruit in to militancy from the
grave will far outstrip anything he could have done on social media,'' a
former chief minister of Jammu and Kashmir State, Omar Abdullah,
\href{https://twitter.com/abdullah_omar/status/751641720195080192}{posted}
on Twitter after the violence broke out.

\hypertarget{what-makes-this-protest-different}{%
\subsection{What makes this protest
different?}\label{what-makes-this-protest-different}}

What is striking about the recent unrest is the speed and scale with
which it grew --- encompassing almost the entire Kashmir valley and
bringing thousands onto the streets. The authorities called a curfew in
the valley and suspended mobile internet services in response.

Further, the protests were set off by the death of a militant at the
hands of the security forces, not violence against civilians, which has
traditionally prompted protests.

\hypertarget{kashmir-seemed-quiet-what-happened}{%
\subsection{Kashmir seemed quiet. What
happened?}\label{kashmir-seemed-quiet-what-happened}}

The last major civilian uprising that engulfed the Kashmir valley took
place in 2010, after a teenage boy was struck by a tear-gas canister and
killed in Srinagar. Stone-throwing protesters filled the streets for
months, and clashes with security forces left more than 100 people dead.
A series of curfews was put in place across the region.

Since then, isolated protests against security forces have continued,
but in recent months, those protests have changed, say observers and
security officers. Protesters have begun to come out in greater numbers
after the deaths of foreign militants, for example, and have been
appearing at the sites of police shootouts with militants.

Gull Mohammad Wani, a professor of political science at the University
of Kashmir, said the alienation among many Kashmiris had been festering
because of a lack of engagement by state and national politicians in the
long-running political crisis in Kashmir.

A polarized atmosphere in India under the government of the Hindu
nationalist Bharatiya Janata Party in New Delhi, he said, has not
helped. Last year, the Bharatiya Janata Party and the Jammu and Kashmir
Peoples Democratic Party formed a government in Jammu and Kashmir State,
which he called ``a very uneasy type of coalition'' that did little to
calm the region.

\hypertarget{what-do-the-protesters-want}{%
\subsection{What do the protesters
want?}\label{what-do-the-protesters-want}}

There has been a constant cry from protesters in Kashmir throughout the
decades: azadi, or freedom. Since the late 1980s, when the militancy was
born, the azadi cry has been heard on the streets whenever there has
been a protest. Recently, it appears to connote as much a feeling of
rebellion against a security apparatus seen as operating with an
unnecessarily heavy hand, as a concrete demand for nationhood. People
are always ``looking for an opportunity to come out and waiting for the
leadership to announce a call,'' said Khurram Parvez, a human-rights
activist. ``People are waiting for an opportunity to express
themselves.''

Advertisement

\protect\hyperlink{after-bottom}{Continue reading the main story}

\hypertarget{site-index}{%
\subsection{Site Index}\label{site-index}}

\hypertarget{site-information-navigation}{%
\subsection{Site Information
Navigation}\label{site-information-navigation}}

\begin{itemize}
\tightlist
\item
  \href{https://help.nytimes.com/hc/en-us/articles/115014792127-Copyright-notice}{©~2020~The
  New York Times Company}
\end{itemize}

\begin{itemize}
\tightlist
\item
  \href{https://www.nytco.com/}{NYTCo}
\item
  \href{https://help.nytimes.com/hc/en-us/articles/115015385887-Contact-Us}{Contact
  Us}
\item
  \href{https://www.nytco.com/careers/}{Work with us}
\item
  \href{https://nytmediakit.com/}{Advertise}
\item
  \href{http://www.tbrandstudio.com/}{T Brand Studio}
\item
  \href{https://www.nytimes.com/privacy/cookie-policy\#how-do-i-manage-trackers}{Your
  Ad Choices}
\item
  \href{https://www.nytimes.com/privacy}{Privacy}
\item
  \href{https://help.nytimes.com/hc/en-us/articles/115014893428-Terms-of-service}{Terms
  of Service}
\item
  \href{https://help.nytimes.com/hc/en-us/articles/115014893968-Terms-of-sale}{Terms
  of Sale}
\item
  \href{https://spiderbites.nytimes.com}{Site Map}
\item
  \href{https://help.nytimes.com/hc/en-us}{Help}
\item
  \href{https://www.nytimes.com/subscription?campaignId=37WXW}{Subscriptions}
\end{itemize}
