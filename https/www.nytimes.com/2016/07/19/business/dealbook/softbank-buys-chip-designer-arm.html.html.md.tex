Sections

SEARCH

\protect\hyperlink{site-content}{Skip to
content}\protect\hyperlink{site-index}{Skip to site index}

\href{https://myaccount.nytimes.com/auth/login?response_type=cookie\&client_id=vi}{}

\href{https://www.nytimes.com/section/todayspaper}{Today's Paper}

\href{/section/business/dealbook}{DealBook}\textbar{}SoftBank's \$32
Billion Deal for Chip Designer ARM Is Britain's Biggest Since Brexit

\url{https://nyti.ms/2a9OMY1}

\begin{itemize}
\item
\item
\item
\item
\item
\end{itemize}

Advertisement

\protect\hyperlink{after-top}{Continue reading the main story}

Supported by

\protect\hyperlink{after-sponsor}{Continue reading the main story}

DealBook Business and Policy

\hypertarget{softbanks-32-billion-deal-for-chip-designer-arm-is-britains-biggest-since-brexit}{%
\section{SoftBank's \$32 Billion Deal for Chip Designer ARM Is Britain's
Biggest Since
Brexit}\label{softbanks-32-billion-deal-for-chip-designer-arm-is-britains-biggest-since-brexit}}

\includegraphics{https://static01.nyt.com/images/2016/07/19/business/19DBSOFTBANK2-web/19DBSOFTBANK2-web-articleLarge.jpg?quality=75\&auto=webp\&disable=upscale}

By Leslie Picker, Mark Scott and Jonathan Soble

\begin{itemize}
\item
  July 18, 2016
\item
  \begin{itemize}
  \item
  \item
  \item
  \item
  \item
  \end{itemize}
\end{itemize}

When Masayoshi Son, the billionaire Japanese technology investor,
\href{http://www.nytimes.com/2016/06/22/business/dealbook/a-departure-leaves-softbanks-founder-firmly-in-charge.html}{solidified
his control over his SoftBank internet conglomerate} last month, he told
shareholders he still wanted to ``work on a few more crazy ideas.''

One of those ideas materialized on Monday, when SoftBank unveiled an
audacious \$32 billion deal to acquire \href{https://www.arm.com/}{ARM
Holdings}, the British semiconductor designer. The deal --- one of the
biggest of the year --- would give the Japanese company control of a
firm whose chip designs can be found in most of the world's mobile
gadgets, from iPhones and drones to a growing array of smart devices and
appliances for the home.

The deal is the first major cross-border transaction in Britain since it
\href{http://www.nytimes.com/news-event/britain-brexit-european-union}{voted
to exit the European Union} last month. Worries over the impact to the
British economy have weakened the value of the country's currency and
made it cheaper for foreign companies like SoftBank to hunt for deals
there. Compared with this same time in 2015, for example,
pound-denominated assets are 30 percent cheaper for buyers holding yen.

For SoftBank, the deal signals another reinvention, this time with a
major bet on a future filled with interconnected devices. While major
technology companies see a future in smart thermostats and toasters, the
technology has not yet become widely available. At the same time, global
sales of smartphones have slowed, showing the mobile future has limits.

``ARM and SoftBank have an overlap on how we see the future,'' Simon A.
Segars, ARM's chief executive, said in an interview. But he left the
door open for another offer. ``Now that the offer is in the public
domain, if anyone wants to make a counteroffer, they are more than
welcome to do so,'' he said. ``There's always a possibility of someone
counterbidding.''

British leaders, under pressure to address global worries about the
country's future outside the European Union, portrayed the deal as an
endorsement. ``Softbank's decision confirms that Britain remains one of
the most attractive destinations globally for investors to create jobs
and wealth,'' Philip Hammond, the new chancellor of the Exchequer, said
in a statement.

Mr. Son said he was a ``strong believer in the U.K.,'' and he added that
he had spoken with Mr. Hammond and Theresa May, Britain's new prime
minister, about the deal on Sunday.

The deal is the third-largest proposed corporate merger this year,
behind Bayer's offer for Monsanto and a Chinese state-owned company's
proposal for Syngenta, according to the deal-tracking firm Dealogic. If
completed, it would also be the second-largest chip deal on record,
after
\href{http://www.nytimes.com/2015/05/29/business/dealbook/avago-agrees-to-acquire-broadcom-for-37-billion.html}{Avago
Technologies' \$37 billion deal for Broadcom}.

SoftBank already has ties to ARM through Sprint, the American wireless
carrier that it controls. Mr. Son said he first spoke with ARM's
chairman about two weeks ago regarding a possible takeover, and added
that the deal came together quickly. The two sides eventually agreed to
a price --- more than 70 times ARM's net earnings in 2015. The deal is
expected to close in November.

Mr. Son described the deal as a bet on the ``internet of things,'' a new
stage in the evolution of network technology, when cars, buildings and
household items may be connected through embedded electronics. He framed
the social and economic implications in grand terms.

``First there was the internet, then the mobile internet and next there
will be the internet of things, which is going to be the biggest
paradigm shift in human history,'' he said at a news conference. ``I'm
making this investment at the very beginning of this shift.''

ARM Holdings may not be a household name, but it is most likely that one
of the company's chip designs powers your smartphone, tablet or other
mobile device. It devises chips and parts of chips that use less power
so that they can be used in smaller gadgets. ARM had a market
capitalization of about \$22 billion as of Friday's close, and the
proposed acquisition represents a 43 percent premium over the company's
closing share price last week.

Started in 1990 as a spinoff from Acorn Computers, a now-defunct British
computer maker, ARM has gone from a small start-up of fewer than 20
people to a global leader whose technology is used in more than 90
percent of smartphones produced by Apple and Samsung, among others.

ARM took an early lead on chips for mobile devices, while the growing
popularity of smartphones and tablets has been more challenging to
traditional chip makers like Intel.

Unlike Intel, ARM forgoes the high margins --- and equally high
production costs --- of directly manufacturing microchips. Instead, its
engineers design chips, which are then licensed to larger technology
companies like Qualcomm that pay ARM fees and royalties for
manufacturing the chips.

The company's revenue totaled a mere \$1.5 billion last year, compared
with \$55.4 billion for Intel over the same period. But as ARM's chips
have become increasingly powerful, the company's stable of customers
have begun to create devices that directly compete with those powered by
Intel. That can be seen in particular in the world of computer servers,
which have become the lifeblood of the internet as people's online
activities move into the cloud.

As smartphone sales have slowed, ARM has invested millions of dollars in
chip designs aimed at new customers, including automakers and household
product companies, that are looking to add internet connectivity to
their existing products.

Mr. Son said he intended to double the number of employees at ARM in the
next five years, and he said he would make that pledge a legally binding
commitment enforceable by Britain's takeover panel.

SoftBank had been signaling that it was preparing for a major move.

Last month, Mr. Son reasserted control over SoftBank's overseas
investment portfolio, easing out a former Google executive whom he had
been grooming as his successor. In a statement announcing the departure
of the executive, Nikesh Arora, Mr. Son said he had decided to stay on
as SoftBank's chief for at least five or 10 more years.

SoftBank has recently been selling assets and raising cash. Last month,
\href{http://www.nytimes.com/2016/06/22/business/dealbook/tencent-softcell-softbank-deal.html}{it
sealed an agreement} to sell its majority stake in Supercell, the
developer of Clash of Clans and other mobile games, to China's Tencent
Holdings for about \$8.6 billion. It also recently sold about \$10
billion of shares in Alibaba, the Chinese internet giant.

Until now, SoftBank has invested mostly in the services side of the
technology business --- internet companies like Yahoo Japan and Alibaba,
and mobile phone carriers like Sprint and Vodafone, whose Japanese arm
Mr. Son bought in 2006 and turned into one of Japan's dominant carriers.

But abrupt changes in direction are part of SoftBank's DNA. Mr. Son
founded the company in the 1980s as a distributor of computer software.
When he broke into the mobile phone market with the Vodafone purchase in
2006, many predicted disaster --- SoftBank lacked experience in the
industry, and the \$15 billion deal loaded it with debt. But the
business, renamed SoftBank Mobile, soon became a cash cow.

Raine Group, Robey Warshaw and Mizuho Securities advised SoftBank on the
deal, while Lazard and Goldman Sachs advised ARM Holdings.

Advertisement

\protect\hyperlink{after-bottom}{Continue reading the main story}

\hypertarget{site-index}{%
\subsection{Site Index}\label{site-index}}

\hypertarget{site-information-navigation}{%
\subsection{Site Information
Navigation}\label{site-information-navigation}}

\begin{itemize}
\tightlist
\item
  \href{https://help.nytimes.com/hc/en-us/articles/115014792127-Copyright-notice}{©~2020~The
  New York Times Company}
\end{itemize}

\begin{itemize}
\tightlist
\item
  \href{https://www.nytco.com/}{NYTCo}
\item
  \href{https://help.nytimes.com/hc/en-us/articles/115015385887-Contact-Us}{Contact
  Us}
\item
  \href{https://www.nytco.com/careers/}{Work with us}
\item
  \href{https://nytmediakit.com/}{Advertise}
\item
  \href{http://www.tbrandstudio.com/}{T Brand Studio}
\item
  \href{https://www.nytimes.com/privacy/cookie-policy\#how-do-i-manage-trackers}{Your
  Ad Choices}
\item
  \href{https://www.nytimes.com/privacy}{Privacy}
\item
  \href{https://help.nytimes.com/hc/en-us/articles/115014893428-Terms-of-service}{Terms
  of Service}
\item
  \href{https://help.nytimes.com/hc/en-us/articles/115014893968-Terms-of-sale}{Terms
  of Sale}
\item
  \href{https://spiderbites.nytimes.com}{Site Map}
\item
  \href{https://help.nytimes.com/hc/en-us}{Help}
\item
  \href{https://www.nytimes.com/subscription?campaignId=37WXW}{Subscriptions}
\end{itemize}
