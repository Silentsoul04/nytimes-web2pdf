Sections

SEARCH

\protect\hyperlink{site-content}{Skip to
content}\protect\hyperlink{site-index}{Skip to site index}

\href{https://www.nytimes.com/section/world/europe}{Europe}

\href{https://myaccount.nytimes.com/auth/login?response_type=cookie\&client_id=vi}{}

\href{https://www.nytimes.com/section/todayspaper}{Today's Paper}

\href{/section/world/europe}{Europe}\textbar{}Theresa May Wins Vote to
Renew Britain's Nuclear Program

\url{https://nyti.ms/2abikob}

\begin{itemize}
\item
\item
\item
\item
\item
\end{itemize}

Advertisement

\protect\hyperlink{after-top}{Continue reading the main story}

Supported by

\protect\hyperlink{after-sponsor}{Continue reading the main story}

\hypertarget{theresa-may-wins-vote-to-renew-britains-nuclear-program}{%
\section{Theresa May Wins Vote to Renew Britain's Nuclear
Program}\label{theresa-may-wins-vote-to-renew-britains-nuclear-program}}

\includegraphics{https://static01.nyt.com/images/2016/07/19/world/19BRITAIN-PRINT1/19BRITAIN-PRINT1-articleLarge.jpg?quality=75\&auto=webp\&disable=upscale}

By \href{http://www.nytimes.com/by/stephen-castle}{Stephen Castle}

\begin{itemize}
\item
  July 18, 2016
\item
  \begin{itemize}
  \item
  \item
  \item
  \item
  \item
  \end{itemize}
\end{itemize}

LONDON --- Prime Minister Theresa May, in her first major parliamentary
appearance since taking office last week, won a vote on Monday to
authorize and update Britain's nuclear arsenal, a move intended to
underscore the nation's commitment to remaining a global power despite
its recent decision to leave the European Union.

The vote in Parliament on maintaining Britain's nuclear missiles and the
submarines that carry them also gave the new British leader a chance to
highlight the deep divisions in the opposition Labour Party over the
issue, and the relative unity of her own Conservative Party after months
in which the Conservatives were deeply split over whether to leave the
European Union.

But the issue also illustrated the deep strains afflicting Britain after
the ``Brexit'' vote. The Scottish National Party, which dominates
representation of Scotland, fiercely opposes the nuclear system as well
as withdrawal from the European Union, and it has indicated that it
might seek another referendum on Scottish independence,
\href{http://www.nytimes.com/2014/09/19/world/europe/scotland-independence-vote.html}{after
a failed vote in 2014}, if Britain goes through with its departure from
the bloc. Britain's nuclear submarines are based in Scotland, which
complicates the question of how the nation could retain its capacity as
a nuclear deterrent if Scotland were to leave the United Kingdom.

Making her first statement in the House of Commons since becoming prime
minister, Ms. May told lawmakers that it would be an act of ``gross
irresponsibility'' not to replace the nation's aging fleet of
nuclear-armed submarines at a time when threats were increasing.
Lawmakers later supported renewal of the Trident nuclear program by a
vote of 472 to 117.

Speaking before the debate, Michael Fallon, the defense secretary,
acknowledged that Britain would have to try harder to reassure allies of
its foreign policy commitment after British voters ignored calls from
international leaders, including President Obama, to remain in the
European Union.

\includegraphics{https://static01.nyt.com/images/2016/07/19/world/19BRITAIN-PRINT2/19BRITAIN-PRINT2-articleLarge.jpg?quality=75\&auto=webp\&disable=upscale}

``We are still around, and we have to demonstrate that leadership all
over again,'' Mr. Fallon told reporters, citing the weapons vote ---
alongside Britain's military commitment to Afghanistan, the dispatching
of 250 troops to help train forces in Iraq and additional deployments in
Eastern Europe --- as evidence that Britain was ``stepping up, not
stepping back.''

``We will do more in NATO to compensate,'' Mr. Fallon added, naming the
United States, France and Germany as countries with which Britain would
seek to deepen defense cooperation.

Mr. Fallon, who will visit Washington this week, said that while the
United States had not anticipated the referendum's result and was
disappointed, it appreciated that Britons were ``pretty practical
people'' who could find their way around challenges. The Americans, he
added, ``know that this does not mean any kind of retreat from the
world.''

In Parliament, Ms. May told lawmakers that she would be willing to order
the use of the Trident system of submarine-based nuclear missiles if
necessary, and she made a broad defense of the program's renewal. ``The
nuclear threat has not gone away. If anything, it has increased,'' she
said, adding that it was impossible to predict that no extreme threats
will emerge in the next 30 or 40 years, and that ``it would be an act of
gross irresponsibility to lose the ability to meet such threats by
discarding the ultimate insurance against those risks in the future.''

Ms. May also promised to spend at least 2 percent of gross domestic
spending on defense, matching a NATO guideline that Mr. Obama has
pressed the European allies to do more to meet.

The renewal of Britain's continuous, at sea, nuclear deterrent, which
includes four Vanguard-class submarines, comes with a price tag of 31
billion pounds (about \$41 billion), with a further £10 billion (\$13.2
billion) set aside as a contingency --- something Mr. Fallon described
as a ``pretty good investment'' for a nuclear deterrent with a life
expectancy of around three decades. Lawmakers voted in 2007 to go ahead
with a nuclear defense program that extends beyond the 2030s, and
Monday's vote was on proceeding with that program.

Most Conservative Party lawmakers support Trident, and enough deputies
in the Labour Party were expected to vote in favor, or to abstain, for
the measure to pass comfortably. However, Labour's left-wing leader,
Jeremy Corbyn, who
\href{http://www.nytimes.com/2015/10/01/world/europe/jeremy-corbyn-labour-party-leader-says-hed-never-use-nuclear-weapons.html}{has
campaigned for decades} for nuclear disarmament, opposes Trident and on
Monday described it as ``an indiscriminate weapon of mass destruction.''

Image

Protesters opposed to the Trident nuclear program demonstrating on
Monday near the House of Commons in London.Credit...Andy Rain/European
Pressphoto Agency

At the last general election, Labour supported a manifesto that accepted
the deterrent, and Mr. Corbyn has not sought to oblige fellow Labour
lawmakers to side with him in opposing the renewal.

Nevertheless, the Trident issue highlights the continuing crisis over
Mr. Corbyn's leadership. He has refused to quit despite losing a
no-confidence motion
\href{http://www.nytimes.com/2016/06/29/world/europe/jeremy-corbyn-labour-party-brexit.html}{among
his own lawmakers}, the resignations of the majority of his team of
senior aides in Parliament, and a looming leadership challenge.

Mr. Corbyn
\href{http://www.nytimes.com/2015/09/13/world/europe/labour-party-election-jeremy-corbyn.html}{was
elected overwhelmingly} by his party's members and supporters last year,
and he is confident that he will be re-elected. Some observers believe
that his re-election could lead to a split in the Labour Party.

Among the difficulties confronting Mr. Corbyn's opponents are
\href{http://www.nytimes.com/2016/07/05/world/europe/brexit-briefing.html}{two
lawmakers who say} they intend to challenge him: Angela Eagle, a former
spokeswoman on business, and Owen Smith, who spoke on welfare issues
before resigning. That could split the vote against the party leader.
Labour lawmakers agreed on Monday to unite behind one of the candidates.

Like many lawmakers on the left wing of the Labour Party, almost all
Scottish lawmakers voted against the Trident program.

Britain's nuclear submarine fleet is based at Faslane in Scotland, and
the Scottish National Party, which has 54 of the 59 Scottish lawmakers
in Parliament, intended to oppose the program's renewal.

``The S.N.P. will be strong and united in voting against Trident --- in
line with the wishes of a clear majority of Scottish society,'' said
Angus Robertson, the party's leader in Parliament. He argued that, with
annual operational expenses taken into account, the cost of Trident ran
to hundreds of billions of pounds.

Advertisement

\protect\hyperlink{after-bottom}{Continue reading the main story}

\hypertarget{site-index}{%
\subsection{Site Index}\label{site-index}}

\hypertarget{site-information-navigation}{%
\subsection{Site Information
Navigation}\label{site-information-navigation}}

\begin{itemize}
\tightlist
\item
  \href{https://help.nytimes.com/hc/en-us/articles/115014792127-Copyright-notice}{©~2020~The
  New York Times Company}
\end{itemize}

\begin{itemize}
\tightlist
\item
  \href{https://www.nytco.com/}{NYTCo}
\item
  \href{https://help.nytimes.com/hc/en-us/articles/115015385887-Contact-Us}{Contact
  Us}
\item
  \href{https://www.nytco.com/careers/}{Work with us}
\item
  \href{https://nytmediakit.com/}{Advertise}
\item
  \href{http://www.tbrandstudio.com/}{T Brand Studio}
\item
  \href{https://www.nytimes.com/privacy/cookie-policy\#how-do-i-manage-trackers}{Your
  Ad Choices}
\item
  \href{https://www.nytimes.com/privacy}{Privacy}
\item
  \href{https://help.nytimes.com/hc/en-us/articles/115014893428-Terms-of-service}{Terms
  of Service}
\item
  \href{https://help.nytimes.com/hc/en-us/articles/115014893968-Terms-of-sale}{Terms
  of Sale}
\item
  \href{https://spiderbites.nytimes.com}{Site Map}
\item
  \href{https://help.nytimes.com/hc/en-us}{Help}
\item
  \href{https://www.nytimes.com/subscription?campaignId=37WXW}{Subscriptions}
\end{itemize}
