Sections

SEARCH

\protect\hyperlink{site-content}{Skip to
content}\protect\hyperlink{site-index}{Skip to site index}

\href{https://www.nytimes.com/section/politics}{Politics}

\href{https://myaccount.nytimes.com/auth/login?response_type=cookie\&client_id=vi}{}

\href{https://www.nytimes.com/section/todayspaper}{Today's Paper}

\href{/section/politics}{Politics}\textbar{}Donald Trump Sets Conditions
for Defending NATO Allies Against Attack

\url{https://nyti.ms/2ai4u3g}

\begin{itemize}
\item
\item
\item
\item
\item
\end{itemize}

Advertisement

\protect\hyperlink{after-top}{Continue reading the main story}

Supported by

\protect\hyperlink{after-sponsor}{Continue reading the main story}

\hypertarget{donald-trump-sets-conditions-for-defending-nato-allies-against-attack}{%
\section{Donald Trump Sets Conditions for Defending NATO Allies Against
Attack}\label{donald-trump-sets-conditions-for-defending-nato-allies-against-attack}}

\includegraphics{https://static01.nyt.com/images/2016/07/21/us/21trumppolicy-web/21trumppolicy-web-articleLarge.jpg?quality=75\&auto=webp\&disable=upscale}

By \href{http://www.nytimes.com/by/david-e-sanger}{David E. Sanger} and
\href{http://www.nytimes.com/by/maggie-haberman}{Maggie Haberman}

\begin{itemize}
\item
  July 20, 2016
\item
  \begin{itemize}
  \item
  \item
  \item
  \item
  \item
  \end{itemize}
\end{itemize}

CLEVELAND --- Donald J. Trump, on the eve of accepting the
\href{http://www.nytimes.com/2016/07/21/us/politics/republican-national-convention.html}{Republican
nomination} for president, explicitly raised new questions on Wednesday
about his commitment to automatically defending NATO allies if they are
attacked, saying he would first look at their contributions to the
alliance.

Asked about Russia's threatening activities, which have unnerved the
small Baltic States that are among the more recent entrants into NATO,
Mr. Trump said that if Russia attacked them, he would decide whether to
come to their aid only after reviewing if those nations have ``fulfilled
their obligations to us.''

``If they fulfill their obligations to us,'' he added, ``the answer is
yes.''

Mr. Trump's statement appeared to be the first time that a major
candidate for president had suggested conditioning the United States'
defense of its major allies. It was consistent, however, with his
previous threat to withdraw American forces from Europe and Asia if
those allies fail to pay more for American protection.

Mr. Trump also said he would not pressure Turkey or other authoritarian
allies about conducting purges of their political adversaries or
cracking down on civil liberties. The United States, he said, has to
``fix our own mess'' before trying to alter the behavior of other
nations.

``I don't think we have a right to lecture,'' Mr. Trump said in a
\href{http://www.nytimes.com/2016/07/22/us/politics/donald-trump-foreign-policy-interview.html}{wide-ranging
interview} in his suite in a downtown hotel here, while keeping an eye
on television broadcasts from the Republican National Convention. ``Look
at what is happening in our country,'' he said. ``How are we going to
lecture when people are shooting policemen in cold blood?''
\emph{(}\href{http://www.nytimes.com/2016/07/22/us/politics/donald-trump-foreign-policy-interview.html}{\emph{Read
the full transcript}}\emph{.)}

During a 45-minute conversation, Mr. Trump re-emphasized the hard-line
nationalist approach that has marked his improbable candidacy,
describing how he would force allies to shoulder defense costs that the
United States has borne for decades, cancel longstanding treaties he
views as unfavorable, and redefine what it means to be a partner of the
United States.

He said the rest of the world would learn to adjust to his approach. ``I
would prefer to be able to continue'' existing agreements, he said, but
only if allies stopped taking advantage of what he called an era of
American largess that was no longer affordable.

Giving a preview of his address to the convention on Thursday night, he
said that he would press the theme of ``America First,'' his rallying
cry for the past four months, and that he was prepared to scrap the
North American Free Trade Agreement with Mexico and Canada if he could
not negotiate radically better terms.

Within hours of Mr. Trump's remarks calling into question whether, as
president, he would automatically defend NATO allies, European officials
who were already nervous about American commitments appeared a little
stunned by his comments.

\href{https://www.nytimes.com/interactive/2016/07/20/us/elections/gop-convention-speakers.html}{}

\includegraphics{https://static01.nyt.com/images/2016/07/20/us/20live-blog-refer2/20live-blog-refer2-videoLarge-v2.jpg}

\hypertarget{republican-convention-night-3-analysis}{%
\subsection{Republican Convention Night 3:
Analysis}\label{republican-convention-night-3-analysis}}

Here's how we analyzed the third night of the Republican National
Convention, which featured Mike Pence, Ted Cruz and more.

``Solidarity among allies is a key value for NATO,'' Jens Stoltenberg,
NATO's secretary general and a former prime minister of Norway, said in
a statement. He said he did not wish to ``interfere'' with the American
election, but added: ``Two world wars have shown that peace in Europe is
also important for the security of the United States.''

The United States created the 28-nation alliance, and Article 5 of the
NATO treaty, signed by President Truman, requires any member to come to
the aid of another that NATO declares was attacked. It has been invoked
only once: NATO pledged to defend the United States after the Sept. 11,
2001, attacks.

That commitment has long been considered a central element of deterring
attacks in Europe, especially against smaller and weaker nations like
Estonia, Latvia and Lithuania, which joined after the breakup of the
Soviet Union.

The president of Estonia, Toomas Hendrik Ilves, one of the most
pro-American allies in the region, quickly posted on Twitter evidence
that his small country was meeting its defense commitments, and noted it
had contributed to the mission in Afghanistan.

Mr. Trump also said he was pleased that the controversy over
\href{https://www.google.com/url?sa=t\&rct=j\&q=\&esrc=s\&source=web\&cd=9\&cad=rja\&uact=8\&ved=0ahUKEwjwj8XHoYPOAhWIKiYKHfotB60QFgg5MAg\&url=http\%3A\%2F\%2Fwww.nytimes.com\%2F2016\%2F07\%2F20\%2Fus\%2Fpolitics\%2Fmelania-trump-convention-speech.html\&usg=AFQjCNG78WkWxO0yF05A0NlaEtc_JcvJ4g\&sig2=bSNM6FRcYkcX9fq9Zwfkiw\&bvm=bv.127521224,d.eWE}{similarities
between passages} in a speech by his wife, Melania, to the convention on
Monday night and one that Michelle Obama gave eight years ago appeared
to be subsiding. ``In retrospect,'' he said, it would have been better
to explain what had happened --- that
\href{https://www.nytimes.com/2016/07/21/us/politics/melania-trump-speech-meredith-mciver.html}{an
aide had incorporated the comments} --- a day earlier.

When asked what he hoped people would take away from the convention, Mr.
Trump said, ``The fact that I'm very well liked.''

Mr. Trump conceded that his approach to dealing with the United States'
allies and adversaries was radically different from the traditions of
the Republican Party --- whose candidates, since the end of World War
II, have almost all pressed for an internationalist approach in which
the United States is the keeper of the peace, the ``indispensable
nation.''

``This is not 40 years ago,'' Mr. Trump said, rejecting
\href{https://www.nytimes.com/2016/07/19/us/politics/donald-trump-portrayed-as-an-heir-to-richard-nixon.html}{comparisons}
of his approaches to law-and-order issues and global affairs to Richard
Nixon's. Reiterating his threat to pull back United States troops
deployed around the world, he said, ``We are spending a fortune on
military in order to lose \$800 billion,'' citing what he called
America's trade losses. ``That doesn't sound very smart to me.''

Mr. Trump repeatedly defined American global interests almost purely in
economic terms. Its roles as a peacekeeper, as a provider of a nuclear
deterrent against adversaries like North Korea, as an advocate of human
rights and as a guarantor of allies' borders were each quickly reduced
to questions of economic benefit to the United States.

No presidential candidate in modern times has ordered American
priorities that way, and even here, several speakers have called for a
far more interventionist policy, more reminiscent of George W. Bush's
party than of Mr. Trump's.

But Mr. Trump gave no ground, whether the subject was countering North
Korea's missile and nuclear threats or dealing with China in the South
China Sea. The forward deployment of American troops abroad, he said,
while preferable, was not necessary.

``If we decide we have to defend the United States, we can always
deploy'' from American soil, Mr. Trump said, ``and it will be a lot less
expensive.''

Many military experts dispute that view, saying the best place to keep
missile defenses against North Korea is in Japan and the Korean
Peninsula. Maintaining such bases only in the United States can be more
expensive because of the financial support provided by Asian nations.

Mr. Trump's discussion of the crisis in Turkey was telling, because it
unfolded at a moment in which he could plainly imagine himself in the
White House, handling an uprising that could threaten a crucial ally in
the Middle East. The United States has a major air base at Incirlik in
Turkey, where it carries out attacks on the Islamic State and keeps a
force of drones and about 50 nuclear weapons.

Mr. Trump had nothing but praise for President Recep Tayyip Erdogan, the
country's increasingly authoritarian but democratically elected leader.
``I give great credit to him for being able to turn that around,'' Mr.
Trump said of the coup attempt on Friday night. ``Some people say that
it was staged, you know that,'' he said. ``I don't think so.''

Asked if Mr. Erdogan was exploiting the coup attempt to purge his
political enemies, Mr. Trump did not call for the Turkish leader to
observe the rule of law, or Western standards of justice. ``When the
world sees how bad the United States is and we start talking about civil
liberties, I don't think we are a very good messenger,'' he said.

The Obama administration has refrained from any concrete measures to
pressure Turkey, fearing for the stability of a crucial ally in a
volatile region. But Secretary of State John F. Kerry has issued several
statements urging Mr. Erdogan to follow the rule of law.

Mr. Trump offered no such caution for restraint to Turkey and nations
like it. However, his argument about America's moral authority is not a
new one: Russia, China, North Korea and other autocratic nations
frequently cite violence and disorder on American streets to justify
their own practices, and to make the case that the United States has no
standing to criticize them.

\includegraphics{https://static01.nyt.com/images/2016/07/20/us/20live-blog-refer2/20live-blog-refer2-videoSixteenByNine3000-v5.jpg}

Mr. Trump said he was convinced that he could persuade Mr. Erdogan to
put more effort into fighting the Islamic State. But the Obama
administration has run up, daily, against the reality that the Kurds ---
among the most effective forces the United States is supporting against
the Islamic State --- are being attacked by Turkey, which fears they
will create a breakaway nation.

Asked how he would solve that problem, Mr. Trump paused, then said:
``Meetings.''

Ousting President Bashar al-Assad of Syria, he said, was a far lower
priority than fighting the Islamic State --- a conclusion the White
House has also reached, but has not voiced publicly.

``Assad is a bad man,'' Mr. Trump said. ``He has done horrible things.''
But the Islamic State, he said, poses a far greater threat to the United
States.

He said he had consulted two former Republican secretaries of state,
James A. Baker III and Henry Kissinger, saying he had gained ``a lot of
knowledge,'' but did not describe any new ideas about national security
that they had encouraged him to explore.

Mr. Trump emphatically underscored his willingness to drop out of Nafta
unless Mexico and Canada agreed to negotiate new terms that would
discourage American companies from moving manufacturing out of the
United States. ``I would pull out of Nafta in a split second,'' he said.

He talked of funding a major military buildup, starting with a
modernization of America's nuclear arsenal. ``We have a lot of obsolete
weapons,'' he said. ``We have nuclear that we don't even know if it
works.''

The Obama administration has a major modernization program underway,
focused on making the nuclear arsenal more reliable, though it has begun
to confront the huge cost of upgrading bombers and submarines. That
staggering bill, estimated at \$500 billion or more, will land on the
desk of the next president.

Mr. Trump used the ``America First'' slogan in
\href{https://www.nytimes.com/2016/03/27/us/politics/donald-trump-foreign-policy.html?_r=0}{an
earlier interview} with The New York Times, but on Wednesday he insisted
he did not mean it in the way that Charles A. Lindbergh and other
isolationists used it before World War II.

``To me, `America First' is a brand-new, modern term,'' he said. ``I
never related it to the past.''

He paused a moment when asked what it meant to him.

``We are going to take care of this country first,'' he said, ``before
we worry about everyone else in the world.''

Advertisement

\protect\hyperlink{after-bottom}{Continue reading the main story}

\hypertarget{site-index}{%
\subsection{Site Index}\label{site-index}}

\hypertarget{site-information-navigation}{%
\subsection{Site Information
Navigation}\label{site-information-navigation}}

\begin{itemize}
\tightlist
\item
  \href{https://help.nytimes.com/hc/en-us/articles/115014792127-Copyright-notice}{©~2020~The
  New York Times Company}
\end{itemize}

\begin{itemize}
\tightlist
\item
  \href{https://www.nytco.com/}{NYTCo}
\item
  \href{https://help.nytimes.com/hc/en-us/articles/115015385887-Contact-Us}{Contact
  Us}
\item
  \href{https://www.nytco.com/careers/}{Work with us}
\item
  \href{https://nytmediakit.com/}{Advertise}
\item
  \href{http://www.tbrandstudio.com/}{T Brand Studio}
\item
  \href{https://www.nytimes.com/privacy/cookie-policy\#how-do-i-manage-trackers}{Your
  Ad Choices}
\item
  \href{https://www.nytimes.com/privacy}{Privacy}
\item
  \href{https://help.nytimes.com/hc/en-us/articles/115014893428-Terms-of-service}{Terms
  of Service}
\item
  \href{https://help.nytimes.com/hc/en-us/articles/115014893968-Terms-of-sale}{Terms
  of Sale}
\item
  \href{https://spiderbites.nytimes.com}{Site Map}
\item
  \href{https://help.nytimes.com/hc/en-us}{Help}
\item
  \href{https://www.nytimes.com/subscription?campaignId=37WXW}{Subscriptions}
\end{itemize}
