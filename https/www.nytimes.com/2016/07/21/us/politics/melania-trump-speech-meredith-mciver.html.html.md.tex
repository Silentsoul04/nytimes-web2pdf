Sections

SEARCH

\protect\hyperlink{site-content}{Skip to
content}\protect\hyperlink{site-index}{Skip to site index}

\href{https://www.nytimes.com/section/politics}{Politics}

\href{https://myaccount.nytimes.com/auth/login?response_type=cookie\&client_id=vi}{}

\href{https://www.nytimes.com/section/todayspaper}{Today's Paper}

\href{/section/politics}{Politics}\textbar{}Behind Melania Trump's
Cribbed Lines, an Ex-Ballerina Who Loved Writing

\url{https://nyti.ms/2agPuTt}

\begin{itemize}
\item
\item
\item
\item
\item
\item
\end{itemize}

Advertisement

\protect\hyperlink{after-top}{Continue reading the main story}

Supported by

\protect\hyperlink{after-sponsor}{Continue reading the main story}

\hypertarget{behind-melania-trumps-cribbed-lines-an-ex-ballerina-who-loved-writing}{%
\section{Behind Melania Trump's Cribbed Lines, an Ex-Ballerina Who Loved
Writing}\label{behind-melania-trumps-cribbed-lines-an-ex-ballerina-who-loved-writing}}

\includegraphics{https://static01.nyt.com/images/2016/07/21/us/21fd-melania/21fd-melania-articleLarge.jpg?quality=75\&auto=webp\&disable=upscale}

By \href{https://www.nytimes.com/by/jason-horowitz}{Jason Horowitz}

\begin{itemize}
\item
  July 20, 2016
\item
  \begin{itemize}
  \item
  \item
  \item
  \item
  \item
  \item
  \end{itemize}
\end{itemize}

CLEVELAND --- In her mid-30s and slowed by injuries, Meredith McIver, a
classically trained ballerina who had danced under the limelight with
Balanchine and the ensembles of Broadway musicals, decided to pursue her
passion for writing.

She tried her hand at short stories and poems in the style of Dylan
Thomas before finding work writing advertising copy. ``She was always
very, very interested in writing as an art form,'' said an ex-boyfriend,
Stephen Palitz.

He said Ms. McIver brought a dancer's discipline, precision and rigor to
her work. ``She's adept at crystallizing phrases and saying things in an
elegant straightforward way.''

\href{https://www.nytimes.com/interactive/2016/07/20/us/politics/trum-aide-statement.html}{}

\includegraphics{https://static01.nyt.com/images/2016/07/20/us/politics/20micivercrop/20micivercrop-videoLarge.png}

\hypertarget{trump-aides-statement-on-melania-trumps-speech}{%
\subsection{Trump Aide's Statement on Melania Trump's
Speech}\label{trump-aides-statement-on-melania-trumps-speech}}

A longtime employee of the Trump Organization, Meredith McIver, took
responsibility for lifting two passages from a speech by Michelle Obama
in 2008 for Melania Trump's Monday address at the Republican National
Convention, saying that it was an innocent mistake.

This week, Ms. McIver returned to center stage for her writing, but not
in the manner she might have hoped.

``My name is Meredith McIver and I'm an in-house staff writer at the
Trump Organization,'' began an extraordinary statement she released
Wednesday morning in which she took the blame for the disastrous
plagiarism of Michelle Obama in Melania Trump's prime-time speech Monday
at the
\href{http://www.nytimes.com/2016/07/21/us/politics/republican-national-convention.html}{Republican
National Convention}.

In the statement, Ms. McIver, a 65-year-old co-author of several books
with Donald J. Trump, said that as she and Ms. Trump were preparing her
speech, Ms. Trump mentioned that she admired Mrs. Obama and read to Ms.
McIver parts of the first lady's 2008 speech at the Democratic
convention.

Ms. McIver said she had inadvertently left portions of the Obama speech
in the final draft.

``This was my mistake,'' she wrote. She wrote that she had offered her
resignation, but that the Trumps had rejected it. ``Mr. Trump told me
that people make innocent mistakes and that we learn and grow from these
experiences.''

\href{https://www.nytimes.com/interactive/2016/07/20/us/elections/gop-convention-speakers.html}{}

\includegraphics{https://static01.nyt.com/images/2016/07/20/us/20live-blog-refer2/20live-blog-refer2-videoLarge-v2.jpg}

\hypertarget{republican-convention-night-3-analysis}{%
\subsection{Republican Convention Night 3:
Analysis}\label{republican-convention-night-3-analysis}}

Here's how we analyzed the third night of the Republican National
Convention, which featured Mike Pence, Ted Cruz and more.

``I feel terrible for the chaos I have caused Melania and the Trumps, as
well as to Mrs. Obama,'' Ms. McIver wrote. ``No harm was meant.''

But harm was of course done.

After a Twitter user discovered the plagiarism, the story of the cribbed
lines hung over the convention and eclipsed the otherwise positive
response to Ms. Trump's speech.

Her husband's warring advisers pointed fingers at one another. His
family was furious. The campaign chairman said that he believed Ms.
Trump wrote the speech herself, as she asserted, and that it would be
``crazy'' to think she would crib lines when all of America was
watching.

As it turned out, Ms. Trump had torn up an early version of her address
done by two professional Republican speechwriters. Instead, in a
campaign that blurs the lines between family, business and politics, Ms.
Trump reached out to one of the most trusted people inside Trump Tower
for help.

Image

Meredith McIver

Now Ms. McIver, a registered Democrat with no known political
experience, is suddenly at the center of one of the biggest political
stories in the country.

Mr. Palitz, a lawyer who has remained friends with Ms. McIver for
decades, said that knowing her generally meticulous attention to detail,
``it sounds like she sort of stepped up and fell on her sword.''

It was not the first time Ms. McIver was faulted for lines she wrote for
the Trumps.

In a 2007 deposition, Mr. Trump was grilled over whether he had
overstated his debt by billions of dollars in a couple of his co-written
books to make his comeback seem more significant.

He acknowledged the exaggeration, but the mistake, he said, was not his.
``This is somebody that wrote it, probably Meredith McIver,'' Mr. Trump
said.

\includegraphics{https://static01.nyt.com/images/2016/07/19/us/19MELANIA-video/19MELANIA-video-videoSixteenByNine3000.jpg}

The daughter of ballroom dancers, Ms. McIver, who did not respond to
messages seeking comment, grew up in Northern California, before coming
to New York at age 14 on a Ford Foundation scholarship for dance.

She studied at the School of American Ballet, the official school of the
New York City Ballet, from 1965 through 1970. She then went to dance out
west, Mr. Palitz said, and enrolled at the University of Utah. An
English major, she graduated magna cum laude in 1976.

She returned to New York and in 1981 danced in the company of the
revival of ``Can Can'' at the Minskoff Theater in New York. It closed
after five performances. ( ``Mediocre material, no matter how it's
sliced, is still mediocre material,'' The New York Times wrote in its
review.)

She settled on the Upper West Side, and her fashionable dress, dancer's
figure and green eyes turned heads at the grocery. She traveled to the
Netherlands and France. In ``How to Get Rich,'' which she co-wrote with
Mr. Trump, she thanked Alain Bernardin, the owner of a famed Paris
striptease saloon, the Crazy Horse.

\includegraphics{https://static01.nyt.com/images/2016/07/21/us/21mciver-jp-sub/21mciver-jp-sub-articleLarge-v2.jpg?quality=75\&auto=webp\&disable=upscale}

But dancing eventually took its toll, and after writing lyrics with Mr.
Palitz, a classical guitarist, she joined her sister Karen, the art
director at the advertising firm Lotas Minard Patton McIver. Around the
time Karen left the firm more than a decade later, her sister entered
Mr. Trump's orbit.

In 2004's ``How to Get Rich,'' Mr. Trump paid tribute to his co-author,
who worked from a desk outside his office.

``As you know, my door is always open, so Meredith has heard everything,
and she's taken good notes,'' he wrote. ``She's done a remarkable job of
helping me put my thoughts and experiences on paper. I am tremendously
grateful to her.''

And Ms. McIver seemed grateful to Mr. Trump, as well as his future wife.
In 2005's ``Trump: Think Like a Billionaire,'' Ms. McIver, again a
co-author, took the opportunity to acknowledge ``Melania Knauss for her
kind assistance.''

As she had once dreamed, her name appeared on the covers of books, and
she sent copies of them signed by Mr. Trump and inscribed with her own
notes to friends, including Mr. Palitz.

``Meredith was a go-to person for a lot of projects --- I often heard
her name,'' said Adam Eisenstat, who wrote for a blog and online
newsletter under Mr. Trump's name for Trump University in 2005 and 2006.
``Like, `Meredith will take care of it.' ''

Georgina Levitt, an associate publisher at Vanguard Press, which
published a collection of Mr. Trump's essays called ``Think Like a
Champion: An Informal Education In Business and Life'' in 2010, recalled
Ms. McIver --- a voracious reader often seen with a bob haircut,
tailored blazers and red lipstick --- as a helpful liaison to Mr. Trump.

``It seemed like there was a history, an element of trust between
them,'' Ms. Levitt said.

Today, Ms. McIver is considered part of the extended Trump family. ``She
is terrific, she's a terrific woman,'' Mr. Trump said in an interview
Wednesday. ``She's been with us a long time and she just made a
mistake.''

``She came in and she said, `Mr. Trump, I'd like to say what happened.'
I thought it was such a nice thing. Who knew this was going to be a big
story?''

Advertisement

\protect\hyperlink{after-bottom}{Continue reading the main story}

\hypertarget{site-index}{%
\subsection{Site Index}\label{site-index}}

\hypertarget{site-information-navigation}{%
\subsection{Site Information
Navigation}\label{site-information-navigation}}

\begin{itemize}
\tightlist
\item
  \href{https://help.nytimes.com/hc/en-us/articles/115014792127-Copyright-notice}{©~2020~The
  New York Times Company}
\end{itemize}

\begin{itemize}
\tightlist
\item
  \href{https://www.nytco.com/}{NYTCo}
\item
  \href{https://help.nytimes.com/hc/en-us/articles/115015385887-Contact-Us}{Contact
  Us}
\item
  \href{https://www.nytco.com/careers/}{Work with us}
\item
  \href{https://nytmediakit.com/}{Advertise}
\item
  \href{http://www.tbrandstudio.com/}{T Brand Studio}
\item
  \href{https://www.nytimes.com/privacy/cookie-policy\#how-do-i-manage-trackers}{Your
  Ad Choices}
\item
  \href{https://www.nytimes.com/privacy}{Privacy}
\item
  \href{https://help.nytimes.com/hc/en-us/articles/115014893428-Terms-of-service}{Terms
  of Service}
\item
  \href{https://help.nytimes.com/hc/en-us/articles/115014893968-Terms-of-sale}{Terms
  of Sale}
\item
  \href{https://spiderbites.nytimes.com}{Site Map}
\item
  \href{https://help.nytimes.com/hc/en-us}{Help}
\item
  \href{https://www.nytimes.com/subscription?campaignId=37WXW}{Subscriptions}
\end{itemize}
