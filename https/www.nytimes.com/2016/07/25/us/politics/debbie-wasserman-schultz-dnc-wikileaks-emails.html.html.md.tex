Sections

SEARCH

\protect\hyperlink{site-content}{Skip to
content}\protect\hyperlink{site-index}{Skip to site index}

\href{https://www.nytimes.com/section/politics}{Politics}

\href{https://myaccount.nytimes.com/auth/login?response_type=cookie\&client_id=vi}{}

\href{https://www.nytimes.com/section/todayspaper}{Today's Paper}

\href{/section/politics}{Politics}\textbar{}Debbie Wasserman Schultz to
Resign D.N.C. Post

\url{https://nyti.ms/2al9inp}

\begin{itemize}
\item
\item
\item
\item
\item
\item
\end{itemize}

Advertisement

\protect\hyperlink{after-top}{Continue reading the main story}

Supported by

\protect\hyperlink{after-sponsor}{Continue reading the main story}

\hypertarget{debbie-wasserman-schultz-to-resign-dnc-post}{%
\section{Debbie Wasserman Schultz to Resign D.N.C.
Post}\label{debbie-wasserman-schultz-to-resign-dnc-post}}

\includegraphics{https://static01.nyt.com/images/2016/07/25/us/25dems-web2/25dems-web2-articleLarge.jpg?quality=75\&auto=webp\&disable=upscale}

By \href{http://www.nytimes.com/by/jonathan-martin}{Jonathan Martin} and
\href{https://www.nytimes.com/by/alan-rappeport}{Alan Rappeport}

\begin{itemize}
\item
  July 24, 2016
\item
  \begin{itemize}
  \item
  \item
  \item
  \item
  \item
  \item
  \end{itemize}
\end{itemize}

\emph{Follow along with our coverage of the}
\href{http://www.nytimes.com/2016/07/25/us/politics/democratic-national-convention.html}{\emph{Democratic
National Covention}}\emph{.}

PHILADELPHIA ---Democrats arrived at their nominating convention on
Sunday under a cloud of discord as Debbie Wasserman Schultz, the
chairwoman of the Democratic National Committee, abruptly said she was
resigning after a
\href{http://www.nytimes.com/2016/07/23/us/politics/dnc-emails-sanders-clinton.html?ref=politics}{trove
of leaked emails} showed party officials conspiring to sabotage the
campaign of Senator Bernie Sanders of Vermont.

The revelation, along with sizable pro-Sanders protests here in the
streets to greet arriving delegates, threatened to undermine the
delicate healing process that followed the contentious fight between Mr.
Sanders and Hillary Clinton. And it raised the prospect that a
convention that was intended to showcase the Democratic Party's optimism
and unity, in contrast to the Republicans, could be marred by dissension
and disorder.

The day also veered extraordinarily into allegations, not easily
dismissed, that
\href{http://www.nytimes.com/2016/07/25/us/politics/donald-trump-russia-emails.html}{Russia
had a hand in the leaks} that helped bring down the head of an American
political party.

Despite those concerns, Democrats are hoping that focusing on Donald J.
Trump, the Republican nominee, will galvanize the party to rally around
Mrs. Clinton, and on Sunday those efforts received a major boost when
Michael R. Bloomberg, the former Republican and independent mayor of New
York, said he
\href{http://www.nytimes.com/2016/07/25/us/politics/michael-bloomberg-hillary-clinton-dnc.html}{would
endorse her}.

In her resignation statement, Ms. Wasserman Schultz, a representative
from Florida, said she would continue to fight for Mrs. Clinton from the
sidelines.

``I know that electing Hillary Clinton as our next president is critical
for America's future,'' Ms. Wasserman Schultz said in a statement. ``I
look forward to serving as a surrogate for her campaign in Florida and
across the country to ensure her victory.''

She added, ``Going forward, the best way for me to accomplish those
goals is to step down as party chair at the end of this convention.''

Donna Brazile, a vice chairwoman of the Democratic National Committee,
will be the interim chairwoman through the election, the committee said.

Ms. Wasserman Schultz has faced a flurry of negative stories during her
five-year tenure as the committee's chairwoman, with critics charging
that she was more focused on promoting her career than on the party, but
she had resisted calls to quit.

\href{https://www.nytimes.com/interactive/2016/us/elections/polls.html}{}

\includegraphics{https://static01.nyt.com/images/2016/06/15/us/elections/polls-1466014214178/polls-1466014214178-articleLarge-v2.jpg}

\hypertarget{latest-election-polls-2016}{%
\subsection{Latest Election Polls
2016}\label{latest-election-polls-2016}}

Get the latest national and state polls on the presidential election
between Hillary Clinton and Donald J. Trump.

Ms. Wasserman Schultz announced her resignation after a private meeting
with advisers and senior aides to Mrs. Clinton at a hotel here a day
before the party's convention was set to begin. She had faced growing
calls for her resignation over the weekend.

``In politics, you need to not only know when to draw your sword, but
also when to fall on it,'' said James Carville, a longtime friend and
adviser to the Clintons.

The breach of the Democratic committee's emails, made public on Friday
\href{https://wikileaks.com/dnc-emails/}{by WikiLeaks},
\href{http://www.nytimes.com/2016/07/23/us/politics/dnc-emails-sanders-clinton.html}{offered
undeniable evidence} of what Mr. Sanders's supporters had complained
about for much of the senator's contentious primary contest with Mrs.
Clinton: that the party was effectively an arm of Mrs. Clinton's
campaign. The messages showed members of the committee's communications
team musing about pushing the narrative that the Sanders campaign was
inept and trying to raise questions publicly about whether he was an
atheist.

Mr. Sanders said the situation was an ``outrage'' on Sunday before the
resignation was announced, and called for Ms. Wasserman Schultz to step
down. Afterward, he said it was the right decision.

``The party leadership must also always remain impartial in the
presidential nominating process, something which did not occur in the
2016 race,'' he said in a statement.

Mrs. Clinton's campaign aides ignored questions as they quickly left a
hotel a few minutes after the resignation was announced. Ms. Brazile
emerged soon after the Clinton aides had left and said in a brief
interview that Ms. Wasserman Schultz had called her Sunday afternoon and
asked her to come to the hotel where the Florida delegation was staying.

Convention organizers had expressed nervousness on Sunday about the
specter of Ms. Wasserman Schultz appearing onstage at all during the
four-day convention. They were worried that what they intended to be a
well-choreographed event, which officials hoped would contrast with the
sometimes chaotic Republican National Convention, could be marred by Mr.
Sanders's backers booing and heckling her.

Ms. Wasserman Schultz recognized the magnitude of the problem on
Saturday and initially planned to offer an apology, one of her advisers
said. But it became clear to her on Sunday that contrition was
insufficient.

Mrs. Clinton's campaign handled the situation delicately, not wanting
the chairwoman to feel intense pressure and dig in. The Clinton aides
told Ms. Wasserman Schultz the choice to resign was hers to make, but
they gently warned her that she would face jeers from Mr. Sanders's
supporters this week in the convention hall, said the adviser to Ms.
Wasserman Schultz, who requested anonymity to discuss private
deliberations.

\includegraphics{https://static01.nyt.com/images/2016/07/25/us/25dems-jp1/25dems-jp1-articleLarge.jpg?quality=75\&auto=webp\&disable=upscale}

The nudge was enough to force Ms. Wasserman Schultz's hand.

The Clinton campaign also suggested on Sunday that
\href{http://www.nytimes.com/2016/07/25/us/politics/donald-trump-russia-emails.html}{Russia
had been responsible} for the leak as part of an effort to help Mr.
Trump, who has made flattering comments about President Vladimir V.
Putin.

The convention will feature a host of prominent attendees, and Monday
will be headlined by speeches by Mr. Sanders and Michelle Obama, the
first lady. President Obama and former President Bill Clinton will
address the delegates later in the week, bringing the kind of
presidential firepower that the Republican convention could not muster
because of opposition to Mr. Trump.

The unexpected decision by Mr. Bloomberg to endorse Mrs. Clinton
reflected his increasing dismay about the rise of Mr. Trump.

Mr. Sanders's supporters were elated by Ms. Wasserman Schultz's
decision, which they said had been long overdue.

``Thank God for WikiLeaks,'' said Dan O'Neal, a delegate from Arizona
who was wearing a ``Bernie for President'' T-shirt. ``The party was
stacked from the beginning with Debbie in charge.''

Benjamin T. Jealous, a frequent surrogate for Mr. Sanders and a former
president of the N.A.A.C.P., said on Sunday in Philadelphia that the
move ``allows us to heal and move on.''

But many of Mr. Sanders's backers were not ready to let go, and their
frustration could be seen on the streets here. A large, boisterous crowd
of his supporters, chanting ``hell, no, D.N.C., we won't vote for
Hillary,''
\href{http://www.nytimes.com/2016/07/25/us/politics/protests-convention-bernie-sanders-philadelphia.html}{marched
toward the site of the Democratic National Convention} on Sunday
afternoon.

The crowd of more than 1,000 people from as far as Seattle and Florida
was much larger than any of the protest marches last week in Cleveland
during the Republican convention.

Further angering some members of the party's left was
\href{http://www.nytimes.com/2016/07/23/us/politics/tim-kaine-hillary-clinton-vice-president.html?action=click\&contentCollection=Politics\&module=RelatedCoverage\&region=Marginalia\&pgtype=article}{Mrs.
Clinton's selection of Senator Tim Kaine} of Virginia as her running
mate, a move that was widely heralded by establishment Democrats but
that provoked some backlash on Sunday from members of the party who
believed he was not sufficiently liberal.

After the announcement on Friday that Mr. Kaine was joining the ticket,
some die-hard supporters of Mr. Sanders complained that the choice of a
centrist Democrat was evidence that Mrs. Clinton had merely been paying
lip service to the party's progressive wing. Many were hoping that she
would select Senator Elizabeth Warren of Massachusetts or Senator
Sherrod Brown of Ohio if Mr. Sanders were not the choice.

Norman Solomon, a delegate for Mr. Sanders who coordinates the Bernie
Delegates Network, said that 90 percent of his delegates had found Mr.
Kaine to be an ``unacceptable'' choice for vice president. Mr. Solomon's
group arrived early in Philadelphia to make its displeasure known.

``Secretary Clinton must know that her choice of Kaine can only inflame
rather than soothe her relations with the huge constituency of Bernie
supporters,'' he said.

Mr. Sanders was conspicuously quiet after the pick was announced before
breaking his silence on Sunday in a CNN interview. While he spoke highly
of Mr. Kaine, he made clear that they had big policy differences.

``Are his political views different than mine? Yes, they are,'' Mr.
Sanders said. ``He is more conservative than I am.''

Ms. Wasserman Schultz's decision to resign could help tamp down open
dissent by Sanders delegates at the convention. Immediately after her
announcement, top party leaders praised her work in an effort to show
solidarity on the eve of the four-day event.

``I am grateful to Debbie for getting the Democratic Party to this
year's historic convention in Philadelphia, and I know that this week's
events will be a success thanks to her hard work and leadership,'' Mrs.
Clinton said.

Mr. Obama said in a statement, ``For the last eight years, Chairwoman
Debbie Wasserman Schultz has had my back.''

As Democrats digested the news, Republicans gloated.

The Republican National Committee sent out headlines that depicted the
Democratic Party in disarray.

Mr. Trump, who spent much of the day egging on supporters of Mr. Sanders
to revolt after the selection of Mr. Kaine,
\href{https://twitter.com/realDonaldTrump/status/757311921095925760?lang=en}{bragged
on Twitter} that he had always known Ms. Wasserman Schultz was
``overrated.''

``The Dems Convention is cracking up, and Bernie is exhausted, no energy
left!'' Mr. Trump wrote.

The fallout from the resignation could jeopardize Ms. Wasserman
Schultz's seat in Congress, as she faces Tim Canova, a law professor who
supports Mr. Sanders, in a primary election next month.

Advertisement

\protect\hyperlink{after-bottom}{Continue reading the main story}

\hypertarget{site-index}{%
\subsection{Site Index}\label{site-index}}

\hypertarget{site-information-navigation}{%
\subsection{Site Information
Navigation}\label{site-information-navigation}}

\begin{itemize}
\tightlist
\item
  \href{https://help.nytimes.com/hc/en-us/articles/115014792127-Copyright-notice}{©~2020~The
  New York Times Company}
\end{itemize}

\begin{itemize}
\tightlist
\item
  \href{https://www.nytco.com/}{NYTCo}
\item
  \href{https://help.nytimes.com/hc/en-us/articles/115015385887-Contact-Us}{Contact
  Us}
\item
  \href{https://www.nytco.com/careers/}{Work with us}
\item
  \href{https://nytmediakit.com/}{Advertise}
\item
  \href{http://www.tbrandstudio.com/}{T Brand Studio}
\item
  \href{https://www.nytimes.com/privacy/cookie-policy\#how-do-i-manage-trackers}{Your
  Ad Choices}
\item
  \href{https://www.nytimes.com/privacy}{Privacy}
\item
  \href{https://help.nytimes.com/hc/en-us/articles/115014893428-Terms-of-service}{Terms
  of Service}
\item
  \href{https://help.nytimes.com/hc/en-us/articles/115014893968-Terms-of-sale}{Terms
  of Sale}
\item
  \href{https://spiderbites.nytimes.com}{Site Map}
\item
  \href{https://help.nytimes.com/hc/en-us}{Help}
\item
  \href{https://www.nytimes.com/subscription?campaignId=37WXW}{Subscriptions}
\end{itemize}
