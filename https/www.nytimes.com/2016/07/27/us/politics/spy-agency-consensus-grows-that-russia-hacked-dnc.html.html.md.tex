Sections

SEARCH

\protect\hyperlink{site-content}{Skip to
content}\protect\hyperlink{site-index}{Skip to site index}

\href{https://www.nytimes.com/section/politics}{Politics}

\href{https://myaccount.nytimes.com/auth/login?response_type=cookie\&client_id=vi}{}

\href{https://www.nytimes.com/section/todayspaper}{Today's Paper}

\href{/section/politics}{Politics}\textbar{}Spy Agency Consensus Grows
That Russia Hacked D.N.C.

\url{https://nyti.ms/2asxxjG}

\begin{itemize}
\item
\item
\item
\item
\item
\end{itemize}

Advertisement

\protect\hyperlink{after-top}{Continue reading the main story}

Supported by

\protect\hyperlink{after-sponsor}{Continue reading the main story}

\hypertarget{spy-agency-consensus-grows-that-russia-hacked-dnc}{%
\section{Spy Agency Consensus Grows That Russia Hacked
D.N.C.}\label{spy-agency-consensus-grows-that-russia-hacked-dnc}}

\includegraphics{https://static01.nyt.com/images/2016/07/27/us/27cyber-jp/27cyber-jp-articleLarge.jpg?quality=75\&auto=webp\&disable=upscale}

By \href{http://www.nytimes.com/by/david-e-sanger}{David E. Sanger} and
\href{http://www.nytimes.com/by/eric-schmitt}{Eric Schmitt}

\begin{itemize}
\item
  July 26, 2016
\item
  \begin{itemize}
  \item
  \item
  \item
  \item
  \item
  \end{itemize}
\end{itemize}

WASHINGTON --- American intelligence agencies have told the White House
they now have ``high confidence'' that the Russian government was behind
the
\href{https://www.nytimes.com/2016/07/26/us/politics/democrats-allege-dnc-hack-is-part-of-russian-effort-to-elect-donald-trump.html}{theft
of emails and documents} from the Democratic National Committee,
according to federal officials who have been briefed on the evidence.

But intelligence officials have cautioned that they are uncertain
whether the electronic break-in at the committee's computer systems was
intended as fairly routine cyberespionage --- of the kind the United
States also conducts around the world --- or as part of an effort to
manipulate the 2016 presidential election.

The emails were released by WikiLeaks, whose founder, Julian Assange,
has made it clear that he hoped to harm Hillary Clinton's chances of
winning the presidency. It is unclear how the documents made their way
to the group. But a large sampling was published before the WikiLeaks
release by several news organizations and someone who called himself
``Guccifer 2.0,'' who investigators now believe was an agent of the
G.R.U., Russia's military intelligence service.

The assessment by the intelligence community of Russian involvement in
the D.N.C. hacking, which largely
\href{https://www.nytimes.com/2016/06/15/us/politics/russian-hackers-dnc-trump.html}{echoes
the findings} of private cybersecurity firms that have examined the
electronic fingerprints left by the intruders, leaves President Obama
and his national security aides with a difficult diplomatic and
political decision: whether to publicly accuse the government of
President Vladimir V. Putin of engineering the hacking.

Such a public accusation could result in a further deterioration of the
already icy relationship between Washington and Moscow, at a moment when
the administration is trying to reach an accord with Mr. Putin on a
cease-fire in Syria and on other issues. It could also doom any effort
to reach some kind of agreement about acceptable behavior in cyberspace,
of the kind the United States has been discussing with China.

\href{http://www.nbcnews.com/nightly-news/video/president-obama-on-russian-dnc-hack-involvement-anything-s-possible-732675139636}{In
an interview with Savannah Guthrie of NBC News} on Tuesday, President
Obama stopped short of accusing the Russian agencies from seeking to
manipulate the election but said, ``Anything's possible.''

He noted that ``on a regular basis, they try to influence elections in
Europe.''

Stealing information about another country's political infighting is
hardly new, and the United States has conducted covert collection from
allies like Germany and adversaries like Russia for decades. Publishing
the documents --- what some have called ``weaponizing'' them --- is a
different issue. Mrs. Clinton's campaign has suggested that Mr. Putin
was trying to even the score after the former secretary of state
denounced a 2011 Russian election as filled with fraud.

``The first thing that the secretary of state did was say that they were
not honest and not fair, but she had not even yet received the material
from the observers,'' Mr. Putin said at the time. ``She set the tone for
some actors in our country and gave them a signal,'' Mr. Putin
continued. ``They heard the signal and, with the support of the U.S.
State Department, began active work.''

Campaign officials have also suggested that Mr. Putin could be trying to
tilt the election to Donald J. Trump. But they acknowledge that they
have no evidence.

\href{https://www.nytimes.com/interactive/2016/07/27/us/politics/trail-of-dnc-emails-russia-hacking.html}{}

\includegraphics{https://static01.nyt.com/images/2016/07/27/us/politics/trail-of-dnc-emails-russia-hacking-1469656463301/trail-of-dnc-emails-russia-hacking-1469656463301-thumbLarge-v6.png}

\hypertarget{following-the-links-from-russian-hackers-to-the-us-election}{%
\subsection{Following the Links From Russian Hackers to the U.S.
Election}\label{following-the-links-from-russian-hackers-to-the-us-election}}

How U.S. intelligence officials have connected the Russian government to
an attempt to disrupt the 2016 presidential election.

Asked on Tuesday at the Democratic convention in Philadelphia whether
``there's more to the Trump/Russian relationship that hasn't come out,''
John Podesta, the Clinton campaign chairman, said, ``Well, he certainly
has a bromance with Mr. Putin, so I don't know.''

Mr. Podesta said that while Russia had a ``history'' of interfering in
democratic elections in Europe, it would be ``unprecedented in the
United States.''

\href{https://www.nytimes.com/2016/07/19/us/politics/republican-party-issues.html}{The
Republican platform}, adopted last week in Cleveland,
\href{https://www.gop.com/the-2016-republican-party-platform/}{calls on
the United States} to ``respond in kind and in greater magnitude'' to
cyberattacks. ``Russia and China see cyber operations as part of a
warfare strategy during peacetime,'' it says. ``Our response should be
to cause diplomatic, financial and legal pain.''

But the Trump campaign has dismissed the accusations about Russia as a
deliberate distraction, meant to draw attention away from the content of
nearly 20,000 emails and documents from the Democratic committee that
WikiLeaks started releasing on Friday. They showed efforts to
\href{https://www.nytimes.com/2016/07/23/us/politics/dnc-emails-sanders-clinton.html}{impugn
Senator Bernie Sanders} of Vermont in his effort to challenge Mrs.
Clinton for the nomination.

On Twitter Tuesday night, Mr. Trump said that in order to deflect ``the
horror and stupidity of the Wikileakes disaster,'' Democrats were
saying: ``Russia is dealing with Trump. Crazy!''

``For the record,'' he said, ``I have ZERO investments in Russia.''

Mr. Obama, in his interview with NBC News, criticized Mr. Trump
\href{http://www.nytimes.com/2016/07/21/us/politics/donald-trump-issues.html}{for
suggesting last week}that the United States should not defend from
Russian aggression any country that does not contribute sufficient funds
to NATO. He said Mr. Trump was ``undermining'' the country's post-World
War II alliances in Europe.

Secretary of State John Kerry raised the cyberattack with his Russian
counterpart, Sergey V. Lavrov, on Tuesday at a meeting of foreign
ministers in Vientiane, Laos. Mr. Lavrov dismissed the idea that Russia
was involved, telling reporters who asked about the charges: ``I don't
want to use four-letter words.''

Mr. Kerry made no accusations, saying he had to allow the F.B.I. to ``do
its work'' before he drew ``any conclusions in terms of what happened or
who's behind it.''

The federal investigation, involving the F.B.I. and the intelligence
agencies, has been going on since the Democratic National Committee
first called in a private cybersecurity firm, Crowdstrike, in April.

Preliminary conclusions were discussed on Thursday at a weekly
cyberintelligence meeting for senior officials. The Crowdstrike report,
supported by several other firms that have examined the same bits of
code and telltale ``metadata'' left on documents that were released
before WikiLeaks' publication of the larger trove, concludes that the
Federal Security Service, known as the F.S.B., entered the committee's
networks last summer.

\includegraphics{https://static01.nyt.com/images/2017/07/08/us/27TRUMP-PUTIN-COMBO/27TRUMP-PUTIN-COMBO-videoSixteenByNine3000.jpg}

The G.R.U., a competing, military intelligence unit, was a later
arrival. Investigators believe it is the G.R.U. that has played a bigger
role in releasing the emails.

In an
\href{https://www.lawfareblog.com/what-does-us-government-know-about-russia-and-dnc-hack}{essay
published on Lawfare}, a blog that often deals with cyberissues, Susan
Hennessey, previously a lawyer for the National Security Agency, called
the published evidence about Russian involvement ``about as close to a
smoking gun as can be expected when a sophisticated nation-state is
involved.''

Mr. Assange's threat to release documents, she wrote, ``means, put
simply, that actors outside the U.S. are using criminal means to
influence the outcome of a US election. That's a problem.''

But American intelligence agencies have their doubts that the Russian
intention, at least initially, was to sway the American election. The
intrusion began just shortly after Mr. Trump announced his candidacy for
the Republican nomination. At the time, his chances looked minuscule.
One senior official noted that while the cyberattack might have been
intended to embarrass Mrs. Clinton, who was the presumptive nominee, it
could not have been aimed at bolstering Mr. Trump.

It is far from clear that Mr. Obama or the F.B.I. director, James B.
Comey, would ever name Russia as the origin of the hack. Mr. Obama has
only once accused a country of attacking an American organization, when
he said North Korea was the source of the 2014 attack against Sony
Pictures Entertainment. But the United States has no relationship with
North Korea, and there was little to lose from identifying it.

In the case of Russia and China --- countries with which the United
States has complex relationships --- Mr. Obama has in the past made the
opposite decision. He never named the Russian intelligence agencies as
the perpetrators of hacks on the State Department and White House
unclassified email systems, or on the Joint Chiefs of Staff.

While the administration has called out the People's Liberation Army of
China for stealing intellectual property --- even indicting officers of
its now-inactive Unit 61398 --- it never publicly accused the Chinese
intelligence services of stealing the security-clearance files on more
than 21 million Americans who held or applied for clearances.

By happenstance, the intelligence report on the Democratic National
Committee hacking was circulating here the day that Mr. Obama issued a
new policy, long in development, to organize the government's response
to major cyberattacks and to set up a six-point ``grading system'' to
assess the severity of strikes against American companies, government
agencies and organizations.

The action against the Democratic committee, they said, would qualify as
a ``significant cyber incident,'' which was defined as one that causes
``demonstrable harm to the national security interests, foreign
relations or economy of the United States, or to the public confidence,
civil liberties or public health and safety of the American people.''

Ranking the hacking in the pantheon of other penetrated networks is
difficult. The top ranking under Mr. Obama's system would be reserved
for an attack that disabled American power grids, for example, akin to
the suspected Russian attack
\href{http://www.nytimes.com/2016/03/01/us/politics/utilities-cautioned-about-potential-for-a-cyberattack-after-ukraines.html}{on
Ukraine's electrical system} in December. The attack on the Office of
Personnel Management and Sony, which destroyed 70 percent of the
studio's computers, would also rank above the ``category 3'' level,
which defines a ``significant'' attack.

But the ranking system does not mandate what kind of response the
president would authorize. And it was designed before many in Washington
imagined the use of cyberattacks to release information in the midst of
a dizzying, and volatile, presidential campaign.

Advertisement

\protect\hyperlink{after-bottom}{Continue reading the main story}

\hypertarget{site-index}{%
\subsection{Site Index}\label{site-index}}

\hypertarget{site-information-navigation}{%
\subsection{Site Information
Navigation}\label{site-information-navigation}}

\begin{itemize}
\tightlist
\item
  \href{https://help.nytimes.com/hc/en-us/articles/115014792127-Copyright-notice}{©~2020~The
  New York Times Company}
\end{itemize}

\begin{itemize}
\tightlist
\item
  \href{https://www.nytco.com/}{NYTCo}
\item
  \href{https://help.nytimes.com/hc/en-us/articles/115015385887-Contact-Us}{Contact
  Us}
\item
  \href{https://www.nytco.com/careers/}{Work with us}
\item
  \href{https://nytmediakit.com/}{Advertise}
\item
  \href{http://www.tbrandstudio.com/}{T Brand Studio}
\item
  \href{https://www.nytimes.com/privacy/cookie-policy\#how-do-i-manage-trackers}{Your
  Ad Choices}
\item
  \href{https://www.nytimes.com/privacy}{Privacy}
\item
  \href{https://help.nytimes.com/hc/en-us/articles/115014893428-Terms-of-service}{Terms
  of Service}
\item
  \href{https://help.nytimes.com/hc/en-us/articles/115014893968-Terms-of-sale}{Terms
  of Sale}
\item
  \href{https://spiderbites.nytimes.com}{Site Map}
\item
  \href{https://help.nytimes.com/hc/en-us}{Help}
\item
  \href{https://www.nytimes.com/subscription?campaignId=37WXW}{Subscriptions}
\end{itemize}
