Sections

SEARCH

\protect\hyperlink{site-content}{Skip to
content}\protect\hyperlink{site-index}{Skip to site index}

\href{https://www.nytimes.com/section/politics}{Politics}

\href{https://myaccount.nytimes.com/auth/login?response_type=cookie\&client_id=vi}{}

\href{https://www.nytimes.com/section/todayspaper}{Today's Paper}

\href{/section/politics}{Politics}\textbar{}Donald Trump Calls on Russia
to Find Hillary Clinton's Missing Emails

\url{https://nyti.ms/2ausQ98}

\begin{itemize}
\item
\item
\item
\item
\item
\item
\end{itemize}

Advertisement

\protect\hyperlink{after-top}{Continue reading the main story}

Supported by

\protect\hyperlink{after-sponsor}{Continue reading the main story}

\hypertarget{donald-trump-calls-on-russia-to-find-hillary-clintons-missing-emails}{%
\section{Donald Trump Calls on Russia to Find Hillary Clinton's Missing
Emails}\label{donald-trump-calls-on-russia-to-find-hillary-clintons-missing-emails}}

\includegraphics{https://static01.nyt.com/images/2016/07/28/us/28fd-trump-sub/28fd-trump-sub-videoSixteenByNineJumbo1600.jpg}

By \href{http://www.nytimes.com/by/ashley-parker}{Ashley Parker} and
\href{http://www.nytimes.com/by/david-e-sanger}{David E. Sanger}

\begin{itemize}
\item
  July 27, 2016
\item
  \begin{itemize}
  \item
  \item
  \item
  \item
  \item
  \item
  \end{itemize}
\end{itemize}

DORAL, Fla. ---
\href{http://www.nytimes.com/interactive/2016/us/elections/donald-trump-on-the-issues.html?inline=nyt-per}{Donald
J. Trump} said on Wednesday that he hoped Russian intelligence services
had successfully hacked
\href{http://www.nytimes.com/interactive/2016/us/elections/hillary-clinton-on-the-issues.html?inline=nyt-per}{Hillary
Clinton}'s email, and encouraged them to publish whatever they may have
stolen, essentially urging a foreign adversary to conduct cyberespionage
against a former secretary of state.

``Russia, if you're listening, I hope you're able to find the 30,000
emails that are missing,'' Mr. Trump said during a news conference here
in an apparent reference to Mrs. Clinton's deleted emails. ``I think you
will probably be rewarded mightily by our press.''

Mr. Trump's call was another bizarre moment in the mystery of
\href{https://www.nytimes.com/2016/07/27/us/politics/spy-agency-consensus-grows-that-russia-hacked-dnc.html}{whether
Vladimir V. Putin's government} has been seeking to influence the United
States' presidential race.

His comments came amid questions about
\href{http://www.nytimes.com/2016/07/25/us/politics/donald-trump-russia-emails.html?_r=1}{the
hacking} of the Democratic National Committee's computer servers, which
American intelligence agencies have told the White House they have
``high confidence'' was the work of the Russian government.

At the same news conference, Mr. Trump also appeared to leave the door
open to accepting Russia's annexation of Crimea two years ago --- which
the United States and its European allies consider an illegal seizure of
territory. That seizure, and the continued efforts of Russian-aided
insurgents to undermine the government of Ukraine, are the reason that
the United States and its allies still have economic sanctions in force
against Moscow.

\href{https://www.nytimes.com/interactive/2016/07/27/us/elections/dnc-speakers.html}{}

\includegraphics{https://static01.nyt.com/images/2016/07/28/us/elections/28dnc-speakers-8/28dnc-speakers-8-videoLarge.jpg}

\hypertarget{democratic-convention-night-3-analysis}{%
\subsection{Democratic Convention Night 3:
Analysis}\label{democratic-convention-night-3-analysis}}

Our real-time analysis of the third night of the Democratic National
Convention, featuring Joseph R. Biden Jr., Michael Bloomberg, Barack
Obama and Tim Kaine.

When asked whether he would recognize Crimea ``as Russian territory''
and lift the sanctions, Mr. Trump said: ``We'll be looking at that.
Yeah, we'll be looking.''

Mr. Trump's apparent willingness to avoid condemning Mr. Putin's
government is a remarkable departure from United States policy and
Republican Party orthodoxy, and has fueled the questions about Russian
meddling in the campaign. Mr. Trump has denied that, saying at the news
conference that he has never met Mr. Putin, and
\href{https://twitter.com/realDonaldTrump/status/758071952498159616}{has
no investments in Russia}.

``I would treat Vladimir Putin firmly, but there's nothing I can think
of that I'd rather do than have Russia friendly as opposed to the way
they are right now,'' he said, ``so that we can go and knock out ISIS
together.''

Mr. Trump later tried to modify his remarks about hacking Mrs. Clinton's
emails, contending they represented an effort to get the Russians to
turn over their trove to the F.B.I.

With the political conventions coming to an end on Thursday, Mr. Trump
is expected to receive his first national security briefings from
American intelligence agencies in coming days. It is unclear whether
those briefings --- which describe the global challenges facing the
United States but not continuing covert operations or especially
sensitive intelligence --- will change any of his views.

\href{https://www.nytimes.com/interactive/2016/07/27/us/politics/trail-of-dnc-emails-russia-hacking.html}{}

\includegraphics{https://static01.nyt.com/images/2016/07/27/us/politics/trail-of-dnc-emails-russia-hacking-1469656463301/trail-of-dnc-emails-russia-hacking-1469656463301-thumbLarge-v6.png}

\hypertarget{following-the-links-from-russian-hackers-to-the-us-election}{%
\subsection{Following the Links From Russian Hackers to the U.S.
Election}\label{following-the-links-from-russian-hackers-to-the-us-election}}

How U.S. intelligence officials have connected the Russian government to
an attempt to disrupt the 2016 presidential election.

His comments about Russian hacking came on a day when Obama
administration officials were already beginning to develop options for
possible retaliation against Russia for the attack on the Democratic
National Committee. As is often the case after cyber incidents, the
options for responding are limited and can be viewed as seeming too mild
or too escalatory.

The administration has not publicly accused the Russian government of
the Democratic National Committee hacking, or presented evidence to back
up such a case. The leaked documents, first published by a hacker who
called himself ``Guccifer 2.0'' and who is now believed to be a
character created by Russian intelligence, portrayed some committee
officials as
\href{https://www.nytimes.com/2016/07/23/us/politics/dnc-emails-sanders-clinton.html}{favoring
Mrs. Clinton's candidacy} while denigrating her opponent, Senator Bernie
Sanders. The release of the internal party emails and documents
\href{https://www.nytimes.com/2016/07/25/us/politics/debbie-wasserman-schultz-dnc-wikileaks-emails.html}{led
to the resignation} of Debbie Wasserman Schultz as chairwoman of the
party.

Mr. Trump contended on Wednesday that the political uproar over whether
Russia was meddling in the election was a ``total deflection'' from the
embarrassing content of the emails. Many Republicans, even some who say
they do not support Mr. Trump, say they agree.

If Mr. Trump is serious in his call for Russian hacking or exposing Mrs.
Clinton's emails, he would be urging a power often hostile to the United
States to violate American law by breaking into a private computer
network. He would also be contradicting the Republican platform, adopted
last week in Cleveland, saying that cyberespionage ``will not be
tolerated,'' and promising to ``respond in kind and in greater
magnitude'' to all Chinese and Russian cyberattacks.

In the past, the Obama administration has stopped short of retaliating
against Russia --- at least in any public fashion --- for its attacks on
the State Department and White House unclassified email systems, or on
networks used by the Joint Chiefs of Staff. It never even publicly
identified Russian intelligence as the source of those intrusions,
though the subject was widely discussed by senior United States
officials when they were not speaking for attribution.

In contrast, the United States did bring indictments against Chinese and
Iranian hackers for thefts of intellectual property and attacks on
American banks, and imposed economic sanctions against North Korea in
early 2015, for hacking into Sony Pictures Entertainment's computers.

Almost as soon as Mr. Trump spoke, other Republicans raced in to try to
reframe his remarks and argue that Russia should be punished. A
spokesman for Speaker Paul D. Ryan termed Russia ``a global menace led
by a devious thug.'' The spokesman, Brendan Buck, added: ``Putin should
stay out of this election.''

Even Gov. Mike Pence of Indiana, Mr. Trump's running mate, issued a
statement, saying that ``if it is Russia and they are interfering in our
elections, I can assure you both parties and the United States
government will ensure there are serious consequences.'' Mr. Pence did
not attend Wednesday's news conference because he was giving local
television interviews, and an aide to Mr. Pence said that his team had
written his statement about Russia before Mr. Trump began speaking.

Shortly after that Mr. Trump sent a message
\href{https://twitter.com/realDonaldTrump/status/758335147183788032}{on
Twitter declaring} ``If Russia or any other country or person has
Hillary Clinton's 33,000 illegally deleted emails, perhaps they should
share them with the FBI!''

The fact that the Democratic committee's servers were targeted --- and,
apparently, not those of the Republican National Committee --- has
brought up inevitable comparisons with the origins of the Watergate
scandal, when burglars found little after breaking into Democratic
committee offices before the 1972 election. The hackers, more than 40
years later, were more successful: A reconstruction of events suggests
the first successful piercing of the Democrats' networks occurred in
June 2015, long before the Russians, or anyone else, could have known
Mr. Trump would get the nomination.

\includegraphics{https://static01.nyt.com/images/2017/07/08/us/27TRUMP-PUTIN-COMBO/27TRUMP-PUTIN-COMBO-videoSixteenByNine3000.jpg}

The Clinton campaign, eager to turn the subject from the chaos caused by
the email release to the question of Russian interference, accused Mr.
Trump of encouraging Russian espionage.

``This has to be the first time that a major presidential candidate has
actively encouraged a foreign power to conduct espionage against his
political opponent,'' said Jake Sullivan, Mrs. Clinton's chief foreign
policy adviser, whose emails from when he was a State Department aide
were among those that were hacked.

``This has gone from being a matter of curiosity, and a matter of
politics, to being a national security issue,'' he added.

For his part, Mr. Trump cast doubt on the conclusion that Russia was
behind the hacking. ``I have no idea,'' he said. He said the ``sad
thing'' is that ``with the genius we have in government, we don't even
know who took the Democratic National Committee emails.''

Mr. Trump then argued that if Russia, or any other foreign government,
was behind the hacking, it showed just how little respect other nations
had for the current administration.

``President Trump would be so much better for U.S.-Russian relations''
than a President Clinton, Mr. Trump said. ``I don't think Putin has any
respect whatsoever for Clinton.''

Former Representative Pete Hoekstra of Michigan, a Republican who had
served as chairman of the House Intelligence Committee, said Mr. Trump
was right to keep hammering Mrs. Clinton on the subject of her private
emails.

Mr. Hoekstra said he was untroubled by Mr. Trump's goading of a foreign
power, particularly in light of Mrs. Clinton's use of a private server
while she was secretary of state.

''Trump is bringing up a fairly valid point: Hillary Clinton, with her
personal email at the State Department, has put the Russians in a very
enviable position,'' Mr. Hoekstra said. ``Most likely the Russians
already have all that info on Hillary.''

But Representative Jason Chaffetz, a Utah Republican who led the House
oversight committee that looked into Mrs. Clinton's emails, was more
critical. If Mr. Trump's comments were meant literally, he said in an
interview, ``I think he was absolutely wrong and out of line. I would
never have said it that way, and I think it was ill-advised.''

If the remark was tongue-in-cheek, he added, it failed at political
humor.

Advertisement

\protect\hyperlink{after-bottom}{Continue reading the main story}

\hypertarget{site-index}{%
\subsection{Site Index}\label{site-index}}

\hypertarget{site-information-navigation}{%
\subsection{Site Information
Navigation}\label{site-information-navigation}}

\begin{itemize}
\tightlist
\item
  \href{https://help.nytimes.com/hc/en-us/articles/115014792127-Copyright-notice}{©~2020~The
  New York Times Company}
\end{itemize}

\begin{itemize}
\tightlist
\item
  \href{https://www.nytco.com/}{NYTCo}
\item
  \href{https://help.nytimes.com/hc/en-us/articles/115015385887-Contact-Us}{Contact
  Us}
\item
  \href{https://www.nytco.com/careers/}{Work with us}
\item
  \href{https://nytmediakit.com/}{Advertise}
\item
  \href{http://www.tbrandstudio.com/}{T Brand Studio}
\item
  \href{https://www.nytimes.com/privacy/cookie-policy\#how-do-i-manage-trackers}{Your
  Ad Choices}
\item
  \href{https://www.nytimes.com/privacy}{Privacy}
\item
  \href{https://help.nytimes.com/hc/en-us/articles/115014893428-Terms-of-service}{Terms
  of Service}
\item
  \href{https://help.nytimes.com/hc/en-us/articles/115014893968-Terms-of-sale}{Terms
  of Sale}
\item
  \href{https://spiderbites.nytimes.com}{Site Map}
\item
  \href{https://help.nytimes.com/hc/en-us}{Help}
\item
  \href{https://www.nytimes.com/subscription?campaignId=37WXW}{Subscriptions}
\end{itemize}
