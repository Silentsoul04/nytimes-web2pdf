Sections

SEARCH

\protect\hyperlink{site-content}{Skip to
content}\protect\hyperlink{site-index}{Skip to site index}

\href{https://www.nytimes.com/section/politics}{Politics}

\href{https://myaccount.nytimes.com/auth/login?response_type=cookie\&client_id=vi}{}

\href{https://www.nytimes.com/section/todayspaper}{Today's Paper}

\href{/section/politics}{Politics}\textbar{}How Paul Manafort Wielded
Power in Ukraine Before Advising Donald Trump

\url{https://nyti.ms/2aFy026}

\begin{itemize}
\item
\item
\item
\item
\item
\item
\end{itemize}

Advertisement

\protect\hyperlink{after-top}{Continue reading the main story}

Supported by

\protect\hyperlink{after-sponsor}{Continue reading the main story}

\hypertarget{how-paul-manafort-wielded-power-in-ukraine-before-advising-donald-trump}{%
\section{How Paul Manafort Wielded Power in Ukraine Before Advising
Donald
Trump}\label{how-paul-manafort-wielded-power-in-ukraine-before-advising-donald-trump}}

\includegraphics{https://static01.nyt.com/images/2016/08/01/us/01MANAFORT1/01MANAFORT1-articleLarge.jpg?quality=75\&auto=webp\&disable=upscale}

By \href{http://www.nytimes.com/by/steven-lee-myers}{Steven Lee Myers}
and \href{http://www.nytimes.com/by/andrew-e-kramer}{Andrew E. Kramer}

\begin{itemize}
\item
  July 31, 2016
\item
  \begin{itemize}
  \item
  \item
  \item
  \item
  \item
  \item
  \end{itemize}
\end{itemize}

WASHINGTON --- Few political consultants have had a client fail quite as
spectacularly as Paul Manafort's did in Ukraine in the winter of 2014.

President Viktor F. Yanukovych, who owed his election to, as an American
\href{https://wikileaks.org/plusd/cables/06KIEV473_a.html}{diplomat put
it}, an ``extreme makeover'' Mr. Manafort oversaw, bolted the country in
the face of violent street protests. He found sanctuary in Russia and
never returned, as his patron, President Vladimir V. Putin, proceeded to
dismember Ukraine,
\href{http://www.nytimes.com/2014/03/19/world/europe/ukraine.html}{annexing
Crimea} and fomenting a war in two other provinces that continues.

Mr. Manafort was undaunted.

Within months of his client's political demise, he went to work seeking
to bring his disgraced party back to power, much as he had Mr.
Yanukovych himself nearly a decade earlier. Mr. Manafort has already had
some success, with former Yanukovych loyalists --- and some Communists
--- forming a new bloc opposing Ukraine's struggling pro-Western
government.

And now Mr. Manafort has taken on a much larger campaign, seeking to
turn Donald J. Trump into a winning presidential candidate.

With Mr. Putin's Russia, and its interference in Ukraine, becoming a
focus of the United States presidential campaign, Mr. Manafort's work in
Ukraine has come under scrutiny --- along with his business dealings
with prominent Ukrainian and Russian tycoons.

After disclosures of a breach of the Democratic National Committee's
emails --- which American intelligence officials
\href{https://www.nytimes.com/2016/07/27/us/politics/spy-agency-consensus-grows-that-russia-hacked-dnc.html}{have
linked to Russian spies} --- both men are facing sharp criticism over
what is seen as an unusually sympathetic view of Mr. Putin and his
policies toward Ukraine. That view has upended decades of party
orthodoxy toward Russia, a country that the previous Republican
presidential nominee, Mitt Romney, called ``our No. 1 geopolitical
foe.''

On Sunday, Mr. Trump even
\href{https://www.nytimes.com/2016/08/01/world/europe/ukraine-trump-crimea-politics.html}{echoed
Mr. Putin's justification} of the annexation of Crimea, saying the
majority of people in the region wanted to be part of Russia, remarks
that were prominently featured on state news channels in Moscow.

\includegraphics{https://static01.nyt.com/images/2016/08/01/us/01manafort-web2/01manafort-web2-articleLarge.jpg?quality=75\&auto=webp\&disable=upscale}

It is far from certain that Mr. Manafort's views have directly shaped
Mr. Trump's, since Mr. Trump spoke favorably of Mr. Putin's leadership
before Mr. Manafort joined the campaign. But it is clear that the two
have a shared view of Russia and neighbors like Ukraine --- an
affection, even --- that, in Mr. Manafort's case, has been shaped by
years of business dealings as much as by any policy or ideology.

``I wouldn't put out any moral arguments about his work,'' said Yevgeny
E. Kopachko, a pollster with Mr. Yanukovych's former party who
cooperated with Mr. Manafort for years and called him a pragmatic and
effective strategist. ``Nobody has a monopoly on truth and morals.''

Mr. Manafort did not respond to requests for an interview. In television
interviews on Sunday, though, he defended Mr. Trump's views on Russia,
saying that as president, Mr. Trump would be firm with Russia but would
deal with it like any other country when doing so suited American
interests.

``He views Russia as a foreign power that has its own interests at
stake,'' Mr. Manafort said on CBS's ``Face the Nation.''

Until he
\href{https://www.nytimes.com/politics/first-draft/2016/03/28/donald-trump-hires-paul-manafort-to-lead-delegate-effort/?_r=0}{joined
Mr. Trump's presidential campaign} this year, Mr. Manafort's work in
Ukraine had been his most significant political campaign in recent
years. He began his career in Republican politics in the 1970s and
extended it overseas to advising authoritarian leaders, including Mobutu
Sese Seko in Zaire, Ferdinand Marcos in the Philippines and Mr.
Yanukovych.

Mr. Manafort, 67, is the scion of an immigrant family that built a
construction business in Connecticut. A lawyer by education, he served
briefly in the Reagan administration before devoting himself to politics
and later to business. A review of his work in Ukraine shows how
politics and business converged in a country still struggling to
function as a democracy, a quarter of a century after it had gained
independence with the collapse of the Soviet Union. In that world in
flux, Mr. Manafort's political strategy had echoes of Mr. Trump's
populist campaign.

Mr. Manafort's influence in the country was significant, and his
political expertise deeply valued, according to Ukrainian politicians
and officials who worked with him. He also had a voice in decisions
about major American investments in Ukraine, said a former spokesman for
Ukraine's foreign ministry, Oleg Voloshyn, who also ran as a candidate
in the new bloc Mr. Manafort helped form.

Image

The Ukrainian opposition leader Viktor A. Yushchenko and his top ally,
Yulia V. Tymoshenko, right, singing the country's national anthem during
a rally in 2004 at Independence Square in Kiev, the
capital.Credit...Oded Balilty/Associated Press

He persuaded the government to lower grain export tariffs, a change that
benefited agribusiness investors like
\href{http://www.cargill.com/worldwide/ukraine/index.jsp}{Cargill}, and
to open negotiations with Chevron and Exxon for oil and natural gas
exploration in the country.

Mr. Manafort began working in Ukraine after the popular uprising in the
winter of 2004-5 that became known as the Orange Revolution. Mr.
Yanukovych, then prime minister, was declared the winner of a
presidential election in 2004 that was marred by fraud and overturned by
the country's highest court after weeks of protests in favor of his
pro-Western rival, Viktor A. Yushchenko.

Mr. Yanukovych had relied disastrously on Russian political advisers who
underestimated voter frustration. After his defeat, he turned to
American experts.

Mr. Manafort had begun working for one of Ukraine's richest men, Rinat
Akhmetov, to improve the image of his companies. Mr. Akhmetov was also a
prominent sponsor of Mr. Yanukovych's party, the Party of Regions, and
he introduced the two men.

With Mr. Manafort's advice, Mr. Yanukovych began a comeback, with the
Party of Regions winning the biggest bloc in parliamentary elections in
2006 and again in 2007, returning him to the post of prime minister. At
the time, Mr. Manafort called Mr. Yanukovych, a former coal trucking
director who was twice convicted of assault as a young man, an
outstanding leader who had been badly misunderstood in the West.

According to State Department cables at the time and later released by
WikiLeaks, Mr. Manafort and his colleagues Phil Griffin and Catherine
Barnes frequently pressed American diplomats in Ukraine to treat Mr.
Yanukovych and his supporters equally so as not to risk being seen as
favoring his opponents in the new elections. With Mr. Manafort's help,
the party was ``working to change its image from that of a haven for
mobsters into that of a legitimate political party,'' the American
ambassador at the time, John E. Herbst,
\href{https://wikileaks.org/plusd/cables/06KIEV473_a.html}{wrote}.

During this time, lucrative side deals opened for Mr. Manafort.

In 2008, he and the developer
\href{https://www.nytimes.com/2014/08/16/nyregion/arthur-g-cohen-real-estate-developer-is-dead-at-84.html}{Arthur
G. Cohen} negotiated a deal to buy the site of the Drake Hotel on Park
Avenue in Manhattan. One partner was Dmytro Firtash, an oligarch who
made billions as a middleman for Gazprom, the Russian natural gas giant,
and who was known for funneling the money into the campaigns of
pro-Russian politicians in Ukraine, including Mr. Yanukovych. The three
men intended to reopen the site as a mall and spa called Bulgari Tower,
according to a lawsuit filed in Manhattan by Yulia V. Tymoshenko, a
former prime minister of Ukraine. In the end, though, the project
unraveled.

Image

Violent demonstrations in February 2014 in Kiev.Credit...Sergey
Ponomarev for The New York Times

A separate deal also funneled Russian-linked oligarchic money into
Ukraine. In 2007, Mr. Manafort and two partners, Rick Gates and Rick
Davis, set up a private equity company in the Cayman Islands to buy
assets in Ukraine, and invited the Russian oligarch Oleg Deripaska to
invest, according to a court filing. Mr. Deripaska agreed to pay a 2
percent annual management fee to Mr. Manafort and his partners, and put
\$100 million into the fund, which bought a cable television station in
the Black Sea port of Odessa, Ukraine, before the agreement unraveled in
disagreements over auditing and Mr. Deripaska sued Mr. Manafort. The
case is still pending.

By 2010, Mr. Yanukovych's revival was complete. He had won a
presidential campaign against Ms. Tymoshenko, who was convicted of abuse
of office and sent to prison.

Mr. Kopachko, the pollster, said Mr. Manafort envisioned an approach
that exploited regional and ethnic peculiarities in voting, tapping the
disenfranchisement of those who felt abandoned by the Orange Revolution
in eastern Ukraine, which has more ethnic Russians and Russian speakers.

Konstantin Grishchenko, a former foreign minister and a deputy prime
minister under Mr. Yanukovych, said in a telephone interview that Mr.
Manafort had ultimately grown disillusioned with his client.

Mr. Manafort pressed Mr. Yanukovych to sign an agreement with the
European Union that would link the country closer to the West --- and
lobbied for the Americans to support Ukraine's membership, as well,
despite deep reservations because of the prosecution of Ms. Tymoshenko.

Mr. Manafort helped draft a report defending the prosecution that Mr.
Yanukovych's government commissioned from the law firm of Skadden, Arps,
Slate, Meagher \& Flom in 2012.

Mr. Manafort's role was disclosed after a document was discovered in a
box in a sauna belonging to a former senior Ukrainian official. Other
documents in that cache are now evidence in a criminal case against a
former justice official, and could shed more light on Mr. Manafort's
role.

Image

A portrait of former President Viktor F. Yanukovych of Ukraine was moved
in April 2014 at the country's National Art Museum.Credit...Joseph
Sywenkyj for The New York Times

Ultimately Mr. Yanukovych disregarded Mr. Manafort's advice and refused
to sign the trade agreement, which Mr. Putin vehemently opposed. Mr.
Yanukovych's decision led to the protests that culminated in two nights
of violence in February 2014 and Mr. Yanukovych's flight.

Mr. Manafort has said little about Mr. Yanukovych's fall. ``I don't
think he's very happy with the outcome,'' Mr. Grishchenko said.

Mr. Manafort's chance for a comeback, however, came sooner than anyone
had expected.

When the government of President Petro O. Poroshenko called snap
parliamentary elections for October 2014, just eight months later, Mr.
Manafort rallied the dispirited remnants of Mr. Yanukovych's party.

He was now on the payroll of Mr. Yanukovych's former chief of staff,
Serhiy Lyovochkin. Mr. Manafort flew to Ukraine in September 2014 and
set to work rebranding a party deeply fractured by the violence and by
Russia's intervention.

Rather than try to resurrect the disgraced party, he supported pitching
a bigger political tent to help his clients and, he argued, to help
stabilize Ukraine. The new bloc would woo everyone in the country angry
at the new Western-backed government.

It was Mr. Manafort who had argued for a new name for the movement ---
the Opposition Bloc, or Oppo Bloc, as it was called. ``He thought to
gather the largest number of people opposed to the current government,
you needed to avoid anything concrete, and just become a symbol of being
opposed,'' recalled Mikhail B. Pogrebinsky, a political analyst in Kiev.

The strategy worked. Under the new name, the Party of Regions kept a
foothold in Parliament. Its new bloc now has 43 members in the 450-seat
chamber.

It is not clear that Mr. Manafort's work in Ukraine ended with his work
with Mr. Trump's campaign. A communications aide for Mr. Lyovochkin, who
financed Mr. Manafort's work, declined to say whether he was still on
retainer or how much he had been paid.

Mr. Manafort has not registered as a lobbyist representing Ukraine,
which would require disclosing his earnings, though at least one company
he subcontracted, the public relations firm
\href{https://www.fara.gov/docs/3657-Exhibit-AB-20080725-6.pdf}{Edelman},
did in 2008. It received a retainer of \$35,000 a month to promote Mr.
Yanukovych's efforts as prime minister ``toward making Ukraine a more
democratic country.''

Advertisement

\protect\hyperlink{after-bottom}{Continue reading the main story}

\hypertarget{site-index}{%
\subsection{Site Index}\label{site-index}}

\hypertarget{site-information-navigation}{%
\subsection{Site Information
Navigation}\label{site-information-navigation}}

\begin{itemize}
\tightlist
\item
  \href{https://help.nytimes.com/hc/en-us/articles/115014792127-Copyright-notice}{©~2020~The
  New York Times Company}
\end{itemize}

\begin{itemize}
\tightlist
\item
  \href{https://www.nytco.com/}{NYTCo}
\item
  \href{https://help.nytimes.com/hc/en-us/articles/115015385887-Contact-Us}{Contact
  Us}
\item
  \href{https://www.nytco.com/careers/}{Work with us}
\item
  \href{https://nytmediakit.com/}{Advertise}
\item
  \href{http://www.tbrandstudio.com/}{T Brand Studio}
\item
  \href{https://www.nytimes.com/privacy/cookie-policy\#how-do-i-manage-trackers}{Your
  Ad Choices}
\item
  \href{https://www.nytimes.com/privacy}{Privacy}
\item
  \href{https://help.nytimes.com/hc/en-us/articles/115014893428-Terms-of-service}{Terms
  of Service}
\item
  \href{https://help.nytimes.com/hc/en-us/articles/115014893968-Terms-of-sale}{Terms
  of Sale}
\item
  \href{https://spiderbites.nytimes.com}{Site Map}
\item
  \href{https://help.nytimes.com/hc/en-us}{Help}
\item
  \href{https://www.nytimes.com/subscription?campaignId=37WXW}{Subscriptions}
\end{itemize}
