Sections

SEARCH

\protect\hyperlink{site-content}{Skip to
content}\protect\hyperlink{site-index}{Skip to site index}

\href{https://www.nytimes.com/section/world/europe}{Europe}

\href{https://myaccount.nytimes.com/auth/login?response_type=cookie\&client_id=vi}{}

\href{https://www.nytimes.com/section/todayspaper}{Today's Paper}

\href{/section/world/europe}{Europe}\textbar{}Donald Trump Gives
Questionable Explanation of Events in Ukraine

\url{https://nyti.ms/2arMCyV}

\begin{itemize}
\item
\item
\item
\item
\item
\end{itemize}

Advertisement

\protect\hyperlink{after-top}{Continue reading the main story}

Supported by

\protect\hyperlink{after-sponsor}{Continue reading the main story}

\hypertarget{donald-trump-gives-questionable-explanation-of-events-in-ukraine}{%
\section{Donald Trump Gives Questionable Explanation of Events in
Ukraine}\label{donald-trump-gives-questionable-explanation-of-events-in-ukraine}}

\includegraphics{https://static01.nyt.com/images/2016/08/01/us/01ukraine-web/01ukraine-web-articleLarge.jpg?quality=75\&auto=webp\&disable=upscale}

By \href{http://www.nytimes.com/by/david-e-sanger}{David E. Sanger} and
\href{http://www.nytimes.com/by/maggie-haberman}{Maggie Haberman}

\begin{itemize}
\item
  July 31, 2016
\item
  \begin{itemize}
  \item
  \item
  \item
  \item
  \item
  \end{itemize}
\end{itemize}

Donald J. Trump on Sunday offered a muddled explanation of his views
about the 2014
\href{http://www.nytimes.com/2014/03/19/world/europe/ukraine.html}{annexation
of Crimea by Russia} and its continued efforts to undermine Ukraine's
control of other parts of the country, and he amplified his earlier
suggestion that, if elected president, he might recognize Russia's claim
and end sanctions against it.

In an interview with George Stephanopoulos on the ABC News program
``\href{http://abcnews.go.com/ThisWeek}{This Week},'' Mr. Trump said
that if he were president, President Vladimir V. Putin of Russia would
not send his forces into Ukraine. He then backpedaled when Mr.
Stephanopoulos pointed out that Russian troops had been there for nearly
two years.

``He's not going into Ukraine, O.K., just so you understand,'' Mr.
Trump, the Republican nominee, said when the issue came up. ``He's not
going to go into Ukraine, all right? You can mark it down. You can put
it down. You can take it anywhere you want.''

``Well, he's already there, isn't he?'' Mr. Stephanopoulos interrupted.

``O.K., well, he's there in a certain way,'' Mr. Trump replied. ``But
I'm not there. You have Obama there. And frankly, that whole part of the
world is a mess under Obama with all the strength that you're talking
about and all of the power of NATO and all of this. In the meantime,
he's going away. He take --- takes Crimea.''

Interpreting Mr. Trump's statements --- what he understands about the
current status of Ukraine, a former Soviet republic, and how it would
change in a Trump administration --- is difficult given the fractured
nature of the exchange. But they were significant because Mr. Trump has
seemingly embraced Mr. Putin, repeatedly called for better relations
with Russia and shown an unwillingness to condemn Mr. Putin for his
aggressive actions against Russia's neighbors and its crackdowns on
freedoms at home.

Questions have been raised about the watering down of a section of the
\href{https://www.gop.com/the-2016-republican-party-platform/}{Republican
platform} dealing with Ukraine amid evidence that wording to support
sending lethal weapons to the Ukrainian government was removed from the
text.

Not since 1976, when President Gerald Ford committed a major gaffe in
one of his debates with Jimmy Carter, declaring that
``\href{https://www.youtube.com/watch?v=PfyL4uQVJLw}{there is no Soviet
domination of Eastern Europe},'' has the issue of American support of
Eastern European states, both those in NATO and those outside it,
emerged as a major presidential campaign issue. It was enormously
harmful to Mr. Ford, because his statement seemed to suggest that he did
not understand the geopolitics of the region, which his staff denied.

Ukraine
\href{http://www.nytimes.com/1991/12/03/world/ex-communist-wins-in-ukraine-yeltsin-recognizes-independence.html}{became
a separate nation in 1991} after the
\href{http://www.nytimes.com/1989/11/10/world/clamor-east-east-germany-opens-frontier-west-for-migration-travel-thousands.html}{fall
of the Berlin Wall} and the
\href{http://www.nytimes.com/1991/12/24/world/end-of-the-soviet-union-yeltsin-asks-bush-to-grant-russians-recognition-by-us.html}{breakup
of the Soviet Union}. It steadily flirted with the West and with NATO,
and Russian officials feared it would be pulled out of Moscow's orbit.
But a pro-Russian president, Viktor F. Yanukovych, was democratically
elected in 2010 and remained in power until he was
\href{http://www.nytimes.com/2014/02/23/world/europe/with\%2Dpresidents\%2Ddeparture\%2Dukraine\%2Dlooks\%2Dtoward\%2Da\%2Dmurky\%2Dfuture.html}{ousted
in 2014}, ultimately taking up exile in Russia.

Mr. Yanukovych had hired a lobbying firm co-founded by Paul Manafort,
now Mr. Trump's campaign manager, to improve his image in the West and
avoid punishment for veering toward Russia.

The annexation of Crimea in 2014 was seen as both a power grab and a
land grab by Mr. Putin. It was condemned by the United States and its
European allies, which all issued sanctions. Since then, Russian troops,
often out of uniform, have been seen, and sometimes killed, in
Russian-speaking parts of Ukraine where a pro-Russia insurgency has
fought the current Ukrainian government.

Republicans in Congress have long pressed for more assistance to Ukraine
to push back against Mr. Putin, including lethal aid. But all references
to giving lethal aid to the Ukrainian government were kept out of the
party platform.

In early July, a delegate offered a platform amendment to support lethal
aid. A delegate for Senator Ted Cruz from Texas, Diana Denman, said in
an interview that she had pushed for inclusion of the language. But Ms.
Denman said her amendment, as proposed, was never voted on because two
men who were observing the panel's deliberations moved to table the
amendment, and suggested that it be discussed later.

``They openly said they were hired by the Trump campaign and worked for
Mr. Trump,'' Ms. Denman said, adding that she did not recall their
names. In the final version of the Republican platform, the words about
weapons were dropped and replaced by the term ``appropriate
assistance.''

Mr. Trump acknowledged in the ABC interview that the language had been
watered down, but he said he had nothing to do with it. (Mr. Manafort
has also said he was unaware of the matter.)

``I wasn't involved in that,'' Mr. Trump said. ``Honestly, I was not
involved.'' But he acknowledged that his supporters were. ``They
softened it, I heard, but I was not involved,'' he said.

Mr. Trump went on to argue that Mr. Putin might have been welcome in
Crimea, sidestepping the issue of whether the Russian leader had
violated the sovereignty of another state to take the territory, where
Russia has a major naval base.

``The people of Crimea, from what I've heard, would rather be with
Russia than where they were,'' Mr. Trump said. ``And you have to look at
that, also.''

He went on to say, ``Ukraine is a mess,'' but he put the blame for that
on Mr. Obama, not on Mr. Putin.

Jake Sullivan, the chief policy adviser to Hillary Clinton, Mr. Trump's
Democratic opponent, said the assessment reinforced his lack of
temperamental fitness for the presidency.

``Today he gamely repeated Putin's argument that Russia was justified in
seizing the sovereign territory of another country by force,'' Mr.
Sullivan said. ``This is scary stuff. But it shouldn't surprise us.''

Advertisement

\protect\hyperlink{after-bottom}{Continue reading the main story}

\hypertarget{site-index}{%
\subsection{Site Index}\label{site-index}}

\hypertarget{site-information-navigation}{%
\subsection{Site Information
Navigation}\label{site-information-navigation}}

\begin{itemize}
\tightlist
\item
  \href{https://help.nytimes.com/hc/en-us/articles/115014792127-Copyright-notice}{©~2020~The
  New York Times Company}
\end{itemize}

\begin{itemize}
\tightlist
\item
  \href{https://www.nytco.com/}{NYTCo}
\item
  \href{https://help.nytimes.com/hc/en-us/articles/115015385887-Contact-Us}{Contact
  Us}
\item
  \href{https://www.nytco.com/careers/}{Work with us}
\item
  \href{https://nytmediakit.com/}{Advertise}
\item
  \href{http://www.tbrandstudio.com/}{T Brand Studio}
\item
  \href{https://www.nytimes.com/privacy/cookie-policy\#how-do-i-manage-trackers}{Your
  Ad Choices}
\item
  \href{https://www.nytimes.com/privacy}{Privacy}
\item
  \href{https://help.nytimes.com/hc/en-us/articles/115014893428-Terms-of-service}{Terms
  of Service}
\item
  \href{https://help.nytimes.com/hc/en-us/articles/115014893968-Terms-of-sale}{Terms
  of Sale}
\item
  \href{https://spiderbites.nytimes.com}{Site Map}
\item
  \href{https://help.nytimes.com/hc/en-us}{Help}
\item
  \href{https://www.nytimes.com/subscription?campaignId=37WXW}{Subscriptions}
\end{itemize}
