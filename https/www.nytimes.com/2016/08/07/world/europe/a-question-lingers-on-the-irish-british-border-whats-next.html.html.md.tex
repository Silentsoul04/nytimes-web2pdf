Sections

SEARCH

\protect\hyperlink{site-content}{Skip to
content}\protect\hyperlink{site-index}{Skip to site index}

\href{https://www.nytimes.com/section/world/europe}{Europe}

\href{https://myaccount.nytimes.com/auth/login?response_type=cookie\&client_id=vi}{}

\href{https://www.nytimes.com/section/todayspaper}{Today's Paper}

\href{/section/world/europe}{Europe}\textbar{}A Question Lingers on the
Irish Border: What's Next?

\url{https://nyti.ms/2aJniEk}

\begin{itemize}
\item
\item
\item
\item
\item
\end{itemize}

Advertisement

\protect\hyperlink{after-top}{Continue reading the main story}

Supported by

\protect\hyperlink{after-sponsor}{Continue reading the main story}

\hypertarget{a-question-lingers-on-the-irish-border-whats-next}{%
\section{A Question Lingers on the Irish Border: What's
Next?}\label{a-question-lingers-on-the-irish-border-whats-next}}

\includegraphics{https://static01.nyt.com/images/2016/08/07/world/07ireland1/07ireland1-articleLarge.jpg?quality=75\&auto=webp\&disable=upscale}

By \href{http://www.nytimes.com/by/stephen-castle}{Stephen Castle}

\begin{itemize}
\item
  Aug. 6, 2016
\item
  \begin{itemize}
  \item
  \item
  \item
  \item
  \item
  \end{itemize}
\end{itemize}

DUNDALK, Ireland --- Gerard McEvoy's daily commute across the border
between his home in the United Kingdom and his workplace in Ireland
takes just 10 minutes, and the only hint that he is in a different
country is when the road signs change from miles to kilometers.

But that all-but-invisible border between Northern Ireland and Ireland
may end up a victim of
\href{http://www.nytimes.com/news-event/britain-brexit-european-union?8qa}{Britain's
decision in June} to leave the European Union. With Ireland still a
member of the bloc, a new arrangement for the flow of people and goods
will have to be negotiated. And the border is just one challenge in what
is likely to be a much broader redefinition of the relationship between
Britain and Ireland, another example of the sprawling and often
unintended consequences of Britain's choice to split from Europe.

Partly as a result of their shared membership in the European Union,
London and Dublin have largely put aside historical animosities and
developed a tightly woven relationship. Now Britain's exit from the bloc
holds the risk of introducing new friction, from the economy to the
management of Northern Ireland's sectarian tensions to the familial,
social and cultural ties that bind the inhabitants of the islands.

``The border has been out of sight, out of mind --- there's been no
trouble,'' Mr. McEvoy said, standing in his department store in Dundalk
and recalling how he grew up during the
\href{http://www.bbc.co.uk/history/troubles}{Troubles}.

Those were years of strife and violence in Northern Ireland between some
Protestants who wanted to remain part of Britain and some Roman
Catholics who favored unification with Ireland. But Mr. McEvoy added,
``With Europe, you could be as Irish as you wanted to be, you could be
British --- whatever you wanted to be.'' His comment reflected the broad
support for membership in the European Union in Northern Ireland, which
\href{http://www.nytimes.com/interactive/2016/06/24/world/europe/how-britain-voted-brexit-referendum.html}{voted
56 percent to 44 percent} to stay in the bloc.

\includegraphics{https://static01.nyt.com/images/2016/08/07/world/07ireland2/07ireland2-articleLarge.jpg?quality=75\&auto=webp\&disable=upscale}

Four decades of European integration have helped Ireland not only escape
the shadow of Britain, but also improve relations with London and work
with the British for peace in Northern Ireland. Now the question is
whether Britain's departure from the bloc will drive a wedge between
them.

The return of a hard border could affect the fragile peace process, with
Sinn Fein, the dominant party among Catholics in Northern Ireland,
already using the so-called Brexit vote as reason to call for a
referendum on uniting Northern Ireland and Ireland. At the same time,
Irish politicians also worry about the economic stability of Northern
Ireland, which depends heavily on subsidies from London and the European
Union. They wonder whether British taxpayers will pick up the tab for
cuts in European funding.

``I think that it's an enormous moment and potentially a catastrophic
moment in terms of Ireland's narrow interests as well as those of the
wider European Union,'' said Eunan O'Halpin, a professor of contemporary
Irish history at Trinity College Dublin. ``Our neighbors have burned the
house down, and once the edifice collapses, we have to see how we can
fix our walls.''

Ireland's foreign minister, Charles Flanagan, acknowledged the gravity
of the change. Ireland, he said, has become ``a totally different
place'' from the country that joined the forerunner of the European
Union on the same day as Britain
\href{https://europa.eu/european-union/about-eu/history/1970-1979/1973_en}{in
1973}.

Yet even an economically transformed Ireland, he said, cannot escape the
ramifications of a referendum by a large neighbor with whom it shares
centuries of troubled history.

Image

The 100th anniversary of the hanging of Roger Casement, a key figure in
the fight for Irish independence, was commemorated in Dublin on
Wednesday. Ireland remains in the European Union, while Northern
Ireland, as part of Britain, would leave as part of the so-called
Brexit.Credit...Paulo Nunes dos Santos for The New York Times

``There are potential negative impacts across every government
department, from energy to agriculture, the environment,'' Mr. Flanagan
said at his office in Dublin. ``Our job in the negotiations will be to
mitigate these losses and minimize the damage.''

Pat Cox, a former president of the European Parliament, said the
European dimension played a crucial role in strengthening relations
between Dublin and London. Membership made Ireland important in its own
right and allowed it to differentiate itself from Britain, and to pick
and choose policies best suiting its interests.

Like the British, the Irish stayed out of the passport-free Schengen
travel zone, but unlike Britain, Ireland joined the European single
currency. A shared, free-market economic perspective, however, made the
two nations allies in Brussels.

``We got to know each other very well over 40 years by showing up at the
same tables in Brussels and elsewhere, not on the basis of equal size
but on the basis of equality of status of member states, which for a
small state matters,'' Mr. Cox said.

``This radically changed the quality of dialogue between Irish and
British leaders over time,'' he added, while fostering the ``mutual
respect and understanding'' that made the Northern Ireland peace process
possible.

Image

Carlingford Lough, a glacial fjord or sea inlet, forms part of the
border between Northern Ireland and Ireland.Credit...Paulo Nunes dos
Santos for The New York Times

Yet its closeness to Britain leaves Ireland exposed to the consequences,
particularly for trade, of Britain's withdrawal from the European Union.
``In a worst-case scenario with the U.K. outside the E.U., the impact
could be 20 percent or more,'' Ireland's Economic and Social Research
Institute
\href{https://www.esri.ie/publications/scoping-the-possible-economic-implications-of-brexit-on-ireland-2/}{concluded
last year}. It added that a Brexit would be a huge blow since ``more
than 15 percent of Irish exports are destined for the U.K.''

If Britain leaves the European Union's single market, as seems likely to
happen, trading with the British would most likely involve new customs
requirements, and possibly tariffs.

Paddy Malone of Dundalk's chamber of commerce said he worried that
customs checks would reverse the integration around the border area.
``Even if it's only extra paperwork, it's still an administration
burden,'' Mr. Malone said. ``It means that people in both jurisdictions
will start looking elsewhere for new suppliers and customers.''

The future of the border is crucial not just for trade but for the free
movement of people across it. Britain and Ireland enjoyed a common
travel area before joining the European Economic Community in 1973.

Yet Britain's exit creates a situation with one country inside the
European Union and one outside. The road from Dundalk to Newry, Northern
Ireland, will therefore cross an external frontier of the bloc. While
Prime Minister Enda Kenny of Ireland and the new British prime minister,
Theresa May, stress that changes can be kept minimal, that was not what
Mrs. May said before Britain's referendum when she warned of
restrictions in the event of a Brexit.

Image

Tom MacGuinness at the Horseware factory and warehouse in Dundalk. The
Brexit vote has left Ireland even less bound to Britain, he
said.Credit...Paulo Nunes dos Santos for The New York Times

For many in Britain, a prime motivation for the vote was to ``take back
control'' of its borders. If Britain does leave the union, it will be
difficult to avoid immigration checks at the Irish border or between
Northern Ireland and mainland Britain, though the latter would be
controversial in Belfast.

Mr. Flanagan described a heavily fortified border between Ireland and
Northern Ireland, with immigration controls, as ``unthinkable,'' but
said he could not rule out some customs checks. ``If there are to be
some customs points, modern technology could be used to the full in
terms of the identification and checking of goods,'' he added.

While Ireland aims to minimize the impact of the Brexit vote, it is just
one voice among the 27 nations remaining in the union that would shape a
deal with Britain.

There are, however, some opportunities for Ireland. If Britain loses
access to the European Union's single market, some of its banks may
shift their operations to Dublin. Ireland could become more important
diplomatically for the United States because of its voice inside the
European Union.

Tom MacGuinness, the founder of Horseware, which sells equine equipment
around the world, said the referendum result had meant uncertainty for
Irish businesses, but he remains upbeat.

``You really can't plan these things,'' he said at his modern factory
and warehouse in Dundalk. ``You have to roll with the punches. You have
to be nimble.''

The Brexit vote, he said, will leave Ireland less tied to Britain.

``The Irish are no longer mentally, physically or psychologically
bound'' to Britain, he said. ``This will accelerate that process, no
question,'' he added. ``This is going to accelerate the Irish
Europeanization.''

Advertisement

\protect\hyperlink{after-bottom}{Continue reading the main story}

\hypertarget{site-index}{%
\subsection{Site Index}\label{site-index}}

\hypertarget{site-information-navigation}{%
\subsection{Site Information
Navigation}\label{site-information-navigation}}

\begin{itemize}
\tightlist
\item
  \href{https://help.nytimes.com/hc/en-us/articles/115014792127-Copyright-notice}{©~2020~The
  New York Times Company}
\end{itemize}

\begin{itemize}
\tightlist
\item
  \href{https://www.nytco.com/}{NYTCo}
\item
  \href{https://help.nytimes.com/hc/en-us/articles/115015385887-Contact-Us}{Contact
  Us}
\item
  \href{https://www.nytco.com/careers/}{Work with us}
\item
  \href{https://nytmediakit.com/}{Advertise}
\item
  \href{http://www.tbrandstudio.com/}{T Brand Studio}
\item
  \href{https://www.nytimes.com/privacy/cookie-policy\#how-do-i-manage-trackers}{Your
  Ad Choices}
\item
  \href{https://www.nytimes.com/privacy}{Privacy}
\item
  \href{https://help.nytimes.com/hc/en-us/articles/115014893428-Terms-of-service}{Terms
  of Service}
\item
  \href{https://help.nytimes.com/hc/en-us/articles/115014893968-Terms-of-sale}{Terms
  of Sale}
\item
  \href{https://spiderbites.nytimes.com}{Site Map}
\item
  \href{https://help.nytimes.com/hc/en-us}{Help}
\item
  \href{https://www.nytimes.com/subscription?campaignId=37WXW}{Subscriptions}
\end{itemize}
