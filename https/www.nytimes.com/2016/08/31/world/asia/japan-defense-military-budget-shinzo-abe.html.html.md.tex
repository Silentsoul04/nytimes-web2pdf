Sections

SEARCH

\protect\hyperlink{site-content}{Skip to
content}\protect\hyperlink{site-index}{Skip to site index}

\href{https://www.nytimes.com/section/world/asia}{Asia Pacific}

\href{https://myaccount.nytimes.com/auth/login?response_type=cookie\&client_id=vi}{}

\href{https://www.nytimes.com/section/todayspaper}{Today's Paper}

\href{/section/world/asia}{Asia Pacific}\textbar{}Japanese Government
Urges Another Increase in Military Spending

\url{https://nyti.ms/2bZ57yJ}

\begin{itemize}
\item
\item
\item
\item
\item
\end{itemize}

Advertisement

\protect\hyperlink{after-top}{Continue reading the main story}

Supported by

\protect\hyperlink{after-sponsor}{Continue reading the main story}

\hypertarget{japanese-government-urges-another-increase-in-military-spending}{%
\section{Japanese Government Urges Another Increase in Military
Spending}\label{japanese-government-urges-another-increase-in-military-spending}}

\includegraphics{https://static01.nyt.com/images/2016/08/31/world/31Japan-web1/31Japan-web1-articleInline.jpg?quality=75\&auto=webp\&disable=upscale}

By \href{http://www.nytimes.com/by/motoko-rich}{Motoko Rich}

\begin{itemize}
\item
  Aug. 30, 2016
\item
  \begin{itemize}
  \item
  \item
  \item
  \item
  \item
  \end{itemize}
\end{itemize}

TOKYO --- The government of Prime Minister Shinzo Abe is requesting
another increase in spending on Japan's armed forces, with a plan to
expand missile defenses that would test the nation's commitment to
pacifism and escalate a regional arms race with China and North Korea.

With rising threats from North Korea's nuclear and ballistic missile
program and repeated incursions by Chinese ships into waters surrounding
a string of islands claimed by Japan, the request would let the Defense
Ministry develop new antiballistic missiles and place troops on southern
islands closer to the chain in dispute with China.

If approved, the budget proposal for 5.17 trillion yen, or \$50.2
billion, formally submitted on Wednesday, would be the nation's
fifth-straight annual increase in military spending. It is a 2.3 percent
rise over last year.

The request includes proposals to develop and potentially purchase new
antiballistic missiles that can be launched from ships or land, and to
upgrade and extend the range of the country's current land-based missile
defense systems, a significant expansion of Japan's missile defense
capabilities.

The budget also details plans to buy an additional submarine and new
fighter aircraft, and to put close to 1,300 soldiers from the
Self-Defense Force, Japan's military, on the southern islands of
Kagoshima and Okinawa. These locations are closer to the Senkaku, the
chain of islands where both China and Japan claim territorial rights.

Despite Japan's longstanding postwar pacifism, initially imposed by a
Constitution that was largely written by American occupiers, the country
has long argued that the Constitution does not prevent it from
maintaining defensive equipment and troops.

But the definition of what is needed to defend the country has evolved
as Japan confronts new dangers. Throughout the 1990s and early 2000s,
government assessments of security in the region led to a decrease in
defense budgets every year.

Yet five years ago, the government began increasing its budget again as
new provocations emerged from China and North Korea.

The budget deliberations come as Mr. Abe's government is reconsidering
the country's pacifist stance. Mr. Abe has long expressed his interest
in revising the clause in the
\href{http://japan.kantei.go.jp/constitution_and_government_of_japan/constitution_e.html}{Constitution}
that says the country must ``forever renounce war,'' and he helped push
through new security laws last year that permit Japan's troops to
participate in overseas combat missions.

\includegraphics{https://static01.nyt.com/images/2016/08/31/world/31Japan-web2/31Japan-web2-articleInline.jpg?quality=75\&auto=webp\&disable=upscale}

A majority of the Japanese public generally opposes amending the
pacifist Constitution;
\href{http://www.nytimes.com/2015/07/17/world/asia/japans-lower-house-passes-bills-giving-military-freer-hand-to-fight.html}{protesters
mounted large demonstrations against the security bills last year}. Yet
some Japanese consider the gradual buildup of military firepower
necessary for their protection.

North Korea continues to develop its nuclear capabilities and test-fire
ballistic missiles that land ever closer to Japan. Just last week,
\href{http://www.nytimes.com/2016/08/25/world/asia/japan-china-korea-missile-test.html}{North
Korea launched a missile} from a submarine off its east coast that flew
310 miles toward Japan, much farther than in previous attempts. By
extending the range of some antiballistic missile systems, the Japanese
would be better equipped to shoot down missiles launched by Pyongyang.

Japan's current land-based missile defense systems have a medium range
for intercepting incoming ballistic missiles. By expanding that range,
the new systems should be able to shoot down missiles before they get so
close.

As for the Chinese, their vessels have repeatedly sailed into disputed
waters surrounding a group of uninhabited Japanese-controlled islands in
the East China Sea, known as the Senkaku to Japan and the Diaoyu to
China. In June,
\href{http://www.nytimes.com/2016/06/10/world/asia/japan-china-navy-protest.html}{China
sent a warship within 24 nautical miles of the islands}; Mr. Abe
responded by putting the Japanese Navy and coast guard on alert.

Japan's defense budget proposal includes funds to help proceed with
development, in conjunction with the United States, of advanced
antiballistic missiles that can be launched from ships and that have
much longer ranges than previous incarnations.

Experts said these missiles could be used not only to shoot down North
Korean missiles, but also to deter China from invading the disputed
islands. Placing more troops on the southern islands of Japan is also
intended to deter China from moving closer to the Senkaku.

``We're in the middle of what is commonly called the security dilemma,''
said
\href{http://web.mit.edu/polisci/people/faculty/richard-samuels.html}{Richard
Samuels}, a Japan specialist and the director of the Center for
International Studies at the Massachusetts Institute of Technology.

``When one nation does something which it believes to be defensive and
in its own interests, its competitor will see it as threatening and see
it as offensive, and then you get this arms race and security dilemma,''
he said. ``That's very much in play here.''

The Defense Ministry's budget request must be reviewed by the Finance
Ministry and approved by Parliament before any purchases can be made.

Analysts said nothing in the new budget request suggested that Japan
would cross the line from a primarily defensive stance to a more
offensive one.

``If they started to procure long-range bombers or intercontinental
ballistic missiles, those would be the things where I would say, `Now we
are seeing something radically different,''' said Jeffrey Hornung, a
research fellow for security and foreign policy at Sasakawa USA, a think
tank in Washington.

The new equipment proposals also seem carefully calibrated to address
current threats. The plan to extend the range of existing PAC-3 missile
defense systems from the current limit of about 19 miles, for example,
would help Japan protect against North Korean missiles but avoid the
appearance of instigating new confrontations, analysts said.

``I think these ranges are very carefully selected,'' said Bonnie S.
Glaser, senior Asia adviser at the Center for Strategic and
International Studies in Washington. She noted that Japan would, for
instance, be aware of China's objection to any hint that Japan might get
involved in disputes over Taiwan. The distances of the missiles
proposed, she said, would not extend to Taiwan.

Amid
\href{http://www.nytimes.com/2016/06/05/world/asia/okinawa-murder-case-heightens-outcry-over-us-militarys-presence.html}{controversy}
over Japan's continued hosting of American bases and troops on the
island of Okinawa, the current budget proposal also includes a request
for a slight increase in spending on American operations to 178.7
billion yen.

All told, the budget request remains less than 1 percent of Japan's
gross domestic product, a self-imposed constraint that few Japanese
administrations have breached.

Some analysts noted that with China rapidly increasing its military
budget, Japan's current military spending might not be sufficient. ``In
the long run, if the military balance in East Asia shifts in favor of
China significantly, we might have to do much more than what we are
doing right now,'' said Narushige Michishita, director of the Security
and International Studies Program at the National Graduate Institute for
Policy Studies in Tokyo.

Tooru Miyamoto, a Communist Party member of the House of
Representatives, said he did not approve of the increased expenditures
at a time when the economy continues to stagnate. ``I want such money to
be spent on day care centers,'' he said.

At an annual review staged by the Ground Self-Defense Force in the
foothills of Mount Fuji last weekend, 25,000 spectators gathered to
watch a parade of tanks, helicopters and other armored vehicles, with
soldiers detonating artillery against artificial targets.

In one segment described as a demonstration of how troops would respond
to an attack on unspecified islands, soldiers dropped from Chinook
helicopters and tanks rolled across a muddy field.

Naoko Matsumaru, 42, who works in a flour mill, attended the drills with
her young daughter and son.

She said that she had been concerned about threats from North Korea and
China, but that ``after seeing today's show, I feel maybe we are
actually O.K.''

Those who value Japan's pacifism said they were concerned about the
expanded military role.

``In these times, I am a little bit worried,'' said Toru Matsuzaki, 71,
a woodworker. He referred to a generation of ``heiwa boke,'' people who
innocently take peace for granted. ``Realistically, it may be necessary
to increase the budget,'' he said, ``but I don't like it.''

Advertisement

\protect\hyperlink{after-bottom}{Continue reading the main story}

\hypertarget{site-index}{%
\subsection{Site Index}\label{site-index}}

\hypertarget{site-information-navigation}{%
\subsection{Site Information
Navigation}\label{site-information-navigation}}

\begin{itemize}
\tightlist
\item
  \href{https://help.nytimes.com/hc/en-us/articles/115014792127-Copyright-notice}{©~2020~The
  New York Times Company}
\end{itemize}

\begin{itemize}
\tightlist
\item
  \href{https://www.nytco.com/}{NYTCo}
\item
  \href{https://help.nytimes.com/hc/en-us/articles/115015385887-Contact-Us}{Contact
  Us}
\item
  \href{https://www.nytco.com/careers/}{Work with us}
\item
  \href{https://nytmediakit.com/}{Advertise}
\item
  \href{http://www.tbrandstudio.com/}{T Brand Studio}
\item
  \href{https://www.nytimes.com/privacy/cookie-policy\#how-do-i-manage-trackers}{Your
  Ad Choices}
\item
  \href{https://www.nytimes.com/privacy}{Privacy}
\item
  \href{https://help.nytimes.com/hc/en-us/articles/115014893428-Terms-of-service}{Terms
  of Service}
\item
  \href{https://help.nytimes.com/hc/en-us/articles/115014893968-Terms-of-sale}{Terms
  of Sale}
\item
  \href{https://spiderbites.nytimes.com}{Site Map}
\item
  \href{https://help.nytimes.com/hc/en-us}{Help}
\item
  \href{https://www.nytimes.com/subscription?campaignId=37WXW}{Subscriptions}
\end{itemize}
