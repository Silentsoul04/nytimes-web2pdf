Sections

SEARCH

\protect\hyperlink{site-content}{Skip to
content}\protect\hyperlink{site-index}{Skip to site index}

\href{https://www.nytimes.com/section/politics}{Politics}

\href{https://myaccount.nytimes.com/auth/login?response_type=cookie\&client_id=vi}{}

\href{https://www.nytimes.com/section/todayspaper}{Today's Paper}

\href{/section/politics}{Politics}\textbar{}Donald Trump Gambles on
Immigration but Sends Conflicting Signals

\url{https://nyti.ms/2c1z8O1}

\begin{itemize}
\item
\item
\item
\item
\item
\item
\end{itemize}

Advertisement

\protect\hyperlink{after-top}{Continue reading the main story}

Supported by

\protect\hyperlink{after-sponsor}{Continue reading the main story}

\hypertarget{donald-trump-gambles-on-immigration-but-sends-conflicting-signals}{%
\section{Donald Trump Gambles on Immigration but Sends Conflicting
Signals}\label{donald-trump-gambles-on-immigration-but-sends-conflicting-signals}}

\includegraphics{https://static01.nyt.com/images/2016/09/01/us/01trump3/01trump3-articleInline.jpg?quality=75\&auto=webp\&disable=upscale}

By \href{http://www.nytimes.com/by/patrick-healy}{Patrick Healy}

\begin{itemize}
\item
  Aug. 31, 2016
\item
  \begin{itemize}
  \item
  \item
  \item
  \item
  \item
  \item
  \end{itemize}
\end{itemize}

Donald J. Trump made an audacious attempt on Wednesday to remake his
image on the divisive issue of immigration, shelving his plan to deport
11 million undocumented people and arguing that a Trump administration
and Mexico would secure the border together.

In a spirited bid for undecided American voters to see him anew, Mr.
Trump swept into Mexico City to make overtures to a nation he has
repeatedly denigrated, then flew to Phoenix to outline in his usual
bullying tone his latest priorities on immigration.

Yet the juxtaposition of Mr. Trump's dual performances was so jarring
that his true vision and intentions on immigration were hard to discern.
He displayed an almost unrecognizable demeanor during his afternoon in
Mexico, appearing measured and diplomatic, while hours later he took the
stage at his campaign rally and denounced illegal immigrants on the
whole as a criminally minded and dangerous group that sows terror in
communities and commits murders, rapes and other heinous violence.

Mr. Trump's mixed messages on whom he would deport and when, and how the
government would go about removing people from the country, were further
muddled by the incendiary language in the Phoenix speech --- a
deliberate effort by campaign advisers to draw attention to his
criticism of illegal immigrants rather than the specifics of his plan.

\includegraphics{https://static01.nyt.com/images/2016/08/31/multimedia/01trump-immigration/01trump-immigration-videoSixteenByNineJumbo1600.jpg}

In his speech, Mr. Trump fervently tried to depict himself as an ally of
average workers, saying their economic interests were far more important
than the needs of undocumented workers. But he left unclear what would
happen to those millions of illegal immigrants, saying only that ``the
appropriate disposition of those individuals'' will take place at some
future date after the criminals are deported and his border wall is
built.

Deporting all illegal immigrants had been his signature political issue
for much of the presidential race, but his caustic tone and harsh
approach has turned off many Republicans and independents, particularly
women. His language was still fiery in Phoenix, yet he also said that
the fate of most illegal immigrants would be handled humanely, and not
right away.

``That discussion can only take place in an atmosphere in which illegal
immigration is a memory of the past, no longer with us, allowing us to
weigh the different options available based on the new circumstances at
the time,'' Mr. Trump said, using the sort of vague phrasing that he
once criticized.

Never had Mr. Trump gambled quite like this. Aiming to appear
statesmanlike, he traveled to politically hostile territory to meet with
a president who might have surprised him with a rebuke, and he also
risked support from some conservatives who do not want him cozying up to
Mexico or softening his immigration plans.

\href{https://www.nytimes.com/interactive/2016/08/31/us/politics/trump-immigration-speech-live.html}{}

\includegraphics{https://static01.nyt.com/images/2016/09/01/us/01TRUMPlivechat/01TRUMPlivechat-largeHorizontalJumbo.jpg}

\hypertarget{donald-trumps-immigration-speech-analysis}{%
\subsection{Donald Trump's Immigration Speech:
Analysis}\label{donald-trumps-immigration-speech-analysis}}

Donald J. Trump delivered a speech in Phoenix on Wednesday that was
expected to clarify his shifting stance on hard-line immigration
policies, following a trip to Mexico to speak with President Enrique
Peña Nieto.

The trip to Mexico City was not without snags. Standing beside President
Peña Nieto, Mr. Trump indicated that he had pulled a punch and chosen
not to discuss his campaign promise to compel Mexico to pay for the
wall. Yet Mr. Peña Nieto saw it somewhat differently, saying later on
Twitter that at the start of their meeting, ``I made it clear that
Mexico will not pay for the wall.''

Mr. Peña Nieto did not dispute Mr. Trump at their news conference,
however, and Mexican officials said that the two men did not dwell on
the wall and that their meeting was conciliatory. Still, campaign
advisers to Hillary Clinton, the Democratic nominee, accused Mr. Trump
of lying, and the Trump campaign issued a statement saying that the
meeting was ``not a negotiation'' and that ``it is unsurprising that
they hold two different views on this issue.''

In Phoenix, Mr. Trump responded to Mr. Peña Nieto with the hectoring
language that has long been part of his strategy to whip up his crowds.

``Mexico will pay for the wall, believe me --- 100 percent --- they
don't know it yet, but they will pay for the wall,'' Mr. Trump said.
``They're great people, and great leaders, but they will pay for the
wall.''

\href{https://www.nytimes.com/interactive/2016/08/31/us/politics/donald-trump-immigration-changes.html}{}

\includegraphics{https://static01.nyt.com/images/2016/08/31/us/donald-trump-immigration-changes/donald-trump-immigration-changes-largeHorizontalJumbo.png}

\hypertarget{a-look-at-trumps-immigration-plan-then-and-now}{%
\subsection{A Look at Trump's Immigration Plan, Then and
Now}\label{a-look-at-trumps-immigration-plan-then-and-now}}

Here's a look at how the Republican candidate's positions on immigration
have changed, or remained the same, throughout the campaign.

Mr. Trump had billed the Phoenix speech as a major address on
immigration, and many Republican leaders and voters had hoped for more
clarity about his positions. Mr. Trump outlined several steps that he
would take to deport criminals and those who overstayed their visas and
end so-called sanctuary cities, while saying that ``the one route and
only route'' for others to obtain legal status would be ``to return home
and apply for re-entry.''

``We will treat everyone living or residing in our country with great
dignity --- so important,'' Mr. Trump said, noting that the status of
most illegal immigrants was no longer a ``core issue'' for him.

Mr. Trump also invited a group of Americans to the stage who, one by
one, shared the names of relatives who they said were killed by illegal
immigrants and insisted that only Mr. Trump could protect the country by
securing its borders and moving swiftly to deport immigrants with
criminal records.

Yet for all the fiery language and stagecraft, it was far from clear if
Mr. Trump's most ardent supporters would stick by him as he moves away
from his original deportation-focused policy on immigration, or if he
would win over many undecided voters with his new approach. But Mr.
Trump went to great lengths to urge voters to view the presidential race
as an epochal moment.

\includegraphics{https://static01.nyt.com/images/2016/09/01/us/01ASSESS/01ASSESS-articleInline.jpg?quality=75\&auto=webp\&disable=upscale}

``We are in the middle of a jobs crisis, a border crisis, and a
terrorism crisis,'' he said. ``This election is our last chance to
secure the border, stop illegal immigration, and reform our laws to make
your life better. This is it. We won't get another opportunity -- it
will be too late.''

The whirlwind day started after Mr. Trump accepted an invitation from
Mr. Peña Nieto to meet him at the presidential palace to discuss
economic and border concerns. For the most part they managed to sidestep
combustible issues and ignore raging hostility from average Mexicans.
Mr. Trump has called them rapists and drug dealers, and he did not
apologize for those remarks during a joint news conference when a
reporter pressed him for any regrets.

Instead, as an impassive Mr. Peña Nieto looked on, Mr. Trump sounded
conciliatory themes about working together to improve border security.
Gone, at least for this foreign trip, were the threats about American
interests and superiority that have defined Mr. Trump's candidacy and
electrified his supporters.

``I think it was an excellent meeting,'' Mr. Trump said.

Mr. Peña Nieto, who pointedly emphasized goals like ``mutual respect''
and ``constructive'' relations several times in his remarks, did Mr.
Trump some favors with his respectful treatment: The Mexican president
acknowledged that every country had a ``right'' to protect its own
border, and suggested that Mr. Trump wanted to move on from his
antagonistic remarks of the past.

``The Mexican people felt aggrieved by those comments,'' Mr. Peña Nieto
said. ``But I am certain that he has a genuine interest in building a
relationship that would lead us to provide better conditions to our
people.''

Mr. Trump's unexpected trip to Mexico was timed to steer attention from
his significant shifts on immigration policy. He flew to Mexico just
hours before he was scheduled to deliver a major speech on immigration
after more than a week of mixed signals about his immigration views,
which he said were ``softening'' and then ``hardening'' in the space of
two days last week.

On a more personal level, Mr. Trump also wanted to show undecided voters
that he had the temperament and self-control of a statesman ---
qualities that many doubt he has --- and also demonstrate that Americans
did not need to worry every time he opened his mouth in a foreign
country. He also hoped to show that he could acquit himself well on the
world stage, something that is a clear strength of Mrs. Clinton, a
former secretary of state, senator and first lady.

Mrs. Clinton's campaign has described Mr. Trump's trip as a hollow
gesture, but it was unclear whether Mrs. Clinton herself will deliver a
more pointed critique of her opponent during his travels.

Mr. Trump, who has little experience with foreign policy statecraft or
news conferences with heads of state, made no obvious mistakes during
his trip to Mexico, nor did he breach any protocol during his public
appearance with Mr. Peña Nieto on a small stage at the presidential
palace. As Mr. Peña Nieto made lengthy opening remarks in Spanish, Mr.
Trump clasped his hands at times, and tapped them against his thighs as
he nodded slightly at other points as he listened to a woman beside him
translate the remarks into English.

Mr. Peña Nieto came across as civil and stolid, defending the North
American Free Trade Agreement --- a frequent target of criticism by Mr.
Trump --- and noting that weak border security also allowed weapons and
cash often to flow from the United States to Mexican gangs and drug
cartels. But for the most part the president took a position of
neutrality, neither chastising Mr. Trump nor indicating that he favored
one American presidential candidate over another.

Yet Mr. Trump, who is known for insisting that only he can fix America's
problems, also suggested that he wanted Mexico to be a partner on border
security.

``I really believe that the president and I will solve those problems,''
Mr. Trump said. ``We will get them solved. Illegal immigration is a
problem for Mexico as well as for us. Drugs are a tremendous problem
from Mexico as well as us. I mean it's not a one-way street.''

Advertisement

\protect\hyperlink{after-bottom}{Continue reading the main story}

\hypertarget{site-index}{%
\subsection{Site Index}\label{site-index}}

\hypertarget{site-information-navigation}{%
\subsection{Site Information
Navigation}\label{site-information-navigation}}

\begin{itemize}
\tightlist
\item
  \href{https://help.nytimes.com/hc/en-us/articles/115014792127-Copyright-notice}{©~2020~The
  New York Times Company}
\end{itemize}

\begin{itemize}
\tightlist
\item
  \href{https://www.nytco.com/}{NYTCo}
\item
  \href{https://help.nytimes.com/hc/en-us/articles/115015385887-Contact-Us}{Contact
  Us}
\item
  \href{https://www.nytco.com/careers/}{Work with us}
\item
  \href{https://nytmediakit.com/}{Advertise}
\item
  \href{http://www.tbrandstudio.com/}{T Brand Studio}
\item
  \href{https://www.nytimes.com/privacy/cookie-policy\#how-do-i-manage-trackers}{Your
  Ad Choices}
\item
  \href{https://www.nytimes.com/privacy}{Privacy}
\item
  \href{https://help.nytimes.com/hc/en-us/articles/115014893428-Terms-of-service}{Terms
  of Service}
\item
  \href{https://help.nytimes.com/hc/en-us/articles/115014893968-Terms-of-sale}{Terms
  of Sale}
\item
  \href{https://spiderbites.nytimes.com}{Site Map}
\item
  \href{https://help.nytimes.com/hc/en-us}{Help}
\item
  \href{https://www.nytimes.com/subscription?campaignId=37WXW}{Subscriptions}
\end{itemize}
