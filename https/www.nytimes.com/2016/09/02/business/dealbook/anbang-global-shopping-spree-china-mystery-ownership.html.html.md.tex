Sections

SEARCH

\protect\hyperlink{site-content}{Skip to
content}\protect\hyperlink{site-index}{Skip to site index}

\href{https://myaccount.nytimes.com/auth/login?response_type=cookie\&client_id=vi}{}

\href{https://www.nytimes.com/section/todayspaper}{Today's Paper}

\href{/section/business/dealbook}{DealBook}\textbar{}A Chinese Mystery:
Who Owns a Firm on a Global Shopping Spree?

\url{https://nyti.ms/2c4Z9MN}

\begin{itemize}
\item
\item
\item
\item
\item
\end{itemize}

Advertisement

\protect\hyperlink{after-top}{Continue reading the main story}

Supported by

\protect\hyperlink{after-sponsor}{Continue reading the main story}

DealBook Business and Policy

\hypertarget{a-chinese-mystery-who-owns-a-firm-on-a-global-shopping-spree}{%
\section{A Chinese Mystery: Who Owns a Firm on a Global Shopping
Spree?}\label{a-chinese-mystery-who-owns-a-firm-on-a-global-shopping-spree}}

\includegraphics{https://static01.nyt.com/images/2016/09/01/business/02DB-ANBANG/02DB-ANBANG-articleInline.jpg?quality=75\&auto=webp\&disable=upscale}

By \href{http://www.nytimes.com/by/michael-forsythe}{Michael Forsythe}
and Jonathan Ansfield

\begin{itemize}
\item
  Sept. 1, 2016
\item
  \begin{itemize}
  \item
  \item
  \item
  \item
  \item
  \end{itemize}
\end{itemize}

Pingyang County's verdant hills still hint at a long-lost China. Rice
paddies and villages surround its bustling towns, and in the fields,
farmers wade into the mud to plant seedlings as they have for thousands
of years.

It is an odd place to find the people behind a Chinese corporate
powerhouse that is turning heads on Wall Street with a global takeover
binge. Yet the area is home to a tiny group of just such people ---
small-time merchants and villagers who happen to control
multibillion-dollar stakes in the Anbang Insurance Group, which
\href{http://dealbook.nytimes.com/2014/10/08/chinese-return-to-the-waldorf-with-2-billion/}{owns}
the
\href{http://www.nytimes.com/2016/07/24/nyregion/chasing-waldorfs-history-as-it-becomes-history-itself.html}{Waldorf
Astoria} in New York and a portfolio of global names and properties.

American regulators are now asking who these shareholders are --- and
whether they are holding their stakes on behalf of others.

The questions add to the mystery surrounding a company that seemed to
come out of nowhere, surprising deal makers with offers to pay more than
\$30 billion for assets around the world.

Anbang's shopping spree is part of an outflow of money from China that
has reshaped global markets but has often been shrouded in secrecy,
sometimes by prominent Chinese looking to shift their wealth abroad
without attracting attention at home. That poses a problem for
international regulators trying to identify the buyers behind major
acquisitions and to assess the riskiness of these deals.

The Anbang shareholders in the Pingyang County area hold their stakes
through a byzantine collection of holding companies. But according to
dozens of interviews and a review of thousands of pages of Anbang
filings by The New York Times, many of them have something in common:
They are family members and acquaintances of
\href{http://www.nytimes.com/2016/03/30/business/dealbook/china-anbang-starwood-wu-xiaohui.html}{Wu
Xiaohui}, Anbang's chairman, a native of the county who married into the
family of Deng Xiaoping, China's paramount leader in the 1980s and '90s.

In many ways, Anbang and Mr. Wu appear to be archetypal products of
China's mix of freewheeling capitalism and Communist Party dominance, a
formula that has fueled nearly four decades of untrammeled growth.

Anbang got its start as an auto insurance company in 2004 in the eastern
Chinese city of Ningbo. For years it was only a minor player. But it
took off as it became more aggressive with its finances, buying stakes
in Chinese banks and bringing in money by selling
\href{http://www.nytimes.com/2016/08/13/business/international/murky-investments-pose-china-risk.html}{high-risk,
high-yield investment funds} to ordinary Chinese.

Mr. Wu, 49, a former car salesman and low-level antismuggling official,
led Anbang through this transformation and is now known as one of
China's most successful businessmen. He wears tailored suits and
polished loafers,
\href{http://www.anbanggroup.com/abic/english/news_detail29.html}{hobnobs}
with the likes of
\href{http://www.nytimes.com/topic/person/stephen-a-schwarzman}{Stephen
A. Schwarzman} of Blackstone, and sometimes holds court at Harvard.

But he does not appear in Anbang's filings as an owner.

It is common in China for the wealthy to have their shares in companies
held in others' names. Known in Chinese as baishoutao, or white gloves,
these people are often trusted relatives or acquaintances. Many defend
the practice as a way to protect their privacy in a nation where riches
can be a political liability. But others say white gloves can be used to
hide ill-gotten gains and thwart corruption investigators.

\includegraphics{https://static01.nyt.com/images/2016/09/02/business/02DB-ANGBANG3/02DB-ANGBANG3-articleLarge.jpg?quality=75\&auto=webp\&disable=upscale}

Anbang did not respond when asked if Mr. Wu was a shareholder and
declined to answer questions about its owners.

The company, a spokesman said, ``has multiple shareholders who have made
all required disclosures under Chinese law. They are a mix of individual
and institutional shareholders who made a commercial decision to invest
in the company. Anbang has now grown to be a global company thanks to
the support of these long-term shareholders.''

For investors and regulators, white gloves can make it difficult to
evaluate the financial health of a Chinese buyer. Ownership may be
concentrated in the hands of a few people, posing hidden risks, and
companies with government connections could be vulnerable to political
shifts or become magnets for corruption.

``It is very important for businesses to know who they are ultimately
doing business with, and for investors, what they are investing in,''
said Keith Williamson, a managing director in Hong Kong at Alvarez \&
Marsal, a firm that carries out corporate fraud investigations.

It is not clear whether the shareholders in the Pingyang County region
are holding large stakes on behalf of anyone else. But on May 27, Anbang
\href{http://www.nytimes.com/2016/06/02/business/dealbook/anbang-fidelity-guaranty-life.html}{withdrew}
its application with New York State to buy an Iowa insurer, Fidelity \&
Guaranty Life, for \$1.6 billion. Regulators had asked about ties
between several shareholders with the same family names, said one person
briefed on the matter who spoke on the condition of anonymity.

A
\href{http://www.nytimes.com/2016/03/14/business/dealbook/chinese-owner-of-waldorf-astoria-bets-big-on-more-us-hotels.html}{\$6.5
billion deal} for a portfolio of hotels that includes the Essex House in
New York and several Four Seasons locations is awaiting results from a
security review by the American government. In March, Anbang withdrew a
\$14 billion bid for Starwood, the operator of Sheraton and Westin
hotels, in a move that
\href{http://www.nytimes.com/2016/04/01/business/dealbook/starwood-hotels-chinese-suitor-backs-out-of-bidding.html}{surprised
Wall Street}.

The company could come under greater scrutiny as it prepares to
\href{http://www.scmp.com/business/companies/article/2008329/chinas-biggest-unlisted-insurer-anbang-poised-go-public}{sell
shares} in its life insurance business on the Hong Kong stock exchange
next year. Already, at least one major New York-based investment bank
has raised concerns about Anbang's ownership after studying its
shareholding structure to evaluate whether to help with its overseas
deals, according to two people involved in the matter who asked not to
be identified because the process was private. The bank did not
participate in Anbang's deals.

Separately, the Chinese magazine Caixin reported in May that Chinese
regulators were examining Anbang's riskier financial products. It is
unclear where that inquiry stands or whether Anbang's ownership
structure is being investigated.

President Xi Jinping has waged a campaign against graft since taking
office, and the use of white gloves has recently come
\href{http://www.nytimes.com/2015/10/31/world/asia/chinese-tycoon-wang-jianlin-defends-xis-relatives-and-himself-on-business-deal.html}{under
scrutiny}. ``White gloves are accompanied by power's black hands,'' the
Communist Party's disciplinary watchdog wrote in a report last year.

Questions about Anbang's owners come as Chinese companies make deals
around the world --- sometimes representing efforts by China's powerful
to move money out of the country, as the economy slows and the party
tightens its grip on everyday life.

Image

Wu Xiaohui, chairman of Anbang, at a global insurance conference in
2015.Credit...Ben Asen/International Insurance Society

China has encouraged some capital outflow to improve the performance of
its investments and expand its influence. But the subject of the elite
moving money overseas is politically sensitive, raising questions about
the source of their wealth and their confidence in the Chinese economy.

Luo Yu, the son of a former chief of staff of China's military, said
China's most politically powerful families had been transferring money
out of the country for some time.

``They don't believe they will hold on to power long enough --- sooner
or later they would collapse,'' said Mr. Luo, a former colonel in the
Chinese Army whose younger brother was a business partner with one of
Anbang's founders. ``So they transfer their money.''

At its founding in 2004, Anbang had an impressive list of politically
connected directors. Records show early Anbang directors included Levin
Zhu, son of a former prime minister, and
\href{http://www.nytimes.com/2013/12/07/world/asia/a-student-leader-in-maos-cultural-revolution.html}{Chen
Xiaolu}, the son of an army marshal who helped bring Communist rule to
China.

Then there was Mr. Wu, who was born Wu Guanghui but was known as Wu
Xiaohui from a young age. Relatives said he grew up in a Catholic
family; a crucifix sat on his aunt's dining room table, and she wears a
necklace with a portrait of the Virgin Mary.

Mr. Wu married Zhuo Ran, a granddaughter of Deng, the Chinese leader who
brought China out of the chaos of the Mao era. Together, Mr. Wu, Ms.
Zhuo, Mr. Chen and their relatives owned or ran the companies that
controlled Anbang, according to company filings.

Anbang leapt onto the global stage with last year's purchase of the
Waldorf Astoria and its aborted bid for the Starwood chain. By this
year, Anbang's assets had swelled to \$295 billion.

It is not clear what prompted Anbang's sudden interest in overseas
assets. But the shift came after a reshuffling of its ownership
structure that also led to the injection of more than \$7.5 billion into
the company.

Company documents filed with Chinese agencies show that the number of
firms holding Anbang's shares jumped to 39, from eight, over six months
in 2014. Most of those firms received large injections of funds. At the
same time, Anbang's capital more than quintupled.

Ms. Zhuo disappeared from the ownership records by the end of that year.
Many of Mr. Wu's relatives did as well. Mr. Wu and Mr. Chen had
disappeared earlier from the records.

Image

The Anbang Insurance Group owns the Waldorf Astoria in New York, above,
and a portfolio of global names and properties.Credit...Chang W. Lee/The
New York Times

Mr. Zhu, who does not appear to have owned shares, disappeared in paper
filings from Anbang's roster of directors by 2009, though he was listed
as a director on online government filings as late as 2014.

Mr. Wu, Mr. Chen and Mr. Zhu did not respond to requests for comment,
and Ms. Zhuo could not be reached. In March, Mr. Zhu told Chinese
reporters that he was not an Anbang director.

Anbang's current shareholding firms are not well-known names in China,
and some appear to have been set up just to hold Anbang shares. One
lists its address as the empty 27th floor of a dusty Beijing office
building. Two more list an address at a mail drop above a Beijing post
office.

Using corporate filings, The Times compiled a list of nearly 100 people
who own shares in the firms and traced about a dozen to Pingyang County
or nearby. Reporters visited the area, in China's eastern Zhejiang
Province, and interviewed dozens of residents, including several whose
names appeared on the list. They also interviewed an uncle, an aunt and
a nephew of Mr. Wu.

The latter two, as well as others in the area, said one name matched
that of his sister, Wu Xiaoxia. The family members said several other
names matched those of Mr. Wu's extended kin, including two cousins and
others on his mother's side of the family. Through their various stakes
in Anbang shareholding companies, these people control a stake
representing more than \$17 billion in assets.

Other names matched local acquaintances of Mr. Wu, including Huang
Maosheng, a local businessman who confirmed in a brief phone interview
that he had a business relationship with Mr. Wu but declined to
elaborate.

One village leader and neighbors identified the names of four of Mr.
Huang's relatives --- including some whom they described as common
workers --- from among those on the list. Their Anbang holdings
represent about \$12 billion in assets.

Another resident, Mei Xiaojing, said two names on the list matched those
of her relatives. Asked if she knew Mr. Wu, she said, ``Well, yes,''
then ended the phone conversation and did not respond to subsequent
calls. Through multiple holding companies, those three people have a
stake representing about \$19 billion in Anbang assets.

As Anbang rose, so did Mr. Wu's profile. In 2013 Mr. Wu secured a
yearlong position as a visiting fellow at the Asia Center of Harvard,
joining
a\href{http://www.nytimes.com/2015/04/29/world/asia/wang-jianlin-abillionaire-at-the-intersection-of-business-and-power-in-china.html}{growing
list}
of\href{http://fairbank.fas.harvard.edu/pages/desmond-and-whitney-shum-fellows-details}{politically
connected} Chinese
\href{https://pubapps.hks.harvard.edu/enrollment/fellowships/fellowship.aspx?id=AshChina}{billionaires}
with ties to Harvard.

Ezra F. Vogel, a professor emeritus at Harvard who wrote
a\href{http://www.hup.harvard.edu/catalog.php?isbn=9780674725867}{biography}
of Deng, said he met Mr. Wu on several occasions.

``He had this staff of sharp people who were working for him,'' Mr.
Vogel said. ``It seems that they were doing the detail work, and he was
the friendly man supplying the connections.''

Advertisement

\protect\hyperlink{after-bottom}{Continue reading the main story}

\hypertarget{site-index}{%
\subsection{Site Index}\label{site-index}}

\hypertarget{site-information-navigation}{%
\subsection{Site Information
Navigation}\label{site-information-navigation}}

\begin{itemize}
\tightlist
\item
  \href{https://help.nytimes.com/hc/en-us/articles/115014792127-Copyright-notice}{©~2020~The
  New York Times Company}
\end{itemize}

\begin{itemize}
\tightlist
\item
  \href{https://www.nytco.com/}{NYTCo}
\item
  \href{https://help.nytimes.com/hc/en-us/articles/115015385887-Contact-Us}{Contact
  Us}
\item
  \href{https://www.nytco.com/careers/}{Work with us}
\item
  \href{https://nytmediakit.com/}{Advertise}
\item
  \href{http://www.tbrandstudio.com/}{T Brand Studio}
\item
  \href{https://www.nytimes.com/privacy/cookie-policy\#how-do-i-manage-trackers}{Your
  Ad Choices}
\item
  \href{https://www.nytimes.com/privacy}{Privacy}
\item
  \href{https://help.nytimes.com/hc/en-us/articles/115014893428-Terms-of-service}{Terms
  of Service}
\item
  \href{https://help.nytimes.com/hc/en-us/articles/115014893968-Terms-of-sale}{Terms
  of Sale}
\item
  \href{https://spiderbites.nytimes.com}{Site Map}
\item
  \href{https://help.nytimes.com/hc/en-us}{Help}
\item
  \href{https://www.nytimes.com/subscription?campaignId=37WXW}{Subscriptions}
\end{itemize}
