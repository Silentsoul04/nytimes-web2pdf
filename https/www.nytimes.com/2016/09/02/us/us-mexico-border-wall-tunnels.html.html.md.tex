Sections

SEARCH

\protect\hyperlink{site-content}{Skip to
content}\protect\hyperlink{site-index}{Skip to site index}

\href{https://www.nytimes.com/section/us}{U.S.}

\href{https://myaccount.nytimes.com/auth/login?response_type=cookie\&client_id=vi}{}

\href{https://www.nytimes.com/section/todayspaper}{Today's Paper}

\href{/section/us}{U.S.}\textbar{}As Donald Trump Calls for Wall on
Mexican Border, Smugglers Dig Tunnels

\url{https://nyti.ms/2c47i3Y}

\begin{itemize}
\item
\item
\item
\item
\item
\end{itemize}

Advertisement

\protect\hyperlink{after-top}{Continue reading the main story}

Supported by

\protect\hyperlink{after-sponsor}{Continue reading the main story}

\hypertarget{as-donald-trump-calls-for-wall-on-mexican-border-smugglers-dig-tunnels}{%
\section{As Donald Trump Calls for Wall on Mexican Border, Smugglers Dig
Tunnels}\label{as-donald-trump-calls-for-wall-on-mexican-border-smugglers-dig-tunnels}}

\includegraphics{https://static01.nyt.com/images/2016/09/02/us/02tunnel1/00tunnel-articleInline.jpg?quality=75\&auto=webp\&disable=upscale}

By \href{http://www.nytimes.com/by/ron-nixon}{Ron Nixon}

\begin{itemize}
\item
  Sept. 1, 2016
\item
  \begin{itemize}
  \item
  \item
  \item
  \item
  \item
  \end{itemize}
\end{itemize}

NOGALES, Ariz. --- On an embankment that runs along a towering steel
fence separating this border town from its Mexican sister city, a patch
of new concrete with a date carved into it stands out, marking the exit
of a tunnel Border Patrol agents sealed in May.

Dozens more like it snake around town, part of a vast underground
network that Mexican drug cartels have used for years to funnel hundreds
of pounds of illicit drugs into the United States. When Border Patrol
agents find the tunnels, they dump concrete to seal them and stamp them
with the date that they are shut down.

But they struggle to stay ahead of the digging. Last Friday, the Border
Patrol, in a joint operation with Mexican authorities, discovered an
unfinished tunnel that started in a Mexican cemetery.

``The clock is ticking as soon as they complete a tunnel,'' said Kevin
Hecht, a Border Patrol tunneling expert. ``They know that we will
eventually find them. But if even one load gets through before we find
it, they consider it a success.''

While Donald J. Trump, the Republican presidential nominee, has made
building a wall at the border a central element of his campaign, the
fence here that reaches up to 30 feet has done little to deter
enterprising drug smugglers. It has simply helped
\href{http://www.nytimes.com/2010/10/03/us/03tunnels.html}{push them
underground}.

\includegraphics{https://static01.nyt.com/images/2016/09/02/us/02tunnel2/02tunnel3-videoSixteenByNineJumbo1600.jpg}

Border Patrol agents cannot hear smugglers digging and do not know how
many tunnels there are, a gap in border security that homeland security
experts say renders talk of a wall moot.

``The Border Patrol has done an incredible job, given its resources,''
said Gen. Barry R. McCaffrey, the White House drug policy director in
the Clinton administration. ``But it would be a stretch to say that the
border and border communities are secure when the agency lacks a
high-confidence ability to detect cross-border tunnels. No wall is going
to fix that.''

During his immigration speech Wednesday in Arizona, Mr. Trump said his
border security plan would use the best technology, including above- and
below-ground sensors to ``find and dislocate tunnels and keep out
criminal cartels.''

But no technology exists to reliably detect the tunnels, and experts say
it may be years before such a system is developed.

Since the
\href{http://www.nytimes.com/1990/05/19/us/agents-find-drug-tunnel-to-us.html}{first
drug tunnel was discovered in 1990} in Douglas, Ariz., border officials
say they have found nearly 200 more along the nearly 2,000-mile
Southwest border, mostly in Arizona and California. Tunnels are so
numerous in the Nogales area that Border Patrol agents described the
ground underneath the city as ``Swiss cheese.''

\includegraphics{https://static01.nyt.com/images/2016/09/02/us/02tunnel3/02tunnel2-articleInline.jpg?quality=75\&auto=webp\&disable=upscale}

Mr. Hecht said smugglers dig tunnels mostly into the drainage system
shared by the two cities. Others are burrowed into the basements of
homes on the American side from buildings in Mexico. One tunnel was dug
under a heavily guarded port of entry.

Technological advances such as ground radar to detect movement, hundreds
of high-tech cameras with night-vision lenses and drones flying overhead
have drastically transformed border security. These tools have helped
federal investigators track and arrest hundreds of smuggling suspects,
and seize tons of marijuana, methamphetamines and cocaine.
Remote-controlled robots help agents explore tunnels that are too risky
for humans to enter.

The American government has poured hundreds of millions of dollars into
research in hopes of finding a way to detect tunnels, but most of these
efforts have ended in disappointment. Most recently, the
\href{https://www.dhs.gov/science-and-technology}{Science and Technology
Directorate} of the Department of Homeland Security concluded that none
of the current methods used to detect underground tunnels were
``necessarily suited to Border Patrol agents' operational needs.''

In the absence of technology to detect tunnels, Border Patrol officials
have worked with Mexican authorities to develop informants and patrol
the border, including water and sewerage infrastructure, looking for
suspicious activity. About half of the Border Patrol agents here have
been trained to work underground.

``But you don't know what you don't know,'' said R. Gil Kerlikowske, the
United States Customs and Border Protection commissioner, who conceded
that many more tunnels might exist.

Image

A machine that sends radar waves about 10 feet into the ground to help
detect tunnels.Credit...Eros Hoagland for The New York Times

Part of the problem in detecting tunnels, say experts like Paul Bauman,
a Canadian geophysicist, is the ground itself. Finding what is under the
surface is not as simple as shooting radar or electromagnetic waves into
the ground, he said.

With underground cracks, water tables, tree roots and caves, it is hard
to tell what is and is not a tunnel, he said.

Mr. Bauman, who has worked with the Israel Defense Forces in their
efforts to find tunnels, said most of the devices used for tunnel
detection were developed for industries to find oil or mineral deposits,
not drug tunnels.

Carey M. Rappaport, a professor of electrical and computer engineering
at Northeastern University in Boston, said the depth of many tunnels
also posed a technological challenge. Some can be as deep as 90 feet,
beyond the reach of most ground-radar devices and sensors.

``Soil is very good at keeping secrets,'' said Mr. Rappaport, who has
also worked with the United States and Israeli governments on
tunnel-detection methods.

Image

Wendi Lee, a Border Patrol agent, driving past an area of San Diego
where several drug tunnels have been discovered.Credit...Eros Hoagland
for The New York Times

In the 2016 defense authorization bill, Congress provided about \$120
million for a joint Defense Department and Israel Defense Forces
tunnel-detection project. Israel is among several nations, including
Egypt, Jordan and South Korea, that have had major problems with hostile
groups using tunnels to stage attacks. American officials hope the
technology developed in Israel can aid efforts on the Mexican border.

A spokeswoman for the Israeli Defense Ministry declined to comment.

The Science and Technology Directorate at Homeland Security is also
spending several million dollars a year to fund tunnel-detection
research.

In San Diego, a task force of agents from Immigration and Customs
Enforcement, Homeland Security Investigations, the Border Patrol and the
Drug Enforcement Administration has tried a variety of technologies to
detect tunnels, much like their colleagues in Nogales have.

One of the tools is a ground-radar machine that looks like a large lawn
mower. The device, which is intended to locate underground utility lines
and flaws in road construction, shoots radar waves about 10 feet into
the ground. A screen displays various shades that identify anomalies
underground that could be tunnels.

But that is often not deep enough. ``We've never found a tunnel using
them,'' said David Shaw, a special agent in charge of Homeland Security
Investigations in San Diego.

\href{https://www.nytimes.com/interactive/2016/09/01/us/document-DHS-Report-to-Congress-on-Cross-Broder-Tunnels.html}{}

\includegraphics{https://static01.nyt.com/images/2016/09/02/us/02tunnel-dhs-doc-promo/02tunnel-dhs-doc-promo-thumbLarge.png}

\hypertarget{a-report-to-congress-on-drug-smuggling-tunnels}{%
\subsection{A Report to Congress on Drug-Smuggling
Tunnels}\label{a-report-to-congress-on-drug-smuggling-tunnels}}

The Department of Homeland Security issued an annual report earlier this
year on efforts to find and close cross-border tunnels.

Mr. Shaw said the task force relied mainly on old-fashioned law
enforcement techniques, such as cultivating informants in cartels or
getting business owners and residents to report suspicious activities.

That's how the most recent
\href{http://www.nytimes.com/2016/04/22/us/drug-smuggling-tunnel-from-tijuana-mexico-found-in-san-diego.html?_r=0}{tunnel
in San Diego was discovered} in April. Its exit was in a fenced-off area
in the heavily industrial Otay Mesa warehouse district, about 500 yards
from the border.

According to Wendi Lee, a Border Patrol agent, business owners across
the street alerted the agency to what they said was suspicious activity.
The tunnel, hidden under a trash bin, was about 800 yards long. The
authorities seized about 2,200 pounds of cocaine and 14,000 pounds of
marijuana. The tunnel was one of about a half-dozen discovered in the
area in recent years.

``Smugglers keep digging them because the tunnels work,'' Mr. Shaw said.

Still, some law enforcement officials say their efforts appear to be
having an effect. Data from Homeland Security shows that fewer tunnels
have been discovered in recent years.

However, Mr. Hecht of the Border Patrol in Nogales said the number of
tunnels discovered should not be used to measure success.

``For every tunnel we find, we feel they're building another one
somewhere, and they might get more creative in concealing it,'' he said.
``Next year, I could find 10. Until there is some device on the market
to help us accurately detect them, we just won't know.''

Advertisement

\protect\hyperlink{after-bottom}{Continue reading the main story}

\hypertarget{site-index}{%
\subsection{Site Index}\label{site-index}}

\hypertarget{site-information-navigation}{%
\subsection{Site Information
Navigation}\label{site-information-navigation}}

\begin{itemize}
\tightlist
\item
  \href{https://help.nytimes.com/hc/en-us/articles/115014792127-Copyright-notice}{©~2020~The
  New York Times Company}
\end{itemize}

\begin{itemize}
\tightlist
\item
  \href{https://www.nytco.com/}{NYTCo}
\item
  \href{https://help.nytimes.com/hc/en-us/articles/115015385887-Contact-Us}{Contact
  Us}
\item
  \href{https://www.nytco.com/careers/}{Work with us}
\item
  \href{https://nytmediakit.com/}{Advertise}
\item
  \href{http://www.tbrandstudio.com/}{T Brand Studio}
\item
  \href{https://www.nytimes.com/privacy/cookie-policy\#how-do-i-manage-trackers}{Your
  Ad Choices}
\item
  \href{https://www.nytimes.com/privacy}{Privacy}
\item
  \href{https://help.nytimes.com/hc/en-us/articles/115014893428-Terms-of-service}{Terms
  of Service}
\item
  \href{https://help.nytimes.com/hc/en-us/articles/115014893968-Terms-of-sale}{Terms
  of Sale}
\item
  \href{https://spiderbites.nytimes.com}{Site Map}
\item
  \href{https://help.nytimes.com/hc/en-us}{Help}
\item
  \href{https://www.nytimes.com/subscription?campaignId=37WXW}{Subscriptions}
\end{itemize}
