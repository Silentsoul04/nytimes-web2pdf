Sections

SEARCH

\protect\hyperlink{site-content}{Skip to
content}\protect\hyperlink{site-index}{Skip to site index}

\href{https://www.nytimes.com/section/fashion}{Fashion}

\href{https://myaccount.nytimes.com/auth/login?response_type=cookie\&client_id=vi}{}

\href{https://www.nytimes.com/section/todayspaper}{Today's Paper}

\href{/section/fashion}{Fashion}\textbar{}Cher's Latest Road Show? The
Campaign Trail

\url{https://nyti.ms/2chx7uJ}

\begin{itemize}
\item
\item
\item
\item
\item
\end{itemize}

Advertisement

\protect\hyperlink{after-top}{Continue reading the main story}

Supported by

\protect\hyperlink{after-sponsor}{Continue reading the main story}

\hypertarget{chers-latest-road-show-the-campaign-trail}{%
\section{Cher's Latest Road Show? The Campaign
Trail}\label{chers-latest-road-show-the-campaign-trail}}

\includegraphics{https://static01.nyt.com/images/2016/09/04/fashion/04COVER2/04COVER2-articleInline.jpg?quality=75\&auto=webp\&disable=upscale}

By \href{http://www.nytimes.com/by/jeremy-w-peters}{Jeremy W. Peters}

\begin{itemize}
\item
  Sept. 3, 2016
\item
  \begin{itemize}
  \item
  \item
  \item
  \item
  \item
  \end{itemize}
\end{itemize}

Cher's first brush with politics was an act of teenage civil
disobedience. It was right before the election of 1960. And she came
home one day horrified to discover that her mother and stepfather had
festooned their house in Hollywood with Richard Nixon paraphernalia.

Cher purged it all.

First she ripped the Nixon yard signs out of their front lawn. And after
her mother fished them out of the garbage and put them back up, Cher
found a more permanent solution.

``I took them to my girlfriend's house, and we threw them in her
trash,'' she told me by phone from California the other day, the
mischievousness and pride still in her voice after all those years.

But the signs weren't all she got her hands on. ``Also threw out those
hideous straw hats with Nixon on them,'' she said.

Fifty-six years later, that is more or less her approach to Donald J.
Trump: trash and destroy. These days she wages her battles against the
Republican Party not from her lawn --- it's in Malibu now --- but on her
\href{https://twitter.com/cher?ref_src=twsrc\%5Egoogle\%7Ctwcamp\%5Eserp\%7Ctwgr\%5Eauthor}{Twitter
feed}, which she acknowledges is not exactly a model of self-restraint.
``If you looked at my tweets, you'd think, `She's crazy,''' she said.
(That ``craziness'' has attracted, at last count, 3.1 million
followers.)

In many ways, Cher is the perfect political counterpart to Mr. Trump.
Like him, she has always been defiantly indifferent toward her critics.
Also like Mr. Trump, no one holds her to the rigid standards of campaign
conduct, giving her license to say what she wants without all the
consequences. Imagine for a minute that another septuagenarian supporter
of Hillary Clinton like Madeleine Albright tweeted side-by-side pictures
of John Gotti and Paul Manafort, Mr. Trump's recently departed campaign
chairman, and offered this critique: ``FYI.. Manafort\ldots{}John Gotti
called..he wants his look back!!''

If there was ever a presidential election perfect for a Cher moment,
this is it. Personality, outrage and a quick-off-the-keyboard insult are
the political currency of 2016.

She has earned a reputation as one of the more effective and
entertaining Trump neutralizers on Twitter, largely because she can go
toe-to-toe with him both in the sheer volume of tweets she fires off
(19,000 and counting) and in her lacerating, no-filter style. She often
won't refer to Mr. Trump by name, for example, but with the toilet
emoji. (And that is one of her few jokes that is printable here.)

\includegraphics{https://static01.nyt.com/images/2016/09/04/fashion/04CHERJP1/04CHERJP1-articleLarge.jpg?quality=75\&auto=webp\&disable=upscale}

Now she is getting marquee billing on the campaign trail. The Clinton
campaign realized the weapon it had at its disposal and reasoned that
Cher would be a hit at fund-raisers --- especially those with a larger
than average guest list of gay men. She headlined three of them last
month, in Miami's Wynwood neighborhood, Fire Island and Provincetown,
Mass. ``I'm such an obvious person,'' she told me, ``to bring that
message to, as I call them, my people.''

After one guest at the Provincetown event uploaded to Facebook
\href{http://www.nytimes.com/2016/08/24/us/politics/cher-hillary-clinton-donald-trump.html}{a
video of her introducing Mrs. Clinton} --- a profane mash-up of insult
comedy, political commentary and world history --- it went viral.

``I think the one thing that I am,'' she said, ``I'm honest. And I say
what I think.''

She is self-aware enough to know her impetuousness may cause trouble for
her and Mrs. Clinton. That is why she sheepishly acknowledged, ``I'm
trying not to use any bad words.''

Public speaking, she insisted, does not come naturally to her. But
anyone who has watched her
\href{https://www.youtube.com/watch?v=gdHdd5ejAeY}{eulogy to her former
husband} Sonny Bono has seen that she has a gift for moving audiences.

I told her this when we had our first phone call because she wouldn't
let me start the interview until I submitted to a minor grilling.

``I have to ask you a question,'' she said before I could get a word
out. ``Because most people think that if you're famous that you don't
deserve an opinion. And so why are you calling me?''

I've followed her career, the farewells, the reinventions, the public
degradation and adulation, for quite a while. When I was in seventh
grade, my performance in the talent show was to lip-sync Sonny and
Cher's ``I Got You Babe.'' I was Cher, dressed in a hip-length black wig
and a pair of my mom's old bell bottoms.

I always thought Cher had a unique voice, and in more than just song.

There was the time in 2003 when
\href{https://www.c-span.org/video/?c4532037/cher-calls-c-span}{she
called in to C-Span's Washington Journal} to describe how heartbroken
and outraged she was after visiting wounded soldiers at Walter Reed Army
Medical Center. After the host tried to press this anonymous caller from
Miami Beach about her identity --- ``an entertainer'' was all Cher would
volunteer --- her cover was blown. ``Is this Cher?'' the host asked,
sounding stunned.

She
\href{https://www.c-span.org/video/?c4452332/cher-calls-c-span}{called
in to the show again in 2006}. ``I just don't understand how this
government can send men into war without the proper helmets,'' she said,
this time as an anonymous caller from Malibu. Again, the host figured it
out. ``And is this Cher?''

Image

Chelsea Clinton and Hillary Clinton, who was running for the Senate, and
Cher in 2000.Credit...David Hume Kennerly/Getty Images

To pigeonhole her as a Hollywood liberal misses some of the nuance of
her politics. She voted for Ross Perot in 1992, which she announced to
Larry King on CNN --- once again dialing in but this time not shrouded
in anonymity. ``I was really nervous, I was really frightened, but my
conscience is clear,''
\href{http://www.apnewsarchive.com/1992/Cher-Calls-In-Kudos-To-Perot/id-3478ce960a33164ccab22c0d7dd2e977}{she
told Mr. Perot}, who was a guest of Mr. King's that night.

She told me she was so strongly moved to oppose Mr. Trump not solely
because of his politics but because she felt he was dangerously
misguided and intemperate. Ronald Reagan, she said, was a Republican she
disagreed with politically but did not fear personally.

``He did nothing for AIDS, and he just stood back when people were
dying,'' she said. ``But there were things that you could say, `He makes
a good president' --- for some people. But I wasn't frightened really
that he could bring the country to his knees because he knew nothing
about how to govern.''

She says she is no fan of the government and has little confidence it
can solve the employment problems that come with a rapidly changing
economy, which she blames on cold, impersonal decisions made by
dollar-driven chief executives.

``It's criminal what's happened to these people who had great jobs in
Detroit, who worked in steel in Pittsburgh,'' she said. ``We've lost
respect for everything that doesn't make money.'' Or for jobs that don't
require four-year college degrees, she added: ``The janitor at my
school, we called him `Mr.'''

Politics have been at the periphery of Cher's life since the 1960s. Her
first husband would become one of the country's most famous Republicans:
Mr. Bono, who was the mayor of Palm Springs and a congressman for two
terms until he died in a skiing accident in 1998.

``When Sonny and I were married, he was a Democrat,'' she said,
recalling how thrilled he sounded after returning from the 1968
Democratic National Convention. ``He said, `Oh my God, Cher, I met this
amazing guy McGovern,''' meaning George McGovern, the former senator
from South Dakota, who ran for the Democratic nomination that year and
lost.

Her second husband, Gregg Allman of the Allman Brothers,
\href{https://www.theguardian.com/music/2012/oct/30/jimmy-carter-president-interview}{helped
raise money for Jimmy Carter}. The Allman Brothers, which had roots in
Georgia, would play concerts for Mr. Carter during the 1976 presidential
campaign. And after Mr. Carter won, he invited the couple to the White
House for a reception for supporters.

Cher did not expect much more than a tour that night. But she said that
as she was downstairs in the White House basement looking at Mary Todd
Lincoln's china collection, Amy Carter, the president's daughter,
summoned her: ``Amy came down and said, `Mama wants you to have
dinner.'''

I wondered, did she ever want to get in the game and run for office?
``Not even for a hot second,'' she assured me. She could never live with
herself, she said, if she had to hold her tongue. ``It would ruin me as
a person.''

In 1988, Cher publicly backed Michael Dukakis, who was not the only
Dukakis she got to know. That year, she won an Oscar for her lead role
in ``Moonstruck.'' Starring as Cher's mother in the film was Olympia
Dukakis, Mr. Dukakis's cousin.

Cher's support for Mrs. Clinton started with her Senate campaign and,
later, in 2008 with her unsuccessful bid for the Democratic presidential
nomination against Barack Obama.

The similarities between the two women struck me as something that must
have affected Cher. They are just a few years apart in age. Cher is 70.
Mrs. Clinton is 68. They both triumphed in male-dominated professions
and faced their own career humiliations and rebirths.

But when I asked her about it, she wouldn't hear it. ``I have to stop
you right there,'' she said. ``The fact that she's a woman does not ---
I don't care.''

When she was a child, she said, the idea of a female president was so
foreign to the way she saw politics that she thought electing a woman
might even be against the law. ``I'm so blessed to be alive to see a
woman president,'' she said. ``But I'm so angry at the fact that I feel
blessed --- that it's not something that's natural.''

Her reasons for feeling so strongly about Mr. Trump, she said, are
maternal: ``I know that women can look into the future, the future for
their children**.''**

``And even if they don't like Hillary, and many women don't,'' she
added, they have to think to themselves ``but I don't want to risk my
children's future to this guy.''

Despite running in overlapping social circles for years --- she and Mr.
Trump were regulars in Aspen, Colo. --- Cher said she has never met this
man she has such disdain for. She said she would occasionally see him at
Mezzaluna, a hot spot in Aspen that they both frequented. That cemented
her opinion early on. ``I just thought, `What an. \ldots{}''

She stopped herself. She is trying not to use bad words anymore.

Advertisement

\protect\hyperlink{after-bottom}{Continue reading the main story}

\hypertarget{site-index}{%
\subsection{Site Index}\label{site-index}}

\hypertarget{site-information-navigation}{%
\subsection{Site Information
Navigation}\label{site-information-navigation}}

\begin{itemize}
\tightlist
\item
  \href{https://help.nytimes.com/hc/en-us/articles/115014792127-Copyright-notice}{©~2020~The
  New York Times Company}
\end{itemize}

\begin{itemize}
\tightlist
\item
  \href{https://www.nytco.com/}{NYTCo}
\item
  \href{https://help.nytimes.com/hc/en-us/articles/115015385887-Contact-Us}{Contact
  Us}
\item
  \href{https://www.nytco.com/careers/}{Work with us}
\item
  \href{https://nytmediakit.com/}{Advertise}
\item
  \href{http://www.tbrandstudio.com/}{T Brand Studio}
\item
  \href{https://www.nytimes.com/privacy/cookie-policy\#how-do-i-manage-trackers}{Your
  Ad Choices}
\item
  \href{https://www.nytimes.com/privacy}{Privacy}
\item
  \href{https://help.nytimes.com/hc/en-us/articles/115014893428-Terms-of-service}{Terms
  of Service}
\item
  \href{https://help.nytimes.com/hc/en-us/articles/115014893968-Terms-of-sale}{Terms
  of Sale}
\item
  \href{https://spiderbites.nytimes.com}{Site Map}
\item
  \href{https://help.nytimes.com/hc/en-us}{Help}
\item
  \href{https://www.nytimes.com/subscription?campaignId=37WXW}{Subscriptions}
\end{itemize}
