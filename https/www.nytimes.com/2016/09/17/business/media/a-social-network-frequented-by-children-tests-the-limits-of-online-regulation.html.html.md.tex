Sections

SEARCH

\protect\hyperlink{site-content}{Skip to
content}\protect\hyperlink{site-index}{Skip to site index}

\href{https://www.nytimes.com/pages/business/media/index.html}{Media}

\href{https://myaccount.nytimes.com/auth/login?response_type=cookie\&client_id=vi}{}

\href{https://www.nytimes.com/section/todayspaper}{Today's Paper}

\href{/pages/business/media/index.html}{Media}\textbar{}Who's Too Young
for an App? Musical.ly Tests the Limits

\url{https://nyti.ms/2cV0DXd}

\begin{itemize}
\item
\item
\item
\item
\item
\end{itemize}

Advertisement

\protect\hyperlink{after-top}{Continue reading the main story}

Supported by

\protect\hyperlink{after-sponsor}{Continue reading the main story}

\hypertarget{whos-too-young-for-an-app-musically-tests-the-limits}{%
\section{Who's Too Young for an App? Musical.ly Tests the
Limits}\label{whos-too-young-for-an-app-musically-tests-the-limits}}

\includegraphics{https://static01.nyt.com/images/2016/09/12/business/12musically-inyt/12musically-inyt-articleLarge.jpg?quality=75\&auto=webp\&disable=upscale}

By \href{https://www.nytimes.com/by/john-herrman}{John Herrman}

\begin{itemize}
\item
  Sept. 16, 2016
\item
  \begin{itemize}
  \item
  \item
  \item
  \item
  \item
  \end{itemize}
\end{itemize}

In the folklore of the start-up world, few figures loom larger than the
teenager. Teenagers see the future, set trends and spend money, or
compel parents to spend it for them. Their behavior has become an
obsession for entrepreneurs.

This would seem to bode well for
\href{http://www.musical.ly/}{Musical.ly}, an app that is young in every
sense of the word. The Shanghai-based company founded in 2014 claims
over 100 million users, most of whom, the company says, are in the 13-20
age bracket. In August, the company teamed up with MTV for a promotion
tied to the Video Music Awards.

What is striking about the app, though, is how many of its users appear
to be even younger than that. Musical.ly hasn't just found the coveted
teenage audience --- it may have gone lower. And it points to a growing
tension between younger users, technology companies, and the norms and
laws that regulate them both.

The app encourages a youthful audience in subtle and obvious ways. It
lets users create short videos in which they can lip-sync, dance or goof
around to popular songs, movie scenes and other audio sources, and then
post the videos to an Instagram-style feed. Its featured feed includes
stars popular with young listeners, including Ariana Grande and Selena
Gomez, as well as lesser-known talent and social media personalities who
have crossed over from services like Vine. And its tool for posting
videos includes an entire category for songs from Disney films and TV
shows.

The app does not collect or show the age of its users, but some of its
top-ranked users, whose posts routinely collect millions of likes,
called hearts, appear from their videos and profile photos to be in
grade school. Until recently, the app had a feature that suggested users
to follow based on their location. In New York, that feature revealed a
list composed largely not just of teenagers, but of children.

``This is no question the youngest social network we've ever seen,''
said Gary Vaynerchuk, the chief executive of
\href{http://www.vaynermedia.com/}{VaynerMedia}, an advertising agency
that focuses on social media. Mr. Vaynerchuk, who has helped clients
produce campaigns for the platform, said he first spotted the app in the
iTunes App Store charts, and through Musical.ly videos reposted to other
services like Instagram.

``I would say that Snapchat and Instagram, they skew a little bit
young,'' he said. But with Musical.ly, ``you're talking about first,
second, third grade.''

This puts Musical.ly in a strange position. Websites and online service
operators that target users under 13 must meet federal requirements
regarding the collection and sharing of personal information, which is
defined broadly to include names, photos or videos, or persistent
identifiers, such as user names. The restrictions are part of the
Children's Online Privacy Protection Rule, often called Coppa, enacted
by the Federal Trade Commission.

Like most social networks that operate in America, Musical.ly says in
its terms of service that it prohibits users under the age of 13. But
the app doesn't collect age information about its users, a convention
often used by networks that, at least anecdotally, are widely used by
children. (Those networks that do ask for an age, like Facebook, tend to
take the user's word for it.)

``The approach taken by most operators is to avoid triggering Coppa ---
to claim ignorance,'' said Denise G. Tayloe, chief executive of
\href{https://privo.com/}{Privo}, a company that assists online services
with Coppa compliance. Enforcement is both difficult and relatively
rare; the rule neither requires social platforms to screen for age, nor
\href{https://www.ftc.gov/tips-advice/business-center/guidance/complying-coppa-frequently-asked-questions}{holds
companies accountable} for users' reporting false ages.

According to a spokesman, the F.T.C., as a law enforcement agency,
cannot comment publicly about individual companies. In an email, the
agency said, ``We support the development of robust, easy-to-use
mechanisms companies that collect kids' personal information can use to
seek the parents' consent.''

Image

Alex Zhu, a founder and co-chief executive of Musical.ly. Like most
social networks that operate in America, Musical.ly's terms of service
prohibit users under the age of 13. But the app doesn't collect age
information about its users.Credit...Gilles Sabrie for The New York
Times

Services that are more openly marketed toward children often stringently
adhere to Coppa's privacy rules. Vine Kids, for example, is a limited
and largely passive service with no user names or video posting
capabilities; similarly, YouTube Kids is essentially an app full of
streaming children's programming, walled off from the rest of YouTube's
ecosystem. In contrast, Musical.ly, like Snapchat or Instagram, is a
full-functioning social network, popular with young people but not
openly marketed to them.

Such discussions about privacy can feel strained against the backdrop of
technological change. The first version of Coppa became law in 1998,
almost a decade before the iPhone was introduced. Last year, the
research firm \href{http://influence-central.com/}{Influence Central}
said that, on average, parents who give their children smartphones do so
at age 12. And once they have a phone, they get apps.

In a study of the law published in 2011 by the academic journal
\href{http://journals.uic.edu/ojs/index.php/fm/article/view/3850/3075}{First
Monday}, researchers suggested that Coppa created intractable issues. To
remain compliant, tech companies either cut off young users or claimed
ignorance of their presence, while parents, for whom the law is meant to
provide guidance and comfort, often ended up helping their children
circumvent sign-up rules. Increasing the current style of enforcement,
the report concluded, would only encourage firms to ``focus on denying
access rather than providing privacy protection or cooperating with
parents.''

In short, children are using their smartphones much like the rest of us,
whether or not they are comprehensively addressed by regulations or by
broader cultural conventions.

Alex Hofmann, president of Musical.ly, said the company tries to be
mindful of its popularity with younger users. ``One of the differences
to other apps,'' he said, ``is that we don't only talk to the musers''
--- the company's term for users --- ``we talk to the parents.''

He keeps close counsel with a network of a few dozen top users, and some
of their families, and frequently asks for feedback from both regarding
everything from user safety to new features. (The company's support page
contains an \href{http://musicallyapp.tumblr.com/parents}{entire
section} directed toward parents --- one that notes the app is
``intended for 13+ only.'')

Ultimately, Mr. Hofmann said, he expects the app to diversify its
audience. ``We really see ourselves as a real social network, and as a
network for different age groups,'' he said.

For now, the company will have to navigate a peculiar if widely envied
situation --- capitalizing on its apparent popularity with an audience
that it cannot fully acknowledge, watched over by wary but increasingly
complicit parents.

``A year ago, there was basically nobody who was 40 years old on
Snapchat,'' Mr. Vaynerchuk said. ``If Musical.ly can hold on, they will
age up.''

Asked whether an app attracting such a young audience might be a
liability, he cast the matter of children using social apps less as an
argument than as an inevitability. Children are using these apps, he
said. And besides, in the beginning, it was the children who chose
Musical.ly, not the other way around.

What remains to be seen, he said, is whether any particular service is
too much, too soon --- something that is largely out of the company's
hands.

``There's no campaign, there's no money to be thrown at it, it will just
become something you get used to,'' Mr. Vaynerchuk said. ``There's
nothing to address. This is about social norms.''

Advertisement

\protect\hyperlink{after-bottom}{Continue reading the main story}

\hypertarget{site-index}{%
\subsection{Site Index}\label{site-index}}

\hypertarget{site-information-navigation}{%
\subsection{Site Information
Navigation}\label{site-information-navigation}}

\begin{itemize}
\tightlist
\item
  \href{https://help.nytimes.com/hc/en-us/articles/115014792127-Copyright-notice}{©~2020~The
  New York Times Company}
\end{itemize}

\begin{itemize}
\tightlist
\item
  \href{https://www.nytco.com/}{NYTCo}
\item
  \href{https://help.nytimes.com/hc/en-us/articles/115015385887-Contact-Us}{Contact
  Us}
\item
  \href{https://www.nytco.com/careers/}{Work with us}
\item
  \href{https://nytmediakit.com/}{Advertise}
\item
  \href{http://www.tbrandstudio.com/}{T Brand Studio}
\item
  \href{https://www.nytimes.com/privacy/cookie-policy\#how-do-i-manage-trackers}{Your
  Ad Choices}
\item
  \href{https://www.nytimes.com/privacy}{Privacy}
\item
  \href{https://help.nytimes.com/hc/en-us/articles/115014893428-Terms-of-service}{Terms
  of Service}
\item
  \href{https://help.nytimes.com/hc/en-us/articles/115014893968-Terms-of-sale}{Terms
  of Sale}
\item
  \href{https://spiderbites.nytimes.com}{Site Map}
\item
  \href{https://help.nytimes.com/hc/en-us}{Help}
\item
  \href{https://www.nytimes.com/subscription?campaignId=37WXW}{Subscriptions}
\end{itemize}
