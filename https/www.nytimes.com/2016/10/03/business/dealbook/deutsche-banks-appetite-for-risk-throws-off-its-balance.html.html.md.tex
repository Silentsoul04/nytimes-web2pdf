Sections

SEARCH

\protect\hyperlink{site-content}{Skip to
content}\protect\hyperlink{site-index}{Skip to site index}

\href{https://myaccount.nytimes.com/auth/login?response_type=cookie\&client_id=vi}{}

\href{https://www.nytimes.com/section/todayspaper}{Today's Paper}

\href{/section/business/dealbook}{DealBook}\textbar{}Deutsche Bank's
Appetite for Risk Throws Off Its Balance

\url{https://nyti.ms/2dnYx3h}

\begin{itemize}
\item
\item
\item
\item
\item
\end{itemize}

Advertisement

\protect\hyperlink{after-top}{Continue reading the main story}

Supported by

\protect\hyperlink{after-sponsor}{Continue reading the main story}

DealBook Business and Policy

\hypertarget{deutsche-banks-appetite-for-risk-throws-off-its-balance}{%
\section{Deutsche Bank's Appetite for Risk Throws Off Its
Balance}\label{deutsche-banks-appetite-for-risk-throws-off-its-balance}}

\includegraphics{https://static01.nyt.com/images/2016/10/03/business/db-deutschejump/db-deutschejump-articleLarge.jpg?quality=75\&auto=webp\&disable=upscale}

By \href{http://www.nytimes.com/by/landon-thomas-jr}{Landon Thomas Jr.}

\begin{itemize}
\item
  Oct. 2, 2016
\item
  \begin{itemize}
  \item
  \item
  \item
  \item
  \item
  \end{itemize}
\end{itemize}

The global banking giants --- think of JPMorgan Chase or HSBC --- make a
nice return by capturing their share of the trillions of dollars that
course through financial markets each day.

But few are as reliant on this business --- be it swapping currencies,
selling bonds or structuring derivatives --- as Deutsche Bank, the giant
lender that has made its name not as a home for German savers but as a
place for hedge funds and other risk-loving investors to put on some of
their boldest financial bets.

And that is why its swooning stock price last week set off alarm bells
in finance ministries, central bank suites and trading floors from Hong
Kong to New York.

More than eight years after the collapse of Lehman Brothers sent shock
waves around the world, the fear is whether Deutsche Bank and its highly
leveraged balance sheet of 1.6 trillion euros might teeter and set off
another bout of financial contagion.

\href{http://www.nytimes.com/2016/10/01/business/dealbook/deutsche-bank-stock-bailout.html?ref=dealbook}{Those
worries} calmed down somewhat late last week as Deutsche Bank's shares
rose after reports that the bank may be close to cutting a deal with the
United States Justice Department regarding the fine it must pay for
selling toxic mortgages during the financial crisis.

There has also been a growing realization that Deutsche Bank, even with
its thin cushion of cash, is in much better financial shape than Lehman
Brothers was. In a letter to employees on Friday, Deutsche's chief
executive, John Cryan, highlighted the ``strong fundamentals'' of the
bank.

It has €220 billion, or \$247 billion, in ready liquidity, compared with
\$45 billion for Lehman in 2007, and the bank can also tap central banks
in the United States and in Europe for a financial lifeline if need be.

That does not mean, however, that traders and regulators will stop
fretting about, among other things, the €42 trillion worth of
derivatives that sit on its books, an amount about 11 times the size of
the German economy. The market value of that figure, €18 billion, is a
lot lower, but as a headline sum it still scares.

More than all other universal banks --- those institutions that collect
deposits and also have investment banking businesses --- Deutsche Bank
has cultivated bankers with an expertise in conceiving of, promoting and
trading some of the most cutting-edge financial instruments around.

Its past leader, Anshu Jain, came to the firm in the 1990s from Merrill
Lynch and rose to the top through his success in selling derivatives to
hedge funds.

Long before he became co-chief executive in 2012, Mr. Jain and a loyal
cadre of like-minded bankers and traders consolidated power in London
(the bank's headquarters are in Frankfurt), where they established the
bank as a behemoth in the markets. Deutsche Bank became one of the few
banks large and aggressive enough to put its capital at risk by trading
and housing derivatives, securitized mortgages and credit default swaps
for those who wanted to deal in them.

``What makes Deutsche Bank systemic is their sheer size combined with
the leverage that is required to stay in the flow and be an
intermediary,'' said Anthony J. Perrotta Jr., an expert in the structure
of financial markets at TABB Group, a consulting firm. ``But as capital
becomes more scarce, this becomes a fragile equation.''

A simple calculation bears this out. Deutsche Bank has €67 billion (\$75
billion) in equity that supports assets of €1.6 trillion (\$1.8
trillion), which mean it is levered at a ratio of 25 to 1.

By comparison, JPMorgan has \$224 billion in cash backing \$2.4 trillion
in assets, which produces a far healthier ratio of 9 to 1.

Making Deutsche Bank's ratio more troubling is that many of these assets
are
\href{http://www.nytimes.com/2016/06/29/business/dealbook/hard-to-sell-assets-complicate-european-banks-brexit-risks.html}{of
the most illiquid variety}, called Level 3 securities, for which
establishing a price is guesswork and finding a buyer near impossible.

According to the last annual report, Level 3 assets stood at €32 billion
--- about half the amount of the bank's equity buffer, although that
figure has come down sharply in recent years.

To get a better sense of how these types of assets ended up on its
balance sheet, it is worth taking a closer look at the professional
makeup of Deutsche Bank's investment bank.

In good times this unit of the bank drove its profits. It was also the
investment bank that was largely responsible for the \$7.5 billion in
losses that Deutsche Bank showed last year.

Because of new European Union regulations, Deutsche Bank is required to
disclose the number of those who take significant risks with the bank's
capital and how much they are paid.

Last year, the bank reported that its investment bank employed 1,871
so-called material risk takers, or M.R.T.s as they have come to be known
in European regulatory circles, who were paid €1.7 billion.

That was down from a year earlier, when 2,057 M.R.T.s took home €2
billion.

Stuart Graham, a banking analyst with Autonomous Research in London,
says that Deutsche Bank has more high-paid risk takers than any other
bank --- including JPMorgan, Goldman Sachs and Credit Suisse.

Barclays, a longtime Deutsche rival, is also a deposit-taking bank that
relies on traders to increase revenues. But the British bank reports
having just 842 maximum risk takers in its investment bank.

While this culture of making big bets with the bank's cash may have
worked in a previous era, in today's market, where a premium is placed
on managing risk as opposed to taking it, such an aptitude becomes a
liability.

So, even though Deutsche Bank may have a €220 billion cash cushion, it
is this unknown liability that is causing some investors to play it safe
and ask for their money back.

``If you are a counterparty, you have to trim back exposure as fast as
you can,'' said David Hendler, a credit analyst with Viola Risk
Advisors.

Most analysts believe that Deutsche Bank needs to raise an extra €5
billion to €7 billion from investors in order to calm concerns about its
financial health, especially in light of the fact that the Justice
Department may require the firm to pay a penalty of at least this
amount.

But there are a few who think that for Deutsche Bank to be able to
survive a crisislike atmosphere, it needs a lot more money than that.

Robert Engle, an economist at New York University who was awarded the
Nobel Prize for his work on volatility and capital markets, has designed
a model that ranks financial institutions in terms of their systemic
risk.

The barometer takes into consideration leverage, the bank's stock price
and its equity base, and as such it represents a real-time measure of
the dangers a bank poses to the financial system at a given moment in
time.

Global regulators use it as well to complement their own internal
models.

As of now, Deutsche Bank is
\href{https://vlab.stern.nyu.edu/analysis/RISK.WORLDFIN-MR.GMES\#risk-graph}{ranked
at the top} among European banks in terms of risk, requiring close to
€100 billion in fresh cash to ensure that it could survive a sustained
sell-off in the markets.

Other European banks, like BNP Paribas and Société Générale, rank
closely behind Deutsche Bank in this regard. What sets Deutsche Bank
apart from its European and American peers is its reliance on debt, the
N.Y.U. model shows.

Since September, the bank's risk, calculated on the basis of its
leverage, has increased sharply relative to other comparable
institutions, according to the N.Y.U. gauge.

``There is just a lot of risk for Deutsche Bank right now,'' Mr. Engle
said. ``We have been worried about European banks for quite a while.''

Advertisement

\protect\hyperlink{after-bottom}{Continue reading the main story}

\hypertarget{site-index}{%
\subsection{Site Index}\label{site-index}}

\hypertarget{site-information-navigation}{%
\subsection{Site Information
Navigation}\label{site-information-navigation}}

\begin{itemize}
\tightlist
\item
  \href{https://help.nytimes.com/hc/en-us/articles/115014792127-Copyright-notice}{©~2020~The
  New York Times Company}
\end{itemize}

\begin{itemize}
\tightlist
\item
  \href{https://www.nytco.com/}{NYTCo}
\item
  \href{https://help.nytimes.com/hc/en-us/articles/115015385887-Contact-Us}{Contact
  Us}
\item
  \href{https://www.nytco.com/careers/}{Work with us}
\item
  \href{https://nytmediakit.com/}{Advertise}
\item
  \href{http://www.tbrandstudio.com/}{T Brand Studio}
\item
  \href{https://www.nytimes.com/privacy/cookie-policy\#how-do-i-manage-trackers}{Your
  Ad Choices}
\item
  \href{https://www.nytimes.com/privacy}{Privacy}
\item
  \href{https://help.nytimes.com/hc/en-us/articles/115014893428-Terms-of-service}{Terms
  of Service}
\item
  \href{https://help.nytimes.com/hc/en-us/articles/115014893968-Terms-of-sale}{Terms
  of Sale}
\item
  \href{https://spiderbites.nytimes.com}{Site Map}
\item
  \href{https://help.nytimes.com/hc/en-us}{Help}
\item
  \href{https://www.nytimes.com/subscription?campaignId=37WXW}{Subscriptions}
\end{itemize}
