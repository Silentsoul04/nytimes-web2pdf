Sections

SEARCH

\protect\hyperlink{site-content}{Skip to
content}\protect\hyperlink{site-index}{Skip to site index}

\href{https://myaccount.nytimes.com/auth/login?response_type=cookie\&client_id=vi}{}

\href{https://www.nytimes.com/section/todayspaper}{Today's Paper}

\href{/section/upshot}{The Upshot}\textbar{}Donald Trump Trashes Nafta.
But Unwinding It Would Come at a Huge Cost.

\url{https://nyti.ms/2dpE1PB}

\begin{itemize}
\item
\item
\item
\item
\item
\item
\end{itemize}

Advertisement

\protect\hyperlink{after-top}{Continue reading the main story}

Supported by

\protect\hyperlink{after-sponsor}{Continue reading the main story}

Upshot

The 2016 Race

\hypertarget{donald-trump-trashes-nafta-but-unwinding-it-would-come-at-a-huge-cost}{%
\section{Donald Trump Trashes Nafta. But Unwinding It Would Come at a
Huge
Cost.}\label{donald-trump-trashes-nafta-but-unwinding-it-would-come-at-a-huge-cost}}

\includegraphics{https://static01.nyt.com/images/2016/10/04/upshot/04UP-NAFTA/02UP-NAFTA-articleInline.jpg?quality=75\&auto=webp\&disable=upscale}

By \href{http://www.nytimes.com/by/neil-irwin}{Neil Irwin}

\begin{itemize}
\item
  Oct. 3, 2016
\item
  \begin{itemize}
  \item
  \item
  \item
  \item
  \item
  \item
  \end{itemize}
\end{itemize}

When you buy an ``American-made'' car, you are probably buying a car
that has an immensely complicated mix of components that were also made
in Mexico and Canada. The same is true for many electronics, and
advanced textiles like carpeting. The beef in your grocery store might
be from a cow that was fattened and slaughtered in the United States,
but that was very likely born across the border in Mexico.

That is the world that has evolved in the almost 23 years since the
North American Free Trade Agreement was enacted. These deep economic
interconnections show why trying to unravel what Donald J. Trump, in
last week's debate, called ``the single worst trade deal ever approved
in this country'' would be no easy feat. It would risk disrupting the
very underpinnings of industries that employ millions of Americans.

The view among mainstream economists is that Nafta, over all, has raised
incomes in the United States while also costing it thousands of
manufacturing jobs. But whether you view the agreement as a net positive
or a net negative for the country, the reality is that the United
States, Canada and Mexico are now for all practical purposes a single
integrated economy. That has wide-ranging consequences --- especially if
the next president tries to reshape or abandon the deal.

At the border between Santa Teresa, N.M., and the Mexican town of San
Jeronimo, up to 5,000 head of cattle a day amble across the border; they
are less likely to become stressed and lose weight when they walk under
their own power than when loaded into semis. After being bred in the
hills of Northern Mexico, and after eating American corn, they become a
key input for the American beef industry. It creates jobs in feedlots
and slaughterhouses in the United States, where the animals are
fattened, and produces less costly beef for consumers in the United
States and in the global markets to which the beef is exported.

A few miles away, Mexican workers in a Foxconn facility assemble Dell
computers that were designed in Texas and will be sold all over the
world. Low labor costs keep Dell on a competitive footing with global
competitors like China's Lenovo. Because of that, Dell can employ
thousands of highly paid engineers and salespeople in the United States.

And American automobile companies have supply chains that are so
thoroughly integrated across the Canadian and Mexican borders that when
huge traffic backups developed on the Ambassador Bridge between Detroit
and Windsor, Ontario, because of intensive security after the Sept. 11
terrorist attacks, Michigan auto factories were at risk of having to
shut down for want of supplies.

The auto industry is so intertwined among the three countries that it's
almost useless to think of a car as being ``made in the United States,''
even if the final assembly takes place within America's borders.

``You have what looks like an American car, with Mexican labor and
materials that went into it and Canadian materials,'' said James
Bookbinder, a professor of logistics and manufacturing at the University
of Waterloo in Ontario. ``It's really from the Nafta region, and in the
process some of those jobs created are in Mexico, but there are also
jobs in the United States because all these pieces fit together.''

As a general rule, Mexican suppliers make parts that can be done with
low-skilled labor and relatively simple assembly --- plastic bumpers,
seats or dashboards. More complex parts like engines, transmissions and
electronics components are more likely to be made in the United States
or Canada, where the workers and suppliers with more advanced skills are
in greater supply.

``A transmission is a complicated piece of machinery and might go back
and forth across the border three or four times as different components
are added at different plants,'' said Gary Clyde Hufbauer, a senior
fellow at the Peterson Institute for International Economics.

There are parallels elsewhere in the world. In Europe, German
manufacturing is extraordinarily successful on the world stage --- but
many of the more labor-intensive, lower-value inputs for German cars and
other manufactured goods come from lower-wage countries in the European
Union like Poland and Hungary.

It's true that there have been fewer auto-making jobs in the United
States since the introduction of Nafta. The 926,000 jobs making motor
vehicles and parts is down 15 percent since December 1993.

But it's hard to untangle the impact of trade and the shifting of some
work to Mexico from the advanced technology like robotics that reduces
the man-hours it takes to build a car. Overall manufacturing employment
in the United States is down 27 percent in the same span.

``There's this tendency to say, `It's all moving to Mexico,' '' said
Bernard Swiecki, a senior analyst at the Center for Automotive Research.
But the center's research finds that automakers have spent \$77 billion
on new or upgraded capital projects in the United States since 2010,
compared with \$26 billion in Mexico.

The improved competitiveness of American companies in the global auto
market since the 1990s has happened in part because they have built more
efficient North American supply chains. Different parts of the car are
made in the place where the local labor force and cost structure are the
best fit.

``If a product uses really sophisticated materials like high-strength
steels and advanced composites, and more sophisticated processes,'' Mr.
Swiecki said, there's a better chance you can find the people with the
skills needed in the United States or Canada. Meanwhile, the lower price
of labor-intensive parts imported from Mexico helps control the cost of
the overall automobile and makes it more competitive with cars built in
Europe or Asia.

Since the 1990s, major automakers based outside the United States have
built American factories to gain access to these local supply networks
and get closer to American consumers. That has to do with a lot more
than Nafta, but the connections between North American parts suppliers
across borders surely helped speed the process.

The North American auto industry is not the only sector where such a
transformation has occurred, Mr. Hufbauer says. American textile
companies make technologically advanced fabrics like those for
carpeting, parachutes or the steel mesh inside tires. They are exported
to Mexico, where the more labor-intensive work of turning those into
finished goods is done. The same goes for many electronics.

But even if you accept that these trends contribute to higher economic
output, and to more high-income jobs and more competitive companies in
the United States, it's not as if the anxiety about losing jobs that Mr.
Trump describes doesn't have a realistic basis.

After all, it's fine to say that Dell is a more competitive, successful
company by assembling computers in Juarez, which creates more
high-paying jobs for the people who design and sell those computers. But
that isn't much solace if you were one of the 905 people
\href{http://www.wral.com/business/story/6156112/}{who lost jobs} at the
plant in Winston-Salem, N.C., doing that same work.

In other words, the narrative promulgated by trade skeptics that a more
integrated global economy has worsened job opportunities for certain
workers isn't wrong. But it's also not a given that the trend will
reverse itself if the next president does seek to renegotiate Nafta.

We don't know exactly what a President Trump would do in seeking to
renegotiate Nafta, or what exactly the consequences would be of a trade
war between the United States and its partners. And a trade war could
well erupt if he were to follow through on some of his aggressive
statements.

What we do know is that even relatively small tariffs can stand in the
way of the kind of supply networks on which many modern industries are
based. With these networks, goods can cross back and forth across
national borders multiple times as part of the pipeline that leads to a
finished automobile or a computer or even a side of beef. It's not that
companies couldn't adjust; over time they could. It's that the networks
evolved this way for a reason, and readjusting would come at a
considerable cost.

More fundamentally, trying to reject the free-trade deal with America's
neighbors entirely would mean upending major industries and a
generation's worth of economic integration. Nafta has had its flaws and
downsides. But either major American industries would have to figure out
how to restructure themselves to rely less on the movement of goods
across borders, or the United States would find itself poorer and more
of an island in the global economy.

Advertisement

\protect\hyperlink{after-bottom}{Continue reading the main story}

\hypertarget{site-index}{%
\subsection{Site Index}\label{site-index}}

\hypertarget{site-information-navigation}{%
\subsection{Site Information
Navigation}\label{site-information-navigation}}

\begin{itemize}
\tightlist
\item
  \href{https://help.nytimes.com/hc/en-us/articles/115014792127-Copyright-notice}{©~2020~The
  New York Times Company}
\end{itemize}

\begin{itemize}
\tightlist
\item
  \href{https://www.nytco.com/}{NYTCo}
\item
  \href{https://help.nytimes.com/hc/en-us/articles/115015385887-Contact-Us}{Contact
  Us}
\item
  \href{https://www.nytco.com/careers/}{Work with us}
\item
  \href{https://nytmediakit.com/}{Advertise}
\item
  \href{http://www.tbrandstudio.com/}{T Brand Studio}
\item
  \href{https://www.nytimes.com/privacy/cookie-policy\#how-do-i-manage-trackers}{Your
  Ad Choices}
\item
  \href{https://www.nytimes.com/privacy}{Privacy}
\item
  \href{https://help.nytimes.com/hc/en-us/articles/115014893428-Terms-of-service}{Terms
  of Service}
\item
  \href{https://help.nytimes.com/hc/en-us/articles/115014893968-Terms-of-sale}{Terms
  of Sale}
\item
  \href{https://spiderbites.nytimes.com}{Site Map}
\item
  \href{https://help.nytimes.com/hc/en-us}{Help}
\item
  \href{https://www.nytimes.com/subscription?campaignId=37WXW}{Subscriptions}
\end{itemize}
