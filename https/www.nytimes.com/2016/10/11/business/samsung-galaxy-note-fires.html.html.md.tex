Sections

SEARCH

\protect\hyperlink{site-content}{Skip to
content}\protect\hyperlink{site-index}{Skip to site index}

\href{https://www.nytimes.com/section/business}{Business}

\href{https://myaccount.nytimes.com/auth/login?response_type=cookie\&client_id=vi}{}

\href{https://www.nytimes.com/section/todayspaper}{Today's Paper}

\href{/section/business}{Business}\textbar{}Samsung Halts Galaxy Note 7
Production as Battery Problems Linger

\url{https://nyti.ms/2dMXKcd}

\begin{itemize}
\item
\item
\item
\item
\item
\end{itemize}

Advertisement

\protect\hyperlink{after-top}{Continue reading the main story}

Supported by

\protect\hyperlink{after-sponsor}{Continue reading the main story}

\hypertarget{samsung-halts-galaxy-note-7-production-as-battery-problems-linger}{%
\section{Samsung Halts Galaxy Note 7 Production as Battery Problems
Linger}\label{samsung-halts-galaxy-note-7-production-as-battery-problems-linger}}

\includegraphics{https://static01.nyt.com/images/2016/10/11/business/11samsung/11samsung-articleInline.jpg?quality=75\&auto=webp\&disable=upscale}

By Daisuke Wakabayashi,
\href{http://www.nytimes.com/by/choe-sang-hun}{Choe Sang-Hun} and
\href{http://www.nytimes.com/by/vindu-goel}{Vindu Goel}

\begin{itemize}
\item
  Oct. 10, 2016
\item
  \begin{itemize}
  \item
  \item
  \item
  \item
  \item
  \end{itemize}
\end{itemize}

In 1995, furious over quality problems with one of his company's mobile
phones, Lee Kun-hee, the chairman of Samsung and arguably the most
famous businessman in South Korea, set a pile of 150,000 defective
phones on fire outside a factory.

The phone bonfire became a turning point for Samsung's two-decade rise
from an electronics maker associated with inexpensive knockoffs to one
considered a leader in product quality, design and sales. But to the
company's critics, that employee motivational moment has also served as
a wry historical foreshadowing of safety problems with one of Samsung's
top-selling smartphones.

The company has temporarily halted production of its Galaxy Note 7, a
high-end answer to the latest iPhones from Apple, a person familiar with
the decision said on Monday. In a statement, the company also asked
retailers and telecommunication carriers to stop selling the phones
until the problem is fixed, and said ``consumers with either an original
Galaxy Note 7 or replacement Galaxy Note 7 device should power down and
stop using the device.''

The phone has been blamed for at least one house fire, a burning Jeep
and several alarming moments on planes when the devices started smoking
mid-flight. The Federal Aviation Administration is so concerned that
airline passengers are routinely warned that they should not turn on or
charge the Galaxy Note 7 during a flight or stow the phone in checked
baggage. Southwest Airlines, which had to
\href{http://www.nytimes.com/2016/10/06/business/southwest-samsung.html}{evacuate
a plane} on Wednesday after a Samsung phone caught fire, said the
details of the incident are still being investigated.

The decision to stop selling the Galaxy Note 7 comes just five weeks
after Samsung said it would
\href{http://www.nytimes.com/2016/09/03/business/samsung-galaxy-note-battery.html}{recall
2.5 million} of them --- the largest ever in the smartphone industry ---
after early reports of battery fires.

Samsung had said it believed it had identified the issue, and allowed
consumers to
\href{http://www.nytimes.com/2016/09/20/technology/personaltech/what-to-do-if-you-have-a-samsung-galaxy-note-7.html}{trade
in their phones} for new ones. But production was halted after the four
major United States carriers said they would stop selling or replacing
Galaxy Note 7 smartphones because of additional reports of fires,
including with the replacement models.

Three of Australia's biggest telecom companies --- Telstra, Optus and
Vodafone Australia --- said they had stopped shipping Galaxy Note 7
phones to customers after reports that the replacement model had caught
fire in the United States.

The company said it hoped to provide an update within a month. The
federal Consumer Product Safety Commission praised Samsung's move and
urged consumers to stop using the phone.

The missteps by Samsung, the world's top seller of smartphones, have
given a rare opportunity to competitors like Apple to close the gap with
the South Korean giant as the holiday shopping season approaches.

``We believe this incident has destroyed billions of dollars of Samsung
brand value,'' said Laura Martin, a technology analyst with Needham \&
Company. ``The consumer says, `Which one blows up? I'm just going to
stay away from Samsung.'''

The Galaxy Note 7 featured a higher-capacity battery to help its
increasingly sophisticated features, like an iris scanner for added
security. It also supported fast wireless charging technologies. It was
the most expensive phone offered by Samsung, putting it in direct
competition with Apple's iPhone.

``Definitely, Apple is the biggest beneficiary'' of Samsung's problems,
said Linda Sui, a director at research firm Strategy Analytics.

\includegraphics{https://static01.nyt.com/images/2016/10/10/business/cnbc-att/cnbc-att-videoSixteenByNineJumbo1600.png}

What's more, Google, the company whose Android software runs on nearly
all of Samsung's smartphones, is now pushing harder to sell its own
phones. Last week, Google unveiled the Pixel --- the first smartphone
that it designed and manufactured. At the same time, aggressive
smartphone manufacturers like Huawei and Xiaomi are looking for ways to
expand beyond their footholds in China to compete with Samsung all over
the world.

It is difficult to say what the impact of the phone problems will be on
the company's overall sales. Before the recall, the research firm
Strategy Analytics had estimated that Samsung would sell 15 million Note
7 units in 2016. But now, the firm is estimating that Samsung, with
about \$180 billion in annual revenue, could lose more than \$10 billion
from the ongoing troubles.

Samsung's reputation is already taking a big hit online, according to an
analysis by Spredfast, a social media marketing firm that helps
businesses analyze chatter on Twitter and other social networks.

Since the Note 7's problems began to receive widespread attention,
negative Twitter messages about the device rose 450 percent compared to
the previous five and a half weeks, the company said.

``While this is itself a huge problem for Samsung, we also found a steep
186 percent rise in negative sentiment about Samsung itself,'' Chris
Kerns, Spredfast's vice president of research and insights said in a
statement. ``Digging deeper, it's clear that this is not just an
isolated issue with one product, but is, in fact, a full-blown brand
crisis.''

Like many Asian companies, Samsung struggled for years to establish a
strong reputation in the West. Shortly after Apple introduced the
iPhone, Samsung went headlong into the smartphone market.

Samsung had been gaining some ground in high-end smartphones with its
latest Galaxy S phones, which have curved edges and offer a premium feel
over the company's budget phones. When it released the Galaxy Note 7 in
August --- with its 5.7-inch screen and a price tag exceeding \$800 ---
it was supposed to add to that momentum.

The recurring problem has led industry experts to wonder whether the
problem went beyond sloppy production and resulted from a faulty battery
or software design.

Technology companies are hardly immune to manufacturing issues. In 1994,
Intel was forced to recall its flagship Pentium chip because of a
mathematical mistake built into it. Dell recalled more than 4 million
laptop computers in 2006 because of exploding lithium ion batteries
produced by Sony. And companies like Fitibit and Microsoft have had
manufacturing problems over the years.

Companies with strong brands can withstand product quality problems.
Over a two-year span starting in 2009, Toyota recalled about 9 million
cars because of issues related to sudden, unintended acceleration. Its
chief executive appeared before Congress, and Toyota paid a \$1.2
billion fine to the Justice Department for concealing information about
defects from consumers and government officials. In 2015, Toyota was the
world's largest automaker.

Samsung is counting on customers like Justin Brooke of Cooper City,
Fla., whose family owns three Note 7 phones as well as Samsung
televisions and tablets, to stay loyal to the brand.

Mr. Brooke said he thinks the fire risk has been overblown. He loves the
Note 7's big screen and pen feature, which he uses to critique websites
for his advertising training business, DMBI Online. ``For me as a
business owner, it's the most productive phone on the market,'' he said.

Still, he admitted to some apprehension. He said his family never
charges the batteries on their phones to 100 percent to reduce the risk
of overheating. ``Maybe we're in denial,'' he said.

He said his father asked him for a phone recommendation on Sunday night,
and he recommended another Samsung model, the S7, which has not been
implicated in the fires, or a Google Pixel phone.

Advertisement

\protect\hyperlink{after-bottom}{Continue reading the main story}

\hypertarget{site-index}{%
\subsection{Site Index}\label{site-index}}

\hypertarget{site-information-navigation}{%
\subsection{Site Information
Navigation}\label{site-information-navigation}}

\begin{itemize}
\tightlist
\item
  \href{https://help.nytimes.com/hc/en-us/articles/115014792127-Copyright-notice}{©~2020~The
  New York Times Company}
\end{itemize}

\begin{itemize}
\tightlist
\item
  \href{https://www.nytco.com/}{NYTCo}
\item
  \href{https://help.nytimes.com/hc/en-us/articles/115015385887-Contact-Us}{Contact
  Us}
\item
  \href{https://www.nytco.com/careers/}{Work with us}
\item
  \href{https://nytmediakit.com/}{Advertise}
\item
  \href{http://www.tbrandstudio.com/}{T Brand Studio}
\item
  \href{https://www.nytimes.com/privacy/cookie-policy\#how-do-i-manage-trackers}{Your
  Ad Choices}
\item
  \href{https://www.nytimes.com/privacy}{Privacy}
\item
  \href{https://help.nytimes.com/hc/en-us/articles/115014893428-Terms-of-service}{Terms
  of Service}
\item
  \href{https://help.nytimes.com/hc/en-us/articles/115014893968-Terms-of-sale}{Terms
  of Sale}
\item
  \href{https://spiderbites.nytimes.com}{Site Map}
\item
  \href{https://help.nytimes.com/hc/en-us}{Help}
\item
  \href{https://www.nytimes.com/subscription?campaignId=37WXW}{Subscriptions}
\end{itemize}
