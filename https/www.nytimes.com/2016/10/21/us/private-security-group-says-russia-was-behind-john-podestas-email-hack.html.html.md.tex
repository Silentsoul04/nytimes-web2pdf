Sections

SEARCH

\protect\hyperlink{site-content}{Skip to
content}\protect\hyperlink{site-index}{Skip to site index}

\href{https://www.nytimes.com/section/us}{U.S.}

\href{https://myaccount.nytimes.com/auth/login?response_type=cookie\&client_id=vi}{}

\href{https://www.nytimes.com/section/todayspaper}{Today's Paper}

\href{/section/us}{U.S.}\textbar{}Private Security Group Says Russia Was
Behind John Podesta's Email Hack

\url{https://nyti.ms/2eqNSVY}

\begin{itemize}
\item
\item
\item
\item
\item
\end{itemize}

Advertisement

\protect\hyperlink{after-top}{Continue reading the main story}

Supported by

\protect\hyperlink{after-sponsor}{Continue reading the main story}

\hypertarget{private-security-group-says-russia-was-behind-john-podestas-email-hack}{%
\section{Private Security Group Says Russia Was Behind John Podesta's
Email
Hack}\label{private-security-group-says-russia-was-behind-john-podestas-email-hack}}

\includegraphics{https://static01.nyt.com/images/2016/10/21/us/21hack/21hack-articleLarge.jpg?quality=75\&auto=webp\&disable=upscale}

By \href{http://www.nytimes.com/by/nicole-perlroth}{Nicole Perlroth} and
\href{http://www.nytimes.com/by/michael-d-shear}{Michael D. Shear}

\begin{itemize}
\item
  Oct. 20, 2016
\item
  \begin{itemize}
  \item
  \item
  \item
  \item
  \item
  \end{itemize}
\end{itemize}

SAN FRANCISCO --- At the start of 2014, President Obama assigned his
trusted counselor, John D. Podesta, to lead a review of the digital
revolution, its potential and its perils. When Mr. Podesta presented his
findings five months later, he called the internet's onslaught of big
data ``a historic driver of progress.'' But two short years later, as
chairman of Hillary Clinton's presidential campaign, Mr. Podesta would
also become one of the internet's most notable victims.

On Thursday, private security researchers said they had concluded that
Mr. Podesta was hacked by Russia's foreign intelligence service, the
GRU, after it tricked him into clicking on a fake Google login page last
March, inadvertently handing over his digital credentials.

For months, the hackers mined Mr. Podesta's inbox for his most sensitive
and potentially embarrassing correspondence, much of which has been
posted on the WikiLeaks website. Additions to the collection on Thursday
included three short email exchanges between Mr. Podesta and Mr. Obama
himself in the days leading up to his election in 2008.

Mr. Podesta's emails were first published by WikiLeaks earlier this
month. The release came just days after James R. Clapper Jr., the
director of national intelligence, and the Department of Homeland
Security publicly blamed Russian officials for cyberattacks on the
Democratic National Committee, in what they described as an effort to
influence the American presidential election.

To date, no government officials have offered evidence that the same
Russian hackers behind the D.N.C. cyberattacks were also behind the hack
of Mr. Podesta's emails, but an investigation by the private security
researchers determined that they were the same.

Threat researchers at Dell SecureWorks, an Atlanta-based security firm,
had been tracking the Russian intelligence group for more than a year.
In June, they reported that they had uncovered a critical tool in the
Russian spy campaign. SecureWorks researchers found that the Russian
hackers were using a popular link shortening service, called Bitly, to
shorten malicious links they used to send targets fake Google login
pages to bait them into submitting their email credentials.

The hackers made a critical error by leaving some of their Bitly
accounts public, making it possible for SecureWorks to trace 9,000 of
their links to nearly 4,000 Gmail accounts targeted between October 2015
and May 2016 with fake Google login pages and security alerts designed
to trick users into turning over their passwords.

Among the list of targets were more than 100 email addresses associated
with Hillary Clinton's presidential campaign, including Mr. Podesta's.
By June, 20 staff members for the campaign had clicked on the short
links sent by Russian spies. In June, SecureWorks disclosed that among
those whose email accounts had been targeted were staff members who
advised Mrs. Clinton on policy and managed her travel, communications
and campaign finances.

Two security researchers who have been tracking the GRU's spearphishing
campaign confirmed Thursday that Mr. Podesta was among those who had
inadvertently turned over his Google email password. The fact that Mr.
Podesta was among those breached by the GRU was
\href{http://www.esquire.com/news-politics/a49791/russian-dnc-emails-hacked/}{first
disclosed}Thursday by Esquire and the Motherboard blog, which published
the link Russian spies used against Mr. Podesta.

``The new public data confirming the Russians are behind the hack of
John Podesta's email is a big deal,'' Jake Sullivan, Mrs. Clinton's
senior policy adviser, said Thursday. ``There is no longer any doubt
that Putin is trying to help Donald Trump by weaponizing WikiLeaks.''

The new release of Mr. Podesta's email exchange with Mr. Obama from 2008
made clear that Mr. Obama's team was confident he would win.

In one of the emails, Mr. Podesta wrote Mr. Obama a lengthy memo in the
evening on Election Day recommending that he not accept an invitation
from President George W. Bush to attend an emergency meeting of the
Group of 20 leaders.

``Attendance alongside President Bush will create an extremely awkward
situation,'' the memo said. ``If you attempt to dissociate yourself from
his positions, you will be subject to criticism for projecting a divided
United States to the rest of the world. But if you adopt a more reserved
posture, you will be associated not only with his policies, but also
with his very tenuous global standing.''

The White House did not respond to questions about the email.

Advertisement

\protect\hyperlink{after-bottom}{Continue reading the main story}

\hypertarget{site-index}{%
\subsection{Site Index}\label{site-index}}

\hypertarget{site-information-navigation}{%
\subsection{Site Information
Navigation}\label{site-information-navigation}}

\begin{itemize}
\tightlist
\item
  \href{https://help.nytimes.com/hc/en-us/articles/115014792127-Copyright-notice}{©~2020~The
  New York Times Company}
\end{itemize}

\begin{itemize}
\tightlist
\item
  \href{https://www.nytco.com/}{NYTCo}
\item
  \href{https://help.nytimes.com/hc/en-us/articles/115015385887-Contact-Us}{Contact
  Us}
\item
  \href{https://www.nytco.com/careers/}{Work with us}
\item
  \href{https://nytmediakit.com/}{Advertise}
\item
  \href{http://www.tbrandstudio.com/}{T Brand Studio}
\item
  \href{https://www.nytimes.com/privacy/cookie-policy\#how-do-i-manage-trackers}{Your
  Ad Choices}
\item
  \href{https://www.nytimes.com/privacy}{Privacy}
\item
  \href{https://help.nytimes.com/hc/en-us/articles/115014893428-Terms-of-service}{Terms
  of Service}
\item
  \href{https://help.nytimes.com/hc/en-us/articles/115014893968-Terms-of-sale}{Terms
  of Sale}
\item
  \href{https://spiderbites.nytimes.com}{Site Map}
\item
  \href{https://help.nytimes.com/hc/en-us}{Help}
\item
  \href{https://www.nytimes.com/subscription?campaignId=37WXW}{Subscriptions}
\end{itemize}
