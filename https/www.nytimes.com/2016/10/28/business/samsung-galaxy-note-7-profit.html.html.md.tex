Sections

SEARCH

\protect\hyperlink{site-content}{Skip to
content}\protect\hyperlink{site-index}{Skip to site index}

\href{https://www.nytimes.com/section/business}{International Business}

\href{https://myaccount.nytimes.com/auth/login?response_type=cookie\&client_id=vi}{}

\href{https://www.nytimes.com/section/todayspaper}{Today's Paper}

\href{/section/business}{International Business}\textbar{}Samsung's
Galaxy Note 7 Debacle Wipes Out Its Mobile Profit

\url{https://nyti.ms/2dMBZaH}

\begin{itemize}
\item
\item
\item
\item
\item
\end{itemize}

Advertisement

\protect\hyperlink{after-top}{Continue reading the main story}

Supported by

\protect\hyperlink{after-sponsor}{Continue reading the main story}

\hypertarget{samsungs-galaxy-note-7-debacle-wipes-out-its-mobile-profit}{%
\section{Samsung's Galaxy Note 7 Debacle Wipes Out Its Mobile
Profit}\label{samsungs-galaxy-note-7-debacle-wipes-out-its-mobile-profit}}

\includegraphics{https://static01.nyt.com/images/2016/10/28/world/28SAMSUNG-web1/28SAMSUNG-web1-articleLarge.jpg?quality=75\&auto=webp\&disable=upscale}

By \href{https://www.nytimes.com/by/paul-mozur}{Paul Mozur}

\begin{itemize}
\item
  Oct. 27, 2016
\item
  \begin{itemize}
  \item
  \item
  \item
  \item
  \item
  \end{itemize}
\end{itemize}

HONG KONG --- It may well take years to tabulate the full damage
Samsung's discontinuation of the Galaxy Note 7 will have on the
company's brand.

But on Thursday, the South Korean electronics giant offered a quick
glimpse of just how costly its combustible phone has been.

The Galaxy Note 7 debacle virtually wiped out profits at its
high-profile mobile communications division during the third quarter,
Samsung said. The division posted an operating profit of \$87.9 million.
By contrast, a year ago the division posted a profit of about \$2.1
billion.

In a statement, Samsung said it hoped that in the next quarter its
mobile operating profits would return to last year's levels, also around
\$2 billion. Previously, Samsung had estimated the cost of the
discontinuation of the Galaxy Note 7 after a number of the phones caught
fire to be around \$3 billion.

Yet, in a reminder that the world's largest maker of smartphones still
has a thriving component and consumer electronics business, Samsung
posted a net profit of more than \$4 billion, in alignment with its
revised forecast following the discontinuation of the Note 7.

Earlier this month, Samsung killed the Galaxy Note 7, a large and
high-priced smartphone that it believed offered its latest chance to
gain ground against its longtime rival, Apple, maker of the iPhone. In
September
\href{http://www.nytimes.com/2016/09/16/business/samsung-galaxy-note-recall.html}{it
voluntarily recalled} more than two million Note 7s following reports
that some caught fire, but as the reports
\href{http://www.nytimes.com/2016/10/12/business/international/samsung-galaxy-note7-terminated.html}{continued
to pour in} it canceled the phone entirely.

During a speech at Samsung's general shareholder meeting, J. K. Shin,
the company's president, said the company was working with regulators
and third-party experts to attempt to diagnose the problems that have
led a small number of the phones to burst into flames.

He said that of the more than 300 incidents reported with the Galaxy
Note 7, a large number were found to have been caused by the battery.
But a smaller portion were the result of some problem other than the
battery.

Along with scrutinizing the batteries, Mr. Shin said Samsung was
``carefully revisiting every aspect of the device, such as its hardware,
software and manufacturing processes, that could possibly have caused
the incident.''

Mr. Shin also apologized to shareholders and consumers and said it was
``not acceptable'' that the company failed to meet its own quality
assurance standards.

As the Note 7 story continues to unfold, China's aviation regulator said
this week that it was banning the phone from all flights. Samsung also
said it was preparing a new software upgrade that would prevent the Note
7 phones from charging more than 60 percent --- a way to both keep
phones still in the market from overheating and a way to get customers
to turn them in.

Samsung said the phones would be automatically updated wirelessly, and
the measure was designed to encourage those who still were holding on to
their Note 7 phones to turn the handsets in ``as soon as possible.''

Advertisement

\protect\hyperlink{after-bottom}{Continue reading the main story}

\hypertarget{site-index}{%
\subsection{Site Index}\label{site-index}}

\hypertarget{site-information-navigation}{%
\subsection{Site Information
Navigation}\label{site-information-navigation}}

\begin{itemize}
\tightlist
\item
  \href{https://help.nytimes.com/hc/en-us/articles/115014792127-Copyright-notice}{©~2020~The
  New York Times Company}
\end{itemize}

\begin{itemize}
\tightlist
\item
  \href{https://www.nytco.com/}{NYTCo}
\item
  \href{https://help.nytimes.com/hc/en-us/articles/115015385887-Contact-Us}{Contact
  Us}
\item
  \href{https://www.nytco.com/careers/}{Work with us}
\item
  \href{https://nytmediakit.com/}{Advertise}
\item
  \href{http://www.tbrandstudio.com/}{T Brand Studio}
\item
  \href{https://www.nytimes.com/privacy/cookie-policy\#how-do-i-manage-trackers}{Your
  Ad Choices}
\item
  \href{https://www.nytimes.com/privacy}{Privacy}
\item
  \href{https://help.nytimes.com/hc/en-us/articles/115014893428-Terms-of-service}{Terms
  of Service}
\item
  \href{https://help.nytimes.com/hc/en-us/articles/115014893968-Terms-of-sale}{Terms
  of Sale}
\item
  \href{https://spiderbites.nytimes.com}{Site Map}
\item
  \href{https://help.nytimes.com/hc/en-us}{Help}
\item
  \href{https://www.nytimes.com/subscription?campaignId=37WXW}{Subscriptions}
\end{itemize}
