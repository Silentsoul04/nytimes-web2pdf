Sections

SEARCH

\protect\hyperlink{site-content}{Skip to
content}\protect\hyperlink{site-index}{Skip to site index}

\href{https://www.nytimes.com/section/business/economy}{Economy}

\href{https://myaccount.nytimes.com/auth/login?response_type=cookie\&client_id=vi}{}

\href{https://www.nytimes.com/section/todayspaper}{Today's Paper}

\href{/section/business/economy}{Economy}\textbar{}How Economic Data Is
Kept Politics-Free

\url{https://nyti.ms/2e5gAyA}

\begin{itemize}
\item
\item
\item
\item
\item
\end{itemize}

Advertisement

\protect\hyperlink{after-top}{Continue reading the main story}

Supported by

\protect\hyperlink{after-sponsor}{Continue reading the main story}

\hypertarget{how-economic-data-is-kept-politics-free}{%
\section{How Economic Data Is Kept
Politics-Free}\label{how-economic-data-is-kept-politics-free}}

\includegraphics{https://static01.nyt.com/images/2016/11/04/business/04STATS/04STATS-articleLarge.jpg?quality=75\&auto=webp\&disable=upscale}

By \href{http://www.nytimes.com/by/patricia-cohen}{Patricia Cohen}

\begin{itemize}
\item
  Nov. 3, 2016
\item
  \begin{itemize}
  \item
  \item
  \item
  \item
  \item
  \end{itemize}
\end{itemize}

Talk about an October surprise. The post on Twitter warning of a
government conspiracy to swing the presidential election in the
Democrats' favor popped up just moments after the Labor Department
reported the biggest monthly decline in the unemployment rate in nine
years.

``Unbelievable jobs numbers..these Chicago guys will do anything,'' it
said, referring to the Obama administration's crew. ``Can't debate so
change numbers.''

The author of the
\href{https://twitter.com/jack_welch/status/254198154260525057}{Oct. 5
post} was not Donald J. Trump, the Republican presidential nominee and
irrepressible Twitter user. It was
\href{http://money.cnn.com/2012/10/05/news/economy/welch-unemployment-rate/}{Jack
Welch}, the legendary retired chief executive of General Electric,
reacting to good economic news in the weeks running up to the 2012
face-off between President Obama and his Republican challenger, Mitt
Romney.

Mr. Welch later admitted that he had no evidence that any numbers had
been cooked --- only that the reported improvements did not match his
own observations of the economy.

Such suspicions about even the routine, day-in and day-out economic
statistics produced by the federal government, voiced by a scattershot
of skeptics in previous years, have turned into a steady roar this
campaign season.

With Mr. Trump insisting, wrongly, that the United States is ``losing
jobs to other countries at a higher rate than ever,'' it may not be a
surprise that nearly half of Mr. Trump's supporters ``completely
distrust the economic data reported by the federal government,''
according to
\href{http://www.marketplace.org/2016/10/13/economy/americans-economic-anxiety-has-reached-new-high}{a
recent Marketplace-Edison Research survey}. (By contrast, 5 percent of
those planning to vote for Hillary Clinton say they distrust the
government information.)

Decades of
\href{http://www.theatlantic.com/magazine/archive/2011/12/i-was-wrong-and-so-are-you/308713/}{psychological
research} have shown that people, regardless of political affinity, tend
to
\href{https://www.psychologytoday.com/blog/science-choice/201504/what-is-confirmation-bias}{embrace
information that confirms their existing beliefs and disregard data that
contradicts them}.

``Partisans tend to credit the information when it reflects well on
their leaders and dismiss it when it doesn't,'' said
\href{http://www.culturalcognition.net/kahan/}{Dan M. Kahan}, a law and
psychology professor at Yale University.

``People don't know where these figures come from, they don't know what
they mean,'' Professor Kahan said. ``They just have an emotional `yay'
or `boo' response to them and anything else that they recognize as
having a political significance.''

Earlier this year, Mr. Trump said, ``Don't believe those phony
numbers,'' contending that the jobless rate was ``probably 28, 29, as
high as 35. In fact, I even heard recently 42 percent.'' More recently,
he declared the official 5 percent jobless rate ``one of the biggest
hoaxes in American modern politics.''

By affirming that view, Trump supporters are in effect signaling: ``I'm
with him.''

So how reliable is the government data on employment, which will be
reported again on Friday? Like all statistical measurements, it can be
both honest and imprecise; a best estimate given the available tools but
nonetheless subject to ambiguity, misinterpretation and error.

``Every data collection comes with a set of strengths and weaknesses,''
said Karen Kosanovich, an economist and 24-year veteran of the Bureau of
Labor Statistics. ``That's part of the business of collecting
information.''

There are some basic ground rules, however, that prevent the process
from spitting out any answers you please and undermine claims that the
results are rigged for a political purpose.

For starters, the people who generate the numbers are all career civil
servants who have churned out reports for both Republicans and
Democrats. And their basic methods do not swerve from one administration
to the next. If the figures are biased, they are consistently biased in
the same way regardless of what party is in office.

``I've never had any outside influence that tells me what to do or how
to collect and interpret information,'' Ms. Kosanovich said. ``Our
approach is based on methodologies that have been proven over time and
approved statistical practices. They are not based on political
influence.''

Image

Jack Welch, the former chief of General Electric, caused a stir in 2012
by questioning favorable jobs data that did not match his personal
observations.Credit...Richard Drew/Associated Press

The \href{http://www.bls.gov/cps/cps_htgm.pdf}{monthly employment
report} is derived from
\href{http://www.bls.gov/cps/documentation.htm}{two separate surveys}
that serve different purposes. The first, collected by the Census Bureau
since 1942, allows the government to estimate the number of people who
are employed and calculate the unemployment rate.

The most common misconception among the public, Ms. Kosanovich said, is
that the unemployment rate is determined solely by counting people who
are receiving unemployment insurance benefits. That is not the case.

Rather, it is based on what is known as the Current Population Survey, a
monthly survey of 60,000 households, or about 110,000 individuals from
all around the country. (By comparison, most respected public opinion
polls depend on a random sample of 1,000 to 2,000 people).

In general, anyone who reports working for pay --- even just an hour ---
during the previous week is considered employed. Anyone who was laid off
or actively looked for work (sending out résumés, responding to
help-wanted ads) during the previous four weeks, regardless of whether
they received any government benefits, is considered unemployed. People
who are not looking --- this includes millions of students enrolled in
college, plenty of parents who are happy to stay home with young
children, and millions more retirees --- are not counted as being in the
labor force.

The \href{http://www.bls.gov/ces/}{second survey} is designed to measure
something else: employment changes over time. This payroll survey
focuses solely on jobs, rather than individuals. Thus, a single person
working two jobs would be counted once by the household survey (one
individual is employed) and twice by the payroll survey (two jobs
exist).

To estimate how many jobs were created and lost, the labor bureau each
month gathers data from 146,000 private business and government agencies
covering about 623,000 work sites throughout the country.

So what gives the best picture of the job market? That depends. The
survey of employers, started in a bare-bones form more than a century
ago, is considered a more reliable measure than the household survey, in
part because the sample size is much larger. But it does not pick up all
the types of jobs (the self-employed, unpaid family workers, domestic
help and agricultural workers) or answer questions about workers' race,
ethnicity, age and educational level. The household survey helps fill in
those gaps.

The official jobless rate comes from the household survey and represents
the number of people who are 1) in the labor force, and 2) unemployed.
In September that was 7.9 million Americans out of the 160 million in
the labor force, or 5 percent.

Many economists, however, consider the official unemployment rate to be
an inadequate gauge of what people are experiencing. It does not take
into account people who are working only part time because they cannot
find a full-time job. Or the former steelworker who, after years of
fruitless searching, has given up looking but would take a job if he
could find one.

The bureau publishes alternative measures of unemployment to help
capture this reality. The broadest one, which includes both discouraged
and underemployed workers, tends to rise and fall with the official rate
but is always larger. In September, the measure, known as
\href{http://www.bls.gov/news.release/empsit.t15.htm}{U-6}, was 9.7
percent.

Even that broader figure does not capture the full picture. In 2003, for
example,
\href{http://www.nytimes.com/2003/11/30/opinion/the-index-of-missing-economic-indicators-the-unemployment-myth.html}{Austan
Goolsbee}, who later went on to serve as a top economic adviser in the
Obama White House, complained that the jobless rate was understated
because of government programs. Social Security disability, in
particular, he argued, had ``effectively been buying people off the
unemployment rolls and reclassifying them as `not in the labor force.'
''

Mr. Goolsbee called this ``a kind of invisible unemployment'' and noted
that ``underreporting unemployment has served the interests of both
political parties.''

Since then, concerns about the shrinking size of the labor force have
sharpened. The proportion of Americans officially in the labor force has
failed to recover significantly after taking a steep dive during the
recession. In 2008, about 66 percent of the population was actively
looking or working; now less than 63 percent is.

Some people have willingly made the choice to stop working. But many,
particularly those in their prime working years, are missing from the
labor force. Recent
\href{http://www.nytimes.com/2016/10/17/opinion/millions-of-men-are-missing-from-the-job-market.html}{research}
by \href{http://www.nytimes.com/by/alan-b-krueger?inline=nyt-per}{Alan
B. Krueger}, a Princeton economist and former Obama administration
adviser, found that nearly half of the seven million men between 25 and
54 not in the labor force were on daily painkillers or disabled.

``Just because people left the work force out of discouragement, doesn't
mean they're not available for work,'' said Patrick J. O'Keefe, director
of economic policy at CohnReznick and a former deputy assistant
secretary in the Labor Department. ``It just means the economy is not
generating sufficient jobs at sufficient pay levels to get them back
in.''

Positive or negative spin, however, is not part of the Labor
Department's brief, Ms. Kosanovich said. She repeated a comment made by
Kathleen Utgoff, a former Bureau of Labor Statistics commissioner
appointed by George W. Bush, that serves as the agency's unofficial
motto: When asked whether the glass is half full or half empty, the
bureau's response is, It's an eight-ounce glass with four ounces of
liquid.

Advertisement

\protect\hyperlink{after-bottom}{Continue reading the main story}

\hypertarget{site-index}{%
\subsection{Site Index}\label{site-index}}

\hypertarget{site-information-navigation}{%
\subsection{Site Information
Navigation}\label{site-information-navigation}}

\begin{itemize}
\tightlist
\item
  \href{https://help.nytimes.com/hc/en-us/articles/115014792127-Copyright-notice}{©~2020~The
  New York Times Company}
\end{itemize}

\begin{itemize}
\tightlist
\item
  \href{https://www.nytco.com/}{NYTCo}
\item
  \href{https://help.nytimes.com/hc/en-us/articles/115015385887-Contact-Us}{Contact
  Us}
\item
  \href{https://www.nytco.com/careers/}{Work with us}
\item
  \href{https://nytmediakit.com/}{Advertise}
\item
  \href{http://www.tbrandstudio.com/}{T Brand Studio}
\item
  \href{https://www.nytimes.com/privacy/cookie-policy\#how-do-i-manage-trackers}{Your
  Ad Choices}
\item
  \href{https://www.nytimes.com/privacy}{Privacy}
\item
  \href{https://help.nytimes.com/hc/en-us/articles/115014893428-Terms-of-service}{Terms
  of Service}
\item
  \href{https://help.nytimes.com/hc/en-us/articles/115014893968-Terms-of-sale}{Terms
  of Sale}
\item
  \href{https://spiderbites.nytimes.com}{Site Map}
\item
  \href{https://help.nytimes.com/hc/en-us}{Help}
\item
  \href{https://www.nytimes.com/subscription?campaignId=37WXW}{Subscriptions}
\end{itemize}
