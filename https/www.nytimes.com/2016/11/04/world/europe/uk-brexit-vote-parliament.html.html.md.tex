Sections

SEARCH

\protect\hyperlink{site-content}{Skip to
content}\protect\hyperlink{site-index}{Skip to site index}

\href{https://www.nytimes.com/section/world/europe}{Europe}

\href{https://myaccount.nytimes.com/auth/login?response_type=cookie\&client_id=vi}{}

\href{https://www.nytimes.com/section/todayspaper}{Today's Paper}

\href{/section/world/europe}{Europe}\textbar{}`Brexit' Will Require a
Vote in Parliament, U.K. Court Rules

\url{https://nyti.ms/2ehcHSo}

\begin{itemize}
\item
\item
\item
\item
\item
\item
\end{itemize}

Advertisement

\protect\hyperlink{after-top}{Continue reading the main story}

Supported by

\protect\hyperlink{after-sponsor}{Continue reading the main story}

\hypertarget{brexit-will-require-a-vote-in-parliament-uk-court-rules}{%
\section{`Brexit' Will Require a Vote in Parliament, U.K. Court
Rules}\label{brexit-will-require-a-vote-in-parliament-uk-court-rules}}

\includegraphics{https://static01.nyt.com/images/2016/11/04/world/04Brexit-web2/04Brexit-web2-videoSixteenByNine3000.jpg}

By \href{http://www.nytimes.com/by/stephen-castle}{Stephen Castle} and
\href{http://www.nytimes.com/by/steven-erlanger}{Steven Erlanger}

\begin{itemize}
\item
  Nov. 3, 2016
\item
  \begin{itemize}
  \item
  \item
  \item
  \item
  \item
  \item
  \end{itemize}
\end{itemize}

LONDON --- The British government's plan for
\href{http://www.nytimes.com/2016/06/21/world/europe/brexit-britain-eu-explained.html}{leaving
the European Union} was thrown into uncertainty on Thursday after the
High Court
\href{https://www.judiciary.gov.uk/judgments/r-miller-v-secretary-of-state-for-exiting-the-european-union/}{ruled}
that Parliament must give its approval before the process can begin.

The court's decision seemed likely to slow --- but not halt --- the
British withdrawal from the bloc, a step approved by
\href{http://www.bbc.com/news/politics/eu_referendum/results}{nearly 52
percent of voters} in a June referendum.

Nevertheless, the court's decision was a significant blow to Prime
Minister Theresa May. She had planned to begin the
\href{http://www.nytimes.com/2016/11/04/world/europe/what-is-article-50-brexit-uk.html}{legal
steps} for leaving the European Union
\href{http://www.nytimes.com/2016/10/03/world/europe/brexit-talks-march-theresa-may-britain.html}{by
the end of March}, and to prepare for the negotiations over Britain's
exit mostly behind closed doors.

If the court's ruling is upheld --- the government immediately vowed to
appeal --- that plan would be thrown into disarray, analysts said.

\includegraphics{https://static01.nyt.com/images/2016/11/04/world/04Brexit-web/04Brexit-web-articleLarge.jpg?quality=75\&auto=webp\&disable=upscale}

Mrs. May would be forced to work with Parliament and consider its
competing priorities for Britain's future. Specifically, she would have
to give it a detailed strategy for negotiating the British departure, or
``\href{http://www.nytimes.com/2016/06/21/world/europe/brexit-britain-eu-explained.html}{Brexit}.''
She has adamantly resisted doing so, arguing that this would impede her
flexibility in the negotiations, preventing Britain from getting the
best possible deal.

Few observers say they believe that Parliament will go so far as to
prevent a departure from the bloc. Lawmakers themselves voted
overwhelmingly to hold the referendum and pledged to abide by the
results.

The more likely impact could be to weaken Mrs. May's hold on the
negotiating process. Her main priority has been controlling immigration
and Britain's borders, even if that hurts the economy by forcing her
nation to leave the European Union's
\href{https://ec.europa.eu/growth/single-market_en}{single market} --- a
``hard Brexit.''

The court's decision may ultimately force her to compromise, a prospect
that
\href{http://www.telegraph.co.uk/business/2016/11/03/pound-smashes-123-ahead-of-brexit-high-court-verdict-and-bank-of/}{led
the pound to rise} Thursday morning, lifting it from the multidecade
lows it had been plumbing in recent weeks.

But it was not immediately clear how the politics would play out. The
Conservative government is already split over what kind of future
relationship it wants with the European Union, and in general, members
of Parliament were not in favor of leaving the bloc in the first place.
The government hoped to get the talks started without a major
parliamentary debate and potential interference, especially in the House
of Lords, where the Conservatives do not have a clear majority.

If Mrs. May should find parliamentary opposition intolerable, she might
ultimately be tempted to seek an early general election to gain a wider
mandate for leaving the bloc, some analysts said. Currently, her
Conservative Party holds a slim majority, with
\href{http://www.parliament.uk/mps-lords-and-offices/mps/current-state-of-the-parties/}{329
seats} in the 650-seat Parliament, and many of those members opposed a
withdrawal.

The ruling unsettled the proponents of exiting the European Union, who
warned against backsliding.
\href{http://www.nytimes.com/2016/07/05/world/europe/nigel-farage-ukip-brexit.html}{Nigel
Farage}, who resigned as leader of the nationalist U.K. Independence
Party after the referendum, said he feared that Britain was heading for
a ``half Brexit,'' and he said he would return to politics in 2019 if
the country had not left the bloc by then.

``I see M.P.s from all parties saying, `Oh well, actually we should stay
part of the single market; we should continue with our daily financial
contributions,' '' he said in an interview on BBC Radio. ``I think we
could be at the beginning, with this ruling, of a process where there is
a deliberate, willful attempt by our political class to betray 17.4
million voters.''

On Thursday, the government said that an expedited appeal would be heard
in December by the Supreme Court, Britain's highest appellate body, and
that it was sticking to its timetable for leaving the bloc for now. Yet
in the growing environment of constitutional, legal and political
uncertainty, the government's strategy could easily be disrupted.

The ruling was ``a severe setback for Theresa May's government,'' said
Mujtaba Rahman, managing director for Europe at the Eurasia Group, a
political consulting firm. But he added that the government's timetable
could still be met if the Supreme Court ruled in its favor.

\href{https://www.nytimes.com/interactive/2016/11/03/world/europe/article-50-high-court-ruling.html}{}

\includegraphics{https://static01.nyt.com/images/2016/11/03/world/europe/image-Summary-of-High-Court-Ruling-on-Article-50/image-Summary-of-High-Court-Ruling-on-Article-50-largeHorizontalJumbo-v2.gif}

\hypertarget{summary-of-high-court-ruling-on-article-50-and-brexit}{%
\subsection{Summary of High Court Ruling on Article 50 and
`Brexit'}\label{summary-of-high-court-ruling-on-article-50-and-brexit}}

The British government must consult Parliament before proceeding with
formal negotiations over its withdrawal from the European Union, the
High Court ruled. The legal action was brought in the name of several
individuals.

Although Parliament approved holding the referendum, Mrs. May's critics
argued in court that failing to give lawmakers a voice would turn them
into bystanders as Britain negotiated its disengagement from the bloc.
They also pointed out that, technically, the referendum is not legally
binding.

The case, brought by several plaintiffs, including
\href{http://www.telegraph.co.uk/news/2016/10/13/who-is-gina-miller-the-woman-leading-the-brexit-legal-battle/}{Gina
Miller}, an investment fund manager, and
\href{http://www.independent.co.uk/news/uk/home-news/brexit-legal-challenge-deir-dos-santos-speech-high-court-david-greene-theresa-may-plans-voting-to-a7395096.html}{Deir
Dos Santos}, a hairdresser, is a constitutional one, about the powers
vested in the government, the crown and Parliament. The case is not
about whether Britain will or will not leave the European Union, but
about the procedure for invoking Article 50, the legal mechanism for
leaving the bloc, which provides a two-year period for negotiations.

The plaintiffs argued successfully that leaving the European Union
involved the revocation of certain rights granted to Britons by
Parliament, and that lawmakers must have a say and a vote before Article
50 is invoked.

In his ruling, the lord chief justice, John Thomas, said, ``The most
fundamental rule of the U.K. Constitution is that Parliament is
sovereign and can make or unmake any law it chooses.''

Oddly enough, this was precisely the case made by those who wanted to
end membership, who argued that only by leaving the European Union could
Parliament's sovereignty be completely restored. Now that same argument
could delay the very exit so desired by those politicians and their
supporters.

\href{https://www.nytimes.com/interactive/2016/business/international/brexit-uk-what-happens-business.html}{}

\includegraphics{https://static01.nyt.com/images/2017/03/29/business/27BREXIT/27BREXIT-articleLarge.jpg}

\hypertarget{how-brexit-could-change-business-in-britain}{%
\subsection{How `Brexit' Could Change Business in
Britain}\label{how-brexit-could-change-business-in-britain}}

Britain has started the clock on leaving the European Union, and will be
out of the bloc by March 2019. Here is how ``Brexit'' has affected
business so far.

The government argued that under residual powers of royal prerogative,
which cover international treaty-making, it had the power to invoke
Article 50 without a vote in Parliament.

But the court found that invoking Article 50 would essentially repeal
the \href{http://www.legislation.gov.uk/ukpga/1972/68/contents}{European
Communities Act}, a 1972 law that allowed for the incorporation of
European law into the British legal system, and that only Parliament had
the power to do so.

Tim Farron, leader of the Liberal Democrats, welcomed the ruling, adding
that it was ``disappointing that this government was so intent on
undermining parliamentary sovereignty and democratic process that they
forced this decision to be made in the court.''

In a statement, he added, ``Given the strict two-year timetable of
exiting the E.U. once Article 50 is triggered, it is critical that the
government now lay out their negotiating to Parliament, before such a
vote is held.''

Cian Murphy, a senior lecturer in public international law at the
University of Bristol,
\href{https://twitter.com/cianmurf/status/794119281105653760}{wrote on
Twitter}: ``The Article 50 judgment from the High Court: a political
bombshell but really rather predicable as a matter of constitutional
law.''

Although Mrs. May has said that lawmakers will eventually be consulted,
many fear it will take place too late to influence the shape of
Britain's new relationship with the European Union.

For example, if Parliament is given a chance to vote on an exit
agreement at the end of the two-year period, lawmakers may be forced to
choose between endorsing a deal they oppose or leaving the bloc without
any formal relationship with it.

The government had dismissed the case as legal ``camouflage,'' regarding
it as a thinly disguised effort to frustrate the democratic outcome of
the June 23 referendum.

The Conservative Party, which was badly split over the referendum, has
now largely embraced its outcome, in many cases enthusiastically.

Many supporters of the opposition Labour Party also voted to leave the
European Union, which will make it harder for their lawmakers to oppose
a withdrawal.

Along with the Supreme Court, the ruling might ultimately be referred to
the European Court of Justice, an institution opposed by many who argued
for Britain to leave the bloc.

Advertisement

\protect\hyperlink{after-bottom}{Continue reading the main story}

\hypertarget{site-index}{%
\subsection{Site Index}\label{site-index}}

\hypertarget{site-information-navigation}{%
\subsection{Site Information
Navigation}\label{site-information-navigation}}

\begin{itemize}
\tightlist
\item
  \href{https://help.nytimes.com/hc/en-us/articles/115014792127-Copyright-notice}{©~2020~The
  New York Times Company}
\end{itemize}

\begin{itemize}
\tightlist
\item
  \href{https://www.nytco.com/}{NYTCo}
\item
  \href{https://help.nytimes.com/hc/en-us/articles/115015385887-Contact-Us}{Contact
  Us}
\item
  \href{https://www.nytco.com/careers/}{Work with us}
\item
  \href{https://nytmediakit.com/}{Advertise}
\item
  \href{http://www.tbrandstudio.com/}{T Brand Studio}
\item
  \href{https://www.nytimes.com/privacy/cookie-policy\#how-do-i-manage-trackers}{Your
  Ad Choices}
\item
  \href{https://www.nytimes.com/privacy}{Privacy}
\item
  \href{https://help.nytimes.com/hc/en-us/articles/115014893428-Terms-of-service}{Terms
  of Service}
\item
  \href{https://help.nytimes.com/hc/en-us/articles/115014893968-Terms-of-sale}{Terms
  of Sale}
\item
  \href{https://spiderbites.nytimes.com}{Site Map}
\item
  \href{https://help.nytimes.com/hc/en-us}{Help}
\item
  \href{https://www.nytimes.com/subscription?campaignId=37WXW}{Subscriptions}
\end{itemize}
