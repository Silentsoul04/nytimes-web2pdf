Sections

SEARCH

\protect\hyperlink{site-content}{Skip to
content}\protect\hyperlink{site-index}{Skip to site index}

\href{https://www.nytimes.com/section/your-money}{Your Money}

\href{https://myaccount.nytimes.com/auth/login?response_type=cookie\&client_id=vi}{}

\href{https://www.nytimes.com/section/todayspaper}{Today's Paper}

\href{/section/your-money}{Your Money}\textbar{}A Stranded \$2 Trillion
Overseas Stash Gets Closer to Coming Home

\url{https://nyti.ms/2em8rkj}

\begin{itemize}
\item
\item
\item
\item
\item
\end{itemize}

Advertisement

\protect\hyperlink{after-top}{Continue reading the main story}

Supported by

\protect\hyperlink{after-sponsor}{Continue reading the main story}

\href{/column/business-strategies}{Strategies}

\hypertarget{a-stranded-2-trillion-overseas-stash-gets-closer-to-coming-home}{%
\section{A Stranded \$2 Trillion Overseas Stash Gets Closer to Coming
Home}\label{a-stranded-2-trillion-overseas-stash-gets-closer-to-coming-home}}

\includegraphics{https://static01.nyt.com/images/2016/11/06/business/06STRA/06STRA-articleInline.jpg?quality=75\&auto=webp\&disable=upscale}

By \href{https://www.nytimes.com/by/jeff-sommer}{Jeff Sommer}

\begin{itemize}
\item
  Nov. 4, 2016
\item
  \begin{itemize}
  \item
  \item
  \item
  \item
  \item
  \end{itemize}
\end{itemize}

The next president may have a rare opportunity to close tax loopholes
that have let American corporations stash more than \$2 trillion in
untaxed profits outside the United States.

This enormous hoard of stranded cash has barely been an issue in the
contentious election campaign of 2016, and precise predictions of deals
that could be made in Washington are foolhardy until the nation goes to
the polls.

But this much is clear: There is a growing political consensus that the
time has come for change in the tax rules to encourage repatriation of
the vast troves of corporate earnings held outside the country.
Companies, ordinary American taxpayers and thousands of investors have
substantial and sometimes conflicting stakes in the outcome.

``Everyone agrees that something is going to be done about this,'' said
Edward D. Kleinbard, the former chief of staff of the congressional
Joint Committee on Taxation, and now a law professor at the University
of Southern California. ``The question, of course, is exactly what.''

Under current rules, by declaring that foreign profits are permanently
or indefinitely reinvested abroad, American companies can defer taxation
on that money. How much money, exactly, is subject to interpretation,
but careful estimates extend from about \$2.4 trillion to roughly \$3
trillion.

Both Hillary Clinton and
\href{http://www.nytimes.com/2016/08/13/upshot/how-hillary-clinton-and-donald-trump-differ-on-taxes.html}{Donald
J. Trump} have indicated that they plan to tax at least some of that
money and induce corporations to bring it home, though details are
scarce. There was
\href{http://www.nytimes.com/2015/07/09/business/end-to-us-taxation-of-overseas-profit-finds-bipartisan-support.html}{bipartisan
support} in Congress for a deal on corporate repatriation in 2015, but
it fizzled. The usual gridlock in Washington --- and the likelihood of
changes in the political firmament after the election --- dimmed
prospects for a deal in 2016. That could soon shift.

\includegraphics{https://static01.nyt.com/images/2016/11/06/business/06JPSTRA2/06JPSTRA2-articleLarge.jpg?quality=75\&auto=webp\&disable=upscale}

One reason is that the sums that could be made available for use by the
government have become staggeringly large. Like the gravity of an
outsized planet, the concentration of so much money creates a nearly
irresistible force: Something needs to be done about it.

An approach called deemed repatriation --- in which untaxed foreign
corporate profits are subject to immediate taxation --- would provide a
gigantic infusion to the Treasury and give corporations a significant
incentive to move money home. Leading plans in Congress include this
approach, Mr. Kleinbard said.

Reforming the tax code is anything but simple, however. The details are
crucial, and there are plenty of them, giving corporate lobbyists ample
opportunity to shape eventual changes in a manner that favors the big
companies.

First, it's not easy to discern the actual size of the stash of
corporate money abroad. One solid figure comes from the congressional
Joint Committee on Taxation, which
\href{http://waysandmeans.house.gov/wp-content/uploads/2016/09/20160831-Barthold-Letter-to-BradyNeal.pdf}{estimated}
in late August that as of 2015, the total of ``undistributed'' and ``not
previously taxed'' foreign earnings of American companies amounted to
\$2.6 trillion.

Consider the implications of that sum for a moment.

On paper, if not in reality, corporations are required to pay a federal
tax rate of 35 percent. If all of that money had been taxed at that
rate, it would amount to \$910 billion in taxes.

In fact, Goldman Sachs research indicates that companies in the Standard
\& Poor's 500-stock index paid a median federal effective tax rate of 28
percent, on average, over the last decade, while companies with high
foreign earnings paid about 22 percent.

But let's stick with the statutory 35 percent rate for a moment. My
calculations show that at that rate, the lost corporate tax revenue
would amount to almost two-thirds of all the money (\$1.39 trillion)
paid by Americans in personal income tax in 2014, according to Treasury
data. And the lost tax revenue is more than 2.5 times the income tax
paid annually by American corporations. Even if corporations were given
a big break --- which is highly likely under any tax code revision ---
the impact of any tax payments on those profits would still be large.

Image

The Senate majority leader, Mitch McConnell, Republican of Kentucky;
Senator Chuck Schumer, Democrat of New York; and the speaker of the
House, Paul Ryan, Republican of Wisconsin, at the Capitol last month.
Mr. Ryan and Mr. Schumer have favored tax changes that would encourage
corporate repatriation.Credit...Andrew Harnik/Associated Press

In reality, current legislative plans for bringing the money home call
for lowering the statutory rate --- to somewhere below 20 percent on a
one-time basis --- as well as for lowering the overall corporate tax
rate permanently. That could be part of an overhaul of the entire tax
code --- a long-thwarted achievement that might gain
\href{http://www.nytimes.com/2016/11/04/business/a-rare-moment-of-unity-on-capitol-hill-thanks-to-trumps-taxes.html}{new
impetus} after the election, as my colleague James B. Stewart has
written.

Taxing the stranded corporate earnings, whatever their amount, is
certainly on the Washington agenda. Goldman Sachs estimated that an
Obama administration proposal to tax American corporations' existing
foreign earnings at a 14 percent rate could
\href{http://www.bloomberg.com/politics/articles/2016-10-25/clinton-readies-post-election-push-on-highways-corporate-taxes}{generate}
\$240 billion in taxes.

And the Clinton campaign has
\href{https://www.hillaryclinton.com/feed/hillary-clintons-275-billion-infrastructure-plan-game-changer-our-economy-heres-why/}{advocated}
using tax revenue from repatriated foreign earnings to help finance an
ambitious domestic infrastructure program. At the same time, Mr. Trump
has proposed a one-time 10 percent tax on American corporate money held
abroad, while reducing the tax on future corporate earnings to 15
percent. Any of these variations would yield a lot of revenue, and the
higher the corporate tax rate, the greater the short-term benefit for
American taxpayers.

Stock investors would also enjoy a windfall under earnings repatriation
plans, but the lower the corporate tax rate, the greater the benefit.
It's easy to see why.

For one thing, studies show that in 2004 when the American Jobs Creation
Act granted a ``tax holiday,'' in which companies were allowed to bring
money home at a 5.25 percent tax rate, they used very little of their
repatriated money to create jobs or develop new businesses or
technologies. Most of the cash simply flowed to investors in the form of
buybacks and dividends.

``The holiday gave multinational firms a signal that there was no reason
to pay the full tax due at repatriation,'' Kimberly Clausing, a
professor at Reed College, wrote in a
\href{http://equitablegrowth.org/report/profit-shifting-and-u-s-corporate-tax-policy-reform/}{recent
paper}. ``Instead, one should wait for the next holiday or lobby for a
tax system that exempts foreign income entirely.''

That's why the details of a tax deal are so important. A tax holiday
could encourage companies to stockpile earnings overseas again and defer
American taxes, while enriching investors. If \$1 trillion were
repatriated and companies funneled nearly all of it to their
shareholders, the windfall would be very large indeed: It could come
close to the \$975 billion in buybacks and dividends for all S.\&P. 500
stocks for the 12 months through June, according to data provided by
Howard Silverblatt, senior index analyst at S\&P Dow Jones Indices.

Image

Senator Rob Portman, an Ohio Republican, who has agreed with many of his
elected colleagues in the House and the Senate about the framework of a
plan to repatriate corporate cash.Credit...Maddie McGarvey for The New
York Times

Dividends and buybacks are important. They have been helping
\href{http://www.nytimes.com/2016/08/28/your-money/some-good-news-for-investors-the-bull-may-still-have-spring-in-its-step.html}{prop
up} the stock market. Bringing money back this way might give the
market, and specific companies, an ephemeral sugar high.

For these reasons, in a recent report for clients, Goldman Sachs
suggested that investors consider buying shares of companies with the
biggest untaxed foreign earnings: Microsoft, General Electric, Apple and
Pfizer, which also top the list of untaxed earnings giants compiled by
Audit Analytics, an accounting research firm.

With earnings stranded overseas, many big corporations have been able to
\href{http://www.nytimes.com/2015/11/08/your-money/microsofts-stock-math-fewer-shares-pricier-shares.html}{borrow}
at very low interest rates to pay dividends and to buy back stock. But
interest rates will rise eventually, and using their own cash that is
parked overseas would be beneficial.

Big companies would benefit in other ways, too. The
\href{http://www.nytimes.com/2016/10/14/business/dealbook/exemptions-made-to-treasurys-tax-saving-restriction-rules.html}{Treasury}
effectively blocked Pfizer last year from a so-called tax inversion
merger with Allergan, a smaller company with Ireland as its tax
domicile. Unproductive, untaxed foreign earnings make that kind of
merger tempting. But a change in the tax code that lets companies lower
their tax burdens while bringing money back home could make inversions
--- as well as ``permanent'' investment of earnings overseas ---
irrelevant strategies.

Furthermore, companies like Apple, which has been drawn into a nasty
dispute with the European Union over the low level of taxes it pays to
Ireland, might not engage in
\href{http://www.nytimes.com/2012/04/29/business/apples-tax-strategy-aims-at-low-tax-states-and-nations.html}{elaborate}
overseas tax maneuvers if the American code were straightforward, and if
the United States and tax haven countries harmonized their rules, making
tax collection more effective.

There is a surprising degree of bipartisan consensus that the American
tax system needs to be fixed and that the stranded earnings should be
brought home. Paul D. Ryan, Republican of Wisconsin and the House
speaker, and Chuck Schumer of New York, who is in line to be the
Senate's Democratic leader, have favored tax changes that would
encourage corporate repatriation. Mr. Schumer and Senator Rob Portman,
an Ohio Republican, agreed on the framework of such a plan, and
President Obama did as well.

A gigantic pot of money is sitting overseas. It will be up to the next
president and Congress --- and their counterparts abroad --- to decide
exactly what to do about it.

Advertisement

\protect\hyperlink{after-bottom}{Continue reading the main story}

\hypertarget{site-index}{%
\subsection{Site Index}\label{site-index}}

\hypertarget{site-information-navigation}{%
\subsection{Site Information
Navigation}\label{site-information-navigation}}

\begin{itemize}
\tightlist
\item
  \href{https://help.nytimes.com/hc/en-us/articles/115014792127-Copyright-notice}{©~2020~The
  New York Times Company}
\end{itemize}

\begin{itemize}
\tightlist
\item
  \href{https://www.nytco.com/}{NYTCo}
\item
  \href{https://help.nytimes.com/hc/en-us/articles/115015385887-Contact-Us}{Contact
  Us}
\item
  \href{https://www.nytco.com/careers/}{Work with us}
\item
  \href{https://nytmediakit.com/}{Advertise}
\item
  \href{http://www.tbrandstudio.com/}{T Brand Studio}
\item
  \href{https://www.nytimes.com/privacy/cookie-policy\#how-do-i-manage-trackers}{Your
  Ad Choices}
\item
  \href{https://www.nytimes.com/privacy}{Privacy}
\item
  \href{https://help.nytimes.com/hc/en-us/articles/115014893428-Terms-of-service}{Terms
  of Service}
\item
  \href{https://help.nytimes.com/hc/en-us/articles/115014893968-Terms-of-sale}{Terms
  of Sale}
\item
  \href{https://spiderbites.nytimes.com}{Site Map}
\item
  \href{https://help.nytimes.com/hc/en-us}{Help}
\item
  \href{https://www.nytimes.com/subscription?campaignId=37WXW}{Subscriptions}
\end{itemize}
