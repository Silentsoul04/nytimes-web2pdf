Sections

SEARCH

\protect\hyperlink{site-content}{Skip to
content}\protect\hyperlink{site-index}{Skip to site index}

\href{https://www.nytimes.com/section/world/europe}{Europe}

\href{https://myaccount.nytimes.com/auth/login?response_type=cookie\&client_id=vi}{}

\href{https://www.nytimes.com/section/todayspaper}{Today's Paper}

\href{/section/world/europe}{Europe}\textbar{}Hungary's Lawmakers Reject
Plan to Block Resettlement of Refugees

\url{https://nyti.ms/2eAc5HC}

\begin{itemize}
\item
\item
\item
\item
\item
\end{itemize}

Advertisement

\protect\hyperlink{after-top}{Continue reading the main story}

Supported by

\protect\hyperlink{after-sponsor}{Continue reading the main story}

\hypertarget{hungarys-lawmakers-reject-plan-to-block-resettlement-of-refugees}{%
\section{Hungary's Lawmakers Reject Plan to Block Resettlement of
Refugees}\label{hungarys-lawmakers-reject-plan-to-block-resettlement-of-refugees}}

\includegraphics{https://static01.nyt.com/images/2016/11/09/world/09hungary-web1/09hungary-web1-articleLarge.jpg?quality=75\&auto=webp\&disable=upscale}

By Balint Bardi and \href{https://www.nytimes.com/by/palko-karasz}{Palko
Karasz}

\begin{itemize}
\item
  Nov. 8, 2016
\item
  \begin{itemize}
  \item
  \item
  \item
  \item
  \item
  \end{itemize}
\end{itemize}

BUDAPEST --- Lawmakers in Hungary on Tuesday rejected a proposed
national ban on refugees relocated from the rest of the European Union,
dealing a rare defeat to Prime Minister Viktor Orban.

Mr. Orban submitted a plan to ban the refugees, in the form of a
constitutional
\href{http://www.kormany.hu/en/the-prime-minister/news/the-prime-minister-has-submitted-to-parliament-a-bill-for-a-constitutional-amendment}{amendment},
last month, after a similar proposal
\href{http://www.nytimes.com/2016/10/03/world/europe/hungary-to-vote-on-accepting-more-migrants-as-europe-watches.html}{failed}
to pass by referendum because of insufficient voter turnout.

He has vowed to block a
\href{http://europa.eu/rapid/press-release_IP-15-6134_en.htm}{European
Union program} that would resettle migrants from the Middle East and
Africa who have gone to countries like Greece and Italy. Under that
program, Hungary, a nation of 10 million, would have to accept
\href{http://ec.europa.eu/dgs/home-affairs/what-we-do/policies/european-agenda-migration/press-material/docs/state_of_play_-_relocation_en.pdf}{1,294}
of a total of about 160,000 migrants.

\includegraphics{https://static01.nyt.com/images/2016/11/09/world/09hungary-web2/09hungary-web2-articleLarge.jpg?quality=75\&auto=webp\&disable=upscale}

The amendment needed two-thirds of sitting members of the 199-member
Parliament to pass. It got 131 votes on Tuesday --- two shy of the
necessary threshold. Three lawmakers voted no, and the rest abstained.

The far-right Jobbik party, which is part of the official opposition but
usually sides with Mr. Orban's Fidesz party on migration issues, was
crucial to the defeat of the amendment.

Gabor Vona, a lawmaker and the leader of Jobbik, said that his party
would support only a solution that ``defends Hungary and Hungarian
people, not just from poor migrants but from rich migrants, not just
from poor terrorists but from rich terrorists.''

He was referring to a rule that allows foreigners who invest over
300,000 euros, or about \$332,000, in Hungarian bonds to acquire
residency. The program dates to 2012, but it has drawn attention
recently, after reports that Hungarian bonds could be bought in places
like Erbil, Iraq. Critics say that allowing migrants to settle in
Hungary could open the door to terrorists from unstable countries like
Iraq, and they say the residency program might open the door to
corruption.

The Fidesz party has been politically weaker since last year, when it
lost its supermajority in Parliament. That advantage had allowed Mr.
Orban's government to rewrite the Constitution and to pass legislation
to rein in the judiciary and the press, packing some of the country's
top institutions with political allies.

Lajos Kosa, an ally of Mr. Orban who leads Fidesz lawmakers in
Parliament, said before the vote on Tuesday that the Jobbik party would
be ``joining the ranks of traitors'' if it rejected the amendment.

``Hungary can only count on Fidesz and K.D.N.P. in the struggle against
migration,'' he said as he emerged from the vote, using the initials for
the Christian Democratic People's Party, which is part of the governing
coalition.

``We are naturally going to continue the struggle,'' he said, noting
that more than three million voters in the referendum had opted to
support the ban on migrants.

\href{https://www.nytimes.com/interactive/2016/05/22/world/europe/europe-right-wing-austria-hungary.html}{}

\includegraphics{https://static01.nyt.com/images/2016/05/22/world/europe/europe-right-wing-austria-hungary-1463897749837/europe-right-wing-austria-hungary-1463897749837-thumbLarge-v5.png}

\hypertarget{how-far-is-europe-swinging-to-the-right}{%
\subsection{How Far Is Europe Swinging to the
Right?}\label{how-far-is-europe-swinging-to-the-right}}

Right-wing parties have been achieving electoral success in a growing
number of nations.

On Tuesday, analysts were cautious in interpreting the defeat of the
amendment as a sign of the government's declining political power.

``I wouldn't say that this is a huge failure for Orban --- it's a
failure, but a minor one,'' said Csaba Toth, the director of strategy
for the Republikon Institute, a research and advocacy group that has
been critical of Mr. Orban's government. ``This is the second time the
government can't have their own way, which is important for a group
whose main governing strategy is power.''

Last year, Mr. Orban and his government began an aggressive campaign
against migrants, particularly those from the Middle East, as hundreds
of thousands crossed Hungary's southern border, most of them en route to
Germany. The number of crossings fell nearly to zero after Hungary built
a razor-wire fence along sections of the border and as the flow of
migrants shifted away from the Balkans.

Nonetheless, Fidesz's campaign against migration has continued, and the
party's stance has seized both attention and voters from Jobbik, which
had been the main voice for right-wing nationalists.

Mr. Toth, the analyst, said that the vote on Tuesday could be seen as an
attempt by Jobbik to improve its political standing. ``It can say that
corruption is more important to Fidesz than the fight against quotas,''
he said.

Advertisement

\protect\hyperlink{after-bottom}{Continue reading the main story}

\hypertarget{site-index}{%
\subsection{Site Index}\label{site-index}}

\hypertarget{site-information-navigation}{%
\subsection{Site Information
Navigation}\label{site-information-navigation}}

\begin{itemize}
\tightlist
\item
  \href{https://help.nytimes.com/hc/en-us/articles/115014792127-Copyright-notice}{©~2020~The
  New York Times Company}
\end{itemize}

\begin{itemize}
\tightlist
\item
  \href{https://www.nytco.com/}{NYTCo}
\item
  \href{https://help.nytimes.com/hc/en-us/articles/115015385887-Contact-Us}{Contact
  Us}
\item
  \href{https://www.nytco.com/careers/}{Work with us}
\item
  \href{https://nytmediakit.com/}{Advertise}
\item
  \href{http://www.tbrandstudio.com/}{T Brand Studio}
\item
  \href{https://www.nytimes.com/privacy/cookie-policy\#how-do-i-manage-trackers}{Your
  Ad Choices}
\item
  \href{https://www.nytimes.com/privacy}{Privacy}
\item
  \href{https://help.nytimes.com/hc/en-us/articles/115014893428-Terms-of-service}{Terms
  of Service}
\item
  \href{https://help.nytimes.com/hc/en-us/articles/115014893968-Terms-of-sale}{Terms
  of Sale}
\item
  \href{https://spiderbites.nytimes.com}{Site Map}
\item
  \href{https://help.nytimes.com/hc/en-us}{Help}
\item
  \href{https://www.nytimes.com/subscription?campaignId=37WXW}{Subscriptions}
\end{itemize}
