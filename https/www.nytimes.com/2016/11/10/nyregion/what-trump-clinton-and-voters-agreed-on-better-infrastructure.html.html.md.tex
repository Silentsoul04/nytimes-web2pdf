Sections

SEARCH

\protect\hyperlink{site-content}{Skip to
content}\protect\hyperlink{site-index}{Skip to site index}

\href{https://www.nytimes.com/section/nyregion}{New York}

\href{https://myaccount.nytimes.com/auth/login?response_type=cookie\&client_id=vi}{}

\href{https://www.nytimes.com/section/todayspaper}{Today's Paper}

\href{/section/nyregion}{New York}\textbar{}What Trump, Clinton and
Voters Agreed On: Better Infrastructure

\url{https://nyti.ms/2eDJReR}

\begin{itemize}
\item
\item
\item
\item
\item
\end{itemize}

Advertisement

\protect\hyperlink{after-top}{Continue reading the main story}

Supported by

\protect\hyperlink{after-sponsor}{Continue reading the main story}

\hypertarget{what-trump-clinton-and-voters-agreed-on-better-infrastructure}{%
\section{What Trump, Clinton and Voters Agreed On: Better
Infrastructure}\label{what-trump-clinton-and-voters-agreed-on-better-infrastructure}}

\includegraphics{https://static01.nyt.com/images/2016/11/10/nyregion/10TRANSPORTATION/10NYREGION-articleLarge.jpg?quality=75\&auto=webp\&disable=upscale}

By \href{http://www.nytimes.com/by/emma-g-fitzsimmons}{Emma G.
Fitzsimmons}

\begin{itemize}
\item
  Nov. 9, 2016
\item
  \begin{itemize}
  \item
  \item
  \item
  \item
  \item
  \end{itemize}
\end{itemize}

At the end of a stunning and divisive election that left many Americans
feeling further apart than ever, there was perhaps one area of common
ground: infrastructure.

In a triumphant victory speech early Wednesday, President-elect Donald
J. Trump cited the issue as a top priority for his administration.

``We are going to fix our inner cities and rebuild our highways,
bridges, tunnels, airports, schools, hospitals,'' Mr. Trump said.
``We're going to rebuild our infrastructure, which will become, by the
way, second to none.''

The sentiment was echoed across the country on Election Day as voters
supported dozens of local ballot measures intended to improve public
transportation. In Los Angeles, Seattle and Atlanta, voters were poised
to approve spending billions of dollars on buses, rail lines and other
projects.

During the presidential campaign, Mr. Trump pledged to spend nearly \$1
trillion on infrastructure,
\href{http://www.nytimes.com/2016/08/03/us/politics/trump-clinton-infrastructure.html}{seeking
to outshine Hillary Clinton} on an issue that is a growing concern for
many Americans.

But with little support from Washington in recent years, many
communities have moved forward with their own plans to improve public
transportation through local ballot measures. In Los Angeles County on
Tuesday, voters approved a half-cent increase in the sales tax to raise
nearly \$120 billion for the transportation system.

In the Seattle region, voters appeared likely to have approved a \$54
billion proposal that would include building 62 new miles of light rail.
The proposition was winning on Wednesday with about 55 percent of the
vote.

``Local regions with some vision are taking matters into their own hands
and going directly to the voters to try to tackle real local problems,''
said Peter Rogoff, the chief executive of Sound Transit, the
transportation agency serving multiple counties in the Seattle region.

While Mr. Trump did not specifically mention trains or buses during his
brief victory speech, he said during the campaign that he supported
spending money on transit. As a resident of New York City and a real
estate developer, Mr. Trump appears to understand how important public
transit is for cities, said Art Guzzetti, the vice president of policy
for the American Public Transportation Association.

``The magnitude of the need is such that you're going to need all the
partners chipping in to the solution,'' Mr. Guzzetti said of the
country's vast infrastructure needs.

Mr. Trump's campaign released a
\href{http://peternavarro.com/sitebuildercontent/sitebuilderfiles/infrastructurereport.pdf}{proposal
last month to pay for infrastructure projects} that included giving tax
credits to private investors. This year, Mr. Trump suggested he might
create a federal infrastructure fund supported by government bonds that
investors and citizens could purchase.

On Wednesday, transit supporters celebrated the success of the local
ballot measures, which consoled some who were disappointed by Mrs.
Clinton's loss. A proposal in Atlanta to increase the sales tax to pay
for transit improvements received about 72 percent support.

New funding will help pay for light rail along the
\href{http://www.nytimes.com/2016/09/12/us/atlanta-beltline.html}{Atlanta
BeltLine}, a popular pedestrian and bike project, said Keith Parker, the
chief executive of the Metropolitan Atlanta Rapid Transit Authority.

``The customers are telling us that they really believe in mass
transit,'' he said.

Of 49 measures on local or state ballots, at least 33 appear to have
passed, said Jason Jordan, the executive director of the
\href{https://www.cfte.org/}{Center for Transportation Excellence}, a
research center that backed the ballot measures.

``We expected a historic year going into Election Day, and we got it in
terms of the largest number of measures we've ever tracked and the
largest dollar amount invested,'' Mr. Jordan said.

Many of the ballot measures were concentrated in the West, where leaders
are pressing for new transit options. On the
\href{http://www.nytimes.com/2016/05/27/nyregion/no-silver-bullet-as-subways-in-the-northeast-show-their-age.html}{East
Coast, older systems} like the century-old subway networks in New York
and Boston are struggling to both maintain aging equipment and expand to
meet rising ridership.

In New Jersey, voters approved a constitutional amendment that would
dedicate all the state's gas tax revenue to transportation. Gov. Chris
Christie, a Republican who is overseeing Mr. Trump's presidential
transition team, had agreed to a deal to raise the
\href{http://www.nytimes.com/2016/11/02/nyregion/as-days-of-cheap-gas-end-in-new-jersey-drivers-descend-for-a-last-fill-up.html}{state's
gas tax this month by 23 cents per gallon} to pay for improvements to
roads, bridges and public transit, but the amendment ensures the money
would not be diverted to other budget needs.

One of the country's largest infrastructure proposals is a plan
supported by the Obama administration to build a new rail tunnel under
the Hudson River from New Jersey to New York, part of a larger project
known as the Gateway program that could cost more than \$20 billion. The
current century-old tunnel used by Amtrak and New Jersey Transit trains
was damaged during Hurricane Sandy.

On Wednesday, supporters of the project praised Mr. Trump's comments,
adding that the tunnel was ``precisely the kind of infrastructure
program that warrants continued and increased public investment,''
according to John D. Porcari, the interim executive director of the
Gateway Development Corporation, which is being created to oversee the
project. Mr. Christie canceled an earlier plan to build a tunnel under
the river, but he supports the new proposal.

In Los Angeles County, a place known for its reliance on cars, nearly 70
percent of voters approved a transportation proposal, known as Measure
M, which required two-thirds of the vote to pass. The financing will
allow the region to expand on recent projects like the new
\href{http://www.nytimes.com/2016/07/18/us/la-metro-expo-line-santa-monica.html}{Expo
Line light rail} connecting downtown Los Angeles and Santa Monica, and
to improve bus service and repave local streets.

Mayor Eric Garcetti of Los Angeles, who supported the transportation
measure, called it a ``new day'' for his city and other American cities.
Many people who live in Los Angeles do not have a car, he said, and the
new rail line there has surpassed expectations for ridership.

``A lot of people are looking for a place they can read the paper, check
their emails and not worry about the headaches of a traffic jam,'' he
said.

Advertisement

\protect\hyperlink{after-bottom}{Continue reading the main story}

\hypertarget{site-index}{%
\subsection{Site Index}\label{site-index}}

\hypertarget{site-information-navigation}{%
\subsection{Site Information
Navigation}\label{site-information-navigation}}

\begin{itemize}
\tightlist
\item
  \href{https://help.nytimes.com/hc/en-us/articles/115014792127-Copyright-notice}{©~2020~The
  New York Times Company}
\end{itemize}

\begin{itemize}
\tightlist
\item
  \href{https://www.nytco.com/}{NYTCo}
\item
  \href{https://help.nytimes.com/hc/en-us/articles/115015385887-Contact-Us}{Contact
  Us}
\item
  \href{https://www.nytco.com/careers/}{Work with us}
\item
  \href{https://nytmediakit.com/}{Advertise}
\item
  \href{http://www.tbrandstudio.com/}{T Brand Studio}
\item
  \href{https://www.nytimes.com/privacy/cookie-policy\#how-do-i-manage-trackers}{Your
  Ad Choices}
\item
  \href{https://www.nytimes.com/privacy}{Privacy}
\item
  \href{https://help.nytimes.com/hc/en-us/articles/115014893428-Terms-of-service}{Terms
  of Service}
\item
  \href{https://help.nytimes.com/hc/en-us/articles/115014893968-Terms-of-sale}{Terms
  of Sale}
\item
  \href{https://spiderbites.nytimes.com}{Site Map}
\item
  \href{https://help.nytimes.com/hc/en-us}{Help}
\item
  \href{https://www.nytimes.com/subscription?campaignId=37WXW}{Subscriptions}
\end{itemize}
