Sections

SEARCH

\protect\hyperlink{site-content}{Skip to
content}\protect\hyperlink{site-index}{Skip to site index}

\href{https://www.nytimes.com/section/world/americas}{Americas}

\href{https://myaccount.nytimes.com/auth/login?response_type=cookie\&client_id=vi}{}

\href{https://www.nytimes.com/section/todayspaper}{Today's Paper}

\href{/section/world/americas}{Americas}\textbar{}Mexico Braces for the
Fallout of a Trump Presidency

\url{https://nyti.ms/2ekebQo}

\begin{itemize}
\item
\item
\item
\item
\item
\end{itemize}

Advertisement

\protect\hyperlink{after-top}{Continue reading the main story}

Supported by

\protect\hyperlink{after-sponsor}{Continue reading the main story}

\hypertarget{mexico-braces-for-the-fallout-of-a-trump-presidency}{%
\section{Mexico Braces for the Fallout of a Trump
Presidency}\label{mexico-braces-for-the-fallout-of-a-trump-presidency}}

\includegraphics{https://static01.nyt.com/images/2016/11/10/world/10MEXICO-2/10MEXICO-2-videoSixteenByNineJumbo1600.jpg}

By \href{http://www.nytimes.com/by/azam-ahmed}{Azam Ahmed},
\href{http://www.nytimes.com/by/kirk-semple}{Kirk Semple} and Paulina
Villegas

\begin{itemize}
\item
  Nov. 9, 2016
\item
  \begin{itemize}
  \item
  \item
  \item
  \item
  \item
  \end{itemize}
\end{itemize}

\href{http://www.nytimes.com/es/2016/11/09/mexico-se-prepara-para-los-efectos-de-una-presidencia-de-trump/}{Leer
en español}

MEXICO CITY --- For Mexico,
\href{http://www.nytimes.com/2016/05/23/world/americas/donald-trump-mexico.html}{the
nightmare} came true.

Perhaps no country aside from the United States itself had as much at
stake in the American presidential election as Mexico did.

Then, early on Wednesday, it watched as Donald J. Trump became the next
American president: a man whose
\href{http://www.nytimes.com/2015/07/03/world/americas/donald-trump-gains-infamy-in-mexico-for-comments-on-immigrants.html}{central
campaign promises} included building a wall between the two countries,
upending decades-old trade deals and deporting millions of Mexican
immigrants.

The peso suffered its largest drop in nearly 20 years, and for many, the
election set back years of carefully cultivated
\href{http://www.nytimes.com/2016/09/01/world/americas/trump-mexico-pena-nieto-reaction.html}{efforts
to improve the cross-border relationship}, one that has been
historically fraught. The outcome promises a turbulent financial future
for Mexico, which relies on America as an economic lifeline, both in
terms of trade and remittances.

``It's an unmitigated disaster,'' said Jorge Castañeda, a former foreign
minister of Mexico and a professor of politics and Latin American
studies at New York University. ``There are very few tools to fix the
relationship.''

For months, Mexico watched the campaign with a
\href{http://www.nytimes.com/2015/07/03/world/americas/donald-trump-gains-infamy-in-mexico-for-comments-on-immigrants.html}{mix
of fear and bemusement}, forced to stare down a raw undercurrent of
American vitriol unleashed by Mr. Trump's candidacy. Now, the election
seems a harbinger of hard days to come for the country, its economy,
migration and even its state of mind.

``This election reminded us of the bad image Mexico has in the U.S.,''
said Jesús Silva-Herzog, a columnist and professor at the Monterrey
Institute of Technology and Higher Education in Mexico. ``It has also
served as a mirror in which we have painfully seen our reflection.''

``We will not have to wait for the presidential baton to be passed to
feel the devastating effects, not only in economic terms, but also the
existential crisis it will cause,'' he added.

Mr. Trump's vow to renegotiate the North American Free Trade Agreement
could greatly affect Canada, too. As a result of the pact, much of
Canada's industry produces for export.

``We're going to get sideswiped by some of this stuff,'' said Mark
Warner, a trade lawyer in Toronto. In particular, he said, he expects
Mr. Trump to promote ``Buy America'' clauses in government
infrastructure projects that would shut out Canadian companies --- a
violation of Nafta.

But for Canada and Mexico, the election was a study in contrasts.
Canadians reacted to Mr. Trump's election with concern, even anxiety,
but also a whiff of pride.

Move-to-Canada memes started spreading as Mr. Trump looked increasingly
likely to win Florida: ``Election Night Starter Kit,'' read a post on
Instagram, above photos of United States passports and an Air Canada
plane. Another post depicted a machine-gun-toting man riding a moose,
with the words ``Canadian Border Patrol Watching for Illegal
Americans.''

Later on Tuesday night, \href{http://www.cic.gc.ca/}{the website} of
Canada's immigration department crashed, fueling speculation that it had
been overloaded by Americans
\href{http://www.nytimes.com/2016/11/10/us/-canada-immigration.html?src=twr}{looking
for a new country} to call home. Lisa Filipps, a spokeswoman for
Citizenship and Immigration Canada, said the site had failed ``as a
result of a significant increase in the volume of traffic.''

\includegraphics{https://static01.nyt.com/images/2016/11/10/world/10MEXICO-3/10MEXICO-3-articleLarge.jpg?quality=75\&auto=webp\&disable=upscale}

In Mexico City, the vote felt like something else entirely: a validation
of Mr. Trump's hostile remarks about Mexican immigrants, and a broad
statement of disrespect.

``Imagine what the U.S. will look like from now,'' said Angelina
González, who sells cosmetics in Mexico City. ``A big wave of
discrimination is coming.''

Among journalists from \href{http://horizontal.mx/}{Horizontal}, a
cultural and political online magazine in Mexico City, spirits were low,
and confusion reigned. Antonio Martínez Velázquez, a co-founder,
reflected on the outcome with shock and a deep sense of uncertainty.

``This moment forces the world, including Mexico, to rethink its
relationship with the U.S.,'' he said. ``This moment, which really is
the end of an era, the end of the U.S. hegemony, is also the beginning
of a new chapter for us in Mexico.''

Mr. Trump has been among the most powerful forces at play in Mexico this
year, infuriating citizens of all stripes and even government officials
with his anti-Mexican campaign. Anger surged when the Mexican president,
Enrique Peña Nieto,
\href{http://www.nytimes.com/2016/09/01/world/americas/trump-mexico-pena-nieto-reaction.html}{invited
Mr. Trump to visit Mexico}, an offer the candidate accepted.

Weeks of
\href{http://www.nytimes.com/2016/09/08/world/americas/mexico-finance-minister-luis-videgaray-resigns.html}{vitriol
and betrayal ensued}, with many Mexicans denouncing Mr. Peña Nieto's
invitation as a needless capitulation from the leader of an insulted
nation.

Now, it turns out Mr. Peña Nieto was right: Mr. Trump was not simply a
candidate who could be ignored.

In a series of Twitter posts on Wednesday morning, Mr. Peña Nieto
congratulated ``the people of the United States for their electoral
process'' and reiterated his willingness to work with Mr. Trump ``in
favor of the bilateral relationship.''

``Mexico and the U.S.A. are friends, partners and allies, who must
continue collaborating for the competitiveness and development of North
America,'' he wrote. ``I trust that Mexico and the United States will
continue to strengthen their bonds of cooperation and mutual respect.''

Mr. Trump has promised to build a wall between the two countries and
make Mexico pay for it. But Foreign Minister Claudia Ruiz Massieu
rejected that notion in a television interview on Wednesday morning.

``Paying for a wall is out of our vision,'' she said. ``The vision that
we have is a vision of integration, of how Mexico and the United States
working together are more competitive.''

In practical terms, most experts suspect, the election will reverberate
most profoundly through the economy.

The United States and Mexico are deeply integrated in matters of
economics, demographics, culture and security, stitched together by the
movement of people, goods and money across a 2,000-mile border.

As one goes, so goes the other. Mexico is America's third-largest
trading partner, after Canada and China, with about \$531 billion in
two-way trade in 2015.

The countries are interdependent, with American goods and parts shipped
to Mexican factories whose products are shipped back into the United
States, and vice versa. Millions of American jobs are directly tied to
trade with Mexico.

Mr. Trump argued that Mexico was the outsize beneficiary of Nafta, while
American workers suffered job losses and stagnant wages, an argument
that played well with segments of the American electorate.

While Mexico is the second-largest destination for American goods,
giving it some leverage in responding to actions taken by Mr. Trump, the
countries have ``a very asymmetrical relationship,'' Mr. Castañeda said,
meaning that in the end, there is little Mexico can do to apply
pressure.

Many Mexicans may lose their jobs. All will suffer from a rapid
depreciation of the peso. But an economic crisis could also turn into a
migration crisis --- exactly what Mr. Trump has campaigned for months to
halt.

About 35 million Mexican citizens and Mexican-Americans live in the
United States, and the vast majority are either American citizens or
legal residents.

Illegal immigration from Mexico has fallen, and the Pew Research Center
estimates that more Mexicans are returning to Mexico than are migrating
to the United States. But a sudden economic shock could send Mexicans
once more to the United States to seek work.

``You generate an economic crisis in Mexico, and all of those gains we
have seen in terms of zero migration go down the tubes,'' said Agustín
Barrios Gómez, a former congressman in Mexico.

Not everyone felt entirely dour about the election results. If there was
a silver lining, some said it was that the threat from the outside would
force Mexicans to come together.

``I believe having a strong, negative factor right across the border
will bring the Mexicans together to work harder, which will be a
positive effect,'' said Arturo Delgado, the retired director of a
technical school.

Some were confident that Mr. Trump's hostile talk as a candidate would
ebb when he took office.

``I don't see a problem with trade or immigration,'' said Raymundo Riva
Palacio, a political analyst and columnist.

On trade, Mr. Riva Palacio argued that business groups and governors who
supported Mr. Trump, including Gov. Greg Abbott of Texas, would impress
upon him the importance of remaining in Nafta.

As for the wall Mr. Trump has vowed to build, ``it will be very
difficult for Donald Trump to obtain the budget,'' he said.

He argued that, ultimately, economics would temper Mr. Trump's policies
toward Mexico. But he added that with the House of Representatives and
the Senate
\href{http://www.nytimes.com/2016/11/09/us/politics/republican-senate.html?action=Click\&contentCollection=BreakingNews\&contentID=64523551\&pgtype=article}{remaining
under Republican control}, Mr. Trump's victory signaled an ideological
realignment that had not occurred in the United States since the
election of President Ronald Reagan.

``The problem isn't for Mexico,'' Mr. Riva Palacio said. ``It's for the
United States.''

Advertisement

\protect\hyperlink{after-bottom}{Continue reading the main story}

\hypertarget{site-index}{%
\subsection{Site Index}\label{site-index}}

\hypertarget{site-information-navigation}{%
\subsection{Site Information
Navigation}\label{site-information-navigation}}

\begin{itemize}
\tightlist
\item
  \href{https://help.nytimes.com/hc/en-us/articles/115014792127-Copyright-notice}{©~2020~The
  New York Times Company}
\end{itemize}

\begin{itemize}
\tightlist
\item
  \href{https://www.nytco.com/}{NYTCo}
\item
  \href{https://help.nytimes.com/hc/en-us/articles/115015385887-Contact-Us}{Contact
  Us}
\item
  \href{https://www.nytco.com/careers/}{Work with us}
\item
  \href{https://nytmediakit.com/}{Advertise}
\item
  \href{http://www.tbrandstudio.com/}{T Brand Studio}
\item
  \href{https://www.nytimes.com/privacy/cookie-policy\#how-do-i-manage-trackers}{Your
  Ad Choices}
\item
  \href{https://www.nytimes.com/privacy}{Privacy}
\item
  \href{https://help.nytimes.com/hc/en-us/articles/115014893428-Terms-of-service}{Terms
  of Service}
\item
  \href{https://help.nytimes.com/hc/en-us/articles/115014893968-Terms-of-sale}{Terms
  of Sale}
\item
  \href{https://spiderbites.nytimes.com}{Site Map}
\item
  \href{https://help.nytimes.com/hc/en-us}{Help}
\item
  \href{https://www.nytimes.com/subscription?campaignId=37WXW}{Subscriptions}
\end{itemize}
