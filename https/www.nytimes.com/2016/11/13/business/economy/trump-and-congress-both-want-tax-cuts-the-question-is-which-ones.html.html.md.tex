Sections

SEARCH

\protect\hyperlink{site-content}{Skip to
content}\protect\hyperlink{site-index}{Skip to site index}

\href{https://www.nytimes.com/section/business/economy}{Economy}

\href{https://myaccount.nytimes.com/auth/login?response_type=cookie\&client_id=vi}{}

\href{https://www.nytimes.com/section/todayspaper}{Today's Paper}

\href{/section/business/economy}{Economy}\textbar{}Trump and Congress
Both Want Tax Cuts. The Question Is Which Ones.

\url{https://nyti.ms/2eNM1Ja}

\begin{itemize}
\item
\item
\item
\item
\item
\end{itemize}

Advertisement

\protect\hyperlink{after-top}{Continue reading the main story}

Supported by

\protect\hyperlink{after-sponsor}{Continue reading the main story}

\hypertarget{trump-and-congress-both-want-tax-cuts-the-question-is-which-ones}{%
\section{Trump and Congress Both Want Tax Cuts. The Question Is Which
Ones.}\label{trump-and-congress-both-want-tax-cuts-the-question-is-which-ones}}

\includegraphics{https://static01.nyt.com/images/2016/11/13/business/13TAXES/13TAXES-articleLarge.jpg?quality=75\&auto=webp\&disable=upscale}

By \href{http://www.nytimes.com/by/patricia-cohen}{Patricia Cohen}

\begin{itemize}
\item
  Nov. 12, 2016
\item
  \begin{itemize}
  \item
  \item
  \item
  \item
  \item
  \end{itemize}
\end{itemize}

Several economic issues divide many Republicans in Congress from Donald
J. Trump, the Republican president-elect. Free trade versus tariffs to
limit imports. Immigration reform versus a border wall. Cutting Social
Security and other benefit programs versus protecting them.

But one economic matter unites just about every member of the Republican
party: support for tax cuts, particularly for those at the top of the
income ladder.

Whatever fault lines have emerged during this campaign, the belief that
lower taxes targeted at ``job creators'' will unleash a roar of economic
growth crosses them. Both
\href{https://www.donaldjtrump.com/policies/tax-plan}{Donald J. Trump}
and
\href{https://abetterway.speaker.gov/_assets/pdf/ABetterWay-Tax-Snapshot.pdf}{Paul
D. Ryan}, the House speaker,
\href{https://abetterway.speaker.gov/_assets/pdf/ABetterWay-Tax-Snapshot.pdf}{have
released tax proposals} that hark back to the supply-side programs of
the Reagan and George W. Bush eras, promising that the
multitrillion-dollar cost will be more than offset by the extra revenue
flowing into the Treasury from the growth that will follow.

``Tax reform is the thing that always unites Republicans,'' said William
Gale, a co-director of the nonpartisan Tax Policy Center and a former
economic adviser to President George H.W. Bush. ``I would guess that
that's Item 1 on the congressional agenda.''

House Republicans already have a fairly detailed blueprint for Congress
and the White House to follow.

``My sense is that Trump doesn't really have the details of a tax reform
package that he wants,'' Mr. Gale said. ``He has broad ideas, and then
the Congress will go at it and pin down the details. The House blueprint
seems like the place to start and may be fairly close to where they
finish.''

That does not mean Mr. Trump will not have his own ideas. Mr. Gale
expects a Trump White House to insist on continuing a deduction for
interest paid on debt-financed projects, a provision dear to real estate
developers. (The House plan proposes ending the deduction, instead
allowing businesses to immediately deduct expenses and investments.)

While sweeping tax cuts were never a crusading theme of Mr. Trump's,
they have long been near the top of Mr. Ryan's agenda. And Mr. Trump has
suggested he would be happy to let Congress take the lead.

``They'll have to take the temperature of the White House to see what
pieces of Trump's campaign promises have to be incorporated into that,''
said Douglas Holtz-Eakin, a former director of the Congressional Budget
Office and now president of the
\href{https://www.americanactionforum.org/}{American Action Forum}, a
conservative economic advocacy group. ``But I assume the House tax plan
is the starting point.''

There is certainly a significant overlap. Both would cut income tax
rates across the board and keep rates low on income from investments, an
approach intended to spur savings that effectively guarantees the
juiciest cuts for the wealthy.

An
\href{http://www.taxpolicycenter.org/publications/analysis-donald-trumps-revised-tax-plan/full}{analysis
of Mr. Trump's latest plan by the Tax Policy Center} calculated that the
top 0.1 percent of the population, those with incomes over \$3.7 million
in 2016, would receive an average 14 percent reduction, or about \$1.1
million. Households in the middle of the scale --- those earning between
about \$48,000 and \$83,000 today --- would get a 1.8 percent tax cut
worth on average \$1,010, while the poorest fifth of Americans will gain
about \$110, or 1 percent of their income.

Both Mr. Trump's and Mr. Ryan's plans eliminate a deep-rooted Republican
bête noire, the
\href{http://topics.nytimes.com/your-money/planning/estate-planning/index.html?inline=nyt-classifier}{estate
tax} on bequests to heirs. Under today's code, it falls on only 0.2
percent of households, since it applies only to estates worth more than
\$10.9 million for a married couple.

Their plans, in conjunction with rejecting the Affordable Care Act, drop
the 3.8 percent surtax on high earners' investment income, which helps
pay for health coverage for lower-income Americans. Both also take aim
at the
\href{http://topics.nytimes.com/top/reference/timestopics/subjects/a/alternative_minimum_tax/index.html?inline=nyt-classifier}{alternative
minimum tax}, which was originally established to make sure that those
earning high incomes do not entirely escape taxes by invoking certain
deductions but now falls mostly on the upper middle class in affluent
regions of the country.

Lowering the tax on capital gains --- which also benefits the wealthy
the most --- draws wide support among the leadership headed for both
ends of Pennsylvania Avenue in 2017 as well.

Republicans say they also want to provide some tax cuts for those lower
on the income ladder. Senator Marco Rubio, who was re-elected to
represent Florida after his failed presidential bid, favors increasing
the child tax credit; Mr. Ryan, who is working closely with Kevin Brady
of Texas, the chairman of the House Ways and Means Committee, supports
expanding the earned-income tax credit to poor working families without
children. Mr. Trump has suggested he wants to provide tax cuts to
two-income families with children as well.

For all the similarities, there are important differences as well. The
biggest contrast between Mr. Trump's and Mr. Ryan's tax approaches can
be seen on the corporate side, where they differ on how to tax capital
investment and debt. They also differ on the proposed tax rate for most
small businesses.

The most compelling target for business tax reform is the roughly \$2.6
trillion that American corporations like Apple, General Electric,
Microsoft and Pfizer have kept abroad on an extended tax holiday, out of
the Internal Revenue Service's reach.

``Everyone agrees that the foreign tax situation is ludicrous because it
doesn't raise any revenue and keeps several trillion dollars abroad,''
said Robert Pozen, a senior lecturer at the M.I.T. Sloan School of
Management.

Mr. Trump has said he was determined to get multinational companies to
pay their American tax bills every year, although the sting would not be
as great since he would also cut corporate rates and allow credits for
foreign taxes paid.

By contrast, the House Republicans have been pushing for what is known
as a territorial system, which would tax all businesses solely on what
goods and services they sold in the United States.

The flaw in a territorial approach, Mr. Pozen and other economists have
pointed out, is that it encourages businesses to shop the world for
lower tax rates, ultimately shifting even more profits and jobs
overseas.

``Why is that good for a president who wants to have more jobs and more
facilities in the U.S.?'' Mr. Pozen asked. ``I don't see how you can
reconcile those goals under a territorial system.''

Mr. Pozen favors a global minimum tax that every American business would
have to pay. Companies that shift their tax home or try to funnel more
profits through low-tax nations would nonetheless be required to make up
the difference between that rate and the minimum by paying the United
States Treasury.

However part of that stash is recaptured, there is a broad consensus in
Congress that some of the new revenue should be used to invest in
repairing and improving public infrastructure. Mr. Trump spoke of
spending \$1 trillion over 10 years on roads, bridges, waterways and
airports, although he said he planned to rely primarily on tax credits
for private companies, equity investments and privately raised debt.

Linking international reform to infrastructure funding could work, said
Janice Mays, a former staff director of the Ways and Means Committee who
is now a managing director at the tax and accounting firm
PricewaterhouseCoopers. ``I do think Trump wants to do infrastructure,
to help people get jobs and stimulate the economy.''

But as a longtime veteran of budget and tax battles on Capitol Hill, Ms.
Mays warned that obstacles continually pop up in both expected and
unexpected places. The size of the projected deficit from the various
tax-cutting plans --- as much as \$7 trillion over a decade --- could
set off an internal war among Republicans who favor restraining spending
on Social Security, Medicare and benefit programs for the poor and
those, like Mr. Trump, who say they want to prevent the blue-collar
families who flocked to his campaign from losing government programs
that help keep them above water.

``I just think they may be a little optimistic at the moment,'' Ms. Mays
said.

Advertisement

\protect\hyperlink{after-bottom}{Continue reading the main story}

\hypertarget{site-index}{%
\subsection{Site Index}\label{site-index}}

\hypertarget{site-information-navigation}{%
\subsection{Site Information
Navigation}\label{site-information-navigation}}

\begin{itemize}
\tightlist
\item
  \href{https://help.nytimes.com/hc/en-us/articles/115014792127-Copyright-notice}{©~2020~The
  New York Times Company}
\end{itemize}

\begin{itemize}
\tightlist
\item
  \href{https://www.nytco.com/}{NYTCo}
\item
  \href{https://help.nytimes.com/hc/en-us/articles/115015385887-Contact-Us}{Contact
  Us}
\item
  \href{https://www.nytco.com/careers/}{Work with us}
\item
  \href{https://nytmediakit.com/}{Advertise}
\item
  \href{http://www.tbrandstudio.com/}{T Brand Studio}
\item
  \href{https://www.nytimes.com/privacy/cookie-policy\#how-do-i-manage-trackers}{Your
  Ad Choices}
\item
  \href{https://www.nytimes.com/privacy}{Privacy}
\item
  \href{https://help.nytimes.com/hc/en-us/articles/115014893428-Terms-of-service}{Terms
  of Service}
\item
  \href{https://help.nytimes.com/hc/en-us/articles/115014893968-Terms-of-sale}{Terms
  of Sale}
\item
  \href{https://spiderbites.nytimes.com}{Site Map}
\item
  \href{https://help.nytimes.com/hc/en-us}{Help}
\item
  \href{https://www.nytimes.com/subscription?campaignId=37WXW}{Subscriptions}
\end{itemize}
