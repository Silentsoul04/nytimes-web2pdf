Sections

SEARCH

\protect\hyperlink{site-content}{Skip to
content}\protect\hyperlink{site-index}{Skip to site index}

\href{/section/world/middleeast}{Middle East}\textbar{}U.S. Fingerprints
on Attacks Obliterating Yemen's Economy

\url{https://nyti.ms/2eRoyGS}

\begin{itemize}
\item
\item
\item
\item
\item
\item
\end{itemize}

\includegraphics{https://static01.nyt.com/images/2016/11/10/world/middleeast/Yemen-slide-AVYH/Yemen-slide-AVYH-articleLarge-v2.jpg?quality=75\&auto=webp\&disable=upscale}

\hypertarget{us-fingerprints-on-attacks-obliterating-yemens-economy}{%
\section{U.S. Fingerprints on Attacks Obliterating Yemen's
Economy}\label{us-fingerprints-on-attacks-obliterating-yemens-economy}}

The Saudi-led coalition is hitting civilian targets, like factories,
bridges and power stations, that critics say have no clear link to the
rebels. In the rubble, the remains of American munitions have been
found.

The remains of a factory, owned by the brothers Khalid and Abdullah
Alsonidar, that was bombed twice in September outside Yemen's capital,
Sana.Credit...Tyler Hicks/The New York Times

Supported by

\protect\hyperlink{after-sponsor}{Continue reading the main story}

By \href{http://www.nytimes.com/by/ben-hubbard}{Ben Hubbard}

\begin{itemize}
\item
  Nov. 13, 2016
\item
  \begin{itemize}
  \item
  \item
  \item
  \item
  \item
  \item
  \end{itemize}
\end{itemize}

SANA, Yemen --- For decades, Mustafa Elaghil's family produced snack
foods popular in Yemen, chips and corn curls in bright packaging
decorated with the image of Ernie from
``\href{http://www.sesamestreet.org/}{Sesame Street}.''

But over the summer, a military coalition led by
\href{http://www.nytimes.com/topic/destination/saudi-arabia?8qa}{Saudi
Arabia} sent warplanes over
\href{http://www.nytimes.com/topic/destination/yemen?8qa}{Yemen} and
bombed the Elaghils' factory. The explosion destroyed it, setting it
ablaze and trapping the workers inside.

The attack killed 10 employees and wiped out a business that had
employed dozens of families.

``It was everything for us,'' Mr. Elaghil said.

The Saudi-led coalition
\href{http://www.nytimes.com/2016/10/31/world/middleeast/airstrikes-kill-dozens-in-western-yemen.html?rref=collection\%2Ftimestopic\%2FYemen\&action=click\&contentCollection=world\&region=stream\&module=stream_unit\&version=latest\&contentPlacement=3\&pgtype=collection}{has
bombed Yemen} for the last 19 months, trying to oust a rebel group
aligned with Iran that
\href{http://www.nytimes.com/2014/09/22/world/middleeast/yemens-prime-minister-resigns-amid-chaos-and-another-cease-fire.html}{took
control of the capital}, Sana, in 2014. The Saudis want to restore the
country's exiled president,
\href{http://www.nytimes.com/2015/02/22/world/africa/yemens-former-president-flees-capital.html}{Abdu
Rabbu Mansour Hadi}, who led an internationally recognized government
more aligned with its interests.

But instead of defeating the rebels, the campaign has sunk into a
grinding stalemate, systematically obliterating Yemen's already
bare-bones economy. The coalition has destroyed a wide variety of
civilian targets that critics say have no clear link to the rebels.

It has hit
\href{http://www.nytimes.com/2016/08/16/world/middleeast/yemen-doctors-without-borders-hospital-bombing.html}{hospitals}
and schools. It has destroyed bridges, power stations, poultry farms, a
key seaport and factories that produce yogurt, tea, tissues, ceramics,
Coca-Cola and potato chips. It has bombed weddings and
\href{http://www.nytimes.com/2016/10/09/world/middleeast/yemen-saudi-arabia-houthis-rebels.html}{a
funeral}.

The bombing campaign has exacerbated a humanitarian crisis in the Arab
world's poorest country, where cholera is spreading, millions of people
are struggling to get enough food, and malnourished babies are
overwhelming hospitals, according to the United Nations. Millions have
been forced from their homes, and since August, the government has been
unable to pay the salaries of most of the 1.2 million civil servants.

\includegraphics{https://static01.nyt.com/images/2016/11/10/world/middleeast/Yemen-slide-G63C/Yemen-slide-G63C-articleInline.jpg?quality=75\&auto=webp\&disable=upscale}

Publicly, the United States has kept its distance from the war, but its
decades-old alliance with Saudi Arabia, underpinned by tens of billions
of dollars in weapons sales, has left American fingerprints on the air
campaign.

Many strikes are carried out by pilots trained by the United States, who
fly American-made jets that are refueled in the air by American planes.
And Yemenis often find the remains of American-made munitions, as they
did in the ruins after a strike that killed more than 100 mourners at a
funeral last month.

Graffiti on walls across Sana reads: ``America is killing the Yemeni
people.''

President-elect Donald J. Trump has not said whether he will continue
United States support for the war, but has been very critical of Saudi
Arabia, saying it does not ``survive without us.'' At a rally in
January, he said Iran was ``going into Yemen'' and was ``going to have
everything'' in the region, but he did not clarify how he would respond.

The sweeping destruction of civilian infrastructure has led analysts and
aid workers to conclude that hitting Yemen's economy is part of the
coalition's strategy.

``The economic dimension of this war has become a tactic,'' said Jamie
McGoldrick, the United Nations' humanitarian coordinator for Yemen. ``It
is all consistent --- the port, the bridges, the factories. They are
getting destroyed, and it is to put pressure on the politics.''

In a written response to questions, a coalition spokesman, Maj. Gen.
Ahmed Asseri, said the air campaign had halted the rebels' advance,
destroyed 90 percent of their rockets and aircraft and pressured them to
join talks aimed at ending the war. He denied that the coalition sought
to inflict suffering on civilians and said only facilities connected to
the war effort had been hit.

He blamed the rebel group, the Houthis, for the humanitarian crisis.

``This is primarily the responsibility of the rebels, who have displaced
Yemen's legitimate government and who are impeding the flow of
humanitarian supplies,'' General Asseri said.

Saudi Arabia and other Persian Gulf countries are also among the top
donors of aid to Yemen. So even as they undermine its self-sufficiency,
they help sustain the population.

The air campaign's civilian toll has led to calls by some American
lawmakers to postpone arms sales to Saudi Arabia.

``It is a significant moral outrage that we continue to provide arms to
Saudi Arabia and to participate in military operations in Yemen,'' said
Representative Ted Lieu, a Democrat from California who was a military
prosecutor in the Air Force. ``The United States is at risk of aiding
and abetting war crimes in Yemen.''

Image

Sana's Old City, where rebels were in control. The Saudi-led coalition
has targeted civilian infrastructure from the air.Credit...Tyler
Hicks/The New York Times

\hypertarget{a-country-in-chaos}{%
\subsection{A Country in Chaos}\label{a-country-in-chaos}}

The difficulty in just getting to Yemen demonstrates how much the war
has upended the country.

The internationally recognized government is based in Saudi Arabia and
in the south of Yemen. For a recent 10-day trip to Sana and surrounding
areas, a photographer and I had to obtain visas from the Houthis.

We could not book flights into Sana because the Saudi-led coalition had
halted all commercial air traffic. The United Nations allowed us onto an
aid flight. As soon as we touched down, we saw traces of the war: the
scattered carcasses of destroyed airplanes along the runway.

Once in Yemen, we were told that we could not go anywhere without a
representative of the Houthis. He was with us whenever we left the
hotel. We did not visit military sites, which the coalition has heavily
bombed to destroy the ballistic missiles that the rebels have fired into
the kingdom, killing civilians.

But the damage and suffering caused by the war were everywhere.

Beggars displaced by the fighting thronged our car, pleading for money
and food. Buildings destroyed by airstrikes dotted the capital: the
Defense and Interior Ministries, the army and central security
headquarters, the Police Academy and Officers' Club, the Sana Chamber of
Commerce and Industry, the homes of officials who had joined the rebels.

The conflict has split the country, with forces backed by gulf nations
and nominally loyal to the exiled president in the south and east, where
Al Qaeda and the Islamic State have staged deadly attacks.

But in the areas we visited in Yemen's northwest, the rebels were firmly
in control, their gunmen running checkpoints alongside police officers
who had joined them. In Sana's Old City, posters of ``martyrs'' killed
in the war covered entire buildings. Trucks with mounted machine guns,
carrying fighters, occasionally sped by.

Image

The cultural center in Hajjah was destroyed from the air last
year.Credit...Tyler Hicks/The New York Times

Spray-painted across the city was the Houthis' rallying cry: ``God is
great. Death to America. Death to Israel. Curse on the Jews. Victory for
Islam.''

On the edge of town, Yemeni families snapped photos of the ruins of
\href{http://www.nytimes.com/video/world/middleeast/100000004736495/in-the-rubble-of-an-airstrike-in-yemen.html}{a
reception center} that the coalition hit with two airstrikes in a single
attack last month while the Houthi-allied interior minister was
receiving condolences for his deceased father. Human Rights Watch called
the attack on the funeral
``\href{https://www.hrw.org/news/2016/10/13/yemen-saudi-led-funeral-attack-apparent-war-crime}{an
apparent war crime}.''

United Nations officials gave us photos of remnants found at the site
that indicated it had been hit with at least one American-made,
500-pound, laser-guided bomb. American warplanes routinely use that
class of bomb, and the United States has provided such bombs to the
Saudi military.

\hypertarget{whats-missing-everything}{%
\subsection{`What's Missing?
Everything!'}\label{whats-missing-everything}}

On an expanse of rocky ground near the town of Khamer northwest of the
capital, where they have been since fleeing their homes last year,
hundreds of families have built shelters out of canvas, plastic sheeting
and mud bricks. Most survive on charity, eating rice and bread cooked on
mud stoves fired with wood or garbage.

Image

A camp of displaced Yemenis in the town of Khamer last month.
International groups have struggled to get them aid.Credit...Tyler
Hicks/The New York Times

In one tent, Farea Gayid, 55, said he had worked as an army engineer
until his unit collapsed when the airstrikes began. An attack near his
home killed his neighbors, so he and his family fled on foot. A trucker
gave them a ride to Khamer, so they settled there, joining the more than
2.5 million Yemenis who the
\href{http://reporting.unhcr.org/node/2647\#_ga=1.211671634.657052128.1471022517}{United
Nations says} are internally displaced.

In August, the government could no longer afford to pay Mr. Gayid his
\$200 monthly salary.

``Now my children beg in the market,'' he said. ``If the situation
continues like this, there is no future.''

While the war spawned Yemen's humanitarian crisis, aid workers say
coalition bombings of critical infrastructure have exacerbated it.

Before the war, Yemen imported 90 percent of its food, mostly through
the Red Sea port of Hodeida.

Last year, the coalition
\href{http://www.reuters.com/article/us-yemen-security-idUSKCN0QN0HX20150819}{bombed
the port}, damaging its cranes. Now ships often wait for weeks at sea to
unload, and some goods are close to expiration by the time they arrive,
said Mr. McGoldrick, the United Nations official.

Image

Yemeni refugees in Khamer, northwest of the capital, Sana.Credit...Tyler
Hicks/The New York Times

The coalition has also bombed key bridges, including
\href{https://twitter.com/AmbassadorPower/status/765682969520418817}{the
main one between the port and the capital}, forcing truckers to take
long detours.

``It is an all-encompassing, applied economic suppression and
strangulation that is causing everyone here to feel it,'' Mr. McGoldrick
said. ``The collapse of the economy is starting to bite very hard.''

According to the \href{https://www.wfp.org/countries/yemen}{World Food
Program}, 14.4 million of Yemen's 26 million people do not have enough
food, and malnutrition is rising.

The suffering is clear in the capital.

``What's missing? Everything!'' said Manal al-Ariqi, a doctor in Sana's
main pediatric hospital. ``We lack medical staff, nurses and medicine.''

Upstairs, nearly every room contained a malnourished baby. Most had been
born to mothers who had fled the war and were too disturbed or
malnourished to breast-feed normally, said Ali al-Faqih, a nurse.

In one room lay 7-month-old twin girls, Ruqaya and Suqaina, both with
sunken cheeks.

``We lost everything because of the war,'' their grandmother Shariya
al-Awaj said when asked why the girls were so small. ``All we brought
with us were our clothes.''

Image

The ancient hilltop town of Kawkaban, a draw for tourists and Yemeni
families before it was bombed.Credit...Tyler Hicks/The New York Times

\hypertarget{the-economic-wreckage}{%
\subsection{The Economic Wreckage}\label{the-economic-wreckage}}

The destruction in Yemen could cripple its economy long into the future,
and it is unclear how the country will rebuild.

``They have hit many factories on the basis of suspicion, but we never
get the real reasons,'' said Abdul-Hakeem Al Manj, a lawyer at the Sana
Chamber of Commerce and Industry who is helping businesses document the
strikes with an eye toward future prosecution. ``Any institution that
has a big hangar, they hit it directly.''

Some businesses said they suspected they were targets only because they
continued to operate after the Houthi takeover.

``For Saudi Arabia, we are all Houthis,'' said Haroon al-Sadi of the
state-owned Amran Cement Factory, which once employed 1,500 people
before
\href{http://www.nytimes.com/2016/02/04/world/middleeast/yemen-bombing-coalition-civilians.html}{it
was bombed} twice.

Plant workers showed us the remains of munitions they had collected,
including pieces of at least one CBU-105, a cluster bomb unit that
contains 10 high-explosive submunitions. They are manufactured by
Textron Defense Systems of Rhode Island.

General Asseri, the coalition spokesman, said it had ``no interest in
damaging any aspect of the Yemeni economy,'' and had made great efforts
to avoid harming civilians. He declined to provide details about
specific sites, but said the coalition had ``accurate intelligence''
that the sites we visited were ``being used by militias to store weapons
and ammunition or a command-and-control center.''

The war has left nothing untouched for the Alsonidar brothers, Khalid
and Abdullah, who own a group of factories outside Sana.

The family works with an Italian company,
\href{http://www.caprari.com/en/welcome.jsp}{Caprari}, to produce
agricultural water pumps. It also owns a brick factory, which was out of
use, and was preparing to open a factory to produce metal pipes to go
with the pumps, also with \href{http://www.addafer.it/}{an Italian
partner}.

Twice in September, the compound was bombed, destroying all three
factories.

Saudi news reports said the factories had produced rockets for the
rebels, a charge the brothers denied. They and their Italian partners
have written to the United Nations to state that the factories could not
produce military technology, and to call for an investigation, which is
continuing, they said.

``We're not talking about something useless,'' Abdullah Alsonidar said.
``We're talking about infrastructure and people's lives. Strikes like
this can bring a family to the ground.''

Remains of munitions that the brothers found at the site indicate that
it was hit with American-made weapons, including one with laser-guidance
equipment that was made in October 2015.

Advertisement

\protect\hyperlink{after-bottom}{Continue reading the main story}

\hypertarget{site-index}{%
\subsection{Site Index}\label{site-index}}

\hypertarget{site-information-navigation}{%
\subsection{Site Information
Navigation}\label{site-information-navigation}}

\begin{itemize}
\tightlist
\item
  \href{https://help.nytimes.com/hc/en-us/articles/115014792127-Copyright-notice}{©~2020~The
  New York Times Company}
\end{itemize}

\begin{itemize}
\tightlist
\item
  \href{https://www.nytco.com/}{NYTCo}
\item
  \href{https://help.nytimes.com/hc/en-us/articles/115015385887-Contact-Us}{Contact
  Us}
\item
  \href{https://www.nytco.com/careers/}{Work with us}
\item
  \href{https://nytmediakit.com/}{Advertise}
\item
  \href{http://www.tbrandstudio.com/}{T Brand Studio}
\item
  \href{https://www.nytimes.com/privacy/cookie-policy\#how-do-i-manage-trackers}{Your
  Ad Choices}
\item
  \href{https://www.nytimes.com/privacy}{Privacy}
\item
  \href{https://help.nytimes.com/hc/en-us/articles/115014893428-Terms-of-service}{Terms
  of Service}
\item
  \href{https://help.nytimes.com/hc/en-us/articles/115014893968-Terms-of-sale}{Terms
  of Sale}
\item
  \href{https://spiderbites.nytimes.com}{Site Map}
\item
  \href{https://help.nytimes.com/hc/en-us}{Help}
\item
  \href{https://www.nytimes.com/subscription?campaignId=37WXW}{Subscriptions}
\end{itemize}
