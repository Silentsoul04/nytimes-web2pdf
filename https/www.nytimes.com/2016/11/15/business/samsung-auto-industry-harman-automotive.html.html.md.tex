Sections

SEARCH

\protect\hyperlink{site-content}{Skip to
content}\protect\hyperlink{site-index}{Skip to site index}

\href{https://myaccount.nytimes.com/auth/login?response_type=cookie\&client_id=vi}{}

\href{https://www.nytimes.com/section/todayspaper}{Today's Paper}

\href{/section/business/dealbook}{DealBook}\textbar{}Samsung to Buy
Harman International in an \$8 Billion Bet on Cars

\url{https://nyti.ms/2ewCbQm}

\begin{itemize}
\item
\item
\item
\item
\item
\end{itemize}

Advertisement

\protect\hyperlink{after-top}{Continue reading the main story}

Supported by

\protect\hyperlink{after-sponsor}{Continue reading the main story}

DealBook Business and Policy

\hypertarget{samsung-to-buy-harman-international-in-an-8-billion-bet-on-cars}{%
\section{Samsung to Buy Harman International in an \$8 Billion Bet on
Cars}\label{samsung-to-buy-harman-international-in-an-8-billion-bet-on-cars}}

\includegraphics{https://static01.nyt.com/images/2016/11/15/world/15DB-SAMSUNG-1/15DB-SAMSUNG-1-articleLarge.jpg?quality=75\&auto=webp\&disable=upscale}

By \href{http://www.nytimes.com/by/amie-tsang}{Amie Tsang}

\begin{itemize}
\item
  Nov. 14, 2016
\item
  \begin{itemize}
  \item
  \item
  \item
  \item
  \item
  \end{itemize}
\end{itemize}

HONG KONG --- Samsung Electronics is spending \$8 billion to get inside
your car.

Samsung, the South Korean electronics giant --- which already makes
popular but
\href{http://www.nytimes.com/2016/10/12/business/international/samsung-galaxy-note7-terminated.html}{recently
problem-plagued} smartphones ---
\href{https://news.samsung.com/global/samsung-electronics-to-acquire-harman-accelerating-growth-in-automotive-and-connected-technologies}{said
on Monday} that it had agreed to buy Harman International Industries, an
American automotive technology company, in an ambitious push into a
whole different kind of mobile.

Harman is best known for making car audio systems under brand names
popular with audiophiles such as Harman/Kardon and JBL. But Harman's
appeal to Samsung comes from what it calls its connected car business
--- an operation that supplies a car's navigation services, its onboard
entertainment systems and its connectivity to the rest of the world.

``The vehicle of tomorrow will be transformed by smart technology and
connectivity in the same way that simple feature phones have become
sophisticated smart devices over the past decade,'' Young Sohn, the
president and chief strategy officer of Samsung Electronics, said in a
news release.

The deal marks the latest ambitious foray by an established name in the
technology world into a new generation of smart objects sometimes
collectively called the internet of things. Under this vision,
everything from home security systems to refrigerators will be connected
to the internet, gathering data and controllable at the touch of a
smartphone icon.

Much of that focus has come down to cars. Last month, the American chip
maker Qualcomm
\href{http://www.nytimes.com/2016/10/28/business/dealbook/qualcomm-acquire-nxp-semiconductors.html}{agreed
to acquire NXP Semiconductors} for \$38.5 billion, which would give it a
presence in the market for making a new generation of chips for smart
cars.

With cars likely to get more screens and more computers, the purchase
gives Samsung a stake in what could be an industrywide boom. It also
could provide insight for the company's varied components businesses.
Samsung can learn firsthand from Harman what it needs to do to sell its
screens, chips and memory to carmakers.

Other major technology names are also betting on mobile and smart
gadgets. In July, SoftBank of Japan struck a deal to
\href{http://www.nytimes.com/2016/07/19/business/dealbook/softbank-buys-chip-designer-arm.html}{acquire
ARM Holdings}, a British chip designer with a focus on mobile devices,
for \$32 billion. Last year, Avago Technologies
\href{http://www.nytimes.com/2015/05/29/business/dealbook/avago-agrees-to-acquire-broadcom-for-37-billion.html}{bought
Broadcom}, which provides chips for the Apple iPhone, for \$37 billion.

It is far from certain whether those technologies will end up being the
ones that power the smart gadgets of tomorrow.
\href{http://www.nytimes.com/2016/09/10/technology/apple-is-said-to-be-rethinking-strategy-on-self-driving-cars.html}{Apple}
and
\href{http://www.nytimes.com/2016/08/06/technology/alphabet-google-autonomous-car-chris-urmson.html}{Google}
have expressed interest in developing cars, while traditional automotive
suppliers have also looked to move up the value chain.

Samsung's \$112-a-share offer for Harman represents a 28 percent premium
from where its shares traded on Friday, but that is still well below the
roughly \$145 that each Harman share was fetching early last year.
Harman's results from its professional solutions business --- which
makes sound and lighting for concerts and other events --- have
weakened. The company has said it will work to bring the operations back
to their previous strength.

Samsung has largely benefited from the new mobile world, as growing
demand for smartphones bolstered sales of its displays and microchips.
But the company has faced difficulties selling its own branded phones,
including a drop in market share, as Apple captured more of the high end
and a new generation of low-cost Chinese manufacturers increased
pressure on the bottom.

The company regained some ground with the Galaxy 7 line of curved
phones. But last month, in an embarrassing turnabout, it discontinued
its new, premium Galaxy Note 7 after several caught fire. The stumble
\href{http://www.nytimes.com/2016/10/28/business/samsung-galaxy-note-7-profit.html}{wiped
\$2 billion off its profit} and cast a shadow over the Samsung brand
name.

The deal for Harman is a rare one for Samsung, which keeps tight control
of its supply chain --- often owning its suppliers outright --- and has
mostly eschewed big deals to fill in holes in its portfolio.

Samsung said that it would also have access to Harman's designers and
engineers, which would allow for more collaboration. It did not give
details on what sorts of services they would aim to build together.

It said that Dinesh Paliwal, Harman's chairman and chief executive,
would continue to run the operation, and that it would keep the
company's facilities.

The deal is expected to close in mid-2017. Samsung was advised by
Evercore, with Paul Hastings as legal counsel. JPMorgan and Lazard
advised Harman, with Wachtell, Lipton, Rosen \& Katz as legal counsel.

Advertisement

\protect\hyperlink{after-bottom}{Continue reading the main story}

\hypertarget{site-index}{%
\subsection{Site Index}\label{site-index}}

\hypertarget{site-information-navigation}{%
\subsection{Site Information
Navigation}\label{site-information-navigation}}

\begin{itemize}
\tightlist
\item
  \href{https://help.nytimes.com/hc/en-us/articles/115014792127-Copyright-notice}{©~2020~The
  New York Times Company}
\end{itemize}

\begin{itemize}
\tightlist
\item
  \href{https://www.nytco.com/}{NYTCo}
\item
  \href{https://help.nytimes.com/hc/en-us/articles/115015385887-Contact-Us}{Contact
  Us}
\item
  \href{https://www.nytco.com/careers/}{Work with us}
\item
  \href{https://nytmediakit.com/}{Advertise}
\item
  \href{http://www.tbrandstudio.com/}{T Brand Studio}
\item
  \href{https://www.nytimes.com/privacy/cookie-policy\#how-do-i-manage-trackers}{Your
  Ad Choices}
\item
  \href{https://www.nytimes.com/privacy}{Privacy}
\item
  \href{https://help.nytimes.com/hc/en-us/articles/115014893428-Terms-of-service}{Terms
  of Service}
\item
  \href{https://help.nytimes.com/hc/en-us/articles/115014893968-Terms-of-sale}{Terms
  of Sale}
\item
  \href{https://spiderbites.nytimes.com}{Site Map}
\item
  \href{https://help.nytimes.com/hc/en-us}{Help}
\item
  \href{https://www.nytimes.com/subscription?campaignId=37WXW}{Subscriptions}
\end{itemize}
