Sections

SEARCH

\protect\hyperlink{site-content}{Skip to
content}\protect\hyperlink{site-index}{Skip to site index}

\href{https://myaccount.nytimes.com/auth/login?response_type=cookie\&client_id=vi}{}

\href{https://www.nytimes.com/section/todayspaper}{Today's Paper}

\href{/section/upshot}{The Upshot}\textbar{}Why Keeping Only the Popular
Parts of Obamacare Won't Work

\url{https://nyti.ms/2eVvS4r}

\begin{itemize}
\item
\item
\item
\item
\item
\item
\end{itemize}

Advertisement

\protect\hyperlink{after-top}{Continue reading the main story}

Supported by

\protect\hyperlink{after-sponsor}{Continue reading the main story}

Upshot

Public Health

\hypertarget{why-keeping-only-the-popular-parts-of-obamacare-wont-work}{%
\section{Why Keeping Only the Popular Parts of Obamacare Won't
Work}\label{why-keeping-only-the-popular-parts-of-obamacare-wont-work}}

\includegraphics{https://static01.nyt.com/images/2016/11/15/upshot/15UP-ACA/15UP-ACA-articleInline.jpg?quality=75\&auto=webp\&disable=upscale}

By \href{http://www.nytimes.com/by/margot-sanger-katz}{Margot
Sanger-Katz}

\begin{itemize}
\item
  Nov. 15, 2016
\item
  \begin{itemize}
  \item
  \item
  \item
  \item
  \item
  \item
  \end{itemize}
\end{itemize}

Before Obamacare, it could be hard to buy your own insurance if you'd
already had a health problem like cancer. An insurance company might
have decided not to sell any insurance to someone like you. It might
have agreed to cover you, but not cover cancer care. Or it might have
offered you a comprehensive policy, but at some incredibly high price
that you could never have paid.

Donald J. Trump says he wants to do away with much of Obamacare, but he
has signaled that parts of the law that banned those practices are good
policy he'd want to keep. ``I like those very much,'' he
\href{http://www.wsj.com/articles/donald-trump-willing-to-keep-parts-of-health-law-1478895339}{told
The Wall Street Journal} last week about the law's rules that prevent
discrimination based on pre-existing conditions.

The pre-existing conditions policies are very popular. Nearly everyone
has relatives or friends with illnesses in their past ---
\href{http://www.coloradohealthagents.com/anthem-bcbs/pre-existing-conditions.html}{cancer},
\href{http://www.indianahealthagents.com/anthem/uninsurable-conditions.pdf}{arthritis},
depression,
\href{http://www.consumerwatchdog.org/newsrelease/pre-existing-health-conditions-cops-firefighters-expectant-dads-and-those-suffering-alle}{even
allergies} --- that could have shut them out of the individual insurance
markets before Obamacare, so it's an issue that hits close to home for
many Americans.

But keeping those provisions while jettisoning others is most likely no
fix at all.

Those policies that make the insurance market feel fairer for sick
Americans who need it can really throw off the prices for everyone else.
That's why Obamacare also includes less popular policies designed to
balance the market with enough young, healthy people.

Imagine you're that patient with cancer. You really want health
insurance, and you're probably willing to pay a lot to get it. If the
law requires insurance companies to offer you a policy, you are very
likely to buy it.

Now imagine you're a young, healthy person without any health problems.
Your budget is tight, and health insurance is expensive. You might
decide you'll be fine without insurance, since you can always buy it
later, when you're the one with a pessimistic diagnosis.

Before Obamacare, several states tried policies like this, and required
insurance companies to sell insurance to everyone at the same price,
regardless of health histories. The results were nearly the same
everywhere:
\href{http://www.sciencedirect.com/science/article/pii/S0047272708001199}{Prices
went way up; enrollment went way down}; and insurance companies fled the
markets.

Some states hobbled along with
\href{http://www.nytimes.com/2014/11/08/upshot/just-because-a-policy-causes-a-death-spiral-doesnt-mean-its-unsustainable.html}{small,
expensive markets}. Some experienced total market collapse and repealed
the policies. Prices in those markets typically became so high that they
were really a good deal only for people who knew they'd use a lot of
health care services. And the sicker the insurance pool got, the more
the companies would charge for their health plans.

The health law attempts to broaden the pool by offering financial
assistance to middle-class people. By limiting how much people can be
asked to pay for insurance, the law's subsidies help make the purchase
more attractive for healthier customers. That's the law's carrot.

Then there's the stick: The law says that if you don't buy insurance,
and you could have afforded it, you have to pay a fine. That rule is
designed to discourage people from gaming the system by waiting until
they're sick. The mandate remains the law's
\href{http://kff.org/health-reform/poll-finding/kaiser-health-policy-tracking-poll-december-2014/}{least
popular} provision.

New York is a great case study. Before Obamacare, it had the
pre-existing conditions policy, but without subsidies or a mandate. When
the Obamacare rules kicked in,
\href{http://www.nytimes.com/2013/07/17/health/health-plan-cost-for-new-yorkers-set-to-fall-50.html?pagewanted=all}{premiums
there went down by 50 percent}.

This year, Obamacare premiums have risen substantially --- an average of
22 percent around the country --- leading many experts and politicians
to question
\href{http://www.nytimes.com/2016/09/25/upshot/football-team-at-the-buffet-why-obamacare-markets-are-in-crisis.html}{whether
the law's incentives were strong enough}. Some, including Hillary
Clinton, have argued that the government should sweeten the carrot, by
making the subsidies more generous. Others have said that the stick
should sting more by forcing the uninsured to pay a bigger penalty for
sitting out of the market.

Republican politicians have tended to criticize both of the incentive
provisions. The subsidies have been attacked as excessive government
spending. The mandate has been criticized as an inappropriate use of
government power. Both have been the subject of big Supreme Court cases
challenging the law. Both would have been eliminated
\href{http://www.nytimes.com/2016/11/10/upshot/the-future-of-obamacare-looks-bleak.html}{under
a bill passed by Congress but vetoed by President Obama} last year.

Taking away those unpopular pieces of the law and keeping the popular
pre-existing conditions piece might seem like a political win. But it
would result in a broken system.

When Mitt Romney was devising the Massachusetts health reform law that
would become the model for Obamacare, he hoped to set up a marketplace
for health plans with some financial assistance for low-income people to
buy insurance. What he didn't want was a mandate.

Then Jonathan Gruber, an M.I.T. economist who had calculated the
results, showed him the numbers: His plan would cover only a third of
the uninsured and cost two-thirds as much as an identical plan with a
mandate. Mr. Romney embraced the mandate.

When Barack Obama ran for president in 2007, he, too, advocated a
market-based health reform system. He, too, said
\href{http://www.nytimes.com/2007/11/16/us/politics/16facts.html}{he did
not support a mandate}. Then he became president, and economists
\href{http://www.newyorker.com/news/news-desk/the-mandate-memo-how-obama-changed-his-mind}{brought
him the numbers}. By the time the Affordable Care Act passed, he had
changed his mind.

We'll see what happens when the economists bring the numbers to Mr.
Trump. His
\href{https://www.greatagain.gov/policy/healthcare.html}{transition
website} suggests that he might develop a different solution to the
problem: a special, separate insurance market just for sick people.

But that plan is different from the more modest amendments to the
Affordable Care Act he described to The Wall Street Journal. It won't be
easy to keep the basic architecture of Obamacare while plucking out its
least popular pieces. (Another provision that Mr. Trump says he likes,
the requirement that insurers cover young adults on their parents'
policies, would be easier to save.)

Last year, I spoke with Mark Hall, a law professor at Wake Forest
University who
\href{http://law.wfu.edu/faculty/assets/profile/cv/cv.hallma.pdf\#page=10}{studied
the states}that had tried pre-existing conditions bans before Obamacare.
One of the Supreme Court cases
\href{http://www.nytimes.com/2015/03/02/upshot/how-an-adverse-supreme-court-ruling-would-send-obamacare-into-a-tailspin.html}{threatened
to wipe out the mandate and the subsidies}, and I asked him what would
happen if the litigants succeeded.

``It would be a big mess,'' he said.

Advertisement

\protect\hyperlink{after-bottom}{Continue reading the main story}

\hypertarget{site-index}{%
\subsection{Site Index}\label{site-index}}

\hypertarget{site-information-navigation}{%
\subsection{Site Information
Navigation}\label{site-information-navigation}}

\begin{itemize}
\tightlist
\item
  \href{https://help.nytimes.com/hc/en-us/articles/115014792127-Copyright-notice}{©~2020~The
  New York Times Company}
\end{itemize}

\begin{itemize}
\tightlist
\item
  \href{https://www.nytco.com/}{NYTCo}
\item
  \href{https://help.nytimes.com/hc/en-us/articles/115015385887-Contact-Us}{Contact
  Us}
\item
  \href{https://www.nytco.com/careers/}{Work with us}
\item
  \href{https://nytmediakit.com/}{Advertise}
\item
  \href{http://www.tbrandstudio.com/}{T Brand Studio}
\item
  \href{https://www.nytimes.com/privacy/cookie-policy\#how-do-i-manage-trackers}{Your
  Ad Choices}
\item
  \href{https://www.nytimes.com/privacy}{Privacy}
\item
  \href{https://help.nytimes.com/hc/en-us/articles/115014893428-Terms-of-service}{Terms
  of Service}
\item
  \href{https://help.nytimes.com/hc/en-us/articles/115014893968-Terms-of-sale}{Terms
  of Sale}
\item
  \href{https://spiderbites.nytimes.com}{Site Map}
\item
  \href{https://help.nytimes.com/hc/en-us}{Help}
\item
  \href{https://www.nytimes.com/subscription?campaignId=37WXW}{Subscriptions}
\end{itemize}
