Sections

SEARCH

\protect\hyperlink{site-content}{Skip to
content}\protect\hyperlink{site-index}{Skip to site index}

\href{https://www.nytimes.com/section/politics}{Politics}

\href{https://myaccount.nytimes.com/auth/login?response_type=cookie\&client_id=vi}{}

\href{https://www.nytimes.com/section/todayspaper}{Today's Paper}

\href{/section/politics}{Politics}\textbar{}Michael Flynn, Anti-Islamist
Ex-General, Offered Security Post, Trump Aide Says

\url{https://nyti.ms/2eM6sek}

\begin{itemize}
\item
\item
\item
\item
\item
\item
\end{itemize}

Advertisement

\protect\hyperlink{after-top}{Continue reading the main story}

Supported by

\protect\hyperlink{after-sponsor}{Continue reading the main story}

\hypertarget{michael-flynn-anti-islamist-ex-general-offered-security-post-trump-aide-says}{%
\section{Michael Flynn, Anti-Islamist Ex-General, Offered Security Post,
Trump Aide
Says}\label{michael-flynn-anti-islamist-ex-general-offered-security-post-trump-aide-says}}

\includegraphics{https://static01.nyt.com/images/2016/11/18/us/18flynn/18flynn-articleLarge.jpg?quality=75\&auto=webp\&disable=upscale}

By \href{http://www.nytimes.com/by/matthew-rosenberg}{Matthew Rosenberg}
and \href{http://www.nytimes.com/by/maggie-haberman}{Maggie Haberman}

\begin{itemize}
\item
  Nov. 17, 2016
\item
  \begin{itemize}
  \item
  \item
  \item
  \item
  \item
  \item
  \end{itemize}
\end{itemize}

WASHINGTON --- President-elect
\href{http://www.nytimes.com/2016/11/18/us/politics/donald-trump-transition.html}{Donald
J. Trump} has offered the post of national security adviser to Lt. Gen.
Michael T. Flynn, potentially putting a retired intelligence officer who
believes Islamist militancy poses an existential threat in one of the
most powerful roles in shaping military and foreign policy, according to
a top official on Mr. Trump's transition team.

General Flynn, 57, a registered Democrat, was Mr. Trump's main national
security adviser during his campaign. If he accepts Mr. Trump's offer,
as expected, he will be a critical gatekeeper for a president with
little experience in military or foreign policy issues.

Mr. Trump and General Flynn both see themselves as brash outsiders who
hustled their way to the big time. They both post on Twitter often about
their own successes, and they have both at times crossed the line into
outright Islamophobia.

They also both exhibit a loose relationship with facts: General Flynn,
for instance, has said that Shariah, or Islamic law, is spreading in the
United States (it is not). His dubious assertions are so common that
when he ran the Defense Intelligence Agency, subordinates came up with a
name for the phenomenon: They called them ``Flynn facts.''

As an adviser, General Flynn has already proved to be a powerful
influence on Mr. Trump, convincing the president-elect that the United
States is in a ``world war'' with Islamist militants and must work with
any willing allies in the fight, including President Vladimir V. Putin
of Russia.

During the transition, General Flynn has been present when Mr. Trump has
received his daily intelligence briefing. As national security adviser,
he would have the last word on how the president should respond to
crises such as a showdown with China over the South China Sea or an
international health crisis like the Ebola epidemic.

But, like Mr. Trump, he would enter the White House with significant
baggage. The Flynn Intel Group, a consulting firm he founded after he
was fired by President Obama as head of the Defense Intelligence Agency,
has hazy business ties to Middle Eastern countries and has appeared to
lobby for the Turkish government. General Flynn also took a paid
speaking engagement last year with Russia Today, a television network
funded by the Kremlin, and attended the network's lavish anniversary
party in Moscow, where he sat at Mr. Putin's elbow.

Those potential conflicts of interest had led Mr. Trump's transition
team to worry that General Flynn might have difficulty winning
confirmation for any post that, unlike the national security adviser
role, requires congressional approval, such as director of the C.I.A.
But for Mr. Trump, he has one overriding virtue: He was an early and
ardent supporter in a campaign during which most of the Washington
national security establishment openly called Mr. Trump unfit to lead.

General Flynn did not respond to repeated interview requests. Yet in
numerous speeches and interviews before the election, and in a book
published in August, he laid out a view of the world that sees the
United States as facing a singular, overarching threat that can be
described in only one way: ``radical Islamic terrorism.''

All else is secondary for General Flynn, and any other description of
the threat is ``the worst kind of political correctness,'' he said in an
interview three weeks before the election.

Islamist militancy poses an existential threat on a global scale, and
the Muslim faith itself is the source of the problem, he said,
describing it as a political ideology, not a religion. He has even at
times gone so far as to call it a political ideology that has
``metastasized'' into a ``malignant cancer.''

For General Flynn, the election of Mr. Trump represents an astounding
career turnaround. Once counted among the most respected military
officers of his generation, General Flynn was fired after serving only
two years as chief of the Defense Intelligence Agency. He then
re-emerged as a vociferous critic of a Washington elite that he
contended could not even properly identify the real enemy --- radical
Islam, that is --- never mind figure out how to defeat it.

\href{https://www.nytimes.com/interactive/2016/us/politics/donald-trump-administration.html}{}

\includegraphics{https://static01.nyt.com/images/2016/11/11/us/politics/donald-trump-administration-1478905372015/donald-trump-administration-1478905372015-square640.jpg}

\hypertarget{donald-trumps-cabinet-is-complete-heres-the-full-list}{%
\subsection{Donald Trump's Cabinet Is Complete. Here's the Full
List.}\label{donald-trumps-cabinet-is-complete-heres-the-full-list}}

A list of appointees and nominees for top posts in the new
administration.

In Mr. Trump, General Flynn found someone who was more than willing to
listen. He readily signed on to the campaign, and quickly emerged as
\href{http://www.nytimes.com/2016/10/19/us/politics/michael-flynn-donald-trump.html}{the
angry voice} of the national security establishment, leading chants of
``lock her up'' against Hillary Clinton at rallies and the Republican
convention. And now, after months of the two men talking to each other,
it can be hard to tell where Mr. Trump's views end and General Flynn's
begin.

They both believe that the United States needs to start working with Mr.
Putin to defeat Islamist militants and stop worrying about his
suppression of critics at home, his attempt to
\href{http://www.nytimes.com/2014/03/19/world/europe/ukraine.html}{dismember
Ukraine} or the Russian military's
\href{http://www.nytimes.com/2016/09/24/world/middleeast/aleppo-syria-airstrikes.html}{indiscriminate
bombing} of Syrian cities. The same goes for President Abdel Fattah
el-Sisi of Egypt, who took power in a coup and who was the
\href{http://www.nytimes.com/2016/11/16/us/politics/trump-transition.html?hp\&action=click\&pgtype=Homepage\&clickSource=story-heading\&module=span-ab-top-region\&region=top-news\&WT.nav=top-news}{first
world leader} to speak with Mr. Trump after the election.

Mr. Trump ``looks at people and leaders of countries and says: `Can I
work with this guy? Do we have a common threat that we can focus on?'''
Mr. Flynn said in the interview before the election. ``He knows that
when it comes to Russia or any other country, the common enemy that we
all have is radical Islam.''

General Flynn and Mr. Trump also agree that the United States needs to
sharply curtail immigration from predominantly Muslim countries, and
possibly even force American Muslims to register with the government.

The similarities run beyond political views. Like the boy from Queens
who made it in Manhattan, General Flynn came into the military without a
West Point pedigree --- he graduated from the Army's Reserve Officer
Training Program at the University of Rhode Island --- and earned a
reputation as outspoken and unconventional as he climbed the ranks to
the top of military intelligence.

Yet General Flynn still nurses the grudge of an outsider, believing he
never quite got the respect he deserves. For example, he has
\href{https://www.washingtonpost.com/world/national-security/nearly-the-entire-national-security-establishment-has-rejected-trumpexcept-for-this-man/2016/08/15/d5072d96-5e4b-11e6-8e45-477372e89d78_story.html}{attributed
his dismissal} from the Defense Intelligence Agency to a pair of
consummate insiders: James R. Clapper Jr., the director of national
intelligence, and Michael Vickers, the undersecretary of defense
intelligence.

His response, like that of his new boss, has been to buck the
establishment. In his view, both the Republican and the Democratic
luminaries who have shaped American defense and foreign policy through
two presidencies have ``gotten us into mess after mess for the wrong
reasons.''

``I would argue with that crowd all day long,'' he said before the
election.

Among the hard-line Republicans who now dominate the party, General
Flynn has become something of a cult figure for what they see as his
brave stand against the Obama administration's perfidy. General Flynn
insists that he was fired from the intelligence agency because he
refused to toe the administration's line that Islamist militants were in
retreat. (He was right, in all fairness.)

``He's an analyst who can get deep into the weeds on the issues and a
lot of this stuff and then is very good at playing three-dimensional
chess,'' said Representative Devin Nunes, the California Republican who
is chairman of the House Intelligence Committee and a close confidant to
Mr. Flynn. ``He was the one who called out the administration for being
wrong on Al Qaeda.''

But many of those who worked with General Flynn attribute his firing to
management problems, saying his attempts to overhaul the sprawling
agency had left it a chaotic, backbiting mess. They also question
whether his tactical acumen --- he was especially good at unraveling
militant networks in Afghanistan and Iraq --- can translate into the
kind of strategic thinking needed at the White House.

``He is a very talented information gatherer,'' said Sarah Chayes of the
Carnegie Endowment in Washington, who worked with General Flynn when he
ran military intelligence in Afghanistan from 2009 to 2011.

``But his thinking process is not sufficiently analytical to test some
streams against others and make sense of it, or draw consistent
conclusions,'' she said. ``If you listen to him, in 10 minutes you'll
hear him contradict himself two or three times.''

Take his views on Islam. In the interview before the election, he
characterized Islam as intolerant.

Then he said that he had many Muslim friends, and that the United States
needed to do a better job of understanding Islamic culture and fostering
its tolerant side.

Advertisement

\protect\hyperlink{after-bottom}{Continue reading the main story}

\hypertarget{site-index}{%
\subsection{Site Index}\label{site-index}}

\hypertarget{site-information-navigation}{%
\subsection{Site Information
Navigation}\label{site-information-navigation}}

\begin{itemize}
\tightlist
\item
  \href{https://help.nytimes.com/hc/en-us/articles/115014792127-Copyright-notice}{©~2020~The
  New York Times Company}
\end{itemize}

\begin{itemize}
\tightlist
\item
  \href{https://www.nytco.com/}{NYTCo}
\item
  \href{https://help.nytimes.com/hc/en-us/articles/115015385887-Contact-Us}{Contact
  Us}
\item
  \href{https://www.nytco.com/careers/}{Work with us}
\item
  \href{https://nytmediakit.com/}{Advertise}
\item
  \href{http://www.tbrandstudio.com/}{T Brand Studio}
\item
  \href{https://www.nytimes.com/privacy/cookie-policy\#how-do-i-manage-trackers}{Your
  Ad Choices}
\item
  \href{https://www.nytimes.com/privacy}{Privacy}
\item
  \href{https://help.nytimes.com/hc/en-us/articles/115014893428-Terms-of-service}{Terms
  of Service}
\item
  \href{https://help.nytimes.com/hc/en-us/articles/115014893968-Terms-of-sale}{Terms
  of Sale}
\item
  \href{https://spiderbites.nytimes.com}{Site Map}
\item
  \href{https://help.nytimes.com/hc/en-us}{Help}
\item
  \href{https://www.nytimes.com/subscription?campaignId=37WXW}{Subscriptions}
\end{itemize}
