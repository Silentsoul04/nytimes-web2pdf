Sections

SEARCH

\protect\hyperlink{site-content}{Skip to
content}\protect\hyperlink{site-index}{Skip to site index}

\href{https://www.nytimes.com/section/business}{Business}

\href{https://myaccount.nytimes.com/auth/login?response_type=cookie\&client_id=vi}{}

\href{https://www.nytimes.com/section/todayspaper}{Today's Paper}

\href{/section/business}{Business}\textbar{}Ford Move, Cited as Victory
by Trump, Has No Effect on U.S. Jobs

\url{https://nyti.ms/2f8uKKX}

\begin{itemize}
\item
\item
\item
\item
\item
\end{itemize}

Advertisement

\protect\hyperlink{after-top}{Continue reading the main story}

Supported by

\protect\hyperlink{after-sponsor}{Continue reading the main story}

\hypertarget{ford-move-cited-as-victory-by-trump-has-no-effect-on-us-jobs}{%
\section{Ford Move, Cited as Victory by Trump, Has No Effect on U.S.
Jobs}\label{ford-move-cited-as-victory-by-trump-has-no-effect-on-us-jobs}}

\includegraphics{https://static01.nyt.com/images/2016/11/19/business/19FORDSUB/19FORDSUB-articleInline.jpg?quality=75\&auto=webp\&disable=upscale}

By Neal E. Boudette

\begin{itemize}
\item
  Nov. 18, 2016
\item
  \begin{itemize}
  \item
  \item
  \item
  \item
  \item
  \end{itemize}
\end{itemize}

Judging from a couple of Twitter messages by President-elect Donald J.
Trump, he had scored a hard-won victory for American autoworkers by
persuading Ford Motor to keep a Lincoln plant in Louisville, Ky., rather
than move it to Mexico.

The reality proved less straightforward.

Ford had never said it was moving a plant to Mexico, only that it was
transferring the production of a small Lincoln sport utility vehicle
there so it could fully dedicate a Louisville plant to a larger-selling
model.

That decision has now been reversed --- but either way, it will have no
impact on jobs at the factory. The plant is already operating virtually
around the clock at full capacity.

The decision, which Ford Motor said it made before Mr. Trump spoke by
phone on Thursday with William Clay Ford Jr., the company's executive
chairman, will simply keep the current product mix in place at the
factory.

The Louisville plant will continue making a far larger number of Ford
Escapes, a small S.U.V. that is a less luxurious vehicle than the
Lincoln model, the MKC.

Ford, which during the election campaign was a frequent target of Mr.
Trump's criticism for moving jobs to Mexico, was no doubt waving a
political olive branch by deciding to keep Lincoln MKC production in
Kentucky. But the move was largely symbolic.

And that Mr. Trump seemingly overstated its impact --- if it proves
emblematic of his future dealings with the industry --- could indicate
that his promises to save and restore auto jobs may not require
significant changes on the part of carmakers.

Mr. Trump's vows to protect manufacturing jobs in the United States
helped him win the support of working-class voters, including many
factory workers in Michigan, Ohio and Kentucky. He sought to underscore
the message in his Twitter dispatches on Thursday night.

``Just got a call from my friend Bill Ford, Chairman of Ford, who
advised me that he will be keeping the Lincoln plant in Kentucky --- no
Mexico,''
\href{https://twitter.com/realDonaldTrump/status/799432403727028224}{Mr.
Trump wrote} in a message.

In a
\href{https://twitter.com/realDonaldTrump/status/799435824622252032}{subsequent
post, he wrote}: ``I worked hard with Bill Ford to keep the Lincoln
plant in Kentucky. I owed it to the great State of Kentucky for their
confidence in me!''

Both posts overstated certain issues.

The plant is not primarily a Lincoln plant --- the MKC represents
roughly 10 percent of its total output. The MKC is a more expensive
version of the Ford Escape, which is a much bigger seller than the MKC.
Production of the Escape alone is enough to keep the Louisville plant
running at full capacity.

Moreover, the decision to keep the MKC in Louisville was made before the
two men spoke on Thursday, not as a result of their conversation,
according to Ford.

``We have been reviewing the sourcing of this product, and Bill Ford
spoke to the President-elect yesterday and shared our recent decision to
keep Lincoln MKC in Kentucky,'' a Ford spokeswoman, Christin Baker, said
in a statement on Friday. ``We are encouraged the economic policies he
will pursue will help improve U.S. competitiveness and make it possible
to keep production of this vehicle here in the U.S.''

Ford's chief financial officer, Robert L. Shanks, held a conference call
with analysts on Thursday morning in which he expressed hope that Mr.
Trump's policies would ``provide an environment where it makes economic
sense to build back up manufacturing jobs here.''

But how Mr. Trump governs may be ``a bit different'' from his campaign
speeches, Mr. Shanks said. ``So let's just wait and see.''

During the campaign, Mr. Trump heavily criticized Ford for deciding to
shuffle its manufacturing operations so that all its small cars are made
in Mexico. At times, he even suggested hitting the company and others
with a 35 percent tariff on vehicles imported from Mexico.

Ford has countered that moving small-car assembly to Mexican plants
would have no impact on American jobs.

For example, a factory in Wayne, Mich., that now makes the weakly
selling Ford Focus compact will be retooled to make trucks and S.U.V.s,
which are selling briskly. Ford said the higher profit margins on trucks
and S.U.V.s allow it to absorb the higher labor costs of building the
vehicles in the United States.

The Wayne plant is expected to remain fully staffed with 3,700 workers.

In October, speaking to reporters at an auto technology conference, Mr.
Ford voiced frustration with the criticism Mr. Trump was then aiming at
the company.

``Look, we are everything he should be celebrating about this country,''
Mr. Ford said, noting that the company makes more cars and trucks in
United States plants than any of its rivals and that it was investing in
its American operations and adding jobs.

``He knows all that,'' Mr. Ford said of Mr. Trump at the time. ``I can't
control what he says.''

This week, Ford's chief executive, Mark Fields, spoke at the Los Angeles
Auto Show and reiterated the company's commitment to shift assembly of
small cars like the Focus to Mexico.

The decision on the MKC gave Mr. Ford some good news to pass on to Mr.
Trump.

Ford wanted to move the MKC to another plant to increase production of
the Escape, a change that had been planned for 2018. It could still
increase output of the Escape by cutting back the number of MKCs it
makes in Louisville, or it could move the Lincoln model to another plant
in the United States.

The Louisville plant's work force would remain unchanged even if the MKC
were moved to a new factory. The factory employs 4,500 hourly workers
and is operating on three shifts, producing vehicles almost around the
clock.

In the first 10 months of this year, the plant made nearly 300,000 Ford
Escapes and just over 37,000 Lincoln MKCs.

In the auto industry, plants are considered to be operating at 100
percent capacity if they are running two shifts a day. Most typically
produce 200,000 to 250,000 vehicles a year.

Advertisement

\protect\hyperlink{after-bottom}{Continue reading the main story}

\hypertarget{site-index}{%
\subsection{Site Index}\label{site-index}}

\hypertarget{site-information-navigation}{%
\subsection{Site Information
Navigation}\label{site-information-navigation}}

\begin{itemize}
\tightlist
\item
  \href{https://help.nytimes.com/hc/en-us/articles/115014792127-Copyright-notice}{©~2020~The
  New York Times Company}
\end{itemize}

\begin{itemize}
\tightlist
\item
  \href{https://www.nytco.com/}{NYTCo}
\item
  \href{https://help.nytimes.com/hc/en-us/articles/115015385887-Contact-Us}{Contact
  Us}
\item
  \href{https://www.nytco.com/careers/}{Work with us}
\item
  \href{https://nytmediakit.com/}{Advertise}
\item
  \href{http://www.tbrandstudio.com/}{T Brand Studio}
\item
  \href{https://www.nytimes.com/privacy/cookie-policy\#how-do-i-manage-trackers}{Your
  Ad Choices}
\item
  \href{https://www.nytimes.com/privacy}{Privacy}
\item
  \href{https://help.nytimes.com/hc/en-us/articles/115014893428-Terms-of-service}{Terms
  of Service}
\item
  \href{https://help.nytimes.com/hc/en-us/articles/115014893968-Terms-of-sale}{Terms
  of Sale}
\item
  \href{https://spiderbites.nytimes.com}{Site Map}
\item
  \href{https://help.nytimes.com/hc/en-us}{Help}
\item
  \href{https://www.nytimes.com/subscription?campaignId=37WXW}{Subscriptions}
\end{itemize}
