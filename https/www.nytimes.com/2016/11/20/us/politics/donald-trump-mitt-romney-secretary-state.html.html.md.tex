Sections

SEARCH

\protect\hyperlink{site-content}{Skip to
content}\protect\hyperlink{site-index}{Skip to site index}

\href{https://www.nytimes.com/section/politics}{Politics}

\href{https://myaccount.nytimes.com/auth/login?response_type=cookie\&client_id=vi}{}

\href{https://www.nytimes.com/section/todayspaper}{Today's Paper}

\href{/section/politics}{Politics}\textbar{}Trump Meets With Romney as
He Starts to Look Outside His Inner Circle

\url{https://nyti.ms/2faYgzB}

\begin{itemize}
\item
\item
\item
\item
\item
\end{itemize}

Advertisement

\protect\hyperlink{after-top}{Continue reading the main story}

Supported by

\protect\hyperlink{after-sponsor}{Continue reading the main story}

\hypertarget{trump-meets-with-romney-as-he-starts-to-look-outside-his-inner-circle}{%
\section{Trump Meets With Romney as He Starts to Look Outside His Inner
Circle}\label{trump-meets-with-romney-as-he-starts-to-look-outside-his-inner-circle}}

\includegraphics{https://static01.nyt.com/images/2016/11/20/us/20TRANSITION-SUB/20TRANSITION-SUB-articleInline.jpg?quality=75\&auto=webp\&disable=upscale}

By \href{http://www.nytimes.com/by/michael-s-schmidt}{Michael S.
Schmidt} and
\href{https://www.nytimes.com/by/julie-hirschfeld-davis}{Julie
Hirschfeld Davis}

\begin{itemize}
\item
  Nov. 19, 2016
\item
  \begin{itemize}
  \item
  \item
  \item
  \item
  \item
  \end{itemize}
\end{itemize}

BEDMINSTER, N.J. --- President-elect Donald J. Trump on Saturday moved
to mend fences with political rivals after a divisive campaign, meeting
with Mitt Romney, who had scathingly criticized him during the race as
``a phony'' and ``a fraud,'' to discuss naming him as secretary of
state.

The outreach signaled a change in tone one day after Mr. Trump moved to
elevate hard-liners to pivotal national security positions. It was not
clear whether Mr. Trump had offered the State Department post to Mr.
Romney, or whether Mr. Romney, who has broken sharply with him on
Russia, free trade and other issues, would accept if he did.

But some strategists argued that merely by reaching out to Mr. Romney,
Mr. Trump was demonstrating an openness to new people and ideas, even
from the unlikeliest of sources. It may also have been intended to
inject the sort of unpredictability and spectacle into the transition
process that the president-elect thrives on.

During a weekend of transition talks at Trump National Golf Club here in
Bedminster, Mr. Trump was scheduled to hold a series of discussions with
what his aides described as a diverse array of potential advisers. The
conversations were aimed at showing that the president-elect was willing
to look beyond his loyal inner circle to fill his administration.

Among the others who sat down with Mr. Trump were Michelle A. Rhee, a
Democrat who served as the chancellor of public schools in the District
of Columbia from 2007 to 2010; Robert L. Woodson, an African-American
conservative who works on community-based anti-poverty programs; James
N. Mattis, a retired Marine Corps general who headed the United States
Central Command and is being considered for the post of defense
secretary; and Todd Ricketts, an owner of the Chicago Cubs.

Mr. Trump met with Mr. Romney for about an hour and a half. Afterward,
both men exited the clubhouse and shook hands for the cameras. ``Went
great,'' Mr. Trump said, cupping his hands at his mouth to project his
voice. Mr. Romney then briefly addressed reporters, declining to say
whether he was interested in a cabinet position.

``We had a far-reaching conversation with regard to the various theaters
of the world with interest to the United States of real significance,''
Mr. Romney said. ``We discussed those areas and exchanged our views on
those topics. A very thorough and in-depth discussion over the time we
had. I appreciate the chance to speak with the president-elect and look
forward to the coming administration.''

\includegraphics{https://static01.nyt.com/images/2016/11/20/us/20TRANSITION/20TRANSITION-articleLarge.jpg?quality=75\&auto=webp\&disable=upscale}

Mr. Romney did not answer reporters' questions about whether he had
apologized to Mr. Trump for his criticism of him during the campaign.

Later in the day, Mr. Trump strongly praised General Mattis after
talking with him for about an hour. ``All I can say is he is the real
deal --- the real deal,'' Mr. Trump said.

Asked whether General Mattis, who was an assertive presence on the
battlefields of Iraq and Afghanistan, would join the administration, Mr.
Trump said: ``We'll see. We'll see.'' He added: ``He's just a brilliant,
wonderful man. What a career --- we are going to see what happens, but
he is the real deal.''

Mr. Trump said on Saturday that there would be some personnel
announcements on Sunday. The president-elect is scheduled to meet with a
similarly wide-ranging group, including Gov. Chris Christie of New
Jersey, whom he removed as the chairman of his transition team days
after the election; Rudolph W. Giuliani, who has also been a contender
for the secretary of state post; and Kris Kobach, the Kansas secretary
of state, who has pressed aggressive measures to crack down on
undocumented immigrants.

The meeting schedule, Jason Miller, a spokesman for Mr. Trump's
transition team, said on Saturday, ``really shows the reach and the
depth to which we are going to pull in diverse ideas and different
perspectives as we form this administration.''

``As we're working to bring the country together and move forward,'' Mr.
Miller added in a conference call with reporters, ``this shows really
where his head is as the next leader of our country.''

On Friday, Mr. Trump moved to install Michael T. Flynn, a retired
lieutenant general bent on destroying Islamic extremism, as his national
security adviser, and he selected Senator Jeff Sessions of Alabama, an
immigration hard-liner, as his attorney general. Both were early
supporters of Mr. Trump's campaign.

Mr. Romney fits a decidedly different mold. Earlier this year, he said
that if Mr. Trump became the Republican nominee, ``the prospects for a
safe and prosperous future are greatly diminished,'' and he suggested
that Mr. Trump was dangerous and unstable. He deplored Mr. Trump's
personal qualities: ``the bullying, the greed, the showing off, the
misogyny, the absurd third-grade theatrics.''

Image

James N. Mattis, a retired Marine Corps general, after meeting with Mr.
Trump on Saturday. General Mattis is being considered for the post of
defense secretary.Credit...Hilary Swift for The New York Times

But if he took a cabinet post, Mr. Romney could serve as a moderating
influence on the hard-liners Mr. Trump has already selected, including
Representative Mike Pompeo of Kansas as C.I.A. director and Stephen K.
Bannon as chief strategist. It could also force Mr. Romney to defend
administration policies he did not believe in.

John Feehery, a Republican strategist, said Mr. Trump was showing
``great magnanimity'' by talking to Mr. Romney, the party's 2012 nominee
and a former governor of Massachusetts. ``I think it is meant to
reassure some of the establishment that he is going to reach out to
them, and that's an important part of healing the party.''

Choosing Ms. Rhee as education secretary ``would show real disruption,''
Mr. Feehery added. But she is ``also someone who has a real track record
of delivering the kind of results he's looking for, and who the
establishment hates.''

Ms. Rhee governed with a brash style as chancellor of Washington's
public school system, which was struggling at the time to improve
underperforming schools and reverse below-average test scores. She
enraged teachers' unions by firing teachers who had received poor
evaluations, renegotiating teacher contracts to weaken seniority
protections and tie their pay to student achievement, and endorsing
vouchers to allow poor students to attend private schools.

Contrary to Mr. Trump, who has said he will scrap the set of education
standards known as Common Core, Ms. Rhee, who founded the education
advocacy group StudentsFirst, supports the program.

Mr. Trump traveled on Friday by motorcade to Bedminster. The trip was
the second time that Mr. Trump has left Manhattan since Election Day.
The other trip was to Washington, where he met with President Obama and
congressional Republicans.

The country club is about an hour's drive from Manhattan and on a rural
stretch. Since buying the property in 2002, Mr. Trump has built two
18-hole golf courses on it.

In the evening, Mr. Trump left the clubhouse with Mr. Pence to attend
meetings on another part of the property. About an hour later, Mr. Trump
returned to the clubhouse to have dinner with Patrick Soon-Shiong, a
billionaire pharmaceutical entrepreneur best known for developing the
cancer drug Abraxane.

Advertisement

\protect\hyperlink{after-bottom}{Continue reading the main story}

\hypertarget{site-index}{%
\subsection{Site Index}\label{site-index}}

\hypertarget{site-information-navigation}{%
\subsection{Site Information
Navigation}\label{site-information-navigation}}

\begin{itemize}
\tightlist
\item
  \href{https://help.nytimes.com/hc/en-us/articles/115014792127-Copyright-notice}{©~2020~The
  New York Times Company}
\end{itemize}

\begin{itemize}
\tightlist
\item
  \href{https://www.nytco.com/}{NYTCo}
\item
  \href{https://help.nytimes.com/hc/en-us/articles/115015385887-Contact-Us}{Contact
  Us}
\item
  \href{https://www.nytco.com/careers/}{Work with us}
\item
  \href{https://nytmediakit.com/}{Advertise}
\item
  \href{http://www.tbrandstudio.com/}{T Brand Studio}
\item
  \href{https://www.nytimes.com/privacy/cookie-policy\#how-do-i-manage-trackers}{Your
  Ad Choices}
\item
  \href{https://www.nytimes.com/privacy}{Privacy}
\item
  \href{https://help.nytimes.com/hc/en-us/articles/115014893428-Terms-of-service}{Terms
  of Service}
\item
  \href{https://help.nytimes.com/hc/en-us/articles/115014893968-Terms-of-sale}{Terms
  of Sale}
\item
  \href{https://spiderbites.nytimes.com}{Site Map}
\item
  \href{https://help.nytimes.com/hc/en-us}{Help}
\item
  \href{https://www.nytimes.com/subscription?campaignId=37WXW}{Subscriptions}
\end{itemize}
