Sections

SEARCH

\protect\hyperlink{site-content}{Skip to
content}\protect\hyperlink{site-index}{Skip to site index}

\href{https://www.nytimes.com/section/world/americas}{Americas}

\href{https://myaccount.nytimes.com/auth/login?response_type=cookie\&client_id=vi}{}

\href{https://www.nytimes.com/section/todayspaper}{Today's Paper}

\href{/section/world/americas}{Americas}\textbar{}Donald Trump and the
U.N.: Signs of Clashing Views on Many Issues

\url{https://nyti.ms/2eQZ4OV}

\begin{itemize}
\item
\item
\item
\item
\item
\end{itemize}

Advertisement

\protect\hyperlink{after-top}{Continue reading the main story}

Supported by

\protect\hyperlink{after-sponsor}{Continue reading the main story}

\hypertarget{donald-trump-and-the-un-signs-of-clashing-views-on-many-issues}{%
\section{Donald Trump and the U.N.: Signs of Clashing Views on Many
Issues}\label{donald-trump-and-the-un-signs-of-clashing-views-on-many-issues}}

\includegraphics{https://static01.nyt.com/images/2016/11/19/world/19nationstrump/19nationstrump-articleLarge.jpg?quality=75\&auto=webp\&disable=upscale}

By \href{http://www.nytimes.com/by/somini-sengupta}{Somini Sengupta} and
\href{https://www.nytimes.com/by/rick-gladstone}{Rick Gladstone}

\begin{itemize}
\item
  Nov. 19, 2016
\item
  \begin{itemize}
  \item
  \item
  \item
  \item
  \item
  \end{itemize}
\end{itemize}

UNITED NATIONS --- In the genteel, carpeted halls of the
\href{http://www.nytimes.com/topic/organization/united-nations?inline=nyt-org}{United
Nations} headquarters, a 20-minute walk from Trump Tower, diplomats from
the world over are holding their breath about the American
president-elect.

The optimists among them are expressing relief that Donald J. Trump said
nothing during the campaign about dismantling the United Nations
altogether --- or turning its iconic tower facing the East River into
condos.

Those who represent the United States' closest allies are trying to
learn who Mr. Trump will appoint to crucial foreign policy jobs and how
he will actually approach the pressing crises that are sure to come up
before the Security Council: Syria, Ukraine, North Korea and the
widening chasm between the Israelis and the Palestinians. Not least,
many are trying to persuade his transition team to respect the
international deals the United States has accepted under the auspices of
the United Nations.

``It's important to reaffirm that, more than ever, we need an America
that's committed to world affairs,'' said the ambassador from France,
François Delattre, who noted, without elaborating, that he had met with
members of the Trump team.

\href{http://cic.nyu.edu/people/sarah-cliffe}{Sarah F. Cliffe}, a former
United Nations assistant secretary general who is now director of the
Center on International Cooperation, a research organization at New York
University, said she expected a reprise of the tensions that erupted
between the United States and the United Nations during the
administration of President George W. Bush. John R. Bolton, who was
ambassador to the United Nations under Mr. Bush, once said the United
Nations would be more effective without its top 10 floors, where its
senior leaders have their offices. Mr. Bolton is one of Mr. Trump's many
would-be candidates for secretary of state.

But Ms. Cliffe said Mr. Trump may also find the United Nations useful.

``He prides himself on making deals,'' she said. ``The U.N. is the forum
where countries make deals in their own national interests but that also
does some collective good.''

His own comments about the United Nations are difficult to parse. At a
\href{https://www.gpo.gov/fdsys/pkg/CHRG-109shrg23164/html/CHRG-109shrg23164.htm}{2005
Senate hearing} about planned renovations of the United Nations
headquarters, he described himself as ``a big fan'' and said that ``the
concept of the United Nations and the fact that the United Nations is in
New York is very important to me and very important to the world as far
as I am concerned.''

As one of Manhattan's pre-eminent builders, he also offered
\href{http://www.cbsnews.com/news/trump-touts-un-renovation-plan/}{to
handle the renovation at half the price.} In 2012, he complained on
Twitter about the marble behind the speaker's lectern at the General
Assembly hall, claiming he could build it better.

In March,
\href{https://www.donaldjtrump.com/media/donald-j-trump-remarks-at-aipac}{during
a campaign speech} at the American Israel Public Affairs Committee, he
excoriated the world body as ``not a friend to freedom,'' adding, ``it
surely isn't a friend to Israel.''

Whatever his specific views, a number of statements made by Mr. Trump
and his loyalists have conflicted with the values and positions of the
United Nations. Here are the major examples:

\hypertarget{climate-change}{%
\subsection{Climate Change}\label{climate-change}}

Mr. Trump, who has called climate change a hoax, has suggested he would
withdraw the United States from the Paris accord, widely regarded as the
most important United Nations achievement in years. Secretary General
Ban Ki-moon and President Obama invested a great deal of political
energy in its success.

This past week, speaking in Marrakesh, Morocco, at an international
meeting on climate change, Mr. Ban said he had spoken by telephone with
Mr. Trump and looked forward to a personal meeting about the importance
of saving the planet from environmental disaster.

``As president of the United States, I am sure that he will understand
this, he will listen, he will evaluate his campaign remarks,'' Mr. Ban
said.

Even if Mr. Trump does not withdraw the United States from the accord,
he could ignore important commitments, including cuts in carbon
emissions or contributions to a global fund to help poor countries deal
with the damage wrought by climate change. American disregard of the
accord could cause other countries to renege on their promises as well.

Diplomats are trying to convince the Trump transition team that adhering
to the deal is good for American businesses.

\includegraphics{https://static01.nyt.com/images/2016/11/19/world/19nationstrump2/19nationstrump2-articleLarge.jpg?quality=75\&auto=webp\&disable=upscale}

\hypertarget{refugees-and-migrants}{%
\subsection{Refugees and Migrants}\label{refugees-and-migrants}}

Mr. Trump has said he wants to bar entry to refugees from certain
countries, and his campaign has defended a
\href{http://www.nytimes.com/2016/09/21/us/politics/donald-trump-jr-skittles.html}{widely
pilloried Twitter message} by his son, Donald Trump Jr., comparing
Syrian refugees to poisoned Skittles.

Barring refugees based on where they come from would be in blatant
violation of international law, which requires countries to offer
protection to all those fleeing war and persecution. Mr. Trump's
proposal has already been denounced by the United Nations high
commissioner for human rights, Zeid Ra'ad al-Hussein, who said during
the campaign that Mr. Trump would be
``\href{https://www.nytimes.com/2016/10/13/world/europe/donald\%2Dtrump\%2Dun\%2Dhuman\%2Drights.html}{dangerous
from an international point of view}.''

The incoming United Nations secretary general,
\href{http://www.nytimes.com/2016/10/07/world/americas/antonio-guterres-united-nations-secretary-general.html}{António
Guterres}, was even more pointed. He said that any proposal to restrict
Muslim refugees would only help jihadist organizations spread their
propaganda and recruit followers. ``It's just telling them: `You are
right. We are against you.' ''

He is to take office less than three weeks before Mr. Trump.

Image

Secretary of State John Kerry and Mohammad Javad Zarif, Iran's foreign
minister, in Lausanne, Switzerland, last year after achieving a major
step toward a deal to limit Iran's nuclear program. Mr. Trump has
repeatedly threatened to scrap the agreement.Credit...Pool photo by
Brendan Smialowski

\hypertarget{iran-nuclear-agreement}{%
\subsection{Iran Nuclear Agreement}\label{iran-nuclear-agreement}}

Mr. Trump has repeatedly threatened to scrap the agreement reached last
year between Iran and six world powers, including the United States,
that severely limited Iran's nuclear activities in exchange for
sanctions relief.

Mr. Ban has expressed the opposite view, calling the agreement a
\href{https://www.un.org/sg/en/content/sg/statement/2016-07-20/statement-secretary-general-occasion-first-anniversary-adoption}{triumph
of diplomacy} that reduced the threat of war. He has urged all
participants to respect its provisions.

The agreement was also endorsed by the Security Council in a unanimous
\href{http://www.un.org/en/ga/search/view_doc.asp?symbol=S/RES/2231(2015)}{resolution}.

Almost immediately after the agreement was announced in July 2015, Mr.
Trump expressed his hostility in a Twitter post.

At that same
March\href{http://time.com/4267058/donald-trump-aipac-speech-transcript/}{meeting
of the American Israel Public Affairs Committee}, a lobbying group, Mr.
Trump asserted that ``my No. 1 priority is to dismantle the disastrous
deal with Iran,'' arguing that the Iranians had outsmarted the United
States in winning concessions and could still develop a nuclear weapon
when the pact's restrictions expire in 15 years.

Despite Mr. Trump's assertions, it remains unclear whether he will seek
to annul or amend the nuclear agreement. The other countries involved
--- Britain, China, France, Germany, Iran and Russia --- have all
expressed their intention to honor it.

Senator Bob Corker, a Tennessee Republican who is under consideration
for a cabinet post in the new administration, said Mr. Trump would
probably be cautious about any change of policy.

``I don't think he will tear it up, and I don't think that's the way to
start,''
\href{http://www.cnn.com/2016/11/16/politics/bob-corker-donald-trump-iran-deal/}{Mr.
Corker said on CNN}.

\hypertarget{human-rights}{%
\subsection{Human Rights}\label{human-rights}}

Mr. Trump has vowed to
restore\href{https://www.washingtonpost.com/news/post-politics/wp/2016/02/17/donald-trump-on-waterboarding-torture-works/}{waterboarding}
as a counterterrorism tool, saying ``torture works.'' That would
contravene an
\href{http://www.ohchr.org/EN/ProfessionalInterest/Pages/CAT.aspx}{international
convention on torture}.

Kris Kobach, a member of Mr. Trump's transition team, has
\href{http://www.reuters.com/article/us-usa-trump-immigration-idUSKBN13B05C}{suggested}
that the new administration could create a national registry for
immigrants from countries where terrorist groups are active. That would
go against an international convention on nondiscrimination, which took
effect nearly 50 years ago.

Western powers have sought to use the United Nations to highlight rights
abuses they attribute to President Bashar al-Assad of Syria. Mr. Trump
has made clear that whatever war crimes Mr. Assad may have carried out,
Mr. Trump's first priority is the Islamic State. ``I don't like Assad at
all, but Assad is killing ISIS,'' he said in October. ``Russia is
killing ISIS. And Iran is killing ISIS.''

\hypertarget{arms-trade-treaty}{%
\subsection{Arms Trade Treaty}\label{arms-trade-treaty}}

Mr. Trump has not specified his view on the
\href{http://www.thearmstradetreaty.org/index.php/en/the-arms-trade-treaty}{Arms
Trade Treaty}, which was overwhelmingly adopted in 2013 by the General
Assembly and entered into legal force nearly two years ago. But
gun-rights advocates in the United States led by the National Rifle
Association, one of Mr. Trump's most powerful supporters, strongly
criticized the treaty and vowed to ensure that the Senate never ratifies
it.

The treaty is an attempt to regulate the enormous global trade in
conventional weapons, and for the first time it links sales to the human
rights records of the buyers.

The N.R.A. has argued that the
\href{https://www.nraila.org/issues/internationalun-gun-control-issues/}{treaty's
provisions amount to an infringement of the Second Amendment}. Kelly
Ayotte, a Republican senator from New Hampshire who is among those being
talked about as Mr. Trump's potential ambassador to United Nations, has
been an outspoken critic of the treaty.

\hypertarget{israel-and-palestinians}{%
\subsection{Israel and Palestinians}\label{israel-and-palestinians}}

Mr. Trump's precise views are unclear on this question, which has vexed
the United Nations for decades. But he has said he is
\href{http://www1.cbn.com/thebrodyfile/archive/2016/01/19/exclusive-donald-trump-tells-the-brody-file-hes-100-in}{``100
percent''} in favor of moving the United States Embassy from Tel Aviv to
Jerusalem, the holy city that Israel claims as its undivided capital.
Such a change would be a major break from longstanding American policy
and would infuriate the Palestinians, who want at least part of
Jerusalem to be the capital of a future Palestinian state. The secretary
general has called Jerusalem
\href{http://www.un.org/press/en/2016/gapal1362.doc.htm}{``the heart of
any negotiated solution to the question of Palestine.''}

\hypertarget{relations-with-cuba}{%
\subsection{Relations With Cuba}\label{relations-with-cuba}}

Mr. Trump has sharply criticized President Obama's reconciliation with
Cuba, a change that was widely welcomed at the United Nations, where an
annual resolution condemning the American trade embargo with Cuba passes
overwhelmingly. Mr. Trump's running mate, Mike Pence, has vowed that the
new administration's policy will be much tougher.

Advertisement

\protect\hyperlink{after-bottom}{Continue reading the main story}

\hypertarget{site-index}{%
\subsection{Site Index}\label{site-index}}

\hypertarget{site-information-navigation}{%
\subsection{Site Information
Navigation}\label{site-information-navigation}}

\begin{itemize}
\tightlist
\item
  \href{https://help.nytimes.com/hc/en-us/articles/115014792127-Copyright-notice}{©~2020~The
  New York Times Company}
\end{itemize}

\begin{itemize}
\tightlist
\item
  \href{https://www.nytco.com/}{NYTCo}
\item
  \href{https://help.nytimes.com/hc/en-us/articles/115015385887-Contact-Us}{Contact
  Us}
\item
  \href{https://www.nytco.com/careers/}{Work with us}
\item
  \href{https://nytmediakit.com/}{Advertise}
\item
  \href{http://www.tbrandstudio.com/}{T Brand Studio}
\item
  \href{https://www.nytimes.com/privacy/cookie-policy\#how-do-i-manage-trackers}{Your
  Ad Choices}
\item
  \href{https://www.nytimes.com/privacy}{Privacy}
\item
  \href{https://help.nytimes.com/hc/en-us/articles/115014893428-Terms-of-service}{Terms
  of Service}
\item
  \href{https://help.nytimes.com/hc/en-us/articles/115014893968-Terms-of-sale}{Terms
  of Sale}
\item
  \href{https://spiderbites.nytimes.com}{Site Map}
\item
  \href{https://help.nytimes.com/hc/en-us}{Help}
\item
  \href{https://www.nytimes.com/subscription?campaignId=37WXW}{Subscriptions}
\end{itemize}
