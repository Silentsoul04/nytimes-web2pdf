Sections

SEARCH

\protect\hyperlink{site-content}{Skip to
content}\protect\hyperlink{site-index}{Skip to site index}

\href{https://myaccount.nytimes.com/auth/login?response_type=cookie\&client_id=vi}{}

\href{https://www.nytimes.com/section/todayspaper}{Today's Paper}

\href{/section/opinion}{Opinion}\textbar{}Mexico Doesn't Have to Appease
Trump. It Can Fight Back.

\url{https://nyti.ms/2ghzyQk}

\begin{itemize}
\item
\item
\item
\item
\item
\end{itemize}

Advertisement

\protect\hyperlink{after-top}{Continue reading the main story}

Supported by

\protect\hyperlink{after-sponsor}{Continue reading the main story}

\href{/section/opinion}{Opinion}

Op-Ed Contributor

\hypertarget{mexico-doesnt-have-to-appease-trump-it-can-fight-back}{%
\section{Mexico Doesn't Have to Appease Trump. It Can Fight
Back.}\label{mexico-doesnt-have-to-appease-trump-it-can-fight-back}}

By Jorge G. Castañeda

\begin{itemize}
\item
  Nov. 22, 2016
\item
  \begin{itemize}
  \item
  \item
  \item
  \item
  \item
  \end{itemize}
\end{itemize}

\includegraphics{https://static01.nyt.com/images/2016/11/23/arts/23castaneda-inyt/23castaneda-inyt-articleLarge.jpg?quality=75\&auto=webp\&disable=upscale}

This year, for the first time since Ronald Reagan assailed the Soviet
Union in 1980, an American presidential candidate actively campaigned
against another country's national interests. By threatening to deport
all undocumented immigrants, about half of whom are Mexican; to build a
wall on the Mexican border; and to rip up the North American Free Trade
Agreement, which is far more important for Mexico than for the United
States, Donald J. Trump made Mexico one of the central issues of the
campaign.

How should Mexicans respond now that Mr. Trump has been elected?

President Enrique Peña Nieto has opted for a nonconfrontational
approach. Since his
\href{http://www.nytimes.com/2016/09/02/opinion/why-did-pena-nieto-invite-trump-to-mexico.html}{embarrassing
invitation to and welcome of} Mr. Trump in August, he has repeatedly
tried to accommodate Mr. Trump's demands. He has accepted reopening
discussion of Nafta and he has limited debate about ``the wall'' to who
will pay for it --- not whether it should be built. Mr. Peña Nieto has
said he will help the Mexicans whom Mr. Trump says his administration
will deport, but he has not taken a firm stand against the deportations
themselves.

Mexico doesn't have to appease Mr. Trump like this. It can fight back.
It will not win every battle, but it may achieve more through
obstruction, and making life miserable for the new president by
increasing the cost of his anti-Mexican policies, than it will achieve
by appeasement.

On Nafta, Mexico should simply tell Washington that it refuses to
renegotiate the treaty. There may be reasons to create side agreements
to supplement the treaty and to address issues like currency devaluation
or wages. But the idea of renegotiating Nafta, as Mr. Trump says he
intends to do, should be completely unacceptable to the government of
Mexico.

If the Trump administration, in return, threatens to leave Nafta, so be
it. Mr. Trump would be responsible for breaking up a deal that was
maintained by three American presidents, five Mexican ones and six
Canadian prime ministers over the past 22 years. And, despite some
flaws, it has worked reasonably well.

The blame for withdrawing from the treaty would be his, and many
American commercial interests and political forces, including numerous
Republicans, would come to resent Mr. Trump for it. The damage to
Mexico's economy would undoubtedly be great. But a prolonged
renegotiation of Nafta could potentially do even more damage, with years
of uncertainty discouraging investment in the country.

\includegraphics{https://static01.nyt.com/images/2016/11/23/opinion/23castaneda01-inyt/23castaneda01-inyt-articleLarge.jpg?quality=75\&auto=webp\&disable=upscale}

Regarding deportations, Mexico can legally maintain that it will welcome
back only those people who the United States can prove are indeed
Mexican. This would have to take place while they were still in the
United States.

Since many unauthorized Mexican immigrants have no documents, this would
shift the political and economic cost of deportation from Mexico to its
northern neighbor. There would be backlogs, litigation and crowded
detention centers. The news media would broadcast scenes of children
separated from their parents who are stuck in legal limbo.

This could amount to a humanitarian catastrophe, something no one wants
to see. But the comparison cannot be with the status quo; rather, it
should be with the millions of deportations promised by Mr. Trump. His
supporters might not care, but many other Americans would. The outcry
could conceivably force him to abandon detestable attempts at mass
deportation.

And what about the wall that was so central to Mr. Trump's campaign? It
is absurd for Mexico to say it doesn't care as long as it doesn't pay
for it. The Mexican government should fully oppose its construction.
Building a border wall is a hostile act. It would send a terrible
message to the world. The cost and danger of crossing without papers
would rise, making smuggling even more lucrative for organized crime
syndicates.

Once Mexico announces that it opposes the wall, the government should
resort to every legal, environmental, political, social, cultural and
regional tool to halt construction. It should mobilize binational
communities in Arizona, California, New Mexico and Texas against the
construction of the wall, until the price of pursuing this nonsensical
idea becomes too high for Mr. Trump. These binational cities, like
Ciudad Juárez-El Paso, should hold demonstrations and file lawsuits to
try to ensure that a hostile American-built wall does not divide them.

Finally, Mexico should take advantage of California's decision to
legalize recreational marijuana. Regardless of Mr. Trump's victory, the
approval of the proposition in the United States' most populous state
makes Mexico's war on drugs ridiculous. What is the purpose of sending
Mexican soldiers to burn fields, search trucks and look for
narco-tunnels if, once our marijuana makes it into California, it can be
sold at the local 7-Eleven?

But with Mr. Trump's aggression against Mexico, there is an additional
reason for the country to adopt a pragmatic ``wink and nod'' attitude on
marijuana exports to the United States: The Mexican government has no
reason to cooperate with a hostile administration in Washington. Our
authorities should instead simply look the other way when it comes to
marijuana.

None of these positions will be risk-free for Mexico. There could be
American reprisals, a backlash in some regions, and humanitarian crises.
A weak and unpopular Mexican government might not resist the Trump
administration's pressure. But if business as usual is not an option,
these suggestions may be. Leaders on both sides of the border should
contemplate them.

Advertisement

\protect\hyperlink{after-bottom}{Continue reading the main story}

\hypertarget{site-index}{%
\subsection{Site Index}\label{site-index}}

\hypertarget{site-information-navigation}{%
\subsection{Site Information
Navigation}\label{site-information-navigation}}

\begin{itemize}
\tightlist
\item
  \href{https://help.nytimes.com/hc/en-us/articles/115014792127-Copyright-notice}{©~2020~The
  New York Times Company}
\end{itemize}

\begin{itemize}
\tightlist
\item
  \href{https://www.nytco.com/}{NYTCo}
\item
  \href{https://help.nytimes.com/hc/en-us/articles/115015385887-Contact-Us}{Contact
  Us}
\item
  \href{https://www.nytco.com/careers/}{Work with us}
\item
  \href{https://nytmediakit.com/}{Advertise}
\item
  \href{http://www.tbrandstudio.com/}{T Brand Studio}
\item
  \href{https://www.nytimes.com/privacy/cookie-policy\#how-do-i-manage-trackers}{Your
  Ad Choices}
\item
  \href{https://www.nytimes.com/privacy}{Privacy}
\item
  \href{https://help.nytimes.com/hc/en-us/articles/115014893428-Terms-of-service}{Terms
  of Service}
\item
  \href{https://help.nytimes.com/hc/en-us/articles/115014893968-Terms-of-sale}{Terms
  of Sale}
\item
  \href{https://spiderbites.nytimes.com}{Site Map}
\item
  \href{https://help.nytimes.com/hc/en-us}{Help}
\item
  \href{https://www.nytimes.com/subscription?campaignId=37WXW}{Subscriptions}
\end{itemize}
