Sections

SEARCH

\protect\hyperlink{site-content}{Skip to
content}\protect\hyperlink{site-index}{Skip to site index}

\href{https://www.nytimes.com/section/us}{U.S.}

\href{https://myaccount.nytimes.com/auth/login?response_type=cookie\&client_id=vi}{}

\href{https://www.nytimes.com/section/todayspaper}{Today's Paper}

\href{/section/us}{U.S.}\textbar{}Nikki Haley's Path: From Daughter of
Immigrants to Trump's Pick for U.N.

\url{https://nyti.ms/2ggQjKi}

\begin{itemize}
\item
\item
\item
\item
\item
\end{itemize}

Advertisement

\protect\hyperlink{after-top}{Continue reading the main story}

Supported by

\protect\hyperlink{after-sponsor}{Continue reading the main story}

\hypertarget{nikki-haleys-path-from-daughter-of-immigrants-to-trumps-pick-for-un}{%
\section{Nikki Haley's Path: From Daughter of Immigrants to Trump's Pick
for
U.N.}\label{nikki-haleys-path-from-daughter-of-immigrants-to-trumps-pick-for-un}}

\includegraphics{https://static01.nyt.com/images/2016/11/24/us/24haleyprofile1/24haleyprofile1-articleInline.jpg?quality=75\&auto=webp\&disable=upscale}

By \href{https://www.nytimes.com/by/richard-fausset}{Richard Fausset}
and \href{http://www.nytimes.com/by/somini-sengupta}{Somini Sengupta}

\begin{itemize}
\item
  Nov. 23, 2016
\item
  \begin{itemize}
  \item
  \item
  \item
  \item
  \item
  \end{itemize}
\end{itemize}

ATLANTA --- Gov. Nikki R. Haley of South Carolina is the daughter of
immigrants, favors free markets and global trade, and earned
international attention for speaking out against the Confederate battle
flag in the aftermath of the 2015 massacre at a black church in
Charleston. During Donald J. Trump's presidential campaign, she sharply
criticized his demeanor and warned what it might mean for American
diplomacy --- even
\href{https://www.washingtonpost.com/opinions/nikki-haley-takes-on-donald-trump/2015/09/02/0ba0dbb0-51b2-11e5-8c19-0b6825aa4a3a_story.html?utm_term=.9c47bf4b5403}{suggesting}
that his tendency to lash out at critics could cause a world war.

But on Wednesday, Mr. Trump named Ms. Haley as his choice for ambassador
to the United Nations, a move that will probably serve to both assuage
and confound the president-elect's critics, raising questions about the
tone and direction of his foreign policy. As an Indian-American woman,
she would also add ethnic and gender diversity to the appointments, so
far predominantly of white men, he has made to other top posts in the
administration.

In a
\href{http://www.jaspersuntimes.com/news/2016-11-23/gov-haleys-statement-being-named-ambassador-united-nations}{statement},
Ms. Haley said she had accepted Mr. Trump's offer because she felt good
about South Carolina's economic standing. She added that this month's
elections had brought ``exciting changes to America.''

``When the president believes you have a major contribution to make to
the welfare of our nation, and to our nation's standing in the world,
that is a calling that is important to heed,'' the statement said.

Little is known about how Ms.
\href{http://www.nytimes.com/2016/11/23/us/politics/nikki-haley-donald-trump-un-ambassador.html}{Haley}
views America's role in the world. But an equally important mystery is
what her clout might be in the Trump administration.

\includegraphics{https://static01.nyt.com/images/2016/11/24/us/24haleyprofile2/24haleyprofile2-articleLarge.jpg?quality=75\&auto=webp\&disable=upscale}

Has Mr. Trump placed her in a post he considers marginal? Or will Ms.
Haley --- along with a still-to-be-named secretary of state --- be able
to temper the more radical views of Mr. Trump's other aides?

Despite the unknowns, many diplomats, scholars and rights advocates
\href{http://www.nytimes.com/2016/11/20/world/americas/united-nations-trump-climate-change-iran-cuba.html}{who
have been anxiously awaiting Mr. Trump's choices} were relieved at the
announcement. They saw in Ms. Haley, a daughter of Indian immigrants,
someone unafraid to express her beliefs even if they differ from Mr.
Trump's.

``If confirmed, we hope she will raise that voice on behalf of the
world's most vulnerable people who suffer from hunger, violence and
injustice around the world,'' the advocacy group Oxfam said in a
\href{https://www.oxfamamerica.org/press/oxfam-reaction-to-the-appointment-of-governor-nikki-haley-as-the-us-ambassador-to-the-un/}{statement}.

Many diplomats said they saw Ms. Haley as something of an enigma, whose
views on the world have not yet been publicized.

But her stance is public on a few issues --- including the
Israeli-Palestinian conflict, refugees, and reproductive rights --- and
they offer a window into how she may carry out her role.

She signed state legislation to thwart a pro-Palestinian disinvestment
campaign against Israel, known as Boycott, Divest, and Sanction, or
B.D.S. --- which made Israel one of the first to welcome her nomination
for the post.

Image

Ms. Haley in January 2015 being sworn into her second term as governor
by Chief Justice Jean H. Toal, while her husband, Michael, center, and
their son, Naline, watched.Credit...Richard Shiro/Associated Press

Ms. Haley has expressed concern about the security checks in place for
Syrian refugees resettled in her state, but she is not among those
Republican governors who have sued the Obama administration to block
resettlement.

She describes herself as ``pro-life'' and has supported legislation in
her state to restrict abortion rights. That position raises questions
about whether the United States would reimpose a funding ban on groups
that promote family planning overseas, and to what extent the United
States would undermine \href{http://indicators.report/targets/5-6/}{a
key United Nations goal} to advance sexual and reproductive rights.

The reaction from Republicans in South Carolina's congressional
delegation on Wednesday demonstrated the broad appeal Ms. Haley has
earned among conservatives statewide and in Washington. Senator Lindsey
Graham, one of Mr. Trump's harshest critics in the past, praised the
nomination,
\href{https://twitter.com/LindseyGrahamSC/status/801429937509040128}{writing
on Twitter} that Ms. Haley would be a ``strong voice for UN reform and
stand for American interests throughout the world.''

But others tempered their admiration for Ms. Haley with concern about
whether her public service credentials, which are limited to South
Carolina government, would translate to the world stage.

``My very practical reaction is that she'd be the least experienced U.N.
ambassador in the history of the country,'' said Bakari Sellers, a CNN
commentator and a Democrat who befriended Ms. Haley when they served
together in the State House of Representatives. ``You go from Samantha
Power'' --- the current United Nations ambassador --- ``who was very
well versed in foreign policy and our geopolitical relationships, to
Nikki Haley, who hasn't been in that depth ever.''

Ms. Haley's admirers note that she repeatedly traveled abroad as
governor to promote the state as a desirable place for investment. Her
highest-profile trip, perhaps, was a 2014 visit to India, the birthplace
of her parents. Her husband, Michael, has served in Afghanistan as an
officer in the South Carolina National Guard.

Image

Ms. Haley embracing her husband, Capt. Michael Haley, in 2013 after he
returned from serving in Afghanistan with the South Carolina National
Guard.Credit...Rainier Ehrhardt/Associated Press

More generally, Ms. Haley has overcome concerns that she would be a
one-dimensional insurgent outsider, similar to worries that dog Mr.
Trump. Her 2010 campaign was given a major lift by an endorsement from
former Gov. Sarah Palin of Alaska, the polarizing darling of the Tea
Party movement. But Ms. Haley has forged a middle path that embraces the
conciliatory racial attitudes favored by the left and the
business-friendly ethos of the right.

This balancing act faced perhaps its greatest test in June 2015, after
nine African-Americans were shot and killed at the historically black
Emanuel African Methodist Episcopal Church in Charleston. The white
supremacist charged in the massacre, Dylann S. Roof, had posed with the
Confederate battle flag in pictures. And for years, blacks and liberals
in South Carolina had pleaded with the conservatives who dominate the
state government to take the flag down from a prominent spot it occupied
in front of the State Capitol.

Ms. Haley, the first ethnic minority and first woman to be elected as
the state's governor, had previously sided with fellow Republicans, who
argued that the flag was not a racist symbol.

But the Charleston massacre hit home
\href{http://www.nytimes.com/2015/06/24/us/politics/south-carolina-governor-nikki-r-haley-points-to-personal-reasons-not-politics-for-shift-on-confederate-flag.html}{personally}.
She had been a friend of State Senator Clementa C. Pinckney, a Democrat
and pastor of the church, who was one of the dead. Ms. Haley had a
change of heart.

``I couldn't look my son or daughter in the face and justify that flag
flying anymore,'' Ms. Haley
\href{https://www.nytimes.com/2015/06/24/us/politics/south-carolina-governor-nikki-r-haley-points-to-personal-reasons-not-politics-for-shift-on-confederate-flag.html}{told
The New York Times} in June 2015.

At her urging, and after much passionate debate, the State Legislature
agreed to remove the flag.

Ms. Haley was born in the small city of Bamberg, S.C., to immigrants
from Punjab State in India. She has said the locals in South Carolina
were often unsure of her place in what is often a Southern binary of
black versus white. When Ms. Haley was about 5, she and her sister
entered a Little Miss Bamberg pageant where, traditionally, a black
queen and a white queen were crowned.

Image

Mitt Romney, the Republican presidential candidate, with Ms. Haley
during a town hall meeting in 2011 in Greenville, S.C.Credit...John
Adkisson for The New York Times

The judges decided the sisters fit neither category, so they were
disqualified.

From a young age, Ms. Haley worked for her family's clothing business,
and she eventually received an accounting degree from Clemson
University. She was elected to the State House in 2004.

In 2009, she declared she was running for governor and prevailed despite
lingering biases. A Republican state senator at the time called her a
``raghead'' on a radio show. A Democratic state representative said that
voters did not consider her a minority, but more of a ``nice
conservative with a tan.''

Ms. Haley, a nimble campaigner who is equally at home among C.E.O.s and
denim-clad bikers, easily won re-election in 2014, arguing that her
maintenance of South Carolina's anti-union, low-regulation atmosphere
had been the key to an economic comeback. Under her leadership, the
state continued looking beyond its boundaries, very often abroad, to
attract new industries to replace a fading textile industry.

It is a record that might seem at odds with Mr. Trump's skepticism of
global trade deals and his promise to subject imports from Mexico and
China to steep tariffs.

But there are other areas of agreement. In 2014, Ms. Haley criticized
federal plans that would force power plants to cut carbon emissions.

Also like Mr. Trump, Ms. Haley has weathered accusations of sexual
impropriety without suffering at the polls. In her first run for
governor, two Republican operatives made separate and unproven
accusations that they had sexual encounters with her. She strongly
denied the assertions.

A few years later, Ms. Haley was among those rumored as a potential
running mate for Mitt Romney, then a Republican presidential candidate.

She said she could not do it because there was too much to be done at
home in South Carolina. But she acknowledged that opportunities
sometimes arose. ``I've never been a planner,'' she said. ``I don't know
what's next, and I love not thinking about it because the doors open at
a certain time.''

Advertisement

\protect\hyperlink{after-bottom}{Continue reading the main story}

\hypertarget{site-index}{%
\subsection{Site Index}\label{site-index}}

\hypertarget{site-information-navigation}{%
\subsection{Site Information
Navigation}\label{site-information-navigation}}

\begin{itemize}
\tightlist
\item
  \href{https://help.nytimes.com/hc/en-us/articles/115014792127-Copyright-notice}{©~2020~The
  New York Times Company}
\end{itemize}

\begin{itemize}
\tightlist
\item
  \href{https://www.nytco.com/}{NYTCo}
\item
  \href{https://help.nytimes.com/hc/en-us/articles/115015385887-Contact-Us}{Contact
  Us}
\item
  \href{https://www.nytco.com/careers/}{Work with us}
\item
  \href{https://nytmediakit.com/}{Advertise}
\item
  \href{http://www.tbrandstudio.com/}{T Brand Studio}
\item
  \href{https://www.nytimes.com/privacy/cookie-policy\#how-do-i-manage-trackers}{Your
  Ad Choices}
\item
  \href{https://www.nytimes.com/privacy}{Privacy}
\item
  \href{https://help.nytimes.com/hc/en-us/articles/115014893428-Terms-of-service}{Terms
  of Service}
\item
  \href{https://help.nytimes.com/hc/en-us/articles/115014893968-Terms-of-sale}{Terms
  of Sale}
\item
  \href{https://spiderbites.nytimes.com}{Site Map}
\item
  \href{https://help.nytimes.com/hc/en-us}{Help}
\item
  \href{https://www.nytimes.com/subscription?campaignId=37WXW}{Subscriptions}
\end{itemize}
