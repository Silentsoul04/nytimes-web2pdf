Sections

SEARCH

\protect\hyperlink{site-content}{Skip to
content}\protect\hyperlink{site-index}{Skip to site index}

\href{https://www.nytimes.com/section/politics}{Politics}

\href{https://myaccount.nytimes.com/auth/login?response_type=cookie\&client_id=vi}{}

\href{https://www.nytimes.com/section/todayspaper}{Today's Paper}

\href{/section/politics}{Politics}\textbar{}Potential Conflicts Around
the Globe for Trump, the Businessman President

\url{https://nyti.ms/2gsCW9H}

\begin{itemize}
\item
\item
\item
\item
\item
\item
\end{itemize}

Advertisement

\protect\hyperlink{after-top}{Continue reading the main story}

Supported by

\protect\hyperlink{after-sponsor}{Continue reading the main story}

\hypertarget{potential-conflicts-around-the-globe-for-trump-the-businessman-president}{%
\section{Potential Conflicts Around the Globe for Trump, the Businessman
President}\label{potential-conflicts-around-the-globe-for-trump-the-businessman-president}}

\includegraphics{https://static01.nyt.com/images/2016/11/27/world/27TRUMPWORLD-FULLBLEED/27TRUMPWORLD-FULLBLEED-articleLarge.jpg?quality=75\&auto=webp\&disable=upscale}

By \href{https://www.nytimes.com/by/richard-c-paddock}{Richard C.
Paddock}, \href{http://www.nytimes.com/by/eric-lipton}{Eric Lipton},
\href{https://www.nytimes.com/by/ellen-barry}{Ellen Barry},
\href{http://www.nytimes.com/by/rod-nordland}{Rod Nordland},
\href{http://www.nytimes.com/by/danny-hakim}{Danny Hakim} and
\href{http://www.nytimes.com/by/simon-romero}{Simon Romero}

\begin{itemize}
\item
  Nov. 26, 2016
\item
  \begin{itemize}
  \item
  \item
  \item
  \item
  \item
  \item
  \end{itemize}
\end{itemize}

MANILA --- On Thanksgiving Day, a Philippine developer
named\href{http://www.nytimes.com/2016/11/10/world/asia/donald-trump-philippines-jose-antonio.html?_r=0}{Jose
E. B. Antonio} hosted a company anniversary bash at one of Manila's
poshest hotels. He had much to be thankful for.

In October, he had quietly been named a special envoy to the United
States by the Philippine president, Rodrigo Duterte. Mr. Antonio was
nearly finished building a \$150 million tower in Manila's financial
district --- a 57-story symbol of affluence and capitalism, which
bluntly promotes itself with the slogan
\href{https://www.facebook.com/pg/trumptowercenturycity/photos/?tab=album\&album_id=1317830394933885}{``Live
Above the Rest.''} And now his partner on the project, Donald J. Trump,
had just been elected president of the United States.

After the election, Mr. Antonio flew to New York for a private meeting
at Trump Tower with the president-elect's children, who have been
involved in the Manila project from the beginning, as have Mr. Antonio's
children. The Trumps and Antonios have other ventures in the works,
including Trump-branded resorts in the Philippines, Mr. Antonio's son
Robbie Antonio said.

``We will continue to give you products that you can enjoy and be proud
of,'' the elder Mr. Antonio, one of
the\href{http://www.forbes.com/profile/jose-antonio/}{richest men in the
Philippines}, told the 500 friends, employees and customers gathered for
his star-studded celebration in Manila.

Mr. Antonio's combination of jobs --- he is a business partner with Mr.
Trump, while also representing the Philippines in its relationship with
the United States and the president-elect --- is hardly inconsequential,
given some of the weighty issues on the diplomatic table.

Among them, Mr. Duterte
\href{http://www.nytimes.com/2016/10/21/world/asia/rodrigo-duterte-philippines-china-xi-jinping.html}{has
urged ``a separation'' from the United States}
and\href{http://www.nytimes.com/2016/10/27/world/asia/philippines-president-rodrigo-duterte-japan.html}{has
called for American troops to exit the country} in two years' time. His
antidrug crusade has resulted
in\href{http://www.nytimes.com/2016/10/29/world/asia/philippines-duterte-mayor-police.html}{the
summary killings of thousands of suspected criminals without trial},
prompting criticism from the Obama administration.

Situations like these are already leading some former government
officials from both parties to ask if America's reaction to events
around the world could potentially be shaded, if only slightly, by the
Trump family's financial ties with foreign players. They worry, too,
that in some countries those connections could compromise American
efforts to criticize the corrupt intermingling of state power with vast
business enterprises controlled by the political elite.

``It is uncharted territory, really in the history of the republic, as
we have never had a president with such an empire both in the United
States and overseas,'' said
\href{https://www.csis.org/people/michael-j-green}{Michael J. Green},
who served on the National Security Council in the administration of
George W. Bush, and before that at the Defense Department.

The globe is dotted with such potential conflicts. Mr. Trump's companies
have business operations in at least 20 countries, with a particular
focus on the developing world, including outposts in nations like India,
Indonesia and Uruguay, according to a New York Times analysis of his
presidential campaign financial disclosures. What's more, the true
extent of Mr. Trump's global financial entanglements is unclear, since
he has refused to release his tax returns and has not made public a list
of his lenders.

In
an\href{http://www.nytimes.com/2016/11/23/us/politics/trump-new-york-times-interview-transcript.html}{interview
with The Times} on Tuesday, Mr. Trump boasted again about the global
reach of his business --- and his family's ability to keep it running
after he takes office.

``I've built a very great company and it's a big company and it's all
over the world,'' Mr. Trump said, adding later: ``I don't care about my
company. It doesn't matter. My kids run it.''

In a written statement, his spokeswoman, Hope Hicks, said Mr. Trump and
his family were committed to addressing any issues related to his
financial holdings.

``Vetting of various structures and immediate transfer of the business
remains a top priority for both President-elect Trump, his adult
children and his executives,'' she said.

But a review by The Times of these business dealings identified a menu
of the kinds of complications that could create a running source of
controversy for Mr. Trump, as well as tensions between his priorities as
president and the needs and objectives of his companies.

In Brazil, for example, the beachfront
\href{https://www.trumphotels.com/rio-de-janeiro}{Trump Hotel Rio de
Janeiro} --- one of Mr. Trump's many branding deals, in which he does
not have an equity stake --- is part of a broad investigation by a
federal prosecutor who is examining whether illicit commissions and
bribes resulted in apparent favoritism by two pension funds that
invested in the project.

Several of Mr. Trump's real estate
\href{http://www.trump.com/real-estate-portfolio/india/trump-towers-pune/}{ventures
in India} --- where he has more projects underway than in any location
outside North America --- are being built through companies with family
ties to India's most important political party. This makes it more
likely that Indian government officials will do special favors
benefiting Mr. Trump's projects, including pressuring state-owned banks
to extend favorable loans.

In \href{http://www.trumpgolfireland.com/}{Ireland} and
\href{http://www.trumpgolfscotland.com/}{Scotland}, executives from Mr.
Trump's golf courses have been waging two separate battles with local
officials. The most recent centers on the Trump Organization's plans to
build a flood-prevention sea wall at the course on the Irish coast. Some
environmentalists say the wall could destroy an endangered snail's
habitat --- a dispute that will soon involve the president of the United
States.

And in Turkey, officials including President Recep Tayyip Erdogan, a
religiously conservative Muslim, demanded that Mr. Trump's name be
removed from
\href{http://www.trump.com/real-estate-portfolio/istanbul/trump-towers/}{Trump
Towers in Istanbul} after he called for a ban on Muslims entering the
United States. More recently, after Mr. Trump came to the defense of Mr.
Erdogan --- suggesting that he had the right to crack down harshly on
dissidents after a failed coup --- the calls for action against Trump
Towers have stopped, fueling worries that Mr. Trump's policies toward
Turkey might be shaped by his commercial interests.

Mr. Trump has acknowledged a conflict of interest in Turkey. ``I have a
little conflict of interest because I have a major, major building in
Istanbul,'' he said during a radio
\href{http://www.breitbart.com/2016-presidential-race/2015/12/01/trump-blasts-obama-warning-world-war-iii/}{interview
last year with Stephen K. Bannon, the Breitbart News executive who has
since been designated his chief White House strategist.}``It's a
tremendously successful job. It's called Trump Towers --- two towers,
instead of one. Not the usual one. It's two.''

These tangled ties already have some members of Congress --- including
at least one Republican representative --- calling on Mr. Trump to
provide more information on his international operations, or
\href{http://www.warren.senate.gov/files/documents/2016-11-23-GAO_Letter_On_Trump_Transition.pdf}{perhaps
for a congressional inquiry into them.}

``You rightly criticized Hillary for Clinton Foundation,''
Representative Justin Amash, Republican of Michigan,
\href{https://twitter.com/justinamash/status/800914868039655425}{said in
a Twitter message} on Monday. ``If you have contracts w/foreign govts,
it's certainly a big deal,
too.\href{https://twitter.com/hashtag/DrainTheSwamp?src=hash}{\#DrainTheSwamp''}

\href{https://www.mccaininstitute.org/staff/david-j-kramer/}{David J.
Kramer}, who served as assistant secretary of state for democracy, human
rights and labor during the Bush administration, said Mr. Trump's
financial entanglements could undermine decades of efforts by Democratic
and Republican presidents to promote government transparency --- and to
use the Foreign Corrupt Practices Act to stop contractors from paying
bribes to secure government work abroad.

``This will make it a little harder to be able to go out and proselytize
around these things,'' Mr. Kramer said.

Even if Mr. Trump and his family seek no special advantages from foreign
governments, officials overseas may feel compelled to help the Trump
family by, say, accelerating building permits or pushing more business
to one of the new president's hotels or golf courses, according to
several former State Department officials.

``The working assumption on behalf of all these foreign government
officials will be that there is an advantage to doing business with the
Trump organization,'' said
\href{http://www.state.gov/r/pa/ei/biog/bureau/204807.htm}{Michael H.
Fuchs}, who was until recently deputy assistant secretary at the bureau
of East Asian and Pacific affairs. ``They will think it will ingratiate
themselves with the Trump administration. And this will significantly
complicate United States foreign policy and our relationships around the
world.''

At the same time, Mr. Fuchs said, American diplomats in countries where
Mr. Trump's companies operate, fearful of a rebuke from Washington, may
be reluctant to take steps that could frustrate business partners or
political allies.

Another question is, who will be responsible for security at the Trump
Towers around the world, especially in the Middle East, which terrorism
experts say may now become more appealing targets as symbols of American
capitalism built in the name of the president?

What is clear is that there has been very little division, in the weeks
since the election, between Mr. Trump's business interests and his
transition effort, with the president-elect or his family greeting real
estate partners from India and the Philippines in his office and Mr.
Trump raising concerns about his golf course in Scotland with a
prominent British politician. Mr. Trump's
daughter\href{http://www.trump.com/the-next-generation/ivanka-trump/}{Ivanka,
who is in charge of planning}and development of the Trump Organization's
global network of hotels, has joined in conversations with at least
three world leaders --- of Turkey, Argentina and Japan --- having access
that could help her expand the brand worldwide.

Mr. Trump, in the interview with The Times on Tuesday, acknowledged that
his move to the Oval Office could help enrich his family. He cited his
new hotel a few blocks from the White House, which the Trump
Organization has urged diplomats to consider patronizing when in town to
meet the president or his team.

Federal law does not prevent Mr. Trump from taking actions that could
benefit him and his family financially; the president is exempt from
most conflict-of-interest laws. But the Constitution, through what is
called the
\href{http://www.nytimes.com/2016/11/21/us/politics/donald-trump-conflict-of-interest.html}{emoluments
clause}, appears to prohibit him from taking payments or gifts from a
foreign government entity, a standard that some legal experts say he may
violate by renting space in Trump Tower in New York to the Bank of China
or if he hosts foreign diplomats in one of his hotels.

``I mean it could be that occupancy at that hotel will be because,
psychologically, occupancy at that hotel will be probably a more
valuable asset now than it was before, O.K.? The brand is certainly a
hotter brand than it was before. I can't help that, but I don't care,''
Mr. Trump said, adding, ``The only thing that matters to me is running
our country.''

\href{http://belfercenter.ksg.harvard.edu/experts/4/robert_d_blackwill.html}{Robert
D. Blackwill,} a former National Security Council member who also served
as ambassador to India during the Bush administration, said Mr. Trump
still had a chance to demonstrate that he could manage these challenges
once he was sworn in.

``Let's listen and not prejudge,'' said Mr. Blackwill, a Republican who
was so critical of Mr. Trump that he endorsed Hillary Clinton. ``I want
to see what he does as president.''

\includegraphics{https://static01.nyt.com/images/2016/11/27/world/27TRUMPWORLD-BRAZIL-1/27TRUMPWORLD-BRAZIL-1-articleLarge.jpg?quality=75\&auto=webp\&disable=upscale}

\hypertarget{brazil}{%
\subsection{BRAZIL}\label{brazil}}

\textbf{Nation Under Pressure, Ventures Under Scrutiny}

Donald Trump Jr., the president-elect's oldest son, gushed with
triumphalism when he announced a deal in 2014 to attach the family name
to the Trump Hotel Rio de Janeiro, a lavish 171-room beachfront
\href{https://www.facebook.com/DonaldJTrumpJr/photos/a.319839844707450.84238.295644160460352/744544555570308/?type=1\&theater}{project
featuring}cavernous suites with private plunge pools and a
4,000-square-foot nightclub.

``This is an exciting time to develop our first project in South America
and the perfect location to do so,'' the younger Mr. Trump (his brother
Eric is also involved in the family business)
\href{http://www.prnewswire.com/news-releases/trump-hotel-collection-announces-trump-hotel-rio-de-janeiro-242363971.html}{said
at the time.}

But just two years later, the venture is embroiled in a
\href{http://www.mpf.mp.br/df/sala-de-imprensa/docs/despacho-trump.pdf}{criminal
investigation} in Brazil, pointing to unfulfilled promises that are
casting a pall over both the Trump business empire and the
president-elect in their dealings in Latin America's largest country.

Anselmo Henrique Cordeiro Lopes, a crusading federal prosecutor in the
capital, Brasília, opened an
\href{http://www.mpf.mp.br/df/sala-de-imprensa/docs/despacho-trump.pdf}{investigation}
in the weeks before the American election into \$40 million in
investments made by two relatively small Brazilian pension funds in the
Trump Hotel Rio.

The Trump hotel inquiry is looking at why the funds --- Serpro, which
invests on behalf of retirees of a state-controlled information
technology firm, and Igeprev, which manages the pensions of public
employees of the sparsely populated Tocantins State --- put so much of
their capital into the venture, which is owned by Mr. Trump's Brazilian
partner, LSH Barra.

Back in 2014, the hotel might have seemed like a good deal. Brazil was
about to host the World Cup soccer tournament that year, while Rio was
preparing to be the venue for the 2016 Summer Olympics. At the same
time, Rio, the nerve center of Brazil's energy industry, had been
bolstered by large offshore oil discoveries.

But Brazil's economy began to weaken in 2014, undermined by falling
commodities prices, colossal graft scandals and political instability
that culminated in the ouster this year of President Dilma Rousseff, who
was replaced by her vice president, Michel Temer. The result: Brazil is
still grappling with its most severe economic crisis in decades.

The hotel officially opened for the Olympics, but months later remains
unfinished. The top floors of the property, whose design evokes a
futuristic pyramid, are closed. Parts of the hotel still resemble a
construction site, including the second floor, where pleasure-seekers
were supposed to mingle in a nightclub overlooking the Atlantic.

The examination of the project by Mr. Lopes, the federal prosecutor, has
already found a series of ``highly suspicious'' potential irregularities
warranting a criminal investigation, according to court documents. ``It
is necessary to verify if the favoritism shown by the pension funds to
LSH and the Trump Organization was due to the payment of illicit
commissions and bribes,'' Mr. Lopes said in documents filed in October.

In his filings, Mr. Lopes said the size of the hotel investments
relative to the overall holdings of the small pension funds reflected a
highly unusual level of risk, especially for an unfinished venture that
failed to capitalize fully on the demand for accommodations during the
Olympics. Going further, Mr. Lopes positioned the inquiry within a
broader investigation of public pension funds, pillars of the Brazilian
economy that often work in tandem with large state-controlled banks and
energy companies.

Mr. Trump first took interest in a Rio hotel venture in 2012, when
Ivanka Trump was having lunch in Florida with Paulo Figueiredo Filho, a
businessman who is
a\href{http://www.nytimes.com/2015/08/21/world/americas/donald-trump-hotel-rio-immigration.html}{grandson}
of João Figueiredo, the last autocrat of Brazil's 21-year military
dictatorship, which ended in 1985. The younger Mr. Figueiredo
spearheaded the hotel venture until recently.

In a statement, Mr. Trump's Brazilian partner, LSH, said it was innocent
of any wrongdoing in connection with the investments by the pension
funds, and was cooperating with the criminal inquiry.

Alan Garten, the Trump Organization's general counsel, said in a
statement issued Friday that the investigation was not targeting Mr.
Trump or his company --- given that it does not own the hotel --- and
``has no knowledge whatsoever regarding any governmental inquiry.''

The investigation of the Trump projects is unfolding at an awkward time
for the Brazilian authorities. Foreign Minister José Serra, Brazil's top
diplomat,
publicly\href{http://www.correiobraziliense.com.br/app/noticia/politica/2016/07/31/internas_polbraeco,542408/a-era-petista-foi-uma-era-de-retrocesso-diz-jose-serra.shtml}{declared}
in July that a Trump presidency would be a ``nightmare.'' Although
President Temer has formally congratulated Mr. Trump on his victory in a
letter, he is still among world leaders who have not yet spoken by
telephone with the president-elect.

Even if Brazil's executive branch actively tries to seek warmer
relations with Mr. Trump, officials will face obstacles if they try to
quell the investigation. Brazil differs from some other countries in
Latin America where presidents can easily exert pressure on prosecutors
and judges, with the judiciary steadily growing more independent.

``Brazilian diplomats could try to avoid the problem of referring to the
investigation when dealing with the Trump administration, but that's
about all they can do,'' said Maurício Santoro, a political scientist at
the State University of Rio de Janeiro. ``This is something that could
hang over relations between the two countries for years.''

Image

Trump Tower Mumbai is one of five Trump Organization projects in India,
a country where connections between developers and officials are
common.Credit...Asmita Parelkar for The New York Times

\hypertarget{india}{%
\subsection{INDIA}\label{india}}

\textbf{Potential Pitfalls in Dual Roles}

On the other side of the world, Donald Trump Jr. had other projects he
was pushing.

In 2012, he flew into Mumbai for a brief meeting with the state's chief
minister at that time, hoping to salvage a residential tower
representing the Trump Organization's first planned project there. He
was hoping the chief minister, Prithviraj Chavan, would intervene on his
behalf to get the permission needed.

The participants recall the meeting differently: Mr. Trump's partner,
Harresh Mehta of Rohan Lifescapes, said development regulations had
changed, leaving the project in limbo, and they hoped Mr. Chavan could
formalize a policy so that the project could continue. Mr. Chavan said
that in a 30-minute meeting, Mr. Trump and his partner were ``requesting
a concession that could not be given.''

By the end of the meeting, in any case, it was fairly clear that the
younger Mr. Trump's presence had not worked any magic. The project was
shelved soon after.

``He thought the name was so big, we would bend backwards to satisfy
him, but that was not the case,'' Mr. Chavan said.

Kalpesh Mehta, managing partner of
\href{http://tribecadevelopers.com/trump/tribeca_trump.php}{Tribeca
Developers of Mumbai}, the Trump Organization's development partner in
India, confirmed that Donald Trump Jr. had met with the chief minister,
but disputed the claim by Mr. Chavan that he had sought a special favor.

``The notion that a request was made by Donald Jr. to waive any
regulations is absolutely false,'' Mr. Mehta said in the statement,
which was issued Friday. ``The Trump Organization does not get involved
in the regulatory aspects and/or interacting with government officials
related to its projects in India.''

This example, analysts here say, points to a potentially serious ethical
hazard for a United States president who is also a real estate mogul in
India, with five projects underway. Mr. Trump was operating much like
other developers in India, who cozy up to politicians --- officially or
unofficially --- to push projects through the bureaucracy.

Often, they must obtain as many as 60 permissions and building permits
from government officials, including bureaucrats ``whose main goal in
life is to attract rent,'' said Saurabh Mukherjea, the chief executive
of institutional equities at Ambit Capital, a leading investment bank in
India.

One of Mr. Trump's projects,
\href{http://www.trump.com/real-estate-portfolio/india/trump-towers-pune/}{Trump
Towers Pune}, is in fact under
\href{https://www.documentcloud.org/documents/3225252-2016-2-24-Complaint-Against-Trump-Tower-Pune.html}{investigation}
by local authorities after another builder alleged that one of its
permits was fraudulent. Panchshil Realty has disputed that accusation,
saying the permit in question was not required for the construction. The
very nature of the country's real estate business, however, underscores
larger concerns about potential damage to American efforts to discourage
corruption in business abroad.

In India, real estate is the main vehicle politicians and businessmen
have used to invest so-called black money, on which taxes have not been
paid. In cities, where land is scarce and extraordinarily valuable,
special favors from top political leaders can lead to windfall profits,
and negotiations between developers and officials are informal affairs.

It is so routine for developers to pay bribes at every step of the
approval process that many bureaucrats have informal rate sheets showing
exactly how much must be paid to each official.

Politicians not only pressure the bureaucracy to approve their pet
projects, sometimes even when they are against local regulations, they
also squeeze government banks to give out favorable loans.

Top officials might ``think in some way the U.S. president will help
them,'' and ``can put in a friendly word with the banks'' to extend
loans for around 8 percent interest, rather than the characteristic 15
percent, said Vikas S. Kasliwal, the chief executive officer and vice
chairman of Shree Ram Urban Infrastructure.

``If the son goes himself, if the son is willing to go and meet the
prime minister of India, or the urban development minister, that is a
very big thing,'' he said. ``They will think the president is meeting
them.''

Image

Donald J. Trump's real estate partners from India met with him in New
York on Nov. 15 as the presidential transition was underway.

Another pitfall is that Donald Trump's partners in major projects are,
in some cases, politicians themselves. Most major Indian developers have
some sort of alignment, direct or indirect, with regional political
leaders, who can assist in acquiring the necessary permits.

Mr. Trump's first projects in India, which are expected to increase in
number over the next year, follow this pattern: His partner for Trump
Towers Pune is \href{http://www.panchshil.com/}{Panchshil Realty,} owned
by a family that has a close and longstanding family relationship with
one of the state's most powerful politicians, Sharad Pawar, the head of
the small but influential Nationalist Congress Party. (Mr. Trump was
photographed --- in an image distributed on Twitter but since taken down
--- with executives from Panchshil Realty on Nov. 15.) Mr. Pawar's
daughter, Supriya Sule, a member of Parliament, holds a 2 percent share
in Panchshil's parent company, she said in an interview.

Mr. Trump's partner in the Trump Tower Mumbai is the Lodha Group,
founded by Mangal Prabhat Lodha, vice president of the Bharatiya Janata
Party --- currently the governing party in Parliament --- in Maharashtra
State. The Lodha Group has already negotiated with the United States
government; it announced a landmark purchase of a property, known as the
Washington House, on tony Altamount Road, from the American government
for 3.75 billion rupees, almost \$70 million.

His partner in an office complex in Gurgaon, near New Delhi, is IREO,
whose managing director, Lalit Goyal, is the brother-in-law of a
Bharatiya Janata member of Parliament, Sudhanshu Mittal. Mr. Mittal, in
an interview, has denied having any connection with the real estate
company.

\href{https://www.facebook.com/pg/AICC-Secretary-SURAJ-HEGDE-687397901292152/about/}{Suraj
Hegde,} the secretary of the All India Congress Committee, a national
body of Indian National Congress party members, said he was troubled by
the dual roles Mr. Trump and his family would play in Indian affairs ---
particularly given real estate's important role in India's fast-growing
economy, and the clout the United States has on the world stage.

``Basically this is the globalization of lobbying across countries,
which then tries to establish monopoly over real estate,'' Mr. Hegde
said in an interview. He added that he was already calling for an
independent parliamentary investigation of such maneuvers, including Mr.
Trump's real estate ventures in India.

``Establishing monopoly at the cost of small players by business
connections to Mr. Trump is very worrisome,'' he said. ``This is not at
all healthy for a democracy.''

Image

The Trump Towers Istanbul. One contains offices, the other luxury
apartments, with a shopping mall connecting them.Credit...Monique Jaques
for The New York Times

\hypertarget{turkey}{%
\subsection{TURKEY}\label{turkey}}

\textbf{Mixing Business, Politics and Islam}

Mr. Trump's business interests in Turkey are emblematic of two weighty
contradictions for a businessman turned politician.

As a candidate, Mr. Trump railed against moving American jobs overseas
and promised to do something about it. As a businessman, he invested in
a partnership with a furniture company here, making luxury furniture in
the firm's factory in western Anatolia and selling it in the United
States and worldwide --- a partnership that apparently remains active.

Mr. Trump the candidate inveighed against Muslims and threatened at
least a temporary ban on their entering the United States. Mr. Trump the
businessman has in recent years had some of his biggest expansions
overseas, including in Muslim countries like Turkey, the United Arab
Emirates and even Azerbaijan.

One of the most visible Muslim-world symbols of that contradiction is in
the bustling commercial district of Sisli, on the European side of
Istanbul, where a pair of
\href{http://www.trump.com/real-estate-portfolio/istanbul/trump-towers/}{cantilevered
modernist towers}, nearly 40 stories high, bear Mr. Trump's name.

Turkey's leader, Mr. Erdogan, visited Trump Towers Istanbul --- one
holds luxury apartments and one office space, with a shopping mall
connecting the two --- after their completion in 2012, with Mr. Trump
and Ivanka Trump appearing as part of the celebration the next day.

``We look forward to this being the first of many world-class
developments undertaken together in Istanbul and throughout Turkey,''
\href{http://www.prnewswire.com/news-releases/donald-j-trump-and-ivanka-trump-visit-istanbul-to-celebrate-the-opening-of-highly-anticipated-trump-towers-mall-150816665.html}{Mr.
Trump said} in a statement issued during the visit.

Beyond real estate, there is the Trump Organization's
\href{http://www.trumphomecollection.com/pages/trump-home-press-releases\#6}{2013
partnership} with Dorya International, a luxury furniture maker with a
factory in Manisa Province, near the city of Izmir,
\href{http://www.trumphomecollection.com/pages/trump-home-press-releases}{to
build pieces} sold under the
\href{http://www.trumphomecollection.com/}{Trump Home Collection.}

But the presidential campaign demonstrated how the goals of his business
and politics ventures can come into direct conflict, particularly once
Mr. Trump in December proposed barring Muslims from entering the United
States, implying that all Muslims might pose a terrorist threat.

``We regret and condemn Trump's discriminatory remarks,'' Bulent Kural,
the manager of the Trump Towers Mall, wrote in an email to a reporter at
the time, as he announced that the mall was considering removing Mr.
Trump's name. ``Such statements bear no value and are products of a mind
that does not understand Islam, a peace religion, at all. Our reaction
has been directly expressed to the Trump family. We are reviewing the
legal dimension of our relation with the Trump brand.''

Mr. Erdogan weighed in on the issue, too,
\href{http://www.yenisafak.com/gundem/erdogandan-trump-towers-cagrisi-2486503}{saying,}
``The ones who put that brand on their building should immediately
remove it.''

Mr. Trump's next move helped re-establish his standing. After a failed
coup in Turkey in July, he defended Mr. Erdogan's crackdown on
dissidents, saying in an interview with The Times that the United States
has to ``fix our own mess'' before trying to alter the behavior of other
nations.

``I don't think we have a right to lecture,'' Mr. Trump
\href{http://www.nytimes.com/2016/07/22/us/politics/donald-trump-foreign-policy-interview.html}{said
in the interview.}``Look at what is happening in our country,'' he
added, referring to violence in the United States. ``How are we going to
lecture when people are shooting policemen in cold blood?''

In between his two remarks --- one infuriating the president of Turkey,
the other comforting him --- the calls for the renaming of the Trump
Towers Mall ended. But much more is at stake in relations between the
United States and Turkey than a shopping mall and two skyscrapers.

Image

Mehmet Ali Yalcindag, Mr. Trump's business associate at Trump Towers
Istanbul, with Ivanka Trump and Mr. Trump.Credit...Trump Organization,
via PR Newswire

Turkey is a
\href{http://www.nytimes.com/2014/10/03/world/europe/turkey-votes-to-allow-operations-against-isis.html}{key
player in United States efforts} to combat the Islamic State in the
Middle East, and sits next door to Syria as the United States has armed
rebel groups
\href{http://www.nytimes.com/2016/09/18/world/middleeast/his-position-still-secure-bashar-al-assad-smiles-as-syria-burns.html}{in
an attempt} to remove Syria's president, Bashar al-Assad, from power.

The recent postelection telephone call between Mr. Trump and Mr. Erdogan
suggests that business and political roles will continue to be mixed.

According to a Turkish journalist, Amberin Zaman, writing in the
independent online news outlet Diken, Mr. Trump told the Turkish leader
that he and his daughter --- who participated in the call --- admired
both Mr. Erdogan and Mehmet Ali Yalcindag, Mr. Trump's business
associate in the towers, whom he called ``a close friend.''

Ms. Zaman, a fellow at the Woodrow Wilson Center in Washington, said no
government officials had disputed her account of the conversation. ``I'm
of the opinion they were quite happy for this to be published,'' she
said. A spokeswoman for Mr. Trump declined to comment about the call.

\href{http://www.cfr.org/experts/china-ukraine-middle-east-and-north-africa/jennifer-m-harris/b18802}{Jennifer
Harris}, who served on the staff of the National Intelligence Council
and on the State Department's policy planning staff, said the twin hats
that Mr. Trump and his family would be wearing in Turkey would almost
certainly complicate the jobs of American diplomats there.

``It makes me wonder if the Trump administration will use the power of
the state to help political or business allies and hurt political
adversaries and business rivals,'' she said.

Image

Jose E.B. Antonio, Mr. Trump's partner on a tower project in Manila,
second from left, with his sons at a company celebration there on
Thursday.Credit...Hannah Reyes Morales for The New York Times

\hypertarget{the-philippines}{%
\subsection{THE PHILIPPINES}\label{the-philippines}}

\textbf{What Stance Toward Duterte?}

President Duterte's antidrug campaign has led to the summary deaths of
thousands of suspected criminals at the hands of police and vigilantes
since he took office June 30. The killing has been condemned by human
rights activists --- and the Obama administration.

In August, Elizabeth Trudeau, a State Department spokeswoman, said the
United States was ``very deeply concerned'' about reports of
``extrajudicial killings by or at the behest of government authorities
of individuals who are suspected to have been in drug activity in the
Philippines.''
\href{http://www.state.gov/r/pa/prs/dpb/2016/08/261275.htm}{She added},
``We have also made our concerns known.''

The question now, former State Department officials say, is just what
kind of a stand the Trump administration will take as Mr. Trump and his
family balance their personal and financial ties with foreign policy
demands.

Mr. Antonio first met Mr. Trump casually in the 1990s and has been his
business partner in the Philippines for five years.
\href{http://www.nytimes.com/2016/11/10/world/asia/donald-trump-philippines-jose-antonio.html?_r=0}{President
Duterte named him} special envoy to the United States as the Philippines
angrily pushed back at President Obama for criticizing his deadly
campaign. At the time of the appointment, Mrs. Clinton was leading in
the polls in the United States presidential election.

Mr. Duterte has made clear that he does not appreciate American meddling
in his country's domestic affairs.

``I am a president of a sovereign state, and we have long ceased to be a
colony,'' Mr. Duterte told reporters in early September, before a
scheduled meeting in Laos with Mr. Obama that never took place. ``I do
not have any master except the Filipino people, nobody but nobody.''

Mr. Duterte handpicked Mr. Antonio as his intermediary with the United
States, said his press secretary, Ernesto Abella, because of his
business success, his previous experience as a special envoy to China
and the Philippine president's ``deep intuition about people.'' The
appointment will be advantageous for the Philippines, Mr. Abella added,
because Mr. Trump already knows Mr. Antonio.

Even before Mr. Trump has been sworn in, Mr. Antonio flew to New York
and visited Trump Tower, where he met with Mr. Trump's children, who are
executives at the Trump Organization --- which oversees the
president-elect's real estate ventures. This was a business trip, not a
diplomatic one, Robbie Antonio, Mr. Antonio's son and the managing
director of the family business, said in an interview.

Image

Donald Trump Jr., left, and Eric Trump, right, with the elder Mr.
Antonio, spoke with reporters as work on the Manila project got underway
in 2012.Credit...Pat Roque/Associated Press

The two families are considering new ventures as they finish work on the
\href{http://www.trumptowerphilippines.com/condo.html}{Trump Tower} in
Makati City, a financial center within metropolitan Manila that is one
of the country's wealthiest enclaves and home to many of the nation's
elite.

The \$150 million tower --- one of the tallest in the Philippines --- is
on the gritty side of Makati about two blocks from Manila's most
notorious red-light district, where it is common to see prostitutes
soliciting business and people sleeping on sidewalks. Completion,
originally scheduled for this year, is now expected in 2017. About 240
of the 260 units have been sold, said Kristina Garcia, the director for
investor relations.

``We are bringing Trump to the Philippines because we believe that Trump
exemplifies the best quality of real estate anywhere in the world,'' Mr.
Antonio said in \href{https://www.youtube.com/watch?v=aKgI4C4EfZg}{a
2011 video promoting the project} --- in which Mr. Antonio is identified
as ``ambassador'' and Mr. Trump also appears. ``It also exemplifies
luxury and it exemplifies exclusivity.''

In the interview at the celebration in Manila on Thursday evening,
Robbie Antonio said he had little doubt of his father's priorities: He
will put the Philippines' interests above those of his company. ``It is
for the good of the country now,'' he said.

But Mr. Fuchs, who helped oversee United States relations with the
Philippines as the deputy assistant secretary of state
\href{https://www.americanprogress.org/press/release/2016/01/20/129095/release-former-deputy-assistant-secretary-of-state-michael-fuchs-joins-cap-as-senior-fellow/}{until
early this year}, said he was deeply troubled by Mr. Trump's overlapping
priorities, particularly given the long list of globally significant
issues in play with the Philippines. These include planned joint
military exercises in the South China Sea, the fight against
\href{http://www.nytimes.com/2016/08/31/world/asia/philippines-abu-sayyaf-jolo.html}{militant
Islamic groups} based in the country's southern islands, and the human
rights abuses taking place.

``What we already have is a blurring of the lines between official and
business activities,'' Mr. Fuchs said. ``The biggest gray area may not
be a President Trump himself advocating for favors for the Trump
Organization. It's the diplomats and career officers who will feel the
need to perhaps not do things that will harm the Trump Organization's
interests. It is seriously disturbing.''

Image

Doughmore Bay as seen from the Trump International Golf Links on the
west coast of Ireland.Credit...Paulo Nunes dos Santos for The New York
Times

\hypertarget{ireland-and-scotland}{%
\subsection{IRELAND and SCOTLAND}\label{ireland-and-scotland}}

\textbf{Over a Tiny Snail, Big Concerns}

The vertigo angustior snail is only two millimeters long. But it punches
above its weight.

The endangered little snail has helped stall Mr. Trump's plans to build
a sea wall to protect the coastline along his Trump International Golf
Links course on the west coast of Ireland, in County Clare.

Environmentalists, as well as surfers, list a host of concerns about the
proposed wall, particularly its potential impact on sand dunes. Along
with the snails, a patch of the dunes near the course is protected by
European Union rules. But Mr. Trump's organization has said the golf
resort development might be dead in the water without the sea wall, and
many locals welcome the business and the jobs it brings.

The battle is likely to be decided next year in front of a national
planning board, in the weeks or months after Mr. Trump is inaugurated on
Jan. 20, several people said.

The planning board was overhauled in the 1980s to insulate it from
political meddling, and it now has the confidence of environmentalists.
But there is little precedent for the Trump situation, which could
involve a public hearing.

``They can be long, they can be lively, and a lot of things could be
aired,'' said Sean O'Leary, the executive director of the Irish Planning
Institute, which represents the majority of the country's professional
planners.

He noted that the national planning board had considered a development
proposed by a politician before, but that was a holiday home that the
Irish president wanted to build.

``The scale is slightly different,'' he said.

Local officials have said the Trump Organization needs to resubmit its
application by the end of the year. In a statement, the Trump
Organization said it was ``considering all potential coastal protection
options at present'' and would be in contact with the local authority
before Christmas. The snail, the statement said, ``is thriving on the
site.''

``Its only material threat is that presented by coastal erosion,'' it
added.

Certainly, Mr. Trump's golf courses in Scotland and Ireland have
remained at the fore in the president-elect's mind, even in recent days.
Shortly after his election, he
\href{http://www.nytimes.com/2016/11/21/business/with-a-meeting-trump-renewed-a-british-wind-farm-fight.html}{urged
a group of ``Brexit'' campaigners}led by Nigel Farage, the head of the
U.K. Independence Party, to fight against wind farms in Britain. Wind
farms have been a favorite target of Mr. Trump's in both Britain and
Ireland, where he has railed against proposed installations as a
potential blight on the views from his resorts.

After a spokeswoman for Mr. Trump initially denied that the matter had
been raised with Mr. Farage's group, Mr. Trump conceded during his
interview with The Times this past week that
``\href{https://twitter.com/maggieNYT/status/801130374566998016}{I might
have brought it up}.''

Tony Lowes, an activist who runs a group called Friends of the Irish
Environment, said Mr. Trump had once called him because Mr. Lowes's
group also happened to oppose a proposed wind farm near Mr. Trump's
Irish course on environmental grounds.

``He certainly hates wind farms, that's for sure,'' Mr. Lowes said about
the call.

His group decided against working with Mr. Trump, and is now a leading
opponent of his planned sea wall.

``The dune system will not be able to develop naturally,'' Mr. Lowes
said. ``It will be starved of the sand it needs to develop and evolve
and it will die.'' He added, ``The whole system there is alive and
mobile and moving, and the wall is intended to stop that.''

Mr. Trump's representatives have advanced a number of rationales for the
sea wall, with the most straightforward being that they simply want to
buffer the land from a continuing erosion problem. The proposal has
previously\href{http://www.politico.com/story/2016/05/donald-trump-climate-change-golf-course-223436}{attracted
attention} because an environmental-impact statement submitted by Mr.
Trump's team highlighted the risks of climate change and its influence
on ``coastal erosion rates.'' That was a noteworthy claim, since Mr.
Trump has called global warming a hoax perpetrated
``\href{https://twitter.com/realdonaldtrump/status/265895292191248385?lang=en-gb}{by
and for the Chinese}.''

The Irish government has zealously courted Mr. Trump. When he visited
the course in 2014, he was greeted on the airport tarmac in Shannon with
a red carpet, a harpist, a violinist and a singer whose voice cut
through the runway clamor.

\href{http://www.irishtimes.com/sport/golf/doonbeg-s-good-news-hardly-worth-the-bowing-and-scraping-1.1792515}{Malachy
Clerkin} of The Irish Times called it ``a preposterous welcome'' and
``the worst kind of forelock-tugging.''

Many locals, however, support Mr. Trump's development. Hugh McNally, the
owner of\href{http://morrisseysdoonbeg.ie/}{Morrissey's Bar} in Doonbeg
Village, about two miles from the course, said the issue had been
``sensationalized by the media'' because of the Trump connection.

``I'll give you an example,'' he said. Roche, the Swiss pharmaceutical
company, announced last year that it
would\href{http://www.irishtimes.com/business/health-pharma/roche-close-irish-plant-with-240-job-losses-1.2427527}{close
a plant} in nearby Clarecastle, causing the loss of more than 200 jobs.
``If someone told them you'd save those jobs by building any wall,
everyone would do it,'' he said. ``The only reason people are objecting
here is because of Trump.''

Image

A red-carpet welcome for Donald J. Trump, on a visit to his Ireland golf
course in 2014.Credit...Niall Carson/PA, via Associated Press

\hypertarget{the-world}{%
\subsection{THE WORLD}\label{the-world}}

\textbf{A Transition and a Business Plan}

Mr. Trump's family appears to have been preparing for the transition to
the Oval Office and ways to capitalize on it both in the United States
and around the globe.

In April, even before Mr. Trump had secured the Republican nomination,
his business
\href{https://trademarks.justia.com/869/70/american-86970999.html}{moved
to trademark the name American Idea} for use in branding hotels, spas
and concierge services, according to the United States Patent and
Trademark Office. It was one of more than two dozen
\href{https://www.documentcloud.org/documents/3225269-Trump-Trademark-Applications-Merged.html}{trademark
applications} that Mr. Trump and members of his family filed in the
United States and around the world while he was running for president.

The applications offer a glimpse of where the Trumps may intend to focus
their business endeavors. Last month, representatives of the Trump
Organization in Indonesia, where Mr. Trump has been pursuing two hotel
deals, filed trademark registrations for use of the Trump name in
connection with hotel management. Similar filings have been made in
Mexico, Canada and the European Union.

Ivanka Trump has filed
\href{https://trademarks.justia.com/owners/ivanka-trump-marks-llc-1365461/}{at
least 25 trademark registrations} for her brand of clothing, cosmetics
and jewelry in the United States, Canada, the European Union and Mexico
since the beginning of the year, most recently in October. Mr. Trump's
wife, Melania, filed an American trademark application for a line of
jewelry in August.

As he prepares for the presidency, Mr. Trump has made at least one
concession so far, he said in the interview with The Times this past
week.

``In theory, I can be president of the United States and run my business
100 percent, sign checks on my business,'' Mr. Trump said, before later
adding, ``but I am phasing that out now, and handing that to Eric Trump
and Don Trump and Ivanka Trump for the most part, and some of my
executives, so that's happening right now.''

Advertisement

\protect\hyperlink{after-bottom}{Continue reading the main story}

\hypertarget{site-index}{%
\subsection{Site Index}\label{site-index}}

\hypertarget{site-information-navigation}{%
\subsection{Site Information
Navigation}\label{site-information-navigation}}

\begin{itemize}
\tightlist
\item
  \href{https://help.nytimes.com/hc/en-us/articles/115014792127-Copyright-notice}{©~2020~The
  New York Times Company}
\end{itemize}

\begin{itemize}
\tightlist
\item
  \href{https://www.nytco.com/}{NYTCo}
\item
  \href{https://help.nytimes.com/hc/en-us/articles/115015385887-Contact-Us}{Contact
  Us}
\item
  \href{https://www.nytco.com/careers/}{Work with us}
\item
  \href{https://nytmediakit.com/}{Advertise}
\item
  \href{http://www.tbrandstudio.com/}{T Brand Studio}
\item
  \href{https://www.nytimes.com/privacy/cookie-policy\#how-do-i-manage-trackers}{Your
  Ad Choices}
\item
  \href{https://www.nytimes.com/privacy}{Privacy}
\item
  \href{https://help.nytimes.com/hc/en-us/articles/115014893428-Terms-of-service}{Terms
  of Service}
\item
  \href{https://help.nytimes.com/hc/en-us/articles/115014893968-Terms-of-sale}{Terms
  of Sale}
\item
  \href{https://spiderbites.nytimes.com}{Site Map}
\item
  \href{https://help.nytimes.com/hc/en-us}{Help}
\item
  \href{https://www.nytimes.com/subscription?campaignId=37WXW}{Subscriptions}
\end{itemize}
