Sections

SEARCH

\protect\hyperlink{site-content}{Skip to
content}\protect\hyperlink{site-index}{Skip to site index}

\href{https://www.nytimes.com/section/world/asia}{Asia Pacific}

\href{https://myaccount.nytimes.com/auth/login?response_type=cookie\&client_id=vi}{}

\href{https://www.nytimes.com/section/todayspaper}{Today's Paper}

\href{/section/world/asia}{Asia Pacific}\textbar{}South Korea's
Impeachment Process, Explained

\url{https://nyti.ms/2g8bYXW}

\begin{itemize}
\item
\item
\item
\item
\item
\end{itemize}

Advertisement

\protect\hyperlink{after-top}{Continue reading the main story}

Supported by

\protect\hyperlink{after-sponsor}{Continue reading the main story}

\hypertarget{south-koreas-impeachment-process-explained}{%
\section{South Korea's Impeachment Process,
Explained}\label{south-koreas-impeachment-process-explained}}

\includegraphics{https://static01.nyt.com/images/2016/12/10/world/28KOREAQA-2/28KOREAQA-2-articleInline.jpg?quality=75\&auto=webp\&disable=upscale}

By \href{http://www.nytimes.com/by/choe-sang-hun}{Choe Sang-Hun}

\begin{itemize}
\item
  Nov. 27, 2016
\item
  \begin{itemize}
  \item
  \item
  \item
  \item
  \item
  \end{itemize}
\end{itemize}

SEOUL, South Korea --- South Korean lawmakers will vote on Friday on the
impeachment of President Park Geun-hye, who is accused of
\href{http://www.nytimes.com/2016/11/06/world/asia/south-koreans-ashamed-over-les-secretive-adviser.html}{helping
a friend commit extortion}. If she is forced from office, it will be a
first for South Korea, but the process is long and uncertain. Here's how
it works.

\textbf{What is Ms. Park accused of?}

Prosecutors say Ms. Park conspired with Choi Soon-sil,
\href{http://www.nytimes.com/2016/10/28/world/asia/south-korea-choi-soon-sil.html}{an
old friend}, to extort tens of millions of dollars from South Korean
businesses. Ms. Park cannot be indicted while in office, but she has
been identified as a criminal suspect, which had never happened to a
president before.

She has also been accused of helping Ms. Choi illegally gain access to
confidential government documents. Opposition parties say the combined
allegations are serious enough to warrant her removal from power; some
members of her own party agree, as do leading South Korean newspapers
and most of the public, according to polls. Huge
\href{http://www.nytimes.com/2016/11/26/world/asia/korea-park-geun-hye-protests.html}{protests
have been held in Seoul demanding that Ms. Park step down}, but she has
refused.

\textbf{What is required for impeachment?}

The 300-member National Assembly is expected to vote on an impeachment
bill on Friday, the last day of the current legislative session. If 200
members vote yes, the National Assembly will formally ask the
Constitutional Court to impeach her and remove her from office. To reach
200 votes, the opposition lawmakers will need at least 28 members of Ms.
Park's conservative party, Saenuri, to join them.

An impeachment motion must accuse an official of violating ``the
Constitution and the laws,'' but the National Assembly is not required
to prove those charges.

\textbf{What happens next?}

If the impeachment motion passes, Ms. Park will be suspended from
office, and the country's No. 2 official, Prime Minister Hwang Kyo-ahn,
will become acting president. The Constitutional Court will then have
180 days to rule on whether to impeach Ms. Park.

The court must decide whether she is guilty of the crimes that the
National Assembly claims she committed and whether they are serious
enough to merit impeachment.

If at least six members of the nine-judge court vote to impeach, Ms.
Park will be impeached and removed from office. South Korea will have 60
days to elect a successor, with Mr. Hwang carrying out her duties in the
meantime.

If fewer than six judges vote for impeachment, Ms. Park will immediately
be returned to office.

Six of the current judges were appointed by Ms. Park or her conservative
predecessor, or are otherwise seen as being close to her party. But
plenty of conservatives think Ms. Park should go.

There could be another complication: Two of the judges are set to retire
by March. If the court has not ruled by then, some legal scholars say,
those judges could not be replaced, because the president formally
appoints them and Ms. Park would still be suspended. That would improve
Ms. Park's odds, because six of the remaining seven judges would have to
vote to impeach her.

\textbf{Has a South Korean president ever faced impeachment before?}

Only once, in 2004, when President Roh Moo-hyun was accused of calling
on voters to support his party in parliamentary elections. The calls
were said to violate a law requiring the president to remain neutral in
the election.

The National Assembly voted for impeachment, but the decision enraged
many South Koreans, who
\href{http://www.nytimes.com/2004/03/13/world/president-s-impeachment-stirs-angry-protests-in-south-korea.html}{demonstrated
in large numbers} and gave Mr. Roh's party a landslide victory at the
polls. The Constitutional Court voted against impeachment, saying Mr.
Roh's breaches of the election law
\href{http://www.nytimes.com/2004/05/14/world/constitutional-court-reinstates-south-korea-s-impeached-president.html}{were
relatively minor}, and he was returned to office.

\textbf{Who is Choi Soon-sil?}

She is the daughter of a cult leader who befriended Ms. Park in the
1970s, when Ms. Park was a young woman and her father, Park Chung-hee,
was South Korea's dictator. Lurid rumors about Ms. Park's connection to
the Choi family have dogged her for years, and many have come to believe
that Ms. Choi wields a sinister, cultlike influence over the president.

\href{http://www.nytimes.com/2016/11/01/world/asia/south-korea-park-geun-hye-choi-soon-sil.html}{Ms.
Choi was arrested} and charged with extortion and fraud, and prosecutors
said they
\href{http://www.nytimes.com/2016/11/20/world/asia/park-geun-hye-south-korea-extortion-accomplice-prosecutors.html}{considered
Ms. Park an accomplice}.

Though Ms. Park cannot be indicted while in office, prosecutors can
pursue charges against her if she is removed from office or after her
term ends in February 2018. The Constitution limits presidents to one
term.

Advertisement

\protect\hyperlink{after-bottom}{Continue reading the main story}

\hypertarget{site-index}{%
\subsection{Site Index}\label{site-index}}

\hypertarget{site-information-navigation}{%
\subsection{Site Information
Navigation}\label{site-information-navigation}}

\begin{itemize}
\tightlist
\item
  \href{https://help.nytimes.com/hc/en-us/articles/115014792127-Copyright-notice}{©~2020~The
  New York Times Company}
\end{itemize}

\begin{itemize}
\tightlist
\item
  \href{https://www.nytco.com/}{NYTCo}
\item
  \href{https://help.nytimes.com/hc/en-us/articles/115015385887-Contact-Us}{Contact
  Us}
\item
  \href{https://www.nytco.com/careers/}{Work with us}
\item
  \href{https://nytmediakit.com/}{Advertise}
\item
  \href{http://www.tbrandstudio.com/}{T Brand Studio}
\item
  \href{https://www.nytimes.com/privacy/cookie-policy\#how-do-i-manage-trackers}{Your
  Ad Choices}
\item
  \href{https://www.nytimes.com/privacy}{Privacy}
\item
  \href{https://help.nytimes.com/hc/en-us/articles/115014893428-Terms-of-service}{Terms
  of Service}
\item
  \href{https://help.nytimes.com/hc/en-us/articles/115014893968-Terms-of-sale}{Terms
  of Sale}
\item
  \href{https://spiderbites.nytimes.com}{Site Map}
\item
  \href{https://help.nytimes.com/hc/en-us}{Help}
\item
  \href{https://www.nytimes.com/subscription?campaignId=37WXW}{Subscriptions}
\end{itemize}
