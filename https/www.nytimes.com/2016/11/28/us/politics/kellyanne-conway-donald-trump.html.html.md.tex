Sections

SEARCH

\protect\hyperlink{site-content}{Skip to
content}\protect\hyperlink{site-index}{Skip to site index}

\href{https://www.nytimes.com/section/politics}{Politics}

\href{https://myaccount.nytimes.com/auth/login?response_type=cookie\&client_id=vi}{}

\href{https://www.nytimes.com/section/todayspaper}{Today's Paper}

\href{/section/politics}{Politics}\textbar{}Challenging the Boss in
Public? For Kellyanne Conway, It's Part of the Job

\url{https://nyti.ms/2gC6DoP}

\begin{itemize}
\item
\item
\item
\item
\item
\end{itemize}

Advertisement

\protect\hyperlink{after-top}{Continue reading the main story}

Supported by

\protect\hyperlink{after-sponsor}{Continue reading the main story}

\hypertarget{challenging-the-boss-in-public-for-kellyanne-conway-its-part-of-the-job}{%
\section{Challenging the Boss in Public? For Kellyanne Conway, It's Part
of the
Job}\label{challenging-the-boss-in-public-for-kellyanne-conway-its-part-of-the-job}}

\includegraphics{https://static01.nyt.com/images/2016/11/29/us/29CONWAY/29CONWAY-articleLarge.jpg?quality=75\&auto=webp\&disable=upscale}

By \href{http://www.nytimes.com/by/michael-d-shear}{Michael D. Shear}
and \href{http://www.nytimes.com/by/maggie-haberman}{Maggie Haberman}

\begin{itemize}
\item
  Nov. 28, 2016
\item
  \begin{itemize}
  \item
  \item
  \item
  \item
  \item
  \end{itemize}
\end{itemize}

WASHINGTON --- Kellyanne Conway, one of President-elect Donald J.
Trump's senior advisers, was about to board a flight back to New York on
Monday morning when she caught a glimpse of the headline crawling across
television screens in the terminal.

``SOURCES: TRUMP `FURIOUS' OVER CONWAY COMMENTS ABOUT ROMNEY,'' screamed
the headline on MSNBC's ``Morning Joe'' program.

Ms. Conway quickly dialed Mr. Trump, as well as Jared Kushner, his
son-in-law and confidant, seeking reassurance that the headline was
wrong.

She got it.

Ms. Conway, the Republican pollster and strategist who managed Mr.
Trump's improbable campaign, said the president-elect was neither
surprised nor angered by her public excoriation a day earlier of former
Gov. Mitt Romney of Massachusetts, a top prospect for secretary of state
in the Trump administration.

``When he's upset with someone, they know it,'' Ms. Conway said in a
telephone interview late Monday afternoon. While her public display may
have bothered some members of Mr. Trump's transition team, by all
accounts, her close relationship with the next occupant of the Oval
Office remains secure.

Mr. Trump, in a statement emailed Monday evening by his spokeswoman,
Hope Hicks, said: ``Kellyanne came to me and asked whether or not she
could go public with her thoughts on the matter. I encouraged her to do
so. Most importantly she fully acknowledged there is only one person
that makes the decision. She has always been a tremendous asset and that
will continue.''

To those on the outside of the Trump transition, her remarks on Sunday
had all the hallmarks of a political staff member gone rogue. Amid
reports of intense closed-door deliberations over who should be
secretary of state, Ms. Conway had seemed intent on committing a
heretical political act by an aide: boxing in her boss. She wrote on
Twitter about a ``deluge'' of concerns from conservatives and appeared
repeatedly on television, insisting that a Romney appointment would be
seen by Mr. Trump's supporters as a ``betrayal.''

But little in Mr. Trump's universe is simple. In fact, people familiar
with the dynamic inside Trump Tower --- who were granted anonymity to
discuss the unusual process that Mr. Trump has allowed for his
transition --- said Ms. Conway had been neither insubordinate nor acting
directly on the president-elect's instruction.

By denouncing Mr. Romney even as Mr. Trump was preparing for their
second meeting, this time over dinner on Tuesday, Ms. Conway was simply
doing what she knows Mr. Trump likes: encouraging a public airing of
conflicting views when he is unsure of what path to take.

``The president-elect would never need to turn on a TV station to find
out how I feel about anything or anyone --- he would already know it,''
she said, noting that she had not said anything publicly that she had
not also shared with Mr. Trump privately. ``It would be a mistake to
think that I communicate with him through the TV.''

What some saw over the weekend as an act of political defiance by Ms.
Conway --- undermining a potential cabinet nominee --- was seen by Mr.
Trump as a demonstration of loyalty, according to people who had talked
to him. Her criticism of Mr. Romney articulated a view her boss had at
times expressed: that Mr. Romney had tried to ``hurt'' him during the
campaign and had yet to fully acknowledge it or apologize.

``There was the Never Trump movement, and then there was Gov. Mitt
Romney,'' she said on ABC, adding later, ``I only wish Governor Romney
had been as critical of Hillary Clinton'' during the general election.
During the primaries, Mr. Romney
\href{http://www.nytimes.com/2016/03/04/us/politics/mitt-romney-speech.html}{called
Mr. Trump a ``fraud'' and a ``phony.''}

A decision on the secretary of state position could come this week,
although Mr. Trump sent more mixed signals on Monday after
\href{http://www.nytimes.com/2016/11/28/us/politics/donald-trump-transition-david-petraeus.html}{interviewing
retired Gen. David H. Petraeus} for the post. ``Was very impressed!''
Mr. Trump posted on Twitter after the hourlong meeting.

But if it is still unclear who will lead the State Department, it is
virtually certain that Ms. Conway will remain close to Mr. Trump,
whether as an influential West Wing aide or, in a move that seems more
likely, as an outside adviser with guaranteed access to the president.
After initially working for a ``super PAC'' supporting Senator Ted
Cruz's presidential campaign, she has become one of the few calming
influences on Mr. Trump, and someone he sees as intensely loyal to him.

Her allies describe her job as ``the Kellyanne role,'' a position in
which the precise title does not completely capture the duties she is
performing or the sway she has. Ms. Conway has a direct line to Mr.
Trump, and she has said that he is supportive of seeing her on
television. The president-elect is also well aware that she is one of
his only female surrogates, and one who has become his ambassador to the
news media.

``She's the first person who was able to develop a strategy internally
and articulate that strategy externally,'' said Frank Luntz, a
Republican pollster who worked with Ms. Conway in the early 1990s. ``She
put a context to everything where there was no context before.''

Mr. Trump made clear throughout the campaign when he was unhappy with
those speaking for him on television. Some cable bookers have been
quietly told not to refer to someone as a ``surrogate'' for the campaign
on a given day if the person has fallen out of favor.

On a conference call with top supporters at one point, Mr. Trump
denounced some of his own aides and said they did not speak for him.

That is not a problem that Ms. Conway has encountered.

Ms. Conway said on Monday that she had spoken with Mr. Trump a number of
times over the weekend, on a range of topics. She had several
conversations with him, she said, about how to be ``a face of the
administration,'' although what that precisely means remains unclear.

And so far, her style of service on Mr. Trump's behalf has left many who
are used to a more traditional definition of spokeswoman confused about
what to expect.

Joe Scarborough, the MSNBC host whose show is closely watched by Mr.
Trump, accused Ms. Conway of trying to ``intimidate the
president-elect,'' adding that ``now all world leaders will be watching
to see if a President Trump can be bullied by his staff.''

Ms. Conway
\href{https://twitter.com/KellyannePolls/status/802963387168727040}{responded
to Mr. Scarborough} on Twitter by saying, ``Repeating 100th time
decision is his \& I'll respect it,'' and adding, ``I already have the
job I want.''

Advertisement

\protect\hyperlink{after-bottom}{Continue reading the main story}

\hypertarget{site-index}{%
\subsection{Site Index}\label{site-index}}

\hypertarget{site-information-navigation}{%
\subsection{Site Information
Navigation}\label{site-information-navigation}}

\begin{itemize}
\tightlist
\item
  \href{https://help.nytimes.com/hc/en-us/articles/115014792127-Copyright-notice}{©~2020~The
  New York Times Company}
\end{itemize}

\begin{itemize}
\tightlist
\item
  \href{https://www.nytco.com/}{NYTCo}
\item
  \href{https://help.nytimes.com/hc/en-us/articles/115015385887-Contact-Us}{Contact
  Us}
\item
  \href{https://www.nytco.com/careers/}{Work with us}
\item
  \href{https://nytmediakit.com/}{Advertise}
\item
  \href{http://www.tbrandstudio.com/}{T Brand Studio}
\item
  \href{https://www.nytimes.com/privacy/cookie-policy\#how-do-i-manage-trackers}{Your
  Ad Choices}
\item
  \href{https://www.nytimes.com/privacy}{Privacy}
\item
  \href{https://help.nytimes.com/hc/en-us/articles/115014893428-Terms-of-service}{Terms
  of Service}
\item
  \href{https://help.nytimes.com/hc/en-us/articles/115014893968-Terms-of-sale}{Terms
  of Sale}
\item
  \href{https://spiderbites.nytimes.com}{Site Map}
\item
  \href{https://help.nytimes.com/hc/en-us}{Help}
\item
  \href{https://www.nytimes.com/subscription?campaignId=37WXW}{Subscriptions}
\end{itemize}
