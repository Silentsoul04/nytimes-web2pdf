Sections

SEARCH

\protect\hyperlink{site-content}{Skip to
content}\protect\hyperlink{site-index}{Skip to site index}

\href{https://www.nytimes.com/section/business/economy}{Economy}

\href{https://myaccount.nytimes.com/auth/login?response_type=cookie\&client_id=vi}{}

\href{https://www.nytimes.com/section/todayspaper}{Today's Paper}

\href{/section/business/economy}{Economy}\textbar{}Trump to Announce
Carrier Plant Will Keep Jobs in U.S.

\url{https://nyti.ms/2ghz1jo}

\begin{itemize}
\item
\item
\item
\item
\item
\end{itemize}

Advertisement

\protect\hyperlink{after-top}{Continue reading the main story}

Supported by

\protect\hyperlink{after-sponsor}{Continue reading the main story}

\hypertarget{trump-to-announce-carrier-plant-will-keep-jobs-in-us}{%
\section{Trump to Announce Carrier Plant Will Keep Jobs in
U.S.}\label{trump-to-announce-carrier-plant-will-keep-jobs-in-us}}

\includegraphics{https://static01.nyt.com/images/2016/11/30/us/30CARRIER/30CARRIER-articleLarge.jpg?quality=75\&auto=webp\&disable=upscale}

By \href{http://www.nytimes.com/by/nelson-d-schwartz}{Nelson D.
Schwartz}

\begin{itemize}
\item
  Nov. 29, 2016
\item
  \begin{itemize}
  \item
  \item
  \item
  \item
  \item
  \end{itemize}
\end{itemize}

From the earliest days of his campaign, Donald J. Trump made keeping
manufacturing jobs in the United States his signature economic issue,
and the decision by Carrier, the big air-conditioner company, to move
over 2,000 of them from Indiana to Mexico was a tailor-made talking
point for him on the stump.

On Thursday, Mr. Trump and Mike Pence, Indiana's governor and the vice
president-elect, plan to appear at Carrier's Indianapolis factory to
announce a deal with the company to keep roughly 1,000 jobs in the
state, according to officials with the transition team as well as
Carrier.

Mr. Trump will be hard-pressed to alter the economic forces that have
hammered the Rust Belt for decades, but forcing Carrier and its parent
company, United Technologies, to reverse course is a powerful tactical
strike that will hearten his followers even before he takes office.

``I'm ready for him to come,'' said Robin Maynard, a 24-year veteran of
Carrier who builds high-efficiency furnaces and earns almost \$24 an
hour. ``Now I can put my daughter through college without having to look
for another job.''

It also signals that Mr. Trump is a different kind of Republican,
willing to take on big business, at least in individual cases.

And just as only a confirmed anti-Communist like Richard Nixon could go
to China, so only a businessman like Mr. Trump could take on corporate
America without being called a Bernie Sanders-style socialist. If Barack
Obama had tried the same maneuver, he'd probably have drawn criticism
for intervening in the free market.

In exchange for keeping the factory running in Indianapolis, Mr. Trump
and Mr. Pence are expected to reiterate their campaign pledges to be
friendlier to businesses by easing regulations and overhauling the
corporate tax code, according to a spokeswoman for Mr. Trump.

The state of Indiana also plans to give economic incentives to Carrier
as part of the deal to stay, according to local officials.

The message from Mr. Trump that captivated the Carrier workers ---
keeping manufacturing jobs in the United States after decades of losses
to overseas factories and automation --- resonated throughout the Rust
Belt. That promise, plus his opposition to pacts like the North American
Free Trade Agreement, were key reasons he was able to edge out Hillary
Clinton in states like Pennsylvania, Michigan and Wisconsin.

Political symbolism aside, saving 1,000 Carrier jobs doesn't loom so
large in an economy that's created an average of 181,000 jobs a month
this year, noted Jared Bernstein, a liberal economist who served as
adviser in the Obama administration from 2009 to 2011.

Still, he confessed a grudging admiration for Mr. Trump's political
jujitsu. ``If I weren't so scared of the damage a Trump administration
might do, I'd find it refreshing to see an administration fighting for
factory jobs like this,'' he said. ``That said, no one should confuse
what Trump is doing here with sustainable economic policy.''

\href{https://www.nytimes.com/interactive/2016/11/30/us/politics/trump-manufacturing-jobs-indiana-carrier.html}{}

\includegraphics{https://static01.nyt.com/images/2016/11/30/us/politics/trump-manufacturing-jobs-indiana-carrier-1480538094077/trump-manufacturing-jobs-indiana-carrier-1480538094077-articleLarge-v3.png}

\hypertarget{what-it-means-for-trump-to-save-1000-jobs-in-indiana}{%
\subsection{What It Means for Trump to Save 1,000 Jobs in
Indiana}\label{what-it-means-for-trump-to-save-1000-jobs-in-indiana}}

How the president-elect's deal measures up to U.S. manufacturing job
losses.

Over the long term, and for less prominent firms, the temptation to move
to cheaper locales for manufacturing will stay great, said Robert Reich,
a prominent liberal Democrat who served as secretary of labor in the
Clinton administration.

``Memories are short but the economic fundamentals remain the same,'' he
said. ``Wall Street is breathing down companies' necks to cut costs, and
the labor savings in Mexico is too great.''

Mr. Trump first announced he was talking to Carrier on Thanksgiving Day
via Twitter, which the company quickly confirmed. The discussions have
continued this week, and with a tentative deal in hand on Tuesday,
transition officials scheduled Mr. Trump's and Mr. Pence's visit to
Indianapolis.

``I didn't think it would be this quick,'' Mr. Maynard said.

While the standoff loomed large in the lives of its employees in
Indiana, for United Technologies the forgone savings is tiny ---
equivalent to about 2 cents per share in earnings.

``Every penny counts, but if we step back and I'm looking at earnings of
\$6.60 per share this year, 2 cents is an easy concession if the
president-elect listens to some of the company's bigger concerns,'' said
Howard Rubel, a senior equity analyst with Jefferies, an investment
banking firm in New York.

When Carrier announced in February that the two Indiana factories would
be closing, it did offer benefits to employees facing layoffs, including
paying for them to go back to school and retrain for other careers. Even
with that, however, once the layoffs were to begin in mid-2017, most of
the workers would have had a hard time finding jobs that paid anywhere
near the \$20 to \$25 an hour that veteran line workers earn.

Carrier is best known for its air-conditioners, but it also sells a
variety of other heating and cooling equipment for homes and businesses,
like the gas furnaces and fan coils for electric furnaces made at the
Indianapolis factory. The jobs in Indiana Mr. Trump has referred to are
in two separate sites --- the Carrier plant in Indianapolis, with 1,400
employees, and a United Technologies factory in Huntington, Ind., with
700.

While Carrier will forfeit some \$65 million a year in savings the move
was supposed to generate, that's a small price to pay to avoid the
public relations damage from moving the jobs as well as a possible
threat to United Technologies' far-larger military contracting business.

Roughly 10 percent of United Technologies' \$56 billion in revenue comes
from the federal government; the Pentagon is its single largest
customer. With \$4 billion in profit last year, the company has the
flexibility to find the savings elsewhere.

Members of Congress have been pressing to punish big military
contractors if they move jobs outside the United States.

Many industrial companies face intense pressure from Wall Street to
increase profits, even when the economy grows slowly --- a major reason
United Technologies decided to move.

That won't change after Mr. Trump takes office --- especially when
hourly pay in the Indianapolis plant is equivalent to what workers in
Mexico make in a day.

``This is a spot solution,'' said Mohan Tatikonda, a professor at the
Kelley School of Business at Indiana University. ``If it goes through it
helps some Carrier employees for a period of time, but it doesn't
address the loss of manufacturing jobs to technological change, which
will continue.''

Advertisement

\protect\hyperlink{after-bottom}{Continue reading the main story}

\hypertarget{site-index}{%
\subsection{Site Index}\label{site-index}}

\hypertarget{site-information-navigation}{%
\subsection{Site Information
Navigation}\label{site-information-navigation}}

\begin{itemize}
\tightlist
\item
  \href{https://help.nytimes.com/hc/en-us/articles/115014792127-Copyright-notice}{©~2020~The
  New York Times Company}
\end{itemize}

\begin{itemize}
\tightlist
\item
  \href{https://www.nytco.com/}{NYTCo}
\item
  \href{https://help.nytimes.com/hc/en-us/articles/115015385887-Contact-Us}{Contact
  Us}
\item
  \href{https://www.nytco.com/careers/}{Work with us}
\item
  \href{https://nytmediakit.com/}{Advertise}
\item
  \href{http://www.tbrandstudio.com/}{T Brand Studio}
\item
  \href{https://www.nytimes.com/privacy/cookie-policy\#how-do-i-manage-trackers}{Your
  Ad Choices}
\item
  \href{https://www.nytimes.com/privacy}{Privacy}
\item
  \href{https://help.nytimes.com/hc/en-us/articles/115014893428-Terms-of-service}{Terms
  of Service}
\item
  \href{https://help.nytimes.com/hc/en-us/articles/115014893968-Terms-of-sale}{Terms
  of Sale}
\item
  \href{https://spiderbites.nytimes.com}{Site Map}
\item
  \href{https://help.nytimes.com/hc/en-us}{Help}
\item
  \href{https://www.nytimes.com/subscription?campaignId=37WXW}{Subscriptions}
\end{itemize}
