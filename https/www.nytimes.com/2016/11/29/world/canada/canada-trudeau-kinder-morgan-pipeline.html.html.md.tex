Sections

SEARCH

\protect\hyperlink{site-content}{Skip to
content}\protect\hyperlink{site-index}{Skip to site index}

\href{https://www.nytimes.com/section/world/canada}{Canada}

\href{https://myaccount.nytimes.com/auth/login?response_type=cookie\&client_id=vi}{}

\href{https://www.nytimes.com/section/todayspaper}{Today's Paper}

\href{/section/world/canada}{Canada}\textbar{}Justin Trudeau Approves
Oil Pipeline Expansion in Canada

\url{https://nyti.ms/2ghvYYi}

\begin{itemize}
\item
\item
\item
\item
\item
\end{itemize}

Advertisement

\protect\hyperlink{after-top}{Continue reading the main story}

Supported by

\protect\hyperlink{after-sponsor}{Continue reading the main story}

\hypertarget{justin-trudeau-approves-oil-pipeline-expansion-in-canada}{%
\section{Justin Trudeau Approves Oil Pipeline Expansion in
Canada}\label{justin-trudeau-approves-oil-pipeline-expansion-in-canada}}

\includegraphics{https://static01.nyt.com/images/2016/11/29/world/30canada/-30canada-articleLarge.jpg?quality=75\&auto=webp\&disable=upscale}

By \href{http://www.nytimes.com/by/ian-austen}{Ian Austen}

\begin{itemize}
\item
  Nov. 29, 2016
\item
  \begin{itemize}
  \item
  \item
  \item
  \item
  \item
  \end{itemize}
\end{itemize}

OTTAWA --- In a decision that will almost surely prompt showdowns with
environmentalists, indigenous groups and some political allies, Prime
Minister
\href{http://www.nytimes.com/2016/09/15/world/canada/justin-trudeau.html}{Justin
Trudeau} of Canada approved on Tuesday the expansion of a pipeline
linking the oil sands in Alberta to a tanker port in British Columbia.

The Kinder Morgan Trans Mountain project will increase the capacity of a
53-year-old pipeline to 890,000 barrels a day from 300,000 and expand
the tanker port. In recent weeks, there have been several large protests
against the project, particularly in Vancouver, British Columbia. But
Rachel Notley, the premier of Alberta, has repeatedly said that the
project is critical to the future of her province's energy industry.

Environmental groups began condemning the decision as Mr. Trudeau was
making his announcement late Tuesday afternoon, but the prime minister
said that the pipeline expansion did not contradict his pledges to
improve environmental protection and mitigate climate change.

``We've heard clearly from Canadians that they don't want to see someone
trying to make a choice between what's good for the environment and
what's good for the economy,'' Mr. Trudeau said at a news conference.
``They need to go together, and the decisions we've made today and
leading up to today are entirely consistent with that.''

He said that spending much of his childhood with his grandparents in
British Columbia and then studying and working in Vancouver as a teacher
helped inform his decision on the pipeline.

``If I thought that this project was unsafe for the B.C. coast, I would
reject it,'' he said.

Mr. Trudeau acknowledged that he was ``under no illusions'' that his
approval of the multibillion-dollar project would not encounter strong
opposition. The expansion's opponents include Gregor Robertson, the
mayor of Vancouver, who is generally a political ally, and some members
of Mr. Trudeau's Liberal caucus in the House of Commons. But while Mr.
Trudeau said the government welcomed people expressing contrary views,
their opposition would not change what he characterized as a decision
based on science.

\includegraphics{https://static01.nyt.com/images/2016/11/30/world/30canada2/30canada2-articleLarge.jpg?quality=75\&auto=webp\&disable=upscale}

``We have not been and we will not be swayed by political arguments,''
Mr. Trudeau said.

A number of factors underlie opposition to the Kinder Morgan project.
Some people in British Columbia fear that the increase in tanker traffic
to the port in the Vancouver suburb of Burnaby will inevitably lead to a
major oil spill. Many environmentalists charge that the pipeline will
make it impossible for Canada to meet the carbon emissions targets set
by Mr. Trudeau's government --- an assertion the government rejects.

And, like the protests that led the Obama administration to block the
\href{http://www.nytimes.com/topic/subject/keystone-xl-pipeline?8qa}{Keystone
XL pipeline project} from Canada, many people see blocking Kinder Morgan
as a way to limit development of the oil sands, which they view as a
particularly dirty energy source.

Environmental groups said they would move to stop the project through a
variety of means.

``People are already standing up and fighting back, and that is only
going to grow,'' said Sven Biggs, the energy and climate campaigner for
Stand.earth, an environmental group previously known as ForestEthics.
``It's going to be in the courts; it's going to be in the streets; it's
also going to be at the ballot box.''

Before Mr. Trudeau's announcement, some opponents of the Kinder Morgan
pipeline predicted that its approval would lead to protests on the scale
of the anti-pipeline demonstrations now underway at the Standing Rock
Sioux Reservation in North Dakota.

After a meeting on Monday with cabinet ministers in Ottawa about the
pipeline, Chief Maureen Thomas of the Tsleil-Waututh Nation, the
indigenous community directly across from the tanker port in British
Columbia, said in an interview that her community would mount a legal
challenge if the project was approved.

Court challenges by indigenous communities have previously stalled the
Enbridge Northern Gateway, a pipeline plan to link the oil sands to the
northern coast of British Columbia. On Tuesday, Mr. Trudeau said the
government had told its energy board to reject that proposal.

Rona Ambrose, the interim leader of the Conservative Party, criticized
Mr. Trudeau for rejecting that pipeline. But she was also pessimistic
about the prospects for the Kinder Morgan project in the face of legal
challenges.

``I don't think that will have a chance of being built,'' she told
reporters.

Advertisement

\protect\hyperlink{after-bottom}{Continue reading the main story}

\hypertarget{site-index}{%
\subsection{Site Index}\label{site-index}}

\hypertarget{site-information-navigation}{%
\subsection{Site Information
Navigation}\label{site-information-navigation}}

\begin{itemize}
\tightlist
\item
  \href{https://help.nytimes.com/hc/en-us/articles/115014792127-Copyright-notice}{©~2020~The
  New York Times Company}
\end{itemize}

\begin{itemize}
\tightlist
\item
  \href{https://www.nytco.com/}{NYTCo}
\item
  \href{https://help.nytimes.com/hc/en-us/articles/115015385887-Contact-Us}{Contact
  Us}
\item
  \href{https://www.nytco.com/careers/}{Work with us}
\item
  \href{https://nytmediakit.com/}{Advertise}
\item
  \href{http://www.tbrandstudio.com/}{T Brand Studio}
\item
  \href{https://www.nytimes.com/privacy/cookie-policy\#how-do-i-manage-trackers}{Your
  Ad Choices}
\item
  \href{https://www.nytimes.com/privacy}{Privacy}
\item
  \href{https://help.nytimes.com/hc/en-us/articles/115014893428-Terms-of-service}{Terms
  of Service}
\item
  \href{https://help.nytimes.com/hc/en-us/articles/115014893968-Terms-of-sale}{Terms
  of Sale}
\item
  \href{https://spiderbites.nytimes.com}{Site Map}
\item
  \href{https://help.nytimes.com/hc/en-us}{Help}
\item
  \href{https://www.nytimes.com/subscription?campaignId=37WXW}{Subscriptions}
\end{itemize}
