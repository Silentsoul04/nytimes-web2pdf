Sections

SEARCH

\protect\hyperlink{site-content}{Skip to
content}\protect\hyperlink{site-index}{Skip to site index}

\href{https://www.nytimes.com/section/world/europe}{Europe}

\href{https://myaccount.nytimes.com/auth/login?response_type=cookie\&client_id=vi}{}

\href{https://www.nytimes.com/section/todayspaper}{Today's Paper}

\href{/section/world/europe}{Europe}\textbar{}What Is U.K.'s `Brexit'
Plan? Glimpse of a Notepad Stirs Up Intrigue

\url{https://nyti.ms/2ggDfYl}

\begin{itemize}
\item
\item
\item
\item
\item
\end{itemize}

Advertisement

\protect\hyperlink{after-top}{Continue reading the main story}

Supported by

\protect\hyperlink{after-sponsor}{Continue reading the main story}

\hypertarget{what-is-uks-brexit-plan-glimpse-of-a-notepad-stirs-up-intrigue}{%
\section{What Is U.K.'s `Brexit' Plan? Glimpse of a Notepad Stirs Up
Intrigue}\label{what-is-uks-brexit-plan-glimpse-of-a-notepad-stirs-up-intrigue}}

\includegraphics{https://static01.nyt.com/images/2016/12/01/world/30Britain/30Britain-articleInline.jpg?quality=75\&auto=webp\&disable=upscale}

By \href{http://www.nytimes.com/by/katrin-bennhold}{Katrin Bennhold}

\begin{itemize}
\item
  Nov. 29, 2016
\item
  \begin{itemize}
  \item
  \item
  \item
  \item
  \item
  \end{itemize}
\end{itemize}

LONDON --- It's official --- or is it? The British strategy for its
divorce from the European Union is to ``have your cake and eat it,''
something critics have derided as delusional.

Ever since voting to leave the bloc, Britons have been trying to figure
out whether their government has an actual plan for negotiating its
exit, or ``Brexit,'' and what that plan might be. This week, an open
notepad, surreptitiously photographed in the grip of an aide leaving the
aptly named Department for Exiting the European Union, has offered at
least a glimpse into the fog.

About halfway down, this line appears: ``What's the model? Have cake \&
eat.''

Boris Johnson, the foreign secretary and leading pro-Brexit campaigner,
once notoriously claimed that Britain could leave behind everything it
did not like about the European Union (like immigration from other
European countries) but keep everything it did like (such as
unencumbered trade with European countries). ``My policy on cake is pro
having it and pro eating it,'' Mr. Johnson once cheerfully proclaimed.

So popular has the analogy become that Donald Tusk, the president of the
European Council, warned Britain in October that ``there will be no
cakes on the table for anyone --- only salt and vinegar.''

With a close-up snapshot of the notepad gracing most front pages in
British newspapers on Tuesday, the government of Prime Minister Theresa
May swiftly let it be known that the notes did not reflect its Brexit
strategy.

The business secretary, Greg Clark, said that ``it would be nice'' to
have the cake and eat it, too, but that this was ``not the policy.''

``I was interested and amused to see it, because it doesn't reflect any
of the conversations that I've been part of in Downing Street,'' Mr.
Clark told the BBC.

Critics were quick to pounce. ``If this is a strategy, it is
incoherent,'' said the leader of the Liberal Democrats, Tim Farron. ``We
can't have our cake and eat it, and there is no certainty on the single
market. This picture shows the government doesn't have a plan or even a
clue.''

However, another entry in the scribbled notes suggested that whoever was
conducting the briefing had a more realistic take on the negotiations,
saying that Britain did not expect to be offered access to the European
single market after leaving the union.

There is concern, the briefer said, that Britain would face a ``very
French negotiating team'' --- the chief negotiator appointed on the
European side, Michel Barnier, is a former French minister --- and that
the French were ``likely to be most difficult.''

A spokesman for Mrs. May's office said the notes did not belong to a
government official or a special adviser. ``They do not reflect the
government's position in relation to Brexit negotiations,'' he said.

But they have certainly increased the pressure on Mrs. May, who has been
fond of repeating that ``Brexit means Brexit,'' to be a little more
specific.

The notebook was carried by an aide to Mark Field, the vice chairman of
the governing Conservative Party whose constituency includes parts of
London's financial district, which is worried about losing business to
Paris and Frankfurt post-Brexit. It displayed a full page of handwritten
notes despite recent reports that a sign at the exit of the Brexit
department reminds all those leaving: ``Stop! Are your documents on
show?''

\href{https://www.nytimes.com/interactive/2016/business/international/brexit-uk-what-happens-business.html}{}

\includegraphics{https://static01.nyt.com/images/2017/03/29/business/27BREXIT/27BREXIT-articleLarge.jpg}

\hypertarget{how-brexit-could-change-business-in-britain}{%
\subsection{How `Brexit' Could Change Business in
Britain}\label{how-brexit-could-change-business-in-britain}}

Britain has started the clock on leaving the European Union, and will be
out of the bloc by March 2019. Here is how ``Brexit'' has affected
business so far.

It is not clear whether Mr. Field, who subsequently went next door into
the prime minister's residence, met the government's Brexit chief, David
Davis, or whether the notes are an account of talks held at the Brexit
department that day.

But they appear to suggest that at least someone in the British
administration believes that while a trade deal on manufacturing should
be ``relatively straightforward,'' agreement on services like London's
lucrative financial and legal sectors would be more complicated. Even a
transitional arrangement, one that would allow Britain continued access
to Europe's single market after a departure from the bloc while it
negotiates a new trade deal, is unlikely.

``Transitional --- loath to do it,'' the notes read. ``Whitehall will
hold onto it,'' they added, referring to the British seat of government.
``We need to bring an end to negotiations.''

Opposition politicians also jumped on the opportunity to demand more
transparency from the government. Keir Starmer, who holds the Brexit
brief for the Labour Party, called on the government ``to come clean, to
end this unnecessary uncertainty and publish a clear plan for Brexit.''

``These disclosures are significant because they suggest that the
government is not even going to fight for the single market or customs
union in the negotiations,'' he said.

The BBC's assistant political editor, Norman Smith, said the way the
government so quickly played down the notes' significance highlighted
how ``awkward'' their publication was.

``The real damage is that phrase: `What is the model? Have cake and eat
it,''' Mr. Smith said. ``The damage is the way that will be read by
other E.U. countries.''

Advertisement

\protect\hyperlink{after-bottom}{Continue reading the main story}

\hypertarget{site-index}{%
\subsection{Site Index}\label{site-index}}

\hypertarget{site-information-navigation}{%
\subsection{Site Information
Navigation}\label{site-information-navigation}}

\begin{itemize}
\tightlist
\item
  \href{https://help.nytimes.com/hc/en-us/articles/115014792127-Copyright-notice}{©~2020~The
  New York Times Company}
\end{itemize}

\begin{itemize}
\tightlist
\item
  \href{https://www.nytco.com/}{NYTCo}
\item
  \href{https://help.nytimes.com/hc/en-us/articles/115015385887-Contact-Us}{Contact
  Us}
\item
  \href{https://www.nytco.com/careers/}{Work with us}
\item
  \href{https://nytmediakit.com/}{Advertise}
\item
  \href{http://www.tbrandstudio.com/}{T Brand Studio}
\item
  \href{https://www.nytimes.com/privacy/cookie-policy\#how-do-i-manage-trackers}{Your
  Ad Choices}
\item
  \href{https://www.nytimes.com/privacy}{Privacy}
\item
  \href{https://help.nytimes.com/hc/en-us/articles/115014893428-Terms-of-service}{Terms
  of Service}
\item
  \href{https://help.nytimes.com/hc/en-us/articles/115014893968-Terms-of-sale}{Terms
  of Sale}
\item
  \href{https://spiderbites.nytimes.com}{Site Map}
\item
  \href{https://help.nytimes.com/hc/en-us}{Help}
\item
  \href{https://www.nytimes.com/subscription?campaignId=37WXW}{Subscriptions}
\end{itemize}
