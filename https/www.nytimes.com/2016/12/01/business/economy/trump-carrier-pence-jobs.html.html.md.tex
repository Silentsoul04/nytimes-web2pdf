Sections

SEARCH

\protect\hyperlink{site-content}{Skip to
content}\protect\hyperlink{site-index}{Skip to site index}

\href{https://www.nytimes.com/section/business/economy}{Economy}

\href{https://myaccount.nytimes.com/auth/login?response_type=cookie\&client_id=vi}{}

\href{https://www.nytimes.com/section/todayspaper}{Today's Paper}

\href{/section/business/economy}{Economy}\textbar{}Trump Sealed Carrier
Deal With Mix of Threat and Incentive

\url{https://nyti.ms/2gQfvav}

\begin{itemize}
\item
\item
\item
\item
\item
\item
\end{itemize}

Advertisement

\protect\hyperlink{after-top}{Continue reading the main story}

Supported by

\protect\hyperlink{after-sponsor}{Continue reading the main story}

\hypertarget{trump-sealed-carrier-deal-with-mix-of-threat-and-incentive}{%
\section{Trump Sealed Carrier Deal With Mix of Threat and
Incentive}\label{trump-sealed-carrier-deal-with-mix-of-threat-and-incentive}}

\includegraphics{https://static01.nyt.com/images/2016/12/02/us/02carrier-video-hp/02carrier-video-hp-videoSixteenByNine3000-v2.jpg}

By \href{http://www.nytimes.com/by/nelson-d-schwartz}{Nelson D.
Schwartz}

\begin{itemize}
\item
  Dec. 1, 2016
\item
  \begin{itemize}
  \item
  \item
  \item
  \item
  \item
  \item
  \end{itemize}
\end{itemize}

INDIANAPOLIS --- The long-promised call from Donald J. Trump to the
heating and cooling giant Carrier came early one morning about a week
after the election, when he unexpectedly won the industrial heartland.

The president-elect warned Gregory Hayes, the chief executive of
Carrier's parent, United Technologies, that he had to find a way to save
a substantial share of the jobs it had vowed to move to Mexico, or he
would face the wrath of the incoming administration.

On Thursday, as he toured the factory floor here to take credit for
saving roughly half of the 2,000 jobs Indiana stood to lose, Mr. Trump
sent a message to other businesses as well that he intended to follow
through on his pledges to impose stiff tariffs on imports from companies
that move production overseas and ship their products back to the United
States.

``This is the way it's going to be,'' Mr. Trump said in an interview
with The New York Times. ``Corporate America is going to have to
understand that we have to take care of our workers also.''

Mr. Trump was accompanied by his vice president-elect, Mike Pence, who
is currently Indiana's governor. He was in the room at Trump Tower when
the president-elect placed his initial call to Mr. Hayes, and he was the
one who sealed the deal with the chief executive with a handshake in the
building on Monday.

\href{https://www.nytimes.com/interactive/2016/11/30/us/politics/trump-manufacturing-jobs-indiana-carrier.html}{}

\includegraphics{https://static01.nyt.com/images/2016/11/30/us/politics/trump-manufacturing-jobs-indiana-carrier-1480538094077/trump-manufacturing-jobs-indiana-carrier-1480538094077-articleLarge-v3.png}

\hypertarget{what-it-means-for-trump-to-save-1000-jobs-in-indiana}{%
\subsection{What It Means for Trump to Save 1,000 Jobs in
Indiana}\label{what-it-means-for-trump-to-save-1000-jobs-in-indiana}}

How the president-elect's deal measures up to U.S. manufacturing job
losses.

``I don't want them moving out of the country without consequences,''
Mr. Trump said, even if that means angering the free-market-oriented
Republicans he beat in the primaries but will have to work with on
Capitol Hill.

``The free market has been sorting it out and America's been losing,''
Mr. Pence added, as Mr. Trump interjected, ``Every time, every time.''

But since the pact was disclosed on Tuesday, critics have pounced on
Carrier's receipt of \$7 million in incentives from the state of Indiana
--- just the kind of corporate giveaways Mr. Trump knocked as he slammed
Carrier on the campaign trail last spring.

Others have pointed out that cutting individual deals with different
companies is a costly and ineffective way to stem the powerful forces
that impel business to move factories and jobs in a highly competitive
global and national economy.

``He has signaled to every corporation in America that they can threaten
to offshore jobs in exchange for business-friendly tax benefits and
incentives,'' Senator Bernie Sanders of Vermont wrote in an op-ed on
Thursday for The Washington Post.

In Washington, Josh Earnest, the White House press secretary, suggested
that Carrier's decision, while ``good news,'' was only a drop in the
bucket.

Image

An image of a letter handed out to Carrier employees on Thursday.
``While this announcement is good news for many, we recognize it is not
good news for everyone,'' the letter stated.

``Mr. Trump would have to make 804 more announcements just like that to
equal the standard of jobs in the manufacturing sector that were created
in this country under President Obama's watch,'' Mr. Earnest said. ``So
this is good news, but the incoming president has a high bar to meet
when it comes to putting in place the kind of economic policies that
would benefit American workers.''

Still, Mr. Trump's appeal is likely to lead to a more fundamental shift
in the way Washington
approaches\href{https://www.nytimes.com/2016/11/30/business/economy/trump-saved-jobs-at-carrier-but-more-midwest-jobs-are-in-jeopardy.html}{dealing
with corporate decisions on where to locate jobs}.

``Contrary to early reactions from the left and the right, the Carrier
deal opens the door to a new approach to U.S. economic growth policy
that is sorely needed,'' the Information Technology and Innovation
Foundation, a research and advocacy group, said in a statement. ``It
sets the precedent that growing, attracting and retaining globally
traded, innovation-based industries that are both high-value and pay
high wages is central to U.S. economic growth.''

Despite the cheers Mr. Trump received as he walked around the factory
floor, where the lines continued to run and he had to shout at times to
be heard, another 1,000 workers for the company in Indiana will be
losing their jobs.

This includes 700 at a United Technologies factory in nearby Huntington,
as well as several hundred here. The 800 or so jobs that are being
preserved are mostly on the lines that build medium- and high-efficiency
gas furnaces.

Not long after Mr. Trump and Mr. Pence departed for the airport and to
another rally in Ohio to celebrate his victory, workers coming in for
the night shift received a letter titled ``Company Update on
Indianapolis Operations.''

\includegraphics{https://static01.nyt.com/images/2016/12/02/business/02CARRIER4/02CARRIER4-articleLarge.jpg?quality=75\&auto=webp\&disable=upscale}

``While this announcement is good news for many, we recognize it is not
good news for everyone,'' the letter stated. ``We are moving forward
with previously announced plans to relocate the fan coil manufacturing
lines, with the expected completion by the end of 2017.''

United Technologies, Carrier's parent, saw the writing on the wall as
soon as Mr. Trump declared victory last month. Offering to preserve
jobs, even at the cost of some of the \$65 million savings the company
expected from the move, could serve its larger corporate interests.

``Every penny counts, but if we step back and I'm looking at earnings of
\$6.60 per share this year, 2 cents is an easy concession if the
president-elect listens to some of the company's bigger concerns,'' said
Howard Rubel, a senior equity analyst with Jefferies, an investment
banking firm in New York.

And Mr. Trump and Mr. Pence, while providing a carrot through the state
incentives and promises of future business tax cuts, held an implicit
stick: the threat of pulling federal contracts from Carrier's parent,
United Technologies. Mr. Trump and his team were well aware that the
amount United Technologies stood to lose in those contracts dwarfed the
savings from moving some of its operations to Monterrey from Indiana.

Despite only accomplishing half of what had been promised in the
campaign, Mr. Trump and Mr. Pence predicted that the number of jobs
ultimately preserved could rise as Carrier follows through on its
promise to invest more than \$16 million in the state.

That provision, plus the incentives, were worked out between Carrier and
officials from the state of Indiana's economic development office, with
Mr. Pence overseeing the process.

The vice president-elect insisted that the incentives did not represent
a giveaway on Mr. Trump's part, and claimed United Technologies had
turned down a similar-size package of breaks in March.

What made the difference, he said, was Mr. Trump's public pressure, as
well his promise to cut corporate taxes and ease regulation.

``These jobs were gone,'' Mr. Pence said. ``I sat the executives down in
my governor's office in the statehouse in early March. They said we
aren't in a position to reconsider this in any way, shape or form.''

Mr. Trump, too, played down the role of the incentives. And he said he
did not directly raise the \$5 billion to \$6 billion in federal
contracts United Technologies receives, much of it from the Pentagon.

But company officials are acutely aware that its Pratt \& Whitney unit,
among other things, supplies jet engines to the Air Force's most
advanced fighter and many other planes, making it much more vulnerable
to political pressure than other, lesser-known manufacturers that have
been steadily closing shop in the Midwest and moving production south of
the border.

``It may have a played a role in their equation,'' Mr. Trump allowed.
``I never mentioned it. I didn't feel I had to.''

What about the fact that Mr. Trump frequently sourced products for his
properties overseas, along with some popular Trump-branded merchandise?
Will he lead by example and buy more goods made in the U.S.A.?

``I buy thousands and thousands of TVs,'' he said. ``I would like to,
but they essentially don't make them in the U.S. You know the hats I do?
It took us forever to find a company that can make a hat in the U.S.''

Advertisement

\protect\hyperlink{after-bottom}{Continue reading the main story}

\hypertarget{site-index}{%
\subsection{Site Index}\label{site-index}}

\hypertarget{site-information-navigation}{%
\subsection{Site Information
Navigation}\label{site-information-navigation}}

\begin{itemize}
\tightlist
\item
  \href{https://help.nytimes.com/hc/en-us/articles/115014792127-Copyright-notice}{©~2020~The
  New York Times Company}
\end{itemize}

\begin{itemize}
\tightlist
\item
  \href{https://www.nytco.com/}{NYTCo}
\item
  \href{https://help.nytimes.com/hc/en-us/articles/115015385887-Contact-Us}{Contact
  Us}
\item
  \href{https://www.nytco.com/careers/}{Work with us}
\item
  \href{https://nytmediakit.com/}{Advertise}
\item
  \href{http://www.tbrandstudio.com/}{T Brand Studio}
\item
  \href{https://www.nytimes.com/privacy/cookie-policy\#how-do-i-manage-trackers}{Your
  Ad Choices}
\item
  \href{https://www.nytimes.com/privacy}{Privacy}
\item
  \href{https://help.nytimes.com/hc/en-us/articles/115014893428-Terms-of-service}{Terms
  of Service}
\item
  \href{https://help.nytimes.com/hc/en-us/articles/115014893968-Terms-of-sale}{Terms
  of Sale}
\item
  \href{https://spiderbites.nytimes.com}{Site Map}
\item
  \href{https://help.nytimes.com/hc/en-us}{Help}
\item
  \href{https://www.nytimes.com/subscription?campaignId=37WXW}{Subscriptions}
\end{itemize}
