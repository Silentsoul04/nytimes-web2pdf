Sections

SEARCH

\protect\hyperlink{site-content}{Skip to
content}\protect\hyperlink{site-index}{Skip to site index}

\href{https://myaccount.nytimes.com/auth/login?response_type=cookie\&client_id=vi}{}

\href{https://www.nytimes.com/section/todayspaper}{Today's Paper}

\href{/section/business/dealbook}{DealBook}\textbar{}Silicon Valley
Chiefs Notably Absent From Trump's Cabinet of Business Advisers

\url{https://nyti.ms/2gWKthp}

\begin{itemize}
\item
\item
\item
\item
\item
\end{itemize}

Advertisement

\protect\hyperlink{after-top}{Continue reading the main story}

Supported by

\protect\hyperlink{after-sponsor}{Continue reading the main story}

DealBook Business and Policy

\hypertarget{silicon-valley-chiefs-notably-absent-from-trumps-cabinet-of-business-advisers}{%
\section{Silicon Valley Chiefs Notably Absent From Trump's Cabinet of
Business
Advisers}\label{silicon-valley-chiefs-notably-absent-from-trumps-cabinet-of-business-advisers}}

\includegraphics{https://static01.nyt.com/images/2016/12/02/business/cnbc-trumpbiz/cnbc-trumpbiz-videoSixteenByNineJumbo1600.png}

By \href{http://www.nytimes.com/by/michael-j-de-la-merced}{Michael J. de
la Merced}

\begin{itemize}
\item
  Dec. 2, 2016
\item
  \begin{itemize}
  \item
  \item
  \item
  \item
  \item
  \end{itemize}
\end{itemize}

In President-elect Donald J. Trump's newly named kitchen cabinet of
business advisers, Wall Street is in. Silicon Valley is out.

Mr. Trump has named 16 business leaders to serve on what's being called
the President's Strategic and Policy Forum, described as a group meant
to guide his administration on economic matters.

The list is notable for leaning toward New York executives and
industries --- finance in particular. The list echoes Mr. Trump's picks
for a number of major economic positions, including Treasury secretary
(the former Goldman Sachs partner and hedge fund manager Steven T.
Mnuchin) and commerce secretary (the billionaire investor Wilbur L.
Ross).

Given his long experience as a New York real estate investor, Mr.
Trump's selections may not come as a surprise.

``Donald comes from the financial services world. I think he tends to
pick people who he's comfortable with,'' Stephen A. Schwarzman, the
co-founder and chief executive of the Blackstone Group, who is leading
the forum effort, told CNBC Friday afternoon.

The forum largely excludes technology, home to some of the nation's
best-known, most innovative and biggest companies by market value, which
represented over
\href{http://www.bea.gov/newsreleases/industry/gdpindustry/2016/pdf/gdpind216.pdf}{8
percent of the private sector economy last year}, according to the
Bureau of Economic Analysis.

It is a bipartisan list, nonetheless, with a number of members who
traditionally have favored Democratic candidates, including Laurence D.
Fink of the asset management colossus BlackRock, Mary T. Barra of
General Motors and Robert Iger of Walt Disney.

``This forum brings together C.E.O.s and business leaders who know what
it takes to create jobs and drive economic growth,'' Mr. Trump said in a
statement. ``My administration is committed to drawing on private sector
expertise.''

The group is expected to meet with Mr. Trump monthly. The first meeting
will be at the White House in early February. Mr. Schwarzman will serve
as chairman.

The private equity titan is a longtime Republican donor who has known
the president-elect for years. Mr. Trump and his wife, Melania, were
among the guests to the Blackstone chief's
\href{http://dealbook.nytimes.com/2007/02/14/inside-stephen-schwarzmans-birthday-bash}{famous
60th birthday party} in 2007.

Mr. Schwarzman was not a vocal supporter or fund-raiser for Mr. Trump.
Since the election, however, he has embraced the notion that the
president-elect's promise to pull back financial regulations and cut
taxes will supercharge the economy.

In public and in private, Mr. Schwarzman has lately said that he is
``excited'' about the potential for economic growth in a Trump
administration.

``The business community now becomes front and center,'' he said at
\href{http://www.wsj.com/video/blackstone-ceo-on-job-growth-under-president-trump/6F1E8405-F028-4ACC-B40B-551CA010EC36.html}{a
recent Wall Street Journal conference}.

At the conference, Mr. Schwarzman listed a number of likely changes
under Mr. Trump, including a loosening of lending regulations, a
lowering of the corporate tax rate from its current 35 percent level,
and a return of the more than a trillion dollars American companies
\href{http://www.cnbc.com/2016/09/20/us-companies-are-hoarding-2-and-a-half-trillion-dollars-in-cash-overseas.html}{currently
hold abroad} to avoid United States taxes.

``There are going to be so many of these changes that I think what's
going to happen, it's really going to force growth from a policy
perspective,'' Mr. Schwarzman said at the conference.

Despite his support for Mr. Trump's policies, Mr. Schwarzman has
signaled privately that he would not commit to a full-time position in
the administration, but was otherwise interested in helping to advise
the president-elect.

Other Wall Street heavyweights on the list are Jamie Dimon, the chief
executive of JPMorgan Chase, and Adebayo Ogunlesi, a former Credit
Suisse executive who is the chairman of the investment firm Global
Infrastructure Partners. Mr. Ogunlesi is also a board member at Goldman
Sachs. No Goldman executives were named to the forum.

Forum members with ties to the finance industry include Paul S. Atkins,
a former commissioner of the Securities and Exchange Commission who is
now the chief executive of a finance regulation consulting firm, and
Kevin Warsh, a former Morgan Stanley executive and a former governor of
the Federal Reserve.

Mr. Fink's presence is somewhat eyebrow-raising given his ties to the
Democratic Party. He had sometimes been bandied about as a potential
Treasury secretary in a Hillary Clinton administration, though his
status as the head of the world's biggest asset management firm made
that seem politically unlikely. Mr. Trump is
\href{http://mobile.reuters.com/article/idUSKCN0WQ0X0}{an investor in a
fund} managed by BlackRock.

Mr. Dimon is easier to understand. Though initially a supporter of
President Obama, the JPMorgan chief has long been critical of the
current administration's financial regulatory overhaul. While his name
had been floated as a potential candidate for Treasury secretary, people
close to him have insisted that he had no interest in taking on a
full-time commitment outside of his bank.

Aside from Virginia M. Rometty of IBM, there is hardly any
representation of technology companies, and certainly none from Silicon
Valley.

Perhaps that's unsurprising, given Mr. Trump's slim public support in
the Bay Area. Among his biggest champions is
\href{http://www.nytimes.com/2016/10/31/technology/peter-thiel-defends-his-most-contrarian-move-yet-supporting-trump.html}{Peter
Thiel}, the PayPal co-founder and Facebook board member, who is now
\href{https://www.bloomberg.com/news/articles/2016-11-11/peter-thiel-joins-trump-s-presidential-transition-team}{a
member of the Trump transition team}.

Indeed, many Silicon Valley luminaries have opposed Mr. Trump since the
presidential campaign. Eric Schmidt, the executive chairman of Google's
parent company, Alphabet, was an enthusiastic supporter of both Mr.
Obama and Mrs. Clinton.

The rest of the advisory group appears to be a bipartisan mixture of
business chiefs drawn from the worlds of finance, media and
manufacturing. They include Ms. Barra, Mr. Iger, Doug McMillon of
Walmart and W. James McNerney Jr., formerly of Boeing.

The most curious selection may be that of John F. Welch Jr., the former
head of General Electric. The current G.E. chief, Jeffrey R. Immelt, a
Republican, is not on the list of advisers. Mr. Immelt had
\href{http://www.vanityfair.com/news/2016/08/the-competitor-amazon-never-saw-coming}{criticized}
Mr. Trump's comments about Mexicans and Muslims.

Mr. Trump's style and speech are sometimes seen as throwbacks to the
Reagan administration, and Mr. Welch was the iconic business leader of
that era.

Advertisement

\protect\hyperlink{after-bottom}{Continue reading the main story}

\hypertarget{site-index}{%
\subsection{Site Index}\label{site-index}}

\hypertarget{site-information-navigation}{%
\subsection{Site Information
Navigation}\label{site-information-navigation}}

\begin{itemize}
\tightlist
\item
  \href{https://help.nytimes.com/hc/en-us/articles/115014792127-Copyright-notice}{©~2020~The
  New York Times Company}
\end{itemize}

\begin{itemize}
\tightlist
\item
  \href{https://www.nytco.com/}{NYTCo}
\item
  \href{https://help.nytimes.com/hc/en-us/articles/115015385887-Contact-Us}{Contact
  Us}
\item
  \href{https://www.nytco.com/careers/}{Work with us}
\item
  \href{https://nytmediakit.com/}{Advertise}
\item
  \href{http://www.tbrandstudio.com/}{T Brand Studio}
\item
  \href{https://www.nytimes.com/privacy/cookie-policy\#how-do-i-manage-trackers}{Your
  Ad Choices}
\item
  \href{https://www.nytimes.com/privacy}{Privacy}
\item
  \href{https://help.nytimes.com/hc/en-us/articles/115014893428-Terms-of-service}{Terms
  of Service}
\item
  \href{https://help.nytimes.com/hc/en-us/articles/115014893968-Terms-of-sale}{Terms
  of Sale}
\item
  \href{https://spiderbites.nytimes.com}{Site Map}
\item
  \href{https://help.nytimes.com/hc/en-us}{Help}
\item
  \href{https://www.nytimes.com/subscription?campaignId=37WXW}{Subscriptions}
\end{itemize}
