Sections

SEARCH

\protect\hyperlink{site-content}{Skip to
content}\protect\hyperlink{site-index}{Skip to site index}

\href{https://www.nytimes.com/section/world/asia}{Asia Pacific}

\href{https://myaccount.nytimes.com/auth/login?response_type=cookie\&client_id=vi}{}

\href{https://www.nytimes.com/section/todayspaper}{Today's Paper}

\href{/section/world/asia}{Asia Pacific}\textbar{}Trump's Call With
Taiwan: A Diplomatic Gaffe or a New Start?

\url{https://nyti.ms/2h5G0ZF}

\begin{itemize}
\item
\item
\item
\item
\item
\end{itemize}

Advertisement

\protect\hyperlink{after-top}{Continue reading the main story}

Supported by

\protect\hyperlink{after-sponsor}{Continue reading the main story}

\hypertarget{trumps-call-with-taiwan-a-diplomatic-gaffe-or-a-new-start}{%
\section{Trump's Call With Taiwan: A Diplomatic Gaffe or a New
Start?}\label{trumps-call-with-taiwan-a-diplomatic-gaffe-or-a-new-start}}

\includegraphics{https://static01.nyt.com/images/2016/12/06/world/06CHINATRUMP-1/06CHINATRUMP-1-articleInline.jpg?quality=75\&auto=webp\&disable=upscale}

By \href{http://www.nytimes.com/by/mark-landler}{Mark Landler} and
\href{http://www.nytimes.com/by/jane-perlez}{Jane Perlez}

\begin{itemize}
\item
  Dec. 5, 2016
\item
  \begin{itemize}
  \item
  \item
  \item
  \item
  \item
  \end{itemize}
\end{itemize}

WASHINGTON --- President-elect Donald J. Trump's motives for his
unorthodox call with the leader of Taiwan remain mysterious, but some
Republicans are hailing his improvisational diplomacy as a refreshing
break with diplomatic rituals that date to the Richard M. Nixon
administration. It could lay the groundwork, they say, for a healthier
relationship with China.

The question, as often with Mr. Trump, is whether he will have the
discipline to see through these changes. In short, was the phone call
the start of a calculated policy, or was it merely a one-off gesture
that will alienate China and leave Taiwan even more isolated?

So far the messages from the Trump camp are mixed. Vice President-elect
Mike Pence insisted on Sunday that people were reading too much into the
\href{https://www.nytimes.com/2016/12/02/us/politics/trump-speaks-with-taiwans-leader-a-possible-affront-to-china.html?_r=0}{conversation
with President Tsai Ing-wen}, and that Mr. Trump was simply accepting
congratulations from the ``democratically elected leader of Taiwan.''
But people advising the transition say Mr. Trump
\href{https://www.washingtonpost.com/politics/trumps-taiwan-phone-call-was-weeks-in-the-planning-say-people-who-were-involved/2016/12/04/f8be4b0c-ba4e-11e6-94ac-3d324840106c_story.html}{knew
exactly what he was doing}.

Certainly, the president-elect has done nothing to defuse the tensions
stirred by the call. A string of vituperative Twitter posts by Mr. Trump
over the weekend on China's trade and military policies has fanned
questions about whether he wants to reset the relationship with Beijing
more fundamentally.

But there is no dispute that several people involved in the Trump
campaign have long favored opening more formal channels between the
United States and Taiwan as a way to put pressure on China. And
Republicans being considered for top jobs applaud what they view as Mr.
Trump's determination to use Taiwan as a chip in a geopolitical contest
with China.

``He's the first president since the Shanghai Communique who provides an
opportunity to look fresh at the cross-strait relationship,'' said Jon
M. Huntsman Jr., who served as ambassador to China under President
Obama. Mr. Huntsman was referring to the 1972 statement, issued after
Nixon's visit to China, which sharply constrained America's dealings
with Taiwan.

Mr. Huntsman's name has recently surfaced as a candidate for secretary
of state, along with that of John R. Bolton, a former ambassador to the
United Nations under President George W. Bush.
\href{https://www.nytimes.com/2016/12/04/us/politics/trump-expands-search-for-secretary-of-state.html}{Mr.
Bolton met with Mr. Trump} on the day he took the phone call from Ms.
Tsai. Later, he said the United States ``should shake up the
relationship'' with China.

\includegraphics{https://static01.nyt.com/images/2016/12/06/world/06CHINATRUMP-2/06CHINATRUMP-2-articleInline.jpg?quality=75\&auto=webp\&disable=upscale}

Mr. Trump himself seemed to take umbrage at the suggestion that he
needed China's approval to speak with Ms. Tsai. In two posts on Twitter,
he wrote: ``Did China ask us if it was O.K. to devalue their currency
(making it hard for our companies to compete), heavily tax our products
going into their country (the U.S. doesn't tax them) or to build a
massive military complex in the middle of the South China Sea? I don't
think so!''

After an initial relatively mild response, China stiffened its protests
of Mr. Trump's freewheeling diplomacy. The Chinese warned Mr. Trump, in
a front-page editorial in the overseas edition of People's Daily, that
``creating troubles for the China-U.S. relationship is creating troubles
for the U.S. itself.''

China often uses the overseas edition of People's Daily to test-run
major policy pronouncements. In a pointed rejoinder to Mr. Trump, the
editorial said that pushing China on Taiwan ``would greatly reduce the
chance to achieve the goal of making America great again.''

By going after China's policies on trade and security, however, Mr.
Trump appeared to be confirming his intent to take a tougher line with
the Chinese leadership across a broader range of issues --- and further
dampened hopes in Beijing that he might step back from the campaign
rhetoric he has used, including threats of punishing trade tariffs.

Such a stance would reflect his foreign policy advisers, who have
criticized the Obama administration for being weak toward Beijing. Alex
Grey, a member of Mr. Trump's State Department transition team, wrote an
article, with another Trump adviser, Peter Navarro, in Foreign Policy
magazine last month in which he described Mr. Obama's treatment of
Taiwan as ``egregious.''

``This beacon of democracy in Asia is perhaps the most militarily
vulnerable U.S. partner anywhere in the world,'' Mr. Grey and Mr.
Navarro wrote, declaring that the island needed a ``comprehensive arms
deal'' with the United States to ``deter China's covetous gaze.''

Reince Priebus, Mr. Trump's designated chief of staff, also has a
history with Taiwan. In October 2015, he met with Ms. Tsai in Taipei as
part of a delegation from the Republican National Committee. After Mr.
Trump selected Mr. Priebus, Taiwan's foreign minister, David Lee, told a
legislative session that the appointment was ``good news for Taiwan.''

Pro-Taiwan groups in Washington played down their role in orchestrating
the call. But they welcomed it as a step to restore balance in the
three-way relationship between Washington, Beijing and Taipei. They also
said it need not provoke a confrontation with China.

``It's absurd that we talk about going to war with China to defend
Taiwan, and our presidents can't talk to each other,'' said Randall
Schriver, the chief executive of the
\href{http://project2049.net/}{Project 2049 Institute}, a Washington
think tank that favors closer American ties with Taiwan.

Image

A man reading a newspaper with the headline ``China wants U.S.
President-elect Donald Trump to use caution in dealing with Taiwan
issue'' at a brokerage house in Beijing on Monday.Credit...Andy
Wong/Associated Press

``The Chinese understand this,'' Mr. Schriver continued. ``They don't
want to see their efforts to isolate Taiwan rolled back, but they also
don't want a bad relationship with the U.S.''

Other China experts said they saw value in Mr. Trump's desire to rethink
old diplomatic protocols. As Mr. Trump has observed, it is difficult to
explain to ordinary people why the United States sells advanced weapons
to Taiwan but the leaders cannot speak to each other. Some China
scholars say it is not clear the policy serves the United States all
that well --- nor is it clear that Beijing would not learn to live with
a closer American relationship with Taiwan.

The trouble, said some, is that Mr. Trump needlessly antagonized China
by trumpeting the phone call and then following it up with a series of
defiant tweets. ``It's not the phone call that's the problem; it's the
making it public that's the problem,'' said Shelley Rigger, a professor
of political science at Davidson College who specializes in Taiwan.

That could put President Xi Jinping in a difficult position, forced to
choose between playing down Mr. Trump's attacks and risking a backlash
at home, or raising the stakes by pushing back more forcefully and
setting China on a potential collision course with the United States.

The Chinese government's initial reaction to Mr. Trump's call has
already drawn a torrent of criticism on social media from Chinese who
complained that it was not tough enough. The statement from the foreign
minister, Wang Yi, which was relatively low key, given the unprecedented
nature of the call, refrained from criticizing Mr. Trump, instead
accusing Taiwan of playing a ``little trick'' on the president-elect.

That offered Mr. Trump a face-saving way out of the imbroglio, and a
chance to de-escalate. But the messages he posted on Twitter late Sunday
stepped up the pressure on China's leaders instead.

``Mr. Trump's tweets and actions appear to be more than a symbolic
act,'' said James E. Fanell, a former director of intelligence and
information operations for the United States Pacific Fleet. ``This will
upset Beijing, but it is going to remind the rest of the Indo-Asia
Pacific region that the new administration is not going to be fettered
by the past.''

Some Chinese analysts see Mr. Trump as striking out on a starkly
different path from Mr. Obama's, determined to strenuously compete with
China on economic issues. But mindful of his career as a negotiator,
some view last week's moves optimistically as an opening gambit.

``By showing strength at the beginning, he may hope to gain advantages
in bargaining later with the Chinese,'' said Zhang Baohui, a professor
of international relations at Lingnan University in Hong Kong. ``He is a
businessman, and he could be bringing his business bargaining tactics to
interstate relations.''

Advertisement

\protect\hyperlink{after-bottom}{Continue reading the main story}

\hypertarget{site-index}{%
\subsection{Site Index}\label{site-index}}

\hypertarget{site-information-navigation}{%
\subsection{Site Information
Navigation}\label{site-information-navigation}}

\begin{itemize}
\tightlist
\item
  \href{https://help.nytimes.com/hc/en-us/articles/115014792127-Copyright-notice}{©~2020~The
  New York Times Company}
\end{itemize}

\begin{itemize}
\tightlist
\item
  \href{https://www.nytco.com/}{NYTCo}
\item
  \href{https://help.nytimes.com/hc/en-us/articles/115015385887-Contact-Us}{Contact
  Us}
\item
  \href{https://www.nytco.com/careers/}{Work with us}
\item
  \href{https://nytmediakit.com/}{Advertise}
\item
  \href{http://www.tbrandstudio.com/}{T Brand Studio}
\item
  \href{https://www.nytimes.com/privacy/cookie-policy\#how-do-i-manage-trackers}{Your
  Ad Choices}
\item
  \href{https://www.nytimes.com/privacy}{Privacy}
\item
  \href{https://help.nytimes.com/hc/en-us/articles/115014893428-Terms-of-service}{Terms
  of Service}
\item
  \href{https://help.nytimes.com/hc/en-us/articles/115014893968-Terms-of-sale}{Terms
  of Sale}
\item
  \href{https://spiderbites.nytimes.com}{Site Map}
\item
  \href{https://help.nytimes.com/hc/en-us}{Help}
\item
  \href{https://www.nytimes.com/subscription?campaignId=37WXW}{Subscriptions}
\end{itemize}
