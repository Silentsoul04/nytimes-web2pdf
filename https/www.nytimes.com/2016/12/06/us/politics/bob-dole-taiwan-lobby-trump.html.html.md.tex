Sections

SEARCH

\protect\hyperlink{site-content}{Skip to
content}\protect\hyperlink{site-index}{Skip to site index}

\href{https://www.nytimes.com/section/politics}{Politics}

\href{https://myaccount.nytimes.com/auth/login?response_type=cookie\&client_id=vi}{}

\href{https://www.nytimes.com/section/todayspaper}{Today's Paper}

\href{/section/politics}{Politics}\textbar{}Bob Dole Worked Behind the
Scenes on Trump-Taiwan Call

\url{https://nyti.ms/2heScYh}

\begin{itemize}
\item
\item
\item
\item
\item
\item
\end{itemize}

Advertisement

\protect\hyperlink{after-top}{Continue reading the main story}

Supported by

\protect\hyperlink{after-sponsor}{Continue reading the main story}

\hypertarget{bob-dole-worked-behind-the-scenes-on-trump-taiwan-call}{%
\section{Bob Dole Worked Behind the Scenes on Trump-Taiwan
Call}\label{bob-dole-worked-behind-the-scenes-on-trump-taiwan-call}}

\includegraphics{https://static01.nyt.com/images/2016/12/07/us/07lobby/07lobby-articleLarge.jpg?quality=75\&auto=webp\&disable=upscale}

By \href{https://www.nytimes.com/by/julie-hirschfeld-davis}{Julie
Hirschfeld Davis} and \href{http://www.nytimes.com/by/eric-lipton}{Eric
Lipton}

\begin{itemize}
\item
  Dec. 6, 2016
\item
  \begin{itemize}
  \item
  \item
  \item
  \item
  \item
  \item
  \end{itemize}
\end{itemize}

\href{http://cn.nytimes.com/usa/20161207/bob-dole-taiwan-lobby-trump/}{阅读简体中文版}

WASHINGTON --- Former Senator Bob Dole, acting as a foreign agent for
the government of Taiwan, worked behind the scenes over the past six
months to establish high-level contact between Taiwanese officials and
President-elect Donald J. Trump's staff, an outreach effort that
culminated last week in an unorthodox telephone call between Mr. Trump
and Taiwan's president.

Mr. Dole, a lobbyist with the Washington law firm Alston \& Bird,
coordinated with Mr. Trump's campaign and the transition team to set up
a series of meetings between Mr. Trump's advisers and officials in
Taiwan, according to disclosure documents filed last week with the
Justice Department. Mr. Dole also assisted in successful efforts by
Taiwan to include language favorable to it in the Republican Party
platform, according to the documents.

Mr. Dole's firm received \$140,000 from May to October for the work, the
forms said.

The disclosures suggest that President-elect Trump's decision to take a
call from the president of Taiwan, Tsai Ing-wen, was less a ham-handed
diplomatic gaffe and more the result of a well-orchestrated plan by
Taiwan to use the election of a new president to deepen its relationship
with the United States --- with an assist from a seasoned lobbyist well
versed in the machinery of Washington.

``They're very optimistic,'' Mr. Dole said of the Taiwanese in an
interview on Tuesday. ``They see a new president, a Republican, and
they'd like to develop a closer relationship.''

The United States' One China policy is nearly four decades old, Mr. Dole
said, referring to the
\href{http://www.nytimes.com/2016/12/03/world/asia/trump-taiwan-and-china-the-controversy-explained.html}{policy}
established in 1979 that denies Taiwan official diplomatic recognition
but maintains close contacts, promoting Taiwan's democracy and selling
it advanced military equipment.

\href{https://www.nytimes.com/interactive/2016/12/06/us/politics/document-bob-dole-lobbying-documents.html}{}

\includegraphics{https://static01.nyt.com/images/2016/12/06/us/politics/lobby-document-promo/lobby-document-promo-largeHorizontalJumbo.png}

\hypertarget{doles-role-in-trump-taiwan-relationship}{%
\subsection{Dole's Role in Trump-Taiwan
Relationship}\label{doles-role-in-trump-taiwan-relationship}}

The law firm Alston \& Bird filed disclosure documents with the Justice
Department, as required by law, at the end of November detailing the
role that former Senator Bob Dole played in cultivating a relationship
between the Taiwanese government and President-elect Donald J. Trump's
staff.

The phone call between Mr. Trump and Ms. Tsai was a striking break from
nearly four decades of diplomatic practice and threatened to precipitate
a major rift with China, which admonished Mr. Trump in a front-page
editorial in the overseas edition of People's Daily.

The disclosure documents were submitted before the call took place and
made no mention of it. But Mr. Dole, 93, a former Senate majority leader
from Kansas, said he had worked with transition officials to facilitate
the conversation.

``It's fair to say that we had some influence,'' he said. ``When you
represent a client and they make requests, you're supposed to respond.''

Officials on Mr. Trump's transition team did not respond to requests for
comment.

The documents suggest that Mr. Dole helped the government of Taiwan
establish early access to Mr. Trump's inner circle during the campaign,
when Mr. Dole worked to involve Mr. Trump's aides in a United States
delegation to Taiwan and to facilitate a Taiwanese delegation to the
Republican National Convention in Cleveland in July.

The effort has continued in the weeks since the election, with Mr. Dole
on Tuesday saying he was trying to fulfill a request from a special
envoy from Taiwan who was visiting Washington to see Reince Priebus,
tapped by Mr. Trump to be White House chief of staff, and Newt Gingrich,
who is close to the president-elect. (The Priebus meeting, Mr. Dole
said, would most likely have to wait until Mr. Trump is inaugurated.)

Mr. Dole, the only former Republican presidential nominee to endorse Mr.
Trump, arranged a meeting between Senator Jeff Sessions, Republican of
Alabama, whom Mr. Trump has chosen to be his attorney general, and
Stanley Kao, Taiwan's envoy to the United States, and convened a meeting
between Taiwanese officials and Mr. Trump's transition team, the
documents say.

Mr. Dole, who said he first took an interest in Taiwan as a senator when
Congress was considering the 1979 Taiwan Relations Act that established
the current policy, has lobbied for the Taiwanese government for nearly
two decades. In a letter in January, Mr. Dole laid out the terms of his
agreement to represent the Taipei Economic and Cultural Representative
Office in the United States, Taiwan's unofficial embassy, including a
\$25,000 monthly retainer.

That letter and the document detailing Mr. Dole's work for the Taiwanese
were filed at the Justice Department, which requires foreign agents to
register and detail their efforts at influencing the United States
government.

Among his duties, the letter said, were helping Taiwan achieve its
``military goals'' and obtain membership in the Trans-Pacific
Partnership, the 12-nation trade deal that Mr. Trump has promised to
withdraw from. Mr. Dole was also to arrange for Taiwanese officials to
meet with members of Congress from both parties and arrange access to
Republican presidential contenders and to the party's national
convention.

The government of Taiwan has retained a powerful bipartisan
constellation of former members of Congress to promote its interests in
Washington. Richard A. Gephardt, a Missouri Democrat and former House
majority leader, also signed a \$25,000-a-month contract to represent
the Taipei office this year, as did Thomas A. Daschle, Democrat of South
Dakota, a former Senate majority leader, in 2015.

\href{https://www.nytimes.com/interactive/2016/12/02/world/trump-calls-to-world-leaders.html}{}

\includegraphics{https://static01.nyt.com/images/2016/12/02/world/trump-calls-to-world-leaders-1480729586148/trump-calls-to-world-leaders-1480729586148-articleLarge.jpg}

\hypertarget{how-trumps-calls-to-world-leaders-are-upsetting-decades-of-diplomacy}{%
\subsection{How Trump's Calls to World Leaders Are Upsetting Decades of
Diplomacy}\label{how-trumps-calls-to-world-leaders-are-upsetting-decades-of-diplomacy}}

President-elect Donald J. Trump has broken with decades of diplomatic
practice in freewheeling calls with foreign leaders.

Mr. Trump's transition team has sent mixed messages about the call with
Ms. Tsai, whether it was meant as a mere gesture of good will or a
provocation aimed at drawing Taiwan closer to the United States as a way
of challenging China, which considers Taiwan a breakaway province.

Vice President-elect Mike Pence suggested in the days after the call
that Mr. Trump had merely been affording a courtesy to another
``democratically elected leader.'' But in a series of Twitter posts on
Sunday, Mr. Trump suggested a more confrontational motive,
\href{http://www.nytimes.com/2016/12/05/world/asia/china-donald-trump-taiwan-twitter.html}{criticizing
China for unfair trade practices and aggressive military moves}.

``Did China ask us if it was OK'' to take such actions, Mr. Trump asked
rhetorically, appearing to counter suggestions that the United States
must ask Beijing's permission to communicate with Taiwan.

Several senior advisers to Mr. Trump have long advocated stronger United
States support for Taiwan, arguing that it would help to counterbalance
Beijing's influence. Alexander Grey and Peter Navarro, Trump transition
advisers, wrote an article last month in
\href{http://foreignpolicy.com/2016/11/07/donald-trumps-peace-through-strength-vision-for-the-asia-pacific/}{Foreign
Policy} branding the Obama administration's treatment of Taiwan
``egregious.''

Over the weekend, Taiwan's official Central News Agency said that Edward
J. Feulner, a member of Mr. Trump's transition team and the former
president of the Heritage Foundation, a conservative think tank that
supports stronger ties with Taiwan, had played a crucial role in
bringing about the call with Mr. Trump. Mr. Feulner met with Ms. Tsai in
Taiwan in October.

Even before the phone call, Taiwan had succeeded in accomplishing
important goals with Mr. Trump's team. At their convention in Cleveland
in July, Republicans adopted a platform that for the first time
enshrined the ``six assurances'' to Taiwan made by President Ronald
Reagan in 1982, including that the United States would not set a date
for ending arms sales to the Taiwanese.

Advertisement

\protect\hyperlink{after-bottom}{Continue reading the main story}

\hypertarget{site-index}{%
\subsection{Site Index}\label{site-index}}

\hypertarget{site-information-navigation}{%
\subsection{Site Information
Navigation}\label{site-information-navigation}}

\begin{itemize}
\tightlist
\item
  \href{https://help.nytimes.com/hc/en-us/articles/115014792127-Copyright-notice}{©~2020~The
  New York Times Company}
\end{itemize}

\begin{itemize}
\tightlist
\item
  \href{https://www.nytco.com/}{NYTCo}
\item
  \href{https://help.nytimes.com/hc/en-us/articles/115015385887-Contact-Us}{Contact
  Us}
\item
  \href{https://www.nytco.com/careers/}{Work with us}
\item
  \href{https://nytmediakit.com/}{Advertise}
\item
  \href{http://www.tbrandstudio.com/}{T Brand Studio}
\item
  \href{https://www.nytimes.com/privacy/cookie-policy\#how-do-i-manage-trackers}{Your
  Ad Choices}
\item
  \href{https://www.nytimes.com/privacy}{Privacy}
\item
  \href{https://help.nytimes.com/hc/en-us/articles/115014893428-Terms-of-service}{Terms
  of Service}
\item
  \href{https://help.nytimes.com/hc/en-us/articles/115014893968-Terms-of-sale}{Terms
  of Sale}
\item
  \href{https://spiderbites.nytimes.com}{Site Map}
\item
  \href{https://help.nytimes.com/hc/en-us}{Help}
\item
  \href{https://www.nytimes.com/subscription?campaignId=37WXW}{Subscriptions}
\end{itemize}
