Sections

SEARCH

\protect\hyperlink{site-content}{Skip to
content}\protect\hyperlink{site-index}{Skip to site index}

\href{https://www.nytimes.com/section/world/canada}{Canada}

\href{https://myaccount.nytimes.com/auth/login?response_type=cookie\&client_id=vi}{}

\href{https://www.nytimes.com/section/todayspaper}{Today's Paper}

\href{/section/world/canada}{Canada}\textbar{}Canada Wonders, if U.S.
Balks, Is Carbon Pricing Still the Answer?

\url{https://nyti.ms/2hqFxSa}

\begin{itemize}
\item
\item
\item
\item
\item
\end{itemize}

Advertisement

\protect\hyperlink{after-top}{Continue reading the main story}

Supported by

\protect\hyperlink{after-sponsor}{Continue reading the main story}

\hypertarget{canada-wonders-if-us-balks-is-carbon-pricing-still-the-answer}{%
\section{Canada Wonders, if U.S. Balks, Is Carbon Pricing Still the
Answer?}\label{canada-wonders-if-us-balks-is-carbon-pricing-still-the-answer}}

\includegraphics{https://static01.nyt.com/images/2016/12/09/world/CANADA/CANADA-articleInline.jpg?quality=75\&auto=webp\&disable=upscale}

By \href{http://www.nytimes.com/by/ian-austen}{Ian Austen}

\begin{itemize}
\item
  Dec. 8, 2016
\item
  \begin{itemize}
  \item
  \item
  \item
  \item
  \item
  \end{itemize}
\end{itemize}

OTTAWA --- When Prime Minister Justin Trudeau meets on Friday with the
leaders of Canada's provinces and territories to work out a national
carbon pricing plan, Donald J. Trump will also be in the room, in a
manner of speaking.

The president-elect has expressed skepticism about climate change,
support for the fossil fuel industry and a desire to pull the United
States out of the Paris climate accord. That has raised a big question
for Canada: Can it move forward with a carbon policy if America is
headed in the opposite direction?

Mr. Trudeau certainly wants to push ahead. His government wants every
province and territory to adopt a plan to reduce carbon emissions by
putting a price on them --- either through a tax on fossil fuels or a
cap-and-trade system of emission allowances for industry. If they refuse
to do one or the other voluntarily, he has warned, the national
government will impose a plan on them.

Many of Mr. Trudeau's political rivals argue that it would be
irresponsible to move ahead now with carbon pricing in Canada if the
United States will not be doing the same thing. But carbon-tax
proponents say the opposite, that Canada could gain a competitive
advantage by acting before its much larger neighbor.

The only regional leader who has balked publicly is Brad Wall, the
premier of Saskatchewan. He has emerged as the effective head of an
informal group that says having Mr. Trump in the White House must mean
no carbon taxes in Canada.

Carbon pricing ``is very risky for us to do, especially when our biggest
trading partner is not going to do it,'' Mr. Wall said in an interview
last week. ``It does not mean that we're not concerned, or we don't want
to move on climate change. I just think there's an orthodoxy that's
grown up around carbon pricing: `If you don't care about carbon pricing,
you don't care about the earth.'''

But next door to Saskatchewan, in the oil-producing province of Alberta,
Shannon Phillips, the environment minister, is equally adamant that Mr.
Trump's election makes no difference to Canada's decision. ``On Nov. 7,
the American price on carbon was something like zero,'' she said. ``So
Nov. 8 changes very little.''

Mr. Trudeau said in early October that the provinces would have to
introduce a minimum carbon price of 10 Canadian dollars per metric ton
(about \$6.80 a ton) starting in 2018. Over the next five years, that
would quintuple.

Several provinces are already well down that road. The
\href{http://www.nytimes.com/2016/03/02/business/does-a-carbon-tax-work-ask-british-columbia.html}{carbon
emissions tax in British Columbia} is now about 30 Canadian dollars per
metric ton, and Christy Clark, the premier, has said it will be raised.
Quebec, Ontario and Manitoba have all agreed to cap-and-trade systems
linked to that of California; in Quebec, that has created an effective
carbon price of about 19 Canadian dollars a ton.

Mr. Wall, who leads the conservative Saskatchewan Party, initially
objected to carbon pricing in part to protect his province's relatively
small coal mining industry and its coal-fired power plants. The province
has been trying to deal with carbon from those plants a different way,
by physically capturing and storing it, an effort that has proved costly
and is
\href{http://www.nytimes.com/2016/03/30/business/energy-environment/technology-to-make-clean-energy-from-coal-is-stumbling-in-practice.html}{off
to a slow start.} Even so, Mr. Wall got the Trudeau government to agree
to let some of the coal-fired plants keep operating after a national
phaseout date of 2030, as long as the province substantially lowers its
overall emissions from power plants.

Now Mr. Wall's chief worry is the Bakken oil field, which straddles the
American border. If Saskatchewan has a carbon tax and the United States
does not, Mr. Wall said, the oil companies working the field ``will
literally go a mile and drill in North Dakota.''

\href{https://www.nytimes.com/interactive/2016/12/08/us/trump-climate-change.html}{}

\includegraphics{https://static01.nyt.com/images/2016/12/08/us/trump-climate-change-1481174650062/trump-climate-change-1481174650062-thumbLarge-v2.png}

\hypertarget{how-trump-can-influence-climate-change}{%
\subsection{How Trump Can Influence Climate
Change}\label{how-trump-can-influence-climate-change}}

A Trump administration could weaken or do away with many of the
Obama-era policies focused on greenhouse gas emissions.

Several economists say that worry is misplaced. Aside from ignoring
Saskatchewan's advantages over North Dakota, most notably its pipeline
network, they said Mr. Wall was exaggerating the impact of carbon taxes
on investment decisions.

For the oil industry, ``the carbon price is really a rounding error,''
said
\href{https://www.ualberta.ca/business/about/contact-us/school-directory/andrew-leach}{Andrew
Leach}, an environmental economist at the University of Alberta in
Edmonton. ``I don't mean to say that it's nothing,'' he added, noting
that some oil operations will feel its effects more than others. ``But
relative to oil prices, it's small.''

Mr. Leach, who was part of a group
\href{http://www.alberta.ca/documents/climate/climate-leadership-report-to-minister.pdf}{that
proposed climate measures} to the provincial government of Alberta,
estimates that they will add about \$2.50 to \$3 in American dollars to
the price of a barrel of crude oil from Alberta's oil sands, after
adjusting for tax write-offs.

The main opposition party in the federal Parliament, the Conservatives,
say that carbon pricing will jeopardize the future of Canadian industry,
which generally exports most of its production to the United States.

``By far, the United States is our biggest trade and investment partner,
and they are also our biggest competitor,'' said
\href{http://edfast.ca/}{Ed Fast}, a former cabinet minister who is now
the Conservative spokesman in Parliament on climate issues. ``Carbon
pricing at this time is going to be devastating to our economy. Some
people are in denial.''

Before Mr. Trudeau's election in November 2015, the Conservative-led
government under Prime Minister Stephen Harper withdrew the country from
international commitments on climate issues, saying they were
unrealistic. Mr. Trudeau campaigned on a promise to reverse that policy.

Like Mr. Wall, Mr. Fast would prefer to focus on cutting emissions with
new technology rather than with taxes or cap-and-trade. But proponents
of carbon pricing say the new technologies will only be developed and
used if carbon prices force companies to act.

Dave Sawyer, an economist with
\href{http://www.enviroeconomics.org/team}{EnviroEconomics}, a
consulting firm in Ottawa, said that long before the American election,
policy makers in Canada were assuming that the United States would not
move ahead soon on carbon pricing.

``So what has Trump changed? Not a lot,'' Mr. Sawyer said.

He noted that most of the existing provincial carbon policies were
intended to not have much effect on industrial plants. ``You're imposing
costs on a new factory,'' he said, adding that by doing so, the
provinces were pushing new factories to be more efficient and keep other
operating costs lower, potentially giving them an edge over plants in
the United States.

\href{http://www.ge.com/ca/en/about-us/leadership/elyse-allan-president-and-ceo}{Elyse
Allan}, chief executive of General Electric Canada, told an
energy-industry group shortly after the election of Mr. Trump that
carbon pricing was ``now a fact of life in Canada'' and would swiftly
force productivity improvements.

``When viewed through this lens,'' Ms. Allan said, ``carbon pricing
moves from threat to opportunity.''

Mr. Sawyer said the advent of a Trump administration would not force
Canada to change course on climate policy, but it might slow the pace a
bit.

``Canada has moved rapidly in the last two years at the federal and
provincial level at implementing a whole series of stringent policies,''
he said. ``So Canada's gotten a little bit ahead of everyone else right
now. It doesn't mean stopping, it doesn't mean getting a little more
aggressive. Canada has to pause and take stock, basically.''

Advertisement

\protect\hyperlink{after-bottom}{Continue reading the main story}

\hypertarget{site-index}{%
\subsection{Site Index}\label{site-index}}

\hypertarget{site-information-navigation}{%
\subsection{Site Information
Navigation}\label{site-information-navigation}}

\begin{itemize}
\tightlist
\item
  \href{https://help.nytimes.com/hc/en-us/articles/115014792127-Copyright-notice}{©~2020~The
  New York Times Company}
\end{itemize}

\begin{itemize}
\tightlist
\item
  \href{https://www.nytco.com/}{NYTCo}
\item
  \href{https://help.nytimes.com/hc/en-us/articles/115015385887-Contact-Us}{Contact
  Us}
\item
  \href{https://www.nytco.com/careers/}{Work with us}
\item
  \href{https://nytmediakit.com/}{Advertise}
\item
  \href{http://www.tbrandstudio.com/}{T Brand Studio}
\item
  \href{https://www.nytimes.com/privacy/cookie-policy\#how-do-i-manage-trackers}{Your
  Ad Choices}
\item
  \href{https://www.nytimes.com/privacy}{Privacy}
\item
  \href{https://help.nytimes.com/hc/en-us/articles/115014893428-Terms-of-service}{Terms
  of Service}
\item
  \href{https://help.nytimes.com/hc/en-us/articles/115014893968-Terms-of-sale}{Terms
  of Sale}
\item
  \href{https://spiderbites.nytimes.com}{Site Map}
\item
  \href{https://help.nytimes.com/hc/en-us}{Help}
\item
  \href{https://www.nytimes.com/subscription?campaignId=37WXW}{Subscriptions}
\end{itemize}
