Sections

SEARCH

\protect\hyperlink{site-content}{Skip to
content}\protect\hyperlink{site-index}{Skip to site index}

\href{https://myaccount.nytimes.com/auth/login?response_type=cookie\&client_id=vi}{}

\href{https://www.nytimes.com/section/todayspaper}{Today's Paper}

\href{/section/business/dealbook}{DealBook}\textbar{}Goldman Sachs to
Extend Its Reach in Trump Administration

\url{https://nyti.ms/2h6MSZY}

\begin{itemize}
\item
\item
\item
\item
\item
\end{itemize}

Advertisement

\protect\hyperlink{after-top}{Continue reading the main story}

Supported by

\protect\hyperlink{after-sponsor}{Continue reading the main story}

DealBook Business and Policy

\hypertarget{goldman-sachs-to-extend-its-reach-in-trump-administration}{%
\section{Goldman Sachs to Extend Its Reach in Trump
Administration}\label{goldman-sachs-to-extend-its-reach-in-trump-administration}}

\includegraphics{https://static01.nyt.com/images/2016/12/10/business/10DB-GOLDMAN/10DB-GOLDMAN-articleLarge.jpg?quality=75\&auto=webp\&disable=upscale}

By \href{http://www.nytimes.com/by/nathaniel-popper}{Nathaniel Popper},
\href{http://www.nytimes.com/by/michael-j-de-la-merced}{Michael J. de la
Merced} and \href{http://www.nytimes.com/by/maggie-haberman}{Maggie
Haberman}

\begin{itemize}
\item
  Dec. 9, 2016
\item
  \begin{itemize}
  \item
  \item
  \item
  \item
  \item
  \end{itemize}
\end{itemize}

Goldman Sachs's outsize influence in Washington is about to get larger.

The longtime second-in-command at Goldman Sachs, Gary D. Cohn, is
expected to be named director of the
\href{https://www.whitehouse.gov/administration/eop/nec}{National
Economic Council}, which oversees economic policy in the White House.

Coming fast on the heels of the nomination of
\href{https://www.nytimes.com/2016/11/29/us/politics/steven-terner-mnuchin-trump-treasury-secretary.html}{Steven
Mnuchin}, a former Goldman partner, as Treasury secretary, it will mean
that economic policy under the president-elect, Donald J. Trump, will be
shaped chiefly by veterans of the Wall Street firm.

The position that Mr. Cohn is expected to take up is one that has long
been identified with Goldman and its influence in the capital.

The role was established by President Bill Clinton and given to
Goldman's co-chairman at the time, Robert E. Rubin. Mr. Rubin's
co-chairman, Stephen Friedman, later held the position under President
George W. Bush.

This time, however, the selection of a Goldman insider is at odds with
statements made by Mr. Trump during the presidential campaign. He
repeatedly attacked the financial elite --- and Goldman Sachs in
particular.

In a commercial that ran in the closing days of the campaign, Mr. Trump
warned about ``a global power structure that is responsible for the
economic decisions that have robbed our working class, stripped our
country of its wealth and put that money into the pockets of a handful
of large corporations and political entities.'' The face of Goldman's
chief executive, Lloyd C. Blankfein --- Mr. Cohn's longtime friend and
collaborator --- was among the images that flashed ominously on the
screen.

And Mr. Trump criticized both Hillary Clinton and a primary opponent,
Senator Ted Cruz, Republican of Texas, over their ties to the investment
bank. ``I know the guys at Goldman Sachs,'' Mr. Trump said at one
primary debate. ``They have total, total control'' over Mr. Cruz, he
said. ``Just like they have total control over Hillary Clinton.''

Since his election, however, Mr. Trump has stocked his future cabinet
with a number of Goldman alumni, including Mr. Mnuchin, a hedge fund
manager and a former Goldman trader, and Stephen K. Bannon, a former
Goldman banker who is now Mr. Trump's chief strategist.

\href{https://www.nytimes.com/interactive/2016/12/05/us/politics/trump-cabinet-insiders-outsiders-millionaires.html}{}

\includegraphics{https://static01.nyt.com/images/2016/12/02/us/politics/trump-cabinet-insiders-outsiders-millionaires-1480717606838/trump-cabinet-insiders-outsiders-millionaires-1480717606838-thumbLarge-v2.png}

\hypertarget{outsiders-insiders-and-multimillionaires-in-trumps-cabinet}{%
\subsection{Outsiders, Insiders and Multimillionaires in Trump's
Cabinet}\label{outsiders-insiders-and-multimillionaires-in-trumps-cabinet}}

President-elect Donald J. Trump's cabinet and top staff are shaping up
to be a mix of wealthy Washington outsiders, Republican insiders and
former military officers who have been critical of the Obama
administration.

Unlike those two, Mr. Cohn is a longtime top Goldman executive who was
helping to guide the firm before and during the financial crisis.

In Mr. Cohn, Mr. Trump is not only turning to yet another Goldman hand
--- and a registered Democrat --- but he is also choosing a financier
whose thinking about the economy has stood in contrast to the
president-elect's more nationalistic views.

At a
\href{http://openmarkets.cmegroup.com/11728/goldman-sachs-cohn-market-looking-predictability}{conference
in Florida} soon after the election, Mr. Cohn said the big problem
facing the country and the world was a ``global growth issue.''

``We're trying to solve it with domestic policy,'' he said. ``It's not
going to work.''

While Mr. Trump has criticized companies that have moved their work
force overseas, Mr. Cohn has been candid about Goldman's international
outlook: ``We have a globalized work force, so when I need to go out and
hire the incremental worker, I go out and look around the world and see
where that incremental worker is available.''

Mr. Cohn, though, has agreed with Mr. Trump about the need to lighten
the regulations that have been imposed on banks like Goldman since the
financial crisis.

In
\href{http://video.cnbc.com/gallery/?video=3000568273\&play=1}{another
interview at the conference}, with CNBC, Mr. Cohn said he was
``cautiously optimistic'' about a Trump administration.

``We're all giving President-elect Trump and his transition team the
benefit of the doubt,'' Mr. Cohn said. ``We're all waiting to see what
happens.''

Mr. Cohn, 56, rose through the ranks of Goldman as a trader and is known
for his gruff, no-nonsense demeanor, which could help explain his
rapport with the plain-spoken Mr. Trump.

In recent years, Mr. Cohn has been in line to take over the firm from
Mr. Blankfein, and his departure will open the door to a new crop of
candidates looking to lead the firm.

While he has been registered as a Democrat, Mr. Cohn has donated to both
political parties. He has given tens of thousands of dollars to
Democrats and Democratic campaign committees.

\href{https://www.nytimes.com/interactive/2016/us/politics/donald-trump-administration.html}{}

\includegraphics{https://static01.nyt.com/images/2016/11/11/us/politics/donald-trump-administration-1478905372015/donald-trump-administration-1478905372015-square640.jpg}

\hypertarget{donald-trumps-cabinet-is-complete-heres-the-full-list}{%
\subsection{Donald Trump's Cabinet Is Complete. Here's the Full
List.}\label{donald-trumps-cabinet-is-complete-heres-the-full-list}}

A list of appointees and nominees for top posts in the new
administration.

Yet more important, he has become friends with Jared Kushner, Mr.
Trump's son-in-law and close adviser.

The decision follows an extended courtship in which the Goldman
executive visited with Mr. Trump three times, most recently on Thursday.

Leaving Wall Street to take a top government post could provide a huge
financial gain for Mr. Cohn.

He would probably have to sell his Goldman holdings to avoid conflicts
of interest with his new role, which would normally generate a big tax
bill immediately. But tax regulations allow executive branch appointees
to roll the proceeds of such a sale into Treasury securities and defer
capital gains taxes.

Mr. Cohn owned 872,712 shares in Goldman as of Nov. 14, according to
Standard \& Poor's Global Market Intelligence. As of Friday afternoon's
stock price, that stake was worth about \$209 million.

Goldman Sachs has already been a beneficiary of the coming Trump
administration. Mr. Trump has promised to push back on financial
regulations passed since the financial crisis, which have come down
particularly hard on Goldman. Since the election, shares of banks and
other financial institutions have risen sharply; Goldman's is up 34
percent.

On Friday, Goldman's shares edged up slightly.

Mr. Cohn grew up in the suburbs of Cleveland, the son of a real estate
developer and electrical contractor. He later attended Gilmour Academy,
a private school in the area, and then American University, though he
has often spoken of his struggles with dyslexia.

After a brief stint at U.S. Steel in his home state --- to appease his
father,
\href{http://www.american.edu/media/20090510_Kogod_School_of_Business_Speaker.cfm}{Mr.
Cohn said in a speech} --- he turned to the New York Mercantile Exchange
in 1983, where he eventually turned to trading silver. He was recruited
to Goldman seven years later and became a fast-rising star at the
investment bank, following the ascent of his friend Mr. Blankfein.

In 1994, Mr. Cohn joined the vaunted partnership at Goldman, in a class
that included a number of financial luminaries: Eric M. Mindich, now a
hedge fund mogul; J. Michael Evans, formerly a top Goldman executive in
China who is now at the Alibaba Group; and Mr. Mnuchin himself.

By 2006, Mr. Cohn had become co-president and co-chief operating officer
of the firm when Mr. Blankfein took the helm as chief executive. He took
sole ownership of the No. 2 spot in 2009 and solidified his role as the
heir apparent.

Over his career, the 6-foot-3 Mr. Cohn has been known for a brusque and
intimidating presence, reportedly looming over traders at their desks.
But he has softened that approach over the years as he became more of a
financial diplomat, flying to Washington, the World Economic Forum in
Davos, Switzerland, and other centers of power.

Advertisement

\protect\hyperlink{after-bottom}{Continue reading the main story}

\hypertarget{site-index}{%
\subsection{Site Index}\label{site-index}}

\hypertarget{site-information-navigation}{%
\subsection{Site Information
Navigation}\label{site-information-navigation}}

\begin{itemize}
\tightlist
\item
  \href{https://help.nytimes.com/hc/en-us/articles/115014792127-Copyright-notice}{©~2020~The
  New York Times Company}
\end{itemize}

\begin{itemize}
\tightlist
\item
  \href{https://www.nytco.com/}{NYTCo}
\item
  \href{https://help.nytimes.com/hc/en-us/articles/115015385887-Contact-Us}{Contact
  Us}
\item
  \href{https://www.nytco.com/careers/}{Work with us}
\item
  \href{https://nytmediakit.com/}{Advertise}
\item
  \href{http://www.tbrandstudio.com/}{T Brand Studio}
\item
  \href{https://www.nytimes.com/privacy/cookie-policy\#how-do-i-manage-trackers}{Your
  Ad Choices}
\item
  \href{https://www.nytimes.com/privacy}{Privacy}
\item
  \href{https://help.nytimes.com/hc/en-us/articles/115014893428-Terms-of-service}{Terms
  of Service}
\item
  \href{https://help.nytimes.com/hc/en-us/articles/115014893968-Terms-of-sale}{Terms
  of Sale}
\item
  \href{https://spiderbites.nytimes.com}{Site Map}
\item
  \href{https://help.nytimes.com/hc/en-us}{Help}
\item
  \href{https://www.nytimes.com/subscription?campaignId=37WXW}{Subscriptions}
\end{itemize}
