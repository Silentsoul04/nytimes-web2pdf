Sections

SEARCH

\protect\hyperlink{site-content}{Skip to
content}\protect\hyperlink{site-index}{Skip to site index}

\href{/section/opinion/sunday}{Sunday Review}\textbar{}Broken Men in
Paradise

\url{https://nyti.ms/2hsPzlU}

\begin{itemize}
\item
\item
\item
\item
\item
\item
\end{itemize}

\includegraphics{https://static01.nyt.com/images/2016/12/11/sunday-review/11Cohen-slide-CD4Q/11Cohen-slide-CD4Q-articleLarge-v6.jpg?quality=75\&auto=webp\&disable=upscale}

\href{/section/opinion}{Opinion}

\hypertarget{broken-men-in-paradise}{%
\section{Broken Men in Paradise}\label{broken-men-in-paradise}}

The world's refugee crisis knows no more sinister exercise in cruelty
than Australia's island prisons.

Benham Satah, an Iranian refugee, in Lorengau, on Manus Island in Papua
New Guinea.Credit...Ashley Gilbertson/VII for The New York Times

Supported by

\protect\hyperlink{after-sponsor}{Continue reading the main story}

\href{https://www.nytimes.com/by/roger-cohen}{\includegraphics{https://static01.nyt.com/images/2014/11/01/opinion/cohen-circular/cohen-circular-thumbLarge-v6.png}}

By \href{https://www.nytimes.com/by/roger-cohen}{Roger Cohen}

\begin{itemize}
\item
  Dec. 9, 2016
\item
  \begin{itemize}
  \item
  \item
  \item
  \item
  \item
  \item
  \end{itemize}
\end{itemize}

MANUS, Papua New Guinea --- The plane banks over the dense tropical
forest of Manus Island, little touched, it seems, by human hand. South
Pacific waters lap onto deserted beaches. The jungle glistens,
impenetrable. At the unfenced airport, built by occupying Japanese
forces during World War II, a sign ``welcomes you to our very beautiful
island paradise in the sun.''

It could be that, a 60-mile-long slice of heaven. But for more than 900
asylum seekers from across the world
\href{http://www.nytimes.com/2016/05/24/opinion/australias-offshore-cruelty.html}{banished}
by Australia to this remote corner of the Papua New Guinea archipelago,
Manus has been hell; a three and a half year exercise in mental and
physical cruelty conducted in near secrecy beneath the green canopy of
the tropics.

A road, newly paved by Australia as part payment to its former colony
for hosting this punitive experiment in refugee management, leads to
Lorengau, a capital of romantic name and unromantic misery. Here I find
Benham Satah, a Kurd who fled persecution in the western Iranian city of
Kermanshah. Detained on Australia's Christmas Island after crossing in a
smuggler's boat from Indonesia and later forced onto a Manus-bound
plane, he has languished here since Aug. 27, 2013.

Endless limbo undoes the mind. But going home could mean facing death:
Refugees do not flee out of choice but because they have no choice.
Satah's light brown eyes are glassy. His legs tremble. A young man with
a college degree in English, he is now nameless, a mere registration
number --- FRT009 --- to Australian officials.

``Sometimes I cut myself,'' he says, ``so that I can see my blood and
remember, `Oh, yes! I am alive.' ''

Reza Barati, his former roommate at what the men's ID badges call the
Offshore Processing Center (Orwell would be proud), is dead. A fellow
Iranian Kurd, he was killed, aged 23, on Feb. 17, 2014. Satah witnessed
the tall, quiet volleyball player being beaten to death after a local
mob scaled the wall of the facility. Protests by asylum seekers had led
to rising tensions with the Australian authorities and their Manus
enforcers.

The murder obsesses Satah but constitutes a mere fraction of the human
cost of a policy that, since July 19, 2013, has sent more than 2,000
asylum seekers and refugees to Manus and the tiny Pacific island nation
of Nauru, far from inquiring eyes. (Unable to obtain a press visa to
visit Manus, I went nonetheless.)

\includegraphics{https://static01.nyt.com/images/2016/12/11/sunday-review/11Cohen-slide-3W2W/11Cohen-slide-3W2W-articleLarge.jpg?quality=75\&auto=webp\&disable=upscale}

The toll among Burmese, Sudanese, Somali, Lebanese, Pakistani, Iraqi,
Afghan, Syrian, Iranian and other migrants is devastating:
self-immolation, overdoses, death from septicemia as a result of medical
negligence, sexual abuse and rampant despair. A recent United Nations
High Commissioner for Refugees
\href{http://www.unhcr.org/58362da34.pdf}{report} by three medical
experts found that 88 percent of the 181 asylum seekers and refugees
examined on Manus were suffering from depressive disorders, including,
in some cases, psychosis.

The world's refugee crisis, with its 65 million people on the move, more
than at any time since 1945, knows no more sustained, sinister or
surreal exercise in cruelty than the South Pacific quasi-prisons
Australia has established for its trickle of the migrant flood.

Australia, like Europe but on a much smaller scale, faces a genuine
dilemma: What to do about desperate migrants trying by any means to gain
asylum? Their journeys across the world have fueled rightist movements
in many developed societies. Anxiety, whether related to jobs or
terrorism, is high and, as Donald Trump demonstrated, scapegoating is
effective. Approaches to the crisis have varied. Angela Merkel, the
German chancellor, has taken in more than a million. But the Australian
government argues that toughness is the only way to prevent the country
from being overwhelmed.

It has ``stopped the boats'' and the Indonesian smugglers behind them:
This is the essence of Australia's case. The government says it has
prevented deaths like those in the Mediterranean, where more than 4,000
migrants have drowned this year. By turning back the ``queue jumpers,''
a phrase that resonates in a nation devoted to a ``fair go'' for all, it
has safeguarded Australia's right to select who gets to people a vast
and empty country. The official vow that those marooned on Manus and
Nauru will never live in Australia has assumed doctrinal vehemence.

In Peter Dutton, the immigration minister, the country has its own
little Trump. Last May he portrayed the asylum seekers as illiterates
bent on stealing Australian jobs, and he has suggested ``mistakes'' were
made in letting in too many Lebanese Muslim immigrants. His soft bigotry
resonates with enough voters to sway elections.

At the same time, Manus and Nauru are a growing embarrassment to
Australia, a party to all major human rights treaties. ``There is an
increasing realization that this is unsustainable,'' Madeline Gleeson,
an Australian human rights lawyer, told me.

Prime Minister Malcolm Turnbull knows this and needs a way out. After
Omid Masoumali, a young Iranian, burned himself to death on Nauru this
year, a cartoon by Cathy Wilcox captured Australia's shame. Above a man
in flames was the caption ``Not drowning.''

The result is a one-time
\href{https://www.nytimes.com/2016/11/25/world/australia/refugee-deal-united-states-nations.html?_r=0}{agreement
with the United States}, announced last month. America will, over an
unspecified period, take in an unspecified number of the refugees, with
priority going to the women, children and families who are on Nauru. The
single men on Manus would presumably bring up the rear, if accepted at
all with Trump in office.

Turnbull has said he's confident Trump will not torpedo the deal. But
when I asked Benham Satah if he thought he would soon be in the United
States, he drew on a cigarette and gazed out to sea: ``After three years
suffering here I know only this: Unless you see it, don't believe it.''

Image

An Iranian refugee praying outside a store in Lorengau, Manus
Island.Credit...Ashley Gilbertson/VII for The New York Times

In the early morning at the Lorengau covered market, another
Australian-funded project, women lay out produce and wares. Pickings are
slim: pineapples, papaya and small bunches of peanuts. Giant turtles
with prices scrawled on their bellies flap in expiration as the sun
rises and flies hover.

Betel nut has pride of place on many tables. Chewing the nut is a Manus
habit often manifested in scarlet lips and rotting teeth. Betel, a mild
stimulant, prompts what June Polomon, who works in the market, called
``our tendency to be nonstop chatterers, just like our noisy
friarbirds.''

Visiting Manus in 1928, Margaret Mead, the American anthropologist,
described a scene little changed nine decades later: ``He puts a betel
nut in his mouth, leisurely rolls a pepper leaf into a long funnel,
bites off the end, and dipping the spatula into the powdered lime, adds
a bit of lime to the mixture he is already chewing vigorously.''

As they chew, the people of Manus discuss property (familial attachment
to land is fierce), daughters' dowries and the many hundreds of asylum
seekers who --- unexpected and unexplained --- were deposited in 2013 at
the island's Lombrum naval base, originally established by United States
forces in 1944 under Gen. Douglas MacArthur.

``If Australia had cared, it would have told us something, talked to our
village leaders, who are important,'' Polomon told me. ``We've been used
in a neocolonial way.''

That is also the view of Charlie Benjamin, the Manus governor, whom I
found in indignant mood. ``It's just morally wrong to dump these people
here and then say, `Never Australia,' '' he said. ``Our understanding
was we'd help a process and genuine refugees would move on, but no
process exists.'' He described endless wrangling with the Australian
authorities over roads he believes they should pay for --- the western
half of the island is still so inaccessible the governor said it took
him six hours to drive the 50 miles to his village.

Image

Charlie Benjamin, the governor of Manus Island.Credit...Ashley
Gilbertson/VII for The New York Times

Under the money-for-migrants deal between Canberra and the Papua New
Guinea government in Port Moresby, Australia promised its former colony
hundreds of millions of dollars, but chiefly for projects outside Manus.
The 60,000 inhabitants of Manus were never consulted. Nor, of course,
were the asylum seekers and refugees. When they arrived, they had no
idea where they were. Seeing black Papuans, many thought they were in
Africa. For almost three years they were held in the detention camp,
humiliated and intermittently terrorized.

Last April, the Papua New Guinea Supreme Court ordered an end to ``the
unconstitutional and illegal detention of the asylum seekers or
transferees at the relocation center on Manus''; it was an offense
``against their rights and freedoms.'' To which Dutton, the immigration
minister, promptly
\href{http://www.nytimes.com/2016/04/27/world/australia/papua-new-guinea-asylum-seeker.html}{responded}
that nobody in Manus ``will settle in Australia.''

The only change resulting from the ruling is that refugees can now leave
the camp during the day and take buses into Lorengau.

``We're just in a bigger prison,'' Abdirahman Ahmed, a Somali refugee,
told me. The Shabab jihadi militia killed his father and brother in
Mogadishu. ``Sometimes I think maybe if I die it's better. If you die
there's no question in front of you, no interpreter between you and God,
no immigration, no Australia. We are not human, just a signpost: If you
want to come to Australia you will end up in Manus with three years of
trauma and torture.''

They are the walking dead, suspended in a dreamland, staring out at
shimmering islets. Abdul Aziz Muhamat's lips are trembling. He is from
Darfur and recalls how Sudanese government forces bound a villager's
limbs to four horses ``and they tore him up.'' The soldiers put children
in a fuel-doused hut and torched it. ``I can see it like yesterday,'' he
says.

With an uncle's help Aziz --- the name he now uses --- fled: Khartoum
airport, Yemenia Airways Flight 632 (``I still remember the number'') to
Sana, on to Dubai, and from there to Jakarta. He is met by a Sudanese
man who whisks him south to Bogor, where he hides in a house with
Iranians, Pakistanis, Burmese and others. It is mid-August 2013.

They move on by truck at night, then paddle in canoes to an island, and
board a rickety boat crammed with 50 people. ``I asked where we were
going,'' Aziz tells me, ``and this guy said Australia.'' But after 12
hours at sea, with the boat foundering, they turn back. Five people
drown.

When Aziz tries again in October, his boat is intercepted by the
Australian Navy and he is thrown into a detention center on Christmas
Island with more than 40 others. Finally an Australian immigration
officer tells them they will be flown to Manus, ``a very dangerous place
full of contagious diseases --- if you touch a local, sanitize
yourself.''

``I have a question,'' Aziz says.

``That's it. I cannot answer questions,'' says the immigration officer.

Image

Abdul Aziz Muhamat, right, and Behrouz Boochani, a refugee from Iran.
``We are here because of all Australia, all the people who are silent,
who have done nothing,'' Mr. Boochani said.Credit...Ashley
Gilbertson/VII for The New York Times

``If you know these things exist on Manus Island, why do you want to
send us there?''

Aziz says he's in a cage. The whole island is a cage. Then he says he's
in a hole. He has no feelings, no desire. There's no point asking why.
It's been too long. At first conditions in the detention center are
primitive, hundreds of men crammed into makeshift compounds or tents,
scant food, bullying expat staff contracted by Australia, constant
threats from a special Papuan police riot squad flown in at Australian
expense --- and no information, no ``process.'' Nothing.

Frustration boils up in early 2014. For weeks, there are peaceful
protests every evening, chants of ``Freedom.'' But they have no effect,
and the asylum seekers are told that ``processing'' could take a decade:
Kafka in the tropics. Anger turns to rage. Two Iranians try to escape
and are beaten up. Local thugs with machetes and bush knives, drunk on
moonshine, goaded and abetted by some international security staff, pile
into the camp. Shots are fired. Reza Barati is killed. Aziz, his toe
broken, finds himself in the clinic among ``170 guys lying on concrete,
some conscious, some unconscious, bodies full of blood. I thought I was
back in Darfur.''

Dump men in the middle of nowhere, confine them, abuse them, suspend
them in limbo, and this is what you get.

The riot changes nothing.

Image

A man injured by a drunken mob, seen in a photo on another refugee's
phone.Credit...Ashley Gilbertson/VII for The New York Times

A year later, in January 2015, hundreds of men begin a hunger strike.
Several sew their mouths shut. The strike persists for two weeks. The
authorities break it by throwing Aziz, Benham Satah and others into
solitary confinement in windowless containers known as the ``chauka''
(named after a bird unique to Manus).

The hunger strike changes nothing.

Australia has relied on the remoteness and secrecy of its program: out
of sight, out of mind.
\href{https://storify.com/tariromze/getting-started}{Keep the press
out}. Impose draconian nondisclosure clauses in contracts for everyone
who works there. Even pass a federal law that can send whistle-blowers
to prison. On the whole, it has worked.

Still, the ugliness is beginning to seep out. In 41 months these
stranded men have had only two pieces of good news: the Papua New Guinea
Supreme Court ruling and now the Australian deal with the United States.

``The deal represents a long overdue concession from the government that
it cannot leave people on Manus and Nauru forever,'' Daniel Webb, a
lawyer at the Melbourne Human Rights Law Center, told me. ``That
concession is way overdue, but it does not end their suffering.''

Aziz, a smart young man who now has dreams of becoming a human rights
lawyer, said the policy is ``not about stopping boats. I think it's
about using innocent people as political tools to win elections.''

Moving the asylum seekers elsewhere to be processed was not in itself
unlawful, so long as the process was fair and efficient and met basic
human rights standards. There should have been explanatory sessions with
the local authorities, clarity over who was running facilities, zero
detention and an Australian-led regional effort to secure a decent life
for the refugees. None of this occurred.

Instead, Australia, briefly under a Labor Party government and then
under the conservatives, has effectively argued that the end
(discouraging human smuggling) justifies the means (cruelty). As Hugh
Mackay, a social researcher, observed, this is ``the very same principle
used to justify torture.'' And even so, boats are still being turned
around by a huge naval deployment.

A strange hysteria about the ``boat people'' seems to have blinded
Australia to what is being perpetrated in its name. The country was
founded on a similar principle to ``offshore processing'': Britain's
dispatch in the late 18th century of convicts to a faraway land in
Oceania, where they, too, would be invisible.

Its subsequent history has included the slaughter and incarceration of
the native Aboriginal people; the White Australia policy, under which a
vast land mass was seen as threatened by black people and other
nonwhites emanating from places like Papua New Guinea; the ``stolen
generation'' policy, under which tens of thousands of Aboriginal
children were taken from their families and placed in white homes; and
now this disgraceful consignment of asylum seekers, many of them
dark-skinned and Muslim, to faraway islands where they are left to
fester with the ``natives.''

``Australians have a tendency to feel vulnerable,'' Amelia Lester, the
editor of the magazine Good Weekend, told me. ``We're so far from
anywhere, it breeds a kind of paranoia.''

Just 24 million people live in Australia, a country twice the size of
India, where 1.25 billion live. Might there be room to squeeze in 2,000
more? Australia has not known a recession in a quarter-century. Perhaps
it is hard to imagine what humiliation and despair are. But it is time
to imagine; they are right here, across the water.

``Whatever the policy challenge, deliberate cruelty to thousands of
innocent people is never the solution,'' Webb told me.

One measure of the government's obsession is that it has introduced
legislation to impose a lifetime ban from Australia for anyone held at
one of the camps. So in theory, a man from Manus could go to the United
States under the recently announced deal, become a Harvard professor,
and never be able to visit Sydney.

Another is that it insists that the roughly 370 people moved from Manus
or Nauru to Australia as ``transitory persons'' because they were
injured in riots, or sexually assaulted, or were dying, or pregnant, or
had broken down (like the wife of the Iranian who self-immolated) cannot
stay in Australia. If they want to be considered for the American deal,
they would have to return to one of the islands to be ``processed.'' The
``transitory persons'' include about 40 children. This is madness.

Lynne Elworthy, a mental health nurse, is one Australian who knows the
agony of Manus and Nauru. She's worked on both islands, and spoke to me
in brave defiance of the nondisclosure rules meant to gag her. ``Some
cope better, focus on gym and seem to do O.K.,'' she said. ``But many
men in Manus are withdrawn, skinny, depressed and worn out, hopeless,
with plummeting lows. It's quite obvious to see this. They exist in a
lifeless pit.''

She continued: ``Apart from the way the whites treated the Aborigines
when they first arrived --- that was worse --- this will come in second
by the time Manus and Nauru are considered for their absolute cruelty. I
imagine one day a royal commission will look into the illegal
imprisonment, the damage caused, the agony and the injury.''

On my last day in Manus I managed to get through the navy checkpoint at
the entrance to the camp. Rain was falling heavily. I drove past General
MacArthur's old house, and an American-built church, and down to the
high metal fences and barbed wire. Dozens of Australian border guards
were exercising in a field. Jeeps and white S.U.V.s splashed by. I saw
the barracks --- Oscar and Delta and Mike and Foxtrot --- and by now it
was easy to imagine the suffering endured within.

Behrouz Boochani, another Iranian refugee, had broken down in front of
me a couple of days earlier, crying uncontrollably. ``I can't sleep,''
he said. ``I want justice,'' he said. ``I have one million pages of
incriminating documents,'' he said. Emaciated, with pale green eyes, a
ponytail and beard, he was a broken but still determined man: ``We are
here because of all Australia, all the people who are silent, who have
done nothing.''

Among the refugees is Nayser Ahmed, a Rohingya who fled persecution in
Myanmar on July 2, 2013. Now 63, he made his way to Indonesia with his
wife and six children. But when they boarded the bus to go to the boat,
he was unable to squeeze in alongside. His family reached Australia
before the imposition of the Manus and Nauru policy, and now live in
Sydney. He did not. Every effort to be reunited with his family since he
arrived in Manus on Nov. 15, 2013, has failed.

Ahmed's nose and ribs were broken in the 2014 riot. A daughter got
married in Sydney two years ago; he told her to stay well and not think
too much. He blames himself for missing the bus.

``I think all the time about what happened,'' he told me. ``When I close
my eyes, I can see that bus leaving.'' He said he was ``shouting and
screaming, `My family is gone, someone help me!' ''

What is incumbent on Australia now is clear enough. Prevail on Trump to
take as many of the refugees as possible. Reunite Nayser Ahmed with his
family. Recognize that the country has incurred a moral debt to the
myriad people it has mistreated on the islands and allow those who do
not go to the United States to build a decent life in Australia. Make
the ``transitory persons'' already in Australia permanent residents.
Close this foul chapter that stains Australia and echoes the darkest
moments in its history.

Aziz had been reading Mandela's biography. One of these men, allowed a
chance, might yet make Australia proud.

Advertisement

\protect\hyperlink{after-bottom}{Continue reading the main story}

\hypertarget{site-index}{%
\subsection{Site Index}\label{site-index}}

\hypertarget{site-information-navigation}{%
\subsection{Site Information
Navigation}\label{site-information-navigation}}

\begin{itemize}
\tightlist
\item
  \href{https://help.nytimes.com/hc/en-us/articles/115014792127-Copyright-notice}{©~2020~The
  New York Times Company}
\end{itemize}

\begin{itemize}
\tightlist
\item
  \href{https://www.nytco.com/}{NYTCo}
\item
  \href{https://help.nytimes.com/hc/en-us/articles/115015385887-Contact-Us}{Contact
  Us}
\item
  \href{https://www.nytco.com/careers/}{Work with us}
\item
  \href{https://nytmediakit.com/}{Advertise}
\item
  \href{http://www.tbrandstudio.com/}{T Brand Studio}
\item
  \href{https://www.nytimes.com/privacy/cookie-policy\#how-do-i-manage-trackers}{Your
  Ad Choices}
\item
  \href{https://www.nytimes.com/privacy}{Privacy}
\item
  \href{https://help.nytimes.com/hc/en-us/articles/115014893428-Terms-of-service}{Terms
  of Service}
\item
  \href{https://help.nytimes.com/hc/en-us/articles/115014893968-Terms-of-sale}{Terms
  of Sale}
\item
  \href{https://spiderbites.nytimes.com}{Site Map}
\item
  \href{https://help.nytimes.com/hc/en-us}{Help}
\item
  \href{https://www.nytimes.com/subscription?campaignId=37WXW}{Subscriptions}
\end{itemize}
