Sections

SEARCH

\protect\hyperlink{site-content}{Skip to
content}\protect\hyperlink{site-index}{Skip to site index}

\href{https://www.nytimes.com/section/world/asia}{Asia Pacific}

\href{https://myaccount.nytimes.com/auth/login?response_type=cookie\&client_id=vi}{}

\href{https://www.nytimes.com/section/todayspaper}{Today's Paper}

\href{/section/world/asia}{Asia Pacific}\textbar{}After Park, Who? A
Guide to Those Who Would Lead South Korea

\url{https://nyti.ms/2hrNjvb}

\begin{itemize}
\item
\item
\item
\item
\item
\end{itemize}

Advertisement

\protect\hyperlink{after-top}{Continue reading the main story}

Supported by

\protect\hyperlink{after-sponsor}{Continue reading the main story}

\hypertarget{after-park-who-a-guide-to-those-who-would-lead-south-korea}{%
\section{After Park, Who? A Guide to Those Who Would Lead South
Korea}\label{after-park-who-a-guide-to-those-who-would-lead-south-korea}}

By \href{http://www.nytimes.com/by/choe-sang-hun}{Choe Sang-Hun}

\begin{itemize}
\item
  Dec. 9, 2016
\item
  \begin{itemize}
  \item
  \item
  \item
  \item
  \item
  \end{itemize}
\end{itemize}

SEOUL, South Korea --- The South Korean Parliament has voted to impeach
President Park Geun-hye, which suspends her from office while the
Constitutional Court decides whether to formally remove her. These are
some of the contenders to replace her, and one who will take her place
temporarily.

Image

Hwang Kyo-ahn.Credit...Luka Conzalez/Agence France-Presse --- Getty
Images

\textbf{HWANG KYO-AHN, 59}

\textbf{Prime minister}

As the No. 2 official in government, Mr. Hwang will serve as acting
president while Ms. Park is suspended. A staunch defender of Ms. Park
and perhaps the second least popular official in government, there is
little chance he would be more than a caretaker until the court decides
to reinstate Ms. Park or an election is held for her successor. A former
prosecutor, he was first appointed by Ms. Park as her justice minister,
a position he used to
\href{https://www.nytimes.com/2014/12/20/world/asia/south-korea-disbands-united-progressive-party-sympathetic-to-north-korea.html}{dismantle
a small, left-wing party} on charges of subscribing to North Korean
ideology.

Image

Moon Jae-in.Credit...Chung Sung-Jun/Getty Images

\textbf{MOON JAE-IN, 63}

\textbf{Democratic Party leader}

As the likely presidential candidate of the main opposition party, the
liberal Mr. Moon tops opinion surveys of potential replacements for Ms.
Park. As a student activist in the 1970s, Mr. Moon was jailed for
opposing the dictatorship of Ms. Park's father, President Park
Chung-hee. He had been a friend and ally of President Roh Moo-hyun since
the 1980s, when they worked together as human rights lawyers defending
students and labor activists persecuted under the military dictatorship.
When Mr. Roh became president, Mr. Moon became his chief of staff. Mr.
Moon ran for president in 2012,
\href{http://www.nytimes.com/2012/12/20/world/asia/south-koreans-vote-in-closely-fought-presidential-race.html}{losing
narrowly to Ms. Park}, which his supporters say was partly because of a
\href{http://www.nytimes.com/2015/02/10/world/asia/former-spy-chief-in-south-korea-sentenced-in-election-case.html}{clandestine
smear campaign} by the government's main intelligence agency.

Mr. Moon supports the country's alliance with Washington, but he has
argued that it needs a more ``balanced diplomacy'' with the United
States and China. His party has criticized the current approach on North
Korea, saying sanctions alone will not end the
\href{http://www.nytimes.com/topic/subject/north-koreas-nuclear-program}{North's
nuclear weapons program}. Many of its members have opposed Ms. Park's
decision to deploy an
\href{http://www.nytimes.com/2016/07/08/world/asia/south-korea-and-us-agree-to-deploy-missile-defense-system.html}{American
missile defense battery} in South Korea. Mr. Moon has not said whether
he would reverse the decision if he became president but has said it
should be subject to parliamentary approval.

Image

Ban Ki-moon.Credit...Mosa'Ab Elshamy/Associated Press

\textbf{BAN KI-MOON, 72}

\textbf{United Nations secretary general}

The departing secretary general is popular in South Korea and is widely
expected to declare his presidential ambition after his term at the
United Nations ends on Dec. 31. There has been speculation that one
reason Ms. Park has refused to step down immediately is to give Mr. Ban,
a conservative, time to prepare for an election while preventing a
progressive like Mr. Moon from taking power. Mr. Ban was a career South
Korean diplomat before winning the top United Nations post in 2007, a
momentous event for South Koreans who saw it as an affirmation of the
country's rising stature. He is considered a supporter of the
partnership with the United States but has not commented on specific
South Korean policies. He is also untested in the cut-and-thrust world
of electoral politics. Another obstacle: He has no party affiliation. He
had been expected to join Ms. Park's party, Saenuri, but its popularity
has plunged in the wake of her scandal. He is now expected to form an
alliance with some members of her party or another party, or he may
start his own.

Image

Lee Jae-myung.Credit...Yonhap/European Pressphoto Agency

\textbf{LEE JAE-MYEONG, 51}

\textbf{Mayor of Seongnam}

A rising star among progressives, Mr. Lee calls himself the
\href{http://www.nytimes.com/topic/person/bernard-sanders}{Bernie
Sanders} of South Korea. But he is more like President-elect Donald J.
Trump in one respect: He uses Twitter, too. He has a huge audience there
and has used his pointed comments to attack Ms. Park and her policies.
He was one of the first major politicians
\href{http://www.nytimes.com/2016/10/31/world/asia/south-korea-choi-soon-sil.html}{to
address the crowds} of antigovernment demonstrators who have filled
central Seoul on recent weekends. He says that Ms. Park ``should be
handcuffed'' on criminal charges the moment she leaves office. Like the
other progressives listed here, his positions on the approach to North
Korea and missile defense are similar to Mr. Moon's. Mr. Lee worked in a
factory as a teenager and did not attend high school, but he taught
himself, winning admission to a college and later passing the bar exam.
Before running for mayor, he worked as a lawyer defending labor
activists and political dissidents.

Image

Ahn Cheol-soo.Credit...Chung Sung-Jun/Getty Images

\textbf{AHN CHEOL-SOO, 54}

\textbf{People's Party leader}

A millionaire software mogul who leads a small opposition party, Mr. Ahn
became
\href{http://www.nytimes.com/2011/11/20/world/asia/a-new-voice-grips-south-korea-with-plain-talk-about-inequality-and-justice.html}{a
political star} for his plain talk about equality and justice and his
searing criticism of the existing political parties and big business.
``Bill Gates wouldn't have become Bill Gates if he were born in South
Korea,'' Mr. Ahn once said, accusing Samsung, Hyundai and other major
corporations of creating ``zoos'' where they have shackled small
entrepreneurs with slavelike contracts. Once considered a top contender
for the 2012 election, he withdrew his candidacy, throwing his support
behind Mr. Moon, with whom he has since parted ways. A medical doctor by
training, Mr. Ahn made a fortune developing antivirus computer software.
He says he wants to heal a country disillusioned with what he calls a
corrupt and out-of-touch political and corporate elite.

Image

Park Won-soon.Credit...Ed Jones/Agence France-Presse --- Getty Images

\textbf{PARK WON-SOON, 60}

\textbf{Mayor of Seoul}

As
\href{http://www.nytimes.com/2011/10/27/world/asia/vote-on-seoul-mayor-seen-as-having-wider-implications.html}{mayor
of the capital}, Mr. Park is considered the second most powerful elected
official in South Korea after the president. A former human rights
lawyer, he is seen as a leader of the civil society movement and founded
the country's most influential civil and political rights group. He has
won many landmark legal cases, including South Korea's first sexual
harassment conviction. He also campaigned for the rights of so-called
\href{http://www.nytimes.com/2015/12/29/world/asia/comfort-women-south-korea-japan.html}{comfort
women}, Korean sex slaves who were lured or forced into working in
brothels for the Japanese Army during World War II. A tireless critic of
what he calls growing social and economic inequality, he has pulled no
punches in attacking Ms. Park, supporting huge rallies against her. Last
month, he showed up at a meeting of cabinet ministers and shouted at
them to ``choose between the people and the president.''

Advertisement

\protect\hyperlink{after-bottom}{Continue reading the main story}

\hypertarget{site-index}{%
\subsection{Site Index}\label{site-index}}

\hypertarget{site-information-navigation}{%
\subsection{Site Information
Navigation}\label{site-information-navigation}}

\begin{itemize}
\tightlist
\item
  \href{https://help.nytimes.com/hc/en-us/articles/115014792127-Copyright-notice}{©~2020~The
  New York Times Company}
\end{itemize}

\begin{itemize}
\tightlist
\item
  \href{https://www.nytco.com/}{NYTCo}
\item
  \href{https://help.nytimes.com/hc/en-us/articles/115015385887-Contact-Us}{Contact
  Us}
\item
  \href{https://www.nytco.com/careers/}{Work with us}
\item
  \href{https://nytmediakit.com/}{Advertise}
\item
  \href{http://www.tbrandstudio.com/}{T Brand Studio}
\item
  \href{https://www.nytimes.com/privacy/cookie-policy\#how-do-i-manage-trackers}{Your
  Ad Choices}
\item
  \href{https://www.nytimes.com/privacy}{Privacy}
\item
  \href{https://help.nytimes.com/hc/en-us/articles/115014893428-Terms-of-service}{Terms
  of Service}
\item
  \href{https://help.nytimes.com/hc/en-us/articles/115014893968-Terms-of-sale}{Terms
  of Sale}
\item
  \href{https://spiderbites.nytimes.com}{Site Map}
\item
  \href{https://help.nytimes.com/hc/en-us}{Help}
\item
  \href{https://www.nytimes.com/subscription?campaignId=37WXW}{Subscriptions}
\end{itemize}
