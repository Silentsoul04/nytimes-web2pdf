Sections

SEARCH

\protect\hyperlink{site-content}{Skip to
content}\protect\hyperlink{site-index}{Skip to site index}

\href{https://www.nytimes.com/section/politics}{Politics}

\href{https://myaccount.nytimes.com/auth/login?response_type=cookie\&client_id=vi}{}

\href{https://www.nytimes.com/section/todayspaper}{Today's Paper}

\href{/section/politics}{Politics}\textbar{}Trump, Mocking Claim That
Russia Hacked Election, at Odds with G.O.P.

\url{https://nyti.ms/2hxJ2Gz}

\begin{itemize}
\item
\item
\item
\item
\item
\end{itemize}

Advertisement

\protect\hyperlink{after-top}{Continue reading the main story}

Supported by

\protect\hyperlink{after-sponsor}{Continue reading the main story}

\hypertarget{trump-mocking-claim-that-russia-hacked-election-at-odds-with-gop}{%
\section{Trump, Mocking Claim That Russia Hacked Election, at Odds with
G.O.P.}\label{trump-mocking-claim-that-russia-hacked-election-at-odds-with-gop}}

\includegraphics{https://static01.nyt.com/images/2016/12/11/us/11INTEL-01/11INTEL-01-articleInline.jpg?quality=75\&auto=webp\&disable=upscale}

By \href{http://www.nytimes.com/by/david-e-sanger}{David E. Sanger}

\begin{itemize}
\item
  Dec. 10, 2016
\item
  \begin{itemize}
  \item
  \item
  \item
  \item
  \item
  \end{itemize}
\end{itemize}

WASHINGTON --- An extraordinary breach has emerged between
President-elect Donald J. Trump and the national security establishment,
with Mr. Trump mocking American intelligence assessments that Russia
interfered in the election on his behalf, and top Republicans vowing
investigations into Kremlin activities.

On Saturday, intelligence officials said it was not until the week after
the election that the C.I.A. altered its formal assessment of Russia's
activities to conclude that the government of President Vladimir V.
Putin was not just trying to undermine the election, but had also acted
to give one candidate an advantage.

Wary of being seen as politicizing their findings, C.I.A. analysts had
been reluctant to come to that conclusion in the midst of the election
--- even as many supporters of Hillary Clinton believed it was obvious,
given the leak of emails from her campaign chairman and others.

One intelligence official said there were indications in early October
that the Russians had shifted their focus to harm Mrs. Clinton. The
C.I.A.'s slowness in shifting its assessment, another official said, was
one reason President Obama
\href{https://www.nytimes.com/2016/12/09/us/obama-russia-election-hack.html}{ordered
a full review} of ``lessons learned'' on the operation to influence the
election.

But the disclosure of the still-classified findings prompted a
blistering attack against the intelligence agencies by Mr. Trump, whose
transition office said in a statement on Friday night that ``these are
the same people that said Saddam Hussein had weapons of mass
destruction,'' adding that the election was over and that it was time to
``move on.''

Mr. Trump has split on the issue with many Republicans on the
congressional intelligence committees, who have said they were presented
with significant evidence, in closed briefings, of a Russian campaign to
meddle in the election.

The rift also raises questions about how Mr. Trump will deal with the
intelligence agencies he will have to rely on for analysis of China,
Russia and the Middle East, as well as for covert drone and
cyberactivities.

At this point in a transition, a president-elect is usually delving into
intelligence he has never before seen, and learning about C.I.A. and
National Security Agency abilities. But Mr. Trump, who has taken
intelligence briefings only sporadically, is questioning not only
analytic conclusions, but also their underlying facts.

``To have the president-elect of the United States simply reject the
fact-based narrative that the intelligence community puts together
because it conflicts with his
\href{https://www.merriam-webster.com/dictionary/a\%20priori}{a priori}
assumptions --- wow,'' said Michael V. Hayden, who was the director of
the N.S.A. and later the C.I.A. under President George W. Bush.

With the partisan emotions on both sides --- Mr. Trump's supporters see
a plot to undermine his presidency, and Mrs. Clinton's supporters see a
conspiracy to keep her from the presidency --- the result is an
environment in which even those basic facts become the basis for
dispute.

Mr. Trump's team lashed out at the agencies after
\href{https://www.washingtonpost.com/world/national-security/obama-orders-review-of-russian-hacking-during-presidential-campaign/2016/12/09/31d6b300-be2a-11e6-94ac-3d324840106c_story.html?utm_term=.9195029a5343}{The
Washington Post reported} that the C.I.A. believed that Russia had
intervened to undercut Mrs. Clinton and lift Mr. Trump, and The New York
Times reported that Russia had broken into Republican National Committee
computer networks just as they had broken into Democratic ones, but had
released documents only on the Democrats.

For months, the president-elect has strenuously rejected all assertions
that Russia was working to help him, though he did at one point
\href{https://www.nytimes.com/2016/07/28/us/politics/donald-trump-russia-clinton-emails.html}{invite
Russia to find} thousands of Mrs. Clinton's emails. There is no evidence
that the Russian meddling affected the outcome of the election or the
legitimacy of the vote, but Mr. Trump and his aides want to shut the
door on any such notion, including the idea that Mr. Putin schemed to
put him in office.

Instead, Mr. Trump casts the issue as an unknowable mystery. ``It could
be Russia,'' he recently told Time magazine. ``And it could be China.
And it could be some guy in his home in New Jersey.''

The Republicans who lead the congressional committees overseeing
intelligence, the Pentagon and the Department of Homeland Security take
the opposite view. They say that Russia was behind the election
meddling, but that the scope and intent of the operation need deep
investigation, hearings and public reports.

One question they may want to explore is why the intelligence agencies
believe that the Republican networks were compromised while the F.B.I.,
which leads domestic cyberinvestigations, has apparently told
Republicans that it has not seen evidence of that breach. Senior
officials say the intelligence agencies' conclusions are not being
widely shared, even with law enforcement.

``We cannot allow foreign governments to interfere in our democracy,''
Representative Michael McCaul, a Texas Republican who is the chairman of
the Homeland Security Committee and was considered by Mr. Trump for
secretary of Homeland Security, said at the conservative Heritage
Foundation. ``When they do, we must respond forcefully, publicly and
decisively.''

He has promised hearings, saying the Russian activity was ``a call to
action,'' as has Senator John McCain of Arizona, one of the few senators
left from the Cold War era, when the Republican Party made opposition to
the Soviet Union --- and later deep suspicion of Russia --- the
centerpiece of its foreign policy.

Representative Peter T. King, Republican of New York and a member of the
House Intelligence Committee, said there was little doubt that the
Russian government was involved in
\href{https://www.nytimes.com/2016/06/15/us/politics/russian-hackers-dnc-trump.html}{hacking
the Democratic National Committee}. ``All of the intelligence analysts
who looked at it came to the conclusion that the tradecraft was very
similar to the Russians,'' he said.

Even one of Mr. Trump's most enthusiastic supporters, Representative
Devin Nunes, Republican of California, said on Friday that he had no
doubt about Russia's culpability. His complaint was with the
intelligence agencies, which he said had ``repeatedly'' failed ``to
anticipate Putin's hostile actions,'' and with the Obama
administration's lack of a punitive response.

Mr. Nunes, the chairman of the House Intelligence Committee, said that
the intelligence agencies had ``ignored pleas by numerous Intelligence
Committee members to take more forceful action against the Kremlin's
aggression.'' He added that the Obama administration had ``suddenly
awoken to the threat.''

Like many Republicans, Mr. Nunes is threading a needle. His statement
puts him in opposition to the position taken by Mr. Trump and his
incoming national security adviser, Michael Flynn, who has traveled to
Russia as a private citizen for RT, the state-controlled news operation,
and attended a dinner with Mr. Putin.

Mr. Nunes's contention that Mr. Obama was captivated by a desire to
``reset'' relations with Russia is also notable, because Mr. Trump has
said he is trying to do the same --- though he is avoiding that term,
which was made popular by Mrs. Clinton in her failed effort as secretary
of state in 2009.

There are splits both within the intelligence agencies and the
congressional committees that oversee them. Officials say the C.I.A. and
the N.S.A. have not always shared their findings with the F.B.I., which
they often distrust. The question of how vigorously to investigate also
has a political tinge: Democrats on the Senate Intelligence Committee,
for example, are pushing hard for a broad investigation, while some
Republicans are resisting.

Intelligence can also get politicized, of course, and one of the running
debates about the disastrously mistaken assessments of Iraq that Mr.
Trump often cites is whether the intelligence itself was tainted or
whether the Bush White House read it selectively to support its march to
war in 2003.

But what is unfolding in the argument over the Russian hacking is more
complex, because tracking the origin of cyberattacks is complicated. It
is made all the harder by the fact that the C.I.A. and the N.S.A. do not
want to reveal human sources or technical abilities, including American
software implants in Russian computer networks.

This much is known: In mid-2015, a hacking group long associated with
the F.S.B. --- the successor to the old Soviet K.G.B. --- got inside the
Democratic National Committee's computer systems. The intelligence
gathering appeared to be fairly routine, and it was unsurprising: The
Chinese, for instance, penetrated Mr. Obama's and Mr. McCain's
presidential campaign communications in 2008.

In the spring of 2016, a second group of Russian hackers, long
associated with the G.R.U., a military intelligence agency, attacked the
D.N.C. again, along with the private email accounts of prominent
Washington figures like John D. Podesta, the chairman of Mrs. Clinton's
campaign. Those emails were ultimately published --- a step the Russians
had never taken before in the United States, though the tactic has been
used often in former Soviet states and elsewhere in Europe. That moved
the issue from espionage to an ``information operation'' with a
political motive.

One person who attended a classified briefing on the intelligence said
that the investigators had explained that the malware used in the
cyberattack on the D.N.C. matched tools previously used by hackers with
proven ties to the Russian government. That sort of ``pattern analysis''
is common in cyberinvestigations, though it is not conclusive.

But the intelligence agencies had more: They had managed to identify the
individuals from the G.R.U. who oversaw the hacking efforts. That may
have come from intercepted conversations, spying efforts, or implants in
computer systems that allow the tracking of emails and text messages.

In briefings to Mr. Obama and on Capitol Hill, intelligence agencies
have said they now believe that what began as an effort to undermine the
credibility of American elections morphed over time into a much more
targeted effort to harm Mrs. Clinton, whom Mr. Putin has long accused of
interfering in Russian parliamentary elections in 2011.

But to hedge their bets before the election, according to the briefings,
the Russians also targeted the Republican National Committee, Republican
operatives and prominent members of the Republican establishment, like
former Secretary of State Colin L. Powell. However, few of those emails
have ever surfaced, save for Mr. Powell's, which were critical of Mrs.
Clinton's campaign for trying to draw him into a defense of her use of a
private computer server.

A spokesman for the Republican National Committee, Sean Spicer, disputed
the report in The Times that the intelligence community had concluded
that the R.N.C. had been hacked.

``The RNC was not `hacked,''' he said on Twitter. ``The
\href{https://twitter.com/nytimes}{@nytimes} was told and chose to
ignore.'' On Friday night, before The Times published its report, the
committee had refused to comment.

Advertisement

\protect\hyperlink{after-bottom}{Continue reading the main story}

\hypertarget{site-index}{%
\subsection{Site Index}\label{site-index}}

\hypertarget{site-information-navigation}{%
\subsection{Site Information
Navigation}\label{site-information-navigation}}

\begin{itemize}
\tightlist
\item
  \href{https://help.nytimes.com/hc/en-us/articles/115014792127-Copyright-notice}{©~2020~The
  New York Times Company}
\end{itemize}

\begin{itemize}
\tightlist
\item
  \href{https://www.nytco.com/}{NYTCo}
\item
  \href{https://help.nytimes.com/hc/en-us/articles/115015385887-Contact-Us}{Contact
  Us}
\item
  \href{https://www.nytco.com/careers/}{Work with us}
\item
  \href{https://nytmediakit.com/}{Advertise}
\item
  \href{http://www.tbrandstudio.com/}{T Brand Studio}
\item
  \href{https://www.nytimes.com/privacy/cookie-policy\#how-do-i-manage-trackers}{Your
  Ad Choices}
\item
  \href{https://www.nytimes.com/privacy}{Privacy}
\item
  \href{https://help.nytimes.com/hc/en-us/articles/115014893428-Terms-of-service}{Terms
  of Service}
\item
  \href{https://help.nytimes.com/hc/en-us/articles/115014893968-Terms-of-sale}{Terms
  of Sale}
\item
  \href{https://spiderbites.nytimes.com}{Site Map}
\item
  \href{https://help.nytimes.com/hc/en-us}{Help}
\item
  \href{https://www.nytimes.com/subscription?campaignId=37WXW}{Subscriptions}
\end{itemize}
