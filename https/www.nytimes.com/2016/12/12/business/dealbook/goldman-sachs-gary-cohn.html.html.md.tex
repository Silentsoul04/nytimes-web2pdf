Sections

SEARCH

\protect\hyperlink{site-content}{Skip to
content}\protect\hyperlink{site-index}{Skip to site index}

\href{https://myaccount.nytimes.com/auth/login?response_type=cookie\&client_id=vi}{}

\href{https://www.nytimes.com/section/todayspaper}{Today's Paper}

\href{/section/business/dealbook}{DealBook}\textbar{}Goldman President
Named Trump Adviser, Opening Door for Younger Executives

\url{https://nyti.ms/2gzuuKm}

\begin{itemize}
\item
\item
\item
\item
\item
\end{itemize}

Advertisement

\protect\hyperlink{after-top}{Continue reading the main story}

Supported by

\protect\hyperlink{after-sponsor}{Continue reading the main story}

DealBook Business and Policy

\hypertarget{goldman-president-named-trump-adviser-opening-door-for-younger-executives}{%
\section{Goldman President Named Trump Adviser, Opening Door for Younger
Executives}\label{goldman-president-named-trump-adviser-opening-door-for-younger-executives}}

\includegraphics{https://static01.nyt.com/images/2016/12/13/business/13DB-GOLDMAN/13DB-GOLDMAN-articleInline-v3.jpg?quality=75\&auto=webp\&disable=upscale}

By \href{http://www.nytimes.com/by/nathaniel-popper}{Nathaniel Popper}

\begin{itemize}
\item
  Dec. 12, 2016
\item
  \begin{itemize}
  \item
  \item
  \item
  \item
  \item
  \end{itemize}
\end{itemize}

The jockeying to be the next chief executive of Goldman Sachs is
expected to intensify after President-elect Donald J. Trump officially
asked the firm's president, Gary D. Cohn, to become his top economic
policy adviser on Monday.

Mr. Cohn's departure from the top ranks of the Wall Street firm removes
a crucial impediment for the next generation of Goldman leaders because
he was seen as the natural heir to the top spot. Mr. Cohn is among a
handful of senior executives who have held sway atop Goldman since
before the financial crisis, and he is among the last of them to leave,
even as Lloyd Blankfein, the chief executive and chairman, seems in no
hurry to step down.

Goldman is expected to announce a new slate of deputies to Mr. Blankfein
in coming days, including the naming of two co-presidents --- a role Mr.
Cohn initially shared with Jon Winkelried until the latter's departure
in 2009.

In the short term, the most likely candidates to become the top deputies
are
\href{http://www.goldmansachs.com/who-we-are/leadership/management-committee/david-m-solomon.html}{David
M. Solomon}, the head of Goldman's advisory division, and
\href{http://www.goldmansachs.com/who-we-are/leadership/executive-officers/harvey-m-schwartz.html}{Harvey
M. Schwartz}, a former executive of the trading division who is now the
chief financial officer.

For those below Mr. Solomon and Mr. Schwartz, there is less clarity
about who is most likely to ascend the ladder. But the next generation
of leaders is thought to include
\href{http://www.goldmansachs.com/who-we-are/leadership/management-committee/stephen-m-scherr.html}{Stephen
M. Scherr}, who has been overseeing Goldman's new consumer banking
operation, and
\href{http://www.goldmansachs.com/who-we-are/leadership/management-committee/r-martin-chavez.html}{R.
Martin Chavez}, the chief technology officer.

Goldman named 84 new partners this year, more than its previous partner
class two years ago. Wall Street banks have been constrained by new
regulations cutting into big businesses, such as trading, and pinching
profit margins --- reasons Goldman has been expanding into new areas
like consumer banking.

Goldman has slowly been losing many of the top names who guided the firm
before the financial crisis and in its immediate aftermath. Last month,
Michael Sherwood, a vice chairman and head of Goldman's European
operation, announced plans to retire at the year's end, after
\href{https://www.nytimes.com/2016/11/21/business/dealbook/michael-sherwood-goldman-sachss-co-head-of-europe-to-step-down.html}{30
years at the firm}. That news came shortly after Mark Schwartz, the head
of Goldman's Asia Pacific region and a vice chairman of the firm,
\href{http://www.nytimes.com/2016/10/18/business/dealbook/goldman-sachss-asia-pacific-chairman-to-retire.html?rref=collection\%2Ftimestopic\%2FGoldman\%20Sachs\%20Group\&action=click\&contentCollection=business\&region=stream\&module=stream_unit\&version=latest\&contentPlacement=5\&pgtype=collection}{announced
plans} to retire at the end of the year, also capping nearly three
decades there. And the firm's longtime chief financial officer, David
Viniar, retired in 2013.

Wall Street insiders will be watching in the coming days to see if any
younger executives are elevated as well.
\href{http://www.goldmansachs.com/who-we-are/leadership/management-committee/pablo-j-salame.html}{Pablo
Salame}, one of the heads of Goldman's trading division, has been seen
as a rising star, as has Sarah Smith, the controller and chief
accounting officer.

The firm is moving into the future with the wind at its back, after
several years spent battling the financial crisis and the regulations
that arose from it, which generally hit Goldman harder than other banks.

During the presidential campaign, it looked as if the pressure on
Goldman would continue. Mr. Trump blasted Hillary Clinton for her ties
to the firm and hinted in ominous tones about the company's outsize role
in shaping the country's policy.

But since Mr. Trump's victory, he has looked to Goldman Sachs veterans
to stock his administration. In addition to the appointment of Mr. Cohn,
the former Goldman executive Steven Mnuchin was tapped as Treasury
secretary, and Mr. Trump's chief strategist, Stephen K. Bannon, is a
Goldman alumnus.

Mr. Trump has also talked about rolling back the financial regulations
that have reined in Goldman's famous trading operations in recent years
and crimped its profits.

In addition, the anticipation of big spending programs by the government
under Mr. Trump has helped to push interest rates up, which generally
helps financial firms like Goldman.

Most banks have had their stocks rise since the election, but Goldman's
shares have experienced immense gains, rising by nearly a third since
early November.

In
\href{http://www.goldmansachs.com/our-thinking/podcasts/episodes/12-12-2016-gary-cohn.html}{a
Goldman podcast released on Monday}, Mr. Cohn said that the firm was
``probably in the best position we have been in the 26 years since I've
been here.''

He added: ``Today, for a variety of reasons, capital and regulatory
reasons, the competitive landscape has changed pretty dramatically. And
we're as uniquely positioned as we ever have been in the competitive
landscape.''

Mr. Trump's transition team officially announced on Monday that Mr. Cohn
would be his assistant on economic policy and the director of his
National Economic Council, a move that had been expected since last
Friday.

``As my top economic adviser, Gary Cohn is going to put his talents as a
highly successful businessman to work for the American people,'' Mr.
Trump said in a statement. ``He fully understands the economy and will
use all of his vast knowledge and experience to make sure the American
people start winning again.''

Mr. Cohn has been with Goldman Sachs since 1990 and rose up as a trader
alongside Mr. Blankfein.

``Gary and I have been partners for more than 25 years, so I know better
than perhaps anyone that he has the intelligence, commitment and
experience to be successful at any endeavor he undertakes,'' Mr.
Blankfein said in a statement on Monday.

In his own statement, Mr. Cohn said, ``I share President-elect Trump's
vision of making sure every American worker has a secure place in a
thriving economy, and we will be completely committed to building a
nation of strength, growth and prosperity.''

Mr. Cohn is a particularly unlikely choice for the Trump administration
because he has been a registered Democrat and has, in public
appearances, reflected the sort of internationalist view of the economy
that Mr. Trump has often criticized.

At a conference in Florida soon after the election, Mr. Cohn said the
big problem facing the country and the world was a ``global growth
issue.''

``We're trying to solve it with domestic policy,'' he said. ``It's not
going to work.''

The job that Mr. Cohn is expected to accept has long been identified
with Goldman and its influence in the capital.

The role of N.E.C. director was established by President Bill Clinton
and given to Robert E. Rubin, Goldman's co-chairman at the time. Stephen
Friedman, who served as co-chairman with Mr. Rubin, later held the
economic adviser position under President George W. Bush.

Advertisement

\protect\hyperlink{after-bottom}{Continue reading the main story}

\hypertarget{site-index}{%
\subsection{Site Index}\label{site-index}}

\hypertarget{site-information-navigation}{%
\subsection{Site Information
Navigation}\label{site-information-navigation}}

\begin{itemize}
\tightlist
\item
  \href{https://help.nytimes.com/hc/en-us/articles/115014792127-Copyright-notice}{©~2020~The
  New York Times Company}
\end{itemize}

\begin{itemize}
\tightlist
\item
  \href{https://www.nytco.com/}{NYTCo}
\item
  \href{https://help.nytimes.com/hc/en-us/articles/115015385887-Contact-Us}{Contact
  Us}
\item
  \href{https://www.nytco.com/careers/}{Work with us}
\item
  \href{https://nytmediakit.com/}{Advertise}
\item
  \href{http://www.tbrandstudio.com/}{T Brand Studio}
\item
  \href{https://www.nytimes.com/privacy/cookie-policy\#how-do-i-manage-trackers}{Your
  Ad Choices}
\item
  \href{https://www.nytimes.com/privacy}{Privacy}
\item
  \href{https://help.nytimes.com/hc/en-us/articles/115014893428-Terms-of-service}{Terms
  of Service}
\item
  \href{https://help.nytimes.com/hc/en-us/articles/115014893968-Terms-of-sale}{Terms
  of Sale}
\item
  \href{https://spiderbites.nytimes.com}{Site Map}
\item
  \href{https://help.nytimes.com/hc/en-us}{Help}
\item
  \href{https://www.nytimes.com/subscription?campaignId=37WXW}{Subscriptions}
\end{itemize}
