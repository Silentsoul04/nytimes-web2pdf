Sections

SEARCH

\protect\hyperlink{site-content}{Skip to
content}\protect\hyperlink{site-index}{Skip to site index}

\href{https://www.nytimes.com/section/politics}{Politics}

\href{https://myaccount.nytimes.com/auth/login?response_type=cookie\&client_id=vi}{}

\href{https://www.nytimes.com/section/todayspaper}{Today's Paper}

\href{/section/politics}{Politics}\textbar{}Trump Chooses Hard-Liner as
Ambassador to Israel

\url{https://nyti.ms/2hMujLX}

\begin{itemize}
\item
\item
\item
\item
\item
\item
\end{itemize}

Advertisement

\protect\hyperlink{after-top}{Continue reading the main story}

Supported by

\protect\hyperlink{after-sponsor}{Continue reading the main story}

\hypertarget{trump-chooses-hard-liner-as-ambassador-to-israel}{%
\section{Trump Chooses Hard-Liner as Ambassador to
Israel}\label{trump-chooses-hard-liner-as-ambassador-to-israel}}

\includegraphics{https://static01.nyt.com/images/2016/12/16/us/16TRUMPISRAEL/16TRUMPISRAEL-articleInline-v2.jpg?quality=75\&auto=webp\&disable=upscale}

By \href{http://www.nytimes.com/by/matthew-rosenberg}{Matthew Rosenberg}

\begin{itemize}
\item
  Dec. 15, 2016
\item
  \begin{itemize}
  \item
  \item
  \item
  \item
  \item
  \item
  \end{itemize}
\end{itemize}

WASHINGTON --- President-elect Donald J. Trump on Thursday named David
M. Friedman, a bankruptcy lawyer aligned with the Israeli far right, as
his nominee for ambassador to Israel, elevating a campaign adviser who
has questioned the need for a two-state solution and has likened
left-leaning Jews in America to the Jews who aided the Nazis in the
Holocaust.

Mr. Friedman, whose outspoken views stand in stark contrast to decades
of American policy toward Israel, did not wait long on Thursday to
signal his intention to upend the American approach. In a statement from
the Trump transition team announcing his nomination, he said he looked
forward to doing the job ``from the U.S. embassy in Israel's eternal
capital, Jerusalem.''

Through decades of Republican and Democratic administrations, the
embassy has been in Tel Aviv, as the State Department insists that
\href{https://www.nytimes.com/2016/11/19/world/middleeast/jerusalem-us-embassy-trump.html}{the
status of Jerusalem} --- which both Israel and the Palestinians see as
their rightful capital --- can be determined only through negotiations
as part of an overall peace deal.

Mr. Friedman, who has no diplomatic experience, has said that he does
not believe it would be illegal for Israel to annex the occupied West
Bank and he supports building new settlements there, which Washington
has long condemned as illegitimate and an obstacle to peace.

The Trump transition team's statement focused on Mr. Friedman's long
history with Israel, portraying him as a friendly supporter of the
country whose views were in line with the United States' position toward
it.

``The two nations have enjoyed a special relationship based on mutual
respect and a dedication to freedom and democracy,'' it said. ``With Mr.
Friedman's nomination, President-elect Trump expressed his commitment to
further enhancing the U.S.-Israel relationship and ensuring there will
be extraordinary strategic, technological, military and intelligence
cooperation between the two countries.'' The statement said that Mr.
Friedman was a fluent speaker of Hebrew and ``a lifelong student of
Israel's history.''

Mr. Friedman's appointment was quickly praised by the Republican Jewish
Coalition, whose executive director, Matt Brooks, called it ``a powerful
signal to the Jewish community.''

But beyond Republicans, there were deep concerns over the choice of Mr.
Friedman. J Street, a dovish lobbying organization that has been
critical of some Israeli policies, said in a statement that it was
``vehemently opposed to the nomination.''

``As someone who has been a leading American friend of the settlement
movement, who lacks any diplomatic or policy credentials,'' it said,
``Friedman should be beyond the pale.''

Mr. Friedman has made clear his disdain for those American Jews ---
especially those connected to J Street --- who support a two-state
solution for the Israelis and the Palestinians. Writing in June on the
website of Arutz Sheva, an Israeli media organization, Mr. Friedman
\href{http://www.israelnationalnews.com/Articles/Article.aspx/18828}{compared
J Street supporters to ``kapos,''} the Jews who cooperated with the
Nazis during the Holocaust.

``The kapos faced extraordinary cruelty,'' he wrote. ``But J Street?
They are just smug advocates of Israel's destruction delivered from the
comfort of their secure American sofas --- it's hard to imagine anyone
worse.''

At a private session this month at the Saban Forum, an annual gathering
of Israeli and American foreign policy figures, Mr. Friedman declined to
disavow the comments and even intensified the sentiment.

Questioned by Jeffrey Goldberg, the editor in chief of the Atlantic, Mr.
Friedman was asked if he would meet with various groups, including J
Street. Mr. Friedman said he would probably meet with individuals but
not with the group, according to several people who attended.

Mr. Goldberg then raised the kapos comparison and asked if he stood by
it. Mr. Friedman did not back away. ``They're not Jewish, and they're
not pro-Israel,'' he said, according to the people in the room.

Daniel Levy, a left-leaning former Israeli peace negotiator, said that
in naming an ambassador with the hard-line views of Mr. Friedman, Mr.
Trump could end up undercutting the security of Israel and the United
States and condemn ``the Palestinians to further disenfranchisement and
dispossession.''

``If an American ambassador stakes out positions that further embolden
an already triumphalist settler elite, then that is likely to cause
headaches for American national security interests across the region and
even for Israel's own security establishment,'' Mr. Levy said.
``Especially an ambassador committed to the ill-advised relocation of
the U.S. embassy to Jerusalem.''

In its statement, the Trump team noted that Mr. Friedman had held his
bar mitzvah 45 years ago in Jerusalem at the Western Wall. The wall, the
holiest place where Jews can pray, is a remnant of the retaining wall
that surrounded the ancient Temple Mount, the most sacred site in
Judaism.

The site today houses the Al Aqsa Mosque compound, the third holiest
site in Islam. Control over the site has been a persistent source of
friction between Israel and the Palestinians, and has sparked violence
between the two sides.

More recently, the Western Wall itself has been a source of tension and
clashes between the Orthodox authorities who control the site and more
liberal Jews, many of whom are from North America and oppose the
restrictions there on prayer by women.

Advertisement

\protect\hyperlink{after-bottom}{Continue reading the main story}

\hypertarget{site-index}{%
\subsection{Site Index}\label{site-index}}

\hypertarget{site-information-navigation}{%
\subsection{Site Information
Navigation}\label{site-information-navigation}}

\begin{itemize}
\tightlist
\item
  \href{https://help.nytimes.com/hc/en-us/articles/115014792127-Copyright-notice}{©~2020~The
  New York Times Company}
\end{itemize}

\begin{itemize}
\tightlist
\item
  \href{https://www.nytco.com/}{NYTCo}
\item
  \href{https://help.nytimes.com/hc/en-us/articles/115015385887-Contact-Us}{Contact
  Us}
\item
  \href{https://www.nytco.com/careers/}{Work with us}
\item
  \href{https://nytmediakit.com/}{Advertise}
\item
  \href{http://www.tbrandstudio.com/}{T Brand Studio}
\item
  \href{https://www.nytimes.com/privacy/cookie-policy\#how-do-i-manage-trackers}{Your
  Ad Choices}
\item
  \href{https://www.nytimes.com/privacy}{Privacy}
\item
  \href{https://help.nytimes.com/hc/en-us/articles/115014893428-Terms-of-service}{Terms
  of Service}
\item
  \href{https://help.nytimes.com/hc/en-us/articles/115014893968-Terms-of-sale}{Terms
  of Sale}
\item
  \href{https://spiderbites.nytimes.com}{Site Map}
\item
  \href{https://help.nytimes.com/hc/en-us}{Help}
\item
  \href{https://www.nytimes.com/subscription?campaignId=37WXW}{Subscriptions}
\end{itemize}
