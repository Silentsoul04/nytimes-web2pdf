Sections

SEARCH

\protect\hyperlink{site-content}{Skip to
content}\protect\hyperlink{site-index}{Skip to site index}

\href{https://www.nytimes.com/section/world/middleeast}{Middle East}

\href{https://myaccount.nytimes.com/auth/login?response_type=cookie\&client_id=vi}{}

\href{https://www.nytimes.com/section/todayspaper}{Today's Paper}

\href{/section/world/middleeast}{Middle East}\textbar{}Hungry, Thirsty
and Bloodied in Battle to Retake Mosul From ISIS

\url{https://nyti.ms/2i4uiiv}

\begin{itemize}
\item
\item
\item
\item
\item
\item
\end{itemize}

Advertisement

\protect\hyperlink{after-top}{Continue reading the main story}

Supported by

\protect\hyperlink{after-sponsor}{Continue reading the main story}

\hypertarget{hungry-thirsty-and-bloodied-in-battle-to-retake-mosul-from-isis}{%
\section{Hungry, Thirsty and Bloodied in Battle to Retake Mosul From
ISIS}\label{hungry-thirsty-and-bloodied-in-battle-to-retake-mosul-from-isis}}

\includegraphics{https://static01.nyt.com/images/2016/12/19/world/19mosul1/19mosul1-articleInline.jpg?quality=75\&auto=webp\&disable=upscale}

By \href{http://www.nytimes.com/by/tim-arango}{Tim Arango},
\href{http://www.nytimes.com/by/eric-schmitt}{Eric Schmitt} and
\href{http://www.nytimes.com/by/rukmini-callimachi}{Rukmini Callimachi}

\begin{itemize}
\item
  Dec. 18, 2016
\item
  \begin{itemize}
  \item
  \item
  \item
  \item
  \item
  \item
  \end{itemize}
\end{itemize}

ISTANBUL --- After two months, the battle to retake the Iraqi city of
Mosul from the Islamic State has settled into a grinding war of
attrition. The front lines have barely budged in weeks. Casualties of
Iraqi security forces are so high that American commanders heading the
United States-led air campaign worry that they are unsustainable.
Civilians are being killed or injured by Islamic State snipers and
growing numbers of suicide bombers.

As the world watches the horrors unfolding in Aleppo, Syria, where
government forces and allied militias bombed civilians and carried out
summary executions as they retook the last rebel-held areas, a different
tragedy is transpiring in Mosul. Up to one million people are trapped
inside the city, running low on food and drinking water and facing the
worsening cruelty of Islamic State fighters.

``ISIS members have become like mad dogs, and every member has the power
of immediate execution,'' Abu Noor said by telephone from his home on
the west side of Mosul, which government forces had not reached,
referring to the terror group by one of its acronyms. ``We live in
constant fear and worry.''

As the fight drags on, it is looking more and more likely that Mosul
will become one of the first national security issues facing
President-elect Donald J. Trump when he takes office next month. While
American forces have largely steered clear of the fighting in Syria,
they are deeply involved in operations just over the border in Iraq,
mainly in training, advising and support roles.

Senior American officials and top commanders in the Middle East say the
brutal urban fight for Mosul is succeeding --- however slowly --- but is
proving to be tougher than expected. These officials predict that the
battle to oust the Islamic State from Iraq's second-largest city could
last two to four more months.

Brett H. McGurk, President Obama's envoy to the international coalition
fighting the Islamic State, noted at a recent White House briefing that
previous battles against the terror group, as in Falluja, in Iraq, or
Ramadi or Kobani, in Syria, took months, and said that eventually the
Islamic State would exhaust its supply of munitions and fighters.

``Eventually they reach a culmination point, they simply cannot
resupply, they run out of suicide bombers,'' Mr. McGurk said. ``In
Mosul, we don't know when that will come. It could come very soon, it
could come a couple months from now, but our momentum will be sustained
and we'll provide relentless pressure on the enemy throughout Mosul.''

Lt. Gen. Stephen Townsend gave Pentagon reporters
\href{https://news.pub.mediadc.com/wta/link.php?AGENCY=AB\&M=40612166\&N=64093\&L=35124\&F=H}{a
year-end update} that made no prediction on how soon Mosul would be
liberated. ``It's progressing. It's probably not progressing as fast as
I, as a U.S. Army officer, would like, but it is progressing, and the
Iraqis are advancing every day,'' he said.

General Townsend said the Iraqis were engaged in discussions ``about how
to inject new energy'' into their assault. ``We're just going to let it
go at the pace'' of the Iraqi military, he said. ``They're the ones
doing the fighting and the dying.''

The battle for Mosul has shaped up unlike any other in Iraq. As Iraqi
forces have advanced, they have uncovered Islamic State bomb-making
facilities that have stunned soldiers and researchers in their
sophistication, indicating that it could be a long time before the group
runs out of arms.

!{[}Azad Hassan, right, and his brother Mohammad each lost a hand,
chopped off by Islamic State militants. They were with an aunt last week
in a village south of Mosul.

Credit...Mohammed
Salem/Reuters{]}(\url{https://static01.nyt.com/images/2016/12/19/world/19MOSUL2/19MOSUL2-articleInline.jpg?quality=75\&auto=webp\&disable=upscale})

\href{http://www.conflictarm.com/publications/}{A recent report} from
Conflict Armament Research, a London-based organization that sent a team
of researchers to eastern Mosul, said the Islamic State had been
producing rockets and mortars on an ``industrial'' scale inside Mosul.

Tens of thousands of armaments have been produced, the organization
said, with supplies from Turkey, which in the last year has tightened
its border security and clamped down on Islamic State smuggling networks
in the face of criticism from allies, including the United States, that
it was turning a blind eye to the terror group.

The findings, the group said, indicate ``a robust supply chain extending
from Turkey, through Syria, to Mosul,'' suggesting that Turkey's efforts
at the border have not been enough to cut off the Islamic State from
suppliers.

Shakir Mahmood, a soldier in Iraq's elite counterterror forces, fought
in battles in Ramadi, Falluja and Tikrit, but they were nothing like the
fight he has faced in Mosul, he said.

``I have never seen or witnessed a battle like the battle for Mosul,''
he said. ``They have so many snipers hiding in the houses among
civilians, and also many car bombs. Our losses in this battle cannot be
compared with the other battles.''

Another veteran soldier, Ibrahim Ali, said: ``I have seen things in
Mosul that I will never forget my whole life. I have seen entire
families get killed because of ISIS car bombs. And I have lost dear
friends in Mosul.''

The Iraqi government does not release figures of military casualties,
but it is clear in talking to officers that they are worrisomely high.
The United Nations reported on Dec. 1 that 1,959 members of the Iraqi
security forces had died in November. But after the Iraqi military
protested, the United Nations issued a new statement saying its figures
were ``largely unverified'' and said it would discontinue releasing
casualty statistics for the military.

When the battle started, in mid-October, it moved fairly quickly as
forces took outlying areas that had mostly been empty of civilians.

Journalists were given wide access to the front lines. But recently,
getting the news out of Mosul has become more difficult; commanders are
prohibiting most front-line embeds.

The tightening of access, apparently, was not an effort to control the
narrative, but a reaction to the recent appearance in Mosul of
Bernard-Henri Lévy, the French philosopher and writer, who is producing
a documentary film about the battle. Why was that controversial? Because
Mr. Levy is Jewish.

His appearance stirred outrage in Iraq, and the authorities in Baghdad
moved to shut down access for all journalists.

``The rumor spread that we were having relations with Israel,'' said Lt.
Gen. Abdulwahab al-Saadi, a special forces commander in Mosul, who said
he had no idea who Mr. Levy was when he arrived. ``In fact, we had no
idea who this was that came to see us.''

He said access for journalists would be restored soon. ``We will solve
this problem,'' he said.

Civilian casualties are soaring, even though the government, at the
outset of the battle, dropped millions of leaflets over the city with
instructions to stay inside their homes. Most civilians have, but those
who have fled --- there are some 90,000 people displaced from their
homes around the city --- have faced harrowing journeys, and many have
been killed or maimed by crossfire.

That so many civilians have remained has hampered the fight, as Iraqi
soldiers move slowly in an effort to protect them. It has also led to
limited use of air power and artillery.

``Essentially, they are trying a different operational approach,'' said
Carl Castellano, a senior analyst at Talos, a consulting firm that
focuses on security in Iraq. ``They don't have the capability to
evacuate all these civilians, and so that's limiting the amount of
firepower they can use in the city. That is limiting their options in
terms of what they can do --- close air support and everything else.''

American air commanders have quickly sought to modify some of their
bombing runs to counter shifting tactics by the Islamic State, cratering
streets in Mosul with bombs to stymie car-bombers or at least slow them
down, and stepping up attacks on car bomb factories in and around Mosul.
Allied warplanes have destroyed nearly 140 car bombs or car-bomb
factories since the Mosul offensive began, American officials said.

In the second week of December, nearly 700 civilians were wounded, from
gunshots, mines and rocket fire, according to the United Nations, a 30
percent increase from the previous week.

Many of the injured wind up in the emergency rooms of hospitals in
Erbil, the capital of the Kurdish region.

On a recent day, Saleh Hassoun sat in a hospital in Erbil, less than an
hour's drive from Mosul, at the bedside of his 1-year-old granddaughter,
who had been wounded by a mine.

``The mines are everywhere,'' he said. ``The one that we set off was on
the ground, and attached to a tiny cable. We didn't see it, and the
explosion killed both of my daughters and injured my granddaughter.''

A woman in the hospital, Umm Ussam, 54, had been shot through the neck.
At first, she obeyed the instructions of the Iraqi Army to stay indoors,
but once the military arrived she ran behind one of the Humvees, only to
be picked off by a sniper, she said.

For those who did stay home and whose areas are now liberated, there are
new challenges, and fears, not to mention a lack of services and a
dwindling supply of safe drinking water.

``The government forces said stay in your houses, but our houses are
without electricity or water,'' said Sabah Kareem, whose neighborhood of
eastern Mosul is now under government control. ``We are amid hell. We
don't know when we will be bombed, or if ISIS will return to kill us.''

Advertisement

\protect\hyperlink{after-bottom}{Continue reading the main story}

\hypertarget{site-index}{%
\subsection{Site Index}\label{site-index}}

\hypertarget{site-information-navigation}{%
\subsection{Site Information
Navigation}\label{site-information-navigation}}

\begin{itemize}
\tightlist
\item
  \href{https://help.nytimes.com/hc/en-us/articles/115014792127-Copyright-notice}{©~2020~The
  New York Times Company}
\end{itemize}

\begin{itemize}
\tightlist
\item
  \href{https://www.nytco.com/}{NYTCo}
\item
  \href{https://help.nytimes.com/hc/en-us/articles/115015385887-Contact-Us}{Contact
  Us}
\item
  \href{https://www.nytco.com/careers/}{Work with us}
\item
  \href{https://nytmediakit.com/}{Advertise}
\item
  \href{http://www.tbrandstudio.com/}{T Brand Studio}
\item
  \href{https://www.nytimes.com/privacy/cookie-policy\#how-do-i-manage-trackers}{Your
  Ad Choices}
\item
  \href{https://www.nytimes.com/privacy}{Privacy}
\item
  \href{https://help.nytimes.com/hc/en-us/articles/115014893428-Terms-of-service}{Terms
  of Service}
\item
  \href{https://help.nytimes.com/hc/en-us/articles/115014893968-Terms-of-sale}{Terms
  of Sale}
\item
  \href{https://spiderbites.nytimes.com}{Site Map}
\item
  \href{https://help.nytimes.com/hc/en-us}{Help}
\item
  \href{https://www.nytimes.com/subscription?campaignId=37WXW}{Subscriptions}
\end{itemize}
