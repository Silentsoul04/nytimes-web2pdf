Sections

SEARCH

\protect\hyperlink{site-content}{Skip to
content}\protect\hyperlink{site-index}{Skip to site index}

\href{https://myaccount.nytimes.com/auth/login?response_type=cookie\&client_id=vi}{}

\href{https://www.nytimes.com/section/todayspaper}{Today's Paper}

\href{/section/business/dealbook}{DealBook}\textbar{}Deutsche Bank to
Settle Mortgage Inquiry for \$7.2 Billion

\url{https://nyti.ms/2heK7FY}

\begin{itemize}
\item
\item
\item
\item
\item
\end{itemize}

Advertisement

\protect\hyperlink{after-top}{Continue reading the main story}

Supported by

\protect\hyperlink{after-sponsor}{Continue reading the main story}

DealBook Business and Policy

\hypertarget{deutsche-bank-to-settle-mortgage-inquiry-for-72-billion}{%
\section{Deutsche Bank to Settle Mortgage Inquiry for \$7.2
Billion}\label{deutsche-bank-to-settle-mortgage-inquiry-for-72-billion}}

\includegraphics{https://static01.nyt.com/images/2016/12/23/business/23db-deutschebank_web1/23db-deutschebank_web1-articleInline.jpg?quality=75\&auto=webp\&disable=upscale}

By \href{http://www.nytimes.com/by/ben-protess}{Ben Protess} and
\href{http://www.nytimes.com/by/landon-thomas-jr}{Landon Thomas Jr.}

\begin{itemize}
\item
  Dec. 22, 2016
\item
  \begin{itemize}
  \item
  \item
  \item
  \item
  \item
  \end{itemize}
\end{itemize}

Deutsche Bank announced late on Thursday that it had reached a tentative
\$7.2 billion deal to resolve a federal investigation into its sale of
toxic mortgage securities, capping months of negotiations that weighed
heavily on the bank's stock price and reputation.

The civil settlement requires the bank, Germany's largest, to pay a
\$3.1 billion penalty and provide relief to American consumers valued at
\$4.1 billion, the
\href{https://www.db.com/newsroom_news/2016/ghp/deutsche-bank-agrees-on-settlement-in-principle-with-the-doj-regarding-rmbs-en-11789.htm}{bank
said in a statement} that came ahead of a formal announcement in the
case. The consumer portion of the settlement, the bank said, is expected
to be ``primarily in the form of loan modifications and other assistance
to homeowners.''

Still, the bank cautioned that the deal was not final, saying that
``there can be no assurance that the U.S. Department of Justice and the
bank will agree on the final documentation.''

The payout would settle claims that the bank, like many other
institutions during the financial crisis of 2008, sold investors
mortgage securities that contributed to the crisis. Collectively, big
banks have paid tens of billions of dollars over their sale of these
securities.

The announcement of the tentative deal with Deutsche Bank came on the
same day that the
\href{https://www.nytimes.com/2016/12/22/business/dealbook/justice-department-sues-barclays-mortgage-backed-securities.html?ref=business}{Justice
Department sued Barclays}, the British bank, over similar accusations.
In that case, negotiations broke down, leading the government to file a
lawsuit in Federal District Court in Brooklyn.

The settlement for Deutsche Bank, abrupt as it was, comes after many
months of uncertainty that cast a pall over the bank's future after
reports last fall that the government had asked the bank to pay \$14
billion as negotiations got underway.

That such an enormous number was even mentioned raised concerns not just
about the bank's business model in the years ahead, but its short-term
prospects as well.

This deal, while substantial, falls far short of that sum.

Few analysts had expected Deutsche Bank to have to pay such an outsize
amount, but the reports gave voice to longstanding
\href{https://www.nytimes.com/2016/10/03/business/dealbook/deutsche-banks-appetite-for-risk-throws-off-its-balance.html}{fears
that investors have had about the bank's trading-driven business model}
and its ability to generate enough cash to pay for its past infractions.

From the summer of 2015 to last summer, Deutsche's stock more than
halved, hitting a low of \$11, as fears over the bank's future mounted,
prompting some hedge funds to withdraw cash they were holding at the
bank.

Since early November, however, shares of Deutsche Bank have taken part
in a broad rally of bank stocks, on hopes that the election of Donald J.
Trump, a longtime and substantial client of the bank, would mean a
dialing back of bank regulations in the United States. The German bank
has substantial operations in New York, as well as in London and
Frankfurt.

Problems with Deutsche Bank manipulating core lending and foreign
exchange rates, pushing toxic mortgages, helping hedge funds lessen
their tax bills and failing to put in place proper risk controls have
persistently infuriated regulators. But not only the watchdogs were
upset.

Ultimately the bank's shareholders supported a business model based on
paying extremely large sums to some of the most brazen risk-takers Wall
Street had ever produced.

According to calculations by Berenberg Bank, since 1995, when Deutsche
Bank's global markets division was first set up, the bank has paid out
65 billion euros, about \$68 billion, in bonuses.

During this same period, the bank's board in Frankfurt asked investors
to come up with €23 billion to bolster its perpetually thin cushion of
cash.

For example, one midlevel trader in the Libor scandal, an attempted
manipulation of benchmark interest rates,
\href{https://www.bloomberg.com/news/articles/2015-04-23/deutsche-bank-trader-bittar-s-libor-messages-revealed-by-u-s-}{earned}
a bonus of \$136 million in 2008.

``Money went everywhere, except to the shareholders,'' said Bernd
Ondruch of Astellon Capital Partners, a hedge fund based in London.
``Deutsche had to pay its way into investment banking, and during the
boom years, the bankers took the balance sheet hostage. This was an
issue for all the banks but Deutsche Bank was the most aggressive by
far.''

These outsize numbers highlight the herculean task before Deutsche
Bank's chief executive, John Cryan, who, has been trying to reinvent the
bank and move it beyond its legal problems.

In recent
\href{https://www.db.com/newsroom_news/2016/ghp/december-message-to-employees-from-john-cryan-en-11775.htm}{communications}
with his employees, he has highlighted ambitions for the bank to be less
of a ``flow monster,'' trading large volumes of securities, and more of
a tech-savvy, focused bank that aims to meet specialized client demands.

It is a strategy that virtually all of Deutsche Bank's peers have been
following, and many of them are already years into a turnaround effort.

And unlike peer institutions like Barclays, Royal Bank of Scotland and
Citigroup, Deutsche Bank cannot fall back on a dominant deposit-taking
brand as it tries to move away from its reliance on trading activities.

Germany's banking market remains highly fragmented, and attempts by
Deutsche Bank to build its domestic franchise via the purchase of the
retail lender Postbank and merger talks with Commerzbank have not been
successful.

In recent weeks, Deutsche Bank, as part of its broader restructuring
efforts, has cut ties with 3,400 trading clients, an extraordinary move
for an institution that over a 20-year span has made its name by
servicing sophisticated investors in London, Hong Kong and the United
Arab Emirates.

Advertisement

\protect\hyperlink{after-bottom}{Continue reading the main story}

\hypertarget{site-index}{%
\subsection{Site Index}\label{site-index}}

\hypertarget{site-information-navigation}{%
\subsection{Site Information
Navigation}\label{site-information-navigation}}

\begin{itemize}
\tightlist
\item
  \href{https://help.nytimes.com/hc/en-us/articles/115014792127-Copyright-notice}{©~2020~The
  New York Times Company}
\end{itemize}

\begin{itemize}
\tightlist
\item
  \href{https://www.nytco.com/}{NYTCo}
\item
  \href{https://help.nytimes.com/hc/en-us/articles/115015385887-Contact-Us}{Contact
  Us}
\item
  \href{https://www.nytco.com/careers/}{Work with us}
\item
  \href{https://nytmediakit.com/}{Advertise}
\item
  \href{http://www.tbrandstudio.com/}{T Brand Studio}
\item
  \href{https://www.nytimes.com/privacy/cookie-policy\#how-do-i-manage-trackers}{Your
  Ad Choices}
\item
  \href{https://www.nytimes.com/privacy}{Privacy}
\item
  \href{https://help.nytimes.com/hc/en-us/articles/115014893428-Terms-of-service}{Terms
  of Service}
\item
  \href{https://help.nytimes.com/hc/en-us/articles/115014893968-Terms-of-sale}{Terms
  of Sale}
\item
  \href{https://spiderbites.nytimes.com}{Site Map}
\item
  \href{https://help.nytimes.com/hc/en-us}{Help}
\item
  \href{https://www.nytimes.com/subscription?campaignId=37WXW}{Subscriptions}
\end{itemize}
