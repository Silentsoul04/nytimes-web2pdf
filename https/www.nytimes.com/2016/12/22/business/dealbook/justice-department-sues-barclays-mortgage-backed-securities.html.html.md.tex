Sections

SEARCH

\protect\hyperlink{site-content}{Skip to
content}\protect\hyperlink{site-index}{Skip to site index}

\href{https://myaccount.nytimes.com/auth/login?response_type=cookie\&client_id=vi}{}

\href{https://www.nytimes.com/section/todayspaper}{Today's Paper}

\href{/section/business/dealbook}{DealBook}\textbar{}Justice Department
Sues Barclays Over Mortgage-Backed Securities

\url{https://nyti.ms/2ilqFVq}

\begin{itemize}
\item
\item
\item
\item
\item
\end{itemize}

Advertisement

\protect\hyperlink{after-top}{Continue reading the main story}

Supported by

\protect\hyperlink{after-sponsor}{Continue reading the main story}

DealBook Business and Policy

\hypertarget{justice-department-sues-barclays-over-mortgage-backed-securities}{%
\section{Justice Department Sues Barclays Over Mortgage-Backed
Securities}\label{justice-department-sues-barclays-over-mortgage-backed-securities}}

\includegraphics{https://static01.nyt.com/images/2016/12/23/business/23DB-BARCLAYS/23DB-BARCLAYS-articleLarge.jpg?quality=75\&auto=webp\&disable=upscale}

By \href{http://www.nytimes.com/by/chad-bray}{Chad Bray}

\begin{itemize}
\item
  Dec. 22, 2016
\item
  \begin{itemize}
  \item
  \item
  \item
  \item
  \item
  \end{itemize}
\end{itemize}

LONDON --- United States authorities have accused the British bank
Barclays and two former executives of fraudulently misleading the public
in the sale of tens of billions of dollars in securities backed by home
mortgages.

The Justice Department
\href{https://www.justice.gov/opa/pr/united-states-sues-barclays-bank-recover-civil-penalties-fraud-sale-residential-mortgage}{filed
a lawsuit} Thursday in Federal District Court in Brooklyn after the two
sides failed to reach a settlement despite months of talks.

The department has also been in settlement talks with other large
European lenders, including Credit Suisse, as the end of the Obama
administration nears. Late Thursday, Deutsche Bank said
that\href{http://www.nytimes.com/2016/12/22/business/dealbook/deutsche-bank-mortgage-securities-justice-department-homeowners.html}{it
had reached a tentative \$7.2 billion deal} with the Justice Department
to resolve a federal investigation into its sale of toxic mortgage
securities after months of negotiations.

On Thursday, the Justice Department said Barclays' actions in the
packaging and sale of what are known as residential mortgage-backed
securities in the years leading to the financial crisis injured tens of
thousands of investors when the housing market collapsed. The lawsuit
claims that Barclays --- which has significant investment banking
operations in New York --- repeatedly misled investors about the quality
of mortgages underlying 36 mortgage-backed securities from 2005 to 2007.
Those deals securitized more than \$31 billion in subprime and other
mortgage loans, according to the complaint. The Justice Department said
they turned out to be ``catastrophic failures.''

``More than half of the underlying loans defaulted, causing investors in
those deals to lose many billions of dollars, with hundreds of millions
more in losses projected during the remaining life of the deals,'' the
complaint says. ``Even many investors in the AAA-rated tranches of these
securitizations, which were rated as safe as investments in U.S.
Treasury bonds, have suffered or are projected to suffer significant
losses.''

Vendors that the bank hired to perform due diligence on the loans
described some of them as ``craptacular,'' others as ``scariest
collateral'' and others as having the ``distinct aroma of default,''
according to the complaint.

Barclays said the claims by the Justice Department were ``disconnected
from the facts.''

``We have an obligation to our shareholders, customers, clients and
employees to defend ourselves against unreasonable allegations and
demands,'' the bank said. ``Barclays will vigorously defend the
complaint and seek its dismissal at the earliest opportunity.''

In its complaint, the Justice Department also named two former Barclays
executives as defendants: Paul K. Menefee, Barclays' lead banker on its
subprime residential mortgage-backed securities; and John T. Carroll,
Barclays' lead trader for subprime loan acquisitions.

``The complaint is a misguided attempt to place blame on Paul Menefee
and others for losses incurred by sophisticated institutional investors
as a result of the industrywide collapse of the housing market,'' said
Barry H. Berke, a lawyer for Mr. Menefee of Austin, Tex. Mr. Berke said
he looked forward to fighting the allegations in court, ``where it will
be shown why the government's own witness described Mr. Menefee as
`honest,' `careful' and `the fairest person he ever dealt with.'''

A lawyer for Mr. Carroll of Port Washington, N.Y., did not immediately
respond to a request for comment.

Since 2013, large banks in the United States, including JPMorgan Chase,
Bank of America and Citigroup, have paid tens of billions of dollars in
settlements over troubled mortgage securities. Goldman Sachs agreed to
pay as much as \$5 billion to
\href{http://www.nytimes.com/2016/01/15/business/dealbook/goldman-to-pay-5-billion-to-settle-claims-of-faulty-mortgages.html}{settle
an inquiry}into mortgage-backed securities in January, while Morgan
Stanley reached a
\href{http://www.nytimes.com/2016/02/12/business/dealbook/morgan-stanley-to-pay-3-2-billion-over-flawed-mortgage-bonds.html}{\$3.2
billion settlement} in February.

For Barclays, the mortgage litigation is one of several legacy issues
that are lingering as James E. Staley, the London-based bank's American
chief executive, seeks to turn it around. The bank also faces several
separate civil lawsuits over the sale of mortgage-backed securities.
Since joining the bank last year, Mr. Staley has accelerated the sale of
businesses that Barclays no longer considers essential and is focusing
its strategy on two divisions: the British consumer bank and a
slimmed-down corporate and investment bank.

The Justice Department litigation was brought under the Financial
Institutions Reform, Recovery and Enforcement Act of 1989, which was
passed after the savings and loan crisis. It has been an important tool
for the government in recent years as it has pursued cases stemming from
the 2008 financial crisis.

Barclays repeatedly assured investors that ``unacceptable'' loans had
been weeded out through its due diligence process and that the
properties underlying the securitized loans had ``sufficient value to
avoid loss in the event of default,'' according to the complaint.

The Justice Department said that Barclays' due diligence on the deals at
issue was ``a sham.'' A large percentage of the properties did not
comply with underwriting guidelines or involved borrowers who lacked the
ability to repay their loans, the Justice Department said.

``Financial institutions like Barclays occupy a position of vital public
trust,'' Attorney General Loretta E. Lynch said. ``Ordinary Americans
depend on their assurances of transparency and legitimacy, and entrust
these banks with their very livelihood.''

Barclays was informed by its due diligence vendors that the appraised
values of significant percentages of the mortgaged properties were
overstated and that thousands of those properties were underwater, or
worth less than the amount of loans against them, the complaint says. In
one instance, the Justice Department said Mr. Menefee lamented to an
unnamed Barclays trader in June 2007 that investors had become smarter
about risk factors in loan pools and the bank would not be able to
continue its past practice of including defective loans in its
residential mortgage-backed securities.

``I just don't think we're able to hide as much as we were last year,
jam things in, you know, bob and weave and hope for the best,'' Mr.
Menefee said, according to the lawsuit. ``And I think those days are
behind us.''

Advertisement

\protect\hyperlink{after-bottom}{Continue reading the main story}

\hypertarget{site-index}{%
\subsection{Site Index}\label{site-index}}

\hypertarget{site-information-navigation}{%
\subsection{Site Information
Navigation}\label{site-information-navigation}}

\begin{itemize}
\tightlist
\item
  \href{https://help.nytimes.com/hc/en-us/articles/115014792127-Copyright-notice}{©~2020~The
  New York Times Company}
\end{itemize}

\begin{itemize}
\tightlist
\item
  \href{https://www.nytco.com/}{NYTCo}
\item
  \href{https://help.nytimes.com/hc/en-us/articles/115015385887-Contact-Us}{Contact
  Us}
\item
  \href{https://www.nytco.com/careers/}{Work with us}
\item
  \href{https://nytmediakit.com/}{Advertise}
\item
  \href{http://www.tbrandstudio.com/}{T Brand Studio}
\item
  \href{https://www.nytimes.com/privacy/cookie-policy\#how-do-i-manage-trackers}{Your
  Ad Choices}
\item
  \href{https://www.nytimes.com/privacy}{Privacy}
\item
  \href{https://help.nytimes.com/hc/en-us/articles/115014893428-Terms-of-service}{Terms
  of Service}
\item
  \href{https://help.nytimes.com/hc/en-us/articles/115014893968-Terms-of-sale}{Terms
  of Sale}
\item
  \href{https://spiderbites.nytimes.com}{Site Map}
\item
  \href{https://help.nytimes.com/hc/en-us}{Help}
\item
  \href{https://www.nytimes.com/subscription?campaignId=37WXW}{Subscriptions}
\end{itemize}
