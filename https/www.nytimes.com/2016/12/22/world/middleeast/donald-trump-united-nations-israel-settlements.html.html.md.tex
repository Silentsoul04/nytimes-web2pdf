Sections

SEARCH

\protect\hyperlink{site-content}{Skip to
content}\protect\hyperlink{site-index}{Skip to site index}

\href{https://www.nytimes.com/section/world/middleeast}{Middle East}

\href{https://myaccount.nytimes.com/auth/login?response_type=cookie\&client_id=vi}{}

\href{https://www.nytimes.com/section/todayspaper}{Today's Paper}

\href{/section/world/middleeast}{Middle East}\textbar{}Trump Pressures
Obama Over U.N. Resolution on Israeli Settlements

\url{https://nyti.ms/2ijBAiu}

\begin{itemize}
\item
\item
\item
\item
\item
\item
\end{itemize}

Advertisement

\protect\hyperlink{after-top}{Continue reading the main story}

Supported by

\protect\hyperlink{after-sponsor}{Continue reading the main story}

\hypertarget{trump-pressures-obama-over-un-resolution-on-israeli-settlements}{%
\section{Trump Pressures Obama Over U.N. Resolution on Israeli
Settlements}\label{trump-pressures-obama-over-un-resolution-on-israeli-settlements}}

\includegraphics{https://static01.nyt.com/images/2016/12/23/world/23ISRAEL/23ISRAEL-articleLarge.jpg?quality=75\&auto=webp\&disable=upscale}

By \href{http://www.nytimes.com/by/peter-baker}{Peter Baker} and
\href{http://www.nytimes.com/by/somini-sengupta}{Somini Sengupta}

\begin{itemize}
\item
  Dec. 22, 2016
\item
  \begin{itemize}
  \item
  \item
  \item
  \item
  \item
  \item
  \end{itemize}
\end{itemize}

JERUSALEM --- President-elect
\href{http://www.nytimes.com/topic/person/donald-trump?inline=nyt-per}{Donald
J. Trump} thrust himself into one of the world's most polarizing debates
on Thursday by pressuring
\href{http://topics.nytimes.com/top/reference/timestopics/people/o/barack_obama/index.html?inline=nyt-per}{President
Obama} to veto a
\href{http://topics.nytimes.com/top/reference/timestopics/organizations/u/united_nations/index.html?inline=nyt-org}{United
Nations} resolution critical of
\href{http://www.nytimes.com/topic/destination/israel}{Israel}, the
newly elected leader's most direct intervention in foreign policy during
his transition to power.

Mr. Trump spoke out after Israeli officials contacted his team for help
in blocking the draft resolution condemning settlement construction even
as they lobbied its sponsor,
\href{http://www.nytimes.com/topic/destination/egypt}{Egypt}. Within a
couple of hours, Egypt withdrew the resolution, at least temporarily,
and its president, Abdel Fattah el-Sisi, called Mr. Trump to discuss how
``to establish true peace in the Middle East,'' according to an aide to
the president-elect.

Mr. Trump's forceful intervention was a rare effort by a new president
to shape international events even before taking office. While new
presidents typically refrain from weighing in on current issues during
the interregnum between their election and inauguration, Mr. Trump's
statement underscored that he does not plan to wait for the swearing in.

He has already upended decades of American policy by
\href{http://www.nytimes.com/2016/12/02/us/politics/trump-speaks-with-taiwans-leader-a-possible-affront-to-china.html?_r=0}{speaking
directly with Taiwan's leader}, and he has spoken out regularly on
events like this week's terrorist attack in Germany. But his push to
stop a United Nations resolution criticizing Israel was more directly
aimed at decisions still being made by his predecessor in his final days
in office.

The move also highlighted the stark shift on Middle East policy ahead
when the new administration takes over in a month. Combined with his
pledge to move the United States Embassy to Jerusalem and his
\href{http://www.nytimes.com/2016/12/15/us/politics/donald-trump-david-friedman-israel-ambassador.html}{selection
of a pro-settlement ambassador} to Israel, Mr. Trump's involvement
Thursday signaled an intent to play an active role in Middle East peace
issues as a strong ally of Israel's.

The Egyptian-sponsored resolution would have condemned Israeli housing
construction in East Jerusalem and the occupied West Bank as a
``flagrant violation under international law'' that was ``dangerously
imperiling the viability'' of a future peace settlement establishing a
Palestinian state. The United States has routinely used its veto at the
Security Council to block similar measures,
\href{http://www.nytimes.com/2011/02/19/world/middleeast/19nations.html}{including
under Mr. Obama in 2011}. But Mr. Obama refused to commit to doing so
again this time.

Mr. Trump said flatly that he should. ``As the United States has long
maintained, peace between the Israelis and the Palestinians will only
come through direct negotiations between the parties and not through the
imposition of terms by the United Nations,'' the president-elect said.
``This puts Israel in a very poor negotiating position and is extremely
unfair to all Israelis.''

Mr. Trump amplified his position by posting the statement on Facebook
and Twitter as well, but a transition official insisted on anonymity to
confirm the president-elect's conversation with Mr. Sisi because of the
sensitivity of the matter. Mr. Trump's words echoed the positions
expressed by Israeli leaders, including Prime Minister Benjamin
Netanyahu, who has welcomed Mr. Trump's election as a breath of fresh
air after years of clashes with Mr. Obama.

According to Security Council Report, an independent research
organization, the United States
\href{http://www.securitycouncilreport.org/atf/cf/\%7B65BFCF9B-6D27-4E9C-8CD3-CF6E4FF96FF9\%7D/SCR-veto-insert-2.pdf}{has
vetoed 30 resolutions} regarding Israel and the Palestinians, plus a
dozen more regarding Israel and Lebanon or Syria, more than half of its
77 vetoes since the United Nations was founded in 1946.

Mr. Netanyahu cited that history on Thursday. ``I hope the U.S. won't
abandon this policy,'' he said. ``I hope it will abide by the principles
set by President Obama himself in his speech in the U.N. in 2011 ---
that peace will come not through U.N. resolutions, but only through
direct negotiations between the parties.''

Frustrated by two failed efforts to broker peace between Israelis and
Palestinians during his tenure, Mr. Obama has been considering an effort
to lay out an American framework during his final days in office.
\href{http://topics.nytimes.com/top/reference/timestopics/subjects/p/palestinians/index.html?inline=nyt-classifier}{Palestinian}
leaders and their allies had hoped he would allow the anti-settlement
resolution at the United Nations to pass as an expression of frustration
at Israeli policies.

A Palestinian delegation traveled to Washington this month to urge Mr.
Obama's team to support the anti-settlement resolution or at least
abstain. Mr. Obama's advisers did not disclose a position and were
holding out until the vote to watch how the matter developed. The
Palestinians were unable to meet with Mr. Trump's aides and expressed
disappointment on Thursday with his position. ``A veto means support of
settlement activities,'' Saeb Erekat, the Palestinian negotiator, said
after the resolution was pulled. ``A veto means abandoning the two-state
solution and peace efforts.''

Asked about Mr. Trump's comments, a visibly upset Palestinian ambassador
to the United Nations, Riyad Mansour, said, ``He is acting on behalf of
Netanyahu.''

The return of the Palestinian cause to the world stage could serve the
interests of some Arab leaders eager to turn public attention away from
troubles at home. The government of Mr. Sisi, which sponsored the
resolution as the Arab representative on the Security Council, faces
domestic challenges stemming from a deteriorating economy, a persistent
Islamic terrorist insurgency and this month's
\href{http://www.nytimes.com/2016/12/11/world/middleeast/cairo-coptic-cathedral-attack.html?ref=world\&_r=0}{bombing
of a Coptic Christian cathedral}.

At the same time, it could distract from Mr. Netanyahu's efforts to
forge stronger relations with Sunni Arab nations on the basis of shared
antipathy toward
\href{http://www.nytimes.com/topic/destination/iran}{Iran}, dominated by
a Shiite theocracy that has threatened Israel's existence and challenged
Arab interests in the region. Arab leaders, who have largely overlooked
the Palestinian issue in recent years, may feel pressured to distance
themselves from Israel again if their own publics are angered at the
treatment of Palestinians.

Egypt backed off on the resolution after Mr. Netanyahu's government put
pressure on Mr. Sisi's government to withdraw it, shortly before Arab
ambassadors meeting at the United Nations endorsed it.

Mr. Netanyahu treated the pending United Nations vote as a crisis,
staying up late into the night discussing it with aides and posting on
his own Twitter account, at 3:28 a.m. local time, a message urging Mr.
Obama to veto what he called the ``anti-Israel'' resolution. ``The
Israelis leaned on the Egyptians this morning to postpone the vote, and
the Egyptians basically caved,'' said a Western official, speaking on
the condition of anonymity because of the diplomatic sensitivity of the
matter.

Arab officials met in Cairo on Thursday night to consider their next
move. ``The negotiations over the Arab proposal for the Israeli
settlements on occupied Palestinian territories are still not finished
at both the United Nations and the Arab League's anti-occupation
committee,'' said Ahmed Abu Zeid, a spokesman for the Egyptian Foreign
Ministry, according to Egyptian state news.

If the White House had let the resolution pass, it would have been a
symbolic blow to the diplomatic shield that the United States has always
offered Israel. It would also have sent a strong signal of international
disapproval over the construction of settlements, regarded by many as
illegal under international law.

A former top Obama adviser suggested that the president should consider
supporting the resolution because settlements are an obstacle to peace
and therefore the real damage to Israel. ``The resolution is about
settlements, not negotiations,'' Martin Indyk, a former special envoy
under Mr. Obama, wrote on Twitter. ``Vetoing would mean vetoing US
policy on settlements.''

But Aaron David Miller, another former Middle East peace negotiator,
said supporting the resolution would have plunged the administration
into an issue that the past several administrations had avoided: the
legality of the settlements.

``The problem with voting for this,'' Mr. Miller said, ``is that Trump
will disavow it and U.S. credibility on the issue will again be
undermined, not to mention what the Israelis might do on the ground in
response, to which the new administration may acquiesce.''

Advertisement

\protect\hyperlink{after-bottom}{Continue reading the main story}

\hypertarget{site-index}{%
\subsection{Site Index}\label{site-index}}

\hypertarget{site-information-navigation}{%
\subsection{Site Information
Navigation}\label{site-information-navigation}}

\begin{itemize}
\tightlist
\item
  \href{https://help.nytimes.com/hc/en-us/articles/115014792127-Copyright-notice}{©~2020~The
  New York Times Company}
\end{itemize}

\begin{itemize}
\tightlist
\item
  \href{https://www.nytco.com/}{NYTCo}
\item
  \href{https://help.nytimes.com/hc/en-us/articles/115015385887-Contact-Us}{Contact
  Us}
\item
  \href{https://www.nytco.com/careers/}{Work with us}
\item
  \href{https://nytmediakit.com/}{Advertise}
\item
  \href{http://www.tbrandstudio.com/}{T Brand Studio}
\item
  \href{https://www.nytimes.com/privacy/cookie-policy\#how-do-i-manage-trackers}{Your
  Ad Choices}
\item
  \href{https://www.nytimes.com/privacy}{Privacy}
\item
  \href{https://help.nytimes.com/hc/en-us/articles/115014893428-Terms-of-service}{Terms
  of Service}
\item
  \href{https://help.nytimes.com/hc/en-us/articles/115014893968-Terms-of-sale}{Terms
  of Sale}
\item
  \href{https://spiderbites.nytimes.com}{Site Map}
\item
  \href{https://help.nytimes.com/hc/en-us}{Help}
\item
  \href{https://www.nytimes.com/subscription?campaignId=37WXW}{Subscriptions}
\end{itemize}
