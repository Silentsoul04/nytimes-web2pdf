Sections

SEARCH

\protect\hyperlink{site-content}{Skip to
content}\protect\hyperlink{site-index}{Skip to site index}

\href{https://www.nytimes.com/section/world/europe}{Europe}

\href{https://myaccount.nytimes.com/auth/login?response_type=cookie\&client_id=vi}{}

\href{https://www.nytimes.com/section/todayspaper}{Today's Paper}

\href{/section/world/europe}{Europe}\textbar{}Intent on Unsettling E.U.,
Russia Taps Foot Soldiers From the Fringe

\url{https://nyti.ms/2iq6964}

\begin{itemize}
\item
\item
\item
\item
\item
\end{itemize}

Advertisement

\protect\hyperlink{after-top}{Continue reading the main story}

Supported by

\protect\hyperlink{after-sponsor}{Continue reading the main story}

\hypertarget{intent-on-unsettling-eu-russia-taps-foot-soldiers-from-the-fringe}{%
\section{Intent on Unsettling E.U., Russia Taps Foot Soldiers From the
Fringe}\label{intent-on-unsettling-eu-russia-taps-foot-soldiers-from-the-fringe}}

\includegraphics{https://static01.nyt.com/images/2016/12/22/world/22Fringe3/22Fringe3-articleLarge.jpg?quality=75\&auto=webp\&disable=upscale}

By \href{http://www.nytimes.com/by/andrew-higgins}{Andrew Higgins}

\begin{itemize}
\item
  Dec. 24, 2016
\item
  \begin{itemize}
  \item
  \item
  \item
  \item
  \item
  \end{itemize}
\end{itemize}

BONY, Hungary --- To his neighbors in a village in western Hungary,
76-year-old Istvan Gyorkos was just an old man who mostly kept to
himself. Hardly anyone looked askance at his passion for guns and for
training youths in paramilitary tactics.

In late October, however, Mr. Gyorkos, a veteran neo-Nazi and the leader
of a tiny fringe outfit called the Hungarian National Front, suddenly
took on a more sinister visage when, according to Hungarian police
officers who raided his home in search of illegal weapons, he shot and
killed a member of the police team with an assault rifle. Members of his
family say the dead policeman was shot by a fellow officer.

The saga then took an even stranger turn: Hungarian intelligence
officials told a parliamentary committee in Budapest that Mr. Gyorkos
had for years been under scrutiny for his role in a network of
extremists linked to and encouraged by Russia. So close was the
relationship, the committee heard, that Russian military intelligence
officers, masquerading as diplomats, staged regular mock combat
exercises using plastic guns with neo-Nazi activists near Mr. Gyorkos's
home.

That Russia, a nation intensely proud of its huge role in the defeat of
Hitler's Germany in World War II, would want anything to do with
marginal, anti-Semitic crackpots who revere Hitler's wartime allies in
Hungary might, at first glance, seem beyond comprehension.

But Andras Racz, a Russia expert at the
\href{http://www.fiia.fi/en/\#tab1}{Finnish Institute of International
Affairs}, said it fit into a scattershot strategy of placing small bets,
directly or through proxies, on ready-made fringe groups in an effort to
destabilize or simply disorient the European Union.

Most of these bets fail, but reaching out to those on the margins costs
little and sometimes hits pay dirt. That happened with Jobbik, a
once-marginal far-right Hungarian group that is now the country's
leading opposition party --- and a big fan of President Vladimir V.
Putin, as is Hungary's prime minister, Viktor Orban.

\includegraphics{https://static01.nyt.com/images/2016/12/22/world/22Fringe6/22Fringe6-articleLarge.jpg?quality=75\&auto=webp\&disable=upscale}

At a time when Russia's relations with the West, or at least with
established parties there, have soured dramatically over Syria, Ukraine
and accusations of interference on all sides, Mr. Putin has enjoyed an
extraordinary run of apparent good luck, as exemplified by the surprise
election victory of Donald J. Trump, who has repeatedly voiced
admiration for the Russian leader. Pro-Russia candidates won
presidential elections recently in
\href{http://www.nytimes.com/2016/11/14/world/europe/pro-russia-candidate-appears-likely-to-win-bulgarian-presidency.html}{Bulgaria}
and
\href{http://www.nytimes.com/2016/10/31/world/europe/moldova-presidential-election-igo-dodon.html}{Moldova},
and France's National Front, which received bank loans worth nearly \$12
million from Russian banks, is now
\href{http://www.nytimes.com/2016/12/17/world/europe/european-union-france-frexit-marine-le-pen.html}{a
serious contender} for the French presidency next year.

Britain, which has generally taken a tough stance on Russia and its
meddling abroad, has turned in on itself amid rancorous internal
struggles over how to leave the European Union after a referendum in
June.

Even in Estonia, a Baltic nation deeply suspicious of Moscow, a party
long reviled as a Russian tool recently
\href{http://www.nytimes.com/2016/11/21/world/europe/estonia-juri-ratas-center-party.html}{took
charge} of a new government.

Each country has its own particular and often very local reasons for its
Russia-friendly turn. Mr. Putin did not engineer the shift
single-handedly, but he has been adept at making his own luck, deploying
Orthodox priests, Russian-funded news media outlets like RT, spies and
computer hackers to ride and help create the wave of populist anger now
battering the foundations of the post-1945 European order.

Mr. Gyorkos, one of the foot soldiers in that assault, is now in jail
but, according to his lawyer, has not yet been formally charged. A few
days before he was accused of opening fire on police officers, a court
in the southern Norwegian town of Tonsberg ordered the detention of Jan
Petrovsky, a longtime local resident of Russian nationality who,
according to a confidential 19-page report by Norway's security service,
belonged to ``a network of people characterized by abnormal interest in
weapons'' and a ``shared enmity towards Norwegian democracy and other
democracies.''

Mr. Petrovsky, the report said, posed a ``threat to fundamental national
interests'' because of his involvement with far-right extremists in
Norway, his trips to eastern Ukraine to fight alongside Russian-backed
separatists and his efforts to recruit Scandinavians to the pro-Russian
cause.

Image

Police officers at the scene of a shooting in Bony, Hungary, in October.
Mr. Gyorkos was accused of fatally shooting an officer.Credit...Csaba
Krizsan/European Pressphoto Agency

There is no evidence that Mr. Petrovsky, 29, acted on instructions from
the Russian state. He instead served a murky Russian nationalist
movement that, under Mr. Putin, has provided muscle for Kremlin-backed
operations to subvert government control in eastern Ukraine and, more
recently,
\href{http://www.nytimes.com/2016/11/26/world/europe/finger-pointed-at-russians-in-alleged-coup-plot-in-montenegro.html}{in
the Balkan nation of Montenegro.}

After his detention in Norway, the immigration authorities stripped him
of his residency permit and sent him back to Russia. Mr. Petrovsky, now
in St. Petersburg helping nationalists there train for combat, declined
to be interviewed. His Oslo lawyer, Nils Christian Nordhus, dismissed
Norway's assessment as untrue and said his client would appeal the
revocation of his Norwegian visa and permanent-residency status.

\href{https://cz.boell.org/sites/default/files/pc_sdi_boll_study_iameurasian.pdf}{Lorant
Gyori,} an analyst with Political Capital, a research group in Budapest
that has studied Russia's outreach to extremist groups, said Russian
methods today mimicked those of the Soviet era, when the K.G.B. had a
department dedicated to ``active measures.'' These went beyond merely
collecting intelligence and included disinformation and subversion,
often involving various front organizations and Moscow-funded fringe
parties that worked to shape, not just spy on, events in foreign
countries.

This department, Section A of the K.G.B.'s First Chief Directorate,
survived the collapse of Communism and now operates as part of Russia's
foreign intelligence service, known as the S.V.R. Russian military
intelligence, the G.R.U., has its own teams expert in subversion,
disinformation and other tools of hybrid warfare.

\hypertarget{casting-a-wide-net}{%
\subsection{Casting a Wide Net}\label{casting-a-wide-net}}

Russia has spread its net wide, reaching out to mainstream parties and
politicians --- like former Chancellor Gerhard Schröder of Germany, who
was given a lucrative job by Russia's state-controlled Gazprom energy
giant --- while also targeting figures widely dismissed as kooks.

Others, like Hungary's prime minister, Mr. Orban, have been attracted by
Mr. Putin's hostility toward liberal democracy and Russia's readiness to
hand out cash, like a \$10 billion loan to Hungary to pay for the
construction by Russia of a nuclear power plant.

Image

Prime Minister Viktor Orban of Hungary, left, meeting with Mr. Putin
outside Moscow in 2014.Credit...Pool photo by Yuri Kochetkov

While polls show that public opinion in Hungary remains far more
favorable to the West than to Russia, which crushed uprisings there in
1848 and 1956, Mr. Orban and the leader of Jobbik have both ditched
their previous hostility toward Moscow and focused their fire on the
West instead, particularly the European Union.

The turnaround by Jobbik has been particularly spectacular and is linked
to the role of Bela Kovacs, an enigmatic Hungarian businessman who
worked for years in Russia. He joined the far-right party when it was
still a struggling band of marginal nationalists in 2005; provided it
with funds to stave off bankruptcy, ostensibly out of his own pocket;
and took charge of its foreign relations. Mr. Kovacs, now a member of
the European Parliament, has been
\href{http://www.nytimes.com/2015/10/15/world/europe/hungary-european-parliament-lifts-immunity-for-spy-suspect.html}{under
investigation} by Hungarian prosecutors since 2014 over suspicions that
he and his Russian-born wife have been recruited as Russian agents.

Widely mocked as KGBela, the businessman has denied any links to Russian
intelligence but has never explained big gaps in his biography, which
include long periods when he disappeared in Russia. Also unexplained is
why he gave Jobbik money and where it came from.

The European Parliament last year lifted his immunity so the
investigation could proceed, but the authorities in Hungary have so far
shown little real interest in pursuing the matter.

The government has shown similar reluctance to probe too deeply into
Russia's links to Mr. Gyorkos, the neo-Nazi in Bony. Those connections
were revealed in October by Index, a well-regarded opposition news media
outlet, and were then confirmed and expanded upon by security officials
who briefed the parliamentary security committee, members of the
committee said.

Mr. Gyorkos, the committee was told, had such close relations with
Russians acting under diplomatic cover at the embassy in Budapest that
they traveled to his remote village as many as five times a year to join
his supporters for games of airsoft, a form of mock combat that involves
the firing of plastic pellets with replica guns. The Russian Embassy in
Budapest did not respond to a request for comment.

Image

Delegates at the International Russian Conservative Forum in St.
Petersburg in 2015.Credit...Anatoly Maltsev/European Pressphoto Agency

Zsolt Molnar, the head of the security committee, said that the
diplomats were believed to be members of Russia's G.R.U. military
intelligence agency, and that the games were a form of military
training.

``It was all entirely legal,'' Mr. Molnar said. ``There was no problem,
and this is precisely the problem.'' He expressed dismay at how easily
and openly supposed Russian diplomats had cultivated ties with violent
and disruptive elements on Hungary's political fringe.

Bernadett Szel, a legislator from Hungary's small green party and a
member of the security committee, said she had this month proposed a
full-scale parliamentary investigation into Russian meddling in Hungary
but the move was blocked by Mr. Orban's governing party, Fidesz. Members
of the security committee from Fidesz declined to comment.

Kolas Gyorkos, the arrested man's son, who is a gunsmith, said he had
not participated in the exercises, organized by his father for
followers, and did not know if any Russians had taken part. He denied
that his father was a neo-Nazi, saying he was simply a Hungarist, a
reference to a Hungarian fascist party set up in the 1930s with much the
same ideology as the Nazis.

The mock military games, which peaked between 2010 and 2012, seem to
have been merely a prelude to what security officials believe was
Russia's primary goal: taking control of a far-right website, Hidfo, or
the Bridgehead, that Mr. Gyorkos's group had set up, and turning it into
a platform for Russian disinformation.

The website, H\href{http://hidfo.net/}{idfo.net}, began as a bulletin
board for rants by members of the Hungarian National Front and other
extremist groups but has since switched its server to Russia --- it is
now Hidfo.ru --- and serves as a portal for more sober but heavily
slanted articles on military and geopolitical affairs with a decidedly
pro-Russian tilt.

Image

Bela Kovacs of Jobbik, a businessman who worked for years in Russia and
is a member of the European Parliament.Credit...Yuri Kochetkov/European
Pressphoto Agency

It is also an outlet for fake news, including an invented report in 2014
that Hungary was sending tanks to Ukraine, which set off a diplomatic
incident. Recent reports, all false, asserted that the United States
Department of Homeland Security had declared the November presidential
election free of any cyberattack; that Austria wanted to lift sanctions
against Russia; and that NATO's secretary general had pledged to make
European nations vassals of Washington. A special section offered a
Russian expert's opinions on how the United States and its allies use
hybrid warfare to undermine their rivals around the world.

\hypertarget{efforts-in-scandinavia}{%
\subsection{Efforts in Scandinavia}\label{efforts-in-scandinavia}}

Russian efforts to disrupt the normal functioning of democracy have also
been on display in Scandinavia. There, an extremist and avowedly
revolutionary outfit called Nordic Resistance has formed a curious
alliance with the Russian Imperial Movement, a far-right group that,
while not sponsored by the Russian state, has helped the Kremlin by
recruiting Russian fighters for the conflict in eastern Ukraine.

The Russian group announced last year that it had given an unspecified
``monetary sum'' to Nordic Resistance, but the Russian group's leader,
Stanislav Vorobyov, said in a recent interview that this amounted to
just 150 euros.

His group has nonetheless played a prominent role in rallying extremists
from Europe and the United States into a common front against what they
see as a globalized elite out of touch with their people and traditional
values. It joined a Russian political party, Rodina, in organizing
\href{http://www.nytimes.com/2015/03/23/world/europe/right-wing-groups-find-a-haven-for-a-day-in-russia.html}{a
conference} in March 2015 in St. Petersburg that was attended by white
supremacists from the United States like Jared Taylor and many of
Europe's most prominent far-right figures. Mr. Petrovsky, the Russian
recently expelled from Norway, also attended.

Thor Bach, a Norwegian youth worker who has followed far-right extremism
in Norway for decades, said the influx of new blood, ideas and possibly
even money from Russia had helped revive what had until recently, at
least in Norway, been a moribund cause.

``The neo-Nazi scene here was dead, but it has had a reawakening this
year,'' he said. ``Someone in Russia thinks it is a good idea to support
neo-Nazis in Scandinavia.''

He said that there was no evidence of direct support by the Russian
state but that there had clearly been an intermingling of Russian and
Scandinavian extremists who all see Mr. Putin as a standard-bearer for
muscular nationalism. ``All the loonies are gathering under the banner
of Putin, and now also Trump,'' he said.

Advertisement

\protect\hyperlink{after-bottom}{Continue reading the main story}

\hypertarget{site-index}{%
\subsection{Site Index}\label{site-index}}

\hypertarget{site-information-navigation}{%
\subsection{Site Information
Navigation}\label{site-information-navigation}}

\begin{itemize}
\tightlist
\item
  \href{https://help.nytimes.com/hc/en-us/articles/115014792127-Copyright-notice}{©~2020~The
  New York Times Company}
\end{itemize}

\begin{itemize}
\tightlist
\item
  \href{https://www.nytco.com/}{NYTCo}
\item
  \href{https://help.nytimes.com/hc/en-us/articles/115015385887-Contact-Us}{Contact
  Us}
\item
  \href{https://www.nytco.com/careers/}{Work with us}
\item
  \href{https://nytmediakit.com/}{Advertise}
\item
  \href{http://www.tbrandstudio.com/}{T Brand Studio}
\item
  \href{https://www.nytimes.com/privacy/cookie-policy\#how-do-i-manage-trackers}{Your
  Ad Choices}
\item
  \href{https://www.nytimes.com/privacy}{Privacy}
\item
  \href{https://help.nytimes.com/hc/en-us/articles/115014893428-Terms-of-service}{Terms
  of Service}
\item
  \href{https://help.nytimes.com/hc/en-us/articles/115014893968-Terms-of-sale}{Terms
  of Sale}
\item
  \href{https://spiderbites.nytimes.com}{Site Map}
\item
  \href{https://help.nytimes.com/hc/en-us}{Help}
\item
  \href{https://www.nytimes.com/subscription?campaignId=37WXW}{Subscriptions}
\end{itemize}
