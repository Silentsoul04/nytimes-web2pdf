Sections

SEARCH

\protect\hyperlink{site-content}{Skip to
content}\protect\hyperlink{site-index}{Skip to site index}

\href{https://www.nytimes.com/section/politics}{Politics}

\href{https://myaccount.nytimes.com/auth/login?response_type=cookie\&client_id=vi}{}

\href{https://www.nytimes.com/section/todayspaper}{Today's Paper}

\href{/section/politics}{Politics}\textbar{}Inside the Trump
Organization, the Company That Has Run Trump's Big World

\url{https://nyti.ms/2hkHkuD}

\begin{itemize}
\item
\item
\item
\item
\item
\end{itemize}

Advertisement

\protect\hyperlink{after-top}{Continue reading the main story}

Supported by

\protect\hyperlink{after-sponsor}{Continue reading the main story}

\hypertarget{inside-the-trump-organization-the-company-that-has-run-trumps-big-world}{%
\section{Inside the Trump Organization, the Company That Has Run Trump's
Big
World}\label{inside-the-trump-organization-the-company-that-has-run-trumps-big-world}}

\includegraphics{https://static01.nyt.com/images/2016/12/26/us/26TRUMPORG1/26TRUMPORG1-articleInline.jpg?quality=75\&auto=webp\&disable=upscale}

By \href{https://www.nytimes.com/by/megan-twohey}{Megan Twohey},
\href{http://www.nytimes.com/by/russ-buettner}{Russ Buettner} and
\href{http://www.nytimes.com/by/steve-eder}{Steve Eder}

\begin{itemize}
\item
  Dec. 25, 2016
\item
  \begin{itemize}
  \item
  \item
  \item
  \item
  \item
  \end{itemize}
\end{itemize}

When Tiah Joo Kim arrived at the Manhattan headquarters of the Trump
Organization to pitch a hotel and condominium project in Vancouver,
British Columbia, he expected the famous company with ventures across
the globe to come with capacious offices and a staff of hundreds.
Instead, he was led through a mere two floors with what appeared to be
no more than a few dozen employees. ``Lean,'' Mr. Tiah, a young
Malaysian developer, remembers thinking as he walked the halls.

The first stop was a conference room, where Mr. Tiah was required to
sell his vision to the boss's three oldest children. Only after securing
their support did he advance to the inner sanctum, with its sweeping
views of Central Park.

Mr. Tiah was not sure what to expect from the man whose face was beamed
around the world through the reality television show ``The Apprentice,''
but the conversation that afternoon in 2012 was casual and warm. Donald
J. Trump spent more time showing off a Shaquille O'Neal shoe and a Mike
Tyson championship belt --- prize artifacts from his display of sports
memorabilia --- than interrogating Mr. Tiah on the details of his
business plan. ``You're a good-looking guy,'' Mr. Tiah recalled Mr.
Trump telling him as he gave the project his blessing.

Then Mr. Trump's trusted lawyers and other top executives swooped in to
play hardball --- working alongside Donald Trump Jr. to negotiate the
confidential agreements that would allow the Vancouver development to be
branded with Mr. Trump's name and managed by his company. The talks
consumed 16-hour days for nearly a week, Mr. Tiah said, explaining: ``It
was tiring. They're tough.''

That is the way business has been done at the Trump Organization, a
relatively small company with a big reach and a bigger self-image that
has come under intense scrutiny as its chief prepares to become
president of the United States.

With extensive entanglements around the world, many packaged in a
network of licensing agreements and limited liability companies, the
Trump Organization poses a raft of potential conflicts of interest for a
president-elect who has long exerted such control over his company that,
as he told The New York Times in
\href{https://www.nytimes.com/2016/11/23/us/politics/trump-new-york-times-interview-transcript.html}{a
recent interview}, he is the one who signs the checks. ``I like to sign
checks so I know what is going on,'' he explained.

Mr. Trump --- owner of all but the smallest sliver of the privately held
company --- has said that, while the law does not require it, he is
formulating plans to remove himself and his older daughter, Ivanka, from
the company's operations. (Ms. Trump's husband, Jared Kushner, is likely
to have a role in the White House.) His sons Donald Jr. and Eric, along
with other executives, will be in charge, the president-elect
\href{https://twitter.com/realDonaldTrump/status/808529888630239232}{wrote
on Twitter} in mid-December, adding that ``no new deals will be done
during my term(s) in office.'' People involved in the planning have said
that Mr. Trump intends to keep a stake in the business.

But in recent weeks, amid rising pressure, Mr. Trump and his advisers
have
been\href{https://www.nytimes.com/2016/12/24/us/politics/trump-family-conflicts.html}{intensely
debating further measures}. Among other things, the president-elect has
agreed to shut down his personal foundation, has ended some
international development deals and has reviewed a plan for an outside
monitor to oversee the Trump Organization.

Yet an examination of the company underscores the complex challenges of
taking Mr. Trump out of Trump the organization.

His company is a distinctly family business fortified with longtime
loyalists that operates less on standardized procedures and more on a
culture of Trump. Mr. Trump may leave the details of contracts to his
deputies, but his name --- and influence --- is stamped on every deal
the company does.

In an interview last spring with The Times, Mr. Trump explained that he
approved new ventures based on his personal ``feel.'' And while in
recent years his three oldest children have taken on more of a
leadership role, Mr. Trump has the final say, sometimes weighing in on
the most minute design details of planned hotels, golf courses or other
properties the company owns or manages.

His other top executives --- many of them natives of Queens, where Mr.
Trump grew up, or Brooklyn, where his father, Fred, expanded a housing
empire many years ago --- have secured power not necessarily through
fancy pedigrees or impressive credentials, but through decades of
devotion to their boss.

Allen Weisselberg, the organization's chief financial officer, started
off as an accountant for Mr. Trump's father. Matthew Calamari, the
organization's chief operating officer, was recruited in 1981 after Mr.
Trump saw him eject some hecklers while working security at the United
States Open tennis tournament.

For some executives, there appears to be little division between their
service to the company and their service to the Trumps.

``We're not a publicly traded company. At the end of the day, I work for
the Trump family,'' Alan Garten, the general counsel,
\href{http://www.corpcounsel.com/id=1202771661378/Trumps-GC-Donald-is-a-Demanding-but-Fair-Boss}{explained
in an interview} with the legal industry publication Corporate Counsel
shortly before the election. ``That's how I view my job. Whether it's
protecting their business interests or protecting their personal
interests. I am here to assist them and represent them in any way they
need.''

When asked to elaborate in an interview last week with The Times, Mr.
Garten said that in any job, ``you want to be as helpful as you can,''
but that ``obviously the interests of the Trumps and the interests of
the company are two distinct things.''

The divisions between business and politics were often fuzzy during the
presidential race: Mr. Garten became a ``liaison'' to Mr. Trump's
campaign; Michael Cohen, an executive vice president, tirelessly
promoted his boss's bid for the White House on television while battling
negative media coverage; and Jason Greenblatt, the company's chief legal
officer, began serving as his adviser on Israel. On Friday, it was
announced that Mr. Greenblatt would be joining Mr. Trump's
administration as a special representative for international
negotiations.

After the election, other lines continued to blur as the president-elect
and his children met with foreign businessmen with connections to their
global ventures and with foreign officials with potential influence over
their business dealings.

Some government-ethics lawyers have warned that unless Mr. Trump fully
divests himself from the company and places someone independent of his
family in charge, he risks entering the White House in violation of a
\href{https://www.nytimes.com/2016/11/21/us/politics/donald-trump-conflict-of-interest.html}{constitutional
clause} that forbids him from taking payments or gifts from a foreign
government entity.

As Mr. Trump assumes the presidency, it is difficult to foresee him
walling himself off from the company entirely, said Michael D'Antonio,
the author of a critical biography, ``The Truth About Trump.''

``I don't think that he could keep himself from inquiring about the
performance of these businesses any more than he can keep himself from
tweeting,'' Mr. D'Antonio said. ``It is just too vital to his identity.
Profit is the way he has always measured himself. I don't see how he can
stop.''

\includegraphics{https://static01.nyt.com/images/2016/12/26/us/26TRUMPORG2/26TRUMPORG2-articleInline.jpg?quality=75\&auto=webp\&disable=upscale}

\hypertarget{mom-and-pop-shop}{%
\subsection{`Mom and Pop' Shop}\label{mom-and-pop-shop}}

Mr. Trump may have business interests around the world, but his power is
concentrated at a single Midtown Manhattan address: 725 Fifth Avenue.
With a gleaming exterior that shoots to the sky, a lobby decked with
marble and a collection of high-end tenants, Trump Tower is his primary
residence as well as his company's headquarters.

To get to work, Mr. Trump steps onto the private elevator in his gilded
three-story penthouse, presses 26 and waits a matter of seconds. When
the doors open, he is at his office, surrounded by Mr. Garten, Mr.
Weisselberg and other top executives. One floor down are the offices of
Donald Jr., Eric and Ivanka Trump, who joined the company in the 2000s
and are now his top deputies and advance guard.

David Brecher, the chief executive of FM Home Loans, visited the Trump
Organization about a decade ago to discuss a potential partnership and
found the aesthetics telling.

``Donald's floor,'' he said, recalling a swirl of gold trim and hues,
``is very his style.''

``The kids,'' by contrast, ``have a very cool floor. Sleek. Marble.''

Image

Ivanka Trump, Mr. Trump's elder daughter, in her office at Trump Tower.
She is considered to be the second-most powerful person at the
organization. The president-elect has said that he is formulating plans
to remove himself and Ivanka from the company's
operations.Credit...Rebecca Greenfield

Photographs of Mr. Trump with the rich and powerful adorn his office
walls, and his desk often overflows with papers, evidence of his refusal
to communicate by email.

When Mr. Trump wants to talk to someone, he calls out to his assistant,
Rhona Graff, a Queens native whose office is right outside his door. She
has been his gatekeeper for decades. Anyone seeking access to him over
the phone has to go through Ms. Graff, sometimes with a secret code.

Mr. Trump often boasts of the size of the Trump Organization. ``It's a
big company,'' he said in the interview last spring. A spokeswoman said
the business employed ``tens of thousands.''

But industry experts estimate that no more than 4,000 people work for
the Trump Organization worldwide. And executives say that the three
floors that make up the headquarters appear to have no more than 150
employees.

It is a family business, as everyone involved is quick to explain. And
the management structure is informal if not confusing, with deputies
constantly buzzing in and out of the boss's office.

``We kind of run a little bit like a mom-and-pop in that sense,'' Donald
Trump Jr. said in a 2011 deposition for a lawsuit involving a Florida
development. ``I guess there is an organizational chart, but in theory,
there is not too many levels.'' He added: ``Could I make one? Yes. Is
there one officially? Not that I'm aware of.''

Indeed, the elder Mr. Trump has tended to collect executives and assign
duties through personal preference.

In 2004, Michelle Carlson was a young lawyer determined to move to
California when a friend suggested that she meet with an acquaintance
who could prove useful. She entered Mr. Trump's office hoping to secure
a recommendation she could use to find work with real estate developers
in Los Angeles, and she encountered a warm welcome.

``I heard there was this nice Atlanta girl in the lobby,'' she
remembered Mr. Trump saying as he offered her a seat. Then came a series
of direct questions: What were her responsibilities at her current job?
How did she view her own strengths? In what areas did she want to grow?

Forty-five minutes later, Mr. Trump was convinced: ``I'm not going to
give you any recommendations in L.A. I'm going to hire you,'' Mr. Trump
told Ms. Carlson, who went on to spend almost four years as his
assistant general counsel, often working 18-hour days with a small team
of lawyers while taking on other responsibilities in the real estate
division.

Andrew Weiss, a Romanian immigrant who grew up in Brooklyn, was hired
straight out of graduate school in 1981, just as Mr. Trump was starting
to make his mark. Thirty-five years later, having weathered many highs
and lows with Mr. Trump, including the spectacular failure of his
Atlantic City casinos, Mr. Weiss is still by his side, as executive vice
president for development and construction.

Mr. Calamari, who started out as a bodyguard, also saw his role expand
as he remained committed to his boss. Five years ago, his son Matthew
Calamari Jr. joined the Trump Organization as a security guard. Today,
he is the director of surveillance. Brian Baudreau, the general manager
of the Trump International Hotel Las Vegas, began as a driver for Mr.
Trump.

``My father knows how to find talent in people,'' Eric Trump said,
recalling how Mr. Baudreau used to chauffeur him to school. ``He's
totally family,'' he added.

Devotion is rewarded.

``To succeed in this company,'' Mr. Garten said, ``you have to be
skilled, highly dedicated and highly loyal.''

Some appear to be hired based on other calculations.

For more than a decade, Ronald C. Lieberman oversaw the concession
contracts for the New York City Department of Parks and Recreation, a
post that required him to represent the interests of the city in a
variety of deals with Mr. Trump.

Then, in 2007, Mr. Lieberman began working for a new employer. In his
job as executive vice president for management and development at the
Trump Organization, he has helped Mr. Trump win contracts to operate the
Central Park carousel and the Ferry Point golf course in the Bronx, the
very projects he handled on behalf of the city for years.

Ivanka Trump is generally seen as the second-most powerful person at the
Trump Organization, while the 11 other executive vice presidents are all
men --- and all white.

There have been other senior female executives, like Cathy Hoffman
Glosser, who oversaw the Trump Organization's expansion into branding
deals, part of its shift from building and buying real estate to selling
the Trump name. (She left the company last year and did not respond to
interview requests. Ms. Carlson said she left by choice to care for her
baby, even though Mr. Trump made earnest attempts to keep her.)

Mr. Garten said that outside the top executive ranks, ``there's greater
diversity in terms of gender and ethnicity,'' adding, ``I don't have the
numbers in front of me.''

Jill Martin, a vice president and assistant general counsel for
litigation and employment, said in an interview last spring that
diversity at the company was ``less forced'' than at the law firms where
she previously worked.

``With the firms, there was a lot of attention placed on gender and
ethnicity and trying to find the balance,'' she said. ``With the Trump
Organization, I just felt like those things really fall by the wayside.
What's important is someone's individual drive and talent.''

When Mr. Tiah was at Trump Tower to discuss the Vancouver partnership,
he could not help noticing that female employees seemed to have
something else in common.

``You have to be attractive?'' he remembers thinking. ``Is that a
requirement?''

Image

Mr. Trump with the casino tycoon Phil Ruffin in 2005 at the
ribbon-cutting ceremony for the Trump International Hotel Las Vegas. Mr.
Ruffin remarked how heavily Mr. Trump was involved in the particulars.
``He didn't want just a TV in the bathroom; it had to be in the mirror
so you can watch when you're shaving,'' he said.Credit...Ethan
Miller/Getty Images

\hypertarget{steeped-in-the-details}{%
\subsection{Steeped in the Details}\label{steeped-in-the-details}}

It was the mid-2000s, and Phil Ruffin was in search of a partner to
develop a combined hotel and condominium tower on the Las Vegas Strip.
Mr. Ruffin, a casino tycoon, owned the land, but he needed an investor,
a brand name to license and a team to manage the construction and
operations of the property.

Mr. Trump did not simply say yes to all three, Mr. Ruffin recalled. He
threw himself into the details of the deal, pushing a bank to cut the
interest rate on a loan by half, insisting that subcontractors lower
their prices and requiring that everything about the 64-story tower
reflect his taste.

``We'd tour, and he'd say, `This is wrong; this is right,''' Mr. Ruffin
said. ``The glass shower had to be etched glass because that's the Trump
way, more expensive. He didn't want just a TV in the bathroom; it had to
be in the mirror so you can watch when you're shaving.''

Mr. Trump, he said, remained actively involved when the financial crisis
hit in 2008, threatening the financial viability of the Las Vegas
venture, and the two men flew to Washington to meet with a tax lawyer.
As they pulled up chairs in his office, the lawyer encouraged the men to
cut their losses and declare bankruptcy. It would provide them with a
handsome tax deduction.

But Mr. Trump was adamant. ``He said: `This is not Atlantic City; this
is Las Vegas. I think it will recover,''' Mr. Ruffin said. Instead, he
and Mr. Trump poured more money into the venture and continued to move
forward.

Mr. Trump's children have taken on increasing responsibility in recent
years; they often solicit new projects and are the primary liaisons with
partners. Two years ago, Eric Trump became the Trump Organization's main
point of contact for the Las Vegas tower, Mr. Ruffin said. Ivanka Trump
initiated the leasing of the old Old Post Office building in Washington,
envisioning it as a new Trump hotel.

But Mr. Trump has the final say on most deals, especially those
involving his own money.

He has signed the licensing agreements, the leases --- and the big
checks. And the tangle of limited liability companies used to structure
all of his deals revolve around a single point of power: Mr. Trump. As
one former executive described it, the company is the ``hub of a wheel,
and he's in the middle.''

The company adheres to few formal corporate guidelines or procedures.

When determining whether and how to enter business partnerships, nothing
is decided by established committee, or through written recommendation
by the children, Donald Trump Jr. explained in the 2011 deposition in
the Florida case.

``Other companies can operate like bureaucrats'' Mr. Tiah said.
``They're not like that.''

Even so, the executives are known for playing tough.

When seeking \$470,000 in outstanding legal bills from the Trump
Organization a decade ago, the lawyer Y. David Scharf accidentally
included a single page of a separate legal bill to another client, the
business magnate Carl C. Icahn.

How did Mr. Weisselberg, Mr. Trump's chief financial officer, respond?

``Mr. Weisselberg threatened to call Mr. Icahn and utilize this
inadvertent clerical error in an effort to embarrass Mr. Scharf and my
firm --- unless my firm agreed to a 50 percent discount on the
outstanding legal bills,'' David A. Piedra, a partner in Mr. Scharf's
firm, Morrison Cohen, wrote in a 2007 letter to a lawyer representing
Mr. Trump.

``As I am sure you realize,'' he wrote, ``this threat, which smacks of
extortion, is entirely inappropriate.''

Mr. Scharf said in an interview that his firm had resolved the matter
and bore no ill will toward Mr. Weisselberg, the Trump Organization or
Mr. Trump.

At the time of the election, Mr. Trump's company was party to at least
75 lawsuits across the country, according to a nationwide tally by USA
Today.

Mr. Garten said that not all of the lawsuits were substantial, but
acknowledged that ``we're extremely hands-on and meticulous in the legal
aspects of the business.''

Image

Donald Trump Jr. with Matthew Calamari, the chief operating officer of
the company, at Trump National Golf Club Westchester in Briarcliff
Manor, N.Y. Mr. Calamari was recruited in 1981 after Mr. Trump saw him
eject some hecklers while working security at the United States Open
tennis tournament.Credit...Bobby Bank/Getty Images

\hypertarget{formidable-loyalties}{%
\subsection{Formidable Loyalties}\label{formidable-loyalties}}

As their boss advanced in the 2016 presidential race, Mr. Trump's
executives remained fierce and aggressive.

When The Daily Beast was preparing to publish an article about Mr.
Trump's first wife, Ivana, alleging in a divorce deposition that he had
raped her, Michael Cohen, one of the organization's executive vice
presidents,
\href{https://www.nytimes.com/politics/first-draft/2015/07/28/donald-trump-aide-apologizes-for-saying-you-cant-rape-your-spouse/?_r=0}{wrongly
insisted it was impossible} for a husband to rape his wife and made
threats. He warned that if the reporter moved ahead with the article,
``I'm going to mess your life up,'' according to
\href{http://www.thedailybeast.com/articles/2015/07/27/ex-wife-donald-trump-made-feel-violated-during-sex.html}{The
Daily Beast's account}.

It was just one of the many ways that Mr. Cohen had cultivated the image
of a pit bull, a reputation he said was well deserved.

``Mr. Trump is more than just a boss to those of us who have been
fortunate enough to be close to him, both professionally and
personally,'' he said in an interview. ``He's more like a patriarch, a
mentor. These qualities make him very endearing to me, which is why I am
so fiercely loyal to him and committed to protecting him at all costs.''

He was not the only seemingly tireless proponent --- and protector ---
of Mr. Trump's political pursuits. Mr. Garten, the general counsel,
defended his boss's record and fought back against allegations that he
had groped women and engaged in other sexual misconduct. At points it
appeared as if he were threatening legal action on a daily basis against
anyone who criticized Mr. Trump, including The Times and other news
outlets.

Last December, after Mr. Garten dangled the possibility of legal action
against a ``super PAC'' promoting Jeb Bush and sent a cease-and-desist
letter to an anti-tax group that ran \$1 million in ads against Mr.
Trump,
\href{http://www.nytimes.com/politics/first-draft/2015/12/09/lawyer-for-pacs-backing-jeb-bush-asks-f-e-c-to-investigate-donald-trump/}{supporters
of Mr. Bush complained to the Federal Election Commission} that the
Trump Organization was illegally acting as an agent for the Trump
campaign.

``Trump and his agents have explicitly directed his corporate attorneys
at the Organization to do the dirty work for the campaign,'' a lawyer
wrote in the complaint, which is pending.

Six months later, Mr. Garten began to appear in the campaign's financial
reports. In the end, he was compensated by Mr. Trump for about \$24,000
of legal work for the campaign and donated thousands more dollars' worth
of services as an in-kind contribution.

Mr. Garten said he saw many of the attacks on Mr. Trump as an attack on
the company. It was his job to fight back, he said.

\hypertarget{a-real-family-affair}{%
\subsection{`A Real Family Affair'}\label{a-real-family-affair}}

The president-elect's deliberations over how to separate himself from
his company coincided with one of its oldest and most celebrated
traditions.

Jill Cremer, a former vice president at the Trump Organization, fondly
recalls company Christmas parties at the Plaza Hotel, the Pierre or the
Rainbow Room. Mr. Trump would hand out prizes --- airline tickets,
luggage, cameras --- and would pose with employees for photos.

``The Christmas party was always the highlight,'' Ms. Cremer said. ``It
was a real family affair.''

This year, the celebration fell on Dec. 14, two nights after Mr. Trump
said he would
\href{https://www.nytimes.com/2016/12/12/us/politics/donald-trump-postpones-announcement-on-business-conflicts.html}{postpone
announcing} the details of his plan for the stewardship of the company.

He was facing a flurry of activity, including making cabinet picks and
navigating calls with foreign leaders, but he found the time to stop by
the atrium of Trump Tower, where the party has been held in recent
years.

As hundreds of Trump Organization employees and guests nibbled on steak
from Trump Grill and sipped wine from Trump Winery, Mr. Trump thanked
the crowd for helping build the company that bears his name.

``You could see the love in the room,'' Eric Trump said.

Advertisement

\protect\hyperlink{after-bottom}{Continue reading the main story}

\hypertarget{site-index}{%
\subsection{Site Index}\label{site-index}}

\hypertarget{site-information-navigation}{%
\subsection{Site Information
Navigation}\label{site-information-navigation}}

\begin{itemize}
\tightlist
\item
  \href{https://help.nytimes.com/hc/en-us/articles/115014792127-Copyright-notice}{©~2020~The
  New York Times Company}
\end{itemize}

\begin{itemize}
\tightlist
\item
  \href{https://www.nytco.com/}{NYTCo}
\item
  \href{https://help.nytimes.com/hc/en-us/articles/115015385887-Contact-Us}{Contact
  Us}
\item
  \href{https://www.nytco.com/careers/}{Work with us}
\item
  \href{https://nytmediakit.com/}{Advertise}
\item
  \href{http://www.tbrandstudio.com/}{T Brand Studio}
\item
  \href{https://www.nytimes.com/privacy/cookie-policy\#how-do-i-manage-trackers}{Your
  Ad Choices}
\item
  \href{https://www.nytimes.com/privacy}{Privacy}
\item
  \href{https://help.nytimes.com/hc/en-us/articles/115014893428-Terms-of-service}{Terms
  of Service}
\item
  \href{https://help.nytimes.com/hc/en-us/articles/115014893968-Terms-of-sale}{Terms
  of Sale}
\item
  \href{https://spiderbites.nytimes.com}{Site Map}
\item
  \href{https://help.nytimes.com/hc/en-us}{Help}
\item
  \href{https://www.nytimes.com/subscription?campaignId=37WXW}{Subscriptions}
\end{itemize}
