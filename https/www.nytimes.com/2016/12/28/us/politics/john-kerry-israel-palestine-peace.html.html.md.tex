Sections

SEARCH

\protect\hyperlink{site-content}{Skip to
content}\protect\hyperlink{site-index}{Skip to site index}

\href{https://www.nytimes.com/section/politics}{Politics}

\href{https://myaccount.nytimes.com/auth/login?response_type=cookie\&client_id=vi}{}

\href{https://www.nytimes.com/section/todayspaper}{Today's Paper}

\href{/section/politics}{Politics}\textbar{}Kerry Rebukes Israel,
Calling Settlements a Threat to Peace

\url{https://nyti.ms/2htfVqH}

\begin{itemize}
\item
\item
\item
\item
\item
\item
\end{itemize}

Advertisement

\protect\hyperlink{after-top}{Continue reading the main story}

Supported by

\protect\hyperlink{after-sponsor}{Continue reading the main story}

\hypertarget{kerry-rebukes-israel-calling-settlements-a-threat-to-peace}{%
\section{Kerry Rebukes Israel, Calling Settlements a Threat to
Peace}\label{kerry-rebukes-israel-calling-settlements-a-threat-to-peace}}

\includegraphics{https://static01.nyt.com/images/2016/12/28/world/29KERRY-video/29KERRY-video-videoSixteenByNine3000.jpg}

By \href{http://www.nytimes.com/by/david-e-sanger}{David E. Sanger}

\begin{itemize}
\item
  Dec. 28, 2016
\item
  \begin{itemize}
  \item
  \item
  \item
  \item
  \item
  \item
  \end{itemize}
\end{itemize}

WASHINGTON --- Secretary of State John Kerry accused Prime Minister
Benjamin Netanyahu of Israel on Wednesday of thwarting peace in the
Middle East, speaking with a clarity and harshness almost never heard
from American diplomats when discussing one of their closest and
strongest allies.

With only 23 days left in his four-year turn as secretary of state,
during which he made the search for peace in the Middle East one of his
driving missions, Mr. Kerry said the Israeli government was undermining
any hope of a two-state solution to its decades-long conflict with the
Palestinians.

The American vote last week in the United Nations allowing the
condemnation of Israel for settlements in the West Bank and East
Jerusalem, he added, was driven by a desire to save Israel from ``the
most extreme elements'' in its own government.

``The status quo is leading toward one state and perpetual occupation,''
Mr. Kerry said, his voice animated.

\href{https://www.state.gov/secretary/remarks/2016/12/266119.htm}{His
speech} was a powerful admonition after years of tension and
frustration, with the Obama administration giving public voice to its
long-held concern that Israel was headed off a cliff toward
international isolation and was condemning itself to a future of
low-level, perpetual warfare with the Palestinians.

Reaction was immediate and harsh, not only from Mr. Netanyahu, but also
from Senators John McCain, Republican of Arizona, and Chuck Schumer,
Democrat of New York. President-elect Donald J. Trump did not even wait
for Mr. Kerry to speak before condemning the secretary of state.

The United States and Israel are in the middle of a breach rarely seen
since President Harry S. Truman recognized the fragile Israeli state in
May 1948. In a direct response to Mr. Netanyahu's barb over the weekend
that ``friends don't take friends to the Security Council'' --- a
reference to the Obama administration's decision to abstain from the
resolution condemning the building of new settlements in disputed
territory --- Mr. Kerry said the United States acted out of a deeper
understanding of the meaning of its alliance.

``Some seem to believe that the U.S. friendship means the U.S. must
accept any policy, regardless of our own interests, our own positions,
our own words, our own principles --- even after urging again and again
that the policy must change,'' he said. ``Friends need to tell each
other the hard truths, and friendships require mutual respect.''

Toward the end of his 70-minute speech in the State Department's
auditorium, Mr. Kerry acknowledged that Mr. Trump may well abandon the
major principles that the United States has used for decades of Middle
East negotiations, including the two-state solution that both
Republicans and Democrats support. Mr. Trump is
\href{https://www.nytimes.com/2016/12/16/world/middleeast/david\%2Dfriedman\%2Dus\%2Dambassador\%2Disrael.html?_r=0}{nominating
a new American ambassador, David M. Friedman}, who has broken with even
the pretense of supporting a two-state negotiated agreement and has
helped fund some of the settlements Mr. Kerry denounced.

\includegraphics{https://static01.nyt.com/images/2016/12/29/world/JP-29ASSESS/JP-29ASSESS-videoSixteenByNineJumbo1600.jpg}

On vacation in Palm Beach, Fla., Mr. Trump posted two Twitter messages
rejecting the speech before it was delivered. ``We cannot continue to
let Israel be treated with such total disdain and disrespect,'' he wrote
on Wednesday morning. After assailing the nuclear deal in Iran and last
week's vote at the Security Council, he said, ``Stay strong Israel,
January 20th is fast approaching!''

He was soon praised --- also on Twitter --- by Mr. Netanyahu, who later
released a video statement that was unsparingly direct and dismissive of
Mr. Kerry.

``The entire Middle East is going up in flames, entire countries are
toppling, terrorism is raging and for an entire hour the secretary of
state attacks the only democracy in the Middle East,'' Mr. Netanyahu
said. ``Maybe Kerry did not notice that Israel is the only place in the
Middle East where Christmas can be celebrated in peace and security.
Sadly, none of this interests the secretary of state.''

Mr. Kerry's speech was criticized at home as well.

Mr. McCain called it a ``pointless tirade,'' while Mr. Schumer, the
incoming Senate Democratic leader, said he feared that Mr. Kerry had
``emboldened extremists on both sides.''

Mr. Kerry did make note of the Palestinian violence, the ``extremist
agenda'' of Hamas, and the Palestinian unwillingness to recognize
Israel. All, he said, were at the heart of the conflict. But Mr.
Netanyahu's continued support of settlements, ``strategically placed in
locations that make two states impossible,'' he said, is driving a
solution further and further away.

Mr. Kerry argued that Israel, with a growing Arab population, could not
survive as both a Jewish state and a democratic state unless it embraced
the two-state approach that a succession of American presidents have
endorsed.

Mahmoud Abbas, the Palestinian president, responded to Mr. Kerry's
speech by calling on Israel to freeze housing construction in order to
restart negotiations. ``The minute the Israeli government agrees to
cease all settlement activities, including in and around occupied East
Jerusalem, and agree to implement the signed agreements on the basis of
mutual reciprocity, the Palestinian leadership stands ready to resume
permanent status negotiations,'' he said.

Mr. Netanyahu has said he is willing to meet Mr. Abbas anytime for talks
as long as there are no preconditions.

It was notable that it was Mr. Kerry who delivered the speech rather
than President Obama, who has long kept a distance from Middle East
peace negotiations, a pursuit he has always doubted would succeed. After
talks at Camp David collapsed in 2000, it was President Bill Clinton
himself who gave a speech laying out the parameters of an ultimate deal,
about 10 days before leaving office in 2001.

At the time, Mr. Clinton also censured Israel for its settlements, but
in far more measured terms. Mr. Kerry called them a violation of
international law, a position he said the State Department had taken
since 1978.

``The Israeli prime minister publicly supports a two-state solution, but
his current coalition is the most right-wing in Israeli history, with an
agenda driven by its most extreme elements,'' he said. ``The result is
that policies of this government --- which the prime minister himself
just described as `more committed to settlements than any in Israel's
history' --- are leading in the opposite direction, towards one state.''

Seldom in modern American diplomacy has an American administration so
directly confronted --- and disavowed --- a close ally's actions as Mr.
Kerry did on Wednesday, dropping most of the restraint he had shown in
public over the past four years. One of the last times was during the
Eisenhower administration, when the United States broke with Britain,
France and Israel over the 1956 invasion of the Egyptian Sinai.
Eisenhower had warned against the invasion and threatened to harm
Britain's financial system in retaliation.

When Mr. Kerry got to the principles for a future settlement, they were
unsurprising. Many date to the 1990s or earlier, and many to past United
Nations resolutions.

The principles he described started with a ``secure and recognized
border between Israel and a viable and contiguous Palestine,'' based on
Israel's withdrawal from territory occupied since the 1967 war and land
swaps to ``reflect practical realities on the ground.''

A second principle was the creation of a state for the Palestinian
people, and a third was a ``fair and realistic solution to the
Palestinian refugee issue,'' including compensation. There was no
mention of a ``right of return'' for refugees and their descendants
forced to leave Israel and the Palestinian territories, back to 1948.

The fourth principle called for Jerusalem to be the recognized capital
of both states, which Mr. Kerry said was ``the most sensitive issue for
both sides.'' The fifth was an agreement to satisfy Israel's security
needs while ending its military occupation of Palestinian territories.

Mr. Kerry, who has cast himself as one of Israel's greatest friends,
said in recent months it became clear he had to ``save the two-state
solution while there was still time.''

``We did not take this decision lightly,'' he said of the vote in the
United Nations Security Council, where the American abstention allowed a
14-to-0 condemnation of Israel go forward. ``Israelis are fully
justified in decrying attempts to delegitimize their state and question
the right of a Jewish state to exist. But this vote was not about that.
It was about actions that Israelis and Palestinians are taking that are
increasingly rendering a two-state solution impossible.''

It was also about Mr. Kerry's own personal disappointment. As soon as he
took over from Hillary Clinton as secretary of state in 2013, Mr. Kerry
plunged into the tar pit of Middle East peace negotiations with an
enthusiasm neither Mrs. Clinton nor Mr. Obama shared. The goal was a
nine-month negotiation leading to a ``final status'' of the
Israeli-Palestinian conflict by the summer of 2014.

It never got that far. Despite scores of meetings between Mr. Kerry and
his two main interlocutors, Mr. Abbas, the Palestinian president, and
Mr. Netanyahu, Mr. Kerry and his lead mediators, Martin S. Indyk and
Frank Lowenstein, could not make progress. They blamed both sides for
taking actions that undermined the process, but the continued expansion
of the settlements was one of their leading complaints --- an effort, in
the American and European view, to establish ``facts on the ground'' so
that territory could not be traded away.

Mr. Netanyahu has accused the United States of ``orchestrating'' the
vote, and his aides have said that Mr. Kerry and Mr. Obama effectively
stabbed Israel in the back. Israeli officials have said they have
evidence that the United States organized the resolution. Mr. Kerry
pushed back at that narrative on Wednesday.

Mr. Netanyahu, for his part, is biding his time and waiting for Mr.
Kerry and Mr. Obama to move on. Israeli leaders postponed plans on
Wednesday to move ahead with new housing in East Jerusalem, just hours
before the speech.

Advertisement

\protect\hyperlink{after-bottom}{Continue reading the main story}

\hypertarget{site-index}{%
\subsection{Site Index}\label{site-index}}

\hypertarget{site-information-navigation}{%
\subsection{Site Information
Navigation}\label{site-information-navigation}}

\begin{itemize}
\tightlist
\item
  \href{https://help.nytimes.com/hc/en-us/articles/115014792127-Copyright-notice}{©~2020~The
  New York Times Company}
\end{itemize}

\begin{itemize}
\tightlist
\item
  \href{https://www.nytco.com/}{NYTCo}
\item
  \href{https://help.nytimes.com/hc/en-us/articles/115015385887-Contact-Us}{Contact
  Us}
\item
  \href{https://www.nytco.com/careers/}{Work with us}
\item
  \href{https://nytmediakit.com/}{Advertise}
\item
  \href{http://www.tbrandstudio.com/}{T Brand Studio}
\item
  \href{https://www.nytimes.com/privacy/cookie-policy\#how-do-i-manage-trackers}{Your
  Ad Choices}
\item
  \href{https://www.nytimes.com/privacy}{Privacy}
\item
  \href{https://help.nytimes.com/hc/en-us/articles/115014893428-Terms-of-service}{Terms
  of Service}
\item
  \href{https://help.nytimes.com/hc/en-us/articles/115014893968-Terms-of-sale}{Terms
  of Sale}
\item
  \href{https://spiderbites.nytimes.com}{Site Map}
\item
  \href{https://help.nytimes.com/hc/en-us}{Help}
\item
  \href{https://www.nytimes.com/subscription?campaignId=37WXW}{Subscriptions}
\end{itemize}
