Sections

SEARCH

\protect\hyperlink{site-content}{Skip to
content}\protect\hyperlink{site-index}{Skip to site index}

\href{https://www.nytimes.com/section/politics}{Politics}

\href{https://myaccount.nytimes.com/auth/login?response_type=cookie\&client_id=vi}{}

\href{https://www.nytimes.com/section/todayspaper}{Today's Paper}

\href{/section/politics}{Politics}\textbar{}Two Russian Compounds,
Caught Up in History's Echoes

\url{https://nyti.ms/2hxQunJ}

\begin{itemize}
\item
\item
\item
\item
\item
\end{itemize}

Advertisement

\protect\hyperlink{after-top}{Continue reading the main story}

Supported by

\protect\hyperlink{after-sponsor}{Continue reading the main story}

\hypertarget{two-russian-compounds-caught-up-in-historys-echoes}{%
\section{Two Russian Compounds, Caught Up in History's
Echoes}\label{two-russian-compounds-caught-up-in-historys-echoes}}

\includegraphics{https://static01.nyt.com/images/2016/12/30/us/30spies_web2/30spies_web2-articleLarge.jpg?quality=75\&auto=webp\&disable=upscale}

By \href{http://www.nytimes.com/by/mark-mazzetti}{Mark Mazzetti} and
\href{http://www.nytimes.com/by/michael-s-schmidt}{Michael S. Schmidt}

\begin{itemize}
\item
  Dec. 29, 2016
\item
  \begin{itemize}
  \item
  \item
  \item
  \item
  \item
  \end{itemize}
\end{itemize}

WASHINGTON --- A pair of luxurious waterfront compounds outside New York
and Washington have for decades been a retreat for Russian diplomats,
places to frolic in the water, play tennis and take lengthy steam baths.

On Thursday, Obama administration officials described the compounds
differently: as beachside spy nests sometimes used by Russian
intelligence operatives to have long conversations on the sand to avoid
being ensnared by American electronic surveillance.

They ordered all Russians out of the compounds within 24 hours.

The move was one of a number of retaliatory measures the White House
announced in response to what it called a Russian campaign to wreak
havoc on the presidential election, and to what it said was systemic
harassment of American officials in Russia.

Besides the shuttering of the two compounds, administration officials
announced the expulsion by Sunday of 35 unnamed Russian officials ---
and their families --- who they said were working undercover as spies.

The announcement had echoes of the tit-for-tat reprisals that were
common during the Cold War, and the government of Vladimir V. Putin
announced within hours that there would be a swift response. That
seemingly ensured that the Obama administration's final days would be
consumed by escalating accusations and retaliation between Washington
and Moscow.

The Obama administration offered no proof on Thursday linking the two
compounds or the Russians being expelled to intelligence activities, and
veterans of the spy games offered mixed assessments of the
administration's actions.

``I think these sanctions are pretty weak. It's more perhaps symbolic,''
said Steven Hall, a former senior C.I.A. official who ran Russia
operations until his retirement in 2015.

But Mr. Hall said that the other measures could cause more irritation in
Moscow. He said that he could not recall a previous move similar to the
shuttering of the two compounds --- one on Long Island and the other on
the Eastern Shore of Maryland --- and that expelling the 35 officers and
their families ``will slow down Russia's activities in the United
States.''

One former senior American official said that the group, made up of
officials working at the embassy in Washington and the consulate in San
Francisco, makes up about a third of the suspected Russian intelligence
officers operating in the United States under diplomatic cover. It is
unclear how many of the expelled Russians spent time at the two
compounds.

The dismissals are believed to be the largest expulsion of Russian
officials from the United States since 2001, when about 50 suspected
Russian intelligence officers were forced to leave after Robert Hanssen,
a senior F.B.I. official, was arrested and charged with spying for
Moscow.

``I would be flabbergasted if the Russians didn't reciprocate and expel
35 American officials from Russia,'' Mr. Hall said. The C.I.A. has long
posted undercover officers in Russia --- and nearly every other country
in the world --- who pose as diplomats, businessmen or other
professionals.

It is unclear whether any of the Russians being expelled played a part
in the hacking that interfered with the American election. Law
enforcement officials said that the White House and the State Department
had come up with the number 35 and had then turned to the F.B.I. to list
Russian officials in the United States who it believed were actually
intelligence officers.

While the expulsions send a powerful message, they will force the F.B.I.
to go back to work trying to identify a new crop of spies the Russians
will almost certainly send to the United States.

After the expulsions were announced, officials described a pattern of
harassment of American government employees working in Russia. In June,
a uniformed officer with Russia's Federal Security Service, called the
F.S.B., attacked an employee of the United States Embassy in Moscow
seemingly without provocation.

``Our diplomats have experienced an unacceptable level of harassment in
Moscow by Russian security services and police over the last year,''
President Obama said in a statement on Thursday.

Before this year, the most notable recent episode to become public was
in 2013, when Russian officials arrested Ryan C. Fogle, an employee of
the embassy in Moscow who the Russian government had discovered was a
C.I.A. officer.

Mr. Fogle, officially posted in Russia as the third secretary of the
embassy's political department, was picked up on the street wearing a
blond wig under a baseball cap. He was carrying a knapsack holding a
compass, a Moscow street atlas and \$130,000 in cash, money that Russian
officials said was meant to be used to recruit a Russian security
officer as a spy.

The episode's amusing details garnered much of the attention, but Mr.
Fogle's arrest and expulsion from Russia was a public indicator that the
spying activity between the United States and Russia was heating up ---
and in many ways had never ended. The F.S.B. took the unusual step of
releasing a video showing Mr. Fogle facedown on a street as a Russian
officer pinned his hands behind his back.

American law enforcement and intelligence agencies have little
indication that the compounds in Maryland and New York played any role
in the cyberattacks against the Democratic National Committee and other
political organizations.

The Long Island retreat is a 14-acre estate in Upper Brookville, known
as Norwich House. Upper Brookville's mayor, Elliot Conway, confirmed on
Friday that the federal government had ordered the house shut.

Federal agents could been seen on Friday morning in the estate's
driveway, shortly before four vehicles with diplomatic license plates
drove away. A few of the people in the cars waved to reporters, in an
apparent goodbye.

The Russian government has owned the property, and another one nearby in
Glen Cove, since the days of the Soviet Union, and relations between the
owners and local residents have at times been strained. In 1982, Reagan
administration officials said Russians were using the Glen Cove mansion
to conduct electronic surveillance of Long Island's defense and
technology industries.

The Glen Cove City Council responded by barring the Russians from
obtaining free beach parking stickers and discounted tennis permits.

The Council's lone dissenting voice described taking away the tennis
permits as ``petty'' and said that ``it's a matter the professionals at
the State Department should handle.''

On Thursday, White House and F.B.I. officials said it was the Glen Cove
estate, Killenworth Mansion, that was being shuttered. On Friday, they
confirmed that it was in fact the Upper Brookville house.

The Maryland compound is a sprawling complex of several buildings
fronting the Corsica River in Centreville, including a three-story brick
Georgian-style mansion that has long been a retreat for Russia's
ambassador to Washington. It has a swimming pool, a soccer field and
lighted tennis courts.

Anatoly Dobrynin, who served as ambassador to Washington from the
Kennedy administration to the Reagan administration, was frequently at
the estate, and the Russian government held on to the property after the
collapse of the Soviet Union.

Unlike on Long Island, relations between the Russians visiting the
Maryland compound and local residents have generally been placid ---
even if the Russians did not always adopt the local folkways.

Julie Patterson, 44, who has lived on a horse farm near the compound for
12 years, said the Russians largely kept to themselves but were cordial
neighbors, inviting locals to an annual Labor Day party.

She said that she believed that the property was largely used as a
weekend retreat and that she often saw children, families and buses
heading to the estate. She said people often held parties there and
hosted a sailing club.

In 1992, a Centreville resident told a reporter for The Associated Press
that the Russians did not cook crabs the local way, by throwing live
crabs into a pot of boiling water.

``They stab them with a screwdriver, break the back shell off, clean
them and then boil the body,'' she said.

Advertisement

\protect\hyperlink{after-bottom}{Continue reading the main story}

\hypertarget{site-index}{%
\subsection{Site Index}\label{site-index}}

\hypertarget{site-information-navigation}{%
\subsection{Site Information
Navigation}\label{site-information-navigation}}

\begin{itemize}
\tightlist
\item
  \href{https://help.nytimes.com/hc/en-us/articles/115014792127-Copyright-notice}{©~2020~The
  New York Times Company}
\end{itemize}

\begin{itemize}
\tightlist
\item
  \href{https://www.nytco.com/}{NYTCo}
\item
  \href{https://help.nytimes.com/hc/en-us/articles/115015385887-Contact-Us}{Contact
  Us}
\item
  \href{https://www.nytco.com/careers/}{Work with us}
\item
  \href{https://nytmediakit.com/}{Advertise}
\item
  \href{http://www.tbrandstudio.com/}{T Brand Studio}
\item
  \href{https://www.nytimes.com/privacy/cookie-policy\#how-do-i-manage-trackers}{Your
  Ad Choices}
\item
  \href{https://www.nytimes.com/privacy}{Privacy}
\item
  \href{https://help.nytimes.com/hc/en-us/articles/115014893428-Terms-of-service}{Terms
  of Service}
\item
  \href{https://help.nytimes.com/hc/en-us/articles/115014893968-Terms-of-sale}{Terms
  of Sale}
\item
  \href{https://spiderbites.nytimes.com}{Site Map}
\item
  \href{https://help.nytimes.com/hc/en-us}{Help}
\item
  \href{https://www.nytimes.com/subscription?campaignId=37WXW}{Subscriptions}
\end{itemize}
