Sections

SEARCH

\protect\hyperlink{site-content}{Skip to
content}\protect\hyperlink{site-index}{Skip to site index}

\href{https://www.nytimes.com/section/world/asia}{Asia Pacific}

\href{https://myaccount.nytimes.com/auth/login?response_type=cookie\&client_id=vi}{}

\href{https://www.nytimes.com/section/todayspaper}{Today's Paper}

\href{/section/world/asia}{Asia Pacific}\textbar{}Uncertainty Over New
Chinese Law Rattles Foreign Nonprofits

\url{https://nyti.ms/2iKSyGF}

\begin{itemize}
\item
\item
\item
\item
\item
\end{itemize}

Advertisement

\protect\hyperlink{after-top}{Continue reading the main story}

Supported by

\protect\hyperlink{after-sponsor}{Continue reading the main story}

\hypertarget{uncertainty-over-new-chinese-law-rattles-foreign-nonprofits}{%
\section{Uncertainty Over New Chinese Law Rattles Foreign
Nonprofits}\label{uncertainty-over-new-chinese-law-rattles-foreign-nonprofits}}

\includegraphics{https://static01.nyt.com/images/2016/12/27/world/27CHINANGO-1/27CHINANGO-1-articleLarge.jpg?quality=75\&auto=webp\&disable=upscale}

By \href{http://www.nytimes.com/by/chris-buckley}{Chris Buckley}

\begin{itemize}
\item
  Dec. 29, 2016
\item
  \begin{itemize}
  \item
  \item
  \item
  \item
  \item
  \end{itemize}
\end{itemize}

BEIJING --- The hotline rings, but nobody answers.

China's Ministry of Public Security opened the line last month to answer
questions about the new law regulating foreign nonprofit organizations,
which takes effect on Sunday.

But this week and last, calls went unanswered, exemplifying the
uncertainty that still surrounds the law, raising concern among
thousands of nongovernmental organizations about their ability to
continue their work in the new year.

The law, which places a raft of new requirements on foreign nonprofits
operating in China, is another building block in President Xi Jinping's
\href{http://www.nytimes.com/2016/10/28/world/asia/xi-jinping-china.html}{fortification
of one-party rule}, which he sees as threatened by foreign influence and
unfettered civil society.

Under the law, foreign nonprofits such as foundations, charities and
many business associations must register with the police, persuade state
agencies and organizations to act as their sponsors, and submit regular,
detailed reports on their activities.

According to an official estimate, there are 7,000 foreign
nongovernmental organizations in China. They range from well-known
institutions like the Ford Foundation and Oxfam to groups of a few
people working on issues like rural education, nature conservation and
health care.

But groups working on politically sensitive issues like human rights,
legal reform and the rule of law, or those concerning ethnic minorities,
are seen as most at risk.

Some foreign organizations have already pulled back. The American Bar
Association, which has a program providing training and support to
strengthen the rule of law, recently closed its Beijing office until it
could gain formal approval for its work.

Elizabeth Andersen, the association's associate executive director,
cited the ``heightened scrutiny of foreign organizations working in
China and the uncertainties and lack of information surrounding how the
new law will be implemented.''

But the uncertainty has also unsettled groups far removed from political
concerns. Numerous aspects of the law remain opaque, and many groups are
anxious about the vagueness and expense of the new requirements, while
some fear their work will be curtailed or even banned.

``Nothing's clear,'' said Corinne Richeux Hua, executive director of
\href{http://steppingstoneschina.net/}{Stepping Stones}, a locally
registered charity in Shanghai that organizes English teachers for
children from the countryside. ``We've got vague directives and
guidelines.''

The local police, with whom her group must now register, had been
helpful, but ``they are still figuring it,'' she said. ``The rules
haven't been made completely clear to them yet.''

Ambiguity about how the law will be enforced is likely to make foreign
groups extra cautious, and the Ministry of Public Security, which
administers the law, ``has every incentive to maintain uncertainty,''
said
\href{http://www.middlebury.edu/academics/ps/faculty/node/25661}{Jessica
C. Teets}, a political scientist at Middlebury College in Vermont who
studies nongovernmental organizations in China.

``This will mean that the government is able to more closely monitor the
foreign NGOs, and, more importantly, the Chinese citizens working and
interacting with them, while allowing them to continue the work that the
government deems beneficial,'' Ms. Teets said by email. ``The NGOs have
every right to fear the closing off of space for advocacy and programs,
but I think the impact will be really differentiated.''

Indeed, a Ministry of Public Security official
\href{http://www.mps.gov.cn/n2253534/n2253535/n2253537/c5542815/content.html}{told
diplomats} in Shanghai last month that ``the Chinese government will
continue welcoming and supporting foreign nongovernmental organizations
coming to China.''

After Deng Xiaoping opened up China in the 1980s, foreign foundations,
associations and charities became important channels for sharing money,
ideas and inspiration. Officials often welcomed their help, especially
in poorer parts of the country, even though the rules governing their
status were murky.

\includegraphics{https://static01.nyt.com/images/2016/12/27/world/27CHINANGO-2/27CHINANGO-2-articleLarge.jpg?quality=75\&auto=webp\&disable=upscale}

But certain kinds of organizations, especially those that work in law
and contentious social issues, have garnered distrust. Through the new
law, the government wants to narrow permissible activities of foreign
groups and monitor their work much more thoroughly.

A list of
\href{http://www.mps.gov.cn/n2254314/n2254409/n4904353/c5579013/content.html?from=timeline\&isappinstalled=0}{permitted
categories} of assistance issued last week suggested that foreign groups
offering technical help on environmental, health and other relatively
uncontroversial issues had strong chances of gaining approval.

Those working on legal issues will have a much narrower foothold.
``Human rights,'' for example, is not on the list of permitted issues.

``Rather than seeing foreign NGOs as potential partners who can help aid
in economic, social and legal development in China, instead they see a
latent threat that needs to be controlled,'' said
\href{https://www.opensocietyfoundations.org/people/thomas-kellogg}{Thomas
Kellogg}, the East Asia director of the Open Society Foundations, which
has financed some work in China. ``People on the international side are
definitely worried. And well they should be. I think it will be
difficult for many foreign NGOs working on legal reform to register. For
those that are able to register, the law will likely restrict what they
are able to do.''

Even before the new law, combative rights lawyers and advocates,
feminists and labor activists have come under Mr. Xi's heavy grip.
\href{http://www.nytimes.com/2016/07/10/world/asia/china-ned-ngo-peter-dahlin.html}{Peter
Dahlin}, a Swedish citizen in Beijing, was arrested, forced to apologize
on television and
\href{http://www.nytimes.com/2016/01/26/world/asia/china-to-expel-swedish-human-rights-advocate.html}{expelled
from China} early this year for working for an unregistered group that
did low-key advocacy for legal rights.

The party sees groups like his as potential Trojan horses of political
subversion. A
\href{http://www.nytimes.com/2016/12/22/world/asia/china-video-communist-party.html}{propaganda
video} promoted by public security agencies this month warned that
anti-party forces were ``using foreign nongovernmental organizations to
nurture `proxies' and to establish a social basis'' for insurrection.

The groups' worries have been compounded by confusion about many
requirements, the belated release of crucial rules, and signs that
public security bureaus are poorly prepared for their new role.

But it is not just activists and charities who are concerned. The opaque
rules mean that organizations such as business groups, universities and
education programs that seemingly pose no political threat are also
unclear whether they must register for some of their activities.

``Business and trade associations, civil society, environmental groups
and educational institutions that are concerned about how their
operations in China may be affected'' have met with American diplomats
to discuss the law, said Mary Beth Polley, a spokeswoman for the
American Embassy in Beijing.

``We remain deeply concerned about the uncertainties and potentially
hostile environment for foreign nonprofit, nongovernment organizations
and their Chinese partners that this law creates,'' she said.

Foreign organizations working in China have long had to seek out
domestic agencies or organizations to act as their sponsors. But the new
law narrows the list of permissible sponsors, and those permitted may be
reluctant to take on the risk of vouching for foreign groups, or feel
they do not have the personnel available for the task.

``Who wants to assume this burden?'' asks
\href{https://www.wilmerhale.com/lester_ross/}{Lester Ross}, a partner
in the Beijing office of the WilmerHale law firm who has been advising
companies and organizations on the new law. ``I think there's a real
issue of capacity. The NGO community serves as an important ballast for
relationships, and if this is mishandled, it won't help.''

While some foreign organizations are resigned to months of uncertainty,
some said they would keep working full time in the country, confident
that public security offices will let them stay open while the
registration is ironed out. Several American trade associations said
they thought they would be allowed to stay, and some groups said they
looked forward to gaining official status under the new rules.

``We see these new regulations as a pretty positive thing for us,'' said
\href{http://wildaid.org/people/steve-blake}{Steve Blake}, the acting
chief representative in Beijing of WildAid, which works with the Chinese
government to fight illegal trading in wildlife. ``We have a big
presence here, but we've never been completely officially on the
books.''

But organizations working on legal issues or social problems said they
were unsure of their futures and may face hard choices. Registration may
mean sacrificing autonomy, but the alternative may be abandoning people
in China who need their help, said Mr. Kellogg of the Open Society
Foundations.

``I would urge foreign NGOs to adopt a wait-and-see attitude before they
make any final decisions about either registering or pulling out of
China,'' he said. ``Once there is more clarity about how the law will be
enforced, it will be at least a bit easier to come up with mitigation
strategies.''

Advertisement

\protect\hyperlink{after-bottom}{Continue reading the main story}

\hypertarget{site-index}{%
\subsection{Site Index}\label{site-index}}

\hypertarget{site-information-navigation}{%
\subsection{Site Information
Navigation}\label{site-information-navigation}}

\begin{itemize}
\tightlist
\item
  \href{https://help.nytimes.com/hc/en-us/articles/115014792127-Copyright-notice}{©~2020~The
  New York Times Company}
\end{itemize}

\begin{itemize}
\tightlist
\item
  \href{https://www.nytco.com/}{NYTCo}
\item
  \href{https://help.nytimes.com/hc/en-us/articles/115015385887-Contact-Us}{Contact
  Us}
\item
  \href{https://www.nytco.com/careers/}{Work with us}
\item
  \href{https://nytmediakit.com/}{Advertise}
\item
  \href{http://www.tbrandstudio.com/}{T Brand Studio}
\item
  \href{https://www.nytimes.com/privacy/cookie-policy\#how-do-i-manage-trackers}{Your
  Ad Choices}
\item
  \href{https://www.nytimes.com/privacy}{Privacy}
\item
  \href{https://help.nytimes.com/hc/en-us/articles/115014893428-Terms-of-service}{Terms
  of Service}
\item
  \href{https://help.nytimes.com/hc/en-us/articles/115014893968-Terms-of-sale}{Terms
  of Sale}
\item
  \href{https://spiderbites.nytimes.com}{Site Map}
\item
  \href{https://help.nytimes.com/hc/en-us}{Help}
\item
  \href{https://www.nytimes.com/subscription?campaignId=37WXW}{Subscriptions}
\end{itemize}
