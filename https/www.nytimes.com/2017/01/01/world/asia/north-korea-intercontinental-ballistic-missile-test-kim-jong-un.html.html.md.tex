Sections

SEARCH

\protect\hyperlink{site-content}{Skip to
content}\protect\hyperlink{site-index}{Skip to site index}

\href{https://www.nytimes.com/section/world/asia}{Asia Pacific}

\href{https://myaccount.nytimes.com/auth/login?response_type=cookie\&client_id=vi}{}

\href{https://www.nytimes.com/section/todayspaper}{Today's Paper}

\href{/section/world/asia}{Asia Pacific}\textbar{}Kim Jong-un Says North
Korea Is Preparing to Test Long-Range Missile

\url{https://nyti.ms/2hG8psv}

\begin{itemize}
\item
\item
\item
\item
\item
\end{itemize}

Advertisement

\protect\hyperlink{after-top}{Continue reading the main story}

Supported by

\protect\hyperlink{after-sponsor}{Continue reading the main story}

\hypertarget{kim-jong-un-says-north-korea-is-preparing-to-test-long-range-missile}{%
\section{Kim Jong-un Says North Korea Is Preparing to Test Long-Range
Missile}\label{kim-jong-un-says-north-korea-is-preparing-to-test-long-range-missile}}

\includegraphics{https://static01.nyt.com/images/2017/01/02/world/asia/02nkorea-photo1/02nkorea-photo1-videoSixteenByNine3000.jpg}

By \href{http://www.nytimes.com/by/choe-sang-hun}{Choe Sang-Hun}

\begin{itemize}
\item
  Jan. 1, 2017
\item
  \begin{itemize}
  \item
  \item
  \item
  \item
  \item
  \end{itemize}
\end{itemize}

SEOUL, South Korea --- North Korea's leader, Kim Jong-un, said on Sunday
that his country was making final preparations to conduct its first test
of an intercontinental ballistic missile --- a bold statement less than
a month before the inauguration of President-elect Donald J. Trump.

Although North Korea has conducted five nuclear tests in the last decade
and more than 20 ballistic missile tests in 2016 alone, and although it
habitually threatens to attack the United States with nuclear weapons,
the country has never flight-tested an intercontinental ballistic
missile, or ICBM.

In his annual New Year's Day speech, which was broadcast on the North's
state-run KCTV on Sunday, Mr. Kim spoke proudly of the strides he said
his country had made in its nuclear weapons and ballistic missile
programs. He said North Korea would continue to bolster its weapons
programs as long as the United States remained hostile and continued its
joint military exercises with South Korea.

``We have reached the final stage in preparations to test-launch an
intercontinental ballistic rocket,'' he said.

Analysts in the region have said Mr. Kim might conduct another weapons
test in coming months, taking advantage of leadership changes in the
United States and South Korea. Mr. Trump will be sworn in on Jan. 20. In
South Korea, President Park Geun-hye, whose powers were suspended in a
Parliamentary impeachment on Dec. 9, is waiting for the Constitutional
Court to rule on whether she should be formally removed from office or
reinstated.

If North Korea conducts a long-range-missile test in coming months, it
will test Mr. Trump's new administration; despite years of increasingly
harsh sanctions, North Korea has been advancing toward Mr. Kim's
professed goal of arming his isolated country with the ability to
deliver a nuclear warhead to the United States.

Mr. Kim's speech on Sunday indicated that North Korea may test-launch a
long-range rocket several times this year to complete its ICBM program,
said
\href{http://www.sejong.org/eng/intro/org_view.php?str_bcode=031240003\&str_no=sccheong}{Cheong
Seong-chang}, a senior research fellow at the Sejong Institute in South
Korea. The first of such tests could come even before Mr. Trump's
inauguration, Mr. Cheong said.

``We need to take note of the fact that this is the first New Year's
speech where Kim Jong-un mentioned an intercontinental ballistic
missile,'' he said.

In his speech, Mr. Kim did not comment on Mr. Trump's election.

Doubt still runs deep that North Korea has mastered all the technology
needed to build a reliable ICBM. But analysts in the region said the
North's launchings of three-stage rockets to put satellites into orbit
in recent years showed that the country had cleared some key
technological hurdles.

After the North's
\href{http://www.nytimes.com/2016/02/07/world/asia/north-korea-moves-up-rocket-launching-plan.html}{satellite
launch} in February, South Korean defense officials said the Unha rocket
used in the launch, if successfully reconfigured as a missile, could fly
more than 7,400 miles with a warhead of 1,100 to 1,300 pounds --- far
enough to reach most of the United States.

North Korea has deployed Rodong ballistic missiles that can reach most
of South Korea and Japan, but it has had a spotty record in
test-launching the Musudan, its intermediate-range ballistic missile
with a range long enough to reach American military bases in the
Pacific, including those on Guam.

The North has also claimed a series of successes in testing various ICBM
technologies, although its claims cannot be verified and are often
disputed by officials and analysts in the region.

It has said it could now make nuclear warheads small enough to fit onto
a ballistic missile. It also claimed success in testing the re-entry
technology that allows a long-range missile to return to the Earth's
atmosphere without breaking up.

In April, North Korea reported the
\href{http://www.nytimes.com/2016/04/10/world/asia/north-korea-says-it-successfully-tested-missile-engine.html}{successful
ground test} of an engine for an intercontinental ballistic missile. At
the time, Mr. Kim said the North ``can tip new-type intercontinental
ballistic rockets with more powerful nuclear warheads and keep any
cesspool of evils in the Earth, including the U.S. mainland, within our
striking range.''

On Sept. 9,
\href{https://www.nytimes.com/2016/09/21/world/asia/north-korea-says-it-has-tested-a-new-long-range-rocket-engine.html}{the
North conducted its fifth}, and most powerful, nuclear test. Mr. Kim
later attended another ground test of a new long-range rocket engine,
exhorting his government to prepare for another rocket launch as soon as
possible. In November, the United Nations Security Council
\href{http://www.nytimes.com/2016/11/30/world/asia/north-korea-un-sanctions.html}{imposed
new sanctions} against the North.

Advertisement

\protect\hyperlink{after-bottom}{Continue reading the main story}

\hypertarget{site-index}{%
\subsection{Site Index}\label{site-index}}

\hypertarget{site-information-navigation}{%
\subsection{Site Information
Navigation}\label{site-information-navigation}}

\begin{itemize}
\tightlist
\item
  \href{https://help.nytimes.com/hc/en-us/articles/115014792127-Copyright-notice}{©~2020~The
  New York Times Company}
\end{itemize}

\begin{itemize}
\tightlist
\item
  \href{https://www.nytco.com/}{NYTCo}
\item
  \href{https://help.nytimes.com/hc/en-us/articles/115015385887-Contact-Us}{Contact
  Us}
\item
  \href{https://www.nytco.com/careers/}{Work with us}
\item
  \href{https://nytmediakit.com/}{Advertise}
\item
  \href{http://www.tbrandstudio.com/}{T Brand Studio}
\item
  \href{https://www.nytimes.com/privacy/cookie-policy\#how-do-i-manage-trackers}{Your
  Ad Choices}
\item
  \href{https://www.nytimes.com/privacy}{Privacy}
\item
  \href{https://help.nytimes.com/hc/en-us/articles/115014893428-Terms-of-service}{Terms
  of Service}
\item
  \href{https://help.nytimes.com/hc/en-us/articles/115014893968-Terms-of-sale}{Terms
  of Sale}
\item
  \href{https://spiderbites.nytimes.com}{Site Map}
\item
  \href{https://help.nytimes.com/hc/en-us}{Help}
\item
  \href{https://www.nytimes.com/subscription?campaignId=37WXW}{Subscriptions}
\end{itemize}
