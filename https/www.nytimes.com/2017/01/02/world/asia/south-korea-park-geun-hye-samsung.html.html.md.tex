Sections

SEARCH

\protect\hyperlink{site-content}{Skip to
content}\protect\hyperlink{site-index}{Skip to site index}

\href{https://www.nytimes.com/section/world/asia}{Asia Pacific}

\href{https://myaccount.nytimes.com/auth/login?response_type=cookie\&client_id=vi}{}

\href{https://www.nytimes.com/section/todayspaper}{Today's Paper}

\href{/section/world/asia}{Asia Pacific}\textbar{}As Scandal Roils South
Korea, Fingers Point to Mixing of Politics and Business

\url{https://nyti.ms/2iVQ8oJ}

\begin{itemize}
\item
\item
\item
\item
\item
\end{itemize}

Advertisement

\protect\hyperlink{after-top}{Continue reading the main story}

Supported by

\protect\hyperlink{after-sponsor}{Continue reading the main story}

\hypertarget{as-scandal-roils-south-korea-fingers-point-to-mixing-of-politics-and-business}{%
\section{As Scandal Roils South Korea, Fingers Point to Mixing of
Politics and
Business}\label{as-scandal-roils-south-korea-fingers-point-to-mixing-of-politics-and-business}}

\includegraphics{https://static01.nyt.com/images/2016/12/22/world/22KOREA-1/22KOREA-1-articleInline.jpg?quality=75\&auto=webp\&disable=upscale}

By \href{http://www.nytimes.com/by/choe-sang-hun}{Choe Sang-Hun} and
\href{http://www.nytimes.com/by/motoko-rich}{Motoko Rich}

\begin{itemize}
\item
  Jan. 2, 2017
\item
  \begin{itemize}
  \item
  \item
  \item
  \item
  \item
  \end{itemize}
\end{itemize}

SEOUL, South Korea --- The heir apparent to the Samsung empire, Jay Y.
Lee, was trying to push through a corporate merger seen as critical to
his plans to succeed his father as chairman.

For months, key shareholders
\href{https://www.nytimes.com/2015/07/05/business/samsung-merger-plan-called-unfair-to-some-investors.html?_r=0}{fought
the move}. Then, suddenly, the standoff broke as South Korea's
government-controlled pension fund, which held the shares to cast the
deciding vote, endorsed Mr. Lee's deal.

A week later, President Park Geun-hye invited Mr. Lee to her office and
asked for Samsung's help with a campaign to promote South Korean culture
and sports. Within months, Samsung had donated \$17.4 million to two
foundations controlled by the president's confidante, Choi Soon-sil, and
\$6.2 million for the training of Korean equestrians, including
\href{http://www.nytimes.com/2017/01/02/world/asia/south-korea-scandal-choi-soon-sil-daughter.html}{Ms.
Choi's daughter}.

Those donations --- and whether they were part of a quid pro quo --- are
now at the heart of the impeachment case against Ms. Park. The nation's
full Constitutional Court will begin formal hearings on Tuesday into the
case, \href{http://nyti.ms/2dQpNWq}{the biggest influence-peddling
scandal} in South Korea's history.

The court has never before ousted a president, though seven of the last
eight have left office tainted by allegations of corruption. Whatever
the court decides, the Park scandal has already put recurring collusion
between big business and government in South Korea under intense
scrutiny and could reshape the nation's flawed, young democracy.

``We created a miracle on the streets,'' said You Jong-il, a professor
of macroeconomics and development policy in Sejong City, referring to
\href{http://www.nytimes.com/2016/11/13/world/asia/korea-park-geun-hye-protests.html}{huge,
peaceful street protests} demanding Ms. Park's resignation. ``But we are
still very worried about whether we will really be able to change Korean
society and Korean politics.''

Public outrage --- initially aimed at the influence that Ms. Choi, the
daughter of a religious sect leader, appeared to exercise over the Park
administration --- has turned to broader concerns about the political
system: the power of the presidency, and its symbiotic relationship with
\href{http://www.nytimes.com/2011/09/14/business/global/south-korean-chaebol-under-increasing-pressure.html}{the
chaebol}, the family-controlled conglomerates like Samsung that dominate
the economy.

``Chaebol are accomplices!'' protesters have chanted, carrying effigies
of their leaders dressed in blue prison uniforms. Damning new details
emerge weekly.

Prosecutors allege that Ms. Choi conspired with Ms. Park to force 53
companies to donate more than \$69 million to the two foundations under
Ms. Choi's control.

\includegraphics{https://static01.nyt.com/images/2016/12/22/world/22KOREA-2/22KOREA-2-articleLarge.jpg?quality=75\&auto=webp\&disable=upscale}

The National Assembly went further in its impeachment motion, describing
the donations as bribes personally benefiting Ms. Choi and paid in
return for favors for the companies, ranging from lucrative licenses to
presidential pardons.

Ms. Park has denied the charges. At an extraordinary parliamentary
hearing, Mr. Lee and eight other chaebol leaders also denied receiving
or seeking special treatment for the donations. But they appeared to
acknowledge that the payments were not entirely voluntary.

``It was difficult to go against the government's wishes,'' testified
Koo Bon-moo, the chairman of LG, the multinational electronics company.

Mr. Lee called Samsung's contribution ``not voluntary'' but
``inevitable.''

\hypertarget{a-history-of-favors}{%
\subsection{A History of Favors}\label{a-history-of-favors}}

The meeting with Mr. Lee was one of eight that Ms. Park held with top
chaebol executives in July 2015. Her lawyers have acknowledged that she
asked for contributions to the two foundations in the meetings but deny
that she promised any favors. Little is known about what else was
discussed in these one-on-one sessions.

The
\href{https://www.nytimes.com/2016/12/08/world/asia/south-korea-park-geun-hye-accusations-impeachment.html}{impeachment
motion} alleges that Ms. Park prepared for the meetings by asking her
chief economic adviser, Ahn Chong-bum, for a memo outlining issues that
the chaebol needed help with.

But there is no doubt the chaebol have benefited from government support
for decades.

Ms. Park's father, Park Chung-hee, the country's first military
dictator, pioneered the economic model before
\href{http://www.nytimes.com/1979/11/04/archives/huge-crowd-mourns-park-at-funeral-rite-in-seoul.html}{his
assassination in 1979}. He showered a handful of businesses with favors
such as tax benefits, cheap electricity, a buy-Korea policy and the
suppression of organized labor.

These companies eventually grew into industrial conglomerates, fueling
the export-driven growth that lifted South Korea out of postwar poverty
and made it one of the world's most dynamic economies.

Some of the firms, such as Samsung, Hyundai and LG, are now global
brands with publicly listed shares. But the founding families still
dominate almost all the conglomerates, in part, critics say, because of
lax enforcement of corporate governance and tax laws.

In return for their support, Park and many of his successors as
president expected the chaebol to contribute to government projects. And
the chaebol did more than that, sometimes channeling money to the
presidents' personal coffers or those of their relatives and associates.

South Koreans are increasingly skeptical of the chaebol and the economic
model they represent. The country's largest shipping line, Hanjin,
recently
\href{http://www.nytimes.com/2016/09/16/business/dealbook/lack-of-planning-hampers-hanjin-shipping-bankruptcy.html}{filed
for bankruptcy}. Samsung, the icon of South Korean technological
prowess, suffered
\href{http://www.nytimes.com/2016/10/23/world/asia/galaxy-note-7-recall-south-korea-samsung.html}{global
humiliation} with its recent
\href{http://www.nytimes.com/2016/09/03/business/samsung-galaxy-note-battery.html}{recall
of exploding Galaxy Note 7 smartphones}.

Image

Choi Soon-sil at a court hearing in Seoul in December. Prosecutors say
she conspired with Ms. Park to force 53 companies to donate more than
\$69 million to two foundations under Ms. Choi's control.Credit...Pool
photo, via Reuters

The chaebol also face competition from China, which has begun producing
many of the same goods, like petrochemicals, more cheaply. Some have
angered the public by shifting manufacturing abroad even as their
tentacle-like grip on the economy at home is blamed for squeezing
start-ups and stifling innovation.

Yet the conglomerates still enjoy some of the benefits that Park
Chung-hee conferred on them more than four decades ago. They are taxed
at lower effective rates than most companies or individuals, and receive
more tax breaks. Businesses also pay lower electricity rates than
individual consumers in South Korea.

The benefits of such policies, Professor You said, ``is a very different
order of magnitude compared to the sums of money that were donated to
the foundations.''

``All decisions are made with the interests of the chaebol in mind,'' he
added of policy-making in recent decades. Politicians and the chaebol,
he said, ``have been relying on each other to maintain their power.''

\hypertarget{president-is-interested}{%
\subsection{`President Is Interested'}\label{president-is-interested}}

Few South Koreans believe the chaebol are innocent victims in the
unfolding case. But while Ms. Choi and Mr. Ahn, the president's economic
adviser, have been arrested, the authorities have not taken action
against executives at any of the businesses.

Historically, the chaebol titans have not been immune from prosecution.
On the contrary, several have been convicted of bribery, tax evasion and
embezzlement --- yet remained at the helm of their businesses.

That is because they are often granted suspended sentences or
presidential pardons. At least six of the nation's top 10 chaebol, which
generate revenue equivalent to more than 80 percent of gross domestic
product, are led by men with criminal records.

Since taking office in 2013, Ms. Park has granted two such pardons. Chey
Tae-won, chairman of the SK Group, which spans chemicals, petroleum,
telecommunications and semiconductors, received one in the summer of
2015. The other went this past summer to Lee Jae-hyun, the chairman of
the CJ Group, which comprises businesses in foods, pharmaceuticals,
entertainment and media.

Both men had been imprisoned on corruption charges. Representatives of
both men met with Ms. Park in 2015. And both the SK Group and the CJ
Group made donations to Ms. Choi's foundations at the president's
request, prosecutors said. The impeachment motion cites the pardons in
accusing Ms. Park of selling favors.

Lawmakers have also noted that the money for the foundations was
collected through the Federation of Korean Industries, which lobbies on
behalf of the chaebol.

Image

Protesters in Seoul last month demanding Ms. Park's
resignation.\href{http://www.nytimes.com/2016/12/10/world/asia/south-korea-protests-history.html}{\\
}Credit...Kim Hong-Ji/Reuters

A special prosecutor is also examining the donation of \$6.2 million by
Samsung to support young equestrians,
\href{http://www.reuters.com/article/us-southkorea-politics-samsung-equestria-idUSKBN14J0T0}{particularly
Ms. Choi's daughter}, Chung Yoo-ra, who trained in Germany using a
thoroughbred purchased for \$830,000.

Ms. Chung, 20, has been living in hiding in Europe, ignoring repeated
calls from South Korean investigators to return home to face criminal
charges. Acting on a tip, the Danish police found her in the northern
city of Aalborg and
\href{https://www.nytimes.com/2017/01/02/world/asia/south-korea-scandal-choi-soon-sil-daughter.html}{detained
her Sunday evening}.

The prosecutor is investigating reports that Ms. Choi used funds donated
by Samsung to buy a house and motel in Germany, as well as to cover her
daughter's personal expenses, including accessories for her pet dogs.

Some of the payments made by the chaebol occurred while the government
was weighing important decisions concerning the companies. For example,
the SK Group and the retail conglomerate Lotte had lost valuable
licenses to run duty-free shops in 2015 and lobbied to regain them last
year. Lotte won back its license in December.

At the same time,
\href{https://www.nytimes.com/2016/10/20/business/international/south-korea-lotte-chaebol-conglomerate-indicted.html}{Lotte's
top executives were under investigation} on tax evasion and embezzlement
charges. In October, prosecutors indicted Lotte's chairman, Shin
Dong-bin, but did not arrest him, allowing him to continue running the
business empire.

While Mr. Shin was under investigation, prosecutors say, Ms. Park and
Mr. Ahn pressured Lotte into donating \$6 million for a sports complex
to be built and managed by a company founded by Ms. Choi. The money was
later returned.

Prosecutors say other chaebol, including Hyundai, directed millions of
dollars' worth of contracts at Mr. Ahn's request to companies owned by
Ms. Choi and her associates.

The chaebol all wrote checks, prosecutors say, usually after Mr. Ahn
uttered the magic words: ``The president is interested in this.''

``What we need is a great national cleanup,'' said Moon Jae-in, an
opposition leader who is the leading candidate to succeed Ms. Park. ``We
must sternly punish politics-business collusion, a legacy of the
dictatorial era, and take this as an opportunity to reform chaebol.''

\hypertarget{concentrated-power}{%
\subsection{Concentrated Power}\label{concentrated-power}}

This is a recurring promise among presidential aspirants in South Korea.
Almost every candidate in recent elections --- including Ms. Park ---
has vowed to end government collusion with the chaebol. But the culture
remains entrenched.

Image

The presidential Blue House in Seoul, where Ms. Park met with top
chaebol executives and asked for contributions to the two
foundations.Credit...Kim Hong-Ji/Reuters

The problem is exacerbated by how much
\href{https://www.nytimes.com/2016/11/12/world/asia/south-korea-park-geun-hye.html?rref=collection\%2Ftimestopic\%2FPark\%20Geun-hye\&action=click\&contentCollection=timestopics\&region=stream\&module=stream_unit\&version=latest\&contentPlacement=54\&pgtype=collection}{power
is concentrated in the presidency}, relative to the legislature or to
the judiciary. The president enjoys considerable influence over
prosecutors, tax collectors and state security agents, whose careers are
largely determined by political loyalty rather than merit.

Some lawmakers are calling for constitutional revisions to shift some of
the president's authority to the prime minister, or even to abolish the
presidency and introduce a parliamentary government.

Another problem is the news media, which can be hesitant to confront the
government and the chaebol, who are major advertisers. The president
effectively handpicks the heads of the two biggest television stations,
and the government can revoke the licenses of cable news channels.

Journalists who tried to investigate Ms. Choi suffered a vicious
official backlash.

As early as 2014, the Segye Ilbo newspaper reported on an intelligence
document alleging influence-peddling by Ms. Choi's family. Ms. Park
attacked the leak, and her office pressed the newspaper to fire its
president, according to the impeachment motion.

Instead of investigating the allegations in the document, prosecutors
interrogated Segye journalists on possible defamation charges, and
reporters at the newspaper said the tax authorities had begun
investigating businesses owned by the paper's parent company.

A police officer accused of leaking the document killed himself.
``Listen, journalists!'' Lt. Choi Kyong-rak wrote in his suicide note.
``The people's right to know is what you live and exist for. Please do
your job.''

Given the authority of the presidency, relatives and close friends often
operate as rainmakers. In the past, the presidents' siblings and sons,
while holding no official titles, often wielded enormous power as
``junior presidents.''

Ms. Park is unmarried, childless and estranged from her siblings, a
status that she said would free her from nepotism and break the pattern.
But she had Ms. Choi, whose family befriended her after the
assassination of her mother in 1974.

Prosecutors did not aggressively investigate the allegations against Ms.
Choi until after Ms. Park delivered her first televised apology in
October, a day after a local cable channel reported that Ms. Choi had
edited the president's speeches.

The story emboldened the press, prompting a flood of other damaging
disclosures and then the huge street protests that eventually led
prosecutors to conclude it was no longer politically tenable to do
nothing.

Cho Eung-cheon, a former prosecutor who is now an opposition lawmaker,
said the authorities had moved too late.

``The prosecutors we see now,'' he wrote on his Facebook page, ``are
nothing more or less than a pack of hyenas attacking a crippled lion.''

Advertisement

\protect\hyperlink{after-bottom}{Continue reading the main story}

\hypertarget{site-index}{%
\subsection{Site Index}\label{site-index}}

\hypertarget{site-information-navigation}{%
\subsection{Site Information
Navigation}\label{site-information-navigation}}

\begin{itemize}
\tightlist
\item
  \href{https://help.nytimes.com/hc/en-us/articles/115014792127-Copyright-notice}{©~2020~The
  New York Times Company}
\end{itemize}

\begin{itemize}
\tightlist
\item
  \href{https://www.nytco.com/}{NYTCo}
\item
  \href{https://help.nytimes.com/hc/en-us/articles/115015385887-Contact-Us}{Contact
  Us}
\item
  \href{https://www.nytco.com/careers/}{Work with us}
\item
  \href{https://nytmediakit.com/}{Advertise}
\item
  \href{http://www.tbrandstudio.com/}{T Brand Studio}
\item
  \href{https://www.nytimes.com/privacy/cookie-policy\#how-do-i-manage-trackers}{Your
  Ad Choices}
\item
  \href{https://www.nytimes.com/privacy}{Privacy}
\item
  \href{https://help.nytimes.com/hc/en-us/articles/115014893428-Terms-of-service}{Terms
  of Service}
\item
  \href{https://help.nytimes.com/hc/en-us/articles/115014893968-Terms-of-sale}{Terms
  of Sale}
\item
  \href{https://spiderbites.nytimes.com}{Site Map}
\item
  \href{https://help.nytimes.com/hc/en-us}{Help}
\item
  \href{https://www.nytimes.com/subscription?campaignId=37WXW}{Subscriptions}
\end{itemize}
