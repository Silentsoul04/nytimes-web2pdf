Sections

SEARCH

\protect\hyperlink{site-content}{Skip to
content}\protect\hyperlink{site-index}{Skip to site index}

\href{https://www.nytimes.com/section/business}{Business}

\href{https://myaccount.nytimes.com/auth/login?response_type=cookie\&client_id=vi}{}

\href{https://www.nytimes.com/section/todayspaper}{Today's Paper}

\href{/section/business}{Business}\textbar{}Even Before He Takes Office,
Trump Knocks Automakers on Their Heels

\url{https://nyti.ms/2j11Tuj}

\begin{itemize}
\item
\item
\item
\item
\item
\end{itemize}

Advertisement

\protect\hyperlink{after-top}{Continue reading the main story}

Supported by

\protect\hyperlink{after-sponsor}{Continue reading the main story}

\hypertarget{even-before-he-takes-office-trump-knocks-automakers-on-their-heels}{%
\section{Even Before He Takes Office, Trump Knocks Automakers on Their
Heels}\label{even-before-he-takes-office-trump-knocks-automakers-on-their-heels}}

\includegraphics{https://static01.nyt.com/images/2017/01/03/business/cnbc-ford/cnbc-ford-videoSixteenByNineJumbo1600.png}

By \href{http://www.nytimes.com/by/bill-vlasic}{Bill Vlasic} and Neal E.
Boudette

\begin{itemize}
\item
  Jan. 3, 2017
\item
  \begin{itemize}
  \item
  \item
  \item
  \item
  \item
  \end{itemize}
\end{itemize}

DETROIT --- Donald J. Trump has promised to change the way American
automakers do business. Less than three weeks before his inauguration as
president, he has already knocked the companies on their heels.

In a stunning reversal, Ford Motor, the nation's second-largest
automaker, said on Tuesday that it would scrap plans to build a
small-car assembly plant in Mexico that Mr. Trump has repeatedly
criticized.

Just a few hours earlier, Mr. Trump threatened to impose tariffs on cars
made in Mexico by General Motors, the nation's largest automaker. His
message forced the company to defend itself.

Both developments indicate how Mr. Trump is having an enormous impact on
how American car companies run their operations, even before he takes
office. They also illustrate that one of Mr. Trump's particular points
of criticism, manufacturing in Mexico, has become particularly
sensitive.

But the moves raise questions about how competitive the country's auto
industry can be if its manufacturing options shrink in Mexico, and what
the implications will be for consumers. For now, at least, some
executives are praising Mr. Trump's economic plans.

``We are encouraged by the pro-growth plans that President-elect Trump
and the new Congress indicate they will pursue,'' Mark Fields, Ford's
chief executive, said at an event on Tuesday.

The decision by Ford to drop plans for a new plant in Mexico --- what
would have been a \$1.6 billion investment --- came at the same time the
company announced it would add 700 jobs to build electric and hybrid
vehicles at a plant in Flat Rock, Mich.

The new Mexican factory was to build Ford Focus sedans currently
manufactured at another Michigan plant near Detroit. Now the company
will build those cars at an existing plant in Mexico.

Ford officials said that the revised plans were tied to market
conditions that have depressed small-car sales, and that they did not
consult with the incoming Trump administration before making the
decision.

They did, though, tell Mr. Trump about the change just before the
announcement. And on Tuesday, Mr. Fields made clear that Mr. Trump's
policies were playing a role in the company's thinking. He added in an
interview that the president-elect's emphasis on tax changes and cutting
regulations should have an overall positive effect on automakers such as
Ford.

\includegraphics{https://static01.nyt.com/images/2017/01/04/business/04AUTO1/04AUTO1-articleLarge.jpg?quality=75\&auto=webp\&disable=upscale}

``We have a president-elect who has said very clearly that one of his
first priorities is to grow the economy,'' he said. ``That should be
music to our ears.''

Ford has been a target of Mr. Trump's criticism since last spring, when
he singled the company out during his campaign for planning to create
jobs in Mexico instead of pushing employment in the United States. After
the election, Ford dropped plans to move production of a Lincoln S.U.V.
to Mexico from Kentucky. That move followed discussions between Mr.
Trump and William C. Ford Jr., the company's chairman.

One industry analyst,
\href{http://www.oliverwyman.com/our-culture/our-people/ron-harbour.html}{Ron
Harbour} of the consulting firm Oliver Wyman, said Ford was under
intense pressure to alter its Mexican plans --- or risk a constant
drumbeat of criticism from Mr. Trump.

``It was an embarrassment for them, and they said, `Let's turn this
thing around,''' Mr. Harbour said.

Now Mr. Trump has turned his attention to G.M. In a
\href{https://twitter.com/realDonaldTrump/status/816260343391514624}{Twitter
post} early Tuesday, he attacked the company for making a hatchback
version of a Chevrolet in Mexico for sale in the American market.

``General Motors is sending Mexican made model Chevy Cruze to U.S. car
dealers tax-free across the border,'' Mr. Trump wrote. ``Make in U.S.A.
or pay big border tax!''

A central tenet of Mr. Trump's economic platform has been to renegotiate
the North American Free Trade Agreement, which allows for the free flow
of manufactured goods between the United States, Canada and Mexico.
Instead, he favors tariffs of up to 35 percent on products made in
Mexico and sold in America.

Industry analysts have questioned whether automakers like G.M. and Ford
can profitably build smaller vehicles in the United States instead of in
Mexico, where wages rarely cross \$10 an hour, compared with the \$29 an
hour earned by a majority of unionized American workers.

For consumers, those higher wages could add up to higher sticker prices.
And that could potentially reduce sales.

But Mr. Trump is hardly backing off on his vow to scrap Nafta, and has
found an unlikely ally in the powerful United Automobile Workers union,
which represents hourly employees at G.M., Ford and Fiat Chrysler in the
United States.

While the U.A.W. leadership supported Mr. Trump's rival, Hillary
Clinton, in the presidential election, the union has consistently
attacked Nafta for encouraging car companies to invest in Mexico.

Image

At a news conference on Tuesday, Ford's chief executive, Mark Fields,
said the company was optimistic that Mr. Trump and the new
Republican-controlled Congress would pursue growth policies that will
strengthen American competitiveness in manufacturing.Credit...Carlos
Osorio/Associated Press

``The U.A.W. has long believed that companies that sell in our country
should build their products in our country,'' the union's president,
\href{https://uaw.org/executive-board/uaw-president-dennis-williams/}{Dennis
Williams}, said on Tuesday.

The hatchback made by G.M. in Mexico is a version of its Cruze compact
car produced primarily at a factory in Lordstown, Ohio.

Sales of the Cruze, like many other passenger cars, have fallen in
recent months because of low gas prices and shifting consumer demand
toward more spacious sport utility vehicles. The Lordstown factory is
among five American plants that G.M. will temporarily idle this month to
reduce its growing inventories of slow-selling cars.

G.M. officials declined to comment on Mr. Trump's Twitter attack, other
than to say in a statement that only a ``small number'' of Cruze
hatchbacks were built in Mexico for the American market.

But G.M. has a large exposure to any potential changes looming on Nafta,
having committed up to \$5 billion in long-term investment in Mexico.
Foreign car companies like Volkswagen and Toyota are also adding jobs
and new products at their Mexican facilities.

With Nafta under fire from the incoming administration, Ford, in
particular, has tried to adapt.

In a recent interview, the company's chief financial officer, Robert L.
Shanks, said the automaker was expecting changes in trade deals, and
increasing its focus on expanding its manufacturing in the United
States. ``The bigger principle is we want to grow the U.S. economy,'' he
said.

On Tuesday, the company packaged a series of announcements on new
electrified vehicles with a promise to invest \$700 million in its Flat
Rock assembly plants.

The addition of 700 jobs at the plant will help it build a new fully
electric S.U.V. to debut in 2020, as well as a new autonomous vehicle
that has no steering wheel and operates entirely by computer.

Ford's vice president of global purchasing,
\href{https://media.ford.com/content/fordmedia/fna/us/en/people/hau-thai-tang.html}{Hau
Thai-Tang}, said the company chose the Flat Rock facility ``to really
show we are making a commitment to the United States and to
technology.''

Mr. Fields, however, was more circumspect on why the company had dropped
its plans for the new Mexican factory. ``We didn't need it anymore,'' he
said. ``We just don't need the capacity anymore given the demand for
small cars.''

Still, the news about the Mexican plant and the new jobs in Michigan
were conveyed directly to Mr. Trump and Vice President-elect Mike Pence
before they were publicly announced.

``We called the president-elect and the vice president-elect this
morning and gave them the news,'' Mr. Fields said Tuesday. ``They were
very pleased, obviously, that we were making these investments in the
U.S.''

Advertisement

\protect\hyperlink{after-bottom}{Continue reading the main story}

\hypertarget{site-index}{%
\subsection{Site Index}\label{site-index}}

\hypertarget{site-information-navigation}{%
\subsection{Site Information
Navigation}\label{site-information-navigation}}

\begin{itemize}
\tightlist
\item
  \href{https://help.nytimes.com/hc/en-us/articles/115014792127-Copyright-notice}{©~2020~The
  New York Times Company}
\end{itemize}

\begin{itemize}
\tightlist
\item
  \href{https://www.nytco.com/}{NYTCo}
\item
  \href{https://help.nytimes.com/hc/en-us/articles/115015385887-Contact-Us}{Contact
  Us}
\item
  \href{https://www.nytco.com/careers/}{Work with us}
\item
  \href{https://nytmediakit.com/}{Advertise}
\item
  \href{http://www.tbrandstudio.com/}{T Brand Studio}
\item
  \href{https://www.nytimes.com/privacy/cookie-policy\#how-do-i-manage-trackers}{Your
  Ad Choices}
\item
  \href{https://www.nytimes.com/privacy}{Privacy}
\item
  \href{https://help.nytimes.com/hc/en-us/articles/115014893428-Terms-of-service}{Terms
  of Service}
\item
  \href{https://help.nytimes.com/hc/en-us/articles/115014893968-Terms-of-sale}{Terms
  of Sale}
\item
  \href{https://spiderbites.nytimes.com}{Site Map}
\item
  \href{https://help.nytimes.com/hc/en-us}{Help}
\item
  \href{https://www.nytimes.com/subscription?campaignId=37WXW}{Subscriptions}
\end{itemize}
