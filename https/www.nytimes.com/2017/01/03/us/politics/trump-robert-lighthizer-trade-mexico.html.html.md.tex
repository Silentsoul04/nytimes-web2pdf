Sections

SEARCH

\protect\hyperlink{site-content}{Skip to
content}\protect\hyperlink{site-index}{Skip to site index}

\href{https://www.nytimes.com/section/politics}{Politics}

\href{https://myaccount.nytimes.com/auth/login?response_type=cookie\&client_id=vi}{}

\href{https://www.nytimes.com/section/todayspaper}{Today's Paper}

\href{/section/politics}{Politics}\textbar{}With Choice of Trade
Negotiator, Trump Prepares to Confront Mexico and China

\url{https://nyti.ms/2hQddeU}

\begin{itemize}
\item
\item
\item
\item
\item
\item
\end{itemize}

Advertisement

\protect\hyperlink{after-top}{Continue reading the main story}

Supported by

\protect\hyperlink{after-sponsor}{Continue reading the main story}

\hypertarget{with-choice-of-trade-negotiator-trump-prepares-to-confront-mexico-and-china}{%
\section{With Choice of Trade Negotiator, Trump Prepares to Confront
Mexico and
China}\label{with-choice-of-trade-negotiator-trump-prepares-to-confront-mexico-and-china}}

By \href{http://www.nytimes.com/by/binyamin-appelbaum}{Binyamin
Appelbaum}

\begin{itemize}
\item
  Jan. 3, 2017
\item
  \begin{itemize}
  \item
  \item
  \item
  \item
  \item
  \item
  \end{itemize}
\end{itemize}

WASHINGTON --- President-elect Donald J. Trump on Tuesday named as his
chief trade negotiator a Washington lawyer who has long advocated
protectionist policies, the latest sign that Mr. Trump intends to
fulfill his campaign promise to get tough with China, Mexico and other
trading partners.

Mr. Trump also renewed his episodic campaign to persuade American
companies to expand domestic manufacturing, criticizing General Motors
\href{https://twitter.com/realDonaldTrump/status/816260343391514624}{via
Twitter on Tuesday morning} for making in Mexico some of the Chevrolet
Cruze hatchbacks it sells domestically. Hours later, Mr. Trump claimed
credit after Ford said
\href{http://www.nytimes.com/2017/01/03/business/ford-general-motors-trump.html?hp\&action=click\&pgtype=Homepage\&clickSource=story-heading\&module=first-column-region\&region=top-news\&WT.nav=top-news}{it
would expand vehicle production in Flat Rock, Mich}.

The choice of Robert Lighthizer (pronounced LIGHT-hi-zer) to be the
United States' trade representative nearly completes Mr. Trump's
selection of top economic advisers and, taken together with the
president-elect's running commentary on Twitter, underscores Mr. Trump's
focus on making things in America. That is causing unease among some
Republicans who regard Mr. Trump's views on trade as
\href{http://www.nytimes.com/2016/03/11/us/politics/-trade-donald-trump-breaks-200-years-economic-orthodoxy-mercantilism.html}{dangerously
retrograde}, even as they embrace the bulk of his economic agenda.

Mainstream economists warn that protectionist policies like import taxes
could impose higher prices on consumers and slow economic growth.

But some Democrats are signaling a readiness to support Mr. Trump. Nine
House Democrats held a news conference Tuesday with the A.F.L.-C.I.O.
president,
\href{https://www.nytimes.com/2016/12/27/opinion/dont-let-trump-speak-for-workers.html?_r=0}{Richard
Trumka}, to urge renegotiation of the North American Free Trade
Agreement with Mexico and Canada.

``We wanted him to know that we'll work with him on doing that,'' Mr.
Trumka said. ``I don't think he has enough Republican support to do it,
and rewriting the rules of trade is a necessary first step in righting
the economy for working people.''

Mr. Trump and his top advisers on trade, including Mr. Lighthizer, share
a view that the United States in recent decades prioritized the ideal of
free trade over its own self-interest. They argue that other countries
are undermining America's industrial base by subsidizing their own
export industries while impeding American importers. They regard this
unfair competition as a key reason for the lackluster growth of the
economy.

In picking Mr. Lighthizer, who has spent much of the last few decades
representing American steel producers in their frequent litigation of
trade disputes, Mr. Trump is seeking to hire one of Washington's top
trade lawyers to enforce international trade agreements more vigorously.
He must be confirmed by the Senate.

``He will do an amazing job helping turn around the failed trade
policies which have robbed so many Americans of prosperity,'' Mr. Trump
said in a statement.

Mainstream Republicans have sought common ground with Mr. Trump,
emphasizing, for example, the importance of enforcing trade rules, but
they have not abandoned the party's longtime advocacy for trade. Senator
Orrin Hatch of Utah, the chairman of the Senate Finance Committee, which
will hold hearings on Mr. Lighthizer's nomination, issued a cautiously
supportive statement Tuesday.

``As the world and our economic competitors move to expand their global
footprints, we can't afford to be left behind in securing strong deals
that will increase our access to new markets for American-made products
and services,'' Mr. Hatch said in a statement. ``I look forward to a
vigorous discussion of Bob's trade philosophy and priorities.''

Mr. Trump has named a number of advisers on trade, leaving some
ambiguity about the division of responsibilities. The president-elect
\href{https://www.nytimes.com/2016/12/21/us/politics/peter-navarro-carl-icahn-trump-china-trade.html?_r=0}{named
the economist Peter Navarro}, an outspoken critic of China, to lead a
new White House office overseeing trade and industrial policy. Mr. Trump
also said Wilbur Ross, the billionaire investor and
\href{https://www.nytimes.com/2016/11/25/business/dealbook/wilbur-ross-commerce-secretary-donald-trump.html}{choice
for commerce secretary}, will play a key role. Mr. Lighthizer, however,
is the only member of the triumvirate with government experience.

``Those who say U.S.T.R. will be subordinated to other agencies are
mistaken,'' said Alan Wolff, another former senior American trade
official who was the steel industry's co-counsel on trade with Mr.
Lighthizer for nearly 20 years, citing Mr. Lighthizer's encyclopedic
knowledge of trade law. ``He'll be a dominant figure on trade, in
harmony with Wilbur Ross and Navarro.''

\includegraphics{https://static01.nyt.com/images/2017/01/04/us/04Trade/04Trade-articleLarge.jpg?quality=75\&auto=webp\&disable=upscale}

There is also an ideological divide between the people Mr. Trump has
named to oversee trade policy and his broader circle of advisers, which
is populated by longstanding trade advocates like
\href{https://www.nytimes.com/2016/12/12/business/dealbook/goldman-sachs-gary-cohn.html}{Gary
D. Cohn}, the president of Goldman Sachs, who will lead the National
Economic Council;
\href{https://www.nytimes.com/2016/12/12/us/politics/rex-tillerson-secretary-of-state-trump.html}{Rex
W. Tillerson}, the chief executive of Exxon Mobil, tapped for secretary
of state; and Gov. Terry Branstad of Iowa,
\href{http://www.nytimes.com/2016/12/07/us/politics/terry-branstad-china-ambassador-trump.html?_r=0}{Mr.
Trump's choice for ambassador to China}.

Proponents of trade hope the broader circle, and congressional
Republicans, will exert a moderating influence.

``You're seeing a pretty clear indication that there will be a focus on
the enforcement of our trade agreements and on the letter of the law,''
said Scott Lincicome, an international trade lawyer at White \& Case.
``But that doesn't necessarily mean a significant turn toward
protectionism. Even free trade guys like me support enforcement.''

Trade opponents on the left and the right, meanwhile, are hoping Mr.
Trump means to break with several decades of pro-trade policy.

``There's going to be a war within the Trump administration on where
they go with trade, and we're hoping to energize the worker base he had
to make sure they go in the right direction to benefit the American
worker,'' Mr. Trumka said.

Mr. Trump's promise to immediately designate China as a currency
manipulator may offer an early test of the administration's intentions.
Economists
\href{http://www.nytimes.com/2016/10/01/business/dealbook/china-trump-yuan-devaluation.html}{see
no evidence China is suppressing the value of its currency}, although it
has done so in the past. Mr. Lincicome said officially labeling China a
currency manipulator despite the lack of recent evidence would signal
that the administration is taking a hard line on trade issues.

A broader shift in trade policy would unfold more slowly. Mr. Trump has
promised to renegotiate Nafta; the original process took most of three
years. He has promised to pursue enforcement actions against other
nations, but it takes time to mount cases. He has threatened to impose
new tariffs on imports, but sweeping changes most likely would require
congressional legislation.

Mr. Trump already is seeking to exert influence by seizing the
presidential bullhorn.

``General Motors is sending Mexican made model of Chevy Cruze to U.S.
car dealers-tax free across border,'' he wrote Tuesday on Twitter.
``Make in U.S.A. or pay big border tax!''

General Motors announced in 2015 that it would make the Cruze in
Coahuila, Mexico. American manufacturers are moving small-car production
to Mexico to take advantage of lower labor costs and because of
declining domestic demand. They continue to build more expensive
vehicles in the United States.

Ford's announcement Tuesday does not reverse that trend. The carmaker
said it still planned to move production of the compact Ford Focus from
Michigan to Mexico. But it said it would invest in a different Michigan
plant to expand production of higher-priced vehicles, including its
F-150 pickup truck and the Mustang sports car, as well as a new
battery-powered sport utility vehicle.

``We are encouraged by the pro-growth policies that President-elect
Trump and the new Congress have indicated they will pursue,'' said the
company's chief executive, Mark Fields.

Mr. Lighthizer served as deputy United States trade representative in
the Ronald Reagan administration, when he was involved in pressing Japan
to reduce its restrictions on American imports and its subsidies for its
own exports. Mr. Trump has criticized China for similar practices,
setting the stage for a new round of confrontations.

Reagan is often remembered as an advocate for free trade, but his
administration in its early hours imposed a quota on Japanese auto
imports. It was the first in a long series of measures aimed at putting
pressure on the nation that was then regarded, like China in recent
years, as a threat to American prosperity.

``President Reagan's pragmatism contrasted strongly with the utopian
dreams of free traders,'' Mr. Lighthizer
\href{http://www.nytimes.com/2008/03/06/opinion/06lighthizer.html}{wrote
in a 2008 piece} criticizing Senator John McCain, Republican of Arizona,
for embracing ``unbridled'' free trade. Conservatives, he argued,
``always understood that trade policy was merely a tool for building a
strong and independent country with a prosperous middle class.''

Advertisement

\protect\hyperlink{after-bottom}{Continue reading the main story}

\hypertarget{site-index}{%
\subsection{Site Index}\label{site-index}}

\hypertarget{site-information-navigation}{%
\subsection{Site Information
Navigation}\label{site-information-navigation}}

\begin{itemize}
\tightlist
\item
  \href{https://help.nytimes.com/hc/en-us/articles/115014792127-Copyright-notice}{©~2020~The
  New York Times Company}
\end{itemize}

\begin{itemize}
\tightlist
\item
  \href{https://www.nytco.com/}{NYTCo}
\item
  \href{https://help.nytimes.com/hc/en-us/articles/115015385887-Contact-Us}{Contact
  Us}
\item
  \href{https://www.nytco.com/careers/}{Work with us}
\item
  \href{https://nytmediakit.com/}{Advertise}
\item
  \href{http://www.tbrandstudio.com/}{T Brand Studio}
\item
  \href{https://www.nytimes.com/privacy/cookie-policy\#how-do-i-manage-trackers}{Your
  Ad Choices}
\item
  \href{https://www.nytimes.com/privacy}{Privacy}
\item
  \href{https://help.nytimes.com/hc/en-us/articles/115014893428-Terms-of-service}{Terms
  of Service}
\item
  \href{https://help.nytimes.com/hc/en-us/articles/115014893968-Terms-of-sale}{Terms
  of Sale}
\item
  \href{https://spiderbites.nytimes.com}{Site Map}
\item
  \href{https://help.nytimes.com/hc/en-us}{Help}
\item
  \href{https://www.nytimes.com/subscription?campaignId=37WXW}{Subscriptions}
\end{itemize}
