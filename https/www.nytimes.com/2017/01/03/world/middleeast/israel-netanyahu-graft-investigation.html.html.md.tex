Sections

SEARCH

\protect\hyperlink{site-content}{Skip to
content}\protect\hyperlink{site-index}{Skip to site index}

\href{https://www.nytimes.com/section/world/middleeast}{Middle East}

\href{https://myaccount.nytimes.com/auth/login?response_type=cookie\&client_id=vi}{}

\href{https://www.nytimes.com/section/todayspaper}{Today's Paper}

\href{/section/world/middleeast}{Middle East}\textbar{}In Graft Inquiry,
Benjamin Netanyahu's Worst Enemy May Be Himself

\url{https://nyti.ms/2hPK7w6}

\begin{itemize}
\item
\item
\item
\item
\item
\end{itemize}

Advertisement

\protect\hyperlink{after-top}{Continue reading the main story}

Supported by

\protect\hyperlink{after-sponsor}{Continue reading the main story}

\hypertarget{in-graft-inquiry-benjamin-netanyahus-worst-enemy-may-be-himself}{%
\section{In Graft Inquiry, Benjamin Netanyahu's Worst Enemy May Be
Himself}\label{in-graft-inquiry-benjamin-netanyahus-worst-enemy-may-be-himself}}

\includegraphics{https://static01.nyt.com/images/2017/01/04/world/04Israel/04Israel-articleLarge.jpg?quality=75\&auto=webp\&disable=upscale}

By \href{https://www.nytimes.com/by/isabel-kershner}{Isabel Kershner}
and \href{http://www.nytimes.com/by/peter-baker}{Peter Baker}

\begin{itemize}
\item
  Jan. 3, 2017
\item
  \begin{itemize}
  \item
  \item
  \item
  \item
  \item
  \end{itemize}
\end{itemize}

JERUSALEM --- He has dominated the Israeli political scene for years,
outmaneuvering or outlasting one opponent after another, the ultimate
survivor in a system that has chewed up many of his predecessors.

But after a nighttime visit by police investigators who read him his
rights before asking him if he was corrupt, Prime Minister Benjamin
Netanyahu now faces a new kind of challenge. The opponent who just might
sink him is none other than Benjamin Netanyahu himself.

Abraham Diskin, a political scientist at the Academic Center of Law and
Science outside Tel Aviv who has advised Mr. Netanyahu, said on Tuesday
that it was ``not very clear that Netanyahu is going to escape some kind
of indictment.'' And ``once there is an indictment,'' Professor Diskin
added, ``he will have to resign.''

Details of the graft investigation remain murky --- ``the truth is that
we don't know anything,'' Professor Diskin said --- but the questions it
raises are clear: Could an appetite for high living abetted by wealthy
business executives ultimately undercut Mr. Netanyahu as it has so many
other politicians in Israel and around the globe?

Past inquiries stretching over Mr. Netanyahu's more than two decades in
public life have examined his family trips and home expenses without
resulting in charges. But allies and enemies alike were already saying
on Tuesday that this case could be different.

The three-hour police interrogation at his official residence on Monday
night came at a high-profile moment for Mr. Netanyahu on the
international stage.

After years of grinding battle, he has engaged in a guns-blazing war
with President Obama during the president's final weeks in office. At
the same time, Mr. Netanyahu has been warmly embraced by President-elect
Donald J. Trump, presenting a closer alignment than any other
president-prime minister pairing in modern times.

A conservative with a stable governing coalition, Mr. Netanyahu emerged
from the police interrogation on Monday with a defiant blast at
left-wing elites and the news media for prematurely counting him out.
Appealing to supporters on the right, he dismissed the allegations
against him, insisting they will all ``come to nothing because there is
nothing.''

But the attorney general, Avichai Mandelblit, who was Mr. Netanyahu's
choice for the post after a tour as his cabinet secretary, has been a
paragon of caution, taking months to turn a preliminary inquiry into a
criminal investigation, something that Israeli legal and political
commentators said added weight to the suspicions.

``Mandelblit tried doing everything he could so as not to reach the
point of having Netanyahu questioned as a suspect,'' Amir Oren, a
columnist, \href{http://www.haaretz.com/opinion/.premium-1.762624}{wrote
in the Haaretz newspaper} on Tuesday. ``If, after all his contortions
and twists, the A.G. was forced into ordering an investigation that he
was hoping to shelve, the default option from here on is an
indictment.''

After the police investigators left Mr. Netanyahu's home late Monday,
Mr. Mandelblit issued a statement close to midnight, vaguely outlining
the contours of the case. It said Mr. Netanyahu was suspected of
receiving gifts and benefits.

By early Tuesday, Israelis were debating how all this might affect Mr.
Netanyahu, whether sitting prime ministers should even be investigated,
and if so, at what point they should resign.

The months of multiple inquiries have centered on issues involving
``moral integrity'' --- a term that legal experts said could include
suspicions of breach of trust, conflict of interest and bribery.

Much of Mr. Mandelblit's statement was devoted to
\href{https://www.nytimes.com/2016/08/01/world/middleeast/israel-netanyahu-scandal-corruption.html?_r=0}{prior
cases} that were closed because of lack of evidence of crimes, including
a travel expense scandal known as Bibi Tours, using Mr. Netanyahu's
nickname, and one about campaign funding in 2009.

Mr. Netanyahu, who is serving his third consecutive term since that
election, and his fourth over all, denounced those inquiries in a
\href{https://www.facebook.com/Netanyahu/}{Facebook post}.

``Bibi Tours --- nothing! Claim of forbidden election funding ---
nothing!'' he wrote, describing reports in the Israeli news media over
the years as ``daily persecution of me and my family'' and ``complete
gibberish.''

Israeli law requires a prime minister to resign if convicted of serious
crimes. But in a precedential ruling in 1993, the Israeli Supreme Court
determined that a cabinet minister, Aryeh Deri, had to resign
immediately after the attorney general submitted an indictment against
him for corruption.

``If everyone who faces an investigation would have to resign, this
could turn into a tool used to politically destroy others,'' Ayelet
Shaked, the justice minister, told Israel Radio. ``This is why the idea
that resigning should only be after an indictment is a correct one,''
she added.

Monday was not the first time Mr. Netanyahu had been questioned as a
criminal suspect while in office. During his first term, in the late
1990s, he was queried in a case that involved the appointment of an
attorney general as part of what the police said was a political deal
involving Mr. Deri. Mr. Netanyahu was never charged in that case, nor in
several others for which he was investigated while out of office.

Other Israeli prime ministers, too, have been the subject of police
investigations into allegations of campaign irregularities and
corruption. Ehud Barak and Ariel Sharon were both questioned. Mr.
Netanyahu's predecessor, Ehud Olmert, was forced from office in 2008,
even before any charges were filed against him, and became the first
former Israeli prime minister to
\href{http://www.nytimes.com/2016/02/16/world/middleeast/ehud-olmert-israel-prison.html}{enter
prison}. He is serving a 19-month term for bribery and obstruction of
justice.

Moshe Katsav, a former president, was recently released from prison
after serving five years of a seven-year term for rape.

Some of Mr. Olmert's supporters felt he was the victim of an overzealous
state prosecutor, though many Israelis take pride in the fact that
nobody is immune from the law.

But with Netanyahu loyalists asserting that he is the victim of a
political witch hunt, some right-wing legislators are now trying to
advance a bill that would ban criminal investigations of incumbent prime
ministers. (Any such legislation would likely come too late to spare Mr.
Netanyahu.)

Critics of the bill argue that such a ban would be problematic in a
small country like Israel, and with a political system that has no term
limit for prime ministers.

``If prime ministers knew there would be an investigation only after
their term was over, they would turn the prime minister's office into a
safe haven and use the powers and resources of the state to extend their
stay in office and avoid prosecution,'' said Yohanan Plesner, the
president of the Israel Democracy Institute, a nonpartisan research
center in Jerusalem.

Mr. Plesner, a former Parliament member, added that suspicions would
still surface in the news media and if they were not formally
investigated, public trust in political institutions could erode.

When Mr. Olmert was under investigation, it was Mr. Barak, then his
defense minister --- though from a different political party --- who
pushed him to leave office. Israel's coalition governments have an
inherent weakness, in that parties participating in the coalition are
often also political rivals competing for votes in the next election.

So far Mr. Netanyahu's party and coalition partners appear to be backing
him, either voicing support or remaining silent.

But the investigation could last for months, if not more, and that could
create political strain.

For now, Mr. Plesner said, ``whoever is critical of Mr. Netanyahu will
use this to continue to be critical, and his supporters will find their
ways to interpret it.''

Advertisement

\protect\hyperlink{after-bottom}{Continue reading the main story}

\hypertarget{site-index}{%
\subsection{Site Index}\label{site-index}}

\hypertarget{site-information-navigation}{%
\subsection{Site Information
Navigation}\label{site-information-navigation}}

\begin{itemize}
\tightlist
\item
  \href{https://help.nytimes.com/hc/en-us/articles/115014792127-Copyright-notice}{©~2020~The
  New York Times Company}
\end{itemize}

\begin{itemize}
\tightlist
\item
  \href{https://www.nytco.com/}{NYTCo}
\item
  \href{https://help.nytimes.com/hc/en-us/articles/115015385887-Contact-Us}{Contact
  Us}
\item
  \href{https://www.nytco.com/careers/}{Work with us}
\item
  \href{https://nytmediakit.com/}{Advertise}
\item
  \href{http://www.tbrandstudio.com/}{T Brand Studio}
\item
  \href{https://www.nytimes.com/privacy/cookie-policy\#how-do-i-manage-trackers}{Your
  Ad Choices}
\item
  \href{https://www.nytimes.com/privacy}{Privacy}
\item
  \href{https://help.nytimes.com/hc/en-us/articles/115014893428-Terms-of-service}{Terms
  of Service}
\item
  \href{https://help.nytimes.com/hc/en-us/articles/115014893968-Terms-of-sale}{Terms
  of Sale}
\item
  \href{https://spiderbites.nytimes.com}{Site Map}
\item
  \href{https://help.nytimes.com/hc/en-us}{Help}
\item
  \href{https://www.nytimes.com/subscription?campaignId=37WXW}{Subscriptions}
\end{itemize}
