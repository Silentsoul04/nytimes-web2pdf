Sections

SEARCH

\protect\hyperlink{site-content}{Skip to
content}\protect\hyperlink{site-index}{Skip to site index}

\href{https://www.nytimes.com/section/world/asia}{Asia Pacific}

\href{https://myaccount.nytimes.com/auth/login?response_type=cookie\&client_id=vi}{}

\href{https://www.nytimes.com/section/todayspaper}{Today's Paper}

\href{/section/world/asia}{Asia Pacific}\textbar{}Japan Recalls
Ambassador to South Korea to Protest `Comfort Woman' Statue

\url{https://nyti.ms/2hZol9w}

\begin{itemize}
\item
\item
\item
\item
\item
\end{itemize}

Advertisement

\protect\hyperlink{after-top}{Continue reading the main story}

Supported by

\protect\hyperlink{after-sponsor}{Continue reading the main story}

\hypertarget{japan-recalls-ambassador-to-south-korea-to-protest-comfort-woman-statue}{%
\section{Japan Recalls Ambassador to South Korea to Protest `Comfort
Woman'
Statue}\label{japan-recalls-ambassador-to-south-korea-to-protest-comfort-woman-statue}}

\includegraphics{https://static01.nyt.com/images/2017/01/07/world/07KOREA-1/07KOREA-1-articleLarge.jpg?quality=75\&auto=webp\&disable=upscale}

By \href{http://www.nytimes.com/by/choe-sang-hun}{Choe Sang-Hun} and
\href{http://www.nytimes.com/by/motoko-rich}{Motoko Rich}

\begin{itemize}
\item
  Jan. 6, 2017
\item
  \begin{itemize}
  \item
  \item
  \item
  \item
  \item
  \end{itemize}
\end{itemize}

SEOUL, South Korea --- Japan recalled its envoy to South Korea on Friday
to protest a statue commemorating Korean women who were forced into
sexual slavery for Japanese soldiers during World War II, in the latest
sign that ties between Washington's two key Asian allies were again
deteriorating over the bitter historical issue.

``The Japanese government finds this situation extremely regrettable,''
Yoshihide Suga, chief cabinet secretary to Prime Minister Shinzo Abe,
said during a news conference in Tokyo, referring to the placement of
the statue
\href{http://www.nytimes.com/2016/12/30/world/asia/south-korea-comfort-women-wwii-japan.html}{outside
the Japanese Consulate in Busan}, South Korea's second-largest city,
last week.

A spokesman for Japan's Ministry of Foreign Affairs, Yasuhisa Kawamura,
said the ambassador, Yasumasa Nagamine, as well as the consul general in
Busan, Yasuhiro Morimoto, had been recalled ``temporarily,'' declining
to say when they would return.

Japan also said it would suspend negotiations over a currency swap meant
to help South Korea stabilize its currency, the won, in times of
financial crisis. It also suspended high-level economic talks and said
staff at the consulate in Busan would not attend events organized by the
city government.

South Korea showed no sign of acquiescing to Japan's demand that it
immediately remove the statue in Busan, a port city in the country's
southeast. ``We want to stress again that despite difficult issues
facing us, both governments must strive to develop bilateral relations
based on mutual trust,'' said Cho June-hyuck, a spokesman for the
Foreign Ministry, who called Japan's announcement ``regrettable.''

South Korea's Finance Ministry urged Tokyo to keep diplomatic disputes
out of economic and financial relations.

Washington has repeatedly appealed to South Korea and Japan to overcome
the persistent, bitter legacies of Japan's brutal colonial rule over
Korea in the first half of the 20th century and to work more closely
together to better address North Korea's advancing threat of nuclear
weapons and China's expanding influence.

But the issue of the comfort women, as the former sex slaves were
euphemistically called in Japan and South Korea, remains seemingly
intractable, despite a 2015 agreement between the countries that was
meant to put the dispute behind them.

Image

Yasumasa Nagamine, Japan's ambassador to South Korea.Credit...Yonhap,
via European Pressphoto Agency

Surviving former sex slaves and their advocates
\href{http://www.nytimes.com/2011/12/16/world/asia/statute-in-seoul-becomes-focal-point-of-dispute-between-south-korea-and-japan.html}{angered
Japan} in 2011 when they installed the first in a series of comfort
woman statues, in front of the Japanese Embassy in Seoul. The bronze,
life-size statue, of a barefoot girl in traditional Korean dress sitting
in a chair, was placed so that diplomats would see it as they left the
office. It is still there, with Korean activists guarding it around the
clock to ensure that it is not removed.

Since then, activists have put up dozens more such statues, in South
Korea and abroad. But the one in Busan was only the second to be
installed near a Japanese diplomatic mission.

Mr. Kawamura said that statue violated the spirit of
\href{http://www.nytimes.com/2015/12/29/world/asia/comfort-women-south-korea-japan.html}{the
deal the countries struck in December 2015} to resolve their dispute
over the extent of Tokyo's responsibility for what the women had to
endure. In that agreement, which both sides called ``a final and
irreversible resolution,'' Japan apologized and promised \$8.3 million
to care for the surviving women, in return for South Korea's promise not
to press any future claims. South Korea also promised to discuss Japan's
complaint about the Seoul statue with activists and survivors.

``Each side, Japan and South Korea respectively, should implement the
agreement with a sense of responsibility,'' Mr. Kawamura said,
specifying that the deal should extend to the statue in Busan.

South Korea also reaffirmed its commitment to the agreement, though it
has proved to be one of the most unpopular decisions made by President
Park Geun-hye, whose powers have been suspended since the National
Assembly
\href{http://www.nytimes.com/2016/12/09/world/asia/south-korea-president-park-geun-hye-impeached.html}{voted
to impeach her} last month over a corruption scandal. The agreement fell
short of the survivors' demand that Japan pay formal reparations and
accept legal responsibility for what happened to them.

On Dec. 28, the first anniversary of the agreement, civic groups in
Busan installed the statue on a sidewalk near the Japanese Consulate,
despite repeated protests from Tokyo and the consulate.

The local government immediately removed it, saying it had been placed
there without permission, but bowed to public pressure two days later
and allowed it to be put back. A visit that week by Japan's defense
minister, Tomomi Inada, to the Yasukuni Shrine in Tokyo, which
commemorates a number of convicted war criminals along with Japan's
other war dead, had deepened resentments in South Korea.

Shinsuke Sugiyama, Japan's vice minister for foreign affairs, who is in
Washington attending talks with his American and South Korean
counterparts to discuss North Korea and other security issues, lodged an
official complaint with his South Korean counterpart, Lim Sung-nam, over
the Busan statue on Thursday. For his part, Mr. Lim strongly protested
Ms. Inada's visit to the shrine, officials here said on Friday.

Japan last recalled its envoy to Seoul in 2012, after South Korea's
president at the time,
\href{http://www.nytimes.com/2012/08/11/world/asia/south-koreans-visit-to-disputed-islets-angers-japan.html}{Lee
Myung-bak, flew to a set of islets} that both countries claim as their
territory. The ambassador returned after 12 days. South Korea
temporarily recalled its own ambassador to Tokyo in 2008, to protest new
guidelines for Japanese textbooks that asserted Japan's
\href{http://www.nytimes.com/2008/08/31/world/asia/31islands.html}{claim
to those islets}.

Advertisement

\protect\hyperlink{after-bottom}{Continue reading the main story}

\hypertarget{site-index}{%
\subsection{Site Index}\label{site-index}}

\hypertarget{site-information-navigation}{%
\subsection{Site Information
Navigation}\label{site-information-navigation}}

\begin{itemize}
\tightlist
\item
  \href{https://help.nytimes.com/hc/en-us/articles/115014792127-Copyright-notice}{©~2020~The
  New York Times Company}
\end{itemize}

\begin{itemize}
\tightlist
\item
  \href{https://www.nytco.com/}{NYTCo}
\item
  \href{https://help.nytimes.com/hc/en-us/articles/115015385887-Contact-Us}{Contact
  Us}
\item
  \href{https://www.nytco.com/careers/}{Work with us}
\item
  \href{https://nytmediakit.com/}{Advertise}
\item
  \href{http://www.tbrandstudio.com/}{T Brand Studio}
\item
  \href{https://www.nytimes.com/privacy/cookie-policy\#how-do-i-manage-trackers}{Your
  Ad Choices}
\item
  \href{https://www.nytimes.com/privacy}{Privacy}
\item
  \href{https://help.nytimes.com/hc/en-us/articles/115014893428-Terms-of-service}{Terms
  of Service}
\item
  \href{https://help.nytimes.com/hc/en-us/articles/115014893968-Terms-of-sale}{Terms
  of Sale}
\item
  \href{https://spiderbites.nytimes.com}{Site Map}
\item
  \href{https://help.nytimes.com/hc/en-us}{Help}
\item
  \href{https://www.nytimes.com/subscription?campaignId=37WXW}{Subscriptions}
\end{itemize}
