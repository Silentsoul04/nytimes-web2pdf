Sections

SEARCH

\protect\hyperlink{site-content}{Skip to
content}\protect\hyperlink{site-index}{Skip to site index}

\href{https://www.nytimes.com/section/politics}{Politics}

\href{https://myaccount.nytimes.com/auth/login?response_type=cookie\&client_id=vi}{}

\href{https://www.nytimes.com/section/todayspaper}{Today's Paper}

\href{/section/politics}{Politics}\textbar{}Mike Pompeo, Trump's C.I.A.
Pick, Faces the Balancing Act of His Career

\url{https://nyti.ms/2ikImHD}

\begin{itemize}
\item
\item
\item
\item
\item
\end{itemize}

Advertisement

\protect\hyperlink{after-top}{Continue reading the main story}

Supported by

\protect\hyperlink{after-sponsor}{Continue reading the main story}

\hypertarget{mike-pompeo-trumps-cia-pick-faces-the-balancing-act-of-his-career}{%
\section{Mike Pompeo, Trump's C.I.A. Pick, Faces the Balancing Act of
His
Career}\label{mike-pompeo-trumps-cia-pick-faces-the-balancing-act-of-his-career}}

\includegraphics{https://static01.nyt.com/images/2017/01/12/us/12pompeo/12pompeo-articleLarge.jpg?quality=75\&auto=webp\&disable=upscale}

By \href{http://www.nytimes.com/by/matthew-rosenberg}{Matthew Rosenberg}

\begin{itemize}
\item
  Jan. 11, 2017
\item
  \begin{itemize}
  \item
  \item
  \item
  \item
  \item
  \end{itemize}
\end{itemize}

WASHINGTON --- The good news for Representative Mike Pompeo of Kansas,
President-elect Donald J. Trump's nominee to run the Central
Intelligence Agency, is that he appears to share the same adversarial
view of Russia as most American spies.

The bad news for Mr. Pompeo is that he will have to square his views
with those of Mr. Trump, who has denigrated American intelligence
agencies, praised President Vladimir V. Putin of Russia and is now
contending with a dossier of
\href{https://www.nytimes.com/2017/01/10/us/politics/donald-trump-russia-intelligence.html?hp\&action=click\&pgtype=Homepage\&clickSource=story-heading\&module=first-column-region\&region=top-news\&WT.nav=top-news}{unsubstantiated
reports} that
\href{http://topics.nytimes.com/top/news/international/countriesandterritories/russiaandtheformersovietunion/index.html?inline=nyt-geo}{Russia}
has collected compromising and salacious personal information about him.

Known as a pugnacious Republican
\href{https://www.nytimes.com/2016/11/19/us/politics/donald-trump-mike-pompeo-cia.html}{partisan}
by his colleagues in Congress, Mr. Pompeo is going to have pull off the
political balancing act of his career to keep the confidence of the
Trump White House while winning over the C.I.A., an agency that is
notoriously hostile to outsiders in the best of times. His Senate
confirmation hearing, scheduled for Thursday before the Senate
intelligence committee, will be the first test of whether he has the
diplomatic finesse to manage it.

Mr. Trump's aides will no doubt be keeping close watch for any signs
that they cannot rely on Mr. Pompeo, while Democrats on the committee
will be eager to look for any daylight between the nominee and Mr.
Trump. That will be especially true when it comes to
\href{https://www.nytimes.com/2017/01/06/us/politics/donald-trump-wall-hack-russia.html?hp\&action=click\&pgtype=Homepage\&clickSource=story-heading\&module=first-column-region\&region=top-news\&WT.nav=top-news}{the
assessment} by intelligence agencies that Russia used cyberattacks and
disinformation to undermine American democracy and promote the candidacy
of Mr. Trump.

At the C.I.A., the concerns are more parochial. The message officials
will be looking to hear from Mr. Pompeo is, ``I'm going to protect you
from these guys,'' said Michael V. Hayden, a former director of both the
C.I.A. and the National Security Agency.

``Every television set in Langley is going to be on for that hearing,''
Mr. Hayden said, referring to the town in Northern Virginia where the
C.I.A. is based.

\includegraphics{https://static01.nyt.com/images/2017/01/11/us/politics/11cong-confirmation-roundup-thumb/11cong-confirmation-roundup-thumb-videoSixteenByNine3000.jpg}

Mr. Trump's
\href{https://www.nytimes.com/2016/12/10/us/politics/trump-mocking-claim-that-russia-hacked-election-at-odds-with-gop.html}{mocking
response} to accusations that Russia meddled in the election has opened
an extraordinary breach between an incoming president and the C.I.A.,
and the revelation that intelligence chiefs briefed Mr. Trump last week
on the dossier of unsubstantiated reports is likely to deepen the
divide. The dispute has hit morale hard at the C.I.A., current and
former agency officials said.

At his news conference on Wednesday, Mr. Trump said that intelligence
agencies were ``vital'' and that they would have 90 days after his
inauguration to produce ``a major report on hacking.''

But he also continued to criticize the agencies, suggesting that they
had leaked the dossier. It was ``disgraceful that the intelligence
agencies allowed any information that turned out to be so false and fake
out,'' he said.

The C.I.A. considers its mission to provide cleareyed information and
analysis that is free of political interference, and it considers the
president its main customer. It is sensitive to slights, and many at the
C.I.A. were especially galled by what they considered Mr. Trump's cheap
shots at the mistaken intelligence in the prelude to the Iraq war.

``It prides itself on being the president's agency, so how does that
feel when your patron suddenly is dumping all over your work,'' said
Mark M. Lowenthal, a former C.I.A. analyst.

Memories run long at the C.I.A., and hanging over the dispute with Mr.
Trump is the
\href{http://www.nytimes.com/2006/05/06/washington/06intel.html}{unhappy
tenure} of Porter J. Goss, the last sitting member of Congress named to
lead the agency. Mr. Goss took over in 2004, when the agency was widely
viewed as being at odds with the Bush administration over the Iraq war,
and his marching orders were to end what the White House viewed as a
campaign of leaks by insiders who opposed administration policies.

Mr. Goss failed to stop the leaks. But his attempted crackdown, which
included rare ``single-issue'' polygraph tests of senior officials,
prompted a wave of departures by veterans. Mr. Goss lasted only 13
months, done in by the spreading discontent with his leadership.

\href{https://www.nytimes.com/interactive/2016/us/politics/donald-trump-administration.html}{}

\includegraphics{https://static01.nyt.com/images/2016/11/11/us/politics/donald-trump-administration-1478905372015/donald-trump-administration-1478905372015-square640.jpg}

\hypertarget{donald-trumps-cabinet-is-complete-heres-the-full-list}{%
\subsection{Donald Trump's Cabinet Is Complete. Here's the Full
List.}\label{donald-trumps-cabinet-is-complete-heres-the-full-list}}

A list of appointees and nominees for top posts in the new
administration.

Mr. Pompeo has not made any public comments since his nomination in
November, and how he will approach his new job remains to be seen.

He is best known for his relentless questioning of Hillary Clinton
during the congressional investigation into the 2012 attacks on the
American consulate in Benghazi, Libya. His quips about the attack being
worse than Watergate and his continued insistence that there was a
``cover-up,'' even after the House Select Committee on Benghazi found
\href{http://www.nytimes.com/2016/06/29/us/politics/hillary-clinton-benghazi.html}{no
new evidence of wrongdoing}, have raised some concerns about such an
overt partisan leading an agency that is supposed to be above politics.

But among both Democrats and Republicans in Congress, Mr. Pompeo, a
former Army tank officer who graduated first in his class from West
Point, is widely seen as smart and thorough --- and a professional who
is capable of rising above the political fray.

``He really enjoys this type of work, so I think he's going to
flourish,'' said Representative Devin Nunes, a Republican from
California. ``He's a military guy, an academy graduate, and this is what
he's been working for his whole career.''

At the C.I.A., there was a sense that Mr. Pompeo's nomination signaled
an end to Mr. Trump's campaign-trail dismissals of accusations about
Russian meddling, and a readiness to start taking intelligence
seriously. A new administration often brings an infusion of energy and
ideas, and most at the agency are eager to get to work under Mr. Pompeo,
according to a current C.I.A. official who spoke on the condition of
anonymity because they could not be quoted by name.

But there are already suggestions from some corners of the Trump camp
about a need to reorganize the intelligence community, which has stoked
concerns at the C.I.A. of a ``hostile takeover'' by leaders who want
political cheerleading, not cleareyed analysis, Mr. Lowenthal said.

How Mr. Pompeo gets along with Lt. Gen. Michael T. Flynn, Mr. Trump's
choice for national security adviser, is likely to prove crucial. Mr.
Flynn, a retired intelligence officer and a former director of the
Defense Intelligence Agency, is a
\href{https://www.nytimes.com/2016/12/12/us/politics/donald-trump-cia-michael-flynn.html}{harsh
judge} of the C.I.A., which he says was overly politicized by the Obama
administration.

That view is not widely shared by Republicans or Democrats in
Washington. But it appears to have been internalized by Mr. Trump.

Still, Mr. Pompeo's hawkish views on a range of other issues are likely
to be welcomed at the agency. He has argued for Congress to permit
domestic surveillance on
\href{http://www.wsj.com/articles/time-for-a-rigorous-national-debate-about-surveillance-1451856106}{a
huge scale}, says waterboarding is legal and does not constitute
torture, and views Russia as the biggest threat facing the United
States.

``I think it's safe to say that Mr. Pompeo is very skeptical of Vladimir
Putin,'' Mr. Nunes said. ``I don't think you can get any more concerned
about Putin's advancement'' than Mr. Pompeo.

Democrats expect to hit the Russia issue hard in an attempt to draw out
differences between Mr. Pompeo and Mr. Trump. But they are also looking
for Mr. Pompeo to take clear stances on controversial issues where he
has made statements that put him in line with the president-elect, like
waterboarding and domestic surveillance.

Senator Ron Wyden, Democrat of Oregon, said he expected Mr. Pompeo to
try to avoid being pinned down, especially on matters on which he may
disagree with Mr. Trump, by saying that the director of the C.I.A. does
not set policy but only executes it. Mr. Pompeo took that tack in his
written responses to questions from some senators on the intelligence
committee, Senator Wyden said.

``I'm going to be respectfully saying that's a lot of baloney,'' he
added.

As director of the C.I.A., Mr. Pompeo would ``have an enormous effect on
surveillance and torture and Russia,'' Senator Wyden said. ``The
American people want policies that are going to better produce security
and liberty.''

The Trump team is ``advancing ideas that give us less of both.''

Advertisement

\protect\hyperlink{after-bottom}{Continue reading the main story}

\hypertarget{site-index}{%
\subsection{Site Index}\label{site-index}}

\hypertarget{site-information-navigation}{%
\subsection{Site Information
Navigation}\label{site-information-navigation}}

\begin{itemize}
\tightlist
\item
  \href{https://help.nytimes.com/hc/en-us/articles/115014792127-Copyright-notice}{©~2020~The
  New York Times Company}
\end{itemize}

\begin{itemize}
\tightlist
\item
  \href{https://www.nytco.com/}{NYTCo}
\item
  \href{https://help.nytimes.com/hc/en-us/articles/115015385887-Contact-Us}{Contact
  Us}
\item
  \href{https://www.nytco.com/careers/}{Work with us}
\item
  \href{https://nytmediakit.com/}{Advertise}
\item
  \href{http://www.tbrandstudio.com/}{T Brand Studio}
\item
  \href{https://www.nytimes.com/privacy/cookie-policy\#how-do-i-manage-trackers}{Your
  Ad Choices}
\item
  \href{https://www.nytimes.com/privacy}{Privacy}
\item
  \href{https://help.nytimes.com/hc/en-us/articles/115014893428-Terms-of-service}{Terms
  of Service}
\item
  \href{https://help.nytimes.com/hc/en-us/articles/115014893968-Terms-of-sale}{Terms
  of Sale}
\item
  \href{https://spiderbites.nytimes.com}{Site Map}
\item
  \href{https://help.nytimes.com/hc/en-us}{Help}
\item
  \href{https://www.nytimes.com/subscription?campaignId=37WXW}{Subscriptions}
\end{itemize}
