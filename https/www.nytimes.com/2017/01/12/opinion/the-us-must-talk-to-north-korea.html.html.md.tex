Sections

SEARCH

\protect\hyperlink{site-content}{Skip to
content}\protect\hyperlink{site-index}{Skip to site index}

\href{https://myaccount.nytimes.com/auth/login?response_type=cookie\&client_id=vi}{}

\href{https://www.nytimes.com/section/todayspaper}{Today's Paper}

\href{/section/opinion}{Opinion}\textbar{}The U.S. Must Talk to North
Korea

\url{https://nyti.ms/2ioljfb}

\begin{itemize}
\item
\item
\item
\item
\item
\end{itemize}

Advertisement

\protect\hyperlink{after-top}{Continue reading the main story}

Supported by

\protect\hyperlink{after-sponsor}{Continue reading the main story}

\href{/section/opinion}{Opinion}

Op-Ed Contributor

\hypertarget{the-us-must-talk-to-north-korea}{%
\section{The U.S. Must Talk to North
Korea}\label{the-us-must-talk-to-north-korea}}

By Siegfried S. Hecker

\begin{itemize}
\item
  Jan. 12, 2017
\item
  \begin{itemize}
  \item
  \item
  \item
  \item
  \item
  \end{itemize}
\end{itemize}

\includegraphics{https://static01.nyt.com/images/2017/01/13/world/12Hecker-inyt-1/12Hecker-inyt-1-articleInline.jpg?quality=75\&auto=webp\&disable=upscale}

STANFORD, Calif. --- Since my first visit to North Korea's Yongbyon
nuclear complex in 2004, I have witnessed the country's nuclear weapons
program grow from a handful of primitive bombs to a formidable nuclear
arsenal that represents one of America's greatest security threats.
After decades of broken policies toward Pyongyang, talking to the North
Koreans is the best option for the Trump administration at this late
date to limit the growing threat.

North Korea broke out to build the bomb because President George W. Bush
was determined to kill President Bill Clinton's 1994 ``Agreed
Framework,'' a bilateral agreement with the North to freeze and
eventually dismantle the North's nuclear program. Hard-liners in the
Bush administration viewed it as appeasement. Mr. Bush labeled the
North, along with Iran and Iraq, part of an ``axis of evil'' in January
2002.

At the first bilateral meeting with Kim Jong-il's regime in Pyongyang in
October 2002, Bush administration officials accused North Korea of
violating the Clinton pact by clandestinely pursuing the uranium path to
the bomb. Washington had already detected this effort in the late 1990s,
but it was deemed an insufficient threat not worthy of jeopardizing the
gains made by the plutonium freeze.

For the Bush administration, the clandestine uranium effort was all it
needed to walk away from the Agreed Framework. Yet Mr. Bush's team
proved unprepared for the consequences and stood by as North Korea
resumed its plutonium program and built the bomb.

During six visits between 2004 and 2009, I watched the North continue to
try to engage Washington, while the Bush administration preferred the
six-party talks led by China, believing that the North would have
greater difficulty cheating in the context of multilateral diplomacy. In
a 2004 visit, I was even allowed to hold a piece of plutonium --- in a
sealed glass jar --- to convince me and Washington that North Korea had
the bomb.

In September 2005, China orchestrated a six-party joint statement
calling for a nuclear-weapon-free Korean Peninsula. When the Bush
administration concurrently slapped financial sanctions on Pyongyang,
the North Koreans walked out of the six-party talks and responded with
their first nuclear test in October 2006.

I was in Pyongyang three weeks later and found that although the test
was only partly successful, it marked a turning point in the North's
nuclear program. North Korea became a nuclear weapon state and insisted
that all future negotiations proceed from that reality. Mr. Bush left
office with the North most likely possessing up to five plutonium-fueled
nuclear weapons and an expanding uranium program.

North Korea greeted the Obama administration with a long-range rocket
launch, followed by a second nuclear test in May 2009 --- this one,
successful. Unlike the Bush administration, which faced the prospect of
the North's violating the Nuclear Nonproliferation Treaty, the Obama
administration faced the North's steady march to an expanding arsenal.

Mr. Obama was also unwilling to engage directly with Pyongyang,
insisting instead that the North denuclearize before starting talks. It
appears the Obama administration also viewed the regimes of Kim Jong-il
and his son and successor, Kim Jong-un, as repugnant and hoped for their
collapse, while also staying in step with two conservative South Korean
administrations. Mr. Obama's preferred path has been to tighten United
Nations and United States sanctions and to pressure Beijing to reign in
Pyongyang. Neither strategy has stopped the Kim regime from expanding
its nuclear program.

Pyongyang upped the ante on its nuclear program with a remarkable
revelation during my seventh and last visit in November 2010: the
existence of a modern uranium centrifuge facility in Yongbyon. That
facility served notice that the North was now capable of pursuing the
second path to the bomb. No outsiders are known to have been in Yongbyon
since my 2010 visit.

Satellite imagery of the Yongbyon complex combined with official North
Korean propaganda photos and three additional successful nuclear tests
point to a robust and rapidly expanding nuclear arsenal. My best
estimate, admittedly highly uncertain, is that North Korea has
sufficient plutonium and highly enriched uranium to build 20 to 25
nuclear weapons.

The North also launched some two dozen missiles in 2016, including
partly successful road-mobile and submarine-based missiles that could
potentially carry nuclear warheads.

President-elect Donald J. Trump faces a much graver threat from the
North than his two predecessors. Pyongyang can most likely already reach
all of South Korea, Japan and possibly even some United States targets
in the Pacific.

The crisis is here. The nuclear clock keeps ticking. Every six to seven
weeks North Korea may be able to add another nuclear weapon to its
arsenal. All in the hands of Kim Jong-un, a young leader about whom we
know little, and a military about which we know less. Both are
potentially prone to overconfidence and miscalculations.

These sensitive nuclear issues require focused discussions in a small,
closed setting. This cannot be achieved at a multilateral negotiating
table, such as the six-party talks.

Mr. Trump should send a presidential envoy to North Korea. Talking is
not a reward or a concession to Pyongyang and should not be construed as
signaling acceptance of a nuclear-armed North Korea. Talking is a
necessary step to re-establishing critical links of communication to
avoid a nuclear catastrophe.

Mr. Trump has little to lose by talking. He can risk the domestic
political downside of appearing to appease the North. He would most
likely get China's support, which is crucial because Beijing prefers
talking to more sanctions. He would also probably get support for
bilateral talks from Seoul, Tokyo and Moscow.

By talking, and especially by listening, the Trump administration may
learn more about the North's security concerns. It would allow
Washington to signal the strength of its resolve to protect its allies
and express its concerns about human rights abuses, as well as to
demonstrate its openness to pragmatic, balanced progress.

Talking will help inform a better negotiating strategy that may
eventually convince the young leader that his country and his regime are
better off without nuclear weapons.

Advertisement

\protect\hyperlink{after-bottom}{Continue reading the main story}

\hypertarget{site-index}{%
\subsection{Site Index}\label{site-index}}

\hypertarget{site-information-navigation}{%
\subsection{Site Information
Navigation}\label{site-information-navigation}}

\begin{itemize}
\tightlist
\item
  \href{https://help.nytimes.com/hc/en-us/articles/115014792127-Copyright-notice}{©~2020~The
  New York Times Company}
\end{itemize}

\begin{itemize}
\tightlist
\item
  \href{https://www.nytco.com/}{NYTCo}
\item
  \href{https://help.nytimes.com/hc/en-us/articles/115015385887-Contact-Us}{Contact
  Us}
\item
  \href{https://www.nytco.com/careers/}{Work with us}
\item
  \href{https://nytmediakit.com/}{Advertise}
\item
  \href{http://www.tbrandstudio.com/}{T Brand Studio}
\item
  \href{https://www.nytimes.com/privacy/cookie-policy\#how-do-i-manage-trackers}{Your
  Ad Choices}
\item
  \href{https://www.nytimes.com/privacy}{Privacy}
\item
  \href{https://help.nytimes.com/hc/en-us/articles/115014893428-Terms-of-service}{Terms
  of Service}
\item
  \href{https://help.nytimes.com/hc/en-us/articles/115014893968-Terms-of-sale}{Terms
  of Sale}
\item
  \href{https://spiderbites.nytimes.com}{Site Map}
\item
  \href{https://help.nytimes.com/hc/en-us}{Help}
\item
  \href{https://www.nytimes.com/subscription?campaignId=37WXW}{Subscriptions}
\end{itemize}
