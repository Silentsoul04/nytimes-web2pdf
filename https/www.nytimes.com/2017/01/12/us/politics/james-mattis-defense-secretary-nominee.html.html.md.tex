Sections

SEARCH

\protect\hyperlink{site-content}{Skip to
content}\protect\hyperlink{site-index}{Skip to site index}

\href{https://www.nytimes.com/section/politics}{Politics}

\href{https://myaccount.nytimes.com/auth/login?response_type=cookie\&client_id=vi}{}

\href{https://www.nytimes.com/section/todayspaper}{Today's Paper}

\href{/section/politics}{Politics}\textbar{}James Mattis Strikes Far
Harsher Tone Than Trump on Russia

\url{https://nyti.ms/2jzMe5g}

\begin{itemize}
\item
\item
\item
\item
\item
\end{itemize}

Advertisement

\protect\hyperlink{after-top}{Continue reading the main story}

Supported by

\protect\hyperlink{after-sponsor}{Continue reading the main story}

\hypertarget{james-mattis-strikes-far-harsher-tone-than-trump-on-russia}{%
\section{James Mattis Strikes Far Harsher Tone Than Trump on
Russia}\label{james-mattis-strikes-far-harsher-tone-than-trump-on-russia}}

\includegraphics{https://static01.nyt.com/images/2017/01/13/us/13mattis/13mattis-videoSixteenByNine3000-v3.jpg}

By \href{http://www.nytimes.com/by/michael-r-gordon}{Michael R. Gordon}
and \href{http://www.nytimes.com/by/helene-cooper}{Helene Cooper}

\begin{itemize}
\item
  Jan. 12, 2017
\item
  \begin{itemize}
  \item
  \item
  \item
  \item
  \item
  \end{itemize}
\end{itemize}

WASHINGTON --- James N. Mattis, the retired Marine Corps general, told
Congress on Thursday that President Vladimir V. Putin of Russia was
trying to ``break the North Atlantic alliance,'' staking out a tougher
stance on Russia during a confirmation hearing for defense secretary
than his prospective commander in chief did on the campaign trail.

``I'm all for engagement, but we also have to recognize reality,''
General Mattis told the Senate Armed Services Committee. ``There's a
decreasing number of areas where we can engage cooperatively and
increasing numbers of areas where we're going to have to confront
Russia.''

In a three-hour hearing, General Mattis argued for expanding the armed
forces, improving the military's readiness to go into battle on short
notice and reinvigorating the North Atlantic Treaty Organization by
maintaining a permanent armed American presence in Baltic nations to
deter a Russian attack, among other steps.

``My view is that nations with allies thrive and nations without allies
don't,'' General Mattis said.

Those remarks were a striking contrast to
\href{https://www.nytimes.com/2016/07/21/us/politics/donald-trump-issues.html}{comments
by President-elect Donald J. Trump} during the campaign that American
support for the alliance should be conditional on financial
contributions by its members.

After the hearing, General Mattis took a major step toward confirmation
when the Republican-led Senate decisively approved a waiver that would
permit him to become defense secretary, by a vote of 81 to 17.
Separately, the House Armed Services Committee narrowly backed a waiver
for him, 34 to 28 along party lines.

\href{https://www.nytimes.com/2017/01/10/us/politics/james-mattis-defense-secretary.html}{Military
officers are barred} by law from serving as defense secretary unless
they have been retired for seven years; General Mattis left active duty
in May 2013. His supporters hope that Mr. Trump will sign legislation
enabling him to begin service as the Pentagon chief on the new
president's first day in office so that his nomination and confirmation
can quickly follow.

In his testimony on Thursday, General Mattis won over most of the Senate
panel's Democrats as they seemed to invest in him their hopes that he
will rein in some of Mr. Trump's more impetuous national security
impulses. In return, General Mattis gave them plenty of reason for hope.

In addition to his remarks skeptical of Russia and supportive of NATO,
General Mattis tacked to the left of President-elect Trump and most of
the Republicans in Congress on whether to keep the agreement
constraining Iran's nuclear program. Mr. Trump has said that his top
priority is to ``dismantle the disastrous deal with Iran.'' But General
Mattis urged the United States to take steps to rigorously enforce it.

``I think it is an imperfect arms control agreement --- it's not a
friendship treaty,'' he said. ``But when America gives her word, we have
to live up to it and work with our allies.''

General Mattis also said he had no intention of revisiting Obama
administration decisions on social issues at the Defense Department,
including the
\href{https://www.nytimes.com/2015/12/04/us/politics/combat-military-women-ash-carter.html}{opening
of combat roles to women} or
\href{http://www.nytimes.com/2010/12/23/us/politics/23military.html}{allowing
openly gay men and women} to serve.

``I believe that military service is a touchstone for patriots of
whatever stripe,'' he said. ``I've never cared much about two consenting
adults and who they go to bed with.''

Trying to reassure lawmakers that he would not be too quick to urge the
use of military force, General Mattis said the nickname he had been
given, ``Mad Dog,'' was an invention of the news media, ignoring the
fact that Mr. Trump seemed to use it at every opportunity. The general's
battlefield call sign was ``Chaos.''

But it was also clear that General Mattis favors a more assertive
response to Iran, which he described in a written submission to the
Senate committee as the ``biggest destabilizing force in the Middle
East.''

General Mattis did not say how many American troops should be kept in
Iraq, but he asserted that the United States needed to maintain its
influence there long after
\href{https://www.nytimes.com/2016/12/18/world/middleeast/iraq-mosul-islamic-state.html}{Mosul
is retaken from the Islamic State}, also known as ISIS or ISIL, to
ensure that Iraq ``does not become a rump state of the regime in
Tehran.''

The United States strategy for taking Raqqa, Syria, the capital of the
Islamic State, he said, ``needs to be reviewed and perhaps energized on
a more aggressive timeline.''

Some notable Democrats were persuaded by General Mattis's independent
streak, fluency with military issues and repeated promises to uphold the
principle of civilian control and consult Congress.

Senators Jack Reed of Rhode Island, the ranking Democrat on the Senate
Armed Services Committee, and Tim Kaine of Virginia, the former
Democratic candidate for vice president and a committee member, were
among the lawmakers who voted for a waiver.

Three Democrats on the committee voted against amending the law:
Senators Kirsten E. Gillibrand of New York, Richard Blumenthal of
Connecticut and Elizabeth Warren of Massachusetts.

Ms. Warren's vote was somewhat surprising as she devoted her question
time to urging General Mattis to consider channeling more military
spending to research laboratories in Massachusetts and thanking him for
his assurances that he would provide the new president with his candid
advice.

``Under what circumstances will you advocate for your views forcefully
and frankly?'' Ms. Warren asked.

``On every circumstance, senator,'' he replied.

``I am very glad to hear that,'' she said. ``Thank you.''

Advertisement

\protect\hyperlink{after-bottom}{Continue reading the main story}

\hypertarget{site-index}{%
\subsection{Site Index}\label{site-index}}

\hypertarget{site-information-navigation}{%
\subsection{Site Information
Navigation}\label{site-information-navigation}}

\begin{itemize}
\tightlist
\item
  \href{https://help.nytimes.com/hc/en-us/articles/115014792127-Copyright-notice}{©~2020~The
  New York Times Company}
\end{itemize}

\begin{itemize}
\tightlist
\item
  \href{https://www.nytco.com/}{NYTCo}
\item
  \href{https://help.nytimes.com/hc/en-us/articles/115015385887-Contact-Us}{Contact
  Us}
\item
  \href{https://www.nytco.com/careers/}{Work with us}
\item
  \href{https://nytmediakit.com/}{Advertise}
\item
  \href{http://www.tbrandstudio.com/}{T Brand Studio}
\item
  \href{https://www.nytimes.com/privacy/cookie-policy\#how-do-i-manage-trackers}{Your
  Ad Choices}
\item
  \href{https://www.nytimes.com/privacy}{Privacy}
\item
  \href{https://help.nytimes.com/hc/en-us/articles/115014893428-Terms-of-service}{Terms
  of Service}
\item
  \href{https://help.nytimes.com/hc/en-us/articles/115014893968-Terms-of-sale}{Terms
  of Sale}
\item
  \href{https://spiderbites.nytimes.com}{Site Map}
\item
  \href{https://help.nytimes.com/hc/en-us}{Help}
\item
  \href{https://www.nytimes.com/subscription?campaignId=37WXW}{Subscriptions}
\end{itemize}
