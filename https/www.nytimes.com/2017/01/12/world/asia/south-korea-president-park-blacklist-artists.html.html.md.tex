Sections

SEARCH

\protect\hyperlink{site-content}{Skip to
content}\protect\hyperlink{site-index}{Skip to site index}

\href{https://www.nytimes.com/section/world/asia}{Asia Pacific}

\href{https://myaccount.nytimes.com/auth/login?response_type=cookie\&client_id=vi}{}

\href{https://www.nytimes.com/section/todayspaper}{Today's Paper}

\href{/section/world/asia}{Asia Pacific}\textbar{}South Korea's
Blacklist of Artists Adds to Outrage Over Presidential Scandal

\url{https://nyti.ms/2jzpAdo}

\begin{itemize}
\item
\item
\item
\item
\item
\end{itemize}

Advertisement

\protect\hyperlink{after-top}{Continue reading the main story}

Supported by

\protect\hyperlink{after-sponsor}{Continue reading the main story}

\hypertarget{south-koreas-blacklist-of-artists-adds-to-outrage-over-presidential-scandal}{%
\section{South Korea's Blacklist of Artists Adds to Outrage Over
Presidential
Scandal}\label{south-koreas-blacklist-of-artists-adds-to-outrage-over-presidential-scandal}}

\includegraphics{https://static01.nyt.com/images/2017/01/13/world/13BLACKLIST-1/13BLACKLIST-1-articleLarge.jpg?quality=75\&auto=webp\&disable=upscale}

By \href{http://www.nytimes.com/by/choe-sang-hun}{Choe Sang-Hun}

\begin{itemize}
\item
  Jan. 12, 2017
\item
  \begin{itemize}
  \item
  \item
  \item
  \item
  \item
  \end{itemize}
\end{itemize}

SEOUL, South Korea --- When the South Korean artist Hong Sung-dam
produced a painting that depicted President
\href{https://www.nytimes.com/topic/person/park-geunhye}{Park Geun-hye}
as a scarecrow manipulated by evil forces, including her dictator
father, her senior aides discussed how to ``punish'' Mr. Hong, according
to a diary one of them kept.

Soon after the painting's completion in August 2014, the retaliation
began as planned in the aide's diary, which surfaced in November in the
investigation into the corruption scandal that has led to Ms. Park's
impeachment trial.

First, a pro-government civic group sued Mr. Hong on
\href{https://www.nytimes.com/2014/08/31/world/asia/an-artist-is-rebuked-for-casting-south-koreas-leader-in-an-unflattering-light.html}{charges
of defaming Ms. Park}. Then his work was excluded from the Gwangju
Biennale, South Korea's best-known international arts festival, an act
Gwangju's mayor later admitted was due to government pressure.

The retaliation did not stop there, Mr. Hong said. ``Dozens of
conservative activists showed up in front of my apartment like a goon
squad, shaking my photographs and calling me a `Communist painter,''' he
said. ``I received death threats on the phone.''

As it turned out, Mr. Hong was one of thousands of artists reportedly
blacklisted by the government of Ms. Park, whose powers have been
suspended as she faces
\href{https://www.nytimes.com/2016/12/22/world/asia/south-korea-president-park-impeachment.html}{an
impeachment trial on charges of corruption and abuse of power}. The
blacklist is just one element in the sprawling case that has infuriated
the public and prompted national introspection about South Korea's young
democracy and its authoritarian past.

On Thursday, three of Ms. Park's former aides, including one of her
former culture ministers, Kim Jong-deok, were arrested on charges of
blacklisting cultural figures deemed unfriendly and barring them from
government-controlled support programs.

So far, two versions of the blacklist have been reported by the news
media, citing anonymous sources. Officials, including the special
prosecutor in the case, Park Young-soo, have confirmed the existence of
the blacklist but have not released it.

A 2015 version of the list included more than 9,000 people, according to
news reports. The list contained some of South Korea's most beloved
filmmakers, actors and writers, including the director of ``Oldboy,''
\href{http://www.imdb.com/title/tt0364569/}{Park Chan-wook}, and the
``Snowpiercer'' actor
\href{http://www.imdb.com/title/tt1706620/?ref_=nv_sr_1}{Song Kang-ho}.

Officially, Ms. Park has made promoting movies and other cultural
products one of her key priorities. But secretly, her government has
blackballed artists, reviving a practice of past military dictators like
her father, Park Chung-hee, and in so doing has ``seriously undermined
the freedom of thought and expression,'' the special prosecutor's office
said.

The revelations about the cultural blacklist added a new layer of
notoriety to the scandal surrounding Ms. Park, and prosecutors planned
to use the list to help strengthen the impeachment charges against her.

\includegraphics{https://static01.nyt.com/images/2017/01/13/world/13BLACKLIST-2/13BLACKLIST-2-articleLarge.jpg?quality=75\&auto=webp\&disable=upscale}

When the National Assembly
\href{https://www.nytimes.com/2016/12/09/world/asia/south-korea-president-park-geun-hye-impeached.html}{voted
to impeach} Ms. Park last month, it accused her of conspiring with her
longtime confidante, Choi Soon-sil, to solicit bribes from businesses
and crack down on uncooperative officials and journalists.

The special prosecutor is investigating whether Ms. Park and Kim
Ki-choon --- her former chief of staff, who was depicted as one of the
dark forces in Mr. Hong's painting --- were involved in the blacklisting
of artists.

Both Ms. Park and Mr. Kim, her former chief of staff, have denied
involvement. However, another of Ms. Park's former culture ministers,
Yoo Jin-ryong, said the list was dictated by the president's office.

On Monday, the current culture minister, Cho Yoon-sun, said, ``I
understand how pained artists must have felt when excluded from
government support just because of their political and ideological
beliefs.''

For many South Koreans, news of the blacklisting of artists reawakened
memories from the nation's dictatorial past.

Ms. Park's father, who ruled South Korea from 1961 to 1979, censored
newspapers and imprisoned dissident writers and publishers. Chun
Doo-hwan, a military dictator during the 1980s, banished a comedian from
TV after people compared the appearances of the two men. (Both were
bald.) Subsequent governments were accused of favoring pro-government
scholars and civic groups when doling out research projects and
subsidies.

Under Ms. Park's conservative predecessor, Lee Myung-bak, some
celebrities and journalists deemed progressive were barred from
state-controlled broadcasters.

But the latest revelations marked the first time the existence of an
extensive government blacklist was revealed since South Korea moved
toward democracy in the late 1980s.

``It's an honor to be on the list,'' Ko Un, one of South Korea's
best-known poets, told the broadcaster SBS last month, when it reported
on another version of the list. ``This shows how disgusting the
government is.''

Under Ms. Park, whose leadership style is
\href{http://www.nytimes.com/2016/11/12/world/asia/south-korea-park-geun-hye.html}{often
compared to her father's}, rumors of a blacklist have been circulating
for years.

The rumors intensified after two award-winning theatrical directors were
mysteriously booted from government subsidy programs: one had campaigned
for Ms. Park's main opponent in the 2012 election; another had produced
a play spoofing Ms. Park and her father.

And after the organizers of the Busan International Film Festival
screened a documentary that delved into what it called Ms. Park's
botched response to the
\href{https://www.nytimes.com/2014/04/30/world/asia/south-korea-ferry-disaster.html}{Sewol
ferry disaster} in 2014 in which more than 300 people died, the festival
lost half of its government funding.

Image

For many South Koreans, reports of the blacklisting of artists have
reawakened memories from the nation's dictatorial past. Park Chung-hee,
Ms. Park's father, who ruled South Korea from 1961 to 1979, censored
newspapers and imprisoned dissident writers and
publishers.Credit...Keystone, via Getty Images

Mr. Yoo, the former culture minister, said Mr. Kim, Ms. Park's chief of
staff at the time, began ordering the culture ministry to blacklist
certain artists in 2013. Last month, a former aide to Ms. Park was
indicted on charges of colluding with her in an attempt to blackmail a
vice chairwoman of CJ, which runs South Korea's biggest film studio,
into retiring in 2013. The company had angered Ms. Park's office by
financing a movie about her ideological enemy, the former President Roh
Moo-hyun, Mr. Yoo said in a radio interview last month.

``I thought this kind of thing happened only under the past military
rule,'' CJ's chairman, Sohn Kyung-shik, told a parliamentary hearing
last month.

Mr. Yoo said that an early version of the blacklist he saw in June 2014
included hundreds of artists. Shortly before he was replaced a month
later, Mr. Yoo said he met Ms. Park to warn against the list. (Ms. Park
has denied being warned.)

By 2015, the list had ballooned to include more than 9,000 visual
artists, musicians, actors, film and musical directors, and writers
deemed critical of Ms. Park, particularly those who took aim at her
handling of the ferry disaster or who were suspected of supporting her
rivals, according to the Hankook Ilbo newspaper, which published what it
claimed was the list in October.

Ms. Park's office zealously pursued her opponents after the ferry
disaster, according to the diary of Kim Young-han, the former
presidential aide who detailed the retaliation against Mr. Hong. The
ferry tragedy is a central motif in Mr. Hong's painting.

During a meeting of senior presidential aides in 2014, Mr. Kim, Ms.
Park's chief of staff at the time, called for a ``combative response to
leftists in the cultural and art circles'' and ordered the aides to
``discover their networks,'' according to the diary. He compared
progressive teachers and journalists ``to poisonous mushrooms.'' The
diary also recorded instructions to punish artists who satirized Ms.
Park, conduct a ``loyalty check'' of senior government officials,
``intimidate'' courts of law and ``induce'' scholars to write
pro-government newspaper columns.

``Make them afraid to challenge the president,'' Mr. Kim was quoted as
saying in a diary entry dated July 4, 2014.

He has denied giving such an order and said that the diary did not
faithfully record what was actually discussed during the aides'
meetings.

The author of the diary, Kim Young-han, died in August, but prosecutors
said they found it useful in building their case against Ms. Park.

Mr. Hong, the painter, said that for poor artists, being cut off from
travel and other government support programs could be crushing.

In 2015, Mr. Hong was invited to show his painting at a Berlin arts
festival. But no domestic logistics company would transport the work for
fear of government retaliation. Mr. Hong had to travel alone and
hurriedly repaint a copy of the original in Berlin. He also suspected
the government was behind a tax audit of his wife's clinic last year.

``It makes me shudder that Park Geun-hye and her cronies tried to tame
artists by holding back a pittance of government support while they
themselves pocket millions,'' Mr. Hong said. ``They showed how depraved
political power can be.''

Advertisement

\protect\hyperlink{after-bottom}{Continue reading the main story}

\hypertarget{site-index}{%
\subsection{Site Index}\label{site-index}}

\hypertarget{site-information-navigation}{%
\subsection{Site Information
Navigation}\label{site-information-navigation}}

\begin{itemize}
\tightlist
\item
  \href{https://help.nytimes.com/hc/en-us/articles/115014792127-Copyright-notice}{©~2020~The
  New York Times Company}
\end{itemize}

\begin{itemize}
\tightlist
\item
  \href{https://www.nytco.com/}{NYTCo}
\item
  \href{https://help.nytimes.com/hc/en-us/articles/115015385887-Contact-Us}{Contact
  Us}
\item
  \href{https://www.nytco.com/careers/}{Work with us}
\item
  \href{https://nytmediakit.com/}{Advertise}
\item
  \href{http://www.tbrandstudio.com/}{T Brand Studio}
\item
  \href{https://www.nytimes.com/privacy/cookie-policy\#how-do-i-manage-trackers}{Your
  Ad Choices}
\item
  \href{https://www.nytimes.com/privacy}{Privacy}
\item
  \href{https://help.nytimes.com/hc/en-us/articles/115014893428-Terms-of-service}{Terms
  of Service}
\item
  \href{https://help.nytimes.com/hc/en-us/articles/115014893968-Terms-of-sale}{Terms
  of Sale}
\item
  \href{https://spiderbites.nytimes.com}{Site Map}
\item
  \href{https://help.nytimes.com/hc/en-us}{Help}
\item
  \href{https://www.nytimes.com/subscription?campaignId=37WXW}{Subscriptions}
\end{itemize}
