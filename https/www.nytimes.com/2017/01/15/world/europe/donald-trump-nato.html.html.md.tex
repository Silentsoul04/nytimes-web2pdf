Sections

SEARCH

\protect\hyperlink{site-content}{Skip to
content}\protect\hyperlink{site-index}{Skip to site index}

\href{https://www.nytimes.com/section/world/europe}{Europe}

\href{https://myaccount.nytimes.com/auth/login?response_type=cookie\&client_id=vi}{}

\href{https://www.nytimes.com/section/todayspaper}{Today's Paper}

\href{/section/world/europe}{Europe}\textbar{}Trump Criticizes NATO and
Hopes for `Good Deals' With Russia

\url{https://nyti.ms/2iBloMA}

\begin{itemize}
\item
\item
\item
\item
\item
\end{itemize}

Advertisement

\protect\hyperlink{after-top}{Continue reading the main story}

Supported by

\protect\hyperlink{after-sponsor}{Continue reading the main story}

\hypertarget{trump-criticizes-nato-and-hopes-for-good-deals-with-russia}{%
\section{Trump Criticizes NATO and Hopes for `Good Deals' With
Russia}\label{trump-criticizes-nato-and-hopes-for-good-deals-with-russia}}

\includegraphics{https://static01.nyt.com/images/2017/01/17/world/16EUROPE-1/17EUROPE-1-articleLarge.jpg?quality=75\&auto=webp\&disable=upscale}

By \href{http://www.nytimes.com/by/michael-r-gordon}{Michael R. Gordon}
and \href{http://www.nytimes.com/by/niraj-chokshi}{Niraj Chokshi}

\begin{itemize}
\item
  Jan. 15, 2017
\item
  \begin{itemize}
  \item
  \item
  \item
  \item
  \item
  \end{itemize}
\end{itemize}

WASHINGTON --- In comments that are likely to create fresh tensions with
the United States' closest European allies, President-elect Donald J.
Trump described NATO as ``obsolete''
\href{http://www.thetimes.co.uk/edition/news/i-ll-do-a-deal-with-britain-6hl2hl73l}{in
an interview published on Sunday} and said other European nations would
probably follow Britain's lead by leaving the European Union.

Mr. Trump has made similar comments before. But the fact that he made
them in a joint interview with two European publications --- The Times
of London and Bild, a German newspaper --- and did so days before
assuming the presidency alarmed European diplomats.

``I took such heat when I said NATO was obsolete,'' Mr. Trump said.
``It's obsolete because it wasn't taking care of terror. I took a lot of
heat for two days. And then they started saying, `Trump is right.'''

Mr. Trump also said that
\href{http://www.thetimes.co.uk/edition/news/brexit-will-be-a-great-thing-you-were-so-smart-to-get-out-09gp9z357}{Britain's
decision to leave the European Union} would ``end up being a great
thing'' and predicted that other countries would follow. ``People,
countries want their own identity, and the U.K. wanted its own
identity,'' he said.

He criticized Chancellor Angela Merkel of Germany over her decision to
welcome more than one million migrants.

``I think she made one very catastrophic mistake, and that was taking
all of these illegals, you know, taking all of the people from wherever
they come from,'' he said. ``And nobody even knows where they come
from.''

On Russia, Mr. Trump said he hoped to strike a deal with President
Vladimir V. Putin to reduce nuclear weapons stockpiles. He suggested
that such an agreement could be part of a broader easing of tensions
that would include lifting economic sanctions imposed after Russia
seized Crimea from Ukraine in 2014.

``They have sanctions on Russia --- let's see if we can make some good
deals with Russia,'' he said. ``For one thing, I think nuclear weapons
should be way down and reduced very substantially, that's part of it.
Russia's hurting very badly right now because of sanctions, but I think
something can happen that a lot of people are going to benefit.''

Mr. Trump was critical of Russia's military intervention in Syria,
including airstrikes in Aleppo that American officials say have hit
hospitals and killed civilians, saying it had led to a ``terrible
humanitarian situation.''

Strikingly, however, Mr. Trump painted Ms. Merkel, the leader of a
staunch American ally, and Mr. Putin, the president of a country who has
often had adversarial relations with Washington, with the same brush. He
described them as leaders he would trust at the beginning of his
presidency, but noted that this could quickly change.

``Well, I start off trusting both --- but let's see how long that
lasts,'' he said. ``It may not last long at all.''

During his hourlong interview with the European publications at Trump
Tower in Manhattan, Mr. Trump sought to temper some of his criticism of
NATO by noting that the alliance ``is very important to me.'' Still, his
characterization of it as divorced from the fight against terrorism was
challenged by NATO experts, who noted that the alliance had joined the
United States in Afghanistan.

``After 9/11, NATO's main vocation became fighting terrorism in
Afghanistan,'' said Alexander Vershbow, the former deputy secretary
general of NATO. ``It is now heavily engaged in training the militaries
of many Middle Eastern countries to help them fight terrorism in their
own backyard.''

Mr. Trump's comments on NATO were also striking because his nominee for
secretary of defense, James N. Mattis, a retired Marine general,
described the alliance as essential for Americans' security
\href{https://www.nytimes.com/2017/01/12/us/politics/james-mattis-defense-secretary-nominee.html}{in
testimony to Congress} just three days ago.

Asked about Mr. Trump's previous criticism of NATO, General Mattis said
he had spoken with him about the value of the alliance.

``I have had discussions with him on this issue,'' General Mattis said.
``He has shown himself open, even to the point of asking more questions,
going deeper into the issue.''

In the interview, Mr. Trump also expressed an eagerness to reach a fair
trade deal with Britain ``very quickly,'' saying he planned to invite
Prime Minister Theresa May to visit him right after he takes office.

Ms. May had reached out to Mr. Trump just after Christmas with a gift: a
copy of an address Winston Churchill gave to the American people in
1941. In a letter, she told Mr. Trump that she hoped the sentiment of
``unity'' in the speech remained ``just as true today as it has ever
been.''

Others in Europe, however, were less hopeful. Two European officials,
who asked not to be identified because they did not want to add strain
to the United States' relations with Europe, expressed frustration with
Mr. Trump's remarks.

The diplomats said they had heard him sound off during the campaign. But
with the inauguration less than a week away, there is a growing
realization in European capitals that Mr. Trump's acerbic criticism of
NATO and the European Union was not just an attempt to win votes.

Advertisement

\protect\hyperlink{after-bottom}{Continue reading the main story}

\hypertarget{site-index}{%
\subsection{Site Index}\label{site-index}}

\hypertarget{site-information-navigation}{%
\subsection{Site Information
Navigation}\label{site-information-navigation}}

\begin{itemize}
\tightlist
\item
  \href{https://help.nytimes.com/hc/en-us/articles/115014792127-Copyright-notice}{©~2020~The
  New York Times Company}
\end{itemize}

\begin{itemize}
\tightlist
\item
  \href{https://www.nytco.com/}{NYTCo}
\item
  \href{https://help.nytimes.com/hc/en-us/articles/115015385887-Contact-Us}{Contact
  Us}
\item
  \href{https://www.nytco.com/careers/}{Work with us}
\item
  \href{https://nytmediakit.com/}{Advertise}
\item
  \href{http://www.tbrandstudio.com/}{T Brand Studio}
\item
  \href{https://www.nytimes.com/privacy/cookie-policy\#how-do-i-manage-trackers}{Your
  Ad Choices}
\item
  \href{https://www.nytimes.com/privacy}{Privacy}
\item
  \href{https://help.nytimes.com/hc/en-us/articles/115014893428-Terms-of-service}{Terms
  of Service}
\item
  \href{https://help.nytimes.com/hc/en-us/articles/115014893968-Terms-of-sale}{Terms
  of Sale}
\item
  \href{https://spiderbites.nytimes.com}{Site Map}
\item
  \href{https://help.nytimes.com/hc/en-us}{Help}
\item
  \href{https://www.nytimes.com/subscription?campaignId=37WXW}{Subscriptions}
\end{itemize}
