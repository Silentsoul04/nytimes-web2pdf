Sections

SEARCH

\protect\hyperlink{site-content}{Skip to
content}\protect\hyperlink{site-index}{Skip to site index}

\href{https://www.nytimes.com/section/world/middleeast}{Middle East}

\href{https://myaccount.nytimes.com/auth/login?response_type=cookie\&client_id=vi}{}

\href{https://www.nytimes.com/section/todayspaper}{Today's Paper}

\href{/section/world/middleeast}{Middle East}\textbar{}World Leaders
Push Israel and Trump to Forge a 2-State Deal

\url{https://nyti.ms/2izELFR}

\begin{itemize}
\item
\item
\item
\item
\item
\end{itemize}

Advertisement

\protect\hyperlink{after-top}{Continue reading the main story}

Supported by

\protect\hyperlink{after-sponsor}{Continue reading the main story}

\hypertarget{world-leaders-push-israel-and-trump-to-forge-a-2-state-deal}{%
\section{World Leaders Push Israel and Trump to Forge a 2-State
Deal}\label{world-leaders-push-israel-and-trump-to-forge-a-2-state-deal}}

\includegraphics{https://static01.nyt.com/images/2017/01/16/world/16PARIS/16PARIS-videoSixteenByNineJumbo1600.jpg}

By \href{https://www.nytimes.com/by/alissa-j-rubin}{Alissa J. Rubin},
\href{https://www.nytimes.com/by/benoit-morenne}{Benoît Morenne} and
\href{https://www.nytimes.com/by/isabel-kershner}{Isabel Kershner}

\begin{itemize}
\item
  Jan. 15, 2017
\item
  \begin{itemize}
  \item
  \item
  \item
  \item
  \item
  \end{itemize}
\end{itemize}

PARIS --- An effort by France to give impetus to the stalled
Israeli-Palestinian peace process through an international conference
always faced long odds. But with an incoming American president who has
vowed to support Israel no matter what, the project seemed even more
quixotic.

As senior representatives from 70 countries gathered in Paris on Sunday
and endorsed anew a two-state solution to the conflict between Israel
and the Palestinians, what they had to offer was mostly symbolic.

Still, that symbol was clear: Though peace efforts seemed doomed, the
leaders, representing every European nation, were signaling that they
still saw them as critical --- and putting President-elect Donald J.
Trump and Prime Minister Benjamin Netanyahu on notice that the process
could be ignored only at their peril.

``The two-state solution, which the international community has agreed
on for many years, appears threatened,'' President François Hollande of
France said as he opened the afternoon session. ``It is physically
threatened on the ground by the acceleration of settlements, it is
politically threatened by the progressive weakening of the peace camp,
it is morally threatened by the distrust that has accumulated between
the parties, and that has certainly been exploited by extremists.''

France held the
\href{https://www.nytimes.com/2016/06/04/world/europe/france-hosts-talks-on-restarting-mideast-peace-process.html}{first
session of the conference} in June, after the United States and others
effectively gave up. Talks had broken down, and Mr. Netanyahu's
government had drawn international anger with its continued settlement
activity in the West Bank and East Jerusalem, a red line for the process
set out years before by Secretary of State John Kerry and other leaders.

The idea was to prevent the peace process from disappearing entirely
from the international agenda, which was dominated by Syria, Ukraine and
Libya, among other hot spots.

At the end of Sunday's meeting, the countries issued a joint communiqué
that reaffirmed support for a two-state solution --- a Palestinian state
existing next to Israel --- and a return to the 1967 boundaries between
the Israelis and Palestinians, including the removal of Israeli
settlements from the West Bank. The statement referenced United Nations
resolutions to that effect, including the condemnation of Israel in
December, which the Obama administration
\href{https://www.nytimes.com/2016/12/23/world/middleeast/israel-settlements-un-vote.html}{declined
to block}, over its continuing settlement of the West Bank.

Most of these goals now seem starkly remote. Israel has continued to
follow an aggressive policy of creating ``facts on the ground'' through
continued settlement of land claimed by the Palestinians. The
Palestinian side, at the same time, is hobbled by a divided leadership,
with one wing, led by the militant group Hamas, still refusing to
recognize Israel at all.

Although the meeting was called a conference, it was always framed more
as a series of diplomatic meetings, since so many people would be in the
room that any real negotiations seemed unlikely.

However, with the Israelis dismissing the meeting as ``rigged'' and
refusing to send representatives, and with the Palestinians absent as
well, it seemed even shakier than before. Tzipi Hotovely, Israel's
deputy foreign minister,
\href{https://www.nytimes.com/2017/01/15/world/middleeast/missing-at-israel-palestinian-peace-conference-israelis-or-palestinians.html}{said
last week} that the conference was ``like a wedding with neither bride
nor groom.''

The Israelis were reluctant to participate because they want a
negotiation that primarily involves only the two principal parties:
Israel and the Palestinians. The Palestinians have lost faith in
bilateral talks and now prefer that any negotiation go on in an
international forum, where they can have more leverage.

The Palestinians welcomed the conference's final communiqué, and Dr.
Saeb Erekat, the Palestine Liberation Organization's secretary general,
said in an emailed statement, ``It is time to stop dealing with Israel
as a country above the law and hold it accountable for its systematic
violation of human rights and international law.''

In another sense, the meeting was also a last shot by a group of world
leaders and diplomats who have driven the current peace process,
fruitless though it has been, to preserve it in the face of major
changes in the American delegation at the heart of the effort.

With Mr. Trump's inauguration days away, his foreign policy is still
mostly a matter of conjecture. But he has repeatedly signaled his
displeasure with Mr. Obama's approach toward Israel and the peace
process.

Israeli officials clearly expect that the pressure to reach an
accommodation with the Palestinians will ease once Mr. Trump is in
office. Some seem to be counting the hours: After Mr. Trump's election
victory, Naftali Bennett, the leader of the pro-settlement Jewish Home
party in Mr. Netanyahu's coalition, exulted, ``The era of a Palestinian
state is over!''

While Mr. Trump has expressed a desire to make what he called the
``ultimate deal'' between Israel and the Palestinians, his staff has
also reached out to the organization that represents the West Bank
settlers, the Yesha Council. The council received multiple invitations
to Mr. Trump's Jan. 20 inauguration, according to a spokesman for the
umbrella organization. A delegation led by Oded Revivi, the chief
foreign envoy of the council and mayor of a large settlement, will be
attending.

Perhaps the clearest signs of a coming change from the Americans are Mr.
Trump's personnel announcements: David M. Friedman,
\href{https://www.nytimes.com/2016/12/15/us/politics/donald-trump-david-friedman-israel-ambassador.html}{his
nominee} for ambassador to Israel and a bankruptcy lawyer
\href{https://www.nytimes.com/2016/12/16/world/middleeast/david-friedman-us-ambassador-israel.html}{aligned
with Israel's far right}, has been an avid supporter of Jewish
settlements in the occupied West Bank.

In Israel, Mr. Netanyahu asserts that he still supports the principle of
a Palestinian state --- but on his terms, not those of Mahmoud Abbas,
the Palestinian leader, whom many Israelis view as too weak to deliver.
Mr. Netanyahu's terms are unacceptable to the Palestinians, who are now
looking to the United Nations and other international forums for
intervention.

Despite the resigned expressions on the faces of many of the
representatives as they left Sunday's meeting --- and Mr. Netanyahu's
declaration, according to Israeli news outlets, that the conference was
the ``final palpitations'' of yesterday's world --- there has been no
official death knell for a two-state solution. No one has yet put
forward an alternative that seems likely to gain international support
and ultimately offer both Israelis and Palestinians more safety.

Still, there was a sense on Sunday that the conference was the close of
an era of negotiations, centered around the two-state principle, that
was driven by the Obama administration, and Mr. Kerry in particular.

In contrast with
\href{https://www.nytimes.com/2016/12/28/us/politics/john-kerry-israel-palestine-peace.html}{his
previous rebukes} of Mr. Netanyahu's policies, Mr. Kerry's role on
Sunday was more conciliatory.

He worked to soften language in the final communiqué and to reassert the
United States' support for Israel. The United States lobbied for
language condemning acts of ``violence and terror and incitement,'' a
reference to Palestinian attacks inside Israel. And over the weekend,
Mr. Kerry called Mr. Netanyahu to assure him that the United States
would not support a United Nations resolution in the wake of the Paris
meeting.

Advertisement

\protect\hyperlink{after-bottom}{Continue reading the main story}

\hypertarget{site-index}{%
\subsection{Site Index}\label{site-index}}

\hypertarget{site-information-navigation}{%
\subsection{Site Information
Navigation}\label{site-information-navigation}}

\begin{itemize}
\tightlist
\item
  \href{https://help.nytimes.com/hc/en-us/articles/115014792127-Copyright-notice}{©~2020~The
  New York Times Company}
\end{itemize}

\begin{itemize}
\tightlist
\item
  \href{https://www.nytco.com/}{NYTCo}
\item
  \href{https://help.nytimes.com/hc/en-us/articles/115015385887-Contact-Us}{Contact
  Us}
\item
  \href{https://www.nytco.com/careers/}{Work with us}
\item
  \href{https://nytmediakit.com/}{Advertise}
\item
  \href{http://www.tbrandstudio.com/}{T Brand Studio}
\item
  \href{https://www.nytimes.com/privacy/cookie-policy\#how-do-i-manage-trackers}{Your
  Ad Choices}
\item
  \href{https://www.nytimes.com/privacy}{Privacy}
\item
  \href{https://help.nytimes.com/hc/en-us/articles/115014893428-Terms-of-service}{Terms
  of Service}
\item
  \href{https://help.nytimes.com/hc/en-us/articles/115014893968-Terms-of-sale}{Terms
  of Sale}
\item
  \href{https://spiderbites.nytimes.com}{Site Map}
\item
  \href{https://help.nytimes.com/hc/en-us}{Help}
\item
  \href{https://www.nytimes.com/subscription?campaignId=37WXW}{Subscriptions}
\end{itemize}
