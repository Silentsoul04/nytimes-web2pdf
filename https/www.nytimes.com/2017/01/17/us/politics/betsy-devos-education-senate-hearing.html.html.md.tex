Sections

SEARCH

\protect\hyperlink{site-content}{Skip to
content}\protect\hyperlink{site-index}{Skip to site index}

\href{https://www.nytimes.com/section/politics}{Politics}

\href{https://myaccount.nytimes.com/auth/login?response_type=cookie\&client_id=vi}{}

\href{https://www.nytimes.com/section/todayspaper}{Today's Paper}

\href{/section/politics}{Politics}\textbar{}Betsy DeVos's Education
Hearing Erupts Into Partisan Debate

\url{https://nyti.ms/2jWZgKz}

\begin{itemize}
\item
\item
\item
\item
\item
\item
\end{itemize}

Advertisement

\protect\hyperlink{after-top}{Continue reading the main story}

Supported by

\protect\hyperlink{after-sponsor}{Continue reading the main story}

\hypertarget{betsy-devoss-education-hearing-erupts-into-partisan-debate}{%
\section{Betsy DeVos's Education Hearing Erupts Into Partisan
Debate}\label{betsy-devoss-education-hearing-erupts-into-partisan-debate}}

\includegraphics{https://static01.nyt.com/images/2017/01/18/us/18devos/18devos-videoSixteenByNine3000.jpg}

By \href{http://www.nytimes.com/by/kate-zernike}{Kate Zernike} and
\href{http://www.nytimes.com/by/yamiche-alcindor}{Yamiche Alcindor}

\begin{itemize}
\item
  Jan. 17, 2017
\item
  \begin{itemize}
  \item
  \item
  \item
  \item
  \item
  \item
  \end{itemize}
\end{itemize}

WASHINGTON --- At her confirmation hearing on Tuesday to be education
secretary, Betsy DeVos vigorously defended her work steering taxpayer
dollars from traditional public schools, arguing that it was time to
move away from a ``one size fits all'' system and toward newer models
for students from preschool to college.

The hearing quickly became a heated and partisan debate that reflected
the nation's political divide on how best to spend public money in
education.

Republicans applauded Ms. DeVos's work to expand charter schools and
school vouchers, which give families public funds to help pay tuition at
private schools. Democrats criticized her for wanting to ``privatize''
public education and pushed her, unsuccessfully, to support making
public colleges and universities tuition-free.

Ms. DeVos,
\href{https://www.nytimes.com/2017/01/09/us/politics/betsy-devos-education-secretary.html}{a
billionaire} with a complex web of investments, including in companies
that stand to win or lose from federal education policy, was the first
nominee of President-elect Donald J. Trump to have a Senate hearing
without completing an ethics review on how she planned to avoid
conflicts of interest. Democrats pointed out that in the past,
Republicans had insisted that no hearings be conducted before those
reviews were complete.

Senator Lamar Alexander, Republican of Tennessee and chairman of the
Committee on Health, Education, Labor and Pensions, limited the
questioning to one round of five minutes for each senator, prompting
howls from Democrats, who noted that previous hearings had included two
rounds of questions.

\href{https://www.nytimes.com/interactive/2016/12/05/us/politics/trump-cabinet-insiders-outsiders-millionaires.html}{}

\includegraphics{https://static01.nyt.com/images/2016/12/02/us/politics/trump-cabinet-insiders-outsiders-millionaires-1480717606838/trump-cabinet-insiders-outsiders-millionaires-1480717606838-thumbLarge-v2.png}

\hypertarget{outsiders-insiders-and-multimillionaires-in-trumps-cabinet}{%
\subsection{Outsiders, Insiders and Multimillionaires in Trump's
Cabinet}\label{outsiders-insiders-and-multimillionaires-in-trumps-cabinet}}

President-elect Donald J. Trump's cabinet and top staff are shaping up
to be a mix of wealthy Washington outsiders, Republican insiders and
former military officers who have been critical of the Obama
administration.

``It suggests that this committee is trying to protect this nominee from
scrutiny,'' said Senator Christopher S. Murphy, Democrat of Connecticut.

With time limited, Democrats confronted Ms. DeVos with rapid-fire
questions, demanding that she explain her family's contributions to
groups that support so-called conversion therapy for gay people; her
donations to Republicans and their causes, which she agreed totaled
about \$200 million over the years; her past statements that government
``sucks'' and that public schools are a ``dead end''; and the poor
performance of charter schools in Detroit, where she resisted
legislation that would have blocked chronically failing charter schools
from expanding.

Under questioning, Ms. DeVos said it would be ``premature'' to say
whether she would continue the Obama administration's policy requiring
uniform reporting standards for sexual assaults on college campuses. She
told Mr. Murphy, whose constituents include families whose children were
killed in the 2012 massacre at Sandy Hook Elementary School, that it
should be ``left to locales'' to decide whether guns are allowed in
schools, and that she supported Mr. Trump's call to ban gun-free zones
around schools. She also denied that she had personally supported
conversion therapy.

Senator Elizabeth Warren of Massachusetts pressed Ms. DeVos on how she
could oversee the Education Department, the largest provider of student
loans, given that she had no experience running a large bureaucracy and
that neither she nor her children had ever taken out a student loan.

``So you have no personal experience with college financial aid?'' Ms.
Warren asked.

Ms. DeVos, who did not attend public schools or send her children to
public schools, argued that vouchers and charter schools were simply a
way of offering poor parents the kind of school choice that wealthy
parents have long been able to afford.

She described a visit she and her husband, an heir to the Amway fortune,
made to a Christian school in her hometown, Grand Rapids, Mich., as a
turning point in her career as a school choice advocate. ``We saw the
struggles and sacrifices many of these families faced when trying to
choose the best educational option for their children,'' she said. ``For
me, this was not just an issue of public policy but of national
injustice.''

But Democrats said research showed that voucher programs had done little
to raise achievement among poor students.

\includegraphics{https://static01.nyt.com/images/2017/01/18/us/18devos2/18devos2-articleInline.jpg?quality=75\&auto=webp\&disable=upscale}

Senator Patty Murray of Washington, the ranking Democrat on the
committee, asked Ms. DeVos, ``Can you commit to us that you will not
work to privatize public schools or cut a single penny from public
education?''

Ms. DeVos began to demur, saying that ``not all schools are working for
the students that are assigned to them'' and that she would work to find
``common ground'' to give parents ``options.''

``I take that as not being willing to commit to not privatizing public
education,'' Ms. Murray said.

Mr. Alexander, himself a former education secretary, argued that Ms.
DeVos's support of charter schools and vouchers put her in the
``mainstream'' of public opinion, and that her critics were outside it.
He noted that charter schools, which are publicly funded but typically
run independently of local school districts and teachers' unions, have
been supported by Republican and Democratic presidents going back to
Bill Clinton.

Democrats, however, argued that Ms. DeVos's support went well beyond
charter schools, to include the more contentious policy of sending
public money to private and religious schools.

``Charter schools are not the issue here,'' said Senator Al Franken of
Minnesota, where Democrats pushed the nation's first law allowing
charter schools nearly three decades ago. He noted that 37 states
prohibit the use of public dollars for religious schools.

One Republican, Senator Lisa Murkowski of Alaska, expressed concern
about Ms. DeVos's enthusiasm for school choice --- a moot point for many
of her constituents, given the vastness of her state.

``When there is no way to get to an alternative option for your child,
the best parent is left relying on a public school system that they
demand to be there for their kids,'' she said, asking Ms. DeVos to
ensure that her commitment to traditional public education was as
``strong and robust'' as her passion for school choice.

Advertisement

\protect\hyperlink{after-bottom}{Continue reading the main story}

\hypertarget{site-index}{%
\subsection{Site Index}\label{site-index}}

\hypertarget{site-information-navigation}{%
\subsection{Site Information
Navigation}\label{site-information-navigation}}

\begin{itemize}
\tightlist
\item
  \href{https://help.nytimes.com/hc/en-us/articles/115014792127-Copyright-notice}{©~2020~The
  New York Times Company}
\end{itemize}

\begin{itemize}
\tightlist
\item
  \href{https://www.nytco.com/}{NYTCo}
\item
  \href{https://help.nytimes.com/hc/en-us/articles/115015385887-Contact-Us}{Contact
  Us}
\item
  \href{https://www.nytco.com/careers/}{Work with us}
\item
  \href{https://nytmediakit.com/}{Advertise}
\item
  \href{http://www.tbrandstudio.com/}{T Brand Studio}
\item
  \href{https://www.nytimes.com/privacy/cookie-policy\#how-do-i-manage-trackers}{Your
  Ad Choices}
\item
  \href{https://www.nytimes.com/privacy}{Privacy}
\item
  \href{https://help.nytimes.com/hc/en-us/articles/115014893428-Terms-of-service}{Terms
  of Service}
\item
  \href{https://help.nytimes.com/hc/en-us/articles/115014893968-Terms-of-sale}{Terms
  of Sale}
\item
  \href{https://spiderbites.nytimes.com}{Site Map}
\item
  \href{https://help.nytimes.com/hc/en-us}{Help}
\item
  \href{https://www.nytimes.com/subscription?campaignId=37WXW}{Subscriptions}
\end{itemize}
