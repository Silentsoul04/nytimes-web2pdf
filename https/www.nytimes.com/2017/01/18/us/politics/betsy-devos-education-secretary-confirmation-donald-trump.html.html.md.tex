Sections

SEARCH

\protect\hyperlink{site-content}{Skip to
content}\protect\hyperlink{site-index}{Skip to site index}

\href{https://www.nytimes.com/section/politics}{Politics}

\href{https://myaccount.nytimes.com/auth/login?response_type=cookie\&client_id=vi}{}

\href{https://www.nytimes.com/section/todayspaper}{Today's Paper}

\href{/section/politics}{Politics}\textbar{}Nominee Betsy DeVos's
Knowledge of Education Basics Is Open to Criticism

\url{https://nyti.ms/2k1MFWf}

\begin{itemize}
\item
\item
\item
\item
\item
\end{itemize}

Advertisement

\protect\hyperlink{after-top}{Continue reading the main story}

Supported by

\protect\hyperlink{after-sponsor}{Continue reading the main story}

\hypertarget{nominee-betsy-devoss-knowledge-of-education-basics-is-open-to-criticism}{%
\section{Nominee Betsy DeVos's Knowledge of Education Basics Is Open to
Criticism}\label{nominee-betsy-devoss-knowledge-of-education-basics-is-open-to-criticism}}

\includegraphics{https://static01.nyt.com/images/2017/01/19/us/19devos/19devos-articleInline.jpg?quality=75\&auto=webp\&disable=upscale}

By \href{http://www.nytimes.com/by/kate-zernike}{Kate Zernike}

\begin{itemize}
\item
  Jan. 18, 2017
\item
  \begin{itemize}
  \item
  \item
  \item
  \item
  \item
  \end{itemize}
\end{itemize}

WASHINGTON --- Until Tuesday,
\href{https://mobile.nytimes.com/2017/01/12/us/politics/betsy-devos-education.html}{the
fight over Betsy DeVos's nomination} to be secretary of education
revolved mostly around her support of contentious school choice
programs.

But her confirmation hearing that night opened her up to new criticism:
that her long battle for school choice, controversial as it has been, is
the sum total of her experience and understanding of education policy.
In questioning by senators, she seemed either unaware or unsupportive of
the longstanding policies and functions of the department she is in line
to lead, from special education rules to the policing of for-profit
universities.

Ms. DeVos admitted that she might have been ``confused'' when she
appeared not to know that the broad statute that has governed special
education for more than four decades is federal law.

A billionaire investor, education philanthropist and Michigan Republican
activist, Ms. DeVos acknowledged that she has no personal experience
with student loans --- the federal government is the largest provider
--- and said she would have to ``review'' the department's policies that
try to prevent fraud by for-profit colleges.

She appeared blank on basic education terms. Asked how school
performance should be assessed, she did not know the difference between
growth, which measures how much students have learned over a given
period, and proficiency, which measures how many students reach a
targeted score.

Ms. DeVos even became something of an internet punch line when she
suggested that some school officials should be allowed to carry guns on
the premises to defend against grizzly bears.

But if she was sometimes rattled on the specifics, Ms. DeVos was
unshakable in her belief that education authority should devolve away
from the federal government and toward state and local authorities.
Whether the issue was allowing guns in schools, how to investigate
sexual assault on college campuses, or how to measure learning, her
answer was always that states and what she called ``locales'' knew best.

Those answers reflected the same instinct that has driven her advocacy
of school choice, primarily vouchers, which take money from public
schools to help families pay tuition at private schools, and charter
schools, which are publicly funded but typically independent of school
district or union rules. As she said bluntly in a 2015 education speech,
``Government really sucks.''

Ms. DeVos's supporters defended her performance at the hearing, saying
she showed herself to be a champion of innovation against Democrats'
defending the status quo.

``They tried their best to turn her into Cruella de Vil, but America got
to see the real Betsy DeVos first hand,'' said Ed Patru, a spokesman for
a group calling itself Friends of Betsy DeVos. ``The country saw an
authentic, compassionate and eminently reasonable education leader who
is committed to empowering parents and putting kids first.''

With a Republican-controlled Senate, Ms. DeVos is still likely to be
confirmed. But her statements on special education could make her
vulnerable; families of children with special needs are a vocal lobby,
one that Republicans do not want to alienate.

The federal Individuals With Disabilities Education Act, or IDEA, has
governed special education in the nation's public schools since the
mid-1970s. Before the law, states and local school districts had been
excluding or essentially warehousing students with disabilities.

Complaints from states and local school districts have been less about
the law than about the federal government's failure to pay its promised
share of the costs. But at the hearing, Ms. DeVos questioned the basic
premise that the federal government has a role in ensuring that any
school receiving taxpayer dollars --- whether a traditional public
school, a charter or a private school accepting vouchers --- comply with
the law's requirements for students with special needs.

Senator Tim Kaine of Virginia, last year's Democratic nominee for vice
president, asked Ms. DeVos whether schools that receive tax dollars
should be required to meet the requirements of IDEA.

``I think that is a matter that's best left to the states,'' Ms. DeVos
replied.

Mr. Kaine came back: ``So some states might be good to kids with
disabilities, and other states might not be so good, and then what?
People can just move around the country if they don't like how their
kids are being treated?''

Ms. DeVos repeated, ``I think that is an issue that's best left to the
states.''

``It's federal law,'' an exasperated Mr. Kaine replied.

Mr. Kaine then asked if all elementary and secondary schools receiving
tax dollars should be required to comply with reporting requirements on
harassment, discipline and bullying.

Ms. DeVos answered, ``I would look forward to reviewing that
provision.''

Senator Maggie Hassan, Democrat of New Hampshire, pressed Ms. DeVos
about whether she really intended to say that states could ignore the
law.

``Were you unaware that it was federal law?'' Ms. Hassan asked.

Ms. DeVos answered, ``I may have confused it.''

Ms. Hassan, who has a son with cerebral palsy, expressed concerns about
private schools that accept vouchers on the condition that students
waive legal rights under federal education law.

``Do you think families should have recourse in the courts if schools
don't meet their needs?'' she asked.

``Senator, I assure you that if confirmed I will be very sensitive to
the needs of special needs students,'' Ms. DeVos said.

``It's not about sensitivity, although that helps,'' Ms. Hassan
countered. ``It's about being willing to enforce the law to make sure
that my child and every child has the same access to public education,
high-quality public education.''

In her opening statement, Ms. DeVos spoke of wanting to expand options
for higher education beyond traditional ``brick and mortar and ivy.''

Senator Elizabeth Warren, a Massachusetts Democrat, asked Ms. DeVos
whether she would enforce Obama-era education policies intended to
prevent fraud at for-profit colleges --- institutions like
\href{https://www.nytimes.com/2016/11/19/us/politics/trump-university.html}{Trump
University}, which paid to settle federal class action lawsuits accusing
it of using marketing tactics to enroll students.

``I will review that rule and see that it is actually achieving what the
intentions are,'' Ms. DeVos said.

Ms. Warren, in apparent disbelief, responded: ``Swindlers and crooks are
out there doing back flips when they hear an answer like this.''

She added, ``If confirmed, you will be the cop on the beat, and if you
can't commit to use the tools already available to you in the Department
of Education, then I don't see how you can be the secretary of
education.''

Advertisement

\protect\hyperlink{after-bottom}{Continue reading the main story}

\hypertarget{site-index}{%
\subsection{Site Index}\label{site-index}}

\hypertarget{site-information-navigation}{%
\subsection{Site Information
Navigation}\label{site-information-navigation}}

\begin{itemize}
\tightlist
\item
  \href{https://help.nytimes.com/hc/en-us/articles/115014792127-Copyright-notice}{©~2020~The
  New York Times Company}
\end{itemize}

\begin{itemize}
\tightlist
\item
  \href{https://www.nytco.com/}{NYTCo}
\item
  \href{https://help.nytimes.com/hc/en-us/articles/115015385887-Contact-Us}{Contact
  Us}
\item
  \href{https://www.nytco.com/careers/}{Work with us}
\item
  \href{https://nytmediakit.com/}{Advertise}
\item
  \href{http://www.tbrandstudio.com/}{T Brand Studio}
\item
  \href{https://www.nytimes.com/privacy/cookie-policy\#how-do-i-manage-trackers}{Your
  Ad Choices}
\item
  \href{https://www.nytimes.com/privacy}{Privacy}
\item
  \href{https://help.nytimes.com/hc/en-us/articles/115014893428-Terms-of-service}{Terms
  of Service}
\item
  \href{https://help.nytimes.com/hc/en-us/articles/115014893968-Terms-of-sale}{Terms
  of Sale}
\item
  \href{https://spiderbites.nytimes.com}{Site Map}
\item
  \href{https://help.nytimes.com/hc/en-us}{Help}
\item
  \href{https://www.nytimes.com/subscription?campaignId=37WXW}{Subscriptions}
\end{itemize}
