Sections

SEARCH

\protect\hyperlink{site-content}{Skip to
content}\protect\hyperlink{site-index}{Skip to site index}

\href{https://www.nytimes.com/section/politics}{Politics}

\href{https://myaccount.nytimes.com/auth/login?response_type=cookie\&client_id=vi}{}

\href{https://www.nytimes.com/section/todayspaper}{Today's Paper}

\href{/section/politics}{Politics}\textbar{}Issues of Riches Trip Up
Steven Mnuchin and Other Nominees

\url{https://nyti.ms/2jEN0kO}

\begin{itemize}
\item
\item
\item
\item
\item
\end{itemize}

Advertisement

\protect\hyperlink{after-top}{Continue reading the main story}

Supported by

\protect\hyperlink{after-sponsor}{Continue reading the main story}

\hypertarget{issues-of-riches-trip-up-steven-mnuchin-and-other-nominees}{%
\section{Issues of Riches Trip Up Steven Mnuchin and Other
Nominees}\label{issues-of-riches-trip-up-steven-mnuchin-and-other-nominees}}

\includegraphics{https://static01.nyt.com/images/2017/01/20/us/20ethics/20ethics-articleInline.jpg?quality=75\&auto=webp\&disable=upscale}

By \href{https://www.nytimes.com/by/alan-rappeport}{Alan Rappeport}

\begin{itemize}
\item
  Jan. 19, 2017
\item
  \begin{itemize}
  \item
  \item
  \item
  \item
  \item
  \end{itemize}
\end{itemize}

WASHINGTON --- Steven T. Mnuchin, President-elect Donald J. Trump's pick
to head the Treasury Department, came under withering fire on Thursday
for foreclosing on homes in California, managing an offshore investment
account and initially
\href{https://www.nytimes.com/2017/01/19/us/politics/steven-mnuchin-treasury-secretary-nominee-assets-confirmation.html}{failing
to disclose almost \$100 million in assets}.

During five hours of contentious questioning, Democrats on the Senate
Finance Committee pressed Mr. Mnuchin to explain how he could fail to
disclose \$95 million of real estate assets, including property in New
York, Los Angeles and Mexico, on his initial committee ethics
questionnaire. The former Goldman Sachs banker also did not initially
disclose that he is the director of Dune Capital International, an
investment fund incorporated on the Cayman Islands, an offshore tax
haven. And he belatedly disclosed that his children own nearly \$1
million in artwork.

Such problems of fabulous wealth seem almost part and parcel to the
confirmation struggles of a Trump government that would be filled with
billionaires and multimillionaires. Mr. Trump's pick for commerce
secretary, Wilbur Ross, admitted this week that he had fired an
undocumented immigrant --- one of the ``dozen or so'' housekeepers the
billionaire investor employs.

Mr. Trump's pick to be White House budget director, Representative Mick
Mulvaney of South Carolina, admitted paying back taxes of more than
\$15,000 for a household employee whose payroll taxes he had not paid,
and his choice for health secretary, Representative Tom Price of
Georgia, spent much of his confirmation hearing defending stock trades
that might have been impacted by his legislating.

But in a Washington where the tone is set by an incoming president who
has refused to disclose his tax returns, it is not clear that any ethics
question could actually derail a nominee, even ones that have caused
problems in the past.

During his prickly hearing, Mr. Mnuchin played down his omissions as
innocent errors that were the products of cumbersome government
bureaucracy.

``I think as you all can appreciate, filling out these government forms
is quite complicated,'' Mr. Mnuchin said when asked about the omissions,
noting that he had handed over 5,000 pages of disclosures. ``Let me
first say, any oversight, it was unintentional.''

Democrats were unsatisfied with the response, assailing Mr. Mnuchin for
the lapse and wondering what he was hiding.

``It doesn't take a rocket scientist to understand the words, `list all
positions,''' said Senator Robert Menendez, Democrat of New Jersey.

Past nominees have stepped aside after such revelations. In 2009, Tom
Daschle, President Obama's pick for health secretary, withdrew after it
emerged that he owed more than \$100,000 in back taxes on the use of a
limousine and chauffeur. Only weeks before, the nomination of Timothy F.
Geithner, Mr. Obama's first Treasury secretary, nearly came undone for
his failure to pay some payroll taxes while at the International
Monetary Fund.

At the dawn of Bill Clinton's presidency, the employment of undocumented
immigrant nannies deprived both Kimba Wood and Zoë Baird of the chance
to become the first female attorneys general.

But such missteps might not matter these days. Just as Mr. Trump
shrugged off the long-established practices of releasing tax returns,
divesting assets that could create conflicts of interest and disclosing
medical records, his cabinet picks might choose to muscle through.

``If Trump's nominees can't get their ethics disclosures right now, how
can we trust them when they are in office?'' asked Norman Eisen, a
fellow at the Brookings Institution who advised the Obama administration
on government ethics.

It appears that ethics concerns will continue to swirl around the Trump
administration. Three Democratic senators on Thursday sent a letter to
Mr. Mnuchin about reports that Anthony Scaramucci, a wealthy financier
whom Mr. Trump appointed to be White House director of engagement and
intergovernmental affairs, was holding talks with the head of the
sanctioned Russian investment fund at the World Economic Forum in Davos
this week. The Russian fund has been under Treasury Department sanctions
since 2015.

``This raises questions about whether Mr. Scaramucci engaged in
discussions to facilitate prohibited transactions with the sanctioned
entity in violation of federal law, and about whether other Trump
administration transition officials were aware of or approved his
activities,'' they wrote, calling on Mr. Mnuchin to investigate the
matter if he is confirmed.

Mr. Mnuchin did not shy away from controversies surrounding his business
record, forcefully defending his use of offshore tax havens as a hedge
fund manager and rejecting accusations that he was churning out
foreclosures when he was chief executive of One West Bank during the
financial crisis.

\includegraphics{https://static01.nyt.com/images/2017/01/20/us/20ethics2/20ethics2-articleInline.jpg?quality=75\&auto=webp\&disable=upscale}

This week progressive groups launched television advertisements painting
him as a cruel predatory lender, and Senator Elizabeth Warren, Democrat
of Massachusetts, hosted a ``shadow hearing'' featuring people who felt
victimized by the bank.

But Mr. Mnuchin went on the offensive, dismissing such accusations in
his opening statement.

``Since I was first nominated to serve as Treasury secretary, I have
been maligned as taking advantage of others' hardships in order to earn
a buck,'' Mr. Mnuchin said. ``Nothing could be further from the truth.''

Mr. Mnuchin spent a good part of the five-hour hearing fending off
charges that One West had been overly aggressive in foreclosing on tens
of thousands of borrowers, many of them elderly. Mr. Mnuchin said the
remnants of IndyMac, which he and a group of wealthy investors bought
from the Federal Deposit Insurance Corporation, was sitting on a
mountain of troubled mortgages. He said the bank did its best to try to
modify loans and keep borrowers in their homes but conceded that OneWest
had made mistakes, such as foreclosing on some military families.

Mr. Mnuchin expressed remorse for not being able to help more homeowners
who struggled during the collapse of the housing market, especially one,
a minor celebrity.

``The most troubling loan we had was actually to the Octomom,'' Mr.
Mnuchin said, referring to a woman who gave birth to octuplets and
became a reality television star. ``That was a terrible situation, and
we worked very, very hard to move her to another home that they could
afford.''

To many Democrats, the actions of One West during the housing and
financial crisis were too much. Senator Sherrod Brown, Democrat of Ohio,
announced his opposition to Mr. Mnuchin hours after the hearing ended.

``Mr. Mnuchin's cozy ties to Wall Street raise serious red flags that
demand serious answers,'' Mr. Brown said in a statement.

When Mr. Mnuchin was pressed on whether he was helping rich clients
avoid paying taxes by setting up shop in the Cayman Islands, he insisted
that he had paid all the taxes that he owed and that he was following
the law.

``I did not use a Cayman Islands entity in any way to avoid paying taxes
for myself,'' Mr. Mnuchin said. ``I would love to work with the I.R.S.
to close these tax issues that make no sense.''

That did not go over well with Democrats.

``One does not go and create offshore entities at the end of the day
other than to avoid, in some form or fashion, the tax laws of the United
States. That's pretty simple,'' Mr. Menendez said.

Although Republicans have been sticklers about ethics when out of power,
they showed little concern about Mr. Mnuchin's failure to make
disclosures. The Finance Committee in particular has prided itself on
its bipartisanship when it comes to the rigorous vetting of nominees,
but on Thursday, that was hardly in evidence.

Republican senators praised Mr. Mnuchin's credentials, showed
appreciation for his willingness to sacrifice his business career and
apologized for the length and tone of the hearing.

``Objectively speaking, I don't believe anyone can reasonably argue that
Mr. Mnuchin is unqualified for the position,'' said Senator Orrin G.
Hatch, Republican of Utah.

``You've been assaulted by innuendo,'' Senator Bill Cassidy, Republican
of Louisiana, told Mr. Mnuchin.

The only time Mr. Mnuchin appeared to be at a loss for words was when he
was peppered with questions about the ethics of Mr. Trump.

Senator Claire McCaskill, Democrat of Missouri, grilled Mr. Mnuchin
about how Americans could be sure that debt the Trump Organization owed
to foreign interests would not influence Mr. Trump's policies. She also
mocked the idea that the president-elect was appointing an independent
ethics officer to ensure he would have nothing to do with his business
empire, asking Mr. Mnuchin what authority a person that Mr. Trump could
hire and fire would have to keep him honest.

``It's a good question,'' Mr. Mnuchin said.

Advertisement

\protect\hyperlink{after-bottom}{Continue reading the main story}

\hypertarget{site-index}{%
\subsection{Site Index}\label{site-index}}

\hypertarget{site-information-navigation}{%
\subsection{Site Information
Navigation}\label{site-information-navigation}}

\begin{itemize}
\tightlist
\item
  \href{https://help.nytimes.com/hc/en-us/articles/115014792127-Copyright-notice}{©~2020~The
  New York Times Company}
\end{itemize}

\begin{itemize}
\tightlist
\item
  \href{https://www.nytco.com/}{NYTCo}
\item
  \href{https://help.nytimes.com/hc/en-us/articles/115015385887-Contact-Us}{Contact
  Us}
\item
  \href{https://www.nytco.com/careers/}{Work with us}
\item
  \href{https://nytmediakit.com/}{Advertise}
\item
  \href{http://www.tbrandstudio.com/}{T Brand Studio}
\item
  \href{https://www.nytimes.com/privacy/cookie-policy\#how-do-i-manage-trackers}{Your
  Ad Choices}
\item
  \href{https://www.nytimes.com/privacy}{Privacy}
\item
  \href{https://help.nytimes.com/hc/en-us/articles/115014893428-Terms-of-service}{Terms
  of Service}
\item
  \href{https://help.nytimes.com/hc/en-us/articles/115014893968-Terms-of-sale}{Terms
  of Sale}
\item
  \href{https://spiderbites.nytimes.com}{Site Map}
\item
  \href{https://help.nytimes.com/hc/en-us}{Help}
\item
  \href{https://www.nytimes.com/subscription?campaignId=37WXW}{Subscriptions}
\end{itemize}
