Sections

SEARCH

\protect\hyperlink{site-content}{Skip to
content}\protect\hyperlink{site-index}{Skip to site index}

\href{https://www.nytimes.com/section/health}{Health}

\href{https://myaccount.nytimes.com/auth/login?response_type=cookie\&client_id=vi}{}

\href{https://www.nytimes.com/section/todayspaper}{Today's Paper}

\href{/section/health}{Health}\textbar{}The Fight Trump Faces Over Drug
Prices

\url{https://nyti.ms/2kl5rIl}

\begin{itemize}
\item
\item
\item
\item
\item
\end{itemize}

Advertisement

\protect\hyperlink{after-top}{Continue reading the main story}

Supported by

\protect\hyperlink{after-sponsor}{Continue reading the main story}

\hypertarget{the-fight-trump-faces-over-drug-prices}{%
\section{The Fight Trump Faces Over Drug
Prices}\label{the-fight-trump-faces-over-drug-prices}}

\includegraphics{https://static01.nyt.com/images/2017/01/24/science/24drugprices/23drugprices-articleInline.jpg?quality=75\&auto=webp\&disable=upscale}

By \href{http://www.nytimes.com/by/katie-thomas}{Katie Thomas}

\begin{itemize}
\item
  Jan. 23, 2017
\item
  \begin{itemize}
  \item
  \item
  \item
  \item
  \item
  \end{itemize}
\end{itemize}

President Trump has made it clear that he thinks drug prices are too
high and that the pharmaceutical industry, as he put it at a news
conference this month, is ``getting away with murder.''

He joins a host of lawmakers and others who have excoriated drug makers
in recent years for high-priced drugs that are out of the reach of many
Americans. On Monday, Sean Spicer, Mr. Trump's press secretary,
reaffirmed that the issue would be a priority.

One of Mr. Trump's proposals --- to force drug makers to bid for the
right to sell their products to Medicare beneficiaries --- has
repeatedly failed to attract enough support in Congress, especially
among his fellow Republicans.

``Pharma has a lot of lobbyists and a lot of power, and there is very
little bidding,'' Mr. Trump said at the news conference this month, in
comments that
\href{http://www.reuters.com/article/us-usa-trump-drugpricing-idUSKBN14V24J}{briefly
sent pharmaceutical stocks tumbling}. ``We're the largest buyer of drugs
in the world, and yet we don't bid properly, and we're going to save
billions of dollars.''

\href{http://kff.org/medicaid/poll-finding/medicare-and-medicaid-at-50/}{Polls
show} that the public is in favor of doing away with a legislative
provision that prohibits the federal government from negotiating
directly with pharmaceutical companies for drugs that are paid for by
Medicare, the government health care program for people who are over 65
or have disabilities.

Removing that prohibition is a favorite cause of liberal politicians
like Senator Bernie Sanders of Vermont, but it has been opposed by the
pharmaceutical industry and Republicans, including Representative Tom
Price of Georgia, the nominee for secretary of health and human
services.

Mr. Price declined to say at
\href{https://www.nytimes.com/2017/01/18/us/politics/confirmation-hearing-cabinet.html}{a
confirmation hearing} last Wednesday whether he supported Mr. Trump's
position, and Democratic senators are likely to question him again on
the issue when he appears before the Senate Finance Committee for
another hearing on Tuesday.

Whether freeing the government to negotiate on drug prices would lower
costs, however, is anything but clear. And its chances of passing a
Republican-led Congress are even less so.

\hypertarget{why-cant-the-government-haggle-with-drug-companies}{%
\subsection{Why can't the government haggle with drug
companies?}\label{why-cant-the-government-haggle-with-drug-companies}}

Medicare did not cover prescription drugs until 2006, after Congress
expanded the program to add a drug benefit, known as Part D. But the
measure included something called the noninterference clause, which was
backed by the pharmaceutical industry and prevented the federal
government from negotiating directly with drug makers.

This prohibition does not mean that pharmaceutical companies can set any
price they want. Privately run prescription drug programs, which control
the benefits for large groups of Medicare beneficiaries, negotiate on
the government's behalf. These programs are run by the same companies
--- including Express Scripts, CVS and UnitedHealth --- that manage the
drug plans for large employers and insurers.

\hypertarget{what-is-the-argument-for-removing-this-clause}{%
\subsection{What is the argument for removing this
clause?}\label{what-is-the-argument-for-removing-this-clause}}

Those who favor letting the federal government negotiate directly on
drug prices argue that other countries, including Canada and Britain,
already have that leverage with many multinational drug corporations.
Their government-run health programs are the only game in town and hold
significant power in setting drug prices. Supporters say that if the
United States government were allowed to negotiate drug prices for all
\href{http://kff.org/medicare/fact-sheet/the-medicare-prescription-drug-benefit-fact-sheet/}{41
million Medicare beneficiaries} enrolled in drug coverage, it would lead
to lower prices across the health care market.

The pharmaceutical industry and Republicans
\href{http://www.phrma.org/advocacy/medicare/partd}{have long opposed
such a change}, however, saying that significantly cutting payments to
drug makers would stifle innovation and prevent them from investing in
new lifesaving drugs.

\hypertarget{would-negotiating-medicares-drug-prices-actually-lower-costs}{%
\subsection{Would negotiating Medicare's drug prices actually lower
costs?}\label{would-negotiating-medicares-drug-prices-actually-lower-costs}}

Many health care experts are skeptical that a repeal of the negotiating
ban would have much impact, and certainly not the ``billions'' of
dollars in reductions that Mr. Trump recently promised. The nonpartisan
Congressional Budget Office
\href{https://www.cbo.gov/sites/default/files/114th-congress-2015-2016/dataandtechnicalinformation/51431-HealthPolicy.pdf}{has
concluded}
\href{https://www.cbo.gov/sites/default/files/110th-congress-2007-2008/reports/drugpricenegotiation.pdf}{more
than once} that a repeal would result in only modest savings, in part
because private drug plans are already negotiating on the government's
behalf, albeit for smaller pools of beneficiaries.

And even if the ban were repealed,
\href{https://www.nytimes.com/2016/02/02/upshot/the-real-reason-medicare-is-a-lousy-drug-negotiator-it-cant-say-no.html}{another
provision} would stand in the way. Under Medicare, the government must
cover all drugs in six ``protected classes'': broad and often expensive
treatment areas for patients with conditions such as cancer, depression,
epilepsy and H.I.V.

The Veterans Health Administration and the Defense Department
\href{http://www.commonwealthfund.org/publications/blog/2016/may/drug-price-control-how-some-government-programs-do-it}{are
able to negotiate lower prices} in part by covering fewer drugs and
pitting drug companies against one another.

``You get your largest negotiating power from your ability to walk
away,'' said Dr. Aaron S. Kesselheim, an associate professor at Harvard
Medical School who has written frequently on drug prices.

Even if Medicare were to remove those protections and agree to cover
fewer drugs --- a politically unpopular move that has failed in the past
--- drug companies might find ways to make up the financial losses, said
Dan Mendelson, president of Avalere Health, a consulting firm that works
for drug companies, hospitals, insurers and others.

Today's most expensive drugs typically treat rare conditions, like
cystic fibrosis and certain types of cancer, for which there are few
options. In those cases, drug companies hold the bargaining power, no
matter who is on the other side of the table.

``I worry that if you force one price across the market, you may well
end up with a higher price in noncompetitive categories than you would
otherwise,'' Mr. Mendelson said.

And even if the negotiating ban were lifted, the United States would
still not have the same clout as, say, Britain, whose universal health
care coverage makes the government the country's only buyer of drugs.

Some worry that if the United States government were to significantly
lower drug prices for Medicare beneficiaries, pharmaceutical companies
would respond by raising prices for the millions of people insured
through large employers or private insurers. Others contend that if the
prices negotiated by Medicare were public, the transparency could level
the playing field and lower prices across the board.

\hypertarget{what-has-mr-trumps-position-been-on-drug-prices}{%
\subsection{What has Mr. Trump's position been on drug
prices?}\label{what-has-mr-trumps-position-been-on-drug-prices}}

During the campaign, Mr. Trump joined his Democratic opponents, Mr.
Sanders and Hillary Clinton, in
\href{http://info.msnbc.com/_news/2016/02/17/35127534-full-transcript-msnbcs-town-hall-with-donald-trump?lite}{calling
for the federal government} to be allowed to negotiate the price of
drugs.

After his election, the drug industry exhaled when the health care
section of his transition website
\href{http://www.politico.com/tipsheets/politico-pulse/2016/11/the-trump-transition-more-details-on-his-health-plan-and-transition-217369}{included
more traditional Republican} (and industry-friendly) priorities and did
not mention the negotiation provision.

In December,
\href{https://www.nytimes.com/2016/12/09/business/donald-trump-drug-prices-pharma-stocks.html}{Mr.
Trump pledged} in an interview with Time magazine to bring down drug
prices, and this month,
\href{http://money.cnn.com/2017/01/11/investing/donald-trump-press-conference-markets-economy/}{he
promised} to ``create new bidding procedures'' but did not offer
specific details.

\hypertarget{will-such-a-proposal-ever-clear-congress}{%
\subsection{Will such a proposal ever clear
Congress?}\label{will-such-a-proposal-ever-clear-congress}}

It seems unlikely, given that Republicans control both the House of
Representatives and the Senate.

``It's been a nonstarter with Republicans, and it's the traditionally
Democratic talking point,'' said Alana Dovner, a research analyst at
Beacon Policy Advisors, which advises investors on developments in
Washington. ``Something as dramatic as allowing Medicare to negotiate
drug prices --- that would be such a radical change, and so opposite the
traditional Republican health care policies, that it just seems
unfathomable at this point.''

Mr. Price opposed a Democratic-led measure in 2007,
\href{http://www.nytimes.com/2007/01/12/washington/12cnd-drug.html}{calling
it} a ``solution in search of a problem.''

Ms. Dovner and others said that even Democrats were unlikely to support
a measure that would allow Medicare to stop covering drugs in protected
areas like cancer. A limited effort to do that in 2014
\href{https://www.nytimes.com/2014/02/22/business/plan-to-alter-medicare-drug-coverage-draws-strong-opposition.html}{failed
amid bipartisan criticism}.

``You are by definition saying yes to some drugs and no to others,'' Mr.
Mendelson said. He called that position politically dangerous, saying it
would be denounced as ``government practice of medicine.''

Advertisement

\protect\hyperlink{after-bottom}{Continue reading the main story}

\hypertarget{site-index}{%
\subsection{Site Index}\label{site-index}}

\hypertarget{site-information-navigation}{%
\subsection{Site Information
Navigation}\label{site-information-navigation}}

\begin{itemize}
\tightlist
\item
  \href{https://help.nytimes.com/hc/en-us/articles/115014792127-Copyright-notice}{©~2020~The
  New York Times Company}
\end{itemize}

\begin{itemize}
\tightlist
\item
  \href{https://www.nytco.com/}{NYTCo}
\item
  \href{https://help.nytimes.com/hc/en-us/articles/115015385887-Contact-Us}{Contact
  Us}
\item
  \href{https://www.nytco.com/careers/}{Work with us}
\item
  \href{https://nytmediakit.com/}{Advertise}
\item
  \href{http://www.tbrandstudio.com/}{T Brand Studio}
\item
  \href{https://www.nytimes.com/privacy/cookie-policy\#how-do-i-manage-trackers}{Your
  Ad Choices}
\item
  \href{https://www.nytimes.com/privacy}{Privacy}
\item
  \href{https://help.nytimes.com/hc/en-us/articles/115014893428-Terms-of-service}{Terms
  of Service}
\item
  \href{https://help.nytimes.com/hc/en-us/articles/115014893968-Terms-of-sale}{Terms
  of Sale}
\item
  \href{https://spiderbites.nytimes.com}{Site Map}
\item
  \href{https://help.nytimes.com/hc/en-us}{Help}
\item
  \href{https://www.nytimes.com/subscription?campaignId=37WXW}{Subscriptions}
\end{itemize}
