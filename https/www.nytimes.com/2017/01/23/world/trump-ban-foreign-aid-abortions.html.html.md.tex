Sections

SEARCH

\protect\hyperlink{site-content}{Skip to
content}\protect\hyperlink{site-index}{Skip to site index}

\href{https://www.nytimes.com/section/world/americas}{Americas}

\href{https://myaccount.nytimes.com/auth/login?response_type=cookie\&client_id=vi}{}

\href{https://www.nytimes.com/section/todayspaper}{Today's Paper}

\href{/section/world/americas}{Americas}\textbar{}Trump Revives Ban on
Foreign Aid to Groups That Give Abortion Counseling

\url{https://nyti.ms/2kkuueB}

\begin{itemize}
\item
\item
\item
\item
\item
\end{itemize}

Advertisement

\protect\hyperlink{after-top}{Continue reading the main story}

Supported by

\protect\hyperlink{after-sponsor}{Continue reading the main story}

\hypertarget{trump-revives-ban-on-foreign-aid-to-groups-that-give-abortion-counseling}{%
\section{Trump Revives Ban on Foreign Aid to Groups That Give Abortion
Counseling}\label{trump-revives-ban-on-foreign-aid-to-groups-that-give-abortion-counseling}}

\includegraphics{https://static01.nyt.com/images/2017/01/24/world/24nations/24nations-articleInline-v3.jpg?quality=75\&auto=webp\&disable=upscale}

By \href{http://www.nytimes.com/by/somini-sengupta}{Somini Sengupta}

\begin{itemize}
\item
  Jan. 23, 2017
\item
  \begin{itemize}
  \item
  \item
  \item
  \item
  \item
  \end{itemize}
\end{itemize}

UNITED NATIONS --- President Trump reinstated a policy on Monday that
originated in the Reagan era, prohibiting the granting of American
foreign aid to health providers abroad who discuss abortion as a
family-planning option.

United States law already prohibits the use of American taxpayer dollars
for abortion services anywhere, including in countries where the
procedure is legal. But Mr. Trump's order takes the prohibition further:
It freezes funding to nongovernmental organizations in poor countries if
they offer abortion counseling or if they advocate the right to seek
abortion in their countries.

The freeze applies even if the organizations use other sources of
funding for these services.

Mr. Trump and Vice President Mike Pence stated their opposition to
abortion during the presidential campaign. Mr. Trump had signaled his
intent to make the order one of his first acts as president, which
pleased anti-abortion activists at home.

``We applaud President Trump for putting an end to taxpayer funding of
groups that promote the killing of unborn children in developing
nations,'' Carol Tobias, president of the National Right to Life
Committee in Washington, the nation's largest anti-abortion
organization,
\href{http://www.nrlc.org/communications/releases/2017/release012317/}{said
in a statement}.

Critics said the order reflected the new administration's disregard of
women's reproductive health rights, whose advocates were an important
force in the
\href{https://www.nytimes.com/2017/01/21/us/women-march-protest-president-trump.html}{protest
marches} in Washington and other cities after Mr. Trump's inauguration.

It revives what is known as the Mexico City policy, so named because
President Ronald Reagan announced it in 1984 during a United Nations
population conference in Mexico City. Critics call it the global gag
rule. Since Reagan, Democratic administrations have suspended the policy
and Republicans have reimposed it.

Some women's health advocates interpreted Mr. Trump's order as a huge
expansion of the policy. Adrienne Lee, a spokeswoman for PAI, a
reproductive rights group in Washington, said the order would cut
funding to ``every program that falls under global health assistance.''

Asked at
\href{https://www.nytimes.com/interactive/2017/01/23/us/politics/spicer-white-house-briefing-live.html}{his
first official briefing} on Monday what message the administration was
sending by reinstating the policy as one of its first orders of
business, Sean Spicer, the White House spokesman, told reporters that
Mr. Trump had ``made it very clear that he's a pro-life president.''

``He wants to stand up for all Americans, including the unborn, and I
think the reinstatement of this policy is not just something that echoes
that value but respects taxpayer funding as well,'' Mr. Spicer said.

Health experts say the policy has not led to a decline in abortions in
the affected countries. Some research suggests that it has had the
opposite effect: increasing abortion rates by forcing health clinics to
close or to restrict contraceptive supplies because of lack of funding.
Others say the restriction only heightens the risk of illegal and often
unsafe abortions.

The impact of Mr. Trump's order is likely to be felt beyond abortion
services, which cannot be carried out with federal funding under a 1973
law known as the Helms Amendment, after former Senator Jesse Helms.

Critics said the order would hinder the ability of women in poor
countries to obtain reproductive health services, including family
planning, by severing American funding to health clinics that offer a
variety of services, including abortion counseling.

The International Planned Parenthood Federation said its partners in
Nepal, Kenya and Ethiopia had lost American funding the last time the
policy was in effect, during the Bush administration. Because
nongovernmental groups in those countries refused to accept the
conditions of the policy, they were compelled to close clinics and offer
fewer contraceptives, said Kelly Castagnaro, a Planned Parenthood
spokeswoman.

A \href{http://www.who.int/bulletin/volumes/89/12/11-091660/en/}{study
of 20 sub-Saharan African countries} by Stanford University researchers
found that in countries that relied heavily on funding from the United
States for reproductive health services, abortion rates rose when the
Reagan-era policy was in place.

``When the policy comes on, fewer women get contraceptives in countries
that depend on U.S. funding for family planning,'' Eran Bendavid, the
lead author of the study, said on Monday.

Ms. Castagnaro said the revival of the Mexico City policy could cost
Planned Parenthood about \$100 million in American funding over the next
four years.

In recent decades, abortion rates have declined sharply in the richest
countries, including the United States, where the rate
\href{https://www.nytimes.com/2017/01/18/health/rate-of-us-abortions-hits-lowest-since-roe-v-wade.html}{has
fallen} to its lowest level since the Supreme Court legalized abortion
in 1973, according to the Guttmacher Institute, a research group that
supports abortion rights. Rates have remained steady in the developing
world since the early 1990s.

The World Health Organization says 225 million women in developing
nations would like to delay childbearing but are not using contraception
for a variety of reasons, including a lack of access.

``President Trump's reinstatement of the global gag rule ignores decades
of research, instead favoring ideological politics over women and
families,'' Senator Jeanne Shaheen, Democrat of New Hampshire, said on
Monday. ``We know that when family planning services and contraceptives
are easily accessible, there are fewer unplanned pregnancies, maternal
deaths and abortions.''

Vicki Saporta, president and chief executive of the National Abortion
Federation, a Washington-based advocacy group for abortion rights, said
in
\href{https://prochoice.org/this-draconian-policy-has-been-devastating/}{a
statement}, ``President Trump's decision to reinstate the global gag
rule will endanger already vulnerable women by further curtailing their
access to accurate information and safe reproductive health care
services.''

Mr. Trump's order repealed one made by President Obama when he took
office in 2009, which had repealed the Bush version of the policy from
2001. In effect, Mr. Trump reinstated the Bush policy.

Democrats in Congress have tried, unsuccessfully, to pass legislation to
scrap the policy. Ms. Shaheen said she would introduce similar
legislation. But with Republicans controlling both houses of Congress,
it is unlikely to pass.

Mr. Trump's pick for ambassador to the United Nations, Gov. Nikki R.
Haley of South Carolina, made clear in her confirmation hearing last
week that she opposed abortion, but said she supported funding for
contraceptive services in foreign aid programs.

Advertisement

\protect\hyperlink{after-bottom}{Continue reading the main story}

\hypertarget{site-index}{%
\subsection{Site Index}\label{site-index}}

\hypertarget{site-information-navigation}{%
\subsection{Site Information
Navigation}\label{site-information-navigation}}

\begin{itemize}
\tightlist
\item
  \href{https://help.nytimes.com/hc/en-us/articles/115014792127-Copyright-notice}{©~2020~The
  New York Times Company}
\end{itemize}

\begin{itemize}
\tightlist
\item
  \href{https://www.nytco.com/}{NYTCo}
\item
  \href{https://help.nytimes.com/hc/en-us/articles/115015385887-Contact-Us}{Contact
  Us}
\item
  \href{https://www.nytco.com/careers/}{Work with us}
\item
  \href{https://nytmediakit.com/}{Advertise}
\item
  \href{http://www.tbrandstudio.com/}{T Brand Studio}
\item
  \href{https://www.nytimes.com/privacy/cookie-policy\#how-do-i-manage-trackers}{Your
  Ad Choices}
\item
  \href{https://www.nytimes.com/privacy}{Privacy}
\item
  \href{https://help.nytimes.com/hc/en-us/articles/115014893428-Terms-of-service}{Terms
  of Service}
\item
  \href{https://help.nytimes.com/hc/en-us/articles/115014893968-Terms-of-sale}{Terms
  of Sale}
\item
  \href{https://spiderbites.nytimes.com}{Site Map}
\item
  \href{https://help.nytimes.com/hc/en-us}{Help}
\item
  \href{https://www.nytimes.com/subscription?campaignId=37WXW}{Subscriptions}
\end{itemize}
