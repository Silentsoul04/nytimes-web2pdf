Sections

SEARCH

\protect\hyperlink{site-content}{Skip to
content}\protect\hyperlink{site-index}{Skip to site index}

\href{https://www.nytimes.com/section/world/americas}{Americas}

\href{https://myaccount.nytimes.com/auth/login?response_type=cookie\&client_id=vi}{}

\href{https://www.nytimes.com/section/todayspaper}{Today's Paper}

\href{/section/world/americas}{Americas}\textbar{}Facing Trump, Mexicans
Think the Unthinkable: Leaving Nafta

\url{https://nyti.ms/2kpSBN4}

\begin{itemize}
\item
\item
\item
\item
\item
\end{itemize}

Advertisement

\protect\hyperlink{after-top}{Continue reading the main story}

Supported by

\protect\hyperlink{after-sponsor}{Continue reading the main story}

\hypertarget{facing-trump-mexicans-think-the-unthinkable-leaving-nafta}{%
\section{Facing Trump, Mexicans Think the Unthinkable: Leaving
Nafta}\label{facing-trump-mexicans-think-the-unthinkable-leaving-nafta}}

\includegraphics{https://static01.nyt.com/images/2017/01/25/us/25MEXICO1/25MEXICO1-articleInline.jpg?quality=75\&auto=webp\&disable=upscale}

By \href{https://www.nytimes.com/by/elisabeth-malkin}{Elisabeth Malkin}

\begin{itemize}
\item
  Jan. 24, 2017
\item
  \begin{itemize}
  \item
  \item
  \item
  \item
  \item
  \end{itemize}
\end{itemize}

MEXICO CITY --- Not long ago, any suggestion that Mexico might walk away
from the North American Free Trade Agreement would have been met with
utter disbelief.

That was
\href{https://www.nytimes.com/2016/11/10/world/americas/mexico-donald-trump-peso.html}{before
Donald J. Trump was elected} president of the United States.

Free trade is a mantra of Mexico's political elite,
\href{https://www.nytimes.com/2017/01/04/world/americas/mexico-donald-trump-nafta.html}{the
core} of the country's development strategy.

But now that Mr. Trump has said he wants to renegotiate Nafta, a growing
number of Mexican officials and businesspeople are asking what price is
worth paying to stay in it. Many of them are concluding that Mexico
could have more to lose from years of haggling and
\href{https://www.nytimes.com/2016/11/27/world/americas/mexico-oil-pemex-donald-trump.html}{economic
uncertainty} than from simply opting out.

``There could be no other option,'' Mexico's economy minister, Ildefonso
Guajardo, said on Tuesday in a televised interview. ``If we go for
something that is less than what we have, well, then there is no sense
in staying.''

About \$1.4 billion in goods go back and forth across the Mexico-United
States border every day. The United States buys almost 80 percent of
Mexico's exports, and
\href{https://www.census.gov/foreign-trade/statistics/highlights/top/top1611yr.html}{Mexico
is the second-largest market} in the world for American goods.

Exactly how Mr. Trump's government is proposing to renegotiate that flow
is still unclear. The Mexican government says that Mr. Guajardo and
Mexico's foreign minister, Luis Videgaray, expect to get a first look
when they arrive in Washington on Wednesday for two days of talks with
officials in the Trump administration.

Mexico's president, Enrique Peña Nieto, is then scheduled to meet Mr.
Trump on Jan 31.

The Mexican government's talk of walking away from Nafta if the Trump
administration demands terms that are too tough could be strategic
bluster, a tactic to begin the discussions on stronger footing.

To drive home Mexico's importance to the United States, the Mexicans
also intend to raise many other issues that
\href{https://www.nytimes.com/2017/01/23/world/americas/trump-pena-nieto-mexico-meeting.html}{bind
the countries} together, including migration, border security and drug
trafficking.

The United States depends on Mexico to fight drug cartels and stop
migrants from Central America and other regions who are trying to reach
the United States. On Wednesday, Mr. Trump is expected to sign an
executive order
\href{https://www.nytimes.com/2017/01/24/us/politics/wall-border-trump.html?ribbon-ad-idx=3\&rref=homepage\&module=Ribbon\&version=origin\&region=Header\&action=click\&contentCollection=Home\%20Page\&pgtype=article}{to
build a wall} on the border.

Mr. Peña Nieto's government is trying to present ``a package deal'' to
the Trump administration, in a tacit warning that the proposed wall
would be an inadequate replacement for Mexico's help on migration and
security.

The message Mexico hopes to deliver is that ``if you build your wall,
the wall will have to substitute everything that we used to do,'' said
Jorge Castañeda, a former Mexican foreign minister.

And by tying trade to security, Mexico will also be able to buy time in
negotiations, contended Rafael Fernández de Castro, an adviser to former
President Felipe Calderón.

More time could allow businesses in the United States that depend on
Nafta, along with states like Texas that send billions of dollars in
exports to Mexico, to make their own case for preserving the agreement.

But the suggestion that ardent free traders in Mexico are even
entertaining the idea of leaving Nafta is
\href{https://www.nytimes.com/2016/11/10/world/americas/mexico-donald-trump-peso.html}{evidence
of the turmoil} Mr. Trump's election has created south of the border.

During the election campaign, Mr. Trump called Nafta ``the single worst
trade deal ever approved in this country'' and promised to renegotiate
it or pull the United States out of it. He has threatened to impose a
``border tax'' on companies that move factories to other countries and
send their products back to the United States.

Mr. Trump built his campaign on a promise to return manufacturing from
lower-wage countries like Mexico to the depressed towns of America's
Rust Belt. Since the election, his warnings have already had
\href{https://www.nytimes.com/2017/01/05/world/americas/mexico-pena-nieto-trump-turmoil-gasoline.html}{a
paralyzing effect in Mexico}.

\includegraphics{https://static01.nyt.com/images/2017/01/25/us/25MEXICO2/25MEXICO2-articleInline.jpg?quality=75\&auto=webp\&disable=upscale}

The peso has sunk to record lows. The
\href{https://www.nytimes.com/2017/01/03/business/ford-general-motors-trump.html}{Ford
Motor Company has canceled} a \$1.6 billion factory, and General Motors
has announced it will send some auto parts work back to the United
States. Uncertainty is sure to stall future foreign direct investment.

So far, Mr. Peña Nieto has offered a ringing affirmation of Mexico's
commitment to free trade.

``We must preserve free trade among Canada, the United States and
Mexico,'' he said Monday, laying out Mexico's foreign policy in the face
of the Trump policy. ``Trade among the three countries must be free of
any tariff or quota.''

But the emerging belief that no Nafta may be preferable to years of
fractious negotiations came into the open during a discussion this month
at the elite university that is the brain trust of Mexico's free trade
gospel.

Many of Mexico's government ministers have studied or taught at the
university, the Autonomous Technological Institute of Mexico, or ITAM,
and some of those who were the architects of Nafta were on hand to hear
the rector, Arturo Fernández, say the unthinkable.

``It would be, perhaps, preferable to leave Nafta aside rather than a
long process of negotiation and tension,'' Mr. Fernández said.

Mr. Castañeda, the former foreign minister, who
\href{https://www.nytimes.com/2016/11/22/opinion/mexico-can-stand-up-to-trump.html}{proposed
in November} that Mexico should consider leaving Nafta if Washington
demanded a full-blown renegotiation, said that policy makers had come
around to the idea that ``they have a choice between a terrible Plan A
and a terrible Plan B.''

Many businesspeople have made peace with that idea.

``We have our own principles that we have to defend,'' said Juan Pablo
Castañón, the president of Mexico's Business Coordinating Council, a
coalition of business groups.

``If there aren't the right conditions, then we have to contemplate the
possibility of not staying within the treaty and working with
international rules instead.''

Without Nafta, trade between Mexico and the United States would be
governed by World Trade Organization rules, Mexican trade experts say.
Tariffs for imports of Mexican goods into the United States would
probably rise to an average of about 3 percent, an increase that experts
say is not enough to deter trade.

``It will not be the end of the world,'' said Mr. Fernández de Castro,
the former adviser. Knowing this, he said, means that ``Mexico can
negotiate standing up.''

Not everybody is convinced that the government is prepared to walk away
from the table.

``In reality, up to now this is a negotiating tactic and not a position
that they intend to take to its final consequences,'' said Carlos
Heredia, a professor at CIDE, a Mexico City university. ``Maybe I am
mistaken.''

Without much besides Mr. Trump's statements to go on, Mexicans have
tried to parse the
\href{https://www.nytimes.com/2017/01/18/us/politics/wilbur-ross-commerce-secretary-trump-trade-nafta.html}{Senate
testimony last week of Wilbur L. Ross Jr.}, Mr. Trump's nominee to lead
the Commerce Department, to understand what Washington's opening
position may be.

Luis de la Calle, a former Mexican trade negotiator, said renegotiation
may be limited to several technical issues, including the complex rules
allowing products like cars and electronics --- which contain parts from
all over the world --- to move duty-free through North America.

``We need to try and keep emotions out of the negotiations,'' he said.

Even with the barrage of Twitter messages and statements against Nafta
from Mr. Trump, Mexico may have leverage in any renegotiation because it
is such an important customer for American goods.

``On the trade side, we understand what Trump says, but on the other
hand, we have cards to play because we're the second-largest market,''
said Andrés Rozental, a former deputy foreign minister. ``We can find
much of what we buy from the U.S. elsewhere.''

Mr. de la Calle argued that Mexico will become the largest market for
American exports within five years, surpassing Canada. ``The structural
reasons for integration won't change,'' he said.

``We Mexicans have patience; we take the long view.''

Advertisement

\protect\hyperlink{after-bottom}{Continue reading the main story}

\hypertarget{site-index}{%
\subsection{Site Index}\label{site-index}}

\hypertarget{site-information-navigation}{%
\subsection{Site Information
Navigation}\label{site-information-navigation}}

\begin{itemize}
\tightlist
\item
  \href{https://help.nytimes.com/hc/en-us/articles/115014792127-Copyright-notice}{©~2020~The
  New York Times Company}
\end{itemize}

\begin{itemize}
\tightlist
\item
  \href{https://www.nytco.com/}{NYTCo}
\item
  \href{https://help.nytimes.com/hc/en-us/articles/115015385887-Contact-Us}{Contact
  Us}
\item
  \href{https://www.nytco.com/careers/}{Work with us}
\item
  \href{https://nytmediakit.com/}{Advertise}
\item
  \href{http://www.tbrandstudio.com/}{T Brand Studio}
\item
  \href{https://www.nytimes.com/privacy/cookie-policy\#how-do-i-manage-trackers}{Your
  Ad Choices}
\item
  \href{https://www.nytimes.com/privacy}{Privacy}
\item
  \href{https://help.nytimes.com/hc/en-us/articles/115014893428-Terms-of-service}{Terms
  of Service}
\item
  \href{https://help.nytimes.com/hc/en-us/articles/115014893968-Terms-of-sale}{Terms
  of Sale}
\item
  \href{https://spiderbites.nytimes.com}{Site Map}
\item
  \href{https://help.nytimes.com/hc/en-us}{Help}
\item
  \href{https://www.nytimes.com/subscription?campaignId=37WXW}{Subscriptions}
\end{itemize}
