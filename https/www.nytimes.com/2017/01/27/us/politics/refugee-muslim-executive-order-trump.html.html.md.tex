Sections

SEARCH

\protect\hyperlink{site-content}{Skip to
content}\protect\hyperlink{site-index}{Skip to site index}

\href{https://www.nytimes.com/section/politics}{Politics}

\href{https://myaccount.nytimes.com/auth/login?response_type=cookie\&client_id=vi}{}

\href{https://www.nytimes.com/section/todayspaper}{Today's Paper}

\href{/section/politics}{Politics}\textbar{}Full Executive Order Text:
Trump's Action Limiting Refugees Into the U.S.

\url{https://nyti.ms/2kcPWG8}

\begin{itemize}
\item
\item
\item
\item
\item
\end{itemize}

Advertisement

\protect\hyperlink{after-top}{Continue reading the main story}

Supported by

\protect\hyperlink{after-sponsor}{Continue reading the main story}

\hypertarget{full-executive-order-text-trumps-action-limiting-refugees-into-the-us}{%
\section{Full Executive Order Text: Trump's Action Limiting Refugees
Into the
U.S.}\label{full-executive-order-text-trumps-action-limiting-refugees-into-the-us}}

\includegraphics{https://static01.nyt.com/images/2017/01/28/us/28eo_web1/28eo_web1-articleInline.jpg?quality=75\&auto=webp\&disable=upscale}

Jan. 27, 2017

\begin{itemize}
\item
\item
\item
\item
\item
\end{itemize}

\emph{President Trump signed an executive order on Friday titled
``Protecting the Nation From Foreign Terrorist Entry Into the United
States.'' Following is the language of that order, as supplied by the
White House.}

By the authority vested in me as President by the Constitution and laws
of the United States of America, including the Immigration and
Nationality Act (INA), 8 U.S.C. 1101 \emph{et seq}., and section 301 of
title 3, United States Code, and to protect the American people from
terrorist attacks by foreign nationals admitted to the United States, it
is hereby ordered as follows:

Section 1. Purpose*.* The visa-issuance process plays a crucial role in
detecting individuals with terrorist ties and stopping them from
entering the United States. Perhaps in no instance was that more
apparent than the terrorist attacks of September 11, 2001, when State
Department policy prevented consular officers from properly scrutinizing
the visa applications of several of the 19 foreign nationals who went on
to murder nearly 3,000 Americans. And while the visa-issuance process
was reviewed and amended after the September 11 attacks to better detect
would-be terrorists from receiving visas, these measures did not stop
attacks by foreign nationals who were admitted to the United States.

Numerous foreign-born individuals have been convicted or implicated in
terrorism-related crimes since September 11, 2001, including foreign
nationals who entered the United States after receiving visitor,
student, or employment visas, or who entered through the United States
refugee resettlement program. Deteriorating conditions in certain
countries due to war, strife, disaster, and civil unrest increase the
likelihood that terrorists will use any means possible to enter the
United States. The United States must be vigilant during the
visa-issuance process to ensure that those approved for admission do not
intend to harm Americans and that they have no ties to terrorism.

In order to protect Americans, the United States must ensure that those
admitted to this country do not bear hostile attitudes toward it and its
founding principles. The United States cannot, and should not, admit
those who do not support the Constitution, or those who would place
violent ideologies over American law. In addition, the United States
should not admit those who engage in acts of bigotry or hatred
(including ``honor'' killings, other forms of violence against women, or
the persecution of those who practice religions different from their
own) or those who would oppress Americans of any race, gender, or sexual
orientation.

Sec. 2. Policy. It is the policy of the United States to protect its
citizens from foreign nationals who intend to commit terrorist attacks
in the United States; and to prevent the admission of foreign nationals
who intend to exploit United States immigration laws for malevolent
purposes.

Sec. 3. Suspension of Issuance of Visas and Other Immigration Benefits
to Nationals of Countries of Particular Concern*.* (a) The Secretary of
Homeland Security, in consultation with the Secretary of State and the
Director of National Intelligence, shall immediately conduct a review to
determine the information needed from any country to adjudicate any
visa, admission, or other benefit under the INA (adjudications) in order
to determine that the individual seeking the benefit is who the
individual claims to be and is not a security or public-safety threat.

(b) The Secretary of Homeland Security, in consultation with the
Secretary of State and the Director of National Intelligence, shall
submit to the President a report on the results of the review described
in subsection (a) of this section, including the Secretary of Homeland
Security's determination of the information needed for adjudications and
a list of countries that do not provide adequate information, within 30
days of the date of this order. The Secretary of Homeland Security shall
provide a copy of the report to the Secretary of State and the Director
of National Intelligence.

(c) To temporarily reduce investigative burdens on relevant agencies
during the review period described in subsection (a) of this section, to
ensure the proper review and maximum utilization of available resources
for the screening of foreign nationals, and to ensure that adequate
standards are established to prevent infiltration by foreign terrorists
or criminals, pursuant to section 212(f) of the INA, 8 U.S.C. 1182(f), I
hereby proclaim that the immigrant and nonimmigrant entry into the
United States of aliens from countries referred to in section 217(a)(12)
of the INA, 8 U.S.C. 1187(a)(12), would be detrimental to the interests
of the United States, and I hereby suspend entry into the United States,
as immigrants and nonimmigrants, of such persons for 90 days from the
date of this order (excluding those foreign nationals traveling on
diplomatic visas, North Atlantic Treaty Organization visas, C-2 visas
for travel to the United Nations, and G-1, G-2, G-3, and G-4 visas).

(d) Immediately upon receipt of the report described in subsection (b)
of this section regarding the information needed for adjudications, the
Secretary of State shall request all foreign governments that do not
supply such information to start providing such information regarding
their nationals within 60 days of notification.

(e) After the 60-day period described in subsection (d) of this section
expires, the Secretary of Homeland Security, in consultation with the
Secretary of State, shall submit to the President a list of countries
recommended for inclusion on a Presidential proclamation that would
prohibit the entry of foreign nationals (excluding those foreign
nationals traveling on diplomatic visas, North Atlantic Treaty
Organization visas, C-2 visas for travel to the United Nations, and G-1,
G-2, G-3, and G-4 visas) from countries that do not provide the
information requested pursuant to subsection (d) of this section until
compliance occurs.

(f) At any point after submitting the list described in subsection (e)
of this section, the Secretary of State or the Secretary of Homeland
Security may submit to the President the names of any additional
countries recommended for similar treatment.

(g) Notwithstanding a suspension pursuant to subsection (c) of this
section or pursuant to a Presidential proclamation described in
subsection (e) of this section, the Secretaries of State and Homeland
Security may, on a case-by-case basis, and when in the national
interest, issue visas or other immigration benefits to nationals of
countries for which visas and benefits are otherwise blocked.

(h) The Secretaries of State and Homeland Security shall submit to the
President a joint report on the progress in implementing this
orderwithin 30 days of the date of this order, a second report within 60
daysof the date of this order, a third report within 90 days of the date
of this order, and a fourth report within 120 days of the date of this
order.

Sec. 4. Implementing Uniform Screening Standards for All Immigration
Programs*.* (a) The Secretary of State, the Secretary of Homeland
Security, the Director of National Intelligence, and the Director of the
Federal Bureau of Investigation shall implement a program, as part of
the adjudication process for immigration benefits, to identify
individuals seeking to enter the United States on a fraudulent basis
with the intent to cause harm, or who are at risk of causing harm
subsequent to their admission. This program will include the development
of a uniform screening standard and procedure, such as in-person
interviews; a database of identity documents proffered by applicants to
ensure that duplicate documents are not used by multiple applicants;
amended application forms that include questions aimed at identifying
fraudulent answers and malicious intent; a mechanism to ensure that the
applicant is who the applicant claims to be; a process to evaluate the
applicant's likelihood of becoming a positively contributing member of
society and the applicant's ability to make contributions to the
national interest; and a mechanism to assess whether or not the
applicant has the intent to commit criminal or terrorist acts after
entering the United States.

(b) The Secretary of Homeland Security, in conjunction with the
Secretary of State, the Director of National Intelligence, and the
Director of the Federal Bureau of Investigation, shall submit to the
President an initial report on the progress of this directive within 60
days of the date of this order, a second report within 100 days of the
date of this order, and a third report within 200 days of the date of
this order.

Sec. 5. Realignment of the U.S. Refugee Admissions Program for Fiscal
Year 2017*.* (a) The Secretary of State shall suspend the U.S. Refugee
Admissions Program (USRAP) for 120 days. During the 120-day period, the
Secretary of State, in conjunction with the Secretary of Homeland
Security and in consultation with the Director of National Intelligence,
shall review the USRAP application and adjudication process to determine
what additional procedures should be taken to ensure that those approved
for refugee admission do not pose a threat to the security and welfare
of the United States, and shall implement such additional procedures.
Refugee applicants who are already in the USRAP process may be admitted
upon the initiation and completion of these revised procedures. Upon the
date that is 120 days after the date of this order, the Secretary of
State shall resume USRAP admissions only for nationals of countries for
which the Secretary of State, the Secretary of Homeland Security, and
the Director of National Intelligence have jointly determined that such
additional procedures are adequate to ensure the security and welfare of
the United States.

(b) Upon the resumption of USRAP admissions, the Secretary of State, in
consultation with the Secretary of Homeland Security, is further
directed to make changes, to the extent permitted by law, to prioritize
refugee claims made by individuals on the basis of religious-based
persecution, provided that the religion of the individual is a minority
religion in the individual's country of nationality. Where necessary and
appropriate, the Secretaries of State and Homeland Security shall
recommend legislation to the President that would assist with such
prioritization.

(c) Pursuant to section 212(f) of the INA, 8 U.S.C. 1182(f), I hereby
proclaim that the entry of nationals of Syria as refugees is detrimental
to the interests of the United States and thus suspend any such entry
until such time as I have determined that sufficient changes have been
made to the USRAP to ensure that admission of Syrian refugees is
consistent with the national interest.

(d) Pursuant to section 212(f) of the INA, 8 U.S.C. 1182(f), I hereby
proclaim that the entry of more than 50,000 refugees in fiscal year 2017
would be detrimental to the interests of the United States, and thus
suspend any such entry until such time as I determine that additional
admissions would be in the national interest.

(e) Notwithstanding the temporary suspension imposed pursuant to
subsection (a) of this section, the Secretaries of State and Homeland
Security may jointly determine to admit individuals to the United States
as refugees on a case-by-case basis, in their discretion, but only so
long as they determine that the admission of such individuals as
refugees is in the national interest --- including when the person is a
religious minority in his country of nationality facing religious
persecution, when admitting the person would enable the United States to
conform its conduct to a preexisting international agreement, or when
the person is already in transit and denying admission would cause undue
hardship --- and it would not pose a risk to the security or welfare of
the United States.

(f) The Secretary of State shall submit to the President an initial
report on the progress of the directive in subsection (b) of this
section regarding prioritization of claims made by individuals on the
basis of religious-based persecution within 100 days of the date of this
order and shall submit a second report within 200 days of the date of
this order.

(g) It is the policy of the executive branch that, to the extent
permitted by law and as practicable, State and local jurisdictions be
granted a role in the process of determining the placement or settlement
in their jurisdictions of aliens eligible to be admitted to the United
States as refugees. To that end, the Secretary of Homeland Security
shall examine existing law to determine the extent to which, consistent
with applicable law, State and local jurisdictions may have greater
involvement in the process of determining the placement or resettlement
of refugees in their jurisdictions, and shall devise a proposal to
lawfully promote such involvement.

Sec. 6. Rescission of Exercise of Authority Relating to the Terrorism
Grounds of Inadmissibility*.* The Secretaries of State and Homeland
Security shall, in consultation with the Attorney General, consider
rescinding the exercises of authority in section 212 of the INA, 8
U.S.C. 1182, relating to the terrorism grounds of inadmissibility, as
well as any related implementing memoranda.

Sec. 7. Expedited Completion of the Biometric Entry-Exit Tracking
System. (a) The Secretary of Homeland Security shall expedite the
completion and implementation of a biometric entry-exit tracking system
for all travelers to the United States, as recommended by the National
Commission on Terrorist Attacks Upon the United States.

(b) The Secretary of Homeland Security shall submit to the President
periodic reports on the progress of the directive contained in
subsection (a) of this section. The initial report shall be submitted
within 100 days of the date of this order, a second report shall be
submitted within 200 days of the date of this order, and a third report
shall be submitted within 365 days of the date of this order. Further,
the Secretary shall submit a report every 180 days thereafter until the
system is fully deployed and operational.

Sec. 8. Visa Interview Security*.* (a) The Secretary of State shall
immediately suspend the Visa Interview Waiver Program and ensure
compliance with section 222 of the INA, 8 U.S.C. 1222, which requires
that all individuals seeking a nonimmigrant visa undergo an in-person
interview, subject to specific statutory exceptions.

(b) To the extent permitted by law and subject to the availability of
appropriations, the Secretary of State shall immediately expand the
Consular Fellows Program, including by substantially increasing the
number of Fellows, lengthening or making permanent the period of
service, and making language training at the Foreign Service Institute
available to Fellows for assignment to posts outside of their area of
core linguistic ability, to ensure that non-immigrant visa-interview
wait times are not unduly affected.

Sec. 9. Visa Validity Reciprocity*.* The Secretary of State shall review
all nonimmigrant visa reciprocity agreements to ensure that they are,
with respect to each visa classification, truly reciprocal insofar as
practicable with respect to validity period and fees, as required by
sections 221(c) and 281 of the INA, 8 U.S.C. 1201(c) and 1351, and other
treatment. If a country does not treat United States nationals seeking
nonimmigrant visas in a reciprocal manner, the Secretary of State shall
adjust the visa validity period, fee schedule, or other treatment to
match the treatment of United States nationals by the foreign country,
to the extent practicable.

Sec. 10. Transparency and Data Collection*.* (a) To be more transparent
with the American people, and to more effectively implement policies and
practices that serve the national interest, the Secretary of Homeland
Security, in consultation with the Attorney General, shall, consistent
with applicable law and national security, collect and make publicly
available within 180 days, and every 180 days thereafter:

(i) information regarding the number of foreign nationals in the United
States who have been charged with terrorism-related offenses while in
the United States; convicted of terrorism-related offenses while in the
United States; or removed from the United States based on
terrorism-related activity, affiliation, or material support to a
terrorism-related organization, or any other national security reasons
since the date of this order or the last reporting period, whichever is
later;

(ii) information regarding the number of foreign nationals in the United
States who have been radicalized after entry into the United States and
engaged in terrorism-related acts, or who have provided material support
to terrorism-related organizations in countries that pose a threat to
the United States, since the date of this order or the last reporting
period, whichever is later; and

(iii) information regarding the number and types of acts of gender-based
violence against women, including honor killings, in the United States
by foreign nationals, since the date of this order or the last reporting
period, whichever is later; and

(iv) any other information relevant to public safety and security as
determined by the Secretary of Homeland Security and the Attorney
General, including information on the immigration status of foreign
nationals charged with major offenses.

(b) The Secretary of State shall, within one year of the date of this
order, provide a report on the estimated long-term costs of the USRAP at
the Federal, State, and local levels.

Sec. 11. General Provisions*.* (a) Nothing in this order shall be
construed to impair or otherwise affect:

(i) the authority granted by law to an executive department or agency,
or the head thereof; or

(ii) the functions of the Director of the Office of Management and
Budget relating to budgetary, administrative, or legislative proposals.

(b) This order shall be implemented consistent with applicable law and
subject to the availability of appropriations.

(c) This order is not intended to, and does not, create any right or
benefit, substantive or procedural, enforceable at law or in equity by
any party against the United States, its departments, agencies, or
entities, its officers, employees, or agents, or any other person.

Advertisement

\protect\hyperlink{after-bottom}{Continue reading the main story}

\hypertarget{site-index}{%
\subsection{Site Index}\label{site-index}}

\hypertarget{site-information-navigation}{%
\subsection{Site Information
Navigation}\label{site-information-navigation}}

\begin{itemize}
\tightlist
\item
  \href{https://help.nytimes.com/hc/en-us/articles/115014792127-Copyright-notice}{©~2020~The
  New York Times Company}
\end{itemize}

\begin{itemize}
\tightlist
\item
  \href{https://www.nytco.com/}{NYTCo}
\item
  \href{https://help.nytimes.com/hc/en-us/articles/115015385887-Contact-Us}{Contact
  Us}
\item
  \href{https://www.nytco.com/careers/}{Work with us}
\item
  \href{https://nytmediakit.com/}{Advertise}
\item
  \href{http://www.tbrandstudio.com/}{T Brand Studio}
\item
  \href{https://www.nytimes.com/privacy/cookie-policy\#how-do-i-manage-trackers}{Your
  Ad Choices}
\item
  \href{https://www.nytimes.com/privacy}{Privacy}
\item
  \href{https://help.nytimes.com/hc/en-us/articles/115014893428-Terms-of-service}{Terms
  of Service}
\item
  \href{https://help.nytimes.com/hc/en-us/articles/115014893968-Terms-of-sale}{Terms
  of Sale}
\item
  \href{https://spiderbites.nytimes.com}{Site Map}
\item
  \href{https://help.nytimes.com/hc/en-us}{Help}
\item
  \href{https://www.nytimes.com/subscription?campaignId=37WXW}{Subscriptions}
\end{itemize}
