Sections

SEARCH

\protect\hyperlink{site-content}{Skip to
content}\protect\hyperlink{site-index}{Skip to site index}

\href{https://www.nytimes.com/section/world/europe}{Europe}

\href{https://myaccount.nytimes.com/auth/login?response_type=cookie\&client_id=vi}{}

\href{https://www.nytimes.com/section/todayspaper}{Today's Paper}

\href{/section/world/europe}{Europe}\textbar{}Donald Trump and Theresa
May: An Odd Couple

\url{https://nyti.ms/2k9XsS6}

\begin{itemize}
\item
\item
\item
\item
\item
\item
\end{itemize}

Advertisement

\protect\hyperlink{after-top}{Continue reading the main story}

Supported by

\protect\hyperlink{after-sponsor}{Continue reading the main story}

\hypertarget{donald-trump-and-theresa-may-an-odd-couple}{%
\section{Donald Trump and Theresa May: An Odd
Couple}\label{donald-trump-and-theresa-may-an-odd-couple}}

\includegraphics{https://static01.nyt.com/images/2017/01/27/world/27Britain1/27Britain1-articleInline.jpg?quality=75\&auto=webp\&disable=upscale}

By \href{http://www.nytimes.com/by/steven-erlanger}{Steven Erlanger}

\begin{itemize}
\item
  Jan. 27, 2017
\item
  \begin{itemize}
  \item
  \item
  \item
  \item
  \item
  \item
  \end{itemize}
\end{itemize}

LONDON --- Prime Minister Theresa May of Britain will meet President
Trump on Friday in Washington for what could be an episode of ``The Odd
Couple'': The Stiff Headmistress meets the Great Salesman.

Reserved, slightly awkward and serious, Mrs. May does not even have a
Twitter account and does her best to remain silent on the key issues of
the day, putting her head above water only when she must.

Normally, American presidents go on to British leaders about ``the
special relationship'' with a sort of patronizing politeness. But Mr.
Trump has already put Mrs. May's teeth on edge with his cheerful support
for a British withdrawal from the European Union, commonly known as
Brexit, which she opposed but must carry out.

She has not appreciated his warm relationship with those like Nigel
Farage, the former leader of the anti-immigrant U.K. Independence Party,
who despises Mrs. May's Conservative Party and who Mr. Trump has
suggested would make a fine ambassador to the United States.

Still, with Britain planning to leave the European Union in two years or
so, Mrs. May needs to show Britons they have big friends out in the
world beyond Europe, and the United States is already Britain's single
largest trading partner, not counting the European Union itself.

Having put such store into being the first foreign leader to meet
President Trump --- Mr. Farage and Arron Banks, UKIP's main financial
backer, have met him several times since the election --- Mrs. May is
determined to put British-American relations on a more traditional
track, as a government-to-government partnership.

But hardly one of equals. Mrs. May ``is coming as a supplicant and Trump
seems to know this,'' Jonathan Freedland, a columnist for The Guardian
newspaper, said in an interview. On trade, ``she's eager to do a deal,
like a house buyer who has already sold her house and has nowhere to
live, and Trump, the real estate man, knows that.''

Mrs. May, he said, ``is the un-Trump.'' Even in ``the comparably
unglitzy realm of British politics, she is unshowy,'' he said. Compared
with her predecessor, David Cameron, ``she is pretty gray and pallid.''
Still, he said, ``history shows that personal chemistry does matter.''

Christopher Meyer, a former British ambassador to Washington, said that
``they look like the odd couple, but you never know --- what's required
is a good working relationship.'' John Major and Bill Clinton were also
an odd couple, he said --- ``there was nothing there at all,'' and
Barack Obama and Gordon Brown never clicked.

As a gift to Mr. Trump, Mrs. May is bringing a quaich (pronounced as
quake), a two-handled Scottish drinking cup for whisky used to symbolize
trust between the giver and recipient. Although intensely proud of his
Scottish ancestry, and his Scottish golf courses, Mr. Trump is a
teetotaler.

The early meeting is important for Mrs. May, Mr. Meyer said, because
``she can put in a British fix on the things that bother us --- NATO,
Putin, security, trade.''

What has upset the British government is Mr. Trump's ``being nasty to
NATO and nice to Putin,'' Mr. Meyer said. But after calming words about
NATO from Defense Secretary James N. Mattis, praise for President
Vladimir V. Putin of Russia is less problematic, he said. ``But May
would like to hear that from the horse's mouth.''

Mrs. May comes with an agenda, her spokeswoman, unidentified under
traditional ground rules, said on Wednesday. Her goal is ``to meet face
to face and establish a personal relationship that leads to an
effective, productive working relationship, not just in the interests of
the U.K. and the U.S., but facing the many global challenges where we
can work together.''

Those include, the spokeswoman said, a start on talks for a bilateral
trade deal post-Brexit, but also a ``frank'' discussion of the
importance of the NATO alliance, which Mr. Trump has sometimes
disparaged; of the European Union (ditto), even though Britain is
leaving it; of Russia and its violations of international law in Crimea
and eastern Ukraine, ``where we've taken a strong position'' and to
which ``we will remain committed.''

Asked about Mr. Trump's more sexist comments, Mrs. May
\href{https://www.nytimes.com/2017/01/22/world/europe/uk-theresa-may-trident-missile.html?_r=0}{has
said} that ``some of the comments that Donald Trump has made in relation
to women are unacceptable, some of those he himself has apologized
for.''

\includegraphics{https://static01.nyt.com/images/2017/01/27/world/27Britain2/27Britain2-articleInline.jpg?quality=75\&auto=webp\&disable=upscale}

When she meets Mr. Trump, she said, ``I think the biggest statement that
will be made about the role of women is the fact that I will be there as
a female prime minister.''

Jeremy Shapiro, a former State Department official who is the director
of research for the European Council on Foreign Relations, said that
Mrs. May had to be careful because Mr. Trump almost never has fights
with someone in the room.

``Then you think that maybe this isn't the person I thought he was, but
48 hours later he tweets something and disappoints you,'' he said.

Mrs. May may be aware that she is a supplicant, Mr. Shapiro said, ``but
Trump has her boxed up in her domestic politics --- the problem of
Farage, her need to control the Brexit wing of her party and her need to
fashion a Brexit that won't destroy her prime ministership.''

Mr. Trump has made reference to the warm, vital relationship between
President Ronald Reagan and Margaret Thatcher. ``But they were an actual
team, they actually worked together, and Trump can't stand that,'' Mr.
Shapiro said.

Mr. Freedland said that the Reagan-Thatcher connection mattered,
``because there was extra leeway and space for both of them, because of
the personal relationship.''

The new president, Mr. Shapiro suggested, will see Mrs. May's desire to
meet him first as a sign of weakness. ``There's no way Trump will say it
that way face-to-face, but later it will come through in the
relationship and in any U.S.-U.K. trade deal,'' which he expects will
not be particularly favorable to Britain.

For Mr. Trump, he suggested, those leaders who do not ask for early
meetings are the ones who show the most strength.

Mr. Meyer, the former ambassador, is less concerned. ``She's completely
aware of the dangers, and I think she might turn out to be a bit of an
iron lady in some of what she says,'' he said. ``She won't sound like a
supplicant. But getting the balance right between saying all the
oleaginous things about the special relationship and saying the things
that matter to us will be the key.''

Mrs. May also addressed Republicans in Philadelphia on Thursday at their
annual retreat, which Mr. Trump also attended, before meeting him at the
White House on Friday afternoon. Then she flies to Ankara, Turkey, for a
meeting with President Recep Tayyip Erdogan.

On issues of trade, terrorism, migration, security, NATO and Cyprus,
Mrs. May's spokeswoman said, Turkey, too, ``is such an important
partner.''

Mr. Trump has emphasized his affection for Britain and for Brexit by
returning a bust of Winston Churchill to the Oval Office.

Mr. Obama's replacement of the bust by one of Martin Luther King Jr.
became an issue in Britain before Brexit, with the current foreign
secretary, Boris Johnson,
\href{https://www.nytimes.com/2016/04/23/world/europe/obama-britain-visit.html?_r=0}{claiming}
that Mr. Obama removed the Churchill bust because he ``is a symbol of
the part-Kenyan president's ancestral dislike of the British Empire.''

As a personal gesture after Christmas, Mrs. May sent Mr. Trump a copy of
Churchill's famous
\href{http://www.dailymail.co.uk/news/article-4123924/Special-relationship-inspired-famous-British-American-Winston-Churchill-tells-Trump-letter-echoing-wartime-leader-s-famous-speech.html}{speech}
to the American people on Christmas Eve 1941, weeks after Japan's attack
on Pearl Harbor brought the United States into the war.

In her letter, she told Mr. Trump that ``the sentiment'' Churchill had
expressed --- ``of a sense of unity and fraternal association between
the United Kingdom and United States --- is just as true today as it has
ever been.''

Maybe. Maybe not.

Advertisement

\protect\hyperlink{after-bottom}{Continue reading the main story}

\hypertarget{site-index}{%
\subsection{Site Index}\label{site-index}}

\hypertarget{site-information-navigation}{%
\subsection{Site Information
Navigation}\label{site-information-navigation}}

\begin{itemize}
\tightlist
\item
  \href{https://help.nytimes.com/hc/en-us/articles/115014792127-Copyright-notice}{©~2020~The
  New York Times Company}
\end{itemize}

\begin{itemize}
\tightlist
\item
  \href{https://www.nytco.com/}{NYTCo}
\item
  \href{https://help.nytimes.com/hc/en-us/articles/115015385887-Contact-Us}{Contact
  Us}
\item
  \href{https://www.nytco.com/careers/}{Work with us}
\item
  \href{https://nytmediakit.com/}{Advertise}
\item
  \href{http://www.tbrandstudio.com/}{T Brand Studio}
\item
  \href{https://www.nytimes.com/privacy/cookie-policy\#how-do-i-manage-trackers}{Your
  Ad Choices}
\item
  \href{https://www.nytimes.com/privacy}{Privacy}
\item
  \href{https://help.nytimes.com/hc/en-us/articles/115014893428-Terms-of-service}{Terms
  of Service}
\item
  \href{https://help.nytimes.com/hc/en-us/articles/115014893968-Terms-of-sale}{Terms
  of Sale}
\item
  \href{https://spiderbites.nytimes.com}{Site Map}
\item
  \href{https://help.nytimes.com/hc/en-us}{Help}
\item
  \href{https://www.nytimes.com/subscription?campaignId=37WXW}{Subscriptions}
\end{itemize}
