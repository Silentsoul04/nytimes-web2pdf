Sections

SEARCH

\protect\hyperlink{site-content}{Skip to
content}\protect\hyperlink{site-index}{Skip to site index}

\href{https://www.nytimes.com/section/politics}{Politics}

\href{https://myaccount.nytimes.com/auth/login?response_type=cookie\&client_id=vi}{}

\href{https://www.nytimes.com/section/todayspaper}{Today's Paper}

\href{/section/politics}{Politics}\textbar{}Trump and Putin Connect, but
Avoid Talk of Lifting U.S. Sanctions

\url{https://nyti.ms/2jCT255}

\begin{itemize}
\item
\item
\item
\item
\item
\end{itemize}

Advertisement

\protect\hyperlink{after-top}{Continue reading the main story}

Supported by

\protect\hyperlink{after-sponsor}{Continue reading the main story}

\hypertarget{trump-and-putin-connect-but-avoid-talk-of-lifting-us-sanctions}{%
\section{Trump and Putin Connect, but Avoid Talk of Lifting U.S.
Sanctions}\label{trump-and-putin-connect-but-avoid-talk-of-lifting-us-sanctions}}

\includegraphics{https://static01.nyt.com/images/2017/01/29/us/29Diplo-1/29Diplo-1485640586834-articleInline.jpg?quality=75\&auto=webp\&disable=upscale}

By \href{http://www.nytimes.com/by/peter-baker}{Peter Baker} and
\href{http://www.nytimes.com/by/andrew-e-kramer}{Andrew E. Kramer}

\begin{itemize}
\item
  Jan. 28, 2017
\item
  \begin{itemize}
  \item
  \item
  \item
  \item
  \item
  \end{itemize}
\end{itemize}

WASHINGTON ---
\href{https://www.nytimes.com/topic/person/donald-trump}{President
Trump} began a new era of diplomacy with
\href{https://www.nytimes.com/topic/destination/russia?inline=nyt-geo}{Russia}
on Saturday as he and President
\href{https://www.nytimes.com/topic/person/vladimir-putin}{Vladimir V.
Putin} conducted an hourlong telephone call, and vowed to repair
relations between the countries after nearly three years of conflict
that threatened a new Cold War between East and West.

The two leaders discussed fighting terrorism and expanding economic
ties, but barely mentioned the wedge that has been driven between
Washington and Moscow since Russia annexed Crimea and sponsored a
separatist war in eastern Ukraine in 2014. Still, although Mr. Trump had
previously expressed a willingness to lift sanctions against Russia, the
issue did not come up, according to officials on both sides.

The tone of the conversation was reported to be warm, indicating a
drastic shift after relations had broken down between Mr. Putin and
former President Barack Obama. ``The positive call was a significant
start to improving the relationship between the United States and Russia
that is in need of repair,'' the Trump administration said in a
statement. ``Both President Trump and President Putin are hopeful that
after today's call, the two sides can move quickly to tackle terrorism
and other important issues of mutual concern.''

In its statement, the Kremlin said: ``Donald Trump asked to convey a
desire for happiness and prosperity for the Russian people, noting that
the people in America relate with sympathy to Russia and its citizens.''
Mr. Putin answered that Russians feel the same way about Americans, the
statement said. Neither side mentioned the Russian hacking of the
American election in their statements.

Over the past two days, Mr. Trump has also had a series of conversations
with the United States' traditional European allies, but those calls
were seemingly not as congenial. After a meeting on Friday with Prime
Minister Theresa May of
\href{https://www.nytimes.com/topic/destination/great-britain}{Britain},
in which she
\href{https://www.nytimes.com/2017/01/27/world/europe/theresa-may-britain-trump.html?_r=0}{warned
against removing sanctions} on Russia, Mr. Trump had on Saturday what
appeared to be a businesslike call with Chancellor
\href{https://www.nytimes.com/topic/person/angela-merkel}{Angela Merkel}
of
\href{https://www.nytimes.com/topic/destination/germany?inline=nyt-geo}{Germany},
and a testier call with President
\href{https://www.nytimes.com/topic/person/francois-hollande}{François
Hollande} of
\href{https://www.nytimes.com/topic/destination/france}{France}.

Mr. Hollande's office said the French president pressed Mr. Trump not to
lift sanctions against Russia and to respect the nuclear agreement with
Iran. He asserted the importance of the Paris climate change pact,
warned of the consequences of protectionism, and added that democratic
values included welcoming refugees --- all in reaction to Mr. Trump's
first week of policy moves. Mr. Hollande also emphasized the importance
of NATO and the United Nations, both of which Mr. Trump has disparaged.

The call came just hours after Mr. Hollande, at a conference in Lisbon,
said that European countries had to stand together against Mr. Trump,
and assert their common values.

``We have to be conscious of the responsibility we have,'' he told
reporters, according to Agence France-Presse. ``Europe is a force,
Europe is a guarantee, Europe is protection, and Europe is also a space
for liberty and democracy.''

Ms. Merkel, who has reacted coolly to Mr. Trump's rise, emphasized the
importance of NATO in their own 45-minute call with the president on
Saturday. The two sides later released similar statements affirming the
American commitment to the alliance, even as they noted that allies
needed to increase their contributions, as Mr. Trump has demanded.

But the tone was spare and lacked warmth. That the statements included a
description of what has been a cornerstone of policy for decades
illustrated just how rattled trans-Atlantic relations have been since
Mr. Trump
\href{https://www.nytimes.com/2017/01/15/world/europe/donald-trump-nato.html}{dismissed
NATO as ``obsolete''} shortly before his inauguration.

Mr. Trump's conversation with Mr. Putin was their first direct
discussion since the inauguration a week ago and aimed at setting the
groundwork for a possible meeting.

American intelligence agencies have concluded that Russia made a
concerted effort --- through hacking, propaganda and other means --- to
influence the November election in favor of Mr. Trump, a conclusion Mr.
Trump initially refused to accept until he received a detailed briefing.
Intelligence officials also briefed both Mr. Obama and his successor on
a
\href{https://www.nytimes.com/2017/01/11/us/politics/trump-intelligence-report-explainer.html}{mysterious
dossier} compiled by political foes of Mr. Trump that included
unsubstantiated claims that Russia had collected compromising
information on the future president.

The Federal Bureau of Investigation reviewed an
\href{https://www.nytimes.com/2017/01/13/us/politics/donald-trump-transition.html}{intercepted
postelection conversation} between
\href{https://www.nytimes.com/2016/12/03/us/politics/in-national-security-adviser-michael-flynn-experience-meets-a-prickly-past.html}{Michael
T. Flynn}, now the national security adviser to Mr. Trump, and Sergei I.
Kislyak, the Russian ambassador. American authorities are also
\href{https://www.nytimes.com/2017/01/19/us/politics/trump-russia-associates-investigation.html}{examining
possible links} between Russian officials and associates of Mr. Trump,
including his former campaign chairman, Paul Manafort.

Mr. Trump's phone call with Mr. Putin came after news reports in Moscow
that two Russian intelligence officers who had worked on
cyberoperations, as well as a Russian computer security expert, had been
\href{https://www.nytimes.com/2017/01/27/world/europe/russia-hacking-us-election.html?ref=todayspaper}{arrested
on treason charges}. One of the Federal Security Service officers
detained has been accused of providing information to the United States,
according to Novaya Gazeta, a Russian opposition newspaper.

The United States and Europe have imposed an array of sanctions on
Russia for its intervention in Ukraine and for human rights abuses at
home. Before leaving office, Mr. Obama imposed additional sanctions and
expelled 35 Russian diplomats in retaliation for that country's
interference in American elections. Lifting them, critics said, would
cause a rift between the United States and Europe.

``That's going to lead to disunity and exactly what Putin wants,'' said
Michael McFaul, who served as ambassador to Russia under Mr. Obama.
``What could be better for him? Not only the act of sanctions' being
lifted but the process of disunity and disarray within the European
Union. That's a giant gift to him.''

Senate Republicans may have slowed Mr. Trump on sanctions after Senator
Mitch McConnell of Kentucky, the majority leader, said he would favor
bipartisan legislation requiring that sanctions stay in place. Mr. Trump
tried to play down the prospect at a news conference with Mrs. May on
Friday, saying it was ``very early to be talking about'' sanctions
relief. But he did not disavow it, and government officials said there
had been discussion of what to do on sanctions since Mr. Trump's team
took over.

Joining Mr. Trump in the Oval Office for his phone call with Mr. Putin
was Vice President
\href{http://topics.nytimes.com/top/reference/timestopics/people/p/mike_pence/index.html}{Mike
Pence}; Reince Priebus, the White House chief of staff; Stephen K.
Bannon, the president's chief strategist; Sean Spicer, the White House
press secretary; and Mr. Flynn.

The Kremlin said the two leaders talked about the Middle East, including
the Syrian civil war; the
\href{https://www.nytimes.com/topic/organization/islamic-state}{Islamic
State}; the Israeli-Palestinian conflict; and Iran's nuclear program,
although it gave no specifics. North Korea and strategic nuclear arms
came up, as did ``key aspects of the Ukraine crisis,'' the statement
said.

``It was agreed to cooperate on these and other issues,'' the Russian
statement said, ``but the priority was placed on uniting forces in the
fight against the main threat --- international terrorism.''

But an American official, who was briefed on the call but not authorized
to discuss it publicly, said Ukraine did not come up in any real detail,
and sanctions did not come up at all.

Still, a leading member of Russia's Parliament, Dmitri Novikov, said
that mention of bolstering trade ties suggested that sanctions relief
was coming. ``To fully develop economic ties, it's necessary to create
the right climate and legal conditions, and that requires canceling
sanctions,'' he told the Interfax news agency.

Others seemed to be expecting the same. Soon after the call, a Russian
government investment fund, the Russian Direct Investment Fund,
announced it had arranged more than 10 projects it hoped could draw
American investment to Russia, and that it planned to open an office in
New York in May.

Mr. Trump also spoke on Saturday with Prime Minister Malcolm Turnbull of
Australia and Prime Minister Shinzo Abe of
\href{https://www.nytimes.com/topic/destination/japan}{Japan}. He
invited Mr. Abe to visit the White House on Feb. 10. ``President Trump
affirmed the ironclad U.S. commitment to ensuring the security of
Japan,'' the administration said in a statement, a commitment that had
been in doubt among some in Tokyo after some of Mr. Trump's statements
during the campaign.

Advertisement

\protect\hyperlink{after-bottom}{Continue reading the main story}

\hypertarget{site-index}{%
\subsection{Site Index}\label{site-index}}

\hypertarget{site-information-navigation}{%
\subsection{Site Information
Navigation}\label{site-information-navigation}}

\begin{itemize}
\tightlist
\item
  \href{https://help.nytimes.com/hc/en-us/articles/115014792127-Copyright-notice}{©~2020~The
  New York Times Company}
\end{itemize}

\begin{itemize}
\tightlist
\item
  \href{https://www.nytco.com/}{NYTCo}
\item
  \href{https://help.nytimes.com/hc/en-us/articles/115015385887-Contact-Us}{Contact
  Us}
\item
  \href{https://www.nytco.com/careers/}{Work with us}
\item
  \href{https://nytmediakit.com/}{Advertise}
\item
  \href{http://www.tbrandstudio.com/}{T Brand Studio}
\item
  \href{https://www.nytimes.com/privacy/cookie-policy\#how-do-i-manage-trackers}{Your
  Ad Choices}
\item
  \href{https://www.nytimes.com/privacy}{Privacy}
\item
  \href{https://help.nytimes.com/hc/en-us/articles/115014893428-Terms-of-service}{Terms
  of Service}
\item
  \href{https://help.nytimes.com/hc/en-us/articles/115014893968-Terms-of-sale}{Terms
  of Sale}
\item
  \href{https://spiderbites.nytimes.com}{Site Map}
\item
  \href{https://help.nytimes.com/hc/en-us}{Help}
\item
  \href{https://www.nytimes.com/subscription?campaignId=37WXW}{Subscriptions}
\end{itemize}
