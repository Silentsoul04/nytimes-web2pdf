Sections

SEARCH

\protect\hyperlink{site-content}{Skip to
content}\protect\hyperlink{site-index}{Skip to site index}

\href{https://www.nytimes.com/section/world/europe}{Europe}

\href{https://myaccount.nytimes.com/auth/login?response_type=cookie\&client_id=vi}{}

\href{https://www.nytimes.com/section/todayspaper}{Today's Paper}

\href{/section/world/europe}{Europe}\textbar{}For Leaders of U.S.
Allies, Getting Close to Trump Can Sting

\url{https://nyti.ms/2jLnJoN}

\begin{itemize}
\item
\item
\item
\item
\item
\end{itemize}

Advertisement

\protect\hyperlink{after-top}{Continue reading the main story}

Supported by

\protect\hyperlink{after-sponsor}{Continue reading the main story}

\hypertarget{for-leaders-of-us-allies-getting-close-to-trump-can-sting}{%
\section{For Leaders of U.S. Allies, Getting Close to Trump Can
Sting}\label{for-leaders-of-us-allies-getting-close-to-trump-can-sting}}

\includegraphics{https://static01.nyt.com/images/2017/01/31/world/31Burned1/31Burned1-articleInline.jpg?quality=75\&auto=webp\&disable=upscale}

By \href{http://www.nytimes.com/by/steven-erlanger}{Steven Erlanger}

\begin{itemize}
\item
  Jan. 30, 2017
\item
  \begin{itemize}
  \item
  \item
  \item
  \item
  \item
  \end{itemize}
\end{itemize}

LONDON --- It had all been going so well.

Prime Minister Theresa May of Britain had just left Washington on Friday
evening after a tense but successful first visit with President Trump
for a 10-hour flight to Ankara, Turkey, for her next awkward encounter,
with the increasingly autocratic Turkish president, Recep Tayyip
Erdogan.

By the time she had landed in Turkey, however, Mr. Trump had
\href{https://www.nytimes.com/2017/01/29/us/politics/white-house-official-in-reversal-says-green-card-holders-wont-be-barred.html}{signed
his executive order} halting entrance to the United States of all Syrian
refugees and of most citizens from seven predominantly Muslim countries.
Mrs. May was beginning to feel the backlash.

After she termed the executive order an American issue, criticism
erupted even among her own members of Parliament. She was
\href{https://twitter.com/TFletcher/status/825405947333976065}{accused
of appeasement} by a former British diplomat. Protesters gathered
outside Downing Street on Monday night, and more than 1.5 million
signatures collected on an internet petition demanding that Mrs. May
\href{https://petition.parliament.uk/petitions/171928}{rescind her
invitation} for Mr. Trump to visit Queen Elizabeth II.

A close relationship with any American president is regarded as crucial
by allies and foes alike, but especially by intimates like Britain,
Canada, Japan and Mexico. Yet like moths to the flame, the leaders of
those nations are finding that they draw close at their peril.

While Mrs. May is the latest prominent figure to suffer repercussions
for her handling of Mr. Trump, the leaders of those other three close
allies have also felt the sting of public anger soon after what seemed
to be friendly telephone calls or encounters. They then find themselves
facing a no-win situation, either openly criticizing the leader of their
superpower ally or pulling their punches and risking severe criticism at
home.

One Western leader to escape this fate so far is the German chancellor,
Angela Merkel, who has kept a cool distance from Mr. Trump. In a
telephone call on Saturday, she reminded him of Washington's obligations
under the Geneva Conventions to accept refugees fleeing war, a view
underlined by her official spokesman.

The danger of playing nice with Mr. Trump should come as little surprise
to his country's allies. Besides campaigning on an ``America First''
platform, he has regularly argued that allies have been taking the
United States for a ride, in trade, security and financial terms.

While he has been cordial in public settings with the leaders of those
allied nations, Mr. Trump has turned on them soon afterward.

``The problem for May is that Trump doesn't value relationships. He
values strength and winning,'' said Jeremy Shapiro, the director of
research at the European Council on Foreign Relations and a former
senior State Department official. ``If you rush to the White House to
offer a weak hand of friendship, you guarantee exploitation.''

While Mr. Trump's executive order was clearly not aimed at Britain, he
signed it on Friday, just a few hours after Mrs. May left. ``You can
show up at his doorstep and hold his hand so he doesn't fall down a
ramp, but that doesn't mean a few hours later when he's signing an order
he thinks at all about how it affects you, your politics or your
citizens,'' Mr. Shapiro said.

Particularly problematic for Mrs. May was her offering the invitation to
Mr. Trump to undertake a state visit with Queen Elizabeth II this year,
which was accepted. The
\href{https://petition.parliament.uk/petitions/171928}{internet
petition} to Parliament calling for the cancellation of the invitation
says the visit ``would cause embarrassment to Her Majesty the Queen.''

\includegraphics{https://static01.nyt.com/images/2017/01/31/world/europe/31london-demo-video2/31london-demo-video2-videoSixteenByNineJumbo1600.jpg}

By Monday evening in Britain, there had been more than 1.5 million
signatures, and some were enjoying themselves watching the numbers rise
in real time. At a large protest outside Downing Street, people urged
Mrs. May to cancel the state visit and said that while relations with
Washington were important, they should be cooler toward Mr. Trump.

Amber Curtis, 21, a film student who is half-British and half-Iranian,
said that she worried for her family and friends in America. ``It sends
a bad message if he comes here after this ban,'' Ms. Curtis said of Mr.
Trump. ``I wouldn't say that I want no relationship at all, but he
cannot come here under the terms of this ban. The terms need to be
renegotiated.''

Negma Yamin, 50, a teacher of Pakistani origin, was in tears. ``I'm so
upset as a fellow Muslim; I hate the persecution,'' she said. Mrs. May
``should absolutely have no relationship with him,'' she added. ``You
can't negotiate with a person like that. What is he going to do with the
people? He's dividing the U.S., he's dividing the world.''

On Monday, Downing Street insisted that the invitation stood. But who
knows how Mr. Trump will react?

The Mexican president,
\href{https://www.nytimes.com/2017/01/27/world/americas/trump-mexican-president-phone-call.html}{Enrique
Peña Nieto}, has had a similar experience to Mrs. May's --- twice. Last
year, in the name of conciliation and dialogue, he invited Mr. Trump to
Mexico, a somewhat questionable move given Mr. Trump's contempt for
Mexico and his promises to renegotiate the North American Free Trade
Agreement, raise tariffs, deport millions of Mexicans, and build (or
finish) a border wall and make the southern neighbor of the United
States pay for it.

The visit was widely viewed in Mexico as a national humiliation. It left
Mr. Trump looking stronger and Mr. Peña Nieto looking weaker, especially
when Mr. Trump,
\href{https://www.nytimes.com/2016/09/01/us/politics/donald-trump-immigration-speech.html}{in
an immigration policy speech in Phoenix} the same day, insisted again
that Mexico would pay for the wall.

Mr. Peña Nieto persisted after Mr. Trump's election, apparently aiming,
like Mrs. May, to influence the new president and to moderate what many
hoped was just hyperbolic campaign talk. But just before the two men
were to meet in Washington, Mr. Trump issued executive orders calling
for the wall and greatly restricting immigration.

Mr. Peña Nieto
\href{https://www.nytimes.com/2017/01/26/world/mexicos-president-cancels-meeting-with-trump-over-wall.html?_r=0}{called
off the meeting} only when Mr. Trump threatened on Twitter to cancel it
unless Mexico agreed to pay for the wall.

Embarrassed and cornered, Mr. Peña Nieto moved first, an act of defiance
that provided a rare moment of public approval for the unpopular
president. But given the importance of bilateral ties, he did
\href{https://www.nytimes.com/2017/01/27/world/americas/trump-mexican-president-phone-call.html}{speak
to Mr. Trump} the next morning for an hour, without setting a new date
to meet.

``This is neither a victory nor a defeat,'' said Fernando Dworak, an
analyst in Mexico City. ``It is the bell ringing in a boxing match.''

Prime Minister Shinzo Abe of Japan has the distinction of being among
the first to feel the sting of Mr. Trump's actions. In
\href{https://www.nytimes.com/2016/11/17/world/asia/shinzo-abe-donald-trump.html}{a
meeting in November} in New York, Mr. Abe urged Mr. Trump, then the
president-elect, not to abandon a major trade deal, the Trans-Pacific
Partnership.

One of Mr. Trump's first actions in office was to abandon the deal,
which many considered a victory for China, even though the pact had
already been blocked in the Senate. Mr. Trump has long questioned the
United States' financial and military commitment to Japan's security,
and he has criticized the automaker Toyota for planning to produce cars
in Mexico.

\includegraphics{https://static01.nyt.com/images/2017/01/31/world/31Burned2/31Burned2-articleInline.jpg?quality=75\&auto=webp\&disable=upscale}

An editorial in the Mainichi Shimbun, a center-right paper in Japan,
questioned why Mr. Abe was not taking a stronger stand against Mr.
Trump: ``It is hard to understand why the prime minister is defending a
president who destroyed the trade accord --- formed after nearly six
years of arduous negotiations --- on his fourth day in office.''

Given the stakes, Mr. Abe has refrained from open criticism of Mr. Trump
and is scheduled to meet with him in Washington early in February.

The Trump effect has been felt even in Australia, where Prime Minister
Malcolm Turnbull has come under criticism for saying it is
\href{http://www.abc.net.au/news/2017-01-30/turnbull-refuses-comment-on-trump-travel-ban/8222616}{not
his job} to comment on the domestic policies of other countries. This
after
\href{https://mobile.nytimes.com/2017/01/30/world/australia/trump-us-refugee-manus-nauru.html}{securing
a pledge} from the president on Sunday to honor an Obama administration
agreement to accept refugees detained on the Pacific islands of Nauru
and Manus.

In Canada, too, the prime minister, Justin Trudeau, has had his Trump
moments. Mr. Trump is deeply unpopular in the country, but as Mr.
Trudeau's father, former Prime Minister Pierre Trudeau, once said,
proximity to America ``is in some ways like sleeping with an elephant;
no matter how friendly or temperate the beast, one is affected by every
twitch and grunt.''

So instead of provoking a fight, Mr. Trudeau moved swiftly to make
contact with officials in the new administration and reshaped his
cabinet to promote ministers with experience in the United States.

Mr. Trump made problems right away for the Canadian leader by giving the
go-ahead to the Keystone XL pipeline, putting Mr. Trudeau in an
\href{https://www.nytimes.com/2017/01/25/world/canada/canada-justin-trudeau-keystone-xl.html}{uncomfortable
position} between environmentalists and oil producers.

If Mr. Trump goes after Canada on trade issues, as seems likely, Mr.
Trudeau is expected to become significantly more vocal and critical.

But to date he has avoided public criticism of the American president, a
reticence that may have helped over the weekend, after Mr. Trump's
executive order on immigration. Canada was able to get quick
clarification from the White House that the directive would not affect
the movement of Canadian citizens and dual nationals into the United
States.

After fumbling its initial response, Britain got essentially the same
clarification 15 hours later, which London hailed as a result of its
special relationship with Mr. Trump. While Britain may have been
influential, however, the White House was already narrowing the initial
interpretations of the executive order.

But not before Mrs. May was attacked for timidity in the face of outrage
by her own legislators and by the opposition.

Still, the ``special relationship'' has never been an equal one, so some
degree of humiliation often goes with the territory.

As one message on Twitter, posted by the user
\href{https://twitter.com/Locke1689/status/826011324270444544}{@Locke1689},
a professed ``progressive conservative,'' read: ``Actively snubbing the
world's only superpower would be gross diplomatic self-harm.''

Advertisement

\protect\hyperlink{after-bottom}{Continue reading the main story}

\hypertarget{site-index}{%
\subsection{Site Index}\label{site-index}}

\hypertarget{site-information-navigation}{%
\subsection{Site Information
Navigation}\label{site-information-navigation}}

\begin{itemize}
\tightlist
\item
  \href{https://help.nytimes.com/hc/en-us/articles/115014792127-Copyright-notice}{©~2020~The
  New York Times Company}
\end{itemize}

\begin{itemize}
\tightlist
\item
  \href{https://www.nytco.com/}{NYTCo}
\item
  \href{https://help.nytimes.com/hc/en-us/articles/115015385887-Contact-Us}{Contact
  Us}
\item
  \href{https://www.nytco.com/careers/}{Work with us}
\item
  \href{https://nytmediakit.com/}{Advertise}
\item
  \href{http://www.tbrandstudio.com/}{T Brand Studio}
\item
  \href{https://www.nytimes.com/privacy/cookie-policy\#how-do-i-manage-trackers}{Your
  Ad Choices}
\item
  \href{https://www.nytimes.com/privacy}{Privacy}
\item
  \href{https://help.nytimes.com/hc/en-us/articles/115014893428-Terms-of-service}{Terms
  of Service}
\item
  \href{https://help.nytimes.com/hc/en-us/articles/115014893968-Terms-of-sale}{Terms
  of Sale}
\item
  \href{https://spiderbites.nytimes.com}{Site Map}
\item
  \href{https://help.nytimes.com/hc/en-us}{Help}
\item
  \href{https://www.nytimes.com/subscription?campaignId=37WXW}{Subscriptions}
\end{itemize}
