Sections

SEARCH

\protect\hyperlink{site-content}{Skip to
content}\protect\hyperlink{site-index}{Skip to site index}

\href{https://www.nytimes.com/section/business/economy}{Economy}

\href{https://myaccount.nytimes.com/auth/login?response_type=cookie\&client_id=vi}{}

\href{https://www.nytimes.com/section/todayspaper}{Today's Paper}

\href{/section/business/economy}{Economy}\textbar{}With Steady Gains in
Economic Outlook, Fed Leaves Interest Rate Unchanged

\url{https://nyti.ms/2jW3aUg}

\begin{itemize}
\item
\item
\item
\item
\item
\end{itemize}

Advertisement

\protect\hyperlink{after-top}{Continue reading the main story}

Supported by

\protect\hyperlink{after-sponsor}{Continue reading the main story}

\hypertarget{with-steady-gains-in-economic-outlook-fed-leaves-interest-rate-unchanged}{%
\section{With Steady Gains in Economic Outlook, Fed Leaves Interest Rate
Unchanged}\label{with-steady-gains-in-economic-outlook-fed-leaves-interest-rate-unchanged}}

\includegraphics{https://static01.nyt.com/images/2017/02/02/business/02FED/02FED-articleInline.jpg?quality=75\&auto=webp\&disable=upscale}

By \href{http://www.nytimes.com/by/binyamin-appelbaum}{Binyamin
Appelbaum}

\begin{itemize}
\item
  Feb. 1, 2017
\item
  \begin{itemize}
  \item
  \item
  \item
  \item
  \item
  \end{itemize}
\end{itemize}

WASHINGTON --- The Federal Reserve is waiting for more information about
the Trump administration's economic plans, just like everyone else.

After its first policy making meeting of the year, the Fed said on
Wednesday that its economic outlook remained essentially unchanged since
its previous meeting in December. The nation's slow-and-steady economic
expansion has continued, with little sign in the latest data that it is
flagging or accelerating.

And as expected, the Federal Open Market Committee, which makes monetary
policy,
\href{https://www.federalreserve.gov/newsevents/press/monetary/20170201a.htm}{left
the Fed's benchmark interest rate unchanged}.

The question is what comes next.

Fed officials said in the weeks before the meeting Wednesday that their
uncertainty about the outlook had increased. President Trump has
proposed significant shifts in economic policy --- including changes in
taxation, regulation and trade --- that could affect growth.

``The statement is written such that the F.O.M.C. will be able to adjust
monetary policy as needed in response to the fiscal and trade policies
of the administration,'' said Michael Gapen, chief United States
economist at Barclays.

At its December meeting,
\href{https://www.nytimes.com/2016/12/14/business/economy/fed-interest-rates-janet-yellen.html?rref=collection\%2Fbyline\%2Fbinyamin-appelbaum\&action=click\&contentCollection=undefined\&region=stream\&module=stream_unit\&version=latest\&contentPlacement=1\&pgtype=collection\&_r=0}{the
Fed raised its benchmark interest rate} for just the second time since
the financial crisis. After the increase of a quarter point, the rate
now ranges from 0.5 percent to 0.75 percent, still very low by
historical standards. Low rates encourage borrowing and risk-taking,
contributing to faster economic growth.

By raising rates, the Fed is gradually reducing the force of that
stimulus.

Fed officials predicted in December that they would raise the benchmark
rate three times this year. But they have cautioned that changes in
fiscal policy could alter those plans. If Mr. Trump and congressional
Republicans seek to increase growth, for example by cutting taxes or
spending a lot on infrastructure and the military, the Fed could raise
rates more quickly.

If Mr. Trump's policies weigh on growth, the Fed could move more slowly.

The only hint of those pressures in the Fed's latest statement was a
mention of increased public optimism about the outlook for the nation's
economy.

``Measures of consumer and business sentiment have improved of late,''
it said.

Fed officials are watching fiscal policy makers closely because the Fed
has concluded that the American economy is growing at something close to
the maximum sustainable pace, meaning that, in the Fed's view, faster
growth would probably lead to higher inflation.

Changes in fiscal policy are most likely to have a gradual impact,
however, so the tension between the Fed and fiscal policy makers may
play out mostly in coming years.

``The committee is probably still in a wait-and-see mode as far as
fiscal policy is concerned,'' said Kevin Logan, chief United States
economist at HSBC.

The Fed's assessment of economic conditions remained upbeat. The latest
data showed ``the labor market has continued to strengthen and that
economic activity has continued to expand at a moderate pace,'' the
statement said. ``Job gains remained solid and the unemployment rate
stayed near its recent low.''

Fed officials spoke in similarly optimistic tones in the weeks before
the meeting.

``All in all, things are looking good,'' Patrick T. Harker, president of
the Federal Reserve Bank of Philadelphia,
\href{https://www.philadelphiafed.org/publications/speeches/harker/2017/01-20-17-nj-bankers-association}{said
in mid-January}. ``We're starting 2017 off on a good foot.''

But seven years of tepid growth have not restored the economy to full
health.

The unemployment rate stood at 4.7 percent in December, a level most Fed
officials regard as nearly normal. Other labor market measures, however,
remain weak. Wage growth is tepid, and the employment to population
ratio for people 25 to 54 was
\href{https://data.bls.gov/timeseries/LNS12300060}{78.2 percent in
December}. The Fed's preferred measure of price inflation, the Bureau of
Economic Analysis'
\href{https://www.bea.gov/newsreleases/national/pi/pinewsrelease.htm}{index
of personal consumption expenditures}, rose by 1.6 percent in 2016, the
strongest performance in more than two years. But inflation remains
below the Fed's goal of a 2 percent annual pace --- a goal the Fed has
not achieved since 2011.

The vote to leave rates unchanged was unanimous, the Fed said. And the
tempered language of the statement led investors to mark down the modest
chance of a rate increase at the Fed's next meeting, in March, from
about 20 percent before the February statement to about 18 percent
afterward, CME Group said.

Janet L. Yellen, chairwoman of the Fed, will have a chance to elaborate
on the central bank's economic outlook and policy plans when she
delivers a semiannual report on monetary policy to Senate and House
committees on Feb. 14 and 15.

In the meantime, the Fed is the rare corner of official Washington where
nothing is happening. Contrasting a lively week at the White House and
on Capitol Hill with the Fed's announcement, Michael Feroli, the chief
United States economist at JPMorgan Chase, declared the Fed's
headquarters ``the most boring spot in Washington.''

Advertisement

\protect\hyperlink{after-bottom}{Continue reading the main story}

\hypertarget{site-index}{%
\subsection{Site Index}\label{site-index}}

\hypertarget{site-information-navigation}{%
\subsection{Site Information
Navigation}\label{site-information-navigation}}

\begin{itemize}
\tightlist
\item
  \href{https://help.nytimes.com/hc/en-us/articles/115014792127-Copyright-notice}{©~2020~The
  New York Times Company}
\end{itemize}

\begin{itemize}
\tightlist
\item
  \href{https://www.nytco.com/}{NYTCo}
\item
  \href{https://help.nytimes.com/hc/en-us/articles/115015385887-Contact-Us}{Contact
  Us}
\item
  \href{https://www.nytco.com/careers/}{Work with us}
\item
  \href{https://nytmediakit.com/}{Advertise}
\item
  \href{http://www.tbrandstudio.com/}{T Brand Studio}
\item
  \href{https://www.nytimes.com/privacy/cookie-policy\#how-do-i-manage-trackers}{Your
  Ad Choices}
\item
  \href{https://www.nytimes.com/privacy}{Privacy}
\item
  \href{https://help.nytimes.com/hc/en-us/articles/115014893428-Terms-of-service}{Terms
  of Service}
\item
  \href{https://help.nytimes.com/hc/en-us/articles/115014893968-Terms-of-sale}{Terms
  of Sale}
\item
  \href{https://spiderbites.nytimes.com}{Site Map}
\item
  \href{https://help.nytimes.com/hc/en-us}{Help}
\item
  \href{https://www.nytimes.com/subscription?campaignId=37WXW}{Subscriptions}
\end{itemize}
