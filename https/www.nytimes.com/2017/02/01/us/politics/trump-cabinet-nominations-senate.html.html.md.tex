Sections

SEARCH

\protect\hyperlink{site-content}{Skip to
content}\protect\hyperlink{site-index}{Skip to site index}

\href{https://www.nytimes.com/section/politics}{Politics}

\href{https://myaccount.nytimes.com/auth/login?response_type=cookie\&client_id=vi}{}

\href{https://www.nytimes.com/section/todayspaper}{Today's Paper}

\href{/section/politics}{Politics}\textbar{}2 G.O.P. Senators to Vote
Against Betsy DeVos as Education Secretary

\url{https://nyti.ms/2jYW5ES}

\begin{itemize}
\item
\item
\item
\item
\item
\end{itemize}

Advertisement

\protect\hyperlink{after-top}{Continue reading the main story}

Supported by

\protect\hyperlink{after-sponsor}{Continue reading the main story}

\hypertarget{2-gop-senators-to-vote-against-betsy-devos-as-education-secretary}{%
\section{2 G.O.P. Senators to Vote Against Betsy DeVos as Education
Secretary}\label{2-gop-senators-to-vote-against-betsy-devos-as-education-secretary}}

\includegraphics{https://static01.nyt.com/images/2017/02/02/us/02confirm_web2/02confirm_web2-articleInline.jpg?quality=75\&auto=webp\&disable=upscale}

By \href{http://www.nytimes.com/by/yamiche-alcindor}{Yamiche Alcindor}
and \href{https://www.nytimes.com/by/emmarie-huetteman}{Emmarie
Huetteman}

\begin{itemize}
\item
  Feb. 1, 2017
\item
  \begin{itemize}
  \item
  \item
  \item
  \item
  \item
  \end{itemize}
\end{itemize}

WASHINGTON --- Two Republican senators on Wednesday said they would vote
against President Trump's nominee for education secretary, delivering a
blow to the White House and raising the possibility that Vice President
Mike Pence would have to break a tie to win her confirmation.

The nominee,
\href{https://www.nytimes.com/2017/01/18/us/politics/betsy-devos-education-secretary-confirmation-donald-trump.html}{Betsy
DeVos}, a billionaire with a complex web of financial investments, had
already faced fierce opposition from Democrats and labor unions because
of her political contributions to Republicans and her involvement in
pushing alternatives to public education. But her confirmation hearing
in front of the Senate Committee on Health, Education, Labor and
Pensions, during which she
\href{https://www.nytimes.com/2017/01/18/us/politics/betsy-devos-education-secretary-confirmation-donald-trump.html?_r=0}{flubbed}
answers on education policy, also brought concerns from Republicans.

Senator Lisa Murkowski, Republican of Alaska, said Ms. DeVos had failed
to demonstrate that she understood what public schools needed to
succeed. ``I have serious concerns about a nominee to be secretary of
education who has been so involved in one side of the equation, so
immersed in the push for vouchers, that she may be unaware of what
actually is successful within the public schools, and also what is
broken and how to fix them,'' Ms. Murkowski said.

Senator Susan Collins, Republican of Maine, said she respected Ms.
DeVos's clear devotion to students and ``valuable work'' in education.
But she said she remained troubled by Ms. DeVos's focus on alternatives
to public education, as well as what Ms. Collins called a ``lack of
familiarity'' with federal laws protecting children with special needs
and disabilities.

``Her concentration on charter schools and vouchers, however, raises the
question about whether or not she fully appreciates that the secretary
of education's primary focus must be on helping states and communities,
parents, teachers, school board administrators, school board members and
administrators strengthen our public schools,'' Ms. Collins said. ``I
will not, I cannot, vote to confirm her as our nation's next secretary
of education.''

\includegraphics{https://static01.nyt.com/images/2017/02/02/us/02confirm2/02confirm2-articleInline.jpg?quality=75\&auto=webp\&disable=upscale}

Senators and education advocates from both sides of the aisle were taken
aback by Ms. DeVos's comments at her Jan. 17 confirmation hearing about
core responsibilities of the Department of Education.
\href{https://www.washingtonpost.com/video/national/tim-kaine-to-betsy-devos-do-you-not-want-to-answer-my-question/2017/01/17/e21d8192-dd1c-11e6-8902-610fe486791c_video.html}{One
exchange with Senator Tim Kaine}, Democrat of Virginia, was later
promoted heavily on social media by the Democratic Party and activist
groups, including national education unions.

In it, Mr. Kaine asked Ms. DeVos whether all schools that receive public
money should have to follow the requirements of the Individuals With
Disabilities Education Act, known as IDEA. Ms. DeVos responded, ``I
think that's a matter that's best left to the states.''

In fact, under IDEA, a landmark 1975 civil rights law, states and school
districts are required to provide special education services to children
with disabilities. During the hearing, Ms. Collins said she had ``heard
repeatedly'' from school officials that the most important action the
federal government could take on education would be ``to fulfill the
promise'' of IDEA by providing more funding for those students.

Ms. DeVos responded to Ms. Collins that she would look at funding
levels, but said, ``Maybe the money should follow individual students
instead of going directly to the states.''

The pushback against Ms. DeVos played out in thousands of emails and
phone calls urging senators to vote against her. Protesters showed up at
her confirmation hearing, outside senators' offices and in Michigan,
where Ms. DeVos has been involved in pushing education policies.

Her nomination now hangs precariously on whether Republicans will rally
the support of a few undeclared colleagues, or woo Democratic
dissenters. Her chances got a boost on Wednesday with the support of two
Republicans who were originally believed to oppose her, Senators Dean
Heller of Nevada and Patrick J. Toomey of Pennsylvania.

Image

Senator Susan Collins, Republican of Maine, said she was troubled by Ms.
DeVos's ``lack of familiarity'' of federal laws protecting children with
special needs and disabilities.Credit...Al Drago/The New York Times

If the Senate's Democrats and independents vote together, just one more
Republican defection would be fatal to Ms. DeVos's prospects. If all
other senators vote along party lines, Mr. Pence could break a 50-50 tie
in his capacity as president of the Senate. But three Republican votes
opposing her confirmation would result in an outright rejection of her
nomination.

Ms. DeVos's opponents said on Wednesday that they would target specific
Republicans, including Mr. Heller and Senator Rob Portman of Ohio.

If enough senators oppose Ms. DeVos, it would be just the second time in
history that the Senate rejected a nominee for a first-term president
assembling his cabinet,
\href{https://www.senate.gov/artandhistory/history/common/briefing/Nominations.htm\#10}{according
to the Senate Historical Office}.

After Ms. Collins's and Ms. Murkowski's announcements, the Senate opened
consideration of Ms. DeVos's nomination Wednesday, scheduling the first
procedural hurdle to her confirmation for Friday. But amid strong
Democratic opposition, it looks unlikely that the Senate's final vote
will come until early next week.

The president of the National Education Association, Lily Eskelsen
García, said educators, parents and students were ``grateful'' for the
opposition by Ms. Collins and Ms. Murkowski. ``The nation is speaking
out; senators need to listen,'' Ms. Eskelsen García said.

Dan Cantor, national director of the Working Families Party, which has
been organizing protests outside senators' offices, said his group would
be holding regular protests at Mr. Heller's office. Opponents of Ms.
DeVos said they would target other Republicans, including Mr. Portman.

``The DeVos family has given millions to elect Republicans, but that
shouldn't buy her a cabinet post,'' Mr. Cantor said. ``The first step
was unity from the Senate Democrats. Now we're going to keep up that
pressure on Senate Republicans.''

Democrats, teachers unions and liberal protesters have voiced concern
about the DeVos family's contributions to groups that support so-called
conversion therapy for gay people; Ms. DeVos's more than \$200 million
in donations to Republicans and their causes; and her past statements
that government ``sucks'' and that public schools are a ``dead end.''
Opponents have also focused on the poor performance of charter schools
in Detroit, which Ms. DeVos has bankrolled even as she resisted
legislation that would have blocked chronically failing charter schools
from expanding.

The White House press secretary, Sean Spicer, swiftly brushed off the
defections as inconsequential, saying he had ``zero'' concern about Ms.
DeVos's confirmation by the Senate.

Ms. DeVos's opponents were undeterred.

``The more people get to know how ill equipped Betsy DeVos is to
strengthen public schools, how disconnected she is from public schools,
and how her record has been focused on pursuing for-profit charters and
vouchers, and not children, the more the people who believe in the
importance of public education are joining to oppose her,'' said Randi
Weingarten, president of the American Federation of Teachers.

Meanwhile, in a display of fierce resistance against Mr. Trump that
played out in a proxy battle against his nominees, Senate Democrats
again boycotted committee votes on the nominations of Representative Tom
Price and Steven T. Mnuchin, the president's picks to lead the
Department of Health and Human Services and the Treasury, respectively.
Democrats argued that new information had emerged to suggest that the
two nominees had not been truthful in their confirmation testimony.

Determined not to be thwarted, the committee's Republicans broke with
the longstanding practice that at least one member of the minority party
be present and held the votes anyway, unanimously agreeing to send the
nominations to the Senate floor.

At the same time, a divided Senate Judiciary Committee greenlighted
Senator Jeff Sessions's nomination as attorney general along a straight
party-line vote, sending it to the full Senate for a final vote.
Democrats also boycotted a planned committee vote Wednesday on Scott
Pruitt, Mr. Trump's choice to lead the Environmental Protection Agency,
and continued to slow down consideration of Representative Mick
Mulvaney, his pick for White House budget director.

Advertisement

\protect\hyperlink{after-bottom}{Continue reading the main story}

\hypertarget{site-index}{%
\subsection{Site Index}\label{site-index}}

\hypertarget{site-information-navigation}{%
\subsection{Site Information
Navigation}\label{site-information-navigation}}

\begin{itemize}
\tightlist
\item
  \href{https://help.nytimes.com/hc/en-us/articles/115014792127-Copyright-notice}{©~2020~The
  New York Times Company}
\end{itemize}

\begin{itemize}
\tightlist
\item
  \href{https://www.nytco.com/}{NYTCo}
\item
  \href{https://help.nytimes.com/hc/en-us/articles/115015385887-Contact-Us}{Contact
  Us}
\item
  \href{https://www.nytco.com/careers/}{Work with us}
\item
  \href{https://nytmediakit.com/}{Advertise}
\item
  \href{http://www.tbrandstudio.com/}{T Brand Studio}
\item
  \href{https://www.nytimes.com/privacy/cookie-policy\#how-do-i-manage-trackers}{Your
  Ad Choices}
\item
  \href{https://www.nytimes.com/privacy}{Privacy}
\item
  \href{https://help.nytimes.com/hc/en-us/articles/115014893428-Terms-of-service}{Terms
  of Service}
\item
  \href{https://help.nytimes.com/hc/en-us/articles/115014893968-Terms-of-sale}{Terms
  of Sale}
\item
  \href{https://spiderbites.nytimes.com}{Site Map}
\item
  \href{https://help.nytimes.com/hc/en-us}{Help}
\item
  \href{https://www.nytimes.com/subscription?campaignId=37WXW}{Subscriptions}
\end{itemize}
