Sections

SEARCH

\protect\hyperlink{site-content}{Skip to
content}\protect\hyperlink{site-index}{Skip to site index}

\href{https://www.nytimes.com/section/world/middleeast}{Middle East}

\href{https://myaccount.nytimes.com/auth/login?response_type=cookie\&client_id=vi}{}

\href{https://www.nytimes.com/section/todayspaper}{Today's Paper}

\href{/section/world/middleeast}{Middle East}\textbar{}Raid in Yemen:
Risky From the Start and Costly in the End

\url{https://nyti.ms/2jXAIBs}

\begin{itemize}
\item
\item
\item
\item
\item
\end{itemize}

Advertisement

\protect\hyperlink{after-top}{Continue reading the main story}

Supported by

\protect\hyperlink{after-sponsor}{Continue reading the main story}

\hypertarget{raid-in-yemen-risky-from-the-start-and-costly-in-the-end}{%
\section{Raid in Yemen: Risky From the Start and Costly in the
End}\label{raid-in-yemen-risky-from-the-start-and-costly-in-the-end}}

\includegraphics{https://static01.nyt.com/images/2017/02/02/world/02MILITARY-PRINT/02MILITARY-PRINT-articleInline.jpg?quality=75\&auto=webp\&disable=upscale}

By \href{http://www.nytimes.com/by/eric-schmitt}{Eric Schmitt} and
\href{http://www.nytimes.com/by/david-e-sanger}{David E. Sanger}

\begin{itemize}
\item
  Feb. 1, 2017
\item
  \begin{itemize}
  \item
  \item
  \item
  \item
  \item
  \end{itemize}
\end{itemize}

WASHINGTON --- Just five days after taking office, over dinner with his
newly installed secretary of defense and the chairman of the Joint
Chiefs of Staff, President Trump was presented with the first of what
will be many life-or-death decisions: whether to approve a commando raid
that risked the lives of American Special Operations forces and foreign
civilians alike.

President Barack Obama's national security aides had reviewed the plans
for a risky attack on a small, heavily guarded brick home of a senior
Qaeda collaborator in a mountainous village in a remote part of central
Yemen. But Mr. Obama did not act because the Pentagon wanted to launch
the attack on a moonless night and the next one would come after his
term had ended.

With two of his closest advisers, Jared Kushner and Stephen K. Bannon,
joining the dinner at the White House along with Defense Secretary Jim
Mattis and Gen. Joseph F. Dunford Jr., Mr. Trump approved sending in the
Navy's SEAL Team 6, hoping the raid early last Sunday would scoop up
cellphones and laptop computers that could yield valuable clues about
one of the world's most dangerous terrorist groups. Vice President Mike
Pence and Michael T. Flynn, the national security adviser, also attended
the dinner.

As it turned out, almost everything that could go wrong did. And on
Wednesday, Mr. Trump flew to Dover Air Force Base in Delaware to be
present as the body of the American commando killed in the raid was
returned home, the first military death on the new commander in chief's
watch.

The death of Chief Petty Officer William Owens came after a chain of
mishaps and misjudgments that plunged the elite commandos into a
ferocious 50-minute firefight that also left three others wounded and a
\$75 million aircraft deliberately destroyed. There are allegations ---
which the Pentagon acknowledged on Wednesday night are most likely
correct --- that the mission also killed several civilians, including
some children. The dead include, by the account of Al Qaeda's branch in
Yemen, the 8-year-old daughter of Anwar al-Awlaki, the American-born
Qaeda leader who was killed in a targeted drone strike in 2011.

Mr. Trump on Sunday hailed his first counterterrorism operation as a
success, claiming the commandos captured ``important intelligence that
will assist the U.S. in preventing terrorism against its citizens and
people around the world.'' A statement by the military's Central Command
on Wednesday night that acknowledged the likelihood of civilian
casualties also said that the recovered materials had provided some
initial information helpful to counterterrorism analysts. The statement
did not provide details.

But the mission's casualties raise doubts about the months of detailed
planning that went into the operation during the Obama administration
and whether the right questions were raised before its approval.
Typically, the president's advisers lay out the risks, but Pentagon
officials declined to characterize any discussions with Mr. Trump.

A senior administration official said on Wednesday night that the
Defense Department had conducted a legal review of the operation that
Mr. Trump approved and that a Pentagon lawyer had signed off on it.

Mr. Trump's new national security team, led by Mr. Flynn, the former
head of the Defense Intelligence Agency and a retired general with
experience in counterterrorism raids, has said that it wants to speed
the decision-making when it comes to such strikes, delegating more power
to lower-level officials so that the military may respond more quickly.
Indeed, the Pentagon is drafting such plans to accelerate activities
against the Qaeda branch in Yemen.

But doing that also raises the possibility of error. ``You can mitigate
risk in missions like this, but you can't mitigate risk down to zero,''
said William Wechsler, a former top counterterrorism official at the
Pentagon.

In this case, the assault force of several dozen commandos, which also
included elite soldiers from the United Arab Emirates, was jinxed from
the start. Qaeda fighters were somehow tipped off to the stealthy
advance toward the village --- perhaps by the whine of American drones
that local tribal leaders said were flying lower and louder than usual.

Through a communications intercept, the commandos knew that the mission
had been somehow compromised, but pressed on toward their target roughly
five miles from where they had been flown into the area. ``They kind of
knew they were screwed from the beginning,'' one former SEAL Team 6
official said.

With the crucial element of surprise lost, the Americans and Emiratis
found themselves in a gun battle with Qaeda fighters who took up
positions in other houses, a clinic, a school and a mosque, often using
women and children as cover, American military officials said in
interviews this week.

The commandos were taken aback when some of the women grabbed weapons
and started firing, multiplying the militant firepower beyond what they
had expected. The Americans called in airstrikes from helicopter
gunships and fighter aircraft that helped kill some 14 Qaeda fighters,
but not before an MV-22 Osprey aircraft involved in the operation
experienced a ``hard landing,'' injuring three more American personnel
on board. The Osprey, which the Marine Corps said cost \$75 million, was
badly damaged and had to be destroyed by an airstrike.

The raid, some details of which were
\href{https://www.washingtonpost.com/news/checkpoint/wp/2017/01/31/how-trumps-first-counter-terror-operation-in-yemen-turned-into-chaos/?utm_term=.77066dbb6dc9}{first
reported by The Washington Post}, also destroyed much of the village of
Yakla, and left senior Yemeni government officials seething. Yemen's
foreign minister, Abdul Malik Al Mekhlafi, condemned the raid on Monday
in a post on his official Twitter account as ``extrajudicial killings.''

Baraa Shiban, a Yemeni fellow for Reprieve, a London-based human rights
group, said he spoke by phone to a tribal sheikh in the village, Jabbr
Abu Soraima, who told him: ``People were afraid to leave their houses
because the sound of choppers and drones were all over the sky. Everyone
feared of being hit by the drones or shot by the soldiers on the
ground.''

After initially denying there were any civilian casualties, Pentagon
officials backtracked somewhat on Sunday after reports from the Yemeni
authorities begin trickling in and grisly photographs of bloody children
purportedly killed in the attack appeared on social media sites
affiliated with Al Qaeda's branch in Yemen.

Capt. Jeff Davis, a Pentagon spokesman, said on Monday that some of the
women were combatants.

The operation was the first known American-led ground mission in Yemen
since December 2014, when members of SEAL Team 6 stormed a village in
southern Yemen in an effort to free an American photojournalist held
hostage by Al Qaeda. But the raid ended with the kidnappers killing the
journalist and a South African held with him.

That mission and the raid over the weekend revealed the shortcomings of
secretive military operations in Yemen. The United States was forced to
withdraw the last 125 Special Operations advisers from the country in
March 2015 after Houthi rebels ousted the government of President Abdu
Rabbu Mansour Hadi, the Americans' main counterterrorism partner.

The loss of Yemen as a base for American counterterrorism training,
advising and intelligence-gathering was a significant blow to blunting
the advance of Al Qaeda's branch in the country and keeping tabs on
their plots. The Pentagon has tried to start rebuilding its
counterterrorism operations in Yemen, however; last year, American
Special Operations forces helped Emirati troops evict Qaeda fighters
from the port city of Mukalla.

Advertisement

\protect\hyperlink{after-bottom}{Continue reading the main story}

\hypertarget{site-index}{%
\subsection{Site Index}\label{site-index}}

\hypertarget{site-information-navigation}{%
\subsection{Site Information
Navigation}\label{site-information-navigation}}

\begin{itemize}
\tightlist
\item
  \href{https://help.nytimes.com/hc/en-us/articles/115014792127-Copyright-notice}{©~2020~The
  New York Times Company}
\end{itemize}

\begin{itemize}
\tightlist
\item
  \href{https://www.nytco.com/}{NYTCo}
\item
  \href{https://help.nytimes.com/hc/en-us/articles/115015385887-Contact-Us}{Contact
  Us}
\item
  \href{https://www.nytco.com/careers/}{Work with us}
\item
  \href{https://nytmediakit.com/}{Advertise}
\item
  \href{http://www.tbrandstudio.com/}{T Brand Studio}
\item
  \href{https://www.nytimes.com/privacy/cookie-policy\#how-do-i-manage-trackers}{Your
  Ad Choices}
\item
  \href{https://www.nytimes.com/privacy}{Privacy}
\item
  \href{https://help.nytimes.com/hc/en-us/articles/115014893428-Terms-of-service}{Terms
  of Service}
\item
  \href{https://help.nytimes.com/hc/en-us/articles/115014893968-Terms-of-sale}{Terms
  of Sale}
\item
  \href{https://spiderbites.nytimes.com}{Site Map}
\item
  \href{https://help.nytimes.com/hc/en-us}{Help}
\item
  \href{https://www.nytimes.com/subscription?campaignId=37WXW}{Subscriptions}
\end{itemize}
