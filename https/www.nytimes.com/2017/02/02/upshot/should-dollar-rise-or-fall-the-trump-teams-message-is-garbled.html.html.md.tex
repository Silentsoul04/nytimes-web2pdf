Sections

SEARCH

\protect\hyperlink{site-content}{Skip to
content}\protect\hyperlink{site-index}{Skip to site index}

\href{https://myaccount.nytimes.com/auth/login?response_type=cookie\&client_id=vi}{}

\href{https://www.nytimes.com/section/todayspaper}{Today's Paper}

\href{/section/upshot}{The Upshot}\textbar{}Should Dollar Rise or Fall?
The Trump Team's Message Is Garbled

\url{https://nyti.ms/2k0kpnw}

\begin{itemize}
\item
\item
\item
\item
\item
\item
\end{itemize}

Advertisement

\protect\hyperlink{after-top}{Continue reading the main story}

Supported by

\protect\hyperlink{after-sponsor}{Continue reading the main story}

Upshot

Economic Trends

\hypertarget{should-dollar-rise-or-fall-the-trump-teams-message-is-garbled}{%
\section{Should Dollar Rise or Fall? The Trump Team's Message Is
Garbled}\label{should-dollar-rise-or-fall-the-trump-teams-message-is-garbled}}

By \href{http://www.nytimes.com/by/neil-irwin}{Neil Irwin}

\begin{itemize}
\item
  Feb. 2, 2017
\item
  \begin{itemize}
  \item
  \item
  \item
  \item
  \item
  \item
  \end{itemize}
\end{itemize}

For a very long time, if you asked United States government officials
their view on the value of the dollar, they would almost certainly
decline to answer. ``The currency is the purview of the Treasury
secretary,'' they'd say.

If you asked the Treasury secretary his view of the dollar, the answer
would be equally rote: ``A strong dollar is in the interest of the
United States.'' Those words have been so standard that when Paul
O'Neill deviated from it in an interview with a German newspaper in
2001, he caused a kerfuffle in global currency markets and
\href{https://www.wsj.com/articles/SB982535321468562836}{quickly
backtracked}.

The Trump administration looks to be taking a different approach.
Officials, including the president himself, have been breaking tradition
and talking about currency markets --- and in many cases, suggesting
that the dollar is too highly valued.

But the Trump administration has also offered contradictory signals
about just what it wants to see happen on currency markets, with tension
building between the administration's stated goals and the policies it
says it advocates. Predicting the direction of currency policy in the
Trump administration requires parsing the words and policy proposals
that come from several different places.

Both before and since the inauguration, Donald J. Trump has weighed in
on currency issues more than any of his recent predecessors. ``Our
dollar is too strong,'' he said in
\href{https://www.wsj.com/articles/trump-comments-signal-shift-in-approach-to-u-s-dollar-1484690469}{an
interview} with The Wall Street Journal in mid-January. ``And our
companies can't compete'' because the currency is too strong. ``It's
killing us,'' he said.

Mr. Trump enlarged on this theme in comments this week at a meeting with
pharmaceutical executives. ``We know nothing about devaluation,'' he
said. ``Every other country lives on devaluation. You look at what
China's doing, you look at what Japan has done over the years --- they
played the money market, they played the devaluation market, and we sit
there like a bunch of dummies.''

Taking the president's words at face value, we would seem to be in for a
deliberate effort by the administration to reduce the value of the
United States currency relative to its neighbors, in hopes of giving
U.S. exporters an advantage and reducing the trade deficit.

\includegraphics{https://static01.nyt.com/images/2017/02/03/upshot/03up-dollar/03up-dollar-articleInline.jpg?quality=75\&auto=webp\&disable=upscale}

That is also the tone that Peter Navarro, the director of the White
House's national trade council, struck in an interview with The
Financial Times this week. He described the euro as ``grossly
undervalued,'' which would imply that the dollar is too strong against
the world's other most widely used currency.

Steven Mnuchin, the Treasury secretary nominee, has tried to take a more
balanced approach to talking about the valuation of the dollar. While
trying not to undercut his boss, he has made the point that over the
longer run a strong domestic economy will cause the dollar to increase
in value.

``The currency is very, very strong, and what you see is people from all
over the world wanting to invest in the U.S. currency,'' Mr. Mnuchin
said in his confirmation hearing. ``I think when the president-elect
made a comment on the U.S. currency, it was not meant to be long-term
comment. It was meant to be that perhaps in the short term, the strength
in the currency as a result of free markets and people wanting to invest
here may have had negative impacts on --- on our ability in trade.''

But even that nuanced thought from Mr. Mnuchin is in tension with the
concrete outcomes in the currency markets that are resulting from his
administration's policies.

Last week, for example, Sean Spicer, the president's press secretary,
\href{https://www.nytimes.com/2017/01/26/upshot/how-to-interpret-the-trump-administrations-latest-signals-on-mexico.html}{suggested}
that a policy known as the ``border adjustment tax'' would be a way to
force Mexico to pay for a border wall. But that policy's adherents
believe it would actually increase the value of the dollar on currency
markets by perhaps 20 percent. (And if it doesn't, the tax would hit
American consumers and retailers hard.)

Wilbur Ross, the commerce secretary nominee, urged a renegotiation of
the North American Free Trade Agreement. Those comments, at his
confirmation hearing,
\href{http://www.businessinsider.com/canadian-dollar-mexican-peso-wilbur-ross-nafta-2017-1}{prompted
a sell-off} of the Canadian and Mexican currencies.

And more broadly, if the administration's plan to cut taxes and increase
infrastructure comes to fruition, it implies that interest rates in the
United States will be higher than they otherwise would be. Higher
interest rates will bring global investors, with their assets, flocking
into the United States and propel the dollar even higher.

Any administration has a range of tools it can use to influence the
value of the currency. It can set trade policies, use diplomatic
pressure on trading partners, make public statements and appoint people
to the Federal Reserve who are inclined toward either looser money or
tighter money.

So in the months ahead, the Trump administration will have to decide if
it really intends to follow through on the president's words about the
need to devalue the dollar. Or it will follow the long tradition in the
United States of viewing a strong currency as a sign of strength, not
weakness.

Advertisement

\protect\hyperlink{after-bottom}{Continue reading the main story}

\hypertarget{site-index}{%
\subsection{Site Index}\label{site-index}}

\hypertarget{site-information-navigation}{%
\subsection{Site Information
Navigation}\label{site-information-navigation}}

\begin{itemize}
\tightlist
\item
  \href{https://help.nytimes.com/hc/en-us/articles/115014792127-Copyright-notice}{©~2020~The
  New York Times Company}
\end{itemize}

\begin{itemize}
\tightlist
\item
  \href{https://www.nytco.com/}{NYTCo}
\item
  \href{https://help.nytimes.com/hc/en-us/articles/115015385887-Contact-Us}{Contact
  Us}
\item
  \href{https://www.nytco.com/careers/}{Work with us}
\item
  \href{https://nytmediakit.com/}{Advertise}
\item
  \href{http://www.tbrandstudio.com/}{T Brand Studio}
\item
  \href{https://www.nytimes.com/privacy/cookie-policy\#how-do-i-manage-trackers}{Your
  Ad Choices}
\item
  \href{https://www.nytimes.com/privacy}{Privacy}
\item
  \href{https://help.nytimes.com/hc/en-us/articles/115014893428-Terms-of-service}{Terms
  of Service}
\item
  \href{https://help.nytimes.com/hc/en-us/articles/115014893968-Terms-of-sale}{Terms
  of Sale}
\item
  \href{https://spiderbites.nytimes.com}{Site Map}
\item
  \href{https://help.nytimes.com/hc/en-us}{Help}
\item
  \href{https://www.nytimes.com/subscription?campaignId=37WXW}{Subscriptions}
\end{itemize}
