Sections

SEARCH

\protect\hyperlink{site-content}{Skip to
content}\protect\hyperlink{site-index}{Skip to site index}

\href{https://www.nytimes.com/section/world/asia}{Asia Pacific}

\href{https://myaccount.nytimes.com/auth/login?response_type=cookie\&client_id=vi}{}

\href{https://www.nytimes.com/section/todayspaper}{Today's Paper}

\href{/section/world/asia}{Asia Pacific}\textbar{}North Korean Leader's
Top Enforcer Is Now the One Getting Purged

\url{https://nyti.ms/2k7v5TQ}

\begin{itemize}
\item
\item
\item
\item
\item
\end{itemize}

Advertisement

\protect\hyperlink{after-top}{Continue reading the main story}

Supported by

\protect\hyperlink{after-sponsor}{Continue reading the main story}

\hypertarget{north-korean-leaders-top-enforcer-is-now-the-one-getting-purged}{%
\section{North Korean Leader's Top Enforcer Is Now the One Getting
Purged}\label{north-korean-leaders-top-enforcer-is-now-the-one-getting-purged}}

\includegraphics{https://static01.nyt.com/images/2017/02/04/world/04korea-web1/04korea-web1-articleInline.jpg?quality=75\&auto=webp\&disable=upscale}

By \href{http://www.nytimes.com/by/choe-sang-hun}{Choe Sang-Hun}

\begin{itemize}
\item
  Feb. 3, 2017
\item
  \begin{itemize}
  \item
  \item
  \item
  \item
  \item
  \end{itemize}
\end{itemize}

SEOUL, South Korea --- The chief of North Korea's powerful secret
police, long considered the right-hand man for the top leader, Kim
Jong-un, has been dismissed on charges of corruption and abuse of power,
the South Korean government said on Friday.

The firing of the chief, Gen. Kim Won-hong, as minister of state
security highlights the turmoil that has engulfed the upper reaches of
Mr. Kim's government. The general is the latest in a series of
high-ranking party and military officials Mr. Kim has
\href{https://www.nytimes.com/2016/09/01/world/asia/north-korea-executes-deputy-premier.html?_r=0}{fired,
demoted or executed} as he tried to consolidate his totalitarian power
through what South Korean officials and North Korean defectors have
called a ``reign of terror.''

General Kim was fired in mid-January after he was demoted to a one-star
general from a four-star one, said Jeong Joon-hee, a spokesman for the
South's Unification Ministry.

The general's surprise downfall was the latest indication that even top
lieutenants are at risk as Mr. Kim has rival agencies monitor one
another to detect and punish any sign of disrespect or disloyalty. Until
his dismissal, General Kim had been Mr. Kim's chief henchman in purging
potential enemies.

``Kim Won-hong has been a key aide to Kim Jong-un and has buttressed his
reign of terror,'' Mr. Jeong said. ``His dismissal could further deepen
unrest among officials and add to the instability of the regime by
weakening its control on the people.''

Mr. Jeong said General Kim was accused of corruption and held
responsible for various human rights violations, including torture,
committed at his agency. But other political machinations could be at
play behind the dismissal, Mr. Jeong said, citing speculation about a
rivalry among different power centers.

Mr. Jeong noted that the dismissal resulted from an investigation by
another powerful agency, the Organization and Guidance Department of the
governing Workers' Party of North Korea. The department supervises all
state agencies and is reportedly directly overseen by Mr. Kim.

Image

Gen. Kim Won-hong in an undated photo released by the North's Korean
Central News Agency last year.Credit...Korean Central News Agency, via
Reuters

``The Organization and Guidance Department is conducting an intensive
investigation of Kim Won-hong and the Ministry of State Security, so the
level of punishment and the number of people affected could be
expanded,'' Mr. Jeong said, without disclosing how the South Korean
government learned of the reported purge transpiring inside a secretive
regime.

The Ministry of State Security, which serves as the secret police and
intelligence agency in the North, is one of the most feared tools of
government there, responsible for arresting dissidents and running a
network of prison gulags.

When Mr. Kim
\href{http://www.nytimes.com/2013/12/14/world/asia/execution-raises-doubts-about-kims-grip-on-north-korea.html}{executed
his own uncle} and No. 2 official, Jang Song-thaek, on charges of
factionalism, corruption and plotting to overthrow his government in
2013, it was General Kim's ministry that arrested and court-martialed
Mr. Jang.

Since taking power after the death of his father in 2011, Mr. Kim has
frequently reshuffled the party and military elites as he has moved
swiftly to establish his monolithic authority in North Korea, which his
family has ruled for seven decades. Mr. Kim has executed at least 140
senior officials, usually killing them with machine guns and even
flamethrowers, according to the Institute for National Security
Strategy, a think tank affiliated with the South's National Intelligence
Service.

Thae Yong-ho, who was the North's No. 2 diplomat in London until his
defection to South Korea last summer, said he fled partly because of Mr.
Kim's
\href{https://www.nytimes.com/2017/01/25/world/asia/north-korea-defector.html}{reign
of terror}.

General Kim, 71, served in various top military positions under Mr. Kim,
said Cheong Seong-chang, a senior analyst at the Sejong Institute, a
South Korea research organization. He had been the top political officer
of the North's People's Army and chief of its general staff, as well as
its minister of armed forces.

Purges and executions remain an important feature of political life in
the North.

In 2015, South Korean officials said that Gen. Hyon Yong-chol, the
defense minister, was
\href{https://www.nytimes.com/2015/05/13/world/asia/north-korea-said-to-execute-a-top-official.html}{executed
with an antiaircraft gun} in Pyongyang, the North's capital, after he
dozed off during military events and second-guessed Mr. Kim's orders.

In August, they said Mr. Kim had found fault with a deputy premier's
``\href{https://www.nytimes.com/2016/09/01/world/asia/north-korea-executes-deputy-premier.html?_r=0}{disrespectful
posture}'' during a meeting and had him executed by a firing squad.

Advertisement

\protect\hyperlink{after-bottom}{Continue reading the main story}

\hypertarget{site-index}{%
\subsection{Site Index}\label{site-index}}

\hypertarget{site-information-navigation}{%
\subsection{Site Information
Navigation}\label{site-information-navigation}}

\begin{itemize}
\tightlist
\item
  \href{https://help.nytimes.com/hc/en-us/articles/115014792127-Copyright-notice}{©~2020~The
  New York Times Company}
\end{itemize}

\begin{itemize}
\tightlist
\item
  \href{https://www.nytco.com/}{NYTCo}
\item
  \href{https://help.nytimes.com/hc/en-us/articles/115015385887-Contact-Us}{Contact
  Us}
\item
  \href{https://www.nytco.com/careers/}{Work with us}
\item
  \href{https://nytmediakit.com/}{Advertise}
\item
  \href{http://www.tbrandstudio.com/}{T Brand Studio}
\item
  \href{https://www.nytimes.com/privacy/cookie-policy\#how-do-i-manage-trackers}{Your
  Ad Choices}
\item
  \href{https://www.nytimes.com/privacy}{Privacy}
\item
  \href{https://help.nytimes.com/hc/en-us/articles/115014893428-Terms-of-service}{Terms
  of Service}
\item
  \href{https://help.nytimes.com/hc/en-us/articles/115014893968-Terms-of-sale}{Terms
  of Sale}
\item
  \href{https://spiderbites.nytimes.com}{Site Map}
\item
  \href{https://help.nytimes.com/hc/en-us}{Help}
\item
  \href{https://www.nytimes.com/subscription?campaignId=37WXW}{Subscriptions}
\end{itemize}
