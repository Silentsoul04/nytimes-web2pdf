Sections

SEARCH

\protect\hyperlink{site-content}{Skip to
content}\protect\hyperlink{site-index}{Skip to site index}

\href{https://www.nytimes.com/section/world/europe}{Europe}

\href{https://myaccount.nytimes.com/auth/login?response_type=cookie\&client_id=vi}{}

\href{https://www.nytimes.com/section/todayspaper}{Today's Paper}

\href{/section/world/europe}{Europe}\textbar{}For Germany, Trump Poses a
Problem With No Clear Solution

\url{https://nyti.ms/2kErrSo}

\begin{itemize}
\item
\item
\item
\item
\item
\end{itemize}

Advertisement

\protect\hyperlink{after-top}{Continue reading the main story}

Supported by

\protect\hyperlink{after-sponsor}{Continue reading the main story}

\href{/column/the-interpreter}{The Interpreter}

\hypertarget{for-germany-trump-poses-a-problem-with-no-clear-solution}{%
\section{For Germany, Trump Poses a Problem With No Clear
Solution}\label{for-germany-trump-poses-a-problem-with-no-clear-solution}}

\includegraphics{https://static01.nyt.com/images/2017/02/07/world/07int-germany/07int-germany-articleInline.jpg?quality=75\&auto=webp\&disable=upscale}

By \href{https://www.nytimes.com/by/max-fisher}{Max Fisher}

\begin{itemize}
\item
  Feb. 6, 2017
\item
  \begin{itemize}
  \item
  \item
  \item
  \item
  \item
  \end{itemize}
\end{itemize}

BERLIN --- As allies across Europe and Asia adjust to changes brought by
President Trump, Germany is in a uniquely difficult position.

Its economy and national security are particularly reliant on American
support, which now seems in doubt, and on European unity, which is under
attack and increasingly up to Germany to maintain.

Yet Germany is constrained by the growing shakiness of allies like
Britain and perhaps even France, by the rise of its own far-right
populist movement and by lingering cultural sensitivities about any
policy that feels militaristic or hegemonic. These dynamics are not new,
but there is a growing tension between the role Germany feels
comfortable with, and the one it feels it needs to play on the world
stage.

A growing number of officials in Germany are asking whether they need a
Plan B for a post-American Europe. But they are finding that any such
plan would require costs and sacrifices almost as great as the
consequences of inaction.

\hypertarget{the-20th-century-is-gone}{%
\subsection{`The 20th century is gone'}\label{the-20th-century-is-gone}}

``We said farewell to your ambassador the other day,'' said Niels Annen,
a lawmaker with the center-left Social Democrats, who are part of the
governing coalition. ``He tried to reassure everybody but I think nobody
believed him.''

Over breakfast in the restored Reichstag, a soaring imperial-era
construction, Mr. Annen worried about ``a return to geopolitics in the
way that we saw in the 20th and maybe 19th centuries.''

A few tables over sat Frank-Walter Steinmeier, the former foreign
minister now designated to become president, who three days later would
declare that, with Mr. Trump's election,
``\href{http://www.bild.de/politik/inland/dr-frank-walter-steinmeier/die-alte-weltordnung-ist-vorueber-49896494.bild.html}{the
old world of the 20th century is gone}'' and that Germans had to prepare
for drastic changes.

The concern vexing Mr. Annen, and much of official Berlin, was that Mr.
Trump might not only withdraw American protection but also actively aid
Europe's growing internal and external threats.

``Someone like Mr. Bannon sitting in the White House who has contacts
with right-wing, up to fascist, groups here in Europe,'' Mr. Annen said,
referring to Mr. Trump's chief strategist, Stephen K. Bannon, ``is
really concerning.''

He also said he feared that Mr. Trump could seek to improve ties with
Moscow ``on Russian terms,'' cutting out Europe, and potentially
emboldening Russia's growing challenge to the continent's unity.

Other lawmakers worried that Mr. Trump was already undermining European
unity, for example by rewarding Britain's exit from the European Union
with promises of a speedy trade deal.

``What I find destabilizing is his announcement that he will
\href{https://insidetrade.com/daily-news/eu-member-states-suggest-trump-bilateral-overtures-product-white-house-influence}{make
deals} with national member states, because that will divide the E.U.,''
said Franziska Brantner, a lawmaker with the Green Party. She also said
she was worried that Mr. Trump was weakening Europe's collective defense
by questioning the value of the North Atlantic Treaty Organization.

\hypertarget{a-new-german-role}{%
\subsection{A new German role}\label{a-new-german-role}}

Most nations, facing such threats, would most likely be moving to
respond.

Germany is unusual. It has secured its place in the world by upholding
the liberal order through consensus-building and peacemaking.

``That's a nice idea,'' Ulrich Kühn, a fellow at the Carnegie Endowment
for International Peace, said of that model. But Germany ``is being
confronted with a reality where we cannot continue that way anymore.''

Most middle powers rely at least partly on traditional forms of power;
Britain has its military might, India its regional dominance, Israel its
nuclear program. Germany has explicitly avoided such assets, leaving it
to rely on soft power tools, particularly its economic strength, that
only work in the framework of a liberal European order that now looks
uncertain.

Now, policy elites and the public in Germany are struggling with whether
and how their nation should develop more traditional forms of power.

Germany has, in the past decade, grown beyond many of the taboos that
stem from the Nazi era and World War II. It leads on eurozone matters,
sends small numbers of troops on overseas NATO missions and has
organized recent sanctions against Russia. National pride remains a
touchy topic but one that can at least be discussed.

Still, the idea of Germany as a military power or even European hegemon
--- likely requirements for taking up the burdens and responsibilities
of a leading European power in the Trump era --- remains difficult.

But Germany may not have the luxury of time to reconcile its
contradictory feelings about its place in the world, especially with an
intransigent America and resurgent Russia, and a Europe rived by
populism.

``We still don't really have a clue who we are in the world and who we
want to be,'' said Jana Puglierin of the German Council on Foreign
Relations. That makes it difficult to face increasingly urgent
questions, she said, over ``what role we should play, who Germany is,
how dominant do we want to be.''

\hypertarget{were-running-out-of-partners}{%
\subsection{`We're running out of
partners'}\label{were-running-out-of-partners}}

As the United States openly questions the European Union's value, its
member states are plagued by populist backlashes that have made its
leaders less willing to address the eurozone and refugee crises at a
moment when they are most urgent.

Germany, as a result, is left carrying a greater share of the burden
just as it is becoming heaviest.

``We're running out of partners,'' Mr. Annen said.

But while the United States could tell Europe hard truths and pressure
its leaders to make difficult decisions, continental politics and memory
mean Germany cannot take over this role. Its regional stature allows it
enough power to push some policies on smaller states, but not enough to
force unity on all issues. And past exercises of power, like pushing
austerity plans that benefited Germany's own economy at the expense of
its poorer neighbors, have not enhanced its regional leadership.

``It will never be a with-us-or-against-us policy, because then these
countries will be against us,'' Mr. Annen said.

An even starker challenge is posed by Russia, which is staging
cyberattacks and aligning with populist movements across Europe. The
fear is that softening American and Western European defense commitments
would compel some Eastern European states to hedge against the alliance
and submit to a degree of Russian influence.

Should Europe's defense unity break under Russian pressure, analysts
worry, its economic and political unity would follow, leaving Germany
isolated at a time when it is unequipped to go it alone.

Eastern European states may look to Germany, whose economy is almost
triple the size of Russia's, to replace American security guarantees.
But the German military lacks many basics such as sufficient ammunition,
for which it relies on American forces. Even with rapid spending
increases, it would take years for the country to be able to play a
major European defense role, Mr. Kühn said.

``There is no substitute for the United States with regard to European
security,'' said Norbert Röttgen, a lawmaker with the Christian
Democratic Union, which is also the party of Chancellor Angela Merkel.

Should Germany seek to uphold European collective defense, the greatest
hurdle may be the German people.

In 2015, a
\href{http://www.pewglobal.org/2015/06/10/nato-publics-blame-russia-for-ukrainian-crisis-but-reluctant-to-provide-military-aid/}{Pew
poll} of NATO member states found that only 38 percent of Germans said
that their country should defend NATO allies bordering Russia if they
were attacked.

\hypertarget{an-american-threat}{%
\subsection{An American threat}\label{an-american-threat}}

Officials say that while their first priority is to establish friendly
relations with the Trump administration, they are not averse to hitting
back if the United States undermines the European Union.

``We cannot allow even our most important ally to dismantle the single
most historic achievement that we have,'' Mr. Annen said. ``That is
something that no government could accept without giving an answer.''

Behind closed doors, according to a senior German government official,
officials are preparing for the day that Berlin could be forced to treat
its longtime ally as a threat, necessitating radical changes in German
foreign policy.

The official asked to remain anonymous because of another predicament
Berlin faces with Mr. Trump: Its leaders must prepare Germany by
enunciating the stakes, but they fear that overtly stepping away from
Mr. Trump would anger him, risking the very breakup they wish to avoid.

Others are beginning to think about the day after.

Roderich Kiesewetter, a former military officer who is now a lawmaker
with the Christian Democratic Union, is among a small but growing group
pushing these questions into the public debate.

Germany should focus on persuading Mr. Trump to drop his hostility
toward Europe, Mr. Kiesewetter said, but ``we should not wait'' to
consider acting.

Mr. Kiesewetter hopes to hasten military integration across the European
Union. He acknowledged that a European-only defense against Russia would
be far weaker than the status quo. Still, he argued it could be a
sufficient deterrent --- if Germany takes enough of a role to bring
along the rest of Europe.

Though few lawmakers have joined Mr. Kiesewetter's public calls for
considering a post-American Europe, policy analysts say that such
discussions are becoming widespread in official Berlin.

Still, Mr. Kiesewetter is hardly optimistic. Should Mr. Trump strike a
rapprochement with Moscow that did not include European leaders, leaving
the continent on its own, he warned this would divide Eastern Europe
between ``zones of influence.''

This possibility seems to torment German officials, who sometimes label
it with the word ``Zwischen-Europa.'' The phrase, which means
``intermediary Europe'' or ``in-between Europe,'' comes from the
interwar era, when Germans used it to describe the borderlands between
it and the Soviet Union. It is remembered here as a partial cause of
World War II.

The phrase is used today not to specifically warn of war but to remind
Germans of the importance of the postwar order that many believe is in
growing peril. It is also a warning: that the liberal system could slip
away and that Germans must remember the dangers of the old order, even
if the rest of the world forgets.

Advertisement

\protect\hyperlink{after-bottom}{Continue reading the main story}

\hypertarget{site-index}{%
\subsection{Site Index}\label{site-index}}

\hypertarget{site-information-navigation}{%
\subsection{Site Information
Navigation}\label{site-information-navigation}}

\begin{itemize}
\tightlist
\item
  \href{https://help.nytimes.com/hc/en-us/articles/115014792127-Copyright-notice}{©~2020~The
  New York Times Company}
\end{itemize}

\begin{itemize}
\tightlist
\item
  \href{https://www.nytco.com/}{NYTCo}
\item
  \href{https://help.nytimes.com/hc/en-us/articles/115015385887-Contact-Us}{Contact
  Us}
\item
  \href{https://www.nytco.com/careers/}{Work with us}
\item
  \href{https://nytmediakit.com/}{Advertise}
\item
  \href{http://www.tbrandstudio.com/}{T Brand Studio}
\item
  \href{https://www.nytimes.com/privacy/cookie-policy\#how-do-i-manage-trackers}{Your
  Ad Choices}
\item
  \href{https://www.nytimes.com/privacy}{Privacy}
\item
  \href{https://help.nytimes.com/hc/en-us/articles/115014893428-Terms-of-service}{Terms
  of Service}
\item
  \href{https://help.nytimes.com/hc/en-us/articles/115014893968-Terms-of-sale}{Terms
  of Sale}
\item
  \href{https://spiderbites.nytimes.com}{Site Map}
\item
  \href{https://help.nytimes.com/hc/en-us}{Help}
\item
  \href{https://www.nytimes.com/subscription?campaignId=37WXW}{Subscriptions}
\end{itemize}
