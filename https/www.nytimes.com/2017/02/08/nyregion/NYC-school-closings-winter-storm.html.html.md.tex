Sections

SEARCH

\protect\hyperlink{site-content}{Skip to
content}\protect\hyperlink{site-index}{Skip to site index}

\href{https://www.nytimes.com/section/nyregion}{New York}

\href{https://myaccount.nytimes.com/auth/login?response_type=cookie\&client_id=vi}{}

\href{https://www.nytimes.com/section/todayspaper}{Today's Paper}

\href{/section/nyregion}{New York}\textbar{}Spring Weather? Snowstorm?
New York Schools to Close on Thursday

\url{https://nyti.ms/2kPC0C4}

\begin{itemize}
\item
\item
\item
\item
\item
\end{itemize}

Advertisement

\protect\hyperlink{after-top}{Continue reading the main story}

Supported by

\protect\hyperlink{after-sponsor}{Continue reading the main story}

\hypertarget{spring-weather-snowstorm-new-york-schools-to-close-on-thursday}{%
\section{Spring Weather? Snowstorm? New York Schools to Close on
Thursday}\label{spring-weather-snowstorm-new-york-schools-to-close-on-thursday}}

\href{https://www.nytimes.com/slideshow/2017/02/08/blogs/a-spring-day-in-february.html}{}

\hypertarget{a-spring-day-in-february}{%
\subsection{A Spring Day in February}\label{a-spring-day-in-february}}

6 Photos

View Slide Show ›

\includegraphics{https://static01.nyt.com/images/2017/02/08/blogs/09weather-ss-slide-8FHH/09weather-ss-slide-8FHH-articleLarge.jpg?quality=75\&auto=webp\&disable=upscale}

Joshua Bright for The New York Times

By \href{https://www.nytimes.com/by/eli-rosenberg}{Eli Rosenberg} and
\href{https://www.nytimes.com/by/christopher-mele}{Christopher Mele}

\begin{itemize}
\item
  Feb. 8, 2017
\item
  \begin{itemize}
  \item
  \item
  \item
  \item
  \item
  \end{itemize}
\end{itemize}

The topsy-turvy weather in New York City saw a record-setting 62 degrees
on Wednesday, but a significant snowstorm expected to move in overnight
prompted Mayor Bill de Blasio to close schools on Thursday.

The announcement came after the
\href{http://www.weather.gov/media/okx/Briefings/CoastalStormBriefing.pdf}{National
Weather Service} on Wednesday posted a winter storm warning for the city
and a blizzard warning for the eastern part of Long Island. Forecasts
called for 8 to 12 inches of snow throughout northern New Jersey, the
Hudson Valley, Long Island, coastal Connecticut and the city.

The heaviest snow was expected from early Thursday morning through
afternoon, with wind gusts up to 35 m.p.h., the service said on its
website.

Nationwide, airports were reporting more than 2,800 canceled flights for
Thursday, according to
\href{http://flightaware.com/live/cancelled/tomorrow/}{FlightAware.com},
a flight-tracking website. Newark Liberty International Airport had 603
canceled flights, La Guardia Airport, 566, and Kennedy International
Airport, 480, the website reported on Thursday night.

On Wednesday, the temperature reached 62 degrees in Central Park,
beating the previous record of 61 degrees set in 1965.

Around the city on Wednesday, people were taking advantage --- walking
around in T-shirts and tank tops, sitting outdoors at cafes, enjoying
parks and promenades.

For children who longed for a snow day, the news of the snowstorm was a
reason to cheer.

Libby Courtemanche had taken her two sons, Christopher, 2, and Bradley,
less than a year old, to a park in Huntington, Long Island, on
Wednesday.

``You know what's gonna happen tomorrow?'' she asked the 2-year-old.
``It's gonna snow. And we're gonna get to play in the snow.''

And this roller coaster ride of extremes is sure to provide fuel for
those who gripe about the weather. It will be unseasonably warm
Wednesday and unpleasantly wintry Thursday, but neither will be just
right, of course.

Tim Morrin, a meteorologist with the National Weather Service, said the
drastic shift in weather was ``unusual, but it's certainly not
unprecedented.''

The extremes should not be seen as a sign that the gods are angry. Mr.
Morrin said it could be explained by two competing weather patterns:
cold air masses descending from the North that will push out a
low-pressure, warm air mass in time to chill the city and turn
precipitation to snow.

``Air masses move,'' he said. ``It's just the timing.''

At Cozy Coffee in Bedford-Stuyvesant, Brooklyn, patrons headed straight
for the back patio, where owner Migdalia Medina ran cups of coffee and
small plates of food plates to people seated at picnic tables under a
canopy of bare tree branches.

``As soon as they feel it they start coming out,'' Ms. Medina said of
her customers. As for Thursday's forecast, she said the patio seating
``also looks good under snow.''

The New York weather historian Steve Fybish, who keeps records of the
city's weather dating to the 19th century, agreed that the weather swing
was not unusual. He rifled through his records to find other days of
similar extremes: a snowstorm and a 70-degree afternoon within a couple
of days in 1984; 69 degrees on a November day followed by five inches of
snow in 1896.

In February 2014, when the Super Bowl was at the Meadowlands, in East
Rutherford, N.J., the weather swung sharply. While there had
\href{https://www.nytimes.com/2014/02/03/sports/football/not-quite-75-and-sunny-but-a-mild-day-dispels-the-weather-worries.html}{been
concerns} that the outdoor game at MetLife Stadium would be affected by
the cold, game day registered a mild 49 degrees at the stadium and 57
degrees in the city. But the next day, the temperature plunged to 27
degrees followed by a cold snap that was punctuated by eight inches of
snow.

On Thursday, the snowfall is likely to be heaviest during the morning
commute, Mr. Morrin said.

``I don't think there's going to be anyone rushing anywhere,'' he said.
``The commute time will be impacted.''

Advertisement

\protect\hyperlink{after-bottom}{Continue reading the main story}

\hypertarget{site-index}{%
\subsection{Site Index}\label{site-index}}

\hypertarget{site-information-navigation}{%
\subsection{Site Information
Navigation}\label{site-information-navigation}}

\begin{itemize}
\tightlist
\item
  \href{https://help.nytimes.com/hc/en-us/articles/115014792127-Copyright-notice}{©~2020~The
  New York Times Company}
\end{itemize}

\begin{itemize}
\tightlist
\item
  \href{https://www.nytco.com/}{NYTCo}
\item
  \href{https://help.nytimes.com/hc/en-us/articles/115015385887-Contact-Us}{Contact
  Us}
\item
  \href{https://www.nytco.com/careers/}{Work with us}
\item
  \href{https://nytmediakit.com/}{Advertise}
\item
  \href{http://www.tbrandstudio.com/}{T Brand Studio}
\item
  \href{https://www.nytimes.com/privacy/cookie-policy\#how-do-i-manage-trackers}{Your
  Ad Choices}
\item
  \href{https://www.nytimes.com/privacy}{Privacy}
\item
  \href{https://help.nytimes.com/hc/en-us/articles/115014893428-Terms-of-service}{Terms
  of Service}
\item
  \href{https://help.nytimes.com/hc/en-us/articles/115014893968-Terms-of-sale}{Terms
  of Sale}
\item
  \href{https://spiderbites.nytimes.com}{Site Map}
\item
  \href{https://help.nytimes.com/hc/en-us}{Help}
\item
  \href{https://www.nytimes.com/subscription?campaignId=37WXW}{Subscriptions}
\end{itemize}
