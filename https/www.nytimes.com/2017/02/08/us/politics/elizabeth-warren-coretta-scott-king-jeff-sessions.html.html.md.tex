Sections

SEARCH

\protect\hyperlink{site-content}{Skip to
content}\protect\hyperlink{site-index}{Skip to site index}

\href{https://www.nytimes.com/section/politics}{Politics}

\href{https://myaccount.nytimes.com/auth/login?response_type=cookie\&client_id=vi}{}

\href{https://www.nytimes.com/section/todayspaper}{Today's Paper}

\href{/section/politics}{Politics}\textbar{}Coretta Scott King's 1986
Statement to the Senate About Jeff Sessions

\url{https://nyti.ms/2k3v1kK}

\begin{itemize}
\item
\item
\item
\item
\item
\end{itemize}

Advertisement

\protect\hyperlink{after-top}{Continue reading the main story}

Supported by

\protect\hyperlink{after-sponsor}{Continue reading the main story}

\hypertarget{coretta-scott-kings-1986-statement-to-the-senate-about-jeff-sessions}{%
\section{Coretta Scott King's 1986 Statement to the Senate About Jeff
Sessions}\label{coretta-scott-kings-1986-statement-to-the-senate-about-jeff-sessions}}

\includegraphics{https://static01.nyt.com/images/2017/02/07/us/08warren-mobile/08warren-mobile-videoSixteenByNine3000-v3.jpg}

By \href{http://www.nytimes.com/by/erin-mccann}{Erin McCann}

\begin{itemize}
\item
  Feb. 8, 2017
\item
  \begin{itemize}
  \item
  \item
  \item
  \item
  \item
  \end{itemize}
\end{itemize}

In 1986, Coretta Scott King wrote
\href{https://www.documentcloud.org/documents/3259988-Scott-King-1986-Letter-and-Testimony-Signed.html\#document/p1}{a
letter} in opposition to Jeff Sessions' nomination to become a federal
judge. It criticized Mr. Sessions's record on civil rights and said that
elevating him to higher office would demean the record of Mrs. King's
husband, the Rev. Dr. Martin Luther King Jr.

The letter resurfaced last month, when BuzzFeed News reported that
Senator Strom Thurmond, a Republican from South Carolina who was then
the chairman of the Judiciary Committee,
\href{https://www.buzzfeed.com/johnstanton/coretta-scott-king-letter-jeff-sessions?utm_term=.de3onv7Lr\#.vlMxGBN0d}{had
failed to enter it} into the Congressional Record. On the same day as
BuzzFeed's report, The Washington Post obtained the letter
\href{https://www.washingtonpost.com/news/powerpost/wp/2017/01/10/read-the-letter-coretta-scott-king-wrote-opposing-sessionss-1986-federal-nomination/?utm_term=.a4855b1d22bd}{and
published it online}.

Three decades ago, the letter was credited with
\href{http://www.nytimes.com/1986/06/06/us/senate-panel-hands-reagan-first-defeat-on-nominee-for-judgeship.html}{helping
to derail Mr. Sessions's nomination}. On Tuesday, it became a rallying
cry for Democrats who are opposed to his nod to become the United States
attorney general.

Image

Coretta Scott King in 1986.Credit...Ric Feld/Associated Press

Senator Elizabeth Warren, Democrat of Massachusetts, was trying to read
the letter aloud in the Senate chamber when her fellow senators,
accusing her of violating a rule that forbids one senator from demeaning
another, invoked a law,
\href{https://www.nytimes.com/2017/02/07/us/politics/republican-senators-vote-to-formally-silence-elizabeth-warren.html}{forcing
her to stop}.

The statement consists of two parts: a cover letter addressed to Mr.
Thurmond, which Ms. Warren did not read aloud, and the statement, part
of which Ms. Warren read on the Senate floor. She later
\href{https://www.facebook.com/senatorelizabethwarren/videos/724337794395383/}{read
it in full on Facebook Live}, uninterrupted. By Wednesday afternoon, her
video had been viewed more than seven million times.

The introduction.

\begin{quote}
Dear Senator Thurmond:

I write to express my sincere opposition to the confirmation of
Jefferson B. Sessions as a federal district court judge for the Southern
District of Alabama. My professional and personal roots in Alabama are
deep and lasting.

Anyone who has used the power of his office as United States Attorney to
intimidate and chill the free exercise of the ballot by citizens should
not be elevated to our courts.

Mr. Sessions has used the awesome powers of his office in a shabby
attempt to intimidate and frighten elderly black voters.

For this reprehensible conduct, he should not be rewarded with a federal
judgeship.

I regret that a long-standing commitment prevents me from appearing in
person to testify against this nominee. However, I have attached a copy
of my statement opposing Mr. Sessions' confirmation and I request that
my statement as well as this letter `be made a part of the' hearing
record.

I do sincerely urge you to oppose the confirmation of Mr. Sessions.

Sincerely,

Coretta Scott King
\end{quote}

Here is the text of the statement.

\textbf{Statement of Coretta Scott King on the Nomination of Jefferson
Beauregard Sessions III for the United States District Court Southern
District of Alabama}

\textbf{Senate Judiciary Committee}

\textbf{Thursday, March 13, 1986}

\begin{quote}
Mr. Chairman and Members of the Committee: Thank you for allowing me
this opportunity to express my strong opposition to the nomination of
Jefferson Sessions for a federal district judgeship for the Southern
District of Alabama. My longstanding commitment which I shared with my
husband, Martin, to protect and enhance the rights of Black Americans,
rights which include equal access to the democratic process, compels me
to testify today.

Civil rights leaders, including my husband and Albert Turner, have
fought long and hard to achieve free and unfettered access to the ballot
box. Mr. Sessions has used the awesome power of his office to chill the
free exercise of the vote by black citizens in the district he now seeks
to serve as a federal judge. This simply cannot be allowed to happen.
Mr. Sessions' conduct as U.S. Attorney, from his politically motivated
voting fraud prosecutions to his indifference toward criminal violations
of civil rights laws, indicates that he lacks the temperament, fairness
and judgment to be a federal judge.

The Voting Rights Act was, and still is, vitally important to the future
of democracy in the United States. I was privileged to join Martin and
many others during the Selma to Montgomery march for voting rights in
1965. Martin was particularly impressed by the determination to get the
franchise of blacks in Selma and neighboring Perry County. As he wrote,
``Certainly no community in the history of the Negro struggle has
responded with the enthusiasm of Selma and her neighboring town of
Marion. Where Birmingham depended largely upon students and unemployed
adults (to participate in non-violent protest of the denial of the
franchise), Selma has involved fully 10 percent of the Negro population
in active demonstrations, and at least half the Negro population of
Marion was arrested on one day.'' Martin was referring of course to a
group that included the defendants recently prosecuted for assisting
elderly and illiterate blacks to exercise that franchise. ln fact,
Martin anticipated from the depth of their commitment twenty years ago,
that a united political organization would remain in Perry County long
after the other marchers had left. This organization, the Perry County
Civic League, started by Mr. Turner, Mr. Hogue, and others as Martin
predicted, continued ``to direct the drive for votes and other rights.''
In the years since the Voting Rights Act was passed, Black Americans in
Marion, Selma and elsewhere have made important strides in their
struggle to participate actively in the electoral process. The number of
Blacks registered to vote in key Southern states has doubled since 1965.
This would not have been possible without the Voting Rights Act.

However, Blacks still fall far short of having equal participation in
the electoral process. Particularly in the South, efforts continue to be
made to deny Blacks access to the polls, even where Blacks constitute
the majority of the voters. It has been a long up-hill struggle to keep
alive the vital legislation that protects the most fundamental right to
vote. A person who has exhibited so much hostility to the enforcement of
those laws, and thus, to the exercise of those rights by Black people
should not be elevated to the federal bench.

The irony of Mr. Sessions' nomination is that, if confirmed, he will be
given life tenure for doing with a federal prosecution what the local
sheriffs accomplished twenty years ago with clubs and cattle prods.
Twenty years ago, when we marched from Selma to Montgomery, the fear of
voting was real, as the broken bones and bloody heads in Selma and
Marion bore witness. As my husband wrote at the time, ``it was not just
a sick imagination that conjured up the vision of a public official,
sworn to uphold the law, who forced an inhuman march upon hundreds of
Negro children; who ordered the Rev. James Bevel to be chained to his
sickbed; who clubbed a Negro woman registrant, and who callously
inflicted repeated brutalities and indignities upon nonviolent Negroes
peacefully petitioning for their constitutional right to vote.''

Free exercise of voting rights is so fundamental to American democracy
that we can not tolerate any form of infringement of those rights. Of
all the groups who have been disenfranchised in our nation's history,
none has struggled longer or suffered more in the attempt to win the
vote than Black citizens. No group has had access to the ballot box
denied so persistently and intently. Over the past century, a broad
array of schemes have been used in attempts to block the Black vote. The
range of techniques developed with the purpose of repressing black
voting rights run the gamut from the --- straightforward application of
brutality against black citizens who tried to vote to such legalized
frauds as ``grandfather clause'' exclusions and rigged literacy tests.

The actions taken by Mr. Sessions in regard to the 1984 voting fraud
prosecutions represent just one more technique used to intimidate Black
voters and thus deny them this most precious franchise. The
investigations into the absentee voting process were conducted only in
the Black Belt counties where blacks had finally achieved political
power in the local government. Whites had been using the absentee
process to their advantage for years, without incident. Then, when
Blacks realizing its strength, began to use it with success, criminal
investigations were begun.

In these investigations, Mr. Sessions, as U.S. Attorney, exhibited an
eagerness to bring to trial and convict three leaders of the Perry
County Civic League including Albert Turner despite evidence clearly
demonstrating their innocence of any wrongdoing. Furthermore, in
initiating the case, Mr. Sessions ignored allegations of similar
behavior by whites, choosing instead to chill the exercise of the
franchise by blacks by his misguided investigation. In fact, Mr.
Sessions sought to punish older black civil rights activists, advisors
and colleagues of my husband, who had been key figures in the civil
rights movement in the 1960's. These were persons who, realizing the
potential of the absentee vote among Blacks, had learned to use the
process within the bounds of legality and had taught others to do the
same. The only sin they committed was being too successful in gaining
votes.

The scope and character of the investigations conducted by Mr. Sessions
also warrant grave concern. Witnesses were selectively chosen in
accordance with the favorability of their testimony to the government's
case. Also, the prosecution illegally withheld from the defense critical
statements made by witnesses. Witnesses who did testify were pressured
and intimidated into submitting the ``correct'' testimony. Many elderly
blacks were visited multiple times by the FBI who then hauled them over
180 miles by bus to a grand jury in Mobile when they could more easily
have testified at a grand jury twenty miles away in Selma. These voters,
and others, have announced they are now never going to vote again.

I urge you to consider carefully Mr. Sessions' conduct in these matters.
Such a review, I believe, raises serious questions about his commitment
to the protection of the voting rights of all American citizens and
consequently his fair and unbiased judgment regarding this fundamental
right. When the circumstances and facts surrounding the indictments of
Al Turner, his wife, Evelyn, and Spencer Hogue are analyzed, it becomes
clear that the motivation was political, and the result frightening ---
the wide-scale chill of the exercise of the ballot for blacks, who
suffered so much to receive that right in the first place. Therefore, it
is my strongly-held view that the appointment of Jefferson Sessions to
the federal bench would irreparably damage the work of my husband, Al
Turner, and countless others who risked their lives and freedom over the
past twenty years to ensure equal participation in our democratic
system.

The exercise of the franchise is an essential means by which our
citizens ensure that those who are governing will be responsible. My
husband called it the number one civil right. The denial of access to
the ballot box ultimately results in the denial of other fundamental
rights. For, it ` is only when the poor and disadvantaged are empowered
that they are able to participate actively in the solutions to their own
problems.

We still have a long way to go before we can say that minorities no
longer need be concerned about discrimination at the polls. Blacks,
Hispanics, Native Americans and Asian Americans are grossly
underrepresented at every level of government in America. If we are
going to make our timeless dream of justice through democracy a reality,
we must take every possible step to ensure that the spirit and intent of
the Voting Rights Act of 1965 and the Fifteenth Amendment of the
Constitution is honored.

The federal courts hold a unique position in our constitutional system,
ensuring that minorities and other citizens without political power have
a forum in which to vindicate their rights. Because of his unique role,
it is essential that the people selected to be federal judges respect
the basic tenets of our legal system: respect for individual rights and
a commitment to equal justice for all. The integrity of the Courts, and
thus the rights they protect, can only be maintained if citizens feel
confident that those selected as federal judges will be able to judge
with fairness others holding differing views.

I do not believe Jefferson Sessions possesses the requisite judgment,
competence, and sensitivity to the rights guaranteed by the federal
civil rights laws to qualify for appointment to the federal district
court. Based on his record, I believe his confirmation would have a
devastating effect on not only the judicial system in Alabama, but also
on the progress we have made everywhere toward fulfilling my husband's
dream that he envisioned over twenty years ago. I therefore urge the
Senate Judiciary Committee to deny his confirmation.

I thank you for allowing me to share my views.
\end{quote}

Advertisement

\protect\hyperlink{after-bottom}{Continue reading the main story}

\hypertarget{site-index}{%
\subsection{Site Index}\label{site-index}}

\hypertarget{site-information-navigation}{%
\subsection{Site Information
Navigation}\label{site-information-navigation}}

\begin{itemize}
\tightlist
\item
  \href{https://help.nytimes.com/hc/en-us/articles/115014792127-Copyright-notice}{©~2020~The
  New York Times Company}
\end{itemize}

\begin{itemize}
\tightlist
\item
  \href{https://www.nytco.com/}{NYTCo}
\item
  \href{https://help.nytimes.com/hc/en-us/articles/115015385887-Contact-Us}{Contact
  Us}
\item
  \href{https://www.nytco.com/careers/}{Work with us}
\item
  \href{https://nytmediakit.com/}{Advertise}
\item
  \href{http://www.tbrandstudio.com/}{T Brand Studio}
\item
  \href{https://www.nytimes.com/privacy/cookie-policy\#how-do-i-manage-trackers}{Your
  Ad Choices}
\item
  \href{https://www.nytimes.com/privacy}{Privacy}
\item
  \href{https://help.nytimes.com/hc/en-us/articles/115014893428-Terms-of-service}{Terms
  of Service}
\item
  \href{https://help.nytimes.com/hc/en-us/articles/115014893968-Terms-of-sale}{Terms
  of Sale}
\item
  \href{https://spiderbites.nytimes.com}{Site Map}
\item
  \href{https://help.nytimes.com/hc/en-us}{Help}
\item
  \href{https://www.nytimes.com/subscription?campaignId=37WXW}{Subscriptions}
\end{itemize}
