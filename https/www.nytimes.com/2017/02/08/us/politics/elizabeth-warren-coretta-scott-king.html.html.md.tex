Sections

SEARCH

\protect\hyperlink{site-content}{Skip to
content}\protect\hyperlink{site-index}{Skip to site index}

\href{https://www.nytimes.com/section/politics}{Politics}

\href{https://myaccount.nytimes.com/auth/login?response_type=cookie\&client_id=vi}{}

\href{https://www.nytimes.com/section/todayspaper}{Today's Paper}

\href{/section/politics}{Politics}\textbar{}Shutting Down Speech by
Elizabeth Warren, G.O.P. Amplifies Her Message

\url{https://nyti.ms/2k4s7Mq}

\begin{itemize}
\item
\item
\item
\item
\item
\end{itemize}

Advertisement

\protect\hyperlink{after-top}{Continue reading the main story}

Supported by

\protect\hyperlink{after-sponsor}{Continue reading the main story}

\hypertarget{shutting-down-speech-by-elizabeth-warren-gop-amplifies-her-message}{%
\section{Shutting Down Speech by Elizabeth Warren, G.O.P. Amplifies Her
Message}\label{shutting-down-speech-by-elizabeth-warren-gop-amplifies-her-message}}

\includegraphics{https://static01.nyt.com/images/2017/02/09/us/09warren/09sessions-web2-videoSixteenByNine3000.jpg}

By \href{http://www.nytimes.com/by/matt-flegenheimer}{Matt Flegenheimer}

\begin{itemize}
\item
  Feb. 8, 2017
\item
  \begin{itemize}
  \item
  \item
  \item
  \item
  \item
  \end{itemize}
\end{itemize}

WASHINGTON --- Republicans seized her microphone. And gave her a
megaphone.

Silenced on the Senate floor for condemning a peer, Senator Elizabeth
Warren, Democrat of Massachusetts, emerged on Wednesday in a coveted
role: the avatar of liberal resistance in the age of President Trump.

Late on Tuesday, Senate Republicans
\href{https://www.nytimes.com/2017/02/07/us/politics/republican-senators-vote-to-formally-silence-elizabeth-warren.html}{voted
to halt the remarks} of Ms. Warren, already a lodestar of the left,
after she criticized a colleague, Senator Jeff Sessions, the nominee for
attorney general, by reading
\href{https://www.nytimes.com/2017/02/08/us/politics/elizabeth-warren-coretta-scott-king-jeff-sessions.html}{a
letter from Coretta Scott King}.

Instantly, the decision --- led by Senator Mitch McConnell, the majority
leader, who invoked a rarely enforced rule prohibiting senators from
impugning the motives and conduct of a peer --- amplified Ms. Warren's
message and further inflamed the angry Senate debate over Mr. Sessions'
nomination. He
\href{https://www.nytimes.com/2017/02/08/us/politics/jeff-sessions-attorney-general-confirmation.html}{was
confirmed on Wednesday}.

For Ms. Warren's supporters, it was the latest and most visceral example
of a woman muzzled by men who seemed unwilling to listen.

Critics saw something else: a senator who has rankled members of both
parties with her nose for the spotlight lobbing a far-too-early salvo in
the next presidential race.

``A lot of that's about 2020 politics,'' Senator John Thune, Republican
of South Dakota, grumbled on MSNBC.

Mr. McConnell's subsequent explanation for his maneuver seemed destined
for a future Warren campaign ad: ``She was warned. She was given an
explanation. Nevertheless, she persisted.'' After an unsuccessful effort
to draft her for the 2016 presidential race, Ms. Warren is considered a
very early front-runner for 2020, should she run.

Mr. McConnell's coda has already been repurposed as a sort of rallying
cry. Across social media, Ms. Warren's allies and supporters
\href{https://www.nytimes.com/2017/02/08/us/politics/elizabeth-warren-republicans-facebook-twitter.html}{posted
with the hashtag \#shepersisted}, calling to mind some Democrats'
embrace of the term ``nasty woman'' after
\href{https://www.nytimes.com/interactive/projects/cp/opinion/clinton-trump-third-debate-election-2016/donald-trumps-nasty-habits}{Mr.
Trump deployed it} to describe Hillary Clinton during a debate.
Appearing with Mrs. Clinton in New Hampshire in October, Ms. Warren
reminded Mr. Trump that ``nasty women vote.''

After the vote to bar Ms. Warren from speaking further about Mr.
Sessions, other senators, including Bernie Sanders of Vermont and Tom
Udall of New Mexico, read Mrs. King's letter without facing any
objection, prompting some activists to raise charges of sexism.

Ms. Warren has long displayed an instinct for capitalizing on highly
visible fights. After she was barred from speaking on the Senate floor,
she began reading the 1986 letter from Mrs. King on Facebook. By
Wednesday evening,
\href{https://www.facebook.com/senatorelizabethwarren/videos/724337794395383/}{the
video} had attracted more than nine million views.

In the letter, Mrs. King, the widow of the Rev. Martin Luther King Jr.,
took aim at Mr. Sessions's record on civil rights as a United States
attorney in Alabama, saying he had used ``the awesome power of his
office to chill the free exercise of the vote by black citizens.'' She
called on the Senate not to confirm Mr. Sessions to a federal judgeship,
and his nomination to that post was ultimately rejected.

On Wednesday morning, in a conference room in the Capitol --- the vote
prohibited Ms. Warren from speaking about the nomination only from the
Senate floor --- Ms. Warren addressed civil rights leaders, recounting
her long night.

``What hit me the hardest was, it is about silence,'' she said. ``It's
about trying to shut people up. It's about saying, `No, no, no, just go
ahead and vote.'''

She went on.

``This is going to be hard,'' she said. ``We don't have the tools.
There's going to be a lot that we will lose. But I guarantee, the one
thing we will not lose, we will not lose our voices.''

\includegraphics{https://static01.nyt.com/images/2017/02/09/us/09spicer-briefing/09spicer-briefing-videoSixteenByNine3000.jpg}

As Democrats strain to navigate the early days of the Trump presidency,
weighing the merits of the blanket opposition that many in their base
seem to crave, the latest rancor appeared to raise the likelihood of
further confrontation in the Senate chamber.

Some left-leaning groups seemed comfortable with that.

``What the public needs to see from Democrats right now is more backbone
and more standing on principle,'' said Adam Green, a co-founder of the
Progressive Change Campaign Committee. ``Elizabeth Warren continues to
be the model for good behavior.''

The timing is fortunate for Ms. Warren, whose fiery denunciations of
corporate greed have long made her a Democratic celebrity.

Her new book deal was announced this week. Its title: ``This Fight Is
Our Fight: The Battle to Save America's Middle Class.'' Shortly after
Mr. McConnell's objection on Tuesday, Ms. Warren called a favorite TV
anchor, MSNBC's Rachel Maddow, and spoke live on the air.

On Wednesday, Republicans betrayed no regret for their move, accusing
Ms. Warren of ignoring repeated warnings to avoid violating the Senate
rule, known as Rule XIX. She had also read a letter from Edward M.
Kennedy, who represented Massachusetts in the Senate, disparaging Mr.
Sessions.

``You don't insult --- whether it be from a letter, or from a message
from God, or on golden tablets,'' said Senator John McCain of Arizona.
``That's the rules of the Senate. They want to complain about it,
complain about it.''

Democrats and their allies resumed their protest against Mr. Sessions on
Wednesday with renewed swagger, despite their long odds of blocking his
confirmation.

``If Mr. McConnell or anybody else wants to deny me the right to debate
Jeff Sessions's qualifications, go for it,'' Mr. Sanders said from the
Senate floor hours before the vote.

Since the election, Mr. Sanders and Ms. Warren have been among the
lawmakers jockeying to be leading messengers for Democratic politics
under the Trump administration.

There have been bumps. Last month, Ms. Warren faced rare criticism from
liberals after voting in a Senate committee to approve Ben Carson as Mr.
Trump's secretary of housing and urban development, infuriating voters
who had hoped for uniform opposition to Mr. Trump's cabinet.

Defending herself on Facebook at the time, Ms. Warren
\href{https://www.facebook.com/senatorelizabethwarren/posts/716640075165155}{wrote
that she appreciated the feedback}. ``Unlike the new administration,''
she said, ``I don't believe in ignoring or silencing people who disagree
with the choices I make or the votes I take.''

This week, it seemed, all had been forgiven. MoveOn.org, the liberal
political group, said it had collected about \$300,000 in contributions
for Ms. Warren since Tuesday night.

And by midafternoon, a fund-raising email from Ms. Warren had arrived in
the inboxes of her supporters.

``I'm still banned from speaking on the Senate floor --- but there's
still time for you to make your voice heard,'' the email read, with a
link to a page for contributions.

She signed off with a familiar message: ``Keep fighting.''

Advertisement

\protect\hyperlink{after-bottom}{Continue reading the main story}

\hypertarget{site-index}{%
\subsection{Site Index}\label{site-index}}

\hypertarget{site-information-navigation}{%
\subsection{Site Information
Navigation}\label{site-information-navigation}}

\begin{itemize}
\tightlist
\item
  \href{https://help.nytimes.com/hc/en-us/articles/115014792127-Copyright-notice}{©~2020~The
  New York Times Company}
\end{itemize}

\begin{itemize}
\tightlist
\item
  \href{https://www.nytco.com/}{NYTCo}
\item
  \href{https://help.nytimes.com/hc/en-us/articles/115015385887-Contact-Us}{Contact
  Us}
\item
  \href{https://www.nytco.com/careers/}{Work with us}
\item
  \href{https://nytmediakit.com/}{Advertise}
\item
  \href{http://www.tbrandstudio.com/}{T Brand Studio}
\item
  \href{https://www.nytimes.com/privacy/cookie-policy\#how-do-i-manage-trackers}{Your
  Ad Choices}
\item
  \href{https://www.nytimes.com/privacy}{Privacy}
\item
  \href{https://help.nytimes.com/hc/en-us/articles/115014893428-Terms-of-service}{Terms
  of Service}
\item
  \href{https://help.nytimes.com/hc/en-us/articles/115014893968-Terms-of-sale}{Terms
  of Sale}
\item
  \href{https://spiderbites.nytimes.com}{Site Map}
\item
  \href{https://help.nytimes.com/hc/en-us}{Help}
\item
  \href{https://www.nytimes.com/subscription?campaignId=37WXW}{Subscriptions}
\end{itemize}
