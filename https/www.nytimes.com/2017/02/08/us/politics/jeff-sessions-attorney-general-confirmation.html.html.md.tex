Sections

SEARCH

\protect\hyperlink{site-content}{Skip to
content}\protect\hyperlink{site-index}{Skip to site index}

\href{https://www.nytimes.com/section/politics}{Politics}

\href{https://myaccount.nytimes.com/auth/login?response_type=cookie\&client_id=vi}{}

\href{https://www.nytimes.com/section/todayspaper}{Today's Paper}

\href{/section/politics}{Politics}\textbar{}Jeff Sessions Confirmed as
Attorney General, Capping Bitter Battle

\url{https://nyti.ms/2k4fbWS}

\begin{itemize}
\item
\item
\item
\item
\item
\item
\end{itemize}

Advertisement

\protect\hyperlink{after-top}{Continue reading the main story}

Supported by

\protect\hyperlink{after-sponsor}{Continue reading the main story}

\hypertarget{jeff-sessions-confirmed-as-attorney-general-capping-bitter-battle}{%
\section{Jeff Sessions Confirmed as Attorney General, Capping Bitter
Battle}\label{jeff-sessions-confirmed-as-attorney-general-capping-bitter-battle}}

\includegraphics{https://static01.nyt.com/images/2017/02/09/us/sub2-09SESSIONS/sub2-09SESSIONS-videoSixteenByNineJumbo1600.jpg}

By \href{http://www.nytimes.com/by/eric-lichtblau}{Eric Lichtblau} and
\href{http://www.nytimes.com/by/matt-flegenheimer}{Matt Flegenheimer}

\begin{itemize}
\item
  Feb. 8, 2017
\item
  \begin{itemize}
  \item
  \item
  \item
  \item
  \item
  \item
  \end{itemize}
\end{itemize}

WASHINGTON --- Senator Jeff Sessions was confirmed on Wednesday as
President Trump's attorney general, capping a bitter and racially
charged nomination battle that crested with the procedural silencing of
a leading Democrat, Senator Elizabeth Warren.

Mr. Sessions, an Alabama Republican, survived a near-party-line vote, 52
to 47, in the latest sign of the extreme partisanship at play as Mr.
Trump strains to install his cabinet. No Republicans broke ranks in
their support of a colleague who will become the nation's top law
enforcement official after two decades in the Senate.

But the confirmation process --- ferocious even by the standards of
moldering decorum that have defined the body's recent years --- laid
bare the Senate's deep divisions at the outset of the Trump presidency.
At the same time, the treatment of Ms. Warren, who was forced to stop
speaking late Tuesday after criticizing Mr. Sessions from the Senate
floor, rekindled the gender-infused politics that animated the
presidential election and the women's march protesting Mr. Trump the day
after his inauguration last month.

Mr. Sessions cast his final vote as a senator to note that he was
present for Wednesday's tally. His confirmation was met by applause from
his colleagues, including a few Democrats, on the Senate floor.

``I can't express how appreciative I am for those of you who stood by me
during this difficult time,'' Mr. Sessions said shortly after the vote.
``By your vote tonight, I have been given a real challenge. I'll do my
best to be worthy of it.''

Democrats spent the hours before the vote on Wednesday seething over the
rebuke of Ms. Warren, of Massachusetts, who had been barred from
speaking on the floor the previous night. Late Tuesday, Republicans
\href{https://www.nytimes.com/2017/02/08/us/politics/elizabeth-warren-coretta-scott-king.html}{voted
to formally silence Ms. Warren} after she read from a 1986 letter by
Coretta Scott King that criticized Mr. Sessions for using ``the awesome
power of his office to chill the free exercise of the vote by black
citizens'' while serving as a United States attorney in Alabama.

Since Mr. Trump announced his choice for attorney general, Mr.
Sessions's history with issues of race had assumed center stage. A
committee hearing on his nomination included searing indictments from
black Democratic lawmakers like Representative John Lewis of Georgia,
the civil rights icon, and Senator Cory Booker of New Jersey, who broke
with Senate tradition to testify against a peer.

\href{https://www.nytimes.com/interactive/2017/02/08/us/politics/jeff-sessions-confirmation-vote.html}{}

\includegraphics{https://static01.nyt.com/images/2017/02/08/us/politics/jeff-sessions-confirmation-vote-1486593238624/jeff-sessions-confirmation-vote-1486593238624-square640-v3.jpg}

\hypertarget{how-senators-voted-on-jeff-sessions}{%
\subsection{How Senators Voted on Jeff
Sessions}\label{how-senators-voted-on-jeff-sessions}}

The Senate voted to confirm Jeff Sessions as attorney general.

For weeks, Republicans rejected suggestions that Mr. Sessions could not
be trusted on civil rights, arguing that he had been tarnished unfairly
over accusations of racial insensitivity that have dogged him since the
1980s.

``Everybody in this body knows Senator Sessions well, knows that he is a
man of integrity, a man of principle,'' Senator Dan Sullivan, Republican
of Alaska, said during the debate on Wednesday afternoon. The
``twisting'' of Mr. Sessions's record offended him, he said, even as
Democrats continued their attacks on the nominee.

As the 84th attorney general, Mr. Sessions brings a sharply conservative
bent to the Justice Department and its 113,000 employees. A former
prosecutor, he promises a focus aligned with Mr. Trump in pushing a
``law and order'' agenda that includes tougher enforcement of laws on
immigration, drugs and gun trafficking.

\href{http://nyti.ms/2jq2IQn}{Civil rights advocates worry}, however,
that he will reverse steps taken by the Obama administration in the last
eight years to bring more accountability to police departments, state
and local governments, and employers. Advocates point to his history of
votes against various civil rights measures, as well as the accusations
of racial insensitivity.

Senator Patty Murray, a Washington Democrat, said on Wednesday that on
civil rights, immigration, abortion, criminal sentencing guidelines and
a range of other issues, Mr. Sessions had been far outside the
mainstream and had pushed ``extreme policies'' often targeting
minorities.

That criticism peaked with Tuesday night's rebuke of Ms. Warren, based
on an arcane Senate rule that prevents members from impugning the
character of a fellow senator, as she read the letter from Mrs. King,
the widow of the Rev. Dr. Martin Luther King Jr. Mrs. King wrote the
letter in response to Mr. Sessions's 1986 nomination for a federal
judgeship, for which he was ultimately rejected in part because of
accusations that he had been insensitive to minorities as a prosecutor.

Senator Mitch McConnell of Kentucky, the Republican majority leader, led
the objection against Ms. Warren. His explanation afterward --- ``She
was warned. She was given an explanation. Nevertheless, she persisted''
--- instantly became a liberal rallying cry, re-establishing Ms. Warren
as a leading voice of Democratic resistance to Mr. Trump.

``What hit me the hardest was, it is about silence,'' Ms. Warren told a
group of civil rights leaders on Wednesday at the Capitol. ``It's about
trying to shut people up. It's about saying: `No, no, no. Just go ahead
and vote.'''

\includegraphics{https://static01.nyt.com/images/2017/02/09/us/09spicer-briefing/09spicer-briefing-videoSixteenByNine3000.jpg}

Senator Chuck Schumer of New York, the Democratic leader, said on
Wednesday that the censure was ``totally, totally uncalled-for'' and
that it reflected an ``anti-free-speech attitude'' emanating from the
White House. He and other Democrats said it served to mute legitimate
criticism of Mr. Sessions's record on civil rights and racial issues ---
\href{http://nyti.ms/2j90h5z}{one of their main avenues of attack} at
his contentious nomination hearing last month.

The vote on Mr. Sessions came a day after Senate Republicans broke
through a bottleneck in Mr. Trump's nominees by
\href{https://www.nytimes.com/2017/02/07/us/politics/betsy-devos-education-secretary-confirmed.html}{approving
Betsy DeVos}, the embattled Republican donor, as education secretary
with the help of a tiebreaking vote by Vice President Mike Pence. With
Mr. Sessions's confirmation, votes are expected in coming days on the
nominations of Representative Tom Price of Georgia for secretary of
health and human services and Steven T. Mnuchin for Treasury secretary.

Mr. Sessions's path to confirmation hit another snag that riled
Democrats and energized opponents of his nomination: Mr. Trump's
dramatic firing of the acting leader of the Justice Department.

Last week, Mr. Trump
\href{https://www.nytimes.com/2017/01/30/us/politics/trump-immigration-ban-memo.html}{abruptly
dismissed Sally Q. Yates}, the acting attorney general, setting off a
fierce backlash from Democrats against Mr. Sessions's nomination to fill
her job permanently. Ms. Yates, a holdover from the Obama
administration, had refused to defend Mr. Trump's controversial order
barring travel by some foreigners, which is now tied up in litigation in
federal courts. Democrats seized on her firing to say that Mr. Sessions
is too close to the president to be independent or stand up to him.

As the first senator to support Mr. Trump's long-shot bid for president
last year, Mr. Sessions became an influential campaign adviser. While he
pledged repeatedly not to be ``a mere rubber stamp'' for the White
House, Democrats asserted that he would not be willing to challenge
legally questionable policies like the travel ban or the president's
threats to reinstitute the use of torture on terrorism suspects.

The arguments failed to sway any Republicans on the Senate Judiciary
Committee, which voted, 11 to 9,
\href{https://www.nytimes.com/2017/02/01/us/politics/jeff-sessions-approved-as-attorney-general-by-senate-committee.html}{along
party lines} last week to approve Mr. Sessions's nomination.

Senator Charles E. Grassley, the Iowa Republican who leads the Senate
Judiciary Committee, expressed confidence that Mr. Sessions would be a
``fair and evenhanded'' attorney general and would make good on his
pledges to enforce even the laws he voted against in the Senate.

``There should be no question,'' Mr. Grassley said, ``that he is more
than qualified to be the nation's top law enforcement officer.''

Advertisement

\protect\hyperlink{after-bottom}{Continue reading the main story}

\hypertarget{site-index}{%
\subsection{Site Index}\label{site-index}}

\hypertarget{site-information-navigation}{%
\subsection{Site Information
Navigation}\label{site-information-navigation}}

\begin{itemize}
\tightlist
\item
  \href{https://help.nytimes.com/hc/en-us/articles/115014792127-Copyright-notice}{©~2020~The
  New York Times Company}
\end{itemize}

\begin{itemize}
\tightlist
\item
  \href{https://www.nytco.com/}{NYTCo}
\item
  \href{https://help.nytimes.com/hc/en-us/articles/115015385887-Contact-Us}{Contact
  Us}
\item
  \href{https://www.nytco.com/careers/}{Work with us}
\item
  \href{https://nytmediakit.com/}{Advertise}
\item
  \href{http://www.tbrandstudio.com/}{T Brand Studio}
\item
  \href{https://www.nytimes.com/privacy/cookie-policy\#how-do-i-manage-trackers}{Your
  Ad Choices}
\item
  \href{https://www.nytimes.com/privacy}{Privacy}
\item
  \href{https://help.nytimes.com/hc/en-us/articles/115014893428-Terms-of-service}{Terms
  of Service}
\item
  \href{https://help.nytimes.com/hc/en-us/articles/115014893968-Terms-of-sale}{Terms
  of Sale}
\item
  \href{https://spiderbites.nytimes.com}{Site Map}
\item
  \href{https://help.nytimes.com/hc/en-us}{Help}
\item
  \href{https://www.nytimes.com/subscription?campaignId=37WXW}{Subscriptions}
\end{itemize}
