Sections

SEARCH

\protect\hyperlink{site-content}{Skip to
content}\protect\hyperlink{site-index}{Skip to site index}

\href{https://www.nytimes.com/section/world/asia}{Asia Pacific}

\href{https://myaccount.nytimes.com/auth/login?response_type=cookie\&client_id=vi}{}

\href{https://www.nytimes.com/section/todayspaper}{Today's Paper}

\href{/section/world/asia}{Asia Pacific}\textbar{}Trump and Abe to Talk:
Why They Are Meeting and What They'll Discuss

\url{https://nyti.ms/2k910js}

\begin{itemize}
\item
\item
\item
\item
\item
\end{itemize}

Advertisement

\protect\hyperlink{after-top}{Continue reading the main story}

Supported by

\protect\hyperlink{after-sponsor}{Continue reading the main story}

\hypertarget{trump-and-abe-to-talk-why-they-are-meeting-and-what-theyll-discuss}{%
\section{Trump and Abe to Talk: Why They Are Meeting and What They'll
Discuss}\label{trump-and-abe-to-talk-why-they-are-meeting-and-what-theyll-discuss}}

\includegraphics{https://static01.nyt.com/images/2017/02/10/world/asia/10japan3/10japan3-articleInline.jpg?quality=75\&auto=webp\&disable=upscale}

By \href{http://www.nytimes.com/by/motoko-rich}{Motoko Rich}

\begin{itemize}
\item
  Feb. 9, 2017
\item
  \begin{itemize}
  \item
  \item
  \item
  \item
  \item
  \end{itemize}
\end{itemize}

HONG KONG --- Prime Minister Shinzo Abe of Japan, the first world leader
to meet Donald J. Trump after the election
\href{https://www.nytimes.com/2016/11/17/world/asia/shinzo-abe-donald-trump.html}{in
November}, plans to meet with the president in Washington on Friday in
the Oval Office. Mr. Abe is then planning to fly to Mr. Trump's
Mar-a-Lago resort in Palm Beach, Fla., to play golf with the president.

\hypertarget{why-are-they-meeting}{%
\subsection{Why are they meeting?}\label{why-are-they-meeting}}

Mr. Abe wants to talk to Mr. Trump about trade and economic issues,
which have already been the subject of some
\href{https://www.nytimes.com/2017/01/06/business/trump-toyota-mexico-twitter-threat.html}{critical
Twitter postings} and
\href{https://www.nytimes.com/2017/02/02/upshot/should-dollar-rise-or-fall-the-trump-teams-message-is-garbled.html}{remarks}
from the president.

Mr. Abe is also eager to pursue a closer personal relationship with Mr.
Trump, with whom he shares some ideological leanings, including a
resistance to immigration.

\hypertarget{what-is-abe-most-worried-about}{%
\subsection{What is Abe most worried
about?}\label{what-is-abe-most-worried-about}}

With Mr. Trump warning Toyota of a
``\href{https://twitter.com/realDonaldTrump/status/817071792711942145}{big
border tax}'' on the company if it built a new plant in Mexico and
telling American business leaders that Japan
\href{https://www.nytimes.com/aponline/2017/02/01/world/asia/ap-as-japan-us-currency-.html}{purposely
devaluates the yen for economic advantage}, Mr. Abe is concerned about
how Mr. Trump's ``America First'' policies could affect Japanese
companies and the country's broader economy.

Now that Mr. Trump has formally
\href{https://www.nytimes.com/2017/01/23/us/politics/tpp-trump-trade-nafta.html}{abandoned
the Trans-Pacific Partnership} multilateral trade deal --- on which Mr.
Abe expended considerable domestic political capital --- Mr. Abe will be
looking to sound out Mr. Trump on the possibility of negotiating a
future bilateral trade deal between the two countries.

\hypertarget{how-will-abe-try-to-persuade-trump-that-economic-alliances-with-japan-could-also-benefit-the-us}{%
\subsection{How will Abe try to persuade Trump that economic alliances
with Japan could also benefit the
U.S.?}\label{how-will-abe-try-to-persuade-trump-that-economic-alliances-with-japan-could-also-benefit-the-us}}

Mr. Abe could point out that Japanese companies invest heavily in the
United States. According to the \href{http://keidanren.us/}{Japan
Business Federation}, Japanese companies have directly invested more
than \$400 billion in building factories and other facilities in the
United States, creating about 1.7 million jobs for American workers.

Mr. Abe has also indicated that Japanese companies or the government
could invest in American infrastructure. He told members of Parliament
that he wanted to sell Japan's bullet train technology in the United
States so that states could build high-speed rail links and create jobs
doing so.

\includegraphics{https://static01.nyt.com/images/2017/02/10/world/asia/10japan2/10japan2-articleInline.jpg?quality=75\&auto=webp\&disable=upscale}

\hypertarget{how-have-japanese-companies-reacted-to-trump}{%
\subsection{How have Japanese companies reacted to
Trump?}\label{how-have-japanese-companies-reacted-to-trump}}

After Mr. Trump attacked Toyota on Twitter, the automaker announced that
it would invest an additional \$10 billion in the United States over the
next five years, although it was not clear if that had already been
planned.

In December, SoftBank, the telecommunications and internet company, said
it would invest \$50 billion in the United States, a move that the
company's founder, Masayoshi Son, said would create 50,000 jobs. Again,
\href{https://www.nytimes.com/2016/12/06/business/dealbook/donald-trump-mayayoshi-son-softbank.html}{it
seems the investment is not entirely new}, but it comes from the
Japanese company's previously announced Vision fund, a \$100 billion
vehicle for investing in technology companies worldwide.

Other companies, including Sharp and Fuji Heavy Industries, have also
recently talked publicly about existing plans to expand or build new
factories in the United States, while Nisshinbo Holdings, an
environmental and energy conglomerate that makes auto parts, said it
would abandon a plan to build a car parts factory in Mexico.

\hypertarget{what-about-the-yen}{%
\subsection{What about the yen?}\label{what-about-the-yen}}

Mr. Abe has said that Mr. Trump's criticisms about the yen are
``undeserved.'' It is true that the value of the yen has fallen steadily
as Japan's central bank has injected cash into the economy, a policy
that Mr. Abe will most likely try to explain to the president.

\hypertarget{are-there-security-issues-the-leaders-might-want-to-discuss}{%
\subsection{Are there security issues the leaders might want to
discuss?}\label{are-there-security-issues-the-leaders-might-want-to-discuss}}

On the campaign trail, Mr. Trump assailed Japan for not paying enough
for its own defense. But during Defense Secretary Jim Mattis's recent
visit to Japan, he described the country as
``\href{https://www.nytimes.com/2017/02/04/world/asia/jim-mattis-defense-iran-persian-gulf.html}{a
model of cost sharing and burden sharing}'' and
\href{https://www.nytimes.com/2017/02/03/world/asia/us-japan-mattis-abe-defense.html}{assured
leaders there} that the United States would stand by its mutual defense
treaty and keep American troops on Okinawa and elsewhere in Japan. Mr.
Abe will most likely want to underscore that view and ensure that Mr.
Trump agrees.

\hypertarget{can-japan-pay-the-us-more-for-defense}{%
\subsection{Can Japan pay the U.S. more for
defense?}\label{can-japan-pay-the-us-more-for-defense}}

Maybe. There are signs that Japan might be willing to pay or do more,
but many experts contend that American troops are defending not only
Japan but also America's own interests in Asia.

The Pentagon is budgeted to spend about \$5.5 billion to support troops
and bases on Okinawa and elsewhere around Japan this year. Japan is set
to spend \$1.8 billion to support the bases, in addition to at least \$4
billion on related expenses, including compensation for the communities
that host the bases and money for relocating American troops.

Under its Constitution, which was written by American occupying forces
after World War II, Japan can keep an army for defensive purposes. But
Mr. Abe has said he wants to
\href{http://www.nytimes.com/2016/07/12/world/asia/japan-election-shinzo-abe.html}{revise
the Constitution and expand the military}. In August, his government
requested
\href{http://www.nytimes.com/2016/08/31/world/asia/japan-defense-military-budget-shinzo-abe.html}{the
latest in a series of increases in military spending}.

Advertisement

\protect\hyperlink{after-bottom}{Continue reading the main story}

\hypertarget{site-index}{%
\subsection{Site Index}\label{site-index}}

\hypertarget{site-information-navigation}{%
\subsection{Site Information
Navigation}\label{site-information-navigation}}

\begin{itemize}
\tightlist
\item
  \href{https://help.nytimes.com/hc/en-us/articles/115014792127-Copyright-notice}{©~2020~The
  New York Times Company}
\end{itemize}

\begin{itemize}
\tightlist
\item
  \href{https://www.nytco.com/}{NYTCo}
\item
  \href{https://help.nytimes.com/hc/en-us/articles/115015385887-Contact-Us}{Contact
  Us}
\item
  \href{https://www.nytco.com/careers/}{Work with us}
\item
  \href{https://nytmediakit.com/}{Advertise}
\item
  \href{http://www.tbrandstudio.com/}{T Brand Studio}
\item
  \href{https://www.nytimes.com/privacy/cookie-policy\#how-do-i-manage-trackers}{Your
  Ad Choices}
\item
  \href{https://www.nytimes.com/privacy}{Privacy}
\item
  \href{https://help.nytimes.com/hc/en-us/articles/115014893428-Terms-of-service}{Terms
  of Service}
\item
  \href{https://help.nytimes.com/hc/en-us/articles/115014893968-Terms-of-sale}{Terms
  of Sale}
\item
  \href{https://spiderbites.nytimes.com}{Site Map}
\item
  \href{https://help.nytimes.com/hc/en-us}{Help}
\item
  \href{https://www.nytimes.com/subscription?campaignId=37WXW}{Subscriptions}
\end{itemize}
