Sections

SEARCH

\protect\hyperlink{site-content}{Skip to
content}\protect\hyperlink{site-index}{Skip to site index}

\href{https://www.nytimes.com/section/world/middleeast}{Middle East}

\href{https://myaccount.nytimes.com/auth/login?response_type=cookie\&client_id=vi}{}

\href{https://www.nytimes.com/section/todayspaper}{Today's Paper}

\href{/section/world/middleeast}{Middle East}\textbar{}Trump May Turn to
Arab Allies for Help With Israeli-Palestinian Relations

\url{https://nyti.ms/2k8rCkp}

\begin{itemize}
\item
\item
\item
\item
\item
\end{itemize}

Advertisement

\protect\hyperlink{after-top}{Continue reading the main story}

Supported by

\protect\hyperlink{after-sponsor}{Continue reading the main story}

\hypertarget{trump-may-turn-to-arab-allies-for-help-with-israeli-palestinian-relations}{%
\section{Trump May Turn to Arab Allies for Help With Israeli-Palestinian
Relations}\label{trump-may-turn-to-arab-allies-for-help-with-israeli-palestinian-relations}}

\includegraphics{https://static01.nyt.com/images/2017/02/10/world/10diplo5-web/10diplo5-web-articleInline.jpg?quality=75\&auto=webp\&disable=upscale}

By \href{http://www.nytimes.com/by/peter-baker}{Peter Baker} and
\href{http://www.nytimes.com/by/mark-landler}{Mark Landler}

\begin{itemize}
\item
  Feb. 9, 2017
\item
  \begin{itemize}
  \item
  \item
  \item
  \item
  \item
  \end{itemize}
\end{itemize}

WASHINGTON --- President Trump and his advisers, venturing for the first
time into the fraught world of Middle East peacemaking, are developing a
strategy on the Israeli-Palestinian conflict that would enlist Arab
nations like Saudi Arabia and Egypt to break years of deadlock.

The emerging approach mirrors the thinking of Prime Minister Benjamin
Netanyahu of Israel, who will visit the United States next week, and
would build on his de facto alignment with Sunni Muslim countries in
trying to counter the rise of Shiite-led Iran. But Arab officials have
warned Mr. Trump and his advisers that if they want cooperation, the
United States cannot make life harder for them with provocative
pro-Israel moves.

The White House seems to be taking the advice. Mr. Trump delayed his
plan to move the United States Embassy to Jerusalem after Arab leaders
told him that doing so would cause angry protests among Palestinians,
who also claim the city as the capital of a future state. And after
meeting with King Abdullah II of Jordan last week, Mr. Trump
\href{https://www.nytimes.com/2017/02/02/world/middleeast/iran-missile-test-trump.html}{authorized
a statement} that, for the first time, cautioned Israel against building
new West Bank settlements beyond existing lines.

``There are some quite interesting ideas circulating on the potential
for U.S.-Israeli-Arab discussions on regional security in which
Israeli-Palestinian issues would play a significant role,'' said Robert
Satloff, the executive director of the Washington Institute for Near
East Policy. ``I don't know if this is going to ripen by next week, but
this stuff is out there.''

The discussions underscore the evolution of the new president's attitude
toward the Israeli-Palestinian conflict as he delves deeper into the
issue. During the campaign and the postelection transition, Mr. Trump
presented himself as an unstinting supporter of Israel who would quickly
move the embassy and support new settlement construction without
reservation. But he has tempered that to a degree.

The notion of recruiting Arab countries to help forge an agreement
between Israelis and Palestinians --- known as the ``outside-in''
approach --- is not a new one. As secretary of state under President
George Bush, James A. Baker III organized the first regional conference
in 1991 at which Arab leaders sat down with Israel's prime minister.
President George W. Bush invited Arab leaders to a summit meeting with
Israel in Annapolis, Md., in 2007. And President Barack Obama's first
special envoy, George Mitchell, spent months in 2009 trying to enlist
Arab partners in a joint effort.

\includegraphics{https://static01.nyt.com/images/2017/02/10/world/10diplo1-web/10diplo1-web-articleInline.jpg?quality=75\&auto=webp\&disable=upscale}

The difference is that in the last eight years, Israel has grown closer
to Sunni Arab nations because of their shared concern about Iranian
hegemony in the region, opening the possibility that this newfound, if
not always public, affiliation could change the dynamics.

``The logic of outside-in is that because the Palestinians are so weak
and divided --- and because there's a new, tacit relationship between
the Sunni Arabs and Israel --- there's the hope the Arabs would be
prepared to do more,'' said Dennis B. Ross, a Middle East peace
negotiator under several presidents, including Mr. Obama.

That is a departure from the countervailing assumption that if Israel
first made peace with the Palestinians, it would lead to peace with the
larger Arab world --- the ``inside-out'' approach. That was at the core
of President Bill Clinton's attempts to bring the two sides together and
was Mr. Obama's fallback position after his efforts to find Arab
partners failed.

Mr. Netanyahu, who is due at the White House on Wednesday, has been
talking about an outside-in approach for a while. His theory is that the
inside-out approach has failed. And so, he argues, if Israel can
transform its relationship with Sunni Arab nations, they can ultimately
lead the way toward a resolution with the Palestinians.

Jared Kushner, the senior White House adviser whom Mr. Trump has
assigned a major role in negotiations, has been intrigued by this logic,
according to people who have spoken with him. Mr. Kushner has grown
close to Ron Dermer, the Israeli ambassador and a close confidant of Mr.
Netanyahu's. Mr. Trump and Mr. Kushner also had dinner at the White
House on Thursday night with Sheldon Adelson, the casino magnate, who is
a key supporter of Mr. Netanyahu.

A series of telephone conversations and personal meetings with Arab and
regional leaders in recent weeks have also shaped Mr. Kushner's thinking
and that of the president. Mr. Trump has talked with President Abdel
Fattah el-Sisi of Egypt; King Salman of Saudi Arabia; Sheikh Mohammed
bin Zayed al-Nahyan, the crown prince of Abu Dhabi, in the United Arab
Emirates; and President Recep Tayyip Erdogan of Turkey. Mr. Kushner has
also met with Arab officials, including Yousef Al Otaiba, the ambassador
from the United Arab Emirates.

Image

The United States Embassy in Tel Aviv. Mr. Trump delayed his plan to
move the embassy to Jerusalem after Arab leaders told him that doing so
would lead to protests among Palestinians.Credit...Jack Guez/Agence
France-Presse --- Getty Images

King Abdullah II of Jordan seems to have played a particularly pivotal
role. Concerned that an embassy move would anger the many Palestinians
living in his country, the king rushed to Washington without an
invitation, in a gamble that he could see Mr. Trump. He visited first
with Vice President Mike Pence, who had him over for breakfast at his
official residence last week. The king appealed to the administration's
fixation with the Islamic State, arguing that the United States should
not alienate Arab allies who could help.

Several days later, the king buttonholed Mr. Trump on the sidelines of
the National Prayer Breakfast and made a similar case. He advised
against a radical shift in American policy and emphasized the risks that
Jordan would face if Israel were to become even more assertive about
building settlements, according to people who spoke with Mr. Kushner and
Stephen K. Bannon, the chief White House strategist.

Mr. Trump had already decided by that point to slow down the embassy
move --- a decision that did not especially trouble Mr. Netanyahu and
his team, who, while publicly supporting a move, privately urged caution
to avoid a violent backlash. The administration had also received
reports from American diplomats in Jordan that the threat level for a
terrorist attack there had been raised to the highest level in years.

But a series of announcements of
\href{https://www.nytimes.com/2017/02/01/world/middleeast/israel-3000-homes-west-bank.html}{new
settlement construction} worried some White House officials, who thought
Mr. Netanyahu was taking action without first meeting with Mr. Trump.

Within hours of Mr. Trump's meeting with King Abdullah, the
administration leaked a statement to The Jerusalem Post saying, ``We
urge all parties from taking unilateral actions that could undermine our
ability to make progress, including settlement announcements.''

After that was posted online, the White House issued a public statement
with softened language: ``While we don't believe the existence of
settlements is an impediment to peace, the construction of new
settlements or the expansion of existing settlements beyond their
current borders may not be helpful in achieving that goal.''

Image

King Abdullah II of Jordan, far left; Secretary of State Rex W.
Tillerson; and the White House adviser Jared Kushner, right, at the
National Prayer Breakfast in Washington last week.Credit...Stephen
Crowley/The New York Times

It was worded in a way that let different parties focus on different
parts. The ``may not be helpful'' phrase was the first time Mr. Trump
had warned against new housing in the West Bank.

But the ``beyond their current borders'' phrase suggested a return to
George W. Bush's policy of essentially acquiescing to additional
construction within existing settlement blocs as long as Israel did not
expand their geographical reach or build entirely new settlements.
Elliott Abrams, one of the authors of that policy under Mr. Bush,
\href{https://www.nytimes.com/2017/02/06/us/politics/donald-trump-elliott-abrams.html}{is
poised} to become deputy secretary of state under Mr. Trump.

Mr. Netanyahu's team focused on that part of the statement. ``I happen
to know they were very pleased with the statement because it was such a
contrast from Obama,'' said Morton A. Klein, the national president of
the Zionist Organization of America, who has been supportive of the
Trump administration.

Indeed, undeterred, Mr. Netanyahu's coalition
\href{https://www.nytimes.com/2017/02/06/world/middleeast/israel-settlement-law-palestinians-west-bank.html}{pushed
through Parliament a bill} to retroactively authorize thousands of homes
in the West Bank that even under Israeli law had been built illegally on
Palestinian-owned land.

Mr. Klein, who argues that settlements are not an obstacle to peace,
said the White House had made the statement too confusing to provide
clear direction. ``I did find it ambiguous, and not as clear as I would
like it to be,'' he said.

The challenge now is whether Mr. Trump can use this ambiguity to his
benefit. If the United States can extract gestures from the Arabs, then
that could provide a basis for Israelis and Palestinians to make
compromises that they could not do by themselves, Mr. Ross said.

``You'd have to have some kind of parallel approach,'' he said. ``This
would be a serious investment of diplomacy to probe what is possible.''

Advertisement

\protect\hyperlink{after-bottom}{Continue reading the main story}

\hypertarget{site-index}{%
\subsection{Site Index}\label{site-index}}

\hypertarget{site-information-navigation}{%
\subsection{Site Information
Navigation}\label{site-information-navigation}}

\begin{itemize}
\tightlist
\item
  \href{https://help.nytimes.com/hc/en-us/articles/115014792127-Copyright-notice}{©~2020~The
  New York Times Company}
\end{itemize}

\begin{itemize}
\tightlist
\item
  \href{https://www.nytco.com/}{NYTCo}
\item
  \href{https://help.nytimes.com/hc/en-us/articles/115015385887-Contact-Us}{Contact
  Us}
\item
  \href{https://www.nytco.com/careers/}{Work with us}
\item
  \href{https://nytmediakit.com/}{Advertise}
\item
  \href{http://www.tbrandstudio.com/}{T Brand Studio}
\item
  \href{https://www.nytimes.com/privacy/cookie-policy\#how-do-i-manage-trackers}{Your
  Ad Choices}
\item
  \href{https://www.nytimes.com/privacy}{Privacy}
\item
  \href{https://help.nytimes.com/hc/en-us/articles/115014893428-Terms-of-service}{Terms
  of Service}
\item
  \href{https://help.nytimes.com/hc/en-us/articles/115014893968-Terms-of-sale}{Terms
  of Sale}
\item
  \href{https://spiderbites.nytimes.com}{Site Map}
\item
  \href{https://help.nytimes.com/hc/en-us}{Help}
\item
  \href{https://www.nytimes.com/subscription?campaignId=37WXW}{Subscriptions}
\end{itemize}
