Sections

SEARCH

\protect\hyperlink{site-content}{Skip to
content}\protect\hyperlink{site-index}{Skip to site index}

\href{https://www.nytimes.com/section/politics}{Politics}

\href{https://myaccount.nytimes.com/auth/login?response_type=cookie\&client_id=vi}{}

\href{https://www.nytimes.com/section/todayspaper}{Today's Paper}

\href{/section/politics}{Politics}\textbar{}Trump Campaign Aides Had
Repeated Contacts With Russian Intelligence

\url{https://nyti.ms/2lhzcxv}

\begin{itemize}
\item
\item
\item
\item
\item
\item
\end{itemize}

Advertisement

\protect\hyperlink{after-top}{Continue reading the main story}

Supported by

\protect\hyperlink{after-sponsor}{Continue reading the main story}

\hypertarget{trump-campaign-aides-had-repeated-contacts-with-russian-intelligence}{%
\section{Trump Campaign Aides Had Repeated Contacts With Russian
Intelligence}\label{trump-campaign-aides-had-repeated-contacts-with-russian-intelligence}}

\includegraphics{https://static01.nyt.com/images/2017/02/15/us/15intercepts/15intercepts-articleLarge.jpg?quality=75\&auto=webp\&disable=upscale}

By \href{http://www.nytimes.com/by/michael-s-schmidt}{Michael S.
Schmidt}, \href{http://www.nytimes.com/by/mark-mazzetti}{Mark Mazzetti}
and \href{http://www.nytimes.com/by/matt-apuzzo}{Matt Apuzzo}

\begin{itemize}
\item
  Feb. 14, 2017
\item
  \begin{itemize}
  \item
  \item
  \item
  \item
  \item
  \item
  \end{itemize}
\end{itemize}

WASHINGTON --- Phone records and intercepted calls show that members of
Donald J. Trump's 2016 presidential campaign and other Trump associates
had repeated contacts with senior Russian intelligence officials in the
year before the election, according to four current and former American
officials.

American law enforcement and intelligence agencies intercepted the
communications around the same time they were discovering evidence that
Russia was trying to disrupt the presidential election by hacking into
the Democratic National Committee, three of the officials said. The
intelligence agencies then sought to learn whether the Trump campaign
was colluding with the Russians on the hacking or other efforts to
influence the election.

The officials interviewed in recent weeks said that, so far, they had
seen no evidence of such cooperation.

But the intercepts alarmed American intelligence and law enforcement
agencies, in part because of the amount of contact that was occurring
while Mr. Trump was
\href{https://www.nytimes.com/2016/09/09/us/politics/hillary-clinton-donald-trump-putin.html}{speaking
glowingly} about the Russian president, Vladimir V. Putin. At one point
last summer, Mr. Trump
\href{http://www.nytimes.com/2016/07/28/us/politics/donald-trump-russia-clinton-emails.html}{said
at a campaign event} that he hoped Russian intelligence services had
stolen Hillary Clinton's emails and would make them public.

The officials said the intercepted communications were not limited to
Trump campaign officials, and included other associates of Mr. Trump. On
the Russian side, the contacts also included members of the government
outside of the intelligence services, they said. All of the current and
former officials spoke on the condition of anonymity because the
continuing investigation is classified.

The officials said that one of the advisers picked up on the calls was
Paul Manafort, who was Mr. Trump's campaign chairman for several months
last year and had worked as a political consultant in Ukraine. The
officials declined to identify the other Trump associates on the calls.

The call logs and intercepted communications are part of a larger trove
of information that the F.B.I. is sifting through as it investigates the
links between Mr. Trump's associates and the Russian government, as well
as
\href{https://www.nytimes.com/2017/01/11/us/politics/trumps-press-conference-highlights-russia.html}{the
hacking of the D.N.C.}, according to federal law enforcement officials.
As part of its inquiry, the F.B.I. has obtained banking and travel
records and conducted interviews, the officials said.

Mr. Manafort, who has not been charged with any crimes, dismissed the
officials' accounts in a telephone interview on Tuesday. ``This is
absurd,'' he said. ``I have no idea what this is referring to. I have
never knowingly spoken to Russian intelligence officers, and I have
never been involved with anything to do with the Russian government or
the Putin administration or any other issues under investigation
today.''

He added, ``It's not like these people wear badges that say, `I'm a
Russian intelligence officer.'''

Several of Mr. Trump's associates, like Mr. Manafort, have done business
in Russia. And it is not unusual for American businessmen to come in
contact with foreign intelligence officials, sometimes unwittingly, in
countries like Russia and Ukraine, where the spy services are deeply
embedded in society. Law enforcement officials did not say to what
extent the contacts might have been about business.

The officials would not disclose many details, including what was
discussed on the calls, the identity of the Russian intelligence
officials who participated, and how many of Mr. Trump's advisers were
talking to the Russians. It is also unclear whether the conversations
had anything to do with Mr. Trump himself.

A report from American intelligence agencies that was made public in
January concluded that the Russian government
\href{https://www.nytimes.com/2016/12/09/us/obama-russia-election-hack.html}{had
intervened in the election} in part to help Mr. Trump, but did not
address whether any members of the Trump campaign had participated in
the effort.

The intercepted calls are different from
\href{https://www.nytimes.com/2017/02/09/us/flynn-is-said-to-have-talked-to-russians-about-sanctions-before-trump-took-office.html}{the
wiretapped conversations last year} between Michael T. Flynn, Mr.
Trump's former national security adviser, and Sergey I. Kislyak,
Russia's ambassador to the United States. In those calls, which led to
\href{https://www.nytimes.com/2017/02/13/us/politics/donald-trump-national-security-adviser-michael-flynn.html}{Mr.
Flynn's resignation} on Monday night, the two men discussed sanctions
that the Obama administration imposed on Russia in December.

But the cases are part of American intelligence and law enforcement
agencies' routine electronic surveillance of the communications of
foreign officials.

The F.B.I. declined to comment. The White House also declined to comment
Tuesday night, but earlier in the day, the press secretary, Sean Spicer,
stood by Mr. Trump's previous comments that nobody from his campaign had
contact with Russian officials before the election.

``There's nothing that would conclude me that anything different has
changed with respect to that time period,'' Mr. Spicer said in response
to a question.

Two days after the election in November, Sergei A. Ryabkov, the deputy
Russian foreign minister,
\href{https://www.nytimes.com/2016/11/11/world/europe/trump-campaign-russia.html}{said}
``there were contacts'' during the campaign between Russian officials
and Mr. Trump's team.

``Obviously, we know most of the people from his entourage,'' Mr.
Ryabkov told Russia's Interfax news agency.

The Trump transition team denied Mr. Ryabkov's statement. ``This is not
accurate,'' Hope Hicks, a spokeswoman for Mr. Trump, said at the time.

The National Security Agency, which monitors the communications of
foreign intelligence services, initially captured the calls between Mr.
Trump's associates and the Russians as part of routine foreign
surveillance. After that, the F.B.I. asked the N.S.A. to collect as much
information as possible about the Russian operatives on the phone calls,
and to search through troves of previous intercepted communications that
had not been analyzed.

The F.B.I. has closely examined at least three other people close to Mr.
Trump, although it is unclear if their calls were intercepted. They are
\href{https://www.nytimes.com/2016/12/08/world/europe/carter-page-donald-trump-moscow-russia.html}{Carter
Page}, a businessman and former foreign policy adviser to the campaign;
Roger Stone, a longtime Republican operative; and Mr. Flynn.

All of the men have strongly denied that they had any improper contacts
with Russian officials.

As part of the inquiry, the F.B.I. is also trying to assess the
credibility of the information contained in a dossier that was given to
the bureau last year by a former British intelligence operative. The
dossier
\href{https://www.nytimes.com/2017/01/10/us/politics/donald-trump-russia-intelligence.html}{contained
a raft of allegations} of a broad conspiracy between Mr. Trump, his
associates and the Russian government. It also included unsubstantiated
claims that the Russians had embarrassing videos that could be used to
blackmail Mr. Trump.

The F.B.I. has spent several months investigating the leads in the
dossier, but has yet to confirm any of its most explosive claims.

Senior F.B.I. officials believe that the former British intelligence
officer who compiled the dossier, Christopher Steele,
\href{https://www.nytimes.com/2017/01/11/us/politics/donald-trump-russia-intelligence.html}{has
a credible track record}, and he briefed investigators last year about
how he obtained the information. One American law enforcement official
said that F.B.I. agents had made contact with some of Mr. Steele's
sources.

The agency's investigation of Mr. Manafort began last spring as an
outgrowth of a criminal investigation into
\href{https://www.nytimes.com/2016/08/01/us/paul-manafort-ukraine-donald-trump.html}{his
work} for a pro-Russian political party in Ukraine and for the country's
former president, Viktor F. Yanukovych. It has focused on why he was in
such close contact with Russian and Ukrainian intelligence officials.

The bureau did not have enough evidence to obtain a warrant for a
wiretap of Mr. Manafort's communications, but it had the N.S.A.
scrutinize the communications of Ukrainian officials he had met.

The F.B.I. investigation is proceeding at the same time that separate
investigations into Russian interference in the election are gaining
momentum on Capitol Hill. Those investigations, by the House and Senate
Intelligence Committees, are examining not only the Russian hacking but
also any contacts that Mr. Trump's team had with Russian officials
during the campaign.

On Tuesday, top Republican lawmakers said that Mr. Flynn should be one
focus of the investigation, and that he should be called to testify
before Congress. Senator Mark Warner of Virginia, the top Democrat on
the Intelligence Committee, said the news about Mr. Flynn underscored
``how many questions still remain unanswered to the American people more
than three months after Election Day, including who was aware of what,
and when.''

Mr. Warner said Mr. Flynn's resignation would not stop the committee
``from continuing to investigate General Flynn, or any other campaign
official who may have had inappropriate and improper contacts with
Russian officials prior to the election.''

Advertisement

\protect\hyperlink{after-bottom}{Continue reading the main story}

\hypertarget{site-index}{%
\subsection{Site Index}\label{site-index}}

\hypertarget{site-information-navigation}{%
\subsection{Site Information
Navigation}\label{site-information-navigation}}

\begin{itemize}
\tightlist
\item
  \href{https://help.nytimes.com/hc/en-us/articles/115014792127-Copyright-notice}{©~2020~The
  New York Times Company}
\end{itemize}

\begin{itemize}
\tightlist
\item
  \href{https://www.nytco.com/}{NYTCo}
\item
  \href{https://help.nytimes.com/hc/en-us/articles/115015385887-Contact-Us}{Contact
  Us}
\item
  \href{https://www.nytco.com/careers/}{Work with us}
\item
  \href{https://nytmediakit.com/}{Advertise}
\item
  \href{http://www.tbrandstudio.com/}{T Brand Studio}
\item
  \href{https://www.nytimes.com/privacy/cookie-policy\#how-do-i-manage-trackers}{Your
  Ad Choices}
\item
  \href{https://www.nytimes.com/privacy}{Privacy}
\item
  \href{https://help.nytimes.com/hc/en-us/articles/115014893428-Terms-of-service}{Terms
  of Service}
\item
  \href{https://help.nytimes.com/hc/en-us/articles/115014893968-Terms-of-sale}{Terms
  of Sale}
\item
  \href{https://spiderbites.nytimes.com}{Site Map}
\item
  \href{https://help.nytimes.com/hc/en-us}{Help}
\item
  \href{https://www.nytimes.com/subscription?campaignId=37WXW}{Subscriptions}
\end{itemize}
