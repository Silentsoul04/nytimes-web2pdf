Sections

SEARCH

\protect\hyperlink{site-content}{Skip to
content}\protect\hyperlink{site-index}{Skip to site index}

\href{https://www.nytimes.com/section/world/europe}{Europe}

\href{https://myaccount.nytimes.com/auth/login?response_type=cookie\&client_id=vi}{}

\href{https://www.nytimes.com/section/todayspaper}{Today's Paper}

\href{/section/world/europe}{Europe}\textbar{}Russia Deploys Missile,
Violating Treaty and Challenging Trump

\url{https://nyti.ms/2lg8khs}

\begin{itemize}
\item
\item
\item
\item
\item
\end{itemize}

Advertisement

\protect\hyperlink{after-top}{Continue reading the main story}

Supported by

\protect\hyperlink{after-sponsor}{Continue reading the main story}

\hypertarget{russia-deploys-missile-violating-treaty-and-challenging-trump}{%
\section{Russia Deploys Missile, Violating Treaty and Challenging
Trump}\label{russia-deploys-missile-violating-treaty-and-challenging-trump}}

\includegraphics{https://static01.nyt.com/images/2017/02/15/us/15missiles/15missiles-articleLarge.jpg?quality=75\&auto=webp\&disable=upscale}

By \href{http://www.nytimes.com/by/michael-r-gordon}{Michael R. Gordon}

\begin{itemize}
\item
  Feb. 14, 2017
\item
  \begin{itemize}
  \item
  \item
  \item
  \item
  \item
  \end{itemize}
\end{itemize}

WASHINGTON --- Russia has secretly deployed a new cruise missile that
American officials say violates a landmark arms control treaty, posing a
major test for President Trump as his administration is facing a crisis
over its ties to Moscow.

The new Russian missile deployment also comes as the Trump
administration is struggling to fill key policy positions at the State
Department and the Pentagon --- and to settle on a permanent replacement
for Michael T. Flynn, the national security adviser who resigned late
Monday. Mr. Flynn stepped down after it was revealed that he had misled
the vice president and other officials over conversations with Moscow's
ambassador to Washington.

The ground-launched cruise missile at the center of American concerns is
one that the Obama administration said in 2014 had been tested in
violation of a 1987 treaty that bans American and Russian
intermediate-range missiles based on land.

The Obama administration had sought to persuade the Russians to correct
the violation while the missile was still in the test phase. Instead,
the Russians have moved ahead with the system, deploying a fully
operational unit.

Administration officials said the Russians now have two battalions of
the prohibited cruise missile. One is still located at Russia's missile
test site at Kapustin Yar in southern Russia near Volgograd. The other
was shifted in December from that test site to an operational base
elsewhere in the country, according to a senior official who did not
provide further details and requested anonymity to discuss recent
intelligence reports about the missile.

American officials had called the cruise missile the SSC-X-8. But the
``X'' has been removed from intelligence reports, indicating that
American intelligence officials consider the missile to be operational
and no longer a system in development.

The missile program has been a major concern for the Pentagon, which has
developed options for how to respond, including deploying additional
missile defenses in Europe or developing air-based or sea-based cruise
missiles.

Russia's actions are politically significant, as well.

It is very unlikely that the Senate, which is already skeptical of
President Vladimir V. Putin's intentions, would agree to ratify a new
strategic arms control accord unless the alleged violation of the
intermediate-range treaty is corrected. Mr. Trump has said the United
States should ``strengthen and expand its nuclear capability.'' But at
the same time, he has talked of reaching a new arms agreement with
Moscow that would reduce arms ``very substantially.''

The deployment of the system could also substantially increase the
military threat to NATO nations, depending on where the highly mobile
system is based and how many more batteries are deployed in the future.
Jim Mattis, the United States defense secretary, is scheduled to meet
with allied defense ministers in Brussels on Wednesday.

Before he left his post last year as the NATO commander and retired from
the military, Gen. Philip M. Breedlove warned that deployment of the
cruise missile would be a militarily significant development that
``can't go unanswered.''

Coming up with an arms control solution would not be easy. Each missile
battalion is believed to have four mobile launchers with about half a
dozen nuclear-tipped missiles allocated to each of the launchers. The
mobile launcher for the cruise missile, however, closely resembles the
mobile launcher used for the Iskander, a nuclear-tipped short-range
system that is permitted under treaties.

``This will make location and verification really tough,'' General
Breedlove said in an interview.

While senior Trump administration officials have not said where the new
unit is based, there has been speculation in press reports that a
missile system with similar characteristics is deployed in central
Russia.

American and Russian relations were on a better footing in December 1987
when President Ronald Reagan and Mikhail S. Gorbachev, the Soviet
leader, signed an arms accord, formally known as the Intermediate-Range
Nuclear Forces Treaty and commonly called the I.N.F. treaty.

As a result of the agreement, Russia and the United States destroyed
2,692 missiles. The missiles the Russians destroyed included the SS-20.
The Americans destroyed their Pershing II ballistic missiles and
ground-launched cruise missiles, which were based in Western Europe.

``We can only hope that this history-making agreement will not be an end
in itself but the beginning of a working relationship that will enable
us to tackle the other urgent issues before us,'' Mr. Reagan said at the
time.

But the Russians developed buyer's remorse. During the George W. Bush
administration, Sergei B. Ivanov, the Russian defense minister,
suggested that the treaty be dropped because Russia still faced threats
from nations on its periphery, including China.

The Bush administration, however, was reluctant to terminate a treaty
that NATO nations valued and whose abrogation would have enabled Russia
to build up forces that could potentially be directed at the United
States' allies in Asia, as well.

In June 2013, Mr. Putin complained that ``nearly all of our neighbors
are developing these kinds of weapons systems'' and described the Soviet
Union's decision to conclude the I.N.F. treaty as ``debatable to say the
least.''

Russia began testing the cruise missile as early as 2008. Rose
Gottemoeller, who was the State Department's top arms control official
during the Obama administration and is now the deputy secretary general
of NATO, first raised the alleged violation with Russian officials in
2013.

After years of frustration, the United States convened a November 2016
meeting in Geneva of a special verification commission established under
the treaty to deal with compliance issues.
\href{https://www.nytimes.com/2016/10/20/world/europe/russia-missiles-inf-treaty.html}{It
was the first meeting in 13 years of the commission}, whose members
include the United States, Russia and three former Soviet republics that
are also party to the accord: Belarus, Kazakhstan and Ukraine.

But Russia denied it had breached the treaty and responded with its own
allegations of American violations, which the Americans asserted were
spurious.

The Obama administration argued that it was in the United States'
interest to preserve the treaty. Having failed to persuade the Russians
to fix the alleged violation, some military experts say, the United
States needs to ratchet up the pressure by announcing plans to expand
missile defenses in Europe and deploy sea-based or air-based nuclear
missiles.

``We have strong tools like missile defense and counterstrike, and we
should not take any of them off the table,'' General Breedlove said.

Franklin C. Miller, a longtime Pentagon official who served on the
National Security Council under Mr. Bush, said the Russian military may
see the cruise missile as a way to expand its target coverage in Europe
and China so it can free its strategic nuclear forces to concentrate on
targets in the United States.

``Clearly, the Russian military thinks this system is very important,
important enough to break the treaty,'' Mr. Miller said.

But he cautioned against responding in kind by seeking to deploy new
American intermediate-range nuclear missiles in Europe.

``The last thing NATO needs is a bruising debate as we had in the late
'70s and early '80s about new missile deployments in Europe,'' Mr.
Miller added. ``The United States should build up its missile defense in
Europe. But if the United States wants to deploy a military response, it
should be sea-based.''

Jon Wolfsthal, who served as a nuclear weapons expert on the National
Security Council during the Obama administration, said the United
States, its NATO allies, Japan and South Korea needed to work together
to put pressure on Russia to correct the violation. The response,
\href{https://twitter.com/JBWolfsthal/status/831585510418157569}{he
wrote on Twitter}, should be taken by the ``alliance as a whole.''

The Trump administration is in the beginning stages of reviewing nuclear
policy and has not said how it plans to respond.

``We do not comment on intelligence matters,'' Mark Toner, the acting
State Department spokesman, said. ``We have made very clear our concerns
about Russia's violation, the risks it poses to European and Asian
security, and our strong interest in returning Russia to compliance with
the treaty.''

Advertisement

\protect\hyperlink{after-bottom}{Continue reading the main story}

\hypertarget{site-index}{%
\subsection{Site Index}\label{site-index}}

\hypertarget{site-information-navigation}{%
\subsection{Site Information
Navigation}\label{site-information-navigation}}

\begin{itemize}
\tightlist
\item
  \href{https://help.nytimes.com/hc/en-us/articles/115014792127-Copyright-notice}{©~2020~The
  New York Times Company}
\end{itemize}

\begin{itemize}
\tightlist
\item
  \href{https://www.nytco.com/}{NYTCo}
\item
  \href{https://help.nytimes.com/hc/en-us/articles/115015385887-Contact-Us}{Contact
  Us}
\item
  \href{https://www.nytco.com/careers/}{Work with us}
\item
  \href{https://nytmediakit.com/}{Advertise}
\item
  \href{http://www.tbrandstudio.com/}{T Brand Studio}
\item
  \href{https://www.nytimes.com/privacy/cookie-policy\#how-do-i-manage-trackers}{Your
  Ad Choices}
\item
  \href{https://www.nytimes.com/privacy}{Privacy}
\item
  \href{https://help.nytimes.com/hc/en-us/articles/115014893428-Terms-of-service}{Terms
  of Service}
\item
  \href{https://help.nytimes.com/hc/en-us/articles/115014893968-Terms-of-sale}{Terms
  of Sale}
\item
  \href{https://spiderbites.nytimes.com}{Site Map}
\item
  \href{https://help.nytimes.com/hc/en-us}{Help}
\item
  \href{https://www.nytimes.com/subscription?campaignId=37WXW}{Subscriptions}
\end{itemize}
