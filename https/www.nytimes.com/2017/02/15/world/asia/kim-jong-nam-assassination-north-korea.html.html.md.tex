Sections

SEARCH

\protect\hyperlink{site-content}{Skip to
content}\protect\hyperlink{site-index}{Skip to site index}

\href{https://www.nytimes.com/section/world/asia}{Asia Pacific}

\href{https://myaccount.nytimes.com/auth/login?response_type=cookie\&client_id=vi}{}

\href{https://www.nytimes.com/section/todayspaper}{Today's Paper}

\href{/section/world/asia}{Asia Pacific}\textbar{}Kim Jong-nam, the
Hunted Heir to a Dictator Who Met Death in Exile

\url{https://nyti.ms/2liPia1}

\begin{itemize}
\item
\item
\item
\item
\item
\end{itemize}

Advertisement

\protect\hyperlink{after-top}{Continue reading the main story}

Supported by

\protect\hyperlink{after-sponsor}{Continue reading the main story}

\hypertarget{kim-jong-nam-the-hunted-heir-to-a-dictator-who-met-death-in-exile}{%
\section{Kim Jong-nam, the Hunted Heir to a Dictator Who Met Death in
Exile}\label{kim-jong-nam-the-hunted-heir-to-a-dictator-who-met-death-in-exile}}

\includegraphics{https://static01.nyt.com/images/2017/02/16/world/16korea-1/16korea-1-articleLarge.jpg?quality=75\&auto=webp\&disable=upscale}

By \href{http://www.nytimes.com/by/choe-sang-hun}{Choe Sang-Hun} and
\href{https://www.nytimes.com/by/richard-c-paddock}{Richard C. Paddock}

\begin{itemize}
\item
  Feb. 15, 2017
\item
  \begin{itemize}
  \item
  \item
  \item
  \item
  \item
  \end{itemize}
\end{itemize}

SEOUL, South Korea --- When North Korea held a state funeral for its
leader, Kim Jong-il, in 2011, one son was conspicuously absent.

The absence of Kim Jong-nam --- the eldest son of the family, who was
bound by Korean tradition to preside over the funeral --- was all the
evidence outside analysts needed to see how isolated he had become from
the center of power in North Korea, the world's most secretive regime.

Never fully accepted by his family, sidelined by his powerful stepmother
and haunted by fears of assassins, Mr. Kim lived much of his life
wandering abroad, in Moscow, Geneva, Beijing, Paris and Macau, the
Chinese gambling enclave.

On Monday, Mr. Kim, 45, met his end at Kuala Lumpur International
Airport in Malaysia. According to the National Intelligence Service of
South Korea, he was poisoned by two women who appeared to be carrying
out an assassination order from Pyongyang, the North Korean capital. Mr.
Kim died on his way to the hospital. Two women have been detained in
connection with the killing.

It remains uncertain if Mr. Kim was traveling alone or if bodyguards
were present. It was also unclear how many people were involved in the
attack.

\href{http://www.thestar.com.my/news/nation/2017/02/15/clearer-image-of-killer-caught-on-cctv/}{Grainy
footage} released on Wednesday showed a woman suspected of being one of
the assassins, who appeared to be of Asian descent and wore a shirt
emblazoned with ``LOL'' in large letters, before she fled the airport.

The Royal Malaysia Police announced late Wednesday afternoon that they
had arrested a woman that morning and that she had been carrying a
Vietnamese passport in Terminal 2, where the attack occurred. They said
she was ``positively identified'' from closed-circuit video, and was
alone at the time of her arrest.

Image

Mr. Kim, the oldest of three known sons of Kim Jong-il, the late
dictator, has been the closest thing North Korea has had to an
international playboy.Credit...Shizuo Kambayashi/Associated Press

She was identified as Doan Thin Hoang, 28, according to the inspector
general of the police, Khalid Abu Bakar.

On Thursday, the Malaysian police said they had detained a second
suspect, a woman with an Indonesian passport. A police official told the
Bernama news agency that more arrests were expected.

The authorities also said that an autopsy on Mr. Kim had been completed.

There were no markers or police tape at Terminal 2 on Wednesday to
indicate that a crime had been committed. Airport workers said that they
had been ordered not to discuss the case.

South Korea's acting president, Hwang Kyo-ahn, said on Wednesday that
his government was working with the Malaysian authorities to find the
assailants. But officials in Seoul quickly pointed fingers at Mr. Kim's
half brother, the North Korean leader Kim Jong-un, who has ordered the
\href{https://www.nytimes.com/2015/04/30/world/asia/north-korea-executed-15-top-officials-in-2015-south-korean-agency-says.html}{executions
of a number of senior officials}, including
\href{http://www.nytimes.com/2013/12/24/world/asia/north-korea-purge.html}{his
own uncle}, who have been deemed potential challenges to his authority.

Ever since Kim Jong-un succeeded his father in 2011, ``there has been a
standing order'' to assassinate his half brother, Lee Byung-ho, the
director of the South's National Intelligence Service, said during a
closed-door briefing at the National Assembly, according to lawmakers
who attended it.

``This is not a calculated action to remove Kim Jong-nam because he was
a challenge to power per se, but rather reflected Kim Jong-un's
paranoia,'' Mr. Lee was quoted as saying.

Kim Jong-un wanted his half brother killed, Mr. Lee said, and there was
an assassination attempt against him in 2012. Mr. Kim was so afraid of
assassins that he begged for his life in a letter to his half brother in
2012.

\includegraphics{https://static01.nyt.com/images/2017/02/16/world/16korea-2/16korea-2-articleLarge.jpg?quality=75\&auto=webp\&disable=upscale}

``Please withdraw the order to punish me and my family,'' Mr. Kim was
quoted as saying in the letter. ``We have nowhere to hide. The only way
to escape is to choose suicide.''

Mr. Lee said that Kim Jong-nam had no power base inside North Korea,
where Kim Jong-un had swiftly established his monolithic rule with what
the South called a reign of terror.

Kim Jong-nam arrived in Malaysia last week, Mr. Lee said. He was in line
at the airport to check in for a flight to Macau on Monday morning when
he was attacked by the two women, Mr. Lee said, citing security camera
footage from the airport. The women fled the airport in a taxi, Mr. Lee
said.

If North Korea's involvement is proved, Washington could face intense
pressure to put the country back on its list of nations that sponsor
terrorism, said Cheong Seong-chang, an analyst at the Sejong Institute,
a think tank in South Korea.

North Korea was first put on the terrorism list after the South caught a
woman from the North who confessed to planting a bomb on a South Korean
airliner that exploded over the Indian Ocean, near Myanmar, in 1987. The
North was taken off the list in 2008, after a deal aimed at ending its
nuclear program.

South Korea's military plans to use loudspeakers along the shared Korean
border to inform North Koreans of Mr. Kim's killing and of their
government's brutality, a South Korean news agency, Yonhap, reported on
Wednesday. The Defense Ministry declined to confirm the report.

``By assassinating Kim Jong-nam, Kim Jong-un may have removed a thorn in
the side, but it will further isolate his country,'' Mr. Cheong said.
``It is also expected to worsen his country's relations with China,
which has been protecting his brother.''

Image

In a 1981 family portrait, Kim Jong-il with his oldest son, Kim
Jong-nam, front right; his sister-in-law Sung Hye-rang, left rear; and
her daughter, Lee Nam-ok, and son, Lee Il-nam.Credit...Agence
France-Presse --- Getty Images

Kim Jong-nam's life illuminates the hidden intrigue in the Kim family,
which has ruled North Korea for almost seven decades.

While the lives of the rest of the family remained shrouded in mystery,
Mr. Kim, the oldest of three known sons of Kim Jong-il, has been the
closest thing the isolated Stalinist state has had to an international
playboy.

He was often seen with fashionably dressed women in international
airports and
\href{https://mobile.nytimes.com/2007/02/01/world/asia/01iht-macao.4431509.html}{spent
much of his time in casinos in Macau}, where he also kept an expensive
house.

Outside analysts often saw him as a possible candidate to replace Kim
Jong-un if the North Korean leadership imploded and China, traditionally
an ally, sought a replacement in its client state.

Chinese experts on North Korea said they doubted that Kim Jong-nam had
special security protection from Beijing.

``Chinese elites had no expectation this guy could play an important
political role,'' said Cheng Xiaohe, an associate professor of
international relations at Renmin University. ``If China wanted to use
him as an alternative leader, China would have offered good protection,
but this assassination shows he had no security protection.''

In Macau, where Mr. Kim was headed, he was safe just by being there,
said Zhang Baohui, director of the Center for Asian Pacific Studies at
Lingnan University in Hong Kong. ``Macau is part of China and is a safe
haven in itself,'' he said.

Image

The North Korean Embassy in Kuala Lumpur on Wednesday. If North Korea's
involvement in Kim Jong-nam's death is proved, the United States could
face intense pressure to put the country back on its list of
terrorism-sponsoring countries.Credit...Ahmad Yusni/European Pressphoto
Agency

Mr. Kim was a prince in exile with little chance of returning home,
analysts and officials in South Korea said. His wife and a daughter and
son are in Macau under Chinese protection, Mr. Lee said.

The South Korean intelligence agency did not disclose how it had
obtained the letter from Mr. Kim begging his half brother to spare his
life. But government sources said that emails Mr. Kim sent home through
North Korean embassies had been obtained in a hacking operation. In one
of the emails, they said, Mr. Kim bitterly complained that the North
Korean government stopped sending him cash after his father died and Kim
Jong-un took over. In 2012, a news report said Mr. Kim was thrown out of
a luxury Macau hotel, unable to pay a \$15,000 bill.

The Kim family has
\href{http://www.nytimes.com/2011/12/23/world/asia/family-intrigue-shadows-north-koreas-secretive-dynasty.html}{never
been known for its togetherness}.

Kim Jong-nam's mother, Sung Hae-rim, a decorated ``people's actress,''
was already married and the mother of a child when Kim Jong-il forced
her to divorce her novelist husband to marry him. Kim Jong-il adored his
first son, Kim Jong-nam. He once seated his young son at his desk and
told him, ``This is the place where you will one day give orders,''
according to Lee Han-young, a relative who defected to the South in
1982.

But Kim Jong-nam's grandfather, the North's founding president, Kim
Il-sung, never approved of the marriage.

``My father was keeping highly secret the fact that he was living with
my mother, who was married, a famous movie actress, so I couldn't get
out of the house or make friends,'' Mr. Kim was quoted as saying in a
2012 book by a Japanese journalist. ``That solitude from childhood may
have made me what I am now, preferring freedom.''

Mr. Kim was born in secret, and when his mother fell out of favor with
Kim Jong-il and was forced to live in Moscow, he was left in the care of
her sister. He was later sent to Geneva, where he learned English and
French. (His mother was alone in Moscow when she died in 2002.)

Kim Jong-il would later begin a relationship with Ko Young-hee, a star
of Pyongyang's premier opera, who gave birth to Kim Jong-chol and then
Kim Jong-un. According to a Japanese sushi chef who published a 2003
memoir about his experience working for the Kim family, Kim Jong-un was
by that time the father's favorite.

Kim Jong-nam squandered what little chance he may have had to succeed
his father when he embarrassed Pyongyang in 2001; he was
\href{http://www.nytimes.com/2001/05/04/world/japan-deports-man-said-to-be-north-korean-leader-s-son.html}{caught
trying to enter Japan} on a fake passport from the Dominican Republic.
He told Japanese investigators that he wanted to visit Tokyo Disneyland.

But rumors of intrigue never left Mr. Kim, as analysts speculated that
if the young, inexperienced Kim Jong-un failed to meet the expectations
of hard-line generals, they might summon home the eldest brother. In a
way, Mr. Kim helped fuel such rumors.

In the 2012 book by the Japanese journalist, Mr. Kim called his younger
brother ``a figurehead.''

Advertisement

\protect\hyperlink{after-bottom}{Continue reading the main story}

\hypertarget{site-index}{%
\subsection{Site Index}\label{site-index}}

\hypertarget{site-information-navigation}{%
\subsection{Site Information
Navigation}\label{site-information-navigation}}

\begin{itemize}
\tightlist
\item
  \href{https://help.nytimes.com/hc/en-us/articles/115014792127-Copyright-notice}{©~2020~The
  New York Times Company}
\end{itemize}

\begin{itemize}
\tightlist
\item
  \href{https://www.nytco.com/}{NYTCo}
\item
  \href{https://help.nytimes.com/hc/en-us/articles/115015385887-Contact-Us}{Contact
  Us}
\item
  \href{https://www.nytco.com/careers/}{Work with us}
\item
  \href{https://nytmediakit.com/}{Advertise}
\item
  \href{http://www.tbrandstudio.com/}{T Brand Studio}
\item
  \href{https://www.nytimes.com/privacy/cookie-policy\#how-do-i-manage-trackers}{Your
  Ad Choices}
\item
  \href{https://www.nytimes.com/privacy}{Privacy}
\item
  \href{https://help.nytimes.com/hc/en-us/articles/115014893428-Terms-of-service}{Terms
  of Service}
\item
  \href{https://help.nytimes.com/hc/en-us/articles/115014893968-Terms-of-sale}{Terms
  of Sale}
\item
  \href{https://spiderbites.nytimes.com}{Site Map}
\item
  \href{https://help.nytimes.com/hc/en-us}{Help}
\item
  \href{https://www.nytimes.com/subscription?campaignId=37WXW}{Subscriptions}
\end{itemize}
