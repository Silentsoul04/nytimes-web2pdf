Sections

SEARCH

\protect\hyperlink{site-content}{Skip to
content}\protect\hyperlink{site-index}{Skip to site index}

\href{https://www.nytimes.com/section/world/europe}{Europe}

\href{https://myaccount.nytimes.com/auth/login?response_type=cookie\&client_id=vi}{}

\href{https://www.nytimes.com/section/todayspaper}{Today's Paper}

\href{/section/world/europe}{Europe}\textbar{}Defense Secretary Mattis
Tells NATO Allies to Spend More, or Else

\url{https://nyti.ms/2lPyyFw}

\begin{itemize}
\item
\item
\item
\item
\item
\end{itemize}

Advertisement

\protect\hyperlink{after-top}{Continue reading the main story}

Supported by

\protect\hyperlink{after-sponsor}{Continue reading the main story}

\hypertarget{defense-secretary-mattis-tells-nato-allies-to-spend-more-or-else}{%
\section{Defense Secretary Mattis Tells NATO Allies to Spend More, or
Else}\label{defense-secretary-mattis-tells-nato-allies-to-spend-more-or-else}}

\includegraphics{https://static01.nyt.com/images/2017/02/16/world/16Mattis/16Mattis-articleLarge.jpg?quality=75\&auto=webp\&disable=upscale}

By \href{http://www.nytimes.com/by/helene-cooper}{Helene Cooper}

\begin{itemize}
\item
  Feb. 15, 2017
\item
  \begin{itemize}
  \item
  \item
  \item
  \item
  \item
  \end{itemize}
\end{itemize}

BRUSSELS --- Defense Secretary Jim Mattis, echoing his boss in
Washington, warned on Wednesday that the amount of American support for
NATO could depend on whether other countries meet their own spending
commitments.

``Americans cannot care more for your children's future security than
you do,'' Mr. Mattis said in his first speech to NATO allies since
becoming defense secretary. ``I owe it to you to give you clarity on the
political reality in the United States and to state the fair demand from
my country's people in concrete terms.''

``America will meet its responsibilities,'' he said, but he made clear
that American support had its limits.

In his speech to NATO defense ministers, Mr. Mattis repeated a call made
by previous American secretaries of defense, for European allies to
spend more on their militaries. His comments on Wednesday give teeth to
President Trump's expressed skepticism about the alliance.

What's more, Mr. Mattis went further than his predecessors in apparently
linking American contributions to the alliance to what other countries
spend.

``If your nations do not want to see America moderate its commitment to
this alliance, each of your capitals needs to show support for our
common defense,'' he said.

Mr. Mattis did not say how the United States might back away from its
obligations to NATO members, though there are several steps the Trump
administration could take short of refusing to come to the aid of an
ally under attack. That would be an abrogation of its treaty
responsibilities, but the United States could reduce the number of
American troops stationed in certain European countries or raise the bar
for what it considers a military attack.

The United States spends more of its gross domestic product on the
military than any other NATO member --- 3.61 percent, or \$664 billion
in 2016. NATO countries have committed to spending 2 percent of their
G.D.P. on the military, but the only other countries that meet that
criteria are Britain, Poland, Estonia and Greece.

During his remarks on Wednesday, Mr. Mattis called for the adoption of a
plan with fixed dates to make progress toward getting to 2 percent.

For decades, the United States has exhorted its allies to put more money
into their military budgets, arguing that if the alliance is called on
to defend a member country, the United States would have to shoulder too
much of the load. But European governments have different priorities
when it comes to military spending than the United States. Iceland, for
instance, has no military. And Germany, which since the end of World War
II has rejected military force outside self-defense, spends only 1.2
percent of its G.D.P. on the military.

European Union budgetary rules also constrain some NATO countries from
increasing their budget deficits.

Martin Stropnicky, the Czech Republic's defense minister, said in an
interview that Mr. Mattis's speech was not a surprise, and he did not
view it as a threat. ``He was absolutely calm and humble and modest,''
Mr. Stropnicky said, adding that his government had increased its
military spending. But the Czech Republic still spends just over 1
percent of its G.D.P. on its military, according to NATO.

Mr. Trump is expected to visit NATO headquarters in May when the
alliance holds its leaders' summit meeting.

This meeting in Brussels was a tough European debut for the Trump
administration, as Mr. Mattis also sought to convince NATO allies that
the United States still values the alliance despite the
\href{https://www.nytimes.com/2017/01/15/world/europe/donald-trump-nato.html}{president's
persistent critiques}.

The latest disclosures, that members of Mr. Trump's 2016 campaign and
other confidants had communicated with Russia intelligence officers, and
the resignation of
\href{https://www.nytimes.com/2017/02/14/us/politics/fbi-interviewed-mike-flynn.html}{Mr.
Trump's national security adviser, Michael T. Flynn}, for misleading
statements about his telephone conversation with Russian Ambassador
Sergey I. Kislyak, raised allies' anxiety.

Privately and publicly, a number of top NATO officials expressed concern
about Russian meddling in elections in Europe and the United States. And
they hung on Mr. Mattis's every word on Wednesday, listening for clues
to understand what the complex entanglements between Trump
administration officials and
\href{https://www.nytimes.com/2017/02/14/us/politics/russia-intelligence-communications-trump.html}{Russian
intelligence officers} might mean for the trans-Atlantic alliance as it
tries to confront a menacing and more aggressive Moscow.

Mr. Mattis sought to persuade the United States' allies that nothing has
changed when it comes to countering President Vladimir V. Putin and
Russia. He said Mr. Flynn's exit over his
\href{https://www.nytimes.com/interactive/2017/02/14/us/politics/flynn-call-russia-timeline.html}{communications
with Russia's ambassador} would not change his message to NATO.
``Frankly this has no impact,'' he said aboard his flight to Brussels.
``No effect at all.''

But Jeanine Hennis-Plasschaert, the Dutch minister of defense, was
pointed. ``There's no such thing as business as usual with Russia,'' she
said on Wednesday before the meeting.

One NATO official characterized the mood in the heavily fortified
compound as tense and said allies were waiting to see if the message Mr.
Mattis presented on Wednesday differed in tone from what Mr. Trump has
said.

In one important way, the defense secretary amplified the president's
previous statements. Though Mr. Mattis acknowledged ``concern in
European capitals about America's commitment to NATO and the security of
Europe,'' he said allies must do more to reach their commitments to
spend 2 percent of their G.D.P. on their militaries. ``No longer can the
American taxpayer carry a disproportionate share of the defense of
western values,'' he said.

Mr. Mattis also struck an assertive tone on Russia, saying the
\href{https://www.nytimes.com/2014/03/19/world/europe/ukraine.html}{2014
annexation of Crimea} dashed any hopes that NATO could have a real
partnership with Russia.

``Events of 2014 were sobering,'' he said in opening remarks made
alongside the NATO secretary general, Jens Stoltenberg. He added, ``As
President Trump has stated, he has strong support for NATO.''

But many officials here remained unconvinced, privately citing Mr.
Trump's previous statements calling the alliance ``obsolete'' and
complaining that it had not ``bothered'' about terrorism. Last year, he
suggested that American support for members of the alliance might be
conditional on whether those members paid their financial share.

Mr. Trump's remarks have deeply rattled NATO's Eastern European members
in particular. But his comments may have spurred a new focus on the
alliance's spending. ``The U.S. has made clear that we need more defense
spending and fairer burden-sharing,'' Mr. Stoltenberg said on Wednesday.
He pointed to numbers released a day earlier that showed that military
spending among European NATO countries and Canada had increased 3.8
percent in 2016 --- around \$10 billion.

``This is significant, but not enough,'' Mr. Stoltenberg said. ``We have
to continue to increase military spending across Europe and Canada.''

Separately, Mr. Stoltenberg expressed concern over news that Russia had
\href{https://www.nytimes.com/2017/02/14/world/europe/russia-cruise-missile-arms-control-treaty.html}{deployed
a new cruise missile} that American officials say violates a landmark
arms control treaty. The ground-launched cruise missile is one that the
\href{https://www.nytimes.com/2014/07/29/world/europe/us-says-russia-tested-cruise-missile-in-violation-of-treaty.html}{Obama
administration said in 2014} had been tested in violation of a 1987
treaty that bans American and Russian intermediate-range missiles based
on land.

The Obama administration tried to persuade the Russians to correct the
violation while the missile was still in the test phase, but instead,
the Russians have moved ahead with the system, deploying a fully
operational unit. ``Any noncompliance of Russia with the I.N.F. treaty
would be a serious concern for the alliance,'' Mr. Stoltenberg said.

Advertisement

\protect\hyperlink{after-bottom}{Continue reading the main story}

\hypertarget{site-index}{%
\subsection{Site Index}\label{site-index}}

\hypertarget{site-information-navigation}{%
\subsection{Site Information
Navigation}\label{site-information-navigation}}

\begin{itemize}
\tightlist
\item
  \href{https://help.nytimes.com/hc/en-us/articles/115014792127-Copyright-notice}{©~2020~The
  New York Times Company}
\end{itemize}

\begin{itemize}
\tightlist
\item
  \href{https://www.nytco.com/}{NYTCo}
\item
  \href{https://help.nytimes.com/hc/en-us/articles/115015385887-Contact-Us}{Contact
  Us}
\item
  \href{https://www.nytco.com/careers/}{Work with us}
\item
  \href{https://nytmediakit.com/}{Advertise}
\item
  \href{http://www.tbrandstudio.com/}{T Brand Studio}
\item
  \href{https://www.nytimes.com/privacy/cookie-policy\#how-do-i-manage-trackers}{Your
  Ad Choices}
\item
  \href{https://www.nytimes.com/privacy}{Privacy}
\item
  \href{https://help.nytimes.com/hc/en-us/articles/115014893428-Terms-of-service}{Terms
  of Service}
\item
  \href{https://help.nytimes.com/hc/en-us/articles/115014893968-Terms-of-sale}{Terms
  of Sale}
\item
  \href{https://spiderbites.nytimes.com}{Site Map}
\item
  \href{https://help.nytimes.com/hc/en-us}{Help}
\item
  \href{https://www.nytimes.com/subscription?campaignId=37WXW}{Subscriptions}
\end{itemize}
