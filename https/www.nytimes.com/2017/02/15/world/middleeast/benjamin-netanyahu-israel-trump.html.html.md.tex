Sections

SEARCH

\protect\hyperlink{site-content}{Skip to
content}\protect\hyperlink{site-index}{Skip to site index}

\href{https://www.nytimes.com/section/world/middleeast}{Middle East}

\href{https://myaccount.nytimes.com/auth/login?response_type=cookie\&client_id=vi}{}

\href{https://www.nytimes.com/section/todayspaper}{Today's Paper}

\href{/section/world/middleeast}{Middle East}\textbar{}Trump, Meeting
With Netanyahu, Backs Away From Palestinian State

\url{https://nyti.ms/2lPZ1CH}

\begin{itemize}
\item
\item
\item
\item
\item
\item
\end{itemize}

Advertisement

\protect\hyperlink{after-top}{Continue reading the main story}

Supported by

\protect\hyperlink{after-sponsor}{Continue reading the main story}

\hypertarget{trump-meeting-with-netanyahu-backs-away-from-palestinian-state}{%
\section{Trump, Meeting With Netanyahu, Backs Away From Palestinian
State}\label{trump-meeting-with-netanyahu-backs-away-from-palestinian-state}}

\includegraphics{https://static01.nyt.com/images/2017/02/16/world/16prexy/16prexy-videoSixteenByNineJumbo1600.jpg}

By \href{http://www.nytimes.com/by/peter-baker}{Peter Baker} and
\href{http://www.nytimes.com/by/mark-landler}{Mark Landler}

\begin{itemize}
\item
  Feb. 15, 2017
\item
  \begin{itemize}
  \item
  \item
  \item
  \item
  \item
  \item
  \end{itemize}
\end{itemize}

WASHINGTON --- President Trump jettisoned two decades of diplomatic
orthodoxy on Wednesday by declaring that the United States would no
longer insist on the creation of a Palestinian state as part of a peace
accord between Israel and the Palestinians.

Hosting Prime Minister Benjamin Netanyahu of Israel for the first time
since becoming president, Mr. Trump promised a concerted effort to bring
the two sides together, suggesting
\href{https://www.nytimes.com/2017/02/09/world/middleeast/trump-arabs-palestinians-israel.html}{a
regional effort} involving an array of Arab nations. But he said that he
was flexible about how an agreement would look and that he would not be
bound by past assumptions.

``I'm looking at two-state and one-state'' formulations, Mr. Trump said
during a White House news conference with Mr. Netanyahu. ``I like the
one that both parties like. I'm very happy with the one that both
parties like. I can live with either one.''

At the same time, Mr. Trump urged Mr. Netanyahu to temporarily stop new
housing construction in the West Bank while he pursues a deal, echoing a
position past presidents have taken. ``I'd like to see you hold back on
settlements for a little bit,'' he told Mr. Netanyahu.

Mr. Trump's position on a two-state solution discarded a policy that has
underpinned America's role in Middle East peacemaking since the Clinton
administration and raised a host of thorny questions.

The Palestinians are highly unlikely to accept anything short of a
sovereign state, and a single Israeli state encompassing the
Palestinians would either leave them as second-class citizens or would
no longer be majority Jewish, given the growth rate of the Arab
population.

Still, long before Mr. Trump's ascension, momentum for side-by-side
states had faded not just in Washington but also in the Middle East,
where many Israelis and Palestinians have given up hope or changed their
minds about the concept. The leaders of both sides face domestic
difficulties and seem unenthusiastic about the compromises that might be
required to get to a mutually agreeable resolution.

The trick is that no one has offered a plausible alternative that would
satisfy both camps, leaving the conflict in a state of suspended
animation. Mr. Netanyahu is under pressure from his right-leaning
coalition to abandon the two-state solution and even annex parts of the
West Bank. And the Palestinian Authority faces pressure from Hamas, the
militant group that controls Gaza and is sworn to Israel's destruction.

Mr. Trump did not address these dynamics, instead emphasizing his
confidence that he could produce a breakthrough. ``I think we're going
to make a deal,'' he said, describing that as personally important to
him. ``It might be a bigger and better deal than people in this room
even understand.''

He emphasized that Israel would have to be flexible in any future talks.
``As with any successful negotiation, both sides will have to make
compromises,'' Mr. Trump said.

Turning to Mr. Netanyahu, he asked, ``You know that, right?''

Mr. Netanyahu responded with a smile. ``Both sides,'' he said,
emphasizing the first word.

Nonetheless, Mr. Netanyahu, who nominally supports a two-state solution,
quickly embraced Mr. Trump's declaration, saying he preferred to deal
with the ``substance'' of a deal rather than ``labels.''

He noted that the concept of a two-state solution meant different things
to different people. And he repeated his two prerequisites: that the
Palestinians recognize Israel as a Jewish state and that Israel maintain
security control over the entire West Bank. He said the obstacle to
peace was Palestinian hatred, demonstrated by the building of statues to
those who carry out terrorist attacks and the payment of salaries to
their families. ``This is the source of the conflict,'' he said.

Mr. Trump's dismissal of the two-state solution seemed reminiscent of
\href{https://www.nytimes.com/2016/12/11/us/politics/trump-taiwan-one-china.html}{his
remark during the transition} that the United States should not be bound
by the decades-old ``one China'' policy that recognizes a single Chinese
government in Beijing and withholds diplomatic ties from Taiwan. That
statement infuriated the Chinese leadership, and Mr. Trump eventually
circled back
\href{http://www.nytimes.com/2017/02/09/world/asia/donald-trump-china-xi-jinping-letter.html}{to
endorse the policy}.

If Mr. Trump is serious about pursuing peace between the Israelis and
the Palestinians, several analysts said, he may inevitably find his way
back to the two-state solution.

``If you do a systematic analysis of the situation, there is no other
option,'' said Daniel C. Kurtzer, a former United States ambassador to
Israel and Egypt. ``There are Israelis who believe they could get away
with giving the Palestinians minimal political rights, but they are
fooling themselves. Unless the Palestinians do a 180, it is just
inconceivable.''

Palestinian leaders lamented Mr. Trump's stance, seeing it as an
abandonment by the United States, which has been the main patron of the
Palestinian Authority. But
\href{http://www.washingtoninstitute.org/experts/view/ghaith-al-omari}{Ghaith
al-Omari}, a senior fellow at the Washington Institute for Near East
Policy, said Palestinians could draw comfort from Mr. Trump's eagerness
for a new peace push and his warning to Israel on settlements.

``They will see an opening in, how do you translate the president's
desire for peace into something concrete?'' Mr. Omari said.

Mr. Trump and Jared Kushner, his son-in-law and senior adviser, have
been
\href{https://www.nytimes.com/2017/02/09/world/middleeast/trump-arabs-palestinians-israel.html}{exploring
an approach} called the ``outside-in'' strategy, which involves
enlisting Arab nations that have already found common cause with Israel
against Iran, their mutual enemy, to help broker a settlement with the
Palestinians.

Until now, Mr. Trump's team has largely avoided conversations with
Palestinian leaders. But Mike Pompeo, the C.I.A. director, met with
Mahmoud Abbas, the Palestinian Authority president, in Ramallah in the
West Bank on Tuesday, according to news reports.

The idea of an independent Palestinian state comprising the West Bank
and Gaza became the central theme of Middle East peacemaking in the
1990s after the Oslo Accords were signed. Bill Clinton was the first
president to endorse a two-state solution, saying in a speech in January
2001, just two weeks before leaving office, that the conflict would
never be settled without ``a sovereign, viable Palestinian state.''

His successor, George W. Bush, picked that up later that year, becoming
the first president to make it official American policy. Barack Obama
considered a two-state solution the unquestionable bedrock of
Washington's approach. But those presidents never got to the point of an
agreement between the two parties, and Mr. Trump picked as his
ambassador to Israel a lawyer, David M. Friedman,
\href{https://www.nytimes.com/2016/12/16/world/middleeast/david-friedman-us-ambassador-israel.html}{who
opposes the two-state solution}.

Mr. Netanyahu looked forward to Mr. Trump's inauguration as the first
time in his four terms as prime minister that he would have a Republican
president as a partner. After years of tension with Mr. Obama, who
pressed Israel for more concessions for peace, Mr. Netanyahu anticipated
vigorous support from the new president.

But Mr. Trump's focus on the Palestinian conflict and his push for a
pause in settlements distracted from the topic Mr. Netanyahu preferred
to address, the threat from Iran. At the news conference, Mr. Trump
again called Mr. Obama's nuclear agreement with Iran ``one of the worst
deals I've ever seen,'' but said nothing about abandoning it or even
renegotiating it. Instead, he simply vowed to keep Iran from becoming a
nuclear power. ``I will do more to prevent Iran from ever developing ---
I mean ever --- a nuclear weapon,'' he said.

Nor did he repeat his campaign vow to move the American Embassy to
Jerusalem, saying only, ``I'd love to see that happen'' and, ``We'll see
what happens.''

But he made a show of warmly welcoming Mr. Netanyahu, even inviting the
prime minister's wife, Sara, to stand during the news conference. The
Israeli first lady was then treated to a museum tour by Mr. Trump's
wife, Melania.

Still, the president was pressed by an Israeli reporter about a rise in
anti-Semitic attacks across the country since his election. The reporter
asked what he would say to those ``who believe and feel that your
administration is playing with xenophobia and maybe racist tones.''

In a meandering response, Mr. Trump cited his victory in the Electoral
College, then promised ``to do everything within our power to stop
long-simmering racism.'' He pointed to Mr. Kushner, who is Jewish, and
his daughter Ivanka, who converted when she married Mr. Kushner, to
dispel suggestions of anti-Semitism.

``As far as Jewish people, so many friends --- a daughter who happens to
be here right now, a son-in-law and three beautiful grandchildren,'' he
said, vowing to promote comity. ``You're going to see a lot of love.''

Advertisement

\protect\hyperlink{after-bottom}{Continue reading the main story}

\hypertarget{site-index}{%
\subsection{Site Index}\label{site-index}}

\hypertarget{site-information-navigation}{%
\subsection{Site Information
Navigation}\label{site-information-navigation}}

\begin{itemize}
\tightlist
\item
  \href{https://help.nytimes.com/hc/en-us/articles/115014792127-Copyright-notice}{©~2020~The
  New York Times Company}
\end{itemize}

\begin{itemize}
\tightlist
\item
  \href{https://www.nytco.com/}{NYTCo}
\item
  \href{https://help.nytimes.com/hc/en-us/articles/115015385887-Contact-Us}{Contact
  Us}
\item
  \href{https://www.nytco.com/careers/}{Work with us}
\item
  \href{https://nytmediakit.com/}{Advertise}
\item
  \href{http://www.tbrandstudio.com/}{T Brand Studio}
\item
  \href{https://www.nytimes.com/privacy/cookie-policy\#how-do-i-manage-trackers}{Your
  Ad Choices}
\item
  \href{https://www.nytimes.com/privacy}{Privacy}
\item
  \href{https://help.nytimes.com/hc/en-us/articles/115014893428-Terms-of-service}{Terms
  of Service}
\item
  \href{https://help.nytimes.com/hc/en-us/articles/115014893968-Terms-of-sale}{Terms
  of Sale}
\item
  \href{https://spiderbites.nytimes.com}{Site Map}
\item
  \href{https://help.nytimes.com/hc/en-us}{Help}
\item
  \href{https://www.nytimes.com/subscription?campaignId=37WXW}{Subscriptions}
\end{itemize}
