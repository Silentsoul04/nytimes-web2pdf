Sections

SEARCH

\protect\hyperlink{site-content}{Skip to
content}\protect\hyperlink{site-index}{Skip to site index}

\href{https://www.nytimes.com/section/business}{Business}

\href{https://myaccount.nytimes.com/auth/login?response_type=cookie\&client_id=vi}{}

\href{https://www.nytimes.com/section/todayspaper}{Today's Paper}

\href{/section/business}{Business}\textbar{}Samsung Heir's Arrest in
South Korea Intensifies Calls for Cleanup

\url{https://nyti.ms/2le3Upl}

\begin{itemize}
\item
\item
\item
\item
\item
\end{itemize}

Advertisement

\protect\hyperlink{after-top}{Continue reading the main story}

Supported by

\protect\hyperlink{after-sponsor}{Continue reading the main story}

\hypertarget{samsung-heirs-arrest-in-south-korea-intensifies-calls-for-cleanup}{%
\section{Samsung Heir's Arrest in South Korea Intensifies Calls for
Cleanup}\label{samsung-heirs-arrest-in-south-korea-intensifies-calls-for-cleanup}}

\includegraphics{https://static01.nyt.com/images/2017/02/18/world/18samsung-1/18samsung-1-articleLarge.jpg?quality=75\&auto=webp\&disable=upscale}

By \href{https://www.nytimes.com/by/paul-mozur}{Paul Mozur} and
\href{http://www.nytimes.com/by/choe-sang-hun}{Choe Sang-Hun}

\begin{itemize}
\item
  Feb. 17, 2017
\item
  \begin{itemize}
  \item
  \item
  \item
  \item
  \item
  \end{itemize}
\end{itemize}

HONG KONG --- A major corporate executive
\href{https://www.nytimes.com/2017/02/16/world/asia/korea-samsung-lee-jae-yong.html}{sits
in jail}, accused of participating in a corruption scandal that
\href{https://www.nytimes.com/2016/12/09/world/asia/south-korea-president-park-geun-hye-impeached.html}{could
topple} a country's top leader. In most places, that image would
probably fuel a campaign to clean up incestuous links between business
and government.

That situation is now unfolding in South Korea --- but the prospects for
a major housecleaning look uncertain.

The unprecedented arrest on Friday of the de facto leader of Samsung,
the largest company in the country, highlighted once again the outsize
political influence of the largest family-run companies there. Critics
of their power cheered the arrest of Lee Jae-yong, Samsung's vice
chairman, as a major step toward curbing that authority.

Still, others warned that the arrest would not necessarily lead to the
sweeping changes they say South Korea needs to fight corruption and
overhaul its economy.

The arrest was ``just the beginning,'' said Sim Sang-jeung, an
opposition lawmaker who has campaigned for transparency at the largest
companies. She warned against a tendency among the law enforcement
agencies to treat major corporate chiefs with kid gloves.

``We needed to see whether prosecutors ask for a sentence befitting his
crimes and whether he is convicted with such a penalty,'' Ms. Sim said.
``Only when he finishes serving such a lengthy sentence will people
believe that the law is alive in their country.''

It also raises questions about the fate of Samsung, a huge company whose
electronics arm alone accounts for one-fifth of South Korea's exports.

Wearing a well-tailored suit, Mr. Lee emerged through the metal
detectors of a court in Seoul, the capital, on Thursday and past a news
media gantlet to his car, which drove him to a detention center to await
a decision. Early Friday morning, he learned that he would be staying at
the detention center through his trial.

South Korea faces a tenuous balancing act. For decades, its growth has
been fueled by companies like Samsung, one of a group of
\href{https://www.nytimes.com/2017/02/17/business/south-korea-chaebol-samsung.html}{family-controlled
conglomerates called chaebol}. Chaebol are now firmly embedded in the
country's economy, with the 10 largest generating annual revenue
exceeding 80 percent of South Korea's gross domestic product. Business
groups warn that disrupting the chaebol could hurt the broader economy.

``We are shocked and deeply worried,'' the Korea Employers Federation, a
pro-business lobby, said in a statement about the arrest.

``Samsung is the global company that represents South Korea, and we fear
that the vacuum in its management will weigh heavily on the economy by
increasing uncertainty and hurt international credibility.''

But the power of the chaebol is coming up against
\href{https://www.nytimes.com/2017/01/02/world/asia/south-korea-park-geun-hye-samsung.html}{rising
public anger} over the perception of corruption and favoritism. Among
those 10 biggest chaebol, six of their leaders have been convicted of
white-collar crimes. Many have been pardoned or had their sentences
suspended or reduced. Chaebol leaders face broader questions as well
about whether their economic dominance squelches small business and
innovation, accusations that their lobbyist denies.

Reflecting the public mood, the governing --- and usually pro-business
--- Liberty Korea Party said it respected the court's decision to arrest
Mr. Lee and expressed ``regrets that the people have been again
disappointed by the deep-rooted collusion between politics and
business.''

Mr. Lee is accused of bribery, embezzlement and perjury as part of
\href{https://www.nytimes.com/2017/01/02/world/asia/south-korea-park-geun-hye-samsung.html}{an
investigation into a confidante} of the country's president, Park
Geun-hye. Ms. Park now faces impeachment. Samsung has said Mr. Lee will
work to clear his name in court.

The police arrested Mr. Lee and took him into custody, an unprecedented
move for a major Samsung official. But in terms of accusations of
wrongdoing against a top executive, Samsung has been there before. Mr.
Lee's father, Lee Kun-hee, Samsung's chairman, has twice been convicted
of bribery and tax evasion.

Still, the elder Mr. Lee never spent time in prison. The fate of the
young Mr. Lee, critics of the chaebol say, will be a test of the
country's young democracy and judicial system.

It will also be a test for Samsung. For the first time in its 79-year
history, the company has been left leaderless. With Mr. Lee gone, there
is no top executive to make long-term plans and strategic decisions.

Samsung has an army of professional executives that manage day-to-day
operations of its 58 subsidiaries. But analysts say that without a
family-appointed leader, decision making will slow.

In chaebol culture, often likened to an imperial monarchy within South
Korea, the chairman must endorse or make corporate decisions. So the
removal of Mr. Lee, who has been the de facto leader since his father
was incapacitated by a
\href{https://www.nytimes.com/2014/05/12/business/international/samsungs-chairman-has-surgery-after-heart-attack.html}{heart
attack in 2014}, is far more serious than the loss of a senior executive
at a conventional company.

Choi Gee-sung, the No. 2 lieutenant in the Samsung hierarchy and
longtime right-hand man for Mr. Lee, will be the closest substitute to a
top manager at the company while Mr. Lee is gone. But Mr. Choi is not a
member of the Lee family and is expected to serve largely as a
``vassal'' caretaker who lacks the kind of sweeping authority and
responsibility that Mr. Lee and his father have wielded in placing
multibillion-dollar bets on investments or new technology.

In one sign of disruption, Samsung delayed its annual reshuffle of
senior managers, which it usually announces in December. Compounding
concerns, Mr. Choi and his deputies are also being investigated by
prosecutors in connection with the bribery scandal.

The arrest comes at a difficult time for Samsung's electronics arm. The
company has faced stiff competition from Apple and cheaper Chinese
smartphone makers alike. It is also still recovering from
\href{https://www.nytimes.com/2017/01/23/business/samsung-galaxy-note7-fires.html}{the
discontinuation of its Galaxy Note 7}, after flaws led some of the
phones to overheat and burst into flames.

Still, few believe Mr. Lee's arrest will challenge the family's ultimate
control of the company. In 2008, facing corruption charges, Mr. Lee's
father resigned from management, leaving the company to be run by loyal
deputies, who served the family for decades and whose responsibilities
were to ensure the father-to-son transfer of power.

For Samsung, one test will be whether the argument that its fate is
important to the South Korean economy carries the same weight. When
\href{https://www.nytimes.com/2016/11/26/world/asia/korea-park-geun-hye-protests.html}{huge
crowds took to the streets} on recent weekends to call for the
impeachment of Ms. Park, they also called for the arrest of chaebol
chairmen accused of playing a crucial role in the presidential scandal.

On Friday, Moon Jae-in, the opposition leader who tops surveys of
potential candidates to replace Ms. Park, called the arrest ``proof that
justice is still alive in South Korea.''

Advertisement

\protect\hyperlink{after-bottom}{Continue reading the main story}

\hypertarget{site-index}{%
\subsection{Site Index}\label{site-index}}

\hypertarget{site-information-navigation}{%
\subsection{Site Information
Navigation}\label{site-information-navigation}}

\begin{itemize}
\tightlist
\item
  \href{https://help.nytimes.com/hc/en-us/articles/115014792127-Copyright-notice}{©~2020~The
  New York Times Company}
\end{itemize}

\begin{itemize}
\tightlist
\item
  \href{https://www.nytco.com/}{NYTCo}
\item
  \href{https://help.nytimes.com/hc/en-us/articles/115015385887-Contact-Us}{Contact
  Us}
\item
  \href{https://www.nytco.com/careers/}{Work with us}
\item
  \href{https://nytmediakit.com/}{Advertise}
\item
  \href{http://www.tbrandstudio.com/}{T Brand Studio}
\item
  \href{https://www.nytimes.com/privacy/cookie-policy\#how-do-i-manage-trackers}{Your
  Ad Choices}
\item
  \href{https://www.nytimes.com/privacy}{Privacy}
\item
  \href{https://help.nytimes.com/hc/en-us/articles/115014893428-Terms-of-service}{Terms
  of Service}
\item
  \href{https://help.nytimes.com/hc/en-us/articles/115014893968-Terms-of-sale}{Terms
  of Sale}
\item
  \href{https://spiderbites.nytimes.com}{Site Map}
\item
  \href{https://help.nytimes.com/hc/en-us}{Help}
\item
  \href{https://www.nytimes.com/subscription?campaignId=37WXW}{Subscriptions}
\end{itemize}
