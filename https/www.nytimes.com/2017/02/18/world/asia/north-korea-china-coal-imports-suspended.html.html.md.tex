Sections

SEARCH

\protect\hyperlink{site-content}{Skip to
content}\protect\hyperlink{site-index}{Skip to site index}

\href{https://www.nytimes.com/section/world/asia}{Asia Pacific}

\href{https://myaccount.nytimes.com/auth/login?response_type=cookie\&client_id=vi}{}

\href{https://www.nytimes.com/section/todayspaper}{Today's Paper}

\href{/section/world/asia}{Asia Pacific}\textbar{}China Suspends All
Coal Imports From North Korea

\url{https://nyti.ms/2m8Ugn3}

\begin{itemize}
\item
\item
\item
\item
\item
\end{itemize}

Advertisement

\protect\hyperlink{after-top}{Continue reading the main story}

Supported by

\protect\hyperlink{after-sponsor}{Continue reading the main story}

\hypertarget{china-suspends-all-coal-imports-from-north-korea}{%
\section{China Suspends All Coal Imports From North
Korea}\label{china-suspends-all-coal-imports-from-north-korea}}

\includegraphics{https://static01.nyt.com/images/2017/02/19/world/19coal1/19coal1-articleLarge.jpg?quality=75\&auto=webp\&disable=upscale}

By \href{http://www.nytimes.com/by/choe-sang-hun}{Choe Sang-Hun}

\begin{itemize}
\item
  Feb. 18, 2017
\item
  \begin{itemize}
  \item
  \item
  \item
  \item
  \item
  \end{itemize}
\end{itemize}

\href{http://cn.nytimes.com/china/20170220/north-korea-china-coal-imports-suspended/}{阅读简体中文版}

SEOUL, South Korea --- China said on Saturday that it was suspending all
imports of coal from North Korea as part of its effort to enact United
Nations Security Council sanctions aimed at stopping the country's
nuclear weapons and ballistic-missile program.

The ban takes effect on Sunday and will last until the end of the year,
the Chinese Commerce Ministry said in a brief statement posted on its
website on Saturday. Chinese trade and aid have long been a vital
economic crutch for North Korea, and the decision strips North Korea of
one of its most important sources of foreign currency.

Coal has accounted for 34 percent to 40 percent of North Korean exports
in the past several years, and almost all of it was shipped to China,
according to South Korean government estimates.

The ban comes six days after the North Korean
\href{https://www.nytimes.com/2017/02/11/world/asia/north-korea-missile-test-trump.html}{test
of a ballistic missile} that the Security Council condemned as a
violation of its resolutions that prohibited the country from developing
and testing ballistic missile technology.

In the test,
\href{https://www.nytimes.com/2017/02/13/world/asia/north-korea-missile-launch-success.html}{North
Korea claimed} that it had successfully launched a new type of
nuclear-capable missile. It said its intermediate-range Pukguksong-2
missile used a solid-fuel technology that American experts say will make
it harder to detect missile attacks from the North.

In
\href{https://www.nytimes.com/2016/11/30/world/asia/north-korea-un-sanctions.html?_r=0}{the
resolution it adopted in November} in response to the North's fifth and
most powerful nuclear test, the Security Council said that North Korea
should not be allowed to export more than 7.5 million metric tons of
coal a year or bring in more than \$400 million in coal sales, whichever
limit is met first. It was unclear whether that cap has already been
reached for this year.

Officials of the United States and its allies, including President
Trump, have suggested that China, North Korea's principal economic
patron, should be more aggressive in enforcing sanctions. But while it
does not approve of the North's weapons program, China has also been
seen as reluctant to inflict crippling pain on North Korea, for fear
that it might destabilize its Communist neighbor.

In April, China announced that it would ban coal imports from North
Korea as part of the United Nations' efforts to squeeze the country's
ability to raise funds for its nuclear and missile programs. But it
allowed exemptions for coal imports for ``livelihood'' reasons, and
deliveries continued.

The Chinese Ministry of Foreign Affairs did not comment on the
suspension after it was announced on Saturday. Calls to the ministry's
press officer were not answered. On Friday, the Chinese minister of
foreign affairs, Wang Yi,
\href{http://www.mfa.gov.cn/web/zyxw/t1439593.shtml}{said at a
conference in Munich} that the United Nations sanctions of North Korea
``must continue to be strictly implemented.''

But Mr. Wang also argued that only renewed negotiations would offer any
hope of curtailing North Korea's nuclear weapons development. China has
hosted six-party talks --- including itself, South Korea, North Korea,
the United States, Japan and Russia --- aimed at a negotiated settlement
of the North Korean nuclear standoff. But those talks have stopped since
2009, and there seems little hope of them restarting anytime soon.

``This situation cannot continue,'' Mr. Wang said, ``because the
ultimate outcome may be intolerable to all sides.''

Last year, China imported 22.5 million metric tons of coal from North
Korea, an increase of 14.5 percent on the amount in 2015, according to
Chinese \href{http://www.sxcoal.com/news/4551922/info}{customs
statistics}. In December, China imported about 2 million tons of North
Korean coal. Mysteel, a Chinese industrial analysis firm, estimated that
under the limits imposed by the sanctions, the coal quota would be used
up by April or May. In 2015, China's cumulative imports of North Korean
coal reached 7.5 million metric tons by May.

The coal suspension also followed
\href{https://www.nytimes.com/2017/02/14/world/asia/kim-jong-un-brother-killed-malaysia.html}{the
assassination of Kim Jong-nam}, the estranged half brother of the North
Korean leader Kim Jong-un, on Monday at an airport in Malaysia. The
Malaysian authorities are continuing to investigate the case. South
Korean officials have suspected North Korean involvement in the killing
of Mr. Kim, who had been living in Macau, the Chinese gambling enclave.

Some analysts have speculated that the killing may have infuriated
Beijing because Mr. Kim was considered a pro-Chinese candidate to
replace Kim Jong-un, the North Korean leader, should the current
government in North Korea fall.

Advertisement

\protect\hyperlink{after-bottom}{Continue reading the main story}

\hypertarget{site-index}{%
\subsection{Site Index}\label{site-index}}

\hypertarget{site-information-navigation}{%
\subsection{Site Information
Navigation}\label{site-information-navigation}}

\begin{itemize}
\tightlist
\item
  \href{https://help.nytimes.com/hc/en-us/articles/115014792127-Copyright-notice}{©~2020~The
  New York Times Company}
\end{itemize}

\begin{itemize}
\tightlist
\item
  \href{https://www.nytco.com/}{NYTCo}
\item
  \href{https://help.nytimes.com/hc/en-us/articles/115015385887-Contact-Us}{Contact
  Us}
\item
  \href{https://www.nytco.com/careers/}{Work with us}
\item
  \href{https://nytmediakit.com/}{Advertise}
\item
  \href{http://www.tbrandstudio.com/}{T Brand Studio}
\item
  \href{https://www.nytimes.com/privacy/cookie-policy\#how-do-i-manage-trackers}{Your
  Ad Choices}
\item
  \href{https://www.nytimes.com/privacy}{Privacy}
\item
  \href{https://help.nytimes.com/hc/en-us/articles/115014893428-Terms-of-service}{Terms
  of Service}
\item
  \href{https://help.nytimes.com/hc/en-us/articles/115014893968-Terms-of-sale}{Terms
  of Sale}
\item
  \href{https://spiderbites.nytimes.com}{Site Map}
\item
  \href{https://help.nytimes.com/hc/en-us}{Help}
\item
  \href{https://www.nytimes.com/subscription?campaignId=37WXW}{Subscriptions}
\end{itemize}
