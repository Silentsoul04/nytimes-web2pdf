Sections

SEARCH

\protect\hyperlink{site-content}{Skip to
content}\protect\hyperlink{site-index}{Skip to site index}

\href{https://www.nytimes.com/section/politics}{Politics}

\href{https://myaccount.nytimes.com/auth/login?response_type=cookie\&client_id=vi}{}

\href{https://www.nytimes.com/section/todayspaper}{Today's Paper}

\href{/section/politics}{Politics}\textbar{}New Trump Deportation Rules
Allow Far More Expulsions

\url{https://nyti.ms/2lrcgKg}

\begin{itemize}
\item
\item
\item
\item
\item
\item
\end{itemize}

Advertisement

\protect\hyperlink{after-top}{Continue reading the main story}

Supported by

\protect\hyperlink{after-sponsor}{Continue reading the main story}

\hypertarget{new-trump-deportation-rules-allow-far-more-expulsions}{%
\section{New Trump Deportation Rules Allow Far More
Expulsions}\label{new-trump-deportation-rules-allow-far-more-expulsions}}

\includegraphics{https://static01.nyt.com/images/2017/02/22/us/22dhs/22dhs-articleLarge.jpg?quality=75\&auto=webp\&disable=upscale}

By \href{http://www.nytimes.com/by/michael-d-shear}{Michael D. Shear}
and \href{http://www.nytimes.com/by/ron-nixon}{Ron Nixon}

\begin{itemize}
\item
  Feb. 21, 2017
\item
  \begin{itemize}
  \item
  \item
  \item
  \item
  \item
  \item
  \end{itemize}
\end{itemize}

\href{https://www.nytimes.com/es/2017/02/21/la-nuevas-politicas-migratorias-de-trump-permiten-mas-deportaciones/}{Leer
en español}

WASHINGTON --- President Trump has directed his administration to
enforce the nation's immigration laws more aggressively, unleashing the
full force of the federal government to find, arrest and deport those in
the country illegally, regardless of whether they have committed serious
crimes.

Documents released on Tuesday by the Department of Homeland Security
revealed the broad scope of the president's ambitions: to publicize
crimes by undocumented immigrants; strip such immigrants of privacy
protections; enlist local police officers as enforcers; erect new
detention facilities; discourage asylum seekers; and, ultimately, speed
up deportations.

The new enforcement policies put into practice language that Mr. Trump
used on the campaign trail, vastly expanding the definition of
``criminal aliens'' and warning that such unauthorized immigrants
``routinely victimize Americans,'' disregard the ``rule of law and pose
a threat'' to people in communities across the United States.

Despite those assertions in the new documents,
\href{https://www.americanimmigrationcouncil.org/research/criminalization-immigration-united-states}{research
shows} lower levels of crime among immigrants than among native-born
Americans.

The president's new immigration policies are likely to be welcomed by
some law enforcement officials around the country, who have called for a
tougher crackdown on unauthorized immigrants, and by some Republicans in
Congress who have argued that lax enforcement encourages a never-ending
flow of unauthorized immigrants.

But taken together, the new policies are a rejection of the sometimes
more restrained efforts by former Presidents Barack Obama and George W.
Bush and their predecessors, who sought to balance protecting the
nation's borders with fiscal, logistical and humanitarian limits on the
exercise of laws passed by Congress.

``The faithful execution of our immigration laws is best achieved by
using all these statutory authorities to the greatest extent
practicable,'' John F. Kelly, the secretary of homeland security, wrote
in one of two memorandums released on Tuesday. ``Accordingly, department
personnel shall make full use of these authorities.''

\href{https://www.nytimes.com/interactive/2017/02/21/us/politics/document-Trump-Immigration-Enforcement-Policies.html}{}

\includegraphics{https://static01.nyt.com/images/2017/02/21/us/politics/image-Trump-Immigration-Enforcement-Policies/image-Trump-Immigration-Enforcement-Policies-thumbLarge.gif}

\hypertarget{memos-on-trumps-immigration-policies}{%
\subsection{Memos on Trump's Immigration
Policies}\label{memos-on-trumps-immigration-policies}}

The Department of Homeland Security has issued two memorandums outlining
how the agency intends to implement and enforce the Trump
administration's immigration policies.

The immediate impact of that shift is not yet fully known. Advocates for
immigrants warned on Tuesday that the new border control and enforcement
directives would create an atmosphere of fear that was likely to drive
those in the country illegally deeper into the shadows.

Administration officials said some of the new policies --- like one
seeking to send unauthorized border crossers from Central America to
Mexico while they await deportation hearings --- could take months to
put in effect and might be limited in scope.

For now, so-called Dreamers, who were brought to the United States as
young children, will not be targeted unless they commit crimes,
officials said on Tuesday.

Mr. Trump has not yet said where he will get the billions of dollars
needed to pay for thousands of new border control agents, a network of
detention facilities to detain unauthorized immigrants and a wall along
the entire southern border with Mexico.

But politically, Mr. Kelly's actions on Tuesday serve to reinforce the
president's standing among a core constituency --- those who blame
unauthorized immigrants for taking jobs away from citizens, committing
heinous crimes and being a financial burden on federal, state and local
governments.

And because of the changes, millions of immigrants in the country
illegally now face a far greater likelihood of being discovered,
arrested and eventually deported.

``The message is: The immigration law is back in business,'' said Mark
Krikorian, the executive director of the Center for Immigration Studies,
which supports restricted immigration. ``That violating immigration law
is no longer a secondary offense.''

Lawyers and advocates for immigrants said the new policies could still
be challenged in court. Maricopa County in Arizona spent years defending
its sheriff at the time, Joseph Arpaio, in federal court, where he was
found to have discriminated against Latinos.

\includegraphics{https://static01.nyt.com/images/2017/02/20/us/politics/lawyers/lawyers-videoSixteenByNineJumbo1600.jpg}

And courts in Illinois, Oregon, Pennsylvania and several other states
have rejected the power given to local and state law enforcement
officers to hold immigrants for up to 48 hours beyond their scheduled
release from detention at the request of federal authorities under a
program known as Secure Communities, which Mr. Trump is reviving.

``When you tell state and local police that their job is to do
immigration enforcement,'' said
\href{https://www.aclu.org/other/biography-omar-jadwat}{Omar Jadwat},
director of the American Civil Liberties Union's Immigrants' Rights
Project, ``it translates into the unwarranted and illegal targeting of
people because of their race, because of their language, because of the
color of their skin.''

Sean Spicer, the White House press secretary, said on Tuesday that the
president wanted to ``take the shackles off'' of the nation's
immigration enforcers. He insisted that the new policies made it clear
that ``the No. 1 priority is that people who pose a threat to our
country are immediately dealt with.''

In fact, that was already the policy under the Obama administration,
which instructed agents that undocumented immigrants convicted of
serious crimes were the priority for deportation. Now, enforcement
officials have been directed to seek the deportation of anyone in the
country illegally.

``Under this executive order, ICE will not exempt classes or categories
of removal aliens from potential enforcement,'' a fact sheet released by
the Department of Homeland Security said, using the acronym for
Immigration and Customs Enforcement. ``All of those present in violation
of the immigration laws may be subject to immigration arrest, detention,
and, if found removable by final order, removal from the United
States.''

That includes people convicted of fraud in any official matter before a
governmental agency and people who ``have abused any program related to
receipt of public benefits.''

The policy also expands a program that lets officials bypass due process
protections such as court hearings in some deportation cases.

Under the Obama administration, the program, known as ``expedited
removal,'' was used only when an immigrant was arrested within 100 miles
of the border and had been in the country no more than 14 days. Now it
will include all those who have been in the country for up to two years,
no matter where they are caught.

``The administration seems to be putting its foot down as far as the gas
pedal will go,'' said
\href{http://www.immigrantjustice.org/nijc-staff}{Heidi Altman}, policy
director for the National Immigrant Justice Center, a Chicago-based
group that offers legal services to immigrants.

In the documents released on Tuesday, the Department of Homeland
Security is directed to begin the process of hiring 10,000 immigration
and customs agents, expanding the number of detention facilities and
creating an office within Immigration and Customs Enforcement to help
families of those killed by undocumented immigrants.

The directives would also revive a program that recruits local police
officers and sheriff's deputies to help with deportation, effectively
making them de facto immigration agents. The effort, called the
\href{https://www.ice.gov/factsheets/287g}{287(g) program}, was scaled
back during the Obama administration.

The program faces resistance from many states and dozens of so-called
\href{https://www.nytimes.com/2017/01/31/us/san-francisco-lawsuit-trump-sanctuary-cities.html}{sanctuary
cities}, which have refused to allow their law enforcement workers to
help round up undocumented individuals. In New York, Mayor Bill de
Blasio in a statement on Tuesday pledged the city's cooperation in cases
involving ``proven public safety threats,'' but vowed that ``what we
will not do is turn our N.Y.P.D. officers into immigration agents.''

Under the new directives, the agency would no longer provide privacy
protections to people who are not American citizens or green card
holders.
\href{https://www.dhs.gov/xlibrary/assets/privacy/privacy_policyguide_2007-1.pdf}{A
policy} established in the last days of the Bush administration in
January 2009 provided some legal protection for information collected by
the Department of Homeland Security on nonresidents.

The new policies also target unauthorized immigrants who smuggle their
children into the country, as happened with Central American children
seeking to reunite with parents living in the United States. Under the
new directives, such parents could face deportation or prosecution for
smuggling or human trafficking.

Officials said that returning Central American refugees to Mexico to
await hearings would be done only in a limited fashion, and only after
discussions with the government of Mexico.

Mexican officials said on Tuesday that such a move could violate Mexican
law and international accords governing repatriation, and immigrants'
advocates questioned Mexico's ability to absorb thousands of Central
Americans in detention centers and shelters.

Advertisement

\protect\hyperlink{after-bottom}{Continue reading the main story}

\hypertarget{site-index}{%
\subsection{Site Index}\label{site-index}}

\hypertarget{site-information-navigation}{%
\subsection{Site Information
Navigation}\label{site-information-navigation}}

\begin{itemize}
\tightlist
\item
  \href{https://help.nytimes.com/hc/en-us/articles/115014792127-Copyright-notice}{©~2020~The
  New York Times Company}
\end{itemize}

\begin{itemize}
\tightlist
\item
  \href{https://www.nytco.com/}{NYTCo}
\item
  \href{https://help.nytimes.com/hc/en-us/articles/115015385887-Contact-Us}{Contact
  Us}
\item
  \href{https://www.nytco.com/careers/}{Work with us}
\item
  \href{https://nytmediakit.com/}{Advertise}
\item
  \href{http://www.tbrandstudio.com/}{T Brand Studio}
\item
  \href{https://www.nytimes.com/privacy/cookie-policy\#how-do-i-manage-trackers}{Your
  Ad Choices}
\item
  \href{https://www.nytimes.com/privacy}{Privacy}
\item
  \href{https://help.nytimes.com/hc/en-us/articles/115014893428-Terms-of-service}{Terms
  of Service}
\item
  \href{https://help.nytimes.com/hc/en-us/articles/115014893968-Terms-of-sale}{Terms
  of Sale}
\item
  \href{https://spiderbites.nytimes.com}{Site Map}
\item
  \href{https://help.nytimes.com/hc/en-us}{Help}
\item
  \href{https://www.nytimes.com/subscription?campaignId=37WXW}{Subscriptions}
\end{itemize}
