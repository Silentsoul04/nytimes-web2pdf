Sections

SEARCH

\protect\hyperlink{site-content}{Skip to
content}\protect\hyperlink{site-index}{Skip to site index}

\href{https://www.nytimes.com/section/world/asia}{Asia Pacific}

\href{https://myaccount.nytimes.com/auth/login?response_type=cookie\&client_id=vi}{}

\href{https://www.nytimes.com/section/todayspaper}{Today's Paper}

\href{/section/world/asia}{Asia Pacific}\textbar{}Kim Jong-nam
Investigators Seek to Question North Korean Embassy Officer

\url{https://nyti.ms/2m6jRAZ}

\begin{itemize}
\item
\item
\item
\item
\item
\end{itemize}

Advertisement

\protect\hyperlink{after-top}{Continue reading the main story}

Supported by

\protect\hyperlink{after-sponsor}{Continue reading the main story}

\hypertarget{kim-jong-nam-investigators-seek-to-question-north-korean-embassy-officer}{%
\section{Kim Jong-nam Investigators Seek to Question North Korean
Embassy
Officer}\label{kim-jong-nam-investigators-seek-to-question-north-korean-embassy-officer}}

\includegraphics{https://static01.nyt.com/images/2017/02/23/world/22MALAYSIA-1/22MALAYSIA-1-articleInline.jpg?quality=75\&auto=webp\&disable=upscale}

By \href{https://www.nytimes.com/by/richard-c-paddock}{Richard C.
Paddock} and \href{http://www.nytimes.com/by/gerry-mullany}{Gerry
Mullany}

\begin{itemize}
\item
  Feb. 21, 2017
\item
  \begin{itemize}
  \item
  \item
  \item
  \item
  \item
  \end{itemize}
\end{itemize}

KUALA LUMPUR, Malaysia --- The Malaysian police said on Wednesday that a
senior diplomat in the North Korean Embassy was wanted for questioning
in the fatal poisoning of
\href{https://www.nytimes.com/2017/02/15/world/asia/kim-jong-nam-assassination-north-korea.html}{Kim
Jong-nam}, the estranged half brother of North Korea's leader, pointing
to possible government involvement in his death.

At a news conference in which investigators gave their fullest public
account to date of the killing, the police also said the attackers had
been trained to wipe toxins on Kim Jong-nam's face and then wash their
hands.

The revelations are sure to escalate pressure on North Korea over
\href{https://www.nytimes.com/2017/02/14/world/asia/kim-jong-un-brother-killed-malaysia.html}{the
killing at the Kuala Lumpur airport} on Feb. 13, which South Korea has
branded a terrorist attack. Evidence of state involvement in Mr. Kim's
death could pressure the United States to put the North back on its list
of countries that sponsor terrorism.

Khalid Abu Bakar, Malaysia's police inspector general, said Wednesday
that North Korean citizens had put toxins on the hands of the two female
attackers, one of whom has been identified as Vietnamese and the other
as Indonesian. He said they had rehearsed the plot at two local shopping
malls.

``The two female suspects knew that the substance they had were toxic,''
he said, undercutting recent reports that the women had thought they
were carrying out a prank. ``We don't know what kind of chemical was
used.''

Mr. Khalid said that four North Koreans suspected of being involved in
the attack were believed to have fled to their homeland. Three others
--- the embassy official, identified as Hyon Kwang Song, the second
secretary at the embassy; an employee of the North Korean airline, Air
Koryo; and a third person --- were still believed to be in Malaysia.

``They're not in custody,'' he said of the three. ``They've been called
in for assistance.''

``We hope that the Korean Embassy will cooperate with us, allow us to
interview them and interview them quickly,'' he said. ``If not, we will
compel them to come to us.''

The embassy had no immediate response, although it later issued a
statement demanding the release of the two women accused of the attack
and a North Korean citizen being held in the case. The statement, which
called the women innocent, said that they could not have applied poison
with their hands, because they would themselves have died.

Malaysia's demand to question the diplomat is sure to further inflame
Malaysia's relations with the North. North Korea has refused to even
acknowledge that the man killed was Kim Jong-nam and has accused
Malaysia of carrying out a politically motivated investigation to
placate South Korea and the United States.

North Korea has rejected any assertion that the victim was the half
brother of its leader, Kim Jong-un, identifying the dead man as Kim
Chol, saying he held a diplomatic passport and rejecting Malaysia's
efforts to involve the victim's family in identifying the body.

North Korea has demanded that its government take part in the inquiry.

The fate of the body itself has become a point of contention. North
Korea has demanded that it be sent to the embassy, while Malaysian
officials say they will release the body only after it is identified by
Kim Jong-nam's next of kin. Mr. Kim's relatives live in the
semiautonomous Chinese territory of Macau, where he was heading at the
time of his death.

Mr. Khalid, the police official, said there had been an attempted
break-in at the morgue where the body was being held.

Still unresolved is what kind of poison was used to kill Mr. Kim and how
the attackers themselves were not hurt while deploying such a deadly
substance.

``These two ladies were trained to swab the deceased's face'' with
``their bare hands'' before cleaning them up, Mr. Khalid said.

The killing of Mr. Kim comes as his half brother has been carrying out
purges of high-ranking government officials and close associates, some
of them
\href{https://www.nytimes.com/2017/02/15/world/asia/north-korea-executions-kim-jong-un.html}{his
own relatives and mentors}. South Korean lawmakers were told last week
that since Kim Jong-un came to power in 2011, there had been a
``standing order'' to kill Kim Jong-nam, who wrote
\href{https://www.nytimes.com/2017/02/15/world/asia/kim-jong-nam-assassination-north-korea.html}{a
desperate plea} to his half brother to spare his life.

Advertisement

\protect\hyperlink{after-bottom}{Continue reading the main story}

\hypertarget{site-index}{%
\subsection{Site Index}\label{site-index}}

\hypertarget{site-information-navigation}{%
\subsection{Site Information
Navigation}\label{site-information-navigation}}

\begin{itemize}
\tightlist
\item
  \href{https://help.nytimes.com/hc/en-us/articles/115014792127-Copyright-notice}{©~2020~The
  New York Times Company}
\end{itemize}

\begin{itemize}
\tightlist
\item
  \href{https://www.nytco.com/}{NYTCo}
\item
  \href{https://help.nytimes.com/hc/en-us/articles/115015385887-Contact-Us}{Contact
  Us}
\item
  \href{https://www.nytco.com/careers/}{Work with us}
\item
  \href{https://nytmediakit.com/}{Advertise}
\item
  \href{http://www.tbrandstudio.com/}{T Brand Studio}
\item
  \href{https://www.nytimes.com/privacy/cookie-policy\#how-do-i-manage-trackers}{Your
  Ad Choices}
\item
  \href{https://www.nytimes.com/privacy}{Privacy}
\item
  \href{https://help.nytimes.com/hc/en-us/articles/115014893428-Terms-of-service}{Terms
  of Service}
\item
  \href{https://help.nytimes.com/hc/en-us/articles/115014893968-Terms-of-sale}{Terms
  of Sale}
\item
  \href{https://spiderbites.nytimes.com}{Site Map}
\item
  \href{https://help.nytimes.com/hc/en-us}{Help}
\item
  \href{https://www.nytimes.com/subscription?campaignId=37WXW}{Subscriptions}
\end{itemize}
