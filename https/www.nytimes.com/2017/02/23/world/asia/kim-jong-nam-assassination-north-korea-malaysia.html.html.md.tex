Sections

SEARCH

\protect\hyperlink{site-content}{Skip to
content}\protect\hyperlink{site-index}{Skip to site index}

\href{https://www.nytimes.com/section/world/asia}{Asia Pacific}

\href{https://myaccount.nytimes.com/auth/login?response_type=cookie\&client_id=vi}{}

\href{https://www.nytimes.com/section/todayspaper}{Today's Paper}

\href{/section/world/asia}{Asia Pacific}\textbar{}Kim Jong-nam Evidence
Being Fabricated by Malaysia, North Korea Says

\url{https://nyti.ms/2malPjO}

\begin{itemize}
\item
\item
\item
\item
\item
\end{itemize}

Advertisement

\protect\hyperlink{after-top}{Continue reading the main story}

Supported by

\protect\hyperlink{after-sponsor}{Continue reading the main story}

\hypertarget{kim-jong-nam-evidence-being-fabricated-by-malaysia-north-korea-says}{%
\section{Kim Jong-nam Evidence Being Fabricated by Malaysia, North Korea
Says}\label{kim-jong-nam-evidence-being-fabricated-by-malaysia-north-korea-says}}

\includegraphics{https://static01.nyt.com/images/2017/02/24/world/24kim-1/24kim-1-articleInline.jpg?quality=75\&auto=webp\&disable=upscale}

By \href{http://www.nytimes.com/by/choe-sang-hun}{Choe Sang-Hun} and
\href{https://www.nytimes.com/by/richard-c-paddock}{Richard C. Paddock}

\begin{itemize}
\item
  Feb. 23, 2017
\item
  \begin{itemize}
  \item
  \item
  \item
  \item
  \item
  \end{itemize}
\end{itemize}

SEOUL, South Korea --- North Korea denied responsibility on Thursday for
Kim Jong-nam's death, accusing the Malaysian authorities of fabricating
evidence of Pyongyang's involvement under the influence of the North's
archrival, South Korea.

With the North's reclusive government on the defensive about the
\href{https://www.nytimes.com/2017/02/14/world/asia/kim-jong-un-brother-killed-malaysia.html}{Feb.
13 killing of Mr. Kim}, the estranged half brother of the country's
leader, Kim Jong-un, at the airport in the Malaysian capital, Kuala
Lumpur, a statement attributed to the North Korean Jurists Committee
said that the greatest share of responsibility for the death ``rests
with the government of Malaysia'' because he died there. And in what
could be seen as a threat to Malaysia, the statement noted that North
Korea is a ``nuclear weapons state.''

But in a case that has been filled with
\href{https://www.nytimes.com/2017/02/22/world/asia/kim-jong-nam-assassination-korea-malaysia.html}{mysteries
and odd plot twists}, North Korea still would not acknowledge that the
man killed was indeed
\href{https://www.nytimes.com/2017/02/15/world/asia/kim-jong-nam-assassination-north-korea.html}{Kim
Jong-nam}. And it gave no indication that it would agree to Malaysia's
demands to question a senior staff member at the North Korean Embassy in
Kuala Lumpur in the investigation into Mr. Kim's death.

Meanwhile, relatives and acquaintances of the two women Malaysia has
accused of carrying out the killing --- by applying poison to Mr. Kim's
face as North Korean agents looked on --- insisted that they must have
been duped into doing so, though the Malaysian authorities say
otherwise.

``I don't believe Huong did such a thing,'' said Doan Van Thanh, father
of Doan Thi Huong, a 28-year-old Vietnamese woman being held in
Malaysia. ``She was a very timid girl. When she saw a rat or frog, she
would scream.''

Mr. Thanh, a 63-year-old veteran who was wounded in 1972 during the war
with the United States, said he had seen little of his daughter in
recent years. He said she left the family's home, in a small farming
village south of Hanoi, at 17 to attend community college, where she
studied to be a pharmacist.

Image

Among the suspects arrested in Mr. Kim's death are Doan Thi Huong, left,
28, and Siti Aisyah, 25, who the authorities say were recruited, trained
and equipped by North Koreans.Credit...Royal Malaysian Police

She later left Vietnam to work in Malaysia without telling her family,
Mr. Thanh said. He said she rarely visited. When she returned home in
January for the Tet holiday, he said, she had little money for the
customary gifts and stayed only a few days.

On Thursday in Nghia Binh, Ms. Huong's hometown, her brother, Doan Van
Binh, said that she
\href{https://www.facebook.com/profile.php?id=100014435959215\&lst=698293796\%253A100014435959215\%253A1487826179}{posted
on Facebook} under the alias Ruby Ruby. Her Facebook photographs and the
attached location information appear to show that she has visited
Malaysia twice since January, and her Facebook friends include several
people who write in Korean.

Mr. Binh said that Ms. Huong had also appeared briefly in a singing
contest on the Vietnamese television show ``Vietnam Idol'' in 2016. In a
\href{http://www.youtube.com/watch?v=vZNqVandYLA}{short video clip} of
the performance, a panel of judges rejected Ms. Huong after she sang
just one line: ``I want to stop breathing gloriously so that the loving
memory will not fade.''

North Korea has called for the release of Ms. Huong, an Indonesian woman
and a North Korean man who are being held by Malaysia in connection with
the death of Mr. Kim.

The statement on Thursday from the Jurists Committee was cited by the
state-run Korean Central News Agency, in the first comment on the
killing from the North's official news media. The statement accused the
Malaysian authorities of pursuing a case ``full of loopholes and
contradictions'' that proved that its investigators ``intended to frame
us.'' It said Malaysia had done so under South Korean influence.

According to the statement, the Malaysian Foreign Ministry and the local
hospital first told the North Korean Embassy in Kuala Lumpur that Mr.
Kim had died of ``heart stroke,'' asking North Korea to take the body
and cremate it.

But Malaysian officials' attitude began changing after the South Korean
news media, citing anonymous sources, reported that Mr. Kim had been
poisoned, according to the North Korean statement.

\includegraphics{https://static01.nyt.com/images/2017/02/24/world/24kim-3/24kim-3-articleInline.jpg?quality=75\&auto=webp\&disable=upscale}

``The Malaysian secret police got involved in the case and recklessly
made it an established fact'' that the death had been a poisoning,
according to the North Korean statement, which did not refer to Mr. Kim
by name.

The statement questioned how Ms. Huong and the Indonesian woman accused
in the killing, Siti Aisyah, 25, could have survived if, as Malaysian
officials said, they used their hands to administer a deadly poison to
Mr. Kim.

Ms. Siti grew up in Indonesia in similar circumstances to Ms. Huong's,
in a small farming village called Sindangsari, about four hours east of
Jakarta. She went to school through sixth grade, married at 16 and
divorced at 20, according to family members and official documents.

Family members and Indonesian officials have said they believe she was
tricked into thinking that the attack on Mr. Kim was part of a comedy
video, involving spraying liquid on unwitting victims in public. The
Malaysian authorities have said that both women were aware that the
liquid was toxic.

Rahmat Yusri, the head of the Jakarta neighborhood where Ms. Siti lived
while she was married, said she was an unlikely assassin. He recalled
that she did not have close friends.

``She is village girl, a naïve girl with a low education,'' he said.
``How can I believe that she's a murderer? Particularly that she killed
a famous person?''

The Malaysian authorities have said that four North Koreans were
believed to have directed the attack and that they fled to their
homeland after it was carried out. On Wednesday, the Malaysian police
said they were seeking to question an official at the North Korean
Embassy, Hyon Kwang Song, in the case.

Channel NewsAsia, a Singaporean news agency,
\href{http://www.channelnewsasia.com/news/asiapacific/north-korean-diplomat-allegedly-sent-off-kim-jong-nam-murder/3541734.html\#.WK44VNn8Hs4.twitter}{reported
on Thursday} that Mr. Hyon had been recorded on closed circuit cameras
at the airport after the killing, seeing off the four North Koreans as
they boarded a flight on the journey back to their homeland.

Advertisement

\protect\hyperlink{after-bottom}{Continue reading the main story}

\hypertarget{site-index}{%
\subsection{Site Index}\label{site-index}}

\hypertarget{site-information-navigation}{%
\subsection{Site Information
Navigation}\label{site-information-navigation}}

\begin{itemize}
\tightlist
\item
  \href{https://help.nytimes.com/hc/en-us/articles/115014792127-Copyright-notice}{©~2020~The
  New York Times Company}
\end{itemize}

\begin{itemize}
\tightlist
\item
  \href{https://www.nytco.com/}{NYTCo}
\item
  \href{https://help.nytimes.com/hc/en-us/articles/115015385887-Contact-Us}{Contact
  Us}
\item
  \href{https://www.nytco.com/careers/}{Work with us}
\item
  \href{https://nytmediakit.com/}{Advertise}
\item
  \href{http://www.tbrandstudio.com/}{T Brand Studio}
\item
  \href{https://www.nytimes.com/privacy/cookie-policy\#how-do-i-manage-trackers}{Your
  Ad Choices}
\item
  \href{https://www.nytimes.com/privacy}{Privacy}
\item
  \href{https://help.nytimes.com/hc/en-us/articles/115014893428-Terms-of-service}{Terms
  of Service}
\item
  \href{https://help.nytimes.com/hc/en-us/articles/115014893968-Terms-of-sale}{Terms
  of Sale}
\item
  \href{https://spiderbites.nytimes.com}{Site Map}
\item
  \href{https://help.nytimes.com/hc/en-us}{Help}
\item
  \href{https://www.nytimes.com/subscription?campaignId=37WXW}{Subscriptions}
\end{itemize}
