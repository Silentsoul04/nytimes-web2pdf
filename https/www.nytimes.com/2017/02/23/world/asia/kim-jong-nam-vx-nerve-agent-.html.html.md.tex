Sections

SEARCH

\protect\hyperlink{site-content}{Skip to
content}\protect\hyperlink{site-index}{Skip to site index}

\href{https://www.nytimes.com/section/world/asia}{Asia Pacific}

\href{https://myaccount.nytimes.com/auth/login?response_type=cookie\&client_id=vi}{}

\href{https://www.nytimes.com/section/todayspaper}{Today's Paper}

\href{/section/world/asia}{Asia Pacific}\textbar{}Kim Jong-nam Was
Killed by VX Nerve Agent, Malaysians Say

\url{https://nyti.ms/2lBkVdf}

\begin{itemize}
\item
\item
\item
\item
\item
\end{itemize}

Advertisement

\protect\hyperlink{after-top}{Continue reading the main story}

Supported by

\protect\hyperlink{after-sponsor}{Continue reading the main story}

\hypertarget{kim-jong-nam-was-killed-by-vx-nerve-agent-malaysians-say}{%
\section{Kim Jong-nam Was Killed by VX Nerve Agent, Malaysians
Say}\label{kim-jong-nam-was-killed-by-vx-nerve-agent-malaysians-say}}

\includegraphics{https://static01.nyt.com/images/2017/02/24/world/24nkorea_web2/24nkorea_web2-articleInline.jpg?quality=75\&auto=webp\&disable=upscale}

By \href{https://www.nytimes.com/by/richard-c-paddock}{Richard C.
Paddock} and \href{http://www.nytimes.com/by/choe-sang-hun}{Choe
Sang-Hun}

\begin{itemize}
\item
  Feb. 23, 2017
\item
  \begin{itemize}
  \item
  \item
  \item
  \item
  \item
  \end{itemize}
\end{itemize}

KUALA LUMPUR, Malaysia --- The poison used to kill Kim Jong-nam, the
half brother of the North Korean leader Kim Jong-un, was
\href{https://www.nytimes.com/2017/02/24/world/asia/vx-nerve-agent-kim-jong-nam.html}{VX
nerve agent}, which is listed as a chemical weapon, the Malaysian police
announced Friday.

In a brief statement, Khalid Abu Bakar, the national police chief, said
the substance was listed as a chemical weapon under the Chemical Weapons
Conventions of 1997 and 2005, to which North Korea is not a party.

South Korea has suggested that the killing was the work of the North
Korean government. The revelation that a banned weapon was used in such
a high-profile killing raises the stakes over how Malaysia and the
international community will respond.

VX nerve agent can be delivered in two compounds that are mixed at the
last moment to create a lethal dose. The police say that two women
approached Mr. Kim at the airport with the poison on their hands and
rubbed it on his face one after the other.

Samples were taken from Mr. Kim's skin and eyes. The poison was
identified in a preliminary analysis by the Center for Chemical Weapons
Analysis of the Chemistry Department of Malaysia, Mr. Khalid said.

The Chemical Weapons Convention bans the use and stockpiling of chemical
weapons, and North Korea is among the world's largest possessors of such
weapons. In 2014, the South Korean Defense Ministry said the North had
stockpiled 2,500 to 5,000 tons of chemical weapons and had a capacity to
produce a variety of biological weapons. (The North has conducted five
nuclear tests since 2006.)

VX is part of a family of nerve agents created decades ago during
research into pesticides. It is tasteless and odorless and kills by
causing uncontrollable muscle contractions, which eventually stop the
victim from breathing. A dose of about 10 milligrams is enough to kill
by skin contact, according to the Federation of American Scientists.

\includegraphics{https://static01.nyt.com/images/2017/02/24/world/24nkorea_web1/24nkorea_web1-articleLarge.jpg?quality=75\&auto=webp\&disable=upscale}

Several world powers, including the United States and the former Soviet
Union, once had large stockpiles of the nerve agent. American stores of
VX were destroyed under the Chemical Weapons Convention of 1997, with
incineration completed in 2012.

In 1994 and 1995, the Japanese cult Aum Shinrikyo used homemade VX to
attack three people, one of whom died.

North Korea is estimated to have a chemical weapons production
capability of up to 4,500 metric tons during a typical year and 12,000
tons during a period of extended crisis. It is widely reported to
possess a large arsenal of chemical weapons, including mustard, phosgene
and sarin gas, a United States Congressional Research Service report
said last year.

The announcement by Malaysia's police chief came just a day after North
Korea denied any responsibility for Mr. Kim's death, accusing the
Malaysian authorities of
\href{https://www.nytimes.com/2017/02/23/world/asia/kim-jong-nam-assassination-north-korea-malaysia.html}{fabricating
evidence of Pyongyang's involvement} under the influence of South Korea.

With the North's reclusive government on the defensive about the
\href{https://www.nytimes.com/2017/02/14/world/asia/kim-jong-un-brother-killed-malaysia.html}{Feb.
13 killing of Mr. Kim}, the estranged half brother of Kim Jong-un, at
the airport for the Malaysian capital, Kuala Lumpur, a statement
attributed to the North Korean Jurists Committee said the greatest share
of responsibility for the death ``rests with the government of
Malaysia'' because Kim Jong-nam died there. And in what could be seen as
a threat to Malaysia, the statement noted that North Korea is a
``nuclear weapons state.''

But in a case that has been filled with
\href{https://www.nytimes.com/2017/02/22/world/asia/kim-jong-nam-assassination-korea-malaysia.html}{mysteries
and odd plot twists}, North Korea still would not acknowledge that the
man killed was indeed
\href{https://www.nytimes.com/2017/02/15/world/asia/kim-jong-nam-assassination-north-korea.html}{Kim
Jong-nam}. And it gave no indication that it would agree to Malaysia's
demands to question a senior staff member at the North Korean Embassy in
Kuala Lumpur in the investigation into Mr. Kim's death.

Relatives and acquaintances of the two women Malaysia has accused of
carrying out the killing, by applying poison to Kim-Jong-nam's face as
North Korean agents looked on, insisted they must have been duped into
doing so, though the Malaysian authorities say otherwise.

``I don't believe Huong did such a thing,'' said Doan Van Thanh, father
of Doan Thi Huong, 28, a Vietnamese woman being held in Malaysia. ``She
was a very timid girl. When she saw a rat or frog, she would scream.''

Mr. Thanh, 63, said he had seen little of his daughter recently. He said
she left the family's home, in a village south of Hanoi, at 17 to attend
community college, where she studied to be a pharmacist.

Image

Mr. Kim, the estranged half brother of North Korea's leader, Kim
Jong-un, in 2010.Credit...Shin In-Seop/JoongAng Ilbo, via Associated
Press

She later left Vietnam to work in Malaysia without telling her family
and rarely visited, Mr. Thanh said. When she returned home in January
for the Tet holiday, he said, she stayed only a few days.

On Thursday in Nghia Binh, Ms. Huong's hometown, her brother, Doan Van
Binh, said that she
\href{https://www.facebook.com/profile.php?id=100014435959215\&lst=698293796\%253A100014435959215\%253A1487826179}{posted
on Facebook} under the alias Ruby Ruby. Her Facebook photographs and the
attached location information appear to show that she had visited
Malaysia twice since January, and her Facebook friends include several
people who write in Korean.

Mr. Binh said that Ms. Huong had also appeared in a singing contest on
the television show ``Vietnam Idol'' in 2016. In a
\href{http://www.youtube.com/watch?v=vZNqVandYLA}{short video clip}, a
panel of judges rejected Ms. Huong after she sang just one line: ``I
want to stop breathing gloriously so that the loving memory will not
fade.''

North Korea has called for the release of Ms. Huong, an Indonesian woman
and a North Korean man who are being held by Malaysia in connection with
the death of Mr. Kim.

The statement on Thursday from the Jurists Committee was cited by the
state-run Korean Central News Agency, in the first comment on the
killing from the North's official news media. The statement accused the
Malaysian authorities of pursuing a case ``full of loopholes and
contradictions'' that proved that its investigators ``intended to frame
us.'' It said Malaysia had done so under South Korean influence.

The statement said Malaysia's Foreign Ministry and the local hospital
first told the North Korean Embassy in Kuala Lumpur that Mr. Kim had
died of ``heart stroke,'' asking North Korea to take the body and
cremate it.

But Malaysian officials' attitude began changing after the South Korean
news media, citing anonymous sources, reported that Mr. Kim had been
poisoned, according to the North Korean statement.

``The Malaysian secret police got involved in the case and recklessly
made it an established fact'' that the death had been a poisoning,
according to the North Korean statement, which did not refer to Mr. Kim
by name.

The statement questioned how Ms. Huong and the Indonesian suspect in the
killing, Siti Aisyah, 25, had survived if, as Malaysian officials said,
they had used their hands to apply a deadly poison.

Advertisement

\protect\hyperlink{after-bottom}{Continue reading the main story}

\hypertarget{site-index}{%
\subsection{Site Index}\label{site-index}}

\hypertarget{site-information-navigation}{%
\subsection{Site Information
Navigation}\label{site-information-navigation}}

\begin{itemize}
\tightlist
\item
  \href{https://help.nytimes.com/hc/en-us/articles/115014792127-Copyright-notice}{©~2020~The
  New York Times Company}
\end{itemize}

\begin{itemize}
\tightlist
\item
  \href{https://www.nytco.com/}{NYTCo}
\item
  \href{https://help.nytimes.com/hc/en-us/articles/115015385887-Contact-Us}{Contact
  Us}
\item
  \href{https://www.nytco.com/careers/}{Work with us}
\item
  \href{https://nytmediakit.com/}{Advertise}
\item
  \href{http://www.tbrandstudio.com/}{T Brand Studio}
\item
  \href{https://www.nytimes.com/privacy/cookie-policy\#how-do-i-manage-trackers}{Your
  Ad Choices}
\item
  \href{https://www.nytimes.com/privacy}{Privacy}
\item
  \href{https://help.nytimes.com/hc/en-us/articles/115014893428-Terms-of-service}{Terms
  of Service}
\item
  \href{https://help.nytimes.com/hc/en-us/articles/115014893968-Terms-of-sale}{Terms
  of Sale}
\item
  \href{https://spiderbites.nytimes.com}{Site Map}
\item
  \href{https://help.nytimes.com/hc/en-us}{Help}
\item
  \href{https://www.nytimes.com/subscription?campaignId=37WXW}{Subscriptions}
\end{itemize}
