Sections

SEARCH

\protect\hyperlink{site-content}{Skip to
content}\protect\hyperlink{site-index}{Skip to site index}

\href{https://www.nytimes.com/section/world/asia}{Asia Pacific}

\href{https://myaccount.nytimes.com/auth/login?response_type=cookie\&client_id=vi}{}

\href{https://www.nytimes.com/section/todayspaper}{Today's Paper}

\href{/section/world/asia}{Asia Pacific}\textbar{}South Korea's Top
Spies Give New Evidence in Plot to Kill Kim Jong-nam

\url{https://nyti.ms/2lM4UTB}

\begin{itemize}
\item
\item
\item
\item
\item
\end{itemize}

Advertisement

\protect\hyperlink{after-top}{Continue reading the main story}

Supported by

\protect\hyperlink{after-sponsor}{Continue reading the main story}

\hypertarget{south-koreas-top-spies-give-new-evidence-in-plot-to-kill-kim-jong-nam}{%
\section{South Korea's Top Spies Give New Evidence in Plot to Kill Kim
Jong-nam}\label{south-koreas-top-spies-give-new-evidence-in-plot-to-kill-kim-jong-nam}}

\includegraphics{https://static01.nyt.com/images/2017/02/28/world/28Kim/29Kim-articleLarge.jpg?quality=75\&auto=webp\&disable=upscale}

By \href{http://www.nytimes.com/by/choe-sang-hun}{Choe Sang-Hun}

\begin{itemize}
\item
  Feb. 27, 2017
\item
  \begin{itemize}
  \item
  \item
  \item
  \item
  \item
  \end{itemize}
\end{itemize}

\href{http://cn.nytimes.com/asia-pacific/20170228/north-korea-kim-jong-nam-state-security/}{阅读简体中文版}

SEOUL, South Korea --- Officials from North Korea's secret police and
Foreign Ministry were involved in the killing of the estranged half
brother of the country's leader, South Korean intelligence officials
told lawmakers on Monday.

Ever since
\href{https://www.nytimes.com/2017/02/14/world/asia/kim-jong-un-brother-killed-malaysia.html?action=click\&contentCollection=Asia\%20Pacific\&module=RelatedCoverage\&region=Marginalia\&pgtype=article}{Kim
Jong-nam}, the eldest brother of the North Korean leader, Kim Jong-un,
was first reported assassinated, the South Korean government has held
the North responsible. On Monday, the National Intelligence Service in
Seoul provided more details of what it described as state-sponsored
terrorism, saying that four of the eight North Koreans identified as
suspects by the Malaysian authorities were agents from North Korea's
Ministry of State Security, the country's secret police.

Speaking on Monday in a closed-door parliamentary hearing, Lee Byung-ho,
the director of the National Intelligence Service, said that two other
suspects worked for the North Korean Ministry of Foreign Affairs. The
remaining two were affiliated with
\href{http://www.airkoryo.com.kp/}{Air Koryo}, the North's state-run
airline company, and Singwang Economics and Trading General Corporation,
Mr. Lee said, according to two lawmakers who attended the briefing.
Singwang is among North Korean companies facing United Nations
sanctions.

The Malaysian authorities have said that Mr. Kim was
\href{https://www.nytimes.com/2017/02/23/world/asia/kim-jong-nam-vx-nerve-agent-.html?action=click\&contentCollection=Asia\%20Pacific\&module=RelatedCoverage\&region=Marginalia\&pgtype=article}{killed
by an extremely toxic nerve agent known as VX}. They said that the North
Koreans had hired and trained two women, one from Indonesia, the other
from Vietnam, to attack Mr. Kim at Kuala Lumpur International Airport.
The women smeared his face with the chemical while he was waiting to
check in for a flight to Macau, where he and his family had a home, they
said.

The two women are now in police custody in Kuala Lumpur.

Mr. Lee, the South Korean intelligence chief, was quoted by the
lawmakers as saying that the eight North Koreans, working as two
four-member teams, converged in Kuala Lumpur to carry out the Feb. 13
assassination.

He said that Ri Jae-nam, a state security agent, and Ri Ji-hyon, a
Foreign Ministry official, had brought
\href{https://www.nytimes.com/2017/02/24/world/asia/kim-jong-nam-suspect-doan-thi-huong.html}{Doan
Thi Huong}, a 28-year-old Vietnamese woman, into the assassination plot,
while
\href{https://www.nytimes.com/2017/02/25/world/asia/north-korea-kim-jong-nam-vx-nerve-agent-siti-aisyah.html}{Siti
Aisyah}, a 25-year-old Indonesian woman, was hired by O Jong-gil, a
state security agent, and by Hong Song-hac, a Foreign Ministry official.

The four North Koreans who made up the assassination team left Malaysia
the same day Mr. Kim was killed and are believed to be back in their
country, Mr. Lee was quoted as saying. The Malaysian police have
confirmed their departure.

\href{https://www.nytimes.com/2017/02/21/world/asia/kim-jong-nam-killing-malaysia-north-korea.html?rref=collection\%2Ftimestopic\%2FKim\%20Jong-un\&action=click\&contentCollection=timestopics\&region=stream\&module=stream_unit\&version=latest\&contentPlacement=9\&pgtype=collection}{Hyon
Kwang-song}, a senior diplomat at the North Korean Embassy in Kuala
Lumpur, and three other North Koreans worked as a support team, Mr. Lee
told the lawmakers, keeping track of Kim Jong-nam's whereabouts and
providing logistical assistance. Mr. Hyon worked for the Ministry of
State Security, he said.

Mr. Hyon and the Air Koryo employee, Kim Uk-il, remain at the embassy in
Malaysia.

A third member of the support team, identified as Ri Jong-chol, has been
arrested in Kuala Lumpur. The fourth, identified as Ri Ji-u, is believed
to be at large in Malaysia.

North Korea's Ministry of State Security specializes in ferreting out
people whose loyalty to Kim Jong-un's totalitarian regime is in doubt.
Mr. Kim is believed to have used the ministry in the arrests and
executions of senior officials, including an uncle,
\href{http://www.nytimes.com/2013/12/13/world/asia/north-korea-says-uncle-of-executed.html}{Jang
Song-thaek}, who was executed on charges of corruption and sedition in
2013. The ministry also runs a network of prison camps.

During the intelligence briefing on Monday, Mr. Lee told the lawmakers
that five senior officials affiliated with the State Security Ministry
had been executed with antiaircraft guns. He also said that
\href{https://www.nytimes.com/2017/02/03/world/asia/north-korea-purge-kim-jong-un-kim-won-hong.html?_r=0}{Gen.
Kim Won-hong}, who was removed as chief of the secret police in January,
was in detention as part of a purge.

Mr. Lee said that General Kim and his five deputies had angered Kim
Jong-un by filing false reports, but he did not elaborate. The South's
intelligence agency had said earlier that General Kim was dismissed on
charges of corruption and abuse of power.

He was the latest in a series of high-ranking party and military
officials that Kim Jong-un has fired, demoted or executed in efforts to
consolidate his power through what South Korean officials have called a
``reign of terror.''

Advertisement

\protect\hyperlink{after-bottom}{Continue reading the main story}

\hypertarget{site-index}{%
\subsection{Site Index}\label{site-index}}

\hypertarget{site-information-navigation}{%
\subsection{Site Information
Navigation}\label{site-information-navigation}}

\begin{itemize}
\tightlist
\item
  \href{https://help.nytimes.com/hc/en-us/articles/115014792127-Copyright-notice}{©~2020~The
  New York Times Company}
\end{itemize}

\begin{itemize}
\tightlist
\item
  \href{https://www.nytco.com/}{NYTCo}
\item
  \href{https://help.nytimes.com/hc/en-us/articles/115015385887-Contact-Us}{Contact
  Us}
\item
  \href{https://www.nytco.com/careers/}{Work with us}
\item
  \href{https://nytmediakit.com/}{Advertise}
\item
  \href{http://www.tbrandstudio.com/}{T Brand Studio}
\item
  \href{https://www.nytimes.com/privacy/cookie-policy\#how-do-i-manage-trackers}{Your
  Ad Choices}
\item
  \href{https://www.nytimes.com/privacy}{Privacy}
\item
  \href{https://help.nytimes.com/hc/en-us/articles/115014893428-Terms-of-service}{Terms
  of Service}
\item
  \href{https://help.nytimes.com/hc/en-us/articles/115014893968-Terms-of-sale}{Terms
  of Sale}
\item
  \href{https://spiderbites.nytimes.com}{Site Map}
\item
  \href{https://help.nytimes.com/hc/en-us}{Help}
\item
  \href{https://www.nytimes.com/subscription?campaignId=37WXW}{Subscriptions}
\end{itemize}
