Sections

SEARCH

\protect\hyperlink{site-content}{Skip to
content}\protect\hyperlink{site-index}{Skip to site index}

\href{https://www.nytimes.com/section/world/asia}{Asia Pacific}

\href{https://myaccount.nytimes.com/auth/login?response_type=cookie\&client_id=vi}{}

\href{https://www.nytimes.com/section/todayspaper}{Today's Paper}

\href{/section/world/asia}{Asia Pacific}\textbar{}U.S. Starts Deploying
Thaad Antimissile System in South Korea, After North's Tests

\url{https://nyti.ms/2n8AwR9}

\begin{itemize}
\item
\item
\item
\item
\item
\end{itemize}

Advertisement

\protect\hyperlink{after-top}{Continue reading the main story}

Supported by

\protect\hyperlink{after-sponsor}{Continue reading the main story}

\hypertarget{us-starts-deploying-thaad-antimissile-system-in-south-korea-after-norths-tests}{%
\section{U.S. Starts Deploying Thaad Antimissile System in South Korea,
After North's
Tests}\label{us-starts-deploying-thaad-antimissile-system-in-south-korea-after-norths-tests}}

\includegraphics{https://static01.nyt.com/images/2017/03/07/world/asia/07thaad-1/07thaad-1-videoSixteenByNine3000.jpg}

By \href{http://www.nytimes.com/by/gerry-mullany}{Gerry Mullany} and
\href{http://www.nytimes.com/by/michael-r-gordon}{Michael R. Gordon}

\begin{itemize}
\item
  March 6, 2017
\item
  \begin{itemize}
  \item
  \item
  \item
  \item
  \item
  \end{itemize}
\end{itemize}

HONG KONG --- Alarmed over North Korea's increasingly provocative
behavior, the United States said Tuesday that it had
\href{https://www.youtube.com/watch?v=BI68rmNQWXQ}{started to deploy} an
antimissile system in
\href{https://www.nytimes.com/topic/destination/south-korea?8qa}{South
Korea} that
\href{https://www.nytimes.com/topic/destination/china?inline=nyt-geo}{China}
has angrily opposed as a threat to its security.

The deployment of the
\href{http://www.lockheedmartin.com/us/products/thaad.html}{Terminal
High Altitude Area Defense system}, or Thaad, came after North Korea
\href{https://www.nytimes.com/2017/03/05/world/north-korea-ballistic-missiles.html?rref=collection\%2Ftimestopic\%2FSouth\%20Korea\&action=click\&contentCollection=world\&region=stream\&module=stream_unit\&version=latest\&contentPlacement=4\&pgtype=collection}{launched
four ballistic missiles} on Monday, apparently in response to joint
naval exercises by South Korea and the United States. Those launchings
led South Korea to call for the accelerated deployment of Thaad.

A spokeswoman for the United States forces in South Korea said that one
of five major components of the missile system had arrived on Monday.
Officials said it could take a couple of months for the system to become
fully operational. Defense Secretary
\href{https://www.nytimes.com/2017/02/02/world/asia/james-mattis-us-korea-thaad.html}{Jim
Mattis had urged} the South Koreans to move ahead with the deployment of
the system during a visit to Seoul in February.

In telephone calls on Monday to South Korean and Japanese leaders,
President Trump said the United States would stand with its Asian allies
and take steps to defend against North Korea's growing ballistic missile
threat.

Mr. Trump emphasized that the United States was taking steps to
``enhance our ability to deter and defend against North Korea's
ballistic missiles using the full range of United States military
capabilities,'' the White House said in a statement.

China has been incensed over the deployment of the system, fearing it
could give the United States military the ability to quickly detect and
track missiles launched in China, according to analysts. A spokesman for
China's Ministry of Foreign Affairs, Geng Shuang, said Tuesday that
China would ``take the necessary steps to safeguard our own security
interests, and the consequences will be shouldered by the United States
and South Korea.''

Mr. Geng warned the two countries not to ``go further and further down
the wrong road.''

Yang Xiyu, a former senior Chinese official who once oversaw talks with
North Korea, said China was worried that the deployment of the system
would open the door to a broader American network of antimissile systems
in the region, possibly in places like Japan and the Philippines, to
counter a growing Chinese military.

``China can see benefits only for a U.S. regional plan, not for South
Korea's national security interest,'' he said.

The state media recently encouraged Chinese citizens to boycott South
Korean products and companies over the Thaad issue. The Chinese
authorities recently forced the closing of 23 stores owned by Lotte, a
South Korean conglomerate that agreed to turn over land that it owned
for use in the Thaad deployment. Hundreds of Chinese protested at Lotte
stores over the weekend, some holding banners that read, ``Get out of
China.''

Adm. Harry B. Harris Jr., the head of the United States Pacific Command,
announced the start of the deployment, saying that ``continued
provocative actions by North Korea, to include yesterday's launch of
multiple missiles, only confirm the prudence of our alliance decision
last year to deploy Thaad to South Korea.''

The developments come as South Korea is consumed by turmoil over the
impeachment of President
\href{http://topics.nytimes.com/top/reference/timestopics/people/p/park_geunhye/index.html?inline=nyt-per}{Park
Geun-hye}, whose administration agreed to the Thaad deployment. But with
the president facing possible removal from office over a corruption
scandal, the fate of the system had been in doubt. Its accelerated
deployment could make it harder, if not impossible, for her successor to
head off its installation.

Moon Jae-in, an opposition leader who is the front-runner in the race to
replace President Park, acknowledged that it would be difficult to
overturn South Korea's agreement to deploy the system. But he has
insisted that the next South Korean government should have the final say
on the matter, saying that Ms. Park's government never allowed a full
debate on it.

Last year,
\href{https://www.nytimes.com/2016/07/14/world/asia/south-korea-thaad-us.html}{thousands
of people in Seongju}, a rural southern county in South Korea, protested
when it was announced that a Thaad battery would be established there.
They said they feared that the system would harm their agricultural
livelihoods. Many South Koreans also worry that any expansion of
military ties with the United States could worsen already festering
tensions with North Korea and China.

Under its deal with Washington, South Korea is providing the land for
the missile system and will build the base, but the United States will
pay for the system, to be built by Lockheed Martin, as well as its
operational costs.

The United States military statement said that ``the first elements'' of
Thaad were deployed on Monday, the same day as the North's missile
launchings.

A C-17 cargo plane landed at the United States military's Osan Air Base,
about 40 miles south of Seoul, on Monday evening, carrying two trucks,
each mounted with a Thaad launchpad. More equipment and personnel will
start arriving in the coming weeks, South Korean military officials
said.

``South Korea and the United States are doing their best to make the
Thaad system operational as soon as possible,'' the South Korean Defense
Ministry said in a statement on Tuesday, adding that the system was
necessary ``to protect South Korea from the nuclear and missile threat
from North Korea.''

The ministry declined to specify when the system would be operational.
But the South Korean news agency Yonhap reported that the deployment was
likely to be completed in one or two months, with the system ready for
use by April.

The arrival of Thaad equipment was announced after South Korea's acting
president, Hwang Kyo-ahn, talked with Mr. Trump on the phone on Tuesday
morning. The two leaders condemned the North's missile tests as a
violation of the United Nations Security Council resolutions and agreed
to beef up the allies' joint defense posture, strengthen sanctions and
step up pressure against the North, Mr. Hwang's office said.

On the phone with Mr. Trump, Mr. Hwang called the North's nuclear and
missile threat a ``present and direct danger'' to its allies, his office
said.

The Japanese prime minister, Shinzo Abe, said he spoke for 25 minutes on
Tuesday with Mr. Trump, who reiterated his pledge to stand by Japan
``100 percent,'' according to the public broadcaster NHK. ``I appreciate
that the United States is showing that all the options are on the
table,'' Mr. Abe said, adding that Japan was ``ready to fulfill larger
roles and responsibilities'' to deter North Korea.

Takashi Kawakami, a professor of international politics and security at
Takushoku University in Tokyo, said the deployment of Thaad could put
the United States in a stronger position to consider a pre-emptive
strike on North Korea. If the United States took such action, he said,
``North Korea is going to make a counterattack on the U.S. or Japan or
another place, so in this case they will use Thaad.''

With tensions increasing over the deployment of the system, some in
China have advocated stern measures, including severing diplomatic
relations with South Korea, or more.

A retired general, Luo Yuan, even
\href{https://www.nytimes.com/2017/03/02/world/asia/china-north-south-korea.html}{suggested
that China destroy the system} with a military strike.

``We could conduct a surgical hard-kill operation that would destroy the
target, paralyzing it and making it unable to hit back,'' General Luo
wrote in Global Times, a state-run newspaper.

Advertisement

\protect\hyperlink{after-bottom}{Continue reading the main story}

\hypertarget{site-index}{%
\subsection{Site Index}\label{site-index}}

\hypertarget{site-information-navigation}{%
\subsection{Site Information
Navigation}\label{site-information-navigation}}

\begin{itemize}
\tightlist
\item
  \href{https://help.nytimes.com/hc/en-us/articles/115014792127-Copyright-notice}{©~2020~The
  New York Times Company}
\end{itemize}

\begin{itemize}
\tightlist
\item
  \href{https://www.nytco.com/}{NYTCo}
\item
  \href{https://help.nytimes.com/hc/en-us/articles/115015385887-Contact-Us}{Contact
  Us}
\item
  \href{https://www.nytco.com/careers/}{Work with us}
\item
  \href{https://nytmediakit.com/}{Advertise}
\item
  \href{http://www.tbrandstudio.com/}{T Brand Studio}
\item
  \href{https://www.nytimes.com/privacy/cookie-policy\#how-do-i-manage-trackers}{Your
  Ad Choices}
\item
  \href{https://www.nytimes.com/privacy}{Privacy}
\item
  \href{https://help.nytimes.com/hc/en-us/articles/115014893428-Terms-of-service}{Terms
  of Service}
\item
  \href{https://help.nytimes.com/hc/en-us/articles/115014893968-Terms-of-sale}{Terms
  of Sale}
\item
  \href{https://spiderbites.nytimes.com}{Site Map}
\item
  \href{https://help.nytimes.com/hc/en-us}{Help}
\item
  \href{https://www.nytimes.com/subscription?campaignId=37WXW}{Subscriptions}
\end{itemize}
