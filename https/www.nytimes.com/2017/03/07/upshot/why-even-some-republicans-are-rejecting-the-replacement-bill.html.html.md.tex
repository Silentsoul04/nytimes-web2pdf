Sections

SEARCH

\protect\hyperlink{site-content}{Skip to
content}\protect\hyperlink{site-index}{Skip to site index}

\href{https://myaccount.nytimes.com/auth/login?response_type=cookie\&client_id=vi}{}

\href{https://www.nytimes.com/section/todayspaper}{Today's Paper}

\href{/section/upshot}{The Upshot}\textbar{}Why Even Some Republicans
Are Rejecting the Replacement Bill

\url{https://nyti.ms/2mfB9sA}

\begin{itemize}
\item
\item
\item
\item
\item
\item
\end{itemize}

Advertisement

\protect\hyperlink{after-top}{Continue reading the main story}

Supported by

\protect\hyperlink{after-sponsor}{Continue reading the main story}

Upshot

Public Health

\hypertarget{why-even-some-republicans-are-rejecting-the-replacement-bill}{%
\section{Why Even Some Republicans Are Rejecting the Replacement
Bill}\label{why-even-some-republicans-are-rejecting-the-replacement-bill}}

By \href{http://www.nytimes.com/by/margot-sanger-katz}{Margot
Sanger-Katz}

\begin{itemize}
\item
  March 7, 2017
\item
  \begin{itemize}
  \item
  \item
  \item
  \item
  \item
  \item
  \end{itemize}
\end{itemize}

Republicans in Congress are fond of
\href{https://www.youtube.com/watch?v=ZFRg14Wg9Bk\&feature=youtu.be\&t=9m17s}{calling
Obamacare a death spiral} of escalating costs and declining coverage.
But their replacement plan could make those problems even worse.

Tuesday, the morning after two House committees released legislation
that would replace the Affordable Care Act with the American Health Care
Act, the plan received a tepid, even hostile reaction from many outside
conservative groups and Republicans in Congress. The bill does less than
many conservatives had hoped to open up the market for health insurance.
And it still offers the kind of subsidies to middle-income people that
they see as too generous. Mike Lee, a conservative senator from Utah,
described the bill as ``not the Obamacare repeal bill we've been waiting
for.'' (It perhaps goes without saying that it has been universally
panned by Democrats.)

And here's why.

Republicans are constrained by their small majority in the Senate.
Instead of passing a normal piece of legislation that could change the
Affordable Care Act's many insurance regulations, they are limited to
changes that can be accomplished through a technical budgetary maneuver,
\href{https://www.nytimes.com/2017/01/04/us/politics/the-parliamentary-trick-that-could-obliterate-obamacare.html}{a
reconciliation bill}.

The result is likely to be higher prices for insurance, and fewer people
with the ability to buy it. "The individual market seems less stable
under this bill than under the Affordable Care Act,'' said Larry Levitt,
a senior vice president at the Kaiser Family Foundation, a health
research group. ``This bill comes across as a bit of a Frankenstein
health bill because of the legislative strategy that's being used.''

\href{http://www.nytimes.com/2017/03/07/us/politics/affordable-care-act-obama-care-health.html}{Many
conservative legislators would prefer} a reprise of a 2015 bill that
would have simply swept away much of Obamacare. But the current bill's
drafters have felt political pressure from President Trump and their
constituents to preserve some of the health law's coverage gains. Their
efforts to preserve popular parts of the law and work within the special
budget rules have led to the uneasy mix of policies in the bill.

The bill keeps many of the Affordable Care Act's rules for insurance
companies that Republicans have decried for raising costs. Here's what
stays.

■ The health law's rule that insurance companies must sell polices to
the healthy and the sick at the same price.

■ Its rule that insurance companies can't limit the benefits they pay
out in a year or a lifetime.

■ Its requirements that all plans cover 10 categories of benefits,
including preventive health services without a co-payment,
rehabilitation services and maternity care.

\includegraphics{https://static01.nyt.com/images/2017/03/09/us/09up-health/up-health-articleInline.jpg?quality=75\&auto=webp\&disable=upscale}

■ The law's caps on how much customers can be asked to pay for health
care through deductibles and co-payments.

Those are popular provisions; they tend to make insurance coverage
comprehensive but also somewhat costly. (The main reason that health
insurance plans are expensive, of course, is because medical care is
expensive, and the bill doesn't do anything about that either.) Because
they all stay, the rest of the policy changes are built atop a chassis
of health insurance products that cost what today's plans cost.

The bill effectively slashes subsidies that help many low-income people
buy insurance, starting in 2020. A 60-year-old earning \$20,000 in
Lincoln, Neb., currently gets a subsidy of \$18,470 to help her buy
insurance, with extra subsidies to help her pay deductibles and
co-payments, according to calculations made by Kaiser. Under the new
legislation, she would get a subsidy of \$4,000, and no help with cost
sharing.

The bill also does away with Obamacare's requirement that people have
insurance or pay a fine. That provision is unpopular, but it is seen as
an important incentive for healthy people to buy insurance every year.
The Congressional Budget Office has estimated that eliminating that
provision would lead to premiums that are 20 to 25 percent higher, even
without any cuts in subsidies.

As a counterweight, the bill does some things that would tend to
stabilize prices. It gives states a big pot of money to help keep
markets working. It allows insurance companies to charge higher prices
to old customers and less to younger ones. That is not so good for our
hypothetical 60-year-old in Nebraska, but might help lure some healthy
20-year-olds into the market who don't buy insurance now.

It also creates a new kind of financial incentive to buy insurance:
People with a lapse in insurance coverage of more than a couple of
months would have to pay a 30 percent higher price for their insurance
when they re-enter the market. Advocates say this provision would get
people to stay insured when they are healthy so they can afford coverage
later. But some critics think it could backfire, since only sick people
would be willing to pay the extra fee, which might not be enough to
cover the extra cost of their care.

``The people who are going to take this gamble are going to be the
healthiest,'' said Craig Garthwaite, the director of the health program
at Northwestern's Kellogg School of Management. ``The only time you are
going to get them into the market is if they get sick.''

Insurers have been vague so far on how they feel about this mix of
policies. And the Congressional Budget Office, Washington's official
scorekeeper, has not weighed in with estimates of how many people would
be covered or what the bill would cost the federal government. But
several health policy experts have said they believe the policy changes
could result in the loss of health insurance for 10 million Americans or
more.

Joe Antos, a scholar at the conservative American Enterprise Institute,
said he sees some good ideas in the health bill, but estimated that it
would cost 10 to 15 million people their insurance over the coming
decade: ``It can't be a cohesive whole, because the things that can't be
in a reconciliation bill aren't here.''

Advertisement

\protect\hyperlink{after-bottom}{Continue reading the main story}

\hypertarget{site-index}{%
\subsection{Site Index}\label{site-index}}

\hypertarget{site-information-navigation}{%
\subsection{Site Information
Navigation}\label{site-information-navigation}}

\begin{itemize}
\tightlist
\item
  \href{https://help.nytimes.com/hc/en-us/articles/115014792127-Copyright-notice}{©~2020~The
  New York Times Company}
\end{itemize}

\begin{itemize}
\tightlist
\item
  \href{https://www.nytco.com/}{NYTCo}
\item
  \href{https://help.nytimes.com/hc/en-us/articles/115015385887-Contact-Us}{Contact
  Us}
\item
  \href{https://www.nytco.com/careers/}{Work with us}
\item
  \href{https://nytmediakit.com/}{Advertise}
\item
  \href{http://www.tbrandstudio.com/}{T Brand Studio}
\item
  \href{https://www.nytimes.com/privacy/cookie-policy\#how-do-i-manage-trackers}{Your
  Ad Choices}
\item
  \href{https://www.nytimes.com/privacy}{Privacy}
\item
  \href{https://help.nytimes.com/hc/en-us/articles/115014893428-Terms-of-service}{Terms
  of Service}
\item
  \href{https://help.nytimes.com/hc/en-us/articles/115014893968-Terms-of-sale}{Terms
  of Sale}
\item
  \href{https://spiderbites.nytimes.com}{Site Map}
\item
  \href{https://help.nytimes.com/hc/en-us}{Help}
\item
  \href{https://www.nytimes.com/subscription?campaignId=37WXW}{Subscriptions}
\end{itemize}
