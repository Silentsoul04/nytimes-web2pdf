Sections

SEARCH

\protect\hyperlink{site-content}{Skip to
content}\protect\hyperlink{site-index}{Skip to site index}

\href{https://www.nytimes.com/section/world/asia}{Asia Pacific}

\href{https://myaccount.nytimes.com/auth/login?response_type=cookie\&client_id=vi}{}

\href{https://www.nytimes.com/section/todayspaper}{Today's Paper}

\href{/section/world/asia}{Asia Pacific}\textbar{}South Korea Removes
President Park Geun-hye

\url{https://nyti.ms/2mrwoxU}

\begin{itemize}
\item
\item
\item
\item
\item
\item
\end{itemize}

Advertisement

\protect\hyperlink{after-top}{Continue reading the main story}

Supported by

\protect\hyperlink{after-sponsor}{Continue reading the main story}

\hypertarget{south-korea-removes-president-park-geun-hye}{%
\section{South Korea Removes President Park
Geun-hye}\label{south-korea-removes-president-park-geun-hye}}

\includegraphics{https://static01.nyt.com/images/2017/03/11/world/10impeachyes-01/10impeachyes-01-videoSixteenByNineJumbo1600.jpg}

By \href{http://www.nytimes.com/by/choe-sang-hun}{Choe Sang-Hun}

\begin{itemize}
\item
  March 9, 2017
\item
  \begin{itemize}
  \item
  \item
  \item
  \item
  \item
  \item
  \end{itemize}
\end{itemize}

\href{https://www.nytimes.com/es/2017/03/10/corea-del-sur-destituye-a-su-presidenta-y-se-reconfigura-el-panorama-politico-en-asia/}{Leer
en español}

SEOUL, South Korea --- A South Korean court removed the president on
Friday, a first in the nation's history, rattling the delicate balance
of relationships across Asia at
\href{https://www.nytimes.com/2017/03/07/world/asia/thaad-missile-defense-us-south-korea-china.html}{a
particularly tense time}.

Her removal capped months of turmoil, as hundreds of thousands of South
Koreans took to the streets, week after week,
\href{https://www.nytimes.com/2016/11/26/world/asia/korea-park-geun-hye-protests.html}{to
protest a sprawling corruption scandal} that shook the top echelons of
business and government.

Park Geun-hye, the nation's first female president and the daughter of
the Cold War military dictator Park Chung-hee, had been an icon of the
conservative establishment that joined Washington in pressing for a hard
line against North Korea's nuclear provocations.

Now, her downfall is expected to shift South Korean politics to the
opposition, whose leaders want more engagement with North Korea and are
wary of a major confrontation in the region. They say they will
re-examine the country's joint strategy on North Korea with the United
States and defuse tensions with China, which has sounded alarms about
the growing American military footprint in Asia.

Ms. Park's powers
\href{https://www.nytimes.com/2016/12/09/world/asia/south-korea-president-park-geun-hye-impeached.html}{were
suspended in December} after a legislative impeachment vote, though she
continued to live in the presidential Blue House, largely alone and
hidden from public view, while awaiting the decision by the
Constitutional Court. The house had been her childhood home: She first
moved in at the age of 9 and left it nearly two decades later after her
mother and father were assassinated in separate episodes.

Eight justices of the Constitutional Court unanimously decided to unseat
Ms. Park for committing ``acts that violated the Constitution and laws''
throughout her time in office, Acting Chief Justice Lee Jung-mi said in
a ruling that was nationally broadcast.

\includegraphics{https://static01.nyt.com/images/2017/03/11/world/10impeachyes-02/10impeachyes-02-articleInline.jpg?quality=75\&auto=webp\&disable=upscale}

Ms. Park's acts ``betrayed the trust of the people and were of the kind
that cannot be tolerated for the sake of protecting the Constitution,''
Justice Lee said.

As the verdict was announced, silence fell over thousands of Park
supporters who rallied near the courthouse waving South Korean flags.
Soon, they tried to march on the court and called for ``destroying'' it.
When the police blocked them, some of the mostly elderly protesters
attacked the officers with flagpoles, hurling water bottles and pieces
of the sidewalk pavement. Two pro-Park demonstrators, ages 60 and 72,
died during the unrest.

Ms. Park did not comment on the ruling, and remained in the presidential
palace after her removal from power. But In Myung-jin, the leader of Ms.
Park's conservative Liberty Korea Party, said he ``humbly respected''
the ruling.

With the immunity conferred by her office now gone, Ms. Park, 65, faces
prosecutors seeking to charge her with bribery, extortion and abuse of
power in connection with allegations of conspiring with a confidante,
her childhood friend Choi Soon-sil, to collect tens of millions of
dollars in bribes from companies like Samsung.

By law, the country must elect a new president within 60 days. The
acting president, Hwang Kyo-ahn, an ally of Ms. Park's, will remain in
office in the interim. The Trump administration is rushing
\href{https://www.nytimes.com/2017/03/07/world/asia/korea\%2Dmissile\%2Ddefense\%2Dchina\%2Dtrump.html?_r=0}{a
missile defense system} to South Korea so that it can be in place before
the election.

After the ruling, Mr. Hwang called key Cabinet ministers to put the
nation on a heightened state of military readiness, saying the lack of a
president represented a national ``emergency.'' He also warned North
Korea against making ``additional provocations.''

Image

Celebrating after the verdict by the Constitutional Court in Seoul, the
capital, on Friday. By law, the country must elect a new president
within 60 days.Credit...Chung Sung-Jun/Getty Images

The last time a South Korean leader was removed from office under
popular pressure was in 1960, when the police fired on crowds calling
for President Syngman Rhee to step down.
(\href{https://timesmachine.nytimes.com/timesmachine/1965/07/20/96708697.html?pageNumber=1}{Mr.
Rhee, a dictator, fled into exile in Hawaii and died there}.)

In a sign of how far South Korea's young democracy has evolved, Ms. Park
was removed without any violence, after large, peaceful protests in
recent months demanding that she step down. In addition to the swell of
popular anger, the legislature and the judiciary --- two institutions
that have been weaker than the presidency historically --- were crucial
to the outcome.

``This is a miracle, a new milestone in the strengthening and
institutionalizing of democracy in South Korea,'' said Kang Won-taek, a
political scientist at Seoul National University.

When crowds took to the streets, they were not just seeking to remove a
leader who had one year left in office. They were also rebelling against
a political order that had held South Korea together for decades but is
now fracturing under pressures both at home and abroad, analysts said.

Ms. Park's father ruled South Korea from 1961 to 1979. He founded its
economic growth model, which transformed the nation into an export
powerhouse and allowed the emergence of family-controlled conglomerates
known as chaebol that benefited from tax cuts and anti-labor policies.

\href{http://www.nytimes.com/2012/12/20/world/asia/south-koreans-vote-in-closely-fought-presidential-race.html}{Ms.
Park was elected in 2012} with the support of older conservative South
Koreans who revered her father for the country's breakneck economic
growth.

Image

Supporters of Ms. Park trying to pass the barricade of police buses to
protest the ruling. As the verdict was announced, silence fell over the
supporters who rallied near the courthouse.Credit...Jung
Ui-Chel/European Pressphoto Agency

But the nexus of industry and political power gave rise to collusive
ties, highlighted by the scandal that led to Ms. Park's fall.

The scandal also swept up the
\href{https://www.nytimes.com/2017/03/09/business/jay-y-lee-samsung-trial.html}{de
facto head of Samsung, Lee Jae-yong}, who was indicted on charges of
bribing Ms. Park and her confidante, Ms. Choi.

Samsung, the nation's largest conglomerate, has been tainted by
corruption before. But the company has been considered too important to
the economy for any of its top leaders to spend time behind bars ---
until now. The jailing of Mr. Lee, who is facing trial, is another
potent sign that the old order is not holding.

In the wake of the Park scandal, all political parties have vowed to
curtail presidential power to pardon chaebol tycoons convicted of
white-collar crimes. They also promised to stop chaebol chairmen from
helping their children amass fortunes through dubious means, like
forcing their companies to do exclusive business with the children's
businesses.

With the conservatives discredited --- and no leading conservative
candidate to succeed Ms. Park --- the left could take power for the
first time in a decade. The dominant campaign issues will probably be
North Korea's nuclear weapons program and South Korea's relations with
the United States and China.

If the opposition takes power, it may try to revive its old ``sunshine
policy'' of building ties with North Korea through aid and exchanges, an
approach favored by China. That would complicate Washington's efforts to
isolate the North at a time other Asian nations like the Philippines are
gravitating toward Beijing.

Image

Supporters of Ms. Park scuffling with the police. Some officers were
attacked with flagpoles, water bottles and pieces of the sidewalk
pavement.Credit...Jung Yeon-Je/Agence France-Presse --- Getty Images

\href{https://www.nytimes.com/2016/12/09/world/asia/south-korea-who-could-replace-park.html}{Moon
Jae-in}, the Democratic Party leader who is leading in opinion surveys,
has said that a decade of applying sanctions on North Korea had failed
to stop its nuclear weapons programs. He has said that sanctions are
necessary, but that ``their goal should be to draw North Korea back to
the negotiating table.''

He believes that Ms. Park's decision to allow the
\href{https://www.nytimes.com/2017/03/06/world/asia/north-korea-thaad-missile-defense-us-china.html}{deployment
of the American missile defense system} --- known as Terminal High
Altitude Area Defense, or Thaad --- has dragged the country into the
dangerous and growing rivalry between Washington and Beijing; China has
called the system a threat to its security and taken steps to punish
South Korea economically for accepting it.

Conservative South Koreans see the deployment of the antimissile system
not only as a guard against the North but also as a symbolic
reaffirmation of the all-important alliance with the United States. Mr.
Moon's party demands that the deployment, which began this week, be
suspended immediately. If it takes power, it says it will review the
deployment of the antimissile system to determine if it is in South
Korea's best interest.

As South Korea has learned, it cannot always keep Washington and Beijing
happy at the same time, as in the case of the country's decision to
accept the American missile defenses.

Yet Ms. Park's impeachment was also a pushback against ``Cold War
conservatives'' like her father, who seized on Communist threats from
North Korea to hide their corruption and silence political opponents,
said Kim Dong-choon, a sociologist at Sungkonghoe University in Seoul.

Ms. Park's father tortured and executed dissidents, framing them with
spying charges. Now, his daughter faces charges that her government
\href{https://www.nytimes.com/2017/01/12/world/asia/south\%2Dkorea\%2Dpresident\%2Dpark\%2Dblacklist\%2Dartists.html}{blacklisted
thousands of unfriendly artists and writers}.

``Her removal means that the curtain is finally drawing on the
authoritarian political and economic order that has dominated South
Korea for decades,'' said Ahn Byong-jin, rector of the Global Academy
for Future Civilizations at Kyung Hee University in Seoul.

Analysts cautioned that political and economic change will come slowly.

As Mr. Moon put it recently: ``We need a national cleanup. We need to
liquidate the old system and build a new South Korea. Only then can we
complete the revolution started by the people who rallied with
candlelight.''

Advertisement

\protect\hyperlink{after-bottom}{Continue reading the main story}

\hypertarget{site-index}{%
\subsection{Site Index}\label{site-index}}

\hypertarget{site-information-navigation}{%
\subsection{Site Information
Navigation}\label{site-information-navigation}}

\begin{itemize}
\tightlist
\item
  \href{https://help.nytimes.com/hc/en-us/articles/115014792127-Copyright-notice}{©~2020~The
  New York Times Company}
\end{itemize}

\begin{itemize}
\tightlist
\item
  \href{https://www.nytco.com/}{NYTCo}
\item
  \href{https://help.nytimes.com/hc/en-us/articles/115015385887-Contact-Us}{Contact
  Us}
\item
  \href{https://www.nytco.com/careers/}{Work with us}
\item
  \href{https://nytmediakit.com/}{Advertise}
\item
  \href{http://www.tbrandstudio.com/}{T Brand Studio}
\item
  \href{https://www.nytimes.com/privacy/cookie-policy\#how-do-i-manage-trackers}{Your
  Ad Choices}
\item
  \href{https://www.nytimes.com/privacy}{Privacy}
\item
  \href{https://help.nytimes.com/hc/en-us/articles/115014893428-Terms-of-service}{Terms
  of Service}
\item
  \href{https://help.nytimes.com/hc/en-us/articles/115014893968-Terms-of-sale}{Terms
  of Sale}
\item
  \href{https://spiderbites.nytimes.com}{Site Map}
\item
  \href{https://help.nytimes.com/hc/en-us}{Help}
\item
  \href{https://www.nytimes.com/subscription?campaignId=37WXW}{Subscriptions}
\end{itemize}
