Sections

SEARCH

\protect\hyperlink{site-content}{Skip to
content}\protect\hyperlink{site-index}{Skip to site index}

\href{https://www.nytimes.com/section/world/asia}{Asia Pacific}

\href{https://myaccount.nytimes.com/auth/login?response_type=cookie\&client_id=vi}{}

\href{https://www.nytimes.com/section/todayspaper}{Today's Paper}

\href{/section/world/asia}{Asia Pacific}\textbar{}Park Geun-hye, Ousted
South Korean Leader, Leaves Presidential Palace

\url{https://nyti.ms/2myTq4B}

\begin{itemize}
\item
\item
\item
\item
\item
\end{itemize}

Advertisement

\protect\hyperlink{after-top}{Continue reading the main story}

Supported by

\protect\hyperlink{after-sponsor}{Continue reading the main story}

\hypertarget{park-geun-hye-ousted-south-korean-leader-leaves-presidential-palace}{%
\section{Park Geun-hye, Ousted South Korean Leader, Leaves Presidential
Palace}\label{park-geun-hye-ousted-south-korean-leader-leaves-presidential-palace}}

\includegraphics{https://static01.nyt.com/images/2017/03/12/world/asia/13KOREA2/13KOREA2-articleInline.jpg?quality=75\&auto=webp\&disable=upscale}

By \href{http://www.nytimes.com/by/choe-sang-hun}{Choe Sang-Hun}

\begin{itemize}
\item
  March 12, 2017
\item
  \begin{itemize}
  \item
  \item
  \item
  \item
  \item
  \end{itemize}
\end{itemize}

SEOUL, South Korea --- For months, hundreds of thousands of
demonstrators have gathered almost weekly near the presidential Blue
House in Seoul, calling for the departure of Park Geun-hye as South
Korea's leader.

On Sunday, two days after the Constitutional Court
\href{https://www.nytimes.com/2017/03/09/world/asia/park-geun-hye-impeached-south-korea.html}{removed
her from office} on charges of corruption and abuse of power, they got
their wish, as Ms. Park left quietly in a motorcade that whisked her to
her two-story red brick house in the southern part of the capital.

Ms. Park became the first South Korean leader to be forced out of office
in response to popular pressure since the country's founding president,
Syngman Rhee, fled into exile in Hawaii in 1960 after protests against
his corrupt, authoritarian rule.

``I am sorry that I could not finish the presidential duty that was
entrusted to me,'' Ms. Park said in a brief statement read by one of her
former aides to reporters outside her home. ``I will bear with me all
the consequences.''

Ms. Park, who has been pressured by the opposition to publicly accept
the court's ruling and whose own party said it ``humbly respected'' the
decision, hinted that she disagreed with it. ``It will take time,'' she
said, ``but I am sure that the truth will be known.''

As the motorcade carrying Ms. Park arrived at the house where she lived
from 1990 to 2013, it pulled past hundreds of supporters lining the
alley and waving national flags.

Ms. Park, who has now lost the privilege of immunity that came with the
presidency, stepped out of the car, smiled and shook hands with former
aides and party lawmakers who waited for her in front of her house.

Supporters said they could not accept the Constitutional Court ruling,
and held up a variety of signs to express that sentiment: ``You are our
president forever!'' ``We love you,'' and ``Park Geun-hye, the president
of the people, welcome back!''

After the court announced its decision on Friday, the flag showing two
phoenixes, the presidential symbol of South Korea, was lowered from a
Blue House flagpole, but despite the huge significance of her removal
from office, Ms. Park could not immediately move out for a prosaic
reason.

Ms. Park's private home in southern Seoul, which has been unoccupied for
the past four years, needed repair. In the past couple of days, workers
have been busy fixing its broken boiler, installing new furniture and
redecorating rooms.

After the ruling was announced on Friday, thousands of Park supporters,
mostly older conservatives,
\href{https://www.nytimes.com/2017/03/10/world/asia/south-korea-president-impeached-protests.html}{tried
to march on the courthouse} and called for its destruction, with some
clashing with police officers who blocked them with a barricade of
buses.

Three men, in their 60s and 70s, died during the clashes. One of the men
died after a steel police speaker fell on his head, and a protester has
been arrested on charges of stealing a police bus and ramming it into
another, causing the speaker to fall.

On Saturday,
\href{https://www.nytimes.com/2017/02/18/world/asia/south-korea-impeached-leader-park-geun-hye.html}{Park
supporters} rallied in central Seoul, vowing to start a political party
to fight ``pro-North Korea'' leftists who they said conspired to bring
down Ms. Park and calling the Constitutional Court ruling ``sedition.''
No violence was reported.

Later, as dusk fell on Saturday, hundreds of thousands of South Koreans
gathered in central Seoul to celebrate Ms. Park's ouster, dancing to the
Queen song ``We Are the Champions'' and releasing firecrackers. They
regard her removal as a key step toward ending what they see as corrupt
ties between government and business that have hindered the country for
decades.

Ms. Park's opponents
\href{https://www.nytimes.com/2016/11/26/world/asia/korea-park-geun-hye-protests.html}{organized
huge candlelight rallies} week after week, for months, forcing
prosecutors to investigate allegations that Ms. Park conspired with her
secretive confidante,
\href{https://www.nytimes.com/2016/11/01/world/asia/south-korea-park-geun-hye-choi-soon-sil.html}{Choi
Soon-sil}, to extort millions of dollars from big businesses. Many
protesters held signs on Saturday that said, ``Now, the next step is to
arrest Park Geun-hye!''

A reinvigorated news media also helped precipitate Ms. Park's downfall
by exposing incriminating details, like a tablet computer belonging to
Ms. Choi that proved her influence in state affairs.

As Ms. Park's approval ratings plummeted, the usually sympathetic
conservative news media also turned against her, leaving her with few
allies beyond right-wing bloggers and some old conservatives who
believed that Ms. Park had been framed and that her downfall would bring
about a pro-North Korean leftist government.

The National Assembly voted to
\href{https://www.nytimes.com/2016/12/09/world/asia/south-korea-president-park-geun-hye-impeached.html}{impeach
Ms. Park on Dec. 9}, asking the Constitutional Court to formally unseat
her. Because she was ousted through impeachment, she lost the privileges
the government provides to a former president --- including a \$10,500
monthly pension payment, an office, a small staff of aides and free
medical service --- but she will receive police protection.

Now an ordinary citizen, Ms. Park is likely to be the subject of a
criminal investigation into whether she engaged in corruption.
Prosecutors have said she conspired with Ms. Choi to collect tens of
millions of dollars from big businesses, like Samsung, and that some of
the money represented bribes for political favors.

When prosecutors indicted Ms. Choi and
\href{https://www.nytimes.com/2017/02/28/world/asia/lee-jae-yong-samsung.html}{Lee
Jae-yong}, the de facto head of Samsung, on bribery and other charges,
the prosecutors formally identified Ms. Park as a criminal accomplice.
But they could not bring charges against her because she was protected
from indictment while in office.

Moon Jae-in, an opposition leader who leads the race to replace Ms.
Park, criticized her on Sunday for failing to announce in public that
she accepts the court ruling.

Speaking at a news conference, he also said prosecutors should open
their corruption investigation into Ms. Park immediately, and warned
that she should not remove any potential evidence while moving out of
the Blue House.

Ms. Park has blocked prosecutors from searching her office. When the
Constitutional Court ruled against her, it criticized Ms. Park for
failing to cooperate with investigators, for trying to hide her
wrongdoings, and for impeding the National Assembly's and the media's
right to know.

When Ms. Park was elected as president in late 2012, it marked a
triumphant return to the Blue House, where she lived from 1961, when her
father, Maj. Gen. Park Chung-hee, seized power in a military coup, until
1979, when he was assassinated.

Now, Ms. Park has left it again --- almost certainly for the final time
--- disgraced, deeply unpopular and as a criminal suspect.

Advertisement

\protect\hyperlink{after-bottom}{Continue reading the main story}

\hypertarget{site-index}{%
\subsection{Site Index}\label{site-index}}

\hypertarget{site-information-navigation}{%
\subsection{Site Information
Navigation}\label{site-information-navigation}}

\begin{itemize}
\tightlist
\item
  \href{https://help.nytimes.com/hc/en-us/articles/115014792127-Copyright-notice}{©~2020~The
  New York Times Company}
\end{itemize}

\begin{itemize}
\tightlist
\item
  \href{https://www.nytco.com/}{NYTCo}
\item
  \href{https://help.nytimes.com/hc/en-us/articles/115015385887-Contact-Us}{Contact
  Us}
\item
  \href{https://www.nytco.com/careers/}{Work with us}
\item
  \href{https://nytmediakit.com/}{Advertise}
\item
  \href{http://www.tbrandstudio.com/}{T Brand Studio}
\item
  \href{https://www.nytimes.com/privacy/cookie-policy\#how-do-i-manage-trackers}{Your
  Ad Choices}
\item
  \href{https://www.nytimes.com/privacy}{Privacy}
\item
  \href{https://help.nytimes.com/hc/en-us/articles/115014893428-Terms-of-service}{Terms
  of Service}
\item
  \href{https://help.nytimes.com/hc/en-us/articles/115014893968-Terms-of-sale}{Terms
  of Sale}
\item
  \href{https://spiderbites.nytimes.com}{Site Map}
\item
  \href{https://help.nytimes.com/hc/en-us}{Help}
\item
  \href{https://www.nytimes.com/subscription?campaignId=37WXW}{Subscriptions}
\end{itemize}
