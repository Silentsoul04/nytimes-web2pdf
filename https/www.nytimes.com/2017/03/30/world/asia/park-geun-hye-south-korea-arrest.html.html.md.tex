Sections

SEARCH

\protect\hyperlink{site-content}{Skip to
content}\protect\hyperlink{site-index}{Skip to site index}

\href{https://www.nytimes.com/section/world/asia}{Asia Pacific}

\href{https://myaccount.nytimes.com/auth/login?response_type=cookie\&client_id=vi}{}

\href{https://www.nytimes.com/section/todayspaper}{Today's Paper}

\href{/section/world/asia}{Asia Pacific}\textbar{}Park Geun-hye, South
Korea's Ousted Leader, Is Arrested and Jailed to Await Trial

\url{https://nyti.ms/2nCbtZO}

\begin{itemize}
\item
\item
\item
\item
\item
\end{itemize}

Advertisement

\protect\hyperlink{after-top}{Continue reading the main story}

Supported by

\protect\hyperlink{after-sponsor}{Continue reading the main story}

\hypertarget{park-geun-hye-south-koreas-ousted-leader-is-arrested-and-jailed-to-await-trial}{%
\section{Park Geun-hye, South Korea's Ousted Leader, Is Arrested and
Jailed to Await
Trial}\label{park-geun-hye-south-koreas-ousted-leader-is-arrested-and-jailed-to-await-trial}}

\includegraphics{https://static01.nyt.com/images/2017/03/31/world/31skorea-sub/31skorea-sub-articleInline.jpg?quality=75\&auto=webp\&disable=upscale}

By \href{http://www.nytimes.com/by/choe-sang-hun}{Choe Sang-Hun}

\begin{itemize}
\item
  March 30, 2017
\item
  \begin{itemize}
  \item
  \item
  \item
  \item
  \item
  \end{itemize}
\end{itemize}

SEOUL, South Korea --- South Korea's recently impeached president, Park
Geun-hye, was arrested on Friday, becoming the first South Korean leader
to be put behind bars since the mid 1990s, when two former military
\href{http://www.nytimes.com/1995/12/22/world/south-korea-indicts-2-former-presidents-in-staging-of-1979-coup.html}{dictators
were imprisoned} on corruption and mutiny charges.

Ms. Park's dramatic downfall capped months of turmoil and intrigue, as
huge crowds took to the streets
\href{https://www.nytimes.com/2016/11/26/world/asia/korea-park-geun-hye-protests.html}{to
protest a sprawling corruption scandal} that shook the interlocking
worlds of government and business --- including
\href{https://www.nytimes.com/2017/03/09/business/jay-y-lee-samsung-trial.html}{the
leadership of Samsung}, the nation's largest conglomerate.

A judge at the Seoul Central District Court issued the warrant early
Friday morning, warning that if Ms. Park were not taken into custody
quickly she might ``destroy evidence.'' The charges against her include
bribery, extortion and abuse of power.

In December, the National Assembly voted overwhelmingly to impeach Ms.
Park, and she was formally removed from office on March 10.

Her removal rattled the delicate balance of relationships across Asia at
a tense moment. Ms. Park's conservatives, in power for four years, had
joined the United States in pressing for a hard line against North
Korea's nuclear program. She had accepted Washington's deployment of an
advanced missile defense system that has angered China, which is
\href{https://www.nytimes.com/2017/03/07/world/asia/thaad-missile-defense-us-south-korea-china.html}{fearful
of an arms race} in the region.

Moon Jae-in, the liberal opposition leader considered most likely to win
the May 9 election to select a new president, has vowed to review that
decision, as well as an unpopular deal she struck with Japan over the
so-called comfort women, or Korean sex slaves, used by Japan's army
during World War II.

Mr. Moon is also viewed as less confrontational toward North Korea and
China, and has advocated dialogue to halt the North's nuclear and
missile threats.

Ms. Park had spent the night in the prosecutor's office waiting to learn
if she would be placed under arrest. Shortly after 3 a.m., Judge Kang
Bu-young issued the warrant and Ms. Park was taken to a jail outside
Seoul, the South Korean capital.

Prosecutors had already said that they would indict the former president
on 13 criminal charges regardless of whether the judge issued an arrest
warrant. Once her trial begins in the coming weeks, Ms. Park will
commute from her cell to a Seoul courthouse.

Ms. Park was accused of conspiring with a longtime
confidante,\href{https://www.nytimes.com/2016/11/06/world/asia/south-koreans-ashamed-over-les-secretive-adviser.html}{Choi
Soon-sil}, to collect tens of millions of dollars from big businesses,
including more than \$38 million in bribes from Samsung. Both Ms. Choi
and Samsung's top executive,
\href{https://www.nytimes.com/2017/02/17/business/samsung-heir-arrested-south-korea.html}{Lee
Jae-yong}, have previously been arrested and are standing trial on
charges including bribery.

On Thursday, hundreds of emotional supporters were gathered near her
two-story, red brick house in southern Seoul as her motorcade pulled out
for the hearing. Some tried to push past police officers lining the
road.

``We can't let you go!'' they shouted, some of them trying to hurl
themselves onto the road. Protesters also threw yellow plastic police
fences at uniformed officers who tried to push them back.

The 10-minute ride was nationally televised, with a fleet of television
crews following her car.

If she is convicted of bribery, Ms. Park, 65, could face between 10
years and life in prison, although her successor has the power to free
her with a special presidential pardon.

Ms. Park's arrest added another tragic chapter to the story of her
family, which has mirrored the country's tumultuous modern history.

Her father, Park Chung-hee, who ruled South Korea from 1961 to 1979, was
considered the initiator of the country's dramatic economic growth but
was also vilified as a dictator who used martial law and torture to
silence dissidents.

\includegraphics{https://static01.nyt.com/images/2017/03/31/world/31skorea-2/31skorea-2-articleLarge.jpg?quality=75\&auto=webp\&disable=upscale}

In 1974, her mother, Yuk Young-soo, was fatally shot by a pro-North
Korean assassin who had targeted her husband. Ms. Park's father was
assassinated five years later by his spy chief.

After leaving the presidential Blue House, Ms. Park lived in seclusion,
unmarried and without children. In the succeeding decades, while the
country moved toward democracy, her family drew little attention.

Then in the late 1990s, as the economy faltered in the financial spasms
that engulfed Asia, and South Koreans began yearning for the kind of
charismatic and stern leadership her father represented, Ms. Park was
catapulted into political stardom.

She seized the opportunity and won a parliamentary seat in 1998, casting
herself as a faithful daughter on a mission to rebuild the nation and
restore the honor of her father and his generation, whom she credited
with fighting Communism and creating a vibrant, world-class economy from
the ashes of the 1950-53 Korean War.

In 2013, she returned to the Blue House, her childhood home, after
winning the presidential election. She was the first child of a former
president to win the presidency.

Older South Koreans were among her most fervent supporters, while
critics warned that she was trying to take the country back to its
authoritarian past.

Four years later, her career ended in disgrace. In a speech last
November, as the growing scandal paralyzed her government, she tearfully
confided that she regretted ever becoming president.

While a majority of South Koreans believe her ouster was merited, Ms.
Park's fall from grace has been hard to accept for many older,
conservative South Koreans who still worshiped her father and family as
Koreans once did their ancient kings.

``Your highness, I am so sorry that I am powerless to protect you,'' an
old woman recently
\href{http://www.wikitree.co.kr/main/news_view.php?id=295753}{wailed on
the pavement} in front of Ms. Park's home. ``Please forgive me, your
highness.''

Since she
\href{https://www.nytimes.com/2017/03/12/world/asia/park-geun-hye-blue-house.html}{returned
to her home} on March 12, hundreds of supporters holding South Korean
flags have been rallying daily outside the house to protest her
impeachment and oppose her arrest.

Ms. Park, who has never accepted the Constitutional Court ruling that
\href{https://www.nytimes.com/2017/03/09/world/asia/park-geun-hye-impeached-south-korea.html}{ended
her presidency}, did not answer questions shouted from reporters as she
walked into the courthouse on Thursday. But she had earlier denied any
legal wrongdoing.

Last fall, local news media and former associates of Ms. Choi's began
exposing lurid details of Ms. Park's alleged conspiracy to abuse her
presidential authority to help Ms. Choi, one of her few friends, to
collect bribes or extort large ``donations'' from big businesses.

With the economy slowing and youth unemployment rising, the allegations
convinced people that Ms. Park had inherited the worst traits of her
father: an authoritarian streak and corrupt ties with the business
elite. They took to the streets in central Seoul, up to two million at a
time, marching peacefully to demand her resignation. When she refused to
step down, the National Assembly
\href{https://www.nytimes.com/2016/12/09/world/asia/south-korea-president-park-geun-hye-impeached.html}{voted
overwhelmingly to impeach her} on Dec. 9, and she was formally removed
from office on March 10.

Ms. Park was the first South Korean leader to be forced from office in
response to popular pressure since the country's founding president,
Syngman Rhee, fled into exile in Hawaii in 1960 after protests against
his corrupt, authoritarian rule.

Two former presidents --- the military dictators Chun Doo-hwan and Roh
Tae-woo --- were arrested in 1995 on charges of each collecting hundreds
of millions of dollars in bribes. The two men, former army generals,
also faced
\href{http://www.nytimes.com/1995/12/22/world/south-korea-indicts-2-former-presidents-in-staging-of-1979-coup.html}{sedition
and mutiny charges} for their roles in the 1979 military coup that
brought them to power and in the 1980 massacre of antigovernment
demonstrators in the southwestern city of Gwangju.

Mr. Chun was sentenced to life in prison while Mr. Roh was sentenced to
17 years. Both were pardoned and released in December 1997.

Advertisement

\protect\hyperlink{after-bottom}{Continue reading the main story}

\hypertarget{site-index}{%
\subsection{Site Index}\label{site-index}}

\hypertarget{site-information-navigation}{%
\subsection{Site Information
Navigation}\label{site-information-navigation}}

\begin{itemize}
\tightlist
\item
  \href{https://help.nytimes.com/hc/en-us/articles/115014792127-Copyright-notice}{©~2020~The
  New York Times Company}
\end{itemize}

\begin{itemize}
\tightlist
\item
  \href{https://www.nytco.com/}{NYTCo}
\item
  \href{https://help.nytimes.com/hc/en-us/articles/115015385887-Contact-Us}{Contact
  Us}
\item
  \href{https://www.nytco.com/careers/}{Work with us}
\item
  \href{https://nytmediakit.com/}{Advertise}
\item
  \href{http://www.tbrandstudio.com/}{T Brand Studio}
\item
  \href{https://www.nytimes.com/privacy/cookie-policy\#how-do-i-manage-trackers}{Your
  Ad Choices}
\item
  \href{https://www.nytimes.com/privacy}{Privacy}
\item
  \href{https://help.nytimes.com/hc/en-us/articles/115014893428-Terms-of-service}{Terms
  of Service}
\item
  \href{https://help.nytimes.com/hc/en-us/articles/115014893968-Terms-of-sale}{Terms
  of Sale}
\item
  \href{https://spiderbites.nytimes.com}{Site Map}
\item
  \href{https://help.nytimes.com/hc/en-us}{Help}
\item
  \href{https://www.nytimes.com/subscription?campaignId=37WXW}{Subscriptions}
\end{itemize}
