Sections

SEARCH

\protect\hyperlink{site-content}{Skip to
content}\protect\hyperlink{site-index}{Skip to site index}

\href{https://www.nytimes.com/section/business}{Business}

\href{https://myaccount.nytimes.com/auth/login?response_type=cookie\&client_id=vi}{}

\href{https://www.nytimes.com/section/todayspaper}{Today's Paper}

\href{/section/business}{Business}\textbar{}Huawei, Chinese Technology
Giant, Is Focus of Widening U.S. Investigation

\url{https://nyti.ms/2pkSOTg}

\begin{itemize}
\item
\item
\item
\item
\item
\end{itemize}

Advertisement

\protect\hyperlink{after-top}{Continue reading the main story}

Supported by

\protect\hyperlink{after-sponsor}{Continue reading the main story}

\hypertarget{huawei-chinese-technology-giant-is-focus-of-widening-us-investigation}{%
\section{Huawei, Chinese Technology Giant, Is Focus of Widening U.S.
Investigation}\label{huawei-chinese-technology-giant-is-focus-of-widening-us-investigation}}

\includegraphics{https://static01.nyt.com/images/2017/04/27/world/28Huaweiprobe-1/28Huaweiprobe-1-articleLarge.jpg?quality=75\&auto=webp\&disable=upscale}

By \href{https://www.nytimes.com/by/paul-mozur}{Paul Mozur}

\begin{itemize}
\item
  April 26, 2017
\item
  \begin{itemize}
  \item
  \item
  \item
  \item
  \item
  \end{itemize}
\end{itemize}

\href{https://cn.nytimes.com/business/20170427/huawei-investigation-sanctions-subpoena/}{阅读简体中文版}

HONG KONG --- As one of the world's biggest sellers of smartphones and
the back-end equipment that makes cellular networks run, Huawei
Technologies has become one of the major symbols of China's global
technology ambitions.

But as it continues its rise, its business with some countries has
fallen under growing scrutiny from investigators in the United States.

American officials are widening their investigation into whether Huawei
broke American trade controls on Cuba, Iran, Sudan and Syria, according
to an administrative subpoena sent to Huawei and reviewed by The New
York Times. The previously unreported subpoena was issued in December by
the United States Treasury Department's Office of Foreign Assets
Control, which oversees compliance with a number of American sanctions
programs.

The Treasury's inquiry follows a subpoena
\href{https://www.nytimes.com/2016/06/03/technology/huawei-technologies-subpoena-iran-north-korea.html}{sent
to Huawei last summer} from the United States Department of Commerce,
which carries out sanctions and also oversees exports of technology that
can have military as well as civilian uses.

Huawei has not been accused of wrongdoing. As an administrative
subpoena, the Treasury document does not indicate that the Chinese
company is part of a criminal investigation.

Still, the widening inquiry puts Huawei in an awkward position at a
moment when sanctions have taken on new import. The Trump administration
has been working to push China to cut back its trade, and in turn
economic support, for North Korea, amid rising tensions over the North's
nuclear and missile programs. The growing investigation also comes after
Huawei's smaller domestic rival, ZTE, in March pleaded guilty to
breaking sanctions and
\href{https://www.nytimes.com/2017/03/07/technology/zte-china-fine.html}{was
fined \$1.19 billion}.

It is not clear why the Treasury Department became involved with the
Huawei investigation. But its subpoena suggests Huawei might also be
suspected of violating American embargoes that broadly restrict the
export of American goods to countries like Iran and Syria.

``The most likely thing happening here is that Commerce figured out
there was more to this than dual-use commodities, and they decided to
notify Treasury,'' said Matthew Brazil, a former United States
commercial officer in Beijing and founder of the Silicon Valley security
firm Madeira Consulting.

Huawei said in a statement that it ``has adhered to international
conventions and all applicable laws and regulations where it operates.''
The company would not comment on the specifics of the investigation but
said it had a ``robust trade compliance program.''

Still, by its own admission, the company has at times struggled with
corporate governance. In a rare 2015 media appearance, Ren Zhengfei,
Huawei's founder, said that 4,000 to 5,000 employees had admitted to
various improprieties as part of a ``confess for leniency'' program the
company set up in 2014.

``The biggest enemy we've run into isn't other people,'' he said
\href{https://bits.blogs.nytimes.com/2015/01/22/thousands-of-huawei-workers-respond-to-internal-anti-fraud-campaign/?_r=0}{at
the time}. ``It's ourselves.''

A Treasury spokeswoman declined to comment on whether it was conducting
an investigation. A Commerce Department spokesman also declined to
comment.

Huawei plays an important strategic role for China. The company is often
a part of Chinese overseas trade delegations and investment deals in
emerging markets like South America and Africa. As a major spender on
research and development, it is also a crucial part of Chinese
industrial policies aimed at building up domestic technological
capabilities.

It has also turned itself into an increasingly recognized smartphone
brand. In the fourth quarter of 2016, Huawei was the third-largest
smartphone maker in the world, with a global market share of about 10
percent.

The subpoena, which was sent to Huawei's Texas offices in the Dallas
suburb of Plano, called for the company to describe technology and
services provided to Cuba, Iran, Sudan and Syria over the past five
years. It also called for the identity of individuals who played a part
in those transactions. North Korea, which was named in the Commerce
Department subpoena issued last year, was not named in the Treasury
Department subpoena.

The scrutiny of Huawei shows the increased importance both the United
States and China are putting on the technology industry. Earlier this
year a Pentagon report distributed at the top levels of the Trump
administration indicated Chinese flows of investment into American
start-ups were a
\href{https://www.nytimes.com/2017/03/22/technology/china-defense-start-ups.html}{new
cause for concern}.

The American authorities have jurisdiction over the trade of companies
like Huawei and ZTE when those companies sell equipment made by or
featuring components from American companies. If Huawei is deemed to
have violated American laws, it could have its access to American
electronic components cut off. Given the company's size --- it is one of
the two largest cellular phone equipment makers in the world --- that
could have an effect on the expansion of mobile networks around the
globe.

When the Department of Commerce first announced its investigation into
ZTE, it released a document in which ZTE executives mapped out a plan
for how to get around American export controls. The document said the
strategy came from a company that ZTE labeled with the code name F7,
which The New York Times reported closely resembled Huawei.

Earlier this month 10 members of Congress sent a letter to the Commerce
Department demanding that F7 be publicly identified and fully
investigated.

``We strongly support holding F7 accountable should the government
conclude that unlawful behavior occurred,'' read a part of the letter.

Advertisement

\protect\hyperlink{after-bottom}{Continue reading the main story}

\hypertarget{site-index}{%
\subsection{Site Index}\label{site-index}}

\hypertarget{site-information-navigation}{%
\subsection{Site Information
Navigation}\label{site-information-navigation}}

\begin{itemize}
\tightlist
\item
  \href{https://help.nytimes.com/hc/en-us/articles/115014792127-Copyright-notice}{©~2020~The
  New York Times Company}
\end{itemize}

\begin{itemize}
\tightlist
\item
  \href{https://www.nytco.com/}{NYTCo}
\item
  \href{https://help.nytimes.com/hc/en-us/articles/115015385887-Contact-Us}{Contact
  Us}
\item
  \href{https://www.nytco.com/careers/}{Work with us}
\item
  \href{https://nytmediakit.com/}{Advertise}
\item
  \href{http://www.tbrandstudio.com/}{T Brand Studio}
\item
  \href{https://www.nytimes.com/privacy/cookie-policy\#how-do-i-manage-trackers}{Your
  Ad Choices}
\item
  \href{https://www.nytimes.com/privacy}{Privacy}
\item
  \href{https://help.nytimes.com/hc/en-us/articles/115014893428-Terms-of-service}{Terms
  of Service}
\item
  \href{https://help.nytimes.com/hc/en-us/articles/115014893968-Terms-of-sale}{Terms
  of Sale}
\item
  \href{https://spiderbites.nytimes.com}{Site Map}
\item
  \href{https://help.nytimes.com/hc/en-us}{Help}
\item
  \href{https://www.nytimes.com/subscription?campaignId=37WXW}{Subscriptions}
\end{itemize}
