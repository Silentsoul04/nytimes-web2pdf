Sections

SEARCH

\protect\hyperlink{site-content}{Skip to
content}\protect\hyperlink{site-index}{Skip to site index}

\href{https://www.nytimes.com/section/politics}{Politics}

\href{https://myaccount.nytimes.com/auth/login?response_type=cookie\&client_id=vi}{}

\href{https://www.nytimes.com/section/todayspaper}{Today's Paper}

\href{/section/politics}{Politics}\textbar{}F.B.I. Director James Comey
Is Fired by Trump

\url{https://nyti.ms/2psUZkc}

\begin{itemize}
\item
\item
\item
\item
\item
\item
\end{itemize}

Advertisement

\protect\hyperlink{after-top}{Continue reading the main story}

Supported by

\protect\hyperlink{after-sponsor}{Continue reading the main story}

\hypertarget{fbi-director-james-comey-is-fired-by-trump}{%
\section{F.B.I. Director James Comey Is Fired by
Trump}\label{fbi-director-james-comey-is-fired-by-trump}}

\includegraphics{https://static01.nyt.com/images/2017/05/10/us/10fbi/10fbi-videoSixteenByNineJumbo1600-v2.jpg}

By \href{http://www.nytimes.com/by/michael-d-shear}{Michael D. Shear}
and \href{http://www.nytimes.com/by/matt-apuzzo}{Matt Apuzzo}

\begin{itemize}
\item
  May 9, 2017
\item
  \begin{itemize}
  \item
  \item
  \item
  \item
  \item
  \item
  \end{itemize}
\end{itemize}

WASHINGTON --- President Trump on Tuesday fired the director of the
F.B.I., James B. Comey, abruptly terminating the top official leading a
criminal investigation into whether Mr. Trump's advisers colluded with
the Russian government to steer the outcome of the 2016 presidential
election.

The stunning development in Mr. Trump's presidency raised the specter of
political interference by a sitting president into an existing
investigation by the nation's leading law enforcement agency. It
immediately ignited Democratic calls for a special counsel to lead the
Russia inquiry.

Mr. Trump explained the firing by citing Mr. Comey's handling of the
investigation into Hillary Clinton's use of a private email server, even
though the president was widely seen to have benefited politically from
that inquiry and had once praised Mr. Comey for his ``guts'' in his
pursuit of Mrs. Clinton during the campaign.

But in his letter to Mr. Comey, released to reporters by the White
House, the president betrayed his focus on the continuing inquiry into
Russia and his aides.

``While I greatly appreciate you informing me, on three separate
occasions, that I am not under investigation, I nevertheless concur with
the judgment of the Department of Justice that you are not able to
effectively lead the bureau,'' Mr. Trump said in a letter to Mr. Comey
dated Tuesday. White House officials refused to say anything more about
the three occasions Mr. Trump cited.

\href{https://www.nytimes.com/interactive/2017/05/09/us/politics/document-White-House-Fires-James-Comey.html}{}

\includegraphics{https://static01.nyt.com/images/2017/05/09/us/politics/image-White-House-Fires-James-Comey/image-White-House-Fires-James-Comey-thumbLarge-v3.png}

\hypertarget{white-house-announces-firing-of-james-comey}{%
\subsection{White House Announces Firing of James
Comey}\label{white-house-announces-firing-of-james-comey}}

Read the letter President Trump sent the F.B.I. director about his
firing.

The officials said that Attorney General Jeff Sessions and the deputy
attorney general, Rod J. Rosenstein, pushed for Mr. Comey's dismissal.
But many in Washington, including veteran F.B.I. officers, saw a
carefully choreographed effort by the president to create a pretense for
a takedown of the president's F.B.I. tormentor.

``I cannot defend the director's handling of the conclusion of the
investigation of Secretary Clinton's emails,'' Mr. Rosenstein wrote in
another letter that was released by the White House, ``and I do not
understand his refusal to accept the nearly universal judgment that he
was mistaken.''

Reaction in Washington was swift and fierce. Senator Chuck Schumer of
New York, the Democratic leader, said the firing could make Americans
suspect a cover-up. Mr. Trump lashed back later Tuesday night in a
Twitter
\href{https://twitter.com/realDonaldTrump/status/862135824745467905}{post}:
``Cryin' Chuck Schumer stated recently, `I do not have confidence in him
(James Comey) any longer.' Then acts so indignant.''

Many Republicans assailed the president for making a rash decision that
could have deep implications for their party. Representative Justin
Amash, Republican of Michigan,
\href{https://twitter.com/justinamash/status/862089192603607041}{said on
Twitter} that he now supports an independent commission to investigate
the Russia links to Mr. Trump. He called Mr. Trump's claim that Mr.
Comey had cleared him three times ``bizarre.''

``I've spent the last several hours trying to find an acceptable
rationale for the timing of Comey's firing,'' Senator Jeff Flake,
Republican of Arizona,
\href{https://twitter.com/JeffFlake/status/862124755339685888}{said on
Twitter}. ``I just can't do it.''

In a sign of the F.B.I.'s intense interest in Mr. Trump's advisers, a
grand jury in Virginia issued subpoenas in recent weeks for records
related to the former White House national security adviser, Michael T.
Flynn, according to an American official familiar with the case. Mr.
Flynn is under investigation for his financial ties to Russia and
Turkey. Grand jury subpoenas are a routine part of federal
investigations and are not a sign that charges are imminent. It was not
clear that the subpoenas, which were first reported by CNN, were related
to Mr. Comey's firing.

The dismissal ended the long-deteriorating relationship of Mr. Trump and
Mr. Comey, who repeatedly collided publicly and privately. For Mr.
Trump, a president who puts a premium on loyalty, Mr. Comey represented
an independent and unpredictable director with enormous power to disrupt
his administration.

Mr. Comey learned from news reports that he had been fired while
addressing bureau employees in Los Angeles. While Mr. Comey spoke,
television screens in the background began flashing the news. In
response to the reports, Mr. Comey laughed, saying that he thought it
was a fairly funny prank. Shortly after, Mr. Trump's letter was
delivered to F.B.I. Headquarters in Washington.

Mr. Comey was four years into a 10-year term, an unusually long tenure
that Congress established to insulate the director from political
pressure. Though the president has the authority to fire the F.B.I.
director for any reason, Mr. Comey is only the second director to be
fired in bureau history. President Bill Clinton fired William S.
Sessions in 1993.

Mr. Trump had already
\href{https://www.nytimes.com/2017/01/30/us/politics/trump-immigration-ban-memo.html}{fired
his acting attorney general} for insubordination and his
\href{https://www.nytimes.com/2017/02/13/us/politics/donald-trump-national-security-adviser-michael-flynn.html}{national
security adviser} for lying to Vice President Mike Pence about contacts
with Russians. But firing Mr. Comey raises much deeper questions about
the independence of the F.B.I. and the future of its investigations
under Mr. Trump.

In an instance of bizarre timing and optics, the White House announced
late Tuesday night that Mr. Trump would meet on Wednesday in the Oval
Office with Sergey V. Lavrov, Russia's foreign minister.

F.B.I. agents were enraged by the firing and worried openly that Mr.
Trump would appoint a White House ally to lead them. Mr. Comey was
widely liked in the F.B.I., even by those who criticized his handling of
the Clinton investigation. Agents regarded him as a good manager and an
independent director.

\includegraphics{https://static01.nyt.com/images/2017/05/10/us/10fbi-2/10fbi-2-articleLarge.jpg?quality=75\&auto=webp\&disable=upscale}

``It is essential that we find new leadership for the F.B.I. that
restores public trust and confidence in its vital law enforcement
mission,'' Mr. Trump wrote, a remark that particularly upset agents who
saw it as an insult to them.

The White House has not said what precipitated the firing, a significant
question because the Justice Department's stated reasons were well known
even when Mr. Trump decided in January to keep Mr. Comey on the job. Mr.
Trump watched last week as Mr. Comey testified on Capitol Hill, offering
his first public explanation of his handling of the Clinton email case.
There, Mr. Comey said that he had no regrets about his decisions but
that
\href{https://www.nytimes.com/2017/05/03/us/politics/james-comey-fbi-senate-hearing.html}{he
felt ``mildly nauseous''}that his actions might have tipped the election
to Mr. Trump.

The Clinton controversy centers on a news conference Mr. Comey held last
July, when he broke with longstanding tradition and policies by
\href{https://www.nytimes.com/2016/07/06/us/politics/hillary-clinton-fbi-email-comey.html}{publicly
discussing the Clinton case} and chastising Mrs. Clinton's ``careless''
handling of classified information. Then, in the campaign's final days,
Mr. Comey announced that the F.B.I. was
\href{https://www.nytimes.com/2016/10/29/us/politics/fbi-hillary-clinton-email.html?_r=0}{reopening
the investigation}, a move that earned him widespread criticism. At the
time, though, Mr. Trump and his attorney general, Mr. Sessions, praised
Mr. Comey for actions that are now at the heart of the F.B.I. director's
firing.

Mr. Trump ``saw an opening'' to fire Mr. Comey after the testimony, a
White House official said. Reince Priebus, the White House chief of
staff, argued against it, delaying --- but not overruling --- the
decision. Mr. Trump received the documents from the Justice Department
on Tuesday. Aides also compiled a stack of news clips criticizing Mr.
Comey.

Mr. Comey's firing came hours after the F.B.I.
\href{https://www.nytimes.com/2017/05/09/us/politics/comey-clinton-emails-testimony.html}{corrected}
his testimony last week about how classified information ended up on the
laptop of the disgraced former congressman Anthony D. Weiner.

Mr. Comey had told the Senate Judiciary Committee that during the
F.B.I.'s investigation into Mrs. Clinton's use of a private email server
while secretary of state, officers uncovered evidence that her aide,
Huma Abedin, had ``forwarded hundreds and thousands of emails, some of
which contain classified information'' to Mr. Weiner, her husband.

But the F.B.I. told Congress that only a few of the emails had been
forwarded and that the vast majority were simply backed up to Mr.
Weiner's laptop.

Mr. Comey's deputy, Andrew G. McCabe, a career F.B.I. officer, became
acting director, the Justice Department said. The White House said the
search for a director will begin immediately.

The firing puts Democrats in a difficult position. Many had hoped that
Mrs. Clinton would fire Mr. Comey soon after taking office, and blamed
him as costing her the election. But under Mr. Trump, Mr. Comey
\href{https://www.nytimes.com/2016/11/19/us/politics/james-comey-fbi-sessions.html}{was
seen as an important check} on the new administration.

Kellyanne Conway, a senior White House adviser, said the firing was not
related to Russia. ``Today's actions had zero to do with that,'' she
said in a contentious interview on CNN. Democrats were unconvinced.

``Any attempt to stop or undermine this F.B.I. investigation would raise
grave constitutional issues,'' said Senator Richard J. Durbin, Democrat
of Illinois.

Mr. Trump's decision to fire Mr. Comey marks the president's latest law
enforcement purge. In late January, Mr. Trump
\href{https://www.nytimes.com/2017/01/30/us/politics/trump-immigration-ban-memo.html}{fired
Sally Q. Yates}, who had worked in the Obama administration but was
serving as acting attorney general. In March, he
\href{https://www.nytimes.com/2017/03/11/us/politics/preet-bharara-us-attorney.html}{fired}
Preet Bharara, the United States attorney in Manhattan, after assuring
the prosecutor months earlier that he could stay on. But the president's
firing of Mr. Comey was far more consequential, as Ms. Yates and Mr.
Bharara both were holdovers, and might only have served in the Trump
administration for a matter of days or weeks.

A longtime prosecutor who served as the deputy attorney general during
the George W. Bush administration, Mr. Comey came into office in 2013
with widespread bipartisan support. The defining moment in Mr. Comey's
career --- until Tuesday night --- came in 2004 during a hospital room
standoff with White House officials who wanted to pressure the Justice
Department to reauthorize a secret wiretapping program. Mr. Comey stood
his ground.

Mr. Trump has been furious with news stories about his campaign's ties
to Russia. The White House has been critical of the leaks at the heart
of those stories and tried unsuccessfully to enlist Mr. Comey in an
effort to rebut the stories.

In a
\href{https://twitter.com/realDonaldTrump/status/859601184285491201}{Twitter
message} last week, Mr. Trump accused Mr. Comey of being ``the best
thing that ever happened to Hillary Clinton,'' and said he gave her ``a
free pass for many bad deeds.'' After the president accused former
President Barack Obama of wiretapping his office, Mr. Comey publicly
declared those claims untrue.

Senator Ron Wyden, Democrat of Oregon and a member of the Senate
Intelligence Committee, said in a
\href{https://twitter.com/RonWyden/status/862064341855686656}{post on
Twitter} that Mr. Comey ``should be immediately called to testify in an
open hearing about the status of Russia/Trump investigation at the time
he was fired.''

Before announcing the firing, Mr. Trump called leaders on Capitol Hill
to tell them of his decision. Mr. Schumer told Mr. Trump that he was
making a ``big mistake.'' Mr. Trump paused.

``O.K.,'' he said. ``We'll see.''

Advertisement

\protect\hyperlink{after-bottom}{Continue reading the main story}

\hypertarget{site-index}{%
\subsection{Site Index}\label{site-index}}

\hypertarget{site-information-navigation}{%
\subsection{Site Information
Navigation}\label{site-information-navigation}}

\begin{itemize}
\tightlist
\item
  \href{https://help.nytimes.com/hc/en-us/articles/115014792127-Copyright-notice}{©~2020~The
  New York Times Company}
\end{itemize}

\begin{itemize}
\tightlist
\item
  \href{https://www.nytco.com/}{NYTCo}
\item
  \href{https://help.nytimes.com/hc/en-us/articles/115015385887-Contact-Us}{Contact
  Us}
\item
  \href{https://www.nytco.com/careers/}{Work with us}
\item
  \href{https://nytmediakit.com/}{Advertise}
\item
  \href{http://www.tbrandstudio.com/}{T Brand Studio}
\item
  \href{https://www.nytimes.com/privacy/cookie-policy\#how-do-i-manage-trackers}{Your
  Ad Choices}
\item
  \href{https://www.nytimes.com/privacy}{Privacy}
\item
  \href{https://help.nytimes.com/hc/en-us/articles/115014893428-Terms-of-service}{Terms
  of Service}
\item
  \href{https://help.nytimes.com/hc/en-us/articles/115014893968-Terms-of-sale}{Terms
  of Sale}
\item
  \href{https://spiderbites.nytimes.com}{Site Map}
\item
  \href{https://help.nytimes.com/hc/en-us}{Help}
\item
  \href{https://www.nytimes.com/subscription?campaignId=37WXW}{Subscriptions}
\end{itemize}
