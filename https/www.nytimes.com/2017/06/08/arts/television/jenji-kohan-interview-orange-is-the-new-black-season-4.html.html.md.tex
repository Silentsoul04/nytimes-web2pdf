Sections

SEARCH

\protect\hyperlink{site-content}{Skip to
content}\protect\hyperlink{site-index}{Skip to site index}

\href{https://www.nytimes.com/section/arts/television}{Television}

\href{https://myaccount.nytimes.com/auth/login?response_type=cookie\&client_id=vi}{}

\href{https://www.nytimes.com/section/todayspaper}{Today's Paper}

\href{/section/arts/television}{Television}\textbar{}Jenji Kohan on the
Prison Riot Driving `Orange Is the New Black'

\href{https://nyti.ms/2sWVIMV}{https://nyti.ms/2sWVIMV}

\begin{itemize}
\item
\item
\item
\item
\item
\end{itemize}

Advertisement

\protect\hyperlink{after-top}{Continue reading the main story}

Supported by

\protect\hyperlink{after-sponsor}{Continue reading the main story}

\hypertarget{jenji-kohan-on-the-prison-riot-driving-orange-is-the-new-black}{%
\section{Jenji Kohan on the Prison Riot Driving `Orange Is the New
Black'}\label{jenji-kohan-on-the-prison-riot-driving-orange-is-the-new-black}}

\includegraphics{https://static01.nyt.com/images/2017/06/11/arts/11KOHAN1/11KOHAN1-articleLarge.jpg?quality=75\&auto=webp\&disable=upscale}

By \href{https://www.nytimes.com/by/kathryn-shattuck}{Kathryn Shattuck}

\begin{itemize}
\item
  June 8, 2017
\item
  \begin{itemize}
  \item
  \item
  \item
  \item
  \item
  \end{itemize}
\end{itemize}

When Season 4 of Netflix's ``Orange Is the New Black'' ended, no one ---
save, perhaps, Jenji Kohan, the show's creator, and her writers ---
could be certain where the series was headed. But wherever it was, it
didn't look good. Poussey Washington, the soulful wisp of an inmate
played by Samira Wiley, had died an episode earlier --- agonizingly, in
a scene echoing the 2014 death of Eric Garner --- when a corrections
officer pinned her to the floor with his knee. Then another corrections
officer lost control of his gun, which a furious Daya (Dascha Polanco)
picked up and pointed at his head.

Season 5, beginning Friday, June 9, steps off with Daya and that gun and
then segues into what Ms. Kohan called ``a natural progression'': a riot
in which the inmates seek justice for Poussey's killing by taking
hostages and demanding what the privatization of their facility took
from them, including G.E.D. classes, adequate health care and even
tampons. The action, covering three days, unfurls across 13 episodes.

``We wanted to slow down to really get into the riot, and also to slow
down in general and get into some of the detail work that's hard to do
if you're skipping through time,'' Ms. Kohan said. In a phone interview
from Los Angeles, she spoke about the loss of Poussey; the real-life
prison system; and her latest Netflix project, ``Glow,'' starring Alison
Brie, about women's professional wrestling. These are edited excerpts
from the conversation.

\textbf{So just how fast are we skipping? In real life, Piper Kerman,
whose story inspired the series, spent 13 months at the Federal
Correctional Institution in Danbury, Conn.}

Our Piper was sentenced to 18. I think we're at about 10 months.

\textbf{There's a point where the riot evolves from chaos to a near
utopia --- baristas serving cold brew, bloggers offering beautification
seminars --- before devolving. What's the message?}

I think it reflects humanity. We can be beautiful, and we can be
incredibly destructive. And sometimes we're our own worst enemy, and be
careful what you wish for, and all those clichés. We wanted to follow
what we thought would be a natural progression of the riot, and we had
done research into other ones. They were mostly men's riots. But we were
going to take the information and see how it would be different in our
environment and how it would be the same.

\textbf{What did you come up with?}

It's less violent --- that's the biggest thing. The men just seemed to
be a lot bloodier, and there was also an opportunity for a lot of people
to settle old scores when no one was looking. Not to say that women
can't be violent, because they certainly can. And there are moments, I
think, where it gets a little domestic when they're in charge, which I
don't think is as common in men's. There's a different tone to it, and I
think it was surprising for the other characters, from the guards to the
administration. Women aren't supposed to riot.

\textbf{You've expounded on privatization and overcrowding in prisons.
What's your take on the system five seasons in?}

I think it's really broken. I think it's increasingly corrupt. I fear
more privatization. I think it's much more punitive than rehabilitative,
and it's a huge waste of money in a lot of areas. It's a very flawed
system.

\includegraphics{https://static01.nyt.com/images/2017/06/11/arts/11KOHAN2/11KOHAN2-articleLarge.jpg?quality=75\&auto=webp\&disable=upscale}

\textbf{Do you worry about increased incarceration under President Trump
and Attorney General Jeff Sessions?}

Absolutely. And we're trying to figure that out because theoretically
we're operating in the past. But we want to address current events and
feelings, so we may abandon the timeline. Piper will still have served
the same amount of time, but we will be in the present day.

\textbf{Even in her absence, Poussey is at the heart of this season. A
lot of ``Orange'' fans miss her.}

We love her, too, and that's part of why she was a sacrificial lamb. We
felt her death would be the most affecting for the audience, and it was
really hard for us because she's terrific to work with, she's a terrific
actress, and we like her personally. But we built this character who had
hope and promise, and extinguishing that light would be the most
devastating.

\textbf{You've been guaranteed seven seasons. Is Season 7 the end?}

I haven't made a final decision yet, but I'm leaning toward ending it
after seven --- although the nature of the show is one that can go on
and on because you can bring in new people.

\textbf{Do you have an ending in mind that you can share?}

I think I do. And I will not share it.

\textbf{Has writing within a prison setting been limiting?}

For the crew and the actors, we spend our days in jail. We spend our
days on a set that looks like a prison. We eat a lot better, we can go
outside, but there is something oppressive about the environment of a
prison that we've written ourselves into. It makes you really grateful
that you're not {[}in a real one{]}. But there are bars and dirt, even
though the set department put the dirt there, and you get a sense of
what you're writing about when you're in that world. And that can take
its toll. That's why we rely on humor, that's why we do the flashbacks
--- to take a break from all that.

\textbf{Netflix doesn't give viewer numbers to producers. What's it like
functioning in a vacuum?}

You just do it because you do it. It makes it certainly a little harder
to negotiate when that time comes because you can't say, ``Look how many
people are watching.'' It's nice to be buffeted by numbers. On the other
hand, I never really thought about numbers before, so it's not a new
experience for me. It's a little frustrating --- how about that?

\textbf{And now you're an executive producer on Netflix's ``Glow,''
about women's wrestling in the 1980s.}

Carly {[}Mensch, one of the show's creators, writers and executive
producers{]} had started on ``Weeds,'' she wrote on ``Orange,'' and then
she went away and did a year on ``Nurse Jackie.'' I'm a fan, and she
said, ``Look, I have this friend who's a writer on `Nurse Jackie,' and
we just watched this documentary about this women's wrestling league,
and we're totally obsessed.'' And I think my role with them is mostly
like godmother because {[}Carly and Liz Flahive, the show's
co-creator{]} are both really talented and capable. But it's also their
first time out, so I guess I'm the bodyguard and adviser. I read scripts
and give notes and go to set sometimes, but they know what they're
doing. We didn't always agree but I think part of my job as protector is
maybe even to protect them from me.

\textbf{People started out viewing the broad diversity on ``Orange'' as
unconventional, which you have said made you sad. How do you see this
array of women?}

I see them as gorgeous and sexy and interesting and normative in a way
that the milieu doesn't always present.

Advertisement

\protect\hyperlink{after-bottom}{Continue reading the main story}

\hypertarget{site-index}{%
\subsection{Site Index}\label{site-index}}

\hypertarget{site-information-navigation}{%
\subsection{Site Information
Navigation}\label{site-information-navigation}}

\begin{itemize}
\tightlist
\item
  \href{https://help.nytimes.com/hc/en-us/articles/115014792127-Copyright-notice}{©~2020~The
  New York Times Company}
\end{itemize}

\begin{itemize}
\tightlist
\item
  \href{https://www.nytco.com/}{NYTCo}
\item
  \href{https://help.nytimes.com/hc/en-us/articles/115015385887-Contact-Us}{Contact
  Us}
\item
  \href{https://www.nytco.com/careers/}{Work with us}
\item
  \href{https://nytmediakit.com/}{Advertise}
\item
  \href{http://www.tbrandstudio.com/}{T Brand Studio}
\item
  \href{https://www.nytimes.com/privacy/cookie-policy\#how-do-i-manage-trackers}{Your
  Ad Choices}
\item
  \href{https://www.nytimes.com/privacy}{Privacy}
\item
  \href{https://help.nytimes.com/hc/en-us/articles/115014893428-Terms-of-service}{Terms
  of Service}
\item
  \href{https://help.nytimes.com/hc/en-us/articles/115014893968-Terms-of-sale}{Terms
  of Sale}
\item
  \href{https://spiderbites.nytimes.com}{Site Map}
\item
  \href{https://help.nytimes.com/hc/en-us}{Help}
\item
  \href{https://www.nytimes.com/subscription?campaignId=37WXW}{Subscriptions}
\end{itemize}
