Sections

SEARCH

\protect\hyperlink{site-content}{Skip to
content}\protect\hyperlink{site-index}{Skip to site index}

\href{https://www.nytimes.com/section/nyregion}{New York}

\href{https://myaccount.nytimes.com/auth/login?response_type=cookie\&client_id=vi}{}

\href{https://www.nytimes.com/section/todayspaper}{Today's Paper}

\href{/section/nyregion}{New York}\textbar{}Five Days of Unrest That
Shaped, and Haunted, Newark

\url{https://nyti.ms/2u6kLkE}

\begin{itemize}
\item
\item
\item
\item
\item
\end{itemize}

Advertisement

\protect\hyperlink{after-top}{Continue reading the main story}

Supported by

\protect\hyperlink{after-sponsor}{Continue reading the main story}

50 years after the uprising

\hypertarget{five-days-of-unrest-that-shaped-and-haunted-newark}{%
\section{Five Days of Unrest That Shaped, and Haunted,
Newark}\label{five-days-of-unrest-that-shaped-and-haunted-newark}}

By \href{http://www.nytimes.com/by/rick-rojas}{Rick Rojas} and Khorri
Atkinson

\begin{itemize}
\item
  July 11, 2017
\item
  \begin{itemize}
  \item
  \item
  \item
  \item
  \item
  \end{itemize}
\end{itemize}

\includegraphics{https://static01.nyt.com/images/2017/07/12/nyregion/12newarkanniversary1/00newarkanniversary1-articleInline.jpg?quality=75\&auto=webp\&disable=upscale}

NEWARK --- The fuse that ignited this city --- violent clashes with the
police, gunfire, flames from burning buildings --- was a rumor. Word
spread that a black cabdriver had been killed inside a police precinct
house. It was not true --- the driver had been arrested and injured.

Still, it was enough to inflame a population filled with years of
pent-up grievances: not only abuse at the hands of the police, but
entrenched, unaddressed poverty, urban renewal policies that bypassed
black residents and a white political power structure that had long
ignored their needs.

The unrest, which started on the night of July 12, 1967, and ended on
July 17, came during a period when racial tensions were exploding into
violent conflagrations across the country: the Watts neighborhood in Los
Angeles, Harlem, Detroit and nearby New Jersey communities, including
Plainfield.

Image

From left: burning businesses doused by firefighters, and an injured
rioter with the police.Credit...Left: Associated Press; Right: Hulton
Archive, via Getty Images

Amid all that, what made Newark's unrest striking was the physical toll,
placing it among the deadliest of the riots. Over several days, 26
people were killed --- many of them black residents, as well as a white
firefighter and a white police detective --- and more than 700 were
injured. The riots caused about \$10 million in damages and reduced
entire blocks to charred ruins, some of which remain vacant
grass-covered lots.

The turbulence also left an enduring legacy in molding perceptions of
the city, even though those impressions vary widely. To some, the flames
and violence were riots, wrecking neighborhoods and driving away white
and middle-class residents, feeding a notion that Newark was a dangerous
and broken place. Or it was a rebellion, the uprising of a
long-oppressed community that had finally had enough, and from that, a
new sense of empowerment was born.

\begin{center}\rule{0.5\linewidth}{\linethickness}\end{center}

Image

Taking cover from sniper fire.Credit...Neal Boenzi/The New York Times

\hypertarget{a-city-pushed-to-the-edge}{%
\subsection{A City Pushed to the Edge}\label{a-city-pushed-to-the-edge}}

On July 12, 1967, residents of a large public housing development saw
the black cabdriver badly beaten by police officers and followed them to
the Fourth Police Precinct house, in Newark's Central Ward. The crowd
ignored calls for a peaceful protest, and the police responded in force.

\hypertarget{junius-w-williams}{%
\subsubsection{Junius W. Williams}\label{junius-w-williams}}

\emph{He was a law student at the time and has long been a civil rights
activist in Newark. He is now director of the Abbott Leadership
Institute at Rutgers University, Newark.}

It was very tense in Newark before this happened, and it was all based
on race and class, but race predominated because all of this was
happening at the expense of and to black people, no matter if you were
poor, middle-class or working. The police were agents of oppression and
the political system was hellbent on keeping us in our place.

The first two or three days there was a sense that this was the relief
and the release that people needed. But in the second three days, once
the combined police force --- local, state and the National Guard~---
had been fully deployed, there was fear, because the police rioted. If
this was a rebellion, the police rioted. They took it in their own hands
to seek retribution against people for the slightest infraction.

\hypertarget{paul-zigo}{%
\subsubsection{Paul Zigo}\label{paul-zigo}}

*He served as a member of the New Jersey National Guard and deployed to
the Central Ward of Newark on July 14, 1967, his 25th birthday. He
recently retired as a professor of American history at Brookdale
Community College and is the director of the World War II Era Studies
Institute.~*

There was nothing but a huge red glow over the entire city. The city was
in flames. You immediately took note of what you may be entering into
--- and that was a war zone. Buildings were smashed, windows were broken
all over the place. You saw a very angry crowd, and when they saw you,
they started coming in and around your Jeep.

I saw two black men fighting like hell among themselves, and what I did
was pull my pistol and fire two shots into the air. {[}He told the men
to leave.{]} They stopped fighting and they scampered down the street.
But to my surprise, when I did that, someone within a blacked-out church
--- St. Rose of Lima Church off Orange Street --- took aim at me and
fired at me. The bullets came toward me and went over my head by about a
foot. I didn't know I was being fired at. Someone called out, ``Don't
you know you're being fired at?'' If it weren't for that individual, who
I considered an angel, I wouldn't have taken cover.~

Image

A man taken from a building from which sniper fire was coming.
Credit...Neal Boenzi/The New York Times

\begin{center}\rule{0.5\linewidth}{\linethickness}\end{center}

\hypertarget{we-took-pride-in-our-police-community-relations-program-this-night-has-destroyed-what-we-thought-was-a-good-relationship-we-cant-sweep-this-under-a-rug-time-is-running-out-it-has-gotten-to-the-point-where-you-cant-take-it-any-longer}{%
\subsubsection{``We took pride in our police-community relations
program. This night has destroyed what we thought was a good
relationship. We can't sweep this under a rug. Time is running out. It
has gotten to the point where you can't take it any
longer.''}\label{we-took-pride-in-our-police-community-relations-program-this-night-has-destroyed-what-we-thought-was-a-good-relationship-we-cant-sweep-this-under-a-rug-time-is-running-out-it-has-gotten-to-the-point-where-you-cant-take-it-any-longer}}

Albert Black

Chairman of Newark's Human Rights Commission, quoted by The New York
Times in a 1967 article about a meeting with Mayor Hugh J. Addonizio on
the second day of unrest, in which black leaders expressed frustration
with the police over racism and brutality.

\begin{center}\rule{0.5\linewidth}{\linethickness}\end{center}

\hypertarget{walter-g-ricciardi}{%
\subsubsection{Walter G. Ricciardi}\label{walter-g-ricciardi}}

\emph{He grew up in Newark, leaving in 1971 for college at Columbia
University. He and his wife raised their four sons in Ridgewood, N.J.,
and they now live there and in New York City, where he works as a lawyer
and adjunct professor at New York University Law School.}

Image

Credit...Joshua Bright for The New York Times

I was a 13-year-old paperboy for The Newark Star-Ledger in the Vailsburg
section of Newark. I saw military vehicles with soldiers riding up and
down South Orange Avenue. Some of the paperboys were sent out of town to
stay with relatives to the dismay of our manager, Mr. Lamb. I carried a
book of poems by Amiri Baraka on my route, hoping I would be seen as an
ally if I ran into any rioters. The father of a friend from the
neighborhood was a fireman. He was shot and killed by a sniper ---
firemen rode on the outside of the truck, clinging to the railing back
then. Mayor Hugh Addonizio was one of my customers. I delivered my
papers every morning unharmed, collected from my customers that Friday
and Saturday, and paid my bill to Mr. Lamb on Saturday afternoon. ~

\hypertarget{john-peoples}{%
\subsubsection{John Peoples}\label{john-peoples}}

\emph{He is a lifelong resident of the city, who was about 4 years old
during the unrest.}

Image

Credit...Misha Friedman for The New York Times

I only can remember hearing my mother saying, ``Boy, move from the
window and get under the bed!''

\hypertarget{jonathan-lazarus}{%
\subsubsection{Jonathan Lazarus}\label{jonathan-lazarus}}

\emph{He worked in Newark from 1966 to 2006 as an editor for The Newark
News,}
\href{http://www.nytimes.com/1981/08/30/nyregion/the-newark-news-in-memoriam.html}{\emph{an
afternoon newspaper that closed in 1972}}\emph{, and The Star-Ledger.
His family lived in the city from 1947 until 1957, when they moved to
the suburb of West Orange, where he still lives.}

Image

Credit...Joshua Bright for The New York Times

What is your most vivid memory from the riots?

Following the National Guard convoys out of the city at night after my
shift as a copy editor ended at The Newark News. I was concerned at the
time that my Army Reserve unit may be activated for riot duty.
Thankfully, we were not. We lacked even rudimentary training and would
have been as ineffective and insensitive as many of the guardsmen and
state troopers on the scene. Working through the riots was both
frightening and surreal.

But the editors failed in the years leading up to the riots to examine
or expose the root causes of the unrest. Very little original reporting
was done in the black community, while downtown business interests and
the suburbs were lavishly covered. When the city exploded, The News
lacked black reporters with deep ties to the community to adequately
explain the ferment.

\hypertarget{ruth-dolinko}{%
\subsubsection{Ruth Dolinko}\label{ruth-dolinko}}

\emph{Ms. Dolinko's husband, Morris,} ****** \emph{who died in 2010, ran
a delicatessen supply business with his father and two other partners.
It is now run by her son and another partner. Ms. Dolinko and her
husband were both born and raised in Newark, and moved to West Orange in
1956. The business was not damaged in the unrest because, she said,
black employees had stationed themselves outside and encouraged rioters
to spare it.}

Image

Credit...Misha Friedman for The New York Times

The people were very distressed. The way the black community was treated
on all levels --- that was very difficult. The political level was not
good, the housing --- you know on all levels. You have to feel a
compassion for these people that were hurt, that had nowhere else to
turn but to raise their voice that way.

\hypertarget{armando-b-fontoura}{%
\subsubsection{Armando B. Fontoura}\label{armando-b-fontoura}}

\emph{He has been the Essex County sheriff since 1990 and is a native of
Newark. His career in law enforcement started in 1967 as a city police
officer, beginning shortly before the riots. He remained with the
department for about two decades, becoming a captain and chief assistant
to the police director.}

I was a rookie police officer at the time. I had just come out of the
academy. You became a seasoned veteran in the war against crime in three
days.

It was chaotic, to say the least. We were not prepared. We had no
equipment. Guys were bringing in their own helmets. Guys were bringing
in their own shotguns.

To see the devastation that was taking place over the next few days ---
the burning, the looting, the breaking into stores. It was devastating
for those of us who love this city, who live in this city. The fear I
had wasn't for my own safety, but for watching the city I loved go up in
flames. You begin to realize this is going to take a really long time to
recover from.

\subsubsection{}

\begin{center}\rule{0.5\linewidth}{\linethickness}\end{center}

Image

Protesters and a guardsman with bayonet.Credit...Neal Boenzi/The New
York Times

\hypertarget{the-people-in-newark-have-to-choose-sides-they-are-either-citizens-of-america-or-criminals-who-would-shoot-down-a-fire-captain-in-the-back-and-then-depend-on-people-to-speak-in-platitudes-about-police-brutality}{%
\subsubsection{``The people in Newark have to choose sides. They are
either citizens of America or criminals who would shoot down a fire
captain in the back and then depend on people to speak in platitudes
about police
brutality.''}\label{the-people-in-newark-have-to-choose-sides-they-are-either-citizens-of-america-or-criminals-who-would-shoot-down-a-fire-captain-in-the-back-and-then-depend-on-people-to-speak-in-platitudes-about-police-brutality}}

Gov. RICHARD J. HUGHES

New Jersey's governor from 1962 to 1970, speaking at a news conference
as the unrest entered its fifth day.

\begin{center}\rule{0.5\linewidth}{\linethickness}\end{center}

\hypertarget{newark-became-a-fearful-place}{%
\subsection{`Newark Became a Fearful
Place'}\label{newark-became-a-fearful-place}}

White residents started leaving Newark well before the riots, drawn by
the advent of suburbs and the highway system. But researchers said the
unrest propelled the exodus and gave the city a dismal reputation. For
those who remained, especially African-Americans, the healing process
has been slow and difficult.

\hypertarget{mr-ricciardi}{%
\subsubsection{Mr. Ricciardi}\label{mr-ricciardi}}

Before the riots, our neighborhood was filled with cops, firemen,
construction workers and teachers. They moved because things were
expected to get worse (home values, safety). We did not have the
financial option to move --- my father was a disabled police officer and
my mother worked nights at a hospital in East Orange to support her five
children. Newark became a fearful place --- above-ground walkways to the
Gateway buildings from Penn Station. No one wanted to set foot in
Newark.

Image

A stretch of Springfield Avenue, a focal point of the riots, in 1974.
Credit...Jerry Mosey/Associated Press

\hypertarget{brian-white}{%
\subsubsection{Brian White}\label{brian-white}}

\emph{He grew up in Newark, where his father had owned a watch repair
and jewelry store downtown. Mr. White left Newark in 1975, and now lives
in Boston after 20 years in Manhattan and the New Jersey suburbs. }

Whatever trust and friendliness existed between blacks and whites eroded
after the riots. White flight accelerated; the streets that were busy in
the downtown area emptied out after 5 p.m., and many businesses closed.
While I haven't lived in Newark for many years, I follow its ups and
(mostly) downs, wondering when it will experience a rebirth much like
Jersey City and Hoboken. It's a shame. My life in Newark was the source
of many wonderful memories and I still am quite fond of the town. ~

\hypertarget{sheriff-fontoura}{%
\subsubsection{Sheriff Fontoura}\label{sheriff-fontoura}}

In the aftermath --- the State Police did their thing and left, and we
were left to carry on here, and I think folks in the community did not
differentiate between the State Police and the Newark police. We all
wore blue. We had to repair some of the damage done by other folks.~

Our relationship wasn't great to begin with. You could cut the tension
with a knife. I remember being on patrol. It took a long time to develop
some trust again. That took long, hard work.

Image

Junius WilliamsCredit...Joshua Bright for The New York Times

\hypertarget{an-empowered-community}{%
\subsection{An Empowered Community}\label{an-empowered-community}}

One of the sources of frustration was a plan to build a medical school
in the Central Ward, which would have occupied more than 150 acres and
displaced thousands of residents, most of them black or Hispanic. After
the unrest, Mr. Williams, the civil rights activist, argued that black
residents were in a position to push back, reducing the school's
footprint and encouraging it to serve the surrounding community.

\hypertarget{mr-williams}{%
\subsubsection{Mr. Williams}\label{mr-williams}}

As the smoke cleared and the last dying embers of the flames receded,
some of us realized the power structure was afraid. First time they had
ever been afraid of us in this city. So we began to think of, how are we
are going to take advantage of this violence that nobody wanted? My
group was formed, the Newark Area Planning Association, and we decided
we were going to work on the medical school. We had to cut that medical
school down. Some people didn't want it at all, but some of us saw it as
something valuable.

The black community was definitely empowered. Nobody wanted that
violence. But at the same time, people were politically adept enough to
see that we had the opportunity to turn that destructive power into
something that was positive for the community, which if they had just
allowed us to do in the beginning, it never would have happened.

\hypertarget{mr-zigo}{%
\subsubsection{Mr. Zigo}\label{mr-zigo}}

As I toured Newark after arriving, seeing the devastation I saw, plus
the conditions under which many, many people were living in the Central
Ward, you could understand why there was an angriness, why there was,
you could say, a rebellion. It was pure, pure impoverishment. The lack
of attention, you could see it.~Go back to Newark now, it's all
different. What an improvement. A positive result did come about because
of the riots, but it took years.

\hypertarget{mildred-c-crump}{%
\subsubsection{Mildred C. Crump}\label{mildred-c-crump}}

\emph{Ms. Crump, the president of the Newark City Council, was the first
African-American woman elected to the council, in 1994. She moved to
Newark from Detroit in 1965, and worked for more than 40 years as a
Braille teacher at the New Jersey Commission for the Blind and Visually
Impaired.}

There has been significant progress, but not enough, trust me. But
there's been progress for African-Americans. Now we're a black and brown
community. Our Hispanic brothers and sisters were part of the
progression that we made. For example, my husband and I bought a house
in the South Ward where the Jewish community was in prominence. That
could never have happened if 1967 had not happened.

Image

Police officers in the Clinton Hill area.Credit...Bryan Anselm for The
New York Times

\hypertarget{enduring-problems}{%
\subsection{Enduring Problems}\label{enduring-problems}}

Last year, the city reached a settlement with the United States Justice
Department that led to the appointment of a federal monitor and the
implementation of sweeping reforms, including establishing a civilian
oversight committee. A three-year investigation highlighted issues
similar to those that fueled the riots: Three-fourths of pedestrian
stops cited by the police were not justified, minorities were stopped
more often than whites and the use of force was underreported.

\hypertarget{ryan-p-haygood}{%
\subsubsection{Ryan P. Haygood}\label{ryan-p-haygood}}

\emph{Mr. Haygood, a civil rights lawyer, is the president and chief
executive of the New Jersey Institute for Social Justice.}

The consent decree is such an important document. It's a 70-page
document with more than 100 provisions that contemplate widespread
reform in almost every area of policing --- in the way officers are
trained, in the way officers are disciplined, in the way officers
collect and store evidence.

Look at the strides the city has made in 50 years, but there's also a
need to look at the underlying issues that led to the Newark rebellion
in the first place. If we fail to address those issues, we leave
ourselves vulnerable to another rebellion.

\hypertarget{derek-glenn}{%
\subsubsection{Derek Glenn}\label{derek-glenn}}

\emph{Mr. Glenn is a Newark police captain and a spokesman for the
department.}

One of the most important lessons that came out of those challenging
days is that the racial composition of the Newark Police Division more
closely reflects the demographics of our residents. Today, we have a
police division comprising 42 percent Hispanic, 37 percent black and the
remaining representing whites, Asians and others. And I'm very proud of
these numbers because I know that the people of the city of Newark
deserve a police division that they can relate to. The us-against-them
mentality is markedly reduced when police reflect the community. This
didn't exist in 1967, when only 11 percent of Newark police officers
were black in contrast to 50 percent of the city's residents' being
black at that time.

\begin{center}\rule{0.5\linewidth}{\linethickness}\end{center}

Image

Henry PunkinCredit...Misha Friedman for The New York Times

\hypertarget{i-havent-seen-any-change-not-in-here-there-may-be-change-but-i-cant-see-it-all-i-can-say-its-getting-real-bad}{%
\subsubsection{``I haven't seen any change. Not in here. There may be
change, but I can't see it. All I can say, it's getting real
bad.''}\label{i-havent-seen-any-change-not-in-here-there-may-be-change-but-i-cant-see-it-all-i-can-say-its-getting-real-bad}}

Henry Punkin

Mr. Punkin, 74, is a retired plumber who moved to Newark from Atlanta
when he was 16 years old, and lives in the Springfield/Belmont
neighborhood.

\begin{center}\rule{0.5\linewidth}{\linethickness}\end{center}

\hypertarget{turning-a-corner}{%
\subsection{Turning a Corner}\label{turning-a-corner}}

\hypertarget{ras-j-baraka}{%
\subsubsection{Ras J. Baraka}\label{ras-j-baraka}}

\emph{He}
\href{https://www.nytimes.com/2014/05/14/nyregion/newark-mayoral-race.html}{\emph{became
mayor of Newark}} \emph{in 2014, the latest in a succession of
African-Americans elected to lead New Jersey's largest city, starting
with Kenneth A. Gibson, who was voted into office three years after the
riots. Mr. Baraka, 47, has deep roots in the city; his father is Amiri
Baraka, the fiery poet and playwright, who}
\href{http://www.nytimes.com/2012/10/11/nyregion/amiri-baraka-newark-poet-looks-back-on-a-bloody-week-in-1967.html}{\emph{took
part in the uprising}} \emph{in 1967, and}
\href{https://www.nytimes.com/2014/01/10/arts/amiri-baraka-polarizing-poet-and-playwright-dies-at-79.html}{\emph{who
died}} \emph{shortly before his son was elected mayor.}

Image

Credit...Joshua Bright for The New York Times

Has it recovered? Not completely. There are still some emotional trauma
and other things we haven't recovered from and social conditions that
led to the rebellion itself. And it hasn't been fully addressed. Some of
the physical and economical circumstances, while they're not exactly the
same as it was, there are vestiges of it and we've been moving very
rapidly to eliminate all of it.

Obviously, some things that exist now didn't exist then that played into
why something like that would not happen again. We don't have a
predominantly white Police Department anymore. You have a predominantly
black and brown Police Department --- in leadership positions. You still
have issues. But some of those officers come from the city of Newark.
Their families are here. Those are the things that changed.

There's the social sector of the community that is basically controlled
by black and brown people, so that kind of helps mitigate the
conditions. But the underlying circumstances that create poverty and
homelessness have not completely gone or been addressed adequately.
That's our job to try to get that done.

Even though we're having all these programs to talk about the 50-year
anniversary of this rebellion, there hasn't been any real conversation
in New Jersey. Most of those folks have left the city. There has been no
conversation with those in the suburbs and those in the city to talk
about what caused {[}the riots{]}, which I think would lead to a deeper
discussion about equality in America.

\hypertarget{mr-lazarus}{%
\subsubsection{Mr. Lazarus}\label{mr-lazarus}}

I grew up in Newark when it was a thriving commercial and manufacturing
hub, a city of vast parks, strong schools, wonderful branch libraries
and viable neighborhoods, all except the Central Ward, the deliberately
overlooked ground-zero ghetto. This all went away with the riots. My
family moved to the suburbs in 1957, so we escaped the immediacy of the
destruction, but felt its impact for a lifetime.

I worked nights in Newark for the remainder of my news career and saw
the scarring effects of those four nights of hell linger for decades.
But Newark has definitely managed to turn a corner. Development, jobs
and commerce are improving. The city has become a higher education
center. Political leadership, while imperfect, is superior to previous
iterations, both black and white. And, after a 40-year absence, the city
ended its food-desert reputation by enticing supermarkets to come in. ~

\hypertarget{andrea-mcchristian}{%
\subsubsection{Andrea McChristian}\label{andrea-mcchristian}}

\emph{She is a lawyer for the New Jersey Institute for Social Justice,
where she works on criminal justice reform. She has lived in the city
for a year and a half.}

Image

Credit...Misha Friedman for The New York Times

I really like the diversity of Newark, and the fact that, even though
it's a very large city --- the largest city in New Jersey --- it feels
like a small town. When you walk around here, I'll bump into 10 people I
know anytime I walk outside.~Each of the wards has a different feel, a
different flavor.

Even though there is skepticism from a great deal of people who have
been hurt before, I think there is also this competing hope, where
people are hopeful that maybe this time, with the vibrancy and
intergenerational nature with a lot of these coalitions, we can get some
reforms going. ​

\hypertarget{sheriff-fontoura-1}{%
\subsubsection{Sheriff Fontoura}\label{sheriff-fontoura-1}}

I'm sick and tired of talking about the riots. It was 50 years ago. The
day we stop talking about the riots, that's the day we make a full
comeback.~

\begin{center}\rule{0.5\linewidth}{\linethickness}\end{center}

Image

Credit...Neal Boenzi/The New York Times

\hypertarget{how-the-times-covered-the-riots-in-1967}{%
\subsection{How The Times covered the riots in
1967}\label{how-the-times-covered-the-riots-in-1967}}

\hypertarget{day-1}{%
\subsubsection{\texorpdfstring{\href{https://timesmachine.nytimes.com/timesmachine/1967/07/13/90372463.html?pageNumber=1}{Day
1}}{Day 1}}\label{day-1}}

\begin{quote}
The violence was believed to have started after the arrest of a Negro
cabdriver on charges of assaulting a policeman. The crowd of Negroes
that gathered outside the Fourth Precinct station house protested the
arrest and some persons shouted ``police brutality!''

The mob that gathered around the station house threw rocks and bottles.
At least five policemen were struck by stones, one policeman reported.
\end{quote}

\hypertarget{day-2}{%
\subsubsection{\texorpdfstring{\href{https://timesmachine.nytimes.com/timesmachine/1967/07/14/83616047.html?pageNumber=1}{Day
2}}{Day 2}}\label{day-2}}

\begin{quote}
Mayor Hugh J. Addonizio telephoned Gov. Richard J. Hughes at 2:20 a.m.
and told the governor that the rampaging Negroes who had looted, burned
and smashed their way through the city in the second straight night of
violence had produced an ``ominous situation.''
\end{quote}

\hypertarget{day-3}{%
\subsubsection{\texorpdfstring{\href{https://timesmachine.nytimes.com/timesmachine/1967/07/15/83616900.html?pageNumber=1}{Day
3}}{Day 3}}\label{day-3}}

\begin{quote}
Gov. Richard J. Hughes declared Newark a city in ``criminal
insurrection'' as the fighting and looting continued despite a curfew
and the presence of armed troops. The violence was the nation's worst
racial outbreak since the rioting in the Watts section of Los Angeles in
\end{quote}

\begin{enumerate}
\def\labelenumi{\arabic{enumi}.}
\setcounter{enumi}{1964}
\item
\end{enumerate}

\hypertarget{day-4}{%
\subsubsection{\texorpdfstring{\href{https://timesmachine.nytimes.com/timesmachine/1967/07/16/83617964.html?pageNumber=1}{Day
4}}{Day 4}}\label{day-4}}

\begin{quote}
The governor, red with anger, said shortly before midnight that it was
time that the people of Newark chose between the terrorists and law and
order.

``This is a criminal insurrection by people who say they hate the white
man but who really hate America,'' Governor Hughes charged.~
\end{quote}

\hypertarget{day-5}{%
\subsubsection{\texorpdfstring{\href{https://timesmachine.nytimes.com/timesmachine/1967/07/17/90374734.html?pageNumber=1}{Day
5}}{Day 5}}\label{day-5}}

\begin{quote}
Nearly half of Newark's 23.7 square miles was an occupied zone. As
sporadic sniping continued, guardsmen and policemen --- weary and
trigger-quick after days and nights of tension --- were reported to have
engaged each other in several accidental gunfights.
\end{quote}

\hypertarget{day-6}{%
\subsubsection{\texorpdfstring{\href{https://timesmachine.nytimes.com/timesmachine/1967/07/18/90375517.html?pageNumber=1}{Day
6}}{Day 6}}\label{day-6}}

\begin{quote}
Although the city appeared to be returning to normal, there was yet
another fatal incident early today. A Negro was shot and killed by the
police in the Negro section while allegedly attempting to loot a store.
His death was the 26th since the rioting began Wednesday.
\end{quote}

\begin{center}\rule{0.5\linewidth}{\linethickness}\end{center}

Some of the interviews included in this article were edited for space
and clarity.

\emph{Jack Begg contributed research.}

Advertisement

\protect\hyperlink{after-bottom}{Continue reading the main story}

\hypertarget{site-index}{%
\subsection{Site Index}\label{site-index}}

\hypertarget{site-information-navigation}{%
\subsection{Site Information
Navigation}\label{site-information-navigation}}

\begin{itemize}
\tightlist
\item
  \href{https://help.nytimes.com/hc/en-us/articles/115014792127-Copyright-notice}{©~2020~The
  New York Times Company}
\end{itemize}

\begin{itemize}
\tightlist
\item
  \href{https://www.nytco.com/}{NYTCo}
\item
  \href{https://help.nytimes.com/hc/en-us/articles/115015385887-Contact-Us}{Contact
  Us}
\item
  \href{https://www.nytco.com/careers/}{Work with us}
\item
  \href{https://nytmediakit.com/}{Advertise}
\item
  \href{http://www.tbrandstudio.com/}{T Brand Studio}
\item
  \href{https://www.nytimes.com/privacy/cookie-policy\#how-do-i-manage-trackers}{Your
  Ad Choices}
\item
  \href{https://www.nytimes.com/privacy}{Privacy}
\item
  \href{https://help.nytimes.com/hc/en-us/articles/115014893428-Terms-of-service}{Terms
  of Service}
\item
  \href{https://help.nytimes.com/hc/en-us/articles/115014893968-Terms-of-sale}{Terms
  of Sale}
\item
  \href{https://spiderbites.nytimes.com}{Site Map}
\item
  \href{https://help.nytimes.com/hc/en-us}{Help}
\item
  \href{https://www.nytimes.com/subscription?campaignId=37WXW}{Subscriptions}
\end{itemize}
