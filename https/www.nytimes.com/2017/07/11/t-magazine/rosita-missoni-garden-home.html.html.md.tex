Sections

SEARCH

\protect\hyperlink{site-content}{Skip to
content}\protect\hyperlink{site-index}{Skip to site index}

\href{https://myaccount.nytimes.com/auth/login?response_type=cookie\&client_id=vi}{}

\href{https://www.nytimes.com/section/todayspaper}{Today's Paper}

At Home (and in the Garden) With Rosita Missoni

\href{https://nyti.ms/2v93Mv7}{https://nyti.ms/2v93Mv7}

\begin{itemize}
\item
\item
\item
\item
\item
\item
\end{itemize}

Advertisement

\protect\hyperlink{after-top}{Continue reading the main story}

Supported by

\protect\hyperlink{after-sponsor}{Continue reading the main story}

\hypertarget{at-home-and-in-the-garden-with-rosita-missoni}{%
\section{At Home (and in the Garden) With Rosita
Missoni}\label{at-home-and-in-the-garden-with-rosita-missoni}}

\href{https://www.nytimes.com/slideshow/2017/07/11/t-magazine/food/a-visit-with-rosita-missoni-in-milan.html}{}

\hypertarget{a-visit-with-rosita-missoni-in-milan}{%
\subsection{A Visit With Rosita Missoni in
Milan}\label{a-visit-with-rosita-missoni-in-milan}}

11 Photos

View Slide Show ›

\includegraphics{https://static01.nyt.com/images/2017/07/11/t-magazine/food/rosita-missoni-slide-ULFV/rosita-missoni-slide-ULFV-articleLarge.jpg?quality=75\&auto=webp\&disable=upscale}

Federico Ciamei

By Laura Rysman

\begin{itemize}
\item
  July 11, 2017
\item
  \begin{itemize}
  \item
  \item
  \item
  \item
  \item
  \item
  \end{itemize}
\end{itemize}

Rosita Missoni, the hardworking 85-year-old matriarch and co-founder of
one of today's great luxury fashion brands, is crouched in the mud of
her vegetable garden, showing me her summer crops. ``It's a little
support for our kitchen,'' she says of the garden, surveying the growth
of her vegetable supply. We are exploring the grounds of her home in
Sumirago, in the forested hills north of Milan. The Missoni factory is
just beyond a row of dogwood bushes and pine trees.

Inside the wooden gates of a small tract of vegetables, Missoni
indicates rows of garlic shoots, underripe tomatoes, clusters of
cucumbers, blooming heads of cabbage, and a queue of sharp-scented
yellow marigolds keeping pests at bay. She leads on through a vast jade
meadow lined with oak and ash trees, where two thoroughly inert dogs
languish on the grass. ``Our guardians,'' Missoni says affectionately,
her braided ducktail falling over one shoulder. Tied with a Missoni
ribbon, the braid is an eccentric flourish to her pragmatic short hair
--- not altogether surprising from the woman who figured out how to turn
knits into a kaleidoscopic whir of zigzags, stipples and stripes.

\includegraphics{https://static01.nyt.com/images/2017/07/11/t-magazine/food/rosita-missoni-slide-WDK7/rosita-missoni-slide-WDK7-articleLarge.jpg?quality=75\&auto=webp\&disable=upscale}

Rosita and Ottavio Missoni married in 1953, founding their knitwear
company the same year; five years later the couple filled the windows of
Milan's La Rinascente department store with the first dresses to bear
their last name. (At Milan's Palazzo Reale, a current exhibition of the
store's 100th anniversary recreates that same installation.) In the
`60s, Rosita invented the warped, woodgrain-like weave of their now
signature \emph{fiammato} pattern, using a cutting-edge knitting machine
and old skeins of fringe from her grandparents' shawl factory. ``That
opened up a world for us,'' she says. ``From there, my husband started
his design experiments with patterns.'' Those patterns became as
singularly recognizable as any logo, establishing Missoni as a luxury
powerhouse, and helping Italy (and Milan in particular) to dominate the
emerging ready-to-wear market. The brand, still family-owned, has been
led by their daughter Angela since 1997; Rosita, after a few months of
``trying to live like a grandmother,'' returned to work, this time as
the creative director of Missoni Home.

Missoni has always been a fanatic for colors, and in the garden she
disappears into a thicket of high-stemmed blue hydrangeas to pick a
particularly purple bloom for her bouquet. Dropping it into her wicker
basket, she heads to her second vegetable plot, this one full of
chicory, mint, San Marzano tomatoes and long-fingered zucchini plants
--- their sunny blossoms in full glory after the earlier morning rain.
Missoni squats in the wet earth and snips off a few for the day's lunch,
depositing the harvest in her wicker basket before cutting a handful of
rosemary branches and waving their piney perfume under my nose. ``For
our little pizzas,'' she says. ``A house specialty.''

As a child, Missoni was scrawny and sickly. Her parents sent her to a
school ``with a marvelous garden'' on the Ligurian coast where she grew
robust on the sea air and a diet of fish. The experience seems to have
worked --- she is as energetic and spry as one could hope to be at any
age. Missoni brags about scuba diving for shellfish (``It's my
passion,'' she says. ``That's why I have such good lungs and can climb
mountains at 85.'') and is tireless in our two-hour tour of the
sprawling garden. But her delicate childhood also intimately acquainted
her with fashion --- kept inside in her youngest days, Missoni passed
the time at her grandparents' factory by making paper dolls from the
atelier's international style magazines. ``There among the fabric and
the patterns, I learned all about `30s fashion, cutting it out in
silhouettes,'' she says. It was the beginning of a lifelong calling.

Image

In the glass-encased dining room facing out over the green slope of the
garden, the table is always set with a medley of pieces from the Missoni
Home line, which Rosita has designed since 1997 when her daughter Angela
became creative director of the brand.Credit...Federico Ciamei

Missoni deposits her vegetables, herbs and flowers in the kitchen as the
cook begins to prepare risotto and church bells chime the noon hour.
Still buzzing around, arranging flowers, carrying chairs from one room
to another, and pulling out books to show me designs from the past,
Missoni proffers some thoughts on her vitality. ``It's luck,
undoubtedly,'' she says. ``I've been passionately dedicated to my
career, so it's never worn me out. I live in the kind of beautiful place
where my husband and I wanted to spend our weekends. I have my children
close by, and friends that come visit on Sundays.''

It's a simple recipe for a happy life --- and an enduring one. ``I've
been able to overcome even terrible times,'' she says, her voice
softening as she recounts a year of tragedy. In 2013, a plane carrying
her eldest son and his wife disappeared in Venezuela. Ottavio passed
away just a few months later. ``At least, I'm rarely alone at the table
here,'' she says, coming back to life.

At lunch, Missoni fills glasses with white wine and takes the head of
the table. After rounds of chicory salad, poached eggs, rosemary-topped
pizzas, and zucchini risotto, she insists on taking me to the factory
next door, where she will spend the rest of the day at work, as she has
for over six decades. ``We had a lovely morning in the garden, didn't
we?'' she says, leading me out by the shoulder. ``But my real place is
in the office.''

Image

``I've been passionately dedicated to my career, so it's never worn me
out,'' Missoni says, proffering some thoughts on her vitality. ``I live
in the kind of beautiful place where my husband and I wanted to spend
our weekends. I have my children close by, and friends that come visit
on Sundays.'' It's a simple recipe for a happy
existence.Credit...Federico Ciamei

\hypertarget{rositas-summer-zucchini-risotto}{%
\subsection{\texorpdfstring{\textbf{Rosita's Summer Zucchini
Risotto}}{Rosita's Summer Zucchini Risotto}}\label{rositas-summer-zucchini-risotto}}

This risotto eschews the white wine of a traditional recipe to leave the
flavor of the fresh vegetables intact. Though best when eaten right
after cooking, it makes a great leftover dish as well: risotto saltato,
which can be made by pan frying a thin pancake of the risotto with olive
oil in a covered, nonstick pan, flipping it when a well-crisped skin
develops. ``The favorite version of children,'' Missoni laughs.

4-5 tablespoons butter\\
1 small white onion, finely chopped\\
1 pound Carnaroli rice\\
Meat or vegetable broth\\
6 small zucchinisA handful of fresh basil and parsley\\
5 tablespoons freshly grated Parmesan cheese\\
Extra virgin olive oil\\
Salt and pepper to taste

1. Bring a pot of broth to a simmer and keep it warm on another burner
as you cook.

2. Place a chunk of butter (around 2-3 tablespoons) in a pot with enough
depth to cook the rice. Melt over medium heat.

3. Add chopped onions and cook lightly, then add the rice and allow it
to toast in the butter for a few minutes.

4. Add four ladles of broth to the rice, then add chopped zucchini.

5. Continue ladling in broth to the rice, stirring all the while, and
cook until the texture of the rice is just cooked through (usually
around 20-25 minutes).

6. Add the herbs when the heat is off.

7. Stir in the cheese, then add a knob of butter and a dash of oil.

8. Stir to combine and then allow the risotto to sit for a couple of
minutes before serving.

Advertisement

\protect\hyperlink{after-bottom}{Continue reading the main story}

\hypertarget{site-index}{%
\subsection{Site Index}\label{site-index}}

\hypertarget{site-information-navigation}{%
\subsection{Site Information
Navigation}\label{site-information-navigation}}

\begin{itemize}
\tightlist
\item
  \href{https://help.nytimes.com/hc/en-us/articles/115014792127-Copyright-notice}{©~2020~The
  New York Times Company}
\end{itemize}

\begin{itemize}
\tightlist
\item
  \href{https://www.nytco.com/}{NYTCo}
\item
  \href{https://help.nytimes.com/hc/en-us/articles/115015385887-Contact-Us}{Contact
  Us}
\item
  \href{https://www.nytco.com/careers/}{Work with us}
\item
  \href{https://nytmediakit.com/}{Advertise}
\item
  \href{http://www.tbrandstudio.com/}{T Brand Studio}
\item
  \href{https://www.nytimes.com/privacy/cookie-policy\#how-do-i-manage-trackers}{Your
  Ad Choices}
\item
  \href{https://www.nytimes.com/privacy}{Privacy}
\item
  \href{https://help.nytimes.com/hc/en-us/articles/115014893428-Terms-of-service}{Terms
  of Service}
\item
  \href{https://help.nytimes.com/hc/en-us/articles/115014893968-Terms-of-sale}{Terms
  of Sale}
\item
  \href{https://spiderbites.nytimes.com}{Site Map}
\item
  \href{https://help.nytimes.com/hc/en-us}{Help}
\item
  \href{https://www.nytimes.com/subscription?campaignId=37WXW}{Subscriptions}
\end{itemize}
