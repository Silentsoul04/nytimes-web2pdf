Sections

SEARCH

\protect\hyperlink{site-content}{Skip to
content}\protect\hyperlink{site-index}{Skip to site index}

\href{https://www.nytimes.com/section/movies}{Movies}

\href{https://myaccount.nytimes.com/auth/login?response_type=cookie\&client_id=vi}{}

\href{https://www.nytimes.com/section/todayspaper}{Today's Paper}

\href{/section/movies}{Movies}\textbar{}`The Big Sick,' South Asian
Identity and Me

\href{https://nyti.ms/2tBnGCC}{https://nyti.ms/2tBnGCC}

\begin{itemize}
\item
\item
\item
\item
\item
\item
\end{itemize}

Advertisement

\protect\hyperlink{after-top}{Continue reading the main story}

Supported by

\protect\hyperlink{after-sponsor}{Continue reading the main story}

\hypertarget{the-big-sick-south-asian-identity-and-me}{%
\section{`The Big Sick,' South Asian Identity and
Me}\label{the-big-sick-south-asian-identity-and-me}}

\includegraphics{https://static01.nyt.com/images/2017/07/22/arts/22SOPANBIGSICK1/22SOPANBIGSICK1-articleInline-v2.jpg?quality=75\&auto=webp\&disable=upscale}

By \href{https://www.nytimes.com/by/sopan-deb}{Sopan Deb}

\begin{itemize}
\item
  July 23, 2017
\item
  \begin{itemize}
  \item
  \item
  \item
  \item
  \item
  \item
  \end{itemize}
\end{itemize}

I never told my Bengali parents about my first kiss. I was 14 and we
were in the basement of my house in Howell, N.J. Her name was Sharon and
she had braces. It didn't go well. Sorry, Sharon.

They never knew about my high school crushes, my dates at Applebee's, or
my first couple of girlfriends. I hid all this because I knew my parents
wouldn't approve. They had an arranged marriage. In India, where they
grew up, choosing your life partner was uncommon.

Right after graduating college, I finally mustered the courage to
introduce my mother to my longtime girlfriend, Michelle, hoping that
after four decades in the United States, my mother might be ready for
the idea that (a) I had chosen my own girlfriend and (b) my girlfriend
might be white.

This is America, after all: You are exposed to choices. You can say what
you want, read what you want and eat what you want. (The actor and
comedian Aasif Mandvi writes in his book
\href{https://www.amazon.com/No-Lands-Man-Perilous-Journey/dp/B00OZ3QUB6}{``No
Land's Man''} that his father brought his family to the United States
because of brunch.)

I was optimistic about how the dinner at my mother's apartment would go,
and it started off well. Michelle brought flowers. But my hope was
misplaced.

The dinner was mostly me being a nervous chatterbox, trying in vain to
spark conversation between two people with little in common. My mother
didn't talk much, if at all. She grew up in a different culture and a
different generation. She was too polite to say it, but I know she
wanted my partner to be Indian, like us. She wanted someone who
understood her world. That wasn't Michelle, or Sharon, or anyone else I
had chosen.

\includegraphics{https://static01.nyt.com/images/2017/07/24/arts/24SOPANBIGSICKJP3/22SOPANBIGSICK6-articleInline.jpg?quality=75\&auto=webp\&disable=upscale}

Crossing the cultural divide can be lonely, particularly when you're
growing up in a mostly white town. Especially when few television shows
and films tell stories of people who look like you.

``The Big Sick'' is a welcome exception.

The romantic complications of South Asian children who grow up here have
rarely been displayed as vividly as in this film, which tells the story
of a Pakistani comic and Uber driver in a relationship with a grad
student, who is white.
\href{https://www.nytimes.com/2017/06/22/movies/the-big-sick-review-kumail-nanjiani.html}{Starring
and co-written by Kumail Nanjiani}, who was born in Karachi, it explores
the South Asian identity in depth, and speaks to conflicts that many of
us face growing up in America.

The film is about Mr. Nanjiani's real-life courtship, breakup and
eventual marriage to Emily V. Gordon, his wife and co-author (played by
Zoe Kazan). And even though no one I've dated fell into a coma, as Ms.
Gordon did, Mr. Nanjiani's struggle was a recognizable one.

He wasn't the first to tell a uniquely South Asian story, of course: Kal
Penn explored similar themes in the 2007 film
\href{http://www.nytimes.com/2007/03/09/movies/09name.html}{``The
Namesake,''} based on the
\href{http://www.nytimes.com/2003/09/02/books/books-of-the-times-from-calcutta-to-suburbia-a-family-s-perplexing-journey.html}{book
by Jhumpa Lahiri}. Ravi Patel lays this conflict bare in his 2015
documentary,
\href{https://www.nytimes.com/2015/09/11/movies/review-in-meet-the-patels-a-son-submits-to-a-marriage-quest.html}{``Meet
the Patels,''} in which he allows his parents to arrange a marriage for
him, at the expense of his true love at the time, a writer named Audrey
Wauchope. And there is Aziz Ansari, who stars in Netflix's ``Master of
None'' and dives into this subject in
\href{https://www.nytimes.com/2017/05/25/watching/master-of-none-critics-roundup.html}{episodic
format}.

Image

Zuleikha Robinson and Kal Penn in the film ``The
Namesake.''Credit...Abbot Genser/Fox Searchlight Pictures

But a number of South Asian women have expressed a reaction completely
different from mine, seeing ``The Big Sick'' as yet another movie that
portrays South Asian women as inherently less desirable.

For the website Jezebel, the Brooklyn artist Aditi Natasha Kini wrote a
\href{http://themuse.jezebel.com/i-m-tired-of-watching-brown-men-fall-in-love-with-white-1796522590}{critique}
of the film, titled ``I'm Tired of Watching Brown Men Fall in Love With
White Women Onscreen.'' On Vice,
\href{https://www.vice.com/en_us/article/zmvmp3/the-big-sick-is-great-and-its-also-stereotypical-toward-brown-women}{Amil
Niazi wrote}, ``I found myself growing increasingly frustrated and then
infuriated with the clichéd, stereotypical depictions of South Asian
women that have unfortunately become the norm in the growing onscreen
narratives of brown men.''

Tanzila Ahmed, writing for The Aerogram, a South Asian culture site,
\href{http://theaerogram.com/big-sick-brown-romance/?utm_content=buffer02d20\&utm_medium=social\&utm_source=facebook.com\&utm_campaign=buffer}{summed
up the critique this way}: ``Once again, Muslim Brown women were crafted
as undesirable, conventional and unmarriageable for the Modern
Muslim-ish Male.''

The word ``erasure'' comes up frequently in the criticism, and it's
neither new nor unfounded. Beyond a few notable exceptions --- such as
Mindy Kaling in ``The Mindy Project,'' and
\href{https://www.theatlantic.com/entertainment/archive/2017/07/all-the-brown-girls-on-tv/530184/}{``Brown
Girls,'' a coming HBO show} featuring a Pakistani-American lead ---
there isn't enough representation of brown women onscreen. Mr. Ansari
faced that complaint after ``Master of None'' was released; his
character often pursues white women.

Image

Mindy Kaling in ``The Mindy Project.''Credit...Jordin Althaus/NBC

``No ethnic requirements on any casting. We just cast the best people,''
\href{https://twitter.com/azizansari?ref_src=twsrc\%5Etfw\&ref_url=http\%3A\%2F\%2Fwww.papermag.com\%2Faziz-ansari-master-of-none-1449383127.html}{he
responded on Twitter}.

Ms. Ahmed also took issue with a dynamic in several American movies
featuring South Asian men: ``Why does there always need to be a white
leading woman? Are we unable to tell Brown romantic narratives without
grounding them in Whiteness?''

On one level, the evidence is clear: Many movies about interracial
relationships feature white women in the lead role (and rarely have
South Asian characters at all). From
\href{http://www.nytimes.com/movie/review?res=9C03E6DE1430E23BBC4A52DFB467838C679EDE}{``Guess
Who's Coming to Dinner''} in 1967 to
``\href{http://www.nytimes.com/movie/review?res=9901E7DC1F3AF931A25752C0A9679C8B63}{Save
the Last Dance}'' in 2001, films with interracial leads often feature a
white woman with a man of color, exploring their relationship almost
exclusively through a racial lens. Some casting has bucked this trend
--- think of Halle Berry's Oscar-winning performance in
\href{http://www.nytimes.com/2001/12/26/movies/film-review-courtesy-and-decency-play-sneaky-with-a-tough-guy.html}{``Monster's
Ball,''} or Zoe Saldana in the remake
\href{http://www.nytimes.com/2005/03/25/movies/shedding-racial-prejudices-but-not-old-ideas-of-virtue.html}{``Guess
Who''} --- but there is a clear pattern of idealizing white women that
often comes at the expense of women of color.

For me, though, ``The Big Sick'' is the wrong target for this
frustration.

Stories of South Asian culture are rarely given voice. We certainly need
more of these, particularly from female writers and producers. But here,
it is Mr. Nanjiani's story being told --- not that of a roomful of white
writers looking to throw some diversity into their screenplay. He is
married to a white woman. The story is about their courtship.

I didn't see ``The Big Sick'' as a rejection of South Asian women, but
rather a rejection of arranged marriage, a difficult and searing subject
for some of us who have experienced it up close.

My parents were a terrible match. But they stayed together because
arranged marriages are often transactional at first; the love, in
theory, comes later. That never happened for my parents. And it took
them 30 years to end the marriage, at least partially because divorce is
stigmatized in South Asian culture. (Estimates for the divorce rates
among Indian-Americans range from 1 to 15 percent.)

My reaction to this, admittedly flawed, was to reject the part of myself
that came from India. Searching for something to blame for my family
ills, I blamed arranged marriage. ``How could I subscribe to a culture
that forced these two together?'' I would often think, silently pulling
away from my roots as I entered adulthood.

Along the way, my mother did her best to keep me grounded. I took
classical Indian vocal lessons. We went to annual Hindu festivals to
offer prayers to various gods and goddesses. Sometimes I performed the
songs I learned, but it was never quite right. (In ``The Big Sick,'' Mr.
Nanjiani's character fake prays to please his parents. Been there.) I
never felt at home. And my parents, especially my mother, felt a
profound sense of loss.

Image

The writer (far left) with his father and brother.Credit...Sopan Deb

This is what I have struggled with in my 20s as I consider my brownness.
When it came to marriage, I've thought: ``If I say no to my Indian
roots, I won't have the marriage my parents had. I won't be like that
with my kids.''

Of course, Mr. Nanjiani, Mr. Patel and many, many other South Asian
children who grew up in the United States didn't have an experience like
mine. For many parents, the love \emph{did} come later --- or they were
not in arranged marriages at all.

Image

Ravi Patel, left, and members of his family in ``Meet the Patels,'' a
documentary film directed by Mr. Patel and his sister, Geeta Patel
(below center).Credit...Alchemy

The critique of ``The Big Sick'' as contributing to stereotypes of South
Asian women is surely understandable. Mr. Nanjiani's mother, played by
\href{http://www.imdb.com/name/nm2801210/?ref_=tt_cl_t6}{Zenobia
Shroff}, lines up women for Mr. Nanjiani, hoping that he will find one
of them suitable for marriage. At least one of the choices, played by an
actress with an ethnic accent, can be reasonably seen as a caricature,
as she tries overly hard to impress Mr. Nanjiani as a fan of ``The
X-Files.''

The inherent awkwardness of an arranged first date around the family
dinner table is highlighted. In my eyes, the point wasn't to relegate
South Asian women to a punch line, but to add levity to a story in which
Mr. Nanjiani struggles with a choice that could isolate him from his
family.

When my brother, Sattik, who is 10 years older than I am, married Erica,
a white woman, they had a Catholic ceremony.

My mother was devastated. She wanted an Indian wedding. I asked why it
was important.

``Because we are Indian. I am Indian,'' my mother said. South Asian
weddings, generally, are about a marriage of two families, rather than
two individuals. My brother's wedding went on, but my mother never fully
embraced the nuptials.

``Shambo, I want you to have an Indian wedding to an Indian girl,'' my
mother said to me, using a childhood nickname. I was 19 at the time and
thought, ``Not gonna happen.''

I was frustrated and baffled by her mind-set. I am sure she felt the
same about mine.

I have since dated South Asian women and been very happy. I may very
well marry one and have the wedding my mother wants for me. But if I do,
it will be because the woman is someone I want to be with, and who wants
to be with me.

Mr. Nanjiani's film explored that freedom to choose --- one brown guy's
experience crossing the chasm between two very different cultures.
Faulting him for telling his story feels like a kind of erasure, too. I
am also not going to deprive Ms. Gordon her due, because it is her story
as well.

Rather than criticizing the film for what it is not, I appreciate what
it offers: a clear, illuminating reflection of the world that I grew up
in, a world that few outsiders see.

And on the other side, my parents have gradually inched across the
divide, accepting more of the cultural differences, as Mr. Nanjiani's
parents ultimately did.

They have even asked --- ever so gingerly --- who I am dating.

Advertisement

\protect\hyperlink{after-bottom}{Continue reading the main story}

\hypertarget{site-index}{%
\subsection{Site Index}\label{site-index}}

\hypertarget{site-information-navigation}{%
\subsection{Site Information
Navigation}\label{site-information-navigation}}

\begin{itemize}
\tightlist
\item
  \href{https://help.nytimes.com/hc/en-us/articles/115014792127-Copyright-notice}{©~2020~The
  New York Times Company}
\end{itemize}

\begin{itemize}
\tightlist
\item
  \href{https://www.nytco.com/}{NYTCo}
\item
  \href{https://help.nytimes.com/hc/en-us/articles/115015385887-Contact-Us}{Contact
  Us}
\item
  \href{https://www.nytco.com/careers/}{Work with us}
\item
  \href{https://nytmediakit.com/}{Advertise}
\item
  \href{http://www.tbrandstudio.com/}{T Brand Studio}
\item
  \href{https://www.nytimes.com/privacy/cookie-policy\#how-do-i-manage-trackers}{Your
  Ad Choices}
\item
  \href{https://www.nytimes.com/privacy}{Privacy}
\item
  \href{https://help.nytimes.com/hc/en-us/articles/115014893428-Terms-of-service}{Terms
  of Service}
\item
  \href{https://help.nytimes.com/hc/en-us/articles/115014893968-Terms-of-sale}{Terms
  of Sale}
\item
  \href{https://spiderbites.nytimes.com}{Site Map}
\item
  \href{https://help.nytimes.com/hc/en-us}{Help}
\item
  \href{https://www.nytimes.com/subscription?campaignId=37WXW}{Subscriptions}
\end{itemize}
