Sections

SEARCH

\protect\hyperlink{site-content}{Skip to
content}\protect\hyperlink{site-index}{Skip to site index}

\href{https://www.nytimes.com/section/politics}{Politics}

\href{https://myaccount.nytimes.com/auth/login?response_type=cookie\&client_id=vi}{}

\href{https://www.nytimes.com/section/todayspaper}{Today's Paper}

\href{/section/politics}{Politics}\textbar{}Anthony Scaramucci's
Uncensored Rant: Foul Words and Threats to Have Priebus Fired

\url{https://nyti.ms/2tLXZiV}

\begin{itemize}
\item
\item
\item
\item
\item
\item
\end{itemize}

Advertisement

\protect\hyperlink{after-top}{Continue reading the main story}

Supported by

\protect\hyperlink{after-sponsor}{Continue reading the main story}

\hypertarget{anthony-scaramuccis-uncensored-rant-foul-words-and-threats-to-have-priebus-fired}{%
\section{Anthony Scaramucci's Uncensored Rant: Foul Words and Threats to
Have Priebus
Fired}\label{anthony-scaramuccis-uncensored-rant-foul-words-and-threats-to-have-priebus-fired}}

\includegraphics{https://static01.nyt.com/images/2017/07/29/us/28dc-scaramucci_web2/28dc-scaramucci_web2-videoSixteenByNine3000.jpg}

By \href{http://www.nytimes.com/by/peter-baker}{Peter Baker} and
\href{http://www.nytimes.com/by/maggie-haberman}{Maggie Haberman}

\begin{itemize}
\item
  July 27, 2017
\item
  \begin{itemize}
  \item
  \item
  \item
  \item
  \item
  \item
  \end{itemize}
\end{itemize}

WASHINGTON --- When Anthony Scaramucci, the new White House
communications director, went on television on Thursday morning to
compare himself and his adversary, Reince Priebus, the chief of staff,
to Cain and Abel, it seemed to encapsulate the fratricidal nature of an
administration riven by biblical rivalries. Cain, after all, killed Abel
as they vied for God's favor.

As it turned out, that was the cleaned-up version. In a vulgarity-laced
telephone call with a New Yorker writer
\href{http://www.newyorker.com/news/ryan-lizza/anthony-scaramucci-called-me-to-unload-about-white-house-leakers-reince-priebus-and-steve-bannon}{reported
on the magazine's website} on Thursday evening, Mr. Scaramucci railed
against Mr. Priebus and Stephen K. Bannon, the president's chief
strategist, both of whom opposed his hiring last week. He even vowed to
get the chief of staff fired. ``Reince Priebus --- if you want to leak
something --- he'll be asked to resign very shortly,'' Mr. Scaramucci
said.

Whether Mr. Scaramucci will turn out to be Cain or Abel, it was clear
that his appointment has added another layer of drama and dissent to a
White House suffused in it --- and revived the perpetual questions about
Mr. Priebus's fate. Sean Spicer, the White House press secretary and an
ally of Mr. Priebus, resigned in protest when Mr. Scaramucci was hired
last week because, he predicted, it would only add more chaos to the
team. On that, at least, he seems to have been proved right.

But President Trump not only tolerates feuds within his team, he fuels
them, playing one courtier off another and leaving them all unsteady. He
chooses favorites and casts others aside, but even those decisions seem
subject to change at any moody moment. And by several accounts, he
personally encouraged Mr. Scaramucci's jihad against Mr. Priebus, once
again subjecting his chief of staff to a ritualistic public lashing even
as he considered pushing him out.

Left to explain all this was Sarah Huckabee Sanders, the new White House
press secretary. ``This is a White House that has lots of different
perspectives because the president hires the very best people,'' she
said gamely, before the New Yorker article posted, asserting that a
``healthy competition'' benefits Mr. Trump. ``With that competition, you
usually get the best results. The president likes that kind of
competition and encourages it.''

That kind of competition has exhausted even some of Mr. Trump's most
loyal defenders. But Mr. Trump has openly told people that he has lost
faith in Mr. Priebus. He has said he wants ``a general'' as chief of
staff, and has focused on John F. Kelly, the retired four-star Marine
now serving as homeland security secretary. Many of his advisers,
however, consider that a bad idea.

Mr. Scaramucci, who has so emulated Mr. Trump's style that colleagues
privately call him ``Mini-Me,'' made clear in his conversation with The
New Yorker's Ryan Lizza that he is trying to push Mr. Priebus out.
``Reince is a fucking paranoid schizophrenic, a paranoiac,'' he said.
Mr. Scaramucci complained that Mr. Priebus had prevented him from
getting a job in the White House until now, saying he ``blocked
Scaramucci for six months.''

In the same telephone call, Mr. Scaramucci disparaged Mr. Bannon. ``I'm
not Steve Bannon. I'm not trying to suck my own cock,'' he said. ``I'm
not trying to build my own brand'' on the president's coattails.

``I'm here to serve the country,'' he added.

Mr. Priebus finds himself isolated inside the White House. He has lost
the support of Mr. Trump's family, and other senior aides have long
bristled at his demeanor or suspected he was trying to undermine them.
Allies like Mr. Spicer are gone or leaving. And some complain that Mr.
Priebus used the White House communications office as his own personal
fief.

Lately Mr. Trump has resumed subjecting him to frequent indignities in
front of the White House staff. According to one aide, the president,
who had ceased for a time, has regularly mentioned how Mr. Priebus
suggested that Mr. Trump consider dropping out of the presidential race
last October after a tape of him boasting about grabbing women by the
genitals emerged. ``Do you remember when Reince did that?'' the
president has asked associates. The issue has always been a sore spot
between the two men.

Mr. Priebus endured the hazing in silence, as he generally has, and the
White House did nothing to defend him against Mr. Scaramucci's tirade.
Mr. Scaramucci released a statement after the New Yorker article was
published that fell well short of an apology.

``I sometimes use colorful language,''
\href{https://twitter.com/Scaramucci/status/890699212022796288}{he said
on Twitter}. ``I will refrain in this arena but not give up the
passionate fight for @RealDonaldTrump's agenda.''

Ms. Sanders said mildly that Mr. Scaramucci was simply expressing strong
feelings, and that his statement made clear that ``he's a passionate guy
and sometimes he lets that passion get the better of him.'' She added,
``I don't think he'll do it again.''

But later in the evening, Mr. Scaramucci shifted blame. ``I made a
mistake in trusting in a reporter,'' he
\href{https://twitter.com/Scaramucci/status/890736308498378753}{wrote on
Twitter}. ``It won't happen again.'' Mr. Lizza wrote that Mr. Scaramucci
never asked to be off the record.

Mr. Priebus's plight was good news for another member of the Trump team.
For the first time in a week, it was not Attorney General Jeff
Sessions's turn to be the presidential punching bag.

During a visit to El Salvador, Mr. Sessions acknowledged
\href{https://www.apnews.com/19a4a6ac62064554ac82a08567d1ef31/Sessions-tells-AP-he's-not-stepping-down-unless-asked}{to
The Associated Press} that ``it hasn't been my best week'' in his
``relationship with the president.'' Speaking to Fox News, he added,
``It's kind of hurtful, but the president of the United States is a
strong leader. He is determined to move this country in the direction
that he believes it needs to go to make it great again.''

So many figures inside Mr. Trump's orbit have been declared on their way
out that it takes a scorecard to keep track. Aside from Mr. Priebus and
Mr. Sessions, many wonder about the future of Lt. Gen. H. R. McMaster,
the national security adviser whose Afghanistan war plan was rejected by
the president last week. Secretary of State Rex W. Tillerson disappeared
for a few days off, stoking speculation that he may leave. (``Rexit,''
it was called on Twitter.) And the president, who has already fired one
F.B.I. director, this week called for the acting head of the bureau to
be dismissed too.

The clash between Mr. Scaramucci and Mr. Priebus offers a case study in
how the Trump White House operates, a conflict divorced from facts,
untethered from the basics of how government works, enabled by the lack
of any organizational structure and driven by ambition, fear, animosity
and envy.

The genesis was a dinner hosted Wednesday night by Mr. Trump at the
White House that included Mr. Scaramucci; Sean Hannity and Kimberly
Guilfoyle, the Fox News hosts; and Bill Shine, a former Fox executive.

Ms. Guilfoyle told the president that Mr. Priebus was a problem and a
leaker, someone who was not serving his agenda, according to a person
briefed on the conversation. (A Fox spokesman did not respond to a
request for comment.)

Mr. Scaramucci grew angry afterward that Mr. Lizza had learned that the
dinner was taking place and that
\href{http://www.politico.com/story/2017/07/26/scaramucci-trump-skybridge-profits-241006}{Politico
had obtained his government financial disclosure form}. At that point,
he called Mr. Lizza, demanding to know his source, whom the reporter
refused to divulge.

``O.K., I'm going to fire every one of them, and then you haven't
protected anybody, so the entire place will be fired over the next two
weeks,'' Mr. Scaramucci replied.

After hanging up, Mr. Scaramucci posted a message on Twitter asserting
that the ``leak'' of his disclosure form was a ``felony'' and that he
would seek an F.B.I. investigation. He included Mr. Priebus's Twitter
handle, a move that was interpreted as blaming the chief of staff.

But it was no leak. The disclosure form is supposed to be made public
under federal law and all Politico did was ask for it under normal
procedures. Mr. Scaramucci deleted the tweet. But on Thursday morning,
he called into CNN with Mr. Trump's encouragement, and threw down the
gauntlet with Mr. Priebus on live television.

``We have had odds. We have had differences,'' Mr. Scaramucci
\href{https://twitter.com/NewDay/status/890532262030213120}{said on
CNN}. ``When I said we were brothers from the podium, that's because
we're rough on each other. Some brothers are like Cain and Abel. Other
brothers can fight with each other and get along. I don't know if this
is reparable or not. That will be up to the president.''

Some of Mr. Trump's supporters said Mr. Scaramucci was causing more harm
than good.

``I would say right now that he's being more pugnacious than
effective,'' Newt Gingrich, the former House speaker, told the radio
host Laura Ingraham. ``I think he ought to slow down a little bit and
learn what he's doing.''

Advertisement

\protect\hyperlink{after-bottom}{Continue reading the main story}

\hypertarget{site-index}{%
\subsection{Site Index}\label{site-index}}

\hypertarget{site-information-navigation}{%
\subsection{Site Information
Navigation}\label{site-information-navigation}}

\begin{itemize}
\tightlist
\item
  \href{https://help.nytimes.com/hc/en-us/articles/115014792127-Copyright-notice}{©~2020~The
  New York Times Company}
\end{itemize}

\begin{itemize}
\tightlist
\item
  \href{https://www.nytco.com/}{NYTCo}
\item
  \href{https://help.nytimes.com/hc/en-us/articles/115015385887-Contact-Us}{Contact
  Us}
\item
  \href{https://www.nytco.com/careers/}{Work with us}
\item
  \href{https://nytmediakit.com/}{Advertise}
\item
  \href{http://www.tbrandstudio.com/}{T Brand Studio}
\item
  \href{https://www.nytimes.com/privacy/cookie-policy\#how-do-i-manage-trackers}{Your
  Ad Choices}
\item
  \href{https://www.nytimes.com/privacy}{Privacy}
\item
  \href{https://help.nytimes.com/hc/en-us/articles/115014893428-Terms-of-service}{Terms
  of Service}
\item
  \href{https://help.nytimes.com/hc/en-us/articles/115014893968-Terms-of-sale}{Terms
  of Sale}
\item
  \href{https://spiderbites.nytimes.com}{Site Map}
\item
  \href{https://help.nytimes.com/hc/en-us}{Help}
\item
  \href{https://www.nytimes.com/subscription?campaignId=37WXW}{Subscriptions}
\end{itemize}
