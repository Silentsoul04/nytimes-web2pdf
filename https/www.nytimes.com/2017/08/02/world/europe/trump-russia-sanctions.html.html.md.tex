Sections

SEARCH

\protect\hyperlink{site-content}{Skip to
content}\protect\hyperlink{site-index}{Skip to site index}

\href{https://www.nytimes.com/section/world/europe}{Europe}

\href{https://myaccount.nytimes.com/auth/login?response_type=cookie\&client_id=vi}{}

\href{https://www.nytimes.com/section/todayspaper}{Today's Paper}

\href{/section/world/europe}{Europe}\textbar{}Trump Signs Russian
Sanctions Into Law, With Caveats

\url{https://nyti.ms/2hnV4qz}

\begin{itemize}
\item
\item
\item
\item
\item
\item
\end{itemize}

Advertisement

\protect\hyperlink{after-top}{Continue reading the main story}

Supported by

\protect\hyperlink{after-sponsor}{Continue reading the main story}

\hypertarget{trump-signs-russian-sanctions-into-law-with-caveats}{%
\section{Trump Signs Russian Sanctions Into Law, With
Caveats}\label{trump-signs-russian-sanctions-into-law-with-caveats}}

\includegraphics{https://static01.nyt.com/images/2017/08/02/us/03dc-prexy/03dc-prexy-videoSixteenByNine3000.jpg}

By \href{http://www.nytimes.com/by/peter-baker}{Peter Baker} and
\href{https://www.nytimes.com/by/sophia-kishkovsky}{Sophia Kishkovsky}

\begin{itemize}
\item
  Aug. 2, 2017
\item
  \begin{itemize}
  \item
  \item
  \item
  \item
  \item
  \item
  \end{itemize}
\end{itemize}

WASHINGTON ---
\href{https://www.nytimes.com/topic/person/donald-trump}{President
Trump} signed legislation on Wednesday imposing sanctions on Russia and
limiting his own authority to lift them, but asserted that the measure
included ``clearly unconstitutional provisions'' and left open the
possibility that he might choose not to enforce them as lawmakers
intended.

The legislation, which also includes sanctions on
\href{https://www.nytimes.com/topic/destination/iran}{Iran} and
\href{https://www.nytimes.com/topic/destination/north-korea}{North
Korea}, represented the first time that Congress had forced Mr. Trump to
sign a bill over his objections by passing it with bipartisan,
veto-proof majorities. Even before he signed it, the
\href{https://www.nytimes.com/2017/07/30/world/europe/russia-sanctions-us-diplomats-expelled.html}{Russian
government retaliated} by seizing two American diplomatic properties and
ordering the United States to reduce its embassy staff members in Russia
by 755 people.

The measure reflected deep skepticism among lawmakers in both parties
about Mr. Trump's friendly approach to President
\href{https://www.nytimes.com/topic/person/vladimir-putin}{Vladimir V.
Putin} of Russia and an effort to prevent Mr. Trump from letting the
Kremlin off the hook for its
\href{https://www.nytimes.com/2014/03/19/world/europe/ukraine.html}{annexation
of Crimea}, military intervention in Ukraine and its
\href{https://www.nytimes.com/2016/12/09/us/obama-russia-election-hack.html}{meddling
in last year's} American election. Rather than the rapprochement Mr.
Trump once envisioned, the United States and Russia now seem locked in a
spiral of increasing tension.

Unlike other bill signings, Mr. Trump did not invite news media
photographers to record the event, nor did he say anything about it to
reporters. He ignored questions about the legislation at an unrelated
event and instead relegated his comments to two written statements, one
meant for Congress to describe caveats in his approval of the bill and
the other issued to reporters to explain his grudging decision to sign.

As other presidents have in the past, Mr. Trump protested that Congress
was improperly interfering with his power to set foreign policy, in this
case by imposing waiting periods before he can suspend or remove
sanctions first imposed by former President
\href{https://www.nytimes.com/topic/person/barack-obama}{Barack Obama}
while Congress reviews and potentially blocks such a move.

In the statement to Congress, Mr. Trump said the bill ``included a
number of clearly unconstitutional provisions.'' Although he added that
``I nevertheless expect to honor'' the waiting periods, he did not
commit to it. Moreover, he took issue with other provisions, saying only
that he ``will give careful and respectful consideration to the
preferences expressed by the Congress.''

``This bill remains seriously flawed --- particularly because it
encroaches on the executive branch's authority to negotiate,'' Mr. Trump
said in the separate statement to reporters. ``Congress could not even
negotiate a health care bill after seven years of talking. By limiting
the executive's flexibility, this bill makes it harder for the United
States to strike good deals for the American people and will drive
\href{https://www.nytimes.com/topic/destination/china}{China}, Russia
and North Korea much closer together.''

``Yet despite its problems,'' he added, ``I am signing this bill for the
sake of national unity. It represents the will of the American people to
see Russia take steps to improve relations with the United States. We
hope there will be cooperation between our two countries on major global
issues so that these sanctions will no longer be necessary.''

Like Mr. Trump, who has offered no public comment or even a Twitter
message about the Russian order to slash the number of United States
Embassy workers, it appears that Mr. Putin has not completely given up
on the idea of establishing closer relations. The Russian government
took its retaliatory action before the president signed the bill so that
it would be a response to Congress, not to Mr. Trump.

After Mr. Trump signed the measure on Wednesday, the Russian government
reaction was mild. ``De facto, this changes nothing,'' said Dmitri S.
Peskov, the Kremlin press secretary, who was traveling with Mr. Putin in
the Russian Far East, according to the Interfax news agency. ``There is
nothing new.''

He added that no new retaliation should be expected. ``Countermeasures
have already been taken,'' he said.

The Russian Foreign Ministry attributed the sanctions to ``Russophobic
hysteria'' and reserved the right to take action if it decided to.
Vasily A. Nebenzya, the Russian ambassador to the United Nations, said
the law would do nothing to change Moscow's policies. ``Those who
invented this bill, if they were thinking that they might change our
policy, they were wrong,'' \href{http://tass.com/politics/958818}{he
told reporters}. ``As history many times proved, they should have known
better that we do not bend, we do not break.''

Dmitri A. Medvedev, the Russian prime minister, declared the ``end to
hope for the improvement of our relations'' and mocked Mr. Trump for
being forced to sign. ``The Trump administration has demonstrated total
impotence, handing over executive functions to Congress in the most
humiliating way possible,'' he wrote on Facebook. He added that ``the
American establishment has totally outplayed Trump'' with the goal ``to
remove him from power.''

American lawmakers said the new law sent an important signal that Russia
would be held to account for its election interference and aggression
toward its neighbors. But the lawmakers expressed concern about whether
Mr. Trump would try to sidestep the measure.

The president's signing statement ``demonstrates that Congress is going
to need to keep a sharp eye on this administration's implementation of
this critical law and any actions it takes with respect to Ukraine,''
said Senator Chuck Schumer of New York, the Democratic minority leader.

Senator Benjamin L. Cardin of Maryland, the senior Democrat on the
Foreign Relations Committee and a prime driver behind the legislation,
said, ``I remain very concerned that this administration will seek to
strike a deal with Moscow that is not in the national security interests
of the United States.''

The Trump administration continues to send mixed messages about Russia.

Vice President
\href{http://topics.nytimes.com/top/reference/timestopics/people/p/mike_pence/index.html}{Mike
Pence}, who has been visiting Eastern Europe in recent days to shore up
allies nervous about an assertive Kremlin,
\href{https://www.nytimes.com/2017/08/02/world/europe/pence-montenegro-markovic-nato.html?ref=topics}{told
a group} of Balkan prime ministers on Wednesday that Russia sought ``to
redraw international borders by force'' and ``undermine your
democracies.''

``The United States will continue to hold Russia accountable for its
actions, and we call on our European allies and friends to do the
same,'' he said in
\href{https://www.nytimes.com/topic/destination/montenegro?inline=nyt-geo}{Montenegro},
the latest Eastern European nation to join NATO. He noted that the
president would sign the sanctions legislation.

``Let me be clear: The United States prefers a constructive relationship
with Russia based on mutual cooperation and common interests,'' Mr.
Pence said. ``But the president and our Congress are unified in our
message to Russia: A better relationship and the lifting of sanctions
will require Russia to reverse the actions and conduct that caused
sanctions to be imposed in the first place.''

But just a day earlier, Secretary of State Rex W. Tillerson offered a
somewhat different take, focusing on the potential for cooperation with
Russia in fighting the Islamic State and finding a resolution to the
civil war in
\href{https://www.nytimes.com/topic/destination/syria}{Syria}. Rather
than sounding unified with Congress, Mr. Tillerson complained that
lawmakers should not have passed the sanctions legislation.

``The action by the Congress to put these sanctions in place and the way
they did, neither the president nor I are very happy about that,'' he
told reporters on Tuesday. ``We were clear that we didn't think it was
going to be helpful to our efforts, but that's the decision they made.
They made it in a very overwhelming way. I think the president accepts
that.''

Advertisement

\protect\hyperlink{after-bottom}{Continue reading the main story}

\hypertarget{site-index}{%
\subsection{Site Index}\label{site-index}}

\hypertarget{site-information-navigation}{%
\subsection{Site Information
Navigation}\label{site-information-navigation}}

\begin{itemize}
\tightlist
\item
  \href{https://help.nytimes.com/hc/en-us/articles/115014792127-Copyright-notice}{©~2020~The
  New York Times Company}
\end{itemize}

\begin{itemize}
\tightlist
\item
  \href{https://www.nytco.com/}{NYTCo}
\item
  \href{https://help.nytimes.com/hc/en-us/articles/115015385887-Contact-Us}{Contact
  Us}
\item
  \href{https://www.nytco.com/careers/}{Work with us}
\item
  \href{https://nytmediakit.com/}{Advertise}
\item
  \href{http://www.tbrandstudio.com/}{T Brand Studio}
\item
  \href{https://www.nytimes.com/privacy/cookie-policy\#how-do-i-manage-trackers}{Your
  Ad Choices}
\item
  \href{https://www.nytimes.com/privacy}{Privacy}
\item
  \href{https://help.nytimes.com/hc/en-us/articles/115014893428-Terms-of-service}{Terms
  of Service}
\item
  \href{https://help.nytimes.com/hc/en-us/articles/115014893968-Terms-of-sale}{Terms
  of Sale}
\item
  \href{https://spiderbites.nytimes.com}{Site Map}
\item
  \href{https://help.nytimes.com/hc/en-us}{Help}
\item
  \href{https://www.nytimes.com/subscription?campaignId=37WXW}{Subscriptions}
\end{itemize}
