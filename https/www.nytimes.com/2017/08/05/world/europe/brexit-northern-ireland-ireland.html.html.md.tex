Sections

SEARCH

\protect\hyperlink{site-content}{Skip to
content}\protect\hyperlink{site-index}{Skip to site index}

\href{https://www.nytimes.com/section/world/europe}{Europe}

\href{https://myaccount.nytimes.com/auth/login?response_type=cookie\&client_id=vi}{}

\href{https://www.nytimes.com/section/todayspaper}{Today's Paper}

\href{/section/world/europe}{Europe}\textbar{}On Irish Border, Worries
That `Brexit' Will Undo a Hard-Won Peace

\url{https://nyti.ms/2utPO68}

\begin{itemize}
\item
\item
\item
\item
\item
\end{itemize}

Advertisement

\protect\hyperlink{after-top}{Continue reading the main story}

Supported by

\protect\hyperlink{after-sponsor}{Continue reading the main story}

\hypertarget{on-irish-border-worries-that-brexit-will-undo-a-hard-won-peace}{%
\section{On Irish Border, Worries That `Brexit' Will Undo a Hard-Won
Peace}\label{on-irish-border-worries-that-brexit-will-undo-a-hard-won-peace}}

\includegraphics{https://static01.nyt.com/images/2017/07/29/world/29borders2/29borders2-articleLarge.jpg?quality=75\&auto=webp\&disable=upscale}

By \href{http://www.nytimes.com/by/sarah-lyall}{Sarah Lyall}

\begin{itemize}
\item
  Aug. 5, 2017
\item
  \begin{itemize}
  \item
  \item
  \item
  \item
  \item
  \end{itemize}
\end{itemize}

LONDONDERRY, Northern Ireland --- Crossing the border between Northern
Ireland and the Irish Republic used to involve delays, checkpoints,
bureaucratic harassment and the lurking threat of violence. That it's
now virtually seamless --- that you can drive across without even
knowing it --- feels close to miraculous.

It is also one of the great successes of the Irish peace process of the
last several decades. ``It was like you had to climb over a locked
gate,'' George Fleming, the president of the Londonderry Chamber of
Commerce, said in an interview. ``And then someone came and opened the
gate.''

But as with so many
\href{https://www.nytimes.com/2017/07/20/world/europe/uk-brexit-conservative-party.html?rref=collection\%2Fnewseventcollection\%2Fbritain-brexit-european-union\&action=click\&contentCollection=europe\&region=stream\&module=stream_unit\&version=latest\&contentPlacement=4\&pgtype=collection}{British-related
matters these days}, ``Brexit'' --- Britain's divorce from the European
Union --- has thrown this hard-won arrangement into jeopardy.

If the British government succeeds in extricating itself from the
European Union, the two parts of Ireland will lose one of their most
important connective threads: a shared membership in the bloc. In an
instant, one part of the island would be in Europe, and the other would
not.

Established nearly 100 years ago according to political expedience
rather than natural logic, the border --- some 300 miles long, with
about 210 crossings --- is not easy to control, police or even always
identify. (Many of the crossings are on tiny back roads.)

Reinstating a hard border, as residents call it, would have both
psychological and practical implications. The movement of goods and
services between north and south, now commonplace and easy, would become
far more complicated with the introduction of new tariffs and customs
regulations.

\includegraphics{https://static01.nyt.com/images/2017/07/29/world/29borders1/29borders1-articleLarge.jpg?quality=75\&auto=webp\&disable=upscale}

There are fears, too, about the return of armed guards and checkpoints,
a resurgence of smuggling and other types of lawlessness, and a renewal
of violence from dissident Irish republicans bound to chafe at signs of
British control at the crossings.

Northern Ireland voted against Brexit in last year's referendum. Polls
show that for practicality's sake, a majority of people in the region,
whether they identify themselves as Irish or British, want the border to
remain porous and fluid.

``To reimpose the border is like putting up the Berlin Wall again, after
you've taken it down,'' said Mr. Fleming, whose farm equipment company
is based just two miles from the border.

He employs people from both north and south; does business in both north
and south (and abroad); and, along with some 325,000 other people per
week, regularly drives back and forth, too many times to count, between
the two places. His 96-year-old mother lives just across the border, in
the republic.

The island has been split in two since 1921 --- the north, part of the
United Kingdom and governed from London, and the south, a sovereign
nation governed from Dublin. Most of the United Kingdom-European Union
border is the waters of the English Channel; the only somewhat
comparable land border is between Spain and
\href{https://www.nytimes.com/2017/04/01/world/europe/gibraltar-brexit-spain-britain-european-union.html?rref=collection\%2Ftimestopic\%2FGibraltar\&action=click\&contentCollection=world\&region=stream\&module=stream_unit\&version=latest\&contentPlacement=2\&pgtype=collection}{the
British territory of Gibraltar}.

The British government has sought to reassure border residents that
their concerns are being heard. ``Nobody wants to return to the borders
of the past,''
\href{https://www.irishtimes.com/news/ireland/irish-news/may-nobody-wants-to-return-to-the-borders-of-the-past-1.2940087}{Prime
Minister Theresa May said in January}, pledging to maintain the
so-called Common Travel Area, which allows citizens of the United
Kingdom and the republic to travel back and forth without being subject
to passport controls.

Image

The pedestrian Peace Bridge over the River Foyle connects Londonderry's
mostly Catholic city center to the more Protestant Waterside
section.Credit...Paulo Nunes dos Santos for The New York Times

But Ms. May's words have convinced few people here. One of the prime
motivations for Brexit was Britain's desire to reestablish sovereignty
and retake control of its borders. People who live on either side of the
divide wonder how Britain can possibly expect to achieve both things ---
put in a new hard border with Europe while maintaining the current
openness.

They say, too, that easy statements from Westminster ignore hundreds of
years of complicated history and show a profound failure to understand
the intense emotions that Brexit has stirred up in a region scarred by
the past.

Londonderry, for instance, is a predominantly Catholic city in a
majority Protestant region with a long and bitter history of violent
sectarian conflict. Ancient problems can seem very close to the surface
here. But in recent years --- and most dramatically since the enactment,
in 1999, of the peace accord known as the Good Friday agreement --- the
city has made a remarkable turnaround.

Few people make a big deal now about the once life-or-death question of
what to call the city: Londonderry, its official name and the one
Protestants traditionally prefer, or Derry, the Catholics' favored name
and the one by which it is generally known. Reflecting that both sides
have a point, government organizations (and the BBC) have succumbed to
practicality and often write it as ``Derry/Londonderry.''

``There's no trouble here anymore,'' said Shauna McClenaghan, a civic
leader in Inishowen, a nearby area of the republic that is intimately
connected to Londonderry politically and culturally, despite being
across the international border. ``Derry's just a city.''

Gerry Lynn, an amateur historian who leads tours at the Guildhall, the
historic downtown building where the City Council meets, unleashed a
condensed version of more than 1,000 extremely complex years of Irish
history by way of explaining how far the country, and the region, have
come since the Troubles (not to mention the 1600s).

Image

Republican murals depicting the Troubles, on houses in the Bogside, a
traditionally Catholic neighborhood of Londonderry known for being the
site of some of the worst outbreaks of violence during that
conflict.Credit...Paulo Nunes dos Santos for The New York Times

``This city, this country, is like a woman who has given birth,'' Mr.
Lynn said. ``All the trauma, the pain and the fighting are over. We've
come out of the Troubles --- out of black and white and into color.''

Now buses full of tourists from China and South America pour in to
admire the 17th-century wall that surrounds the city, whose Protestant
residents are still proud that it was never breached by Catholic forces
during the Siege of Derry, in 1689. In 2013, the city became the
\href{http://www.bbc.com/news/uk-northern-ireland-foyle-west-20849679}{United
Kingdom's first City of Culture}.

In 2011, a pedestrian
\href{http://www.irelands-hidden-gems.com/derry-peace-bridge.html}{Peace
Bridge}, costing 14 million pounds, or about \$21.7 million, and
financed in large part by European money, was built over the River
Foyle, connecting the mostly Catholic city center to the more Protestant
Waterside section in the east.

``Everyone's so content with the peace we have here, and nobody really
makes too much fuss about the politics except the politicians,'' said
Daphne Wilson, 50, who was ambling across the bridge the other day.

Though she voted for Brexit --- ``We don't want pedophiles and
terrorists coming here'' --- she believes that free movement back and
forth has helped the two sides feel like part of a greater whole.

So does Toni Forrester, the chief executive of the chamber of commerce
in Letterkenny, County Donegal, next door in the republic. ``We've
worked so hard and so closely together to get cross-border cooperation
working,'' she said.

Image

The Bogside as seen from Derry Walls.Credit...Paulo Nunes dos Santos for
The New York Times

As an example, she mentioned a new medical-imaging center in Londonderry
that is open to patients from the republic. ``You can have a heart
attack in Donegal and be treated in Derry,'' she said.

Community leaders worry that much of the delicate progress of the last
couple of decades --- the softening of entrenched prejudices, the
gradual
\href{https://www.nytimes.com/2015/03/28/world/europe/using-flames-to-soothe-a-northern-ireland-city-scarred-by-fire.html}{moves
toward reconciliation} --- could be shattered by the reintroduction of
an us-versus-them mentality that a harder border would bring.

``This area benefits from E.U. funding, from peace programs that benefit
north and south promoting the notion that we have more in common than we
have differences,'' Ms. McClenaghan said.

Now 49 and joint chief executive of the Inishowen Development
Partnership, she grew up in Galway, in the republic, when the borders
were pockmarked with checkpoints and the roads patrolled by armed
officers.

``You'd see the army with their tanks and guns, and it was scary and
intimidating,'' she said. ``Passing the border, they'd always ask you
where you were going and where you were from.''

Ms. McClenaghan was chatting over a cup of coffee at a cafe in Bridgend,
at the southern end of the Inishowen peninsula. The border with Northern
Ireland was just down the road, near an intersection that already snarls
up and slows down at rush hour.

``What's going to happen to traffic if there's a hard border?'' she
asked.

Among other logistical awkwardnesses, the impractical way the island is
divided means that unless you take a three- or four-hour detour through
western Ireland, you cannot drive from Inishowen to Dublin without
crossing the border at least twice.

Back at the Guildhall, Mr. Lynn, the tour guide, said that having come
this far, people in the city had no desire to return to the way things
were before. ``History has to be history,'' he said. ``It has to be left
in the past.''

Advertisement

\protect\hyperlink{after-bottom}{Continue reading the main story}

\hypertarget{site-index}{%
\subsection{Site Index}\label{site-index}}

\hypertarget{site-information-navigation}{%
\subsection{Site Information
Navigation}\label{site-information-navigation}}

\begin{itemize}
\tightlist
\item
  \href{https://help.nytimes.com/hc/en-us/articles/115014792127-Copyright-notice}{©~2020~The
  New York Times Company}
\end{itemize}

\begin{itemize}
\tightlist
\item
  \href{https://www.nytco.com/}{NYTCo}
\item
  \href{https://help.nytimes.com/hc/en-us/articles/115015385887-Contact-Us}{Contact
  Us}
\item
  \href{https://www.nytco.com/careers/}{Work with us}
\item
  \href{https://nytmediakit.com/}{Advertise}
\item
  \href{http://www.tbrandstudio.com/}{T Brand Studio}
\item
  \href{https://www.nytimes.com/privacy/cookie-policy\#how-do-i-manage-trackers}{Your
  Ad Choices}
\item
  \href{https://www.nytimes.com/privacy}{Privacy}
\item
  \href{https://help.nytimes.com/hc/en-us/articles/115014893428-Terms-of-service}{Terms
  of Service}
\item
  \href{https://help.nytimes.com/hc/en-us/articles/115014893968-Terms-of-sale}{Terms
  of Sale}
\item
  \href{https://spiderbites.nytimes.com}{Site Map}
\item
  \href{https://help.nytimes.com/hc/en-us}{Help}
\item
  \href{https://www.nytimes.com/subscription?campaignId=37WXW}{Subscriptions}
\end{itemize}
