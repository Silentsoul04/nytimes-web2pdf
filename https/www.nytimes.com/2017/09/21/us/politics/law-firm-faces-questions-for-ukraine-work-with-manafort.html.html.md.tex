Sections

SEARCH

\protect\hyperlink{site-content}{Skip to
content}\protect\hyperlink{site-index}{Skip to site index}

\href{https://www.nytimes.com/section/politics}{Politics}

\href{https://myaccount.nytimes.com/auth/login?response_type=cookie\&client_id=vi}{}

\href{https://www.nytimes.com/section/todayspaper}{Today's Paper}

\href{/section/politics}{Politics}\textbar{}Skadden, Big New York Law
Firm, Faces Questions on Work With Manafort

\url{https://nyti.ms/2jOkhvL}

\begin{itemize}
\item
\item
\item
\item
\item
\item
\end{itemize}

Advertisement

\protect\hyperlink{after-top}{Continue reading the main story}

Supported by

\protect\hyperlink{after-sponsor}{Continue reading the main story}

\hypertarget{skadden-big-new-york-law-firm-faces-questions-on-work-with-manafort}{%
\section{Skadden, Big New York Law Firm, Faces Questions on Work With
Manafort}\label{skadden-big-new-york-law-firm-faces-questions-on-work-with-manafort}}

\includegraphics{https://static01.nyt.com/images/2017/09/22/us/22dc-skadden1/20dc-skadden1-articleLarge.jpg?quality=75\&auto=webp\&disable=upscale}

By \href{https://www.nytimes.com/by/kenneth-p-vogel}{Kenneth P. Vogel}
and \href{http://www.nytimes.com/by/andrew-e-kramer}{Andrew E. Kramer}

\begin{itemize}
\item
  Sept. 21, 2017
\item
  \begin{itemize}
  \item
  \item
  \item
  \item
  \item
  \item
  \end{itemize}
\end{itemize}

WASHINGTON --- Five years ago, Paul Manafort arranged for a prominent
New York-based law firm to draft a report that was used by allies of his
client, Viktor Yanukovych, the Russia-aligned president of Ukraine, to
justify the jailing of a political rival. And now the report is coming
back to haunt it.

The Justice Department, according to two people with direct knowledge of
the situation, recently asked the firm, Skadden, Arps, Slate, Meagher \&
Flom, for information and documents related to its work on behalf of Mr.
Yanukovych's government,
\href{https://www.nytimes.com/2014/02/23/world/europe/ukraine.html}{which
crumbled} after he
\href{https://www.nytimes.com/2016/11/25/world/europe/yanukovych-ukraine-maidan-protests-russia.html}{fled
to Russia} under pressure.

The request comes at a time when Mr. Manafort, his
\href{https://www.nytimes.com/2016/08/01/us/paul-manafort-ukraine-donald-trump.html?mcubz=3}{work
for Mr. Yanukovych's party} and for Russian and Ukrainian oligarchs as
well as the
\href{https://www.nytimes.com/2017/07/15/world/europe/ukraine-paul-manafort-viktor-yanukovych.html?mcubz=3}{handling
of payments} for that work have become focal points in the investigation
of the special counsel, Robert S. Mueller III, into Russian meddling in
the 2016 presidential election, and connections between Russia, Mr.
Trump and his associates.

It's unclear if the Justice Department's request to Skadden, as the firm
is known, is part of Mr. Mueller's inquiry. But the interest from
prosecutors in what Skadden did for the Ukrainian government is one
indication of the wide-ranging nature of the inquiries related to Mr.
Manafort. It also highlights the risks associated with advising
authoritarian governments overseas, a lucrative sideline among
Washington lawyers, lobbyists and public relations consultants.

Mr. Manafort played a central role in the effort to shield Mr.
Yanukovych from international condemnation, according to consultants
involved in the effort. He devised the strategy and recruited lobbyists,
lawyers and public relations consultants from across the political
spectrum, but left the day-to-day implementation of the campaign to
others. Skadden's report was one element of that strategy.

Its conclusions provided a counterpoint to international critics who
said that Mr. Yanukovych's government had prosecuted and convicted the
former Ukrainian prime minister, Yulia V. Tymoshenko, on corruption
charges in 2011 for political reasons and without sufficient evidence.

That kind of international consulting by American firms traditionally
has not drawn much scrutiny from regulators or the media, but that has
changed in the last year, thanks largely to Mr. Manafort's role as Mr.
Trump's campaign chairman in 2016 after years collecting
\href{https://www.nytimes.com/2017/06/27/us/politics/trump-campaign-chiefs-firm-got-17-million-from-pro-russia-party.html?mcubz=3}{multimillion-dollar
paydays} from Russian and Ukrainian oligarchs and political parties.

As part of Mr. Mueller's investigation, prosecutors last month issued
grand jury subpoenas seeking testimony from officials from at least two
lobbying and public relations firms that worked on the team Mr. Manafort
assembled to plead Mr. Yanukovych's case in Washington --- Mercury
Public Affairs and the Podesta Group, according to two people with
direct knowledge of the subpoenas.

\includegraphics{https://static01.nyt.com/images/2017/09/22/us/22dc-skadden2/20dc-skadden2-articleLarge.jpg?quality=75\&auto=webp\&disable=upscale}

The firms were paid more than \$1.1 million each to try to rally support
among American policy makers and opinion leaders for Mr. Yanukovych, and
the firms' lobbyists cited the findings in Skadden's report to quell
mounting concerns about his leadership.

The subpoenas for Mercury and Podesta --- which followed an earlier
round of subpoenas to the firms for documents and information related to
their Ukraine work --- focused on ``Manafort's money --- where it came
from, how he got it, what he did with it,'' according to a person
familiar with the inquiries.

Officials at Mercury and the Podesta Group did not respond to requests
for comment.

Through a spokesman, Mr. Manafort declined to comment.
\href{https://www.nytimes.com/2017/08/09/us/politics/paul-manafort-home-search-mueller.html}{Federal
agents raided his Virginia home} in July, confiscating documents and
copying some of his computer files. Shortly afterward, prosecutors
working for Mr. Mueller told Mr. Manafort they planned to indict him.

The Justice Department's request for information about Skadden's
Ukrainian work came after Ukrainian prosecutors asked their American
counterparts for assistance in pursuing an inquiry into alleged illegal
spending by Mr. Yanukovych's government. That inquiry included payments
to Skadden, though the Ukrainians have not accused the firm of any
crime. The Ukrainians nonetheless requested that the Justice Department
question Mr. Manafort and Skadden's lead lawyer on the case, Gregory B.
Craig, who had served as President Barack Obama's White House counsel.

Mr. Manafort's team hoped that the involvement of Mr. Craig, who
maintained deep connections to Washington's Democratic establishment,
might win Mr. Yanukovych a more favorable reception with the Obama State
department, according to the consultants who worked on the issue. Yet
they said that even employees of Mercury and Podesta regarded the report
as a ``whitewash'' that did little to address valid concerns about Mr.
Yanukovych's government.

The report was concluded in September 2012 --- just before one of Mr.
Manafort's daughters started work as an associate at Skadden --- and
\href{http://www.nytimes.com/2012/12/13/world/europe/failings-found-in-trial-of-ukrainian-ex-premier.html?mcubz=3}{released
in December 2012}.

The day after its release, Victoria Nuland, a State Department official
at the time, called it ``incomplete,'' at a department
\href{https://2009-2017.state.gov/r/pa/prs/dpb/2012/12/202021.htm}{press
briefing}, saying that it ``doesn't give an accurate picture.'' She said
the State Department was concerned that ``Skadden Arps lawyers were
obviously not going to find political motivation if they weren't looking
for it.''

In a recent interview, John E. Herbst, a former United States ambassador
to Ukraine, went further. He said that Skadden ``should have been
ashamed'' of the report, calling it ``a nasty piece of work.''

Mr. Craig declined to comment.

Under the Foreign Agents Registration Act, or F.A.R.A., anyone engaged
in lobbying or public relations for foreign governments must register
with the Justice Department. But in a statement this month, Skadden
contended that ``none of our attorneys engaged in any activity that
required them or the firm to register under F.A.R.A.''

Image

Paul Manafort played a central role in the effort to shield Mr.
Yanukovych from international condemnation, according to consultants
involved in the effort.Credit...Matt Rourke/Associated Press

The firm also asserted that its report ``did not opine about whether the
prosecution was politically motivated or driven by an improper political
objective'' --- an assertion that narrowly avoids directly contradicting
the report's conclusion that ``Tymoshenko has not provided clear and
specific evidence of political motivation that would be sufficient to
overturn her conviction under American standards.''

Rather, the firm's statement said that Ms. Tymoshenko ``was denied basic
rights under Western legal standards,'' was ``improperly incarcerated
during the trial'' and that ``in the West, she would receive a new
trial.''

In June, Skadden refunded \$567,000 to the Ukrainian government ---
about half of the total it was said to have been paid by Mr.
Yanukovych's government. The firm suggested in a statement that it
returned the cash because the money had been placed ``in escrow for
future work'' that never took place.

Less than a year and a half after the release of the Skadden report, Mr.
Yanukovych fled the country amid street protests over his government's
corruption and its pivot toward Moscow. Under the government that
succeeded Mr. Yanukovych, the country's general prosecutors office ---
Ukraine's version of the Justice Department --- opened criminal
corruption investigations into Mr. Yanukovych and members of his
government, including his justice minister, Oleksandr Lavrynovych.

Court documents in the case against Mr. Lavrynovych alleged that Mr.
Manafort ``designed a strategy'' to enlist Skadden to ``confirm the
legality of the criminal prosecution of Yulia Tymoshenko and \ldots{}
reject any political motives of such prosecution.'' Mr. Lavrynovych's
lawyer, Yevgeny V. Solodko, rejected the charges against his client,
characterizing the case as a politically motivated crackdown on
officials from the former government.

The general prosecutor's office, under a mutual legal aid agreement with
the United States, began asking the Justice Department and the F.B.I.
for assistance with the investigation into Mr. Lavrynovych starting in
late 2014.

But neither the Justice Department nor the F.B.I. had responded to the
requests as recently as March, when the F.B.I. director at the time,
James B. Comey, was asked during
\href{https://www.nytimes.com/2017/03/20/us/politics/intelligence-committee-russia-donald-trump.html?mcubz=3}{a
congressional hearing} why the Ukrainian requests for assistance had
gone unheeded.

More recently, Ukraine's prosecutor general, Yuriy Lutsenko,
acknowledged in written responses to The New York Times that his office
had begun working with the Justice Department to investigate the
payments from the Ukrainian Justice Ministry to Skadden.

Asked whether Ukrainian prosecutors are assisting in Mr. Mueller's
investigation, Mr. Lutsenko's office was coy. In a statement, it said
that it had not publicly disclosed any such cooperation, but it also
noted that not all international judicial cooperation can be disclosed.

Representatives for Mr. Mueller's team and the Justice Department
declined to comment.

Advertisement

\protect\hyperlink{after-bottom}{Continue reading the main story}

\hypertarget{site-index}{%
\subsection{Site Index}\label{site-index}}

\hypertarget{site-information-navigation}{%
\subsection{Site Information
Navigation}\label{site-information-navigation}}

\begin{itemize}
\tightlist
\item
  \href{https://help.nytimes.com/hc/en-us/articles/115014792127-Copyright-notice}{©~2020~The
  New York Times Company}
\end{itemize}

\begin{itemize}
\tightlist
\item
  \href{https://www.nytco.com/}{NYTCo}
\item
  \href{https://help.nytimes.com/hc/en-us/articles/115015385887-Contact-Us}{Contact
  Us}
\item
  \href{https://www.nytco.com/careers/}{Work with us}
\item
  \href{https://nytmediakit.com/}{Advertise}
\item
  \href{http://www.tbrandstudio.com/}{T Brand Studio}
\item
  \href{https://www.nytimes.com/privacy/cookie-policy\#how-do-i-manage-trackers}{Your
  Ad Choices}
\item
  \href{https://www.nytimes.com/privacy}{Privacy}
\item
  \href{https://help.nytimes.com/hc/en-us/articles/115014893428-Terms-of-service}{Terms
  of Service}
\item
  \href{https://help.nytimes.com/hc/en-us/articles/115014893968-Terms-of-sale}{Terms
  of Sale}
\item
  \href{https://spiderbites.nytimes.com}{Site Map}
\item
  \href{https://help.nytimes.com/hc/en-us}{Help}
\item
  \href{https://www.nytimes.com/subscription?campaignId=37WXW}{Subscriptions}
\end{itemize}
