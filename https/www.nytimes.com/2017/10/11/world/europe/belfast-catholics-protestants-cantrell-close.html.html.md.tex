Sections

SEARCH

\protect\hyperlink{site-content}{Skip to
content}\protect\hyperlink{site-index}{Skip to site index}

\href{https://www.nytimes.com/section/world/europe}{Europe}

\href{https://myaccount.nytimes.com/auth/login?response_type=cookie\&client_id=vi}{}

\href{https://www.nytimes.com/section/todayspaper}{Today's Paper}

\href{/section/world/europe}{Europe}\textbar{}A New Street Was Meant to
Bridge Belfast's Sectarian Divide. Then the Doorbell Rang.

\url{https://nyti.ms/2z0j9HQ}

\begin{itemize}
\item
\item
\item
\item
\item
\end{itemize}

Advertisement

\protect\hyperlink{after-top}{Continue reading the main story}

Supported by

\protect\hyperlink{after-sponsor}{Continue reading the main story}

Belfast Journal

\hypertarget{a-new-street-was-meant-to-bridge-belfasts-sectarian-divide-then-the-doorbell-rang}{%
\section{A New Street Was Meant to Bridge Belfast's Sectarian Divide.
Then the Doorbell
Rang.}\label{a-new-street-was-meant-to-bridge-belfasts-sectarian-divide-then-the-doorbell-rang}}

\includegraphics{https://static01.nyt.com/images/2017/10/12/world/12Belfast1/07Belfast1-articleLarge.jpg?quality=75\&auto=webp\&disable=upscale}

By \href{https://www.nytimes.com/by/patrick-kingsley}{Patrick Kingsley}

\begin{itemize}
\item
  Oct. 11, 2017
\item
  \begin{itemize}
  \item
  \item
  \item
  \item
  \item
  \end{itemize}
\end{itemize}

BELFAST, Northern Ireland --- When a young Catholic cook moved into a
newly built house in a Protestant part of Belfast last September, it was
a vote of faith in not just his own future --- but that of the province
of Northern Ireland.

Eighteen years on from the peace deal that largely ended three decades
of sectarian violence between people of Catholic and Protestant
backgrounds here, the two communities still live largely apart. But the
cook's new home stood in Cantrell Close, a flagship housing project that
was built in 2016 specifically to accommodate people from all
backgrounds.

In a province where over
\href{http://cain.ulst.ac.uk/issues/education/docs/mag_2013-04-22.pdf}{90
percent} of pupils still receive a largely segregated education, the
cook, then 23, could imagine his infant son attending a mixed school,
several years down the line.

That hope ended just before midnight on a Tuesday in late September,
when two policemen knocked on his door and that of three other Catholic
families in Cantrell Close. ``We believe there is a threat on your
life,'' the cook remembers being told, ``if you're not out of your
property by Friday.''

The cook, his pregnant fiancée and their 14-month-old son were gone by
Wednesday morning. At least two other Catholic families left that day,
too, while others told local politicians that they wanted to leave as
soon as possible.

For several days afterward, those who fled were forced to stay in a
different friend's house each night, said the cook, who asked that he
and his fiancée not be named because he felt their lives were still in
danger.

This kind of intimidation is not new in Northern Ireland, nor does it
appear to be on the rise. Around 30 people have declared themselves
homeless for similar reasons each year for the past half-decade,
according to statistics provided by the Northern Irish government, and
this year's figure of 33 is no departure from that trend.

But the recent episodes at Cantrell Close have made headlines in
Northern Ireland because they occurred in a place that was intended to
be a foundation stone for a post-sectarian society. It has also raised
concerns about certain politicians' commitment to the process of
integration, and about the ability of the Northern Irish police to curb
the influence of sectarian paramilitary groups.

``This is a very good illustration of a much deeper problem,'' said
\href{https://en.wikipedia.org/wiki/Stephen_Farry}{Stephen Farry}, a
lawmaker from the Alliance Party, which tries to bridge the divides
between the province's Unionists and nationalist communities. ``Northern
Ireland is not yet a peaceful society. We have ongoing coercive control
by paramilitary structures at a local level across many communities.''

A recent stroll down Cantrell Close, a tiny T-shaped cul-de-sac, did not
make this instantly obvious. At first sight, it was a picture of prim,
docile suburbia: 41 two-story homes, each with a tidy lawn and a garden
fence, lined a quiet road with a shiny bike rack at one end and a speed
bump at the other. The tensions became apparent only when you looked
toward the sky.

Flying from some of the lampposts were the flags of the
\href{https://en.wikipedia.org/wiki/Ulster_Volunteer_Force}{Ulster
Volunteer Force}, or U.V.F., a banned terrorist group that killed
\href{http://cain.ulst.ac.uk/sutton/tables/Organisation_Responsible.html}{more
than 400 people} during the Northern Irish Troubles, mostly targeting
Catholics and Irish nationalists, or those opposed to Northern Ireland
remaining part of the United Kingdom.

No faction has claimed responsibility, but the police believe that the
threats came from people purporting to be part of the U.V.F. Two men
have been questioned in connection to the crime, on suspicion of
membership in the group.

The East Belfast Community Initiative, which says it mediates on behalf
of former U.V.F. combatants, said the U.V.F. was not involved. Cantrell
Close, meanwhile, is just one of many streets in East Belfast lined with
the group's flags.

\includegraphics{https://static01.nyt.com/images/2017/10/07/world/07Belfast2/07Belfast2-articleLarge.jpg?quality=75\&auto=webp\&disable=upscale}

On Cantrell Close itself, there appears to be an informal omertà in
place. Of those who answered their doors on a recent afternoon, none
would discuss the threats to their former neighbors.

One man even claimed he knew nothing at all about the situation, his
mouth curling into a faint smile.

Some outside the neighborhood have been more outspoken, however, about a
culture of impunity that they believe encouraged whoever made the
threats.

Before moving in, residents of Cantrell Close signed pledges against the
display of controversial flags. The government also says it is illegal
to fly the flags of terrorist groups.

Yet when the U.V.F. flags suddenly appeared in June, no action was taken
either to find the perpetrators or to remove them, creating the
impression that sectarianism would be tolerated even at such a symbolic
housing development.

``The police, who you'd expect to lead, didn't,'' said
\href{https://en.wikipedia.org/wiki/M\%C3\%A1irt\%C3\%ADn_\%C3\%93_Muilleoir}{Mairtin
O Muilleoir}, a lawmaker from Sinn Fein, an Irish nationalist party.
``And the gangs were emboldened.''

Local unionist politicians were also
\href{http://www.belfasttelegraph.co.uk/news/northern-ireland/residents-dont-want-a-fuss-about-uvf-terror-flags-says-dups-pengelly-35846077.html}{perceived}
to take too weak a stance back in June.

``I'll always be cautious about causation and correlation,'' said
\href{https://www.ulster.ac.uk/staff/cp-mcgrattan}{Cillian McGrattan}, a
politics professor at Ulster University. But the
\href{http://www.belfasttelegraph.co.uk/news/northern-ireland/residents-dont-want-a-fuss-about-uvf-terror-flags-says-dups-pengelly-35846077.html}{reaction}
of politicians from the
\href{https://www.nytimes.com/2017/06/10/world/europe/britain-election-dup-northern-ireland.html?_r=0}{Democratic
Unionist Party}, or D.U.P., to sectarian episodes during the summer,
including the Cantrell Close issue, ``really left a lot to be desired,''
he added.

Asked for comment, Emma Little-Pengelly, one of the party lawmakers
\href{http://www.belfasttelegraph.co.uk/news/northern-ireland/dups-pengelly-slammed-over-let-it-be-attitude-to-terrorist-flags-in-mixed-estate-35849293.html}{facing
criticism}, said she had been clear in June about her opposition to
paramilitary flags. She and a D.U.P. colleague also issued an immediate
condemnation of last month's threats to Catholic residents.

Belfast's top policeman meanwhile suggested that the flags had not been
deemed problematic enough to justify their removal back in June.

``The reality is that while we understand the public's frustration in
this matter, police will only act to remove flags if there are
substantial risks to public safety,'' Chief Superintendent Chris Noble
said in an emailed statement.

(The flags were taken down only several days after the threats were
made.)

For the young cook and his fiancée, these explanations mean little. They
feel frustrated at the authorities for doing nothing in the first place
to remove the flags, which they said were clearly intimidating. They
feel abandoned by the police for failing, once the threats were made, to
provide a permanent presence on the street.

And they fault the housing authorities for failing to quickly find them
alternative accommodations --- and for assuring them that the street was
safe to move to.

The cook's fiancée, a 20-year-old waitress, said officials should have
recognized her vulnerability as a young mother with a newborn in tow
before encouraging her to move in.

Politicians like Mr. O Muilleoir hope the episodes at Cantrell Close
will not derail plans for more integrated housing projects. ``In Europe
in 2017, we either let an armed gang dictate policy,'' he said, ``or we
step things up.''

But for the targeted families themselves, the integration dream is over.
From now on, the cook said, his family would stick to neighborhoods with
a Catholic majority.

Advertisement

\protect\hyperlink{after-bottom}{Continue reading the main story}

\hypertarget{site-index}{%
\subsection{Site Index}\label{site-index}}

\hypertarget{site-information-navigation}{%
\subsection{Site Information
Navigation}\label{site-information-navigation}}

\begin{itemize}
\tightlist
\item
  \href{https://help.nytimes.com/hc/en-us/articles/115014792127-Copyright-notice}{©~2020~The
  New York Times Company}
\end{itemize}

\begin{itemize}
\tightlist
\item
  \href{https://www.nytco.com/}{NYTCo}
\item
  \href{https://help.nytimes.com/hc/en-us/articles/115015385887-Contact-Us}{Contact
  Us}
\item
  \href{https://www.nytco.com/careers/}{Work with us}
\item
  \href{https://nytmediakit.com/}{Advertise}
\item
  \href{http://www.tbrandstudio.com/}{T Brand Studio}
\item
  \href{https://www.nytimes.com/privacy/cookie-policy\#how-do-i-manage-trackers}{Your
  Ad Choices}
\item
  \href{https://www.nytimes.com/privacy}{Privacy}
\item
  \href{https://help.nytimes.com/hc/en-us/articles/115014893428-Terms-of-service}{Terms
  of Service}
\item
  \href{https://help.nytimes.com/hc/en-us/articles/115014893968-Terms-of-sale}{Terms
  of Sale}
\item
  \href{https://spiderbites.nytimes.com}{Site Map}
\item
  \href{https://help.nytimes.com/hc/en-us}{Help}
\item
  \href{https://www.nytimes.com/subscription?campaignId=37WXW}{Subscriptions}
\end{itemize}
