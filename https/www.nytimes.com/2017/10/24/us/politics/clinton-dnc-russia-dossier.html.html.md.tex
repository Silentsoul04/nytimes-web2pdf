Sections

SEARCH

\protect\hyperlink{site-content}{Skip to
content}\protect\hyperlink{site-index}{Skip to site index}

\href{https://www.nytimes.com/section/politics}{Politics}

\href{https://myaccount.nytimes.com/auth/login?response_type=cookie\&client_id=vi}{}

\href{https://www.nytimes.com/section/todayspaper}{Today's Paper}

\href{/section/politics}{Politics}\textbar{}Clinton Campaign and
Democratic Party Helped Pay for Russia Trump Dossier

\url{https://nyti.ms/2h7kIxa}

\begin{itemize}
\item
\item
\item
\item
\item
\item
\end{itemize}

Advertisement

\protect\hyperlink{after-top}{Continue reading the main story}

Supported by

\protect\hyperlink{after-sponsor}{Continue reading the main story}

\hypertarget{clinton-campaign-and-democratic-party-helped-pay-for-russia-trump-dossier}{%
\section{Clinton Campaign and Democratic Party Helped Pay for Russia
Trump
Dossier}\label{clinton-campaign-and-democratic-party-helped-pay-for-russia-trump-dossier}}

\includegraphics{https://static01.nyt.com/images/2017/10/25/us/25dc-dossier1/25dc-dossier1-articleLarge.jpg?quality=75\&auto=webp\&disable=upscale}

By \href{https://www.nytimes.com/by/kenneth-p-vogel}{Kenneth P. Vogel}

\begin{itemize}
\item
  Oct. 24, 2017
\item
  \begin{itemize}
  \item
  \item
  \item
  \item
  \item
  \item
  \end{itemize}
\end{itemize}

WASHINGTON --- The presidential campaign of Hillary Clinton and the
Democratic National Committee paid for research that was included in a
\href{https://www.nytimes.com/2017/01/11/us/politics/donald-trump-russia-intelligence.html}{dossier
made public in January} that contained salacious claims about
connections between Donald J. Trump, his associates and Russia.

A spokesperson for a law firm said on Tuesday that it had hired
Washington-based researchers last year to gather damaging information
about Mr. Trump on numerous subjects --- including possible ties to
Russia --- on behalf of the Clinton campaign and the D.N.C.

The revelation, which emerged from a
\href{https://www.documentcloud.org/documents/4116755-PerkinsCoie-Fusion-PrivelegeLetter-102417.html}{letter
filed in court}on Tuesday, is likely to fuel new partisan attacks over
\href{https://www.nytimes.com/2017/10/22/us/politics/russia-investigation-congress-intelligence-committees-gowdy.html}{federal
and congressional investigations} into Russia's attempts to disrupt last
year's election and whether any of Mr. Trump's associates assisted in
the effort.

The president and his allies have argued for months that the
investigations are politically motivated. They have
\href{https://www.nytimes.com/2017/08/30/us/politics/trump-russia-michael-cohen.html}{challenged}
the information contained in the dossier, which was compiled by a former
British spy who had been contracted by the Washington research firm
Fusion GPS.

The letter that was filed in court said that Fusion GPS began working
for the law firm, Perkins Coie, in April 2016. Written by the firm's
general counsel, Matthew J. Gehringer, the letter said that Fusion GPS
had already been conducting the research ``for one or more other clients
during the Republican primary contest.''

Perkins Coie was paid \$12.4 million to represent the Clinton campaign
and the D.N.C. during the 2016 campaign, according to filings. The role
of the Clinton campaign and the national party in funding the research
for the dossier was
\href{https://www.washingtonpost.com/world/national-security/clinton-campaign-dnc-paid-for-research-that-led-to-russia-dossier/2017/10/24/226fabf0-b8e4-11e7-a908-a3470754bbb9_story.html?source=gmail\&usg=AFQjCNH3gY97gLgf21UAC5-BLneZPH2Xxw\&ust=1508970556887000\&utm_term=.73eac5184d14\&utm_term\%3D.50f695ff5f0c}{first
reported} on Tuesday by The Washington Post.

\includegraphics{https://static01.nyt.com/images/2017/10/25/us/25dc-dossier2/25dc-dossier2-articleLarge.jpg?quality=75\&auto=webp\&disable=upscale}

At the time that Democrats began paying for the research, Mr. Trump was
in the process of clinching the Republican presidential nomination, and
Ms. Clinton's allies were scrambling to figure out how to run against a
candidate who had already weathered attacks from Republican rivals about
his shifting policy positions, his character and his business record.

Fusion GPS hired Christopher Steele, a respected former British spy with
extensive experience in Russia, to conduct research into any possible
connections between Mr. Trump, his businesses, campaign team and Russia.

Mr. Steele produced a series of memos that alleged a broad conspiracy
between the Trump campaign and the Russian government to influence the
2016 election on behalf of Mr. Trump. The memos also contained
unsubstantiated accounts of encounters between Mr. Trump and Russian
prostitutes, and real estate deals that were intended as bribes.

The contents of the memos circulated in Washington in late 2016, and
were briefed to Mr. Trump by senior American intelligence officials
during the presidential transition. The memos, which became known as the
``Steele Dossier,'' were made public by Buzzfeed --- sparking an ongoing
debate about their accuracy and about who funded the research.

Fusion GPS was started by three former Wall Street Journal employees.
The firm worked directly with Perkins Coie and its lead election lawyer,
Marc Elias, according to the law firm spokesperson, who spoke on
condition of anonymity to discuss sensitive information about
confidential business relationships. The law firm's payments to Fusion
GPS for the Russia research ended just before Election Day, the
spokesperson said.

The spokesperson said that neither the Clinton campaign, nor the D.N.C.,
was aware that Fusion GPS had been hired to conduct the research.

Earlier this year, Mr. Elias had denied that he had possessed the
dossier before the election.

Anita Dunn, a veteran Democratic operative working with Perkins Coie,
said on Tuesday that Mr. Elias ``was certainly familiar with some of,
but not all, of the information'' in the dossier. But, she said ``he
didn't have and hadn't seen the full document, nor was he involved in
pitching it to reporters.'' And Mr. Elias ``was not at liberty to
confirm Perkins Coie as the client at that point,'' Ms. Dunn said.

Image

Hillary Clinton with Brian Fallon, her campaign's national press
secretary, last year. On Tuesday, Mr. Fallon said he did not know that
the author of a dossier on Mr. Trump was working on behalf of Mrs.
Clinton's campaign.Credit...Brendan Smialowski/Agence France-Presse ---
Getty Images

Brian Fallon, who served as a spokesman for the Clinton campaign, on
Tuesday
\href{https://twitter.com/brianefallon/status/922990478387183616}{wrote
on Twitter} that he did not know that Mr. Steele had been working on
behalf of the Clinton campaign before the election.

``If I had, I would have volunteered to go to Europe and try to help
him,'' Mr. Fallon wrote.

A lawyer and spokeswoman for Fusion GPS did not respond to requests for
comment.

A spokeswoman for the D.N.C. sought to distance the national party from
the work, noting that the party's chairman, Tom Perez, was elected only
after last year's election. He and his leadership team ``were not
involved in any decision-making regarding Fusion GPS, nor were they
aware that Perkins Coie was working with the organization,'' said
Xochitl Hinojosa, the D.N.C. spokeswoman.

The work by Fusion GPS on the dossier has come under scrutiny from
congressional investigators, who have questioned one of its founders and
subpoenaed its banking records to try to determine who funded the
research.

Fusion GPS is fighting the subpoena in federal court, and Mr.
Gehringer's letter was produced in connection with that legal case.

In the letter, Mr. Gehringer praised Fusion GPS for its ``efforts to
fulfill your obligation to maintain client confidentiality. In the
circumstances, however, we believe it is appropriate to release Fusion
GPS from this obligation as it relates to the identity of Perkins
Coie.''

Mr. Gehringer added that**,** ``given the interest in this issue, we
believe it would be appropriate for all parties who hired Fusion GPS in
connection with the 2016 presidential campaign to release Fusion GPS
from this obligation as well.''

Advertisement

\protect\hyperlink{after-bottom}{Continue reading the main story}

\hypertarget{site-index}{%
\subsection{Site Index}\label{site-index}}

\hypertarget{site-information-navigation}{%
\subsection{Site Information
Navigation}\label{site-information-navigation}}

\begin{itemize}
\tightlist
\item
  \href{https://help.nytimes.com/hc/en-us/articles/115014792127-Copyright-notice}{©~2020~The
  New York Times Company}
\end{itemize}

\begin{itemize}
\tightlist
\item
  \href{https://www.nytco.com/}{NYTCo}
\item
  \href{https://help.nytimes.com/hc/en-us/articles/115015385887-Contact-Us}{Contact
  Us}
\item
  \href{https://www.nytco.com/careers/}{Work with us}
\item
  \href{https://nytmediakit.com/}{Advertise}
\item
  \href{http://www.tbrandstudio.com/}{T Brand Studio}
\item
  \href{https://www.nytimes.com/privacy/cookie-policy\#how-do-i-manage-trackers}{Your
  Ad Choices}
\item
  \href{https://www.nytimes.com/privacy}{Privacy}
\item
  \href{https://help.nytimes.com/hc/en-us/articles/115014893428-Terms-of-service}{Terms
  of Service}
\item
  \href{https://help.nytimes.com/hc/en-us/articles/115014893968-Terms-of-sale}{Terms
  of Sale}
\item
  \href{https://spiderbites.nytimes.com}{Site Map}
\item
  \href{https://help.nytimes.com/hc/en-us}{Help}
\item
  \href{https://www.nytimes.com/subscription?campaignId=37WXW}{Subscriptions}
\end{itemize}
