Sections

SEARCH

\protect\hyperlink{site-content}{Skip to
content}\protect\hyperlink{site-index}{Skip to site index}

\href{/section/business}{Business}\textbar{}Detroit: From Motor City to
Housing Incubator

\url{https://nyti.ms/2iZYgtZ}

\begin{itemize}
\item
\item
\item
\item
\item
\item
\end{itemize}

\includegraphics{https://static01.nyt.com/images/2017/11/01/business/00DETROITHOUSING1/00DETROITHOUSING1-articleLarge.jpg?quality=75\&auto=webp\&disable=upscale}

\hypertarget{detroit-from-motor-city-to-housing-incubator}{%
\section{Detroit: From Motor City to Housing
Incubator}\label{detroit-from-motor-city-to-housing-incubator}}

The 2008 financial crisis and 2013 city bankruptcy gutted Detroit's
housing market. Now, Detroit is experimenting with unorthodox ways to
get people to buy homes and renovate houses.

Detroit has been struggling to piece together its broken housing market,
which has been partly frustrated by the unwillingness of banks to make
small mortgages.Credit...Erin Kirkland for The New York Times; Photo
illustration by The New York Times

Supported by

\protect\hyperlink{after-sponsor}{Continue reading the main story}

By \href{http://www.nytimes.com/by/matthew-goldstein}{Matthew Goldstein}

\begin{itemize}
\item
  Nov. 4, 2017
\item
  \begin{itemize}
  \item
  \item
  \item
  \item
  \item
  \item
  \end{itemize}
\end{itemize}

DETROIT --- Bank of America and JPMorgan Chase, the country's two
largest banks, trace their roots in Detroit back decades, when they
helped finance the city's once-booming auto industry.

These days, Detroit is still struggling to recover from the 2008
financial crisis, and the two banks have pledged to help resuscitate the
city and its crippled housing market. So, guess how many home mortgage
loans these two enormous banks made last year in this city of 637,000
people.

Bank of America made 18. JPMorgan did just six.

Detroit's hometown lender, Quicken Loans, made the most --- 170
mortgages.

\includegraphics{https://static01.nyt.com/images/2017/11/05/business/00DETROITHOUSING9/00DETROITHOUSING9-articleLarge.jpg?quality=75\&auto=webp\&disable=upscale}

Midwestern cities like Detroit have long embodied the American can-do
spirit. Over the course of a century, Motor City melded assembly-line
prowess with freedom-of-the-road ideals to help define a nation. In the
postwar years, Detroit became the epitome of the American dream, a place
where factory workers without college degrees could make enough money to
buy a house of their own.

Yet as home prices soar across the United States --- particularly on the
coasts --- Detroit remains a poster child for the economic crisis and
housing collapse of a decade ago. Boarded up homes and rubble-strewn
fields litter the landscape.

Today, a house can be bought here for the price of a used Chevy Caprice.

What is truly surprising about that, though, is how difficult it still
is for buyers to actually buy. Basically, prices are too low for lenders
(who see the deals as too small or risky) but too high for buyers (who
may be cash-poor). There aren't enough houses in move-in-ready condition
--- and not enough money to fix them up.

This strange situation has turned Detroit into an unlikely petri dish
for experiments into how to kick-start a housing market that is,
depending on your perspective, either slumbering or comatose.

Will a neighborhood of ``tiny houses'' for the poor help fix things? Or
how about rehabbing city-owned homes, and selling them at a loss, to
jump-start the action? Other more conventional --- if risky --- ideas
involve providing no-interest financing to fix up tumbledown properties.
Or offering mortgages for homes that normally would be too small to be
worth a banker's trouble.

One local financier is even trying to beautify bulldozed neighborhoods
by planting thousands of trees on 160 acres of vacant land his firm has
gobbled up.

And while Detroit is worse off than most big cities, housing-policy
makers nationwide are keeping a close eye to see what lessons can be
learned.

To understand how far Detroit has fallen, consider the statistics. In
the mid-2000s, banks were writing some 7,000 mortgages a year. Then, the
financial crisis nearly destroyed the American automotive industry,
Detroit's economic heart. Jobs disappeared; citizens fled. Last year,
there were more than 700 mortgages made in Detroit, up from 200 at the
depth of the crisis but barely 10 percent of the level a decade earlier.

Those bleak numbers, however, do not tell the whole story. Behind the
scenes, nonprofit groups, foundations, local officials and a dozen banks
including JPMorgan, Bank of America and Quicken are trying to varying
degrees to reanimate the mortgage market in Michigan's largest city.

Success, however, often comes achingly slow.

At 15455 Winthrop Street, on one of Detroit's better manicured blocks,
there is a freshly rehabbed three-bedroom home. The bungalow-style house
was fixed up by the city itself, through its land bank, which acquired
the house a year ago after the county foreclosed on the owner for
failing to pay taxes. The land bank did a gut renovation with money
provided by a grant from Quicken.

Since August, the land bank has been trying to sell the house, with a
price tag of at least \$79,900. More than 80 people have come to check
it out. But so far there have been no takers.

``We have never not sold one,'' said Craig Fahle, a former radio host
who today is the communications director for the Detroit Land Bank
Authority. ``Detroit likes to do everything kicking and screaming,'' he
said. ``But we get there eventually.''

Image

Erica Wyatt was able to buy this home through a downpayment-assistance
program designed to jump-start the housing market.Credit...Erin Kirkland
for The New York Times

Even happy stories are the product of a slog. Erica Wyatt struggled to
pay down her debts and then searched for two years before she managed to
get a mortgage from Fifth Third Bank to buy a four-bedroom home for
\$92,000. The transaction happened only because Ms. Wyatt, a single
mother with four children, received \$15,000 in down payment assistance.

Ms. Wyatt, who grew up in Detroit, said she was determined to move back
into the city after renting a home in a suburb. ``I wanted to make sure
my children saw that not all of Detroit is bad and there are some
beautiful neighborhoods,'' said Ms. Wyatt, 39, who works for an
insurance company.

Some of the ideas seem like stopgap measures. A social services group's
community of\href{https://casscommunity.org/tinyhomes/}{``tiny homes''}
--- 400-square-foot structures with nothing more than a bedroom, a
bathroom and small kitchen --- is being erected to provide housing to
homeless and handicapped people. The project, led by Reverend Faith
Fowler, executive director of Cass Community Social Services, is taking
place on a plot of vacant land the charitable organization bought from
the city.

The dollhouse-like structures --- seven so far --- are near the
organization's main social services facility, in a rather desolate area
of Detroit off Rosa Parks Boulevard. In all, Ms. Fowler hopes to build
two dozen small homes, which will be rented for as little as \$250 a
month and eventually deeded over after seven years to a select group of
homeless or poor individuals.

Tiny-house living can take adjustment, even for people with no roof over
their heads at all. Ms. Fowler said that one homeless veteran told her
the homes were too small to compete with a traditional homeless shelter.

Still, for some, the homes are perfect. One of the first tenants to move
in this past summer is a former Methodist minister, David Leenhouts, who
was forced to give up his ministry near Cleveland because of health
issues that make it difficult for him to walk and talk.

Mr. Leenhouts, who grew up in the Detroit area, said his college-age son
told him the small home, with a steepled ceiling, was all he needed
because everything is within just a few steps. Mr. Leenhouts, 59, said,
``I have no idea where I would be living if I was not chosen for a tiny
house.''

That said, a cluster of tiny homes hardly seems scalable in a city as
big as Detroit. And almost by definition, a tiny home isn't a viable
option for a family with children.

It's also an example of why the long-term prognosis for Detroit's
housing market remains uncertain at best. Much of the work underway is
taking place block-by-block --- much like the tiny-home homeless
experiment --- and there are a lot of blocks in this 139-square-mile
city.

``The pilot programs help some people, but they are on the margin,''
said Gregory Markus, a professor emeritus of political science at the
University of Michigan and executive director of Detroit Action
Commonwealth, an advocacy group for low-income residents. ```The root
problem is that Detroit is the poorest big city in America.'''

The national poverty rate is 14 percent, and Detroit's is 36 percent.
Mr. Markus said that, without more jobs, home buying will remain a
largely unattainable goal.

Detroit's population peaked in the 1950s at nearly 2 million and has
been falling ever since. The financial crisis and the city's bankruptcy
filing in 2013 hollowed out what was left of its once large,
middle-class African-American community. Over the past decade there have
been more than 150,000 home foreclosures here.

Image

The city of Detroit lacks ``a functioning housing market,'' one report
last year declared. Abandoned, dilapidated houses can be found across
the cityscape.Credit...Erin Kirkland for The New York Times

Detroit lacks ``a functioning housing market,'' a
\href{http://crfusa.com/uploaded/Doc/detroit_homebuying_ecosystem_strategies_final_28_dec_2016_3b2bf0.pdf}{report
last year} bluntly declared.

Things are so difficult that simply finding a contractor to rehab a home
can be an ordeal. ``We had several contractors who didn't want to do
work in the city,'' said Heather McKeon, 35, who along with her husband,
Matthew, recently moved into a fixer-upper in Detroit's up-and-coming
Corktown neighborhood. ``They would say, `I don't trust that I can keep
my tools here.'''

She added: ``It is still sort of flabbergasting to be laughed at.''

Ms. McKeon, an interior designer, said many insurers wouldn't sell them
a homeowner's policy on an unoccupied home under renovation. Ultimately,
they got a policy from a subsidiary of Munich Re Group of Germany.

\hypertarget{detroits-largest-property-owner}{%
\subsection{Detroit's Largest Property
Owner}\label{detroits-largest-property-owner}}

Many of the efforts to resuscitate the housing market begin with the
Detroit Land Bank Authority, a government agency that is the city's
single largest property owner. The land bank owns some 25,000 vacant
homes in various stages of disrepair, another 4,200 occupied homes and
65,000 grass-covered lots where homes once stood before the city tore
them down in an effort to fight blight.

Mr. Fahle, the land bank's communications director, likes to drive
around and point out once-abandoned houses that his employer sold to
people who then fixed them up.

But on a rainy September day, he was particularly interested in showing
off the refurbished three-bedroom house at 15455 Winthrop, which the
land bank spent \$98,000 to renovate. The asking price for the home ---
with its restored hardwood floors and a new granite kitchen countertop
--- was reduced by a few thousand dollars in early September from
\$83,000 to spur more interest.

Image

Craig Fahle, a former radio host who now represents the city's land
bank, in one of the homes that the land bank has fixed up and is trying
to sell --- at a loss --- to help reinvigorate the real estate
market.Credit...Erin Kirkland for The New York Times

Throughout Detroit, the land bank has sold 44 homes under its
\href{https://auctions.buildingdetroit.org/RehabbedAndReady}{``Rehabbed
\& Ready'' pilot program}. The program is funded with a \$5 million
grant from Quicken. At the closing, the buyers get a \$1,500 gift card
from Home Depot to buy appliances.

The program, though, is losing money --- an average of \$21,000 for
every home sold.

Mr. Fahle said the goal wasn't to turn a profit, but to get more
move-in-ready homes into the marketplace and to boost property values in
the process. In all, the land bank has sold more than 2,700 houses, many
in online auctions.

The land bank's operations are not without controversy. Housing
advocates have complained it has focused too much attention on rehabbing
homes in just a few neighborhoods, and on tearing down dilapidated homes
elsewhere. A federal grand jury has been investigating the awarding of
contracts to tear down more than 12,000 dilapidated homes as part of a
war on blight led by Detroit's first-term mayor, Mike Duggan. The
investigation is looking into why costs soared under the demolition
program, with almost \$140 million in mostly federal money being spent.

Image

Overgrown homes in the Brightmoor neighborhood. About half the mortgages
written last year were for properties in just six of 25 Detroit ZIP
codes.Credit...Erin Kirkland for The New York Times

Mr. Fahle said the land bank is cooperating with the investigation. He
said criticism that the rehabbed and ready program has focused on a just
a small part of the city is misguided. Mr. Fahle said a decision was
made to select homes for renovation in four neighborhoods early on, but
over time it is expanding to other parts of the city.

Homes are certainly worth more in Detroit now than they were a few years
ago. Citywide, the median value for a house here is \$47,700, a 40
percent gain over the past two years, according to Zillow. Stately homes
in the Villages, a group of neighborhoods with tree-lined streets,
located not far from the posh suburb of Grosse Pointe, Mich., have sold
for more than \$400,000.

But progress is largely limited to a small cluster of neighborhoods.
About half of the mortgages written in Detroit last year were for homes
purchased in just six ZIP codes, according to data from the real estate
information firm RealtyTrac, part of Attom Data Solutions. There are 25
ZIP codes in Detroit.

One question is whether the money that banks are providing --- a
combination of grants and loans --- signifies a long-term commitment or
an effort to score points with federal regulators. Banks are expected
under the federal Community Reinvestment Act to make loans in
communities with large numbers of poor- or moderate-income residents in
order to spur economic activity.

The downpayment-assistance program that helped Ms. Wyatt buy her home,
for instance, was financed by a settlement Wells Fargo reached a few
years ago in a housing class-action lawsuit. The settlement money is
drying up, though, and the bank said it was not sure if it will renew
the program. So far, it has provided assistance to 180 home buyers in
the city.

Bank of America said it was committed to working in Detroit and is
providing up to \$4 million to fund no-interest loans that have enabled
400 homeowners to fix up properties. The bank, working with two
nonprofit groups, also has said it was willing to finance \$55 million
worth of mortgages in Detroit. So far this year, the bank has issued 23
mortgages in Detroit --- up from 18 in 2016 --- and has increased the
number of loan officers in the city.

JPMorgan said it, too, was here for the long haul. Jamie Dimon, the
bank's chairman and chief executive, regularly promotes its Invested in
Detroit program, which includes up to \$150 million for housing and
commercial development and funds for research by the Urban Institute in
Washington, D.C., to study ways to revive Detroit's economy and housing
market.

Quicken, which moved most of its operations in 2010 to downtown Detroit
from nearby Livonia, Mich., recently committed \$300,000 to a new
government program that will give 80 tenants living in homes that face
tax foreclosure a chance to buy the houses for as little as \$2,500.

Image

Downtown areas have been revitalized with investments from Dan Gilbert,
the founder of Quicken. The Townhouse restaurant is in a building
renovated as part of his work.Credit...Andrew Spear for The New York
Times

Still, the money shelled out by the banks pales in comparison to the
estimated \$2.5 billion that Dan Gilbert, Quicken's founder,
\href{https://www.nytimes.com/2017/01/21/business/dealbook/quicken-loans-dan-gilbert-mortgage-lender.html}{has
spent buying and renovating} over 95 largely vacant properties,
including old department stores, in Detroit's downtown. Now most of
those buildings are filled with new businesses. A company backed by Mr.
Gilbert brought high-speed internet to downtown and Quicken paid \$5
million for the naming rights for a recently opened streetcar system
called the QLine that makes 12 stops along its 3.3-mile path.

The mayoral election on Nov. 7 is to some degree a referendum on Mr.
Duggan's efforts at reviving both downtown and the city's housing
market. Mr. Duggan is seeking a second term and is opposed by Senator
Coleman Young II. Mr. Duggan said one of his top priorities as mayor was
getting home prices up in Detroit.

``Home-sale prices have climbed far faster than anyone could have
predicted,'' Mr. Duggan said.

Perhaps the most vexing issue is the reluctance of banks to give loans
to people to buy cheap homes. It's simple business: The costs of
underwriting a \$50,000 mortgage --- doing all the paperwork, the credit
checks and the inspections --- are the same as for much larger mortgages
that can generate more bank revenue. Plus, when homes are in such
disrepair, often they are appraised for much less than the amount the
borrower needs to fix it up.

That means the collateral on the loan --- the house itself --- is worth
less than the amount the bank is owed. In today's risk-averse banking
culture, that's a big no-no.

The winners in this environment are speculators with lots of cash. Many
local residents, by contrast, are turning to risky seller-financed
transactions such as
\href{https://www.nytimes.com/2016/02/21/business/dealbook/market-for-fixer-uppers-traps-low-income-buyers.html}{contracts
for deed}. Evictions are common after just a few missed payments. Over
the past five years, at least 5,400 homes in Detroit were sold through a
contract for deed and 34,500 in all-cash deals, according to RealtyTrac.

One alternative is the
\href{http://www.detroithomemortgage.org/}{Detroit Home Mortgage
project}. Launched in early 2016, the program works with a handful of
banks to get an appraisal for a house that's based on the ``true value''
of the home after it's been renovated, not in its current dilapidated
state. The process effectively involves two loans --- one to cover the
purchase of a home, and a second mortgage that effectively covers the
renovation work. The second loan is backed by a bank and various
foundations involved with the program.

``DHM wants to be an ambassador for lending in the city,'' said Alex
DeCamp, the mortgage community development manager for Chemical Bank, a
local lender that has funded 15 loans through the program. The program
can take months to complete. Applicants go through a careful screening
and most also complete three mortgage workshops to be eligible for a
loan.

So far, 54 home buyers have bought homes through the program, among them
Ms. McKeon and her husband. So did Ashley and Damon Dickerson, who are
about to move into a renovated two-family home.

The Dickersons, both of whom are architectural designers, closed in
March. But their search began months earlier when they submitted a
\$45,000 bid during one of the land bank's daily online property
auctions.

Winning the bidding for the 107-year-old home was just the start. The
couple found it would cost at least \$180,000 to fully renovate the
six-bedroom, three-story brick structure with a large porch. They were
attracted to the home's hardwood floors, bay windows and potential to
reshape it by knocking down some walls.

In all, they got two mortgages from Chemical Bank, according to property
records: one for \$37,692 to cover the purchase from the land bank and
another for \$207,000 to cover the rehab costs. The
\href{https://www.youtube.com/watch?v=523u2nWWRsI}{Dickersons, who both
graduated from the University of Michigan, said} they never would have
been able to pull the deal off without the mortgage program. But the
process was a bit of an eye-opener because it took longer then
anticipated to close on the home. As with any new program, the couple
said, there were ``growing pains.''

The Detroit Home Mortgage project is now looking to get banks to provide
low-interest loans directly to local contractors, so they can renovate
more homes and get them into move-in-ready condition.

Image

One local investor has snapped up acres of former residential land and
is now planting rows of trees.Credit...Erin Kirkland for The New York
Times

But for now, the lack of move-in ready homes means home buyers like the
Dickersons and the McKeons need to be something of urban pioneers ---
fixing everything from broken water lines to antiquated electrical
wiring.

The prospect of people moving into Detroit from the suburbs or city
residents getting mortgages is of course sweet music to local real
estate agents. Until now, much of the business for them has been
handling all-cash deals. But several said they are looking forward to
getting local residents into homes with traditional financing.

Dorian Harvey, a Detroit native and the incoming president of the
Detroit Association of Realtors, said he would like for the city and
land bank to move quicker to get vacant homes into the hands of local
residents. Mr. Harvey, a Morehouse College graduate, said he came from
the camp that the rebirth of Detroit is going to have to happen from the
ground up with everyone taking part --- contractors, real estate agents
and local investors.

But he isn't necessarily waiting on government largess. ``There are
untapped resources in the city and we need to tap them and the city
needs to tap them,'' said Mr. Harvey, who added there's money to made in
Detroit. ``My heart is liberal but my money is conservative.''

Advertisement

\protect\hyperlink{after-bottom}{Continue reading the main story}

\hypertarget{site-index}{%
\subsection{Site Index}\label{site-index}}

\hypertarget{site-information-navigation}{%
\subsection{Site Information
Navigation}\label{site-information-navigation}}

\begin{itemize}
\tightlist
\item
  \href{https://help.nytimes.com/hc/en-us/articles/115014792127-Copyright-notice}{©~2020~The
  New York Times Company}
\end{itemize}

\begin{itemize}
\tightlist
\item
  \href{https://www.nytco.com/}{NYTCo}
\item
  \href{https://help.nytimes.com/hc/en-us/articles/115015385887-Contact-Us}{Contact
  Us}
\item
  \href{https://www.nytco.com/careers/}{Work with us}
\item
  \href{https://nytmediakit.com/}{Advertise}
\item
  \href{http://www.tbrandstudio.com/}{T Brand Studio}
\item
  \href{https://www.nytimes.com/privacy/cookie-policy\#how-do-i-manage-trackers}{Your
  Ad Choices}
\item
  \href{https://www.nytimes.com/privacy}{Privacy}
\item
  \href{https://help.nytimes.com/hc/en-us/articles/115014893428-Terms-of-service}{Terms
  of Service}
\item
  \href{https://help.nytimes.com/hc/en-us/articles/115014893968-Terms-of-sale}{Terms
  of Sale}
\item
  \href{https://spiderbites.nytimes.com}{Site Map}
\item
  \href{https://help.nytimes.com/hc/en-us}{Help}
\item
  \href{https://www.nytimes.com/subscription?campaignId=37WXW}{Subscriptions}
\end{itemize}
