Sections

SEARCH

\protect\hyperlink{site-content}{Skip to
content}\protect\hyperlink{site-index}{Skip to site index}

\href{https://www.nytimes.com/section/technology}{Technology}

\href{https://myaccount.nytimes.com/auth/login?response_type=cookie\&client_id=vi}{}

\href{https://www.nytimes.com/section/todayspaper}{Today's Paper}

\href{/section/technology}{Technology}\textbar{}AT\&T Deal Puts Trump's
Antitrust Cop at Center of a Political Storm

\url{https://nyti.ms/2hodvwq}

\begin{itemize}
\item
\item
\item
\item
\item
\end{itemize}

Advertisement

\protect\hyperlink{after-top}{Continue reading the main story}

Supported by

\protect\hyperlink{after-sponsor}{Continue reading the main story}

\hypertarget{att-deal-puts-trumps-antitrust-cop-at-center-of-a-political-storm}{%
\section{AT\&T Deal Puts Trump's Antitrust Cop at Center of a Political
Storm}\label{att-deal-puts-trumps-antitrust-cop-at-center-of-a-political-storm}}

\includegraphics{https://static01.nyt.com/images/2017/11/10/business/10ANTITRUST/08TECHREGS-1-articleLarge.jpg?quality=75\&auto=webp\&disable=upscale}

By \href{http://www.nytimes.com/by/cecilia-kang}{Cecilia Kang}

\begin{itemize}
\item
  Nov. 9, 2017
\item
  \begin{itemize}
  \item
  \item
  \item
  \item
  \item
  \end{itemize}
\end{itemize}

WASHINGTON --- A year ago, \href{https://www.justice.gov/atr}{Makan
Delrahim} predicted that AT\&T's \$85.4 billion purchase of Time Warner
would be approved by regulators. ``I don't see this as a major antitrust
problem,'' Mr. Delrahim, then a law professor,
\href{http://www.bnn.ca/video/no-big-worries-in-at-t-deal-for-time-warner~978794}{said
to a Canadian television network}.

Now, five weeks into his job as the top antitrust regulator at the
Justice Department, Mr. Delrahim has taken a different position. The
department has
\href{https://www.nytimes.com/2017/11/08/business/dealbook/att-time-warner.html}{threatened
to block the deal} in court unless AT\&T sells off major assets.

Mr. Delrahim's position has thrown a surprising twist into a blockbuster
deal, which has been watched for signs about how the Trump
administration would handle giant mergers. On Wednesday, the dispute
spilled into the public with conflicting versions about what the Justice
Department wants.

In one account, the agency offered two paths: Sell Turner Broadcasting,
including CNN, or offload DirecTV, according to several people at the
companies. In another, AT\&T offered to sell CNN, according to two
officials at the agency.

Randall L. Stephenson, AT\&T's chief executive, said on Wednesday that
he had never offered to sell CNN. On Thursday,
\href{https://www.nytimes.com/2017/11/09/business/dealbook/att-time-warner-cnn.html}{appearing
at The New York Times's DealBook conference}, he said the company was
ready to go to court against the Justice Department.

``To suggest that selling some of the key franchises of the business
that are the most desired for your business plan makes no business
sense,'' Mr. Stephenson said.

The public spat has put a political cloud over the deal. President Trump
has been critical of CNN and the deal. Several Democratic lawmakers have
called for hearings to determine whether politics played a role in what
is supposed to be an independent process.

The Justice Department declined to comment on Thursday.

Pushing AT\&T to unload a big part of its business to get the deal
approved goes against Mr. Delrahim's history, antitrust experts say. His
past comments have largely been in line with more free-market-oriented
Republican views, and he was widely expected to be more lenient on
mergers than predecessors in the Obama administration.

``This is bold,'' said Diana Moss, president of the American Antitrust
Institute, a nonprofit that generally favors stronger antitrust
enforcement. ``It signals they are willing to consider the
anticompetitive effects of big deals, and on the remedy side they would
use structural remedies, which would signal a potential change in
policy.''

In an interview late last month, Mr. Delrahim strongly rejected the idea
that the White House had tried to or could influence his thinking. At
the time, he would not discuss AT\&T's bid for Time Warner or any other
pending mergers.

``All enforcement decisions will be based on the facts and the law. Not
on politics,'' Mr. Delrahim said. ``That would be antithetical to
everything I've stood for.''

Mr. Delrahim, 48, was born in Tehran, but his family moved to Los
Angeles when he was almost 10, around the time of the Islamic
Revolution. He attended the University of California, Los Angeles, as an
undergraduate and then went to law school at George Washington
University.

After a few years in corporate law, he spent time as an aide on Capitol
Hill, working on some technology and antitrust issues. President George
W. Bush later nominated him to be a top official at the Justice
Department, where he focused on international antitrust enforcement.
After his time there, he returned to corporate law in Los Angeles,
representing numerous large technology, health and telecommunications
companies.

In his recent interview, he said that the Justice Department did not
need to intervene just because a company was big, even a monopoly. He
also said the government should not startle business markets with an
abrupt change in its approach to antitrust legal theory.

``There are people who think big is just bad,'' he said. ``They don't
understand why, but there is an instinctive reaction to big business
these days.'' He added that it was dangerous to go after companies
without clear evidence that they were harming competitors.

\includegraphics{https://static01.nyt.com/images/2017/11/10/business/10TECHREGS-01p/08TECHREGS-3-articleLarge.jpg?quality=75\&auto=webp\&disable=upscale}

Mr. Delrahim has said his comments last year about AT\&T and Time Warner
were taken out of context. He has told lawmakers that the deal deserved
a thorough review. He said there could be antitrust concerns when a
company like AT\&T, which controls distribution of television content
through DirecTV and internet content through its mobile and home
broadband service, bought a big media company.

That view is more aligned with left-leaning consumer groups that have
pushed for greater antitrust enforcement. After the news this week that
the Justice Department was taking a deeper look at the deal, some
Democrats, including Senator Richard Blumenthal of Connecticut, praised
the tough stance.

``AT\&T and Time Warner is an enormous test because we are just seeing
online video disrupting the dominant cable broadband providers,'' said
Gene Kimmelman, president of the nonprofit organization Public Knowledge
and an antitrust official during the Obama administration. ``And it
would be a disaster for consumers and the competitive landscape if those
developments are wiped out or diminished by the approval of this deal.''

For years, though, antitrust officials have generally approved mergers
of companies that do not have competing businesses. These are known as
vertical mergers, and because AT\&T is a telecommunications company,
while Time Warner creates media content like movies, their deal would
fit that category.

Problems with those mergers have generally been resolved with
settlements known as consent decrees, which restrict the new company's
behavior or operations.

Mr. Delrahim is skeptical of such consent decrees, especially demands
for ``behavioral remedies.'' The Obama administration approved Comcast's
purchase of NBC Universal in 2011 with a stack of behavioral
requirements, including a requirement that Comcast make NBC content
available to competing cable and streaming services.

During his interview last month, Mr. Delrahim said industries moved too
fast for those remedies to be effective.

``We have major parts of our industry that are regulated by consent
decrees with the Justice Department,'' he said, referring to settlements
with behavioral remedies.

``I don't think I'm smart enough to figure out where the market and
consumer behavior will be 10 years from now, 20 years from now, 40 years
from now, 100 years from now,'' he continued. The agency, he added, is
still overseeing consent decrees that are a century old.

Instead, Mr. Delrahim said, he prefers so-called structural remedies,
like forcing a company to sell assets before approving a merger. He said
the ultimate test for healthy competition was the consumer welfare test,
which looks at whether consumers are benefiting from lower prices and
options.

Mr. Kimmelman said the Justice Department could be concerned about how
the combined broadband, satellite and media company could hurt
competitors.

The satellite service DirecTV, for example, can reach all American
homes. The potential concern, Mr. Kimmelman said, is that the company
may have an incentive to withhold valuable Time Warner content like HBO,
TNT or CNN from rival cable and satellite providers or streaming
services. The company could also make it harder for rival media
companies like Starz to reach AT\&T customers.

The Justice Department has proposed various options to meet Mr.
Delrahim's demands, according to multiple people close to the
discussions.

In the interview last month, Mr. Delrahim defended the president's
criticism of AT\&T's bid. He said Mr. Trump was unfairly targeted for
remarks about deals when other government officials were allowed to air
their opinions.

``What's interesting is that you see the president getting criticized
for commenting on certain mergers and you have senators who will write
with comments on the same exact merger,'' Mr. Delrahim said. ``I find
that fascinating but maybe not new in Washington.''

And while he insisted that his decisions would be made independently, it
was clear that Mr. Trump was not totally out of mind.

When he arrived at the Justice Department in late September, Mr.
Delrahim received a hat as a welcome gift. It reads: ``Makan Antitrust
Great Again.''

Advertisement

\protect\hyperlink{after-bottom}{Continue reading the main story}

\hypertarget{site-index}{%
\subsection{Site Index}\label{site-index}}

\hypertarget{site-information-navigation}{%
\subsection{Site Information
Navigation}\label{site-information-navigation}}

\begin{itemize}
\tightlist
\item
  \href{https://help.nytimes.com/hc/en-us/articles/115014792127-Copyright-notice}{©~2020~The
  New York Times Company}
\end{itemize}

\begin{itemize}
\tightlist
\item
  \href{https://www.nytco.com/}{NYTCo}
\item
  \href{https://help.nytimes.com/hc/en-us/articles/115015385887-Contact-Us}{Contact
  Us}
\item
  \href{https://www.nytco.com/careers/}{Work with us}
\item
  \href{https://nytmediakit.com/}{Advertise}
\item
  \href{http://www.tbrandstudio.com/}{T Brand Studio}
\item
  \href{https://www.nytimes.com/privacy/cookie-policy\#how-do-i-manage-trackers}{Your
  Ad Choices}
\item
  \href{https://www.nytimes.com/privacy}{Privacy}
\item
  \href{https://help.nytimes.com/hc/en-us/articles/115014893428-Terms-of-service}{Terms
  of Service}
\item
  \href{https://help.nytimes.com/hc/en-us/articles/115014893968-Terms-of-sale}{Terms
  of Sale}
\item
  \href{https://spiderbites.nytimes.com}{Site Map}
\item
  \href{https://help.nytimes.com/hc/en-us}{Help}
\item
  \href{https://www.nytimes.com/subscription?campaignId=37WXW}{Subscriptions}
\end{itemize}
