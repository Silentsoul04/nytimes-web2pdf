Sections

SEARCH

\protect\hyperlink{site-content}{Skip to
content}\protect\hyperlink{site-index}{Skip to site index}

\href{https://www.nytimes.com/section/world/europe}{Europe}

\href{https://myaccount.nytimes.com/auth/login?response_type=cookie\&client_id=vi}{}

\href{https://www.nytimes.com/section/todayspaper}{Today's Paper}

\href{/section/world/europe}{Europe}\textbar{}Northern Ireland Is
Sinking Into a `Profound Crisis'

\url{https://nyti.ms/2hO1cte}

\begin{itemize}
\item
\item
\item
\item
\item
\item
\end{itemize}

Advertisement

\protect\hyperlink{after-top}{Continue reading the main story}

Supported by

\protect\hyperlink{after-sponsor}{Continue reading the main story}

\hypertarget{northern-ireland-is-sinking-into-a-profound-crisis}{%
\section{Northern Ireland Is Sinking Into a `Profound
Crisis'}\label{northern-ireland-is-sinking-into-a-profound-crisis}}

\includegraphics{https://static01.nyt.com/images/2017/11/21/world/21stormont1/21stormont1-articleLarge.jpg?quality=75\&auto=webp\&disable=upscale}

By \href{https://www.nytimes.com/by/patrick-kingsley}{Patrick Kingsley}

\begin{itemize}
\item
  Nov. 20, 2017
\item
  \begin{itemize}
  \item
  \item
  \item
  \item
  \item
  \item
  \end{itemize}
\end{itemize}

BELFAST, Northern Ireland --- The Northern Ireland Assembly's palatial
building in Stormont, in the hills that overlook Belfast, is an eerie
place these days.

This time last year, its grand lobby bustled with lawmakers, lobbyists
and civil servants, but this week it is empty and lifeless. Further
inside, a few tourists warmed the blue seats of the debating chamber,
rather than the 90 lawmakers elected to work there. Even their
microphones had been removed.

``It's a total ghost town,'' said Claire Hanna, one of those lawmakers.
``It's dead.''

And so it has been since as long ago as January, when Northern Ireland's
governing coalition collapsed. That created a power vacuum at Stormont
that has still not been filled, paralyzing the region's already pinched
institutions and threatening
\href{http://www.nytimes.com/1998/10/17/world/2-ulster-peacemakers-win-the-nobel-prize.html}{a
1998 peace deal} that largely ended three decades of fighting between
nationalist and unionist factions.

Since that deal, known as the Good Friday Agreement, Northern Ireland
has been run mostly by a devolved regional government that must, in
effect, be led by a coalition between the region's largest nationalist
party and its largest unionist counterpart.

But in January this delicate arrangement was upended when Sinn Fein,
which hopes for a united Ireland one day,
\href{https://www.nytimes.com/2017/01/16/world/europe/northern-ireland-early-election.html}{withdrew
from a coalition} with the Democratic Unionist Party, which wants
Northern Ireland to remain part of the United Kingdom.

The government stopped working and the assembly stopped meeting. A major
overhaul of the region's ailing health system was postponed, and all
long-term decisions about government spending were put on hold.

``This is a more profound crisis than we've had at other times in the
last 20 years,'' said Stephen Farry, the deputy leader of the Alliance
Party, a centrist group that does not identify as either nationalist or
unionist.

The government was shaken again this weekend, with the news that Gerry
Adams would stand down as president of Sinn Féin. His departure at the
end of the year may help the party achieve its goal of becoming a
palatable coalition partner in the Irish government, but for Northern
Ireland the implications are less clear.

Optimists have expressed hopes that his absence could give the political
parties in Belfast more room for maneuver. But combined with the death
in March of Mr. Adams's former colleague,
\href{https://www.nytimes.com/2017/03/21/world/europe/martin-mcguinness-northern-ireland-ira.html}{Martin
McGuinness}, it deprives the government of established leaders who are
willing or able to make compromises on both sides of the sectarian
divide.

The assembly has previously been suspended, most memorably
\href{http://www.nytimes.com/2002/10/14/world/the-troubles-in-ulster-shift-from-street-to-the-assembly.html}{between
2002 and 2007}. ``But then there was the sense that this was a blip and
the problems would be overcome,'' Mr. Farry said.

\includegraphics{https://static01.nyt.com/images/2017/11/21/world/21stormont2/21stormont2-articleLarge.jpg?quality=75\&auto=webp\&disable=upscale}

``This crisis has a different feel to it,'' he added. ``There's a much
more profound question over the direction of travel and whether
power-sharing is sustainable. And that begs the question: What happens
next?''

In the short term, civil servants have been left in charge of the
day-to-day management of the region. But if the impasse continues, many
fear that decisions will be made by ministers in London, a system known
as ``direct rule'' that the 1998 agreement was meant to end. In the
absence of a Northern Irish finance minister, the British government has
already made the first step toward ``direct rule'' --- passing an
interim budget for Northern Ireland on Monday to ensure that salaries
can be paid until March.

The specter of Brexit compounds the crisis: When the United Kingdom
leaves the European Union,
\href{https://www.nytimes.com/2017/08/05/world/europe/brexit-northern-ireland-ireland.html}{border
controls may be reintroduced} between Northern and southern Ireland for
the first time since the Good Friday Agreement was introduced.

Some now question whether the agreement has in essence collapsed, and
fear, if not a return to anything like a full-blown conflict, then at
least a rise in
\href{https://www.nytimes.com/2017/10/11/world/europe/belfast-catholics-protestants-cantrell-close.html}{sporadic
acts of paramilitary intimidation}.

``I don't mean to be dramatic or anything, but I do think the Good
Friday Agreement is effectively dead,'' said Ms. Hanna, who represents
the Social Democratic and Labour Party, Sinn Fein's main nationalist
rival.

``I don't think there's any real support for violence, but you can see
how quickly things can unravel,'' she added. ``It's very bleak, and it
is something to worry about.''

\hypertarget{a-stagnated-state}{%
\subsection{A Stagnated State}\label{a-stagnated-state}}

For now, civil servants are keeping the machinery of the state in
motion. But as unelected employees, they must simply apply the decisions
made by the elected government during the last financial year, meaning
that they can neither react to changing circumstances nor increase their
budgets in line with inflation.

This has led to stasis and, in some cases, cuts to jobs and services. In
one high-profile case, officials were forced to scrap an intensive
support program for around 1,000 of the region's most vulnerable
children and teenagers, even though the program had led to a significant
drop in antisocial behavior and a rise in school attendance.

These children were ``the first victims of this political impasse,''
said Charlie Mack, the chief executive of
\href{http://www.extern.org/}{Extern}, the charity that ran the program
on the government's behalf. ``Which is disgraceful.''

The health system has also been affected. With the longest wait times in
the United Kingdom, the sector had been due for a major overhaul this
year. But that has been on hold since the start of the crisis, said
\href{https://www.rcn.org.uk/about-us/our-structure/rcn-executive-team}{Janice
Smyth}, the Northern Ireland director at the Royal College of Nursing,
the world's largest nursing union. ``It cannot happen until we get a
minister,'' she said.

Across the region's charitable sector, which is significantly dependent
on state funding, charities have become wary of starting new projects or
hiring staff, making them less able to respond to social problems that
require new strategies.

Image

The Democratic Unionist Party leader, Arlene Foster, and deputy leader,
Nigel Dodds, at a news conference at Stormont Castle in
June.Credit...Charles Mcquillan/Getty Images

``There's been a stagnation,'' said Seamus McAleavey, the chief
executive of \href{http://www.nicva.org/about-us}{Northern Ireland
Council for Voluntary Action}, a group that represents more than 1,000
nongovernmental organizations. ``Treading water is probably a fairly
good description.''

\hypertarget{the-blame-game}{%
\subsection{The Blame Game}\label{the-blame-game}}

Yet these administrative challenges have not yet encouraged politicians
to reach a settlement. In fact, their differences have increased as the
year has developed.

When they left the coalition in January, Sinn Fein originally intended
to force the resignation of the D.U.P. leader, Arlene Foster, who was
accused of corruption. But after Ms. Foster refused to step down, Sinn
Fein expanded its demands, calling on the D.U.P. to agree to give the
Irish language the same legal status as English and to legalize same-sex
marriage. (In Scotland and Wales, the Scottish and Welsh languages
already have an equal legal footing with English, and same-sex marriage
has been permitted in the rest of the United Kingdom since 2014.)

Both sides have been accused of not being truly invested in a
compromise.

After the Conservative party lost its majority this summer in the
national Parliament in Westminster, it turned to
\href{https://www.nytimes.com/2017/06/10/world/europe/britain-election-dup-northern-ireland.html}{the
D.U.P.'s small group of national lawmakers} to ensure that it remained
in power. And there are now suspicions that ``the Westminster deal has
made them less interested in Stormont politics,'' said
\href{https://www.ulster.ac.uk/staff/cp-mcgrattan}{Cillian McGrattan}, a
politics professor at Ulster University. ``I think they have a view that
if they're going to get things done, they're going to get them done in
Westminster.''

The D.U.P. dismisses this argument and says that in fact, it is Sinn
Fein that has ulterior motives for maintaining the standoff. Since Sinn
Fein also operates in the Republic of Ireland, it fears that it will
harm its reputation among left-wing voters in Dublin if it is forced to
pursue neoliberal policies while in government in Belfast, said Sammy
Wilson, a D.U.P. lawmaker.

Sinn Fein scoffs at this. ``Why would it not be great for Sinn Fein to
be running things?'' asked \href{http://newbelfast.com/}{Mairtin O
Muilleoir}, the Sinn Fein finance minister at the time of the
coalition's collapse. ``The months I spent as finance minister burnished
and enhanced Sinn Fein's reputation for getting things done.''

\hypertarget{a-new-paradigm}{%
\subsection{A New Paradigm?}\label{a-new-paradigm}}

As the standoff drags on, and polarization increases, people find it
harder to envisage Northern Ireland as an autonomous entity. ``We're
back to this binary situation where people either see it as a
problematic part of the U.K. or as a part of united Ireland,'' said
\href{http://www.qub.ac.uk/research-centres/CentreforIrishPolitics/Staff/ProfGrahamWalker/}{Graham
Walker}, a politics professor at Queen's University, Belfast.

To resolve the crisis in the long term, some suggest reshaping the Good
Friday Agreement to allow for other kinds of coalitions, instead of a
mandatory partnership between the region's two largest nationalist and
unionist factions. Others predict a referendum on Irish reunification
within a decade, arguing that the current dysfunction, coupled with the
\href{https://www.nytimes.com/2017/08/05/world/europe/brexit-northern-ireland-ireland.html}{fallout
from Brexit}, may encourage moderate nationalists to see a united
Ireland as a more urgent priority than they did previously.

Whereas Stormont was once seen as a symbol of hope and progress,
citizens of all backgrounds now see it as an embarrassment.

Ricky Rowledge, the head of
\href{http://www.chni.org.uk/directorate.html}{a homeless charity} who
meets lawmakers at Stormont as part of her work, said she once felt a
sense of civic pride on her visits there.

``Now, I go up to Stormont and it's this great big echoey, empty,
impotent place,'' Ms. Rowledge said. ``And it's heartbreaking.''

Advertisement

\protect\hyperlink{after-bottom}{Continue reading the main story}

\hypertarget{site-index}{%
\subsection{Site Index}\label{site-index}}

\hypertarget{site-information-navigation}{%
\subsection{Site Information
Navigation}\label{site-information-navigation}}

\begin{itemize}
\tightlist
\item
  \href{https://help.nytimes.com/hc/en-us/articles/115014792127-Copyright-notice}{©~2020~The
  New York Times Company}
\end{itemize}

\begin{itemize}
\tightlist
\item
  \href{https://www.nytco.com/}{NYTCo}
\item
  \href{https://help.nytimes.com/hc/en-us/articles/115015385887-Contact-Us}{Contact
  Us}
\item
  \href{https://www.nytco.com/careers/}{Work with us}
\item
  \href{https://nytmediakit.com/}{Advertise}
\item
  \href{http://www.tbrandstudio.com/}{T Brand Studio}
\item
  \href{https://www.nytimes.com/privacy/cookie-policy\#how-do-i-manage-trackers}{Your
  Ad Choices}
\item
  \href{https://www.nytimes.com/privacy}{Privacy}
\item
  \href{https://help.nytimes.com/hc/en-us/articles/115014893428-Terms-of-service}{Terms
  of Service}
\item
  \href{https://help.nytimes.com/hc/en-us/articles/115014893968-Terms-of-sale}{Terms
  of Sale}
\item
  \href{https://spiderbites.nytimes.com}{Site Map}
\item
  \href{https://help.nytimes.com/hc/en-us}{Help}
\item
  \href{https://www.nytimes.com/subscription?campaignId=37WXW}{Subscriptions}
\end{itemize}
