Sections

SEARCH

\protect\hyperlink{site-content}{Skip to
content}\protect\hyperlink{site-index}{Skip to site index}

\href{https://www.nytimes.com/section/food}{Food}

\href{https://myaccount.nytimes.com/auth/login?response_type=cookie\&client_id=vi}{}

\href{https://www.nytimes.com/section/todayspaper}{Today's Paper}

\href{/section/food}{Food}\textbar{}Mario Batali Steps Away From
Restaurants Amid Sexual Misconduct Allegations

\url{https://nyti.ms/2jNUgKj}

\begin{itemize}
\item
\item
\item
\item
\item
\item
\end{itemize}

Advertisement

\protect\hyperlink{after-top}{Continue reading the main story}

Supported by

\protect\hyperlink{after-sponsor}{Continue reading the main story}

\hypertarget{mario-batali-steps-away-from-restaurants-amid-sexual-misconduct-allegations}{%
\section{Mario Batali Steps Away From Restaurants Amid Sexual Misconduct
Allegations}\label{mario-batali-steps-away-from-restaurants-amid-sexual-misconduct-allegations}}

\includegraphics{https://static01.nyt.com/images/2017/12/12/us/12xp-batali/12xp-batali-articleLarge-v2.jpg?quality=75\&auto=webp\&disable=upscale}

By \href{http://www.nytimes.com/by/christine-hauser}{Christine Hauser},
\href{http://www.nytimes.com/by/kim-severson}{Kim Severson} and
\href{http://www.nytimes.com/by/julia-moskin}{Julia Moskin}

\begin{itemize}
\item
  Dec. 11, 2017
\item
  \begin{itemize}
  \item
  \item
  \item
  \item
  \item
  \item
  \end{itemize}
\end{itemize}

The celebrity chef Mario Batali, one of the country's most high-profile
restaurateurs, is stepping away from the daily operations of his
businesses and the daytime program he co-hosts on ABC, ``The Chew,''
amid allegations of sexual misconduct.

Mr. Batali released a statement after a report was
\href{https://ny.eater.com/2017/12/11/16759540/mario-batali-sexual-misconduct-allegations}{published
Monday on Eater, the food website,} that said four women had alleged
that Mr. Batali touched them inappropriately in a pattern of behavior
that appeared to span at least two decades. Three of the women worked
for Mr. Batali, and the fourth worked in the restaurant industry, Eater
reported.

In his statement, Mr. Batali apologized and said that the accusations
``match up'' with his behavior:

``Although the identities of most of the individuals mentioned in these
stories have not been revealed to me, much of the behavior described
does, in fact, match up with ways I have acted,'' he said in the
statement. ``That behavior was wrong and there are no excuses.''

``I have work to do to try to regain the trust of those I have hurt and
disappointed,'' he added. ``For this reason, I am going to step away
from day-to-day operations of my businesses.''

It was not immediately clear how long Mr. Batali planned to stay away
from his businesses.

Mr. Batali, 57, made his remarks specifically in response to the Eater
report, which included details from interviews with the four women, who
were not identified in the report.

They described behavior that included breast groping and being grabbed
from behind. In one instance, a woman said she was compelled to straddle
Mr. Batali to get past him as he sat blocking an exit.

Mr. Batali had also recently been reprimanded because of a complaint
made in October by an employee at one of the more than 20 restaurants in
the Batali \& Bastianich Hospitality Group, a company spokesman said on
Monday.

The spokesman said that Mr. Batali had been required to undergo sexual
harassment training above what is already required of employees. He then
volunteered to keep away from the restaurant where the employee worked,
and he has done so, said the spokesman, who declined to be identified by
name.

The spokesman would not name the restaurant where the woman worked, and
it was not immediately clear whether the employee was among the women
interviewed in the Eater report.

After the Eater report was released on Monday, Batali \& Bastianich
Hospitality Group said in a statement that it had taken further measures
that extended to all his restaurants.

``Mr. Batali and we have agreed that he will step away from the
company's operations, including the restaurants, and he has already done
so,'' it said. It also said that the company had provided employees with
access to an outside investigations firm if they want to make claims
against corporate officers or owners.

The recent allegations against Mr. Batali led ABC to ask Mr. Batali, who
has been on ``The Chew'' since 2011, to step away ``while we review the
allegations that have just recently come to our attention,'' the network
said in a statement on Monday.

``ABC takes matters like this very seriously as we are committed to a
safe work environment,'' the statement said. ``While we are unaware of
any type of inappropriate behavior involving him and anyone affiliated
with the show, we will swiftly address any alleged violations of our
standards of conduct.''

The allegations against Mr. Batali were among the latest to be made
against prominent men in several industries following a
\href{https://www.nytimes.com/2017/10/05/us/harvey-weinstein-harassment-allegations.html}{New
York Times report}in October about women accusing the Hollywood mogul
Harvey Weinstein of sexual assault and harassment.

The revelations about Mr. Batali have shaken up the food industry, where
he is also a best-selling author of cookbooks. He has long appeared as a
television personality in cooking competitions, including ``Iron Chef
America'' and ``Top Chef.''

Last year, he was enlisted by Michelle Obama to put together the last
state dinner of the Obama presidency.

The allegations concerning Mr. Batali drew quick response from the food
industry, including suggestions that such behavior was widespread.

Tiffani Faison, an American chef who was a finalist on the first season
of Bravo's reality show ``Top Chef,'' suggested that there was a culture
of silence in professional kitchens.

\href{https://www.nytimes.com/interactive/2017/11/10/us/men-accused-sexual-misconduct-weinstein.html}{}

\includegraphics{https://static01.nyt.com/images/2017/11/10/us/men-accused-sexual-misconduct-weinstein-1510370544262/men-accused-sexual-misconduct-weinstein-1510370544262-articleLarge-v2.jpg}

\hypertarget{after-weinstein-71-men-accused-of-sexual-misconduct-and-their-fall-from-power}{%
\subsection{After Weinstein: 71 Men Accused of Sexual Misconduct and
Their Fall From
Power}\label{after-weinstein-71-men-accused-of-sexual-misconduct-and-their-fall-from-power}}

A list of men who have resigned, been fired or otherwise lost power
since the Harvey Weinstein scandal broke.

``I cannot believe we are in a true watershed moment when NOT ONE MAN
has gotten ahead of allegations,'' she
\href{https://twitter.com/tiffanifaison/status/940249384926744576}{wrote
on Twitter}. ``They all know what they did and are just hoping their
number doesn't come up. That is the opposite of integrity.''

Like many of Mr. Batali's colleagues, Traci Des Jardins, the San
Francisco chef and restaurateur who has known Mr. Batali since the late
1980s, was conflicted about the reports.

She and others expressed sympathy for the thousands of people whose
livelihoods depend on the Batali brand, as they did when allegations
surfaced against John Besh, the high-profile New Orleans restaurateur,
in October. In that city, people have vowed not to go to Mr. Besh's
restaurants, and calls for boycotts of Mr. Batali's restaurants arose,
although some expressed caution about such a boycott, saying it would
hurt workers who had nothing to do with his behavior.

The chef and television personality Tom Coliccho, who has been one of
the few male chefs speaking publicly about sexual harassment in the
restaurant business since news about Mr. Weinstein and others came out,
said he was not surprised to hear about Mr. Batali.

``Am I supposed to report rumor and innuendo and suspicion?'' he asked.
``It's not my story to tell. This is about the women and it damages the
`me, too' movement if we don't let them speak for themselves.''

Mr. Batali is among a small group of chefs who helped diners deftly
navigate a transition from decades of Eurocentric dining to one more
distinctly American, and became a celebrity doing it.

Although Italian-American chefs had long been cooking regional
specialties, Mr. Batali is largely credited with educating an entire
nation on the delights of tripe, beef cheek ravioli and spicy squid
through his restaurants and his first television show, ``Molto Mario,''
which ran from 1996 to 2004.

He was raised outside of Seattle, where his family still lives and
operates Salumi, a widely praised cured meat shop in Pioneer Square.
After graduating from Rutgers University and training in kitchens in
Italy, he moved to New York in 1992. From there he built an
international empire with his partner Joe Bastianich that, in addition
to more than 20 restaurants, includes the Eataly Italian market
franchise and a television career that ranged from scrappy shows when
Food Network was in its infancy to a regular spot on ``The Chew.''

At his first restaurant, the tiny Po in Greenwich Village, he made his
name with fresh pastas and other dishes that, while not quite
authentically Italian, were brightly flavored and deeply appealing. Po
opened in 1993, when most American food lovers were just beginning to
understand the world of Italian food in between the two poles of
red-sauce ``Southern'' and cream-rich ``Northern.''

In 1998, he opened Babbo Ristorante in New York's West Village. It was
an immediate hit, receiving three stars from Ruth Reichl, then the New
York Times food critic. His white-tablecloth place in the meatpacking
district, Del Posto, was the first Italian restaurant to earn four stars
from The New York Times, bestowed by Sam Sifton.

Advertisement

\protect\hyperlink{after-bottom}{Continue reading the main story}

\hypertarget{site-index}{%
\subsection{Site Index}\label{site-index}}

\hypertarget{site-information-navigation}{%
\subsection{Site Information
Navigation}\label{site-information-navigation}}

\begin{itemize}
\tightlist
\item
  \href{https://help.nytimes.com/hc/en-us/articles/115014792127-Copyright-notice}{©~2020~The
  New York Times Company}
\end{itemize}

\begin{itemize}
\tightlist
\item
  \href{https://www.nytco.com/}{NYTCo}
\item
  \href{https://help.nytimes.com/hc/en-us/articles/115015385887-Contact-Us}{Contact
  Us}
\item
  \href{https://www.nytco.com/careers/}{Work with us}
\item
  \href{https://nytmediakit.com/}{Advertise}
\item
  \href{http://www.tbrandstudio.com/}{T Brand Studio}
\item
  \href{https://www.nytimes.com/privacy/cookie-policy\#how-do-i-manage-trackers}{Your
  Ad Choices}
\item
  \href{https://www.nytimes.com/privacy}{Privacy}
\item
  \href{https://help.nytimes.com/hc/en-us/articles/115014893428-Terms-of-service}{Terms
  of Service}
\item
  \href{https://help.nytimes.com/hc/en-us/articles/115014893968-Terms-of-sale}{Terms
  of Sale}
\item
  \href{https://spiderbites.nytimes.com}{Site Map}
\item
  \href{https://help.nytimes.com/hc/en-us}{Help}
\item
  \href{https://www.nytimes.com/subscription?campaignId=37WXW}{Subscriptions}
\end{itemize}
