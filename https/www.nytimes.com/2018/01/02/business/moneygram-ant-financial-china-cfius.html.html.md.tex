Sections

SEARCH

\protect\hyperlink{site-content}{Skip to
content}\protect\hyperlink{site-index}{Skip to site index}

\href{https://www.nytimes.com/section/business}{Business}

\href{https://myaccount.nytimes.com/auth/login?response_type=cookie\&client_id=vi}{}

\href{https://www.nytimes.com/section/todayspaper}{Today's Paper}

\href{/section/business}{Business}\textbar{}MoneyGram and Ant Financial
Call Off Merger, Citing Regulatory Concerns

\url{https://nyti.ms/2DRSf7O}

\begin{itemize}
\item
\item
\item
\item
\item
\end{itemize}

Advertisement

\protect\hyperlink{after-top}{Continue reading the main story}

Supported by

\protect\hyperlink{after-sponsor}{Continue reading the main story}

\hypertarget{moneygram-and-ant-financial-call-off-merger-citing-regulatory-concerns}{%
\section{MoneyGram and Ant Financial Call Off Merger, Citing Regulatory
Concerns}\label{moneygram-and-ant-financial-call-off-merger-citing-regulatory-concerns}}

\includegraphics{https://static01.nyt.com/images/2018/01/03/business/03DC-MONEYGRAM/03DC-MONEYGRAM-articleLarge.jpg?quality=75\&auto=webp\&disable=upscale}

By \href{https://www.nytimes.com/by/ana-swanson}{Ana Swanson} and
\href{https://www.nytimes.com/by/paul-mozur}{Paul Mozur}

\begin{itemize}
\item
  Jan. 2, 2018
\item
  \begin{itemize}
  \item
  \item
  \item
  \item
  \item
  \end{itemize}
\end{itemize}

\href{https://cn.nytimes.com/business/20180104/moneygram-ant-financial-china-cfius/}{阅读简体中文版}\href{https://cn.nytimes.com/business/20180104/moneygram-ant-financial-china-cfius/zh-hant/}{閱讀繁體中文版}

WASHINGTON --- United States officials have effectively killed a Chinese
company's \$1.2 billion plan to buy MoneyGram, the money transfer
company, signaling the Trump administration's growing skepticism of
Chinese purchases of American companies and of broader business ties
between the two economic powers.

MoneyGram and Ant Financial, the Chinese electronic payments company,
said Tuesday that they had canceled the deal after failing to win
approval from a Washington panel that reviews foreign purchases of
American companies.

The move comes amid a shifting relationship between the two countries.
The MoneyGram-Ant Financial deal collapsed as tensions have grown over
who will control the technologies of the future. As the scotched deal
shows, China and America are also increasingly at loggerheads over
whether private data will be kept safe as money and corporate ownership
cross borders.

``The geopolitical environment has changed considerably since we first
announced the proposed transaction with Ant Financial nearly a year
ago,'' the chief executive of MoneyGram, Alex Holmes, said.

The deal failed despite a charm offensive by Jack Ma, the Chinese
internet tycoon who controls Ant Financial. Shortly after Mr. Trump won
the 2016 election, Mr. Ma famously stood with the president-elect in
Trump Tower in New York and pledged that his e-commerce empire would
help create one million American jobs.

\includegraphics{https://static01.nyt.com/images/2017/01/09/business/cnbc-alibaba/cnbc-alibaba-videoSixteenByNineJumbo1600.png}

That push apparently could not compete with Mr. Trump's concerns about
Chinese money and deal makers, and with his administration's concerns
about Chinese acquisitions of American know-how. The administration is
finishing an investigation into Chinese theft of intellectual property
owned by American companies, which could result in tariffs on Chinese
imports or further restrictions on Chinese investment.

At a daily news conference on Wednesday, Geng Shuang, a spokesman for
China's Foreign Ministry, said that the government hopes the United
States could create a level playing field for Chinese companies.

The Trump administration could soon take other steps to stop what it
considers unfair Chinese trade practices in China in areas
\href{https://www.nytimes.com/2017/12/01/business/trump-china-trade-solar.html}{like
solar panels}.

For now, the collapse of the MoneyGram deal could spell trouble for
other Chinese companies considering major acquisitions. The effort by
Ant Financial, the payments affiliate of the internet giant Alibaba, was
\href{https://www.nytimes.com/2017/01/31/business/dealbook/moneygram-china-alibaba-ant-financial.html}{seen
as a test} of the Trump administration's political and regulatory
approach to China.

Other deals under review include the sale of SkyBridge Capital --- an
investment firm founded by Anthony Scaramucci, an ally of Mr. Trump ---
to an arm of HNA Group, a Chinese conglomerate.

The collapse of the deal offers the latest sign of a shift --- one that
began during the end of the Obama administration --- over how Washington
treats Chinese acquisitions of American assets.

Mr. Trump and other politicians have long criticized Chinese trade
practices and their impact on traditional, heavy industries in the
United States, such as steel. But officials and lawmakers of both
parties have grown increasingly concerned with Chinese purchases in
high-tech areas, like semiconductors and artificial intelligence. China
--- a country that has grown by leaps and bounds in expertise in recent
years --- has aggressively funded development of next-generation
technology in
\href{https://www.nytimes.com/2017/11/07/business/made-in-china-technology-trade.html}{ways
that have alarmed} some foreign governments and businesses.

That has led to calls to tighten reviews of Chinese purchases of
American assets. Chinese investment surged to more than \$130 billion in
2017 from just \$21.5 billion in 2012, according to Rhodium Group, a
research firm. In 2018, bankers say they expect another big year of
investment, targeted more at sectors considered important to future
economic growth like high technology and renewable energy.

Lawmakers from both parties have
\href{https://www.nytimes.com/2017/11/08/us/politics/china-foreign-investments.html}{introduced
legislation} calling for greater scrutiny of Chinese investments in the
United States. They are pushing for an expansion of the Committee on
Foreign Investment in the United States, or Cfius, the multiagency panel
that reviews foreign deals for potential threats to national security
and that rejected the MoneyGram-Ant Financial plan.

In 2016, American
\href{https://www.nytimes.com/2016/02/05/technology/concern-grows-in-us-over-chinas-drive-to-make-chips.html}{concerns}
over China's interest in microchips scotched
\href{https://www.nytimes.com/2016/12/02/business/dealbook/china-aixtron-obama-cfius.html}{some
Chinese purchases}, and last year Mr. Trump
\href{https://www.nytimes.com/2017/09/13/business/trump-lattice-semiconductor-china.html}{blocked
the purchase of Lattice Semiconductors} by a group with links to China.

The deal's demise also illustrates rising concerns in both the United
States and China over sensitive personal data. In buying a large-scale
money-transfer company like MoneyGram, Ant Financial would have had
access to a large number of records of financial flows within the United
States. That fact, combined with close ties to China's government, could
have created security problems. For example, the flow of funds could
point to possible espionage targets in the United States military who
face financial difficulties.

Ant Financial has disputed such claims, saying that consumer data would
continue to be stored in the United States.

``This deal fell apart because Ant Financial is from China, a country
that has little credibility when it comes to protecting personal data,''
said Scott Kennedy, a fellow at the Center for Strategic and
International Studies. ``Add to that the lack of reciprocity --- China
would never countenance such a deal in reverse --- and one could ask why
Alibaba would expect to find a sympathetic hearing in Washington.''

Critics of the deal also asked whether Ant Financial would be able to
properly police transactions linked to money laundering or terrorism.
Ant Financial had said the deal would give MoneyGram greater resources
to toughen its efforts to prevent money laundering.

China, as well, has taken steps to stanch American company access to
data. A new cybersecurity law there forces American companies to keep
data on Chinese citizens stored within China, while Chinese state-media
have run public service announcements warning about private data
collected by Apple's iPhones.

``Data flows out of and into China are an increasing source of suspicion
in the bilateral relationship,'' said Adam Segal, a senior fellow at the
Council on Foreign Relations in New York. He said Washington was in the
``paradoxical position'' of criticizing the new Chinese law while
opposing deals like MoneyGram because of national security concerns.

For Ant Financial, the decision is a setback to its efforts to extend
its strong position in online payments abroad. Ant Financial, which grew
out of the Chinese payment service Alipay, wanted to buy MoneyGram to
accelerate its global expansion. In the past several years, the company
has pushed into India, South Korea and Thailand.

Mr. Ma, the Chinese internet tycoon, has worked hard to put a positive
face on his companies in the United States. He also controls the Alibaba
Group, China's largest internet company, and last year he hosted a
conference in Detroit aimed at helping American businesses sell on
Alibaba's e-commerce sites.

Amid the heightened sensitivity to overseas purchases, MoneyGram and Ant
Financial said it was clear that Cfius would not approve the deal,
despite their efforts to quell concerns.

Advertisement

\protect\hyperlink{after-bottom}{Continue reading the main story}

\hypertarget{site-index}{%
\subsection{Site Index}\label{site-index}}

\hypertarget{site-information-navigation}{%
\subsection{Site Information
Navigation}\label{site-information-navigation}}

\begin{itemize}
\tightlist
\item
  \href{https://help.nytimes.com/hc/en-us/articles/115014792127-Copyright-notice}{©~2020~The
  New York Times Company}
\end{itemize}

\begin{itemize}
\tightlist
\item
  \href{https://www.nytco.com/}{NYTCo}
\item
  \href{https://help.nytimes.com/hc/en-us/articles/115015385887-Contact-Us}{Contact
  Us}
\item
  \href{https://www.nytco.com/careers/}{Work with us}
\item
  \href{https://nytmediakit.com/}{Advertise}
\item
  \href{http://www.tbrandstudio.com/}{T Brand Studio}
\item
  \href{https://www.nytimes.com/privacy/cookie-policy\#how-do-i-manage-trackers}{Your
  Ad Choices}
\item
  \href{https://www.nytimes.com/privacy}{Privacy}
\item
  \href{https://help.nytimes.com/hc/en-us/articles/115014893428-Terms-of-service}{Terms
  of Service}
\item
  \href{https://help.nytimes.com/hc/en-us/articles/115014893968-Terms-of-sale}{Terms
  of Sale}
\item
  \href{https://spiderbites.nytimes.com}{Site Map}
\item
  \href{https://help.nytimes.com/hc/en-us}{Help}
\item
  \href{https://www.nytimes.com/subscription?campaignId=37WXW}{Subscriptions}
\end{itemize}
