Sections

SEARCH

\protect\hyperlink{site-content}{Skip to
content}\protect\hyperlink{site-index}{Skip to site index}

\href{https://myaccount.nytimes.com/auth/login?response_type=cookie\&client_id=vi}{}

\href{https://www.nytimes.com/section/todayspaper}{Today's Paper}

\href{/section/opinion}{Opinion}\textbar{}No College Kid Needs a Water
Park to Study

\href{https://nyti.ms/2FlBsvo}{https://nyti.ms/2FlBsvo}

\begin{itemize}
\item
\item
\item
\item
\item
\end{itemize}

Advertisement

\protect\hyperlink{after-top}{Continue reading the main story}

Supported by

\protect\hyperlink{after-sponsor}{Continue reading the main story}

\href{/section/opinion}{Opinion}

\href{/column/on-campus}{On Campus}

\hypertarget{no-college-kid-needs-a-water-park-to-study}{%
\section{No College Kid Needs a Water Park to
Study}\label{no-college-kid-needs-a-water-park-to-study}}

By James V. Koch

\begin{itemize}
\item
  Jan. 9, 2018
\item
  \begin{itemize}
  \item
  \item
  \item
  \item
  \item
  \end{itemize}
\end{itemize}

\includegraphics{https://static01.nyt.com/images/2018/01/09/opinion/09oncampusWeb/09oncampusWeb-articleLarge.jpg?quality=75\&auto=webp\&disable=upscale}

In a competition to woo students, public universities are increasingly
offering lavish amenities that have nothing to do with education.

The
\href{https://www.nytimes.com/2014/09/21/fashion/college-recreation-now-includes-pool-parties-and-river-rides.html}{latest
trend} is lazy rivers, which have been installed at several big
institutions, including the Universities of Alabama, Iowa and Missouri.
Last year, Louisiana State University topped them all with a
536-foot-long
\href{https://www.chronicle.com/article/The-Lure-of-the-Lazy-River/241434}{``leisure''
river} in the shape of the letters ``LSU,'' part of an \$85 million
renovation and expansion of its gym. It was L.S.U. students who footed
the bill.

At a time when college has never been more expensive, this is the last
thing students should be paying for. According to the College Board,
tuition and fees at public four-year institutions grew
\href{https://trends.collegeboard.org/college-pricing/figures-tables/published-prices-national}{more
than 60 percent over the past 10 years}. State budgets for higher
education have been slashed, and students have to make up the
difference.

In the case of L.S.U., the lazy river was financed entirely by student
fees, an addendum to their annual tuition.
\href{https://www.chronicle.com/article/The-Lure-of-the-Lazy-River/241434}{According
to the Chronicle of Higher Education}, over the past five years, those
fees increased by 60 percent, nearly triple the amount L.S.U. students
paid in 2000.

Tuition and fee hikes at public universities don't come out of nowhere.
Each has to be approved by a school's governing board, whose trustees
are typically appointed by the governor. Ensuring affordable, quality
education is an essential part of trustees' responsibility, but
unfortunately often not part of their practice.

Trustees of public universities are stewards of a public trust that
rests nobly on the notion that an enlightened citizenry is vital to a
democratic society. They have a fiduciary duty to represent the citizens
and taxpayers who support public institutions of higher education, as
well as the students who attend them. But even though the best interests
of students and taxpayers revolve around college access, affordability
and graduation outcomes, too often presidents and boards are more
focused on the rankings, reputation and popularity of the institution
itself.

In my career as the president of two state universities and a consultant
to nearly 50 higher-education institutions, I've observed dozens of
college presidents skillfully co-opt their governing boards into
approving costly projects that make schools look more attractive. (Of
course, every college president has to increase costs sometimes. But the
goal is to make sure it is necessary, while keeping expenses as low as
possible for students.)

Trustees, who typically meet four to eight times each year, are
entertained as if they are visiting heads of state, flattered for their
service and financial contributions to the institution. College
presidents sweeten requests for new buildings and research centers, as
well as additional student affairs programming, with cleverly branded
words like ``promise'' and ``excellence.'' What board would want to
withhold promise and excellence from its beloved student body?

College presidents also tranquilize trustees into agreement with
impossibly large volumes of reading material. Trustees get binders full
of documentation about institutional successes that are padded with
expensive plans for increasing growth and reputation. Most come away
impressed by their president's expertise and vision and assured that ---
thanks to their efforts --- the university is on the right track.

The unfortunate truth is that while most college presidents care deeply
about their institution's success, an important part of their job is to
shake free more resources. They seldom initiate serious campaigns to
contain costs.

This means it falls on trustees to be better prepared to help challenge
costly proposals that don't add educational value. When it comes to
state schools, the states themselves should educate trustees to
understand their responsibilities to the citizenry and students.
Training on big-picture issues and higher-education trends, such as the
financial trade-off between instruction and research, the costs of
intercollegiate athletics, and the expansion of amenities, would help
trustees develop courage to ask college presidents probing questions
that look beyond institutional narratives and cherry-picked rhetoric.

Our nation's governors must also play a role. As they appoint public
university trustees, they can and should mandate training to make
university boards responsible to taxpayers and students. I don't mean to
imply that trustees should devote themselves to ritual opposition to
presidents, who usually possess an unmatched understanding of the
institutions they lead.

But presidents are not infallible.

Advertisement

\protect\hyperlink{after-bottom}{Continue reading the main story}

\hypertarget{site-index}{%
\subsection{Site Index}\label{site-index}}

\hypertarget{site-information-navigation}{%
\subsection{Site Information
Navigation}\label{site-information-navigation}}

\begin{itemize}
\tightlist
\item
  \href{https://help.nytimes.com/hc/en-us/articles/115014792127-Copyright-notice}{©~2020~The
  New York Times Company}
\end{itemize}

\begin{itemize}
\tightlist
\item
  \href{https://www.nytco.com/}{NYTCo}
\item
  \href{https://help.nytimes.com/hc/en-us/articles/115015385887-Contact-Us}{Contact
  Us}
\item
  \href{https://www.nytco.com/careers/}{Work with us}
\item
  \href{https://nytmediakit.com/}{Advertise}
\item
  \href{http://www.tbrandstudio.com/}{T Brand Studio}
\item
  \href{https://www.nytimes.com/privacy/cookie-policy\#how-do-i-manage-trackers}{Your
  Ad Choices}
\item
  \href{https://www.nytimes.com/privacy}{Privacy}
\item
  \href{https://help.nytimes.com/hc/en-us/articles/115014893428-Terms-of-service}{Terms
  of Service}
\item
  \href{https://help.nytimes.com/hc/en-us/articles/115014893968-Terms-of-sale}{Terms
  of Sale}
\item
  \href{https://spiderbites.nytimes.com}{Site Map}
\item
  \href{https://help.nytimes.com/hc/en-us}{Help}
\item
  \href{https://www.nytimes.com/subscription?campaignId=37WXW}{Subscriptions}
\end{itemize}
