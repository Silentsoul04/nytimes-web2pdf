Sections

SEARCH

\protect\hyperlink{site-content}{Skip to
content}\protect\hyperlink{site-index}{Skip to site index}

\href{https://www.nytimes.com/section/technology}{Technology}

\href{https://myaccount.nytimes.com/auth/login?response_type=cookie\&client_id=vi}{}

\href{https://www.nytimes.com/section/todayspaper}{Today's Paper}

\href{/section/technology}{Technology}\textbar{}Facial Recognition Is
Accurate, if You're a White Guy

\url{https://nyti.ms/2BNurVq}

\begin{itemize}
\item
\item
\item
\item
\item
\end{itemize}

Advertisement

\protect\hyperlink{after-top}{Continue reading the main story}

Supported by

\protect\hyperlink{after-sponsor}{Continue reading the main story}

\hypertarget{facial-recognition-is-accurate-if-youre-a-white-guy}{%
\section{Facial Recognition Is Accurate, if You're a White
Guy}\label{facial-recognition-is-accurate-if-youre-a-white-guy}}

By \href{http://www.nytimes.com/by/steve-lohr}{Steve Lohr}

\begin{itemize}
\item
  Feb. 9, 2018
\item
  \begin{itemize}
  \item
  \item
  \item
  \item
  \item
  \end{itemize}
\end{itemize}

Facial recognition technology is improving by leaps and bounds. Some
commercial software can now tell the gender of a person in a photograph.

When the person in the photo is a white man, the software is right 99
percent of the time.

But the darker the skin, the more errors arise --- up to nearly 35
percent for images of darker skinned women, according to a new study
that breaks fresh ground by measuring how the technology works on people
of different races and gender.

These disparate results, calculated by Joy Buolamwini, a researcher at
the M.I.T. Media Lab, show how some of the biases in the real world can
seep into artificial intelligence, the computer systems that inform
facial recognition.

In modern artificial intelligence, data rules. A.I. software is only as
smart as the data used to train it. If there are many more white men
than black women in the system, it will be worse at identifying the
black women.

One widely used facial-recognition data set was estimated to be more
than 75 percent male and more than 80 percent white, according to
another research study.

The new study also raises broader questions of fairness and
accountability in artificial intelligence at a time when investment in
and adoption of the technology is racing ahead.

Today, facial recognition software is being deployed by companies in
various ways, including to help target product pitches based on social
media profile pictures. But companies are also experimenting with face
identification and other A.I. technology as an ingredient in automated
decisions with higher stakes like hiring and lending.

Researchers at the Georgetown Law School
\href{https://www.perpetuallineup.org/}{estimated that 117 million
American adults} are in face recognition networks used by law
enforcement --- and that African Americans were most likely to be
singled out, because they were disproportionately represented in
mug-shot databases.

Facial recognition technology is lightly regulated so far.

``This is the right time to be addressing how these A.I. systems work
and where they fail --- to make them socially accountable,'' said Suresh
Venkatasubramanian, a professor of computer science at the University of
Utah.

Until now, there was anecdotal evidence of computer vision miscues, and
occasionally in ways that suggested discrimination. In 2015, for
example,
\href{https://bits.blogs.nytimes.com/2015/07/01/google-photos-mistakenly-labels-black-people-gorillas/}{Google
had to apologize} after its image-recognition photo app initially
labeled African Americans as ``gorillas.''

Sorelle Friedler, a computer scientist at Haverford College and a
reviewing editor on Ms. Buolamwini's research paper, said experts had
long suspected that facial recognition software performed differently on
different populations.

``But this is the first work I'm aware of that shows that empirically,''
Ms. Friedler said.

Ms. Buolamwini, a young African-American computer scientist, experienced
the bias of facial recognition firsthand. When she was an undergraduate
at the Georgia Institute of Technology, programs would work well on her
white friends, she said, but not recognize her face at all. She figured
it was a flaw that would surely be fixed before long.

\includegraphics{https://static01.nyt.com/images/2018/02/10/business/12FACES-2/12FACES-2-articleLarge-v2.jpg?quality=75\&auto=webp\&disable=upscale}

But a few years later, after joining the M.I.T. Media Lab, she ran into
the missing-face problem again. Only when she put on a white mask did
the software recognize hers as a face.

By then, face recognition software was increasingly moving out of the
lab and into the mainstream.

``O.K., this is serious,'' she recalled deciding then. ``Time to do
something.''

So she turned her attention to fighting the bias built into digital
technology. Now 28 and a doctoral student, after studying as a Rhodes
scholar and a Fulbright fellow, she is an advocate in the new field of
``algorithmic accountability,'' which seeks to make automated decisions
more transparent, explainable and fair.

Her
\href{https://www.ted.com/talks/joy_buolamwini_how_i_m_fighting_bias_in_algorithms}{short
TED Talk} on coded bias has been viewed more than 940,000 times, and she
founded the \href{https://www.ajlunited.org/}{Algorithmic Justice
League}, a project to raise awareness of the issue.

In her newly
\href{http://proceedings.mlr.press/v81/buolamwini18a/buolamwini18a.pdf}{published
paper}, which will be presented at \href{https://fatconference.org/}{a
conference} this month, Ms. Buolamwini studied the performance of three
leading face recognition systems --- by Microsoft, IBM and Megvii of
China --- by classifying how well they could guess the gender of people
with different skin tones. These companies were selected because they
offered gender classification features in their facial analysis software
--- and their code was publicly available for testing.

She found them all wanting.

To test the commercial systems, Ms. Buolamwini built a data set of 1,270
faces, using faces of lawmakers from countries with a high percentage of
women in office. The sources included three African nations with
predominantly dark-skinned populations, and three Nordic countries with
mainly light-skinned residents.

The African and Nordic faces were scored according to a six-point
labeling system used by dermatologists to classify skin types. The
medical classifications were determined to be more objective and precise
than race.

Then, each company's software was tested on the curated data, crafted
for gender balance and a range of skin tones. The results varied
somewhat. Microsoft's error rate for darker-skinned women was 21
percent, while IBM's and Megvii's rates were nearly 35 percent. They all
had error rates below 1 percent for light-skinned males.

Ms. Buolamwini shared the research results with each of the companies.
IBM said in a statement to her that the company had steadily improved
its facial analysis software and was ``deeply committed'' to
``unbiased'' and ``transparent'' services. This month, the company said,
it will roll out an improved service with a nearly 10-fold increase in
accuracy on darker-skinned women.

Microsoft said that it had ``already taken steps to improve the accuracy
of our facial recognition technology'' and that it was investing in
research ``to recognize, understand and remove bias.''

Ms. Buolamwini's co-author on her paper is Timnit Gebru, who described
her role as an adviser. Ms. Gebru is a scientist at Microsoft Research,
working on its
\href{https://www.microsoft.com/en-us/research/group/fate/}{Fairness
Accountability Transparency and Ethics in A.I.}group.

Image

Timnit Gebru, a scientist at Microsoft Research, was a co-author of the
paper that studied facial recognition software.Credit...Cody O'Loughlin
for The New York Times

Megvii, whose Face++ software is widely used for identification in
online payment and ride-sharing services in China, did not reply to
several requests for comment, Ms. Buolamwini said.

Ms. Buolamwini is releasing her data set for others to build upon. She
describes her research as ``a starting point, very much a first step''
toward solutions.

Ms. Buolamwini is taking further steps in the technical community and
beyond. She is working with the
\href{https://www.ieee.org/index.html}{Institute of Electrical and
Electronics Engineers}, a large professional organization in computing,
to set up a group to create standards for accountability and
transparency in facial analysis software.

She meets regularly with other academics, public policy groups and
philanthropies that are concerned about the impact of artificial
intelligence. Darren Walker, president of the
\href{https://www.fordfoundation.org/}{Ford Foundation}, said that the
new technology could be a ``platform for opportunity,'' but that it
would not happen if it replicated and amplified bias and discrimination
of the past.

``There is a battle going on for fairness, inclusion and justice in the
digital world,'' Mr. Walker said.

Part of the challenge, scientists say, is that there is so little
diversity within the A.I. community.

``We'd have a lot more introspection and accountability in the field of
A.I. if we had more people like Joy,'' said Cathy O'Neil, a data
scientist and author of ``Weapons of Math Destruction.''

Technology, Ms. Buolamwini said, should be more attuned to the people
who use it and the people it's used on.

``You can't have ethical A.I. that's not inclusive,'' she said. ``And
whoever is creating the technology is setting the standards.''

Advertisement

\protect\hyperlink{after-bottom}{Continue reading the main story}

\hypertarget{site-index}{%
\subsection{Site Index}\label{site-index}}

\hypertarget{site-information-navigation}{%
\subsection{Site Information
Navigation}\label{site-information-navigation}}

\begin{itemize}
\tightlist
\item
  \href{https://help.nytimes.com/hc/en-us/articles/115014792127-Copyright-notice}{©~2020~The
  New York Times Company}
\end{itemize}

\begin{itemize}
\tightlist
\item
  \href{https://www.nytco.com/}{NYTCo}
\item
  \href{https://help.nytimes.com/hc/en-us/articles/115015385887-Contact-Us}{Contact
  Us}
\item
  \href{https://www.nytco.com/careers/}{Work with us}
\item
  \href{https://nytmediakit.com/}{Advertise}
\item
  \href{http://www.tbrandstudio.com/}{T Brand Studio}
\item
  \href{https://www.nytimes.com/privacy/cookie-policy\#how-do-i-manage-trackers}{Your
  Ad Choices}
\item
  \href{https://www.nytimes.com/privacy}{Privacy}
\item
  \href{https://help.nytimes.com/hc/en-us/articles/115014893428-Terms-of-service}{Terms
  of Service}
\item
  \href{https://help.nytimes.com/hc/en-us/articles/115014893968-Terms-of-sale}{Terms
  of Sale}
\item
  \href{https://spiderbites.nytimes.com}{Site Map}
\item
  \href{https://help.nytimes.com/hc/en-us}{Help}
\item
  \href{https://www.nytimes.com/subscription?campaignId=37WXW}{Subscriptions}
\end{itemize}
