Sections

SEARCH

\protect\hyperlink{site-content}{Skip to
content}\protect\hyperlink{site-index}{Skip to site index}

\href{https://www.nytimes.com/section/opinion/sunday}{Sunday Review}

\href{https://myaccount.nytimes.com/auth/login?response_type=cookie\&client_id=vi}{}

\href{https://www.nytimes.com/section/todayspaper}{Today's Paper}

\href{/section/opinion/sunday}{Sunday Review}\textbar{}The Student Loan
Serenity Prayer

\href{https://nyti.ms/2BPgp5H}{https://nyti.ms/2BPgp5H}

\begin{itemize}
\item
\item
\item
\item
\item
\item
\end{itemize}

Advertisement

\protect\hyperlink{after-top}{Continue reading the main story}

Supported by

\protect\hyperlink{after-sponsor}{Continue reading the main story}

\href{/section/opinion}{Opinion}

\href{/column/on-campus}{On Campus}

\hypertarget{the-student-loan-serenity-prayer}{%
\section{The Student Loan Serenity
Prayer}\label{the-student-loan-serenity-prayer}}

By Michael Arceneaux

\begin{itemize}
\item
  Feb. 10, 2018
\item
  \begin{itemize}
  \item
  \item
  \item
  \item
  \item
  \item
  \end{itemize}
\end{itemize}

\includegraphics{https://static01.nyt.com/images/2018/02/11/opinion/sunday/11arceneaux/11arceneaux-articleLarge.jpg?quality=75\&auto=webp\&disable=upscale}

Several years ago, while driving around Los Angeles in a state of equal
parts fury and despair, I called my cousin to tell her how sick of it
all I was, that I was tired and couldn't take it anymore. Then I turned
off my phone. I needed silence. When I turned it on again a few hours
later, I was greeted with frantic voice-mail messages --- back then,
most people still checked those --- including a good number from my
mother. She was terrified of what I might have done.

I immediately called her back and apologized in my most soothing
Southern tone for causing concern. ``Ma'am, I had no intention of
killing myself,'' I said. ``I would never do that to you, because you'd
still be on the hook for these loans.''

It was a very handsome college recruiter who, almost a decade earlier,
had come to my public high school in Houston and convinced me that if I
wanted to attend a prestigious college --- private, out of state, even
--- it was possible, no matter what my surroundings or financial
circumstances suggested. My mother made clear that I would go to
college, but neither she nor I had entertained the idea that I might go
to schools like \emph{that}.

Inspired by Recruiter Bae, I applied for as many scholarships as I could
find and won 17 in a semester. But I needed additional money to cover
the cost of Howard University in Washington.

So I turned to Citibank. I took out about \$10,000 for the first year,
but it soon became clear that I'd need much more to keep studying at
Howard and living in Washington until graduation. After I received my
diploma, I immediately owed almost \$800 a month in private loans, with
12 years to pay it off. That's not counting the few hundred dollars I
pay each month in federal loans. Fortunately, the government gives me
more than twice the time to cough it up.

This may be the part of my story where you question my judgment. Why did
I choose a fancy school if I didn't have the money? How could I not have
understood the financial commitment I was making? And if I'm so far in
debt now, why am I writing this and not pursuing a more lucrative career
as a doctor or lawyer --- or, as one relative put it bluntly, ``When are
you going to work in a building?''

I'm not dismissing my responsibility for this. But along with many other
17- and 18-year-olds, when I went to college, I didn't know anything
about student loans, interest rates or rude private debt companies that
hound the living hell out of you. All I knew was what I was told:
College was \emph{the} ticket to social mobility, and good students
deserved to go to schools that matched our talent and ambition. Folks
like me, who come from working-class backgrounds, are told to chase down
a bachelor's degree by any means necessary. But no one mentions just how
expensive and soul-crushing the debt will be.

Still, I get it. I made the decision to take out loans. The voices in my
head don't let me forget that.

At my graduation, on a beautiful day in May 2007, my commencement
speaker and our future president, Oprah Winfrey, spoke passionately on
the toxicity of fear. ``Don't be afraid,'' she told us. ``All you have
to know is who you are. Because there is no such thing as failure.'' I
felt invigorated by her remarks. As I told everyone at the time, just
being in Oprah's presence probably raised my credit score.

But reality set in quickly with a congratulatory letter from Citibank,
stating the terms of my debt and repayment, which could be deferred by
two six-month periods, but that's it, no more, and after that, I'd
better run the bank its money \emph{or else}. It's blasphemous but I've
been entirely unable to follow Oprah's wise words. Instead, I
\emph{live} in fear --- fear that one day I may fall too far behind on
payments, and fear of what that would mean for my mother, who co-signed
my loans with great trepidation.

There's no relief for my wallet or my self-esteem. Every time I fork
over another payment, I think about all of the other ways I could have
financed my education. Why didn't I take more part-time jobs? I was in
Washington --- why didn't I try to date some closeted politician and be
his well-compensated secret? Or spend more time at the campus gym and
land a job stripping? I could have paid for classes in cash!

And for so long, I took to heart the poisonous folklore about
student-debt martyrs who selflessly scrape by to pay off their loans ---
those ``I only ate Spam and paid off my \$160,000 debt in 96 hours''
stories. I blamed myself, thinking that if I had just worked harder and
sacrificed more, I wouldn't be in this situation.

But the truth is, a lot of this was always out of my control. The
student loan industry is a barely regulated, predatory system, and with
Donald Trump in the White House and those equally useless people in
Congress, oversight of the industry is
\href{http://www.latimes.com/business/la-fi-cfpb-overhaul-20180205-story.html}{becoming
nonexistent}.

I was trying to do the right thing for myself and my family. Despite the
cost, going to college is still the only way high-achieving,
lower-income students can hope to get a good job with a decent wage.
It's not our fault that no one told us the system beyond higher
education was set up for us to fail.

I am a member of that class of college students that graduated into a
financial crisis, not long after a
\href{http://www.slate.com/blogs/moneybox/2015/04/16/student_loans_in_bankruptcy_how_the_bush_administration_pointlessly_screwed.html}{2005
bankruptcy bill} was passed that made private student loans
nondischargeable unless borrowers could demonstrate that loan repayment
put an ``undue hardship'' on their finances. Spoiler: The undue-hardship
exception has virtually never applied to anyone. It's so vague that it's
basically meaningless.

I think of that slippery little phrase every time I field a nasty phone
call from my student loan oppressors. If only I were a corporation or a
bank, privy to loopholes, tax havens, lenient bankruptcy provisions and
so many other measures that allow it to be treated far more humanely
than actual human beings.

Like so many others, I'm muddling my way out of a trap. I've come to
accept that I'm simply doing the best that I can with the choices I made
in earnest.

And I have to be good to myself. Not only is not enough attention paid
to the circumstances in which our collective crisis has been created,
but even less is paid to the everyday victories of people trapped in it:
The days we manage to get out of our beds despite feeling completely
weighed down. The times we decide to treat ourselves because we deserve
it. The joy-inducing invention of that block button on the iPhone so
that sometimes we can simply say, ``They'll get that money when I got
it.''

Advertisement

\protect\hyperlink{after-bottom}{Continue reading the main story}

\hypertarget{site-index}{%
\subsection{Site Index}\label{site-index}}

\hypertarget{site-information-navigation}{%
\subsection{Site Information
Navigation}\label{site-information-navigation}}

\begin{itemize}
\tightlist
\item
  \href{https://help.nytimes.com/hc/en-us/articles/115014792127-Copyright-notice}{©~2020~The
  New York Times Company}
\end{itemize}

\begin{itemize}
\tightlist
\item
  \href{https://www.nytco.com/}{NYTCo}
\item
  \href{https://help.nytimes.com/hc/en-us/articles/115015385887-Contact-Us}{Contact
  Us}
\item
  \href{https://www.nytco.com/careers/}{Work with us}
\item
  \href{https://nytmediakit.com/}{Advertise}
\item
  \href{http://www.tbrandstudio.com/}{T Brand Studio}
\item
  \href{https://www.nytimes.com/privacy/cookie-policy\#how-do-i-manage-trackers}{Your
  Ad Choices}
\item
  \href{https://www.nytimes.com/privacy}{Privacy}
\item
  \href{https://help.nytimes.com/hc/en-us/articles/115014893428-Terms-of-service}{Terms
  of Service}
\item
  \href{https://help.nytimes.com/hc/en-us/articles/115014893968-Terms-of-sale}{Terms
  of Sale}
\item
  \href{https://spiderbites.nytimes.com}{Site Map}
\item
  \href{https://help.nytimes.com/hc/en-us}{Help}
\item
  \href{https://www.nytimes.com/subscription?campaignId=37WXW}{Subscriptions}
\end{itemize}
