\href{/section/technology}{Technology}\textbar{}As China Marches Forward
on A.I., the White House Is Silent

\url{https://nyti.ms/2BqgkEz}

\begin{itemize}
\item
\item
\item
\item
\item
\end{itemize}

\includegraphics{https://static01.nyt.com/images/2018/03/02/business/mutfund/AIVacuum-sub/AIVacuum-sub-articleLarge.jpg?quality=75\&auto=webp\&disable=upscale}

Sections

\protect\hyperlink{site-content}{Skip to
content}\protect\hyperlink{site-index}{Skip to site index}

\hypertarget{as-china-marches-forward-on-ai-the-white-house-is-silent}{%
\section{As China Marches Forward on A.I., the White House Is
Silent}\label{as-china-marches-forward-on-ai-the-white-house-is-silent}}

Last summer, China unveiled a plan to become the world's leader in
artificial intelligence, challenging the longtime role of the United
States.

Credit...Zach Meyer

Supported by

\protect\hyperlink{after-sponsor}{Continue reading the main story}

By \href{https://www.nytimes.com/by/cade-metz}{Cade Metz}

\begin{itemize}
\item
  Feb. 12, 2018
\item
  \begin{itemize}
  \item
  \item
  \item
  \item
  \item
  \end{itemize}
\end{itemize}

\href{https://cn.nytimes.com/china/20180213/china-trump-artificial-intelligence/}{阅读简体中文版}\href{https://cn.nytimes.com/china/20180213/china-trump-artificial-intelligence/zh-hant/}{閱讀繁體中文版}

SAN FRANCISCO --- In July, China unveiled a plan to become the world
leader in artificial intelligence and create an industry worth \$150
billion to its economy by 2030.

To technologists working on A.I. in the United States, the statement,
which was 28 pages long in its English translation, was a direct
challenge to America's lead in arguably the most important tech research
to come along in decades. It outlined the Chinese government's
aggressive plan to treat A.I. like the country's own version of the
Apollo 11 lunar mission --- an all-in effort that could stoke national
pride and spark agenda-setting technology breakthroughs.

The manifesto was also remarkably similar to several reports on the
future of artificial intelligence released by the
\href{https://obamawhitehouse.archives.gov/sites/default/files/whitehouse_files/microsites/ostp/NSTC/national_ai_rd_strategic_plan.pdf}{Obama
administration at the end of 2016}.

``It is remarkable to see how A.I. has emerged as a top priority for the
Chinese leadership and how quickly things have been set into motion,''
said Elsa Kania, an adjunct fellow at the Center for a New American
Security who helped translate the manifesto and follows China's work on
artificial intelligence. ``The U.S. plans and policies released in 2016
were seemingly the impetus for the formulation of China's national A.I.
strategy.''

But six months after China seemed to mimic that
\href{https://obamawhitehouse.archives.gov/blog/2016/10/12/administrations-report-future-artificial-intelligence}{Obama-era
road map}, A.I. experts in industry and academia in the United States
say that the Trump White House has done little to follow through on the
previous administration's
\href{https://obamawhitehouse.archives.gov/sites/whitehouse.gov/files/documents/Artificial-Intelligence-Automation-Economy.PDF}{economic
call to arms}.

``We are still waiting on the White House to provide some direction'' on
how to respond to the competition, said Tim Hwang, who worked on A.I.
policy at Google and is now the director of the Ethics and Governance of
AI Initiative, a new organization created by the LinkedIn founder Reid
Hoffman and others to fund ethical research in artificial intelligence.

China's embrace of A.I. comes at a
\href{https://obamawhitehouse.archives.gov/sites/default/files/page/files/20160707_cea_ai_furman.pdf}{crucial
time in the development of the technology} and just as the lead long
enjoyed by the United States has started to dwindle.

For decades, artificial intelligence was more fiction than science. In
the past few years, however, dramatic improvements have prompted some of
the
\href{https://www.nytimes.com/2018/01/04/technology/self-driving-cars-aurora.html}{biggest
companies in Silicon Valley and Detroit} --- and China --- to invest
billions on everything from self-driving cars to home appliances that
can have a conversation with a human.

A.I. has also become a
\href{https://www.nytimes.com/2016/10/26/us/pentagon-artificial-intelligence-terminator.html}{significant
part of national defense policy} as military leaders and ethicists
debate how much autonomy we should give to weapons that can think for
themselves.

American companies like Amazon and Google have done more than anyone to
turn A.I. concepts into real products. But for a number of reasons,
including concerns that the Trump administration will limit the number
of immigrant engineers allowed into the United States, much of the
critical research being done on artificial intelligence is already
migrating to other countries, including tech hot spots like Toronto,
London and Beijing.

\includegraphics{https://static01.nyt.com/images/2018/02/04/business/04AIVACCUM06/merlin_130657928_e991ac4c-a3a9-4a1d-95b9-b087e8d07034-articleLarge.jpg?quality=75\&auto=webp\&disable=upscale}

To China's growing tech community, driving the industry's next big thing
--- a mantra of Silicon Valley --- is becoming a tantalizing
possibility.

``Thanks to the size of the market and the rapid experimentation, China
is going to become one of the most powerful --- if not the most powerful
--- A.I. countries in the world,'' said Kai-Fu Lee, a former Microsoft
and Google executive who now runs a prominent Chinese venture capital
firm dedicated to artificial intelligence.

The 2016 A.I. reports were shepherded by President Barack Obama's Office
of Science and Technology Policy.

The O.S.T.P., which has overseen science and technology activities
across the federal government for more than four decades, is now run by
the deputy chief technology officer Michael Kratsios. He had worked as a
Wall Street analyst before serving as chief of staff for an investment
fund run by Peter Thiel, a venture capitalist who supported Mr. Trump's
presidential run. The administration has yet to name an office director
or fill four other assistant posts.

In a recent interview, Mr. Kratsios was adamant that any concerns over
the administration's approach to A.I. were unfounded.

``Artificial intelligence has been a priority for the Trump
administration since Day 1,'' he said. Mr. Kratsios added that the
administration was particularly concerned with the development of A.I.
in national security and as a way of encouraging economic prosperity.

Many staff members in Mr. Kratsios's office are exploring issues related
to artificial intelligence, he said. Mr. Kratsios also meets with a
committee, set up by the Obama administration, that coordinates A.I.
policy across the government.

``The key thing to remember is that the front line of A.I. policy is at
the agencies,'' he said. ``The White House is a convener and a
coordinator.''

In an echo of plans laid out by the Obama administration, China's
government said it intended to significantly increase long-term funding
for A.I. research and develop a much larger community of A.I.
researchers.

There are several ways to do that, according to the Obama administration
and China. First, educate more students in these technologies. Second,
recruit experts from other countries.

At the same time, both policy statements urged companies to share more
technology and data. Huge pools of data are need to ``train'' A.I.
systems, and in the United States much of this is locked up inside
companies like Facebook and Google. Mr. Lee said China already has an
enormous advantage here because its large population will generate more
data and its companies are more willing to share.

Image

Geoffrey Hinton, a computer scientist and leading expert in artificial
intelligence, has helped make the University of Toronto a center of
innovation in A.I. technology.Credit...Aaron Vincent Elkaim for The New
York Times

Artificial intelligence has been a focus of Chinese technologists for
some time. By 2013, China was already producing more research papers
than the United States in the area of ``deep learning,'' the main
technology driving the rise of A.I., according to the Obama reports.
Deep learning, which allows machines to learn tasks by analyzing vast
amounts of data, is one of the main technologies driving the rise of
artificial intelligence.

It is unclear how much China as a whole is spending. But one Chinese
state has promised to invest \$5 billion in A.I., and the government of
Beijing has committed \$2 billion to an A.I. development park in the
city. South Korea has set aside close to \$1 billion of its own. Canada,
already home to many of the top researchers in the field, has also
committed \$125 million to, in part, attract new talent from other
countries.

It is also difficult to say just how much the government of the United
States is spending. Government organizations like the Intelligence
Advanced Research Projects Activity, the National Institute of Standards
and Technology, and the National Science Foundation continue to fund new
research in universities and the private sector. According to an O.S.T.P
report, the federal government spent about \$1 billion a year in 2015.
The Trump administration says that spending jumped to \$3 billion in
2017. But the current administration said that was not an
apples-to-apples comparison to the 2015 tally, because it was not
certain how the Obama administration made it calculations.

``We may have a bunch of small initiatives inside the government that
are doing good, but we don't have a central national strategy,'' said
Jack Clark, a former journalist who now oversees policy efforts at
OpenAI, the artificial intelligence lab co-founded by Elon Musk, Tesla's
chief executive. ``It is confusing that we have this technology of such
obvious power and merit and we are not hearing full-throated support,
including financial support.''

The Trump administration's budget for 2018 aims to cut science and
technology research funding across the government by 15 percent,
according to a report from the
\href{https://www.aaas.org/page/fy-2018-rd-appropriations-dashboard}{American
Association for the Advancement of Science}.

``They are headed in precisely the wrong direction,'' said Thomas Kalil,
who led O.S.T.P's Technology and Innovation Division under President
Obama. ``That is particularly concerning given that China has identified
this as a strategic priority.''

Over the past five years,
\href{https://www.nytimes.com/2017/10/22/technology/artificial-intelligence-experts-salaries.html}{much
of the progress in A.I. technology} has been led by American companies
like Google, Microsoft, Amazon and Facebook. But these companies don't
need A.I. technologists to work in the United States in order to employ
them.

Take Geoffrey Hinton, a major figure in the rise of A.I. at Google and
across the tech industry. He recently moved back to Toronto, where he
was a professor for many years. He now runs a new Google lab in that
city. Last year,
\href{https://www.nytimes.com/2017/11/28/technology/artificial-intelligence-research-toronto.html}{he
took on an Iranian researcher who was denied a visa by the United States
government}.

Google operates another important lab in Montreal. Its London lab,
DeepMind, may be home to more top-notch A.I. researchers than any other
lab on earth. And Google recently unveiled new labs in both Paris and
Beijing. Facebook, after creating its own lab in Canada, recently pumped
10 million euros, or more than \$12 million, into its existing operation
in Paris. And Amazon is opening a lab in Germany.

Inside these facilities, researchers still create technology for their
American employers. As the labs grow and the products get better, some
employees can be expected to leave to start their own companies and hire
their own employees.

Google's and Microsoft's work in China has already led to Chinese
start-ups like Malong, which is building image recognition systems, and
a major A.I. investment fund run by Mr. Lee.

``When it is close to you, something like Microsoft Research has real
economic value,'' said Mr. Clark, of OpenAI.

Advertisement

\protect\hyperlink{after-bottom}{Continue reading the main story}

\hypertarget{site-index}{%
\subsection{Site Index}\label{site-index}}

\hypertarget{site-information-navigation}{%
\subsection{Site Information
Navigation}\label{site-information-navigation}}

\begin{itemize}
\tightlist
\item
  \href{https://help.nytimes.com/hc/en-us/articles/115014792127-Copyright-notice}{©~2020~The
  New York Times Company}
\end{itemize}

\begin{itemize}
\tightlist
\item
  \href{https://www.nytco.com/}{NYTCo}
\item
  \href{https://help.nytimes.com/hc/en-us/articles/115015385887-Contact-Us}{Contact
  Us}
\item
  \href{https://www.nytco.com/careers/}{Work with us}
\item
  \href{https://nytmediakit.com/}{Advertise}
\item
  \href{http://www.tbrandstudio.com/}{T Brand Studio}
\item
  \href{https://www.nytimes.com/privacy/cookie-policy\#how-do-i-manage-trackers}{Your
  Ad Choices}
\item
  \href{https://www.nytimes.com/privacy}{Privacy}
\item
  \href{https://help.nytimes.com/hc/en-us/articles/115014893428-Terms-of-service}{Terms
  of Service}
\item
  \href{https://help.nytimes.com/hc/en-us/articles/115014893968-Terms-of-sale}{Terms
  of Sale}
\item
  \href{https://spiderbites.nytimes.com}{Site Map}
\item
  \href{https://help.nytimes.com/hc/en-us}{Help}
\item
  \href{https://www.nytimes.com/subscription?campaignId=37WXW}{Subscriptions}
\end{itemize}
