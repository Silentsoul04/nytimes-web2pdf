\href{/section/business}{Business}\textbar{}Kushner's Family Business
Received Loans After White House Meetings

\url{https://nyti.ms/2FEs5aA}

\begin{itemize}
\item
\item
\item
\item
\item
\item
\end{itemize}

\includegraphics{https://static01.nyt.com/images/2018/03/01/business/01kushner-print/merlin_134678906_1b50bb34-08f9-484a-a3af-e4ba0514f7a9-articleLarge.jpg?quality=75\&auto=webp\&disable=upscale}

Sections

\protect\hyperlink{site-content}{Skip to
content}\protect\hyperlink{site-index}{Skip to site index}

\hypertarget{kushners-family-business-received-loans-after-white-house-meetings}{%
\section{Kushner's Family Business Received Loans After White House
Meetings}\label{kushners-family-business-received-loans-after-white-house-meetings}}

Apollo, the private equity firm, and Citigroup made large loans last
year to the family real estate business of Jared Kushner, President
Trump's senior adviser.

Jared Kushner's tenure in the White House has been dogged by questions
about conflicts of interest between his government work and his family
business, in which he remains heavily invested.Credit...Tom Brenner/The
New York Times

Supported by

\protect\hyperlink{after-sponsor}{Continue reading the main story}

By \href{https://www.nytimes.com/by/jesse-drucker}{Jesse Drucker},
\href{https://www.nytimes.com/by/kate-kelly}{Kate Kelly} and
\href{http://www.nytimes.com/by/ben-protess}{Ben Protess}

\begin{itemize}
\item
  Feb. 28, 2018
\item
  \begin{itemize}
  \item
  \item
  \item
  \item
  \item
  \item
  \end{itemize}
\end{itemize}

Early last year, a private equity billionaire started paying regular
visits to the White House.

Joshua Harris, a founder of Apollo Global Management, was advising Trump
administration officials on infrastructure policy. During that period,
he met on multiple occasions with Jared Kushner, President Trump's
son-in-law and senior adviser, said three people familiar with the
meetings. Among other things, the two men discussed a possible White
House job for Mr. Harris.

The job never materialized, but in November, Apollo lent \$184 million
to Mr. Kushner's family real estate firm, Kushner Companies. The loan
was to refinance the mortgage on a Chicago skyscraper.

Even by the standards of Apollo, one of the world's largest private
equity firms, the previously unreported transaction with the Kushners
was a big deal: It was triple the size of the average property loan made
by Apollo's real estate lending arm, securities filings show.

It was one of the largest loans Kushner Companies received last year. An
even larger loan came from Citigroup, which lent the firm and one of its
partners \$325 million to help finance a group of office buildings in
Brooklyn.

That loan was made in the spring of 2017, shortly after Mr. Kushner met
in the White House with Citigroup's chief executive, Michael L. Corbat,
according to people briefed on the meeting. The two men talked about
financial and trade policy and did not discuss Mr. Kushner's family
business, one person said.

There is little precedent for a top White House official meeting with
executives of companies as they contemplate sizable loans to his
business, say government ethics experts.

``This is exactly why senior government officials, for as long back as I
have any experience, don't maintain any active outside business
interests,'' said Don Fox, the former acting director of the Office of
Government Ethics during the Obama administration and, before that, a
lawyer for the Air Force and Navy during Republican and Democratic
administrations. ``The appearance of conflicts of interest is simply too
great.''

The White House referred questions to Mr. Kushner's lawyer, Abbe Lowell,
who did not dispute that the meetings between Mr. Kushner and the
executives took place.

Peter Mirijanian, a spokesman for Mr. Lowell, said in a statement that
Mr. Kushner ``has met with hundreds of business people.'' He said that
Mr. Kushner ``has taken no part of any business, loans or projects with
or for'' Kushner Companies since joining the White House and that he has
followed ethics advice.

Christine Taylor, a spokeswoman for Kushner Companies, said Mr.
Kushner's White House role had not affected the company's relationships
with financial institutions. ``Stories like these attempt to make
insinuating connections that do not exist to disparage the financial
institutions and companies involved,'' she said.

An Apollo spokesman, Charles V. Zehren, said Mr. Harris was not involved
in the decision to loan money to Kushner Companies. He said the loan
``went through the firm's standard approval process.''

A Citigroup spokeswoman, Danielle Romero-Apsilos, said Kushner Companies
had been a bank client since before the election and that the
relationship had no connection to Mr. Kushner's White House role. She
said Citigroup negotiated the 2017 loan with Kushner Companies' business
partner, a real estate developer.

Mr. Kushner's tenure in the White House has been dogged by questions
about conflicts of interest between his government work and his family
business, in which he remains heavily invested. Mr. Kushner steers
American policy in the Middle East, for example, but his family company
continues to do
\href{https://www.nytimes.com/2018/01/07/business/jared-kushner-israel.html}{deals
with Israeli investors}.

This blurring of lines is now a potential liability for Mr. Kushner, who
recently lost his top-secret security clearance
\href{https://www.nytimes.com/2018/02/27/us/politics/jared-kushner-security-clearance-trump.html}{amid
worries} from some United States officials that foreign governments
might try to gain influence with the White House by doing business with
Mr. Kushner.

\includegraphics{https://static01.nyt.com/images/2018/02/28/business/00KUSHNER2/00KUSHNER2-articleLarge.jpg?quality=75\&auto=webp\&disable=upscale}

Investigators working for Robert S. Mueller III, the special counsel
looking into Russian interference in the 2016 election, have asked
questions about Mr. Kushner's interactions with potential investors from
overseas, according to a person familiar with the matter. Mr. Kushner's
firm has sought investments from the Chinese insurer Anbang and from the
former prime minister of Qatar.

Mr. Mirijanian, the spokesman for Mr. Kushner's lawyer, said that Mr.
Mueller had not sought from the Kushner Companies or Mr. Kushner ``any
information or document concerning any business dealing either has
had.''

Mr. Kushner resigned as chief executive of Kushner Companies when he
joined the White House last January, and he sold a small portion of his
stake in the company to a trust controlled by his mother.

But he retained the vast majority of his interest in Kushner Companies.
His real estate holdings and other investments are worth as much as
\$761 million, according to government ethics filings. They are likely
worth much more, because that estimate has his firm's debt subtracted
from the value of his holdings. The company has done at least \$7
billion of deals in the past decade.

Public filings show that Mr. Kushner still owns part of the company that
received the Apollo loan. The loan was used to refinance a Chicago
skyscraper that is the Midwest headquarters for AT\&T. Mr. Kushner also
still holds a stake in the entity that owns the Brooklyn buildings and
received the loan from Citigroup.

Federal ethics regulations restrict government employees from
participating in some matters that involve companies with which the
official is seeking ``a business, contractual or other financial
relationship that involves other than a routine consumer transaction.''

Mr. Fox, the ethics expert, said Mr. Kushner risked violating the
regulations in his meetings with Citigroup and Apollo executives.

Image

Jared Kushner sat next to Leon Black, the chief executive of Apollo
Global Management, at the 2016 US Open, a few months before the
presidential election.Credit...Jean Catuffe/GC Images

``Why does Jared have to take the meeting?'' he asked. ``Is there not
somebody else who doesn't have these financial entanglements who can
brainstorm freely with these folks?''

The Trump administration no longer publicly discloses logs of visitors
to the White House. In February, the White House settled litigation with
Public Citizen, a nonprofit organization, and agreed to disclose some of
that information.

Mr. Kushner has also met at the White House with Stephen A. Schwarzman,
chief executive of the private equity firm Blackstone, which in the past
has lent money to Kushner Companies for several projects, though all
before the election. Until August, Mr. Schwarzman was the head of a
White House business advisory council.

``Blackstone has not done any business with Kushner Companies since the
election, nor has Steve ever discussed private business matters with Mr.
Kushner,'' said a Blackstone spokeswoman.

All of the executives who met with Mr. Kushner have lots to gain or lose
in Washington.

Apollo has sought ways to benefit from the White House's possible
infrastructure plan. And its executives, including Mr. Harris, had tens
of millions of dollars personally at stake in the tax overhaul that was
making its way through Washington last year.

Citigroup, one of the country's largest banks, is heavily regulated by
federal agencies and, like other financial companies, is trying to get
the government to relax its oversight of the industry.

Mr. Kushner has reported owning hundreds of partnerships, limited
liability companies and other entities, but he is not required to
disclose the lifeblood of any real estate firm's business: its lenders
and outside investors.

Some of the lenders can be pieced together through public records. The
Apollo loan is disclosed in Chicago mortgage filings, which list the
lender as ``ACREFI Holdings J-I, LLC.'' That vehicle is owned by
Apollo's real estate investment trust, securities filings show.

The loans' interest rates and other key provisions are not disclosed.

Kushner Companies bought the Chicago building in 2007.

Apollo, best known for investing in companies such as Caesars
Entertainment and Norwegian Cruise Lines, was founded by Mr. Harris,
Marc Rowan and its current chief executive, Leon Black. Mr. Black and
Mr. Kushner sat next to each other at the 2016 US Open, a few months
before the presidential election.

Mr. Harris, a co-owner of the Philadelphia 76ers and the New Jersey
Devils, is Apollo's senior managing director. He also is an adviser to
the Federal Reserve Bank of New York.

Early last year, the White House enlisted Mr. Harris and other
executives to advise the administration on infrastructure policy. Over a
period of months, Mr. Harris met regularly with Mr. Kushner and other
White House officials.

Apollo made the loan to Kushner Companies on Nov. 1, 2017, according to
public records.

Apollo does not make real estate loans directly. Instead, it makes them
through a so-called real estate investment trust, called Apollo
Commercial Real Estate Finance. The trust is a publicly traded company
with its own set of shareholders. It is managed by Apollo, which charges
the trust management fees, and has no employees of its own.

One of the largest investors in Apollo's real estate trust is the Qatari
government's investment fund, the Qatar Investment Authority.

Mr. Kushner's firm previously sought a \$500 million investment from the
former head of that Qatari fund for its headquarters at 666 Fifth Avenue
in Manhattan. That investment never materialized.

Shortly after Kushner Companies received the loan from Apollo, the
private equity firm emerged as a beneficiary of the tax cut package that
the White House championed. Mr. Trump backed down from his earlier
pledge to close a loophole that permits private equity managers to pay
taxes on the bulk of their income at rates that are roughly half of
ordinary income tax rates. The tax law left the loophole largely intact.

Image

Mr. Kushner's firm had sought a \$500 million investment from the former
head of the Qatari government's investment fund for its headquarters at
666 Fifth Avenue in Manhattan.Credit...Karsten Moran for The New York
Times

Advertisement

\protect\hyperlink{after-bottom}{Continue reading the main story}

\hypertarget{site-index}{%
\subsection{Site Index}\label{site-index}}

\hypertarget{site-information-navigation}{%
\subsection{Site Information
Navigation}\label{site-information-navigation}}

\begin{itemize}
\tightlist
\item
  \href{https://help.nytimes.com/hc/en-us/articles/115014792127-Copyright-notice}{©~2020~The
  New York Times Company}
\end{itemize}

\begin{itemize}
\tightlist
\item
  \href{https://www.nytco.com/}{NYTCo}
\item
  \href{https://help.nytimes.com/hc/en-us/articles/115015385887-Contact-Us}{Contact
  Us}
\item
  \href{https://www.nytco.com/careers/}{Work with us}
\item
  \href{https://nytmediakit.com/}{Advertise}
\item
  \href{http://www.tbrandstudio.com/}{T Brand Studio}
\item
  \href{https://www.nytimes.com/privacy/cookie-policy\#how-do-i-manage-trackers}{Your
  Ad Choices}
\item
  \href{https://www.nytimes.com/privacy}{Privacy}
\item
  \href{https://help.nytimes.com/hc/en-us/articles/115014893428-Terms-of-service}{Terms
  of Service}
\item
  \href{https://help.nytimes.com/hc/en-us/articles/115014893968-Terms-of-sale}{Terms
  of Sale}
\item
  \href{https://spiderbites.nytimes.com}{Site Map}
\item
  \href{https://help.nytimes.com/hc/en-us}{Help}
\item
  \href{https://www.nytimes.com/subscription?campaignId=37WXW}{Subscriptions}
\end{itemize}
