Sections

SEARCH

\protect\hyperlink{site-content}{Skip to
content}\protect\hyperlink{site-index}{Skip to site index}

\href{https://myaccount.nytimes.com/auth/login?response_type=cookie\&client_id=vi}{}

\href{https://www.nytimes.com/section/todayspaper}{Today's Paper}

\href{/section/opinion}{Opinion}\textbar{}Women, Booze and the Vote

\url{https://nyti.ms/2D0KYSh}

\begin{itemize}
\item
\item
\item
\item
\item
\item
\end{itemize}

Advertisement

\protect\hyperlink{after-top}{Continue reading the main story}

Supported by

\protect\hyperlink{after-sponsor}{Continue reading the main story}

\href{/section/opinion}{Opinion}

Op-Ed Contributor

\hypertarget{women-booze-and-the-vote}{%
\section{Women, Booze and the Vote}\label{women-booze-and-the-vote}}

By Elaine Weiss

\begin{itemize}
\item
  March 5, 2018
\item
  \begin{itemize}
  \item
  \item
  \item
  \item
  \item
  \item
  \end{itemize}
\end{itemize}

\includegraphics{https://static01.nyt.com/images/2018/03/05/opinion/05weiss1/05weiss1-articleLarge.jpg?quality=75\&auto=webp\&disable=upscale}

Jane Walker will take over her brother Johnny's whiskey label this month
--- in honor of Women's History Month, we're told --- a temporary
rebranding that's fueling comic riffs by Stephen Colbert and other
cynical types. It might seem innocuous enough --- an unsubtle attempt to
lure skittish female drinkers to whiskey --- but there's a back story to
this relationship that's worth noting.

The liquor industry was once the most powerful opponent of granting
women their civil rights. Jane Walker, in a sense, would have been an
anti-suffragist. And over decades, the industry provided much of the
``dark money'' used to fight woman suffrage in Congress and in the
states.

Temperance was a ``woman's issue'' in the 19th and early 20th centuries,
as women and children suffered physical and emotional abuse from
inebriated men, often their own husbands and fathers. Families also
suffered the financial toll of salaries squandered on booze, with not
much left for bread. Early feminists took up the temperance cause, not
just for moral reasons (though that was a rationale for some) but also
as a public health issue and as a way to protect women from domestic
violence and sexual harassment on the street.

Before she was an organizer for woman suffrage, Susan B. Anthony was an
organizer for the Daughters of Temperance. Frances Willard and her
Women's Christian Temperance Union advocated votes for women, and there
was a natural alliance between the movements: Empower women with the
vote so they can protect themselves by placing legal restrictions on
liquor.

The liquor industry tried to protect itself, too, by working strenuously
to keep the ballot out of women's hands.

The industry strengthened its clout by funding the campaigns of members
of Congress, who turned around and obliged the industry by keeping both
the prohibition amendment and woman suffrage amendment stranded in
committee; the suffrage amendment was buried in Congress for 42 years.
Think of Congress, the N.R.A. and gun control.

Image

Johnnie Walker released a limited edition of its Johnnie Walker Black
Label whisky called ``Jane Walker,'' for women's history
month.Credit...Justin Sullivan/Getty Images

The industry held sway in statehouses and city halls, too, especially
where brewing was big business. Whenever woman suffrage legislation
appeared on the docket or a suffrage referendum was on a state ballot
(and only men could vote to decide whether women should have the same
right) the alliance of brewers, bottlers, distributors, saloonkeepers,
hotels and liquor stores (even druggists) was marshaled to insure
defeat. It wasn't unusual for saloons to display anti-suffrage posters
and keep a pile of leaflets on the bar; the promise of a free beer in
exchange for a no vote on a suffrage referendum was common practice.

More nefarious means were also employed: vote alterations, ballot box
dumping, physical intimidation. When a 1912 suffrage referendum was
defeated in Michigan, the governor angrily denounced the role played by
the liquor industry: ``The question seems to be largely one as to
whether the liquor interests own and control and run Michigan,'' he
lamented.

By the second decade of the 20th century, public opinion was swinging
against the liquor industry, and restrictions on the manufacture and
sale of liquor were set in place in counties and states, while the
prohibition amendment picked up steam in Congress. That women had
already been granted the vote in several states helped propel the
momentum: These women were voting for ``dry'' candidates at all levels
of government and demanding they impose legislative restrictions on
liquor sales.

(Today's politicians might want to heed this historical lesson, as newly
energized millennials, outraged by the refusal of legislatures and
Congress to act on any meaningful gun control, reach voting age.)

By 1919 the liquor industry was on the ropes: Prohibition was the law of
the land with the 18th Amendment, and the 19th --- woman suffrage ---
was nearing ratification. The industry's only hope was to limit the
damage by supporting the election of more ``wet'' candidates in
statehouses and Congress, who could blunt the regulations of the
Volstead Act, passed to enforce Prohibition. They poured money into
``wet'' campaigns while also trying to thwart ratification of the 19th
Amendment in the states.

The last stand was made in August 1920 in Tennessee, where the industry
sponsored a ``Jack Daniels Suite'' in a hotel near the statehouse,
dispensing free liquor, day and night, while trying to persuade
inebriated lawmakers to kill the amendment. It almost worked. But by
barely two votes, Tennessee did ratify the 19th Amendment, the necessary
36th state to do so --- and that fall the new female voters made sure
Prohibition was strictly enforced.

Prohibition was not the solution women had hoped for; like the modern
war on drugs, its enforcement spawned a new type of violence. The liquor
industry rebounded after Prohibition's repeal in 1933, seemingly
suffering few long-term consequences for its long stand against women.

And now we have Jane Walker Whiskey, the latest effort of the industry
to expand its consumer base. Beginning with wine and continuing into
pink-themed cocktails, pastel-hued sweet concoctions with a punch and
high-octane bottled fizzy drinks, over the decades the industry has
displayed its marketing creativity. Women are warming to the harder
spirits, too, with bourbon sales zooming.

The success of this industry marketing effort comes at a price. Studies
show that women now consume as much alcohol as men, and it's a problem;
binge drinking and alcoholism among women is on the rise. Falling prey
to sexual assault while under the influence is a growing concern.

A century ago women fought the liquor industry as a menace to public
health, accusing the booze business of putting the welfare of women and
families in danger by selling so much alcohol to their menfolk. Today a
new public health crisis looms, but it's women themselves being
willingly seduced to drink more.

Diageo, the maker of Johnny Walker, is betting that a temporary gender
reassignment on its label will help cultivate female drinkers, who tend
to be ``intimidated'' by scotch, according to the distiller. But the
company professes loftier goals.

``Important conversations about gender continue to be at the forefront
of culture and we strongly believe there is no better time than now to
introduce our Jane Walker icon and contribute to pioneering
organizations that share our mission,'' said Stephanie Jacoby, vice
president of Johnnie Walker. ``We are proud to toast the many
achievements of women and everyone on the journey towards progress in
gender equality.''

``With every step, we all move forward,'' the new Jane Walker campaign
boasts. The appropriation of a feminist theme for selling whiskey to
women is nothing if not ironic, for when the stakes were higher, the
industry chose sales over equal rights; now it may be choosing sales
over women's health.

The kicker: ``In recognition of the women in history who fought for
progress,'' Diageo will donate \$1 of every bottle of Jane Walker sold
this month to ``organizations championing women's causes,'' including
She Should Run, which encourages women to run for office, and the
Elizabeth Cady Stanton and Susan B. Anthony Statue Fund and its
Monumental Women campaign, to place statues of the suffragists who
fought the liquor industry in Central Park.

Advertisement

\protect\hyperlink{after-bottom}{Continue reading the main story}

\hypertarget{site-index}{%
\subsection{Site Index}\label{site-index}}

\hypertarget{site-information-navigation}{%
\subsection{Site Information
Navigation}\label{site-information-navigation}}

\begin{itemize}
\tightlist
\item
  \href{https://help.nytimes.com/hc/en-us/articles/115014792127-Copyright-notice}{©~2020~The
  New York Times Company}
\end{itemize}

\begin{itemize}
\tightlist
\item
  \href{https://www.nytco.com/}{NYTCo}
\item
  \href{https://help.nytimes.com/hc/en-us/articles/115015385887-Contact-Us}{Contact
  Us}
\item
  \href{https://www.nytco.com/careers/}{Work with us}
\item
  \href{https://nytmediakit.com/}{Advertise}
\item
  \href{http://www.tbrandstudio.com/}{T Brand Studio}
\item
  \href{https://www.nytimes.com/privacy/cookie-policy\#how-do-i-manage-trackers}{Your
  Ad Choices}
\item
  \href{https://www.nytimes.com/privacy}{Privacy}
\item
  \href{https://help.nytimes.com/hc/en-us/articles/115014893428-Terms-of-service}{Terms
  of Service}
\item
  \href{https://help.nytimes.com/hc/en-us/articles/115014893968-Terms-of-sale}{Terms
  of Sale}
\item
  \href{https://spiderbites.nytimes.com}{Site Map}
\item
  \href{https://help.nytimes.com/hc/en-us}{Help}
\item
  \href{https://www.nytimes.com/subscription?campaignId=37WXW}{Subscriptions}
\end{itemize}
