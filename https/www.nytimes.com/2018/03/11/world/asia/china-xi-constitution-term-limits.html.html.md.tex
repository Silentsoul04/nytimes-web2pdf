Sections

SEARCH

\protect\hyperlink{site-content}{Skip to
content}\protect\hyperlink{site-index}{Skip to site index}

\href{https://www.nytimes.com/section/world/asia}{Asia Pacific}

\href{https://myaccount.nytimes.com/auth/login?response_type=cookie\&client_id=vi}{}

\href{https://www.nytimes.com/section/todayspaper}{Today's Paper}

\href{/section/world/asia}{Asia Pacific}\textbar{}China's Legislature
Blesses Xi's Indefinite Rule. It Was 2,958 to 2.

\url{https://nyti.ms/2twEISJ}

\begin{itemize}
\item
\item
\item
\item
\item
\end{itemize}

Advertisement

\protect\hyperlink{after-top}{Continue reading the main story}

Supported by

\protect\hyperlink{after-sponsor}{Continue reading the main story}

\hypertarget{chinas-legislature-blesses-xis-indefinite-rule-it-was-2958-to-2}{%
\section{China's Legislature Blesses Xi's Indefinite Rule. It Was 2,958
to
2.}\label{chinas-legislature-blesses-xis-indefinite-rule-it-was-2958-to-2}}

\includegraphics{https://static01.nyt.com/images/2018/03/12/world/12China-congress-sub/merlin_135332079_443c8495-ab66-44f6-a569-faacebb8928d-articleLarge.jpg?quality=75\&auto=webp\&disable=upscale}

By \href{https://www.nytimes.com/by/chris-buckley}{Chris Buckley} and
\href{https://www.nytimes.com/by/steven-lee-myers}{Steven Lee Myers}

\begin{itemize}
\item
  March 11, 2018
\item
  \begin{itemize}
  \item
  \item
  \item
  \item
  \item
  \end{itemize}
\end{itemize}

\href{https://cn.nytimes.com/china/20180311/china-xi-constitution-term-limits/}{阅读简体中文版}\href{https://cn.nytimes.com/china/20180311/china-xi-constitution-term-limits/zh-hant/}{閱讀繁體中文版}

BEIJING --- President
\href{https://www.nytimes.com/topic/person/xi-jinping}{Xi Jinping} set
China on course to follow his hard-line authoritarian rule far into the
future on Sunday, when the national legislature lifted the presidential
term limit and gave constitutional backing to expanding the reach of the
Communist Party.

Under the red-starred dome of the Great Hall of the People in Beijing,
nearly 3,000 delegates of the National People's Congress, the
party-controlled legislature, voted almost unanimously to approve an
amendment to the Constitution to abolish the term limit on the
presidency, opening the way for Mr. Xi to rule indefinitely.

The amendment was among a set of 21 constitutional changes approved by
the congress, which included passages added to the Constitution to
salute Mr. Xi and his drive to entrench party supremacy.

Mr. Xi is using his formidable power to dismantle parts of the political
order set in place in the 1980s and 1990s by Deng Xiaoping, who led
China on a path of economic opening and liberalization. This includes
the system of collective leadership and regular, orderly transitions of
power that became the norm after Deng died in 1997.

Mr. Xi ``has shown the world that he can scrap decades of institutional
building with hardly any public dissent from the elite,''
\href{https://gps.ucsd.edu/faculty-directory/victor-shih.html}{Victor
Shih}, a professor at the University of California, San Diego, who
studies elite Chinese politics, said by email after the vote.

Ever since the party said two weeks ago that it wanted to
\href{https://www.nytimes.com/2018/02/25/world/asia/china-xi-jinping.html}{remove
the 35-year-old line in the Constitution} limiting the president to two
consecutive terms, there was never any real doubt that the congress
would approve the move. But the lopsided outcome --- 2,958 votes in
favor, two against, three abstentions and one invalid vote ---
underlined how much Mr. Xi dominates politics and feels emboldened to
demand drastic changes.

The delegates applauded briefly when an official declared the vote was
over, and clapped again for 20 seconds when the outcome was announced.

``No disagreements, no different points of view,'' Ma Shunnan, a
delegate representing the Chinese navy said in a brief interview shortly
before the vote. ``Every delegate is on the same page.''

Next weekend, the congress is expected to continue that show of lock
step support for Mr. Xi by voting him into a second five-year term as
president, along with electing a new lineup of government officials.

Sunday's constitutional amendments marked a victory not just for Mr.
Xi's own ambitions, but also for his quest to entrench the Communist
Party at the heart of politics, society and the economy as China ascends
globally.

Mr. Xi, 64, has in effect created a new legal basis for ruling for
another decade or longer as president, along with holding his other
posts as Communist Party chief and military chairman. Without the
\href{http://www.npc.gov.cn/npc/dbdhhy/13_1/2018-03/06/content_2042508.htm}{amendments},
he would have been forced to step down as president in 2023, weakening
his control.

``Under Xi Jinping, China is making a U-turn,''
\href{https://gps.ucsd.edu/faculty-directory/susan-shirk.html}{Susan
Shirk}, the head of the 21st Century China Center at the University of
California, San Diego, wrote in a
\href{https://www.journalofdemocracy.org/sites/default/files/media/29.2\%E2\%80\%94Shirk\%E2\%80\%94AdvanceVersion.pdf}{recent
assessment of Mr. Xi}. ``Personalistic rule is back.''

The amendments also reflected his goal of expanding party influence
across China's increasingly complex and wealthy society.

One elevated
``\href{https://www.nytimes.com/2018/02/26/world/asia/xi-jinping-thought-explained-a-new-ideology-for-a-new-era.html}{Xi
Jinping Thought},'' the catchall term for his ideology, into the
\href{http://www.hkhrm.org.hk/english/law/const01.html}{preamble} of the
Constitution, honoring him alongside leaders like China's founding
father, Mao Zedong. Another authorized a
\href{https://www.nytimes.com/2017/11/29/world/asia/china-xi-jinping-anticorruption.html}{new
investigative agency} to step up the anticorruption drive that Mr. Xi
has used to consolidate his control over the party.

``There's an argument to be made that these are the most fundamental
political changes to the Chinese Constitution since it was implemented
in 1982,'' said
\href{https://www.law.cuhk.edu.hk/en/people/info.php?id=237}{Ryan
Mitchell}, an assistant professor of law at the Chinese University of
Hong Kong.

As delegates streamed from the hall after the vote, several said they
hoped Mr. Xi would serve for a third or fourth term. One delegate, Tan
Zeyong of the southern Chinese province of Hunan, said Mr. Xi should
serve until he was age 78, which would keep him in power until 2031.

``This goes a step further to establish Chairman Xi's leadership and the
Party's leadership in the country,'' said another delegate, Wei Xuefeng
from Sichuan Province in southwestern China.

Supporters say ending the term limit will allow Mr. Xi to avoid becoming
a lame duck in his second term, and give him added authority to pursue
other parts of his agenda: overhauling the military, stamping out graft,
reducing extreme poverty and fixing an economy grown dependent on debt
and heavy industry.

Shen Chunyao, a legislative official with the congress, told reporters
after the vote that it made sense to remove the term limits so that Mr.
Xi can continue to steer China as president as well as party chief and
military commission chairman. There are no term limits on those latter
two posts.

But the decision could prompt a backlash among moderate members of the
party elite, who see a dangerous hubris in Mr. Xi's actions, some
experts on Chinese politics have said.

``This is going to cause some serious consternation within certain
circles that are not marginal,'' said
\href{https://www.merton.ox.ac.uk/people/professor-patricia-thornton}{Patricia
Thornton}, a professor at the University of Oxford who studies Chinese
politics. ``It's been clear for some time that Xi doesn't share power
well at all, but it's also clear to me that he genuinely fears
resistance and opposition from within the party.''

Some Chinese people worry that the abrupt change augurs a return to the
strife over succession that troubled the eras of Mao and Deng.

``Abolishing the term limit on the leader of state does not make a
leader but a usurper,'' Wang Yi, a former law lecturer who now works as
a
\href{https://www.theatlantic.com/international/archive/2017/04/china-unregistered-churches-driving-religious-revolution/521544/}{church
pastor in Sichuan}, said via a phone message. ``Writing a living
person's name into the Constitution is not amending the Constitution but
destroying it.''

This was not the direction that many imagined Mr. Xi would take when he
stood at the
\href{http://www.nytimes.com/2012/11/15/world/asia/communists-conclude-party-congress-in-china.html}{congress
in 2013} to accept his first term as president, soon after he became
Communist Party general secretary.

In his first months as leader, Mr. Xi
\href{http://usa.chinadaily.com.cn/china/2012-12/05/content_15985894.htm}{vowed
fidelity} to China's 1982 Constitution, which brought in the two-term
limit on the president, and
\href{http://www.nytimes.com/2012/12/10/world/asia/chinese-leaders-visit-to-shenzhen-hints-at-reform.html}{paid
homage to Deng}, the patriarch who had vowed to end lifelong rule so an
autocrat like Mao could not re-emerge. Such gestures led some to think
that Mr. Xi be would be a relatively mild leader.

Instead, Mr. Xi has proved to be
a\href{https://www.nytimes.com/2018/02/26/world/asia/china-xi-jinping-authoritarianism.html}{strongman}who
now appears intent on at least partially undoing Deng's political
legacy.

``Term limits, avoidance of a cult of personality, and the end of
routine political purges --- all of these were part of China's
reform-era leaders' efforts to steer the nation out of the chaos and
instability of the Maoist era,'' said
\href{https://www.fordham.edu/info/23165/carl_minzner}{Carl Minzner}, a
professor of law at Fordham University in New York and author of a
\href{https://global.oup.com/academic/product/end-of-an-era-9780190672089?cc=us\&lang=en\&}{new
book} on Mr. Xi's authoritarianism. ``As these start to fall like
dominoes, the operative question is: What could go next?''

Removing the term limit follows a series of political victories for Mr.
Xi, including being
\href{https://www.nytimes.com/2016/10/28/world/asia/xi-jinping-china.html?_r=0}{crowned
``core leader}'' of the party in 2016. But none of these steps ignited
as much astonishment, and disquiet, as Mr. Xi's decision to end term
limits on the presidency. Even most experts who thought Mr. Xi might
take that step assumed he would build up to it over several more years.

For now, opposition in China to the constitutional changes has mostly
been smothered. Censorship erases much online discussion, and
\href{https://www.nytimes.com/2018/03/08/world/asia/china-xi-jinping-term-limits-dissent.html}{critics
have been detained}. Many liberal intellectuals and former officials are
privately alarmed, but most also seem pessimistic about the potential to
rein in Mr. Xi unless a crisis breaks his authority.

Sunday's vote showed Mr. Xi's control of the National People's Congress,
which in theory is separate from the Communist Party. The handful of
dissenting votes this time was lower than the 45 votes of no and
abstentions lodged against less important constitutional amendments in
1999, and the 27 no votes and abstentions in 2004.

Delegates are coached and cajoled by party organizers in meetings that
take place away from the television cameras, said
\href{http://wws.princeton.edu/faculty-research/faculty/rtruex}{Rory
Truex}, an assistant professor at Princeton University who studies the
National People's Congress.

``The atmosphere was very solemn and dignified,'' one delegate, Cai
Peihui, said of the discussions about the amendments before the vote.
``The democratic process is flawless.''

Advertisement

\protect\hyperlink{after-bottom}{Continue reading the main story}

\hypertarget{site-index}{%
\subsection{Site Index}\label{site-index}}

\hypertarget{site-information-navigation}{%
\subsection{Site Information
Navigation}\label{site-information-navigation}}

\begin{itemize}
\tightlist
\item
  \href{https://help.nytimes.com/hc/en-us/articles/115014792127-Copyright-notice}{©~2020~The
  New York Times Company}
\end{itemize}

\begin{itemize}
\tightlist
\item
  \href{https://www.nytco.com/}{NYTCo}
\item
  \href{https://help.nytimes.com/hc/en-us/articles/115015385887-Contact-Us}{Contact
  Us}
\item
  \href{https://www.nytco.com/careers/}{Work with us}
\item
  \href{https://nytmediakit.com/}{Advertise}
\item
  \href{http://www.tbrandstudio.com/}{T Brand Studio}
\item
  \href{https://www.nytimes.com/privacy/cookie-policy\#how-do-i-manage-trackers}{Your
  Ad Choices}
\item
  \href{https://www.nytimes.com/privacy}{Privacy}
\item
  \href{https://help.nytimes.com/hc/en-us/articles/115014893428-Terms-of-service}{Terms
  of Service}
\item
  \href{https://help.nytimes.com/hc/en-us/articles/115014893968-Terms-of-sale}{Terms
  of Sale}
\item
  \href{https://spiderbites.nytimes.com}{Site Map}
\item
  \href{https://help.nytimes.com/hc/en-us}{Help}
\item
  \href{https://www.nytimes.com/subscription?campaignId=37WXW}{Subscriptions}
\end{itemize}
