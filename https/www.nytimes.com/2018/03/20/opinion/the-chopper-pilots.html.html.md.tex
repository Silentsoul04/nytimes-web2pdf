Sections

SEARCH

\protect\hyperlink{site-content}{Skip to
content}\protect\hyperlink{site-index}{Skip to site index}

\href{https://myaccount.nytimes.com/auth/login?response_type=cookie\&client_id=vi}{}

\href{https://www.nytimes.com/section/todayspaper}{Today's Paper}

\href{/section/opinion}{Opinion}\textbar{}The Chopper Pilots

\href{https://nyti.ms/2u5NMOB}{https://nyti.ms/2u5NMOB}

\begin{itemize}
\item
\item
\item
\item
\item
\item
\end{itemize}

Advertisement

\protect\hyperlink{after-top}{Continue reading the main story}

Supported by

\protect\hyperlink{after-sponsor}{Continue reading the main story}

\href{/section/opinion}{Opinion}

\href{/column/vietnam-67}{Vietnam '67}

\hypertarget{the-chopper-pilots}{%
\section{The Chopper Pilots}\label{the-chopper-pilots}}

By Bill Lord

\begin{itemize}
\item
  March 20, 2018
\item
  \begin{itemize}
  \item
  \item
  \item
  \item
  \item
  \item
  \end{itemize}
\end{itemize}

\includegraphics{https://static01.nyt.com/images/2018/03/20/opinion/20Vietnam-newsletter/20Vietnam-newsletter-articleLarge.jpg?quality=75\&auto=webp\&disable=upscale}

We were the river people, but we also spent a lot of time on
helicopters. I was a radio operator in the 9th Infantry Division, based
in the Mekong Delta south of Saigon. By the time I left, someone told me
I had made more than 50 combat assaults via chopper. Most but not all of
them were routine insertions that could happen as often as three times
in a day. Occasionally there was light resistance. A few times there was
a good deal of shooting. And since you never really knew if and when the
shooting would start, we all developed our own little formula for when,
under fire, we would decide to jump out of the helicopter.

If I knew what a differential equation was, I would say this might have
been one. There were so many variables. Foremost was altitude. You could
jump from very high up and maybe break your legs. The forward speed of
the chopper was something to take into account. The landing area might
be water, mud or dry land. All were factors. You wanted out of that
chopper in the worst way because the chopper was the target. Still, you
didn't want to get panicky and jump too soon. So each individual had his
own leap point. Mine was probably about the height of jumping from the
roof of a one-story house. Survivable and a good middle ground balancing
all the risks.

The pilots did not have the luxury of jumping out. Helicopter pilots in
Vietnam were among the hardiest of the whole bunch of us. They took a
lot of casualties but they always seemed to be there when you needed
them. Flying us into hot landing zones, flying medevacs to ``dust off''
the wounded and just getting potshots from all over when they were in
the air meant there wasn't much in the way of a routine day for them.
They earned every accolade they received. Many, too many, didn't
survive: 2,165 helicopter pilots were killed in action, and another
2,500 crewmen.

Many of the survivors stuck with flying. Long after Vietnam those pilots
often showed up to fly news helicopters for the television stations
where I worked, and I loved to go flying with them. In uniform or out,
these were very cool customers.

A helicopter is an awkward contraption. There are huge competing
G-forces pulling in different directions, and it seems almost a miracle
it can fly. It takes no small amount of skill to fly one even without
the overlay of ground fire, steep landing zones and various
life-or-death emergencies. And these pilots in Vietnam were never
pampered.

We got a horrifying example of that one afternoon as we lined up to
board choppers coming in to take us to the next landing zone. We were
spread out in what were called pickets of six men each. Five groups were
in a line on the left separated by about 25 yards each. Five more were
on the right as the choppers descended onto our positions. You could
figure out quickly which bird was coming for you and it was easy to
follow it right to the ground. In this case as my eyes followed our
chopper, I noticed a short length of barbed-wire fence just a couple of
feet off the ground. It seemed too low to make any difference but the
chopper came in a little fast, causing the pilot to lift the nose and
drop the tail just enough for the tail rotor to hit that wire. The next
events happened so fast it's hard to imagine even now how we survived.

At the moment the tail rotor hit the strand of wire, the chopper flipped
onto its left side. The main rotor was driven into the ground and
splintered into a thousand pieces. It was just our good fortune to have
been on the right side of the chopper or we probably would not have
survived. We had dived onto the ground but we could still see the right
side door gunner and the co-pilot climbing out just as the now crashed
chopper burst into flames. The co-pilot must have known that was going
to happen because he exited the wreckage with a fire extinguisher. But
it wasn't to put out the fire. The fire was already beyond that. He
sprayed it directly on the plexiglass windshield in front of the pilot
who was struggling to get out. The cold spray of carbon dioxide shrank
the hot plastic and the windshield literally popped out. He pulled the
pilot to safety as the fire raged.

The left-side door gunner never had a chance. He was pinned under the
chopper right next to the fuel tank that was exploding into black smoke.
By now we were all up and everyone thought to flip the burning chopper
upright, but searing heat prevented us from getting near it. The gunner
died very quickly.

The pilot was distraught beyond all description. Anyone would call this
a tragic accident, but in his mind it was pilot error. In his mind his
mistake had taken the life of one of his crew. There isn't much worse
for a guy in his position.

It was a very bad scene. A smoldering chopper. A dead door gunner.
Scared soldiers and this inconsolable pilot sitting on the ground
wailing.

A few minutes into this drama several new choppers arrived on the scene,
one carrying a guy who was clearly the man in charge of this whole
chopper squadron. He was all business. He walked straight over to the
pilot and told him to get up off the ground. He never asked what
happened. No arm around the shoulder. He just walked the crying pilot
over to the helicopter he had just arrived in and ordered the pilot to
get in and take the stick.

The scene drove things home to us. This was a war. If you are going to
be an effective pilot in the future, there is no time for grieving now.
It was the ultimate version of getting back on the bicycle. But that's
how they did things. There was no time for sentiment.

I met up not long ago with a former Vietnam chopper pilot who had been a
few years ahead of me in our high school. He said it was the best job he
ever had, despite all the dangers. He still missed it. As we talked I
could tell that even now, 50 years later, he would happily get back in
the cockpit. He still had that gritty commitment that reminded me of all
the Vietnam pilots I had known. That's why we all trusted them with our
lives.

Advertisement

\protect\hyperlink{after-bottom}{Continue reading the main story}

\hypertarget{site-index}{%
\subsection{Site Index}\label{site-index}}

\hypertarget{site-information-navigation}{%
\subsection{Site Information
Navigation}\label{site-information-navigation}}

\begin{itemize}
\tightlist
\item
  \href{https://help.nytimes.com/hc/en-us/articles/115014792127-Copyright-notice}{©~2020~The
  New York Times Company}
\end{itemize}

\begin{itemize}
\tightlist
\item
  \href{https://www.nytco.com/}{NYTCo}
\item
  \href{https://help.nytimes.com/hc/en-us/articles/115015385887-Contact-Us}{Contact
  Us}
\item
  \href{https://www.nytco.com/careers/}{Work with us}
\item
  \href{https://nytmediakit.com/}{Advertise}
\item
  \href{http://www.tbrandstudio.com/}{T Brand Studio}
\item
  \href{https://www.nytimes.com/privacy/cookie-policy\#how-do-i-manage-trackers}{Your
  Ad Choices}
\item
  \href{https://www.nytimes.com/privacy}{Privacy}
\item
  \href{https://help.nytimes.com/hc/en-us/articles/115014893428-Terms-of-service}{Terms
  of Service}
\item
  \href{https://help.nytimes.com/hc/en-us/articles/115014893968-Terms-of-sale}{Terms
  of Sale}
\item
  \href{https://spiderbites.nytimes.com}{Site Map}
\item
  \href{https://help.nytimes.com/hc/en-us}{Help}
\item
  \href{https://www.nytimes.com/subscription?campaignId=37WXW}{Subscriptions}
\end{itemize}
