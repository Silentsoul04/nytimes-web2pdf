Sections

SEARCH

\protect\hyperlink{site-content}{Skip to
content}\protect\hyperlink{site-index}{Skip to site index}

\href{https://myaccount.nytimes.com/auth/login?response_type=cookie\&client_id=vi}{}

\href{https://www.nytimes.com/section/todayspaper}{Today's Paper}

\href{/section/opinion}{Opinion}\textbar{}Vietnam Wasn't Just an
American War

\href{https://nyti.ms/2G07IUD}{https://nyti.ms/2G07IUD}

\begin{itemize}
\item
\item
\item
\item
\item
\end{itemize}

Advertisement

\protect\hyperlink{after-top}{Continue reading the main story}

Supported by

\protect\hyperlink{after-sponsor}{Continue reading the main story}

\href{/section/opinion}{Opinion}

\href{/column/vietnam-67}{Vietnam '67}

\hypertarget{vietnam-wasnt-just-an-american-war}{%
\section{Vietnam Wasn't Just an American
War}\label{vietnam-wasnt-just-an-american-war}}

By Lan Cao

\begin{itemize}
\item
  March 22, 2018
\item
  \begin{itemize}
  \item
  \item
  \item
  \item
  \item
  \end{itemize}
\end{itemize}

\includegraphics{https://static01.nyt.com/images/2018/03/22/opinion/22cao/22cao-articleLarge.jpg?quality=75\&auto=webp\&disable=upscale}

More than 40 years after its end, the Vietnam War remains, for
Americans, essentially an American experience, or more accurately, an
American metaphor. The continuing American inability and unwillingness
to include the Vietnamese perspective also speaks volumes about how the
United States relates to the rest of the world. Americans want to be
understood but rarely want to understand others.

Hollywood movies have been particularly powerful in shaping America's
Vietnam War narrative. The story line may change, but the backdrop is
the same. Chaos, not just wartime chaos, not just the proverbial ``war
is hell'' chaos, but its Asiatic variant --- with the inscrutable and
unknowable always lurking and pouncing upon the naïve American
protagonist. Predictably, as in ``Apocalypse Now,'' there were fetid
rivers, torrid jungles and impassive brown faces as backdrops to an
American soldier's odyssey through the heart of darkness that is
Vietnam.

The overarching theme was Vietnam's meaninglessness and what it did to
Americans and America. There was always the tortured, psychologically
unglued Vietnam veteran like the one in ``The Deer Hunter.'' Metaphor
takes over as American prisoners of war were forced to endure Russian
roulette, a supposedly popular game played in back alleys by natives who
seemingly have little value for life. Whether this game existed was
irrelevant. It served to symbolize the brutality of war and the lunacies
of this particular war.

In ``Full Metal Jacket,'' the main character, Private Joker, wore both a
peace symbol and the slogan ``Born to Kill.'' The contradictions and
ironies of Vietnam became more obvious when an officer declared that
henceforth ``search and destroy'' missions would be described more
palatably as ``sweep and clear.''

``Platoon'' changed the dominant narrative of the crazed Vietnam vet,
rehabilitating him by recreating the whole picture to depict, as Oliver
Stone put it, ``what it was like to be there.'' In the quest for
authenticity, a former Marine Corps captain put the actors through a
14-day boot camp. They ate military rations, were forbidden to shower,
and had to sleep in the jungle, with real-night watch rotations. Special
packs of Marlboro cigarettes were made with the shade of cherry-red that
match the Marlboros of the 1960s.

No attempt was made to portray the Vietnamese authentically. In scenes
where Vietnamese villagers huddled in the background, the language they
spoke was not Vietnamese. It was not even any real language but rather
deep-throated grunts meant to simulate native talk. To be fair, all this
changed when Mr. Stone made his next Vietnam movie, ``Heaven and
Earth,'' which broke new ground because the central character was a
Vietnamese woman and because the movie paid attention to the details of
Vietnam village custom --- from lacquering teeth as protection against
cavities to chewing betel nut.

\includegraphics{https://static01.nyt.com/images/2018/03/22/opinion/22cao2/22cao2-articleLarge.jpg?quality=75\&auto=webp\&disable=upscale}

But the meticulousness of ``Heaven and Earth'' remains a rarity in
Hollywood. Other Vietnam movies were vehicles for American soul
searching about involvement in the war. ``Born on the Fourth of July''
depicts a soldier's transformation from gung-ho veteran to war
protester. The pendulum swings the other way with ``Rambo,'' about a
one-man army who overcame resistance from pusillanimous politicians at
home to return to Vietnam to rescue American P.O.W.s. There is no
ambiguity or skepticism about America here. Part heroic fantasy, part
bloodthirsty revenge, ``Rambo'' gets to replay the war as it could have
been, asking, ``do we get to win this time?''

The Vietnamese remain irrelevant and invisible even in the universe of
the novel. A novel can accommodate complexity of plot and character. But
American writers for the most part are uninterested in and indifferent
to the Vietnamese. In ``The Things They Carried,'' Tim O'Brien tells the
interrelated stories of men from a single platoon and the things they
took to war, down to the smallest details: can openers, pocketknives,
wristwatches, mosquito repellent, chewing gum, cigarettes, salt tablets,
Kool-Aid, matches, sewing kits, C rations, along with weapons --- and of
course grief. Encounters with the Vietnamese were barely worth a nod,
and as far as allied Vietnamese soldiers were concerned, they were
pithily dismissed by the narrator as ``useless.''

Other notable Vietnam novels, such as ``13th Valley,'' ``Matterhorn,''
``A Rumor of War,'' ``Fields of Fire'' and ``Tree of Smoke,'' have been
celebrated as telling the war as it truly was --- drudgery,
demoralization, disinformation, dehumanization. They also all center on
an American central character transformed in one way or another by a
place that was not a country but a vehicle for American metamorphosis.

But in the realm of movies and literature, one might say creators have
poetic license. Historical endeavors, such as the Ken Burns-Lynn Novick
2017 PBS documentary, or the Vietnam War exhibitions presented by the
New-York Historical Society and the National Archives, would surely be
different. They were not. All presented the Vietnam War along the same
conventional trajectory that treats the Vietnamese perspective and
experience as an aside.

Out of the myriad photos, videos and oral histories compiled and
presented by the historical society and the National Archives, the
Vietnamese were granted minimal space. Certainly, there was no outright
exclusion, which would have been ludicrous and damning. But one got the
sense that the Vietnamese elements were perfunctory, added only to allow
the exhibitions to claim inclusion. By contrast, painstaking care was
devoted to showing the full range of the American experience, from the
mundane and ordinary to the more pressing immediacy of war.

With respect to the PBS documentary, almost universally lauded as
definitive, one might object to my characterization by pointing to the
many South Vietnamese who were in fact interviewed for the series. The
truth is, despite professed desires to include multiple perspectives,
this series privileges the American perspective, whether it is
supportive or critical of the American war effort.

It could have embarked on the worthwhile project of soothing American
souls and healing divisions without sidelining the Vietnamese. The South
Vietnamese were portrayed as incompetent and corrupt, their government
bordering on illegitimate. This was done in a subtle and sophisticated
way, not in the imperial manner of the past in which the West
pontificates about the shortcomings of the native. South Vietnamese
interviewees were allowed to express themselves in their own words.

But hovering above the cacophony is the seemingly more objective and
reliable narrator, Peter Coyote, whose disembodied voice signaled
omniscience. His narrative was peppered with adjectives like corrupt or
weak or beleaguered when describing the South Vietnamese.

As a writer, I know that I write about a minor character differently
from how I do my main character. The protagonist is carefully developed
and infused with an active voice. By contrast, I might use my author's
prerogative of description and prescription for minor characters. From
tone to space to treatment, I know that the South Vietnamese is the
minor character in the American historical exposition of the Vietnam
War.

In the section of the PBS series about the Tet offensive of 1968, for
example, there were hardly any South Vietnamese soldiers whose voices
were included. Instead, their experiences were described and summarized.
Even the postwar ``re-education camps'' and the suffering of the
hundreds of thousands of ``boat people'' merited little time. They
hardly mattered once American troops were pulled out and American
P.O.W.s returned.

But North Vietnamese and Vietcong voices were amply heard, one might
object. How does one explain this? Including enemy voices serves
American interests and elevates the American national character as one
capable of healing and reconciliation. In addition, showing the enemy as
dedicated and fierce also explains why the United States didn't win the
war. Simultaneously marginalizing or presenting the South Vietnamese as
incompetent serves a different but related objective --- it explains why
despite American blood, sweat and tears, the war could not be won. In
fact, the series was much more concerned with demonstrating that the war
could never have been won. On this issue, the Vietnamese point of view
was excluded entirely.

The way the Vietnam War series portrayed Vietnam is emblematic of the
way the United States relates to the world. American suffering is deemed
exceptional and presented with empathy and compassion; it is never a
statistic or a number. American soldiers who died in the Vietnam War
have a monument that names each of them. The dead from Sept. 11 are also
known as individuals, and their names are appropriately read out on the
anniversary. All the American M.I.A.s from the Vietnam War must be
accounted for, because they are American.

American exceptionalism also means American experiences are projected as
universal. Thus, the rules must change when calamity strikes America.
Many countries have suffered terrorist attacks for years, and their
citizens killed and maimed. But after Sept. 11, the United States argues
that the global landscape has changed and new rules must be created.
Waterboarding is not torture. Detention of enemies in Guantánamo away
from American soil is legitimate.

But the truth is that American suffering is not unique and American
tragedies are not more tragic than tragedies in other countries. There
are many mighty countries that are not beloved. America, however, is a
beacon for many in the world because it promises inclusion. It is
exceptional when it embraces others and includes within its heart the
perspectives of others different from it.

Advertisement

\protect\hyperlink{after-bottom}{Continue reading the main story}

\hypertarget{site-index}{%
\subsection{Site Index}\label{site-index}}

\hypertarget{site-information-navigation}{%
\subsection{Site Information
Navigation}\label{site-information-navigation}}

\begin{itemize}
\tightlist
\item
  \href{https://help.nytimes.com/hc/en-us/articles/115014792127-Copyright-notice}{©~2020~The
  New York Times Company}
\end{itemize}

\begin{itemize}
\tightlist
\item
  \href{https://www.nytco.com/}{NYTCo}
\item
  \href{https://help.nytimes.com/hc/en-us/articles/115015385887-Contact-Us}{Contact
  Us}
\item
  \href{https://www.nytco.com/careers/}{Work with us}
\item
  \href{https://nytmediakit.com/}{Advertise}
\item
  \href{http://www.tbrandstudio.com/}{T Brand Studio}
\item
  \href{https://www.nytimes.com/privacy/cookie-policy\#how-do-i-manage-trackers}{Your
  Ad Choices}
\item
  \href{https://www.nytimes.com/privacy}{Privacy}
\item
  \href{https://help.nytimes.com/hc/en-us/articles/115014893428-Terms-of-service}{Terms
  of Service}
\item
  \href{https://help.nytimes.com/hc/en-us/articles/115014893968-Terms-of-sale}{Terms
  of Sale}
\item
  \href{https://spiderbites.nytimes.com}{Site Map}
\item
  \href{https://help.nytimes.com/hc/en-us}{Help}
\item
  \href{https://www.nytimes.com/subscription?campaignId=37WXW}{Subscriptions}
\end{itemize}
