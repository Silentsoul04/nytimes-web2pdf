Sections

SEARCH

\protect\hyperlink{site-content}{Skip to
content}\protect\hyperlink{site-index}{Skip to site index}

\href{https://www.nytimes.com/section/politics}{Politics}

\href{https://myaccount.nytimes.com/auth/login?response_type=cookie\&client_id=vi}{}

\href{https://www.nytimes.com/section/todayspaper}{Today's Paper}

\href{/section/politics}{Politics}\textbar{}Fund-Raiser Held Out Access
to Trump as a Prize for Prospective Clients

\url{https://nyti.ms/2pECyvY}

\begin{itemize}
\item
\item
\item
\item
\item
\item
\end{itemize}

Advertisement

\protect\hyperlink{after-top}{Continue reading the main story}

Supported by

\protect\hyperlink{after-sponsor}{Continue reading the main story}

\hypertarget{fund-raiser-held-out-access-to-trump-as-a-prize-for-prospective-clients}{%
\section{Fund-Raiser Held Out Access to Trump as a Prize for Prospective
Clients}\label{fund-raiser-held-out-access-to-trump-as-a-prize-for-prospective-clients}}

\includegraphics{https://static01.nyt.com/images/2018/03/24/us/24dc-Broidy-1/merlin_135912882_9b7adb6e-7f55-45a8-a909-c77800452712-articleLarge.jpg?quality=75\&auto=webp\&disable=upscale}

By \href{https://www.nytimes.com/by/kenneth-p-vogel}{Kenneth P. Vogel}
and \href{https://www.nytimes.com/by/david-d-kirkpatrick}{David D.
Kirkpatrick}

\begin{itemize}
\item
  March 25, 2018
\item
  \begin{itemize}
  \item
  \item
  \item
  \item
  \item
  \item
  \end{itemize}
\end{itemize}

WASHINGTON --- For Elliott Broidy, Donald J. Trump's presidential
campaign represented an unparalleled political and business opportunity.

An investor and defense contractor, Mr. Broidy became a top fund-raiser
for Mr. Trump's campaign when most elite Republican donors were keeping
their distance, and Mr. Trump in turn overlooked the lingering whiff of
scandal from Mr. Broidy's
\href{https://dealbook.nytimes.com/2009/12/03/guilty-plea-in-new-york-pension-bribery-case/}{2009
guilty plea} in a pension fund bribery case.

After Mr. Trump's election, Mr. Broidy quickly capitalized, marketing
his Trump connections to politicians and governments around the world,
including some with unsavory records, according to interviews and
documents obtained by The New York Times. Mr. Broidy suggested to
clients and prospective customers of his Virginia-based defense
contracting company, Circinus, that he could broker meetings with Mr.
Trump, his administration and congressional allies.

Mr. Broidy's ability to leverage his political connections to boost his
business illuminates how Mr. Trump's unorthodox approach to governing
has spawned a new breed of access peddling in the swamp he
\href{https://www.nytimes.com/2016/11/11/us/politics/trump-government.html}{vowed
to drain}.

Mr. Broidy offered tickets to V.I.P. inauguration events, including a
candlelight dinner attended by Mr. Trump, to a Congolese strongman
accused of funding a lavish lifestyle with public resources. He helped
arrange a meeting with Republican senators and offered a trip to
Mar-a-Lago, the president's private Florida resort, for an Angolan
politician. And he arranged an invitation to a party at Mr. Trump's
Washington hotel for a Romanian parliamentarian facing corruption
charges, who posted a
\href{https://www.facebook.com/liviudragnea.ro/posts/1277208735691204}{photograph
with the president} on Facebook.

This type of access has value on the international stage, where the
\href{https://www.politico.com/story/2017/02/trump-ukraine-russia-sanctions-234631}{perception
of support from an American president} --- or even a photo with one ---
can benefit foreign leaders back home.

Mr. Broidy was open about his business interests, but the administration
made no effort to curtail his offers of access to clients or prospective
clients.

Yet Mr. Broidy was so aggressive, some associates said, that they warned
him to tone down his approach for fear that he might run afoul of the
president, clients or American lobbying and anti-corruption laws.

As with so many other political conventions, Mr. Trump has upended the
traditional system of access to the president, among the most prized
chits in Washington. That is partly because of lax vetting that has
allowed largely unfettered access to Mr. Trump and his White House by
loyalists, friends and hangers-on with their own policy agendas or
business interests.

But it is also because few of Washington's established lobbyists have
close connections to the president. In their place, a new class of
insider has emerged, able to lobby the president directly on behalf of
clients or business partners, an uncommon opportunity in prior
administrations, when lobbyists focused on winning support from
lawmakers or regulators.

In a statement, Mr. Broidy said the success of his company reflected his
recruitment of ``the best people, including a number of former flag
officers,'' not his political connections. ``The idea that our success
derives from activities around last year's presidential inauguration is
not only misplaced but insulting to the careers and capabilities of our
highly trained and decorated veterans,'' he said.

Mr. Broidy, 60, in some ways personifies this new breed of Trump
insider. A lifelong Los Angeles resident who worked as an accountant
before making a fortune as an investor, Mr. Broidy grew more interested
in politics after the Sept. 11 attacks, which shaped his hawkish
approach to foreign policy and concern for the defense of Israel. In
2005, he was appointed to a federal homeland security advisory panel.

Mr. Broidy pleaded guilty in 2009 to giving nearly \$1 million in
illegal gifts to New York state officials to help his company land a
\$250 million contract with the state's public pension fund. He paid
\$18 million in restitution and avoided jail time.

As he worked to rebuild his bank account and reputation, he expanded his
business interests in the defense and security industry, starting a
company called Threat Deterrence in 2013 and purchasing Circinus in July
2015.

\includegraphics{https://static01.nyt.com/images/2018/03/24/us/24dc-Broidy-3/merlin_133856088_766bc7f9-9691-4bef-8577-31004dd96f46-articleLarge.jpg?quality=75\&auto=webp\&disable=upscale}

Mr. Broidy initially supported Senator Ted Cruz of Texas for the
Republican presidential nomination in 2016, embracing Mr. Trump only
after Mr. Cruz
\href{https://www.nytimes.com/2016/05/04/us/politics/ted-cruz.html}{dropped
out}.

His business took off after Mr. Trump's election. Circinus has won
lucrative work from around the world, including contracts worth more
than \$200 million to do defense work for the United Arab Emirates. And
Mr. Broidy openly promoted Circinus's work in meetings with Mr. Trump
and other Republican officials, according to documents and interviews.

But Mr. Broidy faces a sudden backlash.

One of his business partners, George Nader,
\href{https://www.nytimes.com/2018/03/06/us/politics/george-nader-special-counsel-mueller-cooperating-seychelles.html}{is
cooperating} with the special counsel, whose investigators have asked
about Mr. Nader's contacts with top Trump administration officials as
well as his possible role in funneling money from the U.A.E. to Mr.
Trump's political efforts, according to people familiar with the
inquiry.

Mr. Nader helped Circinus gain access to U.A.E.'s crown prince, while
also using Mr. Broidy as a conduit
\href{https://www.nytimes.com/2018/03/21/us/politics/george-nader-elliott-broidy-uae-saudi-arabia-white-house-influence.html}{to
shape Trump administration policy} toward the Persian Gulf on behalf of
the U.A.E. and Saudi Arabia, both American allies. Mr. Broidy, in turn,
appears to have helped Mr. Nader get his photograph taken with Mr. Trump
at a fund-raiser, over the unspecified objections of the Secret Service.
Mr. Nader has been convicted on charges related to child pornography and
\href{http://hosted2.ap.org/PASHA/a5050f4ad4f44dafab85bb41a15281cf/Article_2018-03-15-US-Trump-Russia-Probe/id-93db9ed0b5054340b9acc851355ce56b}{sexual
abuse of minors}.

Hundreds of pages of Mr. Broidy's emails, proposals and contracts were
provided to The Times by an anonymous group critical of Mr. Broidy's
advocacy of American foreign policies in the Middle East. Mr. Broidy's
representatives say
\href{https://assets.documentcloud.org/documents/4417582/Elliott-Broidy-Letter-to-Qatari-Ambassador.pdf}{they
have proof} that the documents were
\href{https://www.nytimes.com/2018/03/05/world/middleeast/qatar-trump-hack-email.html}{stolen
by hackers working for Qatar} in retaliation for his efforts to rally
opposition in Washington to Qatar, a regional nemesis of the Saudis and
the Emiratis. The Qataris dismiss this charge.

The documents reveal that Mr. Broidy, a vice chairman of the finance
committee for Mr. Trump's inauguration, arranged invitations to parties
celebrating the event for foreign leaders with whom Circinus worked to
sign contracts that could have been worth hundreds of millions of
dollars. Mr. Broidy in some cases presented the invitations in a manner
that suggested they were linked to their countries' willingness to do
business with Circinus.

For instance, Mr. Broidy invited Denis Sassou-Nguesso, the longtime
president of the Republic of Congo, to a handful of inauguration week
events, including the candlelight dinner featuring Mr. Trump, Vice
President Mike Pence and their wives. ``Your name has been submitted and
approved,'' Mr. Broidy wrote to Mr. Sassou-Nguesso in a note on Broidy
Capital Management letterhead. Mr. Broidy stressed that the invitation
was from Mr. Broidy and ``is not coming from the Joint Congressional
Committee'' on Inaugural Ceremonies, which oversees the swearing-in and
a luncheon at the Capitol.

The day after the letter was dated, Mr. Broidy asked his team at
Circinus to prepare an invoice for \$2 million to Mr. Sassou-Nguesso's
office for ``military capabilities assessment and review/services,''
according to emails and documents.

People close to Mr. Broidy said no invoice was sent and Circinus never
worked for the Congolese government. Mr. Sassou-Nguesso declined the
invitation to the inauguration festivities, they said.

Donors to previous inaugurations were also able to invite foreign heads
of state to events as their guests, a spokesman for Mr. Broidy noted.

Two officials from Romania, another prospective Circinus client, met Mr.
Trump at a private party to which Mr. Broidy invited them during
inauguration week at the
\href{https://www.nytimes.com/2017/08/25/us/politics/trump-hotel-washington.html}{Trump
International Hotel} near the White House. Liviu Dragnea, a
parliamentary leader who runs the Social Democratic Party, called for
closer ties between Romania and the United States, prompting Mr. Trump
to declare, ``We will make it happen! Romania is important for us!''
according to Mr. Dragnea's Facebook post,
\href{http://www.mcclatchydc.com/news/politics-government/article198952434.html}{reported
by McClatchy}.

At the time, Mr. Dragnea faced corruption charges, which he
\href{https://actmedia.eu/daily/speaker-of-the-chamber-of-deputies-liviu-dragnea-before-the-high-court-of-cassation-and-justice-under-the-accusation-of-abuse-of-office/74617}{continues
to fight}. But for months after the inauguration, Circinus continued
pitching his government partly on the access it could provide to Mr.
Trump, according to a person who works with the Romanian government and
was briefed on the solicitation.

Circinus has yet to receive contracts from Romania, people close to Mr.
Broidy said. The company entered into an agreement last month with a
Romanian government-owned defense company that appears to give it the
inside track for contracts valued at more than \$200 million, according
to Romanian news media and people familiar with the contracting process.

Similarly, in a letter dated Jan. 3, 2017, and emailed to top Angolan
government officials, Mr. Broidy indicated that he was also sending both
an invitation to the inauguration festivities and a proposal for
Circinus to provide security services for Angola. In one version of the
proposal, the company sought nearly \$64 million over five years. ``With
numerous preparations ahead, we request that you kindly return the
executed document no later than January 9, 2017,'' Mr. Broidy wrote in
the letter to João Lourenço, then the Angolan defense minister, and
another Angolan official.

Three days before inauguration, the Angolans submitted a \$6 million
payment to Circinus, which people close to Mr. Broidy say was \$2
million less than the Angolans had agreed to pay. That same day, the
Angolans and Mr. Broidy met on Capitol Hill with senators including Tom
Cotton of Arkansas and Ron Johnson of Wisconsin, both Republicans,
encounters arranged by Mr. Broidy and his team.

Image

President João Lourenço of Angola was invited by Mr. Broidy last year to
Mar-a-Lago, Mr. Trump's Florida resort, even as Mr. Broidy pressed
Angola for payments on a contract. Mr. Lourenço never responded to the
letters.Credit...Ampe Rogerio/Agence France-Presse --- Getty Images

About one month later, Mr. Broidy again wrote to Mr. Lourenço,
discussing plans for him ``to be joining us at Mira Largo'' --- an
apparent reference to Mar-a-Lago, where Mr. Broidy is a member and has
been spotted introducing guests to the president --- ``this coming
weekend.''

``Many preparations have been made in advance of your visit, including
additional meetings at the Capitol and the Department of Treasury,'' Mr.
Broidy wrote.

He went on to remind Mr. Lourenço that Angola was past due on ``the
balance of the payment'' for Circinus's services. ``Please make this
payment immediately,'' he wrote.

Not long after Mr. Lourenço was inaugurated as president of Angola in
September, Mr. Broidy wrote again to congratulate him and offer to help
set up meetings with Mr. Trump and Mr. Pence. ``Before the formalities
which would naturally be in order between our governments, I am able to
assess and advise in this matter, and would be delighted to help foster
such closer relation between Luanda and Washington,'' Mr. Broidy wrote,
referring to the Angolan capital.

Mr. Lourenço did not come to Mar-a-Lago and never responded to the
letters, according to people close to Mr. Broidy. Circinus never
collected any additional funds from Angola beyond the initial \$6
million payment, they said.

Mr. Broidy emphasizes to potential clients how his connections to Mr.
Trump could benefit them, according to several people who have worked
with him or heard his pitches.

``He was mentioning that he has political connections, and it could be
helpful,'' said Eymen Errais, a counselor to the Tunisian Ministry of
Development, Investment and International Cooperation.

Mr. Errais and the department's minister, Fadhel Abdelkefi, met with Mr.
Broidy and Circinus executives in Tunisia a few weeks after Mr. Trump's
election. Circinus sought a contract to build an open-source
intelligence center for the Tunisian government, a person familiar with
the meeting said, but Mr. Abdelkefi's ministry instead sought to
persuade Mr. Broidy to invest in the Tunisian hospitality industry.

Nonetheless, in an email to Mr. Abdelkefi just after the meeting, Mr.
Broidy vowed that the center ``would be an extremely useful and
effective tool for your government'' and again emphasized his connection
to Mr. Trump. He characterized the incoming administration as ``an
unprecedented opportunity, in that the House, Senate and executive
branch of our government are all under Republican control for the first
time in 10 years.''

Circinus prepared a proposal for a five-year contract totaling \$80
million for the Tunisian work. But Mr. Errais said the Tunisians decided
not to do business with Circinus. ``A connection with Trump --- how
would that help us for security?'' he said.

A person close to Mr. Broidy said he discussed the Republicans' control
of Congress and the White House to suggest that the American government
might increase spending on counterterrorism.

Mr. Broidy's allies say he has never hidden the overlap between his
business interests and his political activity. In a memorandum about an
October Oval Office meeting, for example, Mr. Broidy wrote that he had
volunteered to Mr. Trump that his company was seeking a contract from
the United Arab Emirates and had tried to arrange a private meeting for
the president to discuss the matter with their leader, Crown Prince
Mohammed bin Zayed.

But as scrutiny has mounted around Mr. Broidy, Republican officials
signaled that he would be a distraction at a high-dollar fund-raiser
with Mr. Trump this month in Los Angeles. Mr. Broidy helped plan it and
\href{https://twitter.com/kenvogel/status/973623387951124487}{was listed
as a co-host}.

After conversations with Ronna McDaniel, the
\href{https://www.nytimes.com/2018/01/13/us/politics/ronna-romney-mcdaniel.html}{chairwoman
of the Republican National Committee}, he volunteered not to attend,
according to people briefed on the situation.

The dinner and round-table with the president proceeded without him.

Advertisement

\protect\hyperlink{after-bottom}{Continue reading the main story}

\hypertarget{site-index}{%
\subsection{Site Index}\label{site-index}}

\hypertarget{site-information-navigation}{%
\subsection{Site Information
Navigation}\label{site-information-navigation}}

\begin{itemize}
\tightlist
\item
  \href{https://help.nytimes.com/hc/en-us/articles/115014792127-Copyright-notice}{©~2020~The
  New York Times Company}
\end{itemize}

\begin{itemize}
\tightlist
\item
  \href{https://www.nytco.com/}{NYTCo}
\item
  \href{https://help.nytimes.com/hc/en-us/articles/115015385887-Contact-Us}{Contact
  Us}
\item
  \href{https://www.nytco.com/careers/}{Work with us}
\item
  \href{https://nytmediakit.com/}{Advertise}
\item
  \href{http://www.tbrandstudio.com/}{T Brand Studio}
\item
  \href{https://www.nytimes.com/privacy/cookie-policy\#how-do-i-manage-trackers}{Your
  Ad Choices}
\item
  \href{https://www.nytimes.com/privacy}{Privacy}
\item
  \href{https://help.nytimes.com/hc/en-us/articles/115014893428-Terms-of-service}{Terms
  of Service}
\item
  \href{https://help.nytimes.com/hc/en-us/articles/115014893968-Terms-of-sale}{Terms
  of Sale}
\item
  \href{https://spiderbites.nytimes.com}{Site Map}
\item
  \href{https://help.nytimes.com/hc/en-us}{Help}
\item
  \href{https://www.nytimes.com/subscription?campaignId=37WXW}{Subscriptions}
\end{itemize}
