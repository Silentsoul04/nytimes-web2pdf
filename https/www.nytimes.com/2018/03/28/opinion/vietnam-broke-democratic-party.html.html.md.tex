Sections

SEARCH

\protect\hyperlink{site-content}{Skip to
content}\protect\hyperlink{site-index}{Skip to site index}

\href{https://myaccount.nytimes.com/auth/login?response_type=cookie\&client_id=vi}{}

\href{https://www.nytimes.com/section/todayspaper}{Today's Paper}

\href{/section/opinion}{Opinion}\textbar{}How Vietnam Broke the
Democratic Party

\href{https://nyti.ms/2GuocYP}{https://nyti.ms/2GuocYP}

\begin{itemize}
\item
\item
\item
\item
\item
\item
\end{itemize}

Advertisement

\protect\hyperlink{after-top}{Continue reading the main story}

Supported by

\protect\hyperlink{after-sponsor}{Continue reading the main story}

\href{/section/opinion}{Opinion}

\href{/column/vietnam-67}{Vietnam '67}

\hypertarget{how-vietnam-broke-the-democratic-party}{%
\section{How Vietnam Broke the Democratic
Party}\label{how-vietnam-broke-the-democratic-party}}

By Michael Nelson

\begin{itemize}
\item
  March 28, 2018
\item
  \begin{itemize}
  \item
  \item
  \item
  \item
  \item
  \item
  \end{itemize}
\end{itemize}

\includegraphics{https://static01.nyt.com/images/2018/03/28/opinion/28Vietnam-Nelson/28Vietnam-Nelson-articleLarge.jpg?quality=75\&auto=webp\&disable=upscale}

Fifty years later, the 1968 presidential election and the Vietnam War
still shadow American politics. But the war actually had little effect
on the vote that November --- even though surveys showed that Vietnam
was by far the most important issue on voters' minds, they saw little
difference between the Republican nominee, Richard Nixon, and his
Democratic opponent, Hubert H. Humphrey, both of whom they scored
``slightly conservative'' on the University of Michigan's seven-point
``Vietnam scale.''

The real consequences of the election and the war were on the Democratic
Party, with collateral effects on the Republicans. The Democrats'
united, confident and longstanding commitment to the spread of liberal
values throughout the world eroded in the aftermath. Fragmented by
internal divisions over the war, the party also overhauled its process
for choosing presidential candidates in ways that upended its previous
domination by Southerners, unions and big-city bosses.

The decade had begun with the election of John F. Kennedy, whose main
issue in the 1960 presidential campaign was the Cold War, which he
believed America was losing to the Soviet Union. Kennedy pledged in his
Inaugural Address to do whatever was necessary ``to assure the survival
and success of liberty,'' a declaration that placed him squarely in the
tradition of his party.

Ever since Woodrow Wilson and Franklin D. Roosevelt led the United
States onto the world stage as a promoter of liberal values during the
first half of the 20th century, most Democrats had been comfortable with
this role. Indeed, an animating premise of Democratic liberalism was
that the federal government has the ability to solve virtually any
problem it chooses to take on, domestic or foreign.

Soon after taking office, Kennedy began escalating American military
involvement in Vietnam. In June 1961, following the failed Bay of Pigs
invasion of Cuba, he told James Reston of The New York Times that ``we
have a problem in making our power credible'' and ``Vietnam is the
place.'' By November 1963, Kennedy had dispatched 16,300 military
advisers to South Vietnam. As late as the morning of Nov. 22, he said,
``Without the United States, South Vietnam would collapse overnight.''

Assuming the presidency when Kennedy was assassinated a few hours later,
Lyndon B. Johnson felt bound to continue his predecessor's course. Less
confident about Vietnam than he was about domestic matters, Johnson
regularly asked the foreign policy advisers he inherited from Kennedy
--- chiefly Secretary of State Dean Rusk, Secretary of Defense Robert S.
McNamara, and his national security adviser, McGeorge Bundy --- what
Kennedy would have done, and then he did it. Their confident
recommendation was that the American commitment to victory in Vietnam
must be maintained.

After Johnson was elected in 1964 to a term in his own right, hawkish
pressures from the administration's Kennedy alumni intensified. In
February 1965 Bundy returned from South Vietnam and said that the war
was lost unless the United States launched a sustained bombing campaign
against the North. Johnson quickly approved Operation Rolling Thunder,
which lasted, with brief pauses, for three years, the largest sustained
air campaign in the history of warfare. With repeated reassurance from
his advisers that eventually a ``crossover point'' would be reached at
which the Communists decided they couldn't win, Johnson steadily
increased the American troop presence in Vietnam, which rose above a
half-million.

To raise the manpower needed, Johnson saw conscription as a less
politically risky approach than calling up the reserves or National
Guard, which would have forced many married, middle-class men to leave
their jobs and families. College students, whose numbers had swelled to
7.5 million from 2.1 million in 1952, became increasingly alarmed that
the draft soon would extend to them. ``The draft was the best organizing
tool we had,'' said the antiwar activist Sam Brown. Not just male
students but also their sisters and girlfriends joined protests against
the war on campuses and in Washington.

Demonstrations were not the only manifestation of the growing opposition
to the war. Senator Eugene McCarthy and --- after seeing how vulnerable
Johnson was to an intraparty challenge when McCarthy nearly won the
March 12 New Hampshire primary --- Senator Robert F. Kennedy entered the
race for the Democratic nomination. Days afterward, Johnson's ``wise
men,'' 14 high-ranking foreign policy officials from recent
administrations who previously had endorsed his conduct of the war, told
the president that ``we must take steps to disengage.'' Johnson withdrew
from the election at the end of the month.

As the spring primary season unfolded, McCarthy vied with Kennedy to be
the main antiwar alternative to Vice President Humphrey, who delayed
announcing his candidacy until April 27, too late to compete in the
primaries. Kennedy won all but one head-to-head primary contest with
McCarthy before being assassinated on June 4. But even if Kennedy had
lived, Humphrey's nomination was essentially sealed by support from the
Democratic Party's Southern, labor and organizational wings, which
dominated delegate selection and, ultimately, controlled the nomination.

Humphrey wanted to move his party's platform in a slightly dovish
direction to placate Kennedy and McCarthy supporters, but he backed off
when Johnson told him that doing so would ``endanger American troops,''
that he ``would have their blood on my hands.'' The consequence was that
the Democratic platform ended up more hawkish on Vietnam than the
Republican one, which at least called for a ``de-Americanization'' of
the war.

Plagued by war protesters and trailing Nixon badly in the polls,
Humphrey pledged at the end of September to stop the bombing of North
Vietnam. His campaign steadily gained strength from that day on, rising
from a low of 29 percent in a Gallup poll early that fall to near parity
with Nixon by Election Day.

Nixon's victory, however narrow, broke the Democratic Party's record of
seven victories in the previous nine elections, most of them by a
landslide. Once united in support of an assertive foreign policy,
Democrats now were fiercely divided entering the 1972 presidential
nominating contest. One leading contender, the strongly anti-Communist
Senator Henry Jackson, was squarely in the old Democratic tradition.
With support from party heavyweights such as the A.F.L.-C.I.O. president
George Meany and Mayor Richard Daley of Chicago, he may well have been
nominated under the rules that prevailed in 1968.

But in a concession to antiwar delegates at the 1968 convention,
Humphrey had not opposed a resolution requiring that in the future ``all
feasible efforts'' would be ``made to assure that delegates are selected
through primary, convention or committee procedures open to public
participation.'' A postelection commission headed by Senator George
McGovern and Representative Donald Fraser fleshed out this resolution
with rules requiring that every delegate in 1972 and after be chosen in
a primary or caucus open to every Democratic voter.

With grass-roots party activists now driving delegate selection,
McGovern won 15 primaries and Jackson none. Humphrey had not won any
primaries either in 1968, but under the pre-McGovern-Fraser rules that
did not prevent the party's leaders from making him the nominee. In 1972
these leaders were outshouted by war opponents who had carried McGovern
to victory in the primaries.

McGovern lost the election to Nixon, but the new activist-centered party
he midwifed remained. The party lost interest in using American power,
and spent the next generation trying to constrain it. Democratic
Congresses voted to hem in Nixon's war-making power by enacting the War
Powers Resolution and forbade President Gerald Ford to continue to
supply arms to the government of South Vietnam, which quickly fell to
the Communists.

They hamstrung President Ronald Reagan's efforts to roll back Communist
advances in Central America and did their best to impose a ``nuclear
freeze'' on America's arsenal. They opposed President George H. W.
Bush's Gulf war.

Only with victory in the Cold War in the early 1990s did Democrats begin
regain a trace of their old confidence that American involvement in the
world, especially for humanitarian reasons, should sometimes be
embraced. Bill Clinton, the party's victorious nominee in 1992 and 1996,
made clear that he was not a ``McGovern Democrat.''

The Republican Party was a bystander to the changes in the presidential
nominating process, but was still affected by them. The Democrats' move
toward a primaries-based system required alterations to state election
laws that opened Republican nominations to grass-roots capture as well,
albeit grudgingly and with party leaders better able to hold the reins
of power for a longer time.

But just as the Democratic nomination after 1968 was won by political
outsiders such as McGovern in 1972 and, four years later, Jimmy Carter,
so eventually was the Republican. How else to explain the ability of
history's ultimate outsider, Donald Trump, to become the 2016 Republican
nominee despite the near unanimous opposition of established party
leaders?

And is it mere coincidence that what had happened among Democrats
starting with McGovern finally happened among Republicans? Voters in
both parties have always been more reluctant than public officials to
embrace foreign aid, military deployments and multinational agreements.
In 2016, Republican primary voters seized the opportunity to choose a
candidate who disdained his party's previous support of these pillars of
an assertive foreign policy. History doesn't repeat itself, but
sometimes it does rhyme.

Advertisement

\protect\hyperlink{after-bottom}{Continue reading the main story}

\hypertarget{site-index}{%
\subsection{Site Index}\label{site-index}}

\hypertarget{site-information-navigation}{%
\subsection{Site Information
Navigation}\label{site-information-navigation}}

\begin{itemize}
\tightlist
\item
  \href{https://help.nytimes.com/hc/en-us/articles/115014792127-Copyright-notice}{©~2020~The
  New York Times Company}
\end{itemize}

\begin{itemize}
\tightlist
\item
  \href{https://www.nytco.com/}{NYTCo}
\item
  \href{https://help.nytimes.com/hc/en-us/articles/115015385887-Contact-Us}{Contact
  Us}
\item
  \href{https://www.nytco.com/careers/}{Work with us}
\item
  \href{https://nytmediakit.com/}{Advertise}
\item
  \href{http://www.tbrandstudio.com/}{T Brand Studio}
\item
  \href{https://www.nytimes.com/privacy/cookie-policy\#how-do-i-manage-trackers}{Your
  Ad Choices}
\item
  \href{https://www.nytimes.com/privacy}{Privacy}
\item
  \href{https://help.nytimes.com/hc/en-us/articles/115014893428-Terms-of-service}{Terms
  of Service}
\item
  \href{https://help.nytimes.com/hc/en-us/articles/115014893968-Terms-of-sale}{Terms
  of Sale}
\item
  \href{https://spiderbites.nytimes.com}{Site Map}
\item
  \href{https://help.nytimes.com/hc/en-us}{Help}
\item
  \href{https://www.nytimes.com/subscription?campaignId=37WXW}{Subscriptions}
\end{itemize}
