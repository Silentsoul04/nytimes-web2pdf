Sections

SEARCH

\protect\hyperlink{site-content}{Skip to
content}\protect\hyperlink{site-index}{Skip to site index}

\href{https://myaccount.nytimes.com/auth/login?response_type=cookie\&client_id=vi}{}

\href{https://www.nytimes.com/section/todayspaper}{Today's Paper}

\href{/section/opinion}{Opinion}\textbar{}How Vietnam Changed Journalism

\href{https://nyti.ms/2uA0Dcg}{https://nyti.ms/2uA0Dcg}

\begin{itemize}
\item
\item
\item
\item
\item
\item
\end{itemize}

Advertisement

\protect\hyperlink{after-top}{Continue reading the main story}

Supported by

\protect\hyperlink{after-sponsor}{Continue reading the main story}

\href{/section/opinion}{Opinion}

\href{/column/vietnam-67}{Vietnam '67}

\hypertarget{how-vietnam-changed-journalism}{%
\section{How Vietnam Changed
Journalism}\label{how-vietnam-changed-journalism}}

By Andrew Pearson

\begin{itemize}
\item
  March 29, 2018
\item
  \begin{itemize}
  \item
  \item
  \item
  \item
  \item
  \item
  \end{itemize}
\end{itemize}

\includegraphics{https://static01.nyt.com/images/2018/03/29/opinion/29vietnamWeb/29vietnamWeb-popup.jpg?quality=75\&auto=webp\&disable=upscale}

When I first got to Saigon as a journalist, in 1963, I took it for
granted that American policy to counter Communist expansion into the
southern part of Vietnam was the right thing to do. That was the
conventional wisdom from experience in Europe, where the Soviet Union
had established satellite countries on its border. My journalistic
perspective in the beginning was ``normal.'' A good American point of
view.

As I learned more about the complexities of the war, my journalism
became more accurate. The war's defenders might have said I was becoming
more critical, even biased. But in fact I was becoming more objective
--- I set aside the pro-American, anti-Communist filter I brought with
me to Vietnam and reported what I saw. In the news and documentary
reports I did, I showed that despite all the destruction, suffering and
cost, the war was being lost. Now I'd say it shouldn't have started.

My experience, and that of many, even most, American journalists in the
Vietnam War transformed our profession. We realized over the years that
the government was ill-informed and even wrong about issues of life and
death. As a result of that, reporters today are doing a better job
because they know about the evolution of that deception and what its
effects have been on American society.

When I realized that nothing was working out the way the United States
wanted, I began doing TV programs that showed how bad the situation was.
Sometimes people in the New York office said, this can't be right
because of what we're hearing from the White House. I was the bad-news
messenger. But I could see that the body-count war was being lost. It
wasn't possible to kill your way to victory in Vietnam as in World War
II. History, culture, the evolution of the Communist Party under Ho Chi
Minh --- things were different from the situation in Europe.

That breaking point in the ``body count'' that President Lyndon Johnson
wanted, and that Gen. William Westmoreland assured him was close, was a
fiction, because those in charge in Hanoi adjusted the level of fighting
to suit them, and their young men kept coming because they knew what
they were fighting for. President Nguyen van Thieu of South Vietnam
never promoted his best officers because he was afraid of a military
coup unless he kept his friends in charge of the Army. And the
nation-building program, sometimes called pacification, wasn't gaining
enough ground to make any difference.

There was no Saigon government in large parts of the country, and it
wasn't the fault of people who lived there that South Vietnamese
generals had never been able to govern. Elections were sometimes held,
but they were rigged public relations efforts for the benefit of
Washington. The rural population in the South was like a second country,
and those people mostly supported Ho Chi Minh. They had since after
World War II, but especially since Ho beat the French Army in 1954. That
was their war of independence, but Washington misunderstood the
consequences of Ho's victory and decided to reverse it.

Serious journalists today carry with them an awareness of this history
--- not necessarily the specifics, but the way the media's perception of
the war matured --- when they start work. They've read David
Halberstam's book ``The Best and the Brightest,'' describing the
shallow, arrogant views of government officials who didn't think they
needed to know that much about Vietnam, and who were too busy to see the
country except as a chunk of geography on the map they wanted to
control. Reporters today also know how presidents and their advisers,
wary about the next election at home, were afraid of losing some piece
of the world to ``the Communists'' and how paranoia served to perpetuate
the fighting when the war had already been lost.

They've absorbed the message of Daniel Ellsberg and the Pentagon Papers,
revealed through the courageous reporting of Neil Sheehan of The New
York Times. They've read the classic books by the historian Bernard Fall
on the French and American wars in Indochina. They start off determined
to bring insights from the recent past to their reporting about
international situations and understand that American elected officials
are generally ignorant about other countries.

The Vietnam War has ended up putting an unusual burden on young
reporters, their newspapers and TV outlets. Too much stress is placed on
reporting the latest incremental turn in a story, especially regarding
America's current and brewing conflicts abroad. The news media's
business model depends on it. But to serve the public well, the industry
needs the time and space to permit reporters to include a larger context
in their reporting from abroad. Otherwise, news becomes ``Here they
come, there they go'' linear reporting about military action that
doesn't have the meaning it needs to permit Americans to understand
where these new conflicts are taking us.

For me, in my old age, the war comes back, out of order, my subconscious
offering up memories without my asking. Here's one: that time climbing
through a hillside bamboo thicket during a military operation when the
heat felt so intense it got inside me, and yet I couldn't seem to drink
enough water and was vomiting. I thought I was going to die. Here's
another: The troop-carrying choppers settle onto the rice paddy, mud a
foot deep, the men move out, tense because this is a Vietcong area, and
they burn the village down, even though the enemy isn't shooting at
them, ignoring the screaming women and children who are trying to keep
the fire from taking everything as they run to haul water from their
wells in buckets, throwing it up on their burning thatch.

In 1995, 20 years after the end of the war, I was asked to do an
anniversary report. I chose to go to Beallsville, Ohio, a small town
that had too many casualties. One of the veterans I went to see, a
Marine, came to the door but didn't invite me in or shake my hand. I
explained what I was doing, a film for PBS about his experience and the
town's loss. He listened, but his expression was hostile. I said I had
been in Vietnam myself as a TV reporter for five years over a decade. I
tried to think how I could get beyond this awkward moment because
sometimes by sharing experience, you can help people realize that you're
with them, not against them. But I couldn't overcome his hostility so
many years after the war was supposed to be over. I thanked him and
left.

I was probably a reminder of all the bad news that had come out of
Vietnam, especially on TV. The news was always too brief and too blunt
--- the ambush, the mine, the wounded, the medevac helicopter trying to
come in to get the men out. Then cut to a commercial. His family might
have written to him about what they were seeing and asked, is it like
that? It was clear that he resented me, maybe hated what he thought was
journalistic misrepresentation of the war as he knew it. Or maybe he
knew the bad news was correct but didn't want to deal with that either.

There was often resentment among ground troops about ``hit and run''
reporting --- getting news and pictures of the latest firefight by
riding helicopters in and out of the battlefield the same day. That
style of reporting was a response to pressure from editors back home to
get the story out as quickly as possible, to beat the competition, the
other wire services, the other networks. Better reporters stuck around
for a while, spent the night, shared the combat food rations, the heat
and rain, made friends before leaving. So the grunts felt you weren't
just using them, like a stage set to entertain the viewers back home
with a little bang-bang, as we used to call it.

When the Vietnam War was over, the Pentagon decided that there had been
too much negative reporting from the war zone and that it would limit
journalists' access to battlefields. Handouts about what had happened
would be prepared by public affairs staff, and officers, called
``minders,'' would go along with correspondents to supervise their
movements and the information they got. A colleague told me that in
Iraq, he wasn't permitted to speak with local people. It's still
possible to find out what's actually going on under these restrictive
conditions, but it's harder and takes more time. It's also a lot more
dangerous now, because wars have become more violent.

Good journalism is always hard to do, but there's a new generation of
reporters who take nothing for granted because of what they know about
Vietnam. Their work is everywhere in the best daily newspapers, on cable
news and in online newsletters, blogs and websites. Of course journalism
is populated by an assortment of people. There's no entrance exam, so a
lot of reporting is done by people who are ignorant and inexperienced
about the subjects they pretend to know. Journalism is no better or
worse than any other American institution. But the best young reporters
have learned from the Vietnam War to question authority and find out for
themselves what's really going on. And that's how it's supposed to work
in a democracy.

Advertisement

\protect\hyperlink{after-bottom}{Continue reading the main story}

\hypertarget{site-index}{%
\subsection{Site Index}\label{site-index}}

\hypertarget{site-information-navigation}{%
\subsection{Site Information
Navigation}\label{site-information-navigation}}

\begin{itemize}
\tightlist
\item
  \href{https://help.nytimes.com/hc/en-us/articles/115014792127-Copyright-notice}{©~2020~The
  New York Times Company}
\end{itemize}

\begin{itemize}
\tightlist
\item
  \href{https://www.nytco.com/}{NYTCo}
\item
  \href{https://help.nytimes.com/hc/en-us/articles/115015385887-Contact-Us}{Contact
  Us}
\item
  \href{https://www.nytco.com/careers/}{Work with us}
\item
  \href{https://nytmediakit.com/}{Advertise}
\item
  \href{http://www.tbrandstudio.com/}{T Brand Studio}
\item
  \href{https://www.nytimes.com/privacy/cookie-policy\#how-do-i-manage-trackers}{Your
  Ad Choices}
\item
  \href{https://www.nytimes.com/privacy}{Privacy}
\item
  \href{https://help.nytimes.com/hc/en-us/articles/115014893428-Terms-of-service}{Terms
  of Service}
\item
  \href{https://help.nytimes.com/hc/en-us/articles/115014893968-Terms-of-sale}{Terms
  of Sale}
\item
  \href{https://spiderbites.nytimes.com}{Site Map}
\item
  \href{https://help.nytimes.com/hc/en-us}{Help}
\item
  \href{https://www.nytimes.com/subscription?campaignId=37WXW}{Subscriptions}
\end{itemize}
