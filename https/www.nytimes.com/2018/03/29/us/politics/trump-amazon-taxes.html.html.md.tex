Sections

SEARCH

\protect\hyperlink{site-content}{Skip to
content}\protect\hyperlink{site-index}{Skip to site index}

\href{https://www.nytimes.com/section/politics}{Politics}

\href{https://myaccount.nytimes.com/auth/login?response_type=cookie\&client_id=vi}{}

\href{https://www.nytimes.com/section/todayspaper}{Today's Paper}

\href{/section/politics}{Politics}\textbar{}Trump Attacks Amazon, Saying
It Does Not Pay Enough Taxes

\url{https://nyti.ms/2GgqUBP}

\begin{itemize}
\item
\item
\item
\item
\item
\item
\end{itemize}

Advertisement

\protect\hyperlink{after-top}{Continue reading the main story}

Supported by

\protect\hyperlink{after-sponsor}{Continue reading the main story}

\hypertarget{trump-attacks-amazon-saying-it-does-not-pay-enough-taxes}{%
\section{Trump Attacks Amazon, Saying It Does Not Pay Enough
Taxes}\label{trump-attacks-amazon-saying-it-does-not-pay-enough-taxes}}

\includegraphics{https://static01.nyt.com/images/2018/03/30/world/30DC-Amazon/merlin_135448092_122042a7-3cc7-45d4-8004-2ea67a12cac9-articleLarge.jpg?quality=75\&auto=webp\&disable=upscale}

By \href{https://www.nytimes.com/by/michael-d-shear}{Michael D. Shear},
\href{https://www.nytimes.com/by/nick-wingfield}{Nick Wingfield} and
\href{https://www.nytimes.com/by/cecilia-kang}{Cecilia Kang}

\begin{itemize}
\item
  March 29, 2018
\item
  \begin{itemize}
  \item
  \item
  \item
  \item
  \item
  \item
  \end{itemize}
\end{itemize}

WASHINGTON --- President Trump escalated his attack on Amazon on
Thursday, saying in an early-morning tweet that the online retail
behemoth does not pay enough taxes --- and strongly suggesting that he
may use the power of his office to rein in the nation's largest
e-commerce business.

Mr. Trump accused Amazon, one of the country's most recognizable and
successful brands, of putting thousands of local retailers out of
business and said the company was using the United States Postal Service
as its ``Delivery Boy.''

The president has lashed out publicly against the giant company and its
chief executive, Jeff Bezos, on Twitter more than a dozen times since
2015. And privately, people close to him said, Mr. Trump repeatedly
brings up his disdain for the company, often set off by his anger at
negative stories in The Washington Post, which is owned by Mr. Bezos.

By focusing on the threat to small businesses, Mr. Trump has touched on
the unease of Amazon's disruptive force. The company has changed
industries as varied as publishing, groceries and health care. That has
helped the company grow to be worth more than \$700 billion, but it has
also made it a convenient target.

The president has little clear authority to take action against the
company. Some Amazon critics have suggested antitrust actions against
the company, but the moves would need to come from the independent
Federal Trade Commission or the Justice Department, where officials have
promised to keep politics out of its corporate competition cases.

Lindsay Walters, a White House spokeswoman, told reporters on Thursday
that ``the president has expressed his concerns with Amazon. We have no
actions at this time.''

Still, Mr. Trump's willingness to again single out Amazon and
characterize it as a tax cheat and a job killer is a remarkable use of
the presidential bully pulpit that could have serious implications for
the company even without any formal moves by the federal government.

His comments have already spooked investors, sending Amazon stocks
tumbling Wednesday after
\href{https://www.axios.com/newsletters/axios-am-53f09706-9f2f-4dd9-b8dc-480285206e3f.html?chunk=0\#story0}{an
article on the website Axios} about his anger toward the company. The
stock fell further after his tweet on Thursday, though prices rebounded
later.

The tech industry is increasingly on the defensive. Facebook is under
attack in Washington for its handling of personal data and the social
network's role in the 2016 election. And the recent death of a
pedestrian by a self-driving car has renewed criticism of Uber.

``The general principle that I know deeply concerns the president is
that we need to live in a world where the government sets a level
playing field between internet vendors and mom and pop stores,'' Kevin
Hassett, the chair of the president's council on economic advisers, said
Thursday on Fox Business.

Drew Herdener, an Amazon spokesman, declined to comment.

Several current and former officials said that Mr. Trump regularly
conflates Amazon with The Post. Mr. Bezos owns the paper privately,
separate from his role at Amazon.

Brad Parscale, the president's 2020 campaign manager, on Thursday
channeled Mr. Trump's anger about The Post in a tweet of his own,
saying: ``Do not forget to mention that @amazon has probably 10x the
data on every American that @facebook does. All that data and own a
political newspaper, The @washingtonpost. Hmm\ldots{}''

And far-right, conspiracy-fanning websites --- some of which Mr. Trump
is known to read --- have for months stoked the idea that The Post is in
cahoots with the C.I.A. because the agency contracts with Amazon to
provide cloud-based data storage. Last November, a headline at one of
those sites, Infowars, said: ``BEZOS \& DEEP STATE UNITE: AMAZON
LAUNCHES CLOUD SERVICE FOR CIA.''

The Post declined to comment.

One person close to Mr. Trump, who asked for anonymity to discuss
private discussions in the Oval Office, said that the president mused
about the issue of Amazon and taxes at least a half-dozen times in the
last six months. The president has repeatedly claimed that Amazon costs
the Postal Service money even though officials have explained to him
that is not the case. Amazon has said that the Postal Regulatory
Commission, which oversees the service, has consistently found that its
contracts with Amazon are profitable.

People close to Mr. Trump say the president is convinced that Amazon
unfairly benefits from tax laws. The internet giant collects sales taxes
on its own products in all 45 states that have one, but third-party
vendors who sell products on the site often do not collect sales tax, a
fact that rivals say is unfair. In addition,
\href{https://www.nytimes.com/2018/03/25/business/economy/amazon-tax.html}{some
municipalities complain} that the company does not collect local taxes.

In April, the Supreme Court will hear arguments in an case on whether to
allow states to impose sales taxes on all internet sales. Many observers
believe the justices are poised to reverse its 1992 ruling that exempted
online retailers with no physical presence in a state.

Mr. Trump hears the sales tax complaints frequently during visits with
his wealthy friends at his Mar-a-Lago estate in Florida, and in meetings
at the White House, several aides and associates said. One person
familiar with his thinking said that the president believes many of his
core supporters are hurt when Amazon disrupts the local businesses where
they live.

The president's critique is shared by some of his usual adversaries in
the Democratic Party. Like many of Mr. Trump's wealthy Republican
friends, Representative Keith Ellison, Democrat of Minnesota, has been
critical of Amazon's growing power in the marketplace.

``The Trump administration should rein in giants like Amazon because
they have an unfair stranglehold on the competition, not because the
president has a personal feud with a company's C.E.O.,'' Mr. Ellison
said in a statement on Thursday.

There was also some irony in the criticism coming from Mr. Trump, who
has boasted about his dexterity in avoiding paying taxes. ``This is the
guy who said that not paying taxes `makes me smart,''' said Matt
Gardner, senior fellow at the Institute on Taxation and Economic Policy,
a nonpartisan research organization.

How the president could slow Amazon, beyond take to Twitter or complain
about the company in speeches, is unclear.

If Mr. Trump were to pressure the Justice Department to pursue antitrust
enforcement action against Amazon, it would be a sharp break from
tradition, experts say, because the White House has kept a far distance
from those cases for decades.

``It would be a gross violation and abuse of our due process,'' said
Diana Moss, president of the American Antitrust Institute.

But the president has repeatedly crossed lines that his predecessors
have observed. In the investigation into Russian meddling in the 2016
election, he has pressured Justice Department officials to investigate
his Democratic rivals and to abandon the investigation of himself and
his associates.

AT\&T and Time Warner have suggested Mr. Trump played a role in the
Justice Department's decision to block the companies' \$85.4 billion
media merger. Makan Delrahim, the head of antitrust at the Justice
Department, has said under oath in Congress that he would not allow the
White House to impact his decisions over that merger or any other cases.

Mr. Trump also surprised antitrust experts with his swift decision to
block Broadcom's merger with Qualcomm earlier this month, after a review
by a government panel on foreign investments. The president has direct
authority to block a merger with foreign companies that pose national
security concerns.

Amazon has tried, through lobbyists and outside consultants, to meet
with administration officials and members of Congress to counter
arguments about the company's tax collection and its relationship with
the Postal Service.

The size of company's lobbying staff has ballooned in recent years,
according to public filings, focusing largely on drones, transportation,
taxes and cybersecurity. It has hired antitrust consultants over the
past year.

In 2015, shortly after Mr. Trump started his attacks against the
company, Mr. Bezos joked on Twitter about sending the candidate into
space on a rocket made by Blue Origin, a space exploration start-up Mr.
Bezos owns.

But when Mr. Trump became the Republican nominee and then president, Mr.
Bezos and Amazon stopped responding to his attacks, seeing little upside
in a public quarrel with him, according to two people briefed on the
decision. The people, who spoke on condition of anonymity because the
conversations were private, said that these days, Mr. Trump's tweets
were more likely to prompt eye-rolling inside the company than any
serious effort at crisis control.

Advertisement

\protect\hyperlink{after-bottom}{Continue reading the main story}

\hypertarget{site-index}{%
\subsection{Site Index}\label{site-index}}

\hypertarget{site-information-navigation}{%
\subsection{Site Information
Navigation}\label{site-information-navigation}}

\begin{itemize}
\tightlist
\item
  \href{https://help.nytimes.com/hc/en-us/articles/115014792127-Copyright-notice}{©~2020~The
  New York Times Company}
\end{itemize}

\begin{itemize}
\tightlist
\item
  \href{https://www.nytco.com/}{NYTCo}
\item
  \href{https://help.nytimes.com/hc/en-us/articles/115015385887-Contact-Us}{Contact
  Us}
\item
  \href{https://www.nytco.com/careers/}{Work with us}
\item
  \href{https://nytmediakit.com/}{Advertise}
\item
  \href{http://www.tbrandstudio.com/}{T Brand Studio}
\item
  \href{https://www.nytimes.com/privacy/cookie-policy\#how-do-i-manage-trackers}{Your
  Ad Choices}
\item
  \href{https://www.nytimes.com/privacy}{Privacy}
\item
  \href{https://help.nytimes.com/hc/en-us/articles/115014893428-Terms-of-service}{Terms
  of Service}
\item
  \href{https://help.nytimes.com/hc/en-us/articles/115014893968-Terms-of-sale}{Terms
  of Sale}
\item
  \href{https://spiderbites.nytimes.com}{Site Map}
\item
  \href{https://help.nytimes.com/hc/en-us}{Help}
\item
  \href{https://www.nytimes.com/subscription?campaignId=37WXW}{Subscriptions}
\end{itemize}
