Sections

SEARCH

\protect\hyperlink{site-content}{Skip to
content}\protect\hyperlink{site-index}{Skip to site index}

\href{https://www.nytimes.com/section/technology}{Technology}

\href{https://myaccount.nytimes.com/auth/login?response_type=cookie\&client_id=vi}{}

\href{https://www.nytimes.com/section/todayspaper}{Today's Paper}

\href{/section/technology}{Technology}\textbar{}It Built an Empire of
GIFs, Buzzy News and Jokes. China Isn't Amused.

\url{https://nyti.ms/2GUKJLm}

\begin{itemize}
\item
\item
\item
\item
\item
\end{itemize}

Advertisement

\protect\hyperlink{after-top}{Continue reading the main story}

Supported by

\protect\hyperlink{after-sponsor}{Continue reading the main story}

\hypertarget{it-built-an-empire-of-gifs-buzzy-news-and-jokes-china-isnt-amused}{%
\section{It Built an Empire of GIFs, Buzzy News and Jokes. China Isn't
Amused.}\label{it-built-an-empire-of-gifs-buzzy-news-and-jokes-china-isnt-amused}}

\includegraphics{https://static01.nyt.com/images/2018/04/12/world/12chinacensor-1/merlin_136683807_faa3ffb7-ecde-4f4e-906a-20dec96d521c-articleLarge.jpg?quality=75\&auto=webp\&disable=upscale}

By \href{https://www.nytimes.com/by/raymond-zhong}{Raymond Zhong}

\begin{itemize}
\item
  April 11, 2018
\item
  \begin{itemize}
  \item
  \item
  \item
  \item
  \item
  \end{itemize}
\end{itemize}

\href{https://cn.nytimes.com/technology/20180412/china-toutiao-bytedance-censor/}{阅读简体中文版}\href{https://cn.nytimes.com/technology/20180412/china-toutiao-bytedance-censor/zh-hant/}{閱讀繁體中文版}

BEIJING --- A Chinese start-up that appears to have mastered the art of
keeping people glued to their smartphones also has a knack for something
else: drawing the ire of China's censors.

The country's top media regulator on Tuesday ordered the company,
Bytedance, to shut down its app for sharing jokes and silly videos.
Vulgar content on the Neihan Duanzi app had ``caused strong dislike
among internet users,'' a
\href{http://www.sapprft.gov.cn/sapprft/contents/6582/365922.shtml}{brief
notice} from the State Administration of Radio and Television said. The
company was told to clean up its other platforms, too.

The shutdown was only the latest blow for Bytedance, one of the world's
most successful technology start-ups. Just a day earlier, its flagship
app, a
\href{https://www.nytimes.com/2018/01/02/business/china-toutiao-censorship.html}{popular
news aggregator called Jinri Toutiao}, was pulled from app stores for
unspecified reasons.

And last week, Huoshan, the company's platform for sharing slice-of-life
video clips, vanished from app stores after China's official television
broadcaster rapped it for
\href{https://www.nytimes.com/2018/04/06/technology/china-censor-teen-moms.html}{glorifying
underage pregnancy}.

In a statement posted Wednesday morning, Zhang Yiming, Bytedance's
founder and chief executive, said he had spent the previous, sleepless
night in deep reflection, gnawed by ``a guilty conscience.''

``Content had appeared that did not accord with core socialist values
and was not a good guide for public opinion,'' Mr. Zhang wrote. ``Over
the past few years, we put more effort and resources toward expanding
the business, and did not take enough measures to supervise our
platform.''

He added that Bytedance would expand its team for monitoring content to
10,000 people from 6,000 currently.

The company's travails show how the government in Beijing has broadened
its restrictions on what people see and say on the internet. Regulators
are increasingly suppressing content that they deem pornographic or in
poor taste, and not merely material that touches on politically
sensitive topics such as regime change or personal freedoms.

The authorities are also scrambling to keep up as a new wave of Chinese
apps, many of them built around short, spontaneously recorded video
clips or live streams, helps people communicate and express themselves
in new and hard-to-supervise ways.

Bytedance --- which investors valued at more than \$30 billion recently,
putting it more in the financial league of Airbnb or SpaceX than of
Buzzfeed or Vice --- has assembled a confederation of these buzzy new
apps. And it has made no secret of its desire to dominate phone screens
across the rest of the world, too.

The company says it uses artificial intelligence technology to figure
out what users like, then makes sure they are fed more and more of it.
Read a few articles on the trade spat between the United States and
China, and soon your Toutiao feed will be populated with news on
international relations. Watch a bunch of stand-up comedy shows, and
before long the app will suggest new comics who might appeal.

Bytedance has spent top dollar hiring engineers and software experts to
fine-tune its recommendation technology.

``It's like having a chef in your house who knows what kind of food you
like,'' said Xu Qinglu, a 22-year-old student and Toutiao user in
Beijing.

``I think the app is not harmful,'' she added. ``The people who use it
should be responsible for their own behavior.''

At an \href{http://36kr.com/p/5125610.html}{event in Beijing last
month}, Mr. Zhang said he hoped that more than half of the company's
users would come from outside China within the next three years. At the
moment, he said, one in 10 of its users was overseas.

First, though, the company needs to continue thriving in China.
Bytedance's detractors say that salty, unwholesome material --- the sort
that has the Chinese government on edge these days --- is exactly what
the company's apps have specialized in, and is a major reason for its
popularity.

``Will a cleaned-up Toutiao still have an edge?'' said Neil Arora, an
American investor who previously worked in venture capital in Beijing.

``Toutiao's strong team, refined algorithms and locked-in users may help
it adapt,'' said Mr. Arora, who is not a Bytedance shareholder.
``However, the bigger danger is that all news apps may lose out, with
users pulling away from sanitized news feeds for entertainment
elsewhere.''

Hans Tung of GGV Capital, a venture firm that operates in both China and
the United States and is a Bytedance shareholder, said he was confident
the company would continue to add more types of material --- not just
the lowbrow kind --- to its platforms. ``The Toutiao we see today is not
the Toutiao it will be five years from now,'' he said.

``It's better to go through this rodeo a few times,'' Mr. Tung said of
the latest rebuke from regulators. This way, he said, the company will
be motivated to move more quickly in courting users who want
higher-minded stuff.

Toutiao aside, three other popular news apps --- including one run by
Tencent, the giant Chinese conglomerate --- were also taken down from
stores this week.

Another fast-growing video app, Kuaishou, was removed last week
alongside Huoshan, and also for featuring videos made by teenage
mothers. In response, Kuaishou's parent company said it would increase
the size of its content-monitoring team to 5,000 from 2,000.

A \href{https://www.lagou.com/jobs/4377331.html}{posting from Kuaishou
on one hiring website} last week says the company is looking for people
with bachelor's degrees or higher. Candidates with ``good political
awareness'' and ``strong political sensitivity and discernment'' are
preferred. Being a member of the Communist Party or Communist Youth
League is also a plus, the listing says.

Duanzi, Bytedance's now-shuttered humor app, trafficked in dirty jokes,
goofy comedy sketches and well-worn but persistent gender stereotypes.
One post that appeared on the app before it was closed down declared
that the way to know that a man won't cheat on his wife is to place a
beautiful woman before him --- but the way to test a woman's fidelity is
to try seducing her with a lot of money.

Another post, unprintable in a family newspaper, was a ribald joke
involving a seller of fried dough sticks, his wife and an irate
customer.

Even Bytedance's news app, Toutiao, featured plenty of edgy material
that kept users coming back, sometimes reluctantly, for more. Xiao Lin,
a 29-year-old programmer in Beijing, called the app ``spiritual opium.''

``On a typical night, I would keep clicking on news items the app
recommended to me while telling myself, `After this, I will sleep,' ''
Mr. Xiao said. ``But I ended up reading more and more, for hours. I
couldn't stop.''

He deleted the app in January.

Advertisement

\protect\hyperlink{after-bottom}{Continue reading the main story}

\hypertarget{site-index}{%
\subsection{Site Index}\label{site-index}}

\hypertarget{site-information-navigation}{%
\subsection{Site Information
Navigation}\label{site-information-navigation}}

\begin{itemize}
\tightlist
\item
  \href{https://help.nytimes.com/hc/en-us/articles/115014792127-Copyright-notice}{©~2020~The
  New York Times Company}
\end{itemize}

\begin{itemize}
\tightlist
\item
  \href{https://www.nytco.com/}{NYTCo}
\item
  \href{https://help.nytimes.com/hc/en-us/articles/115015385887-Contact-Us}{Contact
  Us}
\item
  \href{https://www.nytco.com/careers/}{Work with us}
\item
  \href{https://nytmediakit.com/}{Advertise}
\item
  \href{http://www.tbrandstudio.com/}{T Brand Studio}
\item
  \href{https://www.nytimes.com/privacy/cookie-policy\#how-do-i-manage-trackers}{Your
  Ad Choices}
\item
  \href{https://www.nytimes.com/privacy}{Privacy}
\item
  \href{https://help.nytimes.com/hc/en-us/articles/115014893428-Terms-of-service}{Terms
  of Service}
\item
  \href{https://help.nytimes.com/hc/en-us/articles/115014893968-Terms-of-sale}{Terms
  of Sale}
\item
  \href{https://spiderbites.nytimes.com}{Site Map}
\item
  \href{https://help.nytimes.com/hc/en-us}{Help}
\item
  \href{https://www.nytimes.com/subscription?campaignId=37WXW}{Subscriptions}
\end{itemize}
