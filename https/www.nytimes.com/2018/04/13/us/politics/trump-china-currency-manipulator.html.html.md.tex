Sections

SEARCH

\protect\hyperlink{site-content}{Skip to
content}\protect\hyperlink{site-index}{Skip to site index}

\href{https://www.nytimes.com/section/politics}{Politics}

\href{https://myaccount.nytimes.com/auth/login?response_type=cookie\&client_id=vi}{}

\href{https://www.nytimes.com/section/todayspaper}{Today's Paper}

\href{/section/politics}{Politics}\textbar{}Trump Declines to Label
China a Currency Manipulator as Trade War Brews

\url{https://nyti.ms/2JKQixN}

\begin{itemize}
\item
\item
\item
\item
\item
\end{itemize}

Advertisement

\protect\hyperlink{after-top}{Continue reading the main story}

Supported by

\protect\hyperlink{after-sponsor}{Continue reading the main story}

\hypertarget{trump-declines-to-label-china-a-currency-manipulator-as-trade-war-brews}{%
\section{Trump Declines to Label China a Currency Manipulator as Trade
War
Brews}\label{trump-declines-to-label-china-a-currency-manipulator-as-trade-war-brews}}

\includegraphics{https://static01.nyt.com/images/2018/04/14/business/14dc-currency-1/14dc-currency-1-articleLarge.jpg?quality=75\&auto=webp\&disable=upscale}

By \href{https://www.nytimes.com/by/alan-rappeport}{Alan Rappeport} and
\href{https://www.nytimes.com/by/ana-swanson}{Ana Swanson}

\begin{itemize}
\item
  April 13, 2018
\item
  \begin{itemize}
  \item
  \item
  \item
  \item
  \item
  \end{itemize}
\end{itemize}

WASHINGTON --- The Trump administration,
\href{https://www.nytimes.com/2018/04/05/business/trump-trade-war-china.html}{which
has been on the verge of a trade war with China}, opted on Friday not to
label the country a currency manipulator, breaking a key campaign
promise by President Trump to punish a government he has called the
``greatest currency manipulators ever.''

The Treasury Department,
\href{https://home.treasury.gov/news/press-releases/sm0348}{in its
biannual currency exchange report}, scolded China for its lack of
progress in reducing the bilateral trade deficit with the United States,
but did not find that it was improperly devaluing its currency, known as
the renminbi.

``Treasury is strongly concerned by the lack of progress by China in
correcting the bilateral trade imbalance and urges China to create a
more level and reciprocal playing field for American workers and
firms,'' the report said.

It was the third time since Mr. Trump assumed the presidency that the
Treasury Department opted not to accuse China of improper meddling.
China has long maintained a strong grip on the value of its currency
and, for years, weakened it compared with the dollar to make Chinese
products cheaper to sell in the United States and other countries. More
recently, China has made a big show of gradually loosening its grip, an
effort meant to mollify
\href{https://www.nytimes.com/2017/04/14/business/china-currency-manipulation-trump.html}{critics
like Mr. Trump} and experts who have long urged Beijing to let markets
fix financial problems in the world's second-largest economy.

China had a \$375 billion trade surplus in goods last year, the largest
of any of America's trading partners. That gap has become a frequent
target of Mr. Trump's,
\href{https://www.nytimes.com/2018/02/06/us/politics/us-china-trade-deficit.html}{who
has cited the trade deficit with China} as a main reason for his
administration's aggressive approach, including the tariffs and
investment restrictions he has threatened.

But Mr. Trump may have been persuaded not to label China a currency
manipulator by business executives, who have warned the president that
such a move could be disastrous for American companies.

In his first months in office, Mr. Trump was surprised by the opposition
to his plan to label China a currency manipulator, always a popular line
on the campaign trail. In a meeting with business leaders last year, an
executive of one of America's largest exporters told the president that
labeling China a currency manipulator could be harmful for his business,
a person briefed on the discussion said.

The president responded incredulously, and was surprised when the other
executives in attendance said they completely agreed, the person said.

The Treasury Department determines if a country should be labeled a
currency manipulator based on bilateral trade deficits and signs that
another country is depressing the value of its currency. The United
States has not officially called another country a manipulator since it
slapped the label on China in 1994, and doing so is supposed to
kick-start negotiations to resolve the problem.

This year China, Germany, Japan, South Korea, Switzerland and India were
placed on Treasury's monitoring list for potential currency
manipulation.

Corporate chiefs and investors are hoping that the U.S. and China can
negotiate a settlement before the tariffs go into effect. But that looks
highly uncertain. The countries are not currently engaging in formal
negotiations, and the United States has not presented China with a list
of actions it could take to avoid the tariff threats.

Instead, the Trump administration appears to be pushing ahead on
tariffs, as well as restrictions on investment that are expected be
announced in the coming months. The United States trade representative
is aiming to publish a list of Chinese goods as early as next week that
would incur additional tariffs if Mr. Trump makes good on his most
recent threat to tax an additional \$100 billion in Chinese goods.

The White House has already begun imposing tariffs on Chinese steel and
aluminum and has outlined another \$50 billion worth of imports that
would be subject to tariffs, including flat-screen TVs. The Chinese, in
return, have begun imposing tariffs on American pork and threatened
levies on additional products, primarily agricultural.

Advertisement

\protect\hyperlink{after-bottom}{Continue reading the main story}

\hypertarget{site-index}{%
\subsection{Site Index}\label{site-index}}

\hypertarget{site-information-navigation}{%
\subsection{Site Information
Navigation}\label{site-information-navigation}}

\begin{itemize}
\tightlist
\item
  \href{https://help.nytimes.com/hc/en-us/articles/115014792127-Copyright-notice}{©~2020~The
  New York Times Company}
\end{itemize}

\begin{itemize}
\tightlist
\item
  \href{https://www.nytco.com/}{NYTCo}
\item
  \href{https://help.nytimes.com/hc/en-us/articles/115015385887-Contact-Us}{Contact
  Us}
\item
  \href{https://www.nytco.com/careers/}{Work with us}
\item
  \href{https://nytmediakit.com/}{Advertise}
\item
  \href{http://www.tbrandstudio.com/}{T Brand Studio}
\item
  \href{https://www.nytimes.com/privacy/cookie-policy\#how-do-i-manage-trackers}{Your
  Ad Choices}
\item
  \href{https://www.nytimes.com/privacy}{Privacy}
\item
  \href{https://help.nytimes.com/hc/en-us/articles/115014893428-Terms-of-service}{Terms
  of Service}
\item
  \href{https://help.nytimes.com/hc/en-us/articles/115014893968-Terms-of-sale}{Terms
  of Sale}
\item
  \href{https://spiderbites.nytimes.com}{Site Map}
\item
  \href{https://help.nytimes.com/hc/en-us}{Help}
\item
  \href{https://www.nytimes.com/subscription?campaignId=37WXW}{Subscriptions}
\end{itemize}
