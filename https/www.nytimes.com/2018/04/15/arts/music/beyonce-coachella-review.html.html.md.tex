Sections

SEARCH

\protect\hyperlink{site-content}{Skip to
content}\protect\hyperlink{site-index}{Skip to site index}

\href{https://www.nytimes.com/section/arts/music}{Music}

\href{https://myaccount.nytimes.com/auth/login?response_type=cookie\&client_id=vi}{}

\href{https://www.nytimes.com/section/todayspaper}{Today's Paper}

\href{/section/arts/music}{Music}\textbar{}Review: Beyoncé Is Bigger
Than Coachella

\url{https://nyti.ms/2JMdPhF}

\begin{itemize}
\item
\item
\item
\item
\item
\end{itemize}

Advertisement

\protect\hyperlink{after-top}{Continue reading the main story}

Supported by

\protect\hyperlink{after-sponsor}{Continue reading the main story}

\hypertarget{review-beyoncuxe9-is-bigger-than-coachella}{%
\section{Review: Beyoncé Is Bigger Than
Coachella}\label{review-beyoncuxe9-is-bigger-than-coachella}}

\includegraphics{https://static01.nyt.com/images/2018/04/17/arts/16COACHELLA-BEYONCE/merlin_136896633_c2bee7b7-60fc-4dac-80f0-a51ed50ba0cd-articleLarge.jpg?quality=75\&auto=webp\&disable=upscale}

\begin{itemize}
\tightlist
\item
  Beyoncé at Coachella Valley Music and Arts Festival\\
  **NYT Critic's Pick
\end{itemize}

By \href{http://www.nytimes.com/by/jon-caramanica}{Jon Caramanica}

\begin{itemize}
\item
  April 15, 2018
\item
  \begin{itemize}
  \item
  \item
  \item
  \item
  \item
  \end{itemize}
\end{itemize}

INDIO, Calif. --- Let's just cut to the chase: There's not likely to be
a more meaningful, absorbing, forceful and radical performance by an
American musician this year, or any year soon, than
\href{https://www.nytimes.com/topic/person/beyonce-knowles}{Beyoncé}'s
headlining set at the Coachella Valley Music and Arts Festival on
Saturday night.

It was rich with history, potently political and visually grand. By
turns uproarious, rowdy, and lush. A gobsmacking marvel of choreography
and musical direction.

And not unimportantly, it obliterated the ideology of the relaxed
festival, the idea that musicians exist to perform in service of a
greater vibe. That is one of the more tragic side effects of the
\href{https://www.nytimes.com/2016/03/19/arts/music/summer-music-festivals.html}{spread
of festival culture over the last two decades}. Beyoncé was having none
of it. The Coachella main stage, on the grounds of the Empire Polo Club
here, was her platform, yes, but her show was in countless ways a
rebuke.

{[} \href{https://www.nytimes.com/newsletters/louder}{\emph{Never miss a
pop music story: Sign up for our weekly newsletter, Louder.}} {]}

It started with the horns: trumpets, trombones, sousaphones. For most of
the night, the 36-year-old star was backed by an ecstatic marching band,
in the manner of
\href{https://www.nytimes.com/2017/01/29/arts/music/honda-battle-of-the-bands.html}{historically
black college football halftime shows}. The choice instantly reoriented
her music, sidelining its connections to pop and framing it squarely in
a lineage of Southern black musical traditions from New Orleans second
line marches to Houston's chopped-and-screwed hip-hop.

Her arrangements were alive with shifts between styles and oodles of
small details, quick musical quotations of songs (Pastor Troy's ``No Mo'
Play in G.A.,'' anyone?) that favored alertness and engagement. As
always, one of the key thrills of a Beyoncé performance is her
willingness to dismantle and rearrange her most familiar hits. ``Drunk
in Love'' began as bass-thick molasses, then erupted into trumpet
confetti. ``Bow Down'' reverberated with nervy techno.
\href{https://www.nytimes.com/2016/02/07/arts/music/beyonce-formation-super-bowl-video.html}{``Formation,''}
already a rapturous march, was a savage low-end stomp here. And during a
brief trip through the Caribbean part of her catalog, she remade ``Baby
Boy'' as startling Jamaican big band jazz.

She does macro, too --- she was joined onstage by approximately 100
dancers, singers and musicians, a stunning tableau that included
fraternity pledges and drumlines and rows of female violinists in
addition to the usual crackerjack backup dancers (which here included
bone breakers and also dancers performing elaborate routines with
cymbals).

Some superstars prize effortlessness, but Beyoncé shows her work --- the
cameras captured the force and determination in her dancing, and also
her sweat. She performed for almost two hours, with only a few breaks,
and her voice rarely flagged. Occasionally her set was punctuated with
fireworks that, compared with what was happening onstage, seemed dull.

Beyoncé was originally meant to perform at Coachella last year, but
\href{https://www.nytimes.com/2017/02/23/arts/music/beyonce-coachella-cancel-pregnant.html}{rescheduled
for this April} after becoming pregnant; her Coachella performances this
weekend and next are her only solo U.S. dates this year. ``Thank you for
allowing me to be the first black woman to headline Coachella,'' she
said midset, then added an aside that was, in fact, the main point:
``Ain't that 'bout a bitch.''

Big-tent festivals, generally speaking, are blithe spaces --- they don't
invite much scrutiny, because they can't stand up to it. But Beyoncé's
simple recitation of fact was searing, especially on the same night
that, in Cleveland, the
\href{https://www.nytimes.com/2017/12/13/arts/music/rock-roll-hall-fame-bon-jovi-nina-simone-cars.html}{Rock
\& Roll Hall of Fame finally inducted} Nina Simone and Sister Rosetta
Tharpe, 15 and 45 years after their deaths, and also Bon Jovi, a band in
which everyone is very much alive.

She was arguing not in defense of herself, but of her forebears. And her
performance was as much ancestral tribute and cultural continuum --- an
uplifting of black womanhood --- as contemporary concert. She sang
``Lift Every Voice and Sing,'' often referred to as the black national
anthem, incorporated vocal snippets of Malcolm X and Chimamanda Ngozi
Adichie, and nodded at Ms. Simone's ``Lilac Wine.''

And she rendered her personal history as well. During the second half of
the show, she unfurled a kind of This Is Your Life in reverse. First
came her husband, Jay-Z, on ``Déjà Vu'' --- with him, she was
affectionate while easily outshining him. Then, a true surprise: a
reunion with her former Destiny's Child groupmates Kelly Rowland and
Michelle Williams, during which she happily ceded the main spotlight.
After that came a playful dance routine with her sister, Solange, on
``Get Me Bodied.'' (Sadly, there was no ``Ring Off'' with her mother,
nor a rendition of ``Daddy Lessons'' with her father.)

As Beyoncé has gotten older, she's been making music that's increasingly
visceral, both emotionally and historically. She is one of the only
working pop stars who need not preoccupy herself with prevailing trends,
or the work of her peers. She is an institution now, and that has
allowed her freedom.
\href{https://www.nytimes.com/2016/04/25/arts/music/beyonce-lemonade.html}{``Lemonade''
is her most accomplished album}, and also a wild and risky one ---
thematically but also musically.

That may be one reason that last year, Beyoncé lost the Grammy for album
of the year to Adele, the sort of upset that
\href{https://www.nytimes.com/2017/02/13/arts/music/grammys-adele-beyonce-black-artists-race.html}{triggered
a storm of criticism about the Grammys' relevance}, and, effectively, an
almost-apology from Adele. In time, though, that moment will feel like a
glitch. That space on the mantel will be filled by a National Medal of
the Arts, or a Presidential Medal of Freedom. Like no other musician of
her generation apart from Kanye West, Beyoncé is performing musicology
in real time. It is bigger than any tribute she might receive. History
is her stage.

Advertisement

\protect\hyperlink{after-bottom}{Continue reading the main story}

\hypertarget{site-index}{%
\subsection{Site Index}\label{site-index}}

\hypertarget{site-information-navigation}{%
\subsection{Site Information
Navigation}\label{site-information-navigation}}

\begin{itemize}
\tightlist
\item
  \href{https://help.nytimes.com/hc/en-us/articles/115014792127-Copyright-notice}{©~2020~The
  New York Times Company}
\end{itemize}

\begin{itemize}
\tightlist
\item
  \href{https://www.nytco.com/}{NYTCo}
\item
  \href{https://help.nytimes.com/hc/en-us/articles/115015385887-Contact-Us}{Contact
  Us}
\item
  \href{https://www.nytco.com/careers/}{Work with us}
\item
  \href{https://nytmediakit.com/}{Advertise}
\item
  \href{http://www.tbrandstudio.com/}{T Brand Studio}
\item
  \href{https://www.nytimes.com/privacy/cookie-policy\#how-do-i-manage-trackers}{Your
  Ad Choices}
\item
  \href{https://www.nytimes.com/privacy}{Privacy}
\item
  \href{https://help.nytimes.com/hc/en-us/articles/115014893428-Terms-of-service}{Terms
  of Service}
\item
  \href{https://help.nytimes.com/hc/en-us/articles/115014893968-Terms-of-sale}{Terms
  of Sale}
\item
  \href{https://spiderbites.nytimes.com}{Site Map}
\item
  \href{https://help.nytimes.com/hc/en-us}{Help}
\item
  \href{https://www.nytimes.com/subscription?campaignId=37WXW}{Subscriptions}
\end{itemize}
