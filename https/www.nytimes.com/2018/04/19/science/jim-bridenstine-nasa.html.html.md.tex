Sections

SEARCH

\protect\hyperlink{site-content}{Skip to
content}\protect\hyperlink{site-index}{Skip to site index}

\href{https://www.nytimes.com/section/science}{Science}

\href{https://myaccount.nytimes.com/auth/login?response_type=cookie\&client_id=vi}{}

\href{https://www.nytimes.com/section/todayspaper}{Today's Paper}

\href{/section/science}{Science}\textbar{}Trump's NASA Nominee, Jim
Bridenstine, Confirmed by Senate on Party-Line Vote

\url{https://nyti.ms/2K0LBQb}

\begin{itemize}
\item
\item
\item
\item
\item
\end{itemize}

Advertisement

\protect\hyperlink{after-top}{Continue reading the main story}

Supported by

\protect\hyperlink{after-sponsor}{Continue reading the main story}

\hypertarget{trumps-nasa-nominee-jim-bridenstine-confirmed-by-senate-on-party-line-vote}{%
\section{Trump's NASA Nominee, Jim Bridenstine, Confirmed by Senate on
Party-Line
Vote}\label{trumps-nasa-nominee-jim-bridenstine-confirmed-by-senate-on-party-line-vote}}

\includegraphics{https://static01.nyt.com/images/2018/04/20/science/20NASA/20NASA-articleLarge.jpg?quality=75\&auto=webp\&disable=upscale}

By \href{http://www.nytimes.com/by/kenneth-chang}{Kenneth Chang}

\begin{itemize}
\item
  April 19, 2018
\item
  \begin{itemize}
  \item
  \item
  \item
  \item
  \item
  \end{itemize}
\end{itemize}

Seven and a half months after being nominated to lead NASA, Jim
Bridenstine finally gets to start his new job. His confirmation
following a vote in the Senate ends the longest span of time that NASA
has operated without a permanent leader, and comes with a vivid reminder
that few posts in Washington are now spared from partisan conflict.

On Thursday, the Senate confirmed Mr. Bridenstine, an Oklahoma
congressman, as the new NASA administrator in a stark partisan vote: 50
Republicans voting for him and 47 Democrats plus two independents
against. The vote lasted more than 45 minutes as Republicans waited for
Senator Jeff Flake of Arizona to cast his lot. The vote was also
punctuated by the
\href{https://www.nytimes.com/2018/04/19/us/politics/baby-duckworth-senate-floor.html}{appearance
of Senator Tammy Duckworth}, Democrat of Illinois, who cast her ``no''
vote on the Senate floor
\href{https://twitter.com/LissandraVilla/status/987037761805930496}{with
her newborn daughter in hand}.

\begin{quote}
Sen. Tammy Duckworth's baby makes her big debut on the Senate floor.

Duckworth is the first sitting senator to give birth while in office,
and the Senate unanimously passed a rule change to allow her to nurse
her newborn on Senate floor. \url{https://t.co/mQPhab19Jk}
\href{https://t.co/pDqc2GiiMh}{pic.twitter.com/pDqc2GiiMh}

--- World News Tonight (@ABCWorldNews)
\href{https://twitter.com/ABCWorldNews/status/987049043779096576?ref_src=twsrc\%5Etfw}{April
19, 2018}
\end{quote}

Many who voted against him expressed concerns about his record of
partisanship as well as some statements questioning climate change, an
area of research in which the space agency plays a central role.

Mr. Bridenstine takes over an agency in transition. While President
Obama talked of sending astronauts to Mars in a couple of decades, the
Trump administration has instead focused on a nearer, quicker goal:
\href{https://www.nytimes.com/2017/12/11/science/trump-moon-space-directive.html}{to
return to the moon}. The administration has also
\href{https://www.nytimes.com/2018/02/11/science/nasa-budget-moon.html}{proposed
getting NASA out of the business of running the International Space
Station} and instead spur commercial alternatives that do not yet exist.

Critics have questioned whether the agency's new administrator is up to
the task.

Mr. Bridenstine, a former Navy pilot who is now in his third term in the
House of Representatives, has become immersed in space issues. In 2016,
he sponsored a bill called the
\href{http://spacerenaissanceact.com/}{American Space Renaissance Act},
which proposed broad, ambitious goals for the nation's space program,
including directing NASA to devise a 20-year plan. Although it did not
reach a vote, some of the ideas were incorporated into other
legislation.

But Democratic senators, led by Bill Nelson of Florida, opposed Mr.
Bridenstine for several reasons. For one, they said Mr. Bridenstine was
too political --- he would be the first elected official to serve as
NASA administrator. During the confirmation hearings in November, Mr.
Nelson read back Mr. Bridenstine's disparaging remarks about other
politicians, even other Republicans.

``I think what's not right for NASA,'' Mr. Nelson said during a speech
on the Senate floor on Wednesday, ``is an administrator who is
politically divisive and who is not prepared to be the last in line to
make that fateful decision on `go' or `no go' for launch.''

Mr. Bridenstine also has no experience running a large government
bureaucracy. The expectation was that someone with that type of
background would be tapped to be NASA's deputy administrator to handle
more of the day-to-day management. However, the Trump administration has
yet to nominate anyone for that position.

On Wednesday, the Project on Government Oversight, an independent
watchdog organization,
\href{https://www.thedailybeast.com/how-trumps-nasa-nominee-used-a-nonprofit-he-ran-to-benefit-himself}{raised
questions about Mr. Bridenstine's actions as executive director of the
Tulsa Air and Space Museum and Planetarium} from 2008 to 2010, before he
ran for Congress.

One of the events that he organized as executive was an air show in 2010
featuring races by rocket-powered airplanes --- by a business he had
personally invested in. That could be considered ``self-dealing,'' where
a nonprofit official directs money from the organization toward a
commercial venture the official has a stake in.

Nick Schwellenbach, director of investigations at the watchdog
organization, said Mr. Bridenstine's actions at the museum raise
concerns given that the administrator oversees an agency with a \$20
billion budget and more than 18,000 employees.

``Someone in that position needs to set a strong ethical tone, from the
top, about the proper use of taxpayer dollars,'' Mr. Schwellenbach said
in an interview.

``This is an old story, no there there,'' said Sheryl Kaufman, Mr.
Bridenstine's communications director. ``The accusations have been fully
refuted by members of the Board of Directors of the museum both in 2012
and again in September 2017.''

Opponents also painted Mr. Bridenstine as a denier of climate change. In
a 2013 speech in the House of Representatives, Mr. Bridenstine sharply
criticized President Obama, saying his administration was spending too
much money on the issue and that Mr. Obama should apologize.

Mr. Bridenstine has since moderated his public views, saying he supports
NASA research into the causes of extreme weather.

During his confirmation hearing, he agreed that human activity
``absolutely'' contributed to climate change, but sparred with Senator
Brian Schatz, Democrat of Hawaii, over whether it was ``a contributor''
or the ``primary cause.''

Some opponents also cite Mr. Bridenstine's conservative social views
like opposition to same-sex marriage. ``I stand squarely in support of
traditional marriage,'' Mr. Bridenstine
\href{https://bridenstine.house.gov/news/documentsingle.aspx?DocumentID=46}{wrote
on his congressional website} in July 2013.

In his confirmation hearing, Mr. Bridenstine tried to make a distinction
between views he espoused as a politician and how he would act as the
manager of a large federal agency. ``I want to make sure that NASA
remains, as you said, apolitical,'' Mr. Bridenstine said to Mr. Nelson.

The previous administrator,
\href{http://www.nytimes.com/2010/02/26/science/space/26nasa.html}{Charles
F. Bolden Jr.}, stepped down on Jan. 20, 2017, the first day of the
Trump presidency. Since then, a longtime NASA official, Robert Lightfoot
Jr., has been filling in. By the time Mr. Bridenstine was officially
nominated, on Sept. 5, Mr. Lightfoot had already served 228 days, the
longest span for an acting administrator.

Mr. Lightfoot then served for another 226 days, until Thursday.

Mr. Bridenstine's nomination languished, because although the
Republicans hold a 51-49 majority, he did not appear to have the
necessary votes for confirmation. Senator John McCain, Republican of
Arizona, was away for cancer treatment, while Marco Rubio, the other
Florida senator, a Republican, also expressed reservations about putting
a politician at the top of NASA.

Then, last month,
\href{https://www.nytimes.com/2018/03/12/science/robert-lightfoot-nasa-retirement.html}{Mr.
Lightfoot announced plans to retire at the end of April}. That appears
to have led Mr. Rubio to reluctantly change his mind.

Still, Mr. Bridenstine's confirmation was accompanied by last-minute
drama.

On Wednesday, a preliminary vote to limit debate on his nomination
unexpectedly deadlocked at 49-49 when Mr. Flake voted against it. (Two
senators, Ms. Duckworth and Mr. McCain, were absent.) Vice President
Mike Pence, who could have broken a 49-49 tie, was also out of town.

Mr. Flake then changed his vote to yes. He did not offer a detailed
explanation.

Other than the confirmation hearing, Mr. Bridenstine has spent much of
the last seven months keeping quiet. He largely stopped making any
public statements and voting on bills to avoid conflicts of interest.

He attended
\href{https://www.nytimes.com/2017/10/05/science/national-space-council-moon-pence.html}{the
first meeting of the National Space Council meeting}, a panel revived by
the Trump administration to coordinate space issues between various
federal agencies, but did not speak or participate.

And during Mr. Trump's State of the Union address in January,
\href{https://bridenstine.house.gov/news/documentsingle.aspx?DocumentID=868}{he
brought a guest}: Bill Nye ``the Science Guy.''

Advertisement

\protect\hyperlink{after-bottom}{Continue reading the main story}

\hypertarget{site-index}{%
\subsection{Site Index}\label{site-index}}

\hypertarget{site-information-navigation}{%
\subsection{Site Information
Navigation}\label{site-information-navigation}}

\begin{itemize}
\tightlist
\item
  \href{https://help.nytimes.com/hc/en-us/articles/115014792127-Copyright-notice}{©~2020~The
  New York Times Company}
\end{itemize}

\begin{itemize}
\tightlist
\item
  \href{https://www.nytco.com/}{NYTCo}
\item
  \href{https://help.nytimes.com/hc/en-us/articles/115015385887-Contact-Us}{Contact
  Us}
\item
  \href{https://www.nytco.com/careers/}{Work with us}
\item
  \href{https://nytmediakit.com/}{Advertise}
\item
  \href{http://www.tbrandstudio.com/}{T Brand Studio}
\item
  \href{https://www.nytimes.com/privacy/cookie-policy\#how-do-i-manage-trackers}{Your
  Ad Choices}
\item
  \href{https://www.nytimes.com/privacy}{Privacy}
\item
  \href{https://help.nytimes.com/hc/en-us/articles/115014893428-Terms-of-service}{Terms
  of Service}
\item
  \href{https://help.nytimes.com/hc/en-us/articles/115014893968-Terms-of-sale}{Terms
  of Sale}
\item
  \href{https://spiderbites.nytimes.com}{Site Map}
\item
  \href{https://help.nytimes.com/hc/en-us}{Help}
\item
  \href{https://www.nytimes.com/subscription?campaignId=37WXW}{Subscriptions}
\end{itemize}
