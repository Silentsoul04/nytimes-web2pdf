Sections

SEARCH

\protect\hyperlink{site-content}{Skip to
content}\protect\hyperlink{site-index}{Skip to site index}

\href{https://myaccount.nytimes.com/auth/login?response_type=cookie\&client_id=vi}{}

\href{https://www.nytimes.com/section/todayspaper}{Today's Paper}

\href{/section/opinion}{Opinion}\textbar{}The Future of Frats

\href{https://nyti.ms/2vVxbOG}{https://nyti.ms/2vVxbOG}

\begin{itemize}
\item
\item
\item
\item
\item
\end{itemize}

Advertisement

\protect\hyperlink{after-top}{Continue reading the main story}

Supported by

\protect\hyperlink{after-sponsor}{Continue reading the main story}

\href{/section/opinion}{Opinion}

\href{/column/on-campus}{On Campus}

\hypertarget{the-future-of-frats}{%
\section{The Future of Frats}\label{the-future-of-frats}}

By Kiley Roache

Ms. Roache is a sorority member at Stanford.

\begin{itemize}
\item
  April 26, 2018
\item
  \begin{itemize}
  \item
  \item
  \item
  \item
  \item
  \end{itemize}
\end{itemize}

\includegraphics{https://static01.nyt.com/images/2018/04/26/opinion/26roache-web/26roache-web-articleLarge.jpg?quality=75\&auto=webp\&disable=upscale}

``Our way of life is under attack,'' the keynote speaker said. The women
filling the ballroom, many of them wearing a Greek letter lapel pin or a
monogrammed sweater, shouted in agreement. It was the 2015 national
convention for my sorority, and college administrators and national
pundits alike had been deliberating the elimination of Greek life.

Debates about the future of fraternities and sororities have only gotten
more intense since then. One part of this discussion is about campus
chapters guilty of hazing, sexual assault or intolerance. Most recently,
a video emerged of fraternity members at Syracuse University making
hateful vows containing racist slurs. There is no debate about what to
do with a group that embraces hate like that: It should be kicked off
campus, immediately (the Syracuse fraternity, Theta Tau,
\href{http://dailyorange.com/2018/04/theta-tau-chapter-permanently-expelled-su/}{was
permanently banished} this month).

But there is also a wider conversation about why fraternities and
sororities continue to exist at all and what role they have at a modern
university. Shouldn't we just ban them all?

I don't think we have to. I am a firm believer that living in a house
with your friends, playing beer-drinking games and dancing to overplayed
pop songs are not fundamentally incompatible with inclusion, respect and
a just society. And on campuses where Greek life remains,
\href{https://www.washingtonpost.com/news/grade-point/wp/2015/01/20/despite-scandals-and-bad-ink-more-and-more-students-want-to-go-greek/?utm_term=.5d2b669d68ec}{it
is increasingly popular}, according to the North-American
Interfraternity Conference.

But Greek life needs to change. In its current form, it fosters not just
fun and friendship but also inequality. At a time when many dorms have
gender-mixed floors, and a full generation after most single-sex schools
began admitting both sexes, these organizations seem like relics.
Fraternities and sororities must make a number of changes to ensure
their survival, starting with going coed.

Those who defend Greek life talk about how it fosters a sense of
community and belonging at a time when many people are far from home and
often unsure of their direction. People meet lifelong friends, and
sometimes spouses, through the Greek system, and after college, alumni
associations can provide networks in new cities. This may explain why
people remain fierce advocates of their ``way of life'' long after
graduation.

Studies have shown that fraternity and sorority members also surpass
their non-Greek counterparts on a number of metrics. They are more
likely to
\href{https://www.sciencedirect.com/science/article/pii/S2214804314000147?via\%3Dihub}{graduate
on time} and go on to
\href{https://papers.ssrn.com/sol3/papers.cfm?abstract_id=2763720}{earn
higher salaries} (though joining a fraternity does seem to have a slight
negative effect on grades). In addition to all that they provide their
members, Greek organizations also typically raise money for charity and
do volunteer work. What's not to love?

Well, a lot. Those who support Greek life often have a blind spot. The
system strictly enforces gender separation and traditional gender roles.
Rules that prevent sorority sisters from hosting parties, drinking
alcohol or having members of the opposite sex in their houses result in
a disproportionate amount of social capital concentrated in
male-dominated spaces. While not all fraternities, and certainly not all
fraternity men, abuse this social power, some do. This means that women
may be excluded from a major social space on campus if they do not
adhere to the expectations of fraternity men. This hasn't just made it
easier to exclude women; it has made it easier to objectify them, too.

I would prefer no Greek life to Greek life that continues to marginalize
women and other groups. The purpose of higher education is to prepare
the next generation to be engaged and democratic citizens. This is why
most universities, even those that are private, receive public funding.
And Title IX prohibits schools that get such funding from discriminating
on the basis of sex. The organizations on these campuses should be held
to the same standard.

Despite the perception that critics of Greek life are attacking a ``way
of life,'' it is quite possible to preserve what is good about the
organizations --- fun with friends and a community based on common
values --- while dispelling the bad. Implicit bias training could reduce
the discrimination based on race and religion that has for years
influenced determinations of which prospective members were the right
``fit,'' and providing financial aid for the expensive dues could bring
in students who would otherwise find membership unaffordable.

And sororities and fraternities should go coed. Greek organizations
would not be the first single-sex college group to become coeducational.
The eating clubs at Princeton were
\href{https://www.nytimes.com/1991/01/28/nyregion/princeton-women-forge-a-new-era.html}{forced
to go coed} in the '90s and more recently,
\href{http://www.thecrimson.com/article/2017/5/25/coping-with-coed/}{some
Harvard final clubs} have become open to all genders. This year at Yale,
the Whiffenpoofs, an a cappella group,
\href{https://www.nytimes.com/2018/02/20/nyregion/yale-whiffenpoofs-first-woman.html}{admitted
the first female member} in its 109-year history.

But this change is not just about keeping up with the times. It is about
resolving a division that disservices all students. Involvement in Greek
life is about building deep friendships, and there is no reason those
friendships should be limited by gender.

Of course, there will be resistance, but this is not just about frats
and sororities surviving in an increasingly diverse and open world; it's
about them living up to what they were meant to be in the first place.
These communities should be based on common values and passions, not on
common privileges of sexual orientation, race, religion or gender.
Organizations chartered to uphold values of character, leadership and
sincere friendship should not object to such a cause; in fact, they
should be at its forefront.

Advertisement

\protect\hyperlink{after-bottom}{Continue reading the main story}

\hypertarget{site-index}{%
\subsection{Site Index}\label{site-index}}

\hypertarget{site-information-navigation}{%
\subsection{Site Information
Navigation}\label{site-information-navigation}}

\begin{itemize}
\tightlist
\item
  \href{https://help.nytimes.com/hc/en-us/articles/115014792127-Copyright-notice}{©~2020~The
  New York Times Company}
\end{itemize}

\begin{itemize}
\tightlist
\item
  \href{https://www.nytco.com/}{NYTCo}
\item
  \href{https://help.nytimes.com/hc/en-us/articles/115015385887-Contact-Us}{Contact
  Us}
\item
  \href{https://www.nytco.com/careers/}{Work with us}
\item
  \href{https://nytmediakit.com/}{Advertise}
\item
  \href{http://www.tbrandstudio.com/}{T Brand Studio}
\item
  \href{https://www.nytimes.com/privacy/cookie-policy\#how-do-i-manage-trackers}{Your
  Ad Choices}
\item
  \href{https://www.nytimes.com/privacy}{Privacy}
\item
  \href{https://help.nytimes.com/hc/en-us/articles/115014893428-Terms-of-service}{Terms
  of Service}
\item
  \href{https://help.nytimes.com/hc/en-us/articles/115014893968-Terms-of-sale}{Terms
  of Sale}
\item
  \href{https://spiderbites.nytimes.com}{Site Map}
\item
  \href{https://help.nytimes.com/hc/en-us}{Help}
\item
  \href{https://www.nytimes.com/subscription?campaignId=37WXW}{Subscriptions}
\end{itemize}
