Sections

SEARCH

\protect\hyperlink{site-content}{Skip to
content}\protect\hyperlink{site-index}{Skip to site index}

\href{https://myaccount.nytimes.com/auth/login?response_type=cookie\&client_id=vi}{}

\href{https://www.nytimes.com/section/todayspaper}{Today's Paper}

\href{/section/opinion}{Opinion}\textbar{}Sweet Briar College Almost
Closed. What Will It Take to Thrive?

\href{https://nyti.ms/2ItMOP4}{https://nyti.ms/2ItMOP4}

\begin{itemize}
\item
\item
\item
\item
\item
\end{itemize}

Advertisement

\protect\hyperlink{after-top}{Continue reading the main story}

Supported by

\protect\hyperlink{after-sponsor}{Continue reading the main story}

\href{/section/opinion}{Opinion}

\href{/column/on-campus}{On Campus}

\hypertarget{sweet-briar-college-almost-closed-what-will-it-take-to-thrive}{%
\section{Sweet Briar College Almost Closed. What Will It Take to
Thrive?}\label{sweet-briar-college-almost-closed-what-will-it-take-to-thrive}}

By Marguerite Joutz

Ms. Joutz is an assistant in the Opinion Department.

\begin{itemize}
\item
  May 17, 2018
\item
  \begin{itemize}
  \item
  \item
  \item
  \item
  \item
  \end{itemize}
\end{itemize}

\includegraphics{https://static01.nyt.com/images/2018/05/17/opinion/17oncampus2web/17oncampus2web-articleLarge.jpg?quality=75\&auto=webp\&disable=upscale}

Three years ago, Sweet Briar, a women's college in rural Virginia,
announced that it would close after more than a century of operation.
Enrollment had declined to fewer than 700 students, and the
administration cited ``insurmountable financial challenges.''

The decision was met with an uproar. Students, faculty members and
alumnae organized amid a national conversation on the future of women's
education. After a legal battle over the closing, the college reversed
its decision.

Today, the student body now numbers about 300, and the impact of that
fight lingers. ``It's an uphill battle to let the word out that we never
really did close, and that Sweet Briar is thriving,'' Meredith Woo, the
college's president, told me.

In 2017, Dr. Woo, a former dean of the University of Virginia, left her
job in London as director of the international higher education support
program at the Open Society Foundations to lead Sweet Briar. Her first
year has not been without controversy: A restructuring of academic
departments prompted complaints, and a commencement speaker
\href{https://www.insidehighered.com/news/2018/05/14/anger-sweet-briar-over-commencement-speech}{made
comments on the Me Too movement} that angered some students. .

I asked Dr. Woo to talk about her new role and plans for Sweet Briar, an
institution at the nexus of women's education, the liberal arts and
international education. This interview has been edited and condensed.

\textbf{Marguerite Joutz:} You've been president for just over a year at
Sweet Briar. What have you been able to do in your first year, and what
is your larger vision for the college?

\textbf{Meredith Woo}: Well, Sweet Briar is a very unusual and
interesting place. For over a century, it produced women who are
leaders, women who are strong, persuasive, and roll up their sleeves to
get things done. And you got some sense of it when three years ago, the
alumnae got together and made a meaningful transformation in the life of
the college. They wrote a very interesting chapter in the history of
higher education in this country.

What we have been able to do in the past year was work with the faculty,
students and other stakeholders to reimagine what a distinctive liberal
arts education at Sweet Briar could look like. That involved an
establishment of what we call an integrated core curriculum under the
theme of leadership.

\textbf{M.J.:} Many people are familiar with the idea of a core
curriculum, but what does it mean to have a core curriculum that's
oriented around not just leadership, but women's leadership?

\textbf{M.W.:} At some level, all colleges profess to inculcate the
qualities of leadership into their students. So in some sense, there is
nothing unusual about leadership as the ultimate purpose of the liberal
arts education we practice and deliver. We decided to really dive in and
explore it with the faculty. And what we decided in the end was that the
qualities of leadership we want in our students align nicely with the
qualities we want to get out of our students through liberal arts
education --- the habits of the mind and other habits that we want to
see in leaders.

So we looked at these things and asked ourselves, ``Is there a way, in a
very parsimonious and powerful way, to deliver on all of these things
through an integrated core curriculum?''

On top of that, we added a couple of different components that we feel
are important for women going forward in the 21st century. One is the
emphasis on what we loosely call financial literacy. We believe that
young women ought to understand the basics of accounting and marketing.
The other aspect that we emphasized in our core is a scientific
vocabulary with which young women can understand great issues that face
our society today, whether it be climate change or the social
consequences of artificial intelligence. We believe that this habit of
understanding science as citizens is very important.

\textbf{M.J.:} Could you tell us about your work in international higher
education at the Open Society Foundations, specifically with the Asian
University for Women in Bangladesh?

Image

Meredith Woo, president of Sweet Briar College.Credit...Parker
Michels-Boyce for The New York Times

\textbf{M.W.:} In my time directing the international higher education
program, we focused on providing higher education for refugees. Higher
education for refugees may sound like a luxury, but that's really not
true. In the year 2000, about 20 percent in the relevant age group
around the world were enrolled in college. By 2014, that jumps up to
about 35 percent, almost the same as in the United States.

Now it doesn't mean that you are sitting in a beautiful classroom such
as one you'd find at Sweet Briar. It might mean that people are getting
their exams delivered on donkeys' backs or learning through TV. But the
fact of the matter is that people are enrolled in higher education. If
you look at a country like Syria, which was a secular nation before the
civil war with a very highly literate population, education becomes
critically important for the refugees who have lost everything. The only
thing they've got is what they have in their brains. Education can
situate them for the better in countries, which are not their home. So
we worked with Syrian refugees to help them get education in bordering
countries such as Turkey, Lebanon and Jordan.

The other refugee group that we worked with was the Rohingyas. They're
often considered the most persecuted people in a world where there is a
lot of persecution going on. The largest group of Rohingyas living as
refugees happens to be in Bangladesh. We believe deeply that the
Rohingya people need leadership. And that they need to be educated, so
that when the time comes, when they are freed from persecution, there
will be a critical mass of people that are educated and empowered. That
can lead the nation forward. And we thought that it was a good idea to
educate the women first.

There is an old African proverb. that if you educate men, you educate
individuals. But if you educate women, you educate the nation. And the
reason, obviously, is that because women will educate their families,
and eventually spread the education throughout the society.

I remember delivering these women to the Asian University for Women in
Chittagong, Bangladesh. The women were shy, shellshocked and in some
ways frightened, although full of hope for the future. Very soon, they
just thrived. For them to speak English, to learn how to debate and
speak in public, and learn all the wonderful things that you learn in a
liberal arts college, was so empowering. The next time I saw them, they
were even in your face a little bit --- very assertive and confident.
And I thought, my God, this is the kind of thing that happens perhaps in
an all-women environment, but may be difficult in a coed environment for
these women.

\textbf{M.J.:} You went from a global perch to one narrowly focused on a
particular type of education. What prompted you to return to Virginia
and assume the presidency at Sweet Briar?

\textbf{M.W.:} The decision to take the job at Sweet Briar was an odd
one since I have been a product of --- at least in terms of teaching and
management --- big research universities. I was working at the Open
Society Foundations on a two-year leave from the University of Virginia.
After two years, I had to fish or cut bait. And at that moment, I got a
call from Sweet Briar College.

I knew that it was an excellent school that produces remarkable women.
Sweet Briar is also an interesting part of American cultural history.
The architect Ralph Adams Cram designed the campus. When you look at the
original design of the campus, you can't help but think that Sweet Briar
was born as a great college. The ambition was to turn it into a
significant American liberal arts college. It is unthinkable that one
would relegate a wonderful place like this to the dustbin of history.

\textbf{M.J.:} There's been a lot of discussion related to the business
model of higher education. And one of the reasons cited for the near
closing of Sweet Briar was its financial situation. What are your
thoughts on the financial model for small liberal arts colleges?

\textbf{M.W.:} Well, we will thrive on the basis of great excellence and
distinction. Our ``tuition reset'' {[}undergraduate tuition, fees and
room and board was lowered to \$34,000 for 2018-2019, from \$50,000 the
year before{]} is designed to create that distinction and communicate
that at Sweet Briar what you get is an education that's excellent,
relevant, but also affordable. And what we are signaling by doing that
in a very dramatic way is to say: A truly excellent and relevant
education at a private institution is as affordable as that for in-state
students in flagship universities.

\textbf{M.J.:} Sweet Briar is one of two women's colleges in the country
to offer programs in engineering. Could you tell me a little bit about
Sweet Briar's engineering program? What is it like to study engineering
at a women's college?

\textbf{M.W.:} We are, along with Smith College, the only women's
college that has an engineering program accredited by ABET {[}the
Accreditation Board for Engineering and Technology{]}. I think it is
deplorable that in the dominant industries of tomorrow, there is such
low representation of women. What we do is offer a truly excellent
engineering education for women engineers of tomorrow in classrooms
devoid of misogyny and intimidation. And when they graduate as
engineers, they really are as good as any engineers being produced by
large state university engineering schools.

\textbf{M.J.:} What do you see as the future of women's education in the
United States?

\textbf{M.W.}: I think there's a real need and purpose to women's
education. There used to be 300 women's colleges, at a time when a lot
of prestigious universities were open only to men. But now, there are
less than 40 women's colleges. For that reason, I think that the
prospects are excellent.

Women's colleges are an American phenomenon, in part, because America
was one of the few nations where women could obtain higher education
fairly early on. Even if there were a lot of schools where they couldn't
go, they went to women's colleges. In the rest of the world, single-sex
or all women's colleges are actually a rarity.

In the U.S., and in the rest of the world, there is recognition that
single-gender colleges or all women's colleges might not be a bad idea,
especially at a time where all over the world --- not only in the United
States --- you have a predominance of women going to colleges. But you
don't have a predominance of women in boardrooms and in the dominant
industries of tomorrow. In that sense, there is recognition that there
is something wrong, that perhaps there ought to be more emphasis on
education that empowers women in a particular setting that's favorable
to them.

Advertisement

\protect\hyperlink{after-bottom}{Continue reading the main story}

\hypertarget{site-index}{%
\subsection{Site Index}\label{site-index}}

\hypertarget{site-information-navigation}{%
\subsection{Site Information
Navigation}\label{site-information-navigation}}

\begin{itemize}
\tightlist
\item
  \href{https://help.nytimes.com/hc/en-us/articles/115014792127-Copyright-notice}{©~2020~The
  New York Times Company}
\end{itemize}

\begin{itemize}
\tightlist
\item
  \href{https://www.nytco.com/}{NYTCo}
\item
  \href{https://help.nytimes.com/hc/en-us/articles/115015385887-Contact-Us}{Contact
  Us}
\item
  \href{https://www.nytco.com/careers/}{Work with us}
\item
  \href{https://nytmediakit.com/}{Advertise}
\item
  \href{http://www.tbrandstudio.com/}{T Brand Studio}
\item
  \href{https://www.nytimes.com/privacy/cookie-policy\#how-do-i-manage-trackers}{Your
  Ad Choices}
\item
  \href{https://www.nytimes.com/privacy}{Privacy}
\item
  \href{https://help.nytimes.com/hc/en-us/articles/115014893428-Terms-of-service}{Terms
  of Service}
\item
  \href{https://help.nytimes.com/hc/en-us/articles/115014893968-Terms-of-sale}{Terms
  of Sale}
\item
  \href{https://spiderbites.nytimes.com}{Site Map}
\item
  \href{https://help.nytimes.com/hc/en-us}{Help}
\item
  \href{https://www.nytimes.com/subscription?campaignId=37WXW}{Subscriptions}
\end{itemize}
