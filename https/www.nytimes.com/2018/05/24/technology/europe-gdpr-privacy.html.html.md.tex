Sections

SEARCH

\protect\hyperlink{site-content}{Skip to
content}\protect\hyperlink{site-index}{Skip to site index}

\href{https://www.nytimes.com/section/technology}{Technology}

\href{https://myaccount.nytimes.com/auth/login?response_type=cookie\&client_id=vi}{}

\href{https://www.nytimes.com/section/todayspaper}{Today's Paper}

\href{/section/technology}{Technology}\textbar{}G.D.P.R., a New Privacy
Law, Makes Europe World's Leading Tech Watchdog

\url{https://nyti.ms/2Lq0rAC}

\begin{itemize}
\item
\item
\item
\item
\item
\item
\end{itemize}

Advertisement

\protect\hyperlink{after-top}{Continue reading the main story}

Supported by

\protect\hyperlink{after-sponsor}{Continue reading the main story}

\hypertarget{gdpr-a-new-privacy-law-makes-europe-worlds-leading-tech-watchdog}{%
\section{G.D.P.R., a New Privacy Law, Makes Europe World's Leading Tech
Watchdog}\label{gdpr-a-new-privacy-law-makes-europe-worlds-leading-tech-watchdog}}

\includegraphics{https://static01.nyt.com/images/2018/05/25/world/25GDPR-1-print/25GDPR-1-articleLarge.jpg?quality=75\&auto=webp\&disable=upscale}

By \href{https://www.nytimes.com/by/adam-satariano}{Adam Satariano}

\begin{itemize}
\item
  May 24, 2018
\item
  \begin{itemize}
  \item
  \item
  \item
  \item
  \item
  \item
  \end{itemize}
\end{itemize}

LONDON --- The
\href{https://www.nytimes.com/2018/05/23/technology/personaltech/what-you-should-look-for-europe-data-law.html?rref=collection\%2Fcolumn\%2Ftech-fix\&action=click\&contentCollection=personaltech\&region=stream\&module=stream_unit\&version=latest\&contentPlacement=1\&pgtype=collection}{notices
are flooding people's inboxes} en masse, from large technology
companies, including Facebook and Uber, and even from parent teacher
associations, children's soccer clubs and yoga instructors. ``Here is an
update to our privacy policy,'' they say.

All are acting because the European Union on Friday enacts the
\href{https://www.nytimes.com/2020/05/22/business/facebook-privacy-law-grandmother.html}{world's
toughest rules to protect people's online data}. And with the internet's
borderless nature, the regulations are set to have an outsize impact far
beyond Europe.

In Silicon Valley, Google, Facebook and other tech companies have been
working for months to comply with the new rules, known as
\href{https://gdpr-info.eu/}{the General Data Protection Regulation}.
The law, which lets people request their online data and restricts how
businesses obtain and handle the information, has set off a panic among
small businesses and local organizations that have an internet presence.

Brazil, Japan and South Korea are set to follow Europe's lead, with some
having already passed similar data protection laws. European officials
are encouraging copycats by tying data protection to some trade deals
and arguing that a unified global approach is the only way to crimp
Silicon Valley's power.

``We want to achieve the same level of restrictions that you have in
Europe,'' said Luiz Fernando Martins Castro, a lawyer based in São Paulo
who advises the Brazilian government on internet policy. Mr. Castro said
Europe was ``pushing the matter and making people realize that we have
to go forward.''

Europe is determined to cement its role as the world's foremost tech
watchdog --- and the region is only getting started. Authorities in
Brussels and in the European Union's 28 member countries are also
setting the bar for stricter enforcement of antitrust laws against tech
behemoths and are paving the way for tougher tax policies on the
companies.

The region's proactive stance is a sharp divergence from the United
States, which has taken little action over the years in regulating the
tech industry. Most recently, the Trump administration has sought to cut
taxes and roll back regulation, while pursuing an increasingly
\href{https://www.nytimes.com/2018/03/23/technology/trump-china-tariffs-tech-cold-war.html}{protectionist
tack to shield tech companies} from competition from China.

``The E.U. is more advanced than the U.S. in protecting consumer
privacy, and what happens there could be a harbinger of the future,''
said Michael Kearns, a computer science professor at the University of
Pennsylvania, who has studied the data collection techniques of
companies including Facebook and Google.

Europe's new privacy measures,
\href{https://www.nytimes.com/2018/05/06/technology/gdpr-european-privacy-law.html}{called
G.D.P.R. for short}, let people reduce the trail of information left
when browsing social media, reading the news or shopping online.
Individuals will be able to request the data that companies hold on
them, and demand it be deleted.

{[}\emph{\href{https://www.nytimes.com/2018/05/06/technology/gdpr-european-privacy-law.html}{Read
more about what the new European privacy rules mean for you.}}{]}

Businesses must also more clearly detail how someone's data is being
handled, while clearing a higher bar to target advertising using
personal information. Companies face fines if they do not comply, with
tech giants risking penalties greater than \$1 billion. Privacy groups
preparing class action-style complaints under the new law may put even
more legal pressure on companies.

European authorities have actively encouraged other countries to adopt
similar laws to G.D.P.R. Officials have been dispatched around the world
to preach the tougher rules. Data protections are becoming part of trade
deals, with the region ready to limit access to its market of 500
million consumers if countries do not rise to meet Europe's standards.

``If we can export this to the world, I will be happy,'' said Vera
Jourova, the European commissioner in charge of consumer protection and
privacy who helped draft G.D.P.R. She said she planned to travel to
Japan and South Korea in the next few weeks for talks about data
protection. Regulating technology, she added, is a ``global challenge.''

\emph{{[}\href{https://www.nytimes.com/2018/05/23/technology/personaltech/what-you-should-look-for-europe-data-law.html}{Read
more about how to parse the flood of G.D.P.R.-related privacy notices in
your inbox.}{]}}

Europe's influence can be seen in Brazil, which has sought advice from
Brussels on its own privacy legislation. The
\href{https://uk.practicallaw.thomsonreuters.com/4-520-1732?transitionType=Default\&contextData=(sc.Default)\&firstPage=true\&bhcp=1}{bill}
closely mirrors Europe's new regulations, including a requirement to get
people's consent before collecting personal data and special protections
for information on political affiliation, religious beliefs, sexual
orientation or health.

Brazil has an incentive to draft tougher privacy laws: One provision of
G.D.P.R. limits the data that companies can transfer outside the
European Union unless that data goes to a country that meets Europe's
standards.

\includegraphics{https://static01.nyt.com/images/2018/05/25/business/25GDPR-2/merlin_137315310_1327e839-9891-4145-a92d-80bb5a8545e7-articleLarge.jpg?quality=75\&auto=webp\&disable=upscale}

``There is almost a reproduction of the European market in our bill,''
said Mr. Castro, a member of Brazil's
\href{https://www.cgi.br/about/}{internet steering committee}.

European officials have also been advising Brazilian authorities.
Giovanni Buttarelli, the European data protection supervisor, is set to
deliver a recorded video message at a policy event in Brazil next week.
And last month, a senior data protection official in the European
Commission testified before the
\href{https://www12.senado.leg.br/ecidadania/visualizacaoaudiencia?id=13368}{Brazilian
Senate committee} drafting the country's legislation.

``Many countries are interested in signing a trade agreement with the
European Union, and then privacy becomes an important precondition,''
said Mr. Buttarelli.

Europe's fingerprints can be seen elsewhere in the world, too. Japan
last year passed a data protection law creating a new independent online
privacy board, and Tokyo and Brussels are finalizing the details of a
data transfer deal. South Korea is considering new privacy rules, while
Israel has adopted updated requirements for disclosures of data breaches
--- both share elements with the European rules.

Europe's influence is not going unnoticed by America's tech giants,
which have long complained that Brussels unfairly focuses on them.

The new privacy rules are part of a ``strong European tradition'' of
policing industries to protect the environment or public health, even if
it does ``constrain business,'' said Margrethe Vestager, Europe's top
antitrust official.

To meet G.D.P.R.'s requirements, Facebook and Google have deployed large
teams to overhaul how they give users access to their own privacy
settings and to redesign certain products that may have sucked up too
much user data. Facebook said it had roughly 1,000 people working on the
initiative globally, including engineers, product managers and lawyers.

In Brussels, the Silicon Valley companies are fast adding lobbyists to
influence other European regulations before they spread. Google and
Microsoft are already among the five biggest spenders on lobbying in the
European Union, with budgets of about 4.5 million euros, or \$5.3
million, each,
\href{https://lobbyfacts.eu/articles/12-12-2016/google-one-brussels\%E2\%80\%99-most-active-lobbyists}{according
to LobbyFacts.eu}, which tracks such spending. Facebook, whose chief
executive,
\href{https://www.nytimes.com/2018/05/22/technology/facebook-eu-parliament-mark-zuckerberg.html}{Mark
Zuckerberg, was in Brussels} this week, doubled its lobbying budget last
year to roughly
\href{https://lobbyfacts.eu/representative/64755e0fc2a14e46aa9d8646df6f8f19/facebook-ireland-limited}{€2.5
million}, the watchdog site said.

Dean C. Garfield, president of the
\href{https://www.itic.org/about/}{Information Technology Industry
Council}, a Washington-based trade group representing Apple, Facebook,
Google and other companies, said his group was adding staff in Brussels
because Europe was ``driving and directing policy.''

``In the absence of another approach, it's easier for other markets to
follow what Europe has done,'' said Mr. Garfield.

On Thursday, a group of Democratic senators
\href{https://www.markey.senate.gov/news/press-releases/senator-markey-introduces-resolution-to-apply-european-privacy-protections-to-americans}{announced
a resolution} to match G.D.P.R., a sign of how United States policy may
change if control of Congress shifts in November.

Whether Europe's tough approach is actually crimping the global tech
giants is unclear. The region's regulators have hit American companies
with hefty fines over antitrust violations, the mishandling of user data
and the payment of taxes, but Amazon, Apple, Google and Facebook have
continued to grow and add customers.

Challenges remain over how G.D.P.R. will be enforced. National
regulators across Europe will be charged with policing the regulations,
but many have woefully fewer resources than the companies they will be
overseeing.

The
\href{https://www.nytimes.com/2018/05/16/technology/gdpr-helen-dixon.html}{data
protection office in Ireland}, for instance, where many tech giants have
their regional headquarters, has a budget of just €7.5 million, or \$8.8
million, but will be responsible for regulating some of the world's
biggest tech firms. That raises concerns that the companies will be able
to avoid tough penalties.

Even if Europe persuades other countries to adopt its policies, it will
be hard to ensure the laws work, said Omer Tene, a vice president at the
International Association of Privacy Professionals, a trade group that
tracks global privacy regulation.

``It's one thing to have rules on the books,'' said Mr. Tene. ``It's
quite another thing to implement these rules on the ground.''

\begin{quote}
\end{quote}

Advertisement

\protect\hyperlink{after-bottom}{Continue reading the main story}

\hypertarget{site-index}{%
\subsection{Site Index}\label{site-index}}

\hypertarget{site-information-navigation}{%
\subsection{Site Information
Navigation}\label{site-information-navigation}}

\begin{itemize}
\tightlist
\item
  \href{https://help.nytimes.com/hc/en-us/articles/115014792127-Copyright-notice}{©~2020~The
  New York Times Company}
\end{itemize}

\begin{itemize}
\tightlist
\item
  \href{https://www.nytco.com/}{NYTCo}
\item
  \href{https://help.nytimes.com/hc/en-us/articles/115015385887-Contact-Us}{Contact
  Us}
\item
  \href{https://www.nytco.com/careers/}{Work with us}
\item
  \href{https://nytmediakit.com/}{Advertise}
\item
  \href{http://www.tbrandstudio.com/}{T Brand Studio}
\item
  \href{https://www.nytimes.com/privacy/cookie-policy\#how-do-i-manage-trackers}{Your
  Ad Choices}
\item
  \href{https://www.nytimes.com/privacy}{Privacy}
\item
  \href{https://help.nytimes.com/hc/en-us/articles/115014893428-Terms-of-service}{Terms
  of Service}
\item
  \href{https://help.nytimes.com/hc/en-us/articles/115014893968-Terms-of-sale}{Terms
  of Sale}
\item
  \href{https://spiderbites.nytimes.com}{Site Map}
\item
  \href{https://help.nytimes.com/hc/en-us}{Help}
\item
  \href{https://www.nytimes.com/subscription?campaignId=37WXW}{Subscriptions}
\end{itemize}
