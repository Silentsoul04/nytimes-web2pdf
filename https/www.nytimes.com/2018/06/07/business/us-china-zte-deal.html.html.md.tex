Sections

SEARCH

\protect\hyperlink{site-content}{Skip to
content}\protect\hyperlink{site-index}{Skip to site index}

\href{https://www.nytimes.com/section/business}{Business}

\href{https://myaccount.nytimes.com/auth/login?response_type=cookie\&client_id=vi}{}

\href{https://www.nytimes.com/section/todayspaper}{Today's Paper}

\href{/section/business}{Business}\textbar{}Trump Strikes Deal to Save
China's ZTE as North Korea Meeting Looms

\url{https://nyti.ms/2JsrPfN}

\begin{itemize}
\item
\item
\item
\item
\item
\item
\end{itemize}

Advertisement

\protect\hyperlink{after-top}{Continue reading the main story}

Supported by

\protect\hyperlink{after-sponsor}{Continue reading the main story}

\hypertarget{trump-strikes-deal-to-save-chinas-zte-as-north-korea-meeting-looms}{%
\section{Trump Strikes Deal to Save China's ZTE as North Korea Meeting
Looms}\label{trump-strikes-deal-to-save-chinas-zte-as-north-korea-meeting-looms}}

\includegraphics{https://static01.nyt.com/images/2018/06/08/business/08zte-2-print/zte-1-articleLarge.jpg?quality=75\&auto=webp\&disable=upscale}

By \href{https://www.nytimes.com/by/ana-swanson}{Ana Swanson}

\begin{itemize}
\item
  June 7, 2018
\item
  \begin{itemize}
  \item
  \item
  \item
  \item
  \item
  \item
  \end{itemize}
\end{itemize}

\href{https://cn.nytimes.com/business/20180608/us-china-zte-deal/}{阅读简体中文版}\href{https://cn.nytimes.com/business/20180608/us-china-zte-deal/zh-hant/}{閱讀繁體中文版}

WASHINGTON --- President Trump handed the Chinese telecommunications
firm ZTE a lifeline on Thursday, agreeing to lift tough American
sanctions over the objections of Republican lawmakers, his defense
advisers and some of his own economic officials.

The deal will help defuse tensions with the Chinese president, Xi
Jinping, who personally asked Mr. Trump to intervene to save ZTE and
whom the president has relied on to help pave the way for next week's
summit meeting with the North Korean leader.

\href{https://www.commerce.gov/news/press-releases/2018/06/secretary-ross-announces-14-billion-zte-settlement-zte-board-management}{The
Commerce Department said} that ZTE had agreed to pay a \$1 billion fine,
replace its board and senior leadership, and allow the United States to
more closely inspect the company by effectively having a handpicked
compliance team embedded inside the firm. The United States would then
lift a seven-year ban that prevented the company from buying American
products and was quickly driving it out of business.

\href{https://www.nytimes.com/2018/06/07/business/what-is-zte.html}{\emph{{[}Read
more about ZTE.{]}}}

But the settlement has inflamed lawmakers, including top Republicans,
who objected to helping a Chinese company that broke American law and
has been accused of posing a national security threat. It also puts the
United States in an awkward position as it punishes allies like Canada,
Mexico and the European Union with stiff tariffs on steel and aluminum,
and insists that countries in Europe and elsewhere abide by American
sanctions on Iran.

In 2016, the United States found the Chinese company guilty of violating
American sanctions on Iran and North Korea. In April, the government
said ZTE had failed to take the necessary actions to rectify the issue,
and had lied about its efforts, prompting the Commerce Department to
implement the ban. Defense officials have also repeatedly expressed
concern about the risk that ZTE's equipment could pose to national
security.

Lawmakers moved swiftly to try to scuttle the agreement on Thursday as a
bipartisan group of senators introduced an amendment that would
automatically reinstate ZTE's ban on purchasing American products until
the president certified to Congress that the company had met certain
conditions.

``I assure you with 100\% confidence that \#ZTE is a much greater
national security threat than steel from Argentina or Europe,'' Senator
Marco Rubio, a Florida Republican who supported the amendment, wrote on
Twitter on Thursday. ``\#VeryBadDeal.''

Senator Mark Warner, Democrat of Virginia, called the deal an ``awful
mistake,'' adding that ``Mr. Trump has done something pretty unique ---
he's built a virtual unanimous bipartisan coalition.''

The commerce secretary, Wilbur Ross, emphasized the toughness of the
agreement on Thursday, saying it was the largest such penalty ever
levied by the agency's Bureau of Industry and Security and included
``unprecedented compliance measures.''

Mr. Ross and other administration officials have repeatedly insisted
that ZTE is being handled as a law enforcement matter that is
independent of trade negotiations. But those statements have been
undercut by the president himself, who has suggested that the company is
a bargaining chip in negotiations between the countries.

In mid-May, the president said he was working with Mr. Xi to give ZTE a
way to get back in business. Two days later, Mr. Trump
\href{https://twitter.com/realDonaldTrump/status/996119678551552000}{described
the ZTE move} as part of ``the larger trade deal we are negotiating with
China and my personal relationship with President Xi.''

Derek Scissors, a resident scholar at the American Enterprise Institute,
said that while ZTE's punishment was sufficiently tough, the path by
which it received a reprieve set a worrying precedent that suggested ``a
month-old decision by a Trump administration cabinet member can be
reversed if you call the president and tug on his heart strings.''

\includegraphics{https://static01.nyt.com/images/2018/06/08/business/08dc-zte-2/merlin_139212126_8b88339e-e525-49d0-8411-c33464ea45ed-articleLarge.jpg?quality=75\&auto=webp\&disable=upscale}

``The process by which we got here suggests you can buy off U.S. law,
and it suggests we're treating the Chinese better than our friends, both
of which are terrible implications,'' Mr. Scissors added.

ZTE's fate has gotten caught up in a bigger web. An American telecom
company, Qualcomm, which sells a large number of semiconductors to ZTE,
is awaiting Chinese approval of a deal to acquire NXP, a Dutch telecom
firm that will help it build the next generation of wireless technology,
known as 5G.

The Trump administration has expressed concerns about China gaining a
leading role in the development of 5G and has singled out Qualcomm as
key to helping the United States retain an edge. China, meanwhile, had
made it clear to the United States that it would not engage in talks to
defuse a brewing trade war between the two economic giants without
putting ZTE's ban on the table for discussion.

During a round of trade talks in Beijing last weekend, the Chinese
offered to make nearly \$70 billion worth of purchases of American
manufactured goods, natural gas, oil, coal, soybeans and other
agricultural products, people familiar with the discussions said. But
that offer was conditional on the Trump administration's not proceeding
with tariffs on \$50 billion worth of Chinese goods.

The Trump administration has not yet announced plans to suspend those
tariffs, which the White House has said would go into effect shortly
after June 15, and the administration's trade advisers remain deeply
divided over whether to proceed. Mr. Trump's advisers have portrayed the
tariffs as leverage to force China to open its markets and make other
concessions, such as dropping demands that American companies hand over
valuable intellectual property in order to operate in China.

A deal that lets ZTE back into business but does little to resolve those
broader concerns would most likely be criticized by hard-liners within
the administration, as well as many lawmakers, who agree that China
needs to change its practices and view a promise by the Chinese to
purchase more American goods as a false victory.

Senator John Kennedy, a Louisiana Republican, said he was undecided
about how lawmakers should proceed now that the administration has
struck a deal.

``I'm not a big ZTE fan; they cheat, they helped Iran and North Korea in
violation of our sanctions, and they are a little too close for my taste
with the Communist Party of China,'' Mr. Kennedy said. ``Call me very
skeptical about the wisdom of what's been done.''

The Trump administration privately
\href{https://www.nytimes.com/2018/05/25/us/politics/trump-trade-zte.html}{told
lawmakers last month} that it had reached a deal to keep the company
alive. On Thursday, the administration went public with its decision.

``At about 6 a.m. this morning, we executed a definitive agreement with
ZTE,'' Mr. Ross said in an interview on CNBC's ``Squawk Box,'' adding,
``This is a pretty strict settlement.''

``We are literally embedding a compliance department of our choosing
into the company to monitor it going forward. They will pay for those
people,'' Mr. Ross said. He went on to say that ZTE would pay a \$1
billion fine, as well as \$400 million in escrow to cover ``any future
violations.''

``We still retain the power to shut them down again,'' Mr. Ross said.

Some in China have speculated that the ZTE penalties are an effort to
gain leverage in other trade matters. But Washington-based experts and
officials point to another issue: that the administration had not
realized what a political problem it would be for Mr. Xi to have the
partly state-owned technology firm, which employs tens of thousands of
people in China, fail.

``You just don't have people in the administration that would recognize
how seriously the Chinese would take it,'' Mr. Scissors said. ``Ross did
not know he was running into a Chinese buzz saw.''

Advertisement

\protect\hyperlink{after-bottom}{Continue reading the main story}

\hypertarget{site-index}{%
\subsection{Site Index}\label{site-index}}

\hypertarget{site-information-navigation}{%
\subsection{Site Information
Navigation}\label{site-information-navigation}}

\begin{itemize}
\tightlist
\item
  \href{https://help.nytimes.com/hc/en-us/articles/115014792127-Copyright-notice}{©~2020~The
  New York Times Company}
\end{itemize}

\begin{itemize}
\tightlist
\item
  \href{https://www.nytco.com/}{NYTCo}
\item
  \href{https://help.nytimes.com/hc/en-us/articles/115015385887-Contact-Us}{Contact
  Us}
\item
  \href{https://www.nytco.com/careers/}{Work with us}
\item
  \href{https://nytmediakit.com/}{Advertise}
\item
  \href{http://www.tbrandstudio.com/}{T Brand Studio}
\item
  \href{https://www.nytimes.com/privacy/cookie-policy\#how-do-i-manage-trackers}{Your
  Ad Choices}
\item
  \href{https://www.nytimes.com/privacy}{Privacy}
\item
  \href{https://help.nytimes.com/hc/en-us/articles/115014893428-Terms-of-service}{Terms
  of Service}
\item
  \href{https://help.nytimes.com/hc/en-us/articles/115014893968-Terms-of-sale}{Terms
  of Sale}
\item
  \href{https://spiderbites.nytimes.com}{Site Map}
\item
  \href{https://help.nytimes.com/hc/en-us}{Help}
\item
  \href{https://www.nytimes.com/subscription?campaignId=37WXW}{Subscriptions}
\end{itemize}
