Sections

SEARCH

\protect\hyperlink{site-content}{Skip to
content}\protect\hyperlink{site-index}{Skip to site index}

\href{https://www.nytimes.com/section/business}{Business}

\href{https://myaccount.nytimes.com/auth/login?response_type=cookie\&client_id=vi}{}

\href{https://www.nytimes.com/section/todayspaper}{Today's Paper}

\href{/section/business}{Business}\textbar{}What Is ZTE? A Chinese
Geopolitical Pawn That Trump Wants to Rescue.

\url{https://nyti.ms/2JnbaKx}

\begin{itemize}
\item
\item
\item
\item
\item
\end{itemize}

Advertisement

\protect\hyperlink{after-top}{Continue reading the main story}

Supported by

\protect\hyperlink{after-sponsor}{Continue reading the main story}

\hypertarget{what-is-zte-a-chinese-geopolitical-pawn-that-trump-wants-to-rescue}{%
\section{What Is ZTE? A Chinese Geopolitical Pawn That Trump Wants to
Rescue.}\label{what-is-zte-a-chinese-geopolitical-pawn-that-trump-wants-to-rescue}}

\includegraphics{https://static01.nyt.com/images/2018/05/15/world/15zte-explain-1/merlin_138124971_6b2efb36-f4b1-4371-848a-9a0f5a53c809-articleLarge.jpg?quality=75\&auto=webp\&disable=upscale}

By \href{https://www.nytimes.com/by/paul-mozur}{Paul Mozur} and
\href{https://www.nytimes.com/by/kevin-granville}{Kevin Granville}

\begin{itemize}
\item
  June 7, 2018
\item
  \begin{itemize}
  \item
  \item
  \item
  \item
  \item
  \end{itemize}
\end{itemize}

ZTE, the Chinese telecommunications giant that was on the brink of
collapse after being hit with tough penalties by the Trump
administration, has become a linchpin of trade relations between the
United States and China. On Thursday, Wilbur Ross, the commerce
secretary, said the
\href{https://www.nytimes.com/2018/06/07/business/us-china-zte-deal.html?action=click\&module=Top\%20Stories\&pgtype=Homepage}{administration
had reached a deal} to lift restrictions imposed on ZTE after it was
found to have violated American sanctions related to doing business with
countries like Iran and North Korea. ZTE's fate has been at the center
of a broader dispute between China and the United States, and the deal
announced on Thursday could inflame a battle with Congress over national
security interests.

Here's what you need to know.

\hypertarget{what-is-zte}{%
\subsection{What Is ZTE?}\label{what-is-zte}}

Zhongxing Telecommunications Equipment, known as ZTE, makes cheap
smartphones that are mostly sold in developing countries, though it also
sells them in the United States.

But ZTE is one of two Chinese companies --- Huawei is the other --- that
sell equipment for cellular networks. It has about 75,000 employees and
says it does business in more than 160 countries.

That makes it an important geopolitical pawn for Beijing, both as an
innovator and as a builder of state-funded projects overseas. If China
wants to improve ties with a government in the developing world, it
often offers loans that can be used to set a ZTE-powered cellular
network.

Longer term, China hopes that companies like ZTE will become powerhouses
that can help the country wean itself from a reliance on American tech
firms, which Beijing views as security threats because of the
possibility that they could help Washington spy.

\hypertarget{how-did-it-break-sanctions}{%
\subsection{How Did It Break
Sanctions?}\label{how-did-it-break-sanctions}}

Tech supply chains are so intertwined these days that just about every
product that ZTE makes has some American components or software in it
--- think microchips, modems and Google's Android operating system. So
if ZTE sells a smartphone to North Korea, it might also be selling a
Qualcomm chip inside that phone. That's illegal under American sanctions
that prohibit the sale of United States tech to embargoed countries.

When the Commerce Department released its findings against ZTE in 2016,
it took the rare step of
\href{https://www.nytimes.com/2016/03/08/technology/us-restricts-sales-to-zte-saying-it-breached-sanctions.html}{disclosing
evidence} of the company's guilt. One document, signed by several senior
ZTE executives, cautioned that American export laws were a risk because
the company was selling to ``all five major embargoed countries ---
Iran, Sudan, North Korea, Syria and Cuba.''

A second company document featured flow charts for best practices to
circumvent American sanctions. Last year, ZTE
\href{https://www.nytimes.com/2017/03/07/technology/zte-china-fine.html}{acknowledged
its guilt} and paid a \$1.19 billion fine.

\hypertarget{how-did-the-us-hobble-zte}{%
\subsection{How Did the U.S. Hobble
ZTE?}\label{how-did-the-us-hobble-zte}}

The Commerce Department wasn't done with that hefty penalty.

In April, officials said ZTE had violated its agreement with the United
States because it didn't punish senior management for having violated
the sanctions. Instead, the Commerce Department said, ZTE paid them
bonuses and lied about it. As punishment, the department
\href{https://www.nytimes.com/2018/04/16/technology/chinese-tech-company-blocked-from-buying-american-components.html}{forbade
American technology companies} to sell their products to ZTE for seven
years.

That means no Qualcomm chips or Android software for its phones, and no
American chips or other components for its cellular gear. Analysts
estimate that four-fifths of ZTE's products have American components.
ZTE went into a tailspin, saying last month that it had shut down major
operations.

\hypertarget{why-did-trump-intervene}{%
\subsection{Why Did Trump Intervene?}\label{why-did-trump-intervene}}

The American president hasn't explained his decision to try to help the
company, other than to cite the potential for lots of Chinese workers to
lose their jobs. But ZTE's troubles come at a complicated moment.

In normal times, the company's fate would be a legal matter for the
Commerce Department. But the Trump administration is pressuring China to
make trade concessions. It may also need Beijing's help to strike a deal
with North Korea as Washington and Pyongyang plan a high-profile meeting
on June 12 in Singapore.

Mr. Trump appears to be using ZTE's punishment as a bargaining chip in
negotiations with China, rather than a matter of law enforcement. It
isn't clear what he will get in return for allowing ZTE to remain in
business.

{[}JIM STEWART'S TAKE*: Trump might cave to China because of*
\href{https://www.nytimes.com/2018/06/07/business/trump-trade-china-iowa-soybeans.html?action=click\&module=Top\%20Stories\&pgtype=Homepage}{\emph{Iowa's
soybean farmers}}{]}

\hypertarget{what-are-the-terms-of-the-deal}{%
\subsection{What Are the Terms of the
Deal?}\label{what-are-the-terms-of-the-deal}}

According to Mr. Ross, who
\href{https://www.commerce.gov/news/press-releases/2018/06/secretary-ross-announces-14-billion-zte-settlement-zte-board-management}{announced
the deal} to end the American sanctions on Thursday morning on CNBC:

■ ZTE must pay a \$1 billion fine plus \$400 million in escrow to cover
``any future violations.''

■ ZTE must allow a compliance team, chosen by the United States, to
monitor the company.

■ ZTE must also change its board of directors and executive team within
30 days.

\hypertarget{some-lawmakers-would-like-to-block-the-deal-can-they}{%
\subsection{Some Lawmakers Would Like to Block the Deal. Can
They?}\label{some-lawmakers-would-like-to-block-the-deal-can-they}}

Some members of Congress, Republicans and Democrats, say that absolving
ZTE of its misdeeds runs counter to national security interests. It is
unclear, however, whether their efforts to block Mr. Trump's moves
regarding ZTE will be successful.

``Their technology is a national security threat, according to our
defense and law enforcement authorities,'' Senator Chuck Schumer of New
York, the minority leader, said on Wednesday. ``Why on earth is the
Trump administration considering relaxing penalties on such a bad
actor?''

On Thursday, Senator Marco Rubio, Republican of Florida,
\href{https://twitter.com/marcorubio/status/1004725599074770944}{said on
Twitter}: ``This `deal' with \#ZTE may keep them from selling to Iran
and North Korea. That's good. But it will do nothing to keep us safe
from corporate \& national security espionage. That is dangerous. Now
Congress will need to act to keep America safe from \#China.''

On Friday morning, Mr. Trump fired back at Mr. Schumer's criticism.
``Schumer failed with North Korea and Iran, we don't need his advice!''
\href{https://twitter.com/realDonaldTrump/status/1005028243760611328}{he
said on Twitter}.

The House and Senate
\href{https://www.nytimes.com/2018/05/24/us/politics/trump-zte-china.html}{have
drafted legislation} that would block the ZTE deal, although it is
uncertain what the practical effect would be. The House passed a bill
last month meant to hamstring the Trump administration's flexibility in
maneuvering on the issue. The Senate Banking Committee approved an
amendment to a bill on foreign investment that would prevent Mr. Trump
from modifying penalties on Chinese companies within a year of their
being found to have violated United States law.

Advertisement

\protect\hyperlink{after-bottom}{Continue reading the main story}

\hypertarget{site-index}{%
\subsection{Site Index}\label{site-index}}

\hypertarget{site-information-navigation}{%
\subsection{Site Information
Navigation}\label{site-information-navigation}}

\begin{itemize}
\tightlist
\item
  \href{https://help.nytimes.com/hc/en-us/articles/115014792127-Copyright-notice}{©~2020~The
  New York Times Company}
\end{itemize}

\begin{itemize}
\tightlist
\item
  \href{https://www.nytco.com/}{NYTCo}
\item
  \href{https://help.nytimes.com/hc/en-us/articles/115015385887-Contact-Us}{Contact
  Us}
\item
  \href{https://www.nytco.com/careers/}{Work with us}
\item
  \href{https://nytmediakit.com/}{Advertise}
\item
  \href{http://www.tbrandstudio.com/}{T Brand Studio}
\item
  \href{https://www.nytimes.com/privacy/cookie-policy\#how-do-i-manage-trackers}{Your
  Ad Choices}
\item
  \href{https://www.nytimes.com/privacy}{Privacy}
\item
  \href{https://help.nytimes.com/hc/en-us/articles/115014893428-Terms-of-service}{Terms
  of Service}
\item
  \href{https://help.nytimes.com/hc/en-us/articles/115014893968-Terms-of-sale}{Terms
  of Sale}
\item
  \href{https://spiderbites.nytimes.com}{Site Map}
\item
  \href{https://help.nytimes.com/hc/en-us}{Help}
\item
  \href{https://www.nytimes.com/subscription?campaignId=37WXW}{Subscriptions}
\end{itemize}
