Sections

SEARCH

\protect\hyperlink{site-content}{Skip to
content}\protect\hyperlink{site-index}{Skip to site index}

\href{https://myaccount.nytimes.com/auth/login?response_type=cookie\&client_id=vi}{}

\href{https://www.nytimes.com/section/todayspaper}{Today's Paper}

\href{/section/opinion}{Opinion}\textbar{}Clay-Court Tennis, the
`Greatest Show on Dirt'

\href{https://nyti.ms/2JwcZF6}{https://nyti.ms/2JwcZF6}

\begin{itemize}
\item
\item
\item
\item
\item
\end{itemize}

Advertisement

\protect\hyperlink{after-top}{Continue reading the main story}

Supported by

\protect\hyperlink{after-sponsor}{Continue reading the main story}

\href{/section/opinion}{Opinion}

\href{/column/sporting}{Sporting}

\hypertarget{clay-court-tennis-the-greatest-show-on-dirt}{%
\section{Clay-Court Tennis, the `Greatest Show on
Dirt'}\label{clay-court-tennis-the-greatest-show-on-dirt}}

By Rowan Ricardo Phillips

Mr. Phillips, the writer of the forthcoming ``The Circuit: A Tennis
Odyssey,'' plays the sport~almost exclusively on red clay.

\begin{itemize}
\item
  June 9, 2018
\item
  \begin{itemize}
  \item
  \item
  \item
  \item
  \item
  \end{itemize}
\end{itemize}

\includegraphics{https://static01.nyt.com/images/2018/06/09/opinion/09sportingWeb/merlin_139275483_febf3a21-89f5-4894-997c-396689457ebd-articleLarge.jpg?quality=75\&auto=webp\&disable=upscale}

The French Open is the quintessential clay court tournament: It's the
greatest show on dirt.

Every year at this time, I fall in love again and again with Roland
Garros's beautiful, burnt sienna courts ringed with emerald green
backdrops. Though it's not really clay. We call it clay because of its
origins. In late 19th-century France, ceramics were crushed into powder
and spread over the grass courts of a Cannes hotel to protect them from
wear and to bring down maintenance costs. (Grass courts are costly to
maintain --- especially with people trampling all over them day after
day.)

The surface proved to be a great hit among both the visitors of the Côte
d'Azur and the hotel owners. Further, the emergence of tennis coincided
with the birth of the modern spectator, and the unique clay-court color
palette became a draw not just for players but for people who simply
wanted to sit courtside and watch a match.

And so, what up until then was an essentially English game played on
defunct croquet courts and known as
``\href{http://www.fundinguniverse.com/company-histories/the-all-england-lawn-tennis-croquet-club-history/}{lawn
tennis}'' swiftly became a global game played on a variety of surfaces
and soon was to be known as simply tennis.

\includegraphics{https://static01.nyt.com/images/2018/06/09/opinion/09sporting2web/09sporting2web-articleLarge.jpg?quality=75\&auto=webp\&disable=upscale}

A tennis court is like a good play: The lines may stay the same, but the
context changes depending on the real world it inhabits. The context of
a tennis court is invariably its surface. The game has been played on
grass, red clay, green clay, blue clay, a myriad of asphalt and concrete
hard courts, wood and even carpet.

You'll find by playing on different courts that the surface changes
everything: what the ball does, how fast it moves, what your body does,
how fast it moves --- even the ideas that enter your head during a point
change (or at least should change) depending on what's under your feet.

I play almost exclusively on red clay. During the cold months, I play
indoors near the East River. And during the warm months, I play outdoors
just off the Hudson River. Red clay might be associated with Western
Europe and South America, but New York is no slouch in that department.
There's something curative about the surface: I'm now in my 40s,
sporting a surgically repaired Achilles' tendon, and the way red clay
gives under me feels less taxing at the end of a couple of hours of
court time.

Conversely, points play out longer, it's harder to hit winners, and so
the mental exertion eventually catches up to my body. You have to think
on clay. Or maybe it's that as the rallies stretch on, you're tempted to
think when you shouldn't think at all. But as I watch Rafael Nadal and
Simona Halep and Dominic Thiem and Sloane Stephens this weekend, I'm
glad that they've figured out the thinking part.

Clay begs your body to come to a different understanding of the game.
Shots that would have whizzed past you on grass hang in the air
invitingly for you on clay --- that is, if you know how to slide to get
there in time and what types of shots you can and cannot make in those
situations.

As we've seen from Serena Williams and Roger Federer, being able to
stand on the baseline and hit shots early and flat is a great advantage
for a player who wants to unleash aggressive, first-strike tennis. But
on clay courts, the ball bounces so remorselessly high that if you don't
play farther back on the court, away from the baseline, you'll be left
to trying to pull off shots coming at you shoulder high and often even
higher.

The French Open luxuriates in its own laws of physics and playlists of
tactics that make it more distinctive than even Wimbledon. Tennis on
grass, after all, is really in essence a bucolic, ultrafast version of
hard court tennis infused with a heavy dose of nostalgia.

Strangely, this is one of the allures of tennis. That as it glances both
backward and forward, the game can revel in the nostalgia of grass and
mandatory all-white outfits while heralding in video replay and
extensive changes such as
\href{https://www.atpworldtour.com/en/news/rule-changes-innovation-for-next-gen-atp-finals-2017}{tiebreakers
and shot clocks}.

This is no doubt in part because of the nature of the schedule.
Professional tennis is one of the few sports that begin on the first day
of the year and progress forward to the end of the year. Within this
calendar year the three main surfaces --- hard court, clay and grass ---
each form their own type of mini-season within the full season,
beginning on hard courts, moving to clay, then grass and finally back to
hardcourts. In other words, a year in tennis mimics a year in our lives:
It's seasonal, and a season is inherently a thing both of renewal and
destruction, welcomes and farewells.

So spare a moment for my favorite surface this weekend and catch the
French Open finals. Come Monday tennis turns the page, from the greatest
show on dirt to leaves of grass.

Advertisement

\protect\hyperlink{after-bottom}{Continue reading the main story}

\hypertarget{site-index}{%
\subsection{Site Index}\label{site-index}}

\hypertarget{site-information-navigation}{%
\subsection{Site Information
Navigation}\label{site-information-navigation}}

\begin{itemize}
\tightlist
\item
  \href{https://help.nytimes.com/hc/en-us/articles/115014792127-Copyright-notice}{©~2020~The
  New York Times Company}
\end{itemize}

\begin{itemize}
\tightlist
\item
  \href{https://www.nytco.com/}{NYTCo}
\item
  \href{https://help.nytimes.com/hc/en-us/articles/115015385887-Contact-Us}{Contact
  Us}
\item
  \href{https://www.nytco.com/careers/}{Work with us}
\item
  \href{https://nytmediakit.com/}{Advertise}
\item
  \href{http://www.tbrandstudio.com/}{T Brand Studio}
\item
  \href{https://www.nytimes.com/privacy/cookie-policy\#how-do-i-manage-trackers}{Your
  Ad Choices}
\item
  \href{https://www.nytimes.com/privacy}{Privacy}
\item
  \href{https://help.nytimes.com/hc/en-us/articles/115014893428-Terms-of-service}{Terms
  of Service}
\item
  \href{https://help.nytimes.com/hc/en-us/articles/115014893968-Terms-of-sale}{Terms
  of Sale}
\item
  \href{https://spiderbites.nytimes.com}{Site Map}
\item
  \href{https://help.nytimes.com/hc/en-us}{Help}
\item
  \href{https://www.nytimes.com/subscription?campaignId=37WXW}{Subscriptions}
\end{itemize}
