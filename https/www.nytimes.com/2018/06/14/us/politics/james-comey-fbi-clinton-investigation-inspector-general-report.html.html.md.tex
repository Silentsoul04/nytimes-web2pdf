Sections

SEARCH

\protect\hyperlink{site-content}{Skip to
content}\protect\hyperlink{site-index}{Skip to site index}

\href{https://www.nytimes.com/section/politics}{Politics}

\href{https://myaccount.nytimes.com/auth/login?response_type=cookie\&client_id=vi}{}

\href{https://www.nytimes.com/section/todayspaper}{Today's Paper}

\href{/section/politics}{Politics}\textbar{}Comey Often Thought He Knew
Best. That May Have Hurt the F.B.I.

\url{https://nyti.ms/2HOvR0D}

\begin{itemize}
\item
\item
\item
\item
\item
\end{itemize}

Advertisement

\protect\hyperlink{after-top}{Continue reading the main story}

Supported by

\protect\hyperlink{after-sponsor}{Continue reading the main story}

\hypertarget{comey-often-thought-he-knew-best-that-may-have-hurt-the-fbi}{%
\section{Comey Often Thought He Knew Best. That May Have Hurt the
F.B.I.}\label{comey-often-thought-he-knew-best-that-may-have-hurt-the-fbi}}

\includegraphics{https://static01.nyt.com/images/2018/06/15/us/politics/15dc-comey/merlin_109623839_69b1a686-37ff-4244-adf8-1ca5227e17d9-articleLarge.jpg?quality=75\&auto=webp\&disable=upscale}

By \href{https://www.nytimes.com/by/adam-goldman}{Adam Goldman}

\begin{itemize}
\item
  June 14, 2018
\item
  \begin{itemize}
  \item
  \item
  \item
  \item
  \item
  \end{itemize}
\end{itemize}

WASHINGTON --- As deputy attorney general during the George W. Bush
administration, James B. Comey clashed repeatedly with the White House
over its interrogation and warrantless wiretapping programs, earning a
reputation of fighting for his view of what was right no matter whom he
angered.

That same impulse --- that he knew best, no matter the consequences ---
underpinned Mr. Comey's decisions in 2016 to flout Justice Department
norms and update the public on the investigation into Hillary Clinton's
handling of classified information. Democrats have said he cost her the
presidential election.

Mr. Comey was
\href{https://www.nytimes.com/2018/06/14/us/politics/fbi-inspector-general-comey-trump-clinton-report.html}{faulted
for those decisions} in
\href{https://int.nyt.com/data/documenthelper/39-justice-department-report-fbi-clinton-comey/5e54a6bfd23e7b94fbad/optimized/full.pdf\#page=1}{a
highly critical Justice Department report} released on Thursday about
the F.B.I.'s handling of the Clinton inquiry. By trying to protect the
bureau, the department's inspector general found, Mr. Comey instead
damaged the F.B.I.'s reputation.

``Comey chose to deviate from the F.B.I.'s and the department's
established procedures and norms and instead engaged in his own
subjective, ad hoc decision making,'' the report said. It added, ``The
decisions negatively impacted the perception of the F.B.I. and the
department as fair administrators of justice.''

An official condemnation of Mr. Comey's go-it-alone approach, the report
is bound to shape his legacy, providing grist for both Republicans and
Democrats as well as F.B.I. agents who disagreed with how he ran the
bureau at a politically perilous time.

Mr. Comey defended his decisions and said the inspector general, Michael
E. Horowitz, had the benefit of hindsight. While Mr. Comey supported the
review, he disagreed with its conclusions.

\hypertarget{read-justice-dept-report-on-the-fbis-handling-of-clinton-inquiry}{%
\subsection{Read: Justice Dept. Report on the F.B.I.'s Handling of
Clinton
Inquiry}\label{read-justice-dept-report-on-the-fbis-handling-of-clinton-inquiry}}

The Justice Department's inspector general released a report on Thursday
detailing the F.B.I.'s handling of the Clinton email investigation
during the 2016 presidential election.

\includegraphics{https://int.nyt.com/data/documenthelper/39-justice-department-report-fbi-clinton-comey/5e54a6bfd23e7b94fbad/optimized/thumbnail.png}

``If a future F.B.I. leadership team ever faces a similar situation ---
something I pray never happens --- it will have the benefit of this
important document,'' he wrote in
\href{https://www.nytimes.com/2018/06/14/opinion/comey-clinton-inspector-general.html}{an
Op-Ed in The New York Times}.

Fired abruptly by President Trump last year as the Russia investigation
engulfed the young Trump administration, Mr. Comey has returned to the
public spotlight, chastening the president on Twitter and
\href{https://www.nytimes.com/2018/04/12/us/politics/trump-comey-book.html}{writing
a best-seller}. Whether he has a third act in another administration or
as a publicly elected official is an open question.

In his tour as F.B.I. director, Mr. Comey ultimately served as a major
figure in the 2016 election,
\href{https://www.nytimes.com/2018/06/14/upshot/did-comey-cost-clinton-the-election-why-well-never-know.html?action=click\&module=Well\&pgtype=Homepage}{possibly
shaping its outcome} even as he sought to navigate the bureau away from
the bitter political atmosphere of the campaign.

Mr. Horowitz determined that Mr. Comey should not have
\href{https://www.nytimes.com/2016/07/06/us/politics/hillary-clinton-fbi-email-comey.html}{announced
unilaterally} in July of that year that he would not recommend charges
against Mrs. Clinton, and he should not have called her ``extremely
careless'' during a highly unusual news conference.

Mr. Comey was insubordinate, the inspector general said, and should have
followed the chain of command and coordinated with his Justice
Department bosses in holding a news conference. Mr. Comey told Attorney
General Loretta E. Lynch that he intended to make an announcement
regarding the investigation but provided no details.

Mr. Comey also should not have sent a pair of letters to Congress just
days before the election saying the F.B.I. had reopened the
investigation to examine new evidence and then closed it days later, the
inspector general said.

The former director's supreme confidence has exposed the F.B.I. to
accusations of political bias and corruption, according to former and
current agents, who predicted the F.B.I. would need years to regain the
public's trust. ``For Comey, this is a stain that will not come out,''
said Tim Weiner, author of ``Enemies: A History of the F.B.I.''

\includegraphics{https://static01.nyt.com/images/2018/06/15/us/politics/15dc-comey-sub/15dc-comey2-articleLarge.jpg?quality=75\&auto=webp\&disable=upscale}

Mr. Comey has refused to say he made a mistake but concedes he might
have done ``some things differently.'' In his best-selling book
published this spring,
``\href{https://www.nytimes.com/2018/04/12/books/review/james-comey-a-higher-loyalty.html}{A
Higher Loyalty},'' Mr. Comey wrote that perhaps he could have found a
better way to describe Mrs. Clinton's conduct but spent little time
second-guessing himself.

Mr. Trump had praised Mr. Comey before and immediately after taking
office. On the campaign trail he said that Mr. Comey had ``a lot of
guts'' for taking on the Clinton investigation, and in a memorable White
House meeting the president embraced Mr. Comey, saying, ``He's become
more famous than me.''

But Mr. Trump quickly soured on him, and in firing Mr. Comey last year,
the president initially cited Justice Department criticism over his
handling of the Clinton investigation.

Mr. Trump later acknowledged that the Russia inquiry was on his mind
during that time and has more recently begun a public campaign to
discredit Mr. Comey, who is a key witness in the obstruction
investigation of the president.

While the new report dented Mr. Comey's standing, Mr. Weiner said the
totality of Mr. Comey's time in public service should not be overlooked.

Mr. Comey had a formidable law enforcement career beginning as a mob
prosecutor in New York. Later, as an assistant United States attorney
\href{https://www.nytimes.com/2001/12/02/nyregion/man-in-the-news-reputation-for-tenacity-james-brien-comey.html}{in
Richmond, Va.}, he cracked down on felons arrested with guns and brought
charges in a major terrorism case. For his efforts, Mr. Comey landed on
the front page of a weekly newspaper under the headline ``One of Good
Guys.'' Mr. Comey had not told his boss about the article ahead of time.

In 2002, Mr. Comey returned to New York as the United States attorney in
Manhattan. He embraced tough cases, prosecuting Martha Stewart on
charges connected to a personal stock trade she made.

Image

Mr. Comey, the deputy attorney general in the George W. Bush
administration, once threatened to resign along with Robert S. Mueller
III over a National Security Agency surveillance program.Credit...J.
Scott Applewhite/Associated Press

``Charging Martha Stewart was my first experience with getting a lot of
hate and heat for a decision that had been carefully and thoughtfully
made,'' Mr. Comey recounted in his book. She was found guilty on all
charges and served five months in federal prison.

Soon, Mr. Comey was tapped to be the deputy attorney general, trying to
protect the country after the Sept. 11 attacks as the Justice
Department's No. 2 official.

He
\href{https://www.nytimes.com/2007/05/16/washington/16nsa.html}{famously
confronted} Mr. Bush's aides as they tried to get Attorney General John
Ashcroft, who was hospitalized with a pancreatic ailment, to reauthorize
a National Security Agency surveillance program that Mr. Comey had found
to be legally dubious. Mr. Comey, who threatened to resign along with
Robert S. Mueller III, then the F.B.I. director, prevailed.

He also fought with the White House over Justice Department memos that
authorized the C.I.A.'s use of harsh interrogation techniques, including
waterboarding. Mr. Comey and others at the Justice Department believed
that the agency might be violating laws against torture because of how
the techniques were being applied.

Mr. Comey did not explicitly say he was thinking about his legacy and
being on the right side of history, but, he wrote, a comment by his wife
resonated with him. ``Don't be the torture guy,'' she said, advising him
to stand up against the program.

Mr. Comey's pleas were ultimately ignored, and no policy changes were
made. He left for the private sector in 2005, then took over the F.B.I.
in 2013.

More than anyone since J. Edgar Hoover, Mr. Comey embraced the persona
of national lawman. He saw himself as the principled leader of not only
the F.B.I. but police officers everywhere. When Mr. Comey traveled, he
mingled with agents in field offices and local police officers. He was
something of a rock star, albeit a tall one at 6-foot-8.

In 2015, the Clinton investigation, with its vast political
implications, began to consume the F.B.I.'s seventh-floor leadership.
Mr. Comey knew the F.B.I. would be attacked no matter the outcome.

The inquiry wrapping up as the 2016 presidential primaries did, Mr.
Comey began to debate whether and how to disclose it. Mr. Comey, who
prides himself on being a great communicator, settled on a public
announcement, a departure from the F.B.I.'s usual practice of silence on
investigations.

He thought he could deliver the right message to the American people,
balancing openness while protecting the bureau from accusations of
favoritism.

Still, before briefing reporters, he later wrote, ``It felt like I was
about to damage my career.''

Indeed, many former F.B.I. agents thought Mr. Comey should have remained
silent and let the Justice Department announce that no charges would be
brought in the case. His overconfidence was his undoing, agents have
said in interviews.

In the fall, when the F.B.I. discovered possible new evidence in the
case, Mr. Comey confronted two ``terrible options'': speak or conceal,
as he titled a chapter in his book.

He ultimately decided to tell lawmakers, which he had promised to do.
The disclosure upended the election in its final days, and Mr. Comey's
subsequent notification that the F.B.I. had closed the investigation
again without finding new evidence earned him a new round of
evisceration.

The inspector general characterized Mr. Comey's dilemma as ``a false
dichotomy.'' In reality, Mr. Horowitz wrote, Mr. Comey could either
follow established practices or policies --- or not. ``Although we
acknowledge that Comey faced a difficult situation with unattractive
choices, in proceeding as he did, we concluded that Comey made a serious
error of judgment,'' the report said.

Mr. Comey disagreed. ``The inspector general weighs it differently, and
that's O.K., even though I respectfully disagree,'' he wrote in his
Op-Ed.

Mr. Comey will probably forever be linked to Mrs. Clinton, said Douglas
M. Charles, an F.B.I. historian. ``People in history are remembered for
one or two things,'' he said. ``He will likely be remembered for
interfering in the 2016 election. I think it is potentially catastrophic
to his legacy.''

Advertisement

\protect\hyperlink{after-bottom}{Continue reading the main story}

\hypertarget{site-index}{%
\subsection{Site Index}\label{site-index}}

\hypertarget{site-information-navigation}{%
\subsection{Site Information
Navigation}\label{site-information-navigation}}

\begin{itemize}
\tightlist
\item
  \href{https://help.nytimes.com/hc/en-us/articles/115014792127-Copyright-notice}{©~2020~The
  New York Times Company}
\end{itemize}

\begin{itemize}
\tightlist
\item
  \href{https://www.nytco.com/}{NYTCo}
\item
  \href{https://help.nytimes.com/hc/en-us/articles/115015385887-Contact-Us}{Contact
  Us}
\item
  \href{https://www.nytco.com/careers/}{Work with us}
\item
  \href{https://nytmediakit.com/}{Advertise}
\item
  \href{http://www.tbrandstudio.com/}{T Brand Studio}
\item
  \href{https://www.nytimes.com/privacy/cookie-policy\#how-do-i-manage-trackers}{Your
  Ad Choices}
\item
  \href{https://www.nytimes.com/privacy}{Privacy}
\item
  \href{https://help.nytimes.com/hc/en-us/articles/115014893428-Terms-of-service}{Terms
  of Service}
\item
  \href{https://help.nytimes.com/hc/en-us/articles/115014893968-Terms-of-sale}{Terms
  of Sale}
\item
  \href{https://spiderbites.nytimes.com}{Site Map}
\item
  \href{https://help.nytimes.com/hc/en-us}{Help}
\item
  \href{https://www.nytimes.com/subscription?campaignId=37WXW}{Subscriptions}
\end{itemize}
