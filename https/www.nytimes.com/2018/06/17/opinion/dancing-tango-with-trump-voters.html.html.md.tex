Sections

SEARCH

\protect\hyperlink{site-content}{Skip to
content}\protect\hyperlink{site-index}{Skip to site index}

\href{https://myaccount.nytimes.com/auth/login?response_type=cookie\&client_id=vi}{}

\href{https://www.nytimes.com/section/todayspaper}{Today's Paper}

\href{/section/opinion}{Opinion}\textbar{}Dancing Tango With Trump
Voters

\href{https://nyti.ms/2JRa9dX}{https://nyti.ms/2JRa9dX}

\begin{itemize}
\item
\item
\item
\item
\item
\end{itemize}

Advertisement

\protect\hyperlink{after-top}{Continue reading the main story}

Supported by

\protect\hyperlink{after-sponsor}{Continue reading the main story}

\href{/section/opinion}{Opinion}

\href{/column/sporting}{Sporting}

\hypertarget{dancing-tango-with-trump-voters}{%
\section{Dancing Tango With Trump
Voters}\label{dancing-tango-with-trump-voters}}

By Meghan Flaherty

Ms.~Flaherty~is a writer who dances tango.

\begin{itemize}
\item
  June 17, 2018
\item
  \begin{itemize}
  \item
  \item
  \item
  \item
  \item
  \end{itemize}
\end{itemize}

\includegraphics{https://static01.nyt.com/images/2018/06/22/opinion/16flahertyWeb/16flahertyWeb-articleLarge.jpg?quality=75\&auto=webp\&disable=upscale}

I fell in love with tango in New York City. I thought of dancing like
riding the subway: I was constantly in the presence of other people, my
body up against their bodies, yet we rarely exchanged words. We got to
know each other's odors, transferred each other's sweat, then parted
ways.

In tango, that body contact is deliberate, methodical. It's both
unspeakably intimate and anonymous. You very often dance with strangers
and you very rarely speak. It is possible to dance a dozen times with
someone and barely learn that person's name. But you learn other things
about them --- not just how their bodies move and feel, but whether they
are arrogant or insecure, light- or heavy-hearted, in love or longing
for it. You get a tactile glimpse of their humanity.

In New York, I danced with executives and activists, divorcées,
handymen, intrepid college kids and engineers. We were old and fat and
white and brown and lithe and young. What mattered, all that mattered,
was the dance.

When my husband and I moved to Chapel Hill, N.C., in the summer of 2014,
I thought it would be different. This was a red state, after all, where
Confederate flags routinely flew and where transgender people would soon
be
\href{https://www.nytimes.com/2016/03/24/us/north-carolina-to-limit-bathroom-use-by-birth-gender.html}{barred
from public bathrooms} corresponding to their gender.

The tango scene in the Research Triangle was small --- maybe 75 dancers.
Milongas (social dances) were infrequent. The heart of the calendar was
a Thursday evening practice session, deep in the suburban office complex
sprawl.

The venue was tucked just off a barren, four-lane stretch of road I
never saw in daylight. The turn into the gravel lot was lodged between
two mailboxes --- one for a spiritualist and one for the Kingdom
Restoration Church for All Nations --- and narrow enough to almost miss,
even after three straight years of Thursday nights.

Our first time, we turned up fashionably late by New York standards, so
as not to look uncool. In North Carolina, that meant the night was
nearly over. The wood-paneled room smelled strongly of an aerosol
disinfectant. A massive industrial fan whirred in the corner, and a few
stock tango posters hung, a little limply, on the walls. We paid \$5 at
the door to someone's grandmother. All eyes were upon us.

We changed our shoes, smiled awkwardly at everyone. We fell a decade
short of the median, mid-40s age and, in athletic gear, were wildly
underdressed. This was small pond tango. But there were bar snacks:
chocolate and nuts and three bottles of wine from Trader Joe's. We
danced a few songs, introduced ourselves, and there we were: part of
this scrappy but devoted band of dancers.

Here, too, was a little slice of everyone: lawyers and construction
workers, tech gurus, game programmers, students, housewives,
schoolteachers, masseurs. For such a small group, we were remarkably
diverse, from 22 different countries: Argentina, Panama, Peru, Japan,
Malaysia, Ecuador, Russia, Scotland, Italy, Korea, Belgium, Belarus,
South Africa, Mauritania, and on and on. We, too, were old and fat and
white and brown and lithe and young.

In that room we were simply tango dancers, the same as anywhere else in
the world. There were the usual dramatis personae: The young Icarus too
eager to advance. The gentle, left-footed older gents. The pushy
peacocks who would rather break a lady's back than admit to their own
flaws. A few truly top-notch talents. And us --- two big-city blow-ins.
We were welcomed, no questions asked. Soon this Thursday crowd became a
kind of family, and the attendance-taking grandmother became a
grandmother, also, to us.

Perhaps what surprised me most about suburban tango was its deafening
indifference. At a time when Americans were said to be incapable of
crossing party lines, when immigrants were demonized, when bigotry was
running rife, there was no trace of this on Thursday nights. Whatever
unrest raged across the country, or even down the road, it wasn't raging
here.

Two days after 2.3 million North Carolina voters filled their bubbles in
for Donald Trump, and the unthinkable occurred, we went to tango. It was
just another Thursday night. People drank their wine and ate their
truffles. They made small talk and smiled and danced. I looked around
the room to see if I could riddle out the voting records of our fellow
dancers; I could not. No matter how well I knew these people's bodies
--- from breastplates to bony joints to fleshy cheeks --- I had no idea
about their politics. In my heartbroken daze, I remember thinking that
was comforting. That tango was the same.

No matter where you travel, or whose embrace you take, tango is tango.
You'll take a stranger in your arms, no questions asked --- and cuddle.
You'll press your softest parts together --- your cheeks, your chests.
You will listen to the same song, and thump along the floor to the same
beat, a pulse that echoes up your feet to thud in unison exactly where
your torsos meet. You'll hold this person's hand. Between songs you
might make even-tempered small talk. And afterward your partner's scent
will linger on your skin.

As the year drew to a close, the punch line of our national catastrophe
appeared to be that we'd lost touch. That we'd all fallen into echo
chambers, right and left, and nevermore the twain would meet. That,
trapped in our respective prisons of like minds, we had forgotten how to
listen.

But what we fail to do in politics --- harness our shared humanity ---
we do so easily in dance. We embrace the other. We take into account
another person's comfort, how our pace and pressure feel. We compromise
to compensate for differences in height, in weight, in skill. We listen
to each other listen to the music. We negotiate our way around the
crowded floor. We are indulgent, expressive, courteous when things go
wrong. We take great mutual care.

If we are capable of this intimacy, then maybe we are not beyond repair.

Advertisement

\protect\hyperlink{after-bottom}{Continue reading the main story}

\hypertarget{site-index}{%
\subsection{Site Index}\label{site-index}}

\hypertarget{site-information-navigation}{%
\subsection{Site Information
Navigation}\label{site-information-navigation}}

\begin{itemize}
\tightlist
\item
  \href{https://help.nytimes.com/hc/en-us/articles/115014792127-Copyright-notice}{©~2020~The
  New York Times Company}
\end{itemize}

\begin{itemize}
\tightlist
\item
  \href{https://www.nytco.com/}{NYTCo}
\item
  \href{https://help.nytimes.com/hc/en-us/articles/115015385887-Contact-Us}{Contact
  Us}
\item
  \href{https://www.nytco.com/careers/}{Work with us}
\item
  \href{https://nytmediakit.com/}{Advertise}
\item
  \href{http://www.tbrandstudio.com/}{T Brand Studio}
\item
  \href{https://www.nytimes.com/privacy/cookie-policy\#how-do-i-manage-trackers}{Your
  Ad Choices}
\item
  \href{https://www.nytimes.com/privacy}{Privacy}
\item
  \href{https://help.nytimes.com/hc/en-us/articles/115014893428-Terms-of-service}{Terms
  of Service}
\item
  \href{https://help.nytimes.com/hc/en-us/articles/115014893968-Terms-of-sale}{Terms
  of Sale}
\item
  \href{https://spiderbites.nytimes.com}{Site Map}
\item
  \href{https://help.nytimes.com/hc/en-us}{Help}
\item
  \href{https://www.nytimes.com/subscription?campaignId=37WXW}{Subscriptions}
\end{itemize}
