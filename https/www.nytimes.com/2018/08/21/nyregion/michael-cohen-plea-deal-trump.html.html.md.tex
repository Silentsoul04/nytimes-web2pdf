Sections

SEARCH

\protect\hyperlink{site-content}{Skip to
content}\protect\hyperlink{site-index}{Skip to site index}

\href{https://www.nytimes.com/section/nyregion}{New York}

\href{https://myaccount.nytimes.com/auth/login?response_type=cookie\&client_id=vi}{}

\href{https://www.nytimes.com/section/todayspaper}{Today's Paper}

\href{/section/nyregion}{New York}\textbar{}Michael Cohen Says He
Arranged Payments to Women at Trump's Direction

\url{https://nyti.ms/2OUBjTy}

\begin{itemize}
\item
\item
\item
\item
\item
\item
\end{itemize}

Advertisement

\protect\hyperlink{after-top}{Continue reading the main story}

Supported by

\protect\hyperlink{after-sponsor}{Continue reading the main story}

\hypertarget{michael-cohen-says-he-arranged-payments-to-women-at-trumps-direction}{%
\section{Michael Cohen Says He Arranged Payments to Women at Trump's
Direction}\label{michael-cohen-says-he-arranged-payments-to-women-at-trumps-direction}}

\includegraphics{https://static01.nyt.com/images/2018/08/22/us/politics/22dc-assess3/22dc-assess3-videoSixteenByNine3000.jpg}

By \href{http://www.nytimes.com/by/william-k-rashbaum}{William K.
Rashbaum}, \href{http://www.nytimes.com/by/maggie-haberman}{Maggie
Haberman}, \href{http://www.nytimes.com/by/ben-protess}{Ben Protess} and
\href{http://www.nytimes.com/by/jim-rutenberg}{Jim Rutenberg}

\begin{itemize}
\item
  Aug. 21, 2018
\item
  \begin{itemize}
  \item
  \item
  \item
  \item
  \item
  \item
  \end{itemize}
\end{itemize}

\href{https://www.nytimes.com/es/2018/08/21/michael-cohen-cargos-trump}{Leer
en español}

Michael D. Cohen, President Trump's former lawyer, made the
extraordinary admission in court on Tuesday that Mr. Trump had directed
him to arrange payments to two women during the 2016 campaign to keep
them from speaking publicly about affairs they said they had with Mr.
Trump.

Mr. Cohen acknowledged the illegal payments while pleading guilty to
breaking campaign finance laws and other charges, a litany of crimes
that revealed both his shadowy involvement in Mr. Trump's circle and his
own corrupt business dealings.

He told a judge in United States District Court in Manhattan that the
payments to the women were made ``in coordination with and at the
direction of a candidate for federal office,'' implicating the president
in a federal crime.

``I participated in this conduct, which on my part took place in
Manhattan, for the principal purpose of influencing the election'' for
president in 2016, Mr. Cohen said.

The plea represented a pivotal moment in the investigation into the
president, and the scene in the Manhattan courtroom was striking. Mr.
Cohen, a longtime lawyer for Mr. Trump --- and loyal confidant ---
described in plain-spoken language how Mr. Trump worked with him to
cover up a potential sex scandal that Mr. Trump feared would endanger
his rising candidacy.

\includegraphics{https://static01.nyt.com/images/2018/12/13/nyregion/13cohen-promo2/13cohen-promo2-videoSixteenByNine3000-v4.jpg}

Mr. Cohen also pleaded guilty to multiple counts of tax evasion and a
single count of bank fraud, capping a monthslong investigation by
Manhattan federal prosecutors who examined his personal business
dealings and his role in helping to arrange the financial deals with
women connected to Mr. Trump.

The plea came shortly before another blow to the president:
His\href{https://www.nytimes.com/2018/08/21/us/politics/paul-manafort-trial-verdict.html}{former
campaign manager, Paul Manafort, was convicted in his financial fraud
trial in Virginia}. The special counsel, Robert S. Mueller III, had
built a case that Mr. Manafort hid millions of dollars in foreign
accounts to evade taxes and lied to banks to obtain millions of dollars
in loans.

Mr. Trump's lawyers have, for months, said privately that they
considered Mr. Cohen's case to be potentially more problematic for the
president than the investigation by the special counsel.

But Mr. Trump's lawyer, Rudolph W. Giuliani, said in a statement after
Mr. Cohen's plea, ``There is no allegation of any wrongdoing against the
president in the government's charges against Mr. Cohen.''

In federal court in Manhattan, Mr. Cohen made the admission about Mr.
Trump's role in the payments to the women --- an adult film actress and
a former Playboy playmate --- as he pleaded guilty to two campaign
finance crimes.

One of those charges stemmed from a \$130,000 payment he made to the
actress, Stephanie Clifford, better known as Stormy Daniels, in the
run-up to the 2016 presidential election.

Prosecutors said that Trump Organization executives were involved in
reimbursing Mr. Cohen for that payment, accepting his phony invoices
that listed it as a legal expense. The other charge concerned a
complicated arrangement in which a tabloid bought the rights to the
story about the former Playboy model, Karen McDougal, then killed it.

\emph{{[}}\href{https://www.nytimes.com/2018/08/21/nyregion/michael-cohen-guilty-glenlivet-trump.html}{\emph{The
scene inside the courtroom}}\emph{: ``I had a glass of Glenlivet 12 on
the rocks,'' Mr. Cohen said when asked whether he had taken medication
or alcohol in the last 24 hours.{]}}

Mr. Cohen's plea was announced by Robert Khuzami, the deputy United
States attorney, along with senior officials from the F.B.I. and the
Internal Revenue Service. Addressing reporters outside the courthouse,
Mr. Khuzami said that Mr. Cohen had ``decided that he was above the law,
and for that, he is going to pay a very, very serious price.''

The plea agreement does not call for Mr. Cohen to cooperate with federal
prosecutors in Manhattan. Still, it does not preclude him from providing
information to them later or to
\href{https://www.nytimes.com/2018/08/16/us/politics/special-counsel-investigation-mueller.html}{the
special counsel,} who is examining the Trump campaign's possible
involvement in Russia's interference in the 2016 campaign.

\hypertarget{cohens-plea-deal-and-charges}{%
\subsection{Cohen's Plea Deal and
Charges}\label{cohens-plea-deal-and-charges}}

The plea agreement between Michael D. Cohen and prosecutors along with
the charges to which Mr. Cohen pleaded guilty.

\includegraphics{https://int.nyt.com/data/documenthelper/182-cohen-plea-deal/9bc6cd47e7c48e9f9469/optimized/thumbnail.png}

If Mr. Cohen were to substantially assist the special counsel's
investigation, Mr. Mueller could recommend a reduction in his sentence.

Mr. Cohen had been the president's longtime fixer, handling some of his
most sensitive personal matters over a decade at the Trump Organization.
He once said he would take a bullet for Mr. Trump.

As Mr. Cohen addressed the judge, admitting to the crimes he had
committed, the packed courtroom remained silent. Even when Mr. Cohen
made obvious references to Mr. Trump, referring to him as ``the
candidate'' and ``a candidate for federal office,'' spectators seemed to
listen raptly, with no gasps or audible reactions.

Mr. Cohen pleaded guilty to five counts of tax evasion for concealing
more than \$4 million in personal income from 2012 to 2016 and to one
count of bank fraud, for failing to disclose \$14 million in debts in an
application for a \$500,000 home equity line of credit --- the source of
his payment to Ms. Clifford.

He also pleaded guilty to making an excessive campaign contribution and
causing an unlawful corporate contribution during the 2016 election
cycle.

He will be sentenced on Dec. 12 before Judge William H. Pauley III.
Though Mr. Cohen faces a maximum of 65 years in prison, the plea
agreement provides for a far more lenient sentence: The government
calculated the sentencing guidelines at from 51 to 63 months and the
defense put them at 46 to 57 months. A final guidelines determination
will be made by the Probation Department, but the ultimate sentence will
be determined by Judge Pauley.

Mr. Cohen's attorney, Lanny J. Davis, said Mr. Cohen had put his family
and country ahead of his loyalty to Mr. Trump.

``He stood up and testified under oath that Donald Trump directed him to
commit a crime by making payments to two women for the principal purpose
of influencing an election,'' Mr. Davis said. ``If those payments were a
crime for Michael Cohen, then why wouldn't they be a crime for Donald
Trump?''

Looming over the negotiations between prosecutors and Mr. Cohen has been
the possibility of a presidential pardon. Mr. Trump reached out to Mr.
Cohen by phone a few days after the F.B.I. raids, and they had dinner
together a month earlier in March, at Mr. Trump's private club in
Florida, Mar-a-Lago.

Mr. Cohen's lawyer had loosely raised the issue of a pardon with an
attorney for Mr. Trump several months ago, according to two people with
knowledge of the conversations.

By striking a deal with Mr. Cohen that includes prison time, federal
authorities were aware of the risk that the president might pardon him,
said another person briefed on the matter. But it is also possible that
Mr. Cohen could eventually cooperate.

Prosecutors charged that Mr. Cohen's \$130,000 payment to Ms. Clifford
was effectively a donation to Mr. Trump's campaign, because by securing
her silence it improved his electoral fortunes, and thus violated 2016
campaign finance law prohibitions against donations of more than \$2,700
in a general election.

Mr. Cohen also pleaded guilty to ``causing'' an illegal corporate
donation to Mr. Trump through his involvement in a \$150,000 payment
American Media Inc. made to Ms. McDougal in late summer 2016 to buy the
rights to her story, effectively securing her silence for the remainder
of the campaign.

Corporations are prohibited from coordinating political spending with
candidates or their representatives. Mr. Cohen signed papers a month
later to purchase the rights to her agreement from A.M.I., but the
publisher backed out of the deal at the last minute.

The prosecutors filled in several blanks in a story that has been
unfolding for months about the lengths to which Mr. Cohen went during
the campaign to help his boss stave off embarrassing news about alleged
affairs ahead of Election Day.

\includegraphics{https://static01.nyt.com/images/2018/03/09/us/politics/PRE_COHEN_COVER-IMAGE_v3_BW/PRE_COHEN_COVER-IMAGE_v3_BW-videoSixteenByNineJumbo1600.jpg}

And the charges confirmed that what might have seemed on the surface to
have been only tawdry allegations involving an adult entertainment star
and a former Playboy model may actually carry legal and political
implications for a sitting president.

Michael

D. Cohen

who paid

\$130,000 to

who

negotiated

payment

between

says he made

the payments

at the direction of

The

National

Enquirer

Stephanie

Clifford

Karen

McDougal

whose

chief executive is

to cover up an

alleged affair with

to cover up an

alleged affair with

is a longtime

friend of

Donald J.

Trump

David J.

Pecker

Michael D. Cohen

who paid

\$130,000

to

says he made

the payments

at the direction of

who

negotiated

payment

between

The

National

Enquirer

Stephanie

Clifford

Karen

McDougal

to cover

up an

alleged

affair with

to cover

up an

alleged

affair with

whose

chief

executive

is

David J.

Pecker

Donald J.

Trump

is a

longtime

friend

of

By The New York Times

Prosecutors left little doubt that A.M.I. Inc., owner of The National
Enquirer, became a de facto campaign proxy for Mr. Cohen in his efforts
on behalf of Mr. Trump.

According to court papers, the publisher agreed in August 2015, months
before the first primaries, to look out for damaging stories about Mr.
Trump and his alleged affairs with women during talks with Mr. Cohen and
``one or more'' members of Mr. Trump's campaign.

The tabloid company agreed to identify those stories ``so they could be
purchased and their publication avoided,'' the prosecutors said on
Tuesday --- an inverted role for a tabloid scandal sheet such as The
Enquirer, which went on to savage Mr. Trump's opponents while promoting
and protecting him.

That deal led to the arrangement with Ms. McDougal, which was struck in
August 2016. It only came together, prosecutors said, after Mr. Cohen
promised A.M.I. it would be reimbursed for the McDougal payment.

But prosecutors also reported for the first time that A.M.I. was
intimately involved in the arrangement with Ms. Clifford. The tabloid
connected Mr. Cohen with the lawyer who had negotiated the McDougal
contract, Keith Davidson. Mr. Davidson also had Ms. Clifford as a client
and later hashed out the agreement for Ms. Clifford's silence.

Prosecutors said in court papers that when Mr. Cohen initially failed to
finalize the deal, an editor at A.M.I. --- a likely reference to Dylan
Howard, the company's chief content officer --- alerted Mr. Cohen that
there was a risk that Ms. Clifford would sell her story to another media
company, one that would publish it.

Mr. Cohen's admission that he broke the law by paying off Ms. Clifford
was a remarkable turnaround from the legal and publicity battle that he
and his lawyers had waged against her. Ms. Clifford and her lawyer,
Michael Avenatti, have hounded Mr. Cohen since May, taunting him on
social media and predicting his indictment.

Mr. Cohen's lawyers frequently fired back, accusing Mr. Avenatti of
``fanning a media storm'' and of ``smearing'' Mr. Cohen in a relentless
series of televised appearances.

``I predicted this a long time ago before the warrants were even
executed,'' Mr. Avenatti said on Tuesday. ``We feel extremely
vindicated.''

Mr. Cohen's plea culminates a long-running inquiry that became publicly
known in April when F.B.I. agents armed with search warrants
\href{https://www.nytimes.com/2018/04/09/us/politics/fbi-raids-office-of-trumps-longtime-lawyer-michael-cohen.html}{raided
his office}, apartment and hotel room, hauling away reams of documents,
including pieces of paper salvaged from a shredder, and millions of
electronic files contained on a series of cellphones, iPads and
computers.

Lawyers for Mr. Cohen and Mr. Trump spent the next four months working
with a court-appointed special master to review the documents and data
files to determine whether any of the materials were subject to
attorney-client privilege and should not be made available to the
government.

The special master, Barbara S. Jones, who completed her review last
week, issued a series of reports in recent months, finding that only a
fraction of the materials were privileged and the rest could be provided
to prosecutors for their investigation.

On Monday, the judge overseeing the review, Kimba M. Wood of Federal
District Court in Manhattan, issued an order adopting Ms. Jones's
findings and ending the review process.

It was unclear on Tuesday what role the materials that Ms. Jones
reviewed, which were made available to prosecutors on a rolling basis,
may have had in the charges against Mr. Cohen.

One collateral effect of Mr. Cohen's plea agreement is that it may allow
Mr. Avenatti, Ms. Clifford's lawyer, to proceed with a deposition of Mr.
Trump in a lawsuit that Ms. Clifford filed accusing the president of
breaking a nondisclosure agreement concerning their affair.

The lawsuit had been stayed by a judge pending the resolution of Mr.
Cohen's criminal case. Mr. Avenatti wrote on Twitter on Tuesday that he
would now seek to force Mr. Trump to testify ``under oath about what he
knew, when he knew it and what he did about it.''

Advertisement

\protect\hyperlink{after-bottom}{Continue reading the main story}

\hypertarget{site-index}{%
\subsection{Site Index}\label{site-index}}

\hypertarget{site-information-navigation}{%
\subsection{Site Information
Navigation}\label{site-information-navigation}}

\begin{itemize}
\tightlist
\item
  \href{https://help.nytimes.com/hc/en-us/articles/115014792127-Copyright-notice}{©~2020~The
  New York Times Company}
\end{itemize}

\begin{itemize}
\tightlist
\item
  \href{https://www.nytco.com/}{NYTCo}
\item
  \href{https://help.nytimes.com/hc/en-us/articles/115015385887-Contact-Us}{Contact
  Us}
\item
  \href{https://www.nytco.com/careers/}{Work with us}
\item
  \href{https://nytmediakit.com/}{Advertise}
\item
  \href{http://www.tbrandstudio.com/}{T Brand Studio}
\item
  \href{https://www.nytimes.com/privacy/cookie-policy\#how-do-i-manage-trackers}{Your
  Ad Choices}
\item
  \href{https://www.nytimes.com/privacy}{Privacy}
\item
  \href{https://help.nytimes.com/hc/en-us/articles/115014893428-Terms-of-service}{Terms
  of Service}
\item
  \href{https://help.nytimes.com/hc/en-us/articles/115014893968-Terms-of-sale}{Terms
  of Sale}
\item
  \href{https://spiderbites.nytimes.com}{Site Map}
\item
  \href{https://help.nytimes.com/hc/en-us}{Help}
\item
  \href{https://www.nytimes.com/subscription?campaignId=37WXW}{Subscriptions}
\end{itemize}
