Sections

SEARCH

\protect\hyperlink{site-content}{Skip to
content}\protect\hyperlink{site-index}{Skip to site index}

\href{https://www.nytimes.com/section/technology}{Technology}

\href{https://myaccount.nytimes.com/auth/login?response_type=cookie\&client_id=vi}{}

\href{https://www.nytimes.com/section/todayspaper}{Today's Paper}

\href{/section/technology}{Technology}\textbar{}Bytedance of China Eyes
\$75 Billion Valuation, Joining Start-Up Giants

\url{https://nyti.ms/2NKVlUl}

\begin{itemize}
\item
\item
\item
\item
\item
\end{itemize}

Advertisement

\protect\hyperlink{after-top}{Continue reading the main story}

Supported by

\protect\hyperlink{after-sponsor}{Continue reading the main story}

\hypertarget{bytedance-of-china-eyes-75-billion-valuation-joining-start-up-giants}{%
\section{Bytedance of China Eyes \$75 Billion Valuation, Joining
Start-Up
Giants}\label{bytedance-of-china-eyes-75-billion-valuation-joining-start-up-giants}}

\includegraphics{https://static01.nyt.com/images/2018/09/28/business/29bytedance/merlin_140511993_590a829f-e07a-46ba-9f7e-81808f06f545-articleLarge.jpg?quality=75\&auto=webp\&disable=upscale}

By \href{https://www.nytimes.com/by/raymond-zhong}{Raymond Zhong}

\begin{itemize}
\item
  Sept. 28, 2018
\item
  \begin{itemize}
  \item
  \item
  \item
  \item
  \item
  \end{itemize}
\end{itemize}

HONG KONG --- In the United States, the fortunes of giant social media
companies such as Facebook and Twitter have wobbled as they have been
blamed for abetting abusive discourse and hosting malicious political
influence campaigns.

But in China, the titans of social media are still getting bigger and
richer.

Bytedance, the creator of the news aggregator Jinri Toutiao, the
video-sharing service Tik Tok and a fleet of other entertainment apps,
is in discussions to raise new funding that would value the business at
\$75 billion, according to people familiar with the matter who requested
anonymity to discuss confidential talks.

The Japanese conglomerate SoftBank is among the investors involved in
the talks, one of these people said.

A \$75 billion valuation would make Bytedance one of the world's most
valuable private tech companies --- Uber was
\href{https://www.nytimes.com/2018/08/27/technology/uber-toyota-partnership.html}{recently
valued at \$76 billion}. It would also give Bytedance financial heft
comparable to that of some publicly traded high-tech leaders. Baidu,
maker of China's dominant search engine, has a market capitalization of
around \$80 billion.

A spokeswoman for Bytedance declined to comment. The talks with SoftBank
were reported earlier by The Information.

In the six years since the company's founding, Bytedance's apps have
become a major draw for users --- and advertisers --- in the world's
largest online population. And the company wants to conquer phone
screens far beyond China, too.

Last year,
\href{https://www.nytimes.com/2017/11/10/business/dealbook/musically-sold-app-video.html}{it
bought Musical.ly}, a video-based social network popular with teenagers
in the United States and Europe, and folded it into Tik Tok. Musical.ly
had been notable for being one of the rare Chinese social media
companies
\href{https://www.nytimes.com/2016/08/10/technology/china-homegrown-internet-companies-rest-of-the-world.html}{to
have attracted a significant following outside China}.

Today, half a billion people worldwide use either Tik Tok or its Chinese
edition, Douyin, every month, Bytedance says. By comparison, Instagram,
where short, homemade videos are also a major draw, has more than a
billion users.

Coming from China, where the government maintains strict and
ever-changing controls on information, Bytedance has weathered unique
turbulence as it has grown. Last year, the company had to suspend
updates to its news app Toutiao after China's internet regulator accused
it of spreading unsavory material. This April, a Bytedance app for
sharing bawdy jokes and videos was
\href{https://www.nytimes.com/2018/04/11/technology/china-toutiao-bytedance-censor.html}{ordered
offline entirely}.

The authorities in Indonesia briefly banned Tik Tok this year for
hosting ``negative'' content.

Zhang Yiming, Bytedance's founder and chief executive, offered a display
of contrition after the joke app, Neihan Duanzi,
\href{https://www.nytimes.com/2018/04/12/business/china-bytedance-duanzi-censor.html}{was
shuttered}. He apologized for the company's failure to respect ``core
socialist values,'' and thanked China's government for enabling tech
start-ups' rapid development.

Mr. Zhang also said that Bytedance would expand its team for monitoring
content to 10,000 people from 6,000.

Advertisement

\protect\hyperlink{after-bottom}{Continue reading the main story}

\hypertarget{site-index}{%
\subsection{Site Index}\label{site-index}}

\hypertarget{site-information-navigation}{%
\subsection{Site Information
Navigation}\label{site-information-navigation}}

\begin{itemize}
\tightlist
\item
  \href{https://help.nytimes.com/hc/en-us/articles/115014792127-Copyright-notice}{©~2020~The
  New York Times Company}
\end{itemize}

\begin{itemize}
\tightlist
\item
  \href{https://www.nytco.com/}{NYTCo}
\item
  \href{https://help.nytimes.com/hc/en-us/articles/115015385887-Contact-Us}{Contact
  Us}
\item
  \href{https://www.nytco.com/careers/}{Work with us}
\item
  \href{https://nytmediakit.com/}{Advertise}
\item
  \href{http://www.tbrandstudio.com/}{T Brand Studio}
\item
  \href{https://www.nytimes.com/privacy/cookie-policy\#how-do-i-manage-trackers}{Your
  Ad Choices}
\item
  \href{https://www.nytimes.com/privacy}{Privacy}
\item
  \href{https://help.nytimes.com/hc/en-us/articles/115014893428-Terms-of-service}{Terms
  of Service}
\item
  \href{https://help.nytimes.com/hc/en-us/articles/115014893968-Terms-of-sale}{Terms
  of Sale}
\item
  \href{https://spiderbites.nytimes.com}{Site Map}
\item
  \href{https://help.nytimes.com/hc/en-us}{Help}
\item
  \href{https://www.nytimes.com/subscription?campaignId=37WXW}{Subscriptions}
\end{itemize}
