Sections

SEARCH

\protect\hyperlink{site-content}{Skip to
content}\protect\hyperlink{site-index}{Skip to site index}

\href{https://myaccount.nytimes.com/auth/login?response_type=cookie\&client_id=vi}{}

\href{https://www.nytimes.com/section/todayspaper}{Today's Paper}

\href{/section/opinion}{Opinion}\textbar{}Why We Can't Stop Rooting for
Tiger Woods

\href{https://nyti.ms/2QjthnJ}{https://nyti.ms/2QjthnJ}

\begin{itemize}
\item
\item
\item
\item
\item
\end{itemize}

Advertisement

\protect\hyperlink{after-top}{Continue reading the main story}

\href{/section/opinion}{Opinion}

Supported by

\protect\hyperlink{after-sponsor}{Continue reading the main story}

Sporting

\hypertarget{why-we-cant-stop-rooting-for-tiger-woods}{%
\section{Why We Can't Stop Rooting for Tiger
Woods}\label{why-we-cant-stop-rooting-for-tiger-woods}}

He's the most talented golfer ever to play the game. But more than ever
he's showing he's also a human being.

By Jeff Benedict and Armen Keteyian

Mr. Benedict and Mr. Keteyian are the authors of ``Tiger Woods.''

\begin{itemize}
\item
  Sept. 29, 2018
\item
  \begin{itemize}
  \item
  \item
  \item
  \item
  \item
  \end{itemize}
\end{itemize}

\includegraphics{https://static01.nyt.com/images/2018/09/29/opinion/29sportingWeb/merlin_144471564_e9a3fde3-212e-4247-9396-e71aaa1dd494-articleLarge.jpg?quality=75\&auto=webp\&disable=upscale}

Our book, ``Tiger Woods,'' a biography of the grittiest and most
mysterious athlete either of us has encountered in our over 50 years
covering sports, came out in March. The final chapter, set in early
2018, sets the scene for Woods as he prepared to return to golf after
his fourth back surgery. The last sentence read: ``A changed man, he
stood poised to show his children --- and a fresh generation of golf
pros and fans --- just what a living legend looks like.''

In all honesty, when we wrote that, our expectations were low ---~very
low. There is no way we would have predicted the scene last weekend at
the Tour Championship, when hundreds of raucous fans overcome with
euphoria swarmed down the final fairway at East Lake Golf Club in
Atlanta, following Woods like a human sea. We were sent back decades
earlier, to 1997, to his triumphant stroll down the 18th fairway at Cog
Hill during the Western Open.

After he tapped in his final putt to win his first P.G.A. tournament in
five years, the raucous crowd chanted ``TI-GER TI-GER.''

At age 42, Woods is writing a new chapter with himself as the central
character in the greatest comeback in the history of sports. But this
comeback is bigger than golf. Woods has emerged, in our deeply divided
country, as a symbol of unity and admiration. When he was announced at
the opening ceremony for this weekend's Ryder Cup at Le Golf National in
France, the enormous crowd gave him a lengthy,
\href{https://www.golfdigest.com/story/ryder-cup-2018-watch-tiger-woods-get-a-long-standing-ovation-at-ryder-cup-opening-ceremony}{spirited
standing ovation}, chanting his name in French amid the waving of
American flags.

As recently as a year ago, we could hardly find anyone who genuinely
believed that Woods would play again on the P.G.A. Tour, much less win.
Woods himself confided to a former Masters winner at the Champions
Dinner at Augusta in April 2017 that he was ``done.'' Nine surgeries
will do that to most people.

But Woods, as we found in our extensive research on him and interviews
with hundreds of people from every facet of his life, has an
unparalleled determination to persevere through disappointment,
adversity and pain. His comeback began this year, and as he found his
swing, he quietly moved up in the World Golf Rankings. Remarkably, in
July he briefly led the British Open with eight holes to play. In August
he finished two shots behind the winner at the P.G.A. Championship.

How unlikely was this? Just 16 months ago, Woods was found on the side
of the road near his home in Florida, asleep behind the wheel of his car
in the middle of the night. He was arrested for D.U.I., and a toxicology
report subsequently revealed that he had taken a potentially lethal
combination of drugs that included the painkillers Vicodin and Dilaudid,
the anti-anxiety medication Xanax and the sleep aid Ambien. His mug shot
ricocheted around the world.

He had hit rock bottom, even lower than the stretch in 2009, following
his infidelity scandal, when his name appeared on the cover of The New
York Post for 21 consecutive days, surpassing the previous record of 20
consecutive covers devoted to the Sept. 11 terror attacks.

Woods hasn't won a major championship since 2008. It erased his aura of
invincibility. Everyone started to look at him differently. He was still
the most talented golfer ever to play the game, but he was also
fallible, a human being with weaknesses and frailties.

Even worse, for a professional athlete, his body broke down. His arrest
last year was ultimately the result of misusing highly addictive pain
medications. He has acknowledged and overcome that situation as well.

The more we thought about what he has been through, the more we admired
him. For so many years, when Woods was at the pinnacle of his sport, he
came off more like a machine than a man. He dispatched opponents with
the cold precision of a trained assassin. He had a distant, often surly
relationship with the press. He kept fans at a distance. He was so much
better than everyone else and so single-minded that spectators couldn't
really relate to him.

But pain has changed Woods. The first visible sign of this came in
January at the Farmers Insurance Open at Torrey Pines. It was his first
P.G.A. tour event since his latest back surgery. As he walked off the
green on the 13th hole on the first day of the tournament, he looked up
and noticed the abundance of military personnel packed into the stands
behind the green. A Marine in full dress had been holding the flag as
players putted out. It was Torrey's tribute to a military town. Pausing,
Woods acknowledged the men and women in uniform.

``He never, ever, ever would have done that before,'' said a tour
insider who witnessed the moment. ``He would have had his head down and
not seen a thing. Now he was looking up and taking it all in, smelling
the roses.''

Similarly, after winning last weekend, Woods admitted, ``I had a hard
time not crying on the last hole.''

That sentence, more than anything, reveals the new Tiger Woods. He is
still the most talented golfer ever to play the game and remains the
toughest competitor we've ever seen. But more than ever he's showing
he's also a human being.

Jeff Benedict (\href{https://twitter.com/authorjeff}{@authorjeff}) ****
and Armen Keteyian
(\href{https://twitter.com/ArmenKeteyian}{@ArmenKeteyian}) are the
authors of
``\href{http://www.simonandschuster.com/books/Tiger-Woods/Jeff-Benedict/9781501126420}{Tiger
Woods}.''

\emph{Follow The New York Times Opinion section on}
\href{https://www.facebook.com/nytopinion}{\emph{Facebook}} \emph{and}
\href{http://twitter.com/NYTOpinion}{\emph{Twitter
(@NYTopinion)}}\emph{, and sign up for the}
\href{http://www.nytimes.com/newsletters/opiniontoday/}{**}
\emph{Opinion Today newsletter.}

Advertisement

\protect\hyperlink{after-bottom}{Continue reading the main story}

\hypertarget{site-index}{%
\subsection{Site Index}\label{site-index}}

\hypertarget{site-information-navigation}{%
\subsection{Site Information
Navigation}\label{site-information-navigation}}

\begin{itemize}
\tightlist
\item
  \href{https://help.nytimes.com/hc/en-us/articles/115014792127-Copyright-notice}{©~2020~The
  New York Times Company}
\end{itemize}

\begin{itemize}
\tightlist
\item
  \href{https://www.nytco.com/}{NYTCo}
\item
  \href{https://help.nytimes.com/hc/en-us/articles/115015385887-Contact-Us}{Contact
  Us}
\item
  \href{https://www.nytco.com/careers/}{Work with us}
\item
  \href{https://nytmediakit.com/}{Advertise}
\item
  \href{http://www.tbrandstudio.com/}{T Brand Studio}
\item
  \href{https://www.nytimes.com/privacy/cookie-policy\#how-do-i-manage-trackers}{Your
  Ad Choices}
\item
  \href{https://www.nytimes.com/privacy}{Privacy}
\item
  \href{https://help.nytimes.com/hc/en-us/articles/115014893428-Terms-of-service}{Terms
  of Service}
\item
  \href{https://help.nytimes.com/hc/en-us/articles/115014893968-Terms-of-sale}{Terms
  of Sale}
\item
  \href{https://spiderbites.nytimes.com}{Site Map}
\item
  \href{https://help.nytimes.com/hc/en-us}{Help}
\item
  \href{https://www.nytimes.com/subscription?campaignId=37WXW}{Subscriptions}
\end{itemize}
