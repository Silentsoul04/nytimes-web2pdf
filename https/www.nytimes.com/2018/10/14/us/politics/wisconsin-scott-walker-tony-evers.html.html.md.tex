Sections

SEARCH

\protect\hyperlink{site-content}{Skip to
content}\protect\hyperlink{site-index}{Skip to site index}

\href{https://www.nytimes.com/section/politics}{Politics}

\href{https://myaccount.nytimes.com/auth/login?response_type=cookie\&client_id=vi}{}

\href{https://www.nytimes.com/section/todayspaper}{Today's Paper}

\href{/section/politics}{Politics}\textbar{}Democrats Want to Beat Scott
Walker. But the Wisconsin Economy Is a Hurdle.

\url{https://nyti.ms/2A9Mwv6}

\begin{itemize}
\item
\item
\item
\item
\item
\item
\end{itemize}

\begin{itemize}
\item
  \href{https://www.nytimes.com/2020/07/31/us/elections/biden-vs-trump.html?action=click\&pgtype=Article\&state=default\&region=TOP_BANNER\&context=storylines_menu}{Election
  Updates}
\item
  \href{https://www.nytimes.com/article/biden-vice-president-2020.html?action=click\&pgtype=Article\&state=default\&region=TOP_BANNER\&context=storylines_menu}{Biden's
  V.P. Search}
\item
  \href{https://www.nytimes.com/interactive/2020/07/24/us/politics/trump-biden-campaign-donors.html?action=click\&pgtype=Article\&state=default\&region=TOP_BANNER\&context=storylines_menu}{Map
  of Donations}
\item
  \href{https://www.nytimes.com/interactive/2020/us/elections/delegate-count-primary-results.html?action=click\&pgtype=Article\&state=default\&region=TOP_BANNER\&context=storylines_menu}{Delegate
  Count}
\item
  \href{https://www.nytimes.com/interactive/2019/us/politics/2020-presidential-candidates.html?action=click\&pgtype=Article\&state=default\&region=TOP_BANNER\&context=storylines_menu}{The
  Candidates}
\item
  \href{https://www.nytimes.com/newsletters/politics?action=click\&pgtype=Article\&state=default\&region=TOP_BANNER\&context=storylines_menu}{Politics
  Newsletter}
\end{itemize}

Advertisement

\protect\hyperlink{after-top}{Continue reading the main story}

Supported by

\protect\hyperlink{after-sponsor}{Continue reading the main story}

\hypertarget{democrats-want-to-beat-scott-walker-but-the-wisconsin-economy-is-a-hurdle}{%
\section{Democrats Want to Beat Scott Walker. But the Wisconsin Economy
Is a
Hurdle.}\label{democrats-want-to-beat-scott-walker-but-the-wisconsin-economy-is-a-hurdle}}

\includegraphics{https://static01.nyt.com/images/2018/10/15/us/politics/15wisconsin-rebound1/merlin_144361422_add5a089-098c-4fe4-b29f-2bf8826f294b-articleLarge.jpg?quality=75\&auto=webp\&disable=upscale}

By \href{https://www.nytimes.com/by/monica-davey}{Monica Davey} and
\href{https://www.nytimes.com/by/nelson-d-schwartz}{Nelson D. Schwartz}

\begin{itemize}
\item
  Oct. 14, 2018
\item
  \begin{itemize}
  \item
  \item
  \item
  \item
  \item
  \item
  \end{itemize}
\end{itemize}

RACINE, Wis. --- This city's downtown was all but empty on a recent
Sunday afternoon, but one storefront office was so packed with Democrats
that people had to wait outside.

One by one, the party's top Wisconsin candidates took a microphone and
fixated on a single villain --- a Republican who drew jeers from the
campaign volunteers preparing to make phone calls and walk door to door
with clipboards and pleas for votes.

The target? Not President Trump, who drew only passing mention. It was
Scott Walker, the state's governor. His eight years in office are the
Democrats' greatest weapon this fall, even as his economic record has
become their greatest complication.

``Forty days is not enough to talk about all the awful things Scott
Walker's done,''
\href{https://www.jsonline.com/story/news/politics/elections/2018/08/15/media-reports-mandela-barnes-dead-white-and-not-ballot/982165002/}{Mandela
Barnes}, the Democrats' nominee for lieutenant governor, called out.

Randy Bryce, a mustachioed former ironworker who hopes to seize the seat
that House Speaker Paul Ryan is leaving, mocked the ``Scott-holes'' that
he said plague Wisconsin's roads.

And Tony Evers, a grandfatherly-looking state schools superintendent who
is running against Mr. Walker, said that it was high time they hold the
governor accountable for cuts to schools, rising health care costs and a
state economy that may look dazzling in headlines but, Mr. Evers says,
doesn't always feel that way to residents.

Wisconsin, which had not picked a Republican for president since 1984,
shocked the country in 2016 by backing Mr. Trump. In hindsight, it
shouldn't have been such a surprise: Mr. Walker and the
Republican-controlled Legislature were re-elected in 2014 after slashing
taxes, and many Republicans, independents and even some fiscally-minded
Democrats saw benefit in a firmer line on the size and costs of
government, not to mention lower tax bills.

But now there's a rising debate over whether this state needs more than
Mr. Walker's unbending rectitude. One question for Democrats is whether
they can successfully make an economic argument at a time when
Wisconsin's economic indicators are strong. By most metrics, Wisconsin's
economy is doing well. At 3 percent, the state's unemployment rate is
well below the national average of 3.7 percent.

Nationally, Democratic leaders are watching Wisconsin closely, in part
to understand how to run against a relatively upbeat economy, and in
part for lessons for winning back the state in the 2020 presidential
election.

As in other once-blue states, the Trump victory in Wisconsin led to new
energy among Democrats this year,
\href{https://www.nytimes.com/2018/04/03/us/wisconsin-election-supreme-court.html}{turning
a state Supreme Court seat over to a liberal candidate}, electing
Democrats in
\href{https://www.nytimes.com/2018/01/17/us/wisconsin-elections-state-senate.html}{special
elections} to two state legislative seats that had long been held by
Republicans, and drawing a
\href{https://www.jsonline.com/story/news/blogs/wisconsin-voter/2018/08/15/democrats-outdraw-republicans-wisconsin-primary-turnout-nears-1-million/996379002/}{higher
primary election voter turnout in August} than the Republicans saw.

But for the moment, just like
\href{https://www.nytimes.com/2018/08/15/us/politics/democrats-house-midterm-campaign.html}{party
candidates in some other places}, many Wisconsin Democrats are running
hard against Mr. Walker and the Republican establishment that has moved
the state firmly to the right, and less overtly against President Trump.
Part of it is calculation: Avoid alienating independent voters who may
have voted for Mr. Trump.

\includegraphics{https://static01.nyt.com/images/2018/10/15/us/politics/15wisconsin-rebound2/merlin_144360477_98a016f4-d076-48da-bb18-a30b1190b416-articleLarge.jpg?quality=75\&auto=webp\&disable=upscale}

Kriss Marion, an organic vegetable farmer and Democrat who is running
for a Wisconsin State Senate seat that Democrats have targeted as one
they might flip, said the president's name rarely comes up when she is
knocking at doors in the small towns and rural, rolling hills of her
senate district, which Mr. Trump won in 2016.

\hypertarget{latest-updates-2020-election}{%
\section{\texorpdfstring{\href{https://www.nytimes.com/2020/07/31/us/elections/biden-vs-trump.html?action=click\&pgtype=Article\&state=default\&region=MAIN_CONTENT_1\&context=storylines_live_updates}{Latest
Updates: 2020
Election}}{Latest Updates: 2020 Election}}\label{latest-updates-2020-election}}

Updated 2020-08-01T01:26:45.732Z

\begin{itemize}
\tightlist
\item
  \href{https://www.nytimes.com/2020/07/31/us/elections/biden-vs-trump.html?action=click\&pgtype=Article\&state=default\&region=MAIN_CONTENT_1\&context=storylines_live_updates\#link-29fdff45}{Kamala
  Harris, a top vice-presidential contender, confronts double
  standards.}
\item
  \href{https://www.nytimes.com/2020/07/31/us/elections/biden-vs-trump.html?action=click\&pgtype=Article\&state=default\&region=MAIN_CONTENT_1\&context=storylines_live_updates\#link-13ec3d9c}{Karen
  Bass and Susan Rice are rising on Biden's vice-presidential
  shortlist.}
\item
  \href{https://www.nytimes.com/2020/07/31/us/elections/biden-vs-trump.html?action=click\&pgtype=Article\&state=default\&region=MAIN_CONTENT_1\&context=storylines_live_updates\#link-49e9a016}{Trump
  says Russian bounties to kill U.S. troops `never took place.'}
\end{itemize}

\href{https://www.nytimes.com/2020/07/31/us/elections/biden-vs-trump.html?action=click\&pgtype=Article\&state=default\&region=MAIN_CONTENT_1\&context=storylines_live_updates}{See
more updates}

``I would never bring that up,'' she said. ``I think it's a
distraction.''

After years of setbacks, including a failed effort
\href{https://www.nytimes.com/2012/06/06/us/politics/walker-survives-wisconsin-recall-effort.html}{to
remove Mr. Walker in a recall election}, Wisconsin Democrats' dreams are
suddenly vast. They hope not only to hold onto Tammy Baldwin's United
States Senate seat but to win the governor's office and control of the
State Senate. With redistricting and advantages going into the 2020
presidential race at stake, Democrats have fielded candidates for more
state legislative seats than in recent memory.

But the Democrats' most challenging problem may not be on the ballot.

``Wisconsin is working,'' Mr. Walker told reporters about the economy,
after a campaign event not long ago. ``Democrats are trying to tell
something to people that's not true with what they actually know and
see.''

He added: ``We can't afford to turn around now.''

\hypertarget{a-blue-wave-meets-scott-walker}{%
\subsection{A `blue wave' meets Scott
Walker}\label{a-blue-wave-meets-scott-walker}}

Image

Wisconsin Democrats running for state office are focusing their efforts
against Gov. Scott Walker and the Republican establishment that has
moved the state firmly to the right, and talking less overtly about
President Trump and Washington.Credit...Lauren Justice for The New York
Times

For months, as Wisconsin residents debated the possibility of a ``blue
wave'' of Democratic votes on Election Day, one unlikely voice has
repeatedly warned that it could happen: Scott Walker himself.

Mr. Walker, who is 50 and seeking a third term in a state where nearly
everyone already has an opinion of him, rocked up and down in his
sneakers the other day as he waited for his introduction at the edge of
a stage in Burlington. He looked impatient to get to work.

Wearing an Aaron Rodgers' Green Bay Packers jersey, Mr. Walker told the
crowd of supporters something that most of them might have argued with
had it come from anyone else.

The polls show a tight race, Mr. Walker warned the crowd, and they
should be believed.

``Don't explain the polls by saying they're wrong,'' he said. ``Explain
the polls by saying that's all the motivation we need to talk to more
voters.''

``The only way we fail to win,'' he said, ``is if the truth gets blurred
over.''

Last week, the \href{https://law.marquette.edu/poll/}{Marquette Law
School Poll} found Mr. Walker a percentage point ahead of Mr. Evers
among likely voters. The poll showed Ms. Baldwin with a 10-point lead in
her Senate race against Leah Vukmir, a Republican ally of Mr. Walker
from the Legislature.

Mr. Walker, a former lawmaker and county executive who ran for president
briefly and knows Wisconsin politics like few others, has countered
shows of mounting Democratic strength with a fierce, unrelenting
campaign: The state Republican Party ran an ominous-sounding
\href{https://www.youtube.com/watch?v=75xbRWAqqrI}{ad} accusing Mr.
Evers, in his role as a schools leader, of failing to properly
discipline teachers accused of sexual misconduct; the Republicans have
issued sharp warnings that Democrats would return the state to days of
high taxes and budget deficits; and, most of all, Mr. Walker ticks off
Wisconsin's economic successes every chance he gets.

In addition to the state's 3 percent unemployment rate, wage growth is
picking up speed after lagging the rest of the country for much of 2016
and 2017. Average hourly earnings in the state are up five percent from
a year ago, compared with a 2.9 percent increase nationally.

``It's hard to argue we need a change economically as people are doing
well,'' said Noah Williams, director of
the\href{https://crowe.wisc.edu/}{Center for Research on the Wisconsin
Economy} at the University of Wisconsin --- Madison.

The turnaround was a long time coming, however. The state lost more than
175,000 jobs between 2008 and 2010 and it took until 2015 for total
employment to get back to where it had been before the recession.

One of Mr. Walker's campaign promises during his initial run for
governor in 2010 was to create 250,000 jobs in his first term, which
ended in early 2015. He reached that goal in April 2018, more than seven
years after taking office.

``Most people feel that the state is really doing well,'' said Rosanne
Hahn, a retired teacher who conferred with Mr. Walker outside his
Burlington event and said she plans to vote for him. ``A lot of times
there are many jobs and people won't take them,'' Ms. Hahn said.

\hypertarget{running-against-the-walker-economy}{%
\subsection{Running against the Walker
economy}\label{running-against-the-walker-economy}}

Image

Mr. Evers won the nomination in a wide field of primary candidates. Some
supporters say he is a safe choice to keep the campaign focus not on
himself --- but on Mr. Walker's legacy.Credit...Lauren Justice for The
New York Times

Mr. Evers, 66, was a teacher and principal before he became the state's
Superintendent of Public Instruction, and he looks the part --- a pen at
the ready in his dress shirt pocket, a flop of white hair and a gentle
smile.

His name is widely known in the state, but his political style is
factual and bland, a concern for some Democratic voters who say they
wonder whether a more inspirational candidate could better seize the
political moment. But some strategists say Mr. Evers, who won the
nomination in a wide field of primary candidates, is a safe, ideal
choice for keeping the focus not on himself --- but on Mr. Walker's
legacy.

By Mr. Evers's telling, the state's economic picture may be the
centerpiece of Mr. Walker's campaign, but that doesn't make it a less
potent argument for the Democrats --- largely, he says, because people
don't feel like things are as upbeat as the statistics imply.

``He can talk about the unemployment rate until the cows come home,''
Mr. Evers said in an interview. ``Most people are just scraping by, so
that doesn't mean anything to them. Many of the people that are employed
are having to get two or three jobs just make ends meet. Also, we're in
a state that people are leaving because of the decisions he has made.
There's a lot of data out there other than the unemployment rate.''

This is how Wisconsin Democrats are pivoting from a strict economic
pitch to a case that may be easier to make: They say that Mr. Walker's
political philosophy has starved Wisconsin of money for needed services.

When Mr. Walker first entered office, he cut spending on schools.
Gradual increases followed and the most recent budget had an infusion of
new dollars, Tamarine Cornelius, an analyst at the Wisconsin Budget
Project, said. But, she added, the new funds were not enough to make up
for the initial cut, after adjusting for inflation.

Aid to the state's university system has also been cut under Mr. Walker.
In the 2010-11 budget year, state funding totaled \$1.179 billion. It
dropped by \$178 million the following year after he took office, and
stood at \$1.06 billion in 2017-2018.

Under Mr. Walker's watch, the state ranked 44th in a
\href{https://www.usnews.com/news/best-states/wisconsin}{U.S. News and
World Report ranking of road quality}. Adjusted for inflation, in fiscal
2019 Wisconsin's
\href{https://wisconsindot.gov/Documents/about-wisdot/performance/budget/trends2018-2019.pdf}{transportation
budget} remains 20 percent below where it was in 2010.

Even as needed services were deprived of money, Mr. Evers says, the
state was agreeing to a deal that would provide
\href{https://www.nytimes.com/2017/07/27/business/wisconsin-foxconn-tax-subsidies.html}{\$3
billion in tax credits} so that Foxconn, a Taiwanese electronics
company, could build a campus in Southeast Wisconsin. ``We're starving
the school systems but giving this Hail Mary version of economic
development with one company?'' Mr. Evers said. ``It's a terrible
deal.''

Image

The Foxconn headquarters in Milwaukee.Credit...Lauren Justice for The
New York Times

Around the state, Democrats said they had not felt the benefit of the
state's impressive economic statistics. Manufacturing plays an outsize
role here; the factory sector is responsible for 18.6 percent of
economic output, compared with 11.7 percent for the United States as a
whole. Nearly half a million Wisconsin residents work in manufacturing,
up 24,000 since the beginning of 2017, but still shy of the total before
the recession.

``Whoopee!'' Denis Olson, who is 62 and fixes tractors, said of the
state's low unemployment rate. ``If you're looking for a job for a buck
over minimum wage you can find them. Every low-paying job is hiring, but
who can live on that?''

\hypertarget{the-state-senate-up-for-grabs}{%
\subsection{The State Senate up for
grabs?}\label{the-state-senate-up-for-grabs}}

Image

Kriss Marion, an organic vegetable farmer and Democrat, is running for a
southwest Wisconsin State Senate seat that the party has targeted as one
they might flip.Credit...Lauren Justice for The New York Times

In Blanchardville, a village of just more than 800 people, homecoming
royalty and homemade floats rolled down Main Street on a sunny, windy
Saturday not long ago. One float bore a canoe and a sign that read
``Come Hell or High Water,'' an allusion to widespread and damaging
flooding in parts of Wisconsin in recent months. Supporters of Ms.
Marion, the Democrat running for state senator, handed out postcards
with a
\href{https://www.cbsnews.com/news/wisconsin-recipes-maple-dunkers/}{Maple
Dunker}cookie recipe from Ms. Marion, who raced through the crowd in a
sparkly T-shirt, shaking hands.

In a state that was once seen as mostly blue but often flipped control
back and forth, Democrats view the Wisconsin Senate as one of their most
winnable openings. Republicans hold control of the chamber with an 18-15
margin, and Democrats have poured attention into several seats they see
as particularly competitive.

The district Ms. Marion is trying to seize from a Republican incumbent
includes small towns and dairy farms in a rolling, rural section of
Southwestern Wisconsin that went for Mr. Trump in 2016, but also Barack
Obama before that; it's a place strategists see as a swing district that
could reflect Democrats' strength in the state this year.

Image

The district Ms. Marion is competing for in southwestern Wisconisn voted
for Mr. Trump in 2016, but previously voted for President Barack
Obama.Credit...Lauren Justice for The New York Times

Ms. Marion, a county board member who tends a small farm with a
bed-and-breakfast, drew statewide attention a few years ago as part of
efforts to pass legislation that became known as the ``Cookie Bill.''
State law didn't authorize bakers to sell their goods at farmers'
markets without commercial licenses, and she wanted to change that, but
struggled to get anywhere with the Republican-controlled Legislature.
Eventually, she was among a group that sued. A court ruling opened the
way for such sales, but Ms. Marion saw the events as telling.

``How simple should that have been to get a cookie bill passed?'' she
said. ``It was eye opening in terms of how you think you know what
democracy is. I'm actually not super into baking cookies, but we just
need more opportunities out here.''

Not far from Ms. Marion along Main Street, supporters of her opponent,
Sen. Howard Marklein, the Republican incumbent and an accountant who
grew up on a dairy farm, were handing out Packers' game schedules and
bright yellow Marklein bags to children collecting candy along the
street. Mr. Marklein said that farmers are struggling but that his
constituents tell him they see the state's economy as growing and
vibrant.

``I've knocked on thousands of doors,'' Mr. Marklein said. ``You talk to
businesses, and these are the best years they've ever had.''

\href{https://www.nytimes.com/interactive/2018/09/28/us/politics/the-campaign-reporter-ul.html?src=hpPromoHeadline}{}

\hypertarget{sign-up-for-the-campaign-reporter}{%
\subsection{Sign up for The Campaign
Reporter}\label{sign-up-for-the-campaign-reporter}}

\includegraphics{https://int.nyt.com/newsgraphics/push-interactive/projects/campaign-reporter/avatars/alex_burns.png}

Hey, I'm Alex Burns, a politics correspondent for The Times. Send me
your questions using the NYT app. I'll give you the latest intel from
the campaign trail.

Sign up via push alert

Still, Ms. Marion's pitch is largely about economic survival and quality
of life: Fix the roads. Expand broadband to the rural reaches. Send more
money to rural schools.

Small towns are struggling, she said, with populations shrinking,
schools closing and farmers wrestling with bankruptcies and a growing
suicide problem.

``In towns like this, we're on the edge all the time,'' she said. ``We
want to stay here, but that is by no means a sure thing. It's sort of
like we're at a crossroads. What's going to happen?''

\hypertarget{our-2020-election-guide}{%
\section{Our 2020 Election Guide}\label{our-2020-election-guide}}

Updated July 31, 2020

\begin{itemize}
\item
  \begin{center}\rule{0.5\linewidth}{\linethickness}\end{center}

  \hypertarget{the-latest}{%
  \subsection{The Latest}\label{the-latest}}

  \begin{itemize}
  \tightlist
  \item
    President Trump's assault on the Postal Service is intersecting with
    his attacks on mail-in voting.
    \href{https://www.nytimes.com/2020/07/31/us/politics/trump-usps-mail-delays.html?action=click\&pgtype=Article\&state=default\&region=BELOW_MAIN_CONTENT\&context=storylines_guide}{Voting
    rights groups say it is a recipe for disaster.}
  \end{itemize}
\item
  \begin{center}\rule{0.5\linewidth}{\linethickness}\end{center}

  \hypertarget{bidens-vp-search}{%
  \subsection{Biden's V.P. Search}\label{bidens-vp-search}}

  \begin{itemize}
  \tightlist
  \item
    \href{https://www.nytimes.com/article/biden-vice-president-2020.html?action=click\&pgtype=Article\&state=default\&region=BELOW_MAIN_CONTENT\&context=storylines_guide}{Here
    are 13 women} who have been under consideration to be Joe Biden's
    running mate, and why each might be chosen --- and might not be.
  \end{itemize}
\item
  \begin{center}\rule{0.5\linewidth}{\linethickness}\end{center}

  \hypertarget{keep-up-with-our-coverage}{%
  \subsection{Keep Up With Our
  Coverage}\label{keep-up-with-our-coverage}}

  \begin{itemize}
  \tightlist
  \item
    Get an
    \href{https://www.nytimes.com/newsletters/politics?action=click\&pgtype=Article\&state=default\&region=BELOW_MAIN_CONTENT\&context=storylines_guide}{email}
    recapping the day's news
  \end{itemize}

  \begin{itemize}
  \tightlist
  \item
    Download our mobile app on
    \href{https://apps.apple.com/us/app/nytimes/id284862083?ls=1\&mat_click_id=5c79ae7455014fd1bd66b5610c05b8f2-20191112-16948\&referrer=mat_click_id\%3D5c79ae7455014fd1bd66b5610c05b8f2-20191112-16948\%26link_click_id\%3D722930677036718082}{iOS}
    and
    \href{http://a.localytics.com/android?id=com.nytimes.android\&referrer=utm_source\%3Dother_nyt_mobile_web\%26utm_medium\%3DWeb\%2520page\%26utm_term\%3DGeneral\%2520Mobile\%2520Page\%26utm_campaign\%3DNYT\%2520Mobile\%2520General\%2520Page}{Android}
    and turn on Breaking News and Politics alerts
  \end{itemize}
\end{itemize}

Advertisement

\protect\hyperlink{after-bottom}{Continue reading the main story}

\hypertarget{site-index}{%
\subsection{Site Index}\label{site-index}}

\hypertarget{site-information-navigation}{%
\subsection{Site Information
Navigation}\label{site-information-navigation}}

\begin{itemize}
\tightlist
\item
  \href{https://help.nytimes.com/hc/en-us/articles/115014792127-Copyright-notice}{©~2020~The
  New York Times Company}
\end{itemize}

\begin{itemize}
\tightlist
\item
  \href{https://www.nytco.com/}{NYTCo}
\item
  \href{https://help.nytimes.com/hc/en-us/articles/115015385887-Contact-Us}{Contact
  Us}
\item
  \href{https://www.nytco.com/careers/}{Work with us}
\item
  \href{https://nytmediakit.com/}{Advertise}
\item
  \href{http://www.tbrandstudio.com/}{T Brand Studio}
\item
  \href{https://www.nytimes.com/privacy/cookie-policy\#how-do-i-manage-trackers}{Your
  Ad Choices}
\item
  \href{https://www.nytimes.com/privacy}{Privacy}
\item
  \href{https://help.nytimes.com/hc/en-us/articles/115014893428-Terms-of-service}{Terms
  of Service}
\item
  \href{https://help.nytimes.com/hc/en-us/articles/115014893968-Terms-of-sale}{Terms
  of Sale}
\item
  \href{https://spiderbites.nytimes.com}{Site Map}
\item
  \href{https://help.nytimes.com/hc/en-us}{Help}
\item
  \href{https://www.nytimes.com/subscription?campaignId=37WXW}{Subscriptions}
\end{itemize}
