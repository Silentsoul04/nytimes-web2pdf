Sections

SEARCH

\protect\hyperlink{site-content}{Skip to
content}\protect\hyperlink{site-index}{Skip to site index}

\href{https://www.nytimes.com/section/food}{Food}

\href{https://myaccount.nytimes.com/auth/login?response_type=cookie\&client_id=vi}{}

\href{https://www.nytimes.com/section/todayspaper}{Today's Paper}

\href{/section/food}{Food}\textbar{}April Bloomfield Breaks Her Silence
About Harassment at Her Restaurants

\url{https://nyti.ms/2Ad9yB5}

\begin{itemize}
\item
\item
\item
\item
\item
\item
\end{itemize}

Advertisement

\protect\hyperlink{after-top}{Continue reading the main story}

Supported by

\protect\hyperlink{after-sponsor}{Continue reading the main story}

\hypertarget{april-bloomfield-breaks-her-silence-about-harassment-at-her-restaurants}{%
\section{April Bloomfield Breaks Her Silence About Harassment at Her
Restaurants}\label{april-bloomfield-breaks-her-silence-about-harassment-at-her-restaurants}}

The Spotted Pig chef finally speaks about her role in the abuse scandal
that has enveloped her and her partner, Ken Friedman.

\includegraphics{https://static01.nyt.com/images/2018/10/17/dining/17Bloomfield1/merlin_145011669_0c4ce04e-a06f-4ac7-9036-00585dfb8460-articleLarge.jpg?quality=75\&auto=webp\&disable=upscale}

\href{https://www.nytimes.com/by/julia-moskin}{\includegraphics{https://static01.nyt.com/images/2018/09/25/multimedia/author-julia-moskin/author-julia-moskin-thumbLarge.png}}\href{https://www.nytimes.com/by/kim-severson}{\includegraphics{https://static01.nyt.com/images/2018/06/13/multimedia/author-kim-severson/author-kim-severson-thumbLarge.jpg}}

By \href{https://www.nytimes.com/by/julia-moskin}{Julia Moskin} and
\href{https://www.nytimes.com/by/kim-severson}{Kim Severson}

\begin{itemize}
\item
  Oct. 16, 2018
\item
  \begin{itemize}
  \item
  \item
  \item
  \item
  \item
  \item
  \end{itemize}
\end{itemize}

April Bloomfield sat with feet planted on the floor of a Manhattan hotel
room, head down, grimly staring at her hands, which she twisted together
until her knuckles turned white.

She fell silent for long stretches, trying to explain how she --- one of
the best-known chefs in the United States --- came to be the first woman
in the culinary world accused of victimizing other women since the
\#MeToo movement exploded.

In a
\href{https://www.nytimes.com/2017/12/12/dining/ken-friedman-sexual-harassment.html}{New
York Times article} last December, more than two dozen people who had
worked at her restaurants described a longstanding pattern of
\href{https://www.nytimes.com/2018/08/20/dining/mario-batali-spotted-pig.html}{sexual
harassment} and verbal abuse by Ken Friedman, her business partner. Some
said she knew about his behavior, which included groping employees and
pressuring them for sex, and did nothing to prevent it.

In an instant, Ms. Bloomfield, a
\href{https://guide.michelin.com/us/new-york/spotted-pig/restaurant}{Michelin-starred}
British chef who had built seven thriving restaurants over decades of
work, including the celebrated
\href{https://www.thespottedpig.com/\#home}{Spotted Pig} in the West
Village, watched her world break apart.

Mr. Friedman, who has disputed some accusations but apologized for
behavior that he called ``abrasive, rude and frankly wrong,''
immediately stepped away from all business operations but kept his
six-figure salary. She had the job of managing the rage and distress of
hundreds of current and former employees, and keeping the restaurants
going.

Ms. Bloomfield said nothing in public except for a few stiffly worded
\href{https://www.instagram.com/p/BcqnG7hhWsW/?utm_source=ig_embed}{apologies}
that were widely criticized as inadequate. Lawyers advised silence while
she and Mr. Friedman negotiated the breakup of their restaurant group,
which has yet to be completed.

But silence, she has come to understand, inflicts its own damage. After
months of requests from The Times, she agreed to be interviewed because
she wants to add her voice to the narrative, and start to rebuild her
reputation.

In a penthouse suite at the sleek James hotel in NoMad, Ms. Bloomfield,
44, recently sat for hours going over what happened, flanked by her wife
and her publicist. She said she now understands that her past silence
contributed to the sexual and emotional harassment of people she should
have protected.

``I failed a lot of people,'' she said. ``That's on my shoulders.''

At the same time, Ms. Bloomfield, like her supporters and some former
employees, said she was a casualty herself --- of her own naïveté,
premature success and a manipulative business partner with whom she
became so entangled that for years she could see no way out.

``I felt like I was in a position where he held all the cards,'' she
said of Mr. Friedman, 59. ``He had so much control, and he was so
dominant and powerful, that I didn't feel like if I stepped away that I
would survive.''

She knows, too, that because she benefited from the partnership for
years, what she says about its dysfunction now may not be believed.

Indeed, several former employees declined to be interviewed for this
article, saying they did not want to contribute to any narrative that
might appear to offer her redemption. Others said Ms. Bloomfield herself
was such a harsh and demanding boss that they simply didn't believe she
was afraid of Mr. Friedman.

``She could be scary and intimidating,'' said Katy Severson, a chef who
worked under Ms. Bloomfield at the Spotted Pig for four years. ``She did
lose her temper, especially with people who didn't care enough about the
food.''

But Ms. Severson, like other employees, said she believed Ms.
Bloomfield's behavior was motivated by perfectionism, while Mr. Friedman
was simply aggressive and volatile.

``I did feel like she truly cared and wanted me to be a better chef,''
she said.

\includegraphics{https://static01.nyt.com/images/2018/10/17/dining/17Bloomfield2/merlin_131167697_6cadce6c-7b4b-4021-980d-96d21dcc0ec2-articleLarge.jpg?quality=75\&auto=webp\&disable=upscale}

In her interview, Ms. Bloomfield broke down in tears once: when she
acknowledged the distance between the leader she had hoped to be and the
leader she became.

At \href{http://www.rivercafe.co.uk/}{the River Café}, the London
restaurant where she acquired her most significant culinary training,
she had learned that it was possible to run a kitchen with civility and
respect. But she said that seemed impossible in her own kitchens ---
partly because of the restaurant group's rapid expansion (eight
restaurants on two coasts in 13 years) and constant turnover, but also
because of her quick temper and untamable perfectionism.

``I have had many moments of anger and frustration in the kitchen,'' she
said. ``It's an intense place to be, for me and for anyone there with
me. And sometimes that's gotten in the way, and it's hurt many people.''

Ms. Bloomfield described the arc of her career in America, when she got
a call (via \href{https://www.jamieoliver.com/}{Jamie Oliver}) about a
job opportunity in New York 15 years ago, through the moment last year
when she said she read to her horror, in the Times article, that the
Spotted Pig's third-floor party space was known to some people as ``the
rape room.''

Ms. Bloomfield arrived in New York in 2003 after a full-court press by
Mr. Friedman, who had decided to open a British-style gastro pub in the
West Village, and by his friend and investor Mario Batali, whom several
women have accused of
\href{https://www.nytimes.com/2017/12/11/dining/mario-batali-sexual-misconduct.html}{sexual
harassment} (and in two cases,
\href{https://www.nytimes.com/2018/05/21/dining/mario-batali-sexual-assault.html}{sexual
assault}) at the Spotted Pig and other restaurants. (Mr. Batali has said
his ``behavior was wrong'' and left daily operations of his restaurants,
but
\href{https://www.nytimes.com/2018/05/21/dining/mario-batali-sexual-assault.html}{denied}
engaging in any nonconsensual sex.)

Mr. Friedman, although he had no restaurant experience, was brimming
with confidence and backed by celebrity investors like Jay-Z. Ms.
Bloomfield was a 28-year-old unknown from Birmingham, England, who had
never been to the United States and never been in charge of a kitchen.
``It's hard to believe now how ignorant I was then,'' she said.

Her introduction to Mr. Friedman's vindictive side came, she said, as
they prepared to open the Spotted Pig and she expressed a mild dislike
for some framed posters on the restaurant's walls. He exploded in anger,
threatening to have her work visa revoked if she criticized his taste
again, she said. (Since the Spotted Pig was her sponsor, she would have
lost her ability to legally work in the United States if she were fired.
At that time, she was an employee, not a partner.)

Through a representative, Mr. Friedman denied that he ever threatened
Ms. Bloomfield's work visa. He added that he was ``personally dismayed
by Ms. Bloomfield's unwarranted and false attacks,'' and that he planned
to comment further soon.

Image

Mr. Friedman, who had worked in the music industry, knew how to draw
people in. The Spotted Pig, seen here in 2006, was perpetually crowded.
Credit...Alex di Suvero for The New York Times

Ms. Bloomfield said she realized early on that to survive in this new
job, she needed an old kitchen skill: the ability to appear tough, harsh
and thick-skinned. She, like most chefs at the time, had been trained in
restaurant kitchens where shouting, sexism and slashing insults were the
norm.

``I had never heard of H.R.,'' she said, referring to company
human-resources operations. ``It just didn't exist in the world I came
from.''

Inside, she recalled, she was terrified of being branded a failure in
the restaurant industry, and convinced that Mr. Friedman had the power
to make that happen. She said Mr. Friedman frequently told her that he
was the reason she had become famous and wealthy, and that he could undo
her success with a few phone calls. (Several people have said that Mr.
Friedman often retaliated against former employees by trying to prevent
them from getting jobs in other restaurants.)

Mr. Friedman had worked in the music industry for years, and knew how to
pull a crowd. The night the Spotted Pig opened in 2004, there was a line
around the block. ``At the time, I couldn't understand how that
happened,'' Ms. Bloomfield said.

For the first two years, the ill-equipped kitchen felt to her like a war
zone. ``All I could think of to do was cook faster, and I realize now I
wasn't doing what I should have done: gather all the tools I needed to
be a leader,'' she said. The crowds and the pressure on her only
intensified as the Spotted Pig won a Michelin star, and as the partners
opened new restaurants like the John Dory and the Breslin.

They informally carved up the responsibilities: In general, Ms.
Bloomfield was in charge of everything to do with food, and Mr. Friedman
handled everything to do with guests. Each kept well away from the
other's staff and sphere of influence. This pattern set the stage for
more than a decade of secrets and silence.

Ms. Bloomfield said that at the beginning, Mr. Friedman's staff ---
hosts, servers, bartenders --- seemed happy to work at the Spotted Pig.
``They were making good money, they worked hard and then they got to sit
down and drink and party with the boss and his friends,'' she said.

As the number of employees increased, so did the chaos in Mr. Friedman's
orbit. Apart from the pattern of sexual harassment, dozens of employees
say he constantly berated them for minor infractions, fired and rehired
them at whim, and created a toxic atmosphere of fear and uncertainty.

Ms. Bloomfield said she knew about some of Mr. Friedman's inappropriate
behavior with female staff members because much of it took place
publicly: hugging and flirting were routine. She knew that the third
floor was a place where Mr. Friedman's friends and guests indulged in
alcohol, drugs, and inappropriate behavior, but said she never knew of
incidents there that were coercive or physically abusive.

She said she was not told about episodes in which women employees said
Mr. Friedman groped and kissed them, persuaded them to get into his car
and tried to touch their breasts, and asked them to send him nude
pictures. She said the staff, at Mr. Friedman's direction, also
concealed the extent of his offenses from her. (Multiple employees
confirmed this; others said they did not inform Ms. Bloomfield because
they believed she didn't want to know.)

Still, Ms. Bloomfield was told about some serious incidents, and said
she also confronted Mr. Friedman many times about his unprofessional
behavior and verbal abuse.

``I would tell him that we need to be a better company and that we need
to treat our staff well and that he needed to stop,'' she said. ``I
thought I could change him. I thought if I was talking to him more and
guided him, he would learn because I was the professional one, I was
trying to teach him the way of the industry.''

(Mr. Friedman, through a representative, confirmed that he and Ms.
Bloomfield had discussions of this nature, but that they also included
employees' complaints about ``Ms. Bloomfield's erratic behavior and
verbal abuse.'')

Image

Ms. Bloomfield, left, with Michelle Petrulio, a chef who worked for the
company on and off for 10 years.Credit...Liz Barclay for The New York
Times

He would agree and promise to do better, she said, then continue as if
nothing had happened. And despite the ever-increasing chaos around her
and the rising distress of the staff, she would put her head down and
bury herself in the kitchen.

``It's like I decided to control what I could control,'' she said.

Those closest to her say it was a survival mechanism, not a heartless
act or a business decision. ``She was not a person who was well-versed
in management,'' said Michelle Petrulio, who worked for the partners on
and off for a decade, and was the company's culinary director when news
of the harassment broke. ``She was just as affected by Ken's behavior as
everyone else. She didn't feel strong in that relationship. She felt
fear.''

Many people confirmed that interpretation. Others scoffed at it, saying
it was impossible that Ms. Bloomfield, especially in recent years, did
not know how much power she had as a star chef.

Trish Nelson, a former server who said she experienced years of verbal
abuse from Ms. Bloomfield and sexual harassment from Mr. Friedman and
his friends at the Spotted Pig, including Mr. Batali, said Ms.
Bloomfield ``has always been out for herself. She was a perpetrator in a
lot of this.''

She and others said Ms. Bloomfield wanted the fame and fortune that came
with being a successful chef and restaurateur, but none of the
management responsibility.

``We had a pretty good rapport, and I had a lot of respect for her,''
said Natalie Saibel, a longtime server who emailed a formal complaint in
2015 to Ms. Bloomfield that Mr. Friedman had groped her. Ms. Bloomfield
didn't respond, passing the complaint to a manager, said Ms. Saibel, who
was fired soon afterward. ``That's why it was doubly shocking and
devastating that she did nothing to stop it.''

Ms. Nelson, Ms. Saibel and others said they had told Ms. Bloomfield
about Mr. Friedman's sexual harassment, but the chef seemed unwilling to
get involved. They said that in the kitchen and in the dining room, the
message from both employers was: ``Suck it up. If you can't handle it,
you don't deserve to work here.''

Image

``I failed a lot of people,'' Ms. Bloomfield said. ``That's on my
shoulders.''Credit...Alex Welsh for The New York Times

Ms. Bloomfield said she had tried countless times to hire a human
resources coordinator, so that she and Mr. Friedman would not be the
only recourse for aggrieved employees. When a coordinator was finally
hired in about 2014, she was let go within months --- a decision by Mr.
Friedman that Ms. Bloomfield said she was not consulted or informed
about.

Finally, Ms. Bloomfield began exploring escape routes. She agreed to
open two restaurants in California, she said, in hopes that she could
put a continent between herself and Mr. Friedman. About two years ago,
she said, she began quietly consulting with lawyers and a few trusted
colleagues about how she might free herself.

``She didn't talk about it very much --- that's April --- but she had
always said it was a very tough relationship and not a very fair
relationship,'' said Gavin Kaysen, a chef in Minneapolis and a longtime
friend. But at a dinner the two cooked together in October 2016, more
than a year before the Spotted Pig revelations, he said she had reached
a new level of despair.

``I'd never seen her so defeated in her life,'' he said.

By then, even some of Ms. Bloomfield's most trusted lieutenants and
loyal employees had begun to turn on her. They say she had made too many
promises that she couldn't or didn't keep: that she would right the ship
and stop staff turnover; that she would help them get the money and
recognition they deserved; that she would get Mr. Friedman to stop the
harassment.

Ms. Bloomfield should have known by then that Mr. Friedman would make it
impossible to keep those promises, said Ms. Petrulio, the culinary
director. ``But it's so simple to say now what she should have done
then.''

The chef Traci Des Jardins, a friend of Ms. Bloomfield, said that early
in her own career, she had partnered with a powerful man to create an
acclaimed restaurant.

``Imagine how difficult it would be to be in partnership in your late
20s when you are so naïve and really don't know anything about business
but you have a burning desire to make great food,'' Ms. Des Jardins
said. ``If you walk away, you would have had to walk away from all the
success and a business you put your heart and soul into.''

She cautioned people not to brand Ms. Bloomfield as a collaborator
because of her reputation as a tough boss. For women in restaurant
kitchens in the 1990s, when both of them began cooking, it was the only
way to survive, she said.

``Being a disciplinarian and being tough in the kitchen does not make
you a tormentor.''

However the public ultimately views Ms. Bloomfield, her reputation is
scarred in ways that will inevitably affect her future. In June, she
\href{https://www.nytimes.com/2018/06/06/dining/april-bloomfield-ken-friedman-split.html}{announced}
that she will retain control of the Breslin and the John Dory Oyster Bar
in New York's Ace Hotel, Tosca Cafe in San Francisco and the Hearth \&
Hound in Los Angeles. (Her new partner is a restaurant management
company that provides structures like a human resources department and
formal hiring and firing procedures.)

Mr. Friedman will keep the Spotted Pig. The fate of
\href{https://www.nytimes.com/2017/02/07/dining/white-gold-butchers-review-april-bloomfield-restaurant.html}{White
Gold Butchers}, which has been closed since August, is unclear. Last
week, GFI Hospitality, the developer of New York's Ace Hotel, sued Mr.
Friedman for \$5 million in damages, financial ``misfeasance'' and back
rent connected to the Breslin and the John Dory Oyster Bar.

Meanwhile, Ms. Bloomfield has begun psychotherapy, is receiving
executive coaching, and has repeatedly gathered her current restaurant
staff in order to listen, reassure and apologize. (Through a
representative, Mr. Friedman said that he also has spent time this past
year in therapy, and that he has been ``listening, thinking and learning
from this experience.'')

Ms. Bloomfield has reached out to several chefs for advice too. Tom
Colicchio said he told her, ``You have to do the hard work, and that
doesn't mean put your head down and make good food. This is different
work.''

But a big hurdle remains: contacting Mr. Friedman's victims, who have
become bitter as her silence stretched out for months.

``These women have been hurting and I feel horrified that I've done
wrong by them,'' she said. ``I know I need to hear what happened to
them.''

She plans to start reaching out soon, she said.

Advertisement

\protect\hyperlink{after-bottom}{Continue reading the main story}

\hypertarget{site-index}{%
\subsection{Site Index}\label{site-index}}

\hypertarget{site-information-navigation}{%
\subsection{Site Information
Navigation}\label{site-information-navigation}}

\begin{itemize}
\tightlist
\item
  \href{https://help.nytimes.com/hc/en-us/articles/115014792127-Copyright-notice}{©~2020~The
  New York Times Company}
\end{itemize}

\begin{itemize}
\tightlist
\item
  \href{https://www.nytco.com/}{NYTCo}
\item
  \href{https://help.nytimes.com/hc/en-us/articles/115015385887-Contact-Us}{Contact
  Us}
\item
  \href{https://www.nytco.com/careers/}{Work with us}
\item
  \href{https://nytmediakit.com/}{Advertise}
\item
  \href{http://www.tbrandstudio.com/}{T Brand Studio}
\item
  \href{https://www.nytimes.com/privacy/cookie-policy\#how-do-i-manage-trackers}{Your
  Ad Choices}
\item
  \href{https://www.nytimes.com/privacy}{Privacy}
\item
  \href{https://help.nytimes.com/hc/en-us/articles/115014893428-Terms-of-service}{Terms
  of Service}
\item
  \href{https://help.nytimes.com/hc/en-us/articles/115014893968-Terms-of-sale}{Terms
  of Sale}
\item
  \href{https://spiderbites.nytimes.com}{Site Map}
\item
  \href{https://help.nytimes.com/hc/en-us}{Help}
\item
  \href{https://www.nytimes.com/subscription?campaignId=37WXW}{Subscriptions}
\end{itemize}
