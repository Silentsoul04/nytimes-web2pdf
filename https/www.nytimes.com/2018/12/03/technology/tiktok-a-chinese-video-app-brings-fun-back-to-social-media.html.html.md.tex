Sections

SEARCH

\protect\hyperlink{site-content}{Skip to
content}\protect\hyperlink{site-index}{Skip to site index}

\href{https://www.nytimes.com/section/technology}{Technology}

\href{https://myaccount.nytimes.com/auth/login?response_type=cookie\&client_id=vi}{}

\href{https://www.nytimes.com/section/todayspaper}{Today's Paper}

\href{/section/technology}{Technology}\textbar{}TikTok, a Chinese Video
App, Brings Fun Back to Social Media

\url{https://nyti.ms/2E0RQTw}

\begin{itemize}
\item
\item
\item
\item
\item
\end{itemize}

Advertisement

\protect\hyperlink{after-top}{Continue reading the main story}

Supported by

\protect\hyperlink{after-sponsor}{Continue reading the main story}

The Shift

\hypertarget{tiktok-a-chinese-video-app-brings-fun-back-to-social-media}{%
\section{TikTok, a Chinese Video App, Brings Fun Back to Social
Media}\label{tiktok-a-chinese-video-app-brings-fun-back-to-social-media}}

\href{https://www.nytimes.com/by/kevin-roose}{\includegraphics{https://static01.nyt.com/images/2018/02/20/multimedia/author-kevin-roose/author-kevin-roose-thumbLarge.jpg}}

By \href{https://www.nytimes.com/by/kevin-roose}{Kevin Roose}

\begin{itemize}
\item
  Dec. 3, 2018
\item
  \begin{itemize}
  \item
  \item
  \item
  \item
  \item
  \end{itemize}
\end{itemize}

\href{https://cn.nytimes.com/technology/20181204/tiktok-a-chinese-video-app-brings-fun-back-to-social-media/}{查看本文中文版}\href{https://cn.nytimes.com/technology/20181204/tiktok-a-chinese-video-app-brings-fun-back-to-social-media/zh-hant/}{閱讀繁體中文版}

About an hour after downloading TikTok, the popular video-sharing app, I
experienced a bizarre sensation, one I haven't felt in a long time while
on the internet. The knot in my chest loosened, my head felt injected
with helium, and the corners of my mouth crept upward into a smile.

Was this \ldots{} \emph{happiness}?

TikTok --- a quirky hybrid of Snapchat, the defunct video app Vine and
the TV segment ``Carpool Karaoke'' --- is a refreshing outlier in the
social media universe.

A so-called challenge video on TikTok, from the user @spellmantwinz,
which encouraged users to make videos of themselves chomping down on
food to the beat of a song.

There are no ads. There's no news, unless you count learning about viral
dance crazes. There are few preening Instagram models hawking
weight-loss tea, and a distinct lack of crazy uncles posting Infowars
clips.

Instead, TikTok --- a Chinese-made app that was known as Musical.ly
until ByteDance, the Chinese internet conglomerate, acquired the company
in 2017 and merged it with a video app it owned --- has a simple
premise. Users create short videos set to music, often lip-syncing
along, dancing or acting out short skits. The app contains templates and
visual effects to spice up the videos. There is also a live-streaming
feature that allows users to send virtual ``gifts'' to their favorite
creators, which can be bought with real money. The rest works like any
other social app --- followers, hashtags, likes and comments.

It doesn't sound like much. But, somehow, it adds up to what might well
be the only truly pleasant social network in existence.

I feel comfortable making that call because I go on social networks for
a living, and I have spent thousands of hours wading through an unholy
slurry of Twitter spammers, Instagram scammers, teenage YouTube fascists
and baby boomers whose brains have been turned to pudding by too many
Facebook memes.

TikTok has none of that. Instead, it's that rarest of internet
creatures: a place where people can let down their guards, act silly
with their friends and sample the fruits of human creativity without
being barraged by abusive trolls or algorithmically amplified
misinformation. It's a throwback to a time before the commercialization
of internet influence, when web culture consisted mainly of harmless
weirdos trying to make each other laugh.

``It's a bit of an escape,'' said Billy Mann, a
\href{https://m.tiktok.com/h5/share/usr/160249092609150976.html}{TikTok
creator} who uses the platform to make comedy videos for his more than
650,000 followers.

``It's a safe haven for people that are seeing the world on fire and
being like, `I need silliness,''' he said.

TikTok's earnest goofiness has
\href{https://theoutline.com/post/6383/tiktok-will-never-replace-vine}{turned
off} some skeptics. But it's hard to argue with the numbers. The app
\href{https://www.theverge.com/2018/11/15/18095446/tiktok-jimmy-fallon-tony-hawk-downloads-revenue}{recently
passed} six million users in the United States, according to a report
from the market research firm Sensor Tower. As of Friday, it ranked No.
4 among free apps in Apple's app store, ahead of Snapchat, Netflix and
Facebook Messenger. Globally, the app, whose Chinese version is called
Douyin, had
\href{http://www.chinadaily.com.cn/a/201807/17/WS5b4d6057a310796df4df6e3c.html}{500
million monthly active users} as of July, making it bigger than Twitter
and about half the size of Instagram.

Currently, TikTok makes money through virtual gift sales and brand
collaborations, such as a Guess-sponsored
``\href{https://www.businesswire.com/news/home/20180831005348/en/GUESS-TikTok-Launch-First-of-Its-Kind-Fashion-Partnership}{fashion
takeover}.'' There are no ads inside the app, although the company's
\href{https://www.tiktok.com/i18n/privacy/\#how-use}{privacy policy}
leaves room for them in the future. TikTok, which is privately held,
does not disclose financial information, but Sensor Tower
\href{https://sensortower.com/blog/tiktok-revenue}{estimates} that it
took in roughly \$3.5 million in October.

TikTok's success has spawned legions of influencers, users with millions
of followers and household-name status among teenagers. And it has
propelled ByteDance, which also owns a suite of other social media and
news apps, to a
\href{https://www.nytimes.com/2018/09/28/technology/bytedance-fundraising-toutiao-tiktok.html}{reported
valuation} of \$75 billion, making it one of the most valuable start-ups
in the world.

TikTok's global head of marketing, Stefan Heinrich, said in a statement
that the company's mission was to ``capture and present the world's
creativity, knowledge and moments that matter, directly from the mobile
phone.''

In perhaps the clearest sign that TikTok is on to something, Facebook is
trying to kill it. Last month, the company
\href{https://www.theverge.com/2018/11/9/18080280/facebook-lasso-tiktok-competitor-app}{quietly
released Lasso}, a clunky clone that borrowed many of TikTok's core
features and even tried to siphon off some of its power users. Lasso got
off to a slow start, and is now the 687th most downloaded photo and
video app in the United States, according to the mobile data company
AppAnnie. The executive leading the Lasso team, Brady Voss,
\href{https://techcrunch.com/2018/11/13/facebook-lasso/}{left the
company} shortly after the app was released. (Facebook declined to
comment, and Mr. Voss did not respond to a request for comment.)

Before I go any further, let's get one thing out of the way: If you're
reading this, you are almost certainly too old to feel at home on
TikTok. The company declined to provide information about its users, but
judging from what's on the platform, the median TikTok user seems to
hover in the midteens. TikTok is full of acne-studded faces, barely
concealed tween angst and impenetrable youth-culture references. As far
as I can tell, there is no way for adults to use it without feeling as
if they are chaperoning a high school dance.

Officially, TikTok users must be 13 or older to join. But the
age-verification process is easy to circumvent, and while browsing the
platform, I stumbled upon several videos starring people who appeared to
be much younger. In its previous incarnation as Musical.ly, TikTok
\href{https://www.nytimes.com/2016/09/17/business/media/a-social-network-frequented-by-children-tests-the-limits-of-online-regulation.html}{drew
fire from some privacy advocates}, who accused it of pushing the limits
of the Children's Online Privacy Protection Act, a law that prohibits
the collection of certain types of information from users younger than
13.

``It's clearly a really popular, cool site, but you also have the issue
of kids being significantly too young for it,'' said James P. Steyer,
the chief executive of Common Sense Media, a nonprofit that reviews tech
products for families. ``It's not that the content on TikTok isn't O.K.
for your 15-year-old. It's what happens to your 6- or 7-year-old.''

While using TikTok, I never saw examples of bullying or harassment.
(Both of which are prohibited by TikTok's community guidelines, as is
sexually explicit content.) There are, however, a decent number of
videos featuring teenage girls dancing suggestively --- which, if you
are a 31-year-old newspaper columnist and not a 16-year-old boy, is
fairly unsettling.

A TikTok spokeswoman said in a statement that promoting safety and
positivity on the platform is ``our top priority.'' She added, ``we
periodically add to and adjust our protective measures, policies and
moderation efforts to support the well-being of our users.''

Last year, after two other apps owned by ByteDance were
\href{https://www.nytimes.com/2018/10/29/technology/bytedance-app-funding-china.html}{criticized
by Chinese officials} for promoting objectionable content, the company's
chief executive, Zhang Yiming, said that it would increase the ranks of
its content moderation team to 10,000 moderators, from 6,000. The TikTok
spokeswoman declined to say how many of those moderators work for
TikTok, or whether content standards for American users differ from
those for users in China, where
\href{https://www.nytimes.com/2018/01/02/business/china-toutiao-censorship.html}{famously
strict censorship laws} apply.

Free-speech advocates might bristle at TikTok's Chinese ownership, and
privacy hawks have raised questions about how the company handles users'
personal data. But perhaps because it is more heavily moderated than
other networks, TikTok mostly feels safe and wholesome. Julia Alexander,
a fellow TikTok convert at The Verge,
\href{https://www.theverge.com/2018/11/5/18009260/tiktok-musically-youtube-challenge-vine}{called
it} ``a rare social app that isn't infested with hateful rhetoric.''

One popular genre of TikTok video is the ``challenge,'' a kind of video
skit that is acted out en masse. One challenge,
\href{http://vm.tiktok.com/JfUMLQ/}{\#eatonthebeat}, encouraged users to
make videos of themselves chomping down on food to the beat of a song.

Another challenge,
\href{http://vm.tiktok.com/Jfp8Br/}{\#chooseyourcharacter}, encouraged
users to mimic a video game's character selection screen.

Then there are the running jokes attached to specific songs --- like
``Good Girls Bad Guys,'' a song by the band Falling in Reverse, which is
used for a genre of video in which a user appears first in nerdy,
unattractive clothes, and then \href{http://vm.tiktok.com/JfVfbM/}{cuts
abruptly to a made-over version} of himself in sunglasses, leather
jackets or other bad-boy attire.

Despite TikTok's teens-only vibe, some adults have started to trickle
on. Jimmy Fallon, the late-night TV host, recently joined the site and
started \href{https://www.youtube.com/watch?v=M_kpmrqssFE}{posting his
own challenges}. The comedian Amy Schumer recently made a
\href{https://m.tiktok.com/v/6627958333363981573.html}{TikTok video},
and prominent YouTubers like Jake Paul have tested the waters.

Is TikTok a Facebook killer? No, probably not. For all the variety in
its videos, it is still a fairly limited app, with a more narrow appeal
than more populist social platforms.

But by purposely limiting its features, by resisting the temptation to
monetize its users aggressively and by keeping trolls and bullies off
its platform, TikTok has done something truly impressive --- it has
built a social network that is genuinely fun to use.

There might be a lesson there.

Advertisement

\protect\hyperlink{after-bottom}{Continue reading the main story}

\hypertarget{site-index}{%
\subsection{Site Index}\label{site-index}}

\hypertarget{site-information-navigation}{%
\subsection{Site Information
Navigation}\label{site-information-navigation}}

\begin{itemize}
\tightlist
\item
  \href{https://help.nytimes.com/hc/en-us/articles/115014792127-Copyright-notice}{©~2020~The
  New York Times Company}
\end{itemize}

\begin{itemize}
\tightlist
\item
  \href{https://www.nytco.com/}{NYTCo}
\item
  \href{https://help.nytimes.com/hc/en-us/articles/115015385887-Contact-Us}{Contact
  Us}
\item
  \href{https://www.nytco.com/careers/}{Work with us}
\item
  \href{https://nytmediakit.com/}{Advertise}
\item
  \href{http://www.tbrandstudio.com/}{T Brand Studio}
\item
  \href{https://www.nytimes.com/privacy/cookie-policy\#how-do-i-manage-trackers}{Your
  Ad Choices}
\item
  \href{https://www.nytimes.com/privacy}{Privacy}
\item
  \href{https://help.nytimes.com/hc/en-us/articles/115014893428-Terms-of-service}{Terms
  of Service}
\item
  \href{https://help.nytimes.com/hc/en-us/articles/115014893968-Terms-of-sale}{Terms
  of Sale}
\item
  \href{https://spiderbites.nytimes.com}{Site Map}
\item
  \href{https://help.nytimes.com/hc/en-us}{Help}
\item
  \href{https://www.nytimes.com/subscription?campaignId=37WXW}{Subscriptions}
\end{itemize}
