Sections

SEARCH

\protect\hyperlink{site-content}{Skip to
content}\protect\hyperlink{site-index}{Skip to site index}

\href{https://www.nytimes.com/section/politics}{Politics}

\href{https://myaccount.nytimes.com/auth/login?response_type=cookie\&client_id=vi}{}

\href{https://www.nytimes.com/section/todayspaper}{Today's Paper}

\href{/section/politics}{Politics}\textbar{}Bush Made Willie Horton an
Issue in 1988, and the Racial Scars Are Still Fresh

\url{https://nyti.ms/2zEVsHQ}

\begin{itemize}
\item
\item
\item
\item
\item
\item
\end{itemize}

Advertisement

\protect\hyperlink{after-top}{Continue reading the main story}

Supported by

\protect\hyperlink{after-sponsor}{Continue reading the main story}

\hypertarget{bush-made-willie-horton-an-issue-in-1988-and-the-racial-scars-are-still-fresh}{%
\section{Bush Made Willie Horton an Issue in 1988, and the Racial Scars
Are Still
Fresh}\label{bush-made-willie-horton-an-issue-in-1988-and-the-racial-scars-are-still-fresh}}

\includegraphics{https://static01.nyt.com/images/2018/12/04/us/politics/04dc-horton/04dc-horton-videoSixteenByNine3000.jpg}

By \href{https://www.nytimes.com/by/peter-baker}{Peter Baker}

\begin{itemize}
\item
  Dec. 3, 2018
\item
  \begin{itemize}
  \item
  \item
  \item
  \item
  \item
  \item
  \end{itemize}
\end{itemize}

WASHINGTON --- The tributes to former President George Bush in recent
days have focused on his essential decency and civility, and his embrace
of others, including even his onetime opponents. But the ``last
gentleman,'' as he has been called, was not always so gentle.

Mr. Bush's successful campaign for the presidency in 1988 was marked in
part by the racially charged politics of crime that continues to
reverberate to this day. The
\href{https://www.nytimes.com/1988/11/04/opinion/george-bush-and-willie-horton.html}{Willie
Horton episode} and the political advertising that came to epitomize it
remain among the most controversial chapters in modern politics, a
precursor to campaigns to come and a decisive force that influenced
criminal justice policy for decades.

Mr. Horton was an African-American prisoner in Massachusetts who, while
released on a furlough program, raped a white Maryland woman and bound
and stabbed her boyfriend. Mr. Bush's campaign and supporters cited the
case as evidence that his Democratic opponent, Gov. Michael S. Dukakis
of Massachusetts, was insufficiently tough on crime.

To many African-American people, the scars from that campaign attack
remain fresh. Whatever Mr. Bush's intentions, they said, the campaign
encouraged more race-based politics and put Democrats on the defensive,
forcing them to prove themselves on crime at the expense of a generation
of African-American men and women who were locked up under tougher
sentencing laws championed by President Bill Clinton, among others.

``The reason why the Willie Horton ad is so important in the political
landscape --- it wasn't just about a racist ad that misrepresented the
furlough process,'' said Marcia Chatelain, a Georgetown University
professor of African-American history who teaches a class on race and
racism in the White House. ``But it also taught the Democrats that in
order to win elections, they have to mirror some of the racially
inflected language of tough on crime.''

Michael Nelson, an editor of a book of essays on the Bush presidency
called ``41,'' said the Horton episode led to far more overt plays to
race in American politics, all the way up to President Trump. ``In some
ways, the Willie Horton ad is the 1.0 version of Trump's relentless
tweets and comments about African-Americans,'' he said.

The wisdom of the Massachusetts furlough program was open to debate
aside from race. Releasing nonviolent offenders on weekends to help ease
re-entry into society was the goal, but freeing violent convicts raised
questions about security, and such release programs have receded in the
decades since 1988.

``What crossed the line was not that he was raising the issue of crime
itself because crime was a big issue, and that's fair game,'' said David
Greenberg, a Rutgers University professor and the author of ``Republic
of Spin,'' about political messaging. ``But to use the image of this
threatening black man --- people call it a dog whistle; it was a pretty
clear whistle.''

The fear of Willie Horton continues to haunt politicians today. When
President Barack Obama was trying to forge a bipartisan coalition to
overhaul the criminal justice system to ease sentencing laws that many
in both parties believe went too far, some lawmakers worried that any
change that resulted in the release of someone who would then go on to
commit another violent crime could be political suicide.

Mr. Bush expressed no regret for the Horton ad, and some of his longtime
allies have long argued that he got a bad rap for something that was not
really of his making. Al Gore, then a senator from Tennessee, was the
first to try to wrap the Horton case around Mr. Dukakis's neck during
the Democratic primaries that year.

By summer, Mr. Bush picked up the theme, citing the case during
speeches, and by fall, his campaign began airing an ad attacking the
Massachusetts furlough program, showing a series of prisoners walking
through a revolving door. But that Bush campaign ad did not mention Mr.
Horton.

The one that would be remembered for years to come was produced not by
the Bush campaign but by an operative named Larry McCarthy working for
an ostensibly independent group called the National Security Political
Action Committee. The ad, called ``Weekend Passes,'' singled out Horton,
showing a picture of his scowling face as the narrator described his
torture and rape of the Maryland couple. In the end, it was shown only
briefly on cable television, but its impact was magnified by repeated
coverage on television newscasts.

When critics called the ad a brazen appeal to racial fears, the Bush
campaign distanced itself from the ad and wrote to the committee that
aired it asking that it be withdrawn. But Mr. Dukakis did not buy the
explanation that the committee was independent. ``Anybody who believes
that believes in the tooth fairy,'' he said at one point.

Indeed, Mr. Bush's advisers had been focused on Mr. Horton for months.
``If I can make Willie Horton a household name, we'll win the
election,'' said Lee Atwater, the campaign strategist. He later referred
to making Horton ``Dukakis's running mate.'' Roger Ailes, another Bush
strategist, said, ``The only question is whether we depict Willie Horton
with a knife in his hand or without it.''

A little more than two years later, when stricken with a cancer that
would take his life, Mr. Atwater
\href{https://www.nytimes.com/1991/01/13/us/gravely-ill-atwater-offers-apology.html}{repented
the hardball tactics} used in 1988. He said he particularly regretted
saying he would make Mr. Horton into Mr. Dukakis's running mate
``because it makes me sound racist, which I am not.''

What was never clear was how involved Mr. Bush was in crafting the
strategy. But as Josh King, the author of ``Off Script,'' a book about
political stagecraft, and a student of the 1988 race, put it, ``He was
willing to employ campaign aides who would use the barest of knuckles in
pursuit of the goal of humiliating and destroying the opposing
candidate.''

Mr. Bush's history with race was complicated. Running for the Senate in
1964 in Texas, he opposed the Civil Rights Act, but later regretted it
and sought to make up for it by supporting the Fair Housing Act in
defiance of conservative supporters.

As president, he vetoed civil rights legislation on the grounds that it
would provide for quotas, but ultimately he signed an updated version of
the bill into law. He appointed only the second African-American person
ever to serve on the Supreme Court, Clarence Thomas, to replace the
first, Thurgood Marshall, but the choice angered African-American
leaders on the left who considered Justice Thomas too conservative.

Mr. Bush was unfailingly gracious and friendly with everyone, black or
white, and gave no indication that it mattered to him. When Mr. Obama
visited Houston as president in 2014, Mr. Bush was there waiting for him
on the tarmac to welcome him to town. As it happens, Mr. Obama then
became one of the last outsiders to see Mr. Bush alive last week when he
visited him at his Houston home three days before his death.

Mr. Nelson said Mr. Bush looked at campaigning and governing
differently.

``He regarded politics and campaigning as just the dirty business,'' Mr.
Nelson said. ``Running for office is the price you have to pay for
holding office.''

Advertisement

\protect\hyperlink{after-bottom}{Continue reading the main story}

\hypertarget{site-index}{%
\subsection{Site Index}\label{site-index}}

\hypertarget{site-information-navigation}{%
\subsection{Site Information
Navigation}\label{site-information-navigation}}

\begin{itemize}
\tightlist
\item
  \href{https://help.nytimes.com/hc/en-us/articles/115014792127-Copyright-notice}{©~2020~The
  New York Times Company}
\end{itemize}

\begin{itemize}
\tightlist
\item
  \href{https://www.nytco.com/}{NYTCo}
\item
  \href{https://help.nytimes.com/hc/en-us/articles/115015385887-Contact-Us}{Contact
  Us}
\item
  \href{https://www.nytco.com/careers/}{Work with us}
\item
  \href{https://nytmediakit.com/}{Advertise}
\item
  \href{http://www.tbrandstudio.com/}{T Brand Studio}
\item
  \href{https://www.nytimes.com/privacy/cookie-policy\#how-do-i-manage-trackers}{Your
  Ad Choices}
\item
  \href{https://www.nytimes.com/privacy}{Privacy}
\item
  \href{https://help.nytimes.com/hc/en-us/articles/115014893428-Terms-of-service}{Terms
  of Service}
\item
  \href{https://help.nytimes.com/hc/en-us/articles/115014893968-Terms-of-sale}{Terms
  of Sale}
\item
  \href{https://spiderbites.nytimes.com}{Site Map}
\item
  \href{https://help.nytimes.com/hc/en-us}{Help}
\item
  \href{https://www.nytimes.com/subscription?campaignId=37WXW}{Subscriptions}
\end{itemize}
