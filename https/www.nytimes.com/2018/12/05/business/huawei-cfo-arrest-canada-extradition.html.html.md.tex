Sections

SEARCH

\protect\hyperlink{site-content}{Skip to
content}\protect\hyperlink{site-index}{Skip to site index}

\href{https://www.nytimes.com/section/business}{Business}

\href{https://myaccount.nytimes.com/auth/login?response_type=cookie\&client_id=vi}{}

\href{https://www.nytimes.com/section/todayspaper}{Today's Paper}

\href{/section/business}{Business}\textbar{}Huawei C.F.O. Is Arrested in
Canada for Extradition to the U.S.

\url{https://nyti.ms/2zLcapk}

\begin{itemize}
\item
\item
\item
\item
\item
\item
\end{itemize}

Advertisement

\protect\hyperlink{after-top}{Continue reading the main story}

Supported by

\protect\hyperlink{after-sponsor}{Continue reading the main story}

\hypertarget{huawei-cfo-is-arrested-in-canada-for-extradition-to-the-us}{%
\section{Huawei C.F.O. Is Arrested in Canada for Extradition to the
U.S.}\label{huawei-cfo-is-arrested-in-canada-for-extradition-to-the-us}}

\includegraphics{https://static01.nyt.com/images/2018/12/06/business/06huawei-promo2/06huawei-promo2-articleLarge-v2.jpg?quality=75\&auto=webp\&disable=upscale}

By \href{https://www.nytimes.com/by/daisuke-wakabayashi}{Daisuke
Wakabayashi} and \href{https://www.nytimes.com/by/alan-rappeport}{Alan
Rappeport}

\begin{itemize}
\item
  Dec. 5, 2018
\item
  \begin{itemize}
  \item
  \item
  \item
  \item
  \item
  \item
  \end{itemize}
\end{itemize}

\href{https://cn.nytimes.com/business/20181206/huawei-cfo-arrest-canada-extradition/}{阅读简体中文版}\href{https://cn.nytimes.com/business/20181206/huawei-cfo-arrest-canada-extradition/zh-hant/}{閱讀繁體中文版}

SAN FRANCISCO --- A top executive and daughter of the founder of the
Chinese tech giant Huawei was arrested on Saturday in Canada at the
request of the United States, in a move likely to escalate tensions
between the two countries at a delicate moment.

The arrest of Meng Wanzhou, the chief financial officer, unfolded on the
same night that President Trump and President Xi Jinping of China dined
together in Buenos Aires and agreed to
\href{https://www.nytimes.com/2018/12/05/us/politics/trump-xi-trade-china.html?action=click\&module=Top\%20Stories\&pgtype=Homepage}{a
90-day trade truce}. The two countries are set to begin tense
negotiations in hopes of ending a trade war that has been pummeling both
economies.

Those talks now face an even steeper challenge. The aim will be for the
United States to ease its tariffs; in exchange, China will be expected
to lower trade barriers and further open its markets to American
businesses.

What's more, Ms. Meng's detention raises questions about the Trump
administration's overall China strategy. Beijing is now likely to
pressure Canada to release her and to press the United States to avoid a
trial.

``The arrest of a family member linked to Huawei's founder indicates how
the tension between the two sides is rapidly escalating,'' said T.J.
Pempel, a professor of political science at the University of
California, Berkeley, who specializes in East Asian politics and
economy.

\href{https://www.huawei.com/us/about-huawei/executives/board-of-directors/meng-wanzhou}{Ms.
Meng}, who joined Huawei in 1993 and is also a deputy chairwoman, was
taken into custody in Vancouver on Dec. 1, said Ian McLeod, a spokesman
for Canada's Justice Department. He said she was ``sought for
extradition by the United States'' but did not give a reason for what
prompted the arrest. He added that a publication ban requested by Ms.
Meng prevented him from providing any further details. A bail hearing
has been set for Friday.

Senator Ben Sasse, a Republican of Nebraska, linked the arrest to the
American sanctions against Iran.

Mr. Sasse said China had been ``working to creatively undermine our
national security interests, and the United States and our allies can't
sit on the sidelines.'' He added that ``Americans are grateful that our
Canadian partners have arrested the chief financial officer of a giant
Chinese telecom company for breaking U.S. sanctions against Iran.''

Huawei, China's largest telecom equipment maker, has been under
investigation into whether it had
\href{https://www.nytimes.com/2017/04/26/business/huawei-investigation-sanctions-subpoena.html}{broken
American trade controls} to countries including Cuba, Iran, Sudan and
Syria.

This year, the Treasury and Commerce Department also asked the Justice
Department to investigate Huawei for possibly violating economic
sanctions against Iran, according to an official who spoke on the
condition of anonymity because he was not authorized to discuss the
investigation. Prosecutors in the Eastern District of New York took on
the case, he said.

\emph{{[}}\href{https://www.nytimes.com/2018/12/06/business/stocks-wall-street-huawei-trade.html}{\emph{Stock
markets were shaken}} \emph{by the arrest, as investors feared the
impact on U.S.-China trade relations.{]}}

In response, a spokesman for the Chinese Embassy in Canada said in a
statement that ``the Chinese side firmly opposes and strongly protests
over such kind of actions'' and urged the authorities ``to immediately
correct the wrongdoing and restore the personal freedom of Ms. Meng.''

Huawei said in a statement that Ms. Meng was arrested while changing
planes in Canada and that she faced unspecified charges from the Eastern
District of New York.

``The company has been provided very little information regarding the
charges and is not aware of any wrongdoing by Ms. Meng,'' Huawei said,
adding that it complies with all laws where it operates.

Press representatives for the Justice Department and the United States
Attorney's Office in the Eastern District of New York declined to
comment. The White House did not immediately respond when asked if Mr.
Trump was aware of the detention during his dinner with President Xi.

Julian Ku, a professor at Hofstra University Law School,
\href{https://twitter.com/julianku/status/1070498880347889664}{wrote on
Twitter} that the move was justifiable. ``US law prohibits exports of
certain US-origin technologies to certain countries,'' he said. ``When
Huawei pays to license certain US tech, it promises not to export to
certain countries like Iran. So it is not unreasonable for the US to
punish Huawei for flouting this US law.''

The arrest meets several major foreign policy aims of the Trump
administration. American officials have sought to persuade other nations
to curb business ventures with Huawei because of security concerns. The
White House has also focused on
\href{https://www.nytimes.com/2018/11/09/sunday-review/trump-sanctions-iran-foreign-policy.html}{tightening
and enforcing economic sanctions on Iran}, months after Mr. Trump
announced he was withdrawing from a multinational agreement reached
under President Barack Obama's administration to freeze Iran's nuclear
program.

Last month, the United States imposed sanctions aimed at reducing
exports of Iranian oil to zero and crippling Iran's economy, though
China is one of a handful of countries allowed to continue to buy oil
for six months.

The United States and China have also been locked in a struggle for
high-tech supremacy, in a race that has increasingly taken on political
undertones this year. While the United States has long claimed an
advantage in the tech industry, China's internet companies,
semiconductor makers and telecom equipment makers have all been growing
rapidly, with many benefiting from government investment.

President Trump has tied national security to advancement in
technologies like wireless networks, and has made protection of the
domestic tech industry a part of his agenda. In March,
\href{https://www.nytimes.com/2018/03/12/technology/trump-broadcom-qualcomm-merger.html}{he
blocked a \$117 billion bid} by Broadcom, a Singapore-based chip maker,
for the American chip maker Qualcomm, citing national security concerns
and how it might allow China ---
\href{https://www.nytimes.com/2018/03/06/business/qualcomm-broadcom-cfius.html}{specifically
citing Huawei} --- to leap ahead in next-generation 5G wireless
networks.

A month later, the Commerce Department
\href{https://www.nytimes.com/2018/04/16/technology/chinese-tech-company-blocked-from-buying-american-components.html?module=inline}{banned
ZTE}, China's second-largest maker of telecommunications equipment, from
using components made in the United States. Federal authorities said ZTE
had violated American sanctions against Iran and North Korea, in a move
that caused the Chinese company to
\href{https://www.nytimes.com/2018/05/09/technology/zte-china-us-trade-war.html}{cease
``major operating activities''} for a time. Mr. Trump ultimately
intervened and ZTE agreed to pay a \$1 billion fine, replace its board
and senior leadership and allow the United States to inspect its
operations with a handpicked compliance team.

Over the last decade, Huawei has grown into a powerhouse. Founded in
1987 by Ren Zhengfei, a former People's Liberation Army engineer, it
generated over \$90 billion in revenue in 2017. Its equipment is the
backbone of mobile networks around the world, and its smartphones are
popular in Europe and China. That has made it a symbol of China's
technological prowess and evolution from a country that makes cheap but
unreliable gadgets to cutting-edge products that can rival the best of
Silicon Valley and other Asian technology giants.

Yet Huawei has long faced scrutiny as a security threat in the United
States. Washington has expressed concern about using Huawei products,
citing spying risk because of the company's close ties to the Chinese
government.

While Huawei has long tried to make inroads into the United States, it
has been bedeviled by the security concerns. In January, Huawei's effort
to sell a new line of smartphones in the United States was derailed when
AT\&T
\href{https://www.nytimes.com/2018/01/09/business/att-huawei-mate-smartphone.html?module=inline}{walked
away from a deal}to distribute the devices.

Eswar Prasad, a trade policy professor at Cornell University, said the
Huawei issue could be a cloud over coming talks. ``A fragile trade truce
between China and the U.S. that was already foundering is now at greater
risk of unraveling in relatively short order,'' Professor Prasad said.

He added: ``It is likely that China will have a measured response to
this incident, although it will certainly add a sharper edge to the
negotiations between the two sides.''

Advertisement

\protect\hyperlink{after-bottom}{Continue reading the main story}

\hypertarget{site-index}{%
\subsection{Site Index}\label{site-index}}

\hypertarget{site-information-navigation}{%
\subsection{Site Information
Navigation}\label{site-information-navigation}}

\begin{itemize}
\tightlist
\item
  \href{https://help.nytimes.com/hc/en-us/articles/115014792127-Copyright-notice}{©~2020~The
  New York Times Company}
\end{itemize}

\begin{itemize}
\tightlist
\item
  \href{https://www.nytco.com/}{NYTCo}
\item
  \href{https://help.nytimes.com/hc/en-us/articles/115015385887-Contact-Us}{Contact
  Us}
\item
  \href{https://www.nytco.com/careers/}{Work with us}
\item
  \href{https://nytmediakit.com/}{Advertise}
\item
  \href{http://www.tbrandstudio.com/}{T Brand Studio}
\item
  \href{https://www.nytimes.com/privacy/cookie-policy\#how-do-i-manage-trackers}{Your
  Ad Choices}
\item
  \href{https://www.nytimes.com/privacy}{Privacy}
\item
  \href{https://help.nytimes.com/hc/en-us/articles/115014893428-Terms-of-service}{Terms
  of Service}
\item
  \href{https://help.nytimes.com/hc/en-us/articles/115014893968-Terms-of-sale}{Terms
  of Sale}
\item
  \href{https://spiderbites.nytimes.com}{Site Map}
\item
  \href{https://help.nytimes.com/hc/en-us}{Help}
\item
  \href{https://www.nytimes.com/subscription?campaignId=37WXW}{Subscriptions}
\end{itemize}
