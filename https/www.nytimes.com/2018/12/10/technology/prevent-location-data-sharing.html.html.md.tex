Sections

SEARCH

\protect\hyperlink{site-content}{Skip to
content}\protect\hyperlink{site-index}{Skip to site index}

\href{https://www.nytimes.com/section/technology}{Technology}

\href{https://myaccount.nytimes.com/auth/login?response_type=cookie\&client_id=vi}{}

\href{https://www.nytimes.com/section/todayspaper}{Today's Paper}

\href{/section/technology}{Technology}\textbar{}How to Stop Apps From
Tracking Your Location

\url{https://nyti.ms/2G93leA}

\begin{itemize}
\item
\item
\item
\item
\item
\end{itemize}

Advertisement

\protect\hyperlink{after-top}{Continue reading the main story}

Supported by

\protect\hyperlink{after-sponsor}{Continue reading the main story}

\hypertarget{how-to-stop-apps-from-tracking-your-location}{%
\section{How to Stop Apps From Tracking Your
Location}\label{how-to-stop-apps-from-tracking-your-location}}

Hundreds of apps can follow your movements and share the details with
advertisers, retailers and even hedge funds. Here's how to limit the
snooping.

By \href{https://www.nytimes.com/by/jennifer-valentino-devries}{Jennifer
Valentino-DeVries} and
\href{https://www.nytimes.com/by/natasha-singer}{Natasha Singer}

\begin{itemize}
\item
  Dec. 10, 2018
\item
  \begin{itemize}
  \item
  \item
  \item
  \item
  \item
  \end{itemize}
\end{itemize}

At least 75 companies receive people's precise location data from
hundreds of apps whose users enable location services for benefits such
as weather alerts,
\href{https://www.nytimes.com/interactive/2018/12/10/business/location-data-privacy-apps.html}{The
New York Times found}. The companies use, store or sell the information
to help advertisers, investment firms and others.

\emph{{[}Read the full investigation:}
\href{https://www.nytimes.com/interactive/2018/12/10/business/location-data-privacy-apps.html}{\emph{Your
Apps Know Where You Were Last Night, and They're Not Keeping It
Secret}}\emph{{]}}

You can head off much of the tracking on your own device by spending a
few minutes changing settings. The information below applies primarily
to people in the United States.

\hypertarget{how-can-i-tell-if-apps-are-sharing-my-location}{%
\subsection{How can I tell if apps are sharing my
location?}\label{how-can-i-tell-if-apps-are-sharing-my-location}}

It's difficult to know for sure whether location data companies are
tracking your phone. Any app that collects location data may share your
information with other companies, as long as it mentions that somewhere
in its privacy policy.

But the language in those policies can be dense, confusing or outright
misleading. Apps that funnel location details to help hedge funds, for
instance, have told users their data would be used for market analysis
--- or simply for ``business purposes.''

\hypertarget{which-apps-gather-and-share-location}{%
\subsection{Which apps gather and share
location?}\label{which-apps-gather-and-share-location}}

There isn't a definitive list. Our tests identified instances of certain
apps collecting precise location data and passing it to other companies
in the moment. But apps can also gather and save the data, and not sell
it until later --- something tests wouldn't catch. Your best bet is to
check your device to see which apps have permission to get your location
in the first place.

The apps most popular among data companies are those that offer services
keyed to people's whereabouts --- including weather, transit, travel,
shopping deals and dating --- because users are more likely to enable
location services on them.

\hypertarget{how-do-i-stop-location-tracking-on-ios}{%
\subsection{How do I stop location tracking on
iOS?}\label{how-do-i-stop-location-tracking-on-ios}}

Some apps have internal settings where you can indicate that you don't
want your location used for targeted advertising or other purposes. But
the easiest method is to go through your device's main privacy menu.

Image

First, open Settings and select Privacy, which has a blue icon with a
white hand.

Image

Then select Location Services, which is at the top and has a little
arrow.

Image

You'll see a list of apps, along with the location setting for each. Tap
on apps you want to adjust. Selecting ``Never'' blocks tracking by that
app.

Image

The option ``While Using the App'' ensures that the app gets location
only while in use. Choosing ``Always,'' allows the app to get location
data even when not in use.

In the device's privacy settings, apps provide brief explanations of how
they will use location data. Do not rely on these descriptions to tell
you whether the location data will be shared or sold. The Times found
that many of these descriptions are incomplete and often don't mention
that the data will be shared.

If you want to disable location tracking entirely, toggle the ``Location
Services" setting to off. With location services switched off entirely,
you may not be able to use certain services, such as finding yourself on
a map.

If you have apps you no longer use, you may want to delete them from
your device.

\hypertarget{how-do-i-stop-it-on-android}{%
\subsection{How do I stop it on
Android?}\label{how-do-i-stop-it-on-android}}

These instructions are for recent Android phones; Google provides more
instructions
\href{https://support.google.com/android/answer/6179507}{here}.

Image

First, open the Settings on your phone. On the main settings page, tap
``Security \& location.''

Image

On the next screen, tap Location, which can be found in the Privacy
section.

Image

On the Location screen, tap ``App-level permissions.''

Image

You'll see a list of apps. To turn off location for an app, slide the
toggle to the left.

Unlike iPhones, Android phones don't allow you to restrict an app's
access to your location to just the moments when you're using it. Any
app on Android that has your permission to track your location can
receive the data even when you're not using it. In newer versions of
Android, the collection of this data is limited to ``a few times an
hour,'' Google says.

To disable location services altogether, switch off ``Use location,''
within the same Location settings described above. Google's instructions
are
\href{https://support.google.com/accounts/answer/3467281?hl=en}{here}.

If you don't enable location services at all, you may not be able to use
certain services, like finding yourself on a map. If you want to be able
to switch periodically between having location services on and off, you
can create a
\href{https://support.google.com/android/answer/9083864}{Quick Setting}.
To see your Quick Settings, swipe down from the top of your screen and
tap the little pencil to edit.

If you have apps you no longer use, you may want to delete them from
your phone.

\hypertarget{can-i-delete-my-location-data-from-these-databases}{%
\subsection{Can I delete my location data from these
databases?}\label{can-i-delete-my-location-data-from-these-databases}}

The location data industry benefits from lack of regulation and little
transparency, making it extremely difficult to get access to or delete
this data. Your information can also be spread among many companies. And
most of them store location data attached not to a person's name or
phone number, but to an ID number, so it may be cumbersome for them to
identify your data if you want to delete it --- and they are under no
obligation to do so.

In the European Union, people now have the legal right to request a copy
of the
\href{https://www.nytimes.com/interactive/2018/05/20/technology/what-data-companies-have-on-you.html}{data
that companies hold about them}, and to ask that it be deleted. The
British data commissioner provides an explanation,
\href{https://ico.org.uk/for-organisations/guide-to-the-general-data-protection-regulation-gdpr/individual-rights/right-of-access/}{here}.

Google, a prominent collector of location data, lets users delete a
segment of that information called their Location History. To do that,
go to \href{https://www.google.com/locationhistory/delete}{this page},
then hit the Delete Location History button. Click it again when
prompted. You can delete another segment of location data associated
with your Google account by logging in and going to
\href{https://myactivity.google.com/myactivity}{My Activity}. Then click
on Activity Controls and turn off Web \& App Activity.

Jennifer Valentino-DeVries is an investigative reporter covering
technology and public policy.
\href{https://twitter.com/jenvalentino}{@jenvalentino}

Advertisement

\protect\hyperlink{after-bottom}{Continue reading the main story}

\hypertarget{site-index}{%
\subsection{Site Index}\label{site-index}}

\hypertarget{site-information-navigation}{%
\subsection{Site Information
Navigation}\label{site-information-navigation}}

\begin{itemize}
\tightlist
\item
  \href{https://help.nytimes.com/hc/en-us/articles/115014792127-Copyright-notice}{©~2020~The
  New York Times Company}
\end{itemize}

\begin{itemize}
\tightlist
\item
  \href{https://www.nytco.com/}{NYTCo}
\item
  \href{https://help.nytimes.com/hc/en-us/articles/115015385887-Contact-Us}{Contact
  Us}
\item
  \href{https://www.nytco.com/careers/}{Work with us}
\item
  \href{https://nytmediakit.com/}{Advertise}
\item
  \href{http://www.tbrandstudio.com/}{T Brand Studio}
\item
  \href{https://www.nytimes.com/privacy/cookie-policy\#how-do-i-manage-trackers}{Your
  Ad Choices}
\item
  \href{https://www.nytimes.com/privacy}{Privacy}
\item
  \href{https://help.nytimes.com/hc/en-us/articles/115014893428-Terms-of-service}{Terms
  of Service}
\item
  \href{https://help.nytimes.com/hc/en-us/articles/115014893968-Terms-of-sale}{Terms
  of Sale}
\item
  \href{https://spiderbites.nytimes.com}{Site Map}
\item
  \href{https://help.nytimes.com/hc/en-us}{Help}
\item
  \href{https://www.nytimes.com/subscription?campaignId=37WXW}{Subscriptions}
\end{itemize}
