Sections

SEARCH

\protect\hyperlink{site-content}{Skip to
content}\protect\hyperlink{site-index}{Skip to site index}

\href{https://www.nytimes.com/section/politics}{Politics}

\href{https://myaccount.nytimes.com/auth/login?response_type=cookie\&client_id=vi}{}

\href{https://www.nytimes.com/section/todayspaper}{Today's Paper}

\href{/section/politics}{Politics}\textbar{}Trump's Intervention in
Huawei Case Would Be Legal, but Bad Precedent, Experts Say

\url{https://nyti.ms/2GezCRw}

\begin{itemize}
\item
\item
\item
\item
\item
\end{itemize}

Advertisement

\protect\hyperlink{after-top}{Continue reading the main story}

Supported by

\protect\hyperlink{after-sponsor}{Continue reading the main story}

News Analysis

\hypertarget{trumps-intervention-in-huawei-case-would-be-legal-but-bad-precedent-experts-say}{%
\section{Trump's Intervention in Huawei Case Would Be Legal, but Bad
Precedent, Experts
Say}\label{trumps-intervention-in-huawei-case-would-be-legal-but-bad-precedent-experts-say}}

\includegraphics{https://static01.nyt.com/images/2018/12/13/us/politics/13dc-trump/merlin_148063839_0591d15d-44ec-4cd8-a6c0-e38b1f87fd80-articleLarge.jpg?quality=75\&auto=webp\&disable=upscale}

By \href{https://www.nytimes.com/by/michael-tackett}{Michael Tackett}
and \href{https://www.nytimes.com/by/charlie-savage}{Charlie Savage}

\begin{itemize}
\item
  Dec. 12, 2018
\item
  \begin{itemize}
  \item
  \item
  \item
  \item
  \item
  \end{itemize}
\end{itemize}

\href{https://cn.nytimes.com/usa/20181214/trump-meng-wanzhou-huawei-extradition/}{阅读简体中文版}\href{https://cn.nytimes.com/usa/20181214/trump-meng-wanzhou-huawei-extradition/zh-hant/}{閱讀繁體中文版}

WASHINGTON --- When President Trump said in an interview this week that
he was willing to intercede in
\href{https://www.nytimes.com/2018/12/11/technology/huawei-executive-canada-bail-decision.html}{the
case of a Chinese telecom executive} facing extradition to the United
States if it helped achieve ``the largest trade deal ever made,'' it was
a clear signal that his White House saw no problem intervening in the
justice system to achieve what it considered economic gain.

A range of experts agreed on Wednesday that the president had the legal
authority to order the government to rescind the extradition request for
the executive, Meng Wanzhou, or even drop the charges against her. But
they could not point to another instance of a president injecting
himself into a criminal proceeding in a similar way.

``It sets a very bad precedent,'' said Nicholas Burns, a former under
secretary of state and ambassador to NATO who served in Republican and
Democratic administrations and now teaches diplomacy and international
relations at Harvard. ``By mixing justice with trade and the rule of law
with trade, it devalues both.''

Ms. Meng, the chief financial officer of the telecommunications giant
Huawei, was arrested last week by Canadian authorities at the request of
the American government on suspicion of fraud related to Iranian
sanctions. Mr. Trump said in
\href{https://www.reuters.com/article/us-usa-trump/trump-says-would-intervene-in-arrest-of-chinese-executive-idUSKBN1OB01P}{an
interview with Reuters} that ``I would certainly intervene if I thought
it was necessary'' for a trade deal.

The White House did not respond to a request to explain how Mr. Trump
arrived at that position.

But John Demers, the assistant attorney general for the Justice
Department's National Security Division, bristled at the notion that the
motivation behind the charges might have anything to do with leverage in
trade talks. ``We are not a tool of trade when we bring the cases,'' Mr.
Demers said.

Mr. Burns and others, however, said the president's remarks could be
seen by other countries as a green light.

``By interfering in a Justice Department decision and giving the
impression he may release her in exchange for concessions on trade
talks, Trump may inspire authoritarian leaders to do the same to
Americans around the world,'' Mr. Burns said. ``You have seen that China
has
\href{https://www.nytimes.com/2018/12/12/world/asia/china-michael-kovrig-detained.html}{detained
a Canadian International Crisis Group leader}. Reciprocity is a
fundamental foundation stone of international politics. Others will do
unto you what you have done unto them.''

Unlike most of his predecessors, Mr. Trump has offered strong opinions
on numerous pending court cases, in ways that lawyers said could shape
everything from jury selection to the ultimate outcome. In the
extradition case involving Ms. Meng, the president seemed willing to
treat it as just another element in a deal.

``It takes the term `transactional' to a new level because he is
basically suggesting everything is for sale,'' said Wendy Cutler, a
trade negotiator in President Barack Obama's administration. ``There's a
reason why these issues are in traditional lanes. And once you signal
you are going to trade one off for the other, then you are in a whole
new world. This undermines the rule of law.''

To be sure, the United States has often been willing to mix issues of
justice to achieve other ends, such as swapping captured spies with
other nations. The Iran nuclear deal offers another example.

As part of the Obama administration's nuclear agreement with Iran, Iran
released five American prisoners and Mr. Obama freed seven Iranian and
Iranian-American prisoners. In addition to that swap, the Justice
Department also quietly
\href{https://www.politico.com/blogs/under-the-radar/2016/01/iran-deal-obama-grants-clemency-to-seven-217879}{dropped
sanctions-related charges and rescinded international arrest warrants}
against 14 other Iranian fugitives.

But Lisa Monaco, who was Mr. Obama's top homeland security and
counterterrorism adviser during the Iran deal --- and the former head of
the Justice Department's national security division --- argued that
dropping the cases against the 14 Iranians was very different from what
Mr. Trump is floating as a possibility in the Huawei case.

For one thing, she said, the Justice Department had concluded it was
unlikely ever to be in a position to extradite the Iranians, while
American prosecutors are now close to having Ms. Meng in their grasp.

For another, the Iran nuclear deal was an international agreement with
numerous other nations, not a common bilateral trade dispute. What is
more, it was also a one-time move made in a unique context: a deal to
end Iran's nuclear ambitions without war. Trade negotiations, on the
other hand, happen all the time --- so the precedent, she said, could
endanger Americans.

``I think as a matter of law Trump could direct the Justice Department
to drop the prosecution that is the basis of the extradition request,''
Ms. Monaco said. ``That would be wrong and outside any acceptable norm,
and it would be open season on our own executives to be detained and
hassled in other countries that want to get leverage in trade talks. But
he could do it.''

\href{https://law.duke.edu/fac/bradleyc/}{Curtis Bradley}, a Duke
University law professor who has written about international law and
extradition issues, said what was most unusual about Mr. Trump's views
was that he expressed them publicly as opposed to making them known
through diplomatic channels. But he said that from a legal perspective,
the decision to request extradition or drop such a request was primarily
a matter of foreign affairs, over which the president had broad legal
authority.

``Extradition is sufficiently political already and part of our foreign
affairs,'' he said. ``I don't think it's comparable to intervening in
domestic criminal decision-making.''

Still, the move could
\href{https://www.nytimes.com/2018/12/12/world/canada/canada-china-meng-huawei.html}{raise
diplomatic tensions with Canada}, one of the largest trading partners of
the United States.

``Our extradition partners should not seek to politicize the extradition
process or use it for ends other than the pursuit of justice,'' Chrystia
Freeland, Canada's foreign minister, said at a news conference on
Wednesday, adding that she had spoken to Secretary of State Mike Pompeo
about the case earlier in the week and had stressed that Canada was a
``rule of law country.''

She said that it would be up to Ms. Meng's ``lawyers whether they choose
to raise comments in the U.S. as part of their defense'' and that it
would then be up to ``Canadian judges how to weigh the significance.''

Mr. Bradley noted that the extradition treaty with Canada made an
exception for crimes of ``a political character.'' Mr. Trump's comments,
he said, might give Ms. Meng's lawyers a hook to argue that Canada
should not send her to the United States.

``If there was evidence that the U.S. prosecution was initiated for
political reasons, I suppose Meng's lawyer could argue that the offense
is of a political character,'' he said. ``And Trump's remarks might at
least be further evidence of that.''

Advertisement

\protect\hyperlink{after-bottom}{Continue reading the main story}

\hypertarget{site-index}{%
\subsection{Site Index}\label{site-index}}

\hypertarget{site-information-navigation}{%
\subsection{Site Information
Navigation}\label{site-information-navigation}}

\begin{itemize}
\tightlist
\item
  \href{https://help.nytimes.com/hc/en-us/articles/115014792127-Copyright-notice}{©~2020~The
  New York Times Company}
\end{itemize}

\begin{itemize}
\tightlist
\item
  \href{https://www.nytco.com/}{NYTCo}
\item
  \href{https://help.nytimes.com/hc/en-us/articles/115015385887-Contact-Us}{Contact
  Us}
\item
  \href{https://www.nytco.com/careers/}{Work with us}
\item
  \href{https://nytmediakit.com/}{Advertise}
\item
  \href{http://www.tbrandstudio.com/}{T Brand Studio}
\item
  \href{https://www.nytimes.com/privacy/cookie-policy\#how-do-i-manage-trackers}{Your
  Ad Choices}
\item
  \href{https://www.nytimes.com/privacy}{Privacy}
\item
  \href{https://help.nytimes.com/hc/en-us/articles/115014893428-Terms-of-service}{Terms
  of Service}
\item
  \href{https://help.nytimes.com/hc/en-us/articles/115014893968-Terms-of-sale}{Terms
  of Sale}
\item
  \href{https://spiderbites.nytimes.com}{Site Map}
\item
  \href{https://help.nytimes.com/hc/en-us}{Help}
\item
  \href{https://www.nytimes.com/subscription?campaignId=37WXW}{Subscriptions}
\end{itemize}
