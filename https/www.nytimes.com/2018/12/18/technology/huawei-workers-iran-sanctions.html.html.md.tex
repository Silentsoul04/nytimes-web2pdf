Sections

SEARCH

\protect\hyperlink{site-content}{Skip to
content}\protect\hyperlink{site-index}{Skip to site index}

\href{https://www.nytimes.com/section/technology}{Technology}

\href{https://myaccount.nytimes.com/auth/login?response_type=cookie\&client_id=vi}{}

\href{https://www.nytimes.com/section/todayspaper}{Today's Paper}

\href{/section/technology}{Technology}\textbar{}Huawei's `Wolf Culture'
Helped It Grow, and Got It Into Trouble

\url{https://nyti.ms/2Gslkge}

\begin{itemize}
\item
\item
\item
\item
\item
\end{itemize}

Advertisement

\protect\hyperlink{after-top}{Continue reading the main story}

Supported by

\protect\hyperlink{after-sponsor}{Continue reading the main story}

\hypertarget{huaweis-wolf-culture-helped-it-grow-and-got-it-into-trouble}{%
\section{Huawei's `Wolf Culture' Helped It Grow, and Got It Into
Trouble}\label{huaweis-wolf-culture-helped-it-grow-and-got-it-into-trouble}}

\includegraphics{https://static01.nyt.com/images/2018/12/17/business/00huawei-01/00huawei-01-articleLarge-v2.jpg?quality=75\&auto=webp\&disable=upscale}

By \href{https://www.nytimes.com/by/raymond-zhong}{Raymond Zhong}

\begin{itemize}
\item
  Dec. 18, 2018
\item
  \begin{itemize}
  \item
  \item
  \item
  \item
  \item
  \end{itemize}
\end{itemize}

\href{https://cn.nytimes.com/technology/20181219/huawei-workers-iran-sanctions/}{阅读简体中文版}\href{https://cn.nytimes.com/technology/20181219/huawei-workers-iran-sanctions/zh-hant/}{閱讀繁體中文版}\href{https://www.nytimes.com/es/2018/12/20/huawei-cultura-arresto}{Leer
en español}

SHENZHEN, China --- Earthquakes, terrorist attacks and low oxygen levels
on Mount Everest could not hold them back.

As the Chinese tech giant Huawei expanded around the globe, supplying
equipment to bring mobile phone and data service to the planet's
farthest reaches, its employees were urged on by a culture that
celebrated daring feats in pursuit of new business.

They worked grueling hours. They were encouraged to bend certain company
rules, as long as doing so enriched the company and not employees
personally, according to Huawei workers interviewed by The New York
Times.

Employees at the company and
\href{https://www.lowyinstitute.org/the-interpreter/weekend-catch-huawei-wolf-culture-and-more}{people
who have studied it} have a name for its hard-charging corporate spirit:
``wolf culture.''

Now, the company's aggressive ways have been cast in a new light. The
United States has accused
\href{https://www.nytimes.com/2018/12/07/technology/meng-wanzhou-huawei-arrest.html}{Meng
Wanzhou}, a top Huawei executive and daughter of its founder, of
\href{https://www.nytimes.com/2018/12/14/business/huawei-meng-hsbc-canada.html}{committing
bank fraud} to help the company's business in Iran.

It is not clear precisely how Huawei's culture shaped its dealings in
Iran. But an intense will to get ahead, which helped propel it to the
head of the global market for telecom network equipment, seems to have
informed employees' actions in previous cases that put the company under
scrutiny.

Huawei workers have been accused of bribing government officials to win
business in Africa, copying an American competitor's source code and
even
\href{https://www.nytimes.com/2014/09/06/business/t-mobile-accuses-huawei-of-theft-from-laboratory.html}{stealing
the fingertip of a robot} in a T-Mobile lab in Bellevue, Wash.
\href{https://bits.blogs.nytimes.com/2015/01/22/thousands-of-huawei-workers-respond-to-internal-anti-fraud-campaign/}{In
2015}, Huawei's founder and chief executive, Ren Zhengfei, said that as
part of a company amnesty program, thousands of employees had admitted
to violations ranging from fraudulent reporting of financial information
to bribery.

In an emailed statement, a spokesman said that Huawei requires all
employees to study and sign guidelines on business conduct every year.
``At the heart of the guidelines is the principle of acting in
accordance with all local laws and regulations,'' said the spokesman,
Joe Kelly. ``Where employees are found to have acted outside these
guidelines, the company takes decisive action which can include
immediate termination of employment.''

Mr. Ren said in 2015 that Huawei had toughened its safeguards against
employee misconduct. But the following year, in a speech that was
emailed to employees,
\href{http://www.c114.com.cn/news/126/a991198.html}{he acknowledged}
that many workers did not pay attention to internal rules and controls
--- perhaps, he said, because Huawei used to evaluate staff solely
according to how much business they won.

\href{https://mp.weixin.qq.com/s/v_d66ls4A2zcd0n92xCh-A}{More recently},
in remarks that were emailed to employees, Mr. Ren said that it was
important to enforce internal standards, but that this should not become
a hindrance.

``If it blocks the business from producing grain, then we all starve to
death,'' he said, according to a transcript of his comments on a Huawei
website.

Ms. Meng's arrest this month has darkened China's relations with the
United States, scrambling efforts by the two nations to ease a tense
economic conflict. Washington has
\href{https://www.nytimes.com/2018/12/14/business/huawei-meng-hsbc-canada.html}{worked
for years to undermine Huawei}, regarding its products as potential
vehicles for
\href{https://www.nytimes.com/2018/12/06/technology/huawei-arrest-meng-wanzhou.html}{espionage
and sabotage} --- something the company denies.

Security concerns about Huawei and other Chinese equipment providers are
mounting among traditional allies of the United States.

At the annual meeting of spy chiefs of the so-called Five Eyes
countries, Huawei was among the topics discussed by senior intelligence
officers from Britain, Australia, New Zealand, Canada and the United
States, including Gina Haspel, the C.I.A. director, according to current
and former officials. There was no formal agreement to seek a ban of
Huawei, but the discussion shows the loose coordination Western security
officials have pursued as they try to push the Chinese company out of
agreements to build the next-generation mobile broadband networks, known
as 5G, some of the officials said.

The pressure on the business is building. In Germany last week, Deutsche
Telekom said it was taking seriously the ``global discussion about the
security of network elements from Chinese manufacturers.'' On Monday,
the Czech intelligence agency warned against the country working with
Huawei and ZTE, another Chinese technology company.

Huawei was founded in the late 1980s, during the tumultuous early years
of China's capitalist revival. Mr. Ren was an engineer in the People's
Liberation Army for nearly a decade before starting Huawei, and military
values --- tenacity, dedication, drive --- have long suffused the
company.

\includegraphics{https://static01.nyt.com/images/2018/12/17/business/00huawei-02/merlin_148266480_a39e844c-20fa-44ad-9c6b-e204081ff3c8-articleLarge.jpg?quality=75\&auto=webp\&disable=upscale}

In the early years, squads of Huawei salesmen crisscrossed China in
sport utility vehicles peddling the company's telephone switches to post
offices. Employees were given mattresses so they could nap while working
late nights.

Company lore, as recounted in employee publications and admiring books
by business professors, is heavy on stories of dogged staff members
enduring physical hardship. They worked to keep telecom services running
despite a terrorist attack in Mumbai and an earthquake in Algeria. They
braved cold and sleeplessness to provide mobile coverage to climbers on
Mount Everest.

Today, the working hours are still long at Huawei, although folding beds
at work are more likely to be used for midday shut-eye than for
all-nighters, according to three employees. Several Huawei staff members
spoke to The New York Times on condition of anonymity, fearing
reprisals.

New hires at Huawei take part in a boot camp-style training course that
involves morning jogs and classes on the company's culture. Employees
also compose and perform skits that illustrate how they would persevere
and serve their customers in difficult environs, such as war zones,
according to three Huawei employees.

In a research lab in Huawei's Shenzhen headquarters, a piece of framed
calligraphy on the wall reads: ``Sacrifice is a soldier's highest cause.
Victory is a soldier's greatest contribution.''

This intense work environment is not universally admired in China.
Internet users savaged Huawei after a 25-year-old employee died of
encephalitis in 2006. A spate of employee suicides led to more outrage
in the Chinese media.

When it comes to staff conduct at Huawei, there are ``red lines'' that
cannot be crossed under any circumstances, four employees told The
Times. These include disclosing company secrets and breaking laws and
sanctions.

But in company parlance, there are also ``yellow lines,'' employees say.
They say they are encouraged to ignore certain internal rules, such as a
ban on using gifts or other inducements to win customers, if it benefits
the firm to do so.

For some people at Huawei, these lines may have become blurred as the
company grew rapidly around the globe.

In 2002, Iraq's government submitted to the United Nations a
\href{https://www.nytimes.com/2002/12/08/world/threats-responses-arms-inspections-iraq-says-report-un-shows-no-banned-arms.html}{12,000-page
declaration} on its weapons program, and Huawei was reported to have
been named as \href{http://news.bbc.co.uk/2/hi/europe/2591351.stm}{one
of dozens of foreign companies} that broke an embargo and sold
technology to Saddam Hussein's regime.
\href{https://www.scmp.com/article/410267/huawei-broke-iraq-embargo}{The
company denied at the time} that it had supplied equipment to Iraq. It
said it had bid on two telecom projects in the country in 1999, but
withdrew for commercial reasons.

Another test came in 2003, when Huawei was
\href{https://www.nytimes.com/2003/01/24/business/technology-cisco-is-suing-a-competitor-based-in-china.html}{sued
by Cisco Systems}, the American maker of computer network equipment, for
allegedly copying its software and even language from its instruction
manuals. The two
sides\href{https://www.nytimes.com/2003/10/02/business/technology-cisco-agrees-to-suspend-patent-suit-for-6-months.html}{settled
out of court}.

A decade later, T-Mobile said that Huawei employees had photographed and
stolen a piece of a smartphone-testing robot named Tappy to help Huawei
produce its own robot.
\href{https://www.nytimes.com/2014/09/06/business/t-mobile-accuses-huawei-of-theft-from-laboratory.html}{Huawei
acknowledged the transgressions} and said the employees had been fired.
A jury later awarded T-Mobile \$4.8 million in damages.

Allegations of impropriety of other kinds trailed Huawei's expansion
into Africa. In Ghana,
\href{https://www.ghanaweb.com/GhanaHomePage/NewsArchive/AFAG-Exposes-NDC-Huawei-Corrupt-Deals-253492}{an
anticorruption group said} in 2012 that the company had sponsored the
governing party's election campaign in exchange for tax breaks. That
year, a Huawei executive was also convicted in Algeria of bribing an
official from a state-run telecom operator.

Huawei did not comment on the accusation in Ghana at the time. After the
Algerian court ruling, the company said it took the court's decision
``seriously'' and was
\href{http://www.information-age.com/huawei-and-zte-execs-convicted-of-bribery-in-algeria-2107858/}{reviewing
the outcome}.

In a \href{https://forum.huawei.com/zh/portal.php?mod=view\&aid=63}{2013
New Year's message that was published in an employee newspaper}, Guo
Ping, Huawei's chief executive at the time, acknowledged that rapid
growth had created problems and risks.

``Not long ago, high-speed growth was Huawei's priority,'' Mr. Guo said.
``This helped Huawei mature quickly, but it also caused Huawei's
management to become negligent.''

Now, he said, ``we must control the impulse to expand, and hold to
account managers who spread themselves too thin.''

By then, Huawei had said it had halted expansion in one particularly
sensitive market: Iran. Still, United States investigators now say the
company broke the law in connection with its business there.

Huawei entered the Iranian market
\href{http://ir.mofcom.gov.cn/article/ztdy/200511/20051100704444.shtml}{in
1999}. Within a decade, the Chinese Embassy in Tehran was boasting that
130 cities in the country were connected to Huawei's fiber optic
network.

``The Iranian telecom market's reliance on Huawei's products is growing
day by day,''
\href{http://ir.mofcom.gov.cn/article/jmxw/200906/20090606329688.shtml}{a
2009 article} on the embassy's website said. ``Huawei has become the
Iranian telecom market's main hardware supplier.''

Soon thereafter,
\href{https://www.nytimes.com/2010/06/10/world/middleeast/10sanctions.html}{the
United Nations} and the
\href{https://www.nytimes.com/2010/07/02/world/middleeast/02sanctions.html}{United
States} imposed new sanctions against Iran's nuclear program.
\href{https://www.bbc.com/zhongwen/trad/business/2011/12/111210_huawei_iran}{In
2011}, Huawei said it would not sign new contracts in the country,
citing the ``complicated'' situation there. It also said it would limit
its business with existing customers.

The accusations against Ms. Meng, Huawei's chief financial officer, stem
from events in 2013.

According to an affidavit that was made public during Ms. Meng's bail
hearing, Huawei used a company called Skycom as an unofficial subsidiary
for doing business in Iran. The filing, which contains information
provided by the United States, says that Ms. Meng concealed Skycom's
link to Huawei to reassure HSBC and other banks that Huawei was not
violating American sanctions against Iran.

As a result, HSBC and its American subsidiary had cleared more than
\$100 million in transactions with Skycom in Iran by 2014, the affidavit
says.

Huawei still has a presence in Iran. At a cellphone bazaar in Tehran is
a store that specializes in the company's devices.

Inside, a shopkeeper, Hamed Hajipour, says Huawei's phones are popular
in Iran. Mr. Hajipour, 29, has even had his name tattooed in Chinese
characters on his arm.

``I love everything about China,'' he said. ``It's a great and powerful
country.''

Advertisement

\protect\hyperlink{after-bottom}{Continue reading the main story}

\hypertarget{site-index}{%
\subsection{Site Index}\label{site-index}}

\hypertarget{site-information-navigation}{%
\subsection{Site Information
Navigation}\label{site-information-navigation}}

\begin{itemize}
\tightlist
\item
  \href{https://help.nytimes.com/hc/en-us/articles/115014792127-Copyright-notice}{©~2020~The
  New York Times Company}
\end{itemize}

\begin{itemize}
\tightlist
\item
  \href{https://www.nytco.com/}{NYTCo}
\item
  \href{https://help.nytimes.com/hc/en-us/articles/115015385887-Contact-Us}{Contact
  Us}
\item
  \href{https://www.nytco.com/careers/}{Work with us}
\item
  \href{https://nytmediakit.com/}{Advertise}
\item
  \href{http://www.tbrandstudio.com/}{T Brand Studio}
\item
  \href{https://www.nytimes.com/privacy/cookie-policy\#how-do-i-manage-trackers}{Your
  Ad Choices}
\item
  \href{https://www.nytimes.com/privacy}{Privacy}
\item
  \href{https://help.nytimes.com/hc/en-us/articles/115014893428-Terms-of-service}{Terms
  of Service}
\item
  \href{https://help.nytimes.com/hc/en-us/articles/115014893968-Terms-of-sale}{Terms
  of Sale}
\item
  \href{https://spiderbites.nytimes.com}{Site Map}
\item
  \href{https://help.nytimes.com/hc/en-us}{Help}
\item
  \href{https://www.nytimes.com/subscription?campaignId=37WXW}{Subscriptions}
\end{itemize}
