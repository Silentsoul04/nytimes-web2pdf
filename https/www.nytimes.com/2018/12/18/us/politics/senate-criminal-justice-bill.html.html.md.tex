Sections

SEARCH

\protect\hyperlink{site-content}{Skip to
content}\protect\hyperlink{site-index}{Skip to site index}

\href{https://www.nytimes.com/section/politics}{Politics}

\href{https://myaccount.nytimes.com/auth/login?response_type=cookie\&client_id=vi}{}

\href{https://www.nytimes.com/section/todayspaper}{Today's Paper}

\href{/section/politics}{Politics}\textbar{}Senate Passes Bipartisan
Criminal Justice Bill

\url{https://nyti.ms/2GqmNnb}

\begin{itemize}
\item
\item
\item
\item
\item
\item
\end{itemize}

Advertisement

\protect\hyperlink{after-top}{Continue reading the main story}

Supported by

\protect\hyperlink{after-sponsor}{Continue reading the main story}

\hypertarget{senate-passes-bipartisan-criminal-justice-bill}{%
\section{Senate Passes Bipartisan Criminal Justice
Bill}\label{senate-passes-bipartisan-criminal-justice-bill}}

\includegraphics{https://static01.nyt.com/images/2018/12/22/us/22dc-criminal3/19dc-criminal3-articleLarge.jpg?quality=75\&auto=webp\&disable=upscale}

By \href{https://www.nytimes.com/by/nicholas-fandos}{Nicholas Fandos}

\begin{itemize}
\item
  Dec. 18, 2018
\item
  \begin{itemize}
  \item
  \item
  \item
  \item
  \item
  \item
  \end{itemize}
\end{itemize}

WASHINGTON --- The Senate overwhelmingly approved on Tuesday the most
substantial changes in a generation to the tough-on-crime prison and
sentencing laws that ballooned the federal prison population and created
a criminal justice system that many conservatives and liberals view as
costly and unfair.

The
\href{https://www.judiciary.senate.gov/download/revised-first-step-act-of-2018}{First
Step Act} would expand job training and other programming aimed at
reducing recidivism rates among federal prisoners. It also expands
early-release programs and modifies sentencing laws, including mandatory
minimum sentences for nonviolent drug offenders, to more equitably
punish drug offenders.

But the legislation falls short of benchmarks set by a more expansive
overhaul proposed in Congress during Barack Obama's presidency and of
the kinds of changes sought by some liberal and conservative activists
targeting mass incarceration.

House leaders have pledged to pass the measure this week, and President
Trump,
\href{https://www.nytimes.com/2018/11/14/us/politics/prison-sentencing-trump.html}{whose
support resuscitated a yearslong overhaul effort last month}, said he
would sign the bill.

Even as both sides acknowledged concessions, Tuesday's vote was an
important first step for the unlikely coalition of liberals and
conservatives --- including the American Civil Liberties Union, the
American Conservative Union, Koch brothers and the liberal Center for
American Progress --- who locked arms in recent years and pushed
lawmakers to reconsider the way the federal government administers
justice three decades after the war on crime peaked. In one of this
Congress's final acts, every Democrat and all but 12 Republicans voted
in favor of the legislation --- an outcome that looked highly unlikely
this month amid skepticism from Republican leaders.

For Republicans preparing to relinquish total control of Washington next
month, the bill's passage offered one final victory on their own terms
and handed Mr. Trump a bipartisan policy achievement that he can tout as
he seeks re-election. Liberals saw reason to celebrate, as well, even as
they called for more aggressive changes: In gaining the support of Mr.
Trump and so many Senate Republicans, they believe they have shifted the
terms of policy debates around criminal justice in a way that could set
the stage for additional changes on the federal level and in the states.

``This bill in its entirety has been endorsed by the political spectrum
of America,'' said Senator Richard J. Durbin, Democrat of Illinois, who
has led the push for changes along with two Republicans, Senators
Charles E. Grassley of Iowa and Mike Lee of Utah. ``I can't remember any
bill that has this kind of support, left and right, liberal and
conservative, Democrat and Republican.''

Mr. Trump quickly touted the vote on Twitter, saying that the changes
would ``keep our communities safer, and provide hope and a second
chance, to those who earn it.''

\includegraphics{https://static01.nyt.com/images/2018/12/20/us/politics/20dc-criminal/merlin_148295469_84478afd-f9a7-4400-af9a-b69d3af68d86-articleLarge.jpg?quality=75\&auto=webp\&disable=upscale}

Proponents of the bill overcame an aggressive campaign by some
conservatives who tried to resurrect the once-resonant charge that
reducing sentences would make the United States less safe. Two
Republicans, Senators Tom Cotton of Arkansas and John Kennedy of
Louisiana, introduced amendments to limit which types of offenders would
be eligible for early-release programs or to water down other changes.
All were narrowly voted down on the Senate floor.

Many of the changes adopted by the Senate and embraced by Mr. Trump are
modeled after successful initiatives at the state level intended to
reduce the costs and improve the outcomes of the criminal justice
system. Congress's action would not directly affect state prisons, where
the majority of the country's offenders are incarcerated, but proponents
believe they could spur more states to change their laws.

Once signed into law, thousands of inmates will be eligible for
immediate sentencing reductions and expanded early-release programs.
Going forward, the effect will grow as thousands of new offenders
receive reduced sentences and enter a changed prison system.

``We're not just talking about money. We're talking about human
potential,'' Senator John Cornyn of Texas, the chamber's No. 2
Republican, said Tuesday during debate on the Senate floor. ``We're
investing in the men and women who want to turn their lives around once
they're released from prison, and we're investing in so doing in
stronger and more viable communities.''

Broadly speaking, the First Step Act makes heavy investments in a
package of incentives and new programs intended to improve prison
conditions and better prepare low-risk prisoners for re-entry into their
communities.

By participating in the programs, eligible prisoners can earn time
credits to reduce their sentence or enter ``prerelease custody,'' such
as home confinement. In recent weeks, conservative senators and law
enforcement groups successfully pushed to limit some violent offenders
from eligibility, including fentanyl traffickers.

The legislation would also prohibit the shackling of pregnant inmates
and the use of solitary confinement for juveniles in almost all cases.
The Bureau of Prisons would be required to place prisoners in facilities
close to their homes, if possible.

In all, it includes four changes to federal sentencing laws. One would
shorten mandatory minimum sentences for some nonviolent drug offenses,
including lowering the mandatory ``three strikes'' penalty from life in
prison to 25 years. Another would provide judges greater liberty to use
so-called safety valves to go around mandatory minimums in some cases.
The bill would also clarify that the so-called stacking mechanism making
it a federal crime to possess a firearm while committing another crime,
like a drug offense, should apply only to individuals who have
previously been convicted.

Finally, the bill would allow offenders sentenced before a 2010
reduction in the sentencing disparity between crack and powder cocaine
to petition for their cases to be re-evaluated. The provision could
alter the sentences of several thousand drug offenders serving lengthy
sentences for crack-cocaine offenses. That would help many
African-American offenders who were disproportionately punished for
crack dealing while white drug dealers got off easier for selling powder
cocaine.

Image

Senator John Kennedy, Republican of Louisiana, had introduced an
amendment to the bill that was narrowly voted down.Credit...Sarah
Silbiger/The New York Times

For the bill's supporters, Tuesday's vote was the culmination of a
five-year campaign on Capitol Hill that only months ago appeared to be
out of reach while Mr. Trump was in office.

Much of the same coalition that pushed the First Step Act had rallied
around similar legislation, the Sentencing Reform and Corrections Act of
2015. With Mr. Obama's support, as well as that of Mr. Grassley and
Speaker Paul D. Ryan, Republican of Wisconsin, the more expansive bill
had appeared destined for passage before Senator Mitch McConnell,
Republican of Kentucky and majority leader, stepped in and refused to
give it a vote in the run-up to the 2016 election.

Mr. McConnell seemed intent on denying proponents another shot this
year, but they secured a powerful ally early on in Mr. Trump's
son-in-law, Jared Kushner.

Over the course of the past year, Mr. Kushner worked with Mr. Grassley,
Mr. Durbin and Senator Mike Lee, Republican of Utah, to draft a
compromise that the president could back. With Mr. Trump's endorsement,
the group brought a strong majority of Senate Republicans on board. By
last week,
\href{https://www.nytimes.com/2018/12/06/us/politics/mcconnell-criminal-justice-bill.html}{under
intense pressure from his own party and the White House}, Mr. McConnell
relented. And on Tuesday, facing his own re-election fight in 2020, he
somewhat unexpectedly cast his own vote in favor of the bill.

\emph{{[}}\href{https://www.nytimes.com/2018/12/14/us/politics/jared-kushner-criminal-justice-bill.html}{\emph{For
Jared Kushner, the criminal justice overhaul has been a personal issue
and a rare victory.}}\emph{{]}}

``This is the biggest thing,'' a jubilant Mr. Grassley said after the
vote, showing off a vote card to reporters. ``Except maybe getting a
Supreme Court justice.''

He embraced another Democrat central to its passage, Senator Cory
Booker, Democrat of New Jersey and one of only three African-American
senators.

``This is a very moving night for me,'' Mr. Booker said. ``This is
literally one of the reasons I came to the United States Senate, to get
something like this done.''

For Democrats and Republicans who favored greater changes, Mr. Trump's
endorsement came at a cost: They had to scale back their proposed
sentencing changes. The 2015 bill made all sentencing reductions
retroactive to include those currently in prison, but the bill passed on
Tuesday limits most of those changes to future offenders.

But by winning the support of a tough-talking, anticrime president who
enjoys deep loyalty among Republican voters, the groups believe they
have shifted the debate in a way that could set the stage for additional
changes and elevate the criminal justice debate before the 2020
Democratic primaries.

Advertisement

\protect\hyperlink{after-bottom}{Continue reading the main story}

\hypertarget{site-index}{%
\subsection{Site Index}\label{site-index}}

\hypertarget{site-information-navigation}{%
\subsection{Site Information
Navigation}\label{site-information-navigation}}

\begin{itemize}
\tightlist
\item
  \href{https://help.nytimes.com/hc/en-us/articles/115014792127-Copyright-notice}{©~2020~The
  New York Times Company}
\end{itemize}

\begin{itemize}
\tightlist
\item
  \href{https://www.nytco.com/}{NYTCo}
\item
  \href{https://help.nytimes.com/hc/en-us/articles/115015385887-Contact-Us}{Contact
  Us}
\item
  \href{https://www.nytco.com/careers/}{Work with us}
\item
  \href{https://nytmediakit.com/}{Advertise}
\item
  \href{http://www.tbrandstudio.com/}{T Brand Studio}
\item
  \href{https://www.nytimes.com/privacy/cookie-policy\#how-do-i-manage-trackers}{Your
  Ad Choices}
\item
  \href{https://www.nytimes.com/privacy}{Privacy}
\item
  \href{https://help.nytimes.com/hc/en-us/articles/115014893428-Terms-of-service}{Terms
  of Service}
\item
  \href{https://help.nytimes.com/hc/en-us/articles/115014893968-Terms-of-sale}{Terms
  of Sale}
\item
  \href{https://spiderbites.nytimes.com}{Site Map}
\item
  \href{https://help.nytimes.com/hc/en-us}{Help}
\item
  \href{https://www.nytimes.com/subscription?campaignId=37WXW}{Subscriptions}
\end{itemize}
