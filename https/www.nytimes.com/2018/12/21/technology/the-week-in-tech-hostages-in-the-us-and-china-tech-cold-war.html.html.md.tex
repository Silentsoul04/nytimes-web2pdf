Sections

SEARCH

\protect\hyperlink{site-content}{Skip to
content}\protect\hyperlink{site-index}{Skip to site index}

\href{https://www.nytimes.com/section/technology}{Technology}

\href{https://myaccount.nytimes.com/auth/login?response_type=cookie\&client_id=vi}{}

\href{https://www.nytimes.com/section/todayspaper}{Today's Paper}

\href{/section/technology}{Technology}\textbar{}The Week in Tech:
Hostages in the U.S.-China Tech Cold War

\url{https://nyti.ms/2Ri5BEN}

\begin{itemize}
\item
\item
\item
\item
\item
\end{itemize}

Advertisement

\protect\hyperlink{after-top}{Continue reading the main story}

Supported by

\protect\hyperlink{after-sponsor}{Continue reading the main story}

\href{/column/bits}{Bits}

\hypertarget{the-week-in-tech-hostages-in-the-us-china-tech-cold-war}{%
\section{The Week in Tech: Hostages in the U.S.-China Tech Cold
War}\label{the-week-in-tech-hostages-in-the-us-china-tech-cold-war}}

\includegraphics{https://static01.nyt.com/images/2018/12/22/business/22technewsletter01/merlin_148071741_eae2830e-ed09-4591-8154-7df2258f9b6c-articleLarge.jpg?quality=75\&auto=webp\&disable=upscale}

By \href{https://www.nytimes.com/by/raymond-zhong}{Raymond Zhong}

\begin{itemize}
\item
  Dec. 21, 2018
\item
  \begin{itemize}
  \item
  \item
  \item
  \item
  \item
  \end{itemize}
\end{itemize}

\emph{Each week, technology reporters and columnists from The New York
Times review}
\href{https://www.nytimes.com/section/technology?module=inline}{\emph{the
week's news}}\emph{, offering analysis and maybe a joke or two about the
most important developments in the tech industry. Want this newsletter
in your inbox?}
\href{https://www.nytimes.com/newsletters/bits?emc=edit_tu_20171005\&nl=bits\&nlid=646676\&te=1\&module=inline}{\emph{Sign
up here}}\emph{.}

Wise readers! I'm Raymond Zhong, a Times reporter in China --- where, I
contend, something is unfolding right now that carries higher stakes
than any other tech story on the planet.

I know, I'm biased. But hear me out.

China detained
\href{https://www.nytimes.com/2018/12/19/us/politics/china-canada-huawei.html}{another
Canadian citizen} this week, the third to have been snatched up in the
country
\href{https://www.nytimes.com/2018/12/14/world/canada/china-detained-spavor-kovrig.html}{this
month}. Beijing denies it, but most people in China see the detentions
as retaliation for
\href{https://www.nytimes.com/2018/12/05/business/huawei-cfo-arrest-canada-extradition.html}{Canada's
arrest of Meng Wanzhou}, a top executive at the Chinese tech giant
Huawei. The United States has accused Ms. Meng of fraud, and is seeking
her extradition. She remains in Canada under
\href{https://www.nytimes.com/2018/12/11/technology/huawei-executive-canada-bail-decision.html}{round-the-clock
surveillance}.

In other words, the United States and China's contest for technological
supremacy has now left four lives hanging in the balance --- hostages to
something much larger than themselves.

To recap: Huawei is the world's leading maker of telecom network
equipment. You probably don't ever think about this technology, but it
powers our mobile age. Which is why the United States is so nervous that
Huawei has become such a powerhouse supplier. Washington has warned for
years, without specific evidence, that Huawei's products can be used for
spying by the Chinese government. Ms. Meng's arrest this month was part
of a
\href{https://www.nytimes.com/2018/12/14/business/huawei-meng-hsbc-canada.html}{yearslong
effort} to curb the company's rise.

It's an unenviable position for any company to be in --- not least one,
like Huawei, that is privately owned and was
\href{https://www.nytimes.com/2018/12/07/technology/meng-wanzhou-huawei-arrest.html}{long
known for secretiveness}.

In an effort at openness, Huawei this week invited me and a bunch of
other reporters to its campuses in southern China, where we toured some
research facilities and chatted with Ken Hu, the company's deputy
chairman.

Mr. Hu reiterated points that the company has made for years as it has
come under
\href{https://www.nytimes.com/2018/12/06/technology/huawei-arrest-meng-wanzhou.html}{scrutiny
around the world}. He said that Huawei had a clean record when it came
to major cybersecurity incidents. He said that ``ideological'' or
``geopolitical'' fears had not cost Huawei the trust of its many
customers.

``Concerns about security should not be aimed without basis at any
specific company,'' Mr. Hu said. He issued a challenge to governments
that raise concerns about Huawei's products: ``If you have evidence,
then make it public.''

He declined to discuss the detained Canadians or Ms. Meng's arrest.

In a way, Huawei is itself a hostage of
\href{https://www.nytimes.com/2018/12/10/business/huawei-meng-arrest-travel.html}{larger
conflicts}. The company has built a globally respected brand. It has won
customers by investing in research and development, providing attentive
service --- and driving its employees really, really hard.
(\href{https://www.nytimes.com/2018/12/18/technology/huawei-workers-iran-sanctions.html}{I
wrote this week about Huawei's intense corporate culture}.)

Yet to the company's fiercest critics, Huawei is tarnished simply by
being Chinese, and hence within arm's reach of a government that
\href{https://www.nytimes.com/2018/11/29/us/politics/china-trump-cyberespionage.html}{conducts
aggressive espionage} against American companies and government
agencies. For some people in Washington, it hardly matters that Huawei
isn't state-owned, or that the Chinese government has never asked it to
spy on its behalf. The mere possibility is enough.

As part of this week's visit to Huawei, we reporters were treated to a
long presentation on the company's processes for evaluating products for
security risks. It was a barrage of details, earnestly presented, that I
suspect would have zero chance of changing the mind of anyone in
Washington about Huawei.

Here's what else caught my eye this week:

■ My colleagues at The Times have produced
\href{https://www.nytimes.com/2018/12/18/technology/facebook-privacy.html}{another
blockbuster article} full of revelations about how user data is
collected and shared by giant tech companies.
\href{https://www.nytimes.com/2018/12/18/us/politics/facebook-data-sharing-deals.html}{Here
are five takeaways}.

■ The Wall Street Journal
\href{https://www.wsj.com/articles/its-been-a-rout-apple-stumbles-in-worlds-largest-untapped-market-11545146399}{took
a look} at Apple's near-total failure to win over smartphone buyers in a
giant, fast-growing market: India. The iPhone is clinging to market
share there of around 1 percent, and the company's revenue in India is
half of what executives once hoped for, according to The Journal's
sources. The country simply doesn't have enough people who are willing
to pay Apple prices.

■ Well, it was fun while it lasted. TikTok,
\href{https://www.nytimes.com/2018/12/03/technology/tiktok-a-chinese-video-app-brings-fun-back-to-social-media.html}{the
quirky short video app} that is now
\href{https://www.nytimes.com/2018/10/29/technology/bytedance-app-funding-china.html}{a
worldwide hit}, has
\href{https://motherboard.vice.com/en_us/article/yw74gy/tiktok-neo-nazis-white-supremacy}{a
Nazi problem}, according to Motherboard. The app's Chinese parent
company, Bytedance, is no stranger to controversies about
\href{https://www.nytimes.com/2018/01/02/business/china-toutiao-censorship.html}{gnarly
content on its platforms}.

■ Finally, I urge you to read this
\href{https://www.nytimes.com/2018/12/17/science/donald-knuth-computers-algorithms-programming.html}{profile
of Donald E. Knuth}, the legendary computer scientist who, for the past
50 years, has been writing ``The Art of Computer Programming'' --- a
multivolume, still-unspooling bible of its field.

\emph{Raymond Zhong is a reporter for The New York Times in China.
Follow him on Twitter at}
\href{https://twitter.com/zhonggg}{\emph{@zhonggg}}

Advertisement

\protect\hyperlink{after-bottom}{Continue reading the main story}

\hypertarget{site-index}{%
\subsection{Site Index}\label{site-index}}

\hypertarget{site-information-navigation}{%
\subsection{Site Information
Navigation}\label{site-information-navigation}}

\begin{itemize}
\tightlist
\item
  \href{https://help.nytimes.com/hc/en-us/articles/115014792127-Copyright-notice}{©~2020~The
  New York Times Company}
\end{itemize}

\begin{itemize}
\tightlist
\item
  \href{https://www.nytco.com/}{NYTCo}
\item
  \href{https://help.nytimes.com/hc/en-us/articles/115015385887-Contact-Us}{Contact
  Us}
\item
  \href{https://www.nytco.com/careers/}{Work with us}
\item
  \href{https://nytmediakit.com/}{Advertise}
\item
  \href{http://www.tbrandstudio.com/}{T Brand Studio}
\item
  \href{https://www.nytimes.com/privacy/cookie-policy\#how-do-i-manage-trackers}{Your
  Ad Choices}
\item
  \href{https://www.nytimes.com/privacy}{Privacy}
\item
  \href{https://help.nytimes.com/hc/en-us/articles/115014893428-Terms-of-service}{Terms
  of Service}
\item
  \href{https://help.nytimes.com/hc/en-us/articles/115014893968-Terms-of-sale}{Terms
  of Sale}
\item
  \href{https://spiderbites.nytimes.com}{Site Map}
\item
  \href{https://help.nytimes.com/hc/en-us}{Help}
\item
  \href{https://www.nytimes.com/subscription?campaignId=37WXW}{Subscriptions}
\end{itemize}
