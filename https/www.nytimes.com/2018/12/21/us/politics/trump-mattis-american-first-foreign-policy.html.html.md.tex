Sections

SEARCH

\protect\hyperlink{site-content}{Skip to
content}\protect\hyperlink{site-index}{Skip to site index}

\href{https://www.nytimes.com/section/politics}{Politics}

\href{https://myaccount.nytimes.com/auth/login?response_type=cookie\&client_id=vi}{}

\href{https://www.nytimes.com/section/todayspaper}{Today's Paper}

\href{/section/politics}{Politics}\textbar{}With the Generals Gone,
Trump's `America First' Could Fully Emerge

\url{https://nyti.ms/2GBeNQh}

\begin{itemize}
\item
\item
\item
\item
\item
\item
\end{itemize}

Advertisement

\protect\hyperlink{after-top}{Continue reading the main story}

Supported by

\protect\hyperlink{after-sponsor}{Continue reading the main story}

News Analysis

\hypertarget{with-the-generals-gone-trumps-america-first-could-fully-emerge}{%
\section{With the Generals Gone, Trump's `America First' Could Fully
Emerge}\label{with-the-generals-gone-trumps-america-first-could-fully-emerge}}

\includegraphics{https://static01.nyt.com/images/2018/12/21/us/politics/21dc-mattis-print-copy/00dc-mattis-hfo-videoSixteenByNine3000.jpg}

By \href{https://www.nytimes.com/by/david-e-sanger}{David E. Sanger}

\begin{itemize}
\item
  Dec. 21, 2018
\item
  \begin{itemize}
  \item
  \item
  \item
  \item
  \item
  \item
  \end{itemize}
\end{itemize}

WASHINGTON --- With the angry
\href{https://www.nytimes.com/2018/12/20/us/politics/jim-mattis-defense-secretary-trump.html?action=click\&module=Spotlight\&pgtype=Homepage}{departure
of Defense Secretary Jim Mattis}, the United States and its shaken
allies are about to discover the true meaning of ``America First.''

Mr. Mattis, a retired four-star general, prided himself on spending four
decades preparing for war while nurturing the alliances needed to
prevent conflict. He was more than the competent grown-up in the
Situation Room, quelling talk of unilateral strikes against North Korea.
In fact, he was the last senior official in the administration deeply
invested in the world order that the United States has led for the 73
years since World War II, and the global footprint needed to keep that
order together.

The breaking point was Syria, where Mr. Trump decided over his defense
secretary's objections to pull out all American troops, and Afghanistan,
where the president seems determined to reduce the American presence by
half in the next few months. By the time Mr. Trump made clear he would
delay those actions no longer, Mr. Mattis was isolated.

He was not alone: Most of the advisers Mr. Trump once called ``my
generals'' also believed in the worldview that Mr. Trump has long
rejected. And now, headed into his third year in office and more
convinced than ever that his initial gut instincts about retreating from
a complex world of civil wars and abstract threats was right, Mr. Trump
has rid himself of the aides who feared the president was undercutting
America's long-term national interests.

Now the president appears determined to assemble a new team of advisers
who will not tell him what he cannot do, but rather embrace his vision
of a powerful America that will amass a military that will enforce
national sovereignty and bolster American deal-making --- but not spend
time nurturing the alliance relationships that Mr. Mattis, in a
\href{https://www.nytimes.com/2018/12/20/us/politics/letter-jim-mattis-trump.html?action=click\&module=Spotlight\&pgtype=Homepage}{remarkable
resignation letter}, makes clear are at the core of American power.

Predicting how far Mr. Trump will take his ``America First'' vision is
risky, but there are some clear hints.

Pulling completely out of Afghanistan is entirely within the realm of
possibility, foreign diplomats and Pentagon officials say --- and Mr.
Trump appears to be halfway there with the planned troop reduction. But
that may be only a start.

Mr. Trump has often threatened to pull back forces from the Pacific,
wondering why he should be paying to defend Japan and South Korea,
especially given the fact that the United States has trade deficits with
both.

He may also be inclined to reassess his approach to that part of the
world in the wake of
\href{https://www.nytimes.com/2018/12/20/world/asia/north-korea-denuclearization.html?module=inline}{North
Korea's declaration the other day that its interpretation of
``denuclearization,''} the goal of Mr. Trump's outreach to Pyongyang,
was far broader than Washington's, further dimming hopes for a deal with
Kim Jong-un, the North Korean leader.

Chafing at the limits imposed by arms control treaties signed by
predecessors back to Ronald Reagan, he could decide to resume a nuclear
arms race --- one aimed more at China than Russia --- if the
administration goes ahead with its threat
\href{https://www.nytimes.com/2018/12/09/us/politics/trump-nuclear-arms-treaty-russia.html}{to
suspend the Intermediate Nuclear Forces treaty in early February.}
Feeling equally encumbered by longstanding agreements with allies,
especially those he views as not carrying their weight, he could
threaten to exit alliances.

``Who will persuade Trump not to withdraw from NATO?'' Daniel B.
Shapiro, the former American ambassador to Israel, asked in a
\href{https://twitter.com/DanielBShapiro/status/1076167846722523137}{tweet}
on Friday as the implications of the Mattis resignation sunk in.
``Really scary possibility, no longer theoretical.''

To Mr. Mattis, alliances were a force-multiplier. To Mr. Trump, they are
mostly a burden.

``I think the question for any future secretary of defense --- or any of
those going onto the Trump team now --- is whether they want to be like
Jim Mattis and try to defend the principles he defended, starting with
alliances, or get on board with the President's approach,'' Leon
Panetta, who served as defense secretary, C.I.A. director and White
House chief of staff during a long career, said by telephone Thursday
night. ``While the president tweeted, Mattis went around the world
reassuring people that they could wink at the statements and know that
America was going to be there to steady the ship.''

Mr. Panetta paused. ``Until he couldn't keep that going any more,'' he
said.

The national security adviser, John R. Bolton, made no secret of his
deep suspicion of international institutions like the United Nations,
NATO, and the European Union. After an initial, awkward meeting at the
Pentagon with Mr. Mattis, the two men often steered clear of each other.

\includegraphics{https://static01.nyt.com/images/2018/12/22/us/politics/22DC-DIPLO1/22DC-DIPLO1-articleLarge.jpg?quality=75\&auto=webp\&disable=upscale}

The secretary of state, Mike Pompeo, is more artful at straddling the
line, talking up Mr. Trump's view of America's role in the world while
quietly working to channel the president's most extreme instincts. But
while he objected to the Syria decision, he defended it, if weakly, on a
series of friendly radio and TV interviews on Thursday.

In retrospect, the clash of world views between Mr. Trump and Mr. Mattis
was inevitable. Mr. Trump made his views clear from the early days of
the campaign, when he railed about the Iran nuclear deal as a
``terrible'' giveaway for the United States, criticized NATO as an
alliance of freeloaders and described the presence of American troops in
Asia as nonsensical, because the United States ran trade deficits with
Japan and South Korea.

Mr. Mattis, in contrast, was an institutionalist --- as were H.R.
McMaster, the retired lieutenant general who served as national security
adviser, and Rex Tillerson, the Exxon Mobil chief who never figured out
how to run the State Department, but recently said he spent most of his
time trying to talk Mr. Trump down from illegal acts on the world stage.

The three men never got along. But they all believed America's strength
lay in its role leading NATO, or the anti-Islamic State alliance, or
keeping the peace in the Pacific by making it clear to North Korea and
China that the Navy was just over the horizon.

Mr. Mattis and Mr. McMaster were authors of a national security strategy
that Mr. Trump issued but never embraced, one that said dealing with the
``revisionist'' powers of Russia and China, not combating terrorism, was
once again the primary objective of American national security policy.

``We are moving back to an earlier conception of America's role in the
world, looking out for ourselves, hoping the two oceans protect us, and
when necessary saying the rest of the world is full of freeloaders who
can go to hell if they don't get on board,'' said Robert Kagan, a
conservative foreign policy expert whose books, ``The World America
Made'' and ``The Jungle Grows Back,'' chronicle the ebbs and flows of
American influence.

``It may be an era more destructive of the world order than in the
1930s,'' he said. ``Back then, at least Britain and France were
responsible for keeping part of the order. Now we are the responsible
world power --- and we are undermining it.''

Mr. Mattis almost never repeated the ``America First'' line that his
boss found so attractive. But he also rarely openly contradicted the
president. He was more subtle. When he received orders he believed
destructive --- for example, the presidential tweet that seemed to ban
transgender soldiers from serving --- he would slow-walk the process,
forming a committee to study the issue, then issuing watered-down
directives.

By dwelling in his resignation letter on the value of the NATO alliance,
the coalition to fight the Islamic state, and the need to be cleareyed
about Russia and China, he was aiming at the heart of his differences
with Mr. Trump.

``Whenever I said `Trump is destroying the Atlantic alliance,' '' Mr.
Kagan said, ``people would tell me, `At least there is Mattis.' ''

In fact, Mr. Mattis was the man who worked up the plans to circulate
troops through Eastern Europe, as a signal to President Vladimir V.
Putin of Russia. It was Mr. Mattis who helped establish what was
supposed to be a long-term presence in Afghanistan, to convince the
Taliban they would have to negotiate a peace.

But Mr. Mattis was also a cautious player. He clashed with Mr. McMaster
over the defense secretary's refusal to order the military to hail and
board North Korean ships suspected of carrying goods that violated the
embargoes on the country. And whenever people talked about unilateral
strikes against North Korea, it was Mr. Mattis who would warn, darkly,
of the potential cost to millions of lives in Seoul.

Now the question is whether Mr. Trump will conclude that his experiment
with the generals --- he told an interviewer during the 2016 campaign
that he liked them because they ``represent power'' --- is truly over.

It turns out that long military careers usually create a different view
of the way the world works than long careers in commercial real estate.
And it is a view that Mr. Trump still rejects, even more vociferously
than he did when he was running for office.

Advertisement

\protect\hyperlink{after-bottom}{Continue reading the main story}

\hypertarget{site-index}{%
\subsection{Site Index}\label{site-index}}

\hypertarget{site-information-navigation}{%
\subsection{Site Information
Navigation}\label{site-information-navigation}}

\begin{itemize}
\tightlist
\item
  \href{https://help.nytimes.com/hc/en-us/articles/115014792127-Copyright-notice}{©~2020~The
  New York Times Company}
\end{itemize}

\begin{itemize}
\tightlist
\item
  \href{https://www.nytco.com/}{NYTCo}
\item
  \href{https://help.nytimes.com/hc/en-us/articles/115015385887-Contact-Us}{Contact
  Us}
\item
  \href{https://www.nytco.com/careers/}{Work with us}
\item
  \href{https://nytmediakit.com/}{Advertise}
\item
  \href{http://www.tbrandstudio.com/}{T Brand Studio}
\item
  \href{https://www.nytimes.com/privacy/cookie-policy\#how-do-i-manage-trackers}{Your
  Ad Choices}
\item
  \href{https://www.nytimes.com/privacy}{Privacy}
\item
  \href{https://help.nytimes.com/hc/en-us/articles/115014893428-Terms-of-service}{Terms
  of Service}
\item
  \href{https://help.nytimes.com/hc/en-us/articles/115014893968-Terms-of-sale}{Terms
  of Sale}
\item
  \href{https://spiderbites.nytimes.com}{Site Map}
\item
  \href{https://help.nytimes.com/hc/en-us}{Help}
\item
  \href{https://www.nytimes.com/subscription?campaignId=37WXW}{Subscriptions}
\end{itemize}
