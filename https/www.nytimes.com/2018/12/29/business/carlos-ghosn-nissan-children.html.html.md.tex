Sections

SEARCH

\protect\hyperlink{site-content}{Skip to
content}\protect\hyperlink{site-index}{Skip to site index}

\href{https://www.nytimes.com/section/business}{Business}

\href{https://myaccount.nytimes.com/auth/login?response_type=cookie\&client_id=vi}{}

\href{https://www.nytimes.com/section/todayspaper}{Today's Paper}

\href{/section/business}{Business}\textbar{}Carlos Ghosn's Daughters See
a Nissan Revolt Behind His Arrest

\url{https://nyti.ms/2GMT8og}

\begin{itemize}
\item
\item
\item
\item
\item
\end{itemize}

Advertisement

\protect\hyperlink{after-top}{Continue reading the main story}

Supported by

\protect\hyperlink{after-sponsor}{Continue reading the main story}

\hypertarget{carlos-ghosns-daughters-see-a-nissan-revolt-behind-his-arrest}{%
\section{Carlos Ghosn's Daughters See a Nissan Revolt Behind His
Arrest}\label{carlos-ghosns-daughters-see-a-nissan-revolt-behind-his-arrest}}

\includegraphics{https://static01.nyt.com/images/2018/12/29/business/29-SUB-ghosnkids-print-dress/29-SUB-ghosnkids-print-dress-articleLarge.jpg?quality=75\&auto=webp\&disable=upscale}

By \href{https://www.nytimes.com/by/amy-chozick}{Amy Chozick}

\begin{itemize}
\item
  Dec. 29, 2018
\item
  \begin{itemize}
  \item
  \item
  \item
  \item
  \item
  \end{itemize}
\end{itemize}

The children of Carlos Ghosn, the jailed auto executive who oversaw an
alliance that sold more than 10 million cars a year, believe accusations
of financial misconduct against him are part of a revolt within Nissan
against exploring a possible merger with Renault.

Caroline Ghosn, the eldest of Mr. Ghosn's four children, said that when
she saw Hiroto Saikawa, the chief executive of Nissan, condemn her
father during a
\href{https://www.youtube.com/watch?v=QNjO4vyjU0w}{televised news
conference} after his arrest, she suspected that Nissan's investigation
was rooted in opposition to proposed changes to the Nissan-Renault
alliance and ``the merger my dad was setting up.''

``For Saikawa to so adamantly denounce someone who had been his mentor
and then immediately without any benefit of the doubt condemns him?''
Ms. Ghosn, 31, said in a phone interview. An entrepreneur, she had
awakened hours before that briefing to the news that her father, who was
Nissan's chairman and Renault's chief executive, had been arrested on
suspicion of violating Japan's financial reporting laws.

\emph{{[}}\href{https://www.nytimes.com/2018/12/30/business/carlos-ghosn-nissan.html}{\emph{Read
about the rise and fall of Mr. Ghosn}}\emph{, who wasn't supposed to
succeed in Japan but never expected to fail like this.{]}}

She and her sister Maya Ghosn, 26, do not have direct knowledge of their
father's business discussions, but both said watching Mr. Saikawa
address the national news media had cemented their belief that internal
company dynamics were at play.

\href{https://www.youtube.com/watch?v=vRkJtTrhQBk}{Mr. Saikawa told the
reporters} that one problem with the alliance was that ``the top of
Renault is concurrently serving as the top of Nissan with 43 percent of
shares.'' In the future, he said, the company would ``look for a more
sustainable structure.''

``Wow,'' Caroline Ghosn said. ``He didn't even waste a breath. He didn't
even try to cover up the fact that the merger had something to do with
this.''

Maya Ghosn, who works in philanthropy, agreed. As Mr. Saikawa was
``talking about the alliance, it was clear to me that there was way more
associated with it,'' she said. ``My gut reaction was that this was
bigger than the accusations against my dad.''

The interviews were the first time since the arrest that the sisters,
now living in San Francisco, have spoken publicly about their father,
who was deposed as chairman of the board of Nissan after creating an
empire that included Nissan, Renault and Mitsubishi.

\includegraphics{https://static01.nyt.com/images/2018/12/28/business/29ghosnkids-02/merlin_147075417_a89c4034-3737-42fb-bbb4-5ead6eb73366-articleLarge.jpg?quality=75\&auto=webp\&disable=upscale}

Mr. Ghosn, 64,
\href{https://www.cnbc.com/2018/11/19/renault-and-nissan-carlos-ghosn-to-be-arrested-for-financial-violations.html}{was
arrested on Nov. 19} as he arrived in Tokyo for a board meeting. He was
later charged with underreporting his compensation for several years in
securities filings and has been detained in Tokyo.

Nissan's internal investigation of what it calls ``substantial and
convincing evidence of misconduct'' has taken on global dimensions,
encompassing teams of compliance people who have tried to secure
potential evidence at residences used by Mr. Ghosn, including an
apartment in Rio de Janeiro.

``Our own investigation is ongoing, and its scope continues to
broaden,'' the company said in a statement Friday, suggesting that Mr.
Ghosn's legal problems could deepen. His family maintains he is
innocent.

Like Mr. Ghosn, Greg Kelly, a Nissan board member, was indicted on
financial misconduct changes. Nissan was indicted, too, and said it
would review its compliance procedures.

Asked to respond to the Ghosn daughters' claims --- that animosity about
a potential merger drove Nissan's investigation --- Nicholas Maxfield, a
company spokesman, said: ``These claims are baseless. The family would
never have had any reason to be privy to discussions related to the
future of Nissan and the alliance.''

``The cause of this chain of events is the misconduct led by Ghosn and
Kelly,'' Mr. Maxfield said. ``During the company's internal
investigation into this misconduct, the prosecutor's office began its
own investigation and took action.'' (Asked specifically whether a
merger had been discussed, Mr. Maxfield said a previously announced
six-year plan had called for ``additional synergies and further
convergence among the member companies.'')

Mr. Ghosn has remained
\href{https://www.nytimes.com/2018/12/19/business/carlos-ghosn-jail.html}{in
a small jail cell} without the opportunity for bail since his arrest.

``That first night, I thought my dad was going to come back within 24
hours,'' Maya Ghosn said. ``In the U.S. you get held for a short period
of time. At least, that was my experience watching `Billions,''' a
Showtime drama about an embattled hedge fund billionaire.

Mr. Ghosn's family said his fall from grace in Japan, a nation that once
celebrated the Brazilian-born son of Lebanese immigrants as a savior,
and where his children spent their formative years, had been
particularly hard to fathom.

In 1999, Mr. Ghosn, then a vice president at Renault, arrived in Tokyo
and applied American-style restructuring to a failing Nissan. He
accomplished what Wall Street analysts deemed impossible --- reviving
Nissan and making it the No. 2 carmaker in Japan.

Image

Mr. Ghosn in 1996 with, from left, Caroline, Anthony, Maya and Nadine.
The family was living in Greenville, S.C., at the time, but the children
consider Tokyo their hometown.Credit...Ghosn Family

The turnaround enshrined Mr. Ghosn in business school studies as the
tough corporate titan who took on the old-line Japanese culture and won.

Mr. Ghosn earned a reputation for championing merit pay --- which he
thought he, too, deserved. He was known for being
\href{https://www.nytimes.com/2018/11/20/business/carlos-ghosn-arrested-nissan.html}{perennially
dissatisfied with his multimillion-dollar salary}, pointing out that it
was not in line with Western auto leaders.

Mr. Ghosn's children, who also include Nadine, 29, and Anthony, 24, live
in the United States and London but considered Tokyo their hometown.
They described a side of Mr. Ghosn the public hasn't seen --- a goofy
and involved parent who would hide chocolate Rolos around the house,
help with homework, go with them to a Pizzakaya hangout, and spend
Sundays buying bagels and roaming the crowded aisles of the Tokyu Hands
store in the Shibuya neighborhood.

``He set that tone of being fully present and creating almost
monastic-like rituals that made his children feel safe and normal,''
Caroline Ghosn said. ``He was a master of creating this scaffolding to
make the turbulence of his life more manageable.''

Last month, that scaffolding came crashing down.

The day after her father's arrest, Maya Ghosn, who was visiting Tokyo,
walked around the upscale Hiroo section, where she grew up, feeling ``a
deep loss of innocence,'' she said.

The Ghosns had hoped to see their father released before the holidays.
On Dec. 20, a Japanese court took the unusual step of rejecting the
prosecutors' request to extend Mr. Ghosn's detention, and his family
prepared to post bail. Hours later, the Japanese authorities
\href{http://fortune.com/2018/12/21/carlos-ghosn-re-arrested-japan/}{rearrested
Mr. Ghosn} on charges that he had temporarily shifted \$16 million in
personal losses incurred during the 2008 financial crisis to Nissan's
balance sheet.

``I called Maya just crying. We didn't say anything for the first three
minutes,'' Caroline Ghosn said. ``The whole situation is absolute
whiplash on the order of a Greek tragedy.''

\href{https://www.nytimes.com/2018/12/19/business/carlos-ghosn-jail.html}{Under
Japanese law}, Mr. Ghosn can be questioned daily by prosecutors and is
allowed interactions with his Japanese lawyer and representatives from
France, Brazil and Lebanon, his three countries of citizenship. His
children are unable to communicate with him.

``Never have we gone this long without hearing my father's voice,'' Maya
Ghosn said.

Her older sister said, ``We just want him to come out of this healthy
and well and have the ability to defend himself, have access to due
process and the ability to use his voice.''

Image

Anthony, Nadine and Caroline, right, with their father in Tennessee in
2015. Referring to his life in a jail cell, Caroline Ghosn said, ``Every
detail we learn is heartbreaking.''Credit...Ghosn Family

They have been told that Mr. Ghosn's cell is unheated and that he has
asked repeatedly for blankets. He was denied pen and paper. They said he
had lost at least 20 pounds.

``He's not a terrorist. He's not El Chapo. Every detail we learn is
heartbreaking,'' Caroline Ghosn said.

Mr. Ghosn moved his family to Brazil, the United States and France
during his years in the auto industry, but the Ghosn sisters said they
had fond memories of Japan. They eat rice and fish bento boxes for
breakfast, and six months ago, Caroline Ghosn had a small wedding in the
Japanese island town of Naoshima.

In the early 2000s, their father was so famous in Japan that the
Japanese would (politely) ask for his autograph at the grocery store.
Nissan was like a member of the family, the sisters said.

``It was such a big part of his life, and he'd grown this child for 20
years, and it had come through blood, sweat and tears when nobody
believed he could do it,'' Maya Ghosn said.

Short of a full merger, Mr. Ghosn had hoped to make the tie-up of Nissan
and Renault permanent, according to a person briefed on his discussions.
It was a move that many Nissan executives, including Mr. Saikawa,
opposed.

Media reports about Mr. Ghosn's lifestyle frustrate his daughters, who
view them as hyped to insinuate wrongdoing. His corporate residences in
Tokyo, Paris, Rio de Janeiro and Beirut, Lebanon, have also come under
scrutiny. They were paid for through a Nissan shell company that Mr.
Ghosn and Mr. Kelly had set up.

Mr. Ghosn's children said that they believed the houses were known to
Nissan and that the homes had helped him efficiently lead two Fortune
Global 500 companies.

``He had those assets accessible to him so he could be more effective,''
Caroline Ghosn said.

Both women said they had been bombarded with media inquiries, including
Japanese reporters pretending to be from the prosecutor's office and
others waiting outside their homes. Mr. Ghosn's lawyers in the United
States have warned his family that it would be unsafe to visit Japan for
fear of arrest or questioning.

Caroline Ghosn said that ever since her father's arrest, she couldn't
bear to drive her beloved Nissan Leaf, an electric coupe in mint green.

``I'm going to walk for the times my father hasn't been able to,'' she
said.

Advertisement

\protect\hyperlink{after-bottom}{Continue reading the main story}

\hypertarget{site-index}{%
\subsection{Site Index}\label{site-index}}

\hypertarget{site-information-navigation}{%
\subsection{Site Information
Navigation}\label{site-information-navigation}}

\begin{itemize}
\tightlist
\item
  \href{https://help.nytimes.com/hc/en-us/articles/115014792127-Copyright-notice}{©~2020~The
  New York Times Company}
\end{itemize}

\begin{itemize}
\tightlist
\item
  \href{https://www.nytco.com/}{NYTCo}
\item
  \href{https://help.nytimes.com/hc/en-us/articles/115015385887-Contact-Us}{Contact
  Us}
\item
  \href{https://www.nytco.com/careers/}{Work with us}
\item
  \href{https://nytmediakit.com/}{Advertise}
\item
  \href{http://www.tbrandstudio.com/}{T Brand Studio}
\item
  \href{https://www.nytimes.com/privacy/cookie-policy\#how-do-i-manage-trackers}{Your
  Ad Choices}
\item
  \href{https://www.nytimes.com/privacy}{Privacy}
\item
  \href{https://help.nytimes.com/hc/en-us/articles/115014893428-Terms-of-service}{Terms
  of Service}
\item
  \href{https://help.nytimes.com/hc/en-us/articles/115014893968-Terms-of-sale}{Terms
  of Sale}
\item
  \href{https://spiderbites.nytimes.com}{Site Map}
\item
  \href{https://help.nytimes.com/hc/en-us}{Help}
\item
  \href{https://www.nytimes.com/subscription?campaignId=37WXW}{Subscriptions}
\end{itemize}
