Sections

SEARCH

\protect\hyperlink{site-content}{Skip to
content}\protect\hyperlink{site-index}{Skip to site index}

\href{https://www.nytimes.com/section/nyregion}{New York}

\href{https://myaccount.nytimes.com/auth/login?response_type=cookie\&client_id=vi}{}

\href{https://www.nytimes.com/section/todayspaper}{Today's Paper}

\href{/section/nyregion}{New York}\textbar{}Early Voting and Other
Changes to Election Laws Are Coming to New York

\url{https://nyti.ms/2H3N2QF}

\begin{itemize}
\item
\item
\item
\item
\item
\end{itemize}

Advertisement

\protect\hyperlink{after-top}{Continue reading the main story}

Supported by

\protect\hyperlink{after-sponsor}{Continue reading the main story}

\hypertarget{early-voting-and-other-changes-to-election-laws-are-coming-to-new-york}{%
\section{Early Voting and Other Changes to Election Laws Are Coming to
New
York}\label{early-voting-and-other-changes-to-election-laws-are-coming-to-new-york}}

\includegraphics{https://static01.nyt.com/images/2019/01/10/nyregion/00nylegis1/merlin_148965030_3e843099-8557-4156-b974-df878f28cc20-articleLarge.jpg?quality=75\&auto=webp\&disable=upscale}

By \href{https://www.nytimes.com/by/jesse-mckinley}{Jesse McKinley}

\begin{itemize}
\item
  Jan. 10, 2019
\item
  \begin{itemize}
  \item
  \item
  \item
  \item
  \item
  \end{itemize}
\end{itemize}

\emph{{[}What you need to know to start the day:}
\href{https://www.nytimes.com/newsletters/newyorktoday?module=inline}{\emph{Get
New York Today in your inbox}}\emph{.{]}}

ALBANY --- For years, the ways in which voters in New York have been
stymied by the state's antiquated voting laws have stood in stark
contrast to the state's liberal reputation.

During last year's contentious midterm elections, New York was
\href{https://www.nytimes.com/2018/06/25/nyregion/new-york-primary-congress-state-federal.html}{the
only state} in the nation that held separate state and federal primary
elections, a bifurcation that almost seemed designed to suppress voter
turnout --- which is generally thought to favor incumbents.

Early voting? Voting by mail? Same-day voter registration? All are
fairly basic voting policies now found in many states,
\href{https://www.nytimes.com/2018/12/19/nyregion/early-voting-reform-laws-ny.html}{but
not in}
\href{https://www.nytimes.com/2018/12/19/nyregion/early-voting-reform-laws-ny.html}{New
York}.

But with Democrats now in control of both chambers of the State Capitol
and the governor's office, things are about to change. Legislative
leaders said they intend to pass what they described as a voting reform
package on Monday to overhaul the state's voting laws, among the more
restrictive in the nation.

The proposals are a veritable wish list for those who have blamed New
York's laws for driving down voter turnout. The measures include
allowing early voting, preregistration of 16- and 17-year-olds and
consolidating state and federal primary elections, which are now held in
different months.

Lawmakers also plan to pass bills to allow vote-by-mail and same-day
voter registration, though those proposals will also require voter
referendums --- and passage by the next Legislature, scheduled to be
seated in 2021 --- as they change the State Constitution.

``This is just the beginning,'' said Senator Michael Gianaris of Queens,
the deputy Democratic leader in the Senate. ``There's a long list of
issues that have been kept on the shelf by Republicans all these years.

``And the issues we're handling in the first week are incredibly
important, and there will be incredibly important issues in the second
week and the third week and the fourth week.''

The changes to the state's voting laws are long overdue, according to
good government groups that have watched as New York has fallen to the
bottom of lists tracking voter participation, despite being one of the
nation's bluest states.

``New York is moving from caboose to locomotive,'' said Blair Horner,
the executive director of the New York Public Interest Research Group,
adding, ``New York has been in the obstacle-creating business, as
opposed to the obstacle-smashing business when it comes to voting.''

The bills, which will be introduced as a package, would place
\href{http://www.ncsl.org/research/elections-and-campaigns/absentee-and-early-voting.aspx}{New
York in the same rank} as other liberal bulwarks like California and
Washington, at a time when Democrats are seeking to enhance voter
involvement.

Gov. Andrew M. Cuomo has expressed support for such measures in the
past, most recently last month, when he vowed to make an election
overhaul a priority in the first 100 days of the new year. A spokesman
for Mr. Cuomo said the administration is hopeful that the voting package
is the start of a range of reforms lawmakers pass this session.

``We look forward to working with them to go further and enact public
campaign financing, make Election Day a state holiday and ban corporate
contributions once and for all,'' the spokesman, Rich Azzopardi, said.

Mr. Cuomo has also favored a bill that would close the so-called
``L.L.C. loophole,'' which allows corporate interests to funnel almost
unlimited amounts of money into campaigns through various anonymous
limited liability companies. The loophole has been a bête noire of
\href{https://www.commoncause.org/new-york/our-work/money-influence/closing-new-yorks-llc-loophole/}{good
government groups} for years, but has been utilized by various powerful
politicians in the state, including Mr. Cuomo, who has been
\href{https://www.politico.com/states/new-york/albany/story/2018/05/06/though-he-backs-reform-cuomo-has-raised-more-from-llcs-than-every-legislator-combined-388190}{one
of its biggest beneficiaries}.

The Democrat-dominated Assembly has voted to close the loophole for
years, but the measure went nowhere in the Republican-led Senate.

On Monday, however, the new Democratic leadership in the Capitol intends
to pass identical bills to cap contributions from L.L.C.s at \$5,000,
bringing them in line with limits on donations from corporations.

L.L.C.s would also be required ``to disclose its beneficial ownership''
with the State Board of Elections, requiring more transparency of
entities that are often purposefully opaque, sometimes identified by
little more than an address of the company that established the L.L.C.

The push on election reform is expected to be followed in short order on
Tuesday by two other progressive agenda items: a ban of treating minors
with so-called conversion therapy, which aims to ``cure'' gay people of
being homosexual; and the Gender Non-Discrimination Act (Genda), which
prohibits discrimination based on ``gender identity or expression'' and
expands the definition of hate crimes to cover transgender people.

The sponsor of both those bills is Senator Brad Hoylman, who said they
would be the first gender-oriented legislation to pass since same-sex
marriage was legalized in 2011. Mr. Hoylman, who is the only openly gay
lawmaker in the upper chamber, said that Republicans had thwarted even
minor bills and resolutions that mentioned gay rights or issues.

Mr. Hoylman, who represents Lower Manhattan and brought his husband,
David, and their 16-month-old daughter, Lucy, to the opening day of
session on Wednesday, said he hoped that the bills would find some votes
from across the aisle as well. ``I wouldn't be surprised if we see
Republican support,'' he said.

On Wednesday, the interim minority leader of the Senate, Joseph Griffo,
said in his introductory speech that the Republicans' ``mission as
public servants remains the same: starting with supporting policies that
help hard-working, middle-class families succeed and businesses grow and
prosper.''

For the time being, Republicans, with only 23 members in a 63-seat
house, have little recourse to stop a Democratic agenda, including the
major changes to the state's voting laws that could amplify demographic
advantages in an already deep-blue state.

Money, however, is a bipartisan need, and the L.L.C. loophole has been
used by both parties to support campaigns. Still, many of the new
members in the Senate Democratic conference ran for office on a pledge
of getting corporate money out of politics, and Monday's plan of action
seems calibrated to make good on those campaign promises.

``I didn't take any L.L.C. money, a number of people coming in didn't
take any L.L.C. money,'' said Zellnor Myrie, a newly elected senator
from Brooklyn. ``And I think that's what we're responding to.''

Advertisement

\protect\hyperlink{after-bottom}{Continue reading the main story}

\hypertarget{site-index}{%
\subsection{Site Index}\label{site-index}}

\hypertarget{site-information-navigation}{%
\subsection{Site Information
Navigation}\label{site-information-navigation}}

\begin{itemize}
\tightlist
\item
  \href{https://help.nytimes.com/hc/en-us/articles/115014792127-Copyright-notice}{©~2020~The
  New York Times Company}
\end{itemize}

\begin{itemize}
\tightlist
\item
  \href{https://www.nytco.com/}{NYTCo}
\item
  \href{https://help.nytimes.com/hc/en-us/articles/115015385887-Contact-Us}{Contact
  Us}
\item
  \href{https://www.nytco.com/careers/}{Work with us}
\item
  \href{https://nytmediakit.com/}{Advertise}
\item
  \href{http://www.tbrandstudio.com/}{T Brand Studio}
\item
  \href{https://www.nytimes.com/privacy/cookie-policy\#how-do-i-manage-trackers}{Your
  Ad Choices}
\item
  \href{https://www.nytimes.com/privacy}{Privacy}
\item
  \href{https://help.nytimes.com/hc/en-us/articles/115014893428-Terms-of-service}{Terms
  of Service}
\item
  \href{https://help.nytimes.com/hc/en-us/articles/115014893968-Terms-of-sale}{Terms
  of Sale}
\item
  \href{https://spiderbites.nytimes.com}{Site Map}
\item
  \href{https://help.nytimes.com/hc/en-us}{Help}
\item
  \href{https://www.nytimes.com/subscription?campaignId=37WXW}{Subscriptions}
\end{itemize}
