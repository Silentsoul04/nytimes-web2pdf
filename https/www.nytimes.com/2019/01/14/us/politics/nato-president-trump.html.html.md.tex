Sections

SEARCH

\protect\hyperlink{site-content}{Skip to
content}\protect\hyperlink{site-index}{Skip to site index}

\href{https://www.nytimes.com/section/politics}{Politics}

\href{https://myaccount.nytimes.com/auth/login?response_type=cookie\&client_id=vi}{}

\href{https://www.nytimes.com/section/todayspaper}{Today's Paper}

\href{/section/politics}{Politics}\textbar{}Trump Discussed Pulling U.S.
From NATO, Aides Say Amid New Concerns Over Russia

\url{https://nyti.ms/2HaZZrK}

\begin{itemize}
\item
\item
\item
\item
\item
\item
\end{itemize}

Advertisement

\protect\hyperlink{after-top}{Continue reading the main story}

Supported by

\protect\hyperlink{after-sponsor}{Continue reading the main story}

\hypertarget{trump-discussed-pulling-us-from-nato-aides-say-amid-new-concerns-over-russia}{%
\section{Trump Discussed Pulling U.S. From NATO, Aides Say Amid New
Concerns Over
Russia}\label{trump-discussed-pulling-us-from-nato-aides-say-amid-new-concerns-over-russia}}

\includegraphics{https://static01.nyt.com/images/2019/01/15/us/15dc-trumpnato-1/merlin_141156909_cfa4e9a4-a737-4809-aae5-718916f5b72c-articleLarge.jpg?quality=75\&auto=webp\&disable=upscale}

By \href{https://www.nytimes.com/by/julian-e-barnes}{Julian E. Barnes}
and \href{https://www.nytimes.com/by/helene-cooper}{Helene Cooper}

\begin{itemize}
\item
  Jan. 14, 2019
\item
  \begin{itemize}
  \item
  \item
  \item
  \item
  \item
  \item
  \end{itemize}
\end{itemize}

WASHINGTON --- There are few things that President Vladimir V. Putin of
Russia desires more than the weakening of NATO, the military alliance
among the United States, Europe and Canada that has deterred Soviet and
Russian aggression for 70 years.

Last year, President Trump suggested a move tantamount to destroying
NATO: the withdrawal of the United States.

Senior administration officials told The New York Times that several
times over the course of 2018, Mr. Trump privately said he wanted to
withdraw from the North Atlantic Treaty Organization. Current and former
officials who support the alliance said they feared Mr. Trump could
return to his threat as allied military spending continued to lag behind
the goals the president had set.

In the days around a
\href{https://www.nytimes.com/2018/07/12/world/europe/trump-nato-russia.html}{tumultuous
NATO summit meeting} last summer, they said, Mr. Trump told his top
national security officials that he did not see the point of the
military alliance, which he presented as a drain on the United States.

At the time, Mr. Trump's national security team, including Jim Mattis,
then the defense secretary, and John R. Bolton, the national security
adviser, scrambled to keep American strategy on track without mention of
a withdrawal that would drastically reduce Washington's influence in
Europe and could embolden Russia for decades.

Now, the president's repeatedly stated desire to withdraw from NATO is
raising new worries among national security officials amid growing
concern about Mr. Trump's efforts to keep his
\href{https://www.nytimes.com/2019/01/13/us/politics/trump-putin-russia-meetings.html}{meetings
with Mr. Putin secret} from even his own aides, and
\href{https://www.nytimes.com/2019/01/11/us/politics/fbi-trump-russia-inquiry.html}{an
F.B.I. investigation} into the administration's Russia ties.

A move to withdraw from the alliance, in place since 1949, ``would be
one of the most damaging things that any president could do to U.S.
interests,'' said Michèle A. Flournoy, an under secretary of defense
under President Barack Obama.

``It would destroy 70-plus years of painstaking work across multiple
administrations, Republican and Democratic, to create perhaps the most
powerful and advantageous alliance in history,'' Ms. Flournoy said in an
interview. ``And it would be the wildest success that Vladimir Putin
could dream of.''

Retired Adm. James G. Stavridis, the former supreme allied commander of
NATO, said an American withdrawal from the alliance would be ``a
geopolitical mistake of epic proportion.''

``Even discussing the idea of leaving NATO --- let alone actually doing
so --- would be the gift of the century for Putin,'' Admiral Stavridis
said.

\includegraphics{https://static01.nyt.com/images/2019/01/15/us/15dc-trumpnato-2/merlin_149179107_c08220b3-9a08-40f3-94d4-66e86f863c0a-articleLarge.jpg?quality=75\&auto=webp\&disable=upscale}

Senior Trump administration officials discussed the internal and highly
sensitive efforts to preserve the military alliance on condition of
anonymity.

After the White House was asked for comment on Monday, a senior
administration official pointed to Mr. Trump's remarks in July when he
called the United States' commitment to NATO ``very strong'' and the
alliance ``very important.'' The official declined to comment further.

American national security officials believe that Russia has largely
focused on undermining solidarity between the United States and Europe
after it annexed Crimea in 2014. Its goal was to upend NATO, which
Moscow views as a threat.

Russia's meddling in American elections and its efforts to prevent
former satellite states from joining the alliance have aimed to weaken
what it views as an enemy next door, the American officials said. With a
weakened NATO, they said, Mr. Putin would have more freedom to behave as
he wishes, setting up Russia as a counterweight to Europe and the United
States.

An American withdrawal from the alliance would accomplish all that Mr.
Putin has been trying to put into motion, the officials said ---
essentially, doing the Russian leader's hardest and most critical work
for him.

When Mr. Trump first raised the possibility of leaving the alliance,
senior administration officials were unsure if he was serious. He has
returned to the idea several times, officials said increasing their
worries.

Mr. Trump's dislike of alliances abroad and American commitments to
international organizations is no secret.

The president has repeatedly and publicly challenged or withdrawn from a
number of military and economic partnerships, from the Paris climate
accord to an Asia-Pacific trade pact. He has questioned the United
States' military alliance with South Korea and Japan, and he has
announced a withdrawal of American troops from Syria without first
consulting allies in the American-led coalition to defeat the Islamic
State.

NATO had planned to hold a leaders meeting in Washington to mark its
70th anniversary in April, akin to the 50-year celebration that was
hosted by President Bill Clinton in 1999. But this year's meeting has
been downgraded to a foreign ministers gathering, as some diplomats
feared that Mr. Trump could use a Washington summit meeting to renew his
attacks on the alliance.

Leaders are now scheduled to meet at the end of 2019, but not in
Washington.

Mr. Trump's threats to withdraw had sent officials scrambling to prevent
the annual gathering of NATO leaders in Brussels last July from turning
into a disaster.

\includegraphics{https://static01.nyt.com/images/2018/07/15/world/europe/putin_still-copy/putin_still-copy-videoSixteenByNineJumbo1600.jpg}

Senior national security officials had already pushed the military
alliance's ambassadors to complete a formal agreement on several NATO
goals --- including shared defenses against Russia --- before the summit
meeting even began, to
\href{https://www.nytimes.com/2018/08/09/us/politics/nato-summit-trump.html}{shield
it from Mr. Trump}.

But Mr. Trump upended the proceedings anyway. One meeting, on July 12,
was ostensibly supposed to be about Ukraine and Georgia --- two non-NATO
members with aspirations to join the alliance.

Accepted protocol dictates that alliance members do not discuss internal
business in front of nonmembers. But as is frequently the case, Mr.
Trump did not adhere to the established norms, according to several
American and European officials who were in the room.

He complained that European governments were not spending enough on the
shared costs of defense, leaving the United States to carry an outsize
burden. He expressed frustration that European leaders would not, on the
spot, pledge to spend more. And he appeared not to grasp the details
when several tried to explain to him that spending levels were set by
parliaments in individual countries, the American and European officials
said.

Then, at another leaders gathering at the same summit meeting, Mr. Trump
appeared to be taken by surprise by Jens Stoltenberg, the NATO secretary
general.

Backing Mr. Trump's position, Mr. Stoltenberg pushed allies to increase
their spending and praised the United States for leading by example ---
including by increasing its military spending in Europe. At that,
according to one official who was in the room, Mr. Trump whipped his
head around and glared at American officials behind him, surprised by
Mr. Stoltenberg's remarks and betraying ignorance of his
administration's own spending plans.

Mr. Trump appeared especially annoyed, officials in the meeting said,
with Chancellor Angela Merkel of Germany and her country's military
spending of 1 percent of its gross domestic product.

By comparison, the United States' military spending is about 4 percent
of G.D.P., and Mr. Trump has railed against allies for not meeting the
NATO spending goal of 2 percent of economic output. At the summit
meeting, he surprised the leaders by demanding 4 percent --- a move that
would essentially put the goal out of reach for many alliance members.
He also threatened that the United States would ``go its own way'' in
2019 if military spending from other NATO countries did not rise.

During the middle of a speech by Ms. Merkel, Mr. Trump again broke
protocol by getting up and leaving, sending ripples of shock across the
room, according to American and European officials who were there. But
before he left, the president walked behind Ms. Merkel and interrupted
her speech to call her a great leader. Startled and relieved that Mr.
Trump had not continued his berating of the leaders, the people in the
room clapped.

In the end, the NATO leaders publicly papered over their differences to
present a unified front. But both European leaders and American
officials emerged from the two days in Brussels shaken and worried that
Mr. Trump would renew his threat to withdraw from the alliance.

Image

Congressional Republicans criticized Mr. Trump's news conference with
President Vladimir V. Putin of Russia in Helsinki, Finland, just days
after the NATO leaders summit meeting in Brussels.Credit...Doug
Mills/The New York Times

Mr. Trump's skepticism of NATO appears to be a core belief,
administration officials said, akin to
\href{https://www.nytimes.com/2017/02/20/world/middleeast/jim-mattis-iraq-oil-trump.html}{his
desire to expropriate Iraq's oil}. While officials have explained
multiple times why the United States cannot take Iraq's oil, Mr. Trump
returns to the issue every few months.

Similarly, just when officials think the issue of NATO membership has
been settled, Mr. Trump again brings up his desire to leave the
alliance.

Any move by Mr. Trump against NATO would most likely invite a response
by Congress. American policy toward Russia is the one area where
congressional Republicans have consistently bucked Mr. Trump, including
with
\href{https://www.nytimes.com/2018/08/09/world/europe/russia-sanctions-ruble.html}{new
sanctions on Moscow} and by criticizing his
\href{https://www.nytimes.com/2018/07/16/world/europe/trump-putin-election-intelligence.html}{warm
July 16 news conference} with Mr. Putin in Helsinki, Finland.

Members of NATO may withdraw after a notification period of a year,
under Article 13 of
\href{https://www.nato.int/cps/ie/natohq/official_texts_17120.htm}{the
Washington Treaty}. Such a delay would give Congress time to try
blocking any attempt by Mr. Trump to leave.

``It's alarming that the president continues to falsely assert that NATO
does not contribute to the overall safety of the United States or the
international community,'' said Senator Jeanne Shaheen, a New Hampshire
Democrat who is among the lawmakers who support legislation to stop Mr.
Trump from withdrawing from the military alliance. ``The Senate knows
better and stands ready to defend NATO.''

NATO's popularity with the public continues to be strong. But the
alliance has become a more partisan issue, with Democrats showing strong
enthusiasm and Republican support softening, according to a survey by
\href{https://www.reaganfoundation.org/reagan-institute/centers/peace-through-strength/}{the
Ronald Reagan Institute.}

Kay Bailey Hutchison, Washington's ambassador to NATO and a former
Republican senator, has sought to build support for the alliance in
Congress, including helping to organize a bipartisan group of backers.

But even if Congress moved to block a withdrawal, a statement by Mr.
Trump that he wanted to leave would greatly damage NATO. Allies feeling
threatened by Russia already have extreme doubts about whether Mr. Trump
would order troops to come to their aid.

In his
\href{https://www.nytimes.com/2018/12/20/us/politics/letter-jim-mattis-trump.html}{resignation
letter} last month, Mr. Mattis specifically cited his own commitment to
America's alliances in an implicit criticism of Mr. Trump's principles.
Mr. Mattis originally said he would stay through the next NATO meeting
at the end of February, but Mr. Trump pushed him out before the new
year.

Acting Defense Secretary Patrick M. Shanahan is believed to support the
alliance. But he has also pointedly said he thinks that the Pentagon
should not be
\href{https://www.nytimes.com/2018/12/24/us/politics/patrick-shanahan-defense-secretary.html}{``the
Department of No''} to the president.

European and American officials said the presence of Mr. Mattis, a
former top NATO commander, had reassured allies that a senior Trump
administration official had their back. His exit from the Pentagon has
increased worries among some European diplomats that the safety blanket
has now been lost.

Advertisement

\protect\hyperlink{after-bottom}{Continue reading the main story}

\hypertarget{site-index}{%
\subsection{Site Index}\label{site-index}}

\hypertarget{site-information-navigation}{%
\subsection{Site Information
Navigation}\label{site-information-navigation}}

\begin{itemize}
\tightlist
\item
  \href{https://help.nytimes.com/hc/en-us/articles/115014792127-Copyright-notice}{©~2020~The
  New York Times Company}
\end{itemize}

\begin{itemize}
\tightlist
\item
  \href{https://www.nytco.com/}{NYTCo}
\item
  \href{https://help.nytimes.com/hc/en-us/articles/115015385887-Contact-Us}{Contact
  Us}
\item
  \href{https://www.nytco.com/careers/}{Work with us}
\item
  \href{https://nytmediakit.com/}{Advertise}
\item
  \href{http://www.tbrandstudio.com/}{T Brand Studio}
\item
  \href{https://www.nytimes.com/privacy/cookie-policy\#how-do-i-manage-trackers}{Your
  Ad Choices}
\item
  \href{https://www.nytimes.com/privacy}{Privacy}
\item
  \href{https://help.nytimes.com/hc/en-us/articles/115014893428-Terms-of-service}{Terms
  of Service}
\item
  \href{https://help.nytimes.com/hc/en-us/articles/115014893968-Terms-of-sale}{Terms
  of Sale}
\item
  \href{https://spiderbites.nytimes.com}{Site Map}
\item
  \href{https://help.nytimes.com/hc/en-us}{Help}
\item
  \href{https://www.nytimes.com/subscription?campaignId=37WXW}{Subscriptions}
\end{itemize}
