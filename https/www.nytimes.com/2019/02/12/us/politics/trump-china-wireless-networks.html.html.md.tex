Sections

SEARCH

\protect\hyperlink{site-content}{Skip to
content}\protect\hyperlink{site-index}{Skip to site index}

\href{https://www.nytimes.com/section/politics}{Politics}

\href{https://myaccount.nytimes.com/auth/login?response_type=cookie\&client_id=vi}{}

\href{https://www.nytimes.com/section/todayspaper}{Today's Paper}

\href{/section/politics}{Politics}\textbar{}Administration Readies Order
to Keep China Out of Wireless Networks

\url{https://nyti.ms/2UXC814}

\begin{itemize}
\item
\item
\item
\item
\item
\end{itemize}

Advertisement

\protect\hyperlink{after-top}{Continue reading the main story}

Supported by

\protect\hyperlink{after-sponsor}{Continue reading the main story}

\hypertarget{administration-readies-order-to-keep-china-out-of-wireless-networks}{%
\section{Administration Readies Order to Keep China Out of Wireless
Networks}\label{administration-readies-order-to-keep-china-out-of-wireless-networks}}

\includegraphics{https://static01.nyt.com/images/2019/02/13/us/politics/13dc-huawei/13dc-huawei-articleLarge.jpg?quality=75\&auto=webp\&disable=upscale}

By \href{https://www.nytimes.com/by/julian-e-barnes}{Julian E. Barnes}

\begin{itemize}
\item
  Feb. 12, 2019
\item
  \begin{itemize}
  \item
  \item
  \item
  \item
  \item
  \end{itemize}
\end{itemize}

\href{https://cn.nytimes.com/usa/20190214/trump-china-wireless-networks/}{阅读简体中文版}\href{https://cn.nytimes.com/usa/20190214/trump-china-wireless-networks/zh-hant/}{閱讀繁體中文版}

WASHINGTON --- The Trump administration is moving closer to completing
an executive order that would ban telecommunications companies in the
United States from using Chinese equipment while building
next-generation wireless networks, according to American officials.

President Trump has been briefed on the proposed ban, which would
prevent the use of equipment from ``adversarial powers,'' and the order
could be issued in the coming days, American government and industry
officials said.

The executive order, which has been under discussion for months, is
aimed largely at preventing Chinese telecom firms like Huawei from
gaining access to the fifth-generation --- or 5G --- wireless networks
that companies are beginning to build in the United States. American
intelligence officials have grown increasingly concerned about Huawei
and other Chinese telecom companies, saying their inclusion in American
networks pose security risks that could jeopardize national security.

5G is expected to be far faster than today's wireless networks, allowing
a broad range of devices like autonomous vehicles to be connected to and
controlled by wireless networks.

The executive order
\href{https://www.nytimes.com/2018/05/02/us/politics/trump-china-telecoms-restrictions.html}{has
been expected} for months and news reports have repeatedly predicted its
imminent rollout, only for the order to be delayed.

But American officials are increasingly ratcheting up pressure on
Huawei, which is China's largest telecom equipment company. The United
States
\href{https://www.nytimes.com/2018/12/05/business/huawei-cfo-arrest-canada-extradition.html}{recently
brought criminal charges} against Huawei's chief financial officer, Meng
Wanzhou, and is
\href{https://www.nytimes.com/2019/01/22/us/politics/meng-wanzhou-extradition.html}{seeking
her extradition} from Canada, where she was arrested at Washington's
behest.

Top American officials, including Secretary of State Mike Pompeo, are in
Europe this week pressing allies to take their own steps to ban Chinese
companies, including Huawei, from their next-generation networks. The
executive order would help strengthen the United States' case to other
nations, according to American officials.

Government and industry officials have expected the executive order to
be announced this month before a mobile technology trade event in
Barcelona. But other people briefed on the effort said the
administration wanted to complete its trade negotiations with China
before introducing the order. American trade negotiators are in Beijing
this week trying to hammer out a trade deal before a March deadline
agreed to by both China and the United States.

American officials have repeatedly emphasized that the executive order
--- and its concerns about Huawei --- is separate from the trade talks.

But Mr. Trump has allowed trade and security issues to be intertwined
before, even when security officials have tried to keep them separate.

During an Oval Office meeting last month with top Chinese trade
officials, Mr. Trump said he expected Huawei to come up during the
talks. And the president
\href{https://www.nytimes.com/2018/06/07/business/what-is-zte.html}{previously
stepped in} to end a ban on another Chinese telecom company, ZTE, which
\href{https://www.nytimes.com/2018/04/16/technology/chinese-tech-company-blocked-from-buying-american-components.html}{was
prevented from buying} American technology for seven years. Security
officials are still smarting from Mr. Trump's capitulation
\href{https://www.nytimes.com/2018/06/07/business/us-china-zte-deal.html}{to
a request by President Xi Jinping of China} that the United States drop
that ban.

Mr. Pompeo is visiting Hungary, Poland and other European countries this
week and has been
\href{https://www.nytimes.com/2019/02/12/world/europe/czech-republic-huawei.html}{making
the case} to allies that they should avoid using Chinese telecom
companies in their 5G networks, saying it is a security risk.

American officials
\href{https://www.nytimes.com/2019/01/26/us/politics/huawei-china-us-5g-technology.html}{have
been privately telling} European allies that decisions about troop
presence and bases could be predicated on which countries have 5G
networks free of Chinese equipment. This week, Mr. Pompeo issued a
public version of that warning, telling allies to steer clear of
Chinese-built 5G networks.

``We have seen this all around the world. It also makes it more
difficult for America to be present,'' Mr. Pompeo said. ``If that
equipment is co-located where we have important American systems, it
makes it more difficult for us to partner alongside them.''

Mr. Pompeo also called out Huawei while in Budapest, warning other
nations about ``risks that Huawei's presence in their networks present
--- actual risks to their people, to the loss of privacy protections.''

The acting secretary of defense, Patrick M. Shanahan, is scheduled to be
in Brussels for meetings on Wednesday and Thursday and will travel to
Munich on Friday for the annual security conference there. He is also
expected to raise the dangers of using Huawei and other Chinese telecom
firms in foreign networks.

Some European officials have considered enlisting the North Atlantic
Treaty Organization in the debate, arguing that decisions about what
companies to use to build wireless networks are a matter of national
sovereignty. While some European countries, like Poland, have embraced
the United States' point of view, Huawei is intertwined in European
networks and many countries have made heavy use of its equipment.
Britain has allowed the company to build equipment outside its core
networks, but has an oversight board that closely examines Huawei's
equipment and software.

Australia was
\href{https://www.nytimes.com/2018/08/23/technology/huawei-banned-australia-5g.html}{one
of the first allies} to move against Huawei, announcing a ban last year.

\href{https://www.nytimes.com/2019/02/06/us/politics/richard-burr-china-huawei-5g.html}{Some
members of Congress} have been skeptical of banning Chinese companies by
executive order, including Senator Richard M. Burr, Republican of North
Carolina, who leads the Senate Intelligence Committee.

Mr. Burr has said discussions with telecom companies could prompt them
to voluntarily block Huawei. Indeed, big carriers like Verizon and AT\&T
have said they will not use Huawei equipment. But administration
officials have said that without an executive order, smaller companies
that serve large parts of the rural United States might use Chinese
equipment.

Government and industry officials have said the move to 5G networks will
be more revolutionary than evolutionary. Unlike earlier generations of
wireless networks, they will run more on software than hardware,
allowing the possibility that companies that control the networks to
divert information without being detected.

This capability comes as Washington and its allies have become more
suspicious of Beijing, arguing that a series of new laws gives it
unfettered access to data that crosses networks built and maintained by
companies based in China.

Much of the attention has been on Huawei, because it makes some of the
best and least expensive equipment that can go into a 5G network. It has
also been the subject of
\href{https://www.nytimes.com/2019/01/28/us/politics/meng-wanzhou-huawei-iran.html}{Department
of Justice indictments} accusing it of stealing competitors' trade
secrets and is at the center of a
\href{https://www.nytimes.com/2019/01/11/world/europe/poland-china-huawei-spy.html}{spy
scandal} in Poland.

But administration officials said the executive order would ban a broad
array of foreign equipment, not just Huawei. It would also prevent any
Russian software from telecommunications networks. It would not ban
European equipment makers like Ericsson or Nokia.

Blocking only Huawei, according to government officials, would simply
result in capital, personnel and know-how shifting to another Chinese
company.

In commercial terms, the effect of the executive order on Huawei and ZTE
is likely to be small. Large American cellular operators, such as AT\&T
and Verizon, have been effectively banned from buying from the Chinese
vendors since a 2012 congressional report said that they could not be
trusted to be free of interference from Beijing. That has left Huawei
and ZTE with small, regional wireless operators as the only customers
for their network equipment in the United States, despite being a large
presence in Europe, Asia and elsewhere.

But the executive order has symbolic value amid the Trump
administration's broad and aggressive campaign to stymie the Chinese
telecom equipment makers.

A Huawei spokesman did not immediately respond to a request for comment.
Addressing reports of the executive order, a spokeswoman for the Chinese
Foreign Ministry, Hua Chunying, said in December, ``Despite not having
any evidence, certain countries have politicized the normal exchanges
and cooperation in science and technology.''

Ms. Hua added, ``This actually amounts to shutting their own door to
openness, progress and fairness.''

American officials had mulled a broader ban that would also prevent the
export of American technology to Huawei and other Chinese telecom
companies.

Such a ban could have set back Huawei's ability to compete for 5G
contracts at a critical time. In the next six months, many allies and
partners will be deciding what technology to use in their
next-generation networks, American officials said.

But the White House rejected the export ban, officials said, believing
that it would hurt American companies who depend on sales to Chinese
companies, threaten high-paying jobs and simply force Huawei and other
companies to make their own competing components on a faster timetable.

Advertisement

\protect\hyperlink{after-bottom}{Continue reading the main story}

\hypertarget{site-index}{%
\subsection{Site Index}\label{site-index}}

\hypertarget{site-information-navigation}{%
\subsection{Site Information
Navigation}\label{site-information-navigation}}

\begin{itemize}
\tightlist
\item
  \href{https://help.nytimes.com/hc/en-us/articles/115014792127-Copyright-notice}{©~2020~The
  New York Times Company}
\end{itemize}

\begin{itemize}
\tightlist
\item
  \href{https://www.nytco.com/}{NYTCo}
\item
  \href{https://help.nytimes.com/hc/en-us/articles/115015385887-Contact-Us}{Contact
  Us}
\item
  \href{https://www.nytco.com/careers/}{Work with us}
\item
  \href{https://nytmediakit.com/}{Advertise}
\item
  \href{http://www.tbrandstudio.com/}{T Brand Studio}
\item
  \href{https://www.nytimes.com/privacy/cookie-policy\#how-do-i-manage-trackers}{Your
  Ad Choices}
\item
  \href{https://www.nytimes.com/privacy}{Privacy}
\item
  \href{https://help.nytimes.com/hc/en-us/articles/115014893428-Terms-of-service}{Terms
  of Service}
\item
  \href{https://help.nytimes.com/hc/en-us/articles/115014893968-Terms-of-sale}{Terms
  of Sale}
\item
  \href{https://spiderbites.nytimes.com}{Site Map}
\item
  \href{https://help.nytimes.com/hc/en-us}{Help}
\item
  \href{https://www.nytimes.com/subscription?campaignId=37WXW}{Subscriptions}
\end{itemize}
