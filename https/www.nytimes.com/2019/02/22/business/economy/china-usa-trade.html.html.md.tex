Sections

SEARCH

\protect\hyperlink{site-content}{Skip to
content}\protect\hyperlink{site-index}{Skip to site index}

\href{https://www.nytimes.com/section/business/economy}{Economy}

\href{https://myaccount.nytimes.com/auth/login?response_type=cookie\&client_id=vi}{}

\href{https://www.nytimes.com/section/todayspaper}{Today's Paper}

\href{/section/business/economy}{Economy}\textbar{}U.S. and China Extend
Talks, but Final Deal Remains Elusive

\url{https://nyti.ms/2GKGfun}

\begin{itemize}
\item
\item
\item
\item
\item
\end{itemize}

Advertisement

\protect\hyperlink{after-top}{Continue reading the main story}

Supported by

\protect\hyperlink{after-sponsor}{Continue reading the main story}

\hypertarget{us-and-china-extend-talks-but-final-deal-remains-elusive}{%
\section{U.S. and China Extend Talks, but Final Deal Remains
Elusive}\label{us-and-china-extend-talks-but-final-deal-remains-elusive}}

\includegraphics{https://static01.nyt.com/images/2019/02/23/business/23china-trade-print/merlin_151071009_c9ea9638-4596-4c39-b66d-d358d98a4737-articleLarge.jpg?quality=75\&auto=webp\&disable=upscale}

By \href{https://www.nytimes.com/by/ana-swanson}{Ana Swanson} and
\href{https://www.nytimes.com/by/alan-rappeport}{Alan Rappeport}

\begin{itemize}
\item
  Feb. 22, 2019
\item
  \begin{itemize}
  \item
  \item
  \item
  \item
  \item
  \end{itemize}
\end{itemize}

\href{https://cn.nytimes.com/business/20190225/china-usa-trade/}{阅读简体中文版}\href{https://cn.nytimes.com/business/20190225/china-usa-trade/zh-hant/}{閱讀繁體中文版}

WASHINGTON --- President Trump said the United States and China were
moving closer toward a trade agreement and suggested that the fate of
Huawei, the Chinese telecom giant facing criminal charges along with its
chief financial officer, could be resolved as part of a final trade deal
with Beijing.

Ringed by his top advisers and a visiting Chinese delegation on Friday
in the Oval Office, Mr. Trump said talks would continue through the
weekend and that there was a ``very, very good chance'' of a deal with
China being reached. He said he expected to meet personally with
President Xi Jinping of China to ``work out the final points,'' most
likely in March at Mar-a-Lago, the
\href{https://www.nytimes.com/2017/02/18/us/mar-a-lago-trump-ethics-winter-white-house.html}{president's
resort} in Palm Beach, Fla.

The talks have been aimed at resolving the Trump administration's
concerns about China's trade barriers and its treatment of American
companies, including forcing firms to hand over valuable technology as a
condition of doing business there. But Mr. Trump again indicated that he
might intertwine a national security case with the trade talks, despite
concerns from his own law enforcement and intelligence officials about
doing so.

Asked by a reporter if criminal charges against Huawei ---
\href{https://www.nytimes.com/2019/01/28/us/politics/meng-wanzhou-huawei-iran.html}{which
has been accused by the Justice Department} of stealing trade secrets
--- could be dropped as part of the trade pact, the president said:
``We'll be talking to the U.S. attorneys and the attorney general. We'll
be making that decision.''

In January, the United States unveiled sweeping charges against Huawei
and its chief financial officer, Meng Wanzhou, outlining a decade-long
attempt by the company to steal trade secrets, obstruct a criminal
investigation and evade economic sanctions on Iran. After Ms. Meng
\href{https://www.nytimes.com/2018/12/05/business/huawei-cfo-arrest-canada-extradition.html}{was
arrested last year} in Canada at the behest of American law enforcement,
Mr. Trump
\href{https://www.reuters.com/article/us-usa-trump-huawei-tech-exclusive/exclusive-trump-says-he-could-intervene-in-u-s-case-against-huawei-cfo-idUSKBN1OA2PQ}{told
Reuters} that he ``would certainly intervene'' in her case ``if I
thought it was necessary'' for a trade deal.

Offering a reprieve to Huawei would come at an awkward time for the
administration, which has been engaged in a
\href{https://www.nytimes.com/2019/01/26/us/politics/huawei-china-us-5g-technology.htmlhttps://www.nytimes.com/2019/01/26/us/politics/huawei-china-us-5g-technology.html}{global
push} to persuade other countries to ban the company's equipment from
the next generation of wireless networks, saying it is a security
threat.

The Trump administration has been
\href{https://www.nytimes.com/2019/02/12/us/politics/trump-china-wireless-networks.html}{preparing
an executive order} for Mr. Trump to sign that would prevent American
companies from using equipment manufactured by Chinese companies,
including Huawei. But on Friday, Mr. Trump appeared to be wavering on
whether to follow through with that effort, which has been criticized by
the Chinese government.

``We may or may not put that in the trade agreement,'' the president
said, quickly adding, ``But we would only do that in conjunction with
the attorney general of the United States.''

``I don't want to block out anybody if we can help it,'' he said.

Administration officials provided varying degrees of optimism about the
chances of actually reaching a trade deal with China. Mr. Trump said
American and Chinese negotiators had reached several areas of agreement
in a series of meetings this week in Washington, and he indicated that
would most likely forestall a planned increase in tariffs the United
States levied on China last year. The two countries have been racing
toward a March 1 deadline, when Mr. Trump had threatened to increase
tariffs on \$200 billion worth of Chinese imports to 25 percent from 10
percent.

``We have a one time shot at making a great deal for both countries,''
Mr. Trump said, adding, ``I would say that it's more likely that a deal
will happen.''

Mr. Trump's advisers were more cautious, with Robert Lighthizer, Mr.
Trump's top trade negotiator, warning that the United States and China
still have ``very big hurdles'' to overcome. Wilbur Ross, the commerce
secretary, said it was ``a little early for champagne.''

While significant obstacles persist, the two sides agreed to extend
negotiations by 48 hours, and American officials said both had reached
what Mr. Trump termed a ``final agreement'' to stabilize China's
currency, though the specifics of the agreement were not immediately
clear.

The United States has been concerned that China could mitigate the
effect of tariffs by weakening its currency, which would make Chinese
products cheaper to purchase. Some international finance experts found
the idea of such an agreement ironic, as the United States for years
urged China to allow its currency to float more freely.

Tony Sayegh, a Treasury Department spokesman, did not respond to a
request for comment on details of the currency deal.

The Chinese delegation also presented Mr. Trump with a letter from Mr.
Xi, as it did last month, which called on both sides to continue their
efforts to reach a deal that benefits both countries. Liu He, who has
been appointed Mr. Xi's special envoy and is leading negotiations,
concurred that he believed a deal was likely.

Mr. Trump, who has focused for years on how unfair competition from
China has hollowed out American manufacturing, has set the bar for his
signature trade deal extremely high. His advisers have pressured China
to reduce the subsidies that are powering its high-tech industries and
to loosen the state's grip on its economy --- moves that would run
counter to Mr. Xi's desire to strengthen China and his control of it.

Despite Mr. Trump's optimism, China does not appear poised to offer the
kind of commitments that would ensure the economic transformation his
administration has sought, people with knowledge of the talks said.
Instead, China has wooed American officials with eye-catching purchases
of American products and vague commitments on opening its economy that
critics say may never come to fruition.

``This is really the last, best chance,'' said Michael R. Wessel, a
member of the U.S.-China Economic and Security Review Commission.

``The real question is whether the president wants a deal that requires
real change in China or just a series of press talking points and some
high-profile sales that add up to little in the long term,'' he added.

Significant gaps remain between the United States and China on
structural issues, such as forced technology transfer, digital trade and
data flows, said Myron Brilliant, an executive vice president and the
head of international affairs at the U.S. Chamber of Commerce.

``If this agreement comes together, it needs to be sustainable, it needs
to have strong enforcement mechanisms,'' he said. ``We want to go back
to business, but not business as usual.''

That the two sides were even putting pen to paper was a sign of progress
for an increasingly rocky relationship between the world's two largest
economies. In meetings on Thursday and Friday, negotiators drafted a
handful of memorandums of understanding, or M.O.U.s, covering
protections for intellectual property, expanded access for foreign
companies in China, Chinese purchases of American goods and other
issues.

However, a person briefed on the negotiations said that the memorandums
had been written by the American side and that China had not yet
formally agreed to anything in writing.

The one thing that was settled on Friday was that a final agreement
would not be referred to as an M.O.U.

``I don't like M.O.U.s because to me they don't mean anything,'' Mr.
Trump said during the Oval Office meeting.

In a remarkable exchange, Mr. Lighthizer tried to explain to the
president that the memorandums would actually be binding contracts that
both countries would have to honor.

Unconvinced, Mr. Trump corrected Mr. Lighthizer, one of the
administration's most ardent hard-liners, in front of senior American
and Chinese officials, saying that he wants a robust trade agreement.

``I disagree,'' he said. ``I think that a memorandum of understanding is
not a contract to the extent that we want.''

Mr. Lighthizer agreed that the term M.O.U. would not be used again.

Advertisement

\protect\hyperlink{after-bottom}{Continue reading the main story}

\hypertarget{site-index}{%
\subsection{Site Index}\label{site-index}}

\hypertarget{site-information-navigation}{%
\subsection{Site Information
Navigation}\label{site-information-navigation}}

\begin{itemize}
\tightlist
\item
  \href{https://help.nytimes.com/hc/en-us/articles/115014792127-Copyright-notice}{©~2020~The
  New York Times Company}
\end{itemize}

\begin{itemize}
\tightlist
\item
  \href{https://www.nytco.com/}{NYTCo}
\item
  \href{https://help.nytimes.com/hc/en-us/articles/115015385887-Contact-Us}{Contact
  Us}
\item
  \href{https://www.nytco.com/careers/}{Work with us}
\item
  \href{https://nytmediakit.com/}{Advertise}
\item
  \href{http://www.tbrandstudio.com/}{T Brand Studio}
\item
  \href{https://www.nytimes.com/privacy/cookie-policy\#how-do-i-manage-trackers}{Your
  Ad Choices}
\item
  \href{https://www.nytimes.com/privacy}{Privacy}
\item
  \href{https://help.nytimes.com/hc/en-us/articles/115014893428-Terms-of-service}{Terms
  of Service}
\item
  \href{https://help.nytimes.com/hc/en-us/articles/115014893968-Terms-of-sale}{Terms
  of Sale}
\item
  \href{https://spiderbites.nytimes.com}{Site Map}
\item
  \href{https://help.nytimes.com/hc/en-us}{Help}
\item
  \href{https://www.nytimes.com/subscription?campaignId=37WXW}{Subscriptions}
\end{itemize}
