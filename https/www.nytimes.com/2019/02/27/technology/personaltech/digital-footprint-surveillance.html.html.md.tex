Sections

SEARCH

\protect\hyperlink{site-content}{Skip to
content}\protect\hyperlink{site-index}{Skip to site index}

\href{https://www.nytimes.com/section/technology/personaltech}{Personal
Tech}

\href{https://myaccount.nytimes.com/auth/login?response_type=cookie\&client_id=vi}{}

\href{https://www.nytimes.com/section/todayspaper}{Today's Paper}

\href{/section/technology/personaltech}{Personal Tech}\textbar{}Limiting
Your Digital Footprints in a Surveillance State

\url{https://nyti.ms/2NwWY4N}

\begin{itemize}
\item
\item
\item
\item
\item
\item
\end{itemize}

Advertisement

\protect\hyperlink{after-top}{Continue reading the main story}

Supported by

\protect\hyperlink{after-sponsor}{Continue reading the main story}

Tech We're Using

\hypertarget{limiting-your-digital-footprints-in-a-surveillance-state}{%
\section{Limiting Your Digital Footprints in a Surveillance
State}\label{limiting-your-digital-footprints-in-a-surveillance-state}}

To protect himself and his sources from prying eyes in China, Paul
Mozur, a technology reporter in Shanghai, leaves just an ``innocent
trace'' of digital exhaust.

\includegraphics{https://static01.nyt.com/images/2019/02/28/business/28techusing9/merlin_151317582_e029bfa0-8b4a-4b81-975d-8cb10070ed8e-articleLarge.jpg?quality=75\&auto=webp\&disable=upscale}

\href{https://www.nytimes.com/by/paul-mozur}{\includegraphics{https://static01.nyt.com/images/2018/10/15/multimedia/author-paul-mozur/author-paul-mozur-thumbLarge.png}}

Featuring \href{https://www.nytimes.com/by/paul-mozur}{Paul Mozur}

\begin{itemize}
\item
  Feb. 27, 2019
\item
  \begin{itemize}
  \item
  \item
  \item
  \item
  \item
  \item
  \end{itemize}
\end{itemize}

\emph{How do New York Times journalists use technology in their jobs and
in their personal lives? Paul Mozur, a technology correspondent based in
Shanghai, discussed the tech he's using.}

\textbf{What are your most important tech tools for reporting in
Shanghai, especially with a government known for surveillance?}

In China, evading the watchful eyes of the government sometimes feels
like an exercise in futility. The place is
\href{https://www.nytimes.com/2018/07/08/business/china-surveillance-technology.html}{wired
with about 200 million surveillance cameras}, Beijing controls the
telecom companies, and every internet company has to hand over data when
the police want it. They also know where journalists live because we
register our address with police. In Shanghai, the police regularly come
to my apartment; once they demanded to come inside.

That said, China is big, and the government less than competent.
Sometimes the police who come to my door have no idea I'm a journalist.
Usually the higher-ups who deal with my visa don't know about the house
visits. The lack of coordination means one of the best things to do is
to try to slip through the cracks. Basically, protect yourself but also
leave an innocent trace.

I use an iPhone because Apple tends to be more secure than Android.
That's especially true in China, where the blocks against Google mean
there are a huge number of third-party Android stores peddling all kinds
of sketchy apps.

It's also important to realize that because Beijing controls the
telecoms, your domestic phone number can be a liability. For
\href{https://www.nytimes.com/2016/12/07/technology/personaltech/worried-about-the-privacy-of-your-messages-download-signal.html}{secure
apps like Signal}, I toggle the registration lock so that if they try to
mirror my phone, my account still has a layer of protection.

Image

The Chinese messaging app WeChat is widely used but is also closely
monitored.Credit...Giulia Marchi for The New York Times

Image

Paul Mozur works on a Mac because, he says, Apple tends to be more
secure than Android.Credit...Giulia Marchi for The New York Times

To get around the
\href{https://www.nytimes.com/2018/08/10/technology/tech-infowars-china-great-firewall.html}{Great
Firewall}, I use a few different VPNs, which I won't name because when
we do bring them up they usually get new government attention.

In some parts of China, the police will demand to check your phone,
usually to delete photos. Having two phones helps with this --- to make
it even trickier, I have the same case on both phones. But it's also
good to have other ways to protect your data. I use a few apps that
disguise themselves as something innocuous but in fact hide and protect
data. It's also always handy to have a USB drive that can plug into your
phone and be used to save stuff quickly.

\textbf{What do you do to keep sources secure?}

As my colleague Li Yuan noted,
\href{https://www.nytimes.com/2019/01/09/technology/personaltech/china-wechat.html}{WeChat
is a reporter's best friend} in China. Everyone spends huge amounts of
time on it. But WeChat is also closely monitored, so when a sensitive
topic comes up, I try to guide people to use more secure apps.

That can also be a problem, though, because setting up an encrypted
messaging app can alert the authorities to the person. In light of that,
personal meetings also work. Often that means leaving the phone at home,
since a device's microphone can become a listening device. There are
also special Faraday bags, which block communications signals and can
help you go dark with your phone on you. Sometimes all the surveillance
here makes me want to put one on my head.

In some cases, it's not possible to get around the government, and you
have to make a judgment call about whether the source understands the
risks and how severe the punishment might be. Sometimes we go places and
it's not possible to safely interview people, so we don't. The
government frequently wins.

\textbf{What is the oddest surveillance tech you've seen in China?}

China is a gadget-loving nation. Often technology that doesn't totally
work is embraced with alacrity. I think my favorite example was the
\href{https://www.nytimes.com/2018/07/16/technology/china-surveillance-state.html}{facial-recognition
sunglasses} that made the rounds last year. The police were rocking
these glasses, with a camera that plugged into a smartphone-like
minicomputer. The idea was the glasses could identify people as the
police looked at them. When I got to try them on, I found out they
didn't work all that well.

\includegraphics{https://static01.nyt.com/images/2019/02/28/business/28techusing5/merlin_151317528_e77b2758-b0d8-457d-a2b0-c58dac99df50-articleLarge.jpg?quality=75\&auto=webp\&disable=upscale}

In some ways, that doesn't matter. If you convince people you know
everything about them, they're less likely to break rules. The police
officer told me about how they had used the glasses to intimidate a drug
smuggler into a full confession. It shows you that tech doesn't always
have to function in order to work.

I actually was talking about that with my editor the other day. When he
calls from Hong Kong, there's a pause of several seconds before the
phone connects. It could be the government listening in, or maybe it's
just a slow network. We can't be sure, but we sure are paranoid, and we
often joke about it.

\textbf{What kind of protections do you take when you travel in China?}

The Chinese police are famous for showing up in hotel rooms. They don't
always do it, but it happens enough that when you're in China, you
shouldn't leave your devices in your room unattended. In 2009, when I
returned to my room while reporting out of the western region of
Xinjiang, I found a police officer reclining on the bed, smoking a
cigarette and casually swiping through the photos on my digital camera.

To avoid lugging a laptop everywhere I go, I now just travel with two
phones and a Logitech Bluetooth keyboard I can use to quickly write up
notes or articles directly on the phone. It's not entirely secure, since
there are devices that intercept Bluetooth communications, so for
passwords I type directly on the phone.

Other than that, I figure that if the Chinese authorities want to
intercept my Bluetooth keyboard signal, they'll just get an appreciation
of how little work my editors have to do to my sterling copy.

Image

"To avoid lugging a laptop everywhere I go, I now just travel with two
phones and a Logitech Bluetooth keyboard," Mr. Mozur
said.Credit...Giulia Marchi for The New York Times

It should also be said that for all these precautions, there's no doubt
the authorities have plenty of ways into our communications. It's almost
impossible to stop a determined state actor, but it's worth making it
more difficult.

\textbf{How do people in China view the privacy awakening in the United
States after}
\textbf{\href{https://www.nytimes.com/2018/03/17/us/politics/cambridge-analytica-trump-campaign.html}{the
Cambridge Analytica scandal}} \textbf{and other data snafus?}

The privacy situation is dreadful in China. Personal data of all stripes
leaks from companies and government alike. While there's an assumption
of some level of monitoring, many Chinese are as appalled as Americans
when it affects them.

Amusingly, even the government doesn't trust the government. In
reporting on data sharing between different ministries, I've found that
it's not uncommon for one part of the government to distrust another to
the point it won't share data. At other times, a government branch might
not even trust itself to handle data.

For a country trying to become a superpower of artificial intelligence,
big data, cyberspace, innovation and whatever buzzword dominates tech
next, it's a pretty big problem.

Then again, when it comes to poor privacy protection, the United States
seems to be doing its best to take on China.

Advertisement

\protect\hyperlink{after-bottom}{Continue reading the main story}

\hypertarget{site-index}{%
\subsection{Site Index}\label{site-index}}

\hypertarget{site-information-navigation}{%
\subsection{Site Information
Navigation}\label{site-information-navigation}}

\begin{itemize}
\tightlist
\item
  \href{https://help.nytimes.com/hc/en-us/articles/115014792127-Copyright-notice}{©~2020~The
  New York Times Company}
\end{itemize}

\begin{itemize}
\tightlist
\item
  \href{https://www.nytco.com/}{NYTCo}
\item
  \href{https://help.nytimes.com/hc/en-us/articles/115015385887-Contact-Us}{Contact
  Us}
\item
  \href{https://www.nytco.com/careers/}{Work with us}
\item
  \href{https://nytmediakit.com/}{Advertise}
\item
  \href{http://www.tbrandstudio.com/}{T Brand Studio}
\item
  \href{https://www.nytimes.com/privacy/cookie-policy\#how-do-i-manage-trackers}{Your
  Ad Choices}
\item
  \href{https://www.nytimes.com/privacy}{Privacy}
\item
  \href{https://help.nytimes.com/hc/en-us/articles/115014893428-Terms-of-service}{Terms
  of Service}
\item
  \href{https://help.nytimes.com/hc/en-us/articles/115014893968-Terms-of-sale}{Terms
  of Sale}
\item
  \href{https://spiderbites.nytimes.com}{Site Map}
\item
  \href{https://help.nytimes.com/hc/en-us}{Help}
\item
  \href{https://www.nytimes.com/subscription?campaignId=37WXW}{Subscriptions}
\end{itemize}
