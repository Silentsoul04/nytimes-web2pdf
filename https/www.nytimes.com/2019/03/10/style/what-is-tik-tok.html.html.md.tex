Sections

SEARCH

\protect\hyperlink{site-content}{Skip to
content}\protect\hyperlink{site-index}{Skip to site index}

\href{https://www.nytimes.com/section/style}{Style}

\href{https://myaccount.nytimes.com/auth/login?response_type=cookie\&client_id=vi}{}

\href{https://www.nytimes.com/section/todayspaper}{Today's Paper}

\href{/section/style}{Style}\textbar{}How TikTok Is Rewriting the World

\url{https://nyti.ms/2Hcg93z}

\begin{itemize}
\item
\item
\item
\item
\item
\end{itemize}

Advertisement

\protect\hyperlink{after-top}{Continue reading the main story}

Supported by

\protect\hyperlink{after-sponsor}{Continue reading the main story}

\hypertarget{how-tiktok-is-rewriting-the-world}{%
\section{How TikTok Is Rewriting the
World}\label{how-tiktok-is-rewriting-the-world}}

TikTok will change the way your social media works --- even if you're
avoiding it.

\includegraphics{https://static01.nyt.com/images/2019/03/07/style/07tiktok-1/07tiktok-still-1-articleLarge.gif?quality=75\&auto=webp\&disable=upscale}

By \href{https://www.nytimes.com/by/john-herrman}{John Herrman}

\begin{itemize}
\item
  March 10, 2019
\item
  \begin{itemize}
  \item
  \item
  \item
  \item
  \item
  \end{itemize}
\end{itemize}

\href{https://cn.nytimes.com/technology/20190312/what-is-tik-tok/}{阅读简体中文版}\href{https://cn.nytimes.com/technology/20190312/what-is-tik-tok/zh-hant/}{閱讀繁體中文版}

Hello, person who is, statistically speaking, a human adult aged
approximately ``millennial'' to ``boomer.'' The analytics suggest a high
likelihood that you're aware there is an app named
\href{https://www.nytimes.com/2020/07/26/technology/tiktok-china-ban-model.html}{TikTok},
and a similarly high likelihood that you're not totally sure what it's
all about. Maybe you asked someone younger in your life, and they tried
to explain and possibly failed. Or maybe you've heard that this
\href{https://www.nytimes.com/2018/12/03/technology/tiktok-a-chinese-video-app-brings-fun-back-to-social-media.html}{new,
extraordinarily popular video app} is ``a refreshing outlier in the
social media universe'' that's ``genuinely fun to use.'' Maybe you even
tried it, but bounced straight out, confused and sapped.

``Fear of missing out'' is a common way to describe how social media can
make people feel like everyone else is part of something --- a concert,
a secret beach, a brunch --- that they're not. A new wrinkle in this
concept is that sometimes that ``something'' is a social media platform
itself. Maybe you saw a photo of some friends on Instagram at a great
party and wondered why you weren't there. But then, next in your feed,
you saw a weird video, watermarked with a vibrating TikTok logo, scored
with a song you'd never heard, starring a person you'd never seen. Maybe
you saw one of the staggering number of ads for TikTok plastered
throughout other social networks, and the real world, and wondered why
you weren't at that party, either, and why it seemed so far away.

It's been a while since a new social app got big enough, quickly enough,
to make nonusers feel they're missing out from an experience. If we
exclude Fortnite, which is very social but also very much a game, the
last time an app inspired such interest from people who weren't on it
was \ldots{} maybe Snapchat? (Not a coincidence that Snapchat's audience
skewed very young, too.)

And while you, perhaps an anxious abstainer, may feel perfectly secure
in your ``choice'' not to join that service, Snapchat has more daily
users than Twitter, changed the course of its industry, and altered the
way people communicate with their phones. TikTok, now
\href{https://www.scmp.com/tech/article/2155580/tik-tok-hits-500-million-global-monthly-active-users-china-social-media-video}{reportedly}
500 million users strong, is not so obvious in its intentions. But that
doesn't mean it doesn't have them! Shall we?

\begin{center}\rule{0.5\linewidth}{\linethickness}\end{center}

\hypertarget{the-basic-human-explanation}{%
\subsection{The basic human
explanation}\label{the-basic-human-explanation}}

of TikTok.

TikTok is an app for making and sharing short videos. The videos are
tall, not square, like on Snapchat or Instagram's stories, but you
navigate through videos by scrolling up and down, like a feed, not by
tapping or swiping side to side.

Video creators have all sorts of tools at their disposal: filters as on
Snapchat (and later, everyone else); the ability to search for sounds to
score your video. Users are also strongly encouraged to engage with
other users, through ``response'' videos or by means of ``duets'' ---
users can duplicate videos and add themselves alongside.

Hashtags play a surprisingly large role on TikTok. In more innocent
times, Twitter hoped its users might congregate around hashtags in a
never-ending series of productive pop-up mini-discourses. On TikTok,
hashtags actually exist as a real, functional organizing principle: not
for news, or even really anything trending anywhere else than TikTok,
but for various ``challenges,'' or jokes, or repeating formats, or other
discernible blobs of activity.

\hypertarget{-in-this-series-of-tiktoks}{%
\subsubsection{\texorpdfstring{► \textbf{In this series of
TikToks,}}{► In this series of TikToks,}}\label{-in-this-series-of-tiktoks}}

\textbf{\href{https://www.tiktok.com/share/video/6666202402800012550}{@DonJuanFutrell
goes shopping and}}\\
\textbf{\href{https://www.tiktok.com/share/video/6666202402800012550}{reinvents
language}.}

\hypertarget{--in-this-tiktok-a-popular-song-for-memes}{%
\subsubsection{\texorpdfstring{► **** In this TikTok,
\href{https://www.tiktok.com/share/video/6664754374910151941}{a popular
song for
memes}}{► **** In this TikTok, a popular song for memes}}\label{--in-this-tiktok-a-popular-song-for-memes}}

\href{https://www.tiktok.com/share/video/6664754374910151941}{is used to
express shopping as a outsider.}

TikTok is, however, a free-for-all. It's easy to make a video on TikTok,
not just because of the tools it gives users, but because of extensive
reasons and prompts it provides for you. You can select from an enormous
range of sounds, from popular song clips to short moments from TV shows,
YouTube videos or \emph{other} TikToks. You can join a dare-like
challenge, or participate in a dance meme, or make a joke. Or you can
make fun of all of these things.

TikTok assertively answers anyone's \emph{what should I watch} with a
flood. In the same way, the app provides plenty of answers for the
paralyzing \emph{what should I post?} The result is an endless
unspooling of material that people, many very young, might be too
self-conscious to post on Instagram, or that they never would have come
up with in the first place without a nudge. It can be hard to watch. It
can be charming. It can be very, very funny. It is frequently, in the
language widely applied outside the platform, from people on
\emph{other} platforms, extremely ``cringe.''

\begin{center}\rule{0.5\linewidth}{\linethickness}\end{center}

\hypertarget{so-thats-whats-on-tiktok}{%
\subsection{So that's what's on
TikTok.}\label{so-thats-whats-on-tiktok}}

What \emph{is} it?

TikTok can feel, to an American audience, a bit like a greatest hits
compilation, featuring only the most engaging elements and experiences
of its predecessors. This is true, to a point. But TikTok --- known as
Douyin in China, where its parent company is based --- must also be
understood as one of the most popular of \emph{many} short-video-sharing
apps in that country. This is a
\href{https://radiichina.com/a-quick-guide-to-chinas-competing-short-video-apps/}{landscape}
that evolved both alongside and at arm's length from the American tech
industry --- Instagram, for example, is banned in China.

Under the hood, TikTok is a fundamentally different app than American
users have used before. It may look and feel like its
friend-feed-centric peers, and you can follow and be followed; of course
there are hugely popular ``stars,'' many
\href{https://technode.com/2018/06/15/8-lessons-douyin/}{cultivated} by
the company itself. There's messaging. Users can and do use it like any
other social app. But the various aesthetic and functional similarities
to Vine or Snapchat or Instagram belie a core difference: TikTok is more
machine than man. In this way, it's from the future --- or at least
\emph{a} future. And it has some messages for us.

Consider the trajectory of what we think of as the major social apps.

\begin{center}\rule{0.5\linewidth}{\linethickness}\end{center}

\hypertarget{instagram-and-twitter-could-only}{%
\subsection{Instagram and Twitter could
only}\label{instagram-and-twitter-could-only}}

take us so far.

Twitter gained popularity as a tool for following people and being
followed by other people and expanded from there. Twitter watched what
its users did with its original concept and formalized the
conversational behaviors they invented. (See: Retweets. See again:
hashtags.) Only then, and after going public, did it start to become
more assertive. It made more recommendations. It started
\href{https://www.vox.com/culture/2018/9/20/17876098/twitter-chronological-timeline-back-finally}{reordering}
users' feeds based on what it thought they might want to see, or might
have missed. Opaque machine intelligence encroached on the original
system.

\hypertarget{-this-tiktok-is-a-piece-of-unlikely-yet}{%
\subsubsection{\texorpdfstring{► \textbf{This TikTok is a piece of}
\textbf{\href{https://www.tiktok.com/share/video/6666104355344616710?refer=embed}{unlikely
yet}}}{► This TikTok is a piece of unlikely yet}}\label{-this-tiktok-is-a-piece-of-unlikely-yet}}

\textbf{\href{https://www.tiktok.com/share/video/6666104355344616710?refer=embed}{sweet
comedy about kids and vaccination}.}

Something similar happened at Instagram, where algorithmic
recommendation is now a very noticeable part of the experience, and on
YouTube, where recommendations shuttle one around the platform in new
and often \ldots{} let's say
\href{https://www.buzzfeednews.com/article/carolineodonovan/down-youtubes-recommendation-rabbithole}{\emph{surprising}}
ways. Some users might feel affronted by these assertive new automatic
features, which are clearly designed to increase interaction. One might
reasonably worry that this trend serves the lowest demands of a brutal
attention economy that is revealing tech companies as cynical
time-mongers and turning us into mindless drones.

These changes have also tended to work, at least on those terms. We
often do spend more time with the apps as they've become more assertive,
and less intimately human, even as we've complained.

What's both crucial and easy to miss about TikTok is how it has stepped
over the midpoint between the familiar self-directed feed and an
experience based
\href{https://motherboard.vice.com/en_us/article/kzdwn9/tiktok-cant-save-us-from-algorithmic-content-hell}{first}
on algorithmic observation and inference. The most obvious clue is right
there when you open the app: the first thing you see isn't a feed of
your friends, but a page called ``For You.'' It's an algorithmic feed
based on videos you've interacted with, or even just watched. It never
runs out of material. It is not, unless you train it to be, full of
people you know, or things you've explicitly told it you want to see.
It's full of things that you seem to have demonstrated you want to
watch, no matter what you actually say you want to watch.

It is constantly learning from you and, over time, builds a presumably
complex but opaque model of what you tend to watch, and shows you more
of that, or things like that, or things related to that, or, honestly,
who knows, but it seems to work. TikTok starts making assumptions the
second you've opened the app, before you've really given it anything to
work with. Imagine an Instagram centered entirely around its ``Explore''
tab, or a Twitter built around, I guess, trending topics or viral
tweets, with ``following'' bolted onto the side.

Imagine a version of Facebook that was able to fill your feed before
you'd friended a single person. That's TikTok.

Its mode of creation is unusual, too. You can make stuff for your
friends, or in response to your friends, sure. But users looking for
something to post about are immediately recruited into group challenges,
or hashtags, or shown popular songs. The bar is low. The stakes are low.
Large audiences feel within reach, and smaller ones are easy to find,
even if you're just messing around.

\hypertarget{--this-tiktok-is-a-great-example}{%
\subsubsection{\texorpdfstring{► ****
\textbf{\href{https://www.tiktok.com/share/video/6652983964904459522}{This
TikTok is a great
example}}}{► **** This TikTok is a great example}}\label{--this-tiktok-is-a-great-example}}

\textbf{\href{https://www.tiktok.com/share/video/6652983964904459522}{of
``Fake Plane Challenge.''}}

On most social networks the first step to showing your content to a lot
of people is grinding to build an audience, or having lots of friends,
or being incredibly beautiful or wealthy or idle and willing to display
that, or getting lucky or striking viral gold. TikTok instead encourages
users to jump from audience to audience, trend to trend, creating
something like simulated temporary friend groups, who get together to do
friend-group things: to share an inside joke; to riff on a song; to talk
idly and aimlessly about whatever is in front of you. Feedback is
instant and frequently abundant; virality has a stiff tailwind.
Stimulation is constant. There is an unmistakable sense that you're
using something that's expanding in every direction. The pool of content
is enormous. Most of it is meaningless. Some of it becomes popular, and
some is great, and some gets to be both. As The Atlantic's Taylor Lorenz
\href{https://www.theatlantic.com/technology/archive/2018/10/what-tiktok-is-cringey-and-thats-fine/573871/}{put
it}, ``Watching too many in a row can feel like you're about to have a
brain freeze. They're incredibly addictive.''

\begin{center}\rule{0.5\linewidth}{\linethickness}\end{center}

\hypertarget{tiktok-is-just-doing}{%
\subsection{TikTok is just doing}\label{tiktok-is-just-doing}}

to you what you told it to do.

In 1994, the artist and software developer Karl Sims demonstrated
``virtual creatures'' that moved in realistic ways discovered through
``genetic algorithms.'' These simulations, through trial and error,
gradually arrived at some
\href{https://www.youtube.com/watch?v=JBgG_VSP7f8}{pre-existing shapes
and movements}: wriggling, slithering, dragging and walking.

But some early models, which emphasized the creatures' ability to cover
a certain distance as quickly as possible, resulted in the evolution of
a very tall, rigid being that simply fell over. In doing so, it
``moved'' more quickly than a wriggling peer. It didn't understand its
evolutionary priority as ``creature-like locomotion.'' It needed to get
to a certain place as efficiently as possible. And it did.

Older social apps are continuously evolving, too. Their models
prioritize growth and discovery, of course, but also assume the
centrality of \emph{your people}: the accounts you follow and which
follow you, or with whom you communicate directly, and are bound up in
their founding myths and structures: Facebook's social graph; the News
Feed; the Instagram feed; Twitter's rigid user relationships.

\hypertarget{-tiktok-is-often-used-for}{%
\subsubsection{\texorpdfstring{► \textbf{TikTok is often used
for}}{► TikTok is often used for}}\label{-tiktok-is-often-used-for}}

\textbf{showing off talent like drawing, building}\\
\textbf{\href{https://www.tiktok.com/share/video/6661522824781450497}{and
``transformation by makeup.''}}

TikTok though is the towering stick falling far and fast, not caring to
wait to evolve through a wriggling, cumbersome social phase, but instead
asking: Why not just start showing people things and see what they do
about it? Why not just ask people to start making things and see what
happens? If engagement is how success is measured, why not just design
the app where taking up time is \emph{the entire point}? There's no
rule, in apps or elsewhere, against engagement for engagement's sake.
Let the creature grow tall and fall upon us all.

\begin{center}\rule{0.5\linewidth}{\linethickness}\end{center}

\hypertarget{in-what-laboratory-was-this}{%
\subsection{In What Laboratory Was
This}\label{in-what-laboratory-was-this}}

Monster Made?

TikTok is far from an evolutionary fluke. Its parent company, ByteDance,
recently
\href{https://www.theverge.com/2018/10/26/18026250/bytedance-china-tiktok-valuation-highest-toutiao}{valued}
at more than \$75 billion, \href{https://bytedance.com/}{bills itself}
first as an artificial intelligence company, not a creator of
mission-driven social platforms. TikTok was merged with Musical.ly, a
social network initially built around lip-syncing and dancing and
adopted by
\href{https://www.nytimes.com/2016/09/17/business/media/a-social-network-frequented-by-children-tests-the-limits-of-online-regulation.html?module=inline}{very
young people}. It still carries a lot of Musical.ly's DNA, and its app
store reviews contain more than a little yearning for Musical.ly's
return. It was the defunct Musical.ly against which the Federal Trade
Commission
\href{https://www.nytimes.com/2019/02/27/technology/ftc-tiktok-child-privacy-fine.html}{recently
levied its largest-ever penalty} for mishandling the private data of
young users.

\hypertarget{-press-twist-and-pull-written}{%
\subsubsection{► Press, twist and pull
written}\label{-press-twist-and-pull-written}}

on a hand are popular instructions to\\
receive a message that is revealed\\
to be written on the palm.
\href{https://www.tiktok.com/share/video/6665427333471554822}{``Almost
got}\\
\href{https://www.tiktok.com/share/video/6665427333471554822}{kicked out
of my house for this one''}\\
\href{https://www.tiktok.com/share/video/6665427333471554822}{claimed a
user.}

``ByteDance's content platforms enable people to enjoy content powered
by AI technology,'' its website says. Its vision is ``to build global
creation and interaction platforms.'' ByteDance's wildly popular news
and entertainment portal, Jinri Toutiao (translated as ``Today's
Headlines,'') relies heavily on AI ---~not human editors, or a
self-selected feed of accounts --- to curate and create customized
streams of largely user-and-partner-generated content tailored to each
of its readers.

These are services where a sort of ``filter'' bubble --- isolating users
into worlds of points of view --- isn't an unintended consequence. It's
the point. And it's extremely effective: Both Toutiao and Douyin have
drawn attention from Chinese regulators for, among many other things,
some
\href{https://www.newyorker.com/news/daily-comment/why-china-cracked-down-on-the-social-media-giant-bytedance}{familiar}
to any large social-ish platform, and others
\href{https://www.newyorker.com/news/daily-comment/why-china-cracked-down-on-the-social-media-giant-bytedance}{unique}
to its speech-constrained political environment, capturing too much user
time. As a result, TikTok's ``Digital Wellbeing'' settings include an
option to enforce a password-protected time limit. The company's other
challenges can be addressed more assertively: an algorithm-first
attention market isn't just centrally ruled, it's centrally allocated.

\begin{center}\rule{0.5\linewidth}{\linethickness}\end{center}

\hypertarget{why-do-people-spend-hours}{%
\subsection{Why Do People Spend Hours}\label{why-do-people-spend-hours}}

on TikTok? It's the Machines.

All of this goes a long way to explain why, at least at first, TikTok
can seem disorienting. ``You're not actually sure why you're seeing what
you're seeing,'' said Ankur Thakkar, the former editorial lead at Vine,
TikTok's other most direct forerunner. On Vine, a new user might not
have had much to watch, or felt much of a reason to create anything, but
they understood their context: the list of people they followed, which
was probably the thing letting them down.

``It's doing the thing that Twitter tried to solve, that everyone tried
to solve,'' he said. ``How do you get people to engage?'' Apparently you
just \ldots{} show them things, and let a powerful artificial
intelligence take notes. You start sending daily notifications
immediately. You tell them what to do. You fake it till you make it,
algorithmically speaking.

Image

A friendly daily reminder to re-engage

American social platforms, each fighting their own desperate and often
stock-price-related fights to increase user engagement, have been
trending in TikTok's general direction for a while. It is possible,
today, to receive highly personalized and effectively infinite content
recommendations in YouTube without ever following a single account,
because Google already watches what you do, and makes guesses about who
you are. And while Facebook and Twitter don't talk about their products
this way, we understand that sometimes ---~maybe a lot of the time
---~we use them just to fill time. They, in turn, want as much of our
time as possible, and are quite obviously doing whatever they can to get
it.

So maybe you'll sit TikTok out. But these things have a way of sneaking
up behind you. Maybe you never joined Snapchat --- but its rise worried
Facebook so much that its prettier product, Instagram, was remade in its
image, and copied concepts from Snapchat reached you there.

And maybe you skipped Twitter --- but it still rewired your entire news
diet, and, besides, it's how the president talks to you, now.

TikTok does away with many of the assumptions other social platforms
have been built upon, and which they are in the process of discarding
anyway. It questions the primacy of individual connections and friend
networks. It unapologetically embraces central control rather than
pretending it doesn't have it. TikTok's real influence going forward may
be that the other social media platforms decide that our friends were
simply holding us back. Or, at least, it was holding \emph{them} back.

Advertisement

\protect\hyperlink{after-bottom}{Continue reading the main story}

\hypertarget{site-index}{%
\subsection{Site Index}\label{site-index}}

\hypertarget{site-information-navigation}{%
\subsection{Site Information
Navigation}\label{site-information-navigation}}

\begin{itemize}
\tightlist
\item
  \href{https://help.nytimes.com/hc/en-us/articles/115014792127-Copyright-notice}{©~2020~The
  New York Times Company}
\end{itemize}

\begin{itemize}
\tightlist
\item
  \href{https://www.nytco.com/}{NYTCo}
\item
  \href{https://help.nytimes.com/hc/en-us/articles/115015385887-Contact-Us}{Contact
  Us}
\item
  \href{https://www.nytco.com/careers/}{Work with us}
\item
  \href{https://nytmediakit.com/}{Advertise}
\item
  \href{http://www.tbrandstudio.com/}{T Brand Studio}
\item
  \href{https://www.nytimes.com/privacy/cookie-policy\#how-do-i-manage-trackers}{Your
  Ad Choices}
\item
  \href{https://www.nytimes.com/privacy}{Privacy}
\item
  \href{https://help.nytimes.com/hc/en-us/articles/115014893428-Terms-of-service}{Terms
  of Service}
\item
  \href{https://help.nytimes.com/hc/en-us/articles/115014893968-Terms-of-sale}{Terms
  of Sale}
\item
  \href{https://spiderbites.nytimes.com}{Site Map}
\item
  \href{https://help.nytimes.com/hc/en-us}{Help}
\item
  \href{https://www.nytimes.com/subscription?campaignId=37WXW}{Subscriptions}
\end{itemize}
