Sections

SEARCH

\protect\hyperlink{site-content}{Skip to
content}\protect\hyperlink{site-index}{Skip to site index}

\href{https://www.nytimes.com/section/reader-center}{Reader Center}

\href{https://myaccount.nytimes.com/auth/login?response_type=cookie\&client_id=vi}{}

\href{https://www.nytimes.com/section/todayspaper}{Today's Paper}

\href{/section/reader-center}{Reader Center}\textbar{}How Our Former
Beijing Bureau Chief Found Himself on a Bullet Train in Saudi Arabia

\url{https://nyti.ms/2W12Vdi}

\begin{itemize}
\item
\item
\item
\item
\item
\item
\end{itemize}

Advertisement

\protect\hyperlink{after-top}{Continue reading the main story}

Supported by

\protect\hyperlink{after-sponsor}{Continue reading the main story}

Times Insider

\hypertarget{how-our-former-beijing-bureau-chief-found-himself-on-a-bullet-train-in-saudi-arabia}{%
\section{How Our Former Beijing Bureau Chief Found Himself on a Bullet
Train in Saudi
Arabia}\label{how-our-former-beijing-bureau-chief-found-himself-on-a-bullet-train-in-saudi-arabia}}

There was a certain dissonance in the fact that the ruling Communist
Party of China, officially atheist and repressive toward the country's
Muslims, had helped build a railway connecting the holiest sites in
Islam.

\includegraphics{https://static01.nyt.com/images/2019/03/15/insider/15insider-saudi-railroad-image4/00insider-saudi-railroad-image4-articleLarge.jpg?quality=75\&auto=webp\&disable=upscale}

\href{https://www.nytimes.com/by/edward-wong}{\includegraphics{https://static01.nyt.com/images/2018/09/24/multimedia/author-edward-wong/author-edward-wong-thumbLarge-v5.png}}

By \href{https://www.nytimes.com/by/edward-wong}{Edward Wong}

\begin{itemize}
\item
  March 14, 2019
\item
  \begin{itemize}
  \item
  \item
  \item
  \item
  \item
  \item
  \end{itemize}
\end{itemize}

\href{https://cn.nytimes.com/world/20190320/saudi-arabia-high-speed-train-medina-mecca/}{阅读简体中文版}\href{https://cn.nytimes.com/world/20190320/saudi-arabia-high-speed-train-medina-mecca/zh-hant/}{閱讀繁體中文版}

\href{http://www.nytimes.com/section/insider?module=inline}{\emph{Times
Insider}} \emph{explains who we are and what we do, and delivers
behind-the-scenes insights into how our journalism comes together.}

MEDINA, Saudi Arabia --- One thing I learned from working for 13 years
as a foreign correspondent is that getting to and from a place can teach
you a lot about a country. Even
\href{https://www.nytimes.com/2005/06/23/world/middleeast/flight-15-to-basra-fewperks-but-no-bombs.html}{in
a war zone}. It's a cliché to say it's the journey that matters and not
the destination, but there is truth in that statement.

So it was that I found myself canceling a plane ticket from Medina to
Jeddah in Saudi Arabia in January in favor of taking the train.

It was not just any train.

I had discovered, while on a trip to report on
\href{https://www.nytimes.com/2019/02/11/world/middleeast/saudi-arabia-tourism-music-festival.html}{a
high-end concert series} in the remote Al Ula region, that in October
Saudi Arabia had opened a high-speed railway between Medina and Mecca,
the most important pilgrimage sites for Muslims. It ran a total distance
of 281 miles.

\includegraphics{https://static01.nyt.com/images/2019/03/14/insider/00insider-saudi-railroad-image8/00insider-saudi-railroad-image8-articleLarge.jpg?quality=75\&auto=webp\&disable=upscale}

I had a flight from Jeddah that night back to Washington, where I now
work as a diplomatic correspondent. If I boarded the train around noon,
I could get to Jeddah by midafternoon, have time to walk around the old
town and seaside corniche, maybe get dinner and catch my flight.

Booking a ticket was as easy as making a reservation on Amtrak. From my
hotel room in a desert canyon in Al Ula, I got on the website of the
\href{https://www.hhr.sa/sites/sro/Pages/home.aspx}{Haramain High Speed
Railway}, clicked on the English-language option and looked at the
schedule. There was a train departing at noon the next day that would
get me into Jeddah at 2:16 p.m. All economy-class tickets were sold out
so, using an American credit card, I booked a business-class seat for
220.5 Saudi riyals, or \$59.

I've always enjoyed train travel, but there was a particular reason this
railway intrigued me. When I was Beijing bureau chief, my job at The
Times before this one, I researched commercial projects abroad that
involved Chinese companies. Chinese state-owned enterprises were getting
infrastructure contracts in many countries, even before President Xi
Jinping began heavily promoting his
\href{https://www.nytimes.com/2019/01/13/world/africa/china-loans-africa-usa.html}{Belt
and Road Initiative}.

I came across the fact that a Chinese state-owned enterprise was
involved in the first phase of building a high-speed railway between
Medina and Mecca. There was a certain dissonance here: The ruling
Communist Party of China, officially atheist and
\href{https://www.nytimes.com/2018/09/10/world/asia/us-china-sanctions-muslim-camps.html}{repressive
toward many of the country's Muslims}, was helping build a railway
connecting the holiest sites in Islam.

I emailed an official at the Saudi Embassy in Washington about getting a
visa to go report in Saudi Arabia, but nothing came of it. The country
usually does not issue tourist visas, and journalist visas are hard to
get.

That was years ago. Now I had a visa, granted so I could cover
\href{https://www.nytimes.com/2019/01/14/world/middleeast/pompeo-saudi-arabia-mohammed-bin-salman.html}{a
trip by Secretary of State Mike Pompeo} to the kingdom. And at the end
of the assignment, after four days of reporting in Riyadh and Ula, I had
a train ticket to Jeddah.

Image

An economy car, showing Makkah (Mecca in English) as the train's
destination.Credit...Edward Wong/The New York Times

On the morning of Jan. 17, a driver took me south from Al Ula to Medina.
We were careful to skirt north around the bustling heart of Medina to
the high-speed train station. Non-Muslims are forbidden from entering
central Medina. Even on the outskirts, I saw more minarets than I had
seen in any other city I had visited in years.

The first thing that struck me at the station were the pilgrims.
Outside, men in white robes and women in full black dress, often with
their faces covered, wheeled suitcases. A few times I saw Muslims who
appeared to be from as far away as Southeast Asia.

The cavernous station was busy but not crowded. A sign pointed to a
mosque. Pillars and archways pulled the eye toward the dark, soaring
ceilings, decorated with diamond-shaped motifs that let in sunlight. The
floor tiles gleamed.

I scanned the electronic ticket on my phone at a turnstile to be let
into the platform area.

In China, I had taken
\href{https://www.nytimes.com/2011/06/23/business/global/23rail.html}{many
high-speed trains}, and these looked similar. There was a bullet-shaped
car at the front and a string of passenger cars behind it. They
resembled, too, the high-speed trains I had taken in Japan and France.

Each train had 417 seats.

Image

In business class, "waiters in white shirts, black vests and white
gloves served dates and Arabic coffee."Credit...Edward Wong/The New York
Times

I walked to a business-class car and got on. The cabin was nearly full.
I sat down in a wide seat. All the seats had a seat-back television
screen. This train had been designed and built by a Spanish company.

Leaving Medina, the train steadily picked up speed, until it reached
about 190 miles per hour. It zipped through flat, dry countryside dotted
with shrubs. Waiters in white shirts, black vests and white gloves
served dates and Arabic coffee. Then they came by with lunch: a chicken
roll, a sweet cake and more coffee.

``What do you think of all this?'' asked an older man with a thick beard
and a traditional red-checkered headdress sitting in front of me. ``It's
a smooth ride?''

The scenery became hillier as we approached the Hijaz Mountains parallel
to the Red Sea, before continuing south along the coast. At 1:30, we
sped through a station in King Abdullah Economic City. Soon afterward,
we pulled into Jeddah.

I wanted to travel on to Mecca, but --- as with central Medina ---
non-Muslims are barred from entering the city. I stepped off the train
with my bag.

Image

A poster at the Jeddah station promoting Saudi Vision 2030, an ambitious
economic development program spearheaded by Crown Prince Mohammed bin
Salman.Credit...Edward Wong/The New York Times

As I walked to the taxi stand with the pilgrims, I saw a poster with
large photos of King Salman, Crown Prince Mohammed bin Salman and the
train, plus the words Saudi Vision 2030. A family took cellphone photos
of themselves standing in front of it.

Saudi Vision 2030 is the catchphrase for an ambitious economic
development program spearheaded by the crown prince. In the West, he is
now known more for violent acts --- carrying on
\href{https://www.nytimes.com/interactive/2018/10/31/magazine/yemen-war-saudi-arabia.html}{a
war against rebels in Yemen} that has resulted in the world's worst
humanitarian crisis, and,
\href{https://www.nytimes.com/2018/11/16/us/politics/cia-saudi-crown-prince-khashoggi.html}{according
to the C.I.A.}, ordering the murder last October of Jamal Khashoggi, a
Virginia resident and Washington Post columnist. The Saudi government
has denied the crown prince's involvement in the murder.

The train presented a different glimpse of the complexities of the
kingdom.

I got into a taxi and looked back at the pilgrims streaming from the
station as we drove off toward the Red Sea.

Follow the \href{https://twitter.com/readercenter}{@ReaderCenter} on
Twitter for more coverage highlighting your perspectives and experiences
and for insight into how we work.

Advertisement

\protect\hyperlink{after-bottom}{Continue reading the main story}

\hypertarget{site-index}{%
\subsection{Site Index}\label{site-index}}

\hypertarget{site-information-navigation}{%
\subsection{Site Information
Navigation}\label{site-information-navigation}}

\begin{itemize}
\tightlist
\item
  \href{https://help.nytimes.com/hc/en-us/articles/115014792127-Copyright-notice}{©~2020~The
  New York Times Company}
\end{itemize}

\begin{itemize}
\tightlist
\item
  \href{https://www.nytco.com/}{NYTCo}
\item
  \href{https://help.nytimes.com/hc/en-us/articles/115015385887-Contact-Us}{Contact
  Us}
\item
  \href{https://www.nytco.com/careers/}{Work with us}
\item
  \href{https://nytmediakit.com/}{Advertise}
\item
  \href{http://www.tbrandstudio.com/}{T Brand Studio}
\item
  \href{https://www.nytimes.com/privacy/cookie-policy\#how-do-i-manage-trackers}{Your
  Ad Choices}
\item
  \href{https://www.nytimes.com/privacy}{Privacy}
\item
  \href{https://help.nytimes.com/hc/en-us/articles/115014893428-Terms-of-service}{Terms
  of Service}
\item
  \href{https://help.nytimes.com/hc/en-us/articles/115014893968-Terms-of-sale}{Terms
  of Sale}
\item
  \href{https://spiderbites.nytimes.com}{Site Map}
\item
  \href{https://help.nytimes.com/hc/en-us}{Help}
\item
  \href{https://www.nytimes.com/subscription?campaignId=37WXW}{Subscriptions}
\end{itemize}
