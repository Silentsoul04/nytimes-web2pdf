Sections

SEARCH

\protect\hyperlink{site-content}{Skip to
content}\protect\hyperlink{site-index}{Skip to site index}

\href{https://www.nytimes.com/section/politics}{Politics}

\href{https://myaccount.nytimes.com/auth/login?response_type=cookie\&client_id=vi}{}

\href{https://www.nytimes.com/section/todayspaper}{Today's Paper}

\href{/section/politics}{Politics}\textbar{}U.S. Campaign to Ban Huawei
Overseas Stumbles as Allies Resist

\url{https://nyti.ms/2ucgGd5}

\begin{itemize}
\item
\item
\item
\item
\item
\item
\end{itemize}

Advertisement

\protect\hyperlink{after-top}{Continue reading the main story}

Supported by

\protect\hyperlink{after-sponsor}{Continue reading the main story}

\hypertarget{us-campaign-to-ban-huawei-overseas-stumbles-as-allies-resist}{%
\section{U.S. Campaign to Ban Huawei Overseas Stumbles as Allies
Resist}\label{us-campaign-to-ban-huawei-overseas-stumbles-as-allies-resist}}

\includegraphics{https://static01.nyt.com/images/2019/03/14/us/politics/17dc-huawei1/merlin_151330467_c260cd99-86ac-476a-ba25-fb4631ab0263-articleLarge.jpg?quality=75\&auto=webp\&disable=upscale}

By \href{https://www.nytimes.com/by/julian-e-barnes}{Julian E. Barnes}
and \href{https://www.nytimes.com/by/adam-satariano}{Adam Satariano}

\begin{itemize}
\item
  March 17, 2019
\item
  \begin{itemize}
  \item
  \item
  \item
  \item
  \item
  \item
  \end{itemize}
\end{itemize}

\href{https://cn.nytimes.com/usa/20190318/huawei-ban/}{阅读简体中文版}\href{https://cn.nytimes.com/usa/20190318/huawei-ban/zh-hant/}{閱讀繁體中文版}

WASHINGTON --- The Trump administration's
\href{https://www.nytimes.com/2019/01/26/us/politics/huawei-china-us-5g-technology.html}{aggressive
campaign} to prevent countries from using Huawei and other Chinese
telecommunications equipment in their next-generation wireless networks
has faltered, with even some of America's closest allies rejecting the
United States' argument that the companies pose a security threat.

Over the past several months, American officials have tried to pressure,
scold and, increasingly, threaten other nations that are considering
using Huawei in building fifth-generation, or 5G, wireless networks.
Mike Pompeo, the secretary of state, has pledged to withhold
intelligence from nations that continue to use Chinese telecom
equipment. The American ambassador to Germany cautioned Berlin this
month that the United States would curtail intelligence sharing if that
country used Huawei.

The warnings stem from the United States' concern that Huawei and other
Chinese telecom companies are a significant security threat given
Beijing's control over the industry. Top officials have pointed to new
Chinese security laws that require Huawei and other companies to provide
information to intelligence officials, arguing China could gain access
to the vast amounts of data that will ultimately travel over 5G,
allowing Beijing to spy on companies, individuals and governments --- an
accusation Huawei has vehemently denied.

But the campaign has run aground.
\href{https://www.nytimes.com/2019/02/20/business/huawei-uk-trump.html}{Britain},
Germany, India and the
\href{https://www.nytimes.com/2019/02/26/technology/huawei-uae-5g-network.html}{United
Arab Emirates} are among the countries signaling they are unlikely to
back the American effort to entirely ban Huawei from building their 5G
networks. While some countries like Britain share the United States'
concerns, they argue that the security risks can be managed by closely
scrutinizing the company and its software.

The decisions are a blow to the Trump administration's efforts to rein
in Beijing's economic and technological ambitions and to stop China from
playing a central role in the next iteration of the internet.

American government officials are now looking for other ways to curb
Huawei's global rise without the cooperation of overseas allies,
including possibly restricting American companies from supplying Huawei
with key components that it needs to build 5G networks across the world.

``It is looking dicey. We are running out of runway,'' said Mike Rogers,
the former Republican congressman who led the House Intelligence
Committee and who has long been a fierce critic of Huawei.

The United States is not ready to admit defeat, but its campaign has
suffered from what foreign officials say is a scolding approach and a
lack of concrete evidence that Huawei poses a real risk. It has also
been hampered by a perception among European and Asian officials that
President Trump may not be fully committed to the fight.

Mr. Trump has repeatedly undercut his own Justice Department, which
unveiled
\href{https://www.nytimes.com/2019/01/28/us/politics/meng-wanzhou-huawei-iran.html}{sweeping
criminal indictments} against Huawei and its chief financial officer
with accusations of fraud, sanctions evasion and obstruction of justice.
Mr. Trump has suggested that the charges
\href{https://www.nytimes.com/2019/02/22/business/economy/china-usa-trade.html}{could
be dropped} as part of a trade deal with China. The president previously
\href{https://www.nytimes.com/2018/07/13/business/zte-ban-trump.html}{eased
penalties} on another Chinese telecom firm accused of violating American
sanctions, ZTE, after a personal appeal by President Xi Jinping of
China.

Those moves have only deepened concerns that the administration's fight
against Huawei is not really about national security and instead
reflects its political and economic ambitions.

European and Asian officials have complained privately that recent
American intelligence briefings for allies did not share any sort of
classified information that clearly demonstrated how the Chinese
government used Huawei to steal information, according to people
familiar with the discussions. European officials have told counterparts
that if the United States has evidence the Chinese government has used
its companies to do so, they should disclose it.

One senior European telecommunications executive said that no American
officials had presented ``actual facts'' about China's abuse of Huawei
networks.

Ren Zhengfei, the founder of Huawei,
\href{https://www.nytimes.com/2019/02/18/technology/huawei-ren-zhengfei-bbc.html}{has
accused the United States} of having political motivations in leveling
criminal charges against the company and has said the firm does not spy
for China.

\includegraphics{https://static01.nyt.com/images/2019/03/18/us/politics/17dc-huawei2/merlin_150768993_d0f86da8-0108-4a89-b0e3-91be9b19d4bd-articleLarge.jpg?quality=75\&auto=webp\&disable=upscale}

Unlike the United States, European wireless networks are much more
dependent on Huawei, so banning its equipment would be far more
consequential. Many of the leading carriers, including Vodafone and
Deutsche Telekom, use the company's equipment, and a widespread ban
would result in costly changes that executives have warned may delay the
debut of 5G in the region.

Garrett Marquis, a spokesman for the National Security Council, said the
United States continued to work ``with our allies and like-minded
partners to mitigate risk in the deployment of 5G and other
communications infrastructure.''

Mr. Rogers said the notion that other nations could adequately manage
the security risk was misplaced. ``They are so convinced they can get
over the security problem. It defies logic,'' he said.

But he said Mr. Trump had not helped his administration's efforts by
suggesting that a national security matter like Huawei could be wrapped
into some type of trade pact with China.

``That is a big mistake,'' Mr. Rogers said. ``You have taken a national
security issue and given it away in a trade deal. This is about the
security of data.''

Europeans have their own China trade worries, which could also factor
into reluctance to ban Huawei. Although European officials have grown
increasingly suspicious of Beijing's growing economic might, China is
still the European Union's
\href{http://ec.europa.eu/trade/policy/countries-and-regions/countries/china/index_en.htm}{second-largest
trading partner} after the United States. This week, Mr. Xi is scheduled
to be in Italy.

``I'm not sure a ban is the solution,'' said Caroline Nagtegaal, a
member of European Parliament from the Netherlands who helped write a
resolution on the cybersecurity risks posed by China that avoided
calling for a Huawei ban. ``We have to be very careful making a step
like that.''

Many countries facing American pressure have not made any final
decisions. In Britain, for instance, intelligence officials say the
threat can be managed, but the government could ultimately overrule
them.

To bolster its campaign, the administration has begun threatening
retaliation against countries that do not agree to its demands.

Mr. Pompeo suggested in Hungary that the presence of Huawei could
influence decisions on where to station troops overseas, noting that its
adoption in wireless networks would make it ``more difficult for America
to be present.'' He
\href{https://www.foxbusiness.com/technology/pompeo-slams-huawei-us-wont-partner-with-countries-that-use-its-technology}{followed
up on Fox Business Network}, saying if countries adopted Huawei
technology, the United States ``won't be able to share information''
with them.

The American ambassador to Germany, Richard Grenell, expanded on Mr.
Pompeo's public messaging with a letter to Berlin, warning of
repercussions should it use Huawei. The letter
\href{https://www.wsj.com/articles/drop-huawei-or-see-intelligence-sharing-pared-back-u-s-tells-germany-11552314827}{was
first reported} by The Wall Street Journal.

Chancellor Angela Merkel of Germany quickly shot back, saying her
country was ``defining our standards for ourselves.''

Andrea Kendall-Taylor, a former American intelligence officer who is now
the director of the Transtlantic Security Program at the Center for a
New American Security, said administration officials had wrongly framed
the decision for European powers as standing with either the United
States or China. Countries in Europe, including Britain and Germany, do
not want to make that choice, and instead want to maintain good trade
relations with China.

``The U.S. needs to approach this not as a black and white issue,'' Ms.
Kendall-Taylor said. ``The U.S. should avoid generating more resentment
in already fraught relations with the Europeans. To manage the China
challenge we will need the Europeans on our side, so we need to work
together.''

Image

A showcase for Huawei in Shenzhen, China. The American push against
Huawei has suffered partly from what foreign officials say is a lack of
concrete evidence that the company poses a real risk.Credit...Wang
Zhao/Agence France-Presse --- Getty Images

The Trump administration has had some small victories, at least
rhetorically. The Czech Republic's cybersecurity agency has
\href{https://www.nytimes.com/2019/02/12/world/europe/czech-republic-huawei.html}{issued
warnings about Huawei} and other Chinese telecom companies, though the
government remains divided over a ban. Poland earned praise from Vice
President Mike Pence for its actions against Huawei, which included
\href{https://www.nytimes.com/2019/01/11/world/europe/poland-china-huawei-spy.html}{arresting
one of its employees} on espionage charges. But as Poland courts Chinese
investment, it is unclear if it will embrace a full ban.

The most decisive action against Huawei by an American ally is outside
Europe, where Australia last year banned the company from its 5G
networks.

The administration continues to
\href{https://www.nytimes.com/2019/02/12/us/politics/trump-china-wireless-networks.html}{look
for other ways} to put Huawei at a global disadvantage, including an
executive order that would prohibit American companies from using
Chinese telecommunications gear in 5G networks. Intelligence and
security officials are also considering a more aggressive presidential
order that would prevent American companies from supplying Huawei with
components that it needs to build 5G networks.

While Huawei would eventually make its own version of those components,
such export restrictions could slow down the company's 5G development,
winning time for competitors to improve their own offerings.

American officials are also exploring ways to counter Huawei's biggest
advantage: its low price and financing deals. Members of Congress and
administration officials have discussed ways for the United States and
its allies to offset the favorable financing deals China offers for its
telecom equipment. Among the options under consideration is providing
some type of financing to allied telecom companies building 5G networks.

While the United States has continued to talk tough, Mr. Trump has yet
to sign any executive order that would curb Huawei's growth and his
recent comments have created doubt about how far he is prepared to go.

Last month, the White House dispatched officials from the State, Defense
and Commerce Departments and from the Federal Communications Commission
to a wireless industry conference in Barcelona, Spain, to make the case
against Huawei. But a few days before the convention started, Mr. Trump
appeared to backtrack on his administration's position, posting on
Twitter that he wanted American companies to win on their merits,
``\href{https://twitter.com/realDonaldTrump/status/1098583029713420288?ref_src=twsrc\%5Etfw\%7Ctwcamp\%5Etweetembed\&ref_url=https\%3A\%2F\%2Fwww.cnbc.com\%2F2019\%2F02\%2F21\%2Ftrump-sends-bizarre-6g-tweet-as-china-trade-talks-resume.html}{not
by blocking out} currently more advanced technologies.''

``The administration policy on Huawei and ZTE has been characterized by
fits and starts and contradictions,'' said Representative Adam B.
Schiff, Democrat of California, who has been a top critic of Mr. Trump.
``I am not sure I can make heads or tails of it.''

American and European officials said that behind the scenes, the
negotiations were far more nuanced than the public threats. Some
European officials believe that privately the White House has been more
receptive to their arguments that the security threat of Chinese telecom
companies can be tempered.

But the efforts to cajole or pressure European powers may have come too
late, say current and former European and American officials. European
officials have also told their American counterparts that there is no
alternative to Huawei that offers better, more secure equipment, even at
a higher price.

British officials have said the risk from Huawei can be mitigated
without a ban, through tough oversight and restricting Huawei to less
critical parts of its networks. The British government operates a
security lab where it inspects Huawei's equipment and code for
cybersecurity flaws. Last year, the inspections discovered problems with
Huawei software code, but the authorities said it was not related to the
Chinese government.

Germany is taking a similar approach, with Huawei opening a research
center in the city of Bonn where security officials can review its
products. The company has also opened a facility in Brussels.

Mark Sedwill, Britain's national security adviser, said it was more
important to focus on the security of the system, not the origin of the
company that made the equipment. Criminal hackers, not the governments
of other countries, remain the biggest threat, he said.

``We think we have a pretty mature approach to this that so far ---
through regulation, through transparency, through setting very close
standards --- is protecting our interests and securing economic
benefits,'' Mr. Sedwill said this month during a speech at the Atlantic
Council.

Advertisement

\protect\hyperlink{after-bottom}{Continue reading the main story}

\hypertarget{site-index}{%
\subsection{Site Index}\label{site-index}}

\hypertarget{site-information-navigation}{%
\subsection{Site Information
Navigation}\label{site-information-navigation}}

\begin{itemize}
\tightlist
\item
  \href{https://help.nytimes.com/hc/en-us/articles/115014792127-Copyright-notice}{©~2020~The
  New York Times Company}
\end{itemize}

\begin{itemize}
\tightlist
\item
  \href{https://www.nytco.com/}{NYTCo}
\item
  \href{https://help.nytimes.com/hc/en-us/articles/115015385887-Contact-Us}{Contact
  Us}
\item
  \href{https://www.nytco.com/careers/}{Work with us}
\item
  \href{https://nytmediakit.com/}{Advertise}
\item
  \href{http://www.tbrandstudio.com/}{T Brand Studio}
\item
  \href{https://www.nytimes.com/privacy/cookie-policy\#how-do-i-manage-trackers}{Your
  Ad Choices}
\item
  \href{https://www.nytimes.com/privacy}{Privacy}
\item
  \href{https://help.nytimes.com/hc/en-us/articles/115014893428-Terms-of-service}{Terms
  of Service}
\item
  \href{https://help.nytimes.com/hc/en-us/articles/115014893968-Terms-of-sale}{Terms
  of Sale}
\item
  \href{https://spiderbites.nytimes.com}{Site Map}
\item
  \href{https://help.nytimes.com/hc/en-us}{Help}
\item
  \href{https://www.nytimes.com/subscription?campaignId=37WXW}{Subscriptions}
\end{itemize}
