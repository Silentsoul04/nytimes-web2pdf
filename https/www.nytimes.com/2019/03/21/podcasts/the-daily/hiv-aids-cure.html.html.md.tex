Sections

SEARCH

\protect\hyperlink{site-content}{Skip to
content}\protect\hyperlink{site-index}{Skip to site index}

\href{https://www.nytimes.com/podcasts/the-daily}{The Daily}

\href{https://myaccount.nytimes.com/auth/login?response_type=cookie\&client_id=vi}{}

\href{https://www.nytimes.com/section/todayspaper}{Today's Paper}

\href{/podcasts/the-daily}{The Daily}\textbar{}A Path to Curing H.I.V.

\href{https://nyti.ms/2Tn96GC}{https://nyti.ms/2Tn96GC}

\begin{itemize}
\item
\item
\item
\item
\item
\item
\end{itemize}

Advertisement

\protect\hyperlink{after-top}{Continue reading the main story}

transcript

Back to The Daily

bars

0:00/28:05

-28:05

transcript

\hypertarget{a-path-to-curing-hiv}{%
\subsection{A Path to Curing H.I.V.}\label{a-path-to-curing-hiv}}

\hypertarget{hosted-by-michael-barbaro-produced-by-andy-mills-and-jonathan-wolfe-and-edited-by-paige-cowett-wendy-dorr-and-larissa-anderson}{%
\subsubsection{Hosted by Michael Barbaro, produced by Andy Mills and
Jonathan Wolfe, and edited by Paige Cowett, Wendy Dorr and Larissa
Anderson}\label{hosted-by-michael-barbaro-produced-by-andy-mills-and-jonathan-wolfe-and-edited-by-paige-cowett-wendy-dorr-and-larissa-anderson}}

\hypertarget{a-second-person-appears-to-have-been-cured-of-infection-with-the-virus-that-causes-aids-we-spoke-to-an-activist-about-what-this-means}{%
\paragraph{A second person appears to have been cured of infection with
the virus that causes AIDS. We spoke to an activist about what this
means.}\label{a-second-person-appears-to-have-been-cured-of-infection-with-the-virus-that-causes-aids-we-spoke-to-an-activist-about-what-this-means}}

Thursday, March 21st, 2019

\begin{itemize}
\item
  michael barbaro\\
  From The New York Times, I'm Michael Barbaro. This is ``The Daily.''

  Earlier this month, a second person was cured of H.I.V., something
  that, after three decades of fighting for a cure, scientists and
  activists had almost given up on. How we got here.

  It's Thursday, March 21.

  So Peter Staley, when did you first hear about H.I.V. and AIDS? Where
  were you? And what was going on in your life?
\item
  peter staley\\
  I think I first heard about AIDS probably a year into the crisis, in
  1982.
\item
  archived recording\\
  It's mysterious. It's deadly. And it's baffling medical science.
\end{itemize}

peter staley

Like, the first mainstream television report.

\begin{itemize}
\item
  archived recording 1\\
  The lifestyle of some male homosexuals has triggered an epidemic.
\item
  archived recording 2\\
  Acquired Immune Deficiency Syndrome, or the acronym by which it's
  frequently identified, AIDS.
\end{itemize}

peter staley

And I was still in college at Oberlin.

\begin{itemize}
\tightlist
\item
  archived recording\\
  Researchers are now studying blood and other samples from the victims,
  trying to learn what is causing the disease. So far they have had no
  luck.
\end{itemize}

peter staley

It seemed like this strange mystery. I'm not sure how applicable it felt
to my life.

\begin{itemize}
\item
  archived recording 1\\
  Investigators have examined the habits of homosexuals for clues.
\item
  archived recording 2\\
  I was in the fast lane at one time in terms of the way that I lived my
  life. And now I'm not.
\end{itemize}

peter staley

I was a closeted homosexual, certainly wasn't hearing about it on campus
at all.

And then I came to New York in about April of `83 for this new job on
Wall Street and had my first conversations with other gay men. And at
least in my generation, the very young twenty-somethings, often
closeted, it still seemed remote. We were telling each other that it was
mostly happening in older gay men and those who had been highly, highly
prolific in their sexuality.

michael barbaro

Mm-hmm.

peter staley

So fast forward a couple of years ---

\begin{itemize}
\tightlist
\item
  archived recording\\
  To begin with, AIDS was just an obscure medical curiosity ---
\end{itemize}

peter staley

--- I'm now a bond trader on Wall Street.

\begin{itemize}
\item
  archived recording 1\\
  --- a strange illness affecting a handful of homosexuals and drug
  abusers in American cities. Today, the virus has claimed thousands of
  lives and threatens millions more.
\item
  archived recording 2\\
  When the AIDS alarm sounded four years ago, most people knew little or
  nothing about the disease, much less knew anyone who had it. Now, as
  the number of cases doubles every six months, that profile is
  changing.
\end{itemize}

peter staley

And AIDS has finally become a massive story in the country.

\begin{itemize}
\tightlist
\item
  archived recording\\
  Some might be led to believe that the disease is leveling off. It is
  not.
\end{itemize}

peter staley

And not in a good way.

\begin{itemize}
\item
  archived recording 1\\
  Many researchers would argue that the number of cases is actually 10
  times higher than those reported by the Centers for Disease Control.
\item
  archived recording 2\\
  You know, they should just come out and tell people.
\end{itemize}

peter staley

The country was in a panic.

\begin{itemize}
\tightlist
\item
  archived recording\\
  You know, because everybody's scared.
\end{itemize}

peter staley

There were cover stories on all the national magazines about AIDS.
Parents were pulling their kids out of schools if there were rumors
about a child having H.I.V.

\begin{itemize}
\item
  archived recording 1\\
  These people are raising money and signing petitions in a fight to
  keep Ryan White out of their school.
\item
  archived recording 2\\
  If he has a spill on a table, a chair, your daughter comes in, touches
  it, what's to say she is not going to get it?
\end{itemize}

peter staley

And I sat down with my kind of new-ish boyfriend at the time ---

\begin{itemize}
\item
  archived recording (speaker 1)\\
  Breakfast is served.
\item
  archived recording (speaker 2)\\
  Thanks. We're almost out of shaving cream.
\item
  archived recording (speaker 1)\\
  Oh, O.K.
\end{itemize}

peter staley

--- to watch the very first television movie about the subject, which
was called ``An Early Frost.''

\begin{itemize}
\tightlist
\item
  archived recording\\
  So how'd it go with your folks?
\end{itemize}

peter staley

And Aidan Quinn was the star.

\begin{itemize}
\tightlist
\item
  archived recording\\
  You didn't tell them, did you?
\end{itemize}

peter staley

He played a closeted gay man ---

\begin{itemize}
\item
  archived recording (speaker 1)\\
  Look. I don't have the same relationship with my parents that you have
  with yours. O.K.? I don't talk about sex with them. They don't talk
  about sex with me.
\item
  archived recording (speaker 2)\\
  Who's talking about sex? We're talking about us.
\end{itemize}

peter staley

--- who gets hit by AIDS with P.C.P. pneumonia, which was the common
killer back then.

\begin{itemize}
\tightlist
\item
  archived recording\\
  {[}COUGHING{]}
\end{itemize}

peter staley

And I'm sitting here watching this thing, and I've got a bad cough.

\begin{itemize}
\item
  archived recording (speaker 1)\\
  Why don't you go home?
\item
  archived recording (speaker 2)\\
  I was hoping you'd say that. How about you?
\item
  archived recording (speaker 1)\\
  {[}COUGHING{]} I want to check these sites first.
\end{itemize}

peter staley

And Aidan Quinn is coughing away, acting like he has P.C.P. pneumonia.

\begin{itemize}
\tightlist
\item
  archived recording\\
  {[}COUGHING{]} {[}BODY FALLING TO FLOOR{]} Michael!
\end{itemize}

peter staley

And during one of the commercial breaks, I'm hacking away. And this
boyfriend leans over and says, you sound just like him.

\begin{itemize}
\tightlist
\item
  archived recording\\
  {[}GROANING{]} {[}DISTANT COUGHING{]}
\end{itemize}

peter staley

Fortunately, I had a gay doctor in New York City, a guy named Dan
William. At that point, Dan was so paranoid about the health of all his
patients that if you went in there with anything, he would run a C.B.C.,
a complete blood count, just a regular blood test.

48 hours after that appointment, I got a call from his nurse while I was
at my trading desk. I remember it very clearly. It was a Friday morning.
And he said, there's an abnormality on your test, and Dan would like you
to come in for some more blood work. I pressed him, and he said, it
might be indication of AIDS.

Right then and there, I just knew that that's what it was.

I felt like I'd been handed a death sentence. That's kind of the classic
story.

I sat down on my bed and started crying because I realized how screwed I
was. So at that point, it became a game of where can I find treatments
to buy me a year, some months, a few years, buy me some time.

michael barbaro

And what was the answer? Where could you and did you go for treatments
at this phase?

peter staley

There was nothing. There were no treatments approved at that point. But
the first thing I did after my diagnosis was go home and tell my family.
And they rallied to my side. It was just wonderful. So I had this
initial support group. But I still felt very alone. I hadn't met another
person with H.I.V. So I finally got myself to an H.I.V. support group at
G.M.H.C.

michael barbaro

Gay Men's Health Crisis.

peter staley

Right. And I actually didn't like what I was hearing.

michael barbaro

Why not?

peter staley

They were living in the stigma. They were all saying they had stopped
having sex. They felt like they were walking around with a scarlet
letter. They were filled with fear about dying any day. They were in
full trauma mode rather than fight mode. But there was this one guy
across from me who spoke up. And {[}CHUCKLES{]} he was a character. He
was in leather. And he had kind of spiked hair and very goth. And he
said, well, I don't know what you're all talking about. I'm not going to
stop living. And my eyes widened. And it was like, yeah, this is the guy
I've got to get to know.

{[}music{]}

His name was Griffin Gold. And it was just such great luck, because he
was one of the leaders of the self-empowerment movement of people with
AIDS in the country.

\begin{itemize}
\tightlist
\item
  archived recording\\
  What's going to happen to stop this epidemic? When is this government
  gonna start to care about the people who are dying?
\end{itemize}

peter staley

And he started tying me into this early activism.

\begin{itemize}
\tightlist
\item
  archived recording\\
  Act up! Fight back! Fight AIDS!
\end{itemize}

peter staley

And then all of us got lucky six months after that when Act Up was born.

\begin{itemize}
\tightlist
\item
  archived recording (larry kramer)\\
  There's a new AIDS death every half hour. There's a new HIV infection
  every single minute.
\end{itemize}

peter staley

Act Up was really in response to a call by Larry Kramer ---

\begin{itemize}
\tightlist
\item
  archived recording (larry kramer)\\
  I think the groundswell of new chapters, of new members, is very
  similar to what happened during the Vietnam War, where people got so
  angry at the government that they were forced into this frustration.
\end{itemize}

peter staley

--- saying that people weren't aware enough and weren't fighting back.
And he wanted a political response.

\begin{itemize}
\tightlist
\item
  archived recording (larry kramer)\\
  We are in the middle of a plague!
\end{itemize}

peter staley

And I had this crazy first year with Act Up where I was still working on
Wall Street. So I had to keep my activism kind of in the closet, as it
were. And then after about a year of that, the game was kind of up. My
CD4 count crashed below 200. In my head I knew from the science that
that usually would start kind of this clock. That people who had less
than 200 CD4 cells were dying of AIDS in about two years. So I felt a
clock had started, and I only had two years left. But I couldn't keep
trying to trade bonds on Wall Street. So the very next morning after
that CD4 count, I walked into my boss's office and told him everything,
and said, I'm walking out of here now, and this is my last day.

michael barbaro

And so what do you do next?

peter staley

I go full in on Act Up.

\begin{itemize}
\tightlist
\item
  archived recording\\
  Gay people and straight people, black people and white people, men and
  women will hear the story that once there was a terrible disease. And
  that a brave group of people stood up and fought and, in some cases,
  died so that others might live and be free.
\end{itemize}

peter staley

It became my church. It became my family. It became my social life. And
it started keeping me alive.

\begin{itemize}
\item
  archived recording 1\\
  And after we kicked the shit out of this disease, I intend to be alive
  to kick the shit out of the system so that this will never happen
  again. Thank you. {[}CHEERING{]}
\item
  archived recording 2\\
  You are all under arrest for disorderly conduct. {[}CROWD CHANTING{]}
\end{itemize}

peter staley

Two weeks after I left Wall Street ---

\begin{itemize}
\tightlist
\item
  archived recording\\
  --- additional charge of resisting arrest will be ---
\end{itemize}

peter staley

--- I got arrested on Wall Street ---

michael barbaro

Wow.

peter staley

--- for Act Up's one-year anniversary.

\begin{itemize}
\tightlist
\item
  archived recording\\
  (CHANTING) Fight back! Fight AIDS! Act up!
\end{itemize}

michael barbaro

What was the next big public action that you took with Act Up?

peter staley

The F.D.A., the Food and Drug Administration.

\begin{itemize}
\tightlist
\item
  archived recording\\
  (CHANTING) --- from AIDS. Where was the F.D.A.? 42,000 deaths from
  AIDS. Where was the F.D.A.?
\end{itemize}

peter staley

October 11, 1988 ---

\begin{itemize}
\tightlist
\item
  archived recording\\
  (CHANTING) No more business as usual! No more business as usual!
\end{itemize}

peter staley

It lasted all day, there were over 1,000 demonstrators from ACT Up
chapters in Rockville, Maryland surrounding the Food and Drug
Administration.

\begin{itemize}
\tightlist
\item
  archived recording\\
  We're here today. We're old. We're young. We're gay. We're straight.
  We're here today because we want to make a difference. We're here
  today because we care.
\end{itemize}

peter staley

And it was surrounded by hundreds of police officers.

\begin{itemize}
\item
  archived recording 1\\
  I see some of your men are wearing plastic gloves. Did the department
  provide them?
\item
  archived recording 2\\
  Yes.
\item
  archived recording 1\\
  Why?
\item
  archived recording 2\\
  That's up to the individual officers to whether he wants to wear them
  or does not want to wear them --- whatever made him feel comfortable.
\end{itemize}

peter staley

And I wanted to do something that would get my fellow activists riled
up.

\begin{itemize}
\tightlist
\item
  archived recording\\
  {[}CROWD CHEERING{]}
\end{itemize}

peter staley

I had noticed this overhang over the front door. And I was trying to
figure out how I could get past 30 cops to get up onto that. And so me
and two friends just walked towards them, like we wanted to negotiate or
talk with them. And we came within about six feet of the front line,
which was just below the outer edge of the overhang. And just when we
got there, my friends clasped their hands together. And I put my right
foot in, and I was launched --- just as the cops sprang forward, trying
to catch me.

\begin{itemize}
\tightlist
\item
  archived recording\\
  {[}CROWD CHEERING{]}
\end{itemize}

peter staley

And I started hanging this huge banner.

\begin{itemize}
\tightlist
\item
  archived recording\\
  Silence equals death! Silence equals death!
\end{itemize}

peter staley

It said ``silence equals death'' over the front entrance of the F.D.A.
And all the activists converged at the front door and started cheering
and hooting and hollering. It was very symbolic of how we were seizing
the building that day.

{[}music{]}

peter staley

For many Americans, it was the first time they had seen large numbers of
gay men and lesbians taking to the streets and determined and angry and
unified. I think we destroyed the American myth that the homosexual is
weak and timid and when threatened, will cower in a corner.

michael barbaro

And what was your message to the F.D.A. in that moment when you are
scaling the building and unfurling this sign that says silence equals
death?

peter staley

We wanted them to create an entirely different regulatory scheme for how
they analyzed drugs for life-threatening illnesses, like AIDS, versus
other drugs. We wanted them to work proactively with pharmaceutical
companies to speed the development of drugs, to make sure that their
clinical trials would enroll quickly and get data quickly. We wanted
them to approve drugs as soon as possible. And we wanted them to open up
a mechanism whereby people who weren't in the clinical trials could
still access some experimental therapies before they were approved. And
to our amazement, within about a year of that action, we got almost all
of that out of the F.D.A. We may not have convinced Americans to love us
or to tolerate us. But Americans do not like to hear that their own
government is letting thousands of their own citizens die.

{[}music{]}

\begin{itemize}
\tightlist
\item
  archived recording\\
  When the living can no longer speak, the dead may speak for them.
\end{itemize}

peter staley

So by `91 and `92, we're an incredibly busy movement with almost weekly
demonstrations. But every week we start our Monday night meetings with
an announcement of who had died that week.

\begin{itemize}
\tightlist
\item
  archived recording\\
  I'd like to read something that Mark wrote and wants us to do today,
  very furiously.
\end{itemize}

peter staley

And all of us are going to memorials every month.

\begin{itemize}
\tightlist
\item
  archived recording\\
  Several months before his death, Mark said, I want my funeral to be
  fierce and defiant.
\end{itemize}

peter staley

So we were winning all these battles, but we were losing the war. The
death count just kept rising and rising.

\begin{itemize}
\tightlist
\item
  archived recording\\
  Let the whole earth hear us now. We beg, we pray, we demand that this
  epidemic end!
\end{itemize}

{[}music{]}

peter staley

We just gave up on the idea that we were going to find a cure. The virus
was just too complex. But we were hoping to shake out enough drugs that
we could find a magic combination that would add up to real time added
to people's lives. That was our new hope in the early `90s. And we kept
fighting for that and fighting for that. And finally --- finally, we got
lucky.

\begin{itemize}
\tightlist
\item
  archived recording (david ho)\\
  We came to the conclusion that it's inevitable for H.I.V. to develop
  drug resistance if you give it one drug at a time. However, if you
  start to combine the drugs and try to force the virus into a corner
  using multiple drugs, it is exceedingly difficult for H.I.V. to become
  ---
\end{itemize}

peter staley

In 1996, scientists announced that if you use three drugs at the same
time --- that was the magic number --- you could stop the replication of
H.I.V. in a patient.

\begin{itemize}
\tightlist
\item
  archived recording (david ho)\\
  By spring of 1996, we had three different trials using a protease
  inhibitor, plus two other drugs.
\end{itemize}

peter staley

And Dr. David Ho, who is very involved with this discovery and presented
the data ---

\begin{itemize}
\tightlist
\item
  archived recording (david ho)\\
  We knew that unlike previous experience, the virus was coming under
  control and staying there.
\end{itemize}

peter staley

--- he started speculating that if patients stayed on these regimens,
that maybe the immune system would knock out the rest of the virus on
its own over time.

\begin{itemize}
\tightlist
\item
  archived recording (david ho)\\
  And so we were pretty excited.
\end{itemize}

peter staley

These current regimens might lead to what they called eradication, i.e.
cure. And we're talking to ourselves, and we're saying, what if this is
real? And we just couldn't believe it after all this time. We all
started on these triple-drug regimens. And within three months, we were
all undetectable. Then it was like, wow, this is it. This is huge.

michael barbaro

That's amazing.

peter staley

Yeah. The death rate in the U.S. over the next two years dropped by 75
percent, just a huge shift in the pandemic.

michael barbaro

But did it mean that H.I.V. was really being eradicated from your
system?

peter staley

No. No, but it took us a couple of years to figure that out. And they
started learning more and more about how even wilier this virus is and
how it is able to protect itself and hide in your body in all sorts of
cellular compartments. It hides in cells that line your G.I. tract. It
hides in your brain. It hides in macrophage cells. The basic science
that started following from this was endlessly depressing. And within, I
would say, five years, the talk of cure had just vanished. So by the
time we get to 2007, there is no cure research going on. Everybody's
kind of resigned to the fact that the virus is too wily and has too many
hiding places in the body to get it out. And then all of a sudden, this
news breaks ---

\begin{itemize}
\tightlist
\item
  archived recording\\
  He is the only person ever to be cured of H.I.V. and AIDS. His cure
  was somewhat of an accident.
\end{itemize}

peter staley

--- that somebody has been cured of AIDS. And it was like, what?
{[}LAUGHS{]}

\begin{itemize}
\tightlist
\item
  archived recording\\
  The only man believed to have been cured of H.I.V. spoke today at the
  International AIDS Conference in Washington. His name is Timothy
  Brown. He is also known as the Berlin patient.
\end{itemize}

peter staley

They had this fellow called the Berlin patient ---

\begin{itemize}
\tightlist
\item
  archived recording\\
  Doctors in Germany used an experimental radiation treatment to wipe
  out Brown's immune system. They gave him two bone-marrow transplants.
  And the results were remarkable.
\end{itemize}

peter staley

--- who also was suffering from very advanced cancer.

\begin{itemize}
\item
  archived recording 1\\
  Doctors in Germany wiped out his immune system with chemotherapy and
  radiation and then gave him two rounds of stem cell transplants.
\item
  archived recording 2\\
  His bone marrow donor had a rare gene mutation that made the donor and
  now Brown's stem cells resistant to H.I.V.
\end{itemize}

peter staley

This new immune system grew out of this bone marrow, and it worked. If
there was any H.I.V. hiding anywhere in his body, it had nothing to
latch onto in order to replicate.

michael barbaro

And so it died.

peter staley

And so it died.

michael barbaro

So the idea is that this person who was immune to H.I.V., that if you
take their bone marrow and introduce it into the system of an
H.I.V.-positive person, the Berlin patient ---

peter staley

Right.

michael barbaro

--- and the Berlin patient's immune system starts to develop around that
bone marrow that's immune to H.I.V, that the entire patient becomes
immune to H.I.V.?

peter staley

Exactly. And so that just cracked open everybody's imaginations. And
cure research was back on the agenda. A whole field of research was
born, trying to figure out how we could do this without wiping out a
person's immune system first, without a bone-marrow transplant.

michael barbaro

Right, because it would be almost impossible to replicate this on a big
scale.

peter staley

Exactly. And there were other attempts to replicate what happened with
Timothy Brown and the Berlin patient. But unfortunately, they failed.
They tried this in other people with H.I.V. that had cancer that needed
bone-marrow transplants. There was a period of time where it looked like
they had undetectable H.I.V. after the transplant, but then it bounced
back. And so there were some disappointments.

{[}music{]}

peter staley

And so what happened two weeks ago ---

\begin{itemize}
\tightlist
\item
  archived recording\\
  Researchers say the latest success confirms that a cure for H.I.V.
  infection is possible.
\end{itemize}

peter staley

--- was proof that what happened in the Berlin patient was not a fluke.

\begin{itemize}
\tightlist
\item
  archived recording\\
  It comes 12 years after American Timothy Brown, known as the Berlin
  patient, became the first known adult to be cured. Both patients
  underwent stem cell transplants from donors who carried a rare genetic
  mutation that made them resistant to H.I.V. Doctors say this second
  success ---
\end{itemize}

peter staley

We're not rushing to the hospital to get bone-marrow transplants. But I
was thrilled to hear this news mostly because I think it's going to
probably get us quicker to that day when we ultimately find a gene
therapy that does the same thing.

michael barbaro

Peter, your entire adult life has been focused on advancing treatment
for H.I.V. and AIDS. And it's remarkable how much advancement has
occurred and that you've witnessed. If you live long enough to
experience a cure, and, of course, we all hope you do, what will you do
then?

peter staley

Well, since 2008, I really do love the idea that I'm probably going to
be there when it happens, that I'm going to witness that.

\begin{itemize}
\item
  archived recording (speaker 1)\\
  I just want to be there if they ever do find a cure.
\item
  archived recording (speaker 2)\\
  Can you imagine what it would be like?
\end{itemize}

peter staley

It's going to be like that last scene in ``Longtime Companion,'' that
great AIDS film, where ---

\begin{itemize}
\tightlist
\item
  archived recording\\
  {[}CHEERING CROWD{]} {[}LAUGHTER{]}
\end{itemize}

peter staley

--- the dream sequence at the end of the movie, where all the people who
have died of AIDS come back and celebrate on a sunny day on the beach in
Fire Island. And everybody's partying. It's like a big gay disco. But
it's everybody we've lost, and they're just celebrating.

\begin{itemize}
\item
  archived recording\\
  Willy! Hey, Will, it's me. {[}GUITAR MUSIC{]}

  {[}LAUGHTER{]}
\end{itemize}

peter staley

It might be like that.

{[}music{]}

peter staley

Now I'm going to be an ex-AIDS activist. {[}LAUGHS{]} And I'm going to
do something else and be glad for it and be proud.

michael barbaro

I hope that you will be an ex-AIDS activist, Peter. And I look forward
to chatting with you when that day comes.

peter staley

Yeah, me too. {[}CHUCKLES{]}

michael barbaro

Thank you very much. We appreciate it.

peter staley

You're welcome.

{[}music{]}

michael barbaro

Peter Staley is now directing his activism toward greater awareness and
access to a drug, Truvada, that is 99 percent effective in preventing
H.I.V. infections but remains unaffordable to large numbers of people in
the U.S. and the world, where H.I.V. infections are on the rise.

\href{https://www.nytimes.com/column/the-daily}{\includegraphics{https://static01.nyt.com/images/2017/01/29/podcasts/the-daily-album-art/the-daily-album-art-square320-v4.png}The
Daily}Subscribe:

\begin{itemize}
\tightlist
\item
  \href{https://itunes.apple.com/us/podcast/id1200361736}{Apple
  Podcasts}
\item
  \href{https://www.google.com/podcasts?feed=aHR0cHM6Ly9yc3MuYXJ0MTkuY29tL3RoZS1kYWlseQ\%3D\%3D}{Google
  Podcasts}
\end{itemize}

\hypertarget{a-path-to-curing-hiv-1}{%
\section{A Path to Curing H.I.V.}\label{a-path-to-curing-hiv-1}}

\hypertarget{a-second-person-appears-to-have-been-cured-of-infection-with-the-virus-that-causes-aids-we-spoke-to-an-activist-about-what-this-means-1}{%
\subsection{A second person appears to have been cured of infection with
the virus that causes AIDS. We spoke to an activist about what this
means.}\label{a-second-person-appears-to-have-been-cured-of-infection-with-the-virus-that-causes-aids-we-spoke-to-an-activist-about-what-this-means-1}}

Hosted by Michael Barbaro, produced by Andy Mills and Jonathan Wolfe,
and edited by Paige Cowett, Wendy Dorr and Larissa Anderson

Transcript

transcript

Back to The Daily

bars

0:00/28:05

-0:00

transcript

\hypertarget{a-path-to-curing-hiv-2}{%
\subsection{A Path to Curing H.I.V.}\label{a-path-to-curing-hiv-2}}

\hypertarget{hosted-by-michael-barbaro-produced-by-andy-mills-and-jonathan-wolfe-and-edited-by-paige-cowett-wendy-dorr-and-larissa-anderson-1}{%
\subsubsection{Hosted by Michael Barbaro, produced by Andy Mills and
Jonathan Wolfe, and edited by Paige Cowett, Wendy Dorr and Larissa
Anderson}\label{hosted-by-michael-barbaro-produced-by-andy-mills-and-jonathan-wolfe-and-edited-by-paige-cowett-wendy-dorr-and-larissa-anderson-1}}

\hypertarget{a-second-person-appears-to-have-been-cured-of-infection-with-the-virus-that-causes-aids-we-spoke-to-an-activist-about-what-this-means-2}{%
\paragraph{A second person appears to have been cured of infection with
the virus that causes AIDS. We spoke to an activist about what this
means.}\label{a-second-person-appears-to-have-been-cured-of-infection-with-the-virus-that-causes-aids-we-spoke-to-an-activist-about-what-this-means-2}}

Thursday, March 21st, 2019

\begin{itemize}
\item
  michael barbaro\\
  From The New York Times, I'm Michael Barbaro. This is ``The Daily.''

  Earlier this month, a second person was cured of H.I.V., something
  that, after three decades of fighting for a cure, scientists and
  activists had almost given up on. How we got here.

  It's Thursday, March 21.

  So Peter Staley, when did you first hear about H.I.V. and AIDS? Where
  were you? And what was going on in your life?
\item
  peter staley\\
  I think I first heard about AIDS probably a year into the crisis, in
  1982.
\item
  archived recording\\
  It's mysterious. It's deadly. And it's baffling medical science.
\end{itemize}

peter staley

Like, the first mainstream television report.

\begin{itemize}
\item
  archived recording 1\\
  The lifestyle of some male homosexuals has triggered an epidemic.
\item
  archived recording 2\\
  Acquired Immune Deficiency Syndrome, or the acronym by which it's
  frequently identified, AIDS.
\end{itemize}

peter staley

And I was still in college at Oberlin.

\begin{itemize}
\tightlist
\item
  archived recording\\
  Researchers are now studying blood and other samples from the victims,
  trying to learn what is causing the disease. So far they have had no
  luck.
\end{itemize}

peter staley

It seemed like this strange mystery. I'm not sure how applicable it felt
to my life.

\begin{itemize}
\item
  archived recording 1\\
  Investigators have examined the habits of homosexuals for clues.
\item
  archived recording 2\\
  I was in the fast lane at one time in terms of the way that I lived my
  life. And now I'm not.
\end{itemize}

peter staley

I was a closeted homosexual, certainly wasn't hearing about it on campus
at all.

And then I came to New York in about April of `83 for this new job on
Wall Street and had my first conversations with other gay men. And at
least in my generation, the very young twenty-somethings, often
closeted, it still seemed remote. We were telling each other that it was
mostly happening in older gay men and those who had been highly, highly
prolific in their sexuality.

michael barbaro

Mm-hmm.

peter staley

So fast forward a couple of years ---

\begin{itemize}
\tightlist
\item
  archived recording\\
  To begin with, AIDS was just an obscure medical curiosity ---
\end{itemize}

peter staley

--- I'm now a bond trader on Wall Street.

\begin{itemize}
\item
  archived recording 1\\
  --- a strange illness affecting a handful of homosexuals and drug
  abusers in American cities. Today, the virus has claimed thousands of
  lives and threatens millions more.
\item
  archived recording 2\\
  When the AIDS alarm sounded four years ago, most people knew little or
  nothing about the disease, much less knew anyone who had it. Now, as
  the number of cases doubles every six months, that profile is
  changing.
\end{itemize}

peter staley

And AIDS has finally become a massive story in the country.

\begin{itemize}
\tightlist
\item
  archived recording\\
  Some might be led to believe that the disease is leveling off. It is
  not.
\end{itemize}

peter staley

And not in a good way.

\begin{itemize}
\item
  archived recording 1\\
  Many researchers would argue that the number of cases is actually 10
  times higher than those reported by the Centers for Disease Control.
\item
  archived recording 2\\
  You know, they should just come out and tell people.
\end{itemize}

peter staley

The country was in a panic.

\begin{itemize}
\tightlist
\item
  archived recording\\
  You know, because everybody's scared.
\end{itemize}

peter staley

There were cover stories on all the national magazines about AIDS.
Parents were pulling their kids out of schools if there were rumors
about a child having H.I.V.

\begin{itemize}
\item
  archived recording 1\\
  These people are raising money and signing petitions in a fight to
  keep Ryan White out of their school.
\item
  archived recording 2\\
  If he has a spill on a table, a chair, your daughter comes in, touches
  it, what's to say she is not going to get it?
\end{itemize}

peter staley

And I sat down with my kind of new-ish boyfriend at the time ---

\begin{itemize}
\item
  archived recording (speaker 1)\\
  Breakfast is served.
\item
  archived recording (speaker 2)\\
  Thanks. We're almost out of shaving cream.
\item
  archived recording (speaker 1)\\
  Oh, O.K.
\end{itemize}

peter staley

--- to watch the very first television movie about the subject, which
was called ``An Early Frost.''

\begin{itemize}
\tightlist
\item
  archived recording\\
  So how'd it go with your folks?
\end{itemize}

peter staley

And Aidan Quinn was the star.

\begin{itemize}
\tightlist
\item
  archived recording\\
  You didn't tell them, did you?
\end{itemize}

peter staley

He played a closeted gay man ---

\begin{itemize}
\item
  archived recording (speaker 1)\\
  Look. I don't have the same relationship with my parents that you have
  with yours. O.K.? I don't talk about sex with them. They don't talk
  about sex with me.
\item
  archived recording (speaker 2)\\
  Who's talking about sex? We're talking about us.
\end{itemize}

peter staley

--- who gets hit by AIDS with P.C.P. pneumonia, which was the common
killer back then.

\begin{itemize}
\tightlist
\item
  archived recording\\
  {[}COUGHING{]}
\end{itemize}

peter staley

And I'm sitting here watching this thing, and I've got a bad cough.

\begin{itemize}
\item
  archived recording (speaker 1)\\
  Why don't you go home?
\item
  archived recording (speaker 2)\\
  I was hoping you'd say that. How about you?
\item
  archived recording (speaker 1)\\
  {[}COUGHING{]} I want to check these sites first.
\end{itemize}

peter staley

And Aidan Quinn is coughing away, acting like he has P.C.P. pneumonia.

\begin{itemize}
\tightlist
\item
  archived recording\\
  {[}COUGHING{]} {[}BODY FALLING TO FLOOR{]} Michael!
\end{itemize}

peter staley

And during one of the commercial breaks, I'm hacking away. And this
boyfriend leans over and says, you sound just like him.

\begin{itemize}
\tightlist
\item
  archived recording\\
  {[}GROANING{]} {[}DISTANT COUGHING{]}
\end{itemize}

peter staley

Fortunately, I had a gay doctor in New York City, a guy named Dan
William. At that point, Dan was so paranoid about the health of all his
patients that if you went in there with anything, he would run a C.B.C.,
a complete blood count, just a regular blood test.

48 hours after that appointment, I got a call from his nurse while I was
at my trading desk. I remember it very clearly. It was a Friday morning.
And he said, there's an abnormality on your test, and Dan would like you
to come in for some more blood work. I pressed him, and he said, it
might be indication of AIDS.

Right then and there, I just knew that that's what it was.

I felt like I'd been handed a death sentence. That's kind of the classic
story.

I sat down on my bed and started crying because I realized how screwed I
was. So at that point, it became a game of where can I find treatments
to buy me a year, some months, a few years, buy me some time.

michael barbaro

And what was the answer? Where could you and did you go for treatments
at this phase?

peter staley

There was nothing. There were no treatments approved at that point. But
the first thing I did after my diagnosis was go home and tell my family.
And they rallied to my side. It was just wonderful. So I had this
initial support group. But I still felt very alone. I hadn't met another
person with H.I.V. So I finally got myself to an H.I.V. support group at
G.M.H.C.

michael barbaro

Gay Men's Health Crisis.

peter staley

Right. And I actually didn't like what I was hearing.

michael barbaro

Why not?

peter staley

They were living in the stigma. They were all saying they had stopped
having sex. They felt like they were walking around with a scarlet
letter. They were filled with fear about dying any day. They were in
full trauma mode rather than fight mode. But there was this one guy
across from me who spoke up. And {[}CHUCKLES{]} he was a character. He
was in leather. And he had kind of spiked hair and very goth. And he
said, well, I don't know what you're all talking about. I'm not going to
stop living. And my eyes widened. And it was like, yeah, this is the guy
I've got to get to know.

{[}music{]}

His name was Griffin Gold. And it was just such great luck, because he
was one of the leaders of the self-empowerment movement of people with
AIDS in the country.

\begin{itemize}
\tightlist
\item
  archived recording\\
  What's going to happen to stop this epidemic? When is this government
  gonna start to care about the people who are dying?
\end{itemize}

peter staley

And he started tying me into this early activism.

\begin{itemize}
\tightlist
\item
  archived recording\\
  Act up! Fight back! Fight AIDS!
\end{itemize}

peter staley

And then all of us got lucky six months after that when Act Up was born.

\begin{itemize}
\tightlist
\item
  archived recording (larry kramer)\\
  There's a new AIDS death every half hour. There's a new HIV infection
  every single minute.
\end{itemize}

peter staley

Act Up was really in response to a call by Larry Kramer ---

\begin{itemize}
\tightlist
\item
  archived recording (larry kramer)\\
  I think the groundswell of new chapters, of new members, is very
  similar to what happened during the Vietnam War, where people got so
  angry at the government that they were forced into this frustration.
\end{itemize}

peter staley

--- saying that people weren't aware enough and weren't fighting back.
And he wanted a political response.

\begin{itemize}
\tightlist
\item
  archived recording (larry kramer)\\
  We are in the middle of a plague!
\end{itemize}

peter staley

And I had this crazy first year with Act Up where I was still working on
Wall Street. So I had to keep my activism kind of in the closet, as it
were. And then after about a year of that, the game was kind of up. My
CD4 count crashed below 200. In my head I knew from the science that
that usually would start kind of this clock. That people who had less
than 200 CD4 cells were dying of AIDS in about two years. So I felt a
clock had started, and I only had two years left. But I couldn't keep
trying to trade bonds on Wall Street. So the very next morning after
that CD4 count, I walked into my boss's office and told him everything,
and said, I'm walking out of here now, and this is my last day.

michael barbaro

And so what do you do next?

peter staley

I go full in on Act Up.

\begin{itemize}
\tightlist
\item
  archived recording\\
  Gay people and straight people, black people and white people, men and
  women will hear the story that once there was a terrible disease. And
  that a brave group of people stood up and fought and, in some cases,
  died so that others might live and be free.
\end{itemize}

peter staley

It became my church. It became my family. It became my social life. And
it started keeping me alive.

\begin{itemize}
\item
  archived recording 1\\
  And after we kicked the shit out of this disease, I intend to be alive
  to kick the shit out of the system so that this will never happen
  again. Thank you. {[}CHEERING{]}
\item
  archived recording 2\\
  You are all under arrest for disorderly conduct. {[}CROWD CHANTING{]}
\end{itemize}

peter staley

Two weeks after I left Wall Street ---

\begin{itemize}
\tightlist
\item
  archived recording\\
  --- additional charge of resisting arrest will be ---
\end{itemize}

peter staley

--- I got arrested on Wall Street ---

michael barbaro

Wow.

peter staley

--- for Act Up's one-year anniversary.

\begin{itemize}
\tightlist
\item
  archived recording\\
  (CHANTING) Fight back! Fight AIDS! Act up!
\end{itemize}

michael barbaro

What was the next big public action that you took with Act Up?

peter staley

The F.D.A., the Food and Drug Administration.

\begin{itemize}
\tightlist
\item
  archived recording\\
  (CHANTING) --- from AIDS. Where was the F.D.A.? 42,000 deaths from
  AIDS. Where was the F.D.A.?
\end{itemize}

peter staley

October 11, 1988 ---

\begin{itemize}
\tightlist
\item
  archived recording\\
  (CHANTING) No more business as usual! No more business as usual!
\end{itemize}

peter staley

It lasted all day, there were over 1,000 demonstrators from ACT Up
chapters in Rockville, Maryland surrounding the Food and Drug
Administration.

\begin{itemize}
\tightlist
\item
  archived recording\\
  We're here today. We're old. We're young. We're gay. We're straight.
  We're here today because we want to make a difference. We're here
  today because we care.
\end{itemize}

peter staley

And it was surrounded by hundreds of police officers.

\begin{itemize}
\item
  archived recording 1\\
  I see some of your men are wearing plastic gloves. Did the department
  provide them?
\item
  archived recording 2\\
  Yes.
\item
  archived recording 1\\
  Why?
\item
  archived recording 2\\
  That's up to the individual officers to whether he wants to wear them
  or does not want to wear them --- whatever made him feel comfortable.
\end{itemize}

peter staley

And I wanted to do something that would get my fellow activists riled
up.

\begin{itemize}
\tightlist
\item
  archived recording\\
  {[}CROWD CHEERING{]}
\end{itemize}

peter staley

I had noticed this overhang over the front door. And I was trying to
figure out how I could get past 30 cops to get up onto that. And so me
and two friends just walked towards them, like we wanted to negotiate or
talk with them. And we came within about six feet of the front line,
which was just below the outer edge of the overhang. And just when we
got there, my friends clasped their hands together. And I put my right
foot in, and I was launched --- just as the cops sprang forward, trying
to catch me.

\begin{itemize}
\tightlist
\item
  archived recording\\
  {[}CROWD CHEERING{]}
\end{itemize}

peter staley

And I started hanging this huge banner.

\begin{itemize}
\tightlist
\item
  archived recording\\
  Silence equals death! Silence equals death!
\end{itemize}

peter staley

It said ``silence equals death'' over the front entrance of the F.D.A.
And all the activists converged at the front door and started cheering
and hooting and hollering. It was very symbolic of how we were seizing
the building that day.

{[}music{]}

peter staley

For many Americans, it was the first time they had seen large numbers of
gay men and lesbians taking to the streets and determined and angry and
unified. I think we destroyed the American myth that the homosexual is
weak and timid and when threatened, will cower in a corner.

michael barbaro

And what was your message to the F.D.A. in that moment when you are
scaling the building and unfurling this sign that says silence equals
death?

peter staley

We wanted them to create an entirely different regulatory scheme for how
they analyzed drugs for life-threatening illnesses, like AIDS, versus
other drugs. We wanted them to work proactively with pharmaceutical
companies to speed the development of drugs, to make sure that their
clinical trials would enroll quickly and get data quickly. We wanted
them to approve drugs as soon as possible. And we wanted them to open up
a mechanism whereby people who weren't in the clinical trials could
still access some experimental therapies before they were approved. And
to our amazement, within about a year of that action, we got almost all
of that out of the F.D.A. We may not have convinced Americans to love us
or to tolerate us. But Americans do not like to hear that their own
government is letting thousands of their own citizens die.

{[}music{]}

\begin{itemize}
\tightlist
\item
  archived recording\\
  When the living can no longer speak, the dead may speak for them.
\end{itemize}

peter staley

So by `91 and `92, we're an incredibly busy movement with almost weekly
demonstrations. But every week we start our Monday night meetings with
an announcement of who had died that week.

\begin{itemize}
\tightlist
\item
  archived recording\\
  I'd like to read something that Mark wrote and wants us to do today,
  very furiously.
\end{itemize}

peter staley

And all of us are going to memorials every month.

\begin{itemize}
\tightlist
\item
  archived recording\\
  Several months before his death, Mark said, I want my funeral to be
  fierce and defiant.
\end{itemize}

peter staley

So we were winning all these battles, but we were losing the war. The
death count just kept rising and rising.

\begin{itemize}
\tightlist
\item
  archived recording\\
  Let the whole earth hear us now. We beg, we pray, we demand that this
  epidemic end!
\end{itemize}

{[}music{]}

peter staley

We just gave up on the idea that we were going to find a cure. The virus
was just too complex. But we were hoping to shake out enough drugs that
we could find a magic combination that would add up to real time added
to people's lives. That was our new hope in the early `90s. And we kept
fighting for that and fighting for that. And finally --- finally, we got
lucky.

\begin{itemize}
\tightlist
\item
  archived recording (david ho)\\
  We came to the conclusion that it's inevitable for H.I.V. to develop
  drug resistance if you give it one drug at a time. However, if you
  start to combine the drugs and try to force the virus into a corner
  using multiple drugs, it is exceedingly difficult for H.I.V. to become
  ---
\end{itemize}

peter staley

In 1996, scientists announced that if you use three drugs at the same
time --- that was the magic number --- you could stop the replication of
H.I.V. in a patient.

\begin{itemize}
\tightlist
\item
  archived recording (david ho)\\
  By spring of 1996, we had three different trials using a protease
  inhibitor, plus two other drugs.
\end{itemize}

peter staley

And Dr. David Ho, who is very involved with this discovery and presented
the data ---

\begin{itemize}
\tightlist
\item
  archived recording (david ho)\\
  We knew that unlike previous experience, the virus was coming under
  control and staying there.
\end{itemize}

peter staley

--- he started speculating that if patients stayed on these regimens,
that maybe the immune system would knock out the rest of the virus on
its own over time.

\begin{itemize}
\tightlist
\item
  archived recording (david ho)\\
  And so we were pretty excited.
\end{itemize}

peter staley

These current regimens might lead to what they called eradication, i.e.
cure. And we're talking to ourselves, and we're saying, what if this is
real? And we just couldn't believe it after all this time. We all
started on these triple-drug regimens. And within three months, we were
all undetectable. Then it was like, wow, this is it. This is huge.

michael barbaro

That's amazing.

peter staley

Yeah. The death rate in the U.S. over the next two years dropped by 75
percent, just a huge shift in the pandemic.

michael barbaro

But did it mean that H.I.V. was really being eradicated from your
system?

peter staley

No. No, but it took us a couple of years to figure that out. And they
started learning more and more about how even wilier this virus is and
how it is able to protect itself and hide in your body in all sorts of
cellular compartments. It hides in cells that line your G.I. tract. It
hides in your brain. It hides in macrophage cells. The basic science
that started following from this was endlessly depressing. And within, I
would say, five years, the talk of cure had just vanished. So by the
time we get to 2007, there is no cure research going on. Everybody's
kind of resigned to the fact that the virus is too wily and has too many
hiding places in the body to get it out. And then all of a sudden, this
news breaks ---

\begin{itemize}
\tightlist
\item
  archived recording\\
  He is the only person ever to be cured of H.I.V. and AIDS. His cure
  was somewhat of an accident.
\end{itemize}

peter staley

--- that somebody has been cured of AIDS. And it was like, what?
{[}LAUGHS{]}

\begin{itemize}
\tightlist
\item
  archived recording\\
  The only man believed to have been cured of H.I.V. spoke today at the
  International AIDS Conference in Washington. His name is Timothy
  Brown. He is also known as the Berlin patient.
\end{itemize}

peter staley

They had this fellow called the Berlin patient ---

\begin{itemize}
\tightlist
\item
  archived recording\\
  Doctors in Germany used an experimental radiation treatment to wipe
  out Brown's immune system. They gave him two bone-marrow transplants.
  And the results were remarkable.
\end{itemize}

peter staley

--- who also was suffering from very advanced cancer.

\begin{itemize}
\item
  archived recording 1\\
  Doctors in Germany wiped out his immune system with chemotherapy and
  radiation and then gave him two rounds of stem cell transplants.
\item
  archived recording 2\\
  His bone marrow donor had a rare gene mutation that made the donor and
  now Brown's stem cells resistant to H.I.V.
\end{itemize}

peter staley

This new immune system grew out of this bone marrow, and it worked. If
there was any H.I.V. hiding anywhere in his body, it had nothing to
latch onto in order to replicate.

michael barbaro

And so it died.

peter staley

And so it died.

michael barbaro

So the idea is that this person who was immune to H.I.V., that if you
take their bone marrow and introduce it into the system of an
H.I.V.-positive person, the Berlin patient ---

peter staley

Right.

michael barbaro

--- and the Berlin patient's immune system starts to develop around that
bone marrow that's immune to H.I.V, that the entire patient becomes
immune to H.I.V.?

peter staley

Exactly. And so that just cracked open everybody's imaginations. And
cure research was back on the agenda. A whole field of research was
born, trying to figure out how we could do this without wiping out a
person's immune system first, without a bone-marrow transplant.

michael barbaro

Right, because it would be almost impossible to replicate this on a big
scale.

peter staley

Exactly. And there were other attempts to replicate what happened with
Timothy Brown and the Berlin patient. But unfortunately, they failed.
They tried this in other people with H.I.V. that had cancer that needed
bone-marrow transplants. There was a period of time where it looked like
they had undetectable H.I.V. after the transplant, but then it bounced
back. And so there were some disappointments.

{[}music{]}

peter staley

And so what happened two weeks ago ---

\begin{itemize}
\tightlist
\item
  archived recording\\
  Researchers say the latest success confirms that a cure for H.I.V.
  infection is possible.
\end{itemize}

peter staley

--- was proof that what happened in the Berlin patient was not a fluke.

\begin{itemize}
\tightlist
\item
  archived recording\\
  It comes 12 years after American Timothy Brown, known as the Berlin
  patient, became the first known adult to be cured. Both patients
  underwent stem cell transplants from donors who carried a rare genetic
  mutation that made them resistant to H.I.V. Doctors say this second
  success ---
\end{itemize}

peter staley

We're not rushing to the hospital to get bone-marrow transplants. But I
was thrilled to hear this news mostly because I think it's going to
probably get us quicker to that day when we ultimately find a gene
therapy that does the same thing.

michael barbaro

Peter, your entire adult life has been focused on advancing treatment
for H.I.V. and AIDS. And it's remarkable how much advancement has
occurred and that you've witnessed. If you live long enough to
experience a cure, and, of course, we all hope you do, what will you do
then?

peter staley

Well, since 2008, I really do love the idea that I'm probably going to
be there when it happens, that I'm going to witness that.

\begin{itemize}
\item
  archived recording (speaker 1)\\
  I just want to be there if they ever do find a cure.
\item
  archived recording (speaker 2)\\
  Can you imagine what it would be like?
\end{itemize}

peter staley

It's going to be like that last scene in ``Longtime Companion,'' that
great AIDS film, where ---

\begin{itemize}
\tightlist
\item
  archived recording\\
  {[}CHEERING CROWD{]} {[}LAUGHTER{]}
\end{itemize}

peter staley

--- the dream sequence at the end of the movie, where all the people who
have died of AIDS come back and celebrate on a sunny day on the beach in
Fire Island. And everybody's partying. It's like a big gay disco. But
it's everybody we've lost, and they're just celebrating.

\begin{itemize}
\item
  archived recording\\
  Willy! Hey, Will, it's me. {[}GUITAR MUSIC{]}

  {[}LAUGHTER{]}
\end{itemize}

peter staley

It might be like that.

{[}music{]}

peter staley

Now I'm going to be an ex-AIDS activist. {[}LAUGHS{]} And I'm going to
do something else and be glad for it and be proud.

michael barbaro

I hope that you will be an ex-AIDS activist, Peter. And I look forward
to chatting with you when that day comes.

peter staley

Yeah, me too. {[}CHUCKLES{]}

michael barbaro

Thank you very much. We appreciate it.

peter staley

You're welcome.

{[}music{]}

michael barbaro

Peter Staley is now directing his activism toward greater awareness and
access to a drug, Truvada, that is 99 percent effective in preventing
H.I.V. infections but remains unaffordable to large numbers of people in
the U.S. and the world, where H.I.V. infections are on the rise.

Previous

More episodes ofThe Daily

\href{https://www.nytimes.com/2020/08/07/podcasts/the-daily/Jack-dorsey-twitter-trump.html?action=click\&module=audio-series-bar\&region=header\&pgtype=Article}{\includegraphics{https://static01.nyt.com/images/2020/08/10/podcasts/10daily-dorsey/merlin_154148610_97295061-f25c-4394-9501-38651a2d4029-thumbLarge.jpg}}

August 7, 2020Jack Dorsey on Twitter's Mistakes

\href{https://www.nytimes.com/2020/08/06/podcasts/the-daily/beirut-explosion-damage.html?action=click\&module=audio-series-bar\&region=header\&pgtype=Article}{\includegraphics{https://static01.nyt.com/images/2020/08/05/video/vi-beirut-promo/vi-beirut-promo-thumbLarge.jpg}}

August 6, 2020~~•~ 23:45The Day That Shook Beirut

\href{https://www.nytimes.com/2020/08/05/podcasts/the-daily/protests-racism-police-brutality.html?action=click\&module=audio-series-bar\&region=header\&pgtype=Article}{\includegraphics{https://static01.nyt.com/images/2020/07/30/us/05daily/portland-mapping-unrest-1596155031199-thumbLarge-v2.jpg}}

August 5, 2020~~•~ 43:46`Stay Black and Die'

\href{https://www.nytimes.com/2020/08/04/podcasts/the-daily/mail-in-voting-president-trump.html?action=click\&module=audio-series-bar\&region=header\&pgtype=Article}{\includegraphics{https://static01.nyt.com/images/2020/07/30/us/politics/04daily/30trump-election1-thumbLarge.jpg}}

August 4, 2020~~•~ 25:27Is the U.S. Ready to Vote by Mail?

\href{https://www.nytimes.com/2020/08/03/podcasts/the-daily/algorithmic-justice-racism.html?action=click\&module=audio-series-bar\&region=header\&pgtype=Article}{\includegraphics{https://static01.nyt.com/images/2020/06/24/business/03daily/24michigan-arrest1-thumbLarge.jpg}}

August 3, 2020~~•~ 28:13Wrongfully Accused by an Algorithm

\href{https://www.nytimes.com/2020/08/02/podcasts/the-daily/on-female-rage.html?action=click\&module=audio-series-bar\&region=header\&pgtype=Article}{\includegraphics{https://static01.nyt.com/images/2018/01/21/magazine/21mag-femaleanger1-copy/21mag-femaleanger1-thumbLarge.jpg}}

August 2, 2020The Sunday Read: `On Female Rage'

\href{https://www.nytimes.com/2020/07/31/podcasts/the-daily/vanessa-guillen-military-metoo.html?action=click\&module=audio-series-bar\&region=header\&pgtype=Article}{\includegraphics{https://static01.nyt.com/images/2020/07/12/us/politics/31daily/00dc-army-metoo-thumbLarge.jpg}}

July 31, 2020A \#MeToo Moment in the Military

\href{https://www.nytimes.com/2020/07/30/podcasts/the-daily/congress-facebook-amazon-google-apple.html?action=click\&module=audio-series-bar\&region=header\&pgtype=Article}{\includegraphics{https://static01.nyt.com/images/2020/07/30/reader-center/30daily/merlin_175077825_5ebc931b-baa1-489a-960c-34e4d845e997-thumbLarge.jpg}}

July 30, 2020~~•~ 35:19The Big Tech Hearing

\href{https://www.nytimes.com/2020/07/29/podcasts/the-daily/china-trump-foreign-policy.html?action=click\&module=audio-series-bar\&region=header\&pgtype=Article}{\includegraphics{https://static01.nyt.com/images/2020/07/26/world/29daily/00china-us-clash1-thumbLarge.jpg}}

July 29, 2020~~•~ 28:40Confronting China

\href{https://www.nytimes.com/2020/07/28/podcasts/the-daily/unemployment-benefits-coronavirus.html?action=click\&module=audio-series-bar\&region=header\&pgtype=Article}{\includegraphics{https://static01.nyt.com/images/2020/07/23/business/28daily/23virus-uiexplain1-thumbLarge.jpg}}

July 28, 2020~~•~ 26:13Why \$600 Checks Are Tearing Republicans Apart

\href{https://www.nytimes.com/2020/07/27/podcasts/the-daily/new-york-hospitals-covid.html?action=click\&module=audio-series-bar\&region=header\&pgtype=Article}{\includegraphics{https://static01.nyt.com/images/2020/07/27/world/27daily-hospitals/27daily-hospitals-thumbLarge.jpg}}

July 27, 2020~~•~ 33:28The Mistakes New York Made

\href{https://www.nytimes.com/2020/07/26/podcasts/the-daily/the-accusation-the-sunday-read.html?action=click\&module=audio-series-bar\&region=header\&pgtype=Article}{\includegraphics{https://static01.nyt.com/images/2020/03/22/magazine/26audm-2/22mag-titleix-thumbLarge.jpg}}

July 26, 2020The Sunday Read: `The Accusation'

\href{https://www.nytimes.com/column/the-daily}{See All Episodes ofThe
Daily}

Next

March 21, 2019

\begin{itemize}
\item
\item
\item
\item
\item
\item
\end{itemize}

\emph{\textbf{L}isten and subscribe to our podcast from your mobile
device:}**\\
\textbf{\href{https://itunes.apple.com/us/podcast/the-daily/id1200361736?mt=2}{\emph{Via
Apple Podcasts}}} \emph{\textbf{\textbar{}}}
\textbf{\href{https://play.radiopublic.com/88f7d8c3-7289-4dc6-b300-5ba71b43f5e5}{\emph{Via
RadioPublic}}} \emph{\textbf{\textbar{}}}
\textbf{\href{http://www.stitcher.com/podcast/the-new-york-times/the-daily-10}{\emph{Via
Stitcher}}}

For only the second time since the start of a global epidemic, a person
was reported this month to have been cured of H.I.V., the virus that
causes AIDS. Scientists and activists had almost given up on reaching
that milestone. Here's a look at how we got to this point.

\emph{{[}For an exclusive look at how the biggest stories on ``The
Daily'' podcast come together,}
\href{https://www.nytimes.com/newsletters/the-daily?module=inline}{\emph{subscribe
to our newsletter}}\emph{. Read the latest edition}
\href{https://www.nytimes.com/2019/03/15/podcasts/daily-newsletter-new-zealand-shooting-medicare-for-all.html?module=inline}{\emph{here}}\emph{.{]}}

\textbf{On today's episode:}

\begin{itemize}
\tightlist
\item
  Peter Staley, a longtime AIDS activist.
\end{itemize}

\includegraphics{https://static01.nyt.com/images/2019/03/21/lens/21daily-actup3/merlin_152364147_80bf5573-b315-4492-9c7b-47027460deb9-articleLarge.jpg?quality=75\&auto=webp\&disable=upscale}

\textbf{Background reading:}

\begin{itemize}
\item
  \href{https://www.nytimes.com/2019/03/04/health/aids-cure-london-patient.html}{This
  milestone comes 12 years after the first person known to be cured of
  H.I.V.} Like the first cure, it was the result of a bone-marrow
  transplant intended to treat cancer.
\item
  Translating this success into a practical treatment will take years,
  if it happens at all. Here are
  \href{https://www.nytimes.com/2019/03/05/health/hiv-aids-cure.html}{answers
  to key questions about the search for a cure}.
\end{itemize}

\includegraphics{https://static01.nyt.com/images/2019/03/05/science/05HIV2/05HIV2-videoSixteenByNine3000.jpg}

\emph{Tune in, and tell us what you think. Email us at}
\href{mailto:thedaily@nytimes.com}{\emph{thedaily@nytimes.com}}\emph{.
Follow Michael Barbaro on Twitter:}
\href{https://twitter.com/mikiebarb}{\emph{@mikiebarb}}\emph{. And if
you're interested in advertising with ``The Daily,'' write to us at}
\href{mailto:thedaily-ads@nytimes.com}{\emph{thedaily-ads@nytimes.com}}\emph{.}

``A Path to Curing H.I.V.'' was produced by Andy Mills and Jonathan
Wolfe, and edited by Paige Cowett, Wendy Dorr and Larissa Anderson.

``The Daily'' is produced by Theo Balcomb, Annie Brown, Jessica Cheung,
Lynsea Garrison, Michael Simon Johnson, Andy Mills, Neena Pathak, Rachel
Quester, Ike Sriskandarajah, Clare Toeniskoetter, Jonathan Wolfe and
Alexandra Leigh Young, and edited by Larissa Anderson, Paige Cowett and
Wendy Dorr. Lisa Tobin is our executive producer. Samantha Henig is our
editorial director. Brad Fisher is our technical manager. Chris Wood is
our sound engineer. Our theme music is by Jim Brunberg and Ben Landsverk
of Wonderly.

Advertisement

\protect\hyperlink{after-bottom}{Continue reading the main story}

\hypertarget{site-index}{%
\subsection{Site Index}\label{site-index}}

\hypertarget{site-information-navigation}{%
\subsection{Site Information
Navigation}\label{site-information-navigation}}

\begin{itemize}
\tightlist
\item
  \href{https://help.nytimes.com/hc/en-us/articles/115014792127-Copyright-notice}{©~2020~The
  New York Times Company}
\end{itemize}

\begin{itemize}
\tightlist
\item
  \href{https://www.nytco.com/}{NYTCo}
\item
  \href{https://help.nytimes.com/hc/en-us/articles/115015385887-Contact-Us}{Contact
  Us}
\item
  \href{https://www.nytco.com/careers/}{Work with us}
\item
  \href{https://nytmediakit.com/}{Advertise}
\item
  \href{http://www.tbrandstudio.com/}{T Brand Studio}
\item
  \href{https://www.nytimes.com/privacy/cookie-policy\#how-do-i-manage-trackers}{Your
  Ad Choices}
\item
  \href{https://www.nytimes.com/privacy}{Privacy}
\item
  \href{https://help.nytimes.com/hc/en-us/articles/115014893428-Terms-of-service}{Terms
  of Service}
\item
  \href{https://help.nytimes.com/hc/en-us/articles/115014893968-Terms-of-sale}{Terms
  of Sale}
\item
  \href{https://spiderbites.nytimes.com}{Site Map}
\item
  \href{https://help.nytimes.com/hc/en-us}{Help}
\item
  \href{https://www.nytimes.com/subscription?campaignId=37WXW}{Subscriptions}
\end{itemize}
