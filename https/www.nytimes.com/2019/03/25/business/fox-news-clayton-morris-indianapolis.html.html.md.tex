Sections

SEARCH

\protect\hyperlink{site-content}{Skip to
content}\protect\hyperlink{site-index}{Skip to site index}

\href{https://www.nytimes.com/section/business}{Business}

\href{https://myaccount.nytimes.com/auth/login?response_type=cookie\&client_id=vi}{}

\href{https://www.nytimes.com/section/todayspaper}{Today's Paper}

\href{/section/business}{Business}\textbar{}An Ex-Fox News Host Pitched
`Financial Freedom.' His Clients Want Their Money Back.

\url{https://nyti.ms/2HYEZUf}

\begin{itemize}
\item
\item
\item
\item
\item
\item
\end{itemize}

Advertisement

\protect\hyperlink{after-top}{Continue reading the main story}

Supported by

\protect\hyperlink{after-sponsor}{Continue reading the main story}

\hypertarget{an-ex-fox-news-host-pitched-financial-freedom-his-clients-want-their-money-back}{%
\section{An Ex-Fox News Host Pitched `Financial Freedom.' His Clients
Want Their Money
Back.}\label{an-ex-fox-news-host-pitched-financial-freedom-his-clients-want-their-money-back}}

Clayton Morris has been sued by nearly two dozen customers who say they
were sold ramshackle homes as investment properties.

\includegraphics{https://static01.nyt.com/images/2019/03/26/business/26morris7/merlin_152354577_81baf501-b443-42e6-848d-74dcf0ded2d7-articleLarge.jpg?quality=75\&auto=webp\&disable=upscale}

\href{https://www.nytimes.com/by/matthew-goldstein}{\includegraphics{https://static01.nyt.com/images/2018/11/06/multimedia/author-matthew-goldstein/author-matthew-goldstein-thumbLarge.png}}

By \href{https://www.nytimes.com/by/matthew-goldstein}{Matthew
Goldstein}

\begin{itemize}
\item
  March 25, 2019
\item
  \begin{itemize}
  \item
  \item
  \item
  \item
  \item
  \item
  \end{itemize}
\end{itemize}

INDIANAPOLIS --- Clayton Morris walked away from his job as a Fox News
host in 2017 to devote himself to the next phase of his professional
life: helping regular people achieve financial independence.

Mr. Morris, a host on ``Fox and Friends Weekend,'' already had a popular
\href{https://itunes.apple.com/us/podcast/investing-in-real-estate-clayton-morris-build-financial/id1115024566?mt=2}{real
estate investing podcast} when he and his wife, Natali, decided to
become full-time real estate advisers. Their plan was to connect
mom-and-pop investors with turnkey investment homes in Detroit,
Indianapolis, Jacksonville, Fla., and several other cities. Their
company, \href{https://morrisinvest.com/}{Morris Invest}, would handle
the details: finding properties, overseeing renovations, hiring property
managers to rent out the houses. All clients had to do was put up the
cash and wait for the checks to arrive.

Morris Invest helped sell at least 1,000 properties over the past two
years, reaping more than \$5 million in referral fees and profits from
the sales, according to resale prices and interviews with investors and
a lawyer for a former business partner. But Mr. Morris's customers said
many of the homes in Indianapolis had cost them dearly.

Nearly two dozen customers are now suing Mr. Morris and his company.
They contend that the properties were in worse shape than advertised,
and that rehab work paid for upfront was done poorly or not at all.
Vacant lots sold on the expectation of new homes being built are strewn
with trash. One house gutted by fire was sold a few days later to an
unwitting investor, according to a lawsuit.

``He comes across as this nice, likable family guy,'' said Brian
Freeman, a California lawyer who plunked down about \$40,000 for a
ramshackle home that was in such bad shape he was issued fines. ``He's
famous and I thought, `He's not going to ruin his entire reputation.'
Obviously, in hindsight, I feel like such an idiot.''

\includegraphics{https://static01.nyt.com/images/2019/03/22/business/00morris2/merlin_152423943_36d9f141-64a0-4cd6-8719-3212273db406-articleLarge.jpg?quality=75\&auto=webp\&disable=upscale}

The Morrises face a half-dozen lawsuits, including one in federal court,
and more will probably follow. Lawyers in Indianapolis are fielding
calls from disgruntled customers and angry renters, and Indiana's
attorney general has opened an investigation. In response to a Freedom
of Information Act request, the Federal Trade Commission said it had
received 21 consumer complaints.

The couple insisted they were not to blame for the properties' problems.

``We were a victim, too,'' Mr. Morris, 42, said during an interview with
his wife at a coffee shop near their suburban New Jersey home.

The couple said they had lost hundreds of thousands of dollars on homes
that they and their relatives bought from a property-management company
that was one of their business partners in Indianapolis. The company,
Oceanpointe Investments, was the seller of the homes the Morrises'
clients bought, and, according to the couple, it was supposed to do the
renovations and manage the properties.

The Morrises said that Oceanpointe, which many Morris Invest clients
said they had never heard of until after buying the homes, is the real
villain and liable for any damages. Few, if any, problems have arisen in
other cities, they said. Oceanpointe blames the Morrises, saying they
are responsible for the promises made to investors.

Also caught in the middle are the renters who lived in some of the
homes. They say that poor upkeep resulted in collapsing ceilings and
frequent plumbing problems. Furnaces often did not work properly,
leaving homes freezing cold in winter. One renter in a pending
landlord-tenant case blamed poor living conditions for the premature
birth of girl who died an hour after delivery.

The Morrises said they were not aware of the extent of the problems.

``We didn't know any were living in abject conditions,'' Ms. Morris, 40,
said.

\hypertarget{incestuous-networks}{%
\subsection{`Incestuous Networks'}\label{incestuous-networks}}

The unfolding affair demonstrates the allure that real estate
speculation still holds for individual investors roughly a decade after
one of the worst housing crises in United States history.

Image

Larry McLeskey and his wife bought this Indianapolis house. They said
they had lost \$40,000 on the property.Credit...Maddie McGarvey for The
New York Times

It can be problematic when such investors look to charismatic
personalities for investing tips. Many of the financial gurus pushing
investors into real estate play off fears of economic insecurity,
according to Philip Garboden, a professor of affordable housing at the
University of Hawaii at Manoa.

In a \href{https://osf.io/preprints/socarxiv/gucw6/}{research working
paper} financed partly by the federal Department of Housing and Urban
Development, Mr. Garboden wrote that amateur investors were vulnerable
to exploitation by those who ``evangelize'' the process and tend to play
down the risks of investing in ``low-end'' urban real estate.

``These are very incestuous networks,'' Mr. Garboden said in an
interview. ``They know the contractors. The property manager. The whole
system thrives on keeping every dollar invested in that network.''

Investment gurus, he said, tend to have a common message: Investing in
real estate can help guide the average investor to financial
independence.

\hypertarget{making-the-pitch}{%
\subsection{Making the Pitch}\label{making-the-pitch}}

Mr. Morris's sales pitch did not lean heavily on his career in broadcast
journalism. But he did not keep it a secret, either. He alluded to his
new career in a
\href{https://insider.foxnews.com/2017/09/04/clayton-morris-leaving-fox-friends-best-moments-tv}{lighthearted
seven-minute video}send-off that Fox News put together and that showed
Mr. Morris competing in an obstacle course competition and grilling
burgers outside the Manhattan studio.

Real estate investing, he said in marketing materials, had given him the
financial security to quit his 9 to 5 broadcast job. ``I'm a big fan of
this radical idea that everyone should be able to achieve total
financial freedom,'' the biography on
\href{https://www.claytonmorris.com/}{one of his websites} says.

With his wife --- a former anchor for CBS Interactive --- he wrote a
book,
``\href{https://www.amazon.com/gp/product/B07FVXN4PF/ref=dbs_a_def_rwt_hsch_vapi_taft_p1_i1}{How
to Pay Off Your Mortgage in 5 Years}.''

Image

Beth Stern, Steve Doocy, Elisabeth Hasselbeck and Mr. Morris on "Fox and
Friends" in 2014.Credit...Rob Kim/Getty Images

But the couple's new venture did more than offer advice: It was a
one-stop shopping experience for investors who wanted to buy rental
homes --- by dipping into their retirement savings, if necessary.

An email sent to one client who signed up last year summed up the pitch:

\begin{quote}
Are you working LONG hours but never quite able to get ahead? Are you
worried about making COSTLY mistakes with a vacant rental property? Are
you intimidated by the thought of DOING IT ALL yourself? Stop worrying,
and let us take care it!
\end{quote}

The Morrises pulled in hundreds of customers from across the United
States and as far away as Israel and South Korea. They helped sell
nearly 700 homes in Indianapolis alone.

``It was his name and his promise,'' said Larry McLeskey, one of nearly
two dozen individual investors suing Mr. Morris in the federal lawsuit.
Mr. McLeskey, who lives in Michigan, said that he and his wife, Kay, had
lost \$40,000 after selling a home in Indianapolis. ``No one was taking
care of the home.''

Danny Gomes, a real estate agent from Redding, Calif., sued Morris
Invest separately after, he said, he lost \$52,000 on an Indianapolis
home he bought last year, just days after it was largely destroyed in a
fire. The house is now boarded up, its back half all but gone.

Image

The house at 1509 Asbury Street in Indianapolis that Danny Gomes bought.
The home had been damaged in a fire days before the deal closed, but Mr.
Gomes did not know that until several months later.Credit...Maddie
McGarvey for The New York Times

Image

Mr. Gomes at his home in California. He said he had paid in advance for
renovations to the Indianapolis property that were never
completed.Credit...Alexandra Hootnick for The New York Times

Mr. Gomes said he had learned about the fire only several months later,
when the city sent him a notice warning that the property was unsafe and
needed to be boarded up.

Until then, he believed the house was being rehabilitated, work he paid
for when he bought it. One of Mr. Morris's employees, who is also Ms.
Morris's sister, told Mr. Gomes that she would be his ``point of contact
for the rehab process,'' according to an email provided to The New York
Times.

Mr. Gomes said it was only after he learned about the fire that he
discovered Morris Invest was simply getting a referral fee for sending
customers to Oceanpointe.

``When it hit the fan,'' Mr. Gomes said, ``they said they were just the
middleman.''

\hypertarget{happy-as-a-clam}{%
\subsection{`Happy as a Clam'}\label{happy-as-a-clam}}

As the Morrises tell it, they were blindsided just like everyone else.
The couple, who have three young children, said they were frustrated
with all the anger directed at them. They have put their own home up for
sale, in part out of concern for their safety.

They place the blame squarely on Oceanpointe and its founder, Bert
Whalen.

The Morrises said they met Mr. Whalen in 2014, when they bought a few
homes in Indianapolis and used Oceanpointe to fix up and manage them. By
2016, Mr. Morris was referring one or two investors a week to Mr.
Whalen's firm. The Morrises said they had eventually formalized the
relationship, sending buyers to Oceanpointe and earning a fee on each
sale.

Image

Another of the Indianapolis homes that Mr. Morris's company, Morris
Invest, sold as investment properties. He is now facing a half-dozen
lawsuits filed by disgruntled buyers.Credit...Maddie McGarvey for The
New York Times

Mr. Morris said it was not until spring 2018 that he became fully aware
of the problems his customers were having with Oceanpointe. The
relationship formally ended in May.

The Morrises said they would not have gotten involved with Mr. Whalen
had they known that Indiana regulators moved to deny a renewal of his
real estate license in December 2015. A state regulator determined that
he had failed to disclose convictions for operating a car and motorboat
while intoxicated and, on at least one occasion, had not turned over
rent money he had collected for a property owner. His real estate
license was permanently revoked in January 2018.

Despite the break with Oceanpointe, the Morrises said they had many
satisfied customers. Renovation work was done on 60 percent of the 700
homes sold in Indianapolis, Mr. Morris estimated.

``There are hundreds of people who are as happy as a clam,'' he said.

There are many, however, who are not.

For clients dissatisfied with the work on their properties, Mr. Morris
said Oceanpointe had agreed to indemnify Morris Invest against all
lawsuits and investor claims. The Morrises' lawyer sent a letter to Mr.
Whalen in October seeking to enforce the indemnification agreement. The
couple declined to provide a copy of the agreement.

John Tompkins, a lawyer for Mr. Whalen, blamed Morris Invest, which he
said had collected a \$6,500 referral fee on every property.

Mr. Whalen's businesses ``have done everything required by their
investment agreements and contracts,'' Mr. Tompkins said.

He disputed the suggestion that the indemnification agreement held
Oceanpointe solely responsible for problems with the selling of the
homes.

``The problems that have come up relate to Morris exclusively,'' Mr.
Tompkins said. ``The misunderstandings of those buyers related to
misstatements by Morris and his sales personnel.''

\hypertarget{a-new-venture}{%
\subsection{A New Venture}\label{a-new-venture}}

The Morrises have largely gotten out of the real estate business in
Indianapolis. They are focused mainly on Detroit, where they have sold
more than 200 homes, largely without the kind of complaints they face in
Indianapolis.

Morris Invest, though, is no longer their top priority.

The Morrises are now selling an online financial advice and planning
program: Financial Freedom Academy. The program offers a ``proven system
for building wealth, guaranteed,'' according to its website.

The site describes a conversation that Mr. Morris said transformed his
life: He was seated next to real estate investor on a trip to New
Zealand several years ago, and learned the man and his wife were on a
two-month vacation after making money acquiring homes, fixing them up
and renting them out.

``From that moment on, I made it my mission to follow in his
footsteps,'' Mr. Morris says on the site.

Mr. Morris is inviting investors to follow in his footsteps with a
\href{https://financialfreedomacademy.com}{nine-session online program}
at a special introductory rate of \$697.

Advertisement

\protect\hyperlink{after-bottom}{Continue reading the main story}

\hypertarget{site-index}{%
\subsection{Site Index}\label{site-index}}

\hypertarget{site-information-navigation}{%
\subsection{Site Information
Navigation}\label{site-information-navigation}}

\begin{itemize}
\tightlist
\item
  \href{https://help.nytimes.com/hc/en-us/articles/115014792127-Copyright-notice}{©~2020~The
  New York Times Company}
\end{itemize}

\begin{itemize}
\tightlist
\item
  \href{https://www.nytco.com/}{NYTCo}
\item
  \href{https://help.nytimes.com/hc/en-us/articles/115015385887-Contact-Us}{Contact
  Us}
\item
  \href{https://www.nytco.com/careers/}{Work with us}
\item
  \href{https://nytmediakit.com/}{Advertise}
\item
  \href{http://www.tbrandstudio.com/}{T Brand Studio}
\item
  \href{https://www.nytimes.com/privacy/cookie-policy\#how-do-i-manage-trackers}{Your
  Ad Choices}
\item
  \href{https://www.nytimes.com/privacy}{Privacy}
\item
  \href{https://help.nytimes.com/hc/en-us/articles/115014893428-Terms-of-service}{Terms
  of Service}
\item
  \href{https://help.nytimes.com/hc/en-us/articles/115014893968-Terms-of-sale}{Terms
  of Sale}
\item
  \href{https://spiderbites.nytimes.com}{Site Map}
\item
  \href{https://help.nytimes.com/hc/en-us}{Help}
\item
  \href{https://www.nytimes.com/subscription?campaignId=37WXW}{Subscriptions}
\end{itemize}
