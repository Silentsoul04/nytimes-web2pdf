Sections

SEARCH

\protect\hyperlink{site-content}{Skip to
content}\protect\hyperlink{site-index}{Skip to site index}

\href{https://www.nytimes.com/section/politics}{Politics}

\href{https://myaccount.nytimes.com/auth/login?response_type=cookie\&client_id=vi}{}

\href{https://www.nytimes.com/section/todayspaper}{Today's Paper}

\href{/section/politics}{Politics}\textbar{}Interior Nominee Intervened
to Block Report on Endangered Species

\begin{itemize}
\item
\item
\item
\item
\item
\item
\end{itemize}

Advertisement

\protect\hyperlink{after-top}{Continue reading the main story}

Supported by

\protect\hyperlink{after-sponsor}{Continue reading the main story}

\hypertarget{interior-nominee-intervened-to-block-report-on-endangered-species}{%
\section{Interior Nominee Intervened to Block Report on Endangered
Species}\label{interior-nominee-intervened-to-block-report-on-endangered-species}}

\includegraphics{https://static01.nyt.com/images/2019/03/22/us/politics/00dc-endanger2/merlin_149288694_429d9924-f617-4f7a-b8c7-db1d626b91e2-articleLarge.jpg?quality=75\&auto=webp\&disable=upscale}

By \href{https://www.nytimes.com/by/eric-lipton}{Eric Lipton}

\begin{itemize}
\item
  March 26, 2019
\item
  \begin{itemize}
  \item
  \item
  \item
  \item
  \item
  \item
  \end{itemize}
\end{itemize}

WASHINGTON --- After years of effort, scientists at the Fish and
Wildlife Service had a moment of celebration as they wrapped up a
comprehensive analysis of the threat that three widely used pesticides
present to hundreds of endangered species, like the
\href{https://www.epa.gov/sites/production/files/2013-08/documents/san-joaquin-kitfox.pdf}{kit
fox} and the
\href{https://www.nps.gov/ever/learn/nature/csss.htm}{seaside sparrow}.

``Woohoo!'' Patrice Ashfield, then a branch chief at Fish and Wildlife
Service headquarters,
\href{https://www.documentcloud.org/documents/5778894-Trump-Era-Shift-Evaluating-Pesticide-Threats-to.html\#document/p42/a488261}{wrote
to her colleagues} in August 2017.

Their analysis found that two of the pesticides, malathion and
chlorpyrifos, were so toxic that they
``\href{https://www.epa.gov/laws-regulations/summary-endangered-species-act}{jeopardize
the continued existence}'' of more than 1,200 endangered birds, fish and
other animals and plants, a conclusion that could lead to tighter
restrictions on use of the chemicals.

But just before the team planned to make its findings public in November
2017, something unexpected happened: Top political appointees of the
Interior Department, which oversees the Fish and Wildlife Service,
blocked the release and set in motion a new process intended to apply a
much narrower standard to determine the risks from the pesticides.

Leading that intervention was David Bernhardt, then the deputy secretary
of the interior and a former lobbyist and oil-industry lawyer. In
October 2017, he abruptly summoned staff members to the first of a rapid
series of meetings in which the Fish and Wildlife Service was directed
to take the new approach, one that pesticide makers and users had
lobbied intensively to promote.

Mr. Bernhardt is now President Trump's nominee to become interior
secretary. The Senate is scheduled to hold a hearing on his confirmation
\href{https://www.energy.senate.gov/public/index.cfm/hearings-and-business-meetings?ID=35F3BA18-1860-4227-8C1B-A7C5891D14E1}{Thursday}.

This sequence of events is detailed in more than
\href{https://www.documentcloud.org/documents/5778894-Trump-Era-Shift-Evaluating-Pesticide-Threats-to.html}{84,000
pages} of Interior Department and Environmental Protection Agency
documents obtained via Freedom of Information requests by The New York
Times and, separately, by the Center for Biological Diversity, an
environmental group that
\href{https://www.biologicaldiversity.org/species/amphibians/California_red-legged_frog/pdfs/RFL2-order_entering_settlement.pdf}{sued
the federal government} to force it to complete the pesticide studies.

The documents provide a case study of how the Trump administration has
been using its power to second-guess or push aside conclusions reached
by career professionals, particularly in the area of public health and
the environment.

The decision to block the release of the report represented a victory
for the pesticide industry, which has industry allies and former
executives sprinkled through the administration. Among those with the
most at stake were Dow AgroSciences, a manufacturer of chlorpyrifos,
which is used on dozens of fruits and vegetables, and FMC Corporation, a
manufacturer of malathion, which is used against mosquitoes as well as
chewing and sucking insects that attack a
\href{http://www.fmccrop.com/grower/Products/Insecticides-Miticides/cheminova-malathion-57ec.aspx}{range
of crops} including tomatoes, strawberries and walnuts.

Dow, which was recently renamed
\href{https://www.corteva.com/who-we-are/our-merger.html}{Corteva},
donated \href{https://www.opensecrets.org/trump/inauguration-donors}{\$1
million} to Mr. Trump's inauguration committee. E.P.A. and Interior
Department records show that top pesticide industry executives had
regular access to senior agency officials, pressing them to reconsider
the way the federal government evaluates the threat pesticides cause to
endangered species.

A Dow spokesman said the shift in policy was unrelated to the \$1
million contribution. The new approach will result in ``a better
understanding of where and how pesticides are being used,'' said Gregg
M. Schmidt, a Corteva spokesman.

Spokesmen for \href{http://www.fmc.com/}{FMC} and
\href{https://www.adama.com/en/}{Adama} --- the other primary makers of
the pesticides being studied --- as well as their lawyers and CropLife
America, the trade group that represents them, declined to comment.

Asked if Mr. Bernhardt's intervention was appropriate or motivated by a
desire to serve the industry's interests, an Interior Department
spokeswoman said his actions had been ``governed solely by legitimate
concerns regarding the legal sufficiency and policy.''

Before he joined the Trump administration, Mr. Bernhardt worked as a
\href{https://www.documentcloud.org/documents/5779173-Bernhardt-Ethics-Letter-and-Disclosure-Report.html}{lawyer
and lobbyist} representing clients including the oil and gas industry.
He was frequently
\href{https://www.nytimes.com/2019/02/12/climate/david-bernhardt-endangered-species.html}{paid
to challenge} endangered species-related matters, including one
involving a tiny silvery blue fish called the delta smelt whose
protection by the federal government has resulted in limits on water use
by California farmers.

\includegraphics{https://static01.nyt.com/images/2019/03/22/us/politics/00dc-endanger1/merlin_148068153_d307417f-e512-4026-8c80-30e83ae053d0-articleLarge.jpg?quality=75\&auto=webp\&disable=upscale}

Agency records suggest Mr. Bernhardt, after having had only limited
involvement in the issue, had nine meetings or calls on his schedule
with Fish and Wildlife staff in October and November 2017, and helped
write the letter saying the Interior Department was no longer prepared
to release the draft.

Wendy Cleland-Hamnett, the E.P.A. official at the time who ran the
office in charge of toxic chemicals and pesticides, said the sudden
change in regulatory philosophy was part of a
\href{https://www.nytimes.com/2017/10/21/us/trump-epa-chemicals-regulations.html}{broader
trend across the government} after Mr. Trump's election.

``It is certainly similar to the pattern we saw in toxic chemicals as
well, where the regulated industry had a more sympathetic ear in the new
administration,'' said Ms. Hamnett, who left the E.P.A. in late 2017,
after a 38-year career with the agency. ``And that resulted in a shift
in approach as to how these issues would be handled.''

Gary Frazer, the top endangered species official at the Fish and
Wildlife Service, whose schedule says he participated in all nine of the
late 2017 discussions with Mr. Bernhardt, and who subsequently directed
his staff to revise the study, said he did not believe the change in
direction was politically driven.

``It was an entirely appropriate role,'' he said in an interview, as two
of the agency's public affairs officials listened in. ``There was no
arm-twisting of any kind.''

The endangered species review is required as part of the re-registration
of pesticides, a process that
\href{https://www.epa.gov/pesticide-reevaluation/registration-review-process}{occurs
every 15 years}.

Experts at the Fish and Wildlife Service and the Commerce Department's
National Marine Fisheries Service were supposed to determine if any of
the pesticides might ``jeopardize the continued existence of any
endangered species or threatened species or result in the destruction or
adverse modification of habitat of such species,'' a standard created
\href{https://www.fws.gov/endangered/laws-policies/section-7.html}{under
federal law}.

Much of the work focuses on questions like whether a wildfire management
program in the Florida Everglades
\href{https://ecos.fws.gov/tails/pub/document/2303919}{hurt endangered
species} such as the American crocodile or the West Indian manatee. The
Fish and Wildlife Service rarely makes so-called jeopardy findings;
\href{https://www.pnas.org/content/112/52/15844}{a 2015 study} of nearly
7,000 cases found that only two concluded with a finding that a species
was in jeopardy.

The pesticide industry, as well as groups representing farmers who rely
on its products, began to mobilize as the endangered species review got
underway during the Obama administration.

With Mr. Trump's election, the industry escalated its campaign. In April
2017, its lawyers
\href{https://www.documentcloud.org/documents/5778894-Trump-Era-Shift-Evaluating-Pesticide-Threats-to.html\#document/p21/a488251}{sent
a letter to Ryan Zinke}, then the interior secretary; Scott Pruitt, then
the E.P.A.'s administrator; and the commerce secretary, Wilbur Ross,
asking them to ``direct that any effort to prepare biological
opinions,'' as the process is called, ``be set aside,'' arguing that the
analysis was ``fundamentally flawed.''

The industry's central argument was that the federal scientists were not
sufficiently taking into account the difference between how the
pesticides could legally be used and how they were actually used.

Staff members at the Fish and Wildlife Service, emails show, did have
access to actual pesticide usage, as well as other information, such as
\href{https://www.documentcloud.org/documents/5779584-2017-05-30-1355-Email-Ambient-Monitoring-for.html}{measurements
of pesticide concentrations} found in salmon-bearing streams in
Washington State.

But the agency staff --- working from dozens of field offices like
Hawaii and Maine as well as the headquarters --- generally built its
predictions of a ``jeopardy'' threat to endangered species by assuming
the pesticides were being used to the maximum extent possible as allowed
by their labels.

That is because ``unlike most other types of product labels, pesticide
labels are legally enforceable,''
\href{https://www.epa.gov/pesticide-labels/introduction-pesticide-labels}{according
to E.P.A. policy}. And historic usage data, the agency staff said in its
documents, is not sufficient to predict how these pesticides might be
used --- and cause harm --- in the coming 15 years.

The pesticides, particularly chlorpyrifos and malathion, are ``high
toxicity'' for all animals, and their effect on endangered species would
be both direct and indirect, via contamination of food sources, for
example,
\href{https://www.documentcloud.org/documents/5778894-Trump-Era-Shift-Evaluating-Pesticide-Threats-to.html\#document/p88/a488287}{the
staff concluded}. The E.P.A.
\href{https://www.nytimes.com/2017/03/29/us/politics/epa-insecticide-chlorpyrifos.html}{has
separately considered} banning chlorpyrifos because of potential harm to
humans.

Image

A San Joaquin kit fox in Bakersfield. Decades ago the species inhabited
large parts of California's San Joaquin Valley, but most of those fox
populations are now gone, in part because of pesticides.Credit...Casey
Christie/The Bakersfield Californian, via Associated Press

The Fish and Wildlife staff cited the San Joaquin kit fox, a tiny animal
that weighs about five pounds, with a slim body, large ears and a long,
bushy tail. Decades ago, it inhabited large parts of California's San
Joaquin Valley, an area today of intensive farming and pesticide use.
But most of those fox populations are now gone, in part because
pesticides like diazinon contaminated birds and grasses the foxes fed
on,
\href{https://www.documentcloud.org/documents/5778894-Trump-Era-Shift-Evaluating-Pesticide-Threats-to.html\#document/p101/a488289}{the
agency concluded}.

The Cape Sable seaside sparrow, another endangered species located
mostly now in Florida, was found to be in jeopardy as a result of
drifting sprays of chlorpyrifos. ``For many vulnerable species, a single
exposure could be catastrophic,'' an October 2017
\href{https://www.documentcloud.org/documents/5778894-Trump-Era-Shift-Evaluating-Pesticide-Threats-to.html\#document/p100/a488288}{summary
of the staff's findings said}.

Agency records show repeated contacts in early 2017 by the pesticide
industry with administration officials. Among those targeted, the emails
show, was Daniel Jorjani, a top Interior Department lawyer who had spent
\href{https://www.documentcloud.org/documents/4873360-Daniel-Jorjani-Resume.html}{six
years working} for groups connected to the billionaire brothers Charles
G. and David H. Koch.

Aaron Hobbs, a
\href{http://disclosures.house.gov/ld/pdfform.aspx?id=300343238}{onetime
lobbyist} for CropLife, the leading pesticide industry trade
association, who now works for an affiliate of the industry-funded
group,
\href{https://www.documentcloud.org/documents/5778894-Trump-Era-Shift-Evaluating-Pesticide-Threats-to.html\#document/p28/a488253}{reached
out to Mr. Jorjani} and invited him to an April 2017 meeting with
industry officials to discuss the endangered species effort --- shortly
after sending the letter asking the agency to kill the Fish and Wildlife
Service's work. He followed up
\href{https://www.documentcloud.org/documents/5778894-Trump-Era-Shift-Evaluating-Pesticide-Threats-to.html\#document/p37/a488258}{again
in July} in an attempt to set up another meeting.

Top officials from the E.P.A. and Interior and Agriculture Departments
began a series of meetings in June 2017, often involving representatives
from the White House.

Among the other
\href{https://www.documentcloud.org/documents/5778894-Trump-Era-Shift-Evaluating-Pesticide-Threats-to.html\#document/p35/a488257}{participants
in these meetings, the records show}, was Rebeckah Adcock, who until
April 2017 had been a
\href{https://www.documentcloud.org/documents/4824481-Rebeckah-Adcock-Resume.html}{director
of government affairs} and registered lobbyist for CropLife and who now
works as a senior adviser at the Agriculture Department.

Ms. Adcock joined the discussions even though the
\href{https://assets.documentcloud.org/documents/4193032/Rebeckah-Adcock-Ethics-Agreement-1.pdf}{ethics
agreement} she signed said she would not participate ``personally and
substantially in any particular matter'' involving CropLife for one
year. An Agriculture Department spokesman said this did not violate that
ban because she had not specifically lobbied on endangered species
matters for CropLife.

Even as these meetings were taking place, staff members inside the Fish
and Wildlife Service were wrapping up the enormous task of assessing the
threat presented by these pesticides, email records show.

The team
\href{https://www.documentcloud.org/documents/5778894-Trump-Era-Shift-Evaluating-Pesticide-Threats-to.html\#document/p1/a488282}{had
concluded} that chlorpyrifos put 1,399 species --- a mixture of animals
and plants --- in jeopardy, while malathion put 1,284 of them in
jeopardy and diazinon, a third pesticide that was evaluated, placed 175
species in jeopardy. There are
\href{https://ecos.fws.gov/ecp0/reports/box-score-report}{1,663 species}
listed as endangered or threatened in the United States, meaning that
two of the pesticides may be putting most of them in jeopardy. (This
information, agency officials said, was accidentally released in a
Freedom of Information response obtained by The Times. They intended to
keep this tally a secret, because the assessment was not final.)

The agency staff was not recommending that the pesticides be banned.
Instead, they were proposing changes in how the pesticides could be
used, including possible restrictions on their use in areas where
endangered species are found, or at certain times of year, the documents
say.

Lawyers who work for the interior secretary's office wanted a very
different approach. They advocated abandoning the presumption that use
of a pesticide by a farmer or a golf course might directly cause the
death of or harm to an endangered species, officials said.

Their argument was that because farmers, for example, do not manufacture
the pesticide, their use of it means that any harm caused is an indirect
effect, as defined under federal law. There is a much higher standard of
proof needed to demonstrate that an indirect effect has harmed an
endangered species. The law requires that this harm be shown to be
``\href{https://www.fws.gov/midwest/endangered/section7/s7process/s7glossary.html}{reasonably
certain to occur}.'' The revised approach is almost certainly going to
result in fewer plants and animals being judged to be in jeopardy of
extinction as a result of continued pesticide use.

The shift in approach goes far beyond a single Fish and Wildlife Service
analysis. In early 2018, Mr. Pruitt, Mr. Zinke and Mr. Ross
\href{https://www.documentcloud.org/documents/5778894-Trump-Era-Shift-Evaluating-Pesticide-Threats-to.html\#document/p125/a488281}{agreed
to work} toward a new framework for all endangered species evaluations,
a move that
\href{https://www.documentcloud.org/documents/5778894-Trump-Era-Shift-Evaluating-Pesticide-Threats-to.html\#document/p133/a488285}{CropLife
called} ``a positive step towards solving this important and complex
issue.''

Documents show that the administration does not now expect to make
public any draft results of the revised
\href{https://www.documentcloud.org/documents/5778894-Trump-Era-Shift-Evaluating-Pesticide-Threats-to.html\#document/p161/a488286}{assessment
until April 2020}, two and a half years later than had been planned.

On Tuesday, after this article was published online, three House
Democrats, including Representative Raul M. Grijalva of Arizona, the
chairman of the House Committee on Natural Resources,
\href{https://www.documentcloud.org/documents/5778894-Trump-Era-Shift-Evaluating-Pesticide-Threats-to.html\#document/p163/a489246}{sent
a letter to Mr. Bernhardt} asking him to release the draft reports
immediately.

Advertisement

\protect\hyperlink{after-bottom}{Continue reading the main story}

\hypertarget{site-index}{%
\subsection{Site Index}\label{site-index}}

\hypertarget{site-information-navigation}{%
\subsection{Site Information
Navigation}\label{site-information-navigation}}

\begin{itemize}
\tightlist
\item
  \href{https://help.nytimes.com/hc/en-us/articles/115014792127-Copyright-notice}{©~2020~The
  New York Times Company}
\end{itemize}

\begin{itemize}
\tightlist
\item
  \href{https://www.nytco.com/}{NYTCo}
\item
  \href{https://help.nytimes.com/hc/en-us/articles/115015385887-Contact-Us}{Contact
  Us}
\item
  \href{https://www.nytco.com/careers/}{Work with us}
\item
  \href{https://nytmediakit.com/}{Advertise}
\item
  \href{http://www.tbrandstudio.com/}{T Brand Studio}
\item
  \href{https://www.nytimes.com/privacy/cookie-policy\#how-do-i-manage-trackers}{Your
  Ad Choices}
\item
  \href{https://www.nytimes.com/privacy}{Privacy}
\item
  \href{https://help.nytimes.com/hc/en-us/articles/115014893428-Terms-of-service}{Terms
  of Service}
\item
  \href{https://help.nytimes.com/hc/en-us/articles/115014893968-Terms-of-sale}{Terms
  of Sale}
\item
  \href{https://spiderbites.nytimes.com}{Site Map}
\item
  \href{https://help.nytimes.com/hc/en-us}{Help}
\item
  \href{https://www.nytimes.com/subscription?campaignId=37WXW}{Subscriptions}
\end{itemize}
