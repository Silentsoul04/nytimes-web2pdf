Sections

SEARCH

\protect\hyperlink{site-content}{Skip to
content}\protect\hyperlink{site-index}{Skip to site index}

\href{https://www.nytimes.com/section/politics}{Politics}

\href{https://myaccount.nytimes.com/auth/login?response_type=cookie\&client_id=vi}{}

\href{https://www.nytimes.com/section/todayspaper}{Today's Paper}

\href{/section/politics}{Politics}\textbar{}Grindr Is Owned by a Chinese
Firm, and the U.S. Is Trying to Force It to Sell

\url{https://nyti.ms/2UX2SPs}

\begin{itemize}
\item
\item
\item
\item
\item
\item
\end{itemize}

Advertisement

\protect\hyperlink{after-top}{Continue reading the main story}

Supported by

\protect\hyperlink{after-sponsor}{Continue reading the main story}

\hypertarget{grindr-is-owned-by-a-chinese-firm-and-the-us-is-trying-to-force-it-to-sell}{%
\section{Grindr Is Owned by a Chinese Firm, and the U.S. Is Trying to
Force It to
Sell}\label{grindr-is-owned-by-a-chinese-firm-and-the-us-is-trying-to-force-it-to-sell}}

\includegraphics{https://static01.nyt.com/images/2019/03/28/us/politics/28dc-grindr/merlin_152724312_a08c49ac-d5ac-465c-8e19-08d94a7b4125-articleLarge.jpg?quality=75\&auto=webp\&disable=upscale}

By \href{https://www.nytimes.com/by/david-e-sanger}{David E. Sanger}

\begin{itemize}
\item
  March 28, 2019
\item
  \begin{itemize}
  \item
  \item
  \item
  \item
  \item
  \item
  \end{itemize}
\end{itemize}

\href{https://cn.nytimes.com/usa/20190329/grindr-china-national-security/}{阅读简体中文版}\href{https://cn.nytimes.com/usa/20190329/grindr-china-national-security/zh-hant/}{閱讀繁體中文版}

WASHINGTON --- The Trump administration is expanding its efforts to
block Chinese acquisitions in the United States, moving to force a
Chinese firm that owns Grindr, the gay dating app, to relinquish control
over concerns that Beijing could use personal information to blackmail
or influence American officials, according to people familiar with the
situation.

The action, which is being driven by the
\href{https://www.nytimes.com/2018/03/05/business/what-is-cfius.html}{Committee
on Foreign Investment in the United States}, is unusual given that the
panel typically investigates mergers that could result in control of an
American business by a foreign individual or company, judging whether
deals could threaten national security. This appears to be the first
case in which the United States has asserted that foreign control of a
social media app could have national security implications.

The administration has not announced the move, which will require that
Grindr be sold, or explained it. But officials familiar with the case,
which was first reported by Reuters, say the concern focused on the
potential for the blackmail of American officials or contractors, if
China threatened to disclose their sexual orientation, or track their
movements or dating habits.

Three years ago, a Chinese firm that owns both gaming and credit
services businesses, Beijing Kunlun Tech Co. Ltd., a public company
listed on the Shenzhen stock exchange, bought a 60 percent stake in
Grindr, which is based in West Hollywood, Calif., for \$93 million.
Early last year, it bought the remaining shares for a little over \$150
million.

While there were news reports about both transactions, the United States
did not take action to block the acquisitions. Since then, the United
States' definition of national security threats has expanded, in part
over concerns by the Trump administration and lawmakers about China's
ability to gain access to critical American technology.

It is unclear why the panel, known as Cfius, acted now, more than three
years after control of the company switched to Chinese hands. And so
far, there is no public evidence that any information on the app has
been used by the Chinese government.

But Senator Ron Wyden, Democrat of Oregon, said he, along with several
other senators, asked Cfius to conduct a review.

``Last year, my office met with a top official from the Treasury
Department to express my serious concerns about the national security
risks associated with a Chinese company buying Grindr,'' he said in a
statement. While he said he could not ``confirm specific actions by
Cfius,'' a highly secretive panel, ``it is high time for the
administration and Cfius to consider the national security impact of
foreign companies acquiring large, sensitive troves of Americans'
private data.''

Congress
\href{https://www.nytimes.com/2018/10/10/business/us-china-investment-cfius.html}{handed
more power to the panel} last year, allowing it to examine transactions
that fell short of majority control of a company and involved just
minority stakes. The expansion was an effort to counter Chinese minority
investments in Silicon Valley companies that gave investors an early
look at emerging technologies.

The Kunlun purchases had never been submitted to Cfius, giving the
government the leverage to go back in after the sale to try to force a
divestment. Calls to Kunlun's office number were not answered, and
emails seeking comment were not returned.

Grindr has already
\href{https://www.nytimes.com/2018/04/03/technology/grindr-sets-off-privacy-firestorm-after-sharing-users-hiv-status-data.html}{faced
questions about its control and use} of personal data. The company faced
a huge backlash for sharing users' H.I.V. status, sexual tastes and
other intimate personal details with outside software vendors. After the
data sharing was made public by European researchers in 2018, the
company said it would stop sharing H.I.V. data with outside companies.

Last year was the first time Cfius appeared to be concerned about the
purchase of companies that contained sensitive data. The government
\href{https://www.nytimes.com/2018/01/02/business/moneygram-ant-financial-china-cfius.html}{killed
a proposed merger last year} between MoneyGram, the money transfer firm,
and Ant Financial, a payments company related to the Chinese e-commerce
giant Alibaba.

The United States has also embarked on a global campaign to block a big
Chinese telecom equipment giant, Huawei, from building the next
generation of wireless networks, known as 5G, over concerns that it
could divert critical data through China, or be forced to turn over data
running through its networks to Beijing. The White House has essentially
accused Huawei of being an arm of the Chinese government that can be
used for spying or to sabotage communications networks, a charge that
Huawei has vehemently denied.

But the administration's efforts to control what kind of personal data
is available to China's intelligence services may have come too late.
China's ministry of state security and other Chinese groups have already
\href{https://www.nytimes.com/2018/12/20/us/politics/us-and-other-nations-to-announce-china-crackdown.html}{been
accused of successfully stealing personal data} from American databases.

The
\href{https://www.nytimes.com/2015/06/05/us/breach-in-a-federal-computer-system-exposes-personnel-data.html}{theft
of 22 million security clearance}
\href{https://www.nytimes.com/2015/06/05/us/breach-in-a-federal-computer-system-exposes-personnel-data.html}{files}
from the Office of Personnel Management in 2014, along with
\href{https://www.nytimes.com/2018/12/11/us/politics/trump-china-trade.html}{similar
theft of data} from the Anthem insurance networks and Marriott hotels,
have all been attributed to Chinese actors by American intelligence
officials, who say they were most likely operating on behalf of the
government.

The files stolen in the 2014 government breach contain far more personal
data than the Chinese could probably find on any individual social media
site: They include work history on sensitive United States projects,
information about bankruptcies, medical conditions, relationship
histories, and any contacts with foreigners. The loss of the information
forced the C.I.A. to reassign personnel headed to China, and was
considered among the largest losses of sensitive security information in
decades. The Obama administration declined to publicly concede that the
breach was committed by Chinese intelligence services.

China has taken steps of its own to limit foreign companies' access to
its citizens' personal information. A recently enacted cybersecurity law
mandates that user data be stored in the country, where it can be kept
under the government's control. In response to the law, Apple said it
would open its first data center in China, and formed a partnership with
a Chinese company to run the center and handle data requests from the
government.

Before the law even came into effect, the Chinese government had
\href{https://www.nytimes.com/2017/07/12/business/apple-china-data-center-cybersecurity.html}{pressured
foreign technology companies to operate servers only within its borders}
--- meaning the data is available to Chinese authorities under Chinese
law. Amazon and Microsoft have partnered with Chinese firms to offer
cloud computing services to Chinese customers.

The United States has also pressed China to allow insurance companies
and other American firms that control personal data to enter the Chinese
market, a demand that goes back nearly two decades. China has agreed to
do so, and that agreement is expected to be part of the larger trade
deal being negotiated between American and Chinese negotiators.

But the Grindr case could give the Chinese government an excuse to make
its own national security claims if American firms sought to purchase a
Chinese insurance company, or any of its social media firms.

Advertisement

\protect\hyperlink{after-bottom}{Continue reading the main story}

\hypertarget{site-index}{%
\subsection{Site Index}\label{site-index}}

\hypertarget{site-information-navigation}{%
\subsection{Site Information
Navigation}\label{site-information-navigation}}

\begin{itemize}
\tightlist
\item
  \href{https://help.nytimes.com/hc/en-us/articles/115014792127-Copyright-notice}{©~2020~The
  New York Times Company}
\end{itemize}

\begin{itemize}
\tightlist
\item
  \href{https://www.nytco.com/}{NYTCo}
\item
  \href{https://help.nytimes.com/hc/en-us/articles/115015385887-Contact-Us}{Contact
  Us}
\item
  \href{https://www.nytco.com/careers/}{Work with us}
\item
  \href{https://nytmediakit.com/}{Advertise}
\item
  \href{http://www.tbrandstudio.com/}{T Brand Studio}
\item
  \href{https://www.nytimes.com/privacy/cookie-policy\#how-do-i-manage-trackers}{Your
  Ad Choices}
\item
  \href{https://www.nytimes.com/privacy}{Privacy}
\item
  \href{https://help.nytimes.com/hc/en-us/articles/115014893428-Terms-of-service}{Terms
  of Service}
\item
  \href{https://help.nytimes.com/hc/en-us/articles/115014893968-Terms-of-sale}{Terms
  of Sale}
\item
  \href{https://spiderbites.nytimes.com}{Site Map}
\item
  \href{https://help.nytimes.com/hc/en-us}{Help}
\item
  \href{https://www.nytimes.com/subscription?campaignId=37WXW}{Subscriptions}
\end{itemize}
