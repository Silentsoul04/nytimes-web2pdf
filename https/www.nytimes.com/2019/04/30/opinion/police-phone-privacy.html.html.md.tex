Sections

SEARCH

\protect\hyperlink{site-content}{Skip to
content}\protect\hyperlink{site-index}{Skip to site index}

\href{https://myaccount.nytimes.com/auth/login?response_type=cookie\&client_id=vi}{}

\href{https://www.nytimes.com/section/todayspaper}{Today's Paper}

\href{/section/opinion}{Opinion}\textbar{}Would You Let the Police
Search Your Phone?

\url{https://nyti.ms/2DDU4HO}

\begin{itemize}
\item
\item
\item
\item
\item
\item
\end{itemize}

Advertisement

\protect\hyperlink{after-top}{Continue reading the main story}

\href{/section/opinion}{Opinion}

Supported by

\protect\hyperlink{after-sponsor}{Continue reading the main story}

\hypertarget{would-you-let-the-police-search-your-phone}{%
\section{Would You Let the Police Search Your
Phone?}\label{would-you-let-the-police-search-your-phone}}

We are much more likely to give consent than we think.

By Roseanna Sommers and Vanessa K. Bohns

Dr. Sommers is a lecturer at the University of Chicago Law School. Dr.
Bohns is an associate professor of organizational behavior at the School
of Industrial and Labor Relations at Cornell.

\begin{itemize}
\item
  April 30, 2019
\item
  \begin{itemize}
  \item
  \item
  \item
  \item
  \item
  \item
  \end{itemize}
\end{itemize}

\includegraphics{https://static01.nyt.com/images/2019/04/30/opinion/30sommersbohn-privacy/30sommersbohn-privacy-articleLarge.jpg?quality=75\&auto=webp\&disable=upscale}

Law enforcement officers on the doorstep threatening to ``come back with
a warrant'' is a cliché of police procedural dramas. Things are much
less dramatic in real life: The officers ask if they can take a look
around, and the civilians say yes without putting up a fight.

A key question in so-called
\href{http://ssrn.com/abstract=3369844}{``}consent\href{http://ssrn.com/abstract=3369844}{}-search\href{http://ssrn.com/abstract=3369844}{''}
cases is why people so readily agree to allow intrusions into their
privacy. The answer, as we argue in a
\href{http://ssrn.com/abstract=3369844}{forthcoming article} in The Yale
Law Journal, is that psychologically, it's much harder to refuse consent
than it seems. The degree of pressure needed to get people to comply is
shockingly minimal --- and our ability to recognize this fact is
limited.

The legal standard for whether a consent search is voluntary --- and
thus whether any contraband police discover is admissible in court ---
is whether a reasonable person would have felt free to refuse the
officers' request. Courts tend to judge the voluntariness of consent by
looking for clear markers of coercion. Did the officer phrase the
request as a demand, instead of a question? Were weapons drawn? If not,
the search is likely to be deemed voluntary.

But this approach misunderstands the psychology of compliance. It takes
much less pressure than it seems to secure people's acquiescence. Police
don't need to use weapons to get people to accede to their requests;
they just need to ask. Our research shows that a simple, polite
face-to-face request is harder to refuse than we think.

To test how the psychology of compliance operates, we recruited hundreds
of participants to the lab. We approached half of them with a polite but
audacious request: ``Before we begin the study, can you please unlock
your phone and hand it to me? I'll just need to take your phone outside
of the room for a moment to check for some things.''

For the other half, we asked what a reasonable person would do if
hypothetically approached by the same experimenter with the same
request.

Our control group, who merely imagined the interaction, said that most
people would refuse to hand over the phone: Only 14 percent thought a
reasonable person would let us search the phone, and only 28 percent
said they would yield their phone to a stranger. But when we actually
approached people, 97 percent handed over their phone.

In addition, these participants reported feeling significantly more
pressured to comply than the control group imagined feeling. The fact
is, saying no is more difficult, and rarer, than we realize. We believe
this same dynamic plays out the in the law surrounding consent searches.

To be sure, saying no to a police officer is different from a saying no
to an experimenter in a laboratory study. So we also tested whether
people underestimate the pressure to comply with the police as well.

We recruited a separate group of survey respondents and offered them a
monetary bonus if they could predict how often drivers grant consent
when stopped by the police. According to traffic data, upward of 90
percent of drivers say yes when the police ask to search their car. But
our survey respondents' average guess was far lower: They thought that
only about 65 percent of drivers say yes. Again, people vastly
understated compliance.

\emph{{[}As technology advances, will it continue to blur the lines
between public and
private?}\href{https://www.nytimes.com/newsletters/privacy-project?action=click\&module=inline\&pgtype=Article}{\emph{Sign
up for Charlie Warzel's limited-run newsletter}} \emph{to explore what's
at stake and what you can do about it.{]}}

This tendency to underappreciate the power of social influence is one of
the most enduring and important findings in all of social psychology. In
Stanley Milgram's famous studies on obedience, for instance, research
participants were willing to heed an experimenter's instructions to
administer dangerous electric shocks to an innocent, protesting victim.

Mr. Milgram showed that normal people would commit violent acts --- not
because they were sadists, but because they were loath to disobey an
authority figure's directives. This was a result that no one, including
expert psychologists, expected.

Critics of consent searches have also been approaching the issue in the
wrong way. Groups like the American Civil Liberties Union have focused
on advocating that the police be required to notify citizens of their
right to refuse consent, much as the police are required to read
custodial suspects their Miranda rights. But this is unlikely to address
the psychological factors at play.

\emph{{[}Technology has made our lives easier. But it also means that
your data is no longer your own. We'll examine who is hoarding your
information --- and give you a guide for what you can do about it.}
\href{https://www.nytimes.com/newsletters/privacy-project?action=click\&module=Intentional\&pgtype=Article}{\emph{Sign
up for our limited-run newsletter}}\emph{.{]}}

In another study, we tested what happens when we tell people they ``have
the right to refuse''the request to search their phone. We found that
this notification altered people's beliefs about the consequences of
refusal, but it did not change how free they felt to refuse. Nor did it
reduce the rates at which they handed over their phone to us --- a
result consistent with previous studies that have found negligible
effects of Miranda warnings on the rates at which suspects confess to
crimes.

The failure of ``know your rights'' interventions makes sense if you
think about the psychology behind police-citizen interactions. Telling
people about their rights addresses information deficits, but the real
reason people comply is social, not informational. The social
imperatives to comply with a police officer's request persist even when
people are properly informed of their rights or given a consent form to
sign --- or just asked politely.

Roseanna Sommers is a lecturer at the University of Chicago Law School.
Vanessa K. Bohns in an associate professor of organizational behavior at
the School of Industrial and Labor Relations at Cornell.

\emph{Follow}
\href{https://twitter.com/privacyproject}{\emph{@privacyproject}}
\emph{on Twitter and The New York Times Opinion Section on}
\href{https://www.facebook.com/nytopinion}{\emph{Facebook}}
\emph{and}\href{https://www.instagram.com/nytopinion/}{\emph{Instagram}}\emph{.}

\hypertarget{glossary-replacer}{%
\subsection{glossary replacer}\label{glossary-replacer}}

Advertisement

\protect\hyperlink{after-bottom}{Continue reading the main story}

\hypertarget{site-index}{%
\subsection{Site Index}\label{site-index}}

\hypertarget{site-information-navigation}{%
\subsection{Site Information
Navigation}\label{site-information-navigation}}

\begin{itemize}
\tightlist
\item
  \href{https://help.nytimes.com/hc/en-us/articles/115014792127-Copyright-notice}{©~2020~The
  New York Times Company}
\end{itemize}

\begin{itemize}
\tightlist
\item
  \href{https://www.nytco.com/}{NYTCo}
\item
  \href{https://help.nytimes.com/hc/en-us/articles/115015385887-Contact-Us}{Contact
  Us}
\item
  \href{https://www.nytco.com/careers/}{Work with us}
\item
  \href{https://nytmediakit.com/}{Advertise}
\item
  \href{http://www.tbrandstudio.com/}{T Brand Studio}
\item
  \href{https://www.nytimes.com/privacy/cookie-policy\#how-do-i-manage-trackers}{Your
  Ad Choices}
\item
  \href{https://www.nytimes.com/privacy}{Privacy}
\item
  \href{https://help.nytimes.com/hc/en-us/articles/115014893428-Terms-of-service}{Terms
  of Service}
\item
  \href{https://help.nytimes.com/hc/en-us/articles/115014893968-Terms-of-sale}{Terms
  of Sale}
\item
  \href{https://spiderbites.nytimes.com}{Site Map}
\item
  \href{https://help.nytimes.com/hc/en-us}{Help}
\item
  \href{https://www.nytimes.com/subscription?campaignId=37WXW}{Subscriptions}
\end{itemize}
