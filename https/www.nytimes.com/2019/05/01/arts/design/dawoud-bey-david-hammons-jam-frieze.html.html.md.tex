Sections

SEARCH

\protect\hyperlink{site-content}{Skip to
content}\protect\hyperlink{site-index}{Skip to site index}

\href{https://www.nytimes.com/section/arts/design}{Art \& Design}

\href{https://myaccount.nytimes.com/auth/login?response_type=cookie\&client_id=vi}{}

\href{https://www.nytimes.com/section/todayspaper}{Today's Paper}

\href{/section/arts/design}{Art \& Design}\textbar{}When Dawoud Bey Met
David Hammons

\href{https://nyti.ms/2J7NxbG}{https://nyti.ms/2J7NxbG}

\begin{itemize}
\item
\item
\item
\item
\item
\item
\end{itemize}

Advertisement

\protect\hyperlink{after-top}{Continue reading the main story}

Supported by

\protect\hyperlink{after-sponsor}{Continue reading the main story}

Frieze Week 2019

\hypertarget{when-dawoud-bey-met-david-hammons}{%
\section{When Dawoud Bey Met David
Hammons}\label{when-dawoud-bey-met-david-hammons}}

In the '80s, a portrait photographer coming into his own captured an
enigmatic artist creating his own legend. Rarely exhibited, the images
are here, and at Frieze New York.

\includegraphics{https://static01.nyt.com/images/2019/05/03/arts/02dawoudbey1/merlin_153599487_66d03505-0fc7-447f-aeaf-e71fd893fb95-articleLarge.jpg?quality=75\&auto=webp\&disable=upscale}

By Max Lakin

\begin{itemize}
\item
  May 1, 2019
\item
  \begin{itemize}
  \item
  \item
  \item
  \item
  \item
  \item
  \end{itemize}
\end{itemize}

In 1981, the artist
\href{http://www.mnuchingallery.com/artists/david-hammons}{David
Hammons} and the photographer
\href{https://www.nytimes.com/2018/12/24/lens/dawoud-bey-seeing-deeply.html}{Dawoud
Bey} found themselves at Richard Serra's
\href{https://www.publicartfund.org/view/exhibitions/5865_twu}{T.W.U.},
a hulking Corten steel monolith installed just the year before in a
pregentrified and sparsely populated TriBeCa. No one really knows the
details of what happened next, or if there were even details to know
aside from what Mr. Bey's images show: Mr. Hammons, wearing Pumas and a
dashiki, standing near the interior of the sculpture, its walls
graffitied and pasted over with fliers, urinating on it.

Another image shows Mr. Hammons presenting identification to a mostly
bemused police officer. Mr. Bey's images are funny and mysterious and
offer proof of something that came to be known as
\href{http://www.thegreatgodpanisdead.com/2013/08/art-out-in-world-jim-nolans-shifting.html}{``Pissed
Off''} and spoken about like a fable --- not exactly photojournalism,
but documentation of a certain Hammons mystique. It wasn't Mr. Hammons'
only act at the site, either. Another Bey image shows a dozen pairs of
sneakers Mr. Hammons lobbed over the Serra sculpture's steel lip,
turning it into something resolutely his own.

Soon after he arrived in New York, from Los Angeles, in 1974, Mr.
Hammons began his practice of creating work whose simplicity belied its
conceptual weight: sculptures rendered from the flotsam of the black
experience --- barbershop clippings and chicken wing bones and bottle
caps bent to resemble cowrie shells --- dense with symbolism and the
freight of history.

His actions, which some called performances, mostly for lack of a more
precise descriptor, were the spiritual stock of Marcel Duchamp and
Marcel Broodthaers --- wily and barbed ready-made sculptures, created by
inverting spent liquor bottles onto branches in empty lots, or slashing
open the backs of mink coats, or inviting people to an empty and unlit
gallery.

The practice for which Mr. Hammons is best known, perhaps, is his own
legend. Not much for holding still, and uninterested in accolades or
institutional attention, he cultivated an enigmatic persona predicated
as much on conceptual rigor as resistance to public life.

\includegraphics{https://static01.nyt.com/images/2019/05/02/arts/02dawoudbey2/merlin_153599478_d5ad0425-edee-4fa1-8b9e-85b289b6edba-articleLarge.jpg?quality=75\&auto=webp\&disable=upscale}

The ``Pissed Off'' images are several in a suite Mr. Bey made in New
York in the early '80s of Mr. Hammons and other artists as they floated
in and around Just Above Midtown, known as JAM, Linda Goode Bryant's
gallery devoted to contemporary African-American artists in a time when
few other institutions were providing such a platform. Mr. Bey's images
of Mr. Hammons, which are set to go on view this week in a special
section at Frieze devoted to JAM, are striking, not least because they
are rarely exhibited, but also because the total visual record of Mr.
Hammons and his work in New York is so spare.

Mr. Bey, who was born and raised in Queens, N.Y., and whose own practice
has been concerned with ideas of community and the continuum of black
life, met Mr. Hammons early in his tenure in New York. Mr.
\href{https://www.studiomuseum.org/exhibition/dawoud-beys-harlem-usa}{Bey
had recently begun photographing street life in Harlem} and was showing
his images at the Studio Museum of Harlem, where Ms. Bryant was working.
When she left to open JAM on 57th Street and Fifth Avenue, Mr. Bey fell
in with its circle of artists, many of whom, including Mr. Hammons,
\href{http://sengasenga.com/}{Senga Nengudi}, and
\href{https://hammer.ucla.edu/now-dig-this/artists/maren-hassinger/}{Maren
Hassinger}, had been working in Los Angeles.

``There are these deep connections she made that I don't think we saw in
commercial spaces,'' said Franklin Sirmans, director of the
\href{https://www.nytimes.com/2015/09/04/arts/design/perez-art-museum-miami-names-franklin-sirmans-as-new-director.html}{Pérez
Art Museum Miami} and curator of the tribute to Ms. Bryant and JAM at
Frieze. ``Dawoud being one of those younger artists she worked with
early on, and someone who obviously was already coming into his own. His
\href{https://www.nytimes.com/2018/12/24/lens/dawoud-bey-seeing-deeply.html}{Harlem
series} was shown at Studio Museum in 1979, so although young, he had a
presence. Thinking about the connection to his friend David Hammons,
whom he's also photographing, then you're talking not only about the
documentation of a friend and artist, but you're widening the circle.
And I think that's one of the big takeaways from Just Above Midtown,
that there was this incredible laboratory of ideas that was being
exchanged between different artists.''

Image

One of the rare formal portraits of David Hammons, under an arch of
empty bottles, a preferred material, taken by Mr. Bey in the artist's
Harlem studio in 1984.Credit...Dawoud Bey, Stephen Daiter Gallery and
Rena Bransten Gallery

Image

Portrait of Dawoud Bey. His images of Mr. Hammons are set to go on view
in a special section at Frieze that will highlight artists from Just
Above Midtown (JAM).Credit...Whitten Sabbatini

As that circle became more defined, it also developed a reflexive
support system. ``We were all part of that community,'' Mr. Bey, who is
65, said. ``So when JAM opened, we knew to show up. I don't know if any
pieces of mail ever went out or anything, you know, it's what you did.
You showed up and supported each other. And you show up at the same
place with the same people long enough, you get to know them, and you
become friends.''

Showing up meant Mr. Bey was usually present when Mr. Hammons unfurled
one of his actions. ``They were spontaneous, unannounced,'' the
photographer recalled. ``Which was the beautiful part about it --- it
wasn't a performance for the art world. He would say, `I think I'm going
to do something. Be at Cooper Square tomorrow, 12 o'clock,' you know,
and I'd say `Sure, man.' It was more about documenting our presence,
because, I thought, if we don't document ourselves, no one will.''

Two such documents concern Mr. Hammons rehearsing a dance piece at JAM.
In one, he and the video artist
\href{https://philipmalloryjones.com/}{Philip Mallory Jones} frame the
dancer and choreographer Bill T. Jones, barefoot and in mid-movement, as
Mr. Hammons is rapt with a folded piece of paper. In the other, the men
pose for Mr. Bey --- Mr. Mallory Jones in the middle, flanked by Mr.
Jones, shirtless, his face turned away and eyes closed in gentle repose.
Mr. Hammons looks directly into the camera, his gaze piercing the
surface, implicating the viewer.

Image

At left, the dancer and choreographer Bill T. Jones, the video artist
Philip Mallory Jones and David Hammons at Just Above Midtown/Downtown
Gallery, 1983. At right, Mr. Hammons and Mr. Mallory Jones frame Mr.
Jones in mid-movement. Credit...Dawoud Bey, Stephen Daiter Gallery and
Rena Bransten Gallery

Image

David Hammons's ``Higher Goals'' (1983), a towering basketball hoop
studded with metal bottle caps in Harlem. ``It was 121st Street and
Frederick Douglass Boulevard, but it's hard to go there now and
visualize that,'' Mr. Bey said. ``It was the Harlem of vacant lots,
which is where David was making a lot of his work.'' Credit...Dawoud
Bey, Stephen Daiter Gallery and Rena Bransten Gallery

Much of Mr. Hammons' work has anticipated the upheaval of urban life,
chiefly black urban life, in forms that collide symbols of race, class,
and wealth.
\href{https://aphelis.net/bliz-aard-ball-sale-david-hammons-1983/}{He
sold snowballs like bootleg luxury goods outside the Cooper Union,} 30
years before the historically free art school, overextended with
construction projects, began charging tuition. He raised
\href{https://www.publicartfund.org/view/exhibitions/5753_higher_goals}{30-foot-tall
basketball hoops studded with bottle caps in Harlem and Downtown
Brooklyn} before those neighborhoods were made smooth with glassy high
rises. \href{http://www.artcritical.com/2013/04/13/nyc-1993/}{He mounted
the hood of a sweatshirt lopped off from its body, like a mask or a
trophy,} in 1993 --- 20 years before that piece of clothing became a
\href{https://www.clydefitchreport.com/2013/07/trayvon-martin-david-hammons-and-how-to-think-about-hoods/}{charged
symbol of a reignited civil rights movement}. In all of it there's a
furious sense of social realism, oriented toward an audience that the
standard gallery and museum system wasn't capable, or willing, to
address.

``I was trying to remember where the first
\href{https://www.publicartfund.org/view/exhibitions/5753_higher_goals}{`Higher
Goals'} was placed,'' Mr. Bey said. ``It was 121st Street and Frederick
Douglass Boulevard, but it's hard to go there now and visualize that. It
was the Harlem of vacant lots, which is where David was making a lot of
his work. That work was of a moment. And of course who knew that? You
can't peer into the future. All you can do is make the work in the
circumstances you had.'' He paused. ``It was about doing something
meaningful at that moment, for the people who would be encountering
it.''

As alive as Mr. Hammons' work was to the fabric of society, he resisted
engaging in the art world machinery, becoming something of a benevolent
ghost. That gave his work the cast of the shamanic, even if its real
power was in the space between what was and wasn't visible. Mr. Bey's
images refocus that visibility, giving shape to a long-gone version of
New York, and to the ephemeral strands of Mr. Hammons' art, which are
discussed now in near-mythological terms.

Image

Mr. Hammons engaging with potential customers in ``Bliz-aard Ball Sale
II,'' 1983, which he set up alongside other sidewalk vendors to hawk
snowballs.Credit...Dawoud Bey, Stephen Daiter Gallery and Rena Bransten
Gallery

``It's like that whisper game that by the time it gets to you it's all
wrong,'' Mr. Bey said, laughing. ``There are very few people who can
provide the firsthand information about any of it. So people just start
filling in the blanks.''

He added, ``In David's case, it's because there was, for the sake of the
work, an understanding that you don't explain it. There were no news
releases. No yakety-yak. No theorizing. What happened before, where
those snowballs came from --- between David and I there's always been an
agreement: don't talk about it. That's part of the aura of the work. And
because David still probably doesn't have a telephone, and probably
wouldn't answer it if he did, it's up to me to at least put that much
out there, to be accountable to and for that history.''

The artist \href{https://www.theastergates.com/}{Theaster Gates} put it
this way: ``He ain't going to be at the parties, he's not going to be at
the openings, he's not going to be at the news conferences. And it means
then that he has more time. There's not very many that have had the
power of resistance that Hammons has had. And that resistance is not a
strategy, there's a powerful lesson that has something to do with the
right to be a maker. It's not about the participation or lack of
participation in the art world. ''

Mr. Bey agreed, recalling the opening of a gallery show in SoHo that
included work by Mr. Hammons. ``Everybody was so sure that David was
going to show up, and people were asking me, `you think he's going to
come?' And I'm like, what makes you think anything's changed? Standing
in the room so everyone can pat him on the back, it was never about
that. We were just doing what we do. We were just two friends making
stuff, you know? At the time, that's what it was. Now, you know, it's at
Frieze.''

\begin{center}\rule{0.5\linewidth}{\linethickness}\end{center}

\textbf{Dawoud Bey at Frieze New York}

Mr. Bey's photographs are at Frieze New York, Booth JAM7, Rena Bransten
Gallery and Stephen Daiter Gallery;
\href{https://frieze.com/fairs/frieze-new-york}{frieze.com}.

\textbf{David Hammons in Los Angeles}

Mr. Hammon's first solo show in Los Angeles in 45 years goes on view May
18 through Aug. 11 at Hauser \& Wirth, 901 East Third Street, Los
Angeles;
\href{https://www.hauserwirth.com/hauser-wirth-exhibitions/24162-david-hammons-los-angeles}{hauserwirth.com}.

Advertisement

\protect\hyperlink{after-bottom}{Continue reading the main story}

\hypertarget{site-index}{%
\subsection{Site Index}\label{site-index}}

\hypertarget{site-information-navigation}{%
\subsection{Site Information
Navigation}\label{site-information-navigation}}

\begin{itemize}
\tightlist
\item
  \href{https://help.nytimes.com/hc/en-us/articles/115014792127-Copyright-notice}{©~2020~The
  New York Times Company}
\end{itemize}

\begin{itemize}
\tightlist
\item
  \href{https://www.nytco.com/}{NYTCo}
\item
  \href{https://help.nytimes.com/hc/en-us/articles/115015385887-Contact-Us}{Contact
  Us}
\item
  \href{https://www.nytco.com/careers/}{Work with us}
\item
  \href{https://nytmediakit.com/}{Advertise}
\item
  \href{http://www.tbrandstudio.com/}{T Brand Studio}
\item
  \href{https://www.nytimes.com/privacy/cookie-policy\#how-do-i-manage-trackers}{Your
  Ad Choices}
\item
  \href{https://www.nytimes.com/privacy}{Privacy}
\item
  \href{https://help.nytimes.com/hc/en-us/articles/115014893428-Terms-of-service}{Terms
  of Service}
\item
  \href{https://help.nytimes.com/hc/en-us/articles/115014893968-Terms-of-sale}{Terms
  of Sale}
\item
  \href{https://spiderbites.nytimes.com}{Site Map}
\item
  \href{https://help.nytimes.com/hc/en-us}{Help}
\item
  \href{https://www.nytimes.com/subscription?campaignId=37WXW}{Subscriptions}
\end{itemize}
