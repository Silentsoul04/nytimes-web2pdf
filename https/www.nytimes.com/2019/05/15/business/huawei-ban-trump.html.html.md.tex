Sections

SEARCH

\protect\hyperlink{site-content}{Skip to
content}\protect\hyperlink{site-index}{Skip to site index}

\href{https://www.nytimes.com/section/business}{Business}

\href{https://myaccount.nytimes.com/auth/login?response_type=cookie\&client_id=vi}{}

\href{https://www.nytimes.com/section/todayspaper}{Today's Paper}

\href{/section/business}{Business}\textbar{}Huawei Is a Target as Trump
Moves to Ban Foreign Telecom Gear

\url{https://nyti.ms/2Jld1U7}

\begin{itemize}
\item
\item
\item
\item
\item
\item
\end{itemize}

Advertisement

\protect\hyperlink{after-top}{Continue reading the main story}

Supported by

\protect\hyperlink{after-sponsor}{Continue reading the main story}

\hypertarget{huawei-is-a-target-as-trump-moves-to-ban-foreign-telecom-gear}{%
\section{Huawei Is a Target as Trump Moves to Ban Foreign Telecom
Gear}\label{huawei-is-a-target-as-trump-moves-to-ban-foreign-telecom-gear}}

\includegraphics{https://static01.nyt.com/images/2019/05/15/us/15dc-Huawei/merlin_154642896_870cd601-8204-4a8f-b607-e54aa3c8b2cc-articleLarge.jpg?quality=75\&auto=webp\&disable=upscale}

By \href{https://www.nytimes.com/by/cecilia-kang}{Cecilia Kang} and
\href{https://www.nytimes.com/by/david-e-sanger}{David E. Sanger}

\begin{itemize}
\item
  May 15, 2019
\item
  \begin{itemize}
  \item
  \item
  \item
  \item
  \item
  \item
  \end{itemize}
\end{itemize}

\href{https://cn.nytimes.com/business/20190516/huawei-ban-trump/}{阅读简体中文版}\href{https://cn.nytimes.com/business/20190516/huawei-ban-trump/zh-hant/}{閱讀繁體中文版}

WASHINGTON --- President Trump moved on Wednesday to ban American
telecommunications firms from installing foreign-made equipment that
could pose a threat to national security, White House officials said,
stepping up a battle against China by effectively barring sales by
Huawei, the country's leading networking company.

Mr. Trump issued an executive order instructing the commerce secretary,
Wilbur Ross, to ban transactions ``posing an unacceptable risk'' but did
not single out any nation or company. The action has long been expected
and is the
\href{https://www.nytimes.com/2019/01/26/us/politics/huawei-china-us-5g-technology.html}{latest
salvo in the administration's economic and security battle with China}.
It is also the most extreme move in the Trump administration's fight
against China's tech sector.

The executive order was ``agnostic,'' White House officials said in a
call with reporters, declining to single out China as the focus. ``This
administration will do what it takes to keep America safe and prosperous
and to protect America from foreign adversaries'' targeting
vulnerabilities in American communications infrastructure, the White
House press secretary, Sarah Huckabee Sanders, said in a statement.

But in a clear strike against Huawei, the Commerce Department separately
announced on Wednesday that it had placed the company and its dozens of
affiliates on a list of firms deemed a risk to national security. The
listing will prevent it from buying American parts and technologies
without seeking United States government approval.

``This will prevent American technology from being used by foreign owned
entities in ways that potentially undermine U.S. national security or
foreign policy interests,'' Mr. Ross said in a statement.

\emph{{[}Read more about how the order takes a shot directly at}
\href{https://www.nytimes.com/2019/05/16/technology/huawei-ban-president-trump.html}{\emph{Huawei}}\emph{'s
business.{]}}

The Commerce Department will also write the rules for reviewing
transactions that fall under the executive order's ban over the next 150
days, the officials said. The department said it would work across the
administration on the new rules, consulting with the attorney general,
Treasury secretary and other agency heads.

The order, which applies only to future transactions, left many
questions unanswered, including how the department will define foreign
adversaries and establish criteria to ban companies from selling
equipment to the United States. The executive action did not address
concerns by rural carriers that the order would hit them particularly
hard. Some of them rely on equipment that already contains parts by
Huawei and other Chinese companies.

Mr. Trump declared the threat posed by foreign adversaries on American
telecommunications networks a national emergency under a law used to
impose sanctions against nations like Iran and Russia.

Led by Secretary of State Mike Pompeo, American officials have warned
allies for months that the United States would stop sharing intelligence
if they use Huawei and other Chinese technology to build the core of
their fifth-generation, or 5G, networks. The networks promise not only
faster cellular service, but also the connection of billions of
``internet of things'' devices --- such as autonomous cars, security
cameras and industrial equipment --- to a new internet architecture.

Pentagon and American intelligence officials have warned that Chinese
firms will be able to control the networks and have expressed concerns
not only that secure messages could be intercepted or secretly diverted
to China, but that the Chinese authorities could order Huawei to shut
down the networks during any conflict, disrupting American
infrastructure as diverse as gas pipelines and cellphone networks.

Huawei has denied those charges, and its chief executive has said he
would shut down the company rather than obey Chinese government orders
to intercept or divert internet traffic. American officials say he would
have no choice: Chinese law requires that the country's firms obey
instructions from the nation's Ministry of State Security.

Mr. Trump has been deeply involved, quietly meeting with American
telecommunications executives at the White House and weighing different
versions of the executive order. He has insisted that the United States
must ``win'' at the 5G competition, only to be told that no American
firms make the core switches that will direct 5G internet traffic.

The executive order came amid an escalating trade war between the United
States and China, with the two sides imposing hundreds of billions of
dollars of tariffs in recent days. Mr. Trump has accused the Chinese
government of unfair trade practices and
\href{https://www.nytimes.com/2019/05/09/us/politics/china-trade-tariffs.html}{announced
increased tariffs} on an additional \$200 billion worth of Chinese goods
last week.

\includegraphics{https://static01.nyt.com/images/2019/05/15/us/15dc-huawei-2/merlin_154823778_1980ffd1-6a12-4144-910f-858841046e02-articleLarge.jpg?quality=75\&auto=webp\&disable=upscale}

But few issues have gained as much bipartisan support in Washington as
the Trump administration's warnings of the security threats posed by
Huawei and ZTE, another company with deep links to the Chinese
government. The calls have gotten so intense that some have warned of a
new red scare, and Chinese officials have said the United States has
moved beyond caution to paranoia.

On Wednesday, Senator Mark Warner of Virginia, the top Democrat on the
Senate Intelligence Committee, called the order ``a needed step'' that
``reflects the reality that Huawei and ZTE represent a threat to the
security of U.S. and allied communications networks.''

He pushed for further action. ``We have yet to see a compelling strategy
from this administration on 5G,'' he said in a statement. ``A coherent
coordinated and global approach is critically needed.''

The major mobile phone companies have renounced the use of Huawei
equipment in their 5G systems, but the order will ensure that smaller,
rural telecom companies avoid Huawei when they build new networks.

Because 5G systems feature high-speed, low-range equipment, the
technology is more suited for now for urban areas than rural areas. But
Huawei remains appealing to rural networks because it is cheaper than
alternatives.

The ban could also help with the Trump administration's campaign to get
European allies to block Huawei. Some allies had questioned why they
should block Huawei if the United States had not. Other European
officials have suspected that Mr. Trump will soften or eliminate the ban
as part of a trade deal with China. Most major allies have resisted the
Trump administration's push, except Australia, which banned Huawei last
summer.

The White House, intelligence officials and lawmakers from both parties
argue that China has already shaped its telecommunications and that tech
industries have also given rise inside Chinese territory to facial
recognition, constant surveillance of the population and human rights
abuses.

American officials have also warned that China's exports of Huawei and
other tech products have allowed other authoritarian nations to spy on
their citizens and access sensitive security and trade secrets.

``We must have a cleareyed view of the threats that we face and be
prepared to do what is necessary to counter those threats,'' Ajit Pai,
the chairman of the Federal Communications Commission, said in a
statement. ``Today's executive order does just that.''

But even if Huawei is banned from the United States, it will likely
control 40 to 60 percent of the networks around the world. It has made a
strong marketing pitch in Africa, Latin America and parts of Asia where
it holds huge economic influence. American officials have said China has
offered subsidized prices and low-interest loans to outmaneuver the few
Western competitors, chiefly Nokia and Ericsson, both European firms.

The United States will have to connect to those nations --- and must
prepare for a day when the American government and companies will have
to live in ``dirty networks,'' Sue Gordon, the deputy director of
national intelligence, recently warned.

In January, prosecutors in Washington State
\href{https://www.justice.gov/opa/pr/chinese-telecommunications-device-manufacturer-and-its-us-affiliate-indicted-theft-trade}{charged}
two units of Huawei of conspiring to steal trade secrets from T-Mobile
and of wire fraud.

Senator Lindsey Graham, Republican of South Carolina and the chairman of
the Senate Judiciary Committee, said he doubted that Chinese companies
could meet American standards and laws on surveillance.

``There is no way in hell China can meet those criteria because of the
way they're governed,'' Mr. Graham said at a hearing this week. ``The
only way China can meet the criteria is to stop being China.''

The Federal Communications Commission is considering regulations that
would bar broadband providers that receive federal subsidies from using
Huawei or ZTE equipment in their networks. It also
\href{https://www.fcc.gov/document/fcc-denies-china-mobile-telecom-services-application}{recently
approved} an order that barred China Mobile Limited from providing
service in the United States. The agency said it was exploring similar
rules against China Unicom and China Telecom Corporation.

Advertisement

\protect\hyperlink{after-bottom}{Continue reading the main story}

\hypertarget{site-index}{%
\subsection{Site Index}\label{site-index}}

\hypertarget{site-information-navigation}{%
\subsection{Site Information
Navigation}\label{site-information-navigation}}

\begin{itemize}
\tightlist
\item
  \href{https://help.nytimes.com/hc/en-us/articles/115014792127-Copyright-notice}{©~2020~The
  New York Times Company}
\end{itemize}

\begin{itemize}
\tightlist
\item
  \href{https://www.nytco.com/}{NYTCo}
\item
  \href{https://help.nytimes.com/hc/en-us/articles/115015385887-Contact-Us}{Contact
  Us}
\item
  \href{https://www.nytco.com/careers/}{Work with us}
\item
  \href{https://nytmediakit.com/}{Advertise}
\item
  \href{http://www.tbrandstudio.com/}{T Brand Studio}
\item
  \href{https://www.nytimes.com/privacy/cookie-policy\#how-do-i-manage-trackers}{Your
  Ad Choices}
\item
  \href{https://www.nytimes.com/privacy}{Privacy}
\item
  \href{https://help.nytimes.com/hc/en-us/articles/115014893428-Terms-of-service}{Terms
  of Service}
\item
  \href{https://help.nytimes.com/hc/en-us/articles/115014893968-Terms-of-sale}{Terms
  of Sale}
\item
  \href{https://spiderbites.nytimes.com}{Site Map}
\item
  \href{https://help.nytimes.com/hc/en-us}{Help}
\item
  \href{https://www.nytimes.com/subscription?campaignId=37WXW}{Subscriptions}
\end{itemize}
