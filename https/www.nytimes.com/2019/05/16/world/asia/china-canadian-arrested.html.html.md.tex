Sections

SEARCH

\protect\hyperlink{site-content}{Skip to
content}\protect\hyperlink{site-index}{Skip to site index}

\href{https://www.nytimes.com/section/world/asia}{Asia Pacific}

\href{https://myaccount.nytimes.com/auth/login?response_type=cookie\&client_id=vi}{}

\href{https://www.nytimes.com/section/todayspaper}{Today's Paper}

\href{/section/world/asia}{Asia Pacific}\textbar{}China Arrests 2
Canadians on Spying Charges, Deepening a Political Standoff

\url{https://nyti.ms/2YvX0hw}

\begin{itemize}
\item
\item
\item
\item
\item
\end{itemize}

Advertisement

\protect\hyperlink{after-top}{Continue reading the main story}

Supported by

\protect\hyperlink{after-sponsor}{Continue reading the main story}

\hypertarget{china-arrests-2-canadians-on-spying-charges-deepening-a-political-standoff}{%
\section{China Arrests 2 Canadians on Spying Charges, Deepening a
Political
Standoff}\label{china-arrests-2-canadians-on-spying-charges-deepening-a-political-standoff}}

\includegraphics{https://static01.nyt.com/images/2019/05/16/world/16china-canada-3/merlin_148030671_f6161c5a-90f0-4782-9820-203bf76d2558-articleLarge.jpg?quality=75\&auto=webp\&disable=upscale}

By \href{https://www.nytimes.com/by/chris-buckley}{Chris Buckley},
\href{https://www.nytimes.com/by/javier-c-hernandez}{Javier C.
Hernández} and \href{https://www.nytimes.com/by/dan-bilefsky}{Dan
Bilefsky}

\begin{itemize}
\item
  May 16, 2019
\item
  \begin{itemize}
  \item
  \item
  \item
  \item
  \item
  \end{itemize}
\end{itemize}

BEIJING --- Two Canadian men detained in China since December have been
formally arrested on espionage charges, the Chinese Foreign Ministry
said on Thursday, a move likely to ratchet up tensions between China and
Canada that broke out with the arrest of a Chinese tech executive in
Vancouver.

Michael Kovrig, a former diplomat who was
\href{https://www.nytimes.com/2018/12/11/world/asia/michael-kovrig-china-canada.html}{detained
while visiting Beijing}, was arrested on suspicion of ``gathering state
secrets and intelligence for abroad,'' and Michael Spavor, a business
consultant who was
\href{https://www.nytimes.com/2018/12/12/world/asia/michael-spavor-canadian-detained-china.html}{detained
in northeastern China}, was accused of ``stealing and providing state
secrets for abroad,'' Lu Kang, a spokesman for the foreign ministry,
said at a regularly scheduled news briefing.

The vague reference to unspecified overseas entities left open the
question of whether the men were suspected of working for a government
or for some other organization.

Mr. Lu did not provide further details and said only that the arrests
had been made recently.

``Everything in China is done in accordance with law,'' Mr. Lu said.

Responding to a reporter's question about Canadian officials' criticism
of how China handled the cases, he said: ``We hope Canada will not
interfere with or comment casually on China's legal system and lawful
practices.''

Canada's prime minister, Justin Trudeau, criticized the initial
detentions of Mr. Kovrig and Mr. Spavor
\href{https://www.nytimes.com/2018/12/12/world/asia/michael-spavor-canadian-detained-china.html}{as
``arbitrary''} and
\href{https://www.citynews1130.com/2019/01/10/trudeau-detention-canadians-politically-motivated/}{politically
motivated}; on Thursday, Global Affairs Canada, the country's foreign
ministry, denounced their arrests.

``Canada strongly condemns their arbitrary arrest as we condemned their
arbitrary detention,'' Brittany Fletcher, a spokeswoman for Global
Affairs Canada, said in an emailed statement. ``Canada continues to
express its appreciation to those who have spoken in support of these
detained Canadians and the rule of law.''

Supporters of the two men and foreign legal experts have said their
detentions appeared to be retaliation for the arrest in Canada in
December of Meng Wanzhou, the chief financial officer of Huawei, China's
biggest telecommunications company.

Ms. Meng was arrested at the request of the United States, which wants
to extradite her on fraud charges.

Mr. Kovrig and Mr. Spavor were seized by the police in December, days
after
\href{https://www.nytimes.com/2018/12/05/business/huawei-cfo-arrest-canada-extradition.html}{Ms.
Meng was arrested} while changing planes in Vancouver. The Chinese
government was incensed by Ms. Meng's arrest, and the charging of Mr.
Kovrig and Mr. Spavor makes it more likely that they will face trial and
conviction, deepening the standoff with Mr. Trudeau's administration.

Many in Canada have reacted with consternation at China's treatment of
the two Canadians, who have been denied access to lawyers and been
confined in secret detention centers, without visits from family
members. Canadian diplomats have been allowed to visit them about once a
month.

Image

Michael KovrigCredit...Julie David de Lossy/Crisis Group

Mr. Lu, the foreign ministry spokesman, did not respond to questions
Thursday about where the two men were being held.

Their circumstances are a striking contrast to those of Ms. Meng, who is
\href{https://www.nytimes.com/2019/05/08/world/canada/huawei-meng-wanzhou-extradition.html}{out
on bail}, has been living in Vancouver in a six-bedroom home valued at 6
million Canadian dollars, and is free to roam largely about the city
with a GPS tracker on her ankle.

In addition to the diplomatic tensions, the Meng case and detentions of
the two Canadians have fueled economic tensions between the two
countries.

China, which bought about \$2.7 billion worth of canola oil from Canada
last year, recently halted shipments of Canadian canola oil, saying they
were contaminated.

According to the
\href{https://ccbc.com/ccbc-canada-china-business-survey-2018-2019/}{Canada
China Business Council}, which promotes business, trade and investment
between Canada and China, 20 percent of Canadian companies have been
adversely affected by the dispute as demand from China for Canadian
products shrinks.

A former Canadian ambassador to China,
\href{https://www.ualberta.ca/china-institute/about/people/senior-fellows/guy-saint-jacques}{Guy
Saint-Jacques,} said the formal arrest of the two men signaled a
worsening in relations with China and would make it even more
challenging for Canada to secure their release.

``In 95 percent of cases the accused are found guilty, and this formal
arrest signals the gravity of the situation,'' he said. ``We are in for
a long period of difficulties with China in which pressure from the
Chinese will increase.''

Mr. Saint-Jacques said that while China's economic might outweighed
Canada's, Canada could retaliate by taking action against China at the
World Trade Organization over the canola issue, and by rallying its
allies internationally to make it clear to China that its actions
against the two Canadians came at a price.

But he also added that the hope by some in Canada that improved trade
relations between the United States and China could help clear the way
for the men's eventual release had diminished given the intensifying
friction between the two countries.

Human rights advocates on Thursday denounced the arrests of Mr. Kovrig
and Mr. Spavor.

``Their cases show again how the Chinese criminal system violates the
human rights of detainees,'' said Patrick Poon, a researcher for Amnesty
International in Hong Kong. He called on Chinese officials to release
the men, absent ``credible and concrete evidence'' of crimes.

Before the latest announcement, Chinese officials had signaled that Mr.
Kovrig and Spavor
\href{https://www.nytimes.com/2019/03/04/world/asia/china-canada-michael-kovrig-huawei.html}{could
be charged with espionage offenses}.

Mr. Kovrig worked for the United Nations and the Canadian foreign
service before 2017, when he
\href{https://www.crisisgroup.org/who-we-are/people/michael-kovrig}{joined
the International Crisis Group}, a nonprofit organization that tries to
defuse conflicts between states.

Image

Michael SpavorCredit...Associated Press

He focused on Chinese foreign policy, Asian regional politics and North
Korea, and was often quoted in foreign news outlets and invited to
meetings in China.

Mr. Spavor was a consultant for companies and people interested in North
Korea,
\href{https://www.macleans.ca/news/world/kim-jong-un-meet-the-nhl/}{including
Dennis Rodman}, the former basketball star who has befriended the
North's leader, Kim Jong-un.

Mr. Spavor was detained in Dandong, the Chinese city on the North Korean
border where he was based.

In early March, a
\href{https://www.nytimes.com/2019/03/04/world/asia/china-canada-michael-kovrig-huawei.html}{legal
affairs committee} within China's ruling Communist Party said
investigators believed that Mr. Kovrig had been ``stealing and spying to
obtain state secrets and intelligence,'' and that Mr. Spavor had
supplied him with information.

But China's definition of state secrets can be sweeping and opaque, and
the International Crisis Group has said Mr. Kovrig's work for it was in
no way nefarious.

``Nothing Michael did was harmful to China,'' the group's president,
Robert Malley, said on Thursday. ``On the contrary, his work helped
inform both China's global policies and those around the world who make
policies toward China in a manner that contributes to preventing and
resolving conflict.''

The cases of Ms. Meng and the two Canadians are playing out as the
United States is pressuring allies not to use Huawei's technology,
arguing that China could use it to spy on other countries.

Those efforts intensified on Wednesday, when President Trump
\href{https://www.nytimes.com/2019/05/15/business/huawei-ban-trump.html?rref=collection\%2Ftimestopic\%2FChina\&action=click\&contentCollection=world\&region=stream\&module=stream_unit\&version=latest\&contentPlacement=1\&pgtype=collection}{moved
to ban} American telecommunications companies from installing
foreign-made equipment that could pose risks to national security. The
measure seemed aimed at blocking sales by Huawei, though it did not
explicitly single out any nation or company.

Canada's security agencies have been undertaking a national security
review to determine whether
\href{https://www.nytimes.com/2019/02/27/world/canada/huawei-5g-meng-wanzhou-china.html}{Huawei's
technology} should be used as Canada develops its 5G telecommunications
network.

In January, American prosecutors
\href{https://www.nytimes.com/2019/01/28/us/politics/meng-wanzhou-huawei-iran.html?module=inline}{indicted}
Ms. Meng and Huawei, laying out what they said were efforts by the
company to steal commercial secrets, obstruct a criminal inquiry and
engage in bank fraud while trying to evade American sanctions on Iran.

Huawei has denied breaking the law, and the Chinese government has
repeatedly said that the charges were politically driven.

The Chinese foreign minister, Wang Yi,
\href{https://www.caixinglobal.com/2019-03-08/beijing-backs-huaweis-lawsuit-against-us-government-101389531.html}{said
at a news conference} in early March that the case against Huawei and
Ms. Meng was ``by no means a purely judicial case, but rather a
deliberate political case'' intended to bring down Huawei.

Mr. Trudeau and Canadian and United States officials have said that the
case against Ms. Meng is a
\href{https://www.nytimes.com/2019/01/23/world/canada/canada-huawei-china-ambassador.html}{legal
matter}, not a political one. But President Trump
\href{https://www.nytimes.com/2018/12/12/us/politics/trump-meng-wanzhou-huawei-extradition.html}{veered
from that position} in December, when he suggested that he could
intervene in the case if that helped to seal a trade agreement with
China.

Advertisement

\protect\hyperlink{after-bottom}{Continue reading the main story}

\hypertarget{site-index}{%
\subsection{Site Index}\label{site-index}}

\hypertarget{site-information-navigation}{%
\subsection{Site Information
Navigation}\label{site-information-navigation}}

\begin{itemize}
\tightlist
\item
  \href{https://help.nytimes.com/hc/en-us/articles/115014792127-Copyright-notice}{©~2020~The
  New York Times Company}
\end{itemize}

\begin{itemize}
\tightlist
\item
  \href{https://www.nytco.com/}{NYTCo}
\item
  \href{https://help.nytimes.com/hc/en-us/articles/115015385887-Contact-Us}{Contact
  Us}
\item
  \href{https://www.nytco.com/careers/}{Work with us}
\item
  \href{https://nytmediakit.com/}{Advertise}
\item
  \href{http://www.tbrandstudio.com/}{T Brand Studio}
\item
  \href{https://www.nytimes.com/privacy/cookie-policy\#how-do-i-manage-trackers}{Your
  Ad Choices}
\item
  \href{https://www.nytimes.com/privacy}{Privacy}
\item
  \href{https://help.nytimes.com/hc/en-us/articles/115014893428-Terms-of-service}{Terms
  of Service}
\item
  \href{https://help.nytimes.com/hc/en-us/articles/115014893968-Terms-of-sale}{Terms
  of Sale}
\item
  \href{https://spiderbites.nytimes.com}{Site Map}
\item
  \href{https://help.nytimes.com/hc/en-us}{Help}
\item
  \href{https://www.nytimes.com/subscription?campaignId=37WXW}{Subscriptions}
\end{itemize}
