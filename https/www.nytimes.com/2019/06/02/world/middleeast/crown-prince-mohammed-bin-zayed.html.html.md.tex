\href{/section/world/middleeast}{Middle East}\textbar{}The Most Powerful
Arab Ruler Isn't M.B.S. It's M.B.Z.

\url{https://nyti.ms/2EMnuE4}

\begin{itemize}
\item
\item
\item
\item
\item
\item
\end{itemize}

\includegraphics{https://static01.nyt.com/images/2019/05/30/world/xxmbz1/xxmbz1-articleLarge.jpg?quality=75\&auto=webp\&disable=upscale}

Sections

\protect\hyperlink{site-content}{Skip to
content}\protect\hyperlink{site-index}{Skip to site index}

\hypertarget{the-most-powerful-arab-ruler-isnt-mbs-its-mbz}{%
\section{The Most Powerful Arab Ruler Isn't M.B.S. It's
M.B.Z.}\label{the-most-powerful-arab-ruler-isnt-mbs-its-mbz}}

Prince Mohammed bin Zayed expanded the U.A.E.'s power by following
America's lead. He now has an increasingly bellicose agenda of his own.
And President Trump seems to be following him.

As de facto ruler of the United Arab Emirates, Prince Mohammed bin Zayed
controls the Arab world's biggest sovereign wealth funds and its most
potent military.Credit...Pool photo by Andrew Caballero-Reynolds

Supported by

\protect\hyperlink{after-sponsor}{Continue reading the main story}

By \href{https://www.nytimes.com/by/david-d-kirkpatrick}{David D.
Kirkpatrick}

\begin{itemize}
\item
  June 2, 2019
\item
  \begin{itemize}
  \item
  \item
  \item
  \item
  \item
  \item
  \end{itemize}
\end{itemize}

ABU DHABI, United Arab Emirates --- Prince Mohammed bin Zayed, the
29-year-old commander of the almost negligible air force of the United
Arab Emirates, had come to Washington shopping for weapons.

In 1991, in the months after Iraq's invasion of Kuwait, the young prince
wanted to buy so much military hardware to protect his own oil-rich
monarchy --- from Hellfire missiles to Apache helicopters to F-16 jets
--- that Congress worried he might destabilize the region.

But the Pentagon, trying to cultivate accommodating allies in the Gulf,
had identified Prince Mohammed as a promising partner. The favorite son
of the semi-literate Bedouin who founded the United Arab Emirates,
Prince Mohammed was a serious-minded, British-trained helicopter pilot
who had persuaded his father to transfer \$4 billion into the United
States Treasury to help pay for the 1991 war in Iraq.

Richard A. Clarke, then an assistant secretary of state, reassured
lawmakers that the young prince would never become ``an aggressor.''

``The U.A.E. is not now and never will be a threat to stability or peace
in the region,'' Mr. Clarke said in congressional testimony. ``That is
very hard to imagine. Indeed, the U.A.E. is a force for peace.''

Thirty years later, Prince Mohammed, now 58, crown prince of Abu Dhabi
and de facto ruler of the United Arab Emirates, is arguably the most
powerful leader in the Arab world. He is also among the most influential
foreign voices in Washington, urging the United States to adopt his
increasingly bellicose approach to the region.

{[}\emph{Here are}
\href{https://www.nytimes.com/2019/06/02/world/middleeast/prince-mohammed-bin-zayed.html}{\emph{five
takeaways from our report}} \emph{on Prince Mohammed.}{]}

Prince Mohammed is almost unknown to the American public and his tiny
country has fewer citizens than Rhode Island. But he may be the richest
man in the world. He controls sovereign wealth funds worth \$1.3
trillion, more than any other country.

His influence operation in Washington is legendary (Mr. Clarke got rich
on his payroll). His military is the Arab world's most potent, equipped
though its work with the United States to conduct
\href{https://www.nytimes.com/2018/08/31/world/middleeast/hacking-united-arab-emirates-nso-group.html}{high-tech
surveillance} and combat operations far beyond its borders.

\includegraphics{https://static01.nyt.com/images/2019/06/02/world/02MBZ/merlin_155639433_75312a24-fb71-4dc1-941a-39b313e6ea13-articleLarge.jpg?quality=75\&auto=webp\&disable=upscale}

For decades, the prince has been a key American ally, following
Washington's lead, but now he is going his own way. His special forces
are active in Yemen, Libya, Somalia and Egypt's North Sinai. He has
worked to thwart democratic transitions in the Middle East, helped
install a reliable autocrat in Egypt and boosted a protégé to power in
Saudi Arabia.

At times, the prince has contradicted American policy and destabilized
neighbors. Rights groups have criticized him for jailing dissidents at
home, for his role in creating
\href{https://www.nytimes.com/interactive/2018/10/26/world/middleeast/saudi-arabia-war-yemen.html}{a
humanitarian crisis in Yemen}, and for backing the Saudi prince whose
agents killed the dissident writer Jamal Khashoggi.

Yet under the Trump administration, his influence in Washington appears
greater than ever. He has a rapport with President Trump, who has
frequently adopted the prince's views on Qatar, Libya and Saudi Arabia,
even over the advice of cabinet officials or senior national security
staff.

Western diplomats who know the prince --- known as M.B.Z. --- say he is
obsessed with two enemies, Iran and
\href{https://www.nytimes.com/2019/05/10/world/middleeast/trump-muslim-brotherhood.html}{the
Muslim Brotherhood}. Mr. Trump has sought to move strongly against both
and last week took steps to
\href{https://www.nytimes.com/2019/05/23/us/politics/trump-saudi-arabia-arms-sales.html}{bypass
congressional opposition} to keep selling weapons to both Saudi Arabia
and the United Arab Emirates.

``M.B.Z. has an extraordinary way of telling Americans his own interests
but making it come across as good advice about the region,'' said Ben
Rhodes, a deputy national security adviser under President Barack Obama,
whose sympathy for the Arab Spring and negotiations with Iran brought
blistering criticism from the Emirati prince. When it comes to influence
in Washington, Mr. Rhodes added, ``M.B.Z. is in a class by himself.''

Prince Mohammed worked assiduously before the presidential election to
crack Mr. Trump's inner circle, and secured a secret meeting during the
transition period with the president's son-in-law, Jared Kushner. The
prince also tried to broker talks between the Trump administration and
Russia, a gambit that later entangled him in the special counsel's
investigation into foreign election interference.

Image

President Trump welcoming Prince Mohammed at the White House in
2017.Credit...Al Drago/The New York Times

Today, at least five people working for Prince Mohammed have been caught
up in criminal investigations growing out of that inquiry. A regular
visitor to the United States for three decades, Prince Mohammed has now
stayed away for two years, in part because he fears prosecutors might
seek to question him or his aides, according to two people familiar with
his thinking. (His brother, the foreign minister, has visited.)

The United Arab Emirates' Embassy in Washington declined to comment. The
prince's many American defenders say it is only prudent of him to try to
shape United States policy, as many governments do, and that he sees his
interventions as an attempt to compensate for an American pullback.

But Prince Mohammed's critics say that his rise is a study in unintended
consequences. The obscure young prince whom Washington adopted as a
pliant ally is now fanning his volatile region's flames.

By arming the United Arab Emirates with such advanced surveillance
technology, commandos and weaponry, argued Tamara Cofman Wittes, a
former State Department official and fellow at the Brookings
Institution. ``We have created a little Frankenstein.''

Image

Prince Mohammed has overseen a construction boom in the Emirati capital,
Abu Dhabi.Credit...Hamad I Mohammed/Reuters

\hypertarget{the-perfect-prince}{%
\subsection{The Perfect Prince}\label{the-perfect-prince}}

Most Arab royals are paunchy, long-winded and prone to keep visitors
waiting. Not Prince Mohammed.

He graduated at the age of 18 from the British officers' training
program at Sandhurst. He stays slim and fit, trades tips with visitors
about workout machines, and never arrives late for a meeting.

American officials invariably describe him as concise, inquisitive, even
humble. He pours his own coffee, and to illustrate his love for America,
sometimes tells visitors that he has taken his grandchildren to Disney
World incognito.

He makes time for low-ranking American officials and greets senior
dignitaries at the airport. With a shy, lopsided smile, he will offer a
tour of his country, then climb into a helicopter to fly his guest over
the skyscrapers and lagoons of Dubai and Abu Dhabi.

``There was always a `wow' factor with M.B.Z.,'' recalled Marcelle
Wahba, a former American ambassador to the United Arab Emirates.

In the capital, Abu Dhabi, he has overseen a construction craze that has
hidden the former coastline behind man-made islands. One is intended to
become a financial district akin to Wall Street. Another includes a
campus of New York University,
\href{https://www.nytimes.com/2019/03/30/arts/design/salvator-mundi-louvre-abu-dhabi.html}{a
franchise of the Louvre} and a planned extension of the Guggenheim.

When he meets Americans, Prince Mohammed emphasizes the things that make
the United Arab Emirates more liberal than their neighbors. Women have
more opportunities: A third of the cabinet ministers are female.

Unlike Saudi Arabia, the United Arab Emirates allow Christian churches
and Hindu or Sikh temples, partly to accommodate a vast foreign work
force. (The country is estimated to have nine million residents, but
fewer than a million citizens; the rest are foreign workers.)

To underscore the point, the prince last year created a Ministry of
Tolerance and declared this the ``Year of Tolerance.'' He has hosted the
Special Olympics and Pope Francis.

Image

Pope Francis celebrated Mass at the Zayed Sports City Stadium in Abu
Dhabi in February.Credit...Ali Haider/EPA, via Shutterstock

``I think he has done admirable work not just in diversifying the
economy but in diversifying the system of thought of the population as
well,'' said Gen. John R. Allen, former commander of United States and
N.A.T.O. forces in Afghanistan, now president of the Brookings
Institution. (In between, General Allen was an adviser to the United
Arab Emirates' Ministry of Defense.)

The United Arab Emirates are a tiny federation of city-states, yet Abu
Dhabi alone accounts for 6 percent of the world's proven oil reserves,
making it a tempting target to a larger neighbor like Iran. In 1971,
when the U.A.E. gained independence from Britain, the shah of Iran
seized three disputed Persian Gulf islands.

The Muslim Brotherhood, a 90-year-old Islamist movement founded in
Egypt,
\href{https://www.nytimes.com/2019/04/30/world/middleeast/is-the-muslim-brotherhood-terrorist.html}{has
become mainstream} in many Arab countries. On that subject, Prince
Mohammed says his dread is more personal.

His father assigned a prominent Brotherhood member, Ezzedine Ibrahim, as
Prince Mohammed's tutor, and he attempted an indoctrination that
backfired, the prince often says.

``I am an Arab, I am a Muslim and I pray. And in the 1970s and early
1980s I was one of them,'' Prince Mohammed told visiting American
diplomats in 2007, as they reported in a classified cable released by
WikiLeaks. ``I believe these guys have an agenda.''

He worries about the appeal of Islamist politics to his population. As
many as 80 percent of the soldiers in his forces would answer the call
of ``some holy man in Mecca,'' he once told American diplomats,
according to a cable released by WikiLeaks.

For that reason, diplomats say, Prince Mohammed has long argued that the
Arab world is not ready for democracy. Islamists would win any
elections.

``In any Muslim country, you will see the same result,'' he said in a
2007 meeting with American officials. ``The Middle East is not
California.''

The United Arab Emirates began allowing American forces to operate from
bases inside the country during the Persian Gulf war of 1991. Since
then, the prince's commandos and air forces have been deployed with the
Americans in Kosovo, Somalia, Afghanistan and Libya, as well as against
the Islamic State.

Image

A demonstration by members of the U.A.E. during the opening of the
International Defence Exhibition \& Conference in Abu Dhabi in
February.Credit...Christopher Pike/Reuters

He has recruited American commanders to run his military and former
spies to set up his intelligence services. He also acquired more
weaponry in the four years before 2010 than the other five Gulf
monarchies combined, including 80 F-16 fighters, 30 Apache combat
helicopters, and 62 French Mirage jets.

Some American officers describe the United Arab Emirates as ``Little
Sparta.''

With advice from former top military commanders including former
Secretary of Defense James Mattis and General Allen, Prince Mohammed has
even developed an Emirati defense industry, producing an amphibious
armored vehicle known as The Beast and others that he is already
supplying to clients in Libya and Egypt.

The United Arab Emirates are also preparing a low-altitude
propeller-driven bomber for counterinsurgency combat --- an idea Mr.
Mattis had long recommended for the United States, a former officer
close to him said.

Prince Mohammed has often told American officials that he saw Israel as
an ally against Iran and the Muslim Brotherhood. Israel trusted him
enough to sell him upgrades for his F-16s, as well as advanced mobile
phone spyware.

To many in Washington, Prince Mohammed had become America's best friend
in the region, a dutiful partner who could be counted on for tasks from
countering Iranian influence in Lebanon to funding construction in Iraq.

``It was well known that if you needed something done in the Middle
East,'' recalled Richard G. Olson, a former United States ambassador to
Abu Dhabi, ``the Emiratis would do it.''

Image

President Barack Obama welcoming Prince Mohammed at the White House in
2015.Credit...Doug Mills/The New York Times

\hypertarget{a-prince-goes-rogue}{%
\subsection{A Prince Goes Rogue}\label{a-prince-goes-rogue}}

Prince Mohammed seemed to find a kindred spirit when President Barack
Obama took office in 2009, White House aides said. Both were detached,
analytic and intrigued by big questions. For a time, Mr. Obama sought
out phone conversations with Prince Mohammed more than with any other
foreign leader, several senior White House officials recalled.

But the Arab Spring came between them. Uprisings swept the region. The
Muslim Brotherhood was winning elections. And Mr. Obama appeared to
endorse the demands for democracy --- though in Syria, where the
uprising threatened a foe of the Emiratis, he balked at military action.

500 mileS

IRAQ

IRAN

UNITED ARAB

EMIRATES

EGYPT

SAUDI

ARABIA

OMAN

SUDAN

YEMEN

Arabian

Sea

ETH.

DJIBOUTI

iran

100 mileS

Dubai

QATAR

Abu Dhabi

United Arab

Emirates

oman

saudi arabia

By The New York Times

Then it emerged that the Obama administration was in secret nuclear
talks with Iran.

``They felt not only ignored --- they felt betrayed by the Obama
administration, and I think Prince Mohammed felt it particularly and
personally,'' said Stephen Hadley, a national security adviser under
President George W. Bush who has stayed close to the prince.

After the uprisings, Prince Mohammed saw the United Arab Emirates as the
only one of the 22 Arab states still on its feet, with a stable
government, functional economy, able military and ``moderate ideology,''
said Abdulkhalleq Abdulla, an Emirati political scientist with access to
the country's senior officials.

``The U.A.E. is part of this very dangerous region that is getting more
dangerous by the day --- full of chaos and wars and extremists,'' he
said. ``So the motivation is this: If we don't go after the bad guys,
they will come after us.''

Image

Tahrir Square in Cairo in 2012. Mr. Obama's sympathy for the Arab Spring
drew blistering criticism from the Emirati prince.Credit...Moises Saman
for The New York Times

At home, Prince Mohammed hired a company linked to Erik Prince, the
founder of the private security company formerly known as Blackwater,
\href{https://www.nytimes.com/2011/05/15/world/middleeast/15prince.html}{to
create a force} of Colombian, South African and other mercenaries. He
crushed any hint of dissent, arresting five activists for organizing a
petition for democratic reforms (signed by only 132 people) and rounding
up dozens suspected of sympathizing with the Muslim Brotherhood.

The United Arab Emirates revved up its influence machine in Washington,
too. They were among the biggest spenders among foreign governments on
Washington advocates and consultants, paying as much \$21 million in
2017, according to a tally by the Center for Responsive Politics. They
earned good will with million-dollar donations after natural disasters,
and they sought to frame public debate by giving millions more to major
think tanks.

The Middle East Institute recently received \$20 million. Its chairman
is Mr. Clarke, the former official who pushed through the U.A.E. defense
contracts. After leaving government in 2003, he had also founded a
consultancy with the United Arab Emirates as a primary client. He did
not respond to requests for comment.

Emirati Ambassador Yousef Otaiba hammered his many contacts in the White
House and on Capitol Hill, arguing that Mr. Obama was ceding the region
to extremists and Iran. The prince himself made the case at the highest
levels. He ``gave me an earful,'' former Secretary of Defense Robert
Gates recalled in a memoir.

In the Middle East, Prince Mohammed did more than talk. In Egypt, he
backed
\href{https://www.nytimes.com/2018/07/27/sunday-review/obama-egypt-coup-trump.html}{a
military takeover in 2013 that removed an elected president who was a
Muslim Brotherhood} leader. In the Horn of Africa, he dispatched a force
to Somalia first to combat piracy and then to fight extremists. He went
on to establish commercial ports or naval bases around the Gulf of Aden.

In Libya, Prince Mohammed defied American pleas and a United Nations
embargo by
\href{https://www.nytimes.com/2015/11/13/world/middleeast/leaked-emirati-emails-could-threaten-peace-talks-in-libya.html}{arming
the forces} of the militia leader and would-be strongman Khalifa Hifter.
Emirati pilots carried out airstrikes in Tripoli and eventually
established an air base in eastern Libya.

In the past, the prince looked for a ``green light'' from Washington,
said Ms. Wahba, the former American ambassador. Now he may send a
heads-up, she said, but ``he is not asking permission anymore.''

Saudi Arabia, the giant next door, had quarreled with the United Arab
Emirates over borders and, as the regional heavyweight, also constrained
U.A.E. foreign policy. By the end of 2014, the position of crown prince
--- next in line for the throne --- had passed to a known foe of the
Emirati prince.

Image

The Saudi crown prince, Mohammed bin Salman, right, with Prince Mohammed
in Abu Dhabi last year.Credit...Bandar Al-Jaloud/Saudi Royal Palace, via
Agence France-Presse --- Getty Images

So he plunged into the internal Saudi succession battle and waged an
all-out lobbying campaign in Washington on behalf of a little-known
alternative: the 29-year-old Prince Mohammed bin Salman, a favorite son
of the aged Saudi king.

``M.B.Z.'s message was, if you trust me and you like me, you will like
this guy because he is cut from the same cloth,'' recalled Mr. Rhodes,
the Obama adviser.

By March 2015, the two princes had invaded Yemen together to roll back a
takeover by a faction aligned with Iran. Then in 2017, as the Saudi
prince consolidated his power, they
\href{https://www.nytimes.com/2018/01/22/world/middleeast/qatar-saudi-emir-boycott.html}{cut
off all trade and diplomatic ties with Qatar} to pressure it into
abandoning support for the Muslim Brotherhood.

Both the Yemen and Qatar conflicts are routinely described as Saudi-led,
but the Emirati prince first sought to sell them to Washington, Mr.
Rhodes and other former officials recalled.

By late 2015, American diplomats say, Prince Mohammed was also
suggesting that the United Arab Emirates and a new Saudi leadership
could be crucial in bringing the Palestinians around to some new peace
agreement --- the so-called ``outside-in'' approach to a deal.

But for that, Prince Mohammed awaited a new administration.

Image

The Russian businessman Kirill Dmitriev acts as a liaison between
President Vladimir V. Putin and the Persian Gulf monarchs, according to
the special counsel's report.Credit...Fayez Nureldine/Agence
France-Presse --- Getty Images

\hypertarget{all-the-princes-men}{%
\subsection{All the Prince's Men}\label{all-the-princes-men}}

It was meant to be a personal farewell.

Despite their sharp differences, Prince Mohammed had remained cordial
with Mr. Obama, and the president thought they shared a mutual respect,
according to four senior White House officials. So when the prince
requested a final meeting, as friends, Mr. Obama agreed to a lunch at
the White House in December 2016.

But Prince Mohammed backed out without much explanation. He flew instead
to New York for his first face-to-face meeting with Jared Kushner and
other advisers to the president-elect, Donald J. Trump.

To arrange the meetings, Prince Mohammed had turned to a financier,
Richard Gerson, founder of Falcon Edge Capital. He had worked with the
prince for years, and he was also a friend of Mr. Kushner.

``I am always here as your trusted family back channel any time you want
to discreetly pass something,'' Mr. Gerson wrote to the prince after the
election in a private text message, one of several provided to The Times
by a third party and corroborated independently. He signed off another
message as ``your loyal soldier.''

The trip was supposed to be secret, but intelligence agencies detected
the prince's arrival. Mr. Obama's advisers were stunned. But Prince
Mohammed was already working to reverse the administration's policies,
talking to Mr. Trump's advisers about the dangers of Iran and about
Palestinian peace talks, according to two people familiar with the
meetings.

``They were deeply impressed with you and already are convinced that you
are their true friend and closest ally,'' Mr. Gerson wrote to the prince
after the meetings.

Prince Mohammed was positioning himself as an intermediary to Russia,
too.

One of Prince Mohammed's younger brothers had introduced Mr. Gerson to a
Russian businessman who acts as a liaison between President Vladimir V.
Putin and the Persian Gulf monarchs, according to the special counsel's
report. The Russian businessman, Kirill Dmitriev, conferred with Mr.
Gerson about a ``reconciliation plan'' for the United States and Russia,
and shortly before the inauguration Mr. Gerson gave a two-page summary
of the plan to Mr. Kushner.

Mr. Gerson declined to comment for this article.

The next month, in January, Prince Mohammed invited Mr. Dmitriev to an
Emirati retreat in the Seychelles to meet with someone else they thought
represented the Trump team: Mr. Prince, the Blackwater founder who had
recruited mercenaries for the United Arab Emirates.

Image

Prince Mohammed hired an American security company linked to Erik Prince
to create a security force of mercenaries.Credit...Zach Gibson for The
New York Times

Why Prince Mohammed would seek to connect Russia with Mr. Trump's circle
remains a matter of debate, but he has worked for years to try to entice
Mr. Putin away from Iran, according to American diplomats and leaked
emails from the Emirati ambassador in Washington.

But prosecutors are also investigating the activities of other
operatives and go-betweens working for the prince who
\href{https://www.nytimes.com/2018/05/19/us/politics/trump-jr-saudi-uae-nader-prince-zamel.html}{tried
to insinuate themselves around Mr. Trump}.

Investigators are still examining the campaign contacts of
\href{https://www.nytimes.com/2018/10/08/us/politics/rick-gates-psy-group-trump.html}{an
Israeli specialist in social media manipulation} who has worked for
Prince Mohammed and of
\href{https://www.nytimes.com/2018/04/04/us/politics/george-nader-russia-uae-special-counsel-investigation.html}{a
Lebanese-American businessman} who
\href{https://www.nytimes.com/2018/03/21/us/politics/george-nader-elliott-broidy-uae-saudi-arabia-white-house-influence.html}{acted
as his emissary}. Other prosecutors are investigating whether another
top Republican donor whose security company worked for the prince should
legally have registered as his agent.

The special counsel's office has also questioned Rashid al-Malik, an
Emirati real-estate developer based in Los Angeles who is close to
Prince Mohammed and to his brother --- the head of Emirati intelligence.
Mr. al-Malik is also close to Mr. Trump's friend Tom Barrack, and
investigators are asking whether Mr. al-Malik was part of an illegal
influence scheme, according to two people familiar with the matter.

Another investigation, prompted by a whistle-blower, is examining the
possibility that the United Arab Emirates used cyberespionage techniques
from former American operatives to spy on American citizens.

Yet the prince's courtship of the Trump administration has not been
damaged. In the two and a half years since his first meeting with Mr.
Kushner, Prince Mohammed has received almost everything he sought from
the White House.

Image

President Abdel Fattah al-Sisi of Egypt and Prince Mohammed in Cairo
last year.Credit...Egyptian Presidency, via Reuters

\hypertarget{a-prince-undaunted}{%
\subsection{A Prince Undaunted}\label{a-prince-undaunted}}

Each winter, Prince Mohammed invites financiers and former officials to
Abu Dhabi for a salon that demonstrates his global influence.

The guest list last December included former British Prime Minister Tony
Blair; former French President Nicolas Sarkozy; former Secretary of
State Condoleezza Rice; Mr. Hadley, the Bush-era national security
adviser; the American investors Mohamed A. El-Erian, David M. Rubenstein
and Thomas S. Kaplan; and the Chinese computer scientist and investor
Kai-Fu Lee.

Undeterred, the prince also included Mr. Dmitriev, the Russian
businessman linked to Mr. Putin.

Prince Mohammed's post-Arab Spring interventions have hardly stabilized
the region. An aide he sent to Cairo to help turn around the moribund
economy has returned in frustration.

Egypt's military-backed government still depends on billions of dollars
a year in assistance from the United Arab Emirates and its Gulf allies,
and despite Emirati help and Israeli airstrikes, Cairo has not yet
quelled a militant backlash centered in the North Sinai.

The isolation of Qatar has failed to change its policies. In Libya,
Khalifa Hifter is mired in a bloody stalemate.

Prince Mohammed's push in the Horn of Africa has set off a competition
for access and influence among rivals like Turkey and Qatar. In Somalia,
after allegations of bribery by the fragile central government, Emirati
forces have shifted to the semiautonomous regions of Puntland and
Somaliland.

Djibouti, alleging neglect, last year replaced its Emirati port managers
with a Chinese rival.

``He thinks he is Machiavelli but he acts more like Mussolini,'' said
Bruce Riedel, a scholar at the Brookings Institution and a former
official in the Central Intelligence Agency.

In Saudi Arabia, the Emirati prince has been embarrassed by the
conclusion of American intelligence agencies that his Saudi protégé had
ordered the brutal murder of Mr. Khashoggi, a Virginia-based Saudi
dissident and Washington Post columnist. Their joint, four-year-old
intervention in Yemen is turning into a quagmire, with horrific civilian
casualties.

Image

A tribute to the Saudi dissident Jamal Khashoggi in Istanbul last
year.Credit...Emrah Gurel/Associated Press

``The U.A.E. is a stain on the world conscience --- the U.A.E. as it is
currently governed is violating every norm of the civilized world,''
said Representative Ro Khanna, Democrat of California.

Yet the prince's standing remains strong inside the Trump
administration. The ``outside-in'' proposals for Israeli-Palestinian
peace passed over by the Obama administration are at the core of Mr.
Kushner's emerging plans.

Mr. Trump has repeatedly backed the positions of the Emirati prince: by
endorsing his Saudi protégé after the Khashoggi killing, by applauding
the isolation of Qatar even as the secretary of state and secretary of
defense publicly opposed it, by canceling the nuclear deal with Iran, by
seeking to designate the Muslim Brotherhood a terrorist group, and by
\href{https://www.nytimes.com/2019/04/16/us/politics/trump-veto-yemen.html}{vetoing
legislation} to cut off American military support for Saudi and Emirati
forces in Yemen.

In April, Mr. Trump publicly endorsed the Emiratis' favored militia
leader in Libya one day after a phone call with Prince Mohammed --- even
through Secretary of State Mike Pompeo had previously urged the same
leader to retreat.

Mr. Mattis, the former secretary of defense, last month delivered a
lecture in Abu Dhabi sponsored by Prince Mohammed. When he joined the
Trump administration, Mr. Mattis disclosed that he had received
\$242,000 in annual fees as well as valuable stock options as a board
member at the defense contractor General Dynamics, which does extensive
business with Abu Dhabi. He had also worked as an unpaid adviser to
Prince Mohammed.

``It's the Year of Tolerance. How many countries in the world right now
are having a year of tolerance?'' Mr. Mattis asked. ``I don't know of
any,'' he said. ``You are an example.''

Image

Jim Mattis, the former United States secretary of defense, in Abu Dhabi
in May.Credit...Eissa Al Hammadi/Saudi Press Agency, via Associated
Press

Advertisement

\protect\hyperlink{after-bottom}{Continue reading the main story}

\hypertarget{site-index}{%
\subsection{Site Index}\label{site-index}}

\hypertarget{site-information-navigation}{%
\subsection{Site Information
Navigation}\label{site-information-navigation}}

\begin{itemize}
\tightlist
\item
  \href{https://help.nytimes.com/hc/en-us/articles/115014792127-Copyright-notice}{©~2020~The
  New York Times Company}
\end{itemize}

\begin{itemize}
\tightlist
\item
  \href{https://www.nytco.com/}{NYTCo}
\item
  \href{https://help.nytimes.com/hc/en-us/articles/115015385887-Contact-Us}{Contact
  Us}
\item
  \href{https://www.nytco.com/careers/}{Work with us}
\item
  \href{https://nytmediakit.com/}{Advertise}
\item
  \href{http://www.tbrandstudio.com/}{T Brand Studio}
\item
  \href{https://www.nytimes.com/privacy/cookie-policy\#how-do-i-manage-trackers}{Your
  Ad Choices}
\item
  \href{https://www.nytimes.com/privacy}{Privacy}
\item
  \href{https://help.nytimes.com/hc/en-us/articles/115014893428-Terms-of-service}{Terms
  of Service}
\item
  \href{https://help.nytimes.com/hc/en-us/articles/115014893968-Terms-of-sale}{Terms
  of Sale}
\item
  \href{https://spiderbites.nytimes.com}{Site Map}
\item
  \href{https://help.nytimes.com/hc/en-us}{Help}
\item
  \href{https://www.nytimes.com/subscription?campaignId=37WXW}{Subscriptions}
\end{itemize}
