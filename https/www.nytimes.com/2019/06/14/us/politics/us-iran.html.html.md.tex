Sections

SEARCH

\protect\hyperlink{site-content}{Skip to
content}\protect\hyperlink{site-index}{Skip to site index}

\href{https://www.nytimes.com/section/politics}{Politics}

\href{https://myaccount.nytimes.com/auth/login?response_type=cookie\&client_id=vi}{}

\href{https://www.nytimes.com/section/todayspaper}{Today's Paper}

\href{/section/politics}{Politics}\textbar{}The U.S. Has Turned Up
Pressure on Iran. See the Timeline of Events.

\url{https://nyti.ms/2XhnAhn}

\begin{itemize}
\item
\item
\item
\item
\item
\end{itemize}

Advertisement

\protect\hyperlink{after-top}{Continue reading the main story}

Supported by

\protect\hyperlink{after-sponsor}{Continue reading the main story}

\hypertarget{the-us-has-turned-up-pressure-on-iran-see-the-timeline-of-events}{%
\section{The U.S. Has Turned Up Pressure on Iran. See the Timeline of
Events.}\label{the-us-has-turned-up-pressure-on-iran-see-the-timeline-of-events}}

\includegraphics{https://static01.nyt.com/images/2019/06/16/us/politics/16DC-IRAN-EXPLAINER/merlin_153232410_aa2bcbd6-ef32-4eec-ae8e-f0b56c7e32e3-articleLarge.jpg?quality=75\&auto=webp\&disable=upscale}

By \href{https://www.nytimes.com/by/helene-cooper}{Helene Cooper}

\begin{itemize}
\item
  June 14, 2019
\item
  \begin{itemize}
  \item
  \item
  \item
  \item
  \item
  \end{itemize}
\end{itemize}

\emph{Updated June 20, 2019}

WASHINGTON --- Iran's shooting down of an American surveillance
\href{https://www.nytimes.com/2019/06/20/us/politics/trump-iran.html}{drone}
on Thursday and the attacks on two tankers in the Gulf of Oman last week
have again sent the United States and
\href{https://www.nytimes.com/2019/06/17/us/politics/iran-nuclear-deal-uranium.html}{Iran},
two longtime adversaries,
\href{https://www.nytimes.com/2019/06/14/us/politics/trump-iran-tanker-hormuz.html}{hurtling
toward potential crisis}. The challenges are diplomatic and economic as
well as military.

But that course was set a year ago, foreign policy experts say, when
\href{https://www.nytimes.com/2019/06/20/us/politics/trump-iran.html}{President
Trump}, enforcing his ``maximum pressure'' campaign against Tehran,
withdrew the United States from an Obama-era agreement meant to rein in
\href{https://www.nytimes.com/2019/06/17/us/politics/iran-nuclear-deal-uranium.html}{Iran's
nuclear} ambitions.

``Iran was getting repeatedly punched in the face by the
\href{https://www.nytimes.com/2019/06/20/us/politics/trump-iran.html}{Trump
administration}, and they've been warning for months there will be
consequences,'' said Karim Sadjadpour, an Iran expert at the Carnegie
Endowment for International Peace. ``The Iranian economy has long been
riddled by endemic mismanagement, corruption, cronyism, and brain drain.
Sanctions makes all these problems worse.''

Here's a look at how the United States turned up the pressure on Iran.

\begin{center}\rule{0.5\linewidth}{\linethickness}\end{center}

\hypertarget{may-8-2018}{%
\subsection{May 8, 2018}\label{may-8-2018}}

Mr. Trump made good on his campaign promise, and announced that he was
\href{https://www.nytimes.com/2018/05/08/world/middleeast/trump-iran-nuclear-deal.html}{withdrawing
the United States} from the Iran nuclear deal, dismantling the signature
foreign policy achievement of President Barack Obama. Mr. Trump said
that the United States would reimpose the stringent sanctions it had
on\href{https://www.nytimes.com/2019/06/17/us/politics/iran-nuclear-deal-uranium.html}{Iran}
before the deal and would consider new sanctions against it.

The nuclear deal had tightly restricted
\href{https://www.nytimes.com/2019/06/20/us/politics/trump-iran.html}{Iran's}
nuclear ambitions in return for ending sanctions that had crippled its
economy.

The other signatories to the deal, France, Germany, Britain, Russia and
China, said they would remain in the agreement. But Mr. Trump put allies
on notice that European companies would face American sanctions if they
did business with Iran, and would have to choose between the United
States and Iran.

\includegraphics{https://static01.nyt.com/images/2019/06/14/us/politics/14dc-iran-explainer2/merlin_156318225_c760e4d1-9d48-427a-8bcc-31321ec38cc2-articleLarge.jpg?quality=75\&auto=webp\&disable=upscale}

``This was a horrible one-sided deal that should have never, ever been
made,''
\href{https://www.nytimes.com/2018/05/08/us/politics/trump-speech-iran-deal.html}{Mr.
Trump said}. The decision drew a public rebuke from Mr. Obama, who
warned that the American withdrawal would leave the world less safe,
forced to face ``a losing choice between a nuclear-armed Iran or another
war in the Middle East.''

\hypertarget{response}{%
\subsubsection{\texorpdfstring{\textbf{Response}}{Response}}\label{response}}

President Hassan Rouhani of Iran said that his country would continue to
abide by the terms of the deal. He criticized Mr. Trump for pulling out
of the pact as well as other international treaties.

\begin{center}\rule{0.5\linewidth}{\linethickness}\end{center}

\hypertarget{april-8-2019}{%
\subsection{April 8, 2019}\label{april-8-2019}}

Mr. Trump, over the objections of Pentagon officials, announced that he
was designating a powerful arm of the Iranian military, its Islamic
Revolutionary Guards Corps, as a
\href{https://www.nytimes.com/2019/04/08/world/middleeast/trump-iran-revolutionary-guard-corps.html}{foreign
terrorist organization}.

It was the first time that the United States named part of another
country's government as that type of official threat. The designation
imposed wide-ranging economic and travel sanctions on the group, which
carries out operations across the Middle East, trains Arab Shiite
militias and oversee businesses in Iran. The sanctions also applied to
organizations, companies and individuals with ties to the Revolutionary
Guards.

The designation, which went into effect April 15, ``will significantly
expand the scope and scale of our maximum pressure on the Iranian
regime,'' Mr. Trump said in a statement.

Gen. Joseph F. Dunford, the chairman of the Joint Chiefs of Staff,
opposed the move, arguing that it would allow Iranian leaders to justify
operations against Americans overseas, especially Special Operations
units and paramilitary units working under the C.I.A. But Secretary of
State Mike Pompeo and John R. Bolton, the national security adviser,
pushed for it.

\hypertarget{response-1}{%
\subsubsection{\texorpdfstring{\textbf{Response}}{Response}}\label{response-1}}

Iran's Supreme National Security Council said it was designating as a
terrorist organization the United States Central Command, which oversees
American military operations in the Middle East.

American officials said they began seeing stepped-up threats against
their forces in the region, as well as reports that Iranian-backed
Shiite militias were considering attacks on American troops in Iraq.

Meanwhile, Iran's oil exports fell by more than half since Mr. Trump
announced he was pulling out of the nuclear deal, to under one million
barrels a day.

\begin{center}\rule{0.5\linewidth}{\linethickness}\end{center}

\hypertarget{may-5-2019}{%
\subsection{May 5, 2019}\label{may-5-2019}}

Image

The aircraft carrier Abraham Lincoln was directed to the Middle East
over a perceived threat from Iran.Credit...Jon Gambrell/Associated Press

Mr. Bolton announced that the United States was
\href{https://www.nytimes.com/2019/05/05/world/middleeast/us-iran-military-threat-.html}{sending
an aircraft carrier strike group} and Air Force bombers to the Middle
East because of ``troubling and escalatory indications and warnings''
related to Iran.

In a statement released on a Sunday night, Mr. Bolton said that the
deployment was intended to ``send a clear and unmistakable message to
the Iranian regime that any attack on United States interests or on
those of our allies will be met with unrelenting force.''

\hypertarget{response-2}{%
\subsubsection{Response}\label{response-2}}

Four oil tankers
\href{https://www.nytimes.com/2019/05/13/world/middleeast/saudi-arabia-oil-tanker-sabotage.html}{were
attacked} in the Persian Gulf a week later; Mr. Pompeo and Mr. Bolton
said Iran was responsible. Two of the tankers belonged to Saudi Arabia
and one to the United Arab Emirates; both are adversaries of Iran and
allies of the United States. The fourth tanker is owned by a Norwegian
company.

\begin{center}\rule{0.5\linewidth}{\linethickness}\end{center}

\hypertarget{may-8-2019}{%
\subsection{May 8, 2019}\label{may-8-2019}}

One year after announcing he was pulling out of the nuclear deal, Mr.
Trump placed a
\href{https://www.nytimes.com/2019/05/08/us/politics/iran-nuclear-deal.html}{new
round of sanctions on Iran}, targeting the country's metals industry,
its largest source of nonpetroleum revenue.

``Today's action targets Iran's revenue from the export of industrial
metals --- 10 percent of its export economy --- and puts other nations
on notice that allowing Iranian steel and other metals into your ports
will no longer be tolerated,'' Mr. Trump said in a statement.

Image

Inside an Iranian steel complex, the industry hit by new American
sanctions.Credit...Vahid Salemi/Associated Press

Around that time, the Trump administration also hit Iranian oil exports,
by
\href{https://www.nytimes.com/2019/04/22/world/middleeast/us-iran-oil-sanctions-.html}{effectively
ordering} countries worldwide to stop buying oil from the country or
face sanctions of their own. Brian Hook, the State Department's special
envoy for Iran, said the United States would not grant waivers on
sanctions to any countries that buy Iranian oil.

``Iran has a choice,'' Tim Morrison, a White House arms control adviser,
told a conference organized by a Washington-based think tank. ``At some
point, even the mullahs will get it.''

\hypertarget{response-3}{%
\subsubsection{\texorpdfstring{\textbf{Response}}{Response}}\label{response-3}}

Iran ramped up its
\href{https://www.nytimes.com/2019/06/10/us/politics/iran-iaea-nuclear-fuel.html}{production
of nuclear fuel}, following through on a threat to begin walking away
from the nuclear deal.

\begin{center}\rule{0.5\linewidth}{\linethickness}\end{center}

\hypertarget{may-24-2019}{%
\subsection{May 24, 2019}\label{may-24-2019}}

Mr. Trump ordered 1,500
\href{https://www.nytimes.com/2019/05/24/world/middleeast/trump-troop-increase-middle-east-iran.html}{additional
troops to the Middle East} to increase protection of American forces
already there. He also circumvented Congress and declared an emergency
over Iran, moving ahead with arms sales to Saudi Arabia, the United Arab
Emirates and Jordan that had been blocked by Congress since last year.

\hypertarget{response-4}{%
\subsubsection{\texorpdfstring{\textbf{Response}}{Response}}\label{response-4}}

Gen. Frank McKenzie, the head of United States Central Command,
\href{https://www.navytimes.com/news/your-navy/2019/06/08/carrier-sends-message-to-iran/}{told
reporters traveling with him aboard the aircraft carrier} Abraham
Lincoln in the Persian Gulf this month that because the United States
deployed the carrier, Iran may now be ``recalculating'' previous plans
to attack American interests in the region.

``They are looking hard at the carrier because they know we are looking
hard at them,'' Gen. McKenzie said.

\begin{center}\rule{0.5\linewidth}{\linethickness}\end{center}

\hypertarget{june-13-2019}{%
\subsection{June 13, 2019}\label{june-13-2019}}

Explosions crippled two oil tankers in the Gulf of Oman on Thursday; the
United States military released video footage that Central Command said
shows a Revolutionary Guards patrol boat pulling up alongside one of the
stricken ships several hours after the initial explosion, and removing
an unexploded limpet mine in broad daylight. The video, American
officials said, proves that Iran was behind the attack.

\includegraphics{https://static01.nyt.com/images/2019/06/14/video/tanker2/tanker2-videoSixteenByNineJumbo1600-v2.jpg}

\hypertarget{response-5}{%
\subsubsection{\texorpdfstring{\textbf{Response}}{Response}}\label{response-5}}

Crude oil prices rose more than 3 percent, indirectly bolstering Iran's
revenue as an oil producer.

\begin{center}\rule{0.5\linewidth}{\linethickness}\end{center}

\hypertarget{june-17-2019}{%
\subsection{June 17, 2019}\label{june-17-2019}}

The Pentagon
\href{https://www.nytimes.com/2019/06/17/world/middleeast/iran-nuclear-deal-compliance.html}{authorized
the deployment} of an additional 1,000 troops to the Middle East,
bringing with them surveillance assets, missile batteries and fighter
jets described as ``deterrence capabilities.''

\hypertarget{response-6}{%
\subsubsection{Response}\label{response-6}}

Hours before the troop announcement in Washington, Iran said that it was
within 10 days of violating the 2015 containment deal. Its stockpile of
low-enriched uranium was set to exceed what was originally authorized.

\begin{center}\rule{0.5\linewidth}{\linethickness}\end{center}

\hypertarget{june-20-2019}{%
\subsection{June 20, 2019}\label{june-20-2019}}

In the early morning hours, an Iranian surface-to-air missile
\href{https://www.nytimes.com/2019/06/20/world/middleeast/iran-us-drone.html}{slammed
into} a Navy RQ-4 Global Hawk, an unpiloted surveillance drone, flying
over the Strait of Hormuz at high altitude. Iran immediately accused the
United States of patrolling over its territorial waters. The Pentagon
called the strike ``unprovoked'' and said that the aircraft was well
within international airspace. Both Washington and Tehran released
images trying to support their claims.

\hypertarget{where-the-us-and-iran-said-drone-was-shot-down}{%
\subsection{Where the U.S. and Iran Said Drone Was Shot
Down}\label{where-the-us-and-iran-said-drone-was-shot-down}}

Source: U.S. Department of Defense, Iranian Foreign Ministry

\hypertarget{response-7}{%
\subsubsection{Response}\label{response-7}}

In the White House and the Pentagon, officials scrambled to come up with
a response. But Mr. Trump, speaking alongside Prime Minister Justin
Trudeau of Canada, seemed to ease off the idea of a retaliatory strike
or an escalation that could lead to war.

\emph{Thomas Gibbons-Neff contributed reporting.}

Advertisement

\protect\hyperlink{after-bottom}{Continue reading the main story}

\hypertarget{site-index}{%
\subsection{Site Index}\label{site-index}}

\hypertarget{site-information-navigation}{%
\subsection{Site Information
Navigation}\label{site-information-navigation}}

\begin{itemize}
\tightlist
\item
  \href{https://help.nytimes.com/hc/en-us/articles/115014792127-Copyright-notice}{©~2020~The
  New York Times Company}
\end{itemize}

\begin{itemize}
\tightlist
\item
  \href{https://www.nytco.com/}{NYTCo}
\item
  \href{https://help.nytimes.com/hc/en-us/articles/115015385887-Contact-Us}{Contact
  Us}
\item
  \href{https://www.nytco.com/careers/}{Work with us}
\item
  \href{https://nytmediakit.com/}{Advertise}
\item
  \href{http://www.tbrandstudio.com/}{T Brand Studio}
\item
  \href{https://www.nytimes.com/privacy/cookie-policy\#how-do-i-manage-trackers}{Your
  Ad Choices}
\item
  \href{https://www.nytimes.com/privacy}{Privacy}
\item
  \href{https://help.nytimes.com/hc/en-us/articles/115014893428-Terms-of-service}{Terms
  of Service}
\item
  \href{https://help.nytimes.com/hc/en-us/articles/115014893968-Terms-of-sale}{Terms
  of Sale}
\item
  \href{https://spiderbites.nytimes.com}{Site Map}
\item
  \href{https://help.nytimes.com/hc/en-us}{Help}
\item
  \href{https://www.nytimes.com/subscription?campaignId=37WXW}{Subscriptions}
\end{itemize}
