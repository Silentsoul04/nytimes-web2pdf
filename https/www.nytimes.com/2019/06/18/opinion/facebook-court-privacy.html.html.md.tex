Sections

SEARCH

\protect\hyperlink{site-content}{Skip to
content}\protect\hyperlink{site-index}{Skip to site index}

\href{https://myaccount.nytimes.com/auth/login?response_type=cookie\&client_id=vi}{}

\href{https://www.nytimes.com/section/todayspaper}{Today's Paper}

\href{/section/opinion}{Opinion}\textbar{}Facebook Under Oath: You Have
No Expectation of Privacy

\url{https://nyti.ms/2MW6Msf}

\begin{itemize}
\item
\item
\item
\item
\item
\end{itemize}

Advertisement

\protect\hyperlink{after-top}{Continue reading the main story}

\href{/section/opinion}{Opinion}

Supported by

\protect\hyperlink{after-sponsor}{Continue reading the main story}

\hypertarget{facebook-under-oath-you-have-no-expectation-of-privacy}{%
\section{Facebook Under Oath: You Have No Expectation of
Privacy}\label{facebook-under-oath-you-have-no-expectation-of-privacy}}

The social media giant thinks privacy is a you problem.

\href{https://www.nytimes.com/by/charlie-warzel}{\includegraphics{https://static01.nyt.com/images/2019/03/15/opinion/charlie-warzel/charlie-warzel-thumbLarge-v3.png}}

By \href{https://www.nytimes.com/by/charlie-warzel}{Charlie Warzel}

Mr. Warzel is an Opinion writer at large.

\begin{itemize}
\item
  June 18, 2019
\item
  \begin{itemize}
  \item
  \item
  \item
  \item
  \item
  \end{itemize}
\end{itemize}

\includegraphics{https://static01.nyt.com/images/2019/06/19/opinion/18warzelWeb/18warzelWeb-articleLarge.jpg?quality=75\&auto=webp\&disable=upscale}

\emph{This article is part of a limited-run newsletter. You can}
\href{https://www.nytimes.com/newsletters/privacy-project?action=click\&module=inline\&pgtype=Article}{\emph{sign
up here}}\emph{.}

In a San Francisco courtroom a few weeks ago, Facebook's lawyers said
the quiet part out loud: Users have no reasonable expectation of
privacy.

The admission came from Orin Snyder, a lawyer representing Facebook in a
litigation stemming from the Cambridge Analytica scandal. In a court
transcript, first surfaced by
\href{https://www.law360.com/articles/1164091/facebook-says-social-media-users-can-t-expect-privacy}{Law360}
and later uploaded in full by Sam Biddle at
\href{https://theintercept.com/2019/06/14/facebook-privacy-policy-court/}{The
Intercept}, Snyder and U.S. District Judge Vince Chhabria debate what
has become an existential platform question: Does posting, even to a
small group of friends, on social media mean that a user is forfeiting
all expectation of privacy? Yes, Facebook argues:

\begin{quote}
There is no privacy interest, because by sharing with a hundred friends
on a social media platform, which is an affirmative social act to
publish, to disclose, to share ostensibly private information with a
hundred people, you have just, under centuries of common law, under the
judgment of Congress, under the SCA, negated any reasonable expectation
of privacy.
\end{quote}

The judge pushed back, suggesting that if a user had painstakingly
tweaked her privacy settings so that only a tight-knit group could see
her posts, it would be a privacy violation if ``Facebook actually
disseminated the photographs and the likes and the posts to hundreds of
companies.'' But Snyder didn't budge, suggesting that sharing any
information with even one human being negates an expectation of privacy.

The
\href{http://www.documentcloud.org/documents/6153329-05-29-2019-Facebook-Inc-Consumer-Privacy.html}{entire
transcript is worth a read}, but this rebuttal from Judge Chhabria set
up what I think might be one of the most revealing exchanges with a tech
company representative in recent memory:

\textbf{Chhabria:} You seem to be treating {[}privacy{]} as a binary
thing, like either you have a full expectation of privacy or you have no
expectation of privacy at all. And I don't understand why we should
think of it in that way.

\textbf{Snyder:} Because, Your Honor, what the plaintiffs are doing here
and what Your Honor's hypothetical suggests is a brand-new right of
privacy that has never been recognized before.

A generous reading of Synder's response is that Facebook's hands are
tied by the legal understanding of privacy. But I'd argue that Facebook
is hiding behind an antiquated definition of the word. Other industry
observers have noticed this recently, too. In a blog post last week,
\href{https://idlewords.com/2019/06/the_new_wilderness.htm}{Maciej
Ceglowski suggested} that the reason companies like Google and Facebook
have taken pro-privacy positions lately is they're not talking about the
status quo but, instead, about this outdated definition of privacy.
``That language, especially as it is codified in law, is not adequate
for the new reality of ubiquitous, mechanized surveillance,'' Ceglowski
wrote.

Ceglowski offered up a different definition, which he calls ``ambient
privacy.'' Basically, it's ``the understanding that there is value in
having our everyday interactions with one another remain outside the
reach of monitoring, and that the small details of our daily lives
should pass by unremembered.''

There is no such ambient privacy in Facebook's world, as evidenced by
the transcript. And for good reason. The very notion of ambient privacy
is an existential threat to Big Tech's business model. Take away that
which violates the ambient privacy and what's left is not Facebook.

Facebook and the rest of Big Tech built their empires by prioritizing
innovation and embracing a mind-set that enormous, systemic challenges
(\href{https://content.time.com/time/subscriber/article/0,33009,2152422,00.html}{``solving
death},'' \href{https://waymo.com/}{driverless cars,}
\href{https://techcrunch.com/2017/06/22/bring-the-world-closer-together/}{bringing
the world closer together}) can be solved through processing power, code
and a reimagining of what's possible. It's a mentality that treats
complex physical world issues as software; everything can be updated.

And yet, Snyder and Facebook appear gobsmacked by the idea of Privacy
2.0 and creating a new definition that reflects the way the tech giants
have altered its very meaning. It's yet another example of what has
become the dismal reality of Silicon Valley: They're very excited to fix
big problems, as long as they're not problems that they created.

\emph{\textbf{Do you think that posting personal content, even to a
small group of friends, on social media means forfeiting all
expectations of privacy? Send me your thoughts at}}
\textbf{\href{mailto:privacynewsletter@nytimes.com}{\emph{privacynewsletter@nytimes.com}}\emph{.
Your responses may be shared in an upcoming edition of this
newsletter.}}

\hypertarget{from-the-archives-police-cameras-in-the-park}{%
\subsection{From the Archives: `Police Cameras in the
Park'}\label{from-the-archives-police-cameras-in-the-park}}

\includegraphics{https://static01.nyt.com/images/2019/06/18/opinion/18warzelarchive/cad92438590b4440b9c484a49ab44cc4-articleLarge.png?quality=75\&auto=webp\&disable=upscale}

In the spirit of expectations of privacy, this week's archive pick is a
\href{https://www.nytimes.com/1998/02/09/opinion/police-cameras-in-the-park.html}{February
1998 editorial} on a decision to put cameras in New York's Washington
Square Park. The similarities to today's debates on surveillance are
striking.

\begin{quote}
Even though there is generally no expectation of privacy in a public
space, most people expect freedom from government monitoring when they
eat lunch on a park bench or stroll down a street. The growing use of
police video monitors in New York City may threaten the free and
anonymous nature of public space \ldots{}

Before Mayor Rudolph Giuliani and Police Commissioner Howard Safir
expand video monitoring to cover more areas of the city, there needs to
be significant public debate about the wisdom of 24-hour videotaping of
lawful movement.
\end{quote}

Sounds a bit like the arguments that led to
\href{https://www.nytimes.com/2019/05/14/us/facial-recognition-ban-san-francisco.html}{San
Francisco's ban on facial recognition} technology, no?

Chaser: This piece 10 months later,
\href{https://www.nytimes.com/1998/12/13/nyregion/secret-surveillance-cameras-growing-in-city-report-says.html}{in
December 1998}, about more cameras infiltrating New York.

\begin{quote}
Mayor Rudolph W. Giuliani has endorsed the use of video surveillance to
enhance public safety. His Police Commissioner, Howard Safir, said
yesterday that such cameras had proven to be ``incredibly effective,''
cutting crime by 30 percent to 50 percent in public housing projects
\ldots{}

``You have no right to privacy in a public place,'' Commissioner Safir
added, and ``no court order is required'' to use cameras.
\end{quote}

Makes for an interesting comparison with a
\href{https://www.nytimes.com/2019/06/09/opinion/facial-recognition-police-new-york-city.html?action=click\&module=privacy\%20footer\%20recirc\%20module\&pgtype=Article}{2019
Op-Ed from New York's police commissioner} arguing that facial
recognition
\href{https://www.nytimes.com/interactive/2019/04/16/opinion/facial-recognition-new-york-city.html}{makes
you}\href{https://www.nytimes.com/interactive/2019/04/16/opinion/facial-recognition-new-york-city.html}{safer}.

\emph{{[}If you're online --- and, well, you are --- chances are someone
is using your information. We'll tell you what you can do about it.}
\href{https://www.nytimes.com/newsletters/privacy-project?action=click\&module=Intentional\&pgtype=Article}{\emph{Sign
up for our limited-run newsletter}}\emph{.{]}}

\hypertarget{tip-of-the-week-how-to-protect-from-bluetooth-snooping}{%
\subsection{Tip of the Week: How to Protect From Bluetooth
Snooping}\label{tip-of-the-week-how-to-protect-from-bluetooth-snooping}}

\emph{Today's tip comes from our Opinion graphics director, Stuart
Thompson.}

In a
\href{https://www.nytimes.com/interactive/2019/06/14/opinion/bluetooth-wireless-tracking-privacy.html}{new
piece for Sunday Review}, Michael Kwet of Yale Law School writes how
grocery stores, sports stadiums and other brick-and-mortar venues are
using small Bluetooth-powered devices called beacons to monitor where
you go.

The devices work by sending out constant wireless Bluetooth signals,
which get picked up by your phone's Bluetooth receiver and can ``wake
up'' apps on your phone --- even when those
\href{https://docs.gimbal.com/proximity_overview.html\#wakeup_app}{apps
are closed}. The apps record your movements, monitoring whether you
lingered by the low-fat ice cream, for example, and transmit your
location to third-party companies for analysis.

How do you protect yourself from this kind of Bluetooth snooping?

For iOS users, the solution might seem simple: Slide up from the bottom
of your screen and press the Bluetooth icon to turn it off. Easy, right?
Except that's
\href{https://www.theverge.com/2018/2/25/17041440/bluetooth-location-tracking-iphone-android-privacy}{not
what those buttons do}. In
\href{https://arstechnica.com/gadgets/2017/11/apple-clarifies-what-control-center-wi-fi-bluetooth-toggles-do-in-ios-11-2-beta/}{2017},
Apple clarified that instead of turning Bluetooth off, the button only
disconnects existing Bluetooth devices --- and only for one day. The
same is true for the nearby button controlling Wi-Fi, which can also be
used for tracking.

Apple gives one pop-up notice explaining how it works the first time you
use the button. But afterward, you only see a small warning at the top
of the screen. The Bluetooth antenna remains active on your phone, so
any app with beacon technology installed could still monitor your
location if you haven't configured your phone properly.

Apple users can turn off Bluetooth entirely by going to Settings
\textgreater{} Bluetooth and sliding the big button at the top to the
``off'' position. (You can turn off Apple's own location tracking --- as
well as location tracking by specific apps --- by toggling options in
Settings \textgreater{} Location Services.) To turn off Bluetooth
tracking entirely on Android phones, make sure
\href{https://qz.com/1169760/phone-data/}{you shut off} both Bluetooth
and Location History. But keep in mind turning these functions back on
can reactivate Bluetooth tracking.

\hypertarget{what-im-reading}{%
\subsection{What I'm Reading}\label{what-im-reading}}

The
\href{http://nymag.com/intelligencer/2019/06/laliga-app-spied-on-users-earning-soccer-league-a-fine.html}{wildest
privacy piece I read last week} is about how a Spanish soccer league's
app is ``using their microphones and location data to listen in and find
bars that were pirating streams of soccer games.''

Some of the Hong Kong protesters are ``going dark'' to prevent being
surveilled while demonstrating.
\href{https://www.france24.com/en/20190613-surveillance-savvy-hong-kong-protesters-go-digitally-dark}{Here's
how they're doing it}.

Amazon
\href{https://twitter.com/seattletimes/status/1139161749909901312}{is
being sued over} its Alexa recordings.

\emph{Like other media companies, The Times collects data on its
visitors when they read stories like this one. For detail please}
\href{https://www.nytimes.com/2019/04/10/opinion/sulzberger-new-york-times-privacy.html?rref=collection\%2Fspotlightcollection\%2Fprivacy-project-does-privacy-matter\&action=click\&contentCollection=opinion\&region=stream\&module=stream_unit\&version=latest\&contentPlacement=8\&pgtype=collection}{\emph{see
our publisher's description}} \emph{of The Times's practices and its
continuing efforts to increase transparency and protections. The Times
privacy policy can be}
\href{https://help.nytimes.com/hc/en-us/articles/115014892108-Privacy-policy}{\emph{found
here}}\emph{.}

\emph{Follow}
\href{https://twitter.com/privacyproject}{\emph{@privacyproject}}
\emph{on Twitter and The New York Times Opinion Section on}
\href{https://www.facebook.com/nytopinion}{\emph{Facebook}}
\emph{and}\href{https://www.instagram.com/nytopinion/}{\emph{Instagram}}\emph{.}

\hypertarget{glossary-replacer}{%
\subsection{glossary replacer}\label{glossary-replacer}}

Advertisement

\protect\hyperlink{after-bottom}{Continue reading the main story}

\hypertarget{site-index}{%
\subsection{Site Index}\label{site-index}}

\hypertarget{site-information-navigation}{%
\subsection{Site Information
Navigation}\label{site-information-navigation}}

\begin{itemize}
\tightlist
\item
  \href{https://help.nytimes.com/hc/en-us/articles/115014792127-Copyright-notice}{©~2020~The
  New York Times Company}
\end{itemize}

\begin{itemize}
\tightlist
\item
  \href{https://www.nytco.com/}{NYTCo}
\item
  \href{https://help.nytimes.com/hc/en-us/articles/115015385887-Contact-Us}{Contact
  Us}
\item
  \href{https://www.nytco.com/careers/}{Work with us}
\item
  \href{https://nytmediakit.com/}{Advertise}
\item
  \href{http://www.tbrandstudio.com/}{T Brand Studio}
\item
  \href{https://www.nytimes.com/privacy/cookie-policy\#how-do-i-manage-trackers}{Your
  Ad Choices}
\item
  \href{https://www.nytimes.com/privacy}{Privacy}
\item
  \href{https://help.nytimes.com/hc/en-us/articles/115014893428-Terms-of-service}{Terms
  of Service}
\item
  \href{https://help.nytimes.com/hc/en-us/articles/115014893968-Terms-of-sale}{Terms
  of Sale}
\item
  \href{https://spiderbites.nytimes.com}{Site Map}
\item
  \href{https://help.nytimes.com/hc/en-us}{Help}
\item
  \href{https://www.nytimes.com/subscription?campaignId=37WXW}{Subscriptions}
\end{itemize}
