Sections

SEARCH

\protect\hyperlink{site-content}{Skip to
content}\protect\hyperlink{site-index}{Skip to site index}

\href{https://myaccount.nytimes.com/auth/login?response_type=cookie\&client_id=vi}{}

\href{https://www.nytimes.com/section/todayspaper}{Today's Paper}

\href{/section/opinion}{Opinion}\textbar{}Launching a Global Currency Is
a Bold, Bad Move for Facebook

\url{https://nyti.ms/2MZ8F7D}

\begin{itemize}
\item
\item
\item
\item
\item
\end{itemize}

Advertisement

\protect\hyperlink{after-top}{Continue reading the main story}

\href{/section/opinion}{Opinion}

Supported by

\protect\hyperlink{after-sponsor}{Continue reading the main story}

\hypertarget{launching-a-global-currency-is-a-bold-bad-move-for-facebook}{%
\section{Launching a Global Currency Is a Bold, Bad Move for
Facebook}\label{launching-a-global-currency-is-a-bold-bad-move-for-facebook}}

The way we structure money and payments is a question for democratic
institutions, not technology companies.

By Matt Stoller

Mr. Stoller is a fellow at the Open Markets Institute.

\begin{itemize}
\item
  June 19, 2019
\item
  \begin{itemize}
  \item
  \item
  \item
  \item
  \item
  \end{itemize}
\end{itemize}

\includegraphics{https://static01.nyt.com/images/2019/06/19/opinion/19stoller-illo/6a0b43c45f394f05b402e7537747e4de-articleLarge.jpg?quality=75\&auto=webp\&disable=upscale}

On Tuesday, Facebook, in partnership with a surfeit of other large and
powerful corporations, including Uber, Spotify, PayPal and VISA,
announced that it would lead the effort to create
\href{https://www.nytimes.com/2019/06/18/technology/facebook-cryptocurrency-libra.html}{a
new global currency} called Libra. ``We believe,'' says the organization
that will govern the currency, ``that the world needs a global,
digitally native currency that brings together the attributes of the
world's best currencies: stability, low inflation, wide global
acceptance and fungibility.''

As far as I can tell, Facebook aims to build a new payments and currency
system using blockchain technology. Facebook is starting a subsidiary,
Calibra, to ``provide financial services'' to individuals and
businesses, including saving, spending and sending money. The actual
standards for the currency will be managed by a nonprofit in Switzerland
called the Libra Association. The currency will have its own central
bank known as the Libra Reserve, and the board will be the committee of
corporations that helped set it up.

\emph{{[}If you're online --- and, well, you are --- chances are someone
is using your information. We'll tell you what you can do about it.}
\href{https://www.nytimes.com/newsletters/privacy-project?action=click\&module=Intentional\&pgtype=Article}{\emph{Sign
up for our limited-run newsletter.}}\emph{{]}}

There are already such alternative currencies --- known as
cryptocurrencies --- in existence, such as Bitcoin and Ripple, but this
one will be different. Today, cryptocurrencies are backed solely by the
willingness of users to accept them, not because they have any intrinsic
value or are backed by any government. This makes such currencies
unstable. Libra, however, will be backed by reserves: If a user buys a
dollar of Libra, that dollar will presumably be held in reserve
somewhere, ready to be honored when someone sells that Libra. Moreover,
while most cryptocurrencies are hard to use, Libra promises to be
user-friendly and embedded into Facebook and WhatsApp.

Or so goes the story. Many of the details of this endeavor are not
public or have not been decided. But creating a global currency is a
bold move on Facebook's part, given that this announcement is happening
as Facebook is being criticized or investigated for
\href{https://www.ftc.gov/news-events/press-releases/2018/03/statement-acting-director-ftcs-bureau-consumer-protection}{massive
privacy violations},
\href{https://www.bundeskartellamt.de/SharedDocs/Meldung/EN/Pressemitteilungen/2019/07_02_2019_Facebook.html}{anti-competitive
practices in the advertising market},
\href{https://cicilline.house.gov/press-release/cicilline-opening-remarks-antitrust-subcommittee-hearing}{eroding
the free press} and
\href{https://www.reuters.com/article/us-myanmar-rohingya-facebook/u-n-investigators-cite-facebook-role-in-myanmar-crisis-idUSKCN1GO2PN}{fomenting
ethnic cleansing}. However, it is consistent with Facebook's goal of
continuing to connect the world no matter the consequences, by creating
a medium of exchange that can potentially bypass central banks, bank
regulators and existing currency systems.

There are four core problems with Facebook's new currency. The first,
and perhaps the simplest, is that organizing a payments system is a
complicated and difficult task, one that requires enormous investment in
compliance systems. Banks pay attention to details, complying with
regulations to prevent money-laundering, terrorist financing, tax
avoidance and counterfeiting. Recreating such a complex system is not a
project that an institution with the level of privacy and technical
problems like Facebook should be leading. (Or worse, failing to recreate
such safeguards could facilitate money-laundering, terrorist financing,
tax avoidance and counterfeiting.)

The second problem is that, since the Civil War, the United States has
had a general prohibition on the intersection between banking and
commerce. Such a barrier has been reinforced many times, such as in 1956
with the Bank Holding Company Act and in 1970 with an amendment to that
law during the conglomerate craze. Both times, Congress blocked banks
from going into nonbanking businesses through holding companies, because
Americans historically didn't want banks competing with their own
customers. Banking and payments is a special business, where a bank gets
access to intimate business secrets of its customers. As one travel
agent told Congress in 1970 when opposing the right of banks to enter
his business, ``Any time I deposited checks from my customers,'' he
said, ``I was providing the banks with the names of my best clients.''

Imagine Facebook's subsidiary Calibra knowing your account balance and
your spending, and offering to sell a retailer an algorithm that will
maximize the price for what you can afford to pay for a product. Imagine
this cartel having this kind of financial visibility into not only many
consumers, but into businesses across the economy. Such conflicts of
interest are why payments and banking are separated from the rest of the
economy in the United States.

It's also possible that insiders belonging to the Libra cartel could
exploit their access to information, business relationships or
technology to give themselves advantages. There are many ways a new
currency system could advantage large businesses over everyone else,
especially when the large ones are sitting on the board of governors for
the payments system. For instance, one of the incentives
\href{https://abc7chicago.com/business/facebook-launching-its-own-currency-for-2-billion-plus-users/5351232/}{being
discussed} to get people to use the currency is discounts on Uber rides;
if this happens, Facebook would be giving an advantage to Uber instead
of other ride-sharing businesses.

The third problem is that the Libra system --- or really any private
currency system --- introduces systemic risk into our economy. The Libra
currency is backed, presumably, by bonds and financial assets held in
reserve at the Libra Reserve. But what happens if there is a theft or
penetration of the system? What happens if all users want to sell their
Libra currency at once, causing the Libra Reserve to hold a fire sale of
assets? If the Libra system becomes intertwined in our global economy in
the way Facebook hopes, we would need to consider a public bailout of a
privately managed system.

Sorry, but no thanks: We should not be setting up a private
international payments network that would need to be backed by taxpayers
because it's too big to fail.

And the fourth problem is that of national security and sovereignty.
Enabling an open flow of money across all borders is a political choice
best made by governments. And openness isn't always good. For instance,
most nations, especially the United States, use economic sanctions to
bar individuals, countries or companies from using our financial system
in ways that harm our interests. Sanctions enforcement flows through the
banking system --- if you can't bank in dollars, you can't use dollars.
With the success of a private parallel currency, government sanctions
could lose their bite. Should Facebook and a supermajority of venture
capitalists and tech executives really be deciding whether North Korean
sanctions can succeed? Of course not.

A permissionless currency system based on a consensus of large private
actors across open protocols sounds nice, but it's not democracy. Today,
American bank regulators and central bankers are hired and fired by
publicly elected leaders. Libra payments regulators would be hired and
fired by a self-selected council of corporations. There are ways to
characterize such a system, but democratic is not one of them.

Years ago, Mark Zuckerberg made it clear that he doesn't think Facebook
is a business. ``In a lot of ways, Facebook is more like a government
than a traditional company,''
\href{https://books.google.com/books?id=n1g1DwAAQBAJ\&pg=PA61\&dq=\%E2\%80\%9CIn+a+lot+of+ways,+Facebook+is+more+like+a+government+than+a+traditional+company\%E2\%80\%9D+Foer\&hl=en\&sa=X\&ved=0ahUKEwiAvO-pi_TiAhXlkOAKHX2jCzYQ6AEIMDAB\#v=onepage\&q=\%E2\%80\%9CIn\%20a\%20lot\%20of\%20ways\%2C\%20Facebook\%20is\%20more\%20like\%20a\%20government\%20than\%20a\%20traditional\%20company\%E2\%80\%9D\%20Foer\&f=false}{said}
Mr. Zuckerberg. ``We're really setting policies.'' He has acted
consistently as a would-be sovereign power. For example, he is
\href{https://www.nytimes.com/2018/11/17/opinion/facebook-supreme-court-speech.html}{attempting
to set up a Supreme Court-style} independent tribunal to handle content
moderation. And now he is setting up a global currency.

The way we structure money and payments is a question for democratic
institutions. Any company big enough to start its own currency is just
too big.

Matt Stoller is a fellow at the Open Markets Institute.

\emph{The Times is committed to publishing}
\href{https://www.nytimes.com/2019/01/31/opinion/letters/letters-to-editor-new-york-times-women.html}{\emph{a
diversity of letters}} \emph{to the editor. We'd like to hear what you
think about this or any of our articles. Here are some}
\href{https://help.nytimes.com/hc/en-us/articles/115014925288-How-to-submit-a-letter-to-the-editor}{\emph{tips}}\emph{.
And here's our email:}
\href{mailto:letters@nytimes.com}{\emph{letters@nytimes.com}}\emph{.}

\emph{Follow}
\href{https://twitter.com/privacyproject}{\emph{@privacyproject}}
\emph{on Twitter and The New York Times Opinion section on}
\href{https://www.facebook.com/nytopinion}{\emph{Facebook}}\emph{,}
\href{http://twitter.com/NYTOpinion}{\emph{Twitter (@NYTopinion)}}
\emph{and}
\href{https://www.instagram.com/nytopinion/}{\emph{Instagram}}\emph{.}

\hypertarget{glossary-replacer}{%
\subsection{glossary replacer}\label{glossary-replacer}}

Advertisement

\protect\hyperlink{after-bottom}{Continue reading the main story}

\hypertarget{site-index}{%
\subsection{Site Index}\label{site-index}}

\hypertarget{site-information-navigation}{%
\subsection{Site Information
Navigation}\label{site-information-navigation}}

\begin{itemize}
\tightlist
\item
  \href{https://help.nytimes.com/hc/en-us/articles/115014792127-Copyright-notice}{©~2020~The
  New York Times Company}
\end{itemize}

\begin{itemize}
\tightlist
\item
  \href{https://www.nytco.com/}{NYTCo}
\item
  \href{https://help.nytimes.com/hc/en-us/articles/115015385887-Contact-Us}{Contact
  Us}
\item
  \href{https://www.nytco.com/careers/}{Work with us}
\item
  \href{https://nytmediakit.com/}{Advertise}
\item
  \href{http://www.tbrandstudio.com/}{T Brand Studio}
\item
  \href{https://www.nytimes.com/privacy/cookie-policy\#how-do-i-manage-trackers}{Your
  Ad Choices}
\item
  \href{https://www.nytimes.com/privacy}{Privacy}
\item
  \href{https://help.nytimes.com/hc/en-us/articles/115014893428-Terms-of-service}{Terms
  of Service}
\item
  \href{https://help.nytimes.com/hc/en-us/articles/115014893968-Terms-of-sale}{Terms
  of Sale}
\item
  \href{https://spiderbites.nytimes.com}{Site Map}
\item
  \href{https://help.nytimes.com/hc/en-us}{Help}
\item
  \href{https://www.nytimes.com/subscription?campaignId=37WXW}{Subscriptions}
\end{itemize}
