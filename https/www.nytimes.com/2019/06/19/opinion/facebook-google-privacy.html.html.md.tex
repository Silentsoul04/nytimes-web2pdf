Sections

SEARCH

\protect\hyperlink{site-content}{Skip to
content}\protect\hyperlink{site-index}{Skip to site index}

\href{https://myaccount.nytimes.com/auth/login?response_type=cookie\&client_id=vi}{}

\href{https://www.nytimes.com/section/todayspaper}{Today's Paper}

\href{/section/opinion}{Opinion}\textbar{}What if We All Just Sold
Non-Creepy Advertising?

\url{https://nyti.ms/2ZzfNsC}

\begin{itemize}
\item
\item
\item
\item
\item
\item
\end{itemize}

Advertisement

\protect\hyperlink{after-top}{Continue reading the main story}

\href{/section/opinion}{Opinion}

Supported by

\protect\hyperlink{after-sponsor}{Continue reading the main story}

\hypertarget{what-if-we-all-just-sold-non-creepy-advertising}{%
\section{What if We All Just Sold Non-Creepy
Advertising?}\label{what-if-we-all-just-sold-non-creepy-advertising}}

The big ad-tech companies know how to sell ads without damaging privacy,
but they choose not to.

By Gabriel Weinberg

Mr. Weinberg is the chief executive and founder of DuckDuckGo, a search
engine and web browser company.

\begin{itemize}
\item
  June 19, 2019
\item
  \begin{itemize}
  \item
  \item
  \item
  \item
  \item
  \item
  \end{itemize}
\end{itemize}

\includegraphics{https://static01.nyt.com/images/2019/06/19/opinion/sunday/19weinberg/19759b8d8ac542d1bc080af9482e4948-articleLarge.jpg?quality=75\&auto=webp\&disable=upscale}

It's easy to see tech companies as a monolithic villain in the battle
over consumer privacy. But in fact, there are
\href{https://spreadprivacy.com/ccpa-privacy-for-all-act/}{countless
tech companies}, like mine, that believe that people have a fundamental
right to avoid being put under surveillance and that it should be easy
for them to exercise that right. By contrast, it is the big ad-tech
companies --- especially Facebook and Google --- that do not want to
make it easy for consumers to avoid profiling, because their business
models rely on it. They are resisting change to other proven models of
advertising, even though other companies are showing that it works.

This distinction between Big Ad Tech and everyone else in tech is
important to keep in mind as policymakers consider new regulations
intended to protect consumers' privacy. Executives of these big
companies may individually make public statements
\href{https://www.washingtonpost.com/opinions/mark-zuckerberg-the-internet-needs-new-rules-lets-start-in-these-four-areas/2019/03/29/9e6f0504-521a-11e9-a3f7-78b7525a8d5f_story.html?utm_term=.0a706aa010da}{welcoming
federal regulation}, but in practice they are doing everything they can
to weaken existing laws and shape new ones in their own interests. This
strategy is very obvious to the rest of us in the tech industry. And
it's essential to get these privacy laws right today, so that people
have the opportunity to opt out of online tracking now.

The most significant example of Big Ad Tech's influence on privacy
regulation is in California, which passed the California Consumer
Privacy Act in 2018. Industry groups representing Big Ad Tech are
leading the charge to weaken the law through
\href{https://leginfo.legislature.ca.gov/faces/billTextClient.xhtml?bill_id=201920200SB753}{amendments}
that would exempt the sharing of personal data for ads, as
\href{https://www.fastcompany.com/90338036/how-big-tech-is-trying-to-shape-californias-landmark-privacy-law}{Fast
Company recently reported.} This type of ``exemption'' is not a minor
change; it would weaken the law so much as to make it almost
meaningless. As Jacob Snow of the A.C.L.U. of Northern California
\href{https://www.fastcompany.com/90338036/how-big-tech-is-trying-to-shape-californias-landmark-privacy-law}{put
it}, ``A privacy law shouldn't have a targeted advertisement exception
for the same reason that an environmental law shouldn't have a coal
mining exception.''

Groups like the Information Technology and Innovation Foundation, which
has board members from \href{https://itif.org/people/board}{several big
tech companies}, are
\href{https://itif.org/publications/2019/01/14/grand-bargain-data-privacy-legislation-america}{lobbying}
in Washington for similar ad-tech exemptions federally, as
\href{https://www.theverge.com/2019/1/14/18182051/data-privacy-congress-grand-bargain-proposal-democrats-amazon-google-facebook-ccpa-coppa-hipaa}{The
Verge has reported.} They argue that strong privacy laws would hurt the
digital ad market, create high costs for businesses and curb innovation.

These are all weak arguments. There is no reason to fear that sites
cannot still make money with advertising. That's because there are
already two kinds of highly profitable online ads: contextual ads, based
on the content being shown on screen, and behavioral ads, based on
personal data collected about the person viewing the ad. Behavioral ads
work by tracking your online behavior and compiling a profile about you
using your internet activities (and even your offline activities in some
cases) to send you targeted ads.

Contextual advertising doesn't need to know anything about you: Search
for ``car'' and you get a car ad. Over the past decade, contextual ads
have been displaced by behavioral ads, aided by the rise of real-time
bidding technology that auctions off each ad on a site based on user
profiling. These behavioral ads are the ones that leave a bad taste in
your mouth. They follow you around from website to mobile app based on
your private information and, intentionally or not, enable online
\href{http://fortune.com/2019/03/28/facebook-hud-discrimination/}{discrimination},
\href{https://www.cbsnews.com/news/facebook-for-the-first-time-acknowledges-election-manipulation/}{manipulation}
and the creation of
\href{https://www.theguardian.com/technology/2017/may/22/social-media-election-facebook-filter-bubbles}{filter
bubbles}.

Strong privacy laws will force the digital advertising industry to
return to its roots in contextual advertising. That's a good thing,
since contextual advertising does not affect privacy in the same way.
(My company uses only contextual advertising, and we compete with
Google.)

In fact, Google search advertising began and still largely operates this
way. When you enter a search request in Google, it displays ads that are
relevant to that particular search, without needing to collect
information about your search, location, purchase or browsing history.
However, Google still collects all this information because it also
powers non-search behavioral ads across the whole internet, on
\href{https://support.google.com/ads/answer/2662856?hl=en-GB}{more than
two million websites and apps} that use Google's ad services and on
Google's non-search properties,
\href{https://support.google.com/youtube/answer/2454017?hl=en}{such as
YouTube}. Once people are allowed to opt out of behavioral advertising,
then online ads for those who do can go back to being more like
non-creepy contextual search ads.

\emph{{[}If you use technology, someone is using your information. We'll
tell you how --- and what you can do about it.}
\href{https://www.nytimes.com/newsletters/privacy-project?action=click\&module=Intentional\&pgtype=Article}{\emph{Sign
up for our limited-run newsletter}}\emph{.{]}}

This shift back to contextual advertising need not reduce profitability.
A
\href{https://digiday.com/media/digiday-research-most-publishers-dont-benefit-from-behavioral-ad-targeting/}{recent
poll} by Digiday of publishing executives found that 45 percent of them
saw no significant benefit from behavioral ads, and 23 percent said they
actually caused a decline in revenue.

What about compliance costs? Companies are quickly realizing that good
privacy practices are a boon for business. People increasingly want to
\href{https://www.pewresearch.org/fact-tank/2018/09/05/americans-are-changing-their-relationship-with-facebook/}{reduce
their digital footprint} and so choose companies that help them do so.
Companies with good privacy practices in their DNA do not face
significant compliance costs.

And if there is anything stifling innovation in Silicon Valley, it is
the dominance of Big Ad Tech, wrought by the
\href{https://www.washingtonpost.com/opinions/surveillance-capitalism-has-gone-rogue-we-must-curb-its-excesses/2019/01/24/be463f48-1ffa-11e9-9145-3f74070bbdb9_story.html?utm_term=.5e058ae81558}{surveillance
capitalism} business model. When contextual advertising regains
prominence, more companies will be able to compete against Facebook's
and Google's ad networks because they won't need huge troves of personal
data to do so. Additionally, strong privacy laws will spur innovations
to help companies use data in a privacy-respecting manner. In some
fields, such as services that help businesses analyze how people use
their sites, this innovation and resurgence of competition has
\href{https://www.fastcompany.com/90300072/its-time-to-ditch-google-analytics}{already
begun}.

I am reminded of the arguments made in the 1960s and '70s about laws to
reduce toxic emissions from cars. Companies profiting from less
regulation lobbied against those laws, and yet,
\href{https://www.epa.gov/transportation-air-pollution-and-climate-change/accomplishments-and-success-air-pollution-transportation}{once
they were enacted}, Americans' health improved, innovations such as the
modern catalytic converter entered the market, and big companies met the
new emissions targets without catastrophic expense. If we enact strong
privacy regulation, I believe we can be similarly hopeful about the
future of privacy.

Gabriel Weinberg is the chief executive and founder of DuckDuckGo, a
search engine and web browser company.

\emph{Like other media companies, The Times collects data on its
visitors when they read stories like this one. For more detail please
see}
\href{https://help.nytimes.com/hc/en-us/articles/115014892108-Privacy-policy?module=inline}{\emph{our
privacy policy}} \emph{and}
\href{https://www.nytimes.com/2019/04/10/opinion/sulzberger-new-york-times-privacy.html?rref=collection\%2Fspotlightcollection\%2Fprivacy-project-does-privacy-matter\&action=click\&contentCollection=opinion\&region=stream\&module=stream_unit\&version=latest\&contentPlacement=8\&pgtype=collection}{\emph{our
publisher's description}} \emph{of The Times's practices and continued
steps to increase transparency and protections. Follow}
\href{https://twitter.com/privacyproject}{\emph{@privacyproject}}
\emph{on Twitter and The New York Times Opinion Section on}
\href{https://www.facebook.com/nytopinion}{\emph{Facebook}}
\emph{and}\href{https://www.instagram.com/nytopinion/}{\emph{Instagram}}\emph{.}

\hypertarget{glossary-replacer}{%
\subsection{glossary replacer}\label{glossary-replacer}}

Advertisement

\protect\hyperlink{after-bottom}{Continue reading the main story}

\hypertarget{site-index}{%
\subsection{Site Index}\label{site-index}}

\hypertarget{site-information-navigation}{%
\subsection{Site Information
Navigation}\label{site-information-navigation}}

\begin{itemize}
\tightlist
\item
  \href{https://help.nytimes.com/hc/en-us/articles/115014792127-Copyright-notice}{©~2020~The
  New York Times Company}
\end{itemize}

\begin{itemize}
\tightlist
\item
  \href{https://www.nytco.com/}{NYTCo}
\item
  \href{https://help.nytimes.com/hc/en-us/articles/115015385887-Contact-Us}{Contact
  Us}
\item
  \href{https://www.nytco.com/careers/}{Work with us}
\item
  \href{https://nytmediakit.com/}{Advertise}
\item
  \href{http://www.tbrandstudio.com/}{T Brand Studio}
\item
  \href{https://www.nytimes.com/privacy/cookie-policy\#how-do-i-manage-trackers}{Your
  Ad Choices}
\item
  \href{https://www.nytimes.com/privacy}{Privacy}
\item
  \href{https://help.nytimes.com/hc/en-us/articles/115014893428-Terms-of-service}{Terms
  of Service}
\item
  \href{https://help.nytimes.com/hc/en-us/articles/115014893968-Terms-of-sale}{Terms
  of Sale}
\item
  \href{https://spiderbites.nytimes.com}{Site Map}
\item
  \href{https://help.nytimes.com/hc/en-us}{Help}
\item
  \href{https://www.nytimes.com/subscription?campaignId=37WXW}{Subscriptions}
\end{itemize}
