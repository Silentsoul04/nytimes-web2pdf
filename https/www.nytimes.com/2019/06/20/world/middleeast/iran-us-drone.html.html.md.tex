Sections

SEARCH

\protect\hyperlink{site-content}{Skip to
content}\protect\hyperlink{site-index}{Skip to site index}

\href{https://www.nytimes.com/section/world/middleeast}{Middle East}

\href{https://myaccount.nytimes.com/auth/login?response_type=cookie\&client_id=vi}{}

\href{https://www.nytimes.com/section/todayspaper}{Today's Paper}

\href{/section/world/middleeast}{Middle East}\textbar{}Strikes on Iran
Approved by Trump, Then Abruptly Pulled Back

\url{https://nyti.ms/2Rq3yff}

\begin{itemize}
\item
\item
\item
\item
\item
\item
\end{itemize}

Advertisement

\protect\hyperlink{after-top}{Continue reading the main story}

Supported by

\protect\hyperlink{after-sponsor}{Continue reading the main story}

\hypertarget{strikes-on-iran-approved-by-trump-then-abruptly-pulled-back}{%
\section{Strikes on Iran Approved by Trump, Then Abruptly Pulled
Back}\label{strikes-on-iran-approved-by-trump-then-abruptly-pulled-back}}

\includegraphics{https://static01.nyt.com/images/2019/06/21/us/politics/20dc-military1-sub/20dc-arms1-sub-videoSixteenByNine3000.jpg}

By \href{https://www.nytimes.com/by/michael-d-shear}{Michael D. Shear},
\href{https://www.nytimes.com/by/eric-schmitt}{Eric Schmitt},
\href{https://www.nytimes.com/by/michael-crowley}{Michael Crowley} and
\href{https://www.nytimes.com/by/maggie-haberman}{Maggie Haberman}

\begin{itemize}
\item
  June 20, 2019
\item
  \begin{itemize}
  \item
  \item
  \item
  \item
  \item
  \item
  \end{itemize}
\end{itemize}

WASHINGTON --- President Trump approved military strikes against
\href{https://www.nytimes.com/2019/07/12/us/politics/trump-iran-vote.html}{Iran}
in retaliation for downing an American surveillance
\href{https://www.nytimes.com/2019/07/18/us/politics/iranian-drone-shot-down.html}{drone},
but pulled back from launching them on Thursday night after a day of
escalating tensions.

As late as 7 p.m., military and diplomatic officials were expecting a
strike, after intense discussions and debate at the White House among
the president's top national security officials and congressional
leaders, according to multiple senior administration officials involved
in or briefed on the deliberations.

Officials said the president had initially approved attacks on a handful
of Iranian targets, like radar and missile batteries.

The operation was underway in its early stages when it was called off, a
senior administration official said. Planes were in the air and ships
were in position, but no missiles had been fired when word came to stand
down, the official said.

\emph{{[}Update:}
\href{https://www.nytimes.com/2019/06/22/world/middleeast/trump-iran.html}{\emph{Facing
intensifying confrontation with Iran}}\emph{, Trump has few appealing
options.{]}}

The abrupt reversal put a halt to what would have been the president's
third military action against targets in the Middle East.
\href{https://www.nytimes.com/2019/07/12/us/politics/trump-iran-vote.html}{Mr.
Trump} had struck twice at targets in Syria, in 2017 and 2018.

It was not clear whether Mr. Trump simply changed his mind on the
strikes or whether the administration altered course because of
logistics or strategy. It was also not clear whether the attacks might
still go forward.

Asked about the plans for a strike and the decision to hold back, the
White House declined to comment, as did Pentagon officials. No
government officials asked The New York Times to withhold the article.

The retaliation plan was intended as a response to the shooting down of
the unmanned, \$130 million surveillance drone, which was struck
Thursday morning by an Iranian surface-to-air missile, according to a
senior administration official who was briefed on the military planning
and spoke on the condition of anonymity to discuss confidential plans.

The strike was set to take place just before dawn Friday in Iran to
minimize risk to the Iranian military and civilians.

But military officials received word a short time later that the strike
was off, at least temporarily.

\hypertarget{where-was-drone-shot-down-us-and-iran-dispute-location}{%
\subsection{Where Was Drone Shot Down? U.S. and Iran Dispute
Location}\label{where-was-drone-shot-down-us-and-iran-dispute-location}}

50 MILES

Bandar Abbas

IRAN

Disputed

boundary

Iran's territorial

waters

Strait of Hormuz

Iran said drone shot down here

U.S. said drone shot down here

Two tankers attacked in June in this area

Dubai

Persian Gulf

Four tankers attacked in May in this area

Abu Dhabi

Gulf of Oman

UNITED

ARAB

EMIRATES

OMAN

Muscat

IRAN

Bandar Abbas

Disputed

boundaries

Strait of Hormuz

Iran said drone

shot down here

U.S. said drone

shot down here

PAKISTAN

Two tankers attacked in June in this area

Dubai

Al Fujairah

Persian Gulf

Four tankers attacked in May in this area

Iran's territorial

waters

Abu Dhabi

UNITED

ARAB

EMIRATES

Gulf of Oman

OMAN

Muscat

50 MILES

50 MILES

IRAN

Bandar Abbas

Disputed

boundary

Iran's territorial

waters

Strait of Hormuz

Iran said drone shot down here

U.S. said drone shot down here

Two tankers attacked in June in this area

Dubai

Four tankers attacked in May in this area

Gulf of Oman

UNITED

ARAB

EMIRATES

OMAN

Bandar Abbas

IRAN

Strait of Hormuz

Iran said drone shot down here

U.S. said drone shot down here

Disputed

boundary

Two tankers attacked in June in this area

Dubai

Al Fujairah

Four tankers attacked in May in this area

Iran's territorial

waters

Abu Dhabi

Gulf of Oman

UNITED

ARAB

EMIRATES

OMAN

Muscat

50 MILES

Sources: Drone locations from the U.S. Department of Defense and the
Iranian Foreign Ministry. Boundaries from Marine Regions and Flanders
Marine Institute.

By Weiyi Cai

\emph{{[}}\href{https://www.nytimes.com/2019/06/14/us/politics/us-iran.html?action=click\&module=Top\%20Stories\&pgtype=Homepage}{\emph{The
U.S. and Iran are once again hurtling toward potential crisis. Here's a
timeline.}}\emph{{]}}

The possibility of a retaliatory strike hung over Washington for much of
the day. Officials in both countries traded accusations about the
location of the drone when it was destroyed by a surface-to-air missile
launched from the Iranian coast along the Gulf of Oman.

Mr. Trump's national security advisers split about whether to respond
militarily. Senior administration officials said Secretary of State Mike
Pompeo; John R. Bolton, the national security adviser; and Gina Haspel,
the C.I.A. director, had favored a military response. But top Pentagon
officials cautioned that such an action could result in a spiraling
escalation with risks for American forces in the region.

Congressional leaders were briefed by administration officials in the
Situation Room.

The destruction of the drone underscored the already tense relations
between the two countries after Mr. Trump's recent accusations that Iran
is to blame for explosions last week that damaged oil tankers traveling
through the strait, the vital waterway for much of the world's oil. Iran
has denied that accusation.

Iran's announcement this week that it would soon breach one of the key
limits it had agreed to
\href{https://www.nytimes.com/2015/07/15/world/middleeast/iran-nuclear-deal-is-reached-after-long-negotiations.html}{in
a 2015 pact} intended to limit its nuclear program has also fueled
tensions. Mr. Trump, who
\href{https://www.nytimes.com/2018/05/08/world/middleeast/trump-iran-nuclear-deal.html}{pulled
the United States out of the 2015 pact}, has vowed that he will not
allow Tehran to build a nuclear weapon.

On Thursday, Mr. Trump insisted that the United States' unmanned
surveillance aircraft was flying over international waters when it was
taken down by an Iranian missile.

``This drone was in international waters, clearly,'' the president told
reporters on Thursday afternoon at the White House as he began a meeting
with Prime Minister Justin Trudeau of Canada. ``We have it all
documented. It's documented scientifically, not just words.''

Asked what would come next, Mr. Trump said, ``Let's see what happens.''

Iran's government fiercely disputed the president's characterization,
insisting that the American drone had strayed into Iranian airspace.
Iran released GPS coordinates that put the drone eight miles off the
country's coast, inside the 12 nautical miles from the shore that Iran
claims as its territorial waters.

Majid Takht-Ravanchi, Iran's ambassador to the United Nations, wrote in
a letter to the Security Council that the drone ignored repeated radio
warnings before it was downed. He said that Tehran ``does not seek war''
but ``is determined to vigorously defend its land, sea and air.''

Congressional Democrats emerged from the president's classified briefing
in the Situation Room and urged Mr. Trump to de-escalate the situation.
They called on the president to seek congressional authorization before
taking any military action.

``This is a dangerous situation,'' Speaker Nancy Pelosi said. ``We are
dealing with a country that is a bad actor in the region. We have no
illusions about Iran in terms of their ballistic missile transfers,
about who they support in the region and the rest.''

Iran's destruction of the drone appeared to provide a boost for
officials inside the Trump administration who have long argued for a
more confrontational approach to Iran, including the possibility of
military actions that could punish the regime for its support of
terrorism and other destabilizing behavior in the region.

\includegraphics{https://static01.nyt.com/images/2019/06/17/video/14-tanker-close/e135963484684192a5094b6158274d22-videoSixteenByNineJumbo1600.png}

But in his public appearance, Mr. Trump initially seemed to be looking
for a way to avoid a potentially serious military crisis. Instead of
directly accusing the leaders of Iran, Mr. Trump said someone ``loose
and stupid'' in Iran was responsible for shooting down the drone.

The president said he suspected it was some individual in Iran who
``made a big mistake,'' even as Iran had taken responsibility for the
strike and asserted that the high-altitude American drone was operating
over Iranian air space, which American officials denied.

Mr. Trump said the episode would have been far more serious if the
aircraft had been a piloted vehicle, and not a drone. It made ``a big,
big difference'' that an American pilot was not threatened, he told
reporters.

Last year, Mr. Trump
\href{https://www.nytimes.com/2018/05/08/world/middleeast/trump-iran-nuclear-deal.html?module=inline}{pulled
the United States out of the 2015 nuclear pact} with Iran, over the
objections of China, Russia and American allies in Europe. He has also
imposed punishing economic sanctions on Iran, trying to cut off its
already limited access to international trade, including oil sales.

Iran has warned of serious consequences
\href{https://www.nytimes.com/2019/06/18/world/europe/iran-us-nuclear-europe.html}{if
Europe does not find a way around those sanctions}, though it has denied
involvement in the attacks on tankers near the vital Strait of Hormuz.
On Monday, Iran said it would soon stop abiding by a central component
of the nuclear deal, the limit on how much enriched uranium it is
allowed to stockpile.

Both Washington and Tehran said the downing of the drone occurred at
4:05 a.m. Thursday in Iran, or 7:35 p.m. Wednesday in Washington. The
drone ``was shot down by an Iranian surface-to-air missile system while
operating in international airspace over the Strait of Hormuz,'' the
United States Central Command said in a statement. ``This was an
unprovoked attack on a U.S. surveillance asset in international
airspace.''

\emph{{[}Update: Almost a month after Iran shot down an American spy
drone, President Trump said the American military}
\href{https://www.nytimes.com/2019/07/18/us/politics/iranian-drone-shot-down.html?action=click\&module=Intentional\&pgtype=Article}{\emph{downed
an Iranian drone}}\emph{.{]}}

Iran's ability to target and destroy the high-altitude American drone,
which was developed to evade the very surface-to-air missiles used to
bring it down, surprised some Defense Department officials, who
interpreted it as a show of how difficult Tehran can make things for the
United States as it deploys more troops and steps up surveillance in the
region.

Lt. Gen. Joseph Guastella, the Air Force commander for the Central
Command region in the Middle East, said the attack could have endangered
``innocent civilians,'' even though officials at Central Command
continued to assert that the drone was over international waters. He
said that the closest that the drone got to the Iranian coast was 21
miles.

Late Thursday, the Defense Department released additional imagery in an
email to support its case that the drone never entered Iranian airspace.
But the department incorrectly called the flight path of the drone the
location of the shooting down and offered little context for an image
that appeared to be the drone exploding in midair.

It was the latest attempt by the Pentagon to try to prove that Iran has
been the aggressor in a series of international incidents.

\includegraphics{https://static01.nyt.com/images/2019/06/21/world/20dc-military3/21iran-articleLarge.jpg?quality=75\&auto=webp\&disable=upscale}

\emph{{[}}\href{https://www.nytimes.com/2019/06/20/us/politics/iran-drone.html}{\emph{What
we know and don't know about Iran shooting down an American
drone}}\emph{.{]}}

Iran's foreign affairs minister, Mohammad Javad Zarif, said in a post on
Twitter that he gave what he said were precise coordinates for where the
American drone was targeted.

``At 00:14 US drone took off from UAE in stealth mode \& violated
Iranian airspace,'' he
\href{https://twitter.com/JZarif/status/1141772824086028288}{said in a
tweet} that included coordinates that he said were near Kouh-e Mobarak.
``We've retrieved sections of the US military drone in OUR territorial
waters where it was shot down.''

Mr. Trump's comments on Thursday afternoon in the Oval Office reflected
the longstanding tension between the president's desire to be seen as
tough on the world stage and his campaign promise to make sure that the
United States did not get tangled in more foreign wars.

The president has embraced a reputation as someone who punches back when
he is challenged. Only months into his tenure, Mr. Trump
\href{https://www.nytimes.com/2017/04/06/world/middleeast/us-said-to-weigh-military-responses-to-syrian-chemical-attack.html}{launched
59 Tomahawk cruise missiles} at an air base in Syria after a chemical
weapon attack.

But he has often talked about ending American involvement in
long-running conflicts abroad, describing his ``America First'' agenda
as having little room for being the world's police force.
\href{https://twitter.com/realDonaldTrump/status/1082484663216730113}{In
a tweet in January}, he said he hoped that ``Endless Wars, especially
those which are fought out of judgement mistakes'' would ``eventually
come to a glorious end!''

According to Iranian news media, a foreign minister spokesman there said
that flying a drone into Iranian airspace was an ``aggressive and
provocative'' move by the United States.

Hossein Salami, the commander in chief of the Islamic Revolutionary
Guards Corps, said crossing the country's border was ``our red line,''
the semiofficial
\href{https://en.mehrnews.com/news/146678/IRGC-releases-details-of-downing-the-intruding-US-drone}{Mehr
news agency reported}.

``We are not going to get engaged in a war with any country, but we are
fully prepared for war,'' Mr. Salami said at a military ceremony in
Sanandaj, Iran, according to a translation from Press TV, a state-run
news outlet. ``Today's incident was a clear sign of this precise
message, so we are continuing our resistance.''

Iranian news media said the drone had flown over Iranian territory
unauthorized, and reported that it had been shot down in the Hormozgan
Province, along the country's southern coast on the Persian Gulf and the
Gulf of Oman.

Both the United States and Iran identified the aircraft as an RQ-4
Global Hawk, a surveillance drone made by Northrop Grumman.

``This was a show of force --- their equivalent of an inside pitch,''
said Derek Chollet, a former assistant secretary of defense for
international security affairs during the Obama administration, speaking
of Iran's decision to shoot down the drone.

\href{https://www.nytimes.com/2016/07/13/us/politics/james-stavridis-hillary-clinton-vice-president.html}{James
G. Stavridis}, who retired as a four-star admiral after serving as the
supreme allied commander at the North Atlantic Treaty Organization,
warned that the two countries were in a dangerous game that could
quickly spiral out of control. He described Iran's downing of the drone,
which costs about \$130 million, as a ``logical albeit highly dangerous
escalatory move by Iran.''

Advertisement

\protect\hyperlink{after-bottom}{Continue reading the main story}

\hypertarget{site-index}{%
\subsection{Site Index}\label{site-index}}

\hypertarget{site-information-navigation}{%
\subsection{Site Information
Navigation}\label{site-information-navigation}}

\begin{itemize}
\tightlist
\item
  \href{https://help.nytimes.com/hc/en-us/articles/115014792127-Copyright-notice}{©~2020~The
  New York Times Company}
\end{itemize}

\begin{itemize}
\tightlist
\item
  \href{https://www.nytco.com/}{NYTCo}
\item
  \href{https://help.nytimes.com/hc/en-us/articles/115015385887-Contact-Us}{Contact
  Us}
\item
  \href{https://www.nytco.com/careers/}{Work with us}
\item
  \href{https://nytmediakit.com/}{Advertise}
\item
  \href{http://www.tbrandstudio.com/}{T Brand Studio}
\item
  \href{https://www.nytimes.com/privacy/cookie-policy\#how-do-i-manage-trackers}{Your
  Ad Choices}
\item
  \href{https://www.nytimes.com/privacy}{Privacy}
\item
  \href{https://help.nytimes.com/hc/en-us/articles/115014893428-Terms-of-service}{Terms
  of Service}
\item
  \href{https://help.nytimes.com/hc/en-us/articles/115014893968-Terms-of-sale}{Terms
  of Sale}
\item
  \href{https://spiderbites.nytimes.com}{Site Map}
\item
  \href{https://help.nytimes.com/hc/en-us}{Help}
\item
  \href{https://www.nytimes.com/subscription?campaignId=37WXW}{Subscriptions}
\end{itemize}
