Sections

SEARCH

\protect\hyperlink{site-content}{Skip to
content}\protect\hyperlink{site-index}{Skip to site index}

\href{https://www.nytimes.com/section/climate}{Climate}

\href{https://myaccount.nytimes.com/auth/login?response_type=cookie\&client_id=vi}{}

\href{https://www.nytimes.com/section/todayspaper}{Today's Paper}

\href{/section/climate}{Climate}\textbar{}Climate Change Fills Storms
With More Rain, Analysis Shows

\url{https://nyti.ms/2LTn3fH}

\begin{itemize}
\item
\item
\item
\item
\item
\item
\end{itemize}

\href{https://www.nytimes.com/section/climate?action=click\&pgtype=Article\&state=default\&region=TOP_BANNER\&context=storylines_menu}{Climate
and Environment}

\begin{itemize}
\tightlist
\item
  \href{https://www.nytimes.com/2020/07/30/climate/sea-level-inland-floods.html?action=click\&pgtype=Article\&state=default\&region=TOP_BANNER\&context=storylines_menu}{Rising
  Seas}
\item
  \href{https://www.nytimes.com/interactive/2020/climate/trump-environment-rollbacks.html?action=click\&pgtype=Article\&state=default\&region=TOP_BANNER\&context=storylines_menu}{Trump's
  Changes}
\item
  \href{https://www.nytimes.com/interactive/2020/04/19/climate/climate-crash-course-1.html?action=click\&pgtype=Article\&state=default\&region=TOP_BANNER\&context=storylines_menu}{Climate
  101}
\item
  \href{https://www.nytimes.com/interactive/2018/08/30/climate/how-much-hotter-is-your-hometown.html?action=click\&pgtype=Article\&state=default\&region=TOP_BANNER\&context=storylines_menu}{Is
  Your Hometown Hotter?}
\item
  \href{https://www.nytimes.com/newsletters/climate-change?action=click\&pgtype=Article\&state=default\&region=TOP_BANNER\&context=storylines_menu}{Newsletter}
\end{itemize}

Advertisement

\protect\hyperlink{after-top}{Continue reading the main story}

Supported by

\protect\hyperlink{after-sponsor}{Continue reading the main story}

\hypertarget{climate-change-fills-storms-with-more-rain-analysis-shows}{%
\section{Climate Change Fills Storms With More Rain, Analysis
Shows}\label{climate-change-fills-storms-with-more-rain-analysis-shows}}

\includegraphics{https://static01.nyt.com/images/2019/07/12/climate/12cli-hurricane/merlin_157775076_c3e8db74-4a14-42da-9711-b9e49beef4b3-articleLarge.jpg?quality=75\&auto=webp\&disable=upscale}

\href{https://www.nytimes.com/by/kendra-pierre-louis}{\includegraphics{https://static01.nyt.com/images/2018/07/16/multimedia/author-kendra-pierre-louis/author-kendra-pierre-louis-thumbLarge.png}}

By \href{https://www.nytimes.com/by/kendra-pierre-louis}{Kendra
Pierre-Louis}

\begin{itemize}
\item
  July 11, 2019
\item
  \begin{itemize}
  \item
  \item
  \item
  \item
  \item
  \item
  \end{itemize}
\end{itemize}

When a tropical storm is approaching, its intensity or wind speed often
gets the bulk of the attention. But as
\href{https://www.nytimes.com/2019/07/11/us/hurricane-barry-tropical-storm-questions.html}{Tropical
Storm Barry bore down on the Gulf Coast}, it was the water that the
storm would bring with it that had weather watchers worried.

The National Weather Service called for roughly 10 to 20 inches of rain
to fall from late
\href{https://www.nytimes.com/2019/07/11/us/hurricane-barry-tropical-storm-questions.html}{Thursday
night through Saturday}. The average rainfall for July in New Orleans,
which is in the path of the storm, is just under six inches.

And Tropical Storm Barry, which may become a Category 1 hurricane before
making landfall, will drop rain on already saturated land. On Wednesday,
the region was hit by severe thunderstorms, which dropped as much as
seven inches of rain according to
\href{https://w2.weather.gov/climate/getclimate.php?date=\&wfo=lix\&sid=LA\&pil=RTPxx\&recent=yes\&specdate=2019-07-11+00\%3A34\%3A52}{preliminary
National Weather Service data}.

``Climate change is in general increasing the frequency and intensity of
heavy rainfall storms,'' said Andreas Prein, a project scientist with
the National Center for Atmospheric Research.

This week's rainfall came after the region experienced an extremely wet
spring, causing the region's rivers to swell, and raising concerns that
the upcoming storm may overtop levees in New Orleans. ``If you look at
the records, mostly it's the water that kills most people,'' Dr. Prein
said.

In an email interview, David Gochis, a hydrometerological scientist at
the National Center for Atmospheric Research, said that flooding of the
Mississippi River had left very little room to accommodate additional
water, and that the storm surge would inhibit river water from flowing
out to sea.

``The ingredients are there for a real catastrophe if the flood control
infrastructure simply gets overwhelmed,'' he said.

In recent years, researchers have found that hurricanes have lingered
longer, as Barry is expected to do, and dumped more rainfall --- a sign
of climate change, said Christina Patricola, a research scientist at
Lawrence Berkeley National Laboratory, and a co-author of
\href{https://www.nature.com/articles/s41586-018-0673-2}{a study that
found that climate chang}e is making tropical cyclones wetter. (Tropical
cyclones include both hurricanes and tropical storms, which are
hurricanes' less speedier kin.)

Researchers have been studying the effects of climate change on tropical
cyclones because those sorts of storms are driven by warm water. Water
in the gulf is 0.5 to 2 degrees Celsius warmer, according to Dr. Prein,
who said: ``This is really increasing the likelihood of a hurricane to
form in this basin. And it will increase the intensity of the hurricane
as well.''

\href{https://www.nytimes.com/section/climate?action=click\&pgtype=Article\&state=default\&region=MAIN_CONTENT_1\&context=storylines_keepup}{}

\hypertarget{climate-and-environment-}{%
\subsubsection{Climate and Environment
›}\label{climate-and-environment-}}

\hypertarget{keep-up-on-the-latest-climate-news}{%
\paragraph{Keep Up on the Latest Climate
News}\label{keep-up-on-the-latest-climate-news}}

Updated July 30, 2020

Here's what you need to know about the latest climate change news this
week:

\begin{itemize}
\item
  \begin{itemize}
  \tightlist
  \item
    \href{https://www.nytimes.com/2020/07/30/climate/bangladesh-floods.html?action=click\&pgtype=Article\&state=default\&region=MAIN_CONTENT_1\&context=storylines_keepup}{Floods
    in}\href{https://www.nytimes.com/2020/07/30/climate/bangladesh-floods.html?action=click\&pgtype=Article\&state=default\&region=MAIN_CONTENT_1\&context=storylines_keepup}{Bangladesh}
    are punishing the people least responsible for climate change.
  \item
    As climate change raises sea levels,
    \href{https://www.nytimes.com/2020/07/30/climate/sea-level-inland-floods.html?action=click\&pgtype=Article\&state=default\&region=MAIN_CONTENT_1\&context=storylines_keepup}{storm
    surges and high tides} are likely to push farther inland.
  \item
    The E.P.A. inspector general plans to investigate whether a rollback
    of fuel efficiency standards
    \href{https://www.nytimes.com/2020/07/27/climate/trump-fuel-efficiency-rule.html?action=click\&pgtype=Article\&state=default\&region=MAIN_CONTENT_1\&context=storylines_keepup}{violated
    government rules}.
  \end{itemize}
\end{itemize}

Though storms can form at any time, the Atlantic hurricane season
stretches from June 1 through Nov. 30 because that is typically when the
Atlantic Ocean's waters are warm enough to sustain storms. But
t\href{https://www.nytimes.com/2019/01/10/climate/ocean-warming-climate-change.html}{he
oceans are now warmer than ever}: They have absorbed more than 90
percent of the heat caused by human-released greenhouse gas emissions.

``We wanted to understand how climate change so far could have
influenced tropical cyclone events,'' Dr. Patricola said about her
study. ``And then the second part is to understand how future warming
could influence these events.''

The researchers used climate models to simulate how tropical cyclone
intensity, or wind speed, and rainfall would change if hurricanes like
Katrina, Irma and Maria had occurred absent climate change and under
future climate scenarios. They found that for all three storms, climate
change increased rainfall by up to 9 percent.

\includegraphics{https://static01.nyt.com/images/2019/07/12/science/12cli-hurricane-3/merlin_157808871_9b9d8323-9d4a-449f-8e38-7b59e92a7fe2-articleLarge.jpg?quality=75\&auto=webp\&disable=upscale}

This study is not the first to find that climate change is causing
tropical cyclones to have more rainfall. Studies on Hurricane Harvey
found that climate change contributed
as\href{https://www.nytimes.com/2017/12/13/climate/hurricane-harvey-climate-change.html}{much
as 38 percent, or 19 inches, of the more than 50 inches of rain that
fell in some places.} Dr. Patricola's study broadens the research by
using global climate models and analyzing a large number of storms.

``What's really interesting is that, regardless of the methodology that
you use, we're starting to see more and more evidence that climate
change so far has been enhancing the rainfall on some of these recent
hurricane events,'' she said.

When the researchers looked at the impact on storms under some possible
future conditions, they found that under scenarios with higher
greenhouse gas emissions there would be more rainfall associated with
storms. The largest increases would occur over regions, like the Gulf
Coast, that also have the heaviest historical rainfalls.

In other words, the wetter places are just going to get wetter.

\href{https://www.nytimes.com/interactive/2019/07/11/us/hurricane-barry-map-tracker.html}{}

\includegraphics{https://static01.nyt.com/images/2019/07/11/us/live-map-hurricane-barry-path-promo-1562889517953/live-map-hurricane-barry-path-promo-1562889517953-articleLarge-v5254.png}

\hypertarget{map-tracking-tropical-storm-barrys-path}{%
\subsection{Map: Tracking Tropical Storm Barry's
Path}\label{map-tracking-tropical-storm-barrys-path}}

Expected rainfall and path for a storm that threatens Louisiana.

And the structure of cities may exacerbate the problem even further,
said Gabriele Villarini, an associate professor of civil and
environmental engineering at the University of Iowa.

At issue: Dirt absorbs water, but paved surfaces such as roads,
sidewalks and even the footprint of building homes that make up cities
don't. The end result is that less water gets absorbed and the excess
inevitably has to go somewhere.

Dr. Villarini and his colleagues researched what might have happened in
Houston in 2017 during
\href{https://www.nytimes.com/video/us/100000005395222/hurricane-harveys-damage-in-texas.html}{Hurricane
Harvey} if the area had been cropland. They looked at both the changes
in rainfall patterns that cities cause as well as differences in how
water behaves based on ground type. They found that the twin effects
increased the likelihood of extreme flooding by 21 times, he said.

In addition to factors faced by most cities, New Orleans has some unique
geological factors at play. There are degraded wetlands and a complex
drainage system that keeps much of the city dry enough for development
but has also contributed to roughly half of the city sinking below sea
level, making it especially vulnerable.

Dr. Villarini noted that in the case of Hurricane Harvey, even absent
the impact of urbanization, there was ``a huge amount of rainfall. And
I'm struggling to think how you would design a city so that basically
you would be able to zero out any effect of flooding.''

Figuring out how to do that is something that researchers are working
on, particularly in places that, unlike New Orleans, are subject both to
intense rainfalls and intense periods of drought.

In places along the Texas Gulf Coast, for example, ``we have too much
water during the floods and not enough water during the drought,'' said
Qian Yang, a research associate in geology at the University of Texas at
Austin.

To help balance out that flow, Dr. Yang looked at a concept known as
\href{https://iopscience.iop.org/article/10.1088/1748-9326/ab148e}{managed
aquifer recharge and studied sites in Texas.}

The idea is that because aquifers, or large bodies of permeable rock
that contain groundwater, get depleted during droughts, cities should
work to refill them during times of significant rainfall. The potential
benefit would be to avoid the large capital construction, and large
geographic footprint, that comes with building new reservoirs because
these aquifers already exist.

For the Houston area, the researchers found that under high flow
scenarios they could recharge aquifers with roughly the same amount of
water as contained in Lake Mead, a
\href{https://www.nytimes.com/2019/03/19/climate/colorado-river-water.html}{reservoir
formed by the Hoover Dam.}

But in the case of extreme rain events like Hurricane Harvey and what is
expected of a potential Hurricane Barry, ``you would need some sort of
interim storage because the aquifers can't take the water in that
fast,'' said Bridget R. Scanlon, a senior research scientist in
geoscience at the University of Texas at Austin and co-author on the
study.

What many scientists and experts agree on: As climate change increases
extreme precipitation,
\href{https://www.nytimes.com/2019/06/19/climate/seawalls-cities-cost-climate-change.html}{cities
will need to adapt}.

For more news on climate and the environment,
\href{https://twitter.com/nytclimate}{follow @NYTClimate on Twitter}.

Advertisement

\protect\hyperlink{after-bottom}{Continue reading the main story}

\hypertarget{site-index}{%
\subsection{Site Index}\label{site-index}}

\hypertarget{site-information-navigation}{%
\subsection{Site Information
Navigation}\label{site-information-navigation}}

\begin{itemize}
\tightlist
\item
  \href{https://help.nytimes.com/hc/en-us/articles/115014792127-Copyright-notice}{©~2020~The
  New York Times Company}
\end{itemize}

\begin{itemize}
\tightlist
\item
  \href{https://www.nytco.com/}{NYTCo}
\item
  \href{https://help.nytimes.com/hc/en-us/articles/115015385887-Contact-Us}{Contact
  Us}
\item
  \href{https://www.nytco.com/careers/}{Work with us}
\item
  \href{https://nytmediakit.com/}{Advertise}
\item
  \href{http://www.tbrandstudio.com/}{T Brand Studio}
\item
  \href{https://www.nytimes.com/privacy/cookie-policy\#how-do-i-manage-trackers}{Your
  Ad Choices}
\item
  \href{https://www.nytimes.com/privacy}{Privacy}
\item
  \href{https://help.nytimes.com/hc/en-us/articles/115014893428-Terms-of-service}{Terms
  of Service}
\item
  \href{https://help.nytimes.com/hc/en-us/articles/115014893968-Terms-of-sale}{Terms
  of Sale}
\item
  \href{https://spiderbites.nytimes.com}{Site Map}
\item
  \href{https://help.nytimes.com/hc/en-us}{Help}
\item
  \href{https://www.nytimes.com/subscription?campaignId=37WXW}{Subscriptions}
\end{itemize}
