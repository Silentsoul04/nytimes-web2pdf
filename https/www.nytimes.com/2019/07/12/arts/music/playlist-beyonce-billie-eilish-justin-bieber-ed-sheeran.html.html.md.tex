Sections

SEARCH

\protect\hyperlink{site-content}{Skip to
content}\protect\hyperlink{site-index}{Skip to site index}

\href{https://www.nytimes.com/section/arts/music}{Music}

\href{https://myaccount.nytimes.com/auth/login?response_type=cookie\&client_id=vi}{}

\href{https://www.nytimes.com/section/todayspaper}{Today's Paper}

\href{/section/arts/music}{Music}\textbar{}The Playlist: Beyoncé's `Lion
King' Anthem, and 9 More New Songs

\url{https://nyti.ms/2XLUwus}

\begin{itemize}
\item
\item
\item
\item
\item
\end{itemize}

Advertisement

\protect\hyperlink{after-top}{Continue reading the main story}

Supported by

\protect\hyperlink{after-sponsor}{Continue reading the main story}

\hypertarget{the-playlist-beyoncuxe9s-lion-king-anthem-and-9-more-new-songs}{%
\section{The Playlist: Beyoncé's `Lion King' Anthem, and 9 More New
Songs}\label{the-playlist-beyoncuxe9s-lion-king-anthem-and-9-more-new-songs}}

Hear tracks by Billie Eilish and Justin Bieber, Jenny Hval, Angelica
Garcia and others.

By \href{https://www.nytimes.com/by/jon-pareles}{Jon Pareles},
\href{https://www.nytimes.com/by/jon-caramanica}{Jon Caramanica},
\href{https://www.nytimes.com/by/caryn-ganz}{Caryn Ganz} and
\href{https://www.nytimes.com/by/giovanni-russonello}{Giovanni
Russonello}

\begin{itemize}
\item
  July 12, 2019
\item
  \begin{itemize}
  \item
  \item
  \item
  \item
  \item
  \end{itemize}
\end{itemize}

\includegraphics{https://static01.nyt.com/images/2019/07/12/arts/12playlist/merlin_157741158_4d27f6fc-0d3c-4b90-a065-1b8ee44a75af-articleLarge.jpg?quality=75\&auto=webp\&disable=upscale}

\emph{Every Friday, pop critics for The New York Times weigh in on the
week's most notable new songs and videos. Just want the music?}
\href{https://open.spotify.com/playlist/35wPR33xiKIX0g8odMcWEX?si=ODVyHv3IRzapQ_5C4_eBNQ}{\emph{Listen
to the Playlist on Spotify here}} \emph{(or find our profile: nytimes).
Like what you hear? Let us know at}
\href{mailto:theplaylist@nytimes.com}{\emph{theplaylist@nytimes.com}}
\emph{and}
\href{https://www.nytimes.com/newsletters/louder?module=inline}{\emph{sign
up for our Louder newsletter}}\emph{, a once-a-week blast of our pop
music coverage.}

\hypertarget{beyoncuxe9-spirit-from-disneys-the-lion-king}{%
\subsection{Beyoncé, `Spirit (From Disney's `The Lion
King')'}\label{beyoncuxe9-spirit-from-disneys-the-lion-king}}

Beyoncé brings full gospel dynamics --- and sets aside the original
Broadway score --- in her showpiece for the remake of ``The Lion King.''
She belts, ``Your destiny is coming close --- stand up and fight!'' The
song's intro chants ``Long live the king'' in Swahili, but Beyoncé's
exhortation is not just for this lion king --- it's for every righteous
striver facing doubts. Her vocal, urged on by a choir, builds in wave
after wave, from breathy eagerness to full-throated cry, then topped by
a different kind of humility at its peak: just piano and Beyoncé's
soprano, envisioning a biblical transcendence, ``to be one with the
Great I Am.'' JON PARELES

\hypertarget{billie-eilish-featuring-justin-bieber-bad-guy}{%
\subsection{Billie Eilish featuring Justin Bieber, `Bad
Guy'}\label{billie-eilish-featuring-justin-bieber-bad-guy}}

The good news is that even though pop music has iterated past him a
couple of times over, Justin Bieber is nimble enough to keep up. On this
new version of Billie Eilish's swinging ``Bad Guy,'' Bieber sings with
the same robot-doing-cabaret cadence as his host. He touches on his
tattoos, his jewelry, his desire for more sleep --- a Sinatra for the
SoundCloud era. JON CARAMANICA

\hypertarget{angelica-garcia-it-dont-hinder-me}{%
\subsection{Angelica Garcia, `It Don't Hinder
Me'}\label{angelica-garcia-it-dont-hinder-me}}

A convincing Southern-rock stomper from Angelica Garcia, who has a
ferociously quavering voice and an even more ferocious sense of purpose.
``It Don't Hinder Me'' is a statement of cultural pride and social
resistance: ``I want the cooking that my grandma made/I want the bed
that I was yelled at to make.'' This sharp song is rowdier and swampier
than Garcia's 2016 debut album ``Medicine for Birds,'' a sign of a
singer getting ever more comfortable, and ever less bothered. CARAMANICA

\hypertarget{lil-nas-x-and-billy-ray-cyrus-featuring-young-thug-and-mason-ramsey-old-town-road-remix}{%
\subsection{Lil Nas X and Billy Ray Cyrus featuring Young Thug and Mason
Ramsey, `Old Town Road
(Remix)'}\label{lil-nas-x-and-billy-ray-cyrus-featuring-young-thug-and-mason-ramsey-old-town-road-remix}}

As of now there is no American award --- not the Grammy, not the
Pulitzer, not the Oscar or the Tony --- that could adequately reward the
miracle that is the never-ending rollout of ``Old Town Road.'' Each time
it courts death, it pivots. This latest version is the one you'd ask for
in a fantasy but never think was possible. Billy Ray Cyrus is still
here, crooning. And then there's Young Thug, less nonsensical than
usual, game to be in on the joke. But the crowning moment is at the end,
with the arrival of the viral yodeling preteen Mason Ramsey, who shows
up to sing about his Razor scooter, cows and his giddy-up. It's the
perfect twist ending to this internet-born-and-enabled saga: meme
recognize meme. CARAMANICA

\hypertarget{lola-marsh-echoes}{%
\subsection{Lola Marsh, `Echoes'}\label{lola-marsh-echoes}}

The Israeli duo Lola Marsh (the singer Yael Shoshana Cohen and the
multi-instrumentalist Gil Landau) makes sweeping, cinematic music
dripping in retro charm and reverb. There are hints of Dum Dum Girls,
Elle King and Lana Del Rey in the group's new track ``Echoes,'' a lush
beach blanket bop wiggling with dramatic energy. CARYN GANZ

\hypertarget{ed-sheeran-featuring-eminem-and-50-cent-remember-the-name}{%
\subsection{Ed Sheeran featuring Eminem and 50 Cent, `Remember the
Name'}\label{ed-sheeran-featuring-eminem-and-50-cent-remember-the-name}}

Ed Sheeran: Successful enough and powerful enough and deep enough into
his career to orchestrate an opportunity to rap on the same song as
Eminem and 50 Cent; rhymes ``misfit'' with ``Ipswich.''

Eminem: Far enough removed from the peak of his success and the peak of
his talent that the opportunity to rap alongside Ed Sheeran is not an
automatic no; over-delivers, perhaps out of mild embarrassment.

50 Cent: Tries out a Sugarhill Gang flow, hopes no one is looking.
CARAMANICA

\hypertarget{nuxe9rija-last-straw}{%
\subsection{Nérija, `Last Straw'}\label{nuxe9rija-last-straw}}

Swelling, underwater electric bass; a grainy, distorted guitar that
travels from woozy chords to neatly chopped rhythm; a high, corkscrewing
horn part that could have been plucked off a radio signal in the
Balkans, or maybe North Africa. You'd expect nothing less than this
absorbing mix from Nérija, a septet to watch of young London musicians.
The group includes the much-discussed young tenor saxophonist Nubya
Garcia and the trumpeter Sheila Maurice-Grey, who wrote this tune and
takes a rewarding solo across most of the track. Nérija has a debut
album, ``Blume,'' due Aug. 2. RUSSONELLO

\hypertarget{jenny-hval-ashes-to-ashes}{%
\subsection{Jenny Hval, `Ashes to
Ashes'}\label{jenny-hval-ashes-to-ashes}}

Here's a song about dreaming about a song about death: ``It had the most
moving chord changes/She was certain the lyrics went about burying
someone's ashes and then having a cigarette.'' Jenny Hval starts it as a
bemused incantation over misty chords; then the beat comes in, the
keyboards start answering her with hooks, and her meta-pop musings verge
on turning into pop themselves. PARELES

\hypertarget{anna-meredith-paramour}{%
\subsection{Anna Meredith, `Paramour'}\label{anna-meredith-paramour}}

Anna Meredith writes chamber music with a minimalist's fondness for
repeated motifs and a rocker's willingness to kick. The ingenious video
for the instrumental ``Paramour'' diagrams it as a frenetic
electric-train ride amid the hard-working ensemble --- clarinet,
keyboards, cello, drums, tuba, electric guitar --- pointing up just how
many motifs Meredith mobilizes and tosses aside in only five minutes.
PARELES

\hypertarget{victor-gould-october}{%
\subsection{Victor Gould, `October'}\label{victor-gould-october}}

Victor Gould never sounds totally at ease at the piano, but that doesn't
mean he's not in control. He plays in shapely, looping harmonies, with
debts to Hank Jones and Cedar Walton, but there's some physical
conflict, some audible work, in each gesture. This is what sets him
apart and underneath the sturdy, well-balanced flow of his arrangements,
it's the real reward. ``October'' gives a glimpse of his skills as an
arranger --- here, for jazz sextet and strings --- and his increasingly
distinctive voice as an improviser. The track comes from Gould's new
album, ``Thoughts Become Things.'' RUSSONELLO

Advertisement

\protect\hyperlink{after-bottom}{Continue reading the main story}

\hypertarget{site-index}{%
\subsection{Site Index}\label{site-index}}

\hypertarget{site-information-navigation}{%
\subsection{Site Information
Navigation}\label{site-information-navigation}}

\begin{itemize}
\tightlist
\item
  \href{https://help.nytimes.com/hc/en-us/articles/115014792127-Copyright-notice}{©~2020~The
  New York Times Company}
\end{itemize}

\begin{itemize}
\tightlist
\item
  \href{https://www.nytco.com/}{NYTCo}
\item
  \href{https://help.nytimes.com/hc/en-us/articles/115015385887-Contact-Us}{Contact
  Us}
\item
  \href{https://www.nytco.com/careers/}{Work with us}
\item
  \href{https://nytmediakit.com/}{Advertise}
\item
  \href{http://www.tbrandstudio.com/}{T Brand Studio}
\item
  \href{https://www.nytimes.com/privacy/cookie-policy\#how-do-i-manage-trackers}{Your
  Ad Choices}
\item
  \href{https://www.nytimes.com/privacy}{Privacy}
\item
  \href{https://help.nytimes.com/hc/en-us/articles/115014893428-Terms-of-service}{Terms
  of Service}
\item
  \href{https://help.nytimes.com/hc/en-us/articles/115014893968-Terms-of-sale}{Terms
  of Sale}
\item
  \href{https://spiderbites.nytimes.com}{Site Map}
\item
  \href{https://help.nytimes.com/hc/en-us}{Help}
\item
  \href{https://www.nytimes.com/subscription?campaignId=37WXW}{Subscriptions}
\end{itemize}
