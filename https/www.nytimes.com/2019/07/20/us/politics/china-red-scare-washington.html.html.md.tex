Sections

SEARCH

\protect\hyperlink{site-content}{Skip to
content}\protect\hyperlink{site-index}{Skip to site index}

\href{https://www.nytimes.com/section/politics}{Politics}

\href{https://myaccount.nytimes.com/auth/login?response_type=cookie\&client_id=vi}{}

\href{https://www.nytimes.com/section/todayspaper}{Today's Paper}

\href{/section/politics}{Politics}\textbar{}A New Red Scare Is Reshaping
Washington

\url{https://nyti.ms/2JIkuLW}

\begin{itemize}
\item
\item
\item
\item
\item
\end{itemize}

Advertisement

\protect\hyperlink{after-top}{Continue reading the main story}

Supported by

\protect\hyperlink{after-sponsor}{Continue reading the main story}

\hypertarget{a-new-red-scare-is-reshaping-washington}{%
\section{A New Red Scare Is Reshaping
Washington}\label{a-new-red-scare-is-reshaping-washington}}

\includegraphics{https://static01.nyt.com/images/2019/07/17/business/00dc-redscare-01/00dc-redscare-01-articleLarge.jpg?quality=75\&auto=webp\&disable=upscale}

By \href{https://www.nytimes.com/by/ana-swanson}{Ana Swanson}

\begin{itemize}
\item
  July 20, 2019
\item
  \begin{itemize}
  \item
  \item
  \item
  \item
  \item
  \end{itemize}
\end{itemize}

\href{https://cn.nytimes.com/usa/20190722/china-red-scare-washington/}{阅读简体中文版}\href{https://cn.nytimes.com/usa/20190722/china-red-scare-washington/zh-hant/}{閱讀繁體中文版}

WASHINGTON --- In a ballroom across from the Capitol building, an
unlikely group of military hawks, populist crusaders, Chinese Muslim
freedom fighters and followers of the Falun Gong has been meeting to
warn anyone who will listen that China poses an existential threat to
the United States that will not end until the Communist Party is
overthrown.

If the warnings sound straight out of the Cold War, they are. The
Committee on the Present Danger, a long-defunct group that campaigned
against the dangers of the Soviet Union in the 1970s and 1980s, has
recently been revived with the help of Stephen K. Bannon, the
president's former chief strategist, to warn against the dangers of
China.

Once dismissed as xenophobes and fringe elements, the group's members
are finding their views increasingly embraced in President Trump's
Washington, where skepticism and mistrust of China have taken hold. Fear
of China has spread across the government, from the White House to
Congress to federal agencies, where Beijing's rise is unquestioningly
viewed as an economic and national security threat and the defining
challenge of the 21st century.

``These are two systems that are incompatible,'' Mr. Bannon said of the
United States and China. ``One side is going to win, and one side is
going to lose.''

The United States and China have been locked in difficult trade
negotiations for the past two years, with talks plagued by a series of
\href{https://www.nytimes.com/2019/05/06/us/politics/trump-tariffs-china.html}{missteps}
and
\href{https://www.nytimes.com/2019/07/10/us/politics/us-china-trade-war.html?rref=collection\%2Fbyline\%2Fana-swanson\&action=click\&contentCollection=undefined\&region=stream\&module=stream_unit\&version=latest\&contentPlacement=2\&pgtype=collection}{misunderstandings}.
Mr. Trump has responded to the lack of progress by steadily ratcheting
up American tariffs on Chinese goods and finding other ways to
retaliate. China has responded in kind.

The two sides now appear far from any agreement that would resolve the
administration's concerns about China, including forcing American
companies operating there to hand over valuable technology. Even if a
deal is reached, the two sides are busy constructing broader economic
barriers.

In addition to placing a 25 percent tariff on roughly half of the goods
China exports, the United States has restricted the kinds of
technologies that can be exported to China, tried to cut off some
Chinese companies, like telecom giant Huawei, from purchasing American
products and rolled out hurdles for Chinese investment in the United
States.

American intelligence agencies have also ratcheted up efforts to combat
Chinese espionage, particularly at universities and research
institutions. Officials from the F.B.I. and the National Security
Council have been dispatched to Ivy League universities to warn
administrators to be vigilant against Chinese students who may be
gathering technological secrets from their laboratories to pass to
Beijing.

The administration paints the crackdown as necessary to protect the
United States. But there are growing concerns that it is stoking a new
red scare, fueling discrimination against students, scientists and
companies with ties to China and risking the collapse of a fraught but
deeply enmeshed trade relationship between the world's two largest
economies.

``I'm worried that some people are going to say, because of this fear,
any policy is justifiable,'' said Scott Kennedy, a China expert at the
Center for Strategic and International Studies. ``The climate of fear
that is being created needs to help generate the conversation, not end
the conversation.''

\includegraphics{https://static01.nyt.com/images/2019/07/17/business/00dc-redscare-02/00dc-redscare-02-articleLarge.jpg?quality=75\&auto=webp\&disable=upscale}

Anti-China sentiment has spread quickly, with Republicans and Democrats,
labor union leaders, Fox News hosts and others warning that China's
efforts to build up its military and advanced industries threaten
America's global leadership, and that the United States should respond
aggressively. Skepticism has seeped into nearly every aspect of China's
interaction with the United States, with officials questioning China's
presence on American stock markets, its construction of American subway
cars and its purchase of social media networks.

Yet there is little agreement on what America can or should do. The
United States has tried for decades to entice and cajole China to become
a more open society, but the Communist Party has steadily tightened its
grip over the Chinese people and the economy. American leaders now face
a choice of whether to continue down a path of engagement that could
leave the country vulnerable to economic and security threats --- or
embark on a path of disengagement that could weaken both economies and
might one day even lead to war.

An increasing number of people in Washington now view the decoupling of
the two economies as inevitable --- including many of the members of the
Committee on the Present Danger. At an inaugural meeting in April, Mr.
Bannon, Senator Ted Cruz of Texas, Newt Gingrich, the former House
speaker, and others issued paeans to Ronald Reagan --- a former member
of the group --- and were met with standing ovations as they called for
vigilance against China.

They praised Mr. Reagan's Cold War victory over the Soviet Union and his
doctrine of ``peace through strength,'' but there was also an air of
inevitability that war might come, only this time with China.

Mr. Bannon was just off the plane from Rome, with a slight shadow of a
mustache and his silver hair brushed back. Clad in a black button-down
and long black suit jacket, he thumped the podium as he described China
as a rising power and the United States as a declining power that would
inevitably clash.

``This is the defining event of our time, and 100 years from now, this
is what they're going to remember us for,'' he said.

The committee's two earliest iterations, in the 1950s and again in the
1970s, called for an arms buildup to counter the Soviets. The second
iteration, formed over a luncheon table at Washington's Metropolitan
Club in 1976, issued documents warning against Soviet expansionism, with
titles like ``Is America Becoming Number 2?''

The group reached the height of its influence during the Reagan
administration, in which
\href{https://www.nytimes.com/1981/11/23/us/group-goes-from-exile-to-influence.html}{dozens
of its members} eventually held posts, including as the national
security adviser and C.I.A. director. But as the Soviet threat faded, so
did the committee.

The group was briefly active again starting in 2004, this time to warn
against the threat of Islamic extremism. The committee's vice chair,
Frank Gaffney, is the founder of the Center for Security Policy, a think
tank that argues that mosques and Muslims across America are engaged in
a ``stealth jihad'' to ``Islamize'' the country by taking advantage of
American pluralism and democracy.

The group's activity largely died down until concern over China
rekindled interest.

Today's committee acknowledges that the threat from China is different
from Soviet Russia because the American and Chinese economies are much
more integrated. But Washington is increasingly reaching back into the
Cold War toolbox to confront the threat.

The administration has
\href{https://www.nytimes.com/2019/06/21/us/politics/us-china-trade-blacklist.html}{placed
Chinese tech companies} on an ``entity list,'' essentially blacklisting
them from doing business with American firms. In keeping with a law
passed last year, the administration has increased its checks of Chinese
investment, including of minority stakes in American companies. Last
June, the administration began
\href{https://www.nytimes.com/2018/07/25/us/politics/visa-restrictions-chinese-students.html}{restricting
visas for Chinese graduate students} in sensitive research fields like
robotics and aviation. And the United States has begun
\href{https://www.nytimes.com/2019/04/14/world/asia/china-academics-fbi-visa-bans.html}{barring
Chinese academics from the United States} if they are suspected of
having links to Chinese intelligence agencies.

``They're not the Soviet Union. But this kind of government control,
statism, never works for long,'' Larry Kudlow, the White House chief
economic adviser, said in a July 16 interview with Sinclair Broadcast
Group. The possibility that China could collapse like the Soviet Union
has ``always been an undercurrent'' in the trade war, he said.

The new Cold War has not been one-sided. Many of the changes in
Washington have been triggered by a darker turn in Beijing.

China has increased its scrutiny of American firms, and many American
companies and their employees in China now fear reprisal. In addition to
detaining millions of Chinese Muslims, democracy activists and others,
Chinese authorities have jailed foreign
\href{https://www.nytimes.com/2019/05/16/world/asia/china-canadian-arrested.html}{diplomats,
academics}
\href{https://www.nytimes.com/2019/07/11/business/american-businesses-china.html?action=click\&module=Top\%20Stories\&pgtype=Homepage}{and
businesspeople} --- prompting some to cancel or delay trips to China.

China is also projecting its power abroad, funding global infrastructure
and constructing an archipelago of artificial islands with giant air
bases reaching almost to the shores of Malaysia and Indonesia. Beijing
has made it clear that it intends to help its companies dominate the
industries of the future, from artificial intelligence and
supercomputers to aerospace equipment. Its policies have sought to
replace imports of high-tech products with Chinese-made goods,
pressuring multinationals to move factories from the United States and
resulting in the loss of American jobs.

China has rejected entreaties by the Trump administration to curb these
activities, arguing that it is simply pursuing its own economic
development. In an interview after trade talks broke down in May, Liu
He, China's top negotiator, said that areas of disagreement between the
United States and China focused on ``major matters of principle'' on
which China was unlikely to bend.

The chill in relations has begun to weigh on Chinese investment in the
United States, along with Chinese students and tourism. Chinese
investment in American residential and commercial real estate has begun
to decline. Companies are increasingly diversifying away from China,
wary of the president's ongoing economic war.

Image

The United States and China have been engaged in a protracted trade war,
with Washington accusing Beijing of breaking a deal earlier this
year.Credit...Erin Schaff/The New York Times

Nintendo, GoPro, Hasbro and other companies are reconsidering factories
in China, choosing to source products from Vietnam, the United States,
Mexico and India instead.

Susan Shirk, the chair of the 21st Century China Center at the
University of California at San Diego, said the United States is at risk
of being gripped by ``an anti-Chinese version of the Red Scare'' that is
driving Chinese talent away and could rupture what little good will is
left between the two countries.

``We've made this mistake once before, during the Cold War,'' Ms. Shirk
said. ``And I don't think we should make it again.''

Chinese nationals and Americans of Chinese heritage say they have felt
the chilling effects. Some suspect they are being passed over for
promotions and grants. Supporters of engagement have been dismissed as
apologists or even traitors.

``Chinese Americans feel targeted,'' said Charlie Woo, chief executive
of Megatoys and a member of the Committee of 100, an organization of
prominent Chinese-Americans. ``And that's really hurtful.''

The Trump administration and the Committee on the Present Danger have
been careful to say their targets are the Chinese government and the
Communist Party, not the Chinese people. But the distinction can be a
difficult one to make. In the rush to protect against new threats from
China, the line between preparedness and paranoia is sometimes unclear.

At a
\href{https://www.intelligence.senate.gov/hearings/open-hearing-worldwide-threats-0}{Senate
hearing} last year, Christopher A. Wray, the F.B.I. director, said the
Trump administration was trying to ``view the China threat as not just a
whole-of-government threat, but a whole-of-society threat,'' adding, ``I
think it's going to take a whole-of-society response by us.''

Many Chinese people and their defenders have bristled at the implication
that the entire Chinese society poses a national security threat.

Toby Smith, vice president for policy at the Association for American
Universities, said that American universities were working hard to
remain vigilant to espionage threats, but that they thrive on openness
and access to talent and science from around the world --- including
from China.

``The situation with China is different than the Cold War,'' he said.
``The concern with the Soviet Union was primarily military. Now it's a
concern about economic competitiveness.''

Advertisement

\protect\hyperlink{after-bottom}{Continue reading the main story}

\hypertarget{site-index}{%
\subsection{Site Index}\label{site-index}}

\hypertarget{site-information-navigation}{%
\subsection{Site Information
Navigation}\label{site-information-navigation}}

\begin{itemize}
\tightlist
\item
  \href{https://help.nytimes.com/hc/en-us/articles/115014792127-Copyright-notice}{©~2020~The
  New York Times Company}
\end{itemize}

\begin{itemize}
\tightlist
\item
  \href{https://www.nytco.com/}{NYTCo}
\item
  \href{https://help.nytimes.com/hc/en-us/articles/115015385887-Contact-Us}{Contact
  Us}
\item
  \href{https://www.nytco.com/careers/}{Work with us}
\item
  \href{https://nytmediakit.com/}{Advertise}
\item
  \href{http://www.tbrandstudio.com/}{T Brand Studio}
\item
  \href{https://www.nytimes.com/privacy/cookie-policy\#how-do-i-manage-trackers}{Your
  Ad Choices}
\item
  \href{https://www.nytimes.com/privacy}{Privacy}
\item
  \href{https://help.nytimes.com/hc/en-us/articles/115014893428-Terms-of-service}{Terms
  of Service}
\item
  \href{https://help.nytimes.com/hc/en-us/articles/115014893968-Terms-of-sale}{Terms
  of Sale}
\item
  \href{https://spiderbites.nytimes.com}{Site Map}
\item
  \href{https://help.nytimes.com/hc/en-us}{Help}
\item
  \href{https://www.nytimes.com/subscription?campaignId=37WXW}{Subscriptions}
\end{itemize}
