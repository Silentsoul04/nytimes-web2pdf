Sections

SEARCH

\protect\hyperlink{site-content}{Skip to
content}\protect\hyperlink{site-index}{Skip to site index}

\href{https://www.nytimes.com/section/movies}{Movies}

\href{https://myaccount.nytimes.com/auth/login?response_type=cookie\&client_id=vi}{}

\href{https://www.nytimes.com/section/todayspaper}{Today's Paper}

\href{/section/movies}{Movies}\textbar{}`Honeyland' Review: The Sting
and the Sweetness

\href{https://nyti.ms/2ydgRqA}{https://nyti.ms/2ydgRqA}

\begin{itemize}
\item
\item
\item
\item
\item
\end{itemize}

Advertisement

\protect\hyperlink{after-top}{Continue reading the main story}

Supported by

\protect\hyperlink{after-sponsor}{Continue reading the main story}

critic's pick

\hypertarget{honeyland-review-the-sting-and-the-sweetness}{%
\section{`Honeyland' Review: The Sting and the
Sweetness}\label{honeyland-review-the-sting-and-the-sweetness}}

A documentary about a Macedonian beekeeper's conflict with her neighbors
becomes a lyrical environmental fable.

\includegraphics{https://static01.nyt.com/images/2019/07/26/arts/26Honeyland-print/merlin_158307867_f210d427-4562-47ae-bff6-93772d1f89b2-articleLarge.jpg?quality=75\&auto=webp\&disable=upscale}

\href{https://www.nytimes.com/by/a-o--scott}{\includegraphics{https://static01.nyt.com/images/2018/02/20/multimedia/author-a-o-scott/author-a-o-scott-thumbLarge.jpg}}

By \href{https://www.nytimes.com/by/a-o--scott}{A.O. Scott}

\begin{itemize}
\item
  July 25, 2019
\item
  \begin{itemize}
  \item
  \item
  \item
  \item
  \item
  \end{itemize}
\end{itemize}

\begin{itemize}
\tightlist
\item
  Honeyland\\
  **NYT Critic's Pick Directed by Tamara Kotevska, Ljubomir Stefanov
  Documentary, Drama 1h 27m
\end{itemize}

\href{https://www.imdb.com/showtimes/title/tt8991268?ref_=ref_ext_NYT}{Find
Tickets}

When you purchase a ticket for an independently reviewed film through
our site, we earn an affiliate commission.

The opening minutes of ``\href{https://honeyland.earth/}{Honeyland}''
are as astonishing --- as sublime and strange and full of human and
natural beauty --- as anything I've ever seen in a movie. A woman makes
her way on foot across wild meadowlands and up a mountainside, carefully
stepping along a narrow ledge to a rocky outcropping, where bees have
made a hive. Without much protective gear, and apparently without being
stung, she extracts several honeycombs and secures them in a sack.

Back home, in the valley --- she is one of two permanent residents of a
hamlet in what is now called the
\href{https://www.nytimes.com/2019/05/17/travel/republic-of-north-macedonia-balkans.html}{Republic
of North Macedonia} --- she transplants the colony to a stone wall near
her house. In the coming months, she sings to the bees and talks to
them, explaining the terms of their relationship. When the time comes to
take the honey, she'll leave half of it for them. It's not a bargain,
exactly, since the bees have no real say in the matter, but it is a
sustainable arrangement, and one that has survived in this region for
generations.

The woman, Hatidze Muratova, is an actual beekeeper, and ``Honeyland,''
directed by \href{https://www.youtube.com/watch?v=_ES4mfYYsII}{Ljubomir
Stefanov and Tamara Kotevska}, is a documentary about her life and
labors. Originally commissioned to make a video about conservation
efforts in Macedonia, the filmmakers spent three years with Hatidze; her
mother, Nazife; and the late-arriving people next door. (More about them
shortly). As a result, Stefanov and Kotevska have done more than record
the rhythms and textures of rural life. They have shaped their
observations --- more than 400 hours of footage --- into a luminous
neorealist fable, a sad and stirring tale of struggle, persistence and
change.

Hatidze, born in 1964, lives with Nazife, who is bedridden and partly
blind; a hound named Jackie; and those industrious bees. It's an austere
existence, but it has its satisfactions, and Hatidze isn't entirely
isolated from the rest of the world. Skopje, the Macedonian capital, is
only a little more than 12 miles away, and she travels there, on foot
and by train, to sell her honey. Because of its superior quality and her
shrewdness, it fetches a good price. She uses the revenue to buy
household goods, including dye for the graying hair she wears tucked
into a kerchief.

Just as you are settling into Hatidze's company, appreciating the edges
and contours of her personality and savoring the deep pleasure of her
work, the pastoral calm is shattered. A family rumbles into the village
in a noisy trailer that turns out to be the quietest thing about it. The
patriarch, Hussein, and his wife, Ljutvie, bring chickens, cows and a
small army of children. At first, Hatidze welcomes these neighbors, who
are Turkish speakers like her. She tutors one of the older boys in the
mysteries of the apiary and tolerates periodic invasions of her privacy
from the littler ones.

Though she is too courteous to say so, you have the feeling that Hatidze
views Hussein as a man who may have overestimated his abilities. He has
a habit of blaming other people --- Ljutvie and their children, mostly
--- for his misfortunes and mistakes. His plan is to herd cows and keep
bees, turning this empty patch of territory into a literal land of milk
and honey, at a scale sufficient to feed his family and his ego.

\includegraphics{https://static01.nyt.com/images/2019/07/26/arts/26HoneyLand-Print3/merlin_158307855_63c8a832-8168-435b-9030-1a4eab6dce04-articleLarge.jpg?quality=75\&auto=webp\&disable=upscale}

Hussein is not exactly a villain. He's doing his best. But his crude
methods, his sandpapery personality and the sheer chaos that surrounds
him conspire to disrupt Hatidze's routines. As relations between them
deteriorate, ``Honeyland'' touches on what may be a universal feature of
modern experience, familiar to urban apartment dwellers and house-proud
suburbanites, as well as to old-school agrarians. The Rolling Stones
wrote a song about
\href{https://www.youtube.com/watch?v=omGDmvNWLVw}{``groaning and
straining with the trouble and strife.''} Neighbors are a nightmare.

But the presence of Hussein and his unruly household is also
serendipitous, a gift from the gods of film narrative. This isn't to
attribute the filmmakers' remarkable achievement to luck. They draw
comedy and pathos out of the conflict. They render the thick complexity
of experience with poignant clarity. Their movie is quiet, intimate and
intense, but touched with a breath of epic grandeur. It's a poem
including history.

And one that regards the future with both apprehension and hope. Hatidze
seems to be the last of her kind. Hussein's rapacious exploitation of
scarce and fragile resources represents the norm, supported by a crude
utilitarian argument. Their struggle feels like a microcosm of the human
predicament at a time of environmental catastrophe, and Hatidze,
precisely because of her modesty and patience, looks like a heroic
figure, offering a warning and an example to the rest of us, her
wasteful, wanton neighbors.

It may be a strange thing to say about a documentary, but ``Honeyland''
reminded me of
\href{https://www.nytimes.com/1984/12/10/us/speaking-for-the-trees.html}{``The
Lorax,''} Dr. Seuss's great, prescient tale of ecological folly. Like
the title character in that story, Hatidze is a kind of prophet. She
speaks for the bees.

\textbf{Honeyland}

Not rated. In Turkish, with English subtitles. Running time: 1 hour 27
minutes.

Advertisement

\protect\hyperlink{after-bottom}{Continue reading the main story}

\hypertarget{site-index}{%
\subsection{Site Index}\label{site-index}}

\hypertarget{site-information-navigation}{%
\subsection{Site Information
Navigation}\label{site-information-navigation}}

\begin{itemize}
\tightlist
\item
  \href{https://help.nytimes.com/hc/en-us/articles/115014792127-Copyright-notice}{©~2020~The
  New York Times Company}
\end{itemize}

\begin{itemize}
\tightlist
\item
  \href{https://www.nytco.com/}{NYTCo}
\item
  \href{https://help.nytimes.com/hc/en-us/articles/115015385887-Contact-Us}{Contact
  Us}
\item
  \href{https://www.nytco.com/careers/}{Work with us}
\item
  \href{https://nytmediakit.com/}{Advertise}
\item
  \href{http://www.tbrandstudio.com/}{T Brand Studio}
\item
  \href{https://www.nytimes.com/privacy/cookie-policy\#how-do-i-manage-trackers}{Your
  Ad Choices}
\item
  \href{https://www.nytimes.com/privacy}{Privacy}
\item
  \href{https://help.nytimes.com/hc/en-us/articles/115014893428-Terms-of-service}{Terms
  of Service}
\item
  \href{https://help.nytimes.com/hc/en-us/articles/115014893968-Terms-of-sale}{Terms
  of Sale}
\item
  \href{https://spiderbites.nytimes.com}{Site Map}
\item
  \href{https://help.nytimes.com/hc/en-us}{Help}
\item
  \href{https://www.nytimes.com/subscription?campaignId=37WXW}{Subscriptions}
\end{itemize}
