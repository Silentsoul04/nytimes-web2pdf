Sections

SEARCH

\protect\hyperlink{site-content}{Skip to
content}\protect\hyperlink{site-index}{Skip to site index}

\href{https://www.nytimes.com/section/politics}{Politics}

\href{https://myaccount.nytimes.com/auth/login?response_type=cookie\&client_id=vi}{}

\href{https://www.nytimes.com/section/todayspaper}{Today's Paper}

\href{/section/politics}{Politics}\textbar{}Trump Friend's Ties to
Mideast at Heart of Lobbying Inquiry

\url{https://nyti.ms/2K6bcZ0}

\begin{itemize}
\item
\item
\item
\item
\item
\item
\end{itemize}

Advertisement

\protect\hyperlink{after-top}{Continue reading the main story}

Supported by

\protect\hyperlink{after-sponsor}{Continue reading the main story}

\hypertarget{trump-friends-ties-to-mideast-at-heart-of-lobbying-inquiry}{%
\section{Trump Friend's Ties to Mideast at Heart of Lobbying
Inquiry}\label{trump-friends-ties-to-mideast-at-heart-of-lobbying-inquiry}}

\includegraphics{https://static01.nyt.com/images/2019/07/28/us/politics/28dc-barrack/28dc-barrack-articleLarge.jpg?quality=75\&auto=webp\&disable=upscale}

By \href{https://www.nytimes.com/by/sharon-lafraniere}{Sharon
LaFraniere}, \href{https://www.nytimes.com/by/maggie-haberman}{Maggie
Haberman}, \href{https://www.nytimes.com/by/william-k-rashbaum}{William
K. Rashbaum}, \href{https://www.nytimes.com/by/ben-protess}{Ben Protess}
and \href{https://www.nytimes.com/by/david-d-kirkpatrick}{David D.
Kirkpatrick}

\begin{itemize}
\item
  July 28, 2019
\item
  \begin{itemize}
  \item
  \item
  \item
  \item
  \item
  \item
  \end{itemize}
\end{itemize}

WASHINGTON --- As Donald J. Trump was preparing to deliver an address on
energy policy in May 2016, Paul Manafort, his campaign chairman, had a
question about the speech's contents for Thomas J. Barrack Jr., a top
campaign fund-raiser and close friend of Mr. Trump.

``Are you running this by our friends?'' Mr. Manafort asked in a
previously undisclosed email to Mr. Barrack, whose real estate and
investment firm does extensive business in the Middle East.

\emph{{[}}\href{https://www.nytimes.com/2019/06/26/the-weekly/trump-inauguration-expensive.html}{\emph{Watch
``The Weekly'': The Money Behind the Most Expensive U.S.
Inauguration}}\emph{{]}}

Mr. Barrack was, in fact, coordinating the language in a draft of the
speech with Persian Gulf contacts including Rashid al-Malik, an Emirati
businessman who is close to the rulers of the United Arab Emirates.

The exchanges about Mr. Trump's energy speech are among a series of
interactions that have come under scrutiny by federal prosecutors
looking at foreign influence over his campaign, his transition and the
early stages of his administration, according to documents and
interviews with people familiar with the case.

Investigators have looked in particular at whether Mr. Barrack or others
violated the law requiring people who try to influence American policy
or opinion at the direction of foreign governments or entities to
disclose their activities to the Justice Department, people familiar
with the case said.

The inquiry had proceeded far enough last month that Mr. Barrack, who
played an influential role in the campaign and acts as an outside
adviser to the White House, was interviewed, at his request, by
prosecutors in the public integrity unit of the United States attorney's
office in Brooklyn.

Mr. Barrack's spokesman, Owen Blicksilver, said that in expectation of
this article, Mr. Barrack's lawyer had again contacted the prosecutors'
office and ``confirmed they have no further questions for Mr. Barrack.''

Mr. Barrack has not been accused of wrongdoing, and his aides said he
never worked on behalf of foreign states or entities. Asked about the
status of the inquiry, a representative for the United States attorney's
office in Brooklyn declined to comment.

The relationship between Mr. Barrack, Mr. Manafort and representatives
of the U.A.E. and Saudi Arabia, including Mr. al-Malik, has been of
interest to federal authorities for at least nine months. The effort to
influence Mr. Trump's energy speech in 2016 was largely unsuccessful.

The special counsel's two-year investigation into Russian interference
in the 2016 presidential election has ended and federal prosecutors in
Manhattan
\href{https://www.nytimes.com/2019/07/18/nyregion/stormy-daniels-michael-cohen-documents.html}{have
signaled} that it is unlikely they would file additional charges in a
separate hush money investigation that ensnared members of Mr. Trump's
inner circle.

But as the scrutiny of Mr. Barrack indicates, prosecutors continue to
pursue questions about foreign influence. Among other lines of inquiry,
they have sought to determine whether Mr. Barrack and others tried to
sway the Trump campaign or the new administration on behalf of the
United Arab Emirates and Saudi Arabia, two closely aligned countries
with huge stakes in United States policy.

Between Mr. Trump's nomination and the end of June, Colony Capital, Mr.
Barrack's real estate investment and private equity firm, received about
\$1.5 billion from Saudi Arabia and the United Arab Emirates through
investments or other transactions like asset sales, Mr. Barrack's aides
said. That included \$474 million in investment from Saudi and Emirati
sovereign wealth funds, out of \$7 billion that Colony raised in
investment worldwide.

An executive familiar with the transactions
\href{https://www.nytimes.com/2018/06/13/world/middleeast/trump-tom-barrack-saudi.html}{had
provided The New York Times with somewhat different figures} last year.

Investigators have also questioned witnesses about Mr. Barrack's
involvement with a proposal from an American group that could give Saudi
Arabia access to nuclear power technology. And they have asked about
another economic development plan for the Arab world, written by Mr.
Barrack and circulated among Mr. Trump's advisers.

Aides to Mr. Barrack, who is of Lebanese descent and speaks Arabic, said
he had always acted as an independent intermediary between Persian Gulf
leaders and the Trump campaign and administration, never on behalf of
any foreign official or entity.

``The ideas he was giving voice to were his ideas,'' said Tommy Davis,
Mr. Barrack's former chief of staff, who continues to work for him.
``These are ideas that he has been advocating for decades.''

\includegraphics{https://static01.nyt.com/images/2019/07/28/us/politics/28dc-barrack02/merlin_138491229_be193c2e-3c62-4ed0-8759-054baa703c3a-articleLarge.jpg?quality=75\&auto=webp\&disable=upscale}

He said Mr. Barrack had no incentive to lobby on behalf of any
particular country or countries in the Persian Gulf because his business
interests and policy concerns span the entire region and countries at
odds with one another.

Nor is there any evidence, Mr. Barrack's aides said, that either Mr.
Barrack or his Los Angeles-based company has profited from his efforts.

``There is zero pay to play here,'' Mr. Blicksilver, Mr. Barrack's
spokesman, said. ``That is supported by the facts and the numbers.''

For Mr. Barrack, 72, the inquiry has unfolded amid a series of other
setbacks. A friend of Mr. Trump since the 1980s, he had anticipated that
his efforts to elect Mr. Trump, help run his transition team and manage
his inauguration would land him a prominent role in the administration.

But Jared Kushner, the president's son-in-law, blocked Mr. Barrack from
becoming a special envoy to the Middle East. A proposed role as a kind
of superambassador to Central and South America did not materialize
either.

At the same time, Colony Capital encountered substantial difficulties
after a troubled merger drove down its stock price and forced a series
of management changes.

Mr. Trump's inauguration in January 2017 was a high point for Mr.
Barrack: The inaugural committee he led set records for the amount of
money raised and spent to celebrate an inauguration.

But critics claimed the inaugural became a hub for peddling access to
foreign officials and business leaders, or people acting on their
behalf. The United States attorney's office in Manhattan opened an
investigation into possible violations of campaign finance law, focusing
partly on whether foreigners, who were barred from contributing to the
\$107 million inaugural fund, illegally funneled donations through
Americans.

Questions about whether Mr. Barrack complied with the Foreign Agents
Registration Act, commonly known as FARA, arose during the Russia
inquiry led by the special counsel, Robert S. Mueller III, and were
referred to the United States attorney's office in Brooklyn.

Three of the six former Trump aides who were charged by the special
counsel acknowledged violating the foreign lobbying statute in their
guilty pleas: Mr. Manafort, Rick Gates, who served as deputy campaign
chairman for Mr. Trump in 2016, and Michael T. Flynn, Mr. Trump's former
national security adviser.

But while the Justice Department has been trying for several years to
step up criminal enforcement of FARA requirements, such cases are
typically difficult to prove. ``Are you acting under someone's direction
or control?'' Adam S. Hickey, the deputy assistant attorney general in
charge of the national security division, said in a recent interview.
``Are you working on their behalf? That can be a very hard thing to
investigate or to decide.''

Central to the inquiry into Mr. Barrack are his dealings with Mr.
al-Malik, who is well connected in the court of Crown
\href{https://www.nytimes.com/2019/06/02/world/middleeast/crown-prince-mohammed-bin-zayed.html}{Prince
Mohammed bin Zayed}, the de facto ruler of the United Arab Emirates
widely known by his initials, M.B.Z., and is close to the prince's
brother, Sheikh Hamdan bin Zayed, who oversees the United Arab Emirates'
intelligence services. Sheikh Hamdan is considered to be Mr. al-Malik's
patron and a major financier of his business activities.

When Mr. Trump was elected, Mr. al-Malik received a coveted invitation
to the inaugural's most exclusive event --- the chairman's dinner,
hosted by Mr. Barrack.

In early 2018, Mr. al-Malik gave an interview and provided documents to
federal prosecutors who questioned whether he had been acting as an
unregistered foreign agent in the United States, according to two people
familiar with the matter. After he was interviewed, Mr. al-Malik left
for the United Arab Emirates and has not returned to the United States.

William F. Coffield, a lawyer for Mr. al-Malik, said that he
``voluntarily cooperated with the special council's office,'' adding,
``They accepted his cooperation and they certainly aren't going after
him.''

Investigators have documented a string of instances in which Mr. Barrack
appears to have tried, with feedback from Mr. al-Malik and others, to
shape the message of the Trump campaign or new administration in ways
that were more friendly to Middle East interests.

Although he was not always successful, Mr. Barrack had substantial sway
within the campaign when it was overseen by Mr. Manafort, a longtime
friend, and Mr. Manafort's deputy, Mr. Gates.

Image

Mr. Trump's inauguration in January 2017 was a high point for Mr.
Barrack. The inaugural committee he led set records for the amount of
money raised and spent to celebrate an inauguration.Credit...Chang W.
Lee/The New York Times

Mr. Barrack recommended that Mr. Trump hire Mr. Manafort, who rose to
campaign chairman before
\href{https://www.nytimes.com/2016/08/20/us/politics/paul-manafort-resigns-donald-trump.html}{he
was fired} over a separate foreign lobbying scandal. Mr. Manafort, who
was awash in debt and had no income, had hoped that after the campaign
Mr. Barrack would use his deep ties to the oil-rich nations to drum up
business for them both, according to people familiar with the situation.

In one email to the U.A.E.'s ambassador in Washington, Mr. Barrack
promoted Mr. Manafort as someone who was ``totally programmed'' on the
alliance between the Saudis and Emiratis.

Mr. Manafort, in turn, was willing to describe Mr. Barrack to foreign
officials as someone who could speak for the campaign on all subjects.

The Times learned of some of Mr. Barrack's electronic correspondence
from people critical of Emirati foreign policy and from people familiar
with his work with the Trump campaign.

In early May 2016, Mr. Barrack asked Mr. al-Malik and other Persian Gulf
contacts to propose language for a draft of an energy speech that Mr.
Trump was to deliver in Bismarck, N.D., that month.

Mr. Barrack's draft of the speech cited a new generation of leaders in
the Gulf region, naming both the Emirati crown prince and his ally,
Mohammed bin Salman, then deputy crown prince of Saudi Arabia. The Saudi
prince, often referred to by his initials, M.B.S., has now consolidated
his control of the kingdom.

Mr. Barrack's aides said he tried to influence Mr. Trump's address
because he cares deeply about United States relations with the Persian
Gulf region and was worried that Mr. Trump's inflammatory campaign
messaging would damage them. Among other provocative statements, Mr.
Trump had vowed that, if elected, he would bar Muslims from entering the
United States.

When Mr. Trump and a campaign speechwriter rejected Mr. Barrack's draft,
Mr. Manafort wrote to Mr. Barrack, ``Send me an insert that works for
our friends and I will fight for it.''

In the end, to Mr. Barrack's disappointment, Mr. Trump made only a
passing reference to the need to work with ``gulf allies'' on ``a
positive energy relationship as part of our antiterrorism strategy.''

A few days later, Mr. Manafort emailed Mr. Barrack that ``on the
platform issue there is another chance to make our gulf friends happy.''
He was referring to language in the Republican Party platform to be
approved at the convention where Mr. Trump would formally become the
nominee.

In late June, Mr. Manafort alerted Mr. Barrack that Mr. Trump had
softened his stance on a Muslim ban. Mr. Barrack quickly forwarded the
email to Yousef al-Otaiba, the Emirates' powerful ambassador in
Washington.

Then in July, Mr. Barrack informed Mr. Otaiba that the Trump team had
removed language from the proposed Republican platform that would have
called for the disclosure of redacted pages related to Saudi Arabia in a
report on the Sept. 11, 2001, terrorist attacks on the United States.

``Really confidential but important,'' he wrote, enclosing campaign
emails on the subject. ``Please do not distribute.''

Two days later,
\href{https://www.nytimes.com/2016/07/16/us/28-pages-saudi-arabia-september-11.html}{Congress
released the passages}, which detailed contacts between Saudi officials
and some of the hijackers.

Mr. Barrack tried to set up a meeting that summer between Mr. Manafort
and Mohammed bin Salman, the Saudi deputy crown prince, but it was
canceled at the last moment.

The month after Mr. Trump clinched the Republican nomination, Mr.
Barrack traveled to the Persian Gulf and met with the Saudi prince and
the Emirati crown prince, aides said. At a dinner meeting in Saudi
Arabia, he was briefed on the kingdom's economic plan.

In a subsequent text to Mr. Manafort, Mr. Barrack sounded elated.

``Amazing meetings. Off the map,'' he wrote. ``A lot to talk about and
do.''

Advertisement

\protect\hyperlink{after-bottom}{Continue reading the main story}

\hypertarget{site-index}{%
\subsection{Site Index}\label{site-index}}

\hypertarget{site-information-navigation}{%
\subsection{Site Information
Navigation}\label{site-information-navigation}}

\begin{itemize}
\tightlist
\item
  \href{https://help.nytimes.com/hc/en-us/articles/115014792127-Copyright-notice}{©~2020~The
  New York Times Company}
\end{itemize}

\begin{itemize}
\tightlist
\item
  \href{https://www.nytco.com/}{NYTCo}
\item
  \href{https://help.nytimes.com/hc/en-us/articles/115015385887-Contact-Us}{Contact
  Us}
\item
  \href{https://www.nytco.com/careers/}{Work with us}
\item
  \href{https://nytmediakit.com/}{Advertise}
\item
  \href{http://www.tbrandstudio.com/}{T Brand Studio}
\item
  \href{https://www.nytimes.com/privacy/cookie-policy\#how-do-i-manage-trackers}{Your
  Ad Choices}
\item
  \href{https://www.nytimes.com/privacy}{Privacy}
\item
  \href{https://help.nytimes.com/hc/en-us/articles/115014893428-Terms-of-service}{Terms
  of Service}
\item
  \href{https://help.nytimes.com/hc/en-us/articles/115014893968-Terms-of-sale}{Terms
  of Sale}
\item
  \href{https://spiderbites.nytimes.com}{Site Map}
\item
  \href{https://help.nytimes.com/hc/en-us}{Help}
\item
  \href{https://www.nytimes.com/subscription?campaignId=37WXW}{Subscriptions}
\end{itemize}
