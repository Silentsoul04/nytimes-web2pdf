Sections

SEARCH

\protect\hyperlink{site-content}{Skip to
content}\protect\hyperlink{site-index}{Skip to site index}

\href{https://www.nytimes.com/section/politics}{Politics}

\href{https://myaccount.nytimes.com/auth/login?response_type=cookie\&client_id=vi}{}

\href{https://www.nytimes.com/section/todayspaper}{Today's Paper}

\href{/section/politics}{Politics}\textbar{}How a Trump Ally Tested the
Boundaries of Washington's Influence Game

\url{https://nyti.ms/2H5gyDj}

\begin{itemize}
\item
\item
\item
\item
\item
\item
\end{itemize}

Advertisement

\protect\hyperlink{after-top}{Continue reading the main story}

Supported by

\protect\hyperlink{after-sponsor}{Continue reading the main story}

\hypertarget{how-a-trump-ally-tested-the-boundaries-of-washingtons-influence-game}{%
\section{How a Trump Ally Tested the Boundaries of Washington's
Influence
Game}\label{how-a-trump-ally-tested-the-boundaries-of-washingtons-influence-game}}

\includegraphics{https://static01.nyt.com/images/2019/08/05/us/00broidy-1/merlin_157636071_4265db01-1dac-4734-9725-b8aeb32fb175-articleLarge.jpg?quality=75\&auto=webp\&disable=upscale}

By \href{https://www.nytimes.com/by/kenneth-p-vogel}{Kenneth P. Vogel}

\begin{itemize}
\item
  Aug. 13, 2019
\item
  \begin{itemize}
  \item
  \item
  \item
  \item
  \item
  \item
  \end{itemize}
\end{itemize}

WASHINGTON --- Elliott Broidy had the kind of past that might have given
a more traditional White House reason to keep him at a distance: A
wealthy businessman, he had
\href{https://dealbook.nytimes.com/2009/12/03/guilty-plea-in-new-york-pension-bribery-case/}{pleaded
guilty in 2009} to giving nearly \$1 million in illegal gifts to New
York State officials to help land a \$250 million investment from the
state's pension fund.

But on a fall day in 2017, Mr. Broidy was ushered into the West Wing.
For about two hours, he met with a handful of the most powerful people
on earth, including President Trump, his chief of staff, his national
security adviser and Jared Kushner, his son-in-law, discussing
everything from personnel recommendations to the Republican Party's
finances.

Mostly, though, according to a detailed account he later sent to an
associate,
\href{https://www.nytimes.com/2018/03/21/us/politics/george-nader-elliott-broidy-uae-saudi-arabia-white-house-influence.html}{Mr.
Broidy talked about the Middle East}, a subject that had long been
important to him personally and was becoming increasingly important to
him financially.

As he sat with Mr. Trump, Mr. Broidy promoted a plan for a
counterterrorism force backed by Saudi Arabia and the United Arab
Emirates, which he said would be supported by his private security and
intelligence company, Circinus, under the leadership of Stanley A.
McChrystal, the retired Army general and former commander in
Afghanistan.

And at a time when Mr. Broidy was running a multimillion-dollar advocacy
campaign to turn Washington against Qatar, a regional rival of the
Saudis and the Emiratis, he took the opportunity to tell Mr. Trump that
Qatar was part of an ``axis of evil,'' according to his account of the
meeting.

\href{https://www.rollingstone.com/politics/politics-features/elliott-broidy-donald-trump-swamp-793159/}{That
meeting} was one of the high points of a comeback by Mr. Broidy, who
after having been shunned by some Republicans in the wake of his 2009
guilty plea had worked himself into Mr. Trump's inner circle as a top
fund-raiser for his 2016 campaign and inauguration.

The stature he suddenly assumed when Mr. Trump won the election allowed
him to position himself as a premier broker of influence and access to
the new administration. In the process, his international business came
to overlap with his efforts to influence government policy in ways that
have now made him the subject of an
\href{https://www.nytimes.com/2019/07/08/us/politics/elliott-broidy-trump.html}{intensifying
federal investigation}.

But Mr. Broidy's tour through the White House that day was also further
evidence of how Mr. Trump --- who initially lacked an established
network of high-dollar fund-raisers, held unformed positions on many
issues and had difficulty attracting top-tier talent --- came to rely on
people whose backgrounds and activities would have raised red flags in
other campaigns and administrations.

Among them were Paul Manafort, who was the chairman of Mr. Trump's
campaign and was
\href{https://www.nytimes.com/2017/10/30/us/politics/paul-manafort-indicted.html}{later
indicted} for lobbying and financial crimes, and Mr. Manafort's deputy,
Rick Gates, who also helped run Mr. Trump's inauguration. Prosecutors
are
\href{https://www.nytimes.com/2019/07/28/us/politics/thomas-barrack-foreign-lobbying.html}{still
investigating} whether the chairman of the inaugural committee and a
close friend of the president, Thomas J. Barrack Jr., violated lobbying
laws.

Few figures exploited the moment more ambitiously than Mr. Broidy, whose
Oval Office meeting was just one element of a sophisticated effort to
amass and exert influence in Mr. Trump's Washington.

Bolstering his own access to the administration, Mr. Broidy enlisted a
host of prominent figures to advance the interests of his companies, his
clients or his causes. In addition to General McChrystal, there was the
former Trump adviser Stephen K. Bannon; former defense secretaries
including Robert M. Gates and Leon E. Panetta; David H. Petraeus, the
former C.I.A. director; and the longtime diplomat Dennis B. Ross. They
gave paid speeches to groups he was funding, wrote op-eds or advised Mr.
Broidy, wittingly or unwittingly becoming public faces of his efforts.

While Mr. Broidy seemed to find a sympathetic audience for his positions
in the upper reaches of the administration, including his campaign
against Qatar, other efforts appeared to yield little action, like an
arrangement to help a Malaysian financier with legal problems in the
United States. And some of Mr. Broidy's proposals, like his plan to help
set up the counterterrorism force in the Persian Gulf, went nowhere.

\includegraphics{https://static01.nyt.com/images/2019/08/14/us/14broidy3/merlin_158902074_de96803f-23f3-4fff-bfb3-4b00c21652ca-articleLarge.jpg?quality=75\&auto=webp\&disable=upscale}

The Justice Department has been investigating, among other issues,
whether Mr. Broidy violated the law by not registering as an agent of
foreign interests at a time when he was promoting their causes and being
paid by them, and whether, in one case, he was
\href{https://www.nytimes.com/2018/11/30/us/politics/broidy-trump-foreign-money.html}{paid
with laundered money} to lobby. The Foreign Agents Registration Act, or
FARA, requires Americans to disclose efforts to shape government policy
or public opinion on behalf of foreign governments and political
interests. Enforcing FARA
\href{https://www.nytimes.com/2019/03/06/us/politics/fara-task-force-justice-department.html}{has
become an increasing priority} for the Justice Department.

While Mr. Broidy's advocacy efforts could have benefited his paying
clients, his representatives say the efforts were not directed or funded
by those clients in a way that would require FARA registration.

``Elliott Broidy has never agreed to work for, been retained or
compensated by, nor taken direction from any foreign government directly
or indirectly for any interaction with the United States government,
ever,'' said his lawyer, Chris Clark. ``Any implication to the contrary
is a lie.''

But the full scope and intensity of Mr. Broidy's activities, and the
investigations into them, are only now coming into focus. Interviews and
records show that:

• Federal investigators are homing in on the question of whether his
involvement with the government of the United Arab Emirates and the
Malaysian financier may have run afoul of FARA.

• Investigators are exploring the financial links between Mr. Broidy,
the government of the United Arab Emirates and one of that government's
advisers, George Nader. According to previously unreported banking
records, Mr. Nader was paid millions of dollars by the United Arab
Emirates as he was working closely with Mr. Broidy on two fronts: to win
security and intelligence contracts from the Emirate and Saudi
governments, and to direct and fund the campaign in Washington against
Qatar.

• Other banking records show that the government of the United Arab
Emirates continued to pay Mr. Broidy's company tens of millions of
dollars, including a payment of \$24 million in late March, even as it
became public that prosecutors were looking into his activities.

• Officials from one country with which Mr. Broidy has worked, Angola,
say they believed his company was being paid to lobby on their behalf,
rather than to provide private intelligence services, as Mr. Broidy's
representatives say.

• His efforts to help his clients in Washington were more extensive than
previously known. They involved not just prominent political figures but
also payments to influential think tanks, lobbyists and a nonprofit
conservative media outlet that produced articles promoting his clients'
agendas and criticizing their rivals.

Four people Mr. Broidy worked with on business or advocacy efforts have
been indicted. He
\href{https://www.nytimes.com/2018/04/13/us/politics/elliott-broidy-michael-cohen-payout.html?module=inline}{resigned
as deputy finance chairman} of the Republican National Committee last
year after it was revealed he had agreed to pay \$1.6 million in hush
money to a former Playboy model he impregnated, in a deal arranged by
Michael D. Cohen, the president's former lawyer.

\hypertarget{business-was-good-and-then-it-wasnt}{%
\subsection{Business Was Good, and Then It
Wasn't}\label{business-was-good-and-then-it-wasnt}}

Mr. Broidy's current situation is a sharp turnabout from two and a half
years ago, when he helped raise a
\href{https://www.nytimes.com/2017/04/18/us/politics/trump-inauguration-fundraising.html}{record
\$107 million} for Mr. Trump's inauguration. He offered to arrange
inaugural tickets for politicians from Angola, the Republic of Congo and
Romania --- countries from which he sought intelligence contracts worth
as much as \$266 million, documents and interviews show.

He greatly increased his giving to Republicans. He socialized with Mr.
Trump at the president's Mar-a-Lago resort, where he was a member.

Business was good. Mr. Broidy's company won deals worth more than \$200
million from the United Arab Emirates alone. The company established an
office there that employs 60 people who compile intelligence reports for
the U.A.E. government.

After The New York Times,
\href{https://www.apnews.com/a3521859cf8d4c199cb9a8567abd2b71}{The
Associated Press} and other news media outlets revealed last year that
he had
\href{https://www.nytimes.com/2018/03/25/us/politics/elliott-broidy-trump-access-circinus-lobbying.html}{marketed
his access} to the Trump team to prospective foreign clients, his
company lost lucrative United States government subcontracts. Members of
Congress returned donations, as did the Hudson Institute, a think tank,
which returned funding for a research project on Qatari influence. Mr.
Ross returned \$20,000 in consulting fees he had accepted in early 2018,
when he was advising Mr. Broidy on how to pursue contracts with foreign
governments and how to shape American foreign policy toward those
governments.

Image

Mr. Broidy offered inaugural tickets to politicians from Angola, the
Republic of Congo and Romania --- countries from which he was seeking
defense intelligence contracts worth as much as \$266
million.Credit...Todd Heisler/The New York Times

``There was a cloud that was created, and it made sense just to
dissociate,'' said Mr. Ross, who worked on Middle Eastern policy for
administrations of both parties.

Some of the activities of Mr. Broidy and his associates are detailed in
hundreds of documents and emails from the private accounts of Mr. Broidy
and his wife, which were distributed to reporters anonymously starting
in early 2018. Mr. Broidy
\href{https://www.nytimes.com/2018/03/26/world/middleeast/elliott-broidy-qatar-lawsuit.html}{sued
Qatar} and some of its lobbyists, accusing them of orchestrating the
theft and dissemination of those documents, which Qatar denies.

Mr. Broidy's spokesman, Nathan Miller, said those documents ``have been
altered and cherry-picked out of context to present a false narrative
about his business activities and public educational efforts that were
entirely legitimate and legal.''

But this account also relies on dozens of interviews, banking records
provided by people familiar with Mr. Broidy's work and other documents
submitted in court cases or obtained through the Freedom of Information
Act.

``He was certainly trying to influence the administration to adopt a
policy that served his political preference,'' Mr. Ross said in a July
interview with The Times about his work with Mr. Broidy, some of which
was subsequently
\href{https://www.thedailybeast.com/gop-moneyman-elliott-broidy-enlisted-veteran-diplomat-amid-secret-influence-campaign}{reported
by The Daily Beast}. ``Was he doing it because it would serve his
business interests as well? Presumably yes.''

\hypertarget{from-guilty-plea-to-trump-fund-raiser}{%
\subsection{From Guilty Plea to Trump
Fund-Raiser}\label{from-guilty-plea-to-trump-fund-raiser}}

Mr. Broidy, 62, made his own fortune. He grew up middle class in Los
Angeles, and paid his way through the University of Southern California
by operating a laundromat. After earning a bachelor's degree in
accounting and finance, he went to work for an accounting firm, before
he was hired to handle the personal investments of one of the firm's
clients, Taco Bell's founder,
\href{https://www.nytimes.com/2010/01/19/business/19bell.html}{Glen Bell
Jr.}, in the early 1980s.

After about a decade, Mr. Broidy started his own investment firm, Broidy
Capital Management. He built a mansion in the hills of Bel Air and
established a reputation as a generous philanthropist and pillar of Los
Angeles's Jewish community.

He assembled a large wine collection and indulged a fondness for
expensive wristwatches, according to people who know him. They said he
boasted that he was among the biggest private buyers of a type of
25-year-old whisky that retails for \$1,800 a bottle.

After the Sept. 11, 2001, terrorist attacks, Mr. Broidy's political and
business focus turned toward national security in the United States and
Israel.

In 2006, he was appointed by President George W. Bush, for whom Mr.
Broidy had become a top fund-raiser, to a homeland security advisory
panel and the Kennedy Center board of trustees. In October 2006, Mr.
Bush attended a dinner at the Bel Air mansion that
\href{https://www.latimes.com/politics/la-na-elliott-broidy-trump-20180811-story.html}{raised
\$1 million} for the Republican Party.

Weeks later, Mr. Broidy and his wife, Robin Rosenzweig,
\href{https://georgewbush-whitehouse.archives.gov/news/releases/2006/12/text/20061203.html}{were
on the guest list for a White House reception} for the Kennedy Center
Honors.

After his 2009 guilty plea in the New York State pension fund case,
which a court later
\href{https://www.nydailynews.com/news/crime/no-jail-time-guilty-israeli-investor-article-1.1209146}{reduced
from a felony to a misdemeanor}, Mr. Broidy retreated from the
spotlight. Politicians whose campaigns he once funded
\href{https://www.latimes.com/archives/la-xpm-2009-dec-08-la-me-broidy8-2009dec08-story.html}{turned
their backs on him}.

But his business ventures continued. He
\href{https://www.documentcloud.org/documents/6205003-The-Announcement-of-the-Formation-of-the.html}{helped
start a national security nonprofit group} and a cyberdefense
contracting company called Threat Deterrence, then purchased Circinus in
2015. Started after the 2001 terrorist attacks, Circinus says it
provides cybersecurity, ``force protection and operational training,''
and open source intelligence services to governments.

Image

Some of the activities of Mr. Broidy and his associates have come to
light through the circulation of documents and emails from the private
accounts of Mr. Broidy and his wife, Robin Rosenzweig.Credit...Alex
Berliner/BEI, via Shutterstock

As the 2016 presidential campaign got underway, Mr. Broidy edged back
into high-profile electoral politics, supporting a succession of
senators seeking the Republican nomination, including Lindsey Graham of
South Carolina, Marco Rubio of Florida and Ted Cruz of Texas.

When Mr. Cruz
\href{https://www.nytimes.com/2016/05/04/us/politics/ted-cruz.html?module=inline}{dropped
out}, Mr. Broidy enthusiastically began raising money for the Trump
campaign.

\hypertarget{on-top-of-the-world-at-the-inaugural}{%
\subsection{On Top of the World at the
Inaugural}\label{on-top-of-the-world-at-the-inaugural}}

In the weeks before Mr. Trump's inauguration, Mr. Broidy was in the
center of the action.

He helped organize and fund a private breakfast at the Trump
International Hotel two days before the inauguration that was attended
by 50 to 60 people, according to people familiar with the event.

The
\href{https://www.thedailybeast.com/mueller-probes-an-event-with-nunes-flynn-and-foreign-officials-at-trumps-dc-hotel}{guest
list} featured officials from Africa, Eastern Europe and Arab nations,
as well as Republicans with ties to the incoming administration,
including Mr. Trump's choice for national security adviser, Michael T.
Flynn.

Mr. Broidy teamed with a Nigerian-American entrepreneur to pursue an
intelligence contract with the Angolan government. An early draft of the
deal called for payments of as much as \$64 million over five years, but
someone familiar with it said the final contract was for a smaller
amount.

He offered to arrange access in Washington for a pair of powerful
Angolan officials who had a hand in the contract.

Days before the inauguration, the Angolans paid \$6 million to Circinus.
And Mr. Broidy escorted an Angolan official, André de Oliveira João
Sango, then the director of external intelligence, to introductory
meetings with Republican lawmakers.

A couple of days later, Mr. Sango sat at a table adjacent to Mr.
Broidy's at an exclusive ``candlelight'' donor dinner sponsored by Mr.
Trump's inaugural committee and attended by the president-elect,
according to another Angolan official.

While Mr. Broidy's representatives say he was not required to register
as a lobbyist because he did not accept funds for lobbying, Angolan
diplomats in Washington saw things differently.

``It was basically to help assist in approaching the Trump
administration,'' Lucombo Joaquim Luveia, a counselor at the embassy,
said of the payment to Circinus.

Mr. Luveia said that ``all those arrangements were back-channeled
between the lobbyist Broidy and the central government, at the
presidential level.'' The Angolan president at the time, José Eduardo
dos Santos,
\href{https://www.nytimes.com/2018/09/08/world/africa/angola-dos-santos.html}{was
replaced} last year.

Mr. Broidy also provided access during inauguration week to a pair of
Romanian politicians seen as critical to Circinus's chances for doing
business in the country. Mr. Broidy arranged an impromptu introduction
to Mr. Trump during an informal dinner at the Trump hotel for Liviu
Dragnea, then a powerful Romanian parliamentary leader.

Circinus subsequently competed for Romanian government contracts valued
at more than \$200 million, according to the Romanian news media and
people familiar with the contracting process. But the contracts did not
materialize. Mr. Dragnea, who was facing unrelated corruption charges in
Romania at the time of the inauguration, has since been convicted. And
Romanian and American officials have questioned a former Circinus
executive in Romania.

Image

George Nader presented himself as a liaison to Crown Prince Mohammed bin
Zayed, center, the de facto ruler of the United Arab Emirates, and Saudi
Arabia's crown prince, Mohammed bin Salman, right.Credit...via
Shutterstock

Hours after Mr. Trump's swearing-in, Mr. Broidy was abuzz as he and his
wife, holding hands, walked into a late-night party in a private room at
the Trump hotel.

He approached a fellow Republican donor and, in a move the donor
interpreted as an early flexing of new status, Mr. Broidy suggested it
was time to settle a lingering business dispute between them.

``He was exuding hubris,'' said the donor, Yuri Vanetik, a
characterization disputed by Mr. Broidy's representatives. ``He wanted
to show that it was his world now.''

\hypertarget{a-flurry-of-deal-discussions}{%
\subsection{A Flurry of Deal
Discussions}\label{a-flurry-of-deal-discussions}}

Through the transition and the early days of the administration, Mr.
Broidy entertained discussions about using his newfound connections in
Washington to help an array of foreign clients.

After being approached by a lawyer working with Russian executives who
were under sanctions, Mr. Broidy devised a plan to try to lift the
sanctions in exchange for \$11 million --- a deal that ultimately was
not pursued.

Separately, Mr. Broidy discussed helping to end a Justice Department
investigation into a flamboyant Malaysian financier who was
\href{https://www.nytimes.com/2018/08/24/world/asia/jho-low-malaysia-1mdb.html}{suspected
of embezzling billions of dollars} from a Malaysian investment fund.

The financier, Low Taek Jho, known as Jho Low, transferred \$6 million
to the law firm of Mr. Broidy's wife, Ms. Rosenzweig, to finance the
effort, according to
\href{https://www.nytimes.com/2018/11/30/us/politics/broidy-trump-foreign-money.html}{a
guilty plea for bank fraud} by a former Justice Department employee in a
related case.

Allies of Mr. Low also talked with Mr. Broidy about using his
connections to force the extradition of a
\href{https://www.nytimes.com/2018/01/10/magazine/the-mystery-of-the-exiled-billionaire-whistleblower.html}{Chinese
dissident living in the United States}, according to the court filings.

Mr. Broidy's lawyers said their client never discussed assisting Mr. Low
in any criminal matters and never lobbied to resolve the civil issues
facing the financier.

\hypertarget{a-key-partnership}{%
\subsection{A Key Partnership}\label{a-key-partnership}}

Mr. Trump took office signaling a new approach to the Middle East,
setting off a scramble by governments in the region to assure that their
voices would be heard by the new administration. A key figure in Mr.
Broidy's activities was Mr. Nader.

An American citizen born in Lebanon, Mr. Nader, 60, entered Mr. Broidy's
life at a fortuitous moment for both men and for Mr. Nader's patrons ---
primarily
\href{https://www.nytimes.com/2019/06/02/world/middleeast/crown-prince-mohammed-bin-zayed.html}{Crown
Prince Mohammed bin Zayed}, the de facto ruler of the United Arab
Emirates, though Mr. Nader also presented himself as a liaison to Saudi
Arabia's crown prince, Mohammed bin Salman.

To the princes, whose countries are closely allied, Mr. Broidy was a
perfect messenger to try to turn the new American administration against
Qatar.

And to Mr. Broidy, Mr. Nader was a perfect messenger to pitch Circinus's
services to the wealthy governments of the Emirates and Saudi Arabia.

Image

Rick Gates, the former deputy chairman of the Trump campaign, is one of
a number of Trump aides to have run into legal problems.Credit...Erin
Schaff for The New York Times

Not long after meeting at the Trump hotel during inauguration week, Mr.
Broidy and Mr. Nader were exchanging messages about Circinus's efforts
to win hundreds of millions of dollars' worth of defense contracts with
the Persian Gulf nations, and discussing the anti-Qatar campaign,
according to documents and interviews.

Mr. Nader wired Mr. Broidy \$2.4 million in three installments, starting
less than three months after the inauguration, for the anti-Qatar public
policy effort. Mr. Broidy contributed his own money, according to people
familiar with the campaign. They said other donors contributed as well.

Mr. Broidy donated to two Washington think tanks --- the Foundation for
Defense of Democracies and the Hudson Institute --- to fund conferences
he intended to be critical of Qatar. Featured speakers included the
former defense secretaries Mr. Panetta and Mr. Gates, as well as Mr.
Bannon and Mr. Petraeus.

Mr. Gates and Mr. Bannon were paid about \$100,000 each, while Mr.
Petraeus was paid \$50,000, according to interviews and contracts, which
stipulated that Mr. Gates and Mr. Petraeus would meet privately with Mr.
Broidy on the sidelines of the conference. The think tanks paid the
speakers and were reimbursed by Mr. Broidy. Mr. Nader helped arrange Mr.
Bannon's appearance,
\href{https://www.thedailybeast.com/accused-sex-trafficker-george-nader-helped-steve-bannon-land-dollar100k-payday}{The
Daily Beast reported}.

Mr. Broidy assured the think tanks that he was using only his own money
and that it was not from foreign sources, according to people familiar
with the conferences, who said he did not disclose that he was
simultaneously pursuing business in the region.

But updates sent by Mr. Broidy to Mr. Nader list Circinus as the entity
overseeing the advocacy campaign, which included plans for the
conferences, op-eds, articles and congressional and media outreach,
including to the Fox News host Sean Hannity, a favorite of Mr. Trump.

One update lists the Emirati and Saudi governments as the ``clients'' of
the campaign, and a senior Saudi general, Maj. Gen. Ahmed al-Assiri, who
would
\href{https://www.nytimes.com/2018/10/18/world/middleeast/jamal-khashoggi-killing-saudi-arabia.html}{later
be blamed} by his country's leadership for the killing of the journalist
Jamal Khashoggi, as a consultant. Mr. Broidy's lawyers say that the
updates were early drafts and that references to the involvement of
Circinus and the Saudi and Emirati governments were errors that were
corrected in subsequent drafts.

Banking records obtained by The Times show that, months after the first
think-tank conference, and days before the second, Mr. Nader received
the first of two payments of about \$5 million worth of Emirati currency
from an entity controlled by the government of the United Arab Emirates.

``Any payments by the U.A.E. to Mr. Nader had absolutely nothing to do
with the conferences or the broader educational initiative,'' said Tim
McCarten, a lawyer with the firm Latham \& Watkins, who represents both
Mr. Nader and Mr. Broidy. Mr. McCarten declined to specify the purpose
of the payments.

The second \$5 million payment came months after Mr. Nader began
\href{https://www.nytimes.com/2018/03/06/us/politics/george-nader-special-counsel-mueller-cooperating-seychelles.html}{cooperating
with prosecutors} looking into whether Emirati money was funneled into
Mr. Trump's political operation.

The Justice Department has asked witnesses about the funding of the
anti-Qatar campaign, as well as whether foreign money flowed into Mr.
Trump's inaugural.

In April, federal prosecutors in Brooklyn issued a subpoena for
documents from the inaugural committee naming Mr. Broidy and companies
with which he is associated, as well as Mr. Nader. Among others named
were Mr. Dragnea, the Angolan politician Mr. Sango and Angola's current
president, João Lourenço. Mr. Lourenço previously served as the head of
the Angolan Defense Ministry, and was also invited by Mr. Broidy to
attend the inauguration, but did not go, according to the Angolan
diplomats.

Mr. Nader was
\href{https://www.nytimes.com/2019/06/03/us/politics/george-nader-child-pornography-arrest.html}{charged
in June} with possession of child pornography, to which he has pleaded
not guilty.

Image

Leon E. Panetta, a former defense secretary, is among the prominent
figures Mr. Broidy enlisted to advance the interests of his companies,
his clients or his causes.Credit...Damon Winter/The New York Times

\hypertarget{putting-washington-to-work}{%
\subsection{Putting Washington to
Work}\label{putting-washington-to-work}}

The direct impact of the anti-Qatar advocacy campaign is not clear. It
coincided with Mr. Trump's
\href{https://www.nytimes.com/2017/06/06/world/middleeast/trump-qatar-saudi-arabia.html}{public
criticism of Qatar}, and his expression of support for Qatar's rivals,
the Emiratis and the Saudis, though his administration attempted to walk
back some of the criticism.

Mr. Broidy paid \$10,000 a month to a Democratic firm, Bluelight
Strategies, which worked to harness the center-left to press the
administration to be tough on Qatar, according to emails and interviews.

Mr. Broidy gave \$25,000 to a nonprofit group called the Jewish
Institute for National Security of America to write op-eds and host news
conferences criticizing Qatar, including with a retired Air Force
general, Charles F. Wald.

Another nonprofit listed by Mr. Broidy as part of the advocacy campaign,
the American Media Institute, received \$240,000 from Mr. Broidy in
2017, according to its
\href{https://www.documentcloud.org/documents/6234112-American-Media-Institute-s-2017-Tax-Filing.html}{tax
returns}. Mr. Broidy and his allies were in close contact with the
group's staff as it produced articles and op-eds that advanced the
interests of his clients and prospective clients, including the
government of Malaysia, while criticizing their rivals, including Qatar
and the Chinese dissident.

Richard Miniter, the institute's chief executive, said its decisions
were based on news judgment, rather than Mr. Broidy's wishes. ``We get
tons of ideas from both donors and nondonors, but there were no
conditions on the grant to do those stories,'' he said.

Mr. Miniter said he was unaware before being alerted by The Times of
overlap between Mr. Broidy's business and the subjects he wanted
covered.

In correspondence around the time of the Hudson Institute conference,
Mr. Broidy cited Mr. Panetta and General Wald --- as well as General
McChrystal --- as members of Circinus's team.

The men or their representatives say those claims were exaggerated or
false.

General McChrystal acknowledged that he accompanied Mr. Broidy and his
team on a trip to the Middle East, where they met with Prince Mohammed
bin Zayed in the summer of 2017.

The trip came after Mr. McChrystal was offered \$100,000 by Mr. Broidy,
according to documents and interviews.

When Mr. Broidy later dropped the general's name in the Oval Office, Mr.
Trump interjected to say that ``he thinks highly of General
McChrystal,'' according to Mr. Broidy's readout.

Mr. McChrystal said he accompanied Mr. Broidy to the United Arab
Emirates because it seemed as if his company was pursuing worthwhile
work. But he said he declined a subsequent offer for a leadership role
in the company because ``it didn't fit into my time or my interests to
do any more.''

Mr. Panetta's office said he ``is not and has never been involved in''
Mr. Broidy's business.

General Wald said he turned down Mr. Broidy's invitation to join
Circinus because he felt the company's work was ``mercenary,'' and
because of concerns about Mr. Broidy.

``Broidy is playing for both political and financial reasons,'' he said,
``and it's hard to figure out which one he is interested in mostly.''

Advertisement

\protect\hyperlink{after-bottom}{Continue reading the main story}

\hypertarget{site-index}{%
\subsection{Site Index}\label{site-index}}

\hypertarget{site-information-navigation}{%
\subsection{Site Information
Navigation}\label{site-information-navigation}}

\begin{itemize}
\tightlist
\item
  \href{https://help.nytimes.com/hc/en-us/articles/115014792127-Copyright-notice}{©~2020~The
  New York Times Company}
\end{itemize}

\begin{itemize}
\tightlist
\item
  \href{https://www.nytco.com/}{NYTCo}
\item
  \href{https://help.nytimes.com/hc/en-us/articles/115015385887-Contact-Us}{Contact
  Us}
\item
  \href{https://www.nytco.com/careers/}{Work with us}
\item
  \href{https://nytmediakit.com/}{Advertise}
\item
  \href{http://www.tbrandstudio.com/}{T Brand Studio}
\item
  \href{https://www.nytimes.com/privacy/cookie-policy\#how-do-i-manage-trackers}{Your
  Ad Choices}
\item
  \href{https://www.nytimes.com/privacy}{Privacy}
\item
  \href{https://help.nytimes.com/hc/en-us/articles/115014893428-Terms-of-service}{Terms
  of Service}
\item
  \href{https://help.nytimes.com/hc/en-us/articles/115014893968-Terms-of-sale}{Terms
  of Sale}
\item
  \href{https://spiderbites.nytimes.com}{Site Map}
\item
  \href{https://help.nytimes.com/hc/en-us}{Help}
\item
  \href{https://www.nytimes.com/subscription?campaignId=37WXW}{Subscriptions}
\end{itemize}
