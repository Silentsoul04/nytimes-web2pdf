Sections

SEARCH

\protect\hyperlink{site-content}{Skip to
content}\protect\hyperlink{site-index}{Skip to site index}

\href{https://www.nytimes.com/section/style}{Style}

\href{https://myaccount.nytimes.com/auth/login?response_type=cookie\&client_id=vi}{}

\href{https://www.nytimes.com/section/todayspaper}{Today's Paper}

\href{/section/style}{Style}\textbar{}Beyond Androgyny: Nonbinary
Teenage Fashion

\href{https://nyti.ms/2H4pd92}{https://nyti.ms/2H4pd92}

\begin{itemize}
\item
\item
\item
\item
\item
\end{itemize}

Advertisement

\protect\hyperlink{after-top}{Continue reading the main story}

Supported by

\protect\hyperlink{after-sponsor}{Continue reading the main story}

cultural studies

\hypertarget{beyond-androgyny-nonbinary-teenage-fashion}{%
\section{Beyond Androgyny: Nonbinary Teenage
Fashion}\label{beyond-androgyny-nonbinary-teenage-fashion}}

Clothes make the person. Maybe now parents understand.

By Hayley Krischer

\begin{itemize}
\item
  Aug. 14, 2019
\item
  \begin{itemize}
  \item
  \item
  \item
  \item
  \item
  \end{itemize}
\end{itemize}

\includegraphics{https://static01.nyt.com/images/2019/08/15/fashion/15NONBINARYTEENS-zai/merlin_159180222_5f79fc8f-0989-4485-ad19-2f4a1b9f839b-articleLarge.jpg?quality=75\&auto=webp\&disable=upscale}

Anna Kinlock, 17, was at the Brooklyn Museum the other day wearing
mid-calf black leather platform boots with small silver spikes, buckle
straps and five-inch heels. Fishnet stockings, a mini gray lace dress
and a long black wool cardigan completed the look. Anna identifies as
queer and non-binary and uses neutral pronouns.

Winged black eyeliner was precisely drawn past the corner of their
eyelid; layers of silver and gold chains draped around their neck. Anna
is passionate about androgynous fashion.

They speak about ``androgynous ancestors'' like the godmother of goth,
Siouxsie Sioux; Lydia Lunch; and Peter Murphy, the lead singer of
Bauhaus whose look gained cult status in the 1980s.

Fishnet stockings and black platform boots used to mean you were a goth
chick. But fashion definitions have changed. So have gendered
descriptions. (Just try using ``chick'' without irony to refer to
``woman'' in the public square.)

Mx. Kinlock, an intern at the Brooklyn Museum's gender and sexuality
teen program, InterseXtions, was at the museum not long ago to prepare
for the programs' sixth annual LGBTQ+ Teen Night. Last year, they put
together a talk on androgyny and queer fashion, inviting Chella Man, an
activist, actor and artist, as well as handing out a `zine filled with
pictures of the singers David Bowie and FKA twigs and the self-described
``\href{https://www.gq.com/story/ezra-miller-gq-style-cover-story}{gender-fluid}''
actor Ezra Miller.

They spoke excitedly about the Bowie exhibit at the museum last year.
About the goth and punk movements of the 1980s, and how androgynous
fashion has come from a history of people locating themselves outside
the mainstream.

Despite stares and catcalls on the street, Anna is excited to use
clothes as a method of empowerment. ``Androgynous fashion isn't only
about looking boxy, and flow and looking ambiguous,'' they said. ``The
androgynous fashion movement is about expressing yourself without the
confines of gender.''

\hypertarget{whats-new}{%
\subsection{What's New?}\label{whats-new}}

In the beginning of this decade, gender confines felt fixed. Nonbinary
was hardly part of the lexicon. Androgyny was reserved for subcultures
and didn't have a place in the teen and tween marketing machine.

The early aughts were focused on hyper-femininity.
\href{https://slate.com/human-interest/2012/05/pre-teen-runway-models-tracking-the-trend.html}{Very
young models} walked the runway. A Vogue Paris
\href{https://jezebel.com/fashion-industry-salivates-over-creepy-photos-of-10-yea-5827092}{2011
photo shoot} featured a 10-year-old
\href{http://www.jezebel.com/5827092/fashion-industry-salivates-over-creepy-photos-of-10+year+old-french-girl}{Thylane
Blondeau} in heavy makeup, staring at the camera with a come-hither
look.

Peggy Orenstein's book ``Cinderella Ate my Daughter,'' about raising a
girl in a culture of Disney and sparkly, was on the New York Times
best-seller list. Kylie Jenner was on her way to creating a newer,
hyper-femme version of herself with a little army of Kylies to follow.

\href{https://www.nytimes.com/2010/06/13/magazine/13fob-wwln-t.html}{Articles}
and
\href{https://wp.nyu.edu/steinhardt-appliedpsychology/wp-content/uploads/sites/72/2015/10/Guidance-Expo-Sexualization-Final.pdf}{psychological
reports} called for the media to stop oversexualizing and
hyperfeminizing young girls. It was time for girls to go back to being
tomboys, many adults felt. But was
``\href{https://www.nytimes.com/2017/04/18/opinion/my-daughter-is-not-transgender-shes-a-tomboy.html}{tomboy}''
even the right word anymore?

The word ``nonbinary''
\href{https://trends.google.com/trends/explore?date=all\&geo=US\&q=nonbinary}{became
something people asked the internet about} around 2014, making a steady
upward climb to present day. Gender identity has become an international
conversation, especially among teenagers. In 2017, a
\href{https://williamsinstitute.law.ucla.edu/wp-content/uploads/CHIS-Transgender-Teens-FINAL.pdf}{University
of California, Los Angeles study found} that 27 percent (796,000) of
California youth between the ages of 12-17 believed they were seen by
others as gender nonconforming.

More teenagers overall are identifying with nontraditional gender
labels, according to a March 2018 study published in the journal
\href{https://pediatrics.aappublications.org/content/141/3/e20171683}{Pediatrics}.
Some progressive synagogues and Jewish communities are holding
\href{https://www.nytimes.com/2019/03/27/style/gender-fluid-bar-bat-mitzvah.html}{nonbinary
mitzvahs}. Nonbinary teenagers are
\href{https://www.nytimes.com/2019/05/29/us/nonbinary-drivers-licenses.html}{choosing
non-gendered for driver's licenses}.

``When we're looking at trends that we might see in the community of
youth who are identifying as nonbinary, what we really are seeing is a
community of people who are just accepting the diversity of gender
expression,'' said Jeremy Wernick, a clinical assistant professor in the
department of child and adolescent psychiatry at N.Y.U. Langone. Mr.
Wernick's work focuses on gender-expansive children and adolescents.

``Yes, nonbinary kiddos are sort of leading the way in pushing the
boundaries of those binary stereotypes,'' Mr. Wernick said. ``But what
they're really doing is modeling for other young people and adults the
reality that gender expression can inevitably have an impact on the rest
of the world if things are accepted and celebrated.''

\hypertarget{gender-is-over}{%
\subsection{Gender Is Over}\label{gender-is-over}}

Just because a teenager is painting their nails a certain way or
trimming a beard a certain way, he added, doesn't mean they're going to
develop any specific type of gender identity.

But still, clothes, makeup and hair are manifesting this change. Because
if people don't have to exist in a binary, then why should fashion?

One photograph that sums up the nonbinary youth movement can be found on
the \href{https://www.instagram.com/p/Bqk3o6SHq7h/}{Instagram account}
of Lachlan Watson, an 18-year-old actor who stars in ``Chilling
Adventures of Sabrina'' and who, in the photo wears a John Lennon- and
Yoko Ono-inspired T-shirt that reads: ``Gender is over. If you want
it.''

Billie Eilish, the 17-year-old whose music has been streamed more than a
billion times, is a current focus of the teenage gaze. Ms. Eilish is the
anti-Britney Spears, the anti-Katy Perry. (Though, in a 2017
\href{https://www.vogue.com/article/katy-perry-interview-religion-childhood-may-vogue-cover}{Vogue}
article, even Ms. Perry announced wanting to transcend ``cutesy'' and
that she was going for more ``androgynous, architectural'' looks.) Ms.
Eilish calls gender roles
``\href{https://www.thefader.com/2019/03/05/billie-eilish-cover-story}{ancient}.''

She is known for wearing baggy, oversize clothing and on the red carpet
wears
\href{https://www.usatoday.com/story/life/entertainthis/2019/05/14/billie-eilish-says-she-wears-baggy-clothes-so-people-cant-body-shame/3664386002/}{gigantic
jackets} and big furry pants. Onstage, Ms. Eilish is often seen wearing
hoodies, large athletic-looking shorts and tube socks.

In a recent \href{https://www.youtube.com/watch?v=JeMmUglv6wA}{Calvin
Klein ad campaign}, she wore her go-to oversize look. ``I never want the
world to know everything about me,'' she says softly in the
\href{https://twitter.com/CalvinKlein/status/1126937436477632512}{video},
gazing into the mirror, her green hair matching the color of her eyes,
in stark contrast to a teenage Brooke Shields's ads for the same brand
in the 1980s. ``That's why I wear big baggy clothes.''

Image

Boxy is beautiful: Billie Eilish.Credit...Scott Dudelson/Getty Images

Ms. Eilish's stylist, Samantha Burkhart, who also dresses Sia in unisex
silhouettes and who dressed
\href{https://people.com/style/kesha-grammys-2018-suit/}{Kesha} in the
suit that she wore to the Grammys in 2018, simply doesn't feel
comfortable putting the teenager in a dress or even in women's clothing.
``I think it didn't feel like who she was,'' Ms. Burkhart said. ``The
gender stereotypes of clothing just didn't seem to encapsulate her.''

Fashion designers, who have resisted sending men's wear to some of Ms.
Burkhart's other female clients, have had no problem sending selections
from their men's wear collection to Ms. Eilish. ``They see she's the
future of things that are going on, and there's something really nice
about it,'' Ms. Burkhart said.

\hypertarget{fashion-for-all}{%
\subsection{Fashion for All}\label{fashion-for-all}}

For nonbinary teenagers, Ms. Eilish is a revelation. Zai Nixon-Reid, 19,
a student at the New School, who is female-aligned nonbinary and goes by
the pronouns they/them, says people often compared their style to Ms.
Eilish's androgynous look.

``It's definitely why I like her, because everything she wears is really
oversized and that's kind of how I wish my closet was,'' they said. To
scroll through Mx. Nixon-Reid's
\href{https://www.instagram.com/pixieshawty/}{Instagram} is to see a
style that consists of oversize buttoned shirts and chains, but also
suits and scarves and fedora hats.

Expressing themselves through fashion is something new for Mx.
Nixon-Reid. It wasn't always easy. Their mother is hyper-feminine, so
most of their childhood clothes had been traditionally female.

``These days, sort of at the end of 2018, I've been able to explore
gender through fashion and it's helped me understand my own gender
through clothes,'' they said.

The moment now is that mall fixtures like H \& M carry unisex lines, but
gender nonconforming youth are still at high risk for
\href{https://www.hrc.org/blog/new-cdc-report-highlights-need-to-support-trans-and-gender-non-conforming-y}{bullying
and suicide}, in both in cosmopolitan areas and, especially, outside of
them. In other words, a goth androgynous person may appear, as the kids
say, dope, in Brooklyn, but could easily be a target somewhere else.

Deborah Tolman, a psychology professor at the City University of New
York whose work focuses on teenage sexuality, thinks this wider-spread
fashion movement is, for many teenagers, about playing with masculinity
and femininity ``while maintaining it at the same time.'' True
androgyny, she said, would suggest that the binary goes away. That there
is no binary.

Dr. Tolman called what is happening now ``queering'' fashion, because
when you ``queer'' something --- fashion, whatever --- you're getting
out of those boxes. ``And the point of queering things is not to be in
those boxes,'' she said. ``Because if you keep your head in the boxes,
you can't actually think about this.''

Many Gen X parents, raised on ``Free to Be You and Me,'' were determined
to break gender stereotypes. They dressed their baby daughters in black.
They rejected pink. They read books like ``My Princess Boy'' and
``Jacob's New Dress.''

And yet now they are forced to reckon with having become the
finger-wagging, clueless adults in DJ Jazzy Jeff and the Fresh Prince's
``\href{https://www.youtube.com/watch?v=jW3PFC86UNI}{Parents Just Don't
Understand.}''

``Now these kids are teenagers and young adults,'' said Jo B. Paoletti,
an emeritus professor of American studies at the University of Maryland
in College Park, who has written extensively on the topic of gender and
children's clothing. She first started seeing pushback against the
pink-blue binary (itself
\href{https://www.smithsonianmag.com/arts-culture/when-did-girls-start-wearing-pink-1370097/}{historically
arbitrary}), in the early aughts, around the time she was researching
her book ``Pink and Blue: Telling the Boys from the Girls in America.''
``It's been part of a conversation that's been going on their whole
lives,'' she said.

That's at least how it feels for Isobel Middleton, 14, who wore a forest
green baggy camp sweatshirt, a pair of loose jeans, short hair and
glasses the other day her house in Glen Ridge, N. J. She has always had
an aversion to Daisy Dukes and cold shoulders. ``I just couldn't wear
those,'' she said.

Last summer she asked her mother, Rebecca, to get her ``man pants'' for
a summer concert and so she did what any mother would do to please her
teenage daughter. She shopped for a pair of pants for her daughter in
the men's section. They were a perfect fit.

Advertisement

\protect\hyperlink{after-bottom}{Continue reading the main story}

\hypertarget{site-index}{%
\subsection{Site Index}\label{site-index}}

\hypertarget{site-information-navigation}{%
\subsection{Site Information
Navigation}\label{site-information-navigation}}

\begin{itemize}
\tightlist
\item
  \href{https://help.nytimes.com/hc/en-us/articles/115014792127-Copyright-notice}{©~2020~The
  New York Times Company}
\end{itemize}

\begin{itemize}
\tightlist
\item
  \href{https://www.nytco.com/}{NYTCo}
\item
  \href{https://help.nytimes.com/hc/en-us/articles/115015385887-Contact-Us}{Contact
  Us}
\item
  \href{https://www.nytco.com/careers/}{Work with us}
\item
  \href{https://nytmediakit.com/}{Advertise}
\item
  \href{http://www.tbrandstudio.com/}{T Brand Studio}
\item
  \href{https://www.nytimes.com/privacy/cookie-policy\#how-do-i-manage-trackers}{Your
  Ad Choices}
\item
  \href{https://www.nytimes.com/privacy}{Privacy}
\item
  \href{https://help.nytimes.com/hc/en-us/articles/115014893428-Terms-of-service}{Terms
  of Service}
\item
  \href{https://help.nytimes.com/hc/en-us/articles/115014893968-Terms-of-sale}{Terms
  of Sale}
\item
  \href{https://spiderbites.nytimes.com}{Site Map}
\item
  \href{https://help.nytimes.com/hc/en-us}{Help}
\item
  \href{https://www.nytimes.com/subscription?campaignId=37WXW}{Subscriptions}
\end{itemize}
