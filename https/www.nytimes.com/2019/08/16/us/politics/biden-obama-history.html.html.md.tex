Sections

SEARCH

\protect\hyperlink{site-content}{Skip to
content}\protect\hyperlink{site-index}{Skip to site index}

\href{https://www.nytimes.com/section/politics}{Politics}

\href{https://myaccount.nytimes.com/auth/login?response_type=cookie\&client_id=vi}{}

\href{https://www.nytimes.com/section/todayspaper}{Today's Paper}

\href{/section/politics}{Politics}\textbar{}Obama and Biden's
Relationship Looks Rosy. It Wasn't Always That Simple.

\url{https://nyti.ms/2KBMDob}

\begin{itemize}
\item
\item
\item
\item
\item
\item
\end{itemize}

\begin{itemize}
\item
  \href{https://www.nytimes.com/2020/07/31/us/elections/biden-vs-trump.html?action=click\&pgtype=Article\&state=default\&region=TOP_BANNER\&context=storylines_menu}{Election
  Updates}
\item
  \href{https://www.nytimes.com/article/biden-vice-president-2020.html?action=click\&pgtype=Article\&state=default\&region=TOP_BANNER\&context=storylines_menu}{Biden's
  V.P. Search}
\item
  \href{https://www.nytimes.com/interactive/2020/07/24/us/politics/trump-biden-campaign-donors.html?action=click\&pgtype=Article\&state=default\&region=TOP_BANNER\&context=storylines_menu}{Map
  of Donations}
\item
  \href{https://www.nytimes.com/interactive/2020/us/elections/delegate-count-primary-results.html?action=click\&pgtype=Article\&state=default\&region=TOP_BANNER\&context=storylines_menu}{Delegate
  Count}
\item
  \href{https://www.nytimes.com/interactive/2019/us/politics/2020-presidential-candidates.html?action=click\&pgtype=Article\&state=default\&region=TOP_BANNER\&context=storylines_menu}{The
  Candidates}
\item
  \href{https://www.nytimes.com/newsletters/politics?action=click\&pgtype=Article\&state=default\&region=TOP_BANNER\&context=storylines_menu}{Politics
  Newsletter}
\end{itemize}

Advertisement

\protect\hyperlink{after-top}{Continue reading the main story}

Supported by

\protect\hyperlink{after-sponsor}{Continue reading the main story}

The Long Run

\hypertarget{obama-and-bidens-relationship-looks-rosy-it-wasnt-always-that-simple}{%
\section{Obama and Biden's Relationship Looks Rosy. It Wasn't Always
That
Simple.}\label{obama-and-bidens-relationship-looks-rosy-it-wasnt-always-that-simple}}

\includegraphics{https://static01.nyt.com/images/2019/08/17/us/politics/00BidenObama1/00BidenObama1-articleLarge.jpg?quality=75\&auto=webp\&disable=upscale}

By \href{https://www.nytimes.com/by/glenn-thrush}{Glenn Thrush}

\begin{itemize}
\item
  Aug. 16, 2019
\item
  \begin{itemize}
  \item
  \item
  \item
  \item
  \item
  \item
  \end{itemize}
\end{itemize}

WASHINGTON ---
\href{https://www.nytimes.com/2020/07/23/arts/television/biden-obama-reunion-video.html}{Barack
Obama} was riding his call for generational change to the Democratic
presidential nomination in the spring of 2008 when he began musing about
potential running mates with aides traveling with him on the trail.

\emph{{[}Election 2020:}
\href{https://www.nytimes.com/2020/07/23/us/politics/barack-obama-joe-biden-video.html}{\emph{Joe
Biden and Barack Obama join forces against Trump}}\emph{.{]}}

``I want somebody with gray in his hair,''
\href{https://www.nytimes.com/2020/04/14/us/politics/obama-endorses-biden.html}{Mr.
Obama}, then 46, told one of them. He was thinking about an ``older
guy,'' he told another.

That older guy, people around the candidate would soon learn, was
\href{https://www.nytimes.com/2020/07/23/arts/television/biden-obama-reunion-video.html}{Joseph
R. Biden Jr.}, 65, a has-been to pundits but to Mr. Obama a sweet-spot
pick --- a policy heavyweight with limited political horizons, assuming
that would ensure loyalty and minimal drama. Mr. Obama was already
phoning Mr. Biden two or three times a week to solicit advice, and to
decide whether the Delaware senator's many positive attributes
outweighed his singular liability, a notoriously self-tangling tongue.

Over the next several months, Mr. Obama's top advisers would present 30
alternatives, all of whom he respectfully considered. But his preference
was clear from the start. When it came time to decide in August, Mr.
Obama chose Mr. Biden over two younger finalists --- Tim Kaine, the
governor of Virginia, and Senator Evan Bayh of Indiana, peers in the
mold of Bill Clinton's choice of Al Gore in 1992.

``You are the pick of my heart, but Joe is the pick of my head,'' Mr.
Obama told Mr. Kaine after he made his choice, according to people with
knowledge of the exchange.

Eleven years later, Mr. Obama's cautions and calculations have come to
roost.

Mr. Obama, standard-bearer of change but conscious of the racial
dynamics of his candidacy, was wary of asking voters to digest too much
at once. In Mr. Biden, he found a running mate who would conjure the
comforting past and provide experience he did not possess, but would not
maneuver for the presidency from the No. 2 slot.

While Mr. Biden exceeded the first two expectations, he never abandoned
his aspirations for the top job. He has leveraged his steady vice
presidency into a fragile front-runner status in the 2020 primary, at
the even more advanced, and politically vulnerable, age of 76.

What's more, the choice of Mr. Biden as a hedge against change has left
the demographically and ideologically evolving Democrats profoundly
divided as they desperately seek to unseat President Trump. Even as Mr.
Biden casts himself as the man to complete and cement the Obama legacy,
that legacy has moved to the center of the Democrats' fractious debate.

While paying homage to Mr. Obama, who remains popular among Democratic
voters, many candidates, including Senators Elizabeth Warren, Bernie
Sanders, Kamala Harris and Cory Booker, are urging the party to move far
beyond his administration's positions on immigration, criminal justice,
health care, the regulation of banks, the environment, income inequality
and race, which they now view as timid, conservative or dated.

``You invoke President Obama more than anybody in this campaign. You
can't do it when it's convenient and then dodge it when it's not,'' Mr.
Booker said to Mr. Biden during the
\href{https://www.nytimes.com/2019/07/31/us/politics/live-updates-democratic-debate.html?module=inline}{most
recent Democratic debate}.

The Obama-Biden origin story has been often told, and often
sentimentalized. But a re-examination at this crystallizing moment of
the primary campaign, based on more than two dozen interviews with Obama
and Biden aides and others with knowledge of the relationship, reveals a
more complicated dynamic between the two men, and one that is still
evolving.

Mr. Biden and his advisers initially thought he might be a better fit as
Mr. Obama's secretary of state, and he bridled at the Obama campaign's
attempt to control his every utterance and personnel move. He exploded
when campaign researchers began asking questions about the private life
of his family, especially his younger son, Hunter.

Mr. Obama, for his part, took a long time to warm to Mr. Biden, and kept
him at arms' length, and on a leash, in the early days. Up until earlier
this year, he suggested Mr. Biden would be better off sticking with his
vague promise, made during the audition for the vice presidency, that
his short-lived 2008 presidential campaign would be his last.

That has changed: While initially skeptical of Mr. Biden's decision to
run, Mr. Obama, driven by affection and loyalty, has been more active in
advising his campaign than previously known --- going so far as to
request a briefing from the campaign before his friend officially joined
the fray, according to people close to both men.

``It's an incredible turn of events, when you think about it,'' said Mr.
Bayh, who retired from the Senate in 2011. ``The question then was, `Do
you happen to fit the moment?' The question now is, `After all these
years, can you turn yourself into an independent source of power, as
opposed to being just a loyal and faithful wingman?' Only time will
tell.''

\includegraphics{https://static01.nyt.com/images/2019/08/17/us/politics/00BidenObama2/merlin_158790564_78c8f85b-842a-4daf-bd11-0905d0307dcf-articleLarge.jpg?quality=75\&auto=webp\&disable=upscale}

\hypertarget{a-rocky-start}{%
\subsection{A Rocky Start}\label{a-rocky-start}}

Mr. Biden ran for president in 2008 because he thought he could do a
better job than anyone else, saw no real downside, and as chairman of
the powerful Senate Foreign Relations Committee wanted to advance a
cherished policy idea: a plan for partitioning Iraq into three ethnic
enclaves.

\hypertarget{latest-updates-2020-election}{%
\section{\texorpdfstring{\href{https://www.nytimes.com/2020/07/31/us/elections/biden-vs-trump.html?action=click\&pgtype=Article\&state=default\&region=MAIN_CONTENT_1\&context=storylines_live_updates}{Latest
Updates: 2020
Election}}{Latest Updates: 2020 Election}}\label{latest-updates-2020-election}}

Updated 2020-08-01T01:26:45.732Z

\begin{itemize}
\tightlist
\item
  \href{https://www.nytimes.com/2020/07/31/us/elections/biden-vs-trump.html?action=click\&pgtype=Article\&state=default\&region=MAIN_CONTENT_1\&context=storylines_live_updates\#link-29fdff45}{Kamala
  Harris, a top vice-presidential contender, confronts double
  standards.}
\item
  \href{https://www.nytimes.com/2020/07/31/us/elections/biden-vs-trump.html?action=click\&pgtype=Article\&state=default\&region=MAIN_CONTENT_1\&context=storylines_live_updates\#link-13ec3d9c}{Karen
  Bass and Susan Rice are rising on Biden's vice-presidential
  shortlist.}
\item
  \href{https://www.nytimes.com/2020/07/31/us/elections/biden-vs-trump.html?action=click\&pgtype=Article\&state=default\&region=MAIN_CONTENT_1\&context=storylines_live_updates\#link-49e9a016}{Trump
  says Russian bounties to kill U.S. troops `never took place.'}
\end{itemize}

\href{https://www.nytimes.com/2020/07/31/us/elections/biden-vs-trump.html?action=click\&pgtype=Article\&state=default\&region=MAIN_CONTENT_1\&context=storylines_live_updates}{See
more updates}

``He felt a responsibility to do it,'' recalled Ted Kaufman, one of Mr.
Biden's oldest friends and advisers. ``He looked around at the potential
people who would run, and he concluded, `It's my time to run.' It wasn't
a complex set of decisions. If he lost, he lost.''

Others in Mr. Biden's orbit discerned a deeper motive: Here was his
final chance to exorcise the humiliating memories of a promising 1988
campaign demolished by reports he had plagiarized speeches.

{[}\emph{Read more about}
\href{https://www.nytimes.com/2019/06/03/us/politics/biden-1988-presidential-campaign.html}{\emph{how
Joe Biden's '88 campaign fell apart}}\emph{.}{]}

Mr. Biden drove his 2008 campaign from the lot directly into a pothole.
On the eve of its rollout, in January 2007, he told a reporter for The
New York Observer that Mr. Obama was ``the first mainstream
African-American who is articulate and bright and clean and a
nice-looking guy.'' Mr. Biden did not bother to tell any of his aides
that the interview had gone catastrophically wrong.

Mr. Obama laughed it off. But it did little to improve his overall
impression of Mr. Biden as condescending to him, forged when Mr. Obama
was assigned to the Foreign Relations Committee shortly after being
elected to the Senate in 2004. During their first private meeting, Mr.
Biden suggested the two men have dinner, and offered to pick up the
bill.

The freshman senator, who was earning millions from his memoirs, shot
back that he could afford to pay, according to Mr. Biden's retelling of
the episode in his own autobiography.

Their relationship had nowhere to go but up, and as the 2008 primary
race rolled on, it did. Mr. Biden's witty self-confidence and command of
policy at the debates elicited admiration from Mr. Obama, who struggled
with the format despite his greater oratorical gifts. Over time, Mr.
Obama dropped his guard a bit, and Mr. Biden treated him respectfully on
the debate stage.

A little too respectfully, in the eyes of one particular ally of Hillary
Clinton, who was also seeking the nomination. At one of the final
debates before the Iowa caucuses, Mr. Biden and Christopher Dodd, the
Connecticut senator who shared a plane with his friend Mr. Biden to cut
costs on their shoestring campaigns, were approached by Mr. Clinton.

``You know, I really thought you guys would do a better job pointing out
Obama's lack of experience,'' Mr. Dodd recalled Mr. Clinton saying.

Image

From left, Hillary Clinton, Mr. Biden and Senator Chris Dodd during the
final debate before the Iowa caucus.Credit...Charlie
Neibergall/Associated Press

Mr. Biden managed just 4 percent of the vote in Iowa on the night of
Jan. 3, 2008. His communications adviser, Larry Rasky, suggested he
soldier on in New Hampshire, and the candidate wavered for a moment. But
his sister, Valerie Owens Biden, shot it down, and he dropped out.

Mr. Biden slipped back to the Senate, and seemed at peace. Shortly after
dropping out, he approached Mr. Bayh, who had briefly considered running
for president, in the Senate gym. ``You were a whole hell of a lot
smarter than I was!'' Mr. Biden said, according to Mr. Bayh. ``I never
had a goddamn chance!''

Plus there was the pull of the Senate. One hallmark of the generational
difference between Mr. Biden and Mr. Obama was in their view of the
place. The younger man saw it as a steppingstone. Mr. Biden had spent 36
years of his life there, and shared a bygone belief that the ability to
work a single, stable job was at the heart of the American dream.

For all that, Mr. Biden had begun to hold what's-next strategy sessions
at his home in Delaware, known as the ``The Lake'' house with Mr.
Kaufman, his adviser; his sister; his older son, Beau; and his two most
trusted political advisers, Mike Donilon and John Marttila. In his
increasingly frequent phone calls, Mr. Obama suggested that he was
trying to figure out ``a way to get you into my administration one way
or another,'' according to a former Biden aide. The Biden team discussed
the possibility that he might be tapped to be secretary of state, said a
person involved in some of the talks.

It was around that time, in February or March, that Mr. Donilon raised
the idea of the vice presidency. Mr. Biden, according to two people in
his orbit, initially dismissed the idea, saying he had no interest in
being anyone's ``second banana.''

``It wasn't that easy for him to move on,'' said Terrell McSweeny, a
longtime Biden policy adviser. Still, ``he was starting to ask himself,
`What can I do that will make the biggest difference for my country?'''

Mr. Biden did not dissuade his people from exploring the opportunity.

Image

Mr. Biden with his son Beau at the 2008 Democratic convention. Mr.
Obama's advisers were impressed with Mr. Biden's ease and affection with
his family.Credit...Damon Winter/The New York Times

\hypertarget{a-difficult-decision}{%
\subsection{A Difficult Decision}\label{a-difficult-decision}}

By the summer, Mr. Obama's two top strategists, David Plouffe and David
Axelrod, had Mr. Biden at the top of their list. The choice was not just
about politics and optics. Mr. Obama, confident to the point of
cockiness about his political chops, was privately expressing anxiety
about his ability to govern --- conceding that Mrs. Clinton, his chief
rival for the nomination, had made valid points about his inexperience.

``He needed somebody in the Situation Room, and somebody who would deal
with Mitch so he wouldn't have to,'' said Mr. Axelrod, referring to
Mitch McConnell, the combative Senate Republican leader.

Mr. Biden's relationship with Mr. McConnell would come with its own
complications. In late 2012, Mr. Obama tapped his vice president to
negotiate one-on-one with Mr. McConnell what was known as the ``fiscal
cliff,'' a budget-cutting deal.

It produced one previously unreported incident that left White House and
Senate staff scratching their heads: During a follow-up meeting in the
Oval Office in early 2013, Mr. McConnell ruled out a big deal before the
2014 midterms, when he would be running for re-election in Kentucky. Mr.
Biden responded by saying, ``Mitch, we want to see you come back,'' an
off-the-cuff endorsement of one of their biggest adversaries.

``That was Joe Biden being Joe Biden,'' said Harry Reid, then the Senate
Democratic leader, who was in the room, adding that it was an attempt to
put Mr. McConnell at ease.

One of the first decisions Mr. Obama's search team made was to exclude
Mrs. Clinton from consideration, despite a tepid public claim that she
was in the running. The protracted primary fight had simply been too
bitter, and the president would soon offer her the State Department, to
put her near, but not so very near, the seat of power.

The extended list of hopefuls included Ohio's governor, Ted Strickland;
a moderate Texas congressman, Chet Edwards; and Mark Warner, a former
governor of Virginia who was running for Senate that year. Kathleen
Sebelius, the progressive governor of Kansas, was added later to
compensate for the gender imbalance. In the end, however, it came down
to Mr. Biden and two men who would have represented generational change:
Mr. Bayh, then 52, and Mr. Kaine, 50.

Image

Gov. Tim Kaine of Virginia with Mr. Obama in June 2008. ``You are the
pick of my heart, but Joe is the pick of my head,'' Mr. Obama told Mr.
Kaine after he made his choice.Credit...Mandel Ngan/Agence France-Presse
--- Getty Images

It was indicative of Mr. Obama's unsentimental approach that his
personal favorite, Mr. Kaine, finished third. The nominee viewed him as
too young and too unschooled in foreign affairs to help him in the
campaign or White House. Mr. Obama was also deeply worried about a
backlash against a black man at the top of the ticket, and believed that
an older white running mate would ease fears in battleground states like
Pennsylvania, Michigan and Indiana that he had lost in the primaries.

``Barack Hussein Obama is change enough for most people,'' Mr. Obama
said of passing over Mr. Kaine, according to Mr. Axelrod.

In late July, Mr. Plouffe and Mr. Axelrod embarked on a one-day trip
from the campaign's Chicago headquarters to audition all three, starting
with Mr. Biden in Delaware.

``Basically I said, `Forgive me for being so blunt, but how do we know
you know how to shut up?''' Mr. Axelrod recalled asking. ``An hour
later, he finished answering. So I asked him another question.''

\emph{{[}}\href{https://www.nytimes.com/2019/08/12/us/politics/joe-biden-gaffes.html?action=click\&module=Intentional\&pgtype=Article}{\emph{Joe
Biden has a long history of verbal flubs and gaffes. And he knows
it.}}\emph{{]}}

Mr. Biden was candid about his struggle to maintain verbal discipline,
and he repeatedly interrupted himself to ask, ``Am I making sense?'' But
the quantity of his advice was offset by its quality. Mr. Obama's
political magi were especially impressed with his insights into the
Republican nominee, Senator John McCain.

The former Navy pilot valued unpredictability, Mr. Biden said,
anticipating Mr. McCain's selection of Sarah Palin as his running mate
and Mr. McCain's disastrous decision to suspend his campaign that fall
to focus on the global financial crisis.

What most impressed Mr. Obama's advisers, however, was Mr. Biden's ease
with his family; he was comfortable expressing affection to his wife and
grown children in a way that most politicians, including Mr. Obama, were
not.

The intensity of those bonds would become apparent after Mr. Obama
picked Mr. Biden, and campaign researchers uncovered potential public
relations problems stemming from Mr. Biden's son Hunter, including
complications from his lobbying work and indications of marital, legal
and substance-abuse problems. (Those issues were examined in detail by
The New Yorker
\href{https://www.newyorker.com/magazine/2019/07/08/will-hunter-biden-jeopardize-his-fathers-campaign}{earlier
this year.})

When an Obama campaign official flagged the issue, Mr. Biden grew angry
and warned, ``Keep my family out of this.'' The issue was dropped,
according to a person involved in the vetting process.

Image

Senator Evan Bayh of Indiana and Mr. Obama in July 2008. Mr. Obama was
warned that picking Mr. Bayh would guarantee his Senate seat would flip
Republican.Credit...Michael Conroy/Associated Press

The talk later in the day with Mr. Bayh, who was vacationing at the tony
Greenbrier resort in West Virginia with his wife and young children, did
not go well. The visitors caught him barefoot, emerging from a shower
--- and assumed it was an attempt to appear Kennedyesque. In reality,
Mr. Bayh was less diffident than disoriented by being thrust into the
national spotlight.

Mr. Bayh had another major liability. Mr. Reid, the Senate Democratic
leader, had advised Mr. Obama that picking Mr. Bayh would guarantee his
Senate seat would flip Republican --- which could imperil the new
president's legislative agenda. Mr. Biden's seat in Democratic Delaware
was much safer.

The meeting with Mr. Kaine in Richmond was respectful, friendly and a
bit bittersweet. (Eight years later, Mr. Kaine would be Mrs. Clinton's
running mate in the losing campaign against Mr. Trump.)

In early August, Mr. Obama arranged to have Mr. Biden quietly shuttled
to his suite at the Graves 601 Hotel in Minneapolis, where he was
campaigning. The conversation lasted well into the night.

Mr. Obama agreed that Mr. Biden would be the final person he spoke to
before making a big decision, and the two men would have weekly lunches.
Mr. Biden also made a loyalty pledge that would become the basis of
their deeper personal bond. ``You make a decision, and I will follow it
to my death,'' Mr. Biden said, according to Mr. Kaufman.

At some point, Mr. Biden also told Obama aides that ``Barack would never
have to worry'' about him positioning himself for another presidential
run. He was too old, he told them, and he viewed his new job as a
capstone, not a catapult. But while both sides assumed that vow covered
the duration of Mr. Obama's presidency, what might happen after that was
never explicitly stated.

Mr. Biden was the only one of the finalists to make such a promise.
``That was helpful,'' Mr. Plouffe said.

Before parting, Mr. Obama popped a surprise, intended to test Mr.
Biden's commitment to being a wingman: ``Would you prefer being
secretary of state to vice president?'' he asked.

Mr. Biden chose the latter. Mr. Obama formally offered him the job after
he flew back to Washington. Neither man has ever spoken publicly about
exactly what was said, but one Biden aide who was watching the little
red switchboard light for the senator's private line said it stayed on
too long for it to have simply been a perfunctory call of
congratulations.

Image

In discussions about the 2020 campaign, Mr. Obama has expressed
frustration to Mr. Biden that his closest advisers are too old and out
of touch with the current political climate.Credit...Al Drago/The New
York Times

\hypertarget{a-protective-partner}{%
\subsection{A Protective Partner}\label{a-protective-partner}}

The next eight years are the stuff of buddy-movie lore --- ``a shotgun
marriage that gradually turned into a love story,'' in Mr. Axelrod's
telling.

Still, Mr. Biden's simmering ambition was a source of unease for both
men. Mr. Plouffe shut down an early move made by Mr. Biden as vice
president to assemble a presidential team-in-waiting, blocking Mr.
Biden's attempts to court the party's West Coast fund-raising elite and
rejecting an attempt to hire Kevin Sheekey, a veteran Democratic
operative.

In 2016, Mr. Obama quietly pressured Mr. Biden to sit out the race,
partly because he believed Mrs. Clinton had a better chance of building
on his agenda, and partly because he thought Mr. Biden was in no shape
emotionally following the illness and death of his son Beau in May 2015.

By now, the line between heart and head, between the personal and
political, so clear a decade ago, has blurred completely.

The two men spoke at least a half dozen times before Mr. Biden decided
to run, and Mr. Obama took pains to cast his doubts about the campaign
in personal terms.

``You don't have to do this, Joe, you really don't,'' Mr. Obama told Mr.
Biden earlier this year, according to a person familiar with the
exchange.

Mr. Biden --- who thinks he could have defeated Donald Trump four years
ago --- responded by telling Mr. Obama he could never forgive himself if
he turned down a second shot at Mr. Trump.

Mr. Obama has said he will not make an endorsement in the primary, and
has offered every candidate his counsel. But he has taken an active
interest in the inner workings of his friend's campaign, to an extent
beyond anything offered to other candidates.

In his interactions with Mr. Biden --- the pair had a quiet lunch in
Washington last month --- Mr. Obama has hammered away at the need for
his campaign to expand his aging inner circle.

He has communicated his frustration that Mr. Biden's closest advisers
are too old and out of touch with the current political climate ---
urging him to include more younger aides, according to three Democrats
with direct knowledge of the discussion.

In March, Mr. Obama took the unusual step of summoning Mr. Biden's top
campaign advisers, including the former White House communications
director Anita Dunn and Mr. Biden's longtime spokeswoman, Kate
Bedingfield, to his Washington office for a briefing on the campaign's
digital and communications strategy with members of his own staff,
including his senior adviser, Eric Schultz.

When they were done, Mr. Obama offered a pointed reminder, according to
two people with knowledge of his comments:

Win or lose, they needed to make sure Mr. Biden did not ``embarrass
himself'' or ``damage his legacy'' during the campaign.

\hypertarget{our-2020-election-guide}{%
\section{Our 2020 Election Guide}\label{our-2020-election-guide}}

Updated July 31, 2020

\begin{itemize}
\item
  \begin{center}\rule{0.5\linewidth}{\linethickness}\end{center}

  \hypertarget{the-latest}{%
  \subsection{The Latest}\label{the-latest}}

  \begin{itemize}
  \tightlist
  \item
    President Trump's assault on the Postal Service is intersecting with
    his attacks on mail-in voting.
    \href{https://www.nytimes.com/2020/07/31/us/politics/trump-usps-mail-delays.html?action=click\&pgtype=Article\&state=default\&region=BELOW_MAIN_CONTENT\&context=storylines_guide}{Voting
    rights groups say it is a recipe for disaster.}
  \end{itemize}
\item
  \begin{center}\rule{0.5\linewidth}{\linethickness}\end{center}

  \hypertarget{bidens-vp-search}{%
  \subsection{Biden's V.P. Search}\label{bidens-vp-search}}

  \begin{itemize}
  \tightlist
  \item
    \href{https://www.nytimes.com/article/biden-vice-president-2020.html?action=click\&pgtype=Article\&state=default\&region=BELOW_MAIN_CONTENT\&context=storylines_guide}{Here
    are 13 women} who have been under consideration to be Joe Biden's
    running mate, and why each might be chosen --- and might not be.
  \end{itemize}
\item
  \begin{center}\rule{0.5\linewidth}{\linethickness}\end{center}

  \hypertarget{keep-up-with-our-coverage}{%
  \subsection{Keep Up With Our
  Coverage}\label{keep-up-with-our-coverage}}

  \begin{itemize}
  \tightlist
  \item
    Get an
    \href{https://www.nytimes.com/newsletters/politics?action=click\&pgtype=Article\&state=default\&region=BELOW_MAIN_CONTENT\&context=storylines_guide}{email}
    recapping the day's news
  \end{itemize}

  \begin{itemize}
  \tightlist
  \item
    Download our mobile app on
    \href{https://apps.apple.com/us/app/nytimes/id284862083?ls=1\&mat_click_id=5c79ae7455014fd1bd66b5610c05b8f2-20191112-16948\&referrer=mat_click_id\%3D5c79ae7455014fd1bd66b5610c05b8f2-20191112-16948\%26link_click_id\%3D722930677036718082}{iOS}
    and
    \href{http://a.localytics.com/android?id=com.nytimes.android\&referrer=utm_source\%3Dother_nyt_mobile_web\%26utm_medium\%3DWeb\%2520page\%26utm_term\%3DGeneral\%2520Mobile\%2520Page\%26utm_campaign\%3DNYT\%2520Mobile\%2520General\%2520Page}{Android}
    and turn on Breaking News and Politics alerts
  \end{itemize}
\end{itemize}

Advertisement

\protect\hyperlink{after-bottom}{Continue reading the main story}

\hypertarget{site-index}{%
\subsection{Site Index}\label{site-index}}

\hypertarget{site-information-navigation}{%
\subsection{Site Information
Navigation}\label{site-information-navigation}}

\begin{itemize}
\tightlist
\item
  \href{https://help.nytimes.com/hc/en-us/articles/115014792127-Copyright-notice}{©~2020~The
  New York Times Company}
\end{itemize}

\begin{itemize}
\tightlist
\item
  \href{https://www.nytco.com/}{NYTCo}
\item
  \href{https://help.nytimes.com/hc/en-us/articles/115015385887-Contact-Us}{Contact
  Us}
\item
  \href{https://www.nytco.com/careers/}{Work with us}
\item
  \href{https://nytmediakit.com/}{Advertise}
\item
  \href{http://www.tbrandstudio.com/}{T Brand Studio}
\item
  \href{https://www.nytimes.com/privacy/cookie-policy\#how-do-i-manage-trackers}{Your
  Ad Choices}
\item
  \href{https://www.nytimes.com/privacy}{Privacy}
\item
  \href{https://help.nytimes.com/hc/en-us/articles/115014893428-Terms-of-service}{Terms
  of Service}
\item
  \href{https://help.nytimes.com/hc/en-us/articles/115014893968-Terms-of-sale}{Terms
  of Sale}
\item
  \href{https://spiderbites.nytimes.com}{Site Map}
\item
  \href{https://help.nytimes.com/hc/en-us}{Help}
\item
  \href{https://www.nytimes.com/subscription?campaignId=37WXW}{Subscriptions}
\end{itemize}
