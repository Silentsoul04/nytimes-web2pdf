Sections

SEARCH

\protect\hyperlink{site-content}{Skip to
content}\protect\hyperlink{site-index}{Skip to site index}

\href{https://www.nytimes.com/section/politics}{Politics}

\href{https://myaccount.nytimes.com/auth/login?response_type=cookie\&client_id=vi}{}

\href{https://www.nytimes.com/section/todayspaper}{Today's Paper}

\href{/section/politics}{Politics}\textbar{}The Education of Elizabeth
Warren

\url{https://nyti.ms/2U0VTW8}

\begin{itemize}
\item
\item
\item
\item
\item
\end{itemize}

\begin{itemize}
\item
  \href{https://www.nytimes.com/2020/07/31/us/elections/biden-vs-trump.html?action=click\&pgtype=Article\&state=default\&region=TOP_BANNER\&context=storylines_menu}{Election
  Updates}
\item
  \href{https://www.nytimes.com/article/biden-vice-president-2020.html?action=click\&pgtype=Article\&state=default\&region=TOP_BANNER\&context=storylines_menu}{Biden's
  V.P. Search}
\item
  \href{https://www.nytimes.com/interactive/2020/07/24/us/politics/trump-biden-campaign-donors.html?action=click\&pgtype=Article\&state=default\&region=TOP_BANNER\&context=storylines_menu}{Map
  of Donations}
\item
  \href{https://www.nytimes.com/interactive/2020/us/elections/delegate-count-primary-results.html?action=click\&pgtype=Article\&state=default\&region=TOP_BANNER\&context=storylines_menu}{Delegate
  Count}
\item
  \href{https://www.nytimes.com/interactive/2019/us/politics/2020-presidential-candidates.html?action=click\&pgtype=Article\&state=default\&region=TOP_BANNER\&context=storylines_menu}{The
  Candidates}
\item
  \href{https://www.nytimes.com/newsletters/politics?action=click\&pgtype=Article\&state=default\&region=TOP_BANNER\&context=storylines_menu}{Politics
  Newsletter}
\end{itemize}

Advertisement

\protect\hyperlink{after-top}{Continue reading the main story}

Supported by

\protect\hyperlink{after-sponsor}{Continue reading the main story}

The Long Run

\hypertarget{the-education-of-elizabeth-warren}{%
\section{The Education of Elizabeth
Warren}\label{the-education-of-elizabeth-warren}}

\includegraphics{https://static01.nyt.com/images/2019/08/20/multimedia/00Warren-01/merlin_159488103_78d8b82f-6658-429a-a055-5c963abc9ce6-articleLarge.jpg?quality=75\&auto=webp\&disable=upscale}

By \href{https://www.nytimes.com/by/stephanie-saul}{Stephanie Saul}

\begin{itemize}
\item
  Published Aug. 25, 2019Updated Jan. 7, 2020
\item
  \begin{itemize}
  \item
  \item
  \item
  \item
  \item
  \end{itemize}
\end{itemize}

Never one to shy away from a fight, Elizabeth Warren had found a new
sparring partner. She had only recently started teaching at the
University of Texas School of Law, but her colleague Calvin H. Johnson
already knew her well enough to brace for a lively exchange as they
commuted to work.

Indeed, on this morning in 1981, Ms. Warren again wanted to debate, this
time arguing on the side of giant utilities over their customers.

Her position was ``savagely anti-consumer,'' Mr. Johnson recalled
recently, adding that it wasn't unusual for her to espouse similar
pro-business views on technical legal issues.

Then something changed. He calls it Ms. Warren's ``road to Damascus''
moment.

``She started flipping --- `I'm pro-consumer,''' Mr. Johnson said.

That something, as Ms. Warren often tells the story, was her deepening
academic research into consumer bankruptcy, its causes, and lenders'
efforts to restrict it. Through the 1980s, the work took her to
courthouses across the country. There, she said in a recent interview,
she found not only the dusty bankruptcy files she had gone looking for
but heart-wrenching scenes she hadn't imagined --- average working
Americans, tearful and humiliated, admitting they were failures:

``People dressed in their Sunday best, hands shaking, women clutching a
handful of tissues, trying to stay under control. Big beefy men whose
faces were red and kept wiping their eyes, who showed up in court to
declare themselves losers in the great American game of life.''

\includegraphics{https://static01.nyt.com/images/2019/08/20/multimedia/00Warren-06/merlin_158665434_40f5ffc3-42a9-4bb5-acb5-a5922a1f5dd3-articleLarge.jpg?quality=75\&auto=webp\&disable=upscale}

Nearly 40 years after she began her bankruptcy research, Ms. Warren is
among the leading candidates for the Democratic presidential nomination.
Along with Bernie Sanders, her fellow senator, she represents a
progressive wing whose profound ideological division from their moderate
rivals have turned this primary into
\href{https://www.nytimes.com/2019/07/28/us/politics/democrats-2020-trump.html}{a
contest over the future of the party}.

Ms. Warren's political awakening didn't simply happen all at once. Her
road to Damascus was a long one. But over several decades, she
transformed from a largely pro-business and politically disengaged
academic --- a sort of default Republican --- to a
\href{https://www.nytimes.com/2020/01/07/us/politics/elizabeth-warren-bankruptcy-plan-biden.html}{fierce
consumer advocate and bankruptcy expert} whose advice was sought on
Capitol Hill, and then, finally, to a Democratic force on the Hill
herself.

Her bankruptcy work with two Texas colleagues, Jay L. Westbrook and
Teresa A. Sullivan, resulted in a 1989 book, ``As We Forgive Our
Debtors,'' regarded as a landmark among many bankruptcy lawyers and
academics for its depth and conclusions. One central finding --- that
bankrupt debtors represented a social cross-section of society ---
dispelled the popular narrative at the time. Even more controversial was
the book's uncompromising criticism of the credit card industry for
enticing consumers to take on ever more high-interest debt.

Ms. Warren, who said she began the study on the lookout for ``cheaters
and deadbeats,'' quickly realized that the people she was studying
seemed familiar. Her own family in Oklahoma had
\href{https://www.nytimes.com/2019/06/24/us/politics/elizabeth-warren-republican-conservative-democrat.html}{teetered
on the brink} of financial ruin. It is a part of the biography she
discusses in folksy speeches across the country --- her father's
unemployment, her mother's effort to save the family home with a
minimum-wage job, and how that wouldn't be possible today, with minimum
wage paying below the poverty rate.

The revelations from her bankruptcy research, by her account, became the
seeds of her worldview, laid out in her campaign plans for everything
from a new tax on the wealthiest Americans to a breakup of the big
technology companies.

Yet as Ms. Warren's candidacy has gained traction, critics have
complained that she is too rigid and
\href{https://www.nytimes.com/2019/07/30/us/politics/democratic-presidential-debate-recap.html}{radical
in her liberal ideas}, dug in to a polarizing degree against
\href{https://dealbook.nytimes.com/2013/02/14/at-senate-hearing-warren-comes-out-swinging/}{others
with differing views}, and can come across as dogmatic and
intellectually strident at times. Mr. Johnson, who admires Ms. Warren,
describes her as ``hard as nails.'' Some Democrats
\href{https://www.nytimes.com/2019/08/15/us/politics/elizabeth-warren-2020-campaign.html}{worry}
that such perceptions make her seem too left-wing and hard to work with,
and could make it difficult for her to build an Electoral College
majority if she is the nominee.

But a look at Ms. Warren's philosophical and political metamorphosis
provides yet another perspective on her personality, revealing a woman
who searched for answers and found something she had never expected,
then altered her thinking accordingly.

As Mr. Westbrook put it, ``She is really someone who is willing to learn
and willing to be persuaded.''

\hypertarget{law-and-economics}{%
\subsection{Law and Economics}\label{law-and-economics}}

In 1979, Ms. Warren recruited her parents from her native Oklahoma to
her home in the Houston suburbs to help babysit her two young children.

\hypertarget{latest-updates-2020-election}{%
\section{\texorpdfstring{\href{https://www.nytimes.com/2020/07/31/us/elections/biden-vs-trump.html?action=click\&pgtype=Article\&state=default\&region=MAIN_CONTENT_1\&context=storylines_live_updates}{Latest
Updates: 2020
Election}}{Latest Updates: 2020 Election}}\label{latest-updates-2020-election}}

Updated 2020-08-01T01:26:45.732Z

\begin{itemize}
\tightlist
\item
  \href{https://www.nytimes.com/2020/07/31/us/elections/biden-vs-trump.html?action=click\&pgtype=Article\&state=default\&region=MAIN_CONTENT_1\&context=storylines_live_updates\#link-29fdff45}{Kamala
  Harris, a top vice-presidential contender, confronts double
  standards.}
\item
  \href{https://www.nytimes.com/2020/07/31/us/elections/biden-vs-trump.html?action=click\&pgtype=Article\&state=default\&region=MAIN_CONTENT_1\&context=storylines_live_updates\#link-13ec3d9c}{Karen
  Bass and Susan Rice are rising on Biden's vice-presidential
  shortlist.}
\item
  \href{https://www.nytimes.com/2020/07/31/us/elections/biden-vs-trump.html?action=click\&pgtype=Article\&state=default\&region=MAIN_CONTENT_1\&context=storylines_live_updates\#link-49e9a016}{Trump
  says Russian bounties to kill U.S. troops `never took place.'}
\end{itemize}

\href{https://www.nytimes.com/2020/07/31/us/elections/biden-vs-trump.html?action=click\&pgtype=Article\&state=default\&region=MAIN_CONTENT_1\&context=storylines_live_updates}{See
more updates}

Then a professor at the University of Houston, she would be spending
several weeks at a luxury resort near Miami, one of 22 law professors
selected to study an increasingly popular discipline known as ``law and
economics.'' One of its central ideas is that markets perform more
efficiently than courts.

Mr. Johnson, Ms. Warren's former Texas commuting partner, believes that
it was an important influence on her early thinking.

``Before Liz converted, she came to us from the decidedly
anti-government side of law and economics,'' he said.

The summer retreat was colloquially known as a ``Manne camp,'' after its
organizer, the libertarian legal scholar Henry G. Manne. With financial
support from industry and conservative foundations, Mr. Manne had formed
a Law and Economics Center at the University of Miami. (He would later
move operations to Emory University and then to George Mason
University.)

Image

Henry G. Manne, the libertarian legal scholar, had a formative role in
Ms. Warren's academic career and served as a sort of
mentor.Credit...Benjamin Myers/Reuters

The mission of the retreat was to spread the gospel of free-market
microeconomics among law professors. One participant, John Price, a
former dean of the law school at the University of Washington, described
it as ``sort of pure proselytizing on the part of dedicated, very
conservative law and economics folks,'' with an emphasis on an
anti-regulatory agenda. One faculty member, he recalled, suggested
eliminating the Consumer Product Safety Commission.

In the reputational pecking order of those attending its 10th annual
edition that summer in Key Biscayne, Ms. Warren, with her law degree
from Rutgers, was near the bottom.

The group included such legal luminaries as Abraham D. Sofaer, a
Columbia law professor who had just been named a federal judge in
Manhattan, and Frank Iacobucci, a professor at the University of Toronto
who would later serve on the Canadian Supreme Court. Ms. Warren was one
of only two women in attendance, along with Grace Ganz Blumberg, then a
professor at the State University of New York at Buffalo.

While some in the group have said Ms. Warren expressed skepticism at the
libertarian ideology, Ms. Blumberg remembers someone very much
developing the early stages of her career, who was ``far more captivated
than I'' with the theories.

Afterward, Ms. Warren remained in contact with Mr. Manne, writing
periodically to update him on her life and career, appreciative notes
that suggest something of a mentor-mentee relationship.

In November 1979, during a visit by Mr. Manne to Houston, Ms. Warren
confided that she was going through a divorce from her husband, Jim
Warren, an IBM mathematical wizard who had chafed at Ms. Warren's
pursuit of a career outside the home.

Two months after her divorce was finalized,
\href{https://int.nyt.com/data/documenthelper/1648-warren-to-manne-3-7-1980-1-1/a741af593ce52226ffe3/optimized/full.pdf\#page=1}{in
a letter dated March 7, 1980}, Ms. Warren thanked Mr. Manne for
providing ``lots of advice and comfort'' during that Houston visit. She
also shared another development in her life: She was engaged to be
married again --- to Bruce Mann, a Brown- and Yale-trained scholar in
the history of law who had become her frequent tennis partner at the
Florida retreat.

``You even provided (indirectly) my husband,'' she wrote. ``Can I ever
thank you enough, or should I just request that in lieu of wedding gifts
guests just send a donation to LEC?''

Ms. Warren again
\href{https://int.nyt.com/data/documenthelper/1650-gmu-00006434-pdf/a741af593ce52226ffe3/optimized/full.pdf\#page=2}{wrote
to Mr. Manne in 1981}, attaching a copy of her latest published article.
She was sending him one article a summer, she wrote, and each
``increasingly reflects my time at LEC.''

The exact topic of the article is not reflected in the correspondence,
which was released to The New York Times by the Manne Center at George
Mason University. (Mr. Manne died in 2015.) Based on the letter's date,
however, it was probably a 68-page treatise on judicial interpretation
of contracts published in the University of Pittsburgh Law Review.

``This is really hard-core law \& econ analysis,'' Todd J. Zywicki, a
law professor at George Mason who formerly served as executive director
of the Manne Center, wrote in an email. ``If you had given me this
article with the author anonymized and asked me who wrote it, I would
have answered that it was one of the leading scholars in the law \&
economics of commercial and contract law. Never, in a million years,
would I have thought this article was written by EW.''

Several articles Ms. Warren wrote during that period, on utility
regulation, promote a pro-industry position, experts said.

A 1978 article in the Rutgers Law Review --- which would inspire some of
her early debates with Mr. Johnson --- argued in favor of a utility
rate-setting model that Mr. Johnson said would allow utilities to charge
customers for taxes that hadn't yet been paid. A piece in 1980 for
Public Utilities Fortnightly, an industry magazine, took a similar
position, according to its editor, Steven Mitnick, who added, ``It's
quite surprising that this was written by the person we know as
Elizabeth Warren.''

But Mr. Price, the former University of Washington dean who became
friends with Ms. Warren in Key Biscayne and is a supporter and
contributor to her presidential campaign, sees nothing unusual in what
he describes as Ms. Warren's ``reasonable evolution from what they were
espousing'' at the retreat.

Reflecting on his own shift from ``Rockefeller Republican'' to Democrat,
he added, ``I believe we just go through an educational process.''

As Ms. Warren sees that process, her early interest in law and economics
revealed a period of ``learning another tool, but then coming to
understand the values that underlie that system.''

\hypertarget{traveling-with-r2-d2}{%
\subsection{Traveling With R2-D2}\label{traveling-with-r2-d2}}

By 1981, Ms. Warren and her husband had secured temporary teaching posts
at the University of Texas, where she agreed to teach bankruptcy law.
She quickly earned a reputation for lively lectures, putting students on
the spot and peppering them with questions and follow-up questions ---
the consummate practitioner of the Socratic method.

Even visitors to her class got the treatment. One of them was Stefan A.
Riesenfeld, a
\href{https://www.nytimes.com/1999/03/13/us/stefan-a-riesenfeld-90-expert-on-international-law.html}{renowned}
bankruptcy professor who had come to lecture on the Bankruptcy Reform
Act of 1978. The law, which had expanded bankruptcy protection for
consumers, was already under attack by the credit industry, which argued
that it made personal bankruptcy too attractive.

Even so, Mr. Riesenfeld explained to Ms. Warren's class, those who filed
personal bankruptcy were ``mostly day laborers and housemaids who had
lived at the economic margins and always would,'' she wrote in her 2014
memoir.

``I asked the obvious follow-up question: `How did he know?''' Ms.
Warren wrote. After more questioning, it became clear that not only did
Mr. Riesenfeld have no real answer, he was irritated by Ms. Warren's
probing.

The subject struck close to home.
\href{https://www.nytimes.com/2019/12/23/us/politics/elizabeth-warren-oklahoma-native-american.html}{When
she was growing up in Oklahoma}, her father's heart attack had thrown
their household into precarious financial territory, forcing her mother
to take a minimum-wage job answering telephones at Sears.

Image

Ms. Warren's childhood home in Oklahoma. When she was young, her
father's heart attack threw their household into precarious financial
territory.Credit...Nick Oxford for The New York Times

She remembers being fearful as she lay in bed at night listening to her
mother cry. ``She thought I had gone to sleep. I didn't know for sure
the details of why she was crying, but I knew it was bad and that we
could lose everything,'' Ms. Warren said.

(Later, the oil glut of the 1980s would destroy her brother David's
once-thriving business delivering supplies to oil rigs. Her brother
John, a construction worker, would also struggle after the oil market
collapsed. Her family ended up in such dire straits that Ms. Warren and
her husband would ultimately provide financial assistance to some
relatives, including help buying their homes.)

She wanted answers, more than Professor Riesenfeld could provide. She
began discussing her questions with colleagues.

``There was this new bankruptcy code, and nobody knew much about what
was happening out there in the world,'' Mr. Westbrook said. ``We got to
talking and decided it would be kind of interesting to go down and take
a look at some of the cases on file in the Western District of Texas,
San Antonio to El Paso. We literally went to the courthouses and talked
to the judges.''

As the study expanded, the researchers began visiting other states to
collect data from the court files --- then available only on paper.

``We got two portable Xerox machines, which in those days were a big
deal --- high technology,'' Mr. Westbrook remembers. ``We had to buy a
ticket for them. We didn't trust them to baggage.''

They nicknamed the equipment R2-D2.

Dozens of people would eventually be involved in the effort, an analysis
of a quarter million pieces of data gathered from bankruptcy cases filed
from 1981 through 1985.

Among the researchers was Kimberly S. Winick, then a University of Texas
law student and now a lawyer in Los Angeles. While Ms. Warren didn't
talk a lot about her views, Ms. Winick said she believed that the
project's initial theory was that, ``If you filed bankruptcy, you must
be cheating.''

``Liz was from a more conservative place,'' Ms. Winick said. ``And she
was somebody who had worked very, very, very hard all her life. And she
had never walked away from a debt. And I think she kind of started with
the view --- let's see what people are doing and how they're cadging on
their debts and screwing their creditors.''

That was the conventional thinking of the day, promoted in a
\href{https://www.nytimes.com/1993/12/15/business/at-purdue-a-wealth-of-data-on-consumer-debt.html}{study}
by Purdue University researchers that was being widely circulated on
Capitol Hill as evidence the 1978 bankruptcy law needed to be toughened,
Ms. Warren said in the interview.

But when she and her colleagues analyzed the study, she said, they
concluded not only that its methodology was flawed, but that it had been
funded with a sizable grant from the credit industry.

``That's where it starts to shift for me,'' Ms. Warren said. ``Once we
figured out that this was a bought-and-paid-for piece of credit industry
advertising, now I was a little more neutral.''

\hypertarget{in-the-fight}{%
\subsection{In the Fight}\label{in-the-fight}}

And there were the personal stories, voices from that cross-section of
society forced into bankruptcy.

``There was an elderly couple who told the story of their child, an
adult man, who had a drug problem,'' Ms. Warren said. ``And how they had
sold everything, he'd fallen back in again, and they cashed out their
retirement accounts to put him through rehab.''

Image

A yearbook photo of Ms. Warren from the University of Texas in
1985.Credit...The University of Texas at Austin

Another couple had left their jobs in the previous year. Even though
they had quickly found new ones --- as a psychiatric aide at a state
hospital and a manager at a state agency --- their income dropped 25
percent. They fell behind on their \$45,000 mortgage and other debts.
Their experience illustrated the study's crucial finding that lenders
had extended an enormous amount of credit to people who were clearly bad
risks.

While the files did not tell the whole story, they provided enough
evidence for Ms. Warren and her co-authors to write, ``Repeatedly, we
have been surprised by the data and forced to rethink our own
understanding of bankruptcy.''

The study's findings, laid out in their 1989 book, also explicitly
raised questions about a central tenet of law and economics: that
individuals respond rationally to economic incentives. As applied to
bankruptcy --- the more generous the bankruptcy provisions, the more
people would file --- that idea had been the rationale behind the
campaign that succeeded in toughening the bankruptcy code in 1984.

Over the years, the research elevated Ms. Warren's status, from
little-known Texas professor to sought-after lecturer, writer and
consultant in bankruptcy law. It also set the stage for her career in
politics.

In 1995, Mike Synar, a former Democratic congressman from her home
state, asked Ms. Warren, by then a Harvard professor, to advise a
special commission reviewing the bankruptcy system. She balked, fearing
the Washington work would render her scholarship impure, but signed on
when Mr. Synar promised to keep her insulated from politics.

It was during that period, in 1996, that she switched her party
affiliation from Republican to Democrat, though she insists that her
essential conversion was from ``not political'' to ``political.''

``I didn't come from a political family,'' she said. ``I hadn't been
political as an adult. I was raising a family, teaching school and doing
my research,'' she said.

Image

Ms. Warren, who switched her party affiliation from Republican to
Democrat in the 1990s, says her conversion is better described as ``not
political'' to ``political.''Credit...Tom Brenner for The New York Times

Then she went to Capitol Hill.

``I quickly discovered that every single Republican was on the side of
the banks and half the Democrats were,'' she said. ``But whenever there
was someone who would stand up for those working families, it was a
Democrat.''

She added, ``I picked sides, got in the fight, and I've been in the
fight ever since.''

\hypertarget{our-2020-election-guide}{%
\section{Our 2020 Election Guide}\label{our-2020-election-guide}}

Updated July 31, 2020

\begin{itemize}
\item
  \begin{center}\rule{0.5\linewidth}{\linethickness}\end{center}

  \hypertarget{the-latest}{%
  \subsection{The Latest}\label{the-latest}}

  \begin{itemize}
  \tightlist
  \item
    President Trump's assault on the Postal Service is intersecting with
    his attacks on mail-in voting.
    \href{https://www.nytimes.com/2020/07/31/us/politics/trump-usps-mail-delays.html?action=click\&pgtype=Article\&state=default\&region=BELOW_MAIN_CONTENT\&context=storylines_guide}{Voting
    rights groups say it is a recipe for disaster.}
  \end{itemize}
\item
  \begin{center}\rule{0.5\linewidth}{\linethickness}\end{center}

  \hypertarget{bidens-vp-search}{%
  \subsection{Biden's V.P. Search}\label{bidens-vp-search}}

  \begin{itemize}
  \tightlist
  \item
    \href{https://www.nytimes.com/article/biden-vice-president-2020.html?action=click\&pgtype=Article\&state=default\&region=BELOW_MAIN_CONTENT\&context=storylines_guide}{Here
    are 13 women} who have been under consideration to be Joe Biden's
    running mate, and why each might be chosen --- and might not be.
  \end{itemize}
\item
  \begin{center}\rule{0.5\linewidth}{\linethickness}\end{center}

  \hypertarget{keep-up-with-our-coverage}{%
  \subsection{Keep Up With Our
  Coverage}\label{keep-up-with-our-coverage}}

  \begin{itemize}
  \tightlist
  \item
    Get an
    \href{https://www.nytimes.com/newsletters/politics?action=click\&pgtype=Article\&state=default\&region=BELOW_MAIN_CONTENT\&context=storylines_guide}{email}
    recapping the day's news
  \end{itemize}

  \begin{itemize}
  \tightlist
  \item
    Download our mobile app on
    \href{https://apps.apple.com/us/app/nytimes/id284862083?ls=1\&mat_click_id=5c79ae7455014fd1bd66b5610c05b8f2-20191112-16948\&referrer=mat_click_id\%3D5c79ae7455014fd1bd66b5610c05b8f2-20191112-16948\%26link_click_id\%3D722930677036718082}{iOS}
    and
    \href{http://a.localytics.com/android?id=com.nytimes.android\&referrer=utm_source\%3Dother_nyt_mobile_web\%26utm_medium\%3DWeb\%2520page\%26utm_term\%3DGeneral\%2520Mobile\%2520Page\%26utm_campaign\%3DNYT\%2520Mobile\%2520General\%2520Page}{Android}
    and turn on Breaking News and Politics alerts
  \end{itemize}
\end{itemize}

Advertisement

\protect\hyperlink{after-bottom}{Continue reading the main story}

\hypertarget{site-index}{%
\subsection{Site Index}\label{site-index}}

\hypertarget{site-information-navigation}{%
\subsection{Site Information
Navigation}\label{site-information-navigation}}

\begin{itemize}
\tightlist
\item
  \href{https://help.nytimes.com/hc/en-us/articles/115014792127-Copyright-notice}{©~2020~The
  New York Times Company}
\end{itemize}

\begin{itemize}
\tightlist
\item
  \href{https://www.nytco.com/}{NYTCo}
\item
  \href{https://help.nytimes.com/hc/en-us/articles/115015385887-Contact-Us}{Contact
  Us}
\item
  \href{https://www.nytco.com/careers/}{Work with us}
\item
  \href{https://nytmediakit.com/}{Advertise}
\item
  \href{http://www.tbrandstudio.com/}{T Brand Studio}
\item
  \href{https://www.nytimes.com/privacy/cookie-policy\#how-do-i-manage-trackers}{Your
  Ad Choices}
\item
  \href{https://www.nytimes.com/privacy}{Privacy}
\item
  \href{https://help.nytimes.com/hc/en-us/articles/115014893428-Terms-of-service}{Terms
  of Service}
\item
  \href{https://help.nytimes.com/hc/en-us/articles/115014893968-Terms-of-sale}{Terms
  of Sale}
\item
  \href{https://spiderbites.nytimes.com}{Site Map}
\item
  \href{https://help.nytimes.com/hc/en-us}{Help}
\item
  \href{https://www.nytimes.com/subscription?campaignId=37WXW}{Subscriptions}
\end{itemize}
