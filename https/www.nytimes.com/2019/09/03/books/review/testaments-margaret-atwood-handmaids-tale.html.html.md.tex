Sections

SEARCH

\protect\hyperlink{site-content}{Skip to
content}\protect\hyperlink{site-index}{Skip to site index}

\href{https://www.nytimes.com/section/books/review}{Book Review}

\href{https://myaccount.nytimes.com/auth/login?response_type=cookie\&client_id=vi}{}

\href{https://www.nytimes.com/section/todayspaper}{Today's Paper}

\href{/section/books/review}{Book Review}\textbar{}The Handmaid's
Thriller: In `The Testaments,' There's a Spy in Gilead

\url{https://nyti.ms/34nE73U}

\begin{itemize}
\item
\item
\item
\item
\item
\end{itemize}

Advertisement

\protect\hyperlink{after-top}{Continue reading the main story}

Supported by

\protect\hyperlink{after-sponsor}{Continue reading the main story}

Fiction

\hypertarget{the-handmaids-thriller-in-the-testaments-theres-a-spy-in-gilead}{%
\section{The Handmaid's Thriller: In `The Testaments,' There's a Spy in
Gilead}\label{the-handmaids-thriller-in-the-testaments-theres-a-spy-in-gilead}}

\includegraphics{https://static01.nyt.com/images/2019/09/22/books/review/22Kakutani/22Kakutani-articleLarge.jpg?quality=75\&auto=webp\&disable=upscale}

By \href{https://www.nytimes.com/by/michiko-kakutani}{Michiko Kakutani}

\begin{itemize}
\item
  Published Sept. 3, 2019Updated Sept. 10, 2019
\item
  \begin{itemize}
  \item
  \item
  \item
  \item
  \item
  \end{itemize}
\end{itemize}

\textbf{THE TESTAMENTS}\\
By Margaret Atwood

The most chilling --- and timely --- lines in ``The Handmaid's Tale''
occur near the beginning of Margaret Atwood's 1985 novel. Offred and her
shopping partner Ofglen are walking past the Wall --- a landmark that
once belonged to a famous university in Cambridge, Mass., and is now
used by the rulers of Gilead to display the corpses of people executed
as traitors. As she looks at six new bodies hanging there, Offred
remembers the unnerving words of their warden and teacher Aunt Lydia:
``Ordinary,'' she said, is ``what you are used to. This may not seem
ordinary to you now, but after a time it will. It will become
ordinary.''

Nothing changes instantaneously, Offred observes: ``In a gradually
heating bathtub you'd be boiled to death before you knew it.'' How did
the United States of America become the totalitarian state of Gilead ---
a place where women are treated as ``two-legged wombs''; where nonwhite
residents and unbelievers (that is, Jews, Catholics, Quakers, Baptists,
anyone who does not embrace the fundamentalist extremism of Gilead) are
resettled, exiled or disappeared; where the leadership deliberately uses
gender, race and class to divide the country? It started before ordinary
citizens like herself were paying attention, Offred remembers: ``We
lived, as usual, by ignoring. Ignoring isn't the same as ignorance, you
have to work at it.''

In ``The Testaments,'' Atwood's compelling sequel to ``The Handmaid's
Tale'' --- which takes place a decade and a half later --- Offred makes
only the briefest of appearances, speaking a scant three sentences. But
she has attained almost mythic status in Gilead, where she's been
declared a terrorist and enemy of the state: The regime has already made
at least two assassination attempts on her life, and it's turned Baby
Nicole, the daughter Offred (in the TV series adaptation) had smuggled
across the border to Canada, into a poster girl martyr.

The main story line in ``The Testaments'' is a kind of spy thriller
about a mole inside Gilead, who is working with the Mayday resistance to
help bring down the evil empire. It's a contrived and heavily
stage-managed premise --- but contrived in a Dickensian sort of way with
coincidences that reverberate with philosophical significance. And
Atwood's sheer assurance as a storyteller makes for a fast, immersive
narrative that's as propulsive as it is melodramatic.

\emph{{[} This book was one of our most anticipated titles of
September.}
\href{https://www.nytimes.com/2019/08/28/books/new-september-books.html}{\emph{See
the full list}}\emph{. {]}}

That story unfolds through three overlapping narratives. One is told by
Nicole, now a young woman of 16 living in Canada under another name. One
is told by Agnes Jemima, Offred's older daughter (known as Hannah in the
TV series), who, at the age of 5, was snatched away from Offred as she
and her husband, Luke, were attempting to flee to Canada, and who has
since grown up in Gilead with foster parents. And one is told by Aunt
Lydia, the implacable enforcer, who has imposed Gilead's draconian rules
on the Handmaids with vengeful relish.

The Hulu TV adaptation of ``The Handmaid's Tale'' has awkwardly tried to
make Aunt Lydia more than a cartoonish villain by sketching in her back
story, suggesting that loneliness and shame in her own life fueled her
cruelty. In ``The Testaments,'' Atwood makes a more convincing case for
Lydia's complexity: She has made Lydia (like Offred, Offred's daughters
and so many characters in the author's earlier novels like ``Surfacing''
and ``Cat's Eye'') a survivor, someone who's done what she thinks is
necessary to avoid death or further loss. To save herself in the early
days of Gilead, Lydia became a collaborator with the regime, and she
rises within the leadership by playing coldblooded, hardball politics.
Her involvement in a Mayday resistance plot to undermine Gilead has as
much to do with deadly rivalries within the regime's elite as it does
with her own disillusionment over growing corruption and hypocrisy in
the theocracy.

This is only one of myriad differences between Atwood's fleet-footed
sequel and the television adaptation. Under the guidance of the
showrunner Bruce Miller, the TV series did a brilliant job in Season 1
of translating the novel to the screen, but in generating new story
lines for Seasons 2 and 3, the show's writers have subjected Offred
(played by the gifted Elisabeth Moss) to a wearisome ``Groundhog Day''
loop of tribulations, including several failed escape attempts,
repetitious, soap-opera confrontations with Serena and Aunt Lydia and
more and more preposterous situations calling for bad-ass heroics.

In the interests of heightening the depravity of the Gilead regime, the
TV writers have told an increasingly grisly story, which dwells, at
gruesome length, on sadistic tortures inflicted upon the Handmaids: In
addition to the ritualized rapes described in the novel, there are
finger amputations, Taser assaults, an excised eyeball, hands scorched
on hot stoves, muzzles and metal rings used to keep the women's mouths
clamped shut --- the sort of abominations more likely to be found in the
misogynistic horror porn that Offred's activist mother wanted to burn,
than in a feminist allegory.

\emph{{[} ``I'm too old to be scared by much'':}
\href{https://www.nytimes.com/2019/09/05/books/handmaids-tale-sequel-testaments-margaret-atwood.html}{\emph{Margaret
Atwood talks about closure, literary reputations and ``The
Testaments.''}} \emph{{]}}

In both ``The Handmaid's Tale'' (the novel) and ``The Testaments,''
Atwood wisely focuses less on the viciousness of the Gilead regime
(though there is one harrowing and effective sequence about its use of
emotional manipulation to win over early converts to its cause), and
more on how temperament and past experiences shape individual
characters' very different responses to these dire circumstances.

\includegraphics{https://static01.nyt.com/images/2019/09/22/books/review/22Kakutani1/22Kakutani1-articleLarge.jpg?quality=75\&auto=webp\&disable=upscale}

Atwood understands that the fascist crimes of Gilead speak for
themselves --- they do not need to be italicized, just as their
relevance to our own times does not need to be put in boldface. Many
American readers and viewers of ``The Handmaid's Tale'' are already
heavily invested in the story of Gilead because we've come to identify
with the Handmaids' hopes that the nightmare will end and the United
States --- with its democratic norms and constitutional guarantees ---
will soon be restored. We identify because the events in Atwood's novel
--- which not so long ago felt like something that could only happen in
the distant past or in distant parts of the globe --- now feel
frighteningly real. Because news segments on television in 2019 are
filled with images of children being torn from their parents' arms, a
president using racist language to sow fear and hatred and reports of
accelerating climate change jeopardizing life as we know it on the
planet. This also explains why the scenes in ``The Handmaid's Tale''
that feel most haunting today are the flashbacks in which Offred
remembers her former life in America, when she and her friends took for
granted the rights and freedoms they enjoyed, when people reassured one
another that whatever emergency measures taken by the new government (in
the name of protecting against Islamic terrorism) were temporary and
that normalcy would soon return.

Enduring dystopian novels look backward and forward at the same time.
\href{https://www.nytimes.com/2017/01/26/books/why-1984-is-a-2017-must-read.html}{Orwell's
``1984''} was at once a savage satire of the Soviet Union under Stalin
--- from its rewriting of history to its cult of personality to its use
of torture and propaganda --- and a shrewd anatomy of totalitarianism
that foretold the rise of the surveillance state and the fire hose of
falsehood spewed forth daily by Putin's Kremlin and Trump's White House
in attempts to redefine reality. Aldous Huxley's ``Brave New World''
reflected its author's worries in the 1930s that individual freedom was
threatened by both communism and assembly-line capitalism, and it
anticipated a technology-driven future in which people would be
narcotized and distracted to death by trivia and entertainment.

Atwood, who began ``The Handmaid's Tale'' in the Orwellian year of 1984,
decided that she would include nothing in the novel ``that had not
already happened'' somewhere, some time in history, or any technology
``not already available.'' She extrapolated some of the trends she saw
in the 1970s and early '80s, like the rising fundamentalist movement in
America. She imagined what she called ``the heavy-handed theocracy of
17th-century Puritan New England --- with its marked bias against
women'' --- reasserting itself during a period of social chaos. And she
drew upon historical horrors like the Nazis' Lebensborn program and
public executions in countries like North Korea and Saudi Arabia to
delineate the malign machinery of the Gilead regime.

Atwood's creation of Gilead, like her creation of the futuristic
wasteland that serves as a backdrop to ``Oryx and Crake'' and ``The Year
of the Flood,'' was informed by her wide-ranging reading in dystopian
literature and related genres, lending these novels a faintly
postmodernist density and gloss. The stories of Nicole and Agnes in
``The Testaments'' similarly reflect Atwood's easy familiarity with
Victorian literature, which she studied in graduate school at Harvard in
the 1960s.

The sisters' quests to discover their family roots, for instance, mirror
the efforts made by so many of the orphans in 19th-century literature,
like Dickens's Pip and Oliver Twist. Much the way that Offred struggled
in ``The Handmaid's Tale'' to come to terms with the expectations of her
radical feminist mother, in ``The Testaments'' Nicole and Agnes find
their search for self-definition tied up in questions about their real
mother's identity. Nicole is shocked to learn that the life she's been
leading in Canada --- as Daisy, the daughter of owners of a
used-clothing store --- is ``a forgery.'' Agnes realizes that much of
what she grew up believing about Gilead has been a lie --- that the
regime's corrupt leaders have been rewriting the Bible in a Big
Brother-like endeavor to market and justify their dictatorial rule.

If Agnes comes across as willfully naïve in the opening sections of
``The Testaments,'' Atwood appears to be making the point that Agnes
begins as a very ordinary girl. Ordinary in the way that Offred was
ordinary in ``The Handmaid's Tale'' --- a smart, resourceful young
woman, more concerned, initially, with the travails of daily life than
with politics or the larger world. Atwood's Offred was not a rebel like
her friend Moira and not an ideologue like her mother. In fact, she
asserted that she didn't want to be ``the incarnation'' of her feminist
mother's ideas, didn't want to have to ``vindicate her life for her.''

Perhaps because some fans had complained that Atwood's Offred was too
passive, the TV writers have been transforming her, in Seasons 2 and 3,
into a ferocious (and at times ruthless) warrior queen, willing to
compromise her own morals if it furthers her ends; a committed member of
the resistance whose quest has evolved from getting her own daughter
back, to trying to evacuate dozens of children to Canada.

But while this makes for a more dramatic heroine who can grow and change
over multiple seasons of TV, it's worth remembering that the very
ordinariness of Atwood's Offred gave readers an immediate understanding
of how Gilead's totalitarian rule affected regular people's lives. The
same holds true of Agnes's account in ``The Testaments,'' which is less
an exposé of the hellscape that is Gilead than a young girl's chronicle
of her family life and education there, and the unexpected turn of
events that lead her to play a pivotal role in determining the regime's
fate.

\href{https://www.nytimes.com/2017/03/10/books/review/margaret-atwood-handmaids-tale-age-of-trump.html}{In
a 2017 essay}, Atwood described writing Offred's story in the tradition
of ``the literature of witness'' --- referring to those accounts left by
people bearing witness to the calamities of history they've experienced
firsthand: wars, atrocities, disasters, social upheavals, hinge moments
in civilization. It's a genre that includes the diary of Anne Frank, the
writings of Primo Levi, the choral histories assembled by the Nobel
Prize winner Svetlana Alexievich from intensive interviews with
Russians, remembering their daily lives during World War II, the
Chernobyl accident or the Afghanistan war. Agency and strength, Atwood
seems to be suggesting, do not require a heroine with the visionary
gifts of Joan of Arc, or the ninja skills of a Katniss Everdeen or
Lisbeth Salander --- there are other ways of defying tyranny,
participating in the resistance or helping ensure the truth of the
historical record.

{[}
\emph{\href{https://www.nytimes.com/2017/03/10/books/review/margaret-atwood-handmaids-tale-age-of-trump.html}{Read
Atwood's essay ``What `The Handmaid's Tale' Means in the Age of
Trump.''}} {]}

The very act of writing or recording one's experiences, Atwood argues,
is ``an act of hope.'' Like messages placed in bottles tossed into the
sea, witness testimonies count on someone, somewhere, being there to
read their words --- even if it's the pompous, myopic Gileadean scholars
who narrate the satirical epilogues to both ``The Handmaid's Tale'' and
``The Testaments.''

As Atwood no doubt knows, one of the definitions given by Bible
dictionaries for ``Gilead'' is ``hill of testimony.'' And in testifying
to what they have witnessed, Offred, Nicole, Agnes and, yes, Lydia are
leaving behind accounts that will challenge official Gileadean
narratives, and in doing so, they are standing up to the regime's
determination to silence women by telling their own stories in their own
voices.

Advertisement

\protect\hyperlink{after-bottom}{Continue reading the main story}

\hypertarget{site-index}{%
\subsection{Site Index}\label{site-index}}

\hypertarget{site-information-navigation}{%
\subsection{Site Information
Navigation}\label{site-information-navigation}}

\begin{itemize}
\tightlist
\item
  \href{https://help.nytimes.com/hc/en-us/articles/115014792127-Copyright-notice}{©~2020~The
  New York Times Company}
\end{itemize}

\begin{itemize}
\tightlist
\item
  \href{https://www.nytco.com/}{NYTCo}
\item
  \href{https://help.nytimes.com/hc/en-us/articles/115015385887-Contact-Us}{Contact
  Us}
\item
  \href{https://www.nytco.com/careers/}{Work with us}
\item
  \href{https://nytmediakit.com/}{Advertise}
\item
  \href{http://www.tbrandstudio.com/}{T Brand Studio}
\item
  \href{https://www.nytimes.com/privacy/cookie-policy\#how-do-i-manage-trackers}{Your
  Ad Choices}
\item
  \href{https://www.nytimes.com/privacy}{Privacy}
\item
  \href{https://help.nytimes.com/hc/en-us/articles/115014893428-Terms-of-service}{Terms
  of Service}
\item
  \href{https://help.nytimes.com/hc/en-us/articles/115014893968-Terms-of-sale}{Terms
  of Sale}
\item
  \href{https://spiderbites.nytimes.com}{Site Map}
\item
  \href{https://help.nytimes.com/hc/en-us}{Help}
\item
  \href{https://www.nytimes.com/subscription?campaignId=37WXW}{Subscriptions}
\end{itemize}
