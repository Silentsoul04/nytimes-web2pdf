Sections

SEARCH

\protect\hyperlink{site-content}{Skip to
content}\protect\hyperlink{site-index}{Skip to site index}

\href{https://www.nytimes.com/section/world/asia}{Asia Pacific}

\href{https://myaccount.nytimes.com/auth/login?response_type=cookie\&client_id=vi}{}

\href{https://www.nytimes.com/section/todayspaper}{Today's Paper}

\href{/section/world/asia}{Asia Pacific}\textbar{}How Trump's Plan to
Secretly Meet With the Taliban Came Together, and Fell Apart

\url{https://nyti.ms/34xv8NS}

\begin{itemize}
\item
\item
\item
\item
\item
\item
\end{itemize}

Advertisement

\protect\hyperlink{after-top}{Continue reading the main story}

Supported by

\protect\hyperlink{after-sponsor}{Continue reading the main story}

\hypertarget{how-trumps-plan-to-secretly-meet-with-the-taliban-came-together-and-fell-apart}{%
\section{How Trump's Plan to Secretly Meet With the Taliban Came
Together, and Fell
Apart}\label{how-trumps-plan-to-secretly-meet-with-the-taliban-came-together-and-fell-apart}}

\includegraphics{https://static01.nyt.com/images/2019/09/09/world/09dc-prexy-1/merlin_160248603_f840f8be-d470-478d-ba5a-aca59960a1b0-articleLarge.jpg?quality=75\&auto=webp\&disable=upscale}

By \href{https://www.nytimes.com/by/peter-baker}{Peter Baker},
\href{https://www.nytimes.com/by/mujib-mashal}{Mujib Mashal} and
\href{https://www.nytimes.com/by/michael-crowley}{Michael Crowley}

\begin{itemize}
\item
  Published Sept. 8, 2019Updated Feb. 14, 2020
\item
  \begin{itemize}
  \item
  \item
  \item
  \item
  \item
  \item
  \end{itemize}
\end{itemize}

WASHINGTON --- On the Friday before Labor Day,
\href{https://www.nytimes.com/2019/09/10/us/politics/john-bolton-national-security-adviser-trump.html}{President
Trump} gathered top advisers in the Situation Room to consider what
could be among the profound decisions of his presidency --- a peace plan
with the Taliban after 18 years of grinding, bloody war in Afghanistan.

The meeting brought to a head a bristling conflict dividing his foreign
policy team for months, pitting Secretary of State Mike Pompeo against
\href{https://www.nytimes.com/2019/09/10/us/politics/john-bolton-national-security-adviser-trump.html}{John
R. Bolton}, the national security adviser, in a battle for the competing
instincts of a president who relishes tough talk but promised to wind
down America's endless wars.

\emph{{[}Update:}
\emph{\href{https://www.nytimes.com/2020/02/14/world/asia/taliban-cease-fire-step-afghanistan.html}{U.S.
and Taliban agree to reduce violence, in first step toward
cease-fire}.{]}}

As they discussed terms of the agreement, Mr. Pompeo and his negotiator,
Zalmay Khalilzad, made the case that it would enable Mr. Trump to begin
withdrawing troops while securing a commitment from the Taliban not to
shelter terrorists. Mr. Bolton, beaming in by video from Warsaw, where
he was visiting, argued that Mr. Trump could keep his campaign pledge to
draw down forces without getting in bed with killers swathed in American
blood.

Mr. Trump made no decision on the spot, but at some point during the
meeting the idea was floated to finalize the negotiations in Washington,
a prospect that appealed to the president's penchant for dramatic
spectacle. Mr. Trump suggested that he would even invite President
Ashraf Ghani of Afghanistan, whose government has not been party to the
talks, and get him to sign on.

\emph{{[}Our reporters}
\href{https://www.nytimes.com/2019/09/10/reader-center/afghanistan-casualty-report.html}{\emph{explain
how they keep track}} \emph{of the civilian and security-force
casualties of the Afghanistan War.{]}}

In the days that followed, Mr. Trump came up with an even more
remarkable idea --- he would not only bring the Taliban to Washington,
but to Camp David, the crown jewel of the American presidency. The
leaders of a rugged militant organization deemed terrorists by the
United States would be hosted in the mountain getaway used for
presidents, prime ministers and kings just three days before the
anniversary of the Sept. 11, 2001, attacks that led to the Afghan war.

Thus began an extraordinary few days of ad hoc diplomatic wrangling that
upended the talks in a weekend Twitter storm. On display were all of the
characteristic traits of the Trump presidency --- the yearning ambition
for the grand prize, the endless quest to achieve what no other
president has achieved, the willingness to defy convention, the volatile
mood swings and the tribal infighting.

\includegraphics{https://static01.nyt.com/images/2019/09/09/world/09dc-prexy-2/merlin_156790473_7a708a42-ed75-4941-bd86-0b2350694951-articleLarge.jpg?quality=75\&auto=webp\&disable=upscale}

What would have been one of the biggest headline-grabbing moments of his
tenure was put together on the spur of the moment and then canceled on
the spur of the moment. The usual National Security Council process was
dispensed with; only a small circle of advisers was even clued in.

\includegraphics{https://static01.nyt.com/images/2017/01/29/podcasts/the-daily-album-art/the-daily-album-art-articleInline-v2.jpg?quality=75\&auto=webp\&disable=upscale}

\hypertarget{listen-to-the-daily-a-historic-peace-plan-collapses}{%
\subsubsection{Listen to `The Daily': A Historic Peace Plan
Collapses}\label{listen-to-the-daily-a-historic-peace-plan-collapses}}

Talks with the Taliban were intended to end 18 years of war in
Afghanistan. But President Trump called off the negotiations just as
they appeared to be nearing an agreement.

transcript

Back to The Daily

bars

0:00/24:48

-24:48

transcript

\hypertarget{listen-to-the-daily-a-historic-peace-plan-collapses-1}{%
\subsection{Listen to `The Daily': A Historic Peace Plan
Collapses}\label{listen-to-the-daily-a-historic-peace-plan-collapses-1}}

\hypertarget{hosted-by-michael-barbaro-produced-by-rachel-quester-and-luke-vander-ploeg-and-edited-by-lisa-tobin-and-mj-davis-lin}{%
\subsubsection{Hosted by Michael Barbaro, produced by Rachel Quester and
Luke Vander Ploeg, and edited by Lisa Tobin and M.J. Davis
Lin}\label{hosted-by-michael-barbaro-produced-by-rachel-quester-and-luke-vander-ploeg-and-edited-by-lisa-tobin-and-mj-davis-lin}}

\hypertarget{talks-with-the-taliban-were-intended-to-end-18-years-of-war-in-afghanistan-but-president-trump-called-off-the-negotiations-just-as-they-appeared-to-be-nearing-an-agreement}{%
\paragraph{Talks with the Taliban were intended to end 18 years of war
in Afghanistan. But President Trump called off the negotiations just as
they appeared to be nearing an
agreement.}\label{talks-with-the-taliban-were-intended-to-end-18-years-of-war-in-afghanistan-but-president-trump-called-off-the-negotiations-just-as-they-appeared-to-be-nearing-an-agreement}}

\begin{itemize}
\item
  michael barbaro\\
  From The New York Times, I'm Michael Barbaro. This is ``The Daily.''

  Today. President Trump has abruptly called off the negotiations
  between the U.S. and the Taliban they could have ended the war in
  Afghanistan, saying that the talks are dead. The story of how a
  historic peace deal went off the rails.

  It's Tuesday, September 10.

  Peter Baker, tell me about this meeting in the White House Situation
  Room.
\item
  peter baker\\
  Well on the Friday before Labor Day, the president sat down in the
  Situation Room with his top national security advisors talk about
  whether or not to make peace with the Taliban. Now his administration
  had been negotiating a deal for nearly a year with these militants
  we've been fighting since 9/11. And in that room at this moment, he
  was presented with a choice --- do you go ahead and make peace with
  them or not. On the one side was Mike Pompeo, Secretary of State, and
  his special envoy, Zalmay Khalilzad, who has negotiated this deal.
  They said, look, this is a way to fulfill your campaign promise, pull
  troops out, and get a security agreement with the Taliban that will
  make sure that Afghanistan is not a haven for terrorists. But not
  everybody agreed, in particular, John Bolton. He's the president's
  national security advisor, a longtime well-known hawk. And he said,
  wait a minute, you don't have to have agreement with the Taliban to
  pull troops out and fulfill your campaign promise. If you want to pull
  out for these 5600 troops the way you are talking about doing, you can
  do that on your own without getting in bed with these killers who have
  American blood on their hands.
\item
  michael barbaro\\
  And like you said, in the background of this, it seems, is the
  president's campaign pledge to get every U.S. troop out of
  Afghanistan.
\item
  archived recording (donald trump)\\
  For years, we have been caught up in endless wars and conflicts under
  the leadership of failed politicians, and a failed --- totally failed
  foreign policy.
\end{itemize}

peter baker

Yeah, he had said that he wanted to end these endless wars, right? That
Iraq, and Afghanistan, Syria, and all these other places we have troops
around the world. That the American public was tired of it, we spent too
much money, lost too many lives, and for nothing.

\begin{itemize}
\tightlist
\item
  archived recording (donald trump)\\
  If we didn't do anything with Iraq, if we never went --- if our
  presidents went to the beach, we'd be much better off. If they just
  went every single day to the beach.
\end{itemize}

peter baker

Now that's a very radically different view than the conventional
Republican view up until that point, as represented by George W. Bush,
and Dick Cheney, and others in the Republican Party, including John
Bolton.

michael barbaro

I understand the Bolton perspective here, but it seems a little bit late
for this sort of elemental division, right? I mean, these negotiations
led by the U.S. government have been going on for, as you just said,
almost a year. Why was the U.S. government so deeply engaged in these
negotiations if members of the administration weren't aligned in
agreeing that they should be doing so?

peter baker

Yeah, what President Trump's aides would tell you is he wants to hear
disagreement. There is disagreement within his team on a lot of issues,
and he doesn't shy away from hearing it. He doesn't like it when it's
made public and it's aimed at him, right? But if people are disagreeing
in front of him, he's OK with that, as long as he is the one who makes
the final choice. And you're right, it's late in the game, but it's not
the first time this debate has come up. It's just sort of the last
chance that the opponents have to register their objections before the
president goes forward, if he goes forward.

michael barbaro

Do you think that the Taliban would have been surprised to learn that
amid all these negotiations, the president and those around him weren't
even sure that there should be a deal?

peter baker

I don't know. That's a good question. I think that they themselves, my
guess is having a similar debate as to what they're willing to accept
and what they're not willing to accept. Remember, you know, we're kind
of circling back here 18 years later to where we started.

\begin{itemize}
\tightlist
\item
  archived recording (george w. bush)\\
  Tonight, we are a country awakened to danger and called to defend
  freedom.
\end{itemize}

peter baker

After 9/11, George W. Bush, the president at the time, said to the
Taliban, which was the government in Afghanistan ---

\begin{itemize}
\tightlist
\item
  archived recording (george w. bush)\\
  Deliver to United States authorities all the leaders of Al Qaeda who
  hide in your land.
\end{itemize}

peter baker

You have to give us Osama bin Laden, and you have to give us Al Qaeda,
and expel them for your country.

\begin{itemize}
\tightlist
\item
  archived recording (george w. bush)\\
  These demands are not open to negotiation or discussion.
\end{itemize}

peter baker

And if you do that, we'll leave you alone.

\begin{itemize}
\tightlist
\item
  archived recording (george w. bush)\\
  The Taliban must act and act immediately.
\end{itemize}

peter baker

If you don't do that, there'll be a price to pay.

\begin{itemize}
\tightlist
\item
  archived recording (george w. bush)\\
  They will hand over the terrorists or they will share in their fate.
\end{itemize}

peter baker

And the Taliban didn't do that. Well, we're circling back to that now
same position 18 years later. And a lot of blood and treasure under the
bridge between 2001 and 2019.

michael barbaro

So what does President Trump decide to do in that room dealing with
these warring advisors?

peter baker

Well, he's not convinced by John Bolton's argument, but he doesn't make
a decision to go forward yet, either. They do begin talking about how
they would go about finalizing a deal. And the idea comes up in this
meeting of doing it in Washington. Which appeals to President Trump's
sense of spectacle and drama. The idea of making peace in Afghanistan
after all his time in Washington was very appealing to him. And over the
days that follow, he came up with an even more dramatic idea, which is
to actually bring the leader of the Taliban as well as the president of
Afghanistan to Camp David.

Now Camp David, of course, has a historic legacy. It's where Ike, and
F.D.R., and Reagan used to relax.

\begin{itemize}
\tightlist
\item
  archived recording (jimmy carter)\\
  When we first arrived at Camp David, the first thing upon which we
  agreed was to ask the people of the world to pray that our
  negotiations would be successful.
\end{itemize}

peter baker

And where Jimmy Carter made peace between Egypt and Israel.

\begin{itemize}
\item
  archived recording (jimmy carter)\\
  Those prayers have been answered far beyond any expectations.
\item
  archived recording (bill clinton)\\
  We have an opportunity to bring about a just and enduring end to the
  Israeli-Palestinian conflict.
\end{itemize}

peter baker

Bill Clinton tried to make peace between Israel and the Palestinians
there.

\begin{itemize}
\item
  archived recording (bill clinton)\\
  Of course, there is no guarantee of success, but not to try is to
  guarantee failure.
\item
  archived recording (george w. bush)\\
  I've asked the highest levels of our government to come to discuss the
  current tragedy that has so deeply affected our nation.
\end{itemize}

peter baker

George W. Bush set a path of the war that we are talking about now in
Afghanistan there at Camp David.

\begin{itemize}
\tightlist
\item
  archived recording (george w. bush)\\
  We will not only deal with those who dare attack America. We will deal
  with those who harbor them, and feed them, and house them.
\end{itemize}

peter baker

So it's a place with a real resonance is special. The idea that you
would invite the leaders of this band of terrorists --- that's what
they're designated by the United States government --- to the mountains
of Maryland, and in effect, host them in this special setting, really
struck some as a bad idea. In particular, John Bolton. We're told that
the vice president, Mike Pence, was worried that it would be bad optics,
particularly coming just a few days before 9/11. But the president
pulled forward. He loved this idea, and he began pushing it forward so
that they began even telling the Afghan government about a plane coming
to pick up the president of Afghanistan, Ashraf Ghani. The trick is I
don't think the Taliban had necessarily the same idea that the president
had, right? The Taliban said we're open to coming, but only after the
deal is announced. In other words, they wanted the assurance that it was
done before they traveled to the heart of the enemy, the United States,
and showed up at an event with President Trump. That's a fundamental
disagreement of what Trump thought. What Trump wants to do is he wants
to be the dealmaker, he wants to be who closes the deal, or at least be
perceived to be the person who closes the deal. So he doesn't want to
announce before you have it at Camp David. He wants to bring you to Camp
David, seal the deal, and then say I've had this great success.

michael barbaro

It's sort of funny Peter that the question goes from whether to make
peace with the Taliban because of who they are, to whether to take them
to Camp David to do it.

peter baker

Yes. I mean, look, this is how life is in the Trump White House. You saw
literally over the course of a few days, a debate about whether to go
forward the agreement to this remarkable idea for how to finalize an
agreement that hadn't yet been fully finalized.

michael barbaro

Peter, besides the fact that the president wants to invite the Taliban
to Camp David, I'm struck by the fact that he wants to invite the leader
of the Afghan government. Because my sense from talking to our
colleagues is that the Taliban feels really strongly about these being
negotiations with the U.S. to end the war, and then it will negotiate
with the government of Afghanistan. Which by the way, it views as
illegitimate.

peter baker

Right, exactly. And this is a structural conundrum of this process
that's been going on. You're leaving out the very people who are
supposed to be running the country. And from their point of view, they
are potentially being sold out, right? The Americans want to get out,
but then they're left to battle the Taliban on their own. Ashraf Ghani,
the president, he was very skeptical of this whole thing from the
beginning. And the idea of bringing him to Camp David, putting him in
one lodge, putting the Taliban leaders in another lodge, trying to find
a way to bring them together would have been a big gamble for frankly,
all three parties.

michael barbaro

So Peter, as the president is secretly planning for this spectacle of a
summit at Camp David, what is the state of the actual peace talks
between the U.S. and the Taliban that have been going on in Doha?

peter baker

So while the president is planning for this Camp David extravaganza, his
special envoy, Zalmay Khalilzad, is shuttling back and forth in the
region between Doha and Qatar which is where the talks of the Taliban
have been taking place, and Kabul, the Afghan capital. And he's telling
everybody they've got a deal. In fact, he and the Taliban negotiators
initial a document, outlining their deal that would involve America
pulling out the remaining 14,000 troops. In exchange, the Taliban would
make commitments about ensuring that Afghanistan is never again a haven
for Al Qaeda or other terrorist groups that have America as their
target. And this would be basically a way out of 18 years of war while
in theory, securing some sort of guarantee of American safety.

michael barbaro

We'll be right back.

OK, so Peter, from everything you've just described, these peace talks
are going quite well. The president has a vision of a grand public
announcement of a deal. And as he's planning this, the two sides seem to
be coming together to actually make a deal. So then what happens?

peter baker

Well, on Thursday ---

\begin{itemize}
\tightlist
\item
  archived recording\\
  A brazen attack just outside the entrance to NATO headquarters in
  Kabul.
\end{itemize}

peter baker

In Kabul, a suicide car bomber attack, kills 12 people, including an
American soldier.

\begin{itemize}
\tightlist
\item
  archived recording\\
  It's part of a string of attacks launched by the Taliban this week as
  the militant group reportedly edges closer to a peace deal with the
  U.S.
\end{itemize}

peter baker

The president is told about this by his aides. The Taliban takes credit.

\begin{itemize}
\tightlist
\item
  archived recording\\
  And while they may be talking peace in Doha, here in Afghanistan,
  they're still waging war in the most brutal way.
\end{itemize}

peter baker

And the idea of bringing the Taliban leaders to Camp David just a few
days after an American soldier has been killed is just politically
untenable. And he says to his aides, this is it, we can't do this. It's
over, we can't do it. Now he's not calling off the peace drive
altogether, but it is the idea of having the enemy to your home in
effect just three days after one of your soldiers is killed was just too
much for him.

michael barbaro

But Peter, hasn't the Taliban been setting off suicide bombs and waging
war in Afghanistan, and even horribly killing American soldiers for
years and months right up and through these negotiations with the U.S.?

peter baker

Yes, and as far as we can tell, there was no agreement to have a
ceasefire during the talks. In fact, Secretary Mike Pompeo was on
television on Sunday ---

\begin{itemize}
\tightlist
\item
  archived recording (mike pompeo)\\
  Jake, it's also the case we haven't been negotiating while they've
  been killing us, and we've been standing still. We've been taken into
  the Taliban as well. Over 1,000 Taliban killed in just the last 10
  days alone.
\end{itemize}

peter baker

Specifically, boasting that the American side had killed 1,000 Taliban
fighters just in the last week.

\begin{itemize}
\tightlist
\item
  archived recording (mike pompeo)\\
  So the American people should know we're going to defend American
  national interests. We're going to be tough.
\end{itemize}

peter baker

Whether that's true or not, that doesn't sound like a situation where we
expect violence to go away in the midst of the talks, right?

michael barbaro

Right.

peter baker

We had one soldier killed, they had 1,000. Having said that, as a matter
of politics, as a matter of visuals, as a matter of optics, it would
have been hard for a president to go forward with the Camp David meeting
in the immediate aftermath.

michael barbaro

And is your understanding, Peter, that this car bombing in Kabul is the
main reason that the Camp David talks are called off? Or was it sort of
just accumulation of complexities that were starting to signal that this
was just a little bit too dicey?

peter baker

Yeah, I think it's more complex than just the car bomb. It was a work in
progress, it was coming together so quickly. And then it falls apart
just as quickly. But nobody knows about it outside of a handful of
people. The president didn't have to say anything. And yet on Saturday
night ---

\begin{itemize}
\tightlist
\item
  archived recording\\
  Breaking news tonight, the president revealing he was going to have a
  secret meeting tomorrow ---
\end{itemize}

peter baker

For whatever reason, he sends out a handful of tweets announcing to the
world that he had issued this invitation and canceled it.

\begin{itemize}
\tightlist
\item
  archived recording\\
  Trump wrote, quote ``Unbeknownst to almost everyone, the major Taliban
  leaders and separately, the president of Afghanistan, were going to
  secretly meet with me at Camp David on Sunday. They were coming to the
  United States tonight. Unfortunately, in order to build false
  leverage, they admitted to an attack in Kabul that killed one of our
  great soldiers and 11 other people. I immediately canceled the meeting
  and called off peace negotiations.''
\end{itemize}

peter baker

That's how most of America learns for the first time that this was even
a possibility. And it was a shock, I think, to a lot of people. It was 7
o'clock at night on a Saturday. And most of Washington doesn't know
about this. Congress doesn't know about this. Most of the State
Department and the Defense Department don't know about this. And it was
a real jaw-dropper, I think, for a lot of people.

michael barbaro

What exactly was the reaction beyond the initial shock of its very
existence?

peter baker

Well, you can imagine the Democrats would be critical. But what was
really striking was of course, Republicans were nervous and
uncomfortable about the whole idea of this peace negotiation to begin
with. The idea that it would be at Camp David was just too much. So
Republicans ended up criticizing the president by praising him.

\begin{itemize}
\tightlist
\item
  archived recording\\
  I think that the president was right to cancel the meeting. I think
  what he's trying to do is fulfill a campaign promise ---
\end{itemize}

peter baker

They put out tweets and they issue statements saying, the president did
the right thing not to do the thing he was thinking about doing.

\begin{itemize}
\tightlist
\item
  archived recording\\
  Well, the president, you know, leads through action. He's looking for
  results. Looking for solutions. He's done that in terms of North
  Korea. He's done that all across the world. I think he is right to
  have done it here and right to have pulled back ---
\end{itemize}

peter baker

The president was right not to do it.

michael barbaro

So in that praise is really, a profound critique that the president
should never, ever have invited the Taliban to Camp David in the first
place.

peter baker

Exactly, exactly. And this is the thing that kind of wakes Washington
up, right? Again, these peace talks have been going on for a while,
everybody kind of knows about it, but it hasn't been a source of major
debate in Washington. But now for the first time, it's sort of thrust
out into the conversation in a significant way. Now suddenly, it's sort
of more front and center. Now suddenly, they're going to have to
confront the idea whether it's even a good idea in the first place.
Forget the Camp David idea --- is the very idea of this agreement wise
or not? And we don't know the answer to how Trump views this. He was
certainly willing to go forward with the deal on some level, and we
assume that he is still open to it. He has shown in the past a
willingness to go back and forth. He did this with Kim Jong-un. He
canceled a summit meeting with him one day, and a few days later,
reinstated it. Would not surprise me if at some point in the not too
distant future they resume talks or they resume and effort to still
finalize this deal.

michael barbaro

I wonder if Camp David had never entered the picture, is it likely that
he would now be signing a peace deal between the United States and the
Taliban?

peter baker

I think it's certainly possible. I mean, again, the bombing would have
still made it difficult. But they could have postponed by a week and
nobody would have known. Or they could have come back and negotiated,
hey, if we're going to do this kind of agreement, we have to have a
period of calm and no violence between us. And if we do something like
that, then we can sign a deal. I suspect the president hasn't changed
his mind about the fundamentals, which is that he wants to get out of
Afghanistan. And whether he, in fact, has a deal with the Taliban or
not, he seems still determined to bring troops home, at least some of
them.

michael barbaro

And does Washington, does Congress, and perhaps even the entire
Republican Party, suddenly play a much greater role in all of this
because the president, as you said, kind of awakened everyone to this?

peter baker

Yeah, I think that they're going weigh in. Now the question of course,
is this is the Republican Party has been remarkably deferential to the
president even when he has gone against traditional Republican
orthodoxy. He's going against it now and they have been mostly quiet up
until this point. The Camp David angle gives them a way in to register
their discontent with making too much of a deal with the Taliban, with
pulling out too fast in their view, or too precipitously. And we'll see
if they you know jump into that void. But for now, the debate is front
and center.

michael barbaro

So it sounds like the debate you told us about at the beginning of our
conversation, in the Situation Room between Mike Pompeo on one end and
John Bolton on the other about whether to sign a peace deal with the
Taliban, that is representative of a much bigger and still unresolved
question. Not just in the Republican Party, but maybe in the whole
country, which is whether we should ever make peace with the Taliban,
given who they are and what they did.

peter baker

I think that's exactly right. That debate in the Situation Room now may
play out in the broader public, as you say. Because in fact, America's
got to make a decision. They're tired of war, right? Poll after poll
shows that Americans want to get out. They're done with it. Last poll I
saw showed that around 59 percent, something like that, think the
Afghanistan war wasn't worth it. But the trade off is does that mean
you're willing to pull out, does it means you're willing to hand the
country over to the Taliban if that's the consequence of pulling out?
And do you want to get in bed with these guys who harbored the Al Qaeda
terrorists who were responsible for 9/11. These are difficult choices
and now instead of just in the Situation Room, this choice is being
presented to the American public.

michael barbaro

Peter, thank you very much.

peter baker

Thank you. It's great talking to you.

michael barbaro

After we spoke with Peter, President Trump was asked about the state of
negotiations with the Taliban by reporters outside the White House.

\begin{itemize}
\tightlist
\item
  archived recording (donald trump)\\
  They're dead, they're dead. As far as I'm concerned, they're dead.
\end{itemize}

michael barbaro

Trump defended his original plan to host the Taliban and the Afghan
government at Camp David, saying it was less controversial than holding
the summit at the White House.

\begin{itemize}
\tightlist
\item
  archived recording (donald trump)\\
  I like the idea of meeting. I've met with a lot of bad people and a
  lot of good people during the course of the last almost three years.
  And I think meeting is a great thing.
\end{itemize}

michael barbaro

The Times reports that despite the president's emphatic claim that the
negotiations are dead, it's unclear whether the peace talks have
permanently ended, something the president seemed to hint at Monday,
when he reiterated his desire to withdraw American troops from
Afghanistan.

\begin{itemize}
\tightlist
\item
  archived recording (donald trump)\\
  Yeah, we'd would like to get out, but we'll get out at the right time.
\end{itemize}

michael barbaro

We'll be right back.

Here's what else you need to know today.

\begin{itemize}
\tightlist
\item
  archived recording\\
  Order. Order.
\end{itemize}

michael barbaro

On Monday night, British lawmakers delivered their latest and most
devastating defeat yet to Prime Minister Boris Johnson.

\begin{itemize}
\tightlist
\item
  archived recording\\
  The ayes to the right, 293. In the nos to the left, 46. So the ayes
  have it, the ayes it.
\end{itemize}

michael barbaro

By voting to reject his plan to hold new elections.

\begin{itemize}
\tightlist
\item
  archived recording\\
  The majority does not satisfy the requirements of the fixed term
  Parliament's act for the purpose of engendering the election that some
  seeks.
\end{itemize}

michael barbaro

Thanks to his own tactics, there is virtually no way for Johnson to
fulfill his promise to leave the European Union with or without a
negotiated exit by October 31. Parliament is now suspended for the next
five weeks at Johnson's behest, and has passed a law in response,
blocking the prime minister from leaving the E.U. without a negotiated
deal. And The Times reports that the Secretary of Commerce, Wilbur Ross,
threatened to fire top officials at the National Oceanic and Atmospheric
Administration for contradicting President Trump's claim that Hurricane
Dorian might hit Alabama. The threat prompted the agency to disavow a
message that assured Alabama that it was not at risk, in a move widely
seen as caving to the White House. The president's claim that Alabama
was in the storm's path was inaccurate from the moment he made it. But
for the past week, his administration has aggressively defended it.

That's it for ``The Daily.'' I'm Michael Barbaro. See you tomorrow.

And even after it fell apart, Mr. Trump took it upon himself to disclose
the secret machinations in a string of Saturday night Twitter messages
that surprised not only many national security officials across the
government but even some of the few who were part of the deliberations.

\hypertarget{almost-a-done-deal}{%
\subsection{Almost a Done Deal}\label{almost-a-done-deal}}

For Mr. Trump,
\href{https://www.nytimes.com/2019/09/08/us/politics/pompeo-trump-afghan-peace-negotiations.html}{ending
the war in Afghanistan} has been a focus since taking office, a
signature accomplishment that could help him win re-election next year.
For nearly a year, Mr. Khalilzad, a former ambassador to Afghanistan,
has engaged in talks with the Taliban to make that happen.

In recent weeks, it had been increasingly clear that the United States
and the Taliban, after nine rounds of painstaking negotiations in Doha,
Qatar, had ironed out most of the issues between them. Mr. Khalilzad
declared that the agreement document had been finalized ``in
principle.''

The deal called for
\href{https://www.nytimes.com/2019/09/02/world/asia/us-withdrawal-afghanistan-taliban.html?module=inline}{a
gradual withdrawal of the remaining 14,000 American troops} over 16
months, with about 5,000 of them leaving within 135 days. In return, the
Taliban would provide counterterrorism assurances to ease American fears
of a repeat of Sept. 11 from Afghan soil.

But the negotiations left out Afghanistan's government, and Mr. Ghani's
officials criticized it for lacking measures that would ensure
stability. At home, Mr. Trump was cautioned by Senator Lindsey Graham,
Republican of South Carolina; Gen. Jack Keane, a retired Army vice chief
of staff; and Gen. David Petraeus, the retired Afghanistan and Iraq
commander.

Mr. Bolton was the leading voice against the deal on the inside as Mr.
Pompeo's allies increasingly tried to isolate the national security
adviser. Mr. Bolton argued that Mr. Trump could pull out 5,000 troops
while still leaving enough forces to assist counterterrorism efforts
without a deal with the Taliban, a group he argued could not be trusted.

In an interview on Sunday, Mr. Graham said he shared Mr. Trump's desire
``to end the war in Afghanistan between the Taliban and the Afghan
people.'' But he added that no deal could include withdrawing all
American forces or trusting the Taliban to confront Al Qaeda or the
Islamic State.

Image

Zalmay Khalilzad, the top American negotiator, had declared that an
agreement document between the United States and the Taliban had been
finalized ``in principle.''Credit...Jim Huylebroek for The New York
Times

``My advice to the administration is, let's focus on trying to shore up
our relationship with Pakistan,'' he said, adding that it should include
a free-trade agreement. He said that the Taliban must be prevented from
believing it can seek safe harbor in Pakistan.

\hypertarget{a-dividing-point}{%
\subsection{A Dividing Point}\label{a-dividing-point}}

When Mr. Khalilzad left Doha after the last round of talks concluded on
Sept. 1, two days after the Situation Room meeting, he and his Taliban
counterparts had finalized the text of the agreement, according to
people involved. Leaders of both teams initialed their copies and handed
them to their Qatari hosts.

Before the end of the meeting, Mr. Khalilzad brought up the idea of a
Taliban trip to Washington. Taliban leaders said they accepted the idea
--- as long as the visit came after the deal was announced.

That would become a fundamental dividing point contributing to the
collapse of the talks. Mr. Trump did not want the Camp David meeting to
be a celebration of the deal; after staying out of the details of what
has been a delicate effort in a complicated region, Mr. Trump wanted to
be the dealmaker who would put the final parts together himself, or at
least be perceived to be.

The idea was for Mr. Trump to hold separate meetings at Camp David with
the Taliban and with Mr. Ghani, leading to a more global resolution.

Even as talks were wrapping up in Doha, the American ambassador to
Afghanistan arrived at the presidential palace in Kabul with the
proposal of a Camp David meeting, Afghan officials said.

Details were sorted out between the Afghan president and the American
side when Mr. Khalilzad arrived from Doha and held four rounds of talks
with Mr. Ghani. A plane would arrive to take Mr. Ghani and his
delegation to the United States, according to the initial plan.

Mr. Ghani's ministers knew that a Taliban delegation would most likely
be arriving, too, but were unclear on the details. They had three
priorities: the fate of presidential elections scheduled for Sept. 28,
how the peace talks would move forward to include them and how they
would bolster security forces to reduce the cost for the United States.

Image

Members of the Taliban delegation in Doha, Qatar, in July. Taliban
leaders said the Americans were tricking them into political suicide
with the Camp David meeting.Credit...Karim Jaafar/Agence France-Presse
--- Getty Images

As a sign of how important the event was for the United States, Mr.
Ghani got the Americans to agree to include on the trip his national
security adviser, Hamdullah Mohib, who had essentially been
\href{https://www.nytimes.com/2019/03/30/world/asia/afghanistan-hamdullah-mohib-zalmay-khalilzad.html?module=inline}{kept
out of the American meetings} after lashing out at the peace process.

For months, the Americans had essentially held Mr. Ghani's re-election
campaign hostage to a deal that they projected was imminent. Mr. Ghani
was reduced to pretending that the September elections were still on by
holding a couple of daily ``virtual rallies'' at which he addressed
small gatherings around the country via video chat. If the
American-Taliban deal were finalized, it would most likely push the
elections back.

If Mr. Ghani had refused the Camp David meeting, he would have been
called a spoiler of peace, a senior Afghan official said. So he took his
chances; it was to be hosted by an ally on friendly turf, and it could
help clarify whether there would be a peace deal, and whether the
elections would proceed.

\hypertarget{intensifying-bloodshed}{%
\subsection{Intensifying Bloodshed}\label{intensifying-bloodshed}}

But Taliban leaders, having refused to negotiate directly with the
Afghan government until after the group had an agreement with the United
States, said the Americans were tricking them into political suicide.

A senior Taliban leader said on Sunday that Mr. Trump was fooling
himself to think he could bring the Taliban and Mr. Ghani together at
Camp David ``because we do not recognize the stooge government'' in
Kabul.

The Americans were also rushing to finalize outstanding issues in the
days before the last-minute proposed Camp David meeting. Among the most
significant was a disagreement over the release of thousands of Taliban
prisoners in Afghan prisons.

Afghan officials said the Americans had taken the liberty of negotiating
on their behalf by agreeing to the release. Mr. Ghani's government found
that unacceptable, saying it would agree only if the Taliban
reciprocated with an extensive cease-fire --- something the insurgents
are reluctant to do at this stage of the talks since violence is their
main leverage.

The final negotiations occurred during a period of intensifying
bloodshed. In response to Taliban attacks, American negotiators made
clear they were prioritizing the agreement, not looking to boycott the
talks. Their negotiations were undergirded by increasing battlefield
pressure by the American military.

\includegraphics{https://static01.nyt.com/images/2018/02/01/world/asia/afghanistan-cover-4/afghanistan-cover-4-videoSixteenByNineJumbo1600.jpg}

When Mr. Khalilzad and Gen. Austin S. Miller, the American commander in
Afghanistan, returned to Doha on Thursday, it was to finalize technical
appendices to the main text. The Taliban negotiators got no sense that
anything was amiss and later posted on Twitter that the atmosphere was
good.

But the same day, aides told Mr. Trump about a suicide car bomb attack
that killed an American soldier and 11 others. At this point, according
to senior officials, Mr. Trump and his team were unified. He could not
host Taliban leaders at Camp David just days after an American was
killed even though it was unclear whether the insurgents had actually
agreed to come in the first place.

``This is off; we can't do this,'' Mr. Trump told his aides, according
to one official.

{[}\emph{To follow the Afghan war peace talks,}
\href{https://www.nytimes.com/newsletters/at-war}{\emph{sign up for the
weekly At War newsletter}}\emph{.}{]}

No announcement was made by the White House. In Kabul on Friday, Mr.
Ghani's officials told reporters that he planned to travel to the United
States, and then hours later said he would not go.

But little was made of that at the time. The endgame of the talks seemed
near, if not the timetable. Only then came Mr. Trump's tweets on
Saturday night disclosing that he had invited the Taliban and Mr. Ghani
to Camp David --- but called it off, citing the bombing.

The tweets took many in the administration by surprise; there was no
reason for Mr. Trump to reveal what had happened, several officials
said, especially since he has not given up on the idea of a negotiated
settlement.

Hours later, Mr. Pompeo visited Dover Air Force Base for the arrival of
the coffin of Army Sgt. First Class Elis Angel Barreto Ortiz, who was
killed in the Kabul bombing. His presence was unusual for a secretary of
state; the return of fallen American soldiers would be more
traditionally attended by presidents or defense secretaries.

On Sunday, after their negotiating team held an emergency internal
meeting in Doha, the Taliban said Mr. Trump's decision to cancel the
talks would hurt only the United States. The Afghan government blamed
the Taliban, saying that the violence was making the peace process
difficult.

American officials stressed that the peace drive was not over and the
deal had been neither rejected nor accepted. With Mr. Trump especially,
anything can happen.

But for the moment, at least, all sides seemed certain of one thing:
Violence will now intensify. The war will go on.

Advertisement

\protect\hyperlink{after-bottom}{Continue reading the main story}

\hypertarget{site-index}{%
\subsection{Site Index}\label{site-index}}

\hypertarget{site-information-navigation}{%
\subsection{Site Information
Navigation}\label{site-information-navigation}}

\begin{itemize}
\tightlist
\item
  \href{https://help.nytimes.com/hc/en-us/articles/115014792127-Copyright-notice}{©~2020~The
  New York Times Company}
\end{itemize}

\begin{itemize}
\tightlist
\item
  \href{https://www.nytco.com/}{NYTCo}
\item
  \href{https://help.nytimes.com/hc/en-us/articles/115015385887-Contact-Us}{Contact
  Us}
\item
  \href{https://www.nytco.com/careers/}{Work with us}
\item
  \href{https://nytmediakit.com/}{Advertise}
\item
  \href{http://www.tbrandstudio.com/}{T Brand Studio}
\item
  \href{https://www.nytimes.com/privacy/cookie-policy\#how-do-i-manage-trackers}{Your
  Ad Choices}
\item
  \href{https://www.nytimes.com/privacy}{Privacy}
\item
  \href{https://help.nytimes.com/hc/en-us/articles/115014893428-Terms-of-service}{Terms
  of Service}
\item
  \href{https://help.nytimes.com/hc/en-us/articles/115014893968-Terms-of-sale}{Terms
  of Sale}
\item
  \href{https://spiderbites.nytimes.com}{Site Map}
\item
  \href{https://help.nytimes.com/hc/en-us}{Help}
\item
  \href{https://www.nytimes.com/subscription?campaignId=37WXW}{Subscriptions}
\end{itemize}
