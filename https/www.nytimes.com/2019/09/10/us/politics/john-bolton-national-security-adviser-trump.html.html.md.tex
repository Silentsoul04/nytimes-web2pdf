Sections

SEARCH

\protect\hyperlink{site-content}{Skip to
content}\protect\hyperlink{site-index}{Skip to site index}

\href{https://www.nytimes.com/section/politics}{Politics}

\href{https://myaccount.nytimes.com/auth/login?response_type=cookie\&client_id=vi}{}

\href{https://www.nytimes.com/section/todayspaper}{Today's Paper}

\href{/section/politics}{Politics}\textbar{}Trump Ousts John Bolton as
National Security Adviser

\url{https://nyti.ms/2A4wbGT}

\begin{itemize}
\item
\item
\item
\item
\item
\item
\end{itemize}

Advertisement

\protect\hyperlink{after-top}{Continue reading the main story}

Supported by

\protect\hyperlink{after-sponsor}{Continue reading the main story}

\hypertarget{trump-ousts-john-bolton-as-national-security-adviser}{%
\section{Trump Ousts John Bolton as National Security
Adviser}\label{trump-ousts-john-bolton-as-national-security-adviser}}

\includegraphics{https://static01.nyt.com/images/2019/09/11/us/politics/10dc-bolton-hfo/10dc-bolton-hfo-videoSixteenByNine3000.jpg}

By \href{https://www.nytimes.com/by/peter-baker}{Peter Baker}

\begin{itemize}
\item
  Published Sept. 10, 2019Updated Sept. 11, 2019
\item
  \begin{itemize}
  \item
  \item
  \item
  \item
  \item
  \item
  \end{itemize}
\end{itemize}

WASHINGTON --- President Trump on Tuesday pushed out John R. Bolton, his
third national security adviser, amid fundamental disputes over how to
handle major foreign policy challenges like Iran, North Korea and most
recently Afghanistan.

The departure ended a 17-month partnership that had grown so tense that
the two men even disagreed over how they parted ways, as
\href{https://twitter.com/realDonaldTrump/status/1171452880055746560}{Mr.
Trump announced on Twitter} that he had fired the adviser only to be
\href{https://twitter.com/AmbJohnBolton/status/1171455806069305346?s=20}{rebutted
by Mr. Bolton}, who insisted he
\href{https://int.nyt.com/data/documenthelper/1697-bolton-resignation-letter/1efe19fa7a0e6a93abd9/optimized/full.pdf\#page=1}{had
resigned of his own accord}.

A longtime Republican hawk known for a combative style, Mr. Bolton spent
much of his tenure trying to restrain the president from making what he
considered unwise agreements with America's enemies. Mr. Trump bristled
at what he viewed as Mr. Bolton's militant approach, to the point that
he made barbed jokes in meetings about his adviser's desire to get the
United States into more wars.

Their differences came to a climax in recent days as Mr. Bolton waged a
last-minute campaign to stop the president from signing a peace
agreement at Camp David with leaders of the radical Taliban group. He
won the policy battle as
\href{https://twitter.com/realDonaldTrump/status/1170469618177236992?s=20}{Mr.
Trump scrapped the deal} but lost the larger war when the president grew
angry about the way the matter played out.

\includegraphics{https://static01.nyt.com/images/2017/01/29/podcasts/the-daily-album-art/the-daily-album-art-articleInline-v2.jpg?quality=75\&auto=webp\&disable=upscale}

\hypertarget{listen-to-the-daily-john-bolton-is-fired-or-did-he-resign}{%
\subsubsection{Listen to `The Daily': John Bolton Is Fired. Or Did He
Resign?}\label{listen-to-the-daily-john-bolton-is-fired-or-did-he-resign}}

The national security adviser was ousted after clashing with President
Trump over issues like Afghanistan, Iran and North Korea. But whose
decision was it?

transcript

Back to The Daily

bars

0:00/22:14

-22:14

transcript

\hypertarget{listen-to-the-daily-john-bolton-is-fired-or-did-he-resign-1}{%
\subsection{Listen to `The Daily': John Bolton Is Fired. Or Did He
Resign?}\label{listen-to-the-daily-john-bolton-is-fired-or-did-he-resign-1}}

\hypertarget{hosted-by-michael-barbaro-produced-by-luke-vander-ploeg-alexandra-leigh-young-and-julia-longoria-and-edited-by-lisa-tobin-and-marc-georges}{%
\subsubsection{Hosted by Michael Barbaro, produced by Luke Vander Ploeg,
Alexandra Leigh Young and Julia Longoria, and edited by Lisa Tobin and
Marc
Georges}\label{hosted-by-michael-barbaro-produced-by-luke-vander-ploeg-alexandra-leigh-young-and-julia-longoria-and-edited-by-lisa-tobin-and-marc-georges}}

\hypertarget{the-national-security-adviser-was-ousted-after-clashing-with-president-trump-over-issues-like-afghanistan-iran-and-north-korea-but-whose-decision-was-it}{%
\paragraph{The national security adviser was ousted after clashing with
President Trump over issues like Afghanistan, Iran and North Korea. But
whose decision was
it?}\label{the-national-security-adviser-was-ousted-after-clashing-with-president-trump-over-issues-like-afghanistan-iran-and-north-korea-but-whose-decision-was-it}}

\begin{itemize}
\item
  {[}PHONE RINGING{]}
\item
  peter baker\\
  Hello.
\item
  michael barbaro\\
  Hey, Peter. It's Michael Barbaro.
\item
  peter baker\\
  Hey, how are you?
\item
  michael barbaro\\
  Good, good, good. I imagine I'm catching you absolutely in the thick
  of it.
\item
  peter baker\\
  Yes.
\item
  michael barbaro\\
  Just how in the thick of it?
\item
  peter baker\\
  Like crashing. What's going on?
\item
  michael barbaro\\
  OK. We saw the president's tweet about John Bolton being out as
  National Security Advisor, and I wonder if I could just ask you a few
  questions about it, or is now just not a good time?
\item
  peter baker\\
  Yeah. I think it would be better not to.
\item
  michael barbaro\\
  OK.
\item
  peter baker\\
  Is that all right? Sorry.
\item
  michael barbaro\\
  Yes. We'll talk in a little bit.
\item
  peter baker\\
  OK. Thanks, bye.
\item
  michael barbaro\\
  OK, bye. {[}SNAZZY MUSIC{]} From the New York Times, I'm Michael
  Barbaro. And this is ``The Daily.'' Today, Peter Baker eventually
  takes the call and explains what happened to John Bolton. It's
  Wednesday, September 11th.

  Peter.
\item
  peter baker\\
  Hello.
\item
  michael barbaro\\
  Hi.
\item
  peter baker\\
  Hi there. How are you?
\item
  michael barbaro\\
  What a difference a day makes.

  Little 24 hours.
\item
  peter baker\\
  Just another day in crazy town, as John Kelly would call it.
\item
  michael barbaro\\
  Right. John Kelly. Him. So yesterday, Peter, you were telling us that
  President Trump was calling off a peace deal with the Taliban, which
  is exactly what his National Security Advisor, John Bolton, was
  pushing for. And that felt like a John Bolton win. Now today, Bolton
  has either been fired or he quit, depending on who you believe. So how
  do you square those two?
\item
  peter baker\\
  Well, sometimes you can win a policy fight and lose the war, right? In
  this case, Bolton did get what he wanted in terms of ending the
  negotiations, at least for now, with the Taliban. But he had so worn
  down his relationship with the president that, by less than 24 hours
  later, he's out of a job.
\item
  archived recording (stephen colbert)\\
  Welcome to The Late Show. I'm your host, Stephen Colbert.
\end{itemize}

peter baker

After we talked yesterday, there was a confrontation between the
president and Bolton over this very topic.

\begin{itemize}
\tightlist
\item
  archived recording (stephen colbert)\\
  Donald Trump's invited the Taliban to Camp David the weekend before
  9/11. That's like --- there's nothing that's like that.
\end{itemize}

peter baker

People in Vice President Mike Pence's camp were upset at Bolton because
of a story ---

\begin{itemize}
\tightlist
\item
  archived recording (stephen colbert)\\
  Both Vice President Mike Pence and National Security Advisor John
  Bolton thought it was a mistake, but according to ---
\end{itemize}

peter baker

--- that had come out saying that Pence had also been against the Camp
David meeting with the Taliban. That was perceived by Pence's people as
a way from Bolton's camp to basically enlist allies to say, hey, it
wasn't just him.

\begin{itemize}
\tightlist
\item
  archived recording (stephen colbert)\\
  But according to people familiar with the talks, Trump wanted to be
  the deal maker who would put the final parts together himself, or at
  least be perceived to be. So.
\end{itemize}

peter baker

They deeply resented that. That was blamed on Bolton, fairly or not, and
so by the time the president talked with Bolton last night, feelings
were really raw.

michael barbaro

Right, because Bolton had opposed the Camp David meeting, and so the
thinking is that Bolton would have been the one, or people around him,
to get the word out that, oh, I'm not alone. Look, even the Vice
President opposed this meeting.

peter baker

Exactly, and for months, the president had been kind of bristling at
what he perceived to be John Bolton's overly hawkish view of the world.

\begin{itemize}
\tightlist
\item
  archived recording (donald trump)\\
  John Bolton is absolutely a hawk. If it was up to him, he'd take on
  the whole world at one time, OK?
\end{itemize}

peter baker

You know, he would even joke about that.

\begin{itemize}
\tightlist
\item
  archived recording (donald trump)\\
  I actually temper John, which is pretty amazing, isn't it?
\end{itemize}

peter baker

He was the peacemaker among the two.

\begin{itemize}
\tightlist
\item
  archived recording (donald trump)\\
  Nobody thought that was going to --- I'm the one that tempers him, but
  that's OK. I have different sides. I mean, I have John Bolton, and I
  have other people that are a little more dovish than him.
\end{itemize}

peter baker

And in some ways, that was true because Bolton didn't like a lot of this
diplomacy that was going on, didn't like dealing with the North Koreans
or the Iranians or the Taliban. He didn't trust any of them. He didn't
think that the United States should get in bed with these bad actors.

\begin{itemize}
\tightlist
\item
  archived recording (john bolton)\\
  Yeah, here's an all-purpose insult that you can use. I'll apply it to
  the North Koreans. Question, how do you know when the North Korean
  regime is lying? Answer, when their lips are moving.
\end{itemize}

peter baker

Whereas the president, as you know, is somebody who's looking for the
big deal. He's going to make a deal with the Taliban. Maybe he'll make a
deal with Iran.

\begin{itemize}
\tightlist
\item
  archived recording (donald trump)\\
  Again, I think Iran has tremendous economic potential, and I look
  forward to letting them get back to the stage where they can show
  that.
\end{itemize}

peter baker

And that was at the core of the very big difference between him and his
National Security Advisor.

michael barbaro

So what happens this morning?

peter baker

Well, the morning begins actually kind of normal in the White House.

There was a meeting of the national security team in the Situation Room.
It was chaired by John Bolton, as it should normally be without the
president. A couple hours later, the White House scheduled a briefing
that Bolton was going to give in the White House briefing room to the
press, along with Secretary of State, Mike Pompeo, and Secretary of the
Treasury, Steve Mnuchin, to talk about terrorism efforts.

Then ---

\begin{itemize}
\tightlist
\item
  archived recording\\
  Dramatic breaking news, just as we begin the hour. The President of
  the United States announcing on Twitter, his National Security
  Advisor, John Bolton, is leaving.
\end{itemize}

peter baker

--- right around noon comes the tweet from the president.

\begin{itemize}
\tightlist
\item
  archived recording\\
  See the tweet from the president right there. I asked for John, for
  his resignation, which was given to me this morning. I thank John very
  much for his service. I will be naming a new National Security Advisor
  this week.
\end{itemize}

peter baker

Within minutes, though, comes another tweet.

\begin{itemize}
\tightlist
\item
  archived recording\\
  In a surreal moment, 12 minutes later ---
\end{itemize}

peter baker

This one from John Bolton.

\begin{itemize}
\tightlist
\item
  archived recording\\
  --- Bolton denied he was fired, tweeting, I offered to resign last
  night and President Trump said, let's talk tomorrow.
\end{itemize}

peter baker

He's saying that he offered his resignation and that the president had
accepted. In other words, it was his idea, not the president's. So I
went ahead and texted him just to make sure we were understanding that
he's giving us a different version. He was disputing the president.

michael barbaro

You texted John Bolton.

peter baker

I texted John Bolton on his phone, and he texted back, offered last
night without his asking. Slept on it and gave it to him this morning.
So John Bolton is disputing the version that the president gave. He's
saying it's not true.

michael barbaro

And Peter, what happens to that press conference where Bolton was
supposed to talk?

peter baker

Well, they still had a briefing ---

\begin{itemize}
\tightlist
\item
  archived recording (steve mnuchin)\\
  Hello, everybody. So Secretary Pompeo and I are here today to talk
  about the president's new executive order.
\end{itemize}

peter baker

--- with Pompeo and Mnuchin, the secretaries of State and Treasury, and
they're there to talk about ---

\begin{itemize}
\tightlist
\item
  archived recording (steve mnuchin)\\
  Fighting global terrorism.
\end{itemize}

peter baker

--- terrorism, financing, and how they're planning to be tougher on
terrorism, as we have the anniversary of 9/11.

\begin{itemize}
\tightlist
\item
  archived recording (mike pompeo)\\
  At this time, Secretary Mnuchin and I are happy to take a couple of
  questions on this topic.
\end{itemize}

peter baker

But of course, everybody in the room wants to ask about ---

\begin{itemize}
\tightlist
\item
  archived recording (reporter)\\
  Did John Bolton get fired or did he quit, and did he ---
\end{itemize}

peter baker

--- what happened to Bolton? And what the back story is, of course, is
that Pompeo and Bolton have been at odds for months. They have been in
this epic feud over, basically, the ear of the president. They share
some of the same policy views. They're both pretty hawkish
conservatives. But Pompeo has done more to stay within the president's
good graces, I would say, than John Bolton.

\begin{itemize}
\tightlist
\item
  archived recording (reporter)\\
  Did he leave the White House because he disagreed with you in
  particular over talks with the Taliban?
\end{itemize}

peter baker

So we asked Pompeo about this in the briefing room.

\begin{itemize}
\tightlist
\item
  archived recording (mike pompeo)\\
  Uh, I'll leave it to the president to talk about the reasons he made
  decision, but I would say this. The president's entitled to the staff
  that he wants.
\end{itemize}

peter baker

And I had to say, there were no tears shed on his part for John Bolton.
He said, look, the president deserved to have somebody he trusts and
values.

\begin{itemize}
\tightlist
\item
  archived recording (mike pompeo)\\
  There were many times Ambassador Bolton I disagreed. That's to be
  sure.
\end{itemize}

peter baker

And he very openly said, look, I had a lot of disagreements. So he
wasn't trying to even pretend that they didn't have a rivalry.

\begin{itemize}
\tightlist
\item
  archived recording (mike pompeo)\\
  There were definitely places that ambassador and I had different views
  about how we should proceed.
\end{itemize}

peter baker

At one point, they were asked ---

\begin{itemize}
\tightlist
\item
  archived recording (reporter)\\
  Were you two blindsided by what occurred today?
\end{itemize}

peter baker

--- were they blindsided by this. And both of them grinned.

\begin{itemize}
\tightlist
\item
  archived recording (mike pompeo)\\
  I'm never surprised.
\end{itemize}

peter baker

And Pompeo said, I'm not surprised by anything. I'm never surprised, he
said.

\begin{itemize}
\tightlist
\item
  archived recording (mike pompeo)\\
  And I don't mean that on just this issue.
\end{itemize}

michael barbaro

If Pompeo and Bolton are both hawks, help me understand what a
difference might look like for them. For example, how did Bolton handle
Afghanistan and the Taliban talks versus Pompeo?

peter baker

Well that's one area where they were at odds. Pompeo was more favorable
toward the talks because he knew the president was for it. In other
words, if he were left to his own devices, it might not be his
particular choice, but Pompeo was more willing to subordinate his views
to his president's. And that was the lesson he learned from Rex
Tillerson, who didn't do that, the first Secretary of State, and ended
up getting fired as a result. Bolton, in some ways, was, policy wise,
very different than Rex Tillerson, but in terms of not simply going
along to get along, he was less willing to simply go along with policy
ideas that he didn't favor. Where he probably went crosswise is, the
president and his people value loyalty. And they never quite accepted
Bolton as a member of the team.

michael barbaro

Does all of that, in the end, kind of suggest to you that despite this
dispute over exactly what happened --- no, I wasn't fired. I resigned.
No, I fired you --- that in the end it sounds like it was Trump who
pushed out Bolton?

peter baker

Well, look, I mean, the relationship was broken. It was inevitable that
this was going to happen. Whether it was going to happen today or some
other day, it doesn't really matter that much because we knew that he
overstayed his welcome. In other words, he was no longer really going to
be welcome in that White House.

michael barbaro

We'll be right back.

\^{}archived recording (alicia burke)\^{}

Sometimes, a single moment can change everything, like when filmmaker
Ken Burns first saw his father cry.

\begin{itemize}
\tightlist
\item
  archived recording (ken burns)\\
  It was after my mom died. This one film had given my dad permission to
  express real emotions.
\end{itemize}

\^{}archived recording (alicia burke)\^{}

Or when Sal Khan, a hedge fund analyst, decided to tackle the
impossible.

\begin{itemize}
\tightlist
\item
  archived recording (sal khan)\\
  When the IRS asks you the mission statement, I filled out, free world
  class education for anyone, anywhere.
\end{itemize}

\^{}archived recording (alicia burke)\^{}

I'm Alicia Burke, host of That Made All The Difference, a new podcast
from Bank of America about the defining moments in the lives of
achievers. What would you like the power to do?

\begin{itemize}
\tightlist
\item
  archived recording (larissa anderson)\\
  I'm Larissa Anderson. I am one of the people that make ``The Daily,''
  and I'm also one of the people who made a show called ``Caliphate.''
  It's a show that follows Rukmini Callimachi as she reports on ISIS. I
  can remember the first time I heard the interview Rick Meaney and our
  producer, Andy Mills, had done with a former ISIS member. He confesses
  to things he does as part of the group's religious police. It felt
  like such a rare interview. When we had our doubts about the story he
  told us, we pulled on every lever we had to get to the bottom of it.
  We leaned on reporters with sources in national security agencies. We
  used the full muscle of The New York Times newsroom to get to the
  truth of his story. I can't think of another place this story could
  come together, and that is what you get when you subscribe to The New
  York Times. We are setting out to make stories that only The New York
  Times can tell. So if you want to support our efforts to keep bringing
  shows like this to you, please subscribe to The New York Times. And
  thank you.
\end{itemize}

michael barbaro

So Peter, I guess the question here is, why hire a National Security
Advisor who is so fundamentally different from you on the issue of
national security?

peter baker

Well, that's a great question. What President Trump liked about John
Bolton ---

\begin{itemize}
\tightlist
\item
  archived recording anchor\\
  Joining me right now, former U.S. ambassador to the U.N., Ambassador
  John Bolton. Good to see you, ambassador.
\end{itemize}

peter baker

We're seeing him on Fox News, and very aggressively articulating a
conservative point of view.

\begin{itemize}
\tightlist
\item
  archived recording (john bolton)\\
  People have talked about closing this border for a long time. I was in
  the Reagan administration when we passed the 1986 Immigration Reform
  and Control Act, a lot of which was, close the borders.
\end{itemize}

peter baker

He thought that Bolton was kind of a like-minded pugilistic version of a
political figure. They do share some things. This idea of America First
does play into John Bolton's philosophy as well. He describes himself as
an American nationalist. He's not all that thrilled with the allies.
Neither is President Trump. He doesn't really believe in the U.N. or
allies or international organizations. He doesn't think that they're
very effective or that the United States should be subordinating itself
to them. Neither does President Trump. What he missed was that Bolton's
view is very different than his own view, than President Trump's view,
on a lot of these big issues. On North Korea, where President Trump
wanted to negotiate with Kim Jong Un, and John Bolton thought that was
probably unwise.

\begin{itemize}
\tightlist
\item
  archived recording (john bolton)\\
  There's not a lot of time to waste here. Talking to the North Koreans
  is a waste of time.
\end{itemize}

peter baker

On Iran ---

\begin{itemize}
\tightlist
\item
  archived recording (john bolton)\\
  How long would it take for Iran to get a deliverable nuclear weapon?
  Roughly one day after North Korea gets it.
\end{itemize}

peter baker

--- where just a couple of months ago, John Bolton teed up a retaliatory
strike for the downing of an American surveillance drone, and the
president pulled it back at the last minute, right? On Russia, John
Bolton, much, much tougher, much more skeptical of Vladimir Putin's
Russia than the president, would never invite them back into the G7 the
way the president has talked about. And finally, of course, these last
few days, we see highlighted in this very big dramatic way ---

\begin{itemize}
\tightlist
\item
  archived recording (john bolton)\\
  There's no blind trust in the Taliban in this administration. That's
  for sure.
\end{itemize}

peter baker

--- the idea of negotiating with the Taliban.

michael barbaro

Peter, it seems like Bolton did ultimately have an outsized influence
over these huge foreign policy issues that you just outlined. He wanted
the U.S. to be out of the Iran nuclear deal. The U.S. is now out. He
opposed peace talks with Kim Jong Un in North Korea. Those talks have
more or less stalled, right? He opposed a peace deal with the Taliban.
The president just said, those are over. Not bad.

peter baker

Yeah, in some ways, that's true, obviously, but it depends on the issue.
On Iran, for instance ---

\begin{itemize}
\tightlist
\item
  archived recording (donald trump)\\
  And I'm not looking to hurt Iran at all. I'm looking to have Iran say,
  no nuclear weapons. We have enough problems in this world right now
  with nuclear weapons. No nuclear weapons for Iran. And I think we'll
  make a deal.
\end{itemize}

peter baker

He was pushing on an open door. The president agreed with John Bolton's
view of the Iran nuclear deal. They both thought it was a terrible idea,
that it was giving Iran too much leeway, and that they should get out.
So they agreed on that. But they didn't agree on, for instance, regime
change. That's something that Bolton has always favored in Iran. The
president has now publicly said, several times, I'm not for regime
change.

\begin{itemize}
\tightlist
\item
  archived recording (donald trump)\\
  These are great people. It has a chance to be a great country, with
  the same leadership. We're not looking for regime change. I just want
  to make that clear. We're looking for no nuclear weapons.
\end{itemize}

peter baker

On North Korea, you're right. The talks are stalled. That certainly
pleased John Bolton because he thought that they're counterproductive
and dangerous, but it probably wasn't because of him. It was really more
because of Kim not coming to the table with anything meaningful. So in a
lot of ways, John Bolton did have a lot of influence. He particularly
helped, for instance, pull the United States out of some treaties, like
the I.N.F. Treaty with Russia, but his successes were in areas where he
was working with the president's own instinct, where he was going in the
same direction. Where he got in trouble was where he was fighting
against the tide.

michael barbaro

I'm struck by the fact that the thing you say the president liked most
about Bolton, his pugilism, his combative style, is also kind of what
cost Bolton his job, right?

peter baker

Yeah. I mean, this was something people said even 17 months ago when he
was hired. When he was hired, people would say, well, I don't know. I
put those two in a room, the inevitable clash is going to eventually
drive them apart. To some people, what's actually surprising is that it
took this long. But he viewed his job as how to stop bad deals from
happening. In fact, within minutes of his resigning, I talked with a
person who's close to Bolton, who said, look, for the 17 months that
John Bolton was National Security Advisor, there were no bad deals.
That's his view. And we've seen this before, where people who surround
the president view their job as stopping the president from doing things
that they consider to be bad, right? In Rex Tillerson's case, for
instance, it was the other way around. He wanted to stop the president
from being too combative in the world. John Bolton wanted to stop the
president from being too naive, in his view, too willing to get in bed
with bad actors who can't be trusted. In both cases, it didn't end up
well.

michael barbaro

So Bolton is leaving with all these same foreign policy matters largely
unresolved, which is, I guess, a problem for the next National Security
Advisor. So what do you think ends up happening next? And what can you
tell us about who you suspect the president will choose for that job and
what it will tell us about how he's thinking about those issues?

peter baker

Right. Exactly. I mean, the president said today that he will name
somebody next week. Now it suggests he has, perhaps, somebody in mind,
and one of the names, for instance, we've heard is a guy named Steve
Biegun. Steve Biegun Is a former George W. Bush administration official
who has been President Trump's chief negotiator with the North Koreans.

\begin{itemize}
\tightlist
\item
  archived recording (steve biegun)\\
  I fully understand the importance of this job. The issues are tough,
  and they will be tough to resolve. But the president has created an
  opening, and it's one that we must take by seizing every possible
  opportunity to realize the vision for a peaceful future for the people
  of North Korea.
\end{itemize}

peter baker

That would seem to tell you, if he picks Biegun, that he's looking for
somebody on the opposite side of Bolton when it comes to some of these
diplomacy issue, somebody who's willing to talk with some of these bad
actors in order to try to negotiate some sort of an agreement. On the
other hand, if he picks a different figure, somebody who's more hardline
like John Bolton, maybe you'll see that the policies won't change. And
the question really would be more about personality and fit and
chemistry. So we're looking to see who the choice is because it will be
telling in figuring out where the president will go in this next year
before re-election.

michael barbaro

So Peter, finally, what are you thinking about on this fine afternoon,
besides the fact that ``The Daily'' calls you all the time now?

peter baker

Well, this is a really interesting moment in foreign policy because we
do have all these balls up in the air. And we can't tell whether or not
any of them are going to come to fruition or not, right? The president
is talking with, or talking about talking with, North Korea, Iran, the
Taliban, Afghanistan.

michael barbaro

Literally, some of our greatest adversaries in history.

peter baker

Biggest enemies, greatest adversaries. If he were to pull off what he
wanted to pull off, it would reshape the world and America's place in
it. But, and there's a big but, none these are guaranteed. These are all
super hard, super entrenched disputes that have gone through many, many
presidencies before without being resolved. And so he has set himself up
for either a big win or a big loss, depending on how it turns out over
these months to come.

michael barbaro

I get the sense that John Bolton would like to tell the story of what
really happened, and I hope you will then share with us.

peter baker

I'm looking forward to sitting down with him and hearing the story.

michael barbaro

We'll talk to you then. Thank you, Peter.

peter baker

Thank you.

michael barbaro

We'll be right back.

\begin{itemize}
\tightlist
\item
  archived recording\\
  ThirdLove designs bras to fit you and not the other way around. They
  use millions of real women's measurements, not size templates, to
  create products that fit perfectly and make you feel great inside and
  out. ThirdLove believes that everybody deserves the perfect fit, so
  they're offering 15\% off your order. Go to ThirdLove.com/daily to
  find your perfect fitting bra and get 15\% off your purchase. That's
  ThirdLove.com/daily for 15\% off.
\end{itemize}

michael barbaro

Here's what else you need to know today.

\begin{itemize}
\tightlist
\item
  archived recording\\
  Today, I informed my intention ---
\end{itemize}

\^{}archived recording (benjamin netanyahu)\^{}

{[}HEBREW{]}

\begin{itemize}
\tightlist
\item
  archived recording\\
  --- with the establishment of the next government, to apply Israeli
  sovereignty over the Jordan Valley and North Dead Sea.
\end{itemize}

michael barbaro

In a speech on Tuesday, Israeli Prime Minister, Benjamin Netanyahu,
promised that if re-elected next week, he would annex nearly a third of
the occupied West Bank.

\begin{itemize}
\tightlist
\item
  archived recording\\
  So I speak to you, citizens of the state of Israel ---
\end{itemize}

\^{}archived recording (benjamin netanyahu)\^{}

{[}HEBREW{]}

\begin{itemize}
\tightlist
\item
  archived recording\\
  --- for the sake of future generations, and future generations give me
  the power to guarantee Israel's security, give me the power to
  determine Israel's borders. Thank you very much.
\end{itemize}

michael barbaro

Netanyahu, who failed to create a governing coalition after his last
election, prompting a new one on September 17, said he would take over
the territory in the name of Israeli security, and with what he
described as the approval of the Trump administration. But the move was
widely seen as a last-minute appeal to right-wing voters that would
reduce any future Palestinian state to a small area encircled by Israel.
And later today, congressional Democrats plan to introduce gun safety
legislation that would make it easier for police to take away guns from
those deemed dangerous, bar gun purchases by people convicted of hate
crimes, and ban magazines that hold more than 10 rounds. The bills,
crafted in the wake of the latest mass shootings, could pass in the
Democratically controlled House, but face significant opposition in the
Republican-controlled Senate.

That's it for ``The Daily.'' I'm Michael Barbaro. See you tomorrow.

Mr. Trump and his aides privately blamed the national security adviser
for news reports describing Mr. Bolton's opposition to the deal. Vice
President Mike Pence and his camp likewise grew angry at reports
suggesting he had agreed with Mr. Bolton, seeing them as an effort to
bolster the adviser's position.

``I informed John Bolton last night that his services are no longer
needed at the White House,''
\href{https://twitter.com/realDonaldTrump/status/1171452880055746560}{the
president tweeted}. ``I disagreed strongly with many of his suggestions,
as did others in the Administration, and therefore I asked John for his
resignation, which was given to me this morning. I thank John very much
for his service.''

Mr. Bolton disputed the president's version of events in his own tweet
12 minutes later. ``I offered to resign last night and President Trump
said, `Let's talk about it tomorrow,'''
\href{https://twitter.com/AmbJohnBolton/status/1171455806069305346}{Mr.
Bolton wrote, without elaborating}.

Responding to a question from The New York Times via text, Mr. Bolton
said his resignation was his own initiative, not the president's.
``Offered last night without his asking,'' he wrote. ``Slept on it and
gave it to him this morning.''

Mr. Trump said he
\href{https://www.nytimes.com/2019/09/10/us/john-bolton-replacement.html}{would
appoint a replacement} ``next week,'' setting off a process that should
offer clues to where he wants to take his foreign policy. In the
meantime, the White House said Charles M. Kupperman, the deputy national
security adviser, would serve in an acting capacity. No other president
has had four national security advisers in his first three years in
office.

\emph{{[}News Analysis:}
\href{https://www.nytimes.com/2019/09/10/us/politics/bolton-firing.html?action=click\&module=Intentional\&pgtype=Article}{\emph{John
Bolton's departure removes one of the last constraints on President
Trump's sense of the possible in world affairs.}}\emph{{]}}

While it was clear for months that Mr. Bolton was on thin ice, the end
came with a brutal suddenness typical of the Trump White House. On
Tuesday morning, Mr. Bolton led a meeting of the national security
principals in the Situation Room, with no sign that anything was about
to break.

At 11 a.m., the White House even scheduled a 1:30 p.m. news briefing
where Mr. Bolton would talk about terrorism alongside Secretary of State
Mike Pompeo and Treasury Secretary Steven Mnuchin. But then came Mr.
Trump's tweet two minutes before noon, and Mr. Bolton left the White
House.

The briefing went forward without him, and Mr. Pompeo, who has feuded
with Mr. Bolton for months, shed no tears about the president's
decision. ``He should have people that he trusts and values, and whose
efforts and judgments benefit him in delivering American foreign
policy,'' Mr. Pompeo told reporters.

The secretary also made no effort to hide his rivalry with Mr. Bolton.
``There were definitely places that Ambassador Bolton and I had
different views about how we should proceed,'' he said. Asked if he was
blindsided by the decision, Mr. Pompeo said, ``I'm never surprised,'' as
he and Mr. Mnuchin grinned broadly.

Mr. Pompeo and Mr. Bolton generally shared a conservative policy
outlook, but the secretary of state has proved more adept at managing
the president and subordinating his views to Mr. Trump's, while Mr.
Bolton kept pushing his beliefs even after they were rejected.

Mr. Pompeo did not see Mr. Bolton as a team player, but as someone who
undermined the president's policies. Mr. Bolton saw Mr. Pompeo as a
politician more interested in currying Mr. Trump's favor to have his
support in a future run for Senate.

Mr. Bolton's adversaries inside the administration have been after him
for weeks, spreading stories about how the national security adviser had
been excluded from meetings and was on the outs with the president.

When Mr. Bolton declined to appear on two Sunday talk shows during the
Group of 7 summit last month, his internal critics said it was because
he refused to defend the president's policies on Russia. Mr. Bolton
denied that, saying he did not go on the shows because he anticipated
that the main topic would be the trade war with China, which is not his
area of specialty.

\href{https://www.nytimes.com/interactive/2018/03/16/us/politics/all-the-major-firings-and-resignations-in-trump-administration.html}{}

\includegraphics{https://static01.nyt.com/images/2018/07/05/us/all-the-major-firings-and-resignations-in-trump-administration-promo-1530825933054/all-the-major-firings-and-resignations-in-trump-administration-promo-1530825933054-articleLarge-v2.jpg}

\hypertarget{the-turnover-at-the-top-of-the-trump-administration}{%
\subsection{The Turnover at the Top of the Trump
Administration}\label{the-turnover-at-the-top-of-the-trump-administration}}

Since President Trump's inauguration, White House staffers and cabinet
officials have left in firings and resignations, one after the other.

Mr. Bolton, the hard-liner, saw his job as keeping Mr. Trump from going
soft in what he considered fuzzy-headed diplomacy. ``While John Bolton
was national security adviser for the last 17 months, there have been no
bad deals,'' a person close to Mr. Bolton said minutes after the
president's announcement on Tuesday, reflecting the ousted adviser's
view.

To Mr. Bolton's aggravation, the president has continued to court Kim
Jong-un, the repressive leader of North Korea, despite Mr. Kim's refusal
to surrender his nuclear program and despite
\href{https://www.nytimes.com/2019/07/30/world/asia/north-korea-projectiles.html}{repeated
short-range missile tests} by the North that have rattled its neighbors.

In recent days, Mr. Trump has also
\href{https://www.nytimes.com/2019/08/26/world/europe/g7-trump-china-trade-war.html}{expressed
a willingness to meet} with President Hassan Rouhani of Iran under the
right circumstances, and even to extend short-term financing to Tehran.
Mr. Pompeo confirmed on Tuesday that it was possible such a meeting
could take place this month on the sidelines of the United Nations
General Assembly session in New York.

The tension between Mr. Trump and Mr. Bolton was aggravated in recent
months by the president's decisions to call off
\href{https://www.nytimes.com/2019/06/20/world/middleeast/iran-us-drone.html}{a
planned airstrike on Iran} in retaliation for the downing of an American
surveillance drone and to meet with Mr. Kim at the Demilitarized Zone
and
\href{https://www.nytimes.com/2019/06/30/world/asia/trump-north-korea-dmz.html}{cross
over into North Korea}.

Mr. Bolton favored the strike on Iran and publicly criticized the recent
North Korean missile tests that Mr. Trump brushed off. Mr. Trump
disavowed regime change in Iran, a long-held goal of Mr. Bolton's. After
the president arranged the DMZ meeting with Mr. Kim via
\href{https://twitter.com/realDonaldTrump/status/1144740178948493314}{a
last-minute tweet}, Mr. Bolton did not accompany him and instead
proceeded on a previously scheduled trip to Mongolia.

The day after the DMZ meeting, Mr. Bolton pushed an internal policy
debate into the open by disputing a Times story reporting that some
administration officials were considering an agreement with North Korea
for a nuclear freeze as an intermediate step toward full disarmament.

Mr. Bolton, on Twitter,
\href{https://twitter.com/AmbJohnBolton/status/1145646367865528320?s=20}{accused
someone of trying to ``box in the President''} and said ``there should
be consequences.'' It soon became clear those officials were Mr. Pompeo
and his special envoy, Stephen E. Biegun, making Mr. Bolton's tweet a
veiled attack on them.

The same day, Mr. Bolton's aides obtained a copy of notes taken by State
Department reporters during an off-the-record briefing with Mr. Biegun
discussing the nuclear freeze. Mr. Bolton tried to use those notes as a
cudgel in the internal policy battle, administration officials said.
Details of Mr. Biegun's meeting were leaked to the news outlet Axios.

Mr. Bolton's resignation on Tuesday caught allies and adversaries
off-guard. Senator Mitt Romney, Republican of Utah, called the news ``an
extraordinary loss for our nation and the White House.''

``John Bolton is a brilliant man with decades of experience in foreign
policy,'' he said. ``His point of view was not always the same as
everybody else in the room. That's why you wanted him there. The fact
that he was a contrarian from time to time was an asset, not a
liability.''

But Republicans like Senator Rand Paul of Kentucky who have tried to
push Mr. Trump away from foreign intervention were openly gleeful.

``The threat of war worldwide goes down exponentially with John Bolton
out of the White House,'' Mr. Paul told reporters. ``I think his
advocacy for regime change around the world is a naïve worldview, and I
think that the world will be a much better place with new advisers to
the president.''

Among others pleased to be rid of Mr. Bolton were Iran's leaders, who
viewed him as an enemy of peace. Hesameddin Ashena, Mr. Rouhani's top
political adviser, tweeted that Mr. Bolton getting sidelined was ``a
definitive sign that Washington's maximum pressure on Iran has failed''
and that ``Iran's blockade will end.''

A former under secretary of state and ambassador to the United Nations
under President George W. Bush, Mr. Bolton, 70, known for his trademark
bushy mustache, was tapped as national security adviser in March 2018
after impressing Mr. Trump with
\href{https://www.nytimes.com/2018/04/08/us/politics/john-bolton-trump.html}{his
outspoken performances} on Fox News.

Mr. Bolton followed two military officers who held the post before him:
Michael T. Flynn, a retired lieutenant general who stepped down after 24
days and later pleaded guilty to lying to the F.B.I.; and his successor,
Lt. Gen. H. R. McMaster, who never forged a strong connection with the
president and was forced out.

Long before Mr. Trump popularized his ``America First'' slogan, Mr.
Bolton termed himself an ``Americanist'' who prioritized a cold-eyed
view of national interests and sovereignty over what they both saw as a
starry-eyed fixation on democracy promotion and human rights. They
shared a deep skepticism of globalism and multilateralism, a commonality
that empowered Mr. Bolton to use his time in the White House to
orchestrate the withdrawal of the United States from
\href{https://www.nytimes.com/2019/02/01/us/politics/trump-inf-nuclear-treaty.html}{arms
control treaties} and other
\href{https://www.nytimes.com/2018/10/03/world/middleeast/us-withdraws-treaty-iran.html}{international
agreements}.

But if Mr. Trump's original national security team was seen as
restraining a mercurial new commander in chief, the president found
himself sometimes restraining Mr. Bolton. Behind the scenes, he joked
about Mr. Bolton's penchant for confrontation. ``If it was up to John,
\href{https://www.nytimes.com/2019/05/16/world/middleeast/iran-war-donald-trump.html?module=inline}{we'd
be in four wars now},'' one senior official recalled the president
saying.

Mr. Trump also grew disenchanted with Mr. Bolton over the failed effort
to push out President Nicolás Maduro of Venezuela. Rather than the easy
victory he was led to anticipate, Mr. Trump has found himself bogged
down in a conflict over which he has less influence than he had assumed.

Russia was another flash point for the two. While Mr. Trump seeks to woo
President Vladimir V. Putin, Mr. Bolton considers Moscow a hostile
player. After Mr. Trump last month suggested inviting Russia back into
the Group of 7 despite its annexation of Crimea, Mr. Bolton traveled to
Ukraine to reassure its leaders of American support against Russian
aggression.

Advertisement

\protect\hyperlink{after-bottom}{Continue reading the main story}

\hypertarget{site-index}{%
\subsection{Site Index}\label{site-index}}

\hypertarget{site-information-navigation}{%
\subsection{Site Information
Navigation}\label{site-information-navigation}}

\begin{itemize}
\tightlist
\item
  \href{https://help.nytimes.com/hc/en-us/articles/115014792127-Copyright-notice}{©~2020~The
  New York Times Company}
\end{itemize}

\begin{itemize}
\tightlist
\item
  \href{https://www.nytco.com/}{NYTCo}
\item
  \href{https://help.nytimes.com/hc/en-us/articles/115015385887-Contact-Us}{Contact
  Us}
\item
  \href{https://www.nytco.com/careers/}{Work with us}
\item
  \href{https://nytmediakit.com/}{Advertise}
\item
  \href{http://www.tbrandstudio.com/}{T Brand Studio}
\item
  \href{https://www.nytimes.com/privacy/cookie-policy\#how-do-i-manage-trackers}{Your
  Ad Choices}
\item
  \href{https://www.nytimes.com/privacy}{Privacy}
\item
  \href{https://help.nytimes.com/hc/en-us/articles/115014893428-Terms-of-service}{Terms
  of Service}
\item
  \href{https://help.nytimes.com/hc/en-us/articles/115014893968-Terms-of-sale}{Terms
  of Sale}
\item
  \href{https://spiderbites.nytimes.com}{Site Map}
\item
  \href{https://help.nytimes.com/hc/en-us}{Help}
\item
  \href{https://www.nytimes.com/subscription?campaignId=37WXW}{Subscriptions}
\end{itemize}
