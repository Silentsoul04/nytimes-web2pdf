Sections

SEARCH

\protect\hyperlink{site-content}{Skip to
content}\protect\hyperlink{site-index}{Skip to site index}

\href{https://myaccount.nytimes.com/auth/login?response_type=cookie\&client_id=vi}{}

\href{https://www.nytimes.com/section/todayspaper}{Today's Paper}

\href{/section/opinion}{Opinion}\textbar{}When Is Impeachment Not
Impeachment?

\href{https://nyti.ms/2AsuOSK}{https://nyti.ms/2AsuOSK}

\begin{itemize}
\item
\item
\item
\item
\item
\item
\end{itemize}

Advertisement

\protect\hyperlink{after-top}{Continue reading the main story}

\href{/section/opinion}{Opinion}

Supported by

\protect\hyperlink{after-sponsor}{Continue reading the main story}

\hypertarget{when-is-impeachment-not-impeachment}{%
\section{When Is Impeachment Not
Impeachment?}\label{when-is-impeachment-not-impeachment}}

When the speaker of the House thinks it is politically foolhardy.

By
\href{https://www.nytimes.com/interactive/opinion/editorialboard.html}{The
Editorial Board}

The editorial board represents the opinions of the board, its editor and
the publisher. It is separate from the newsroom and the Op-Ed section.

\begin{itemize}
\item
  Sept. 17, 2019
\item
  \begin{itemize}
  \item
  \item
  \item
  \item
  \item
  \item
  \end{itemize}
\end{itemize}

\includegraphics{https://static01.nyt.com/images/2019/09/17/opinion/17impeachment/17impeachment-articleLarge.jpg?quality=75\&auto=webp\&disable=upscale}

To clarify: The House Judiciary Committee has begun an inquiry to
determine whether to recommend the impeachment of President Trump. The
effort has been underway since March 4, when the committee
\href{https://judiciary.house.gov/news/press-releases/house-judiciary-committee-unveils-investigation-threats-against-rule-law}{announced}
it would look into ``the alleged obstruction of justice, public
corruption, and other abuses of power'' on the part of the president.
Last Thursday, committee members passed a
\href{https://bass.house.gov/sites/karenbass.house.gov/files/Resolution\%20for\%20Investigative\%20Procedures.pdf}{resolution}
setting the parameters for the investigation ``to determine whether to
recommend articles of impeachment.'' On Tuesday, the panel began what
its chairman, Representative Jerry Nadler, has said will be an
``aggressive series of hearings'' to this end.

This does not mean that the committee will necessarily recommend
impeachment. But Mr. Nadler's team is working to establish whether that
step makes sense.

Unfortunately, there is tremendous confusion about what the Judiciary
Committee is up to --- largely because of conflicting signals from House
Democrats, who have been struggling with their public statements on
impeachment. Mr. Nadler has said repeatedly that his committee is
engaged in
\href{https://www.nytimes.com/2019/09/13/us/politics/trump-impeachment-definition.html}{an
impeachment investigation} --- or, if you prefer, an impeachment
inquiry. He insists the
``\href{https://nadler.house.gov/news/documentsingle.aspx?DocumentID=394065}{nomenclature}''
does not matter. The House speaker, Nancy Pelosi, and her leadership
team clearly disagree. They assiduously avoid the ``I'' word, painting
the committee's work as
\href{https://www.nytimes.com/2019/04/22/us/politics/impeaching-trump-pelosi.html}{garden-variety
oversight.}

As a result, even Democratic lawmakers don't seem to know whether they
are engaged in an impeachment inquiry. Representative Pramila Jayapal
\href{https://www.politico.com/story/2019/09/10/impeachment-democrats-trump-1488401}{has
said} ``yes.'' Representative Jim Himes has said ``no.'' Last week,
Steny Hoyer, the House majority leader, said ``no'' --- then
\href{https://www.cnn.com/2019/09/12/politics/impeachment-investigation-vote-house-judiciary-committee/index.html}{backtracked},
claimed he'd misheard the question and offered a non-answer instead.

This is more than semantic hairsplitting. It is a reflection of the
Democrats' divisions over the wisdom of impeaching Mr. Trump. Advocates
of impeachment are eager to play up, and skeptics to play down, the
possibility of the Judiciary Committee's work leading in that direction.
\href{https://www.needtoimpeach.com}{Need to Impeach}, the advocacy
group founded by the Democratic presidential candidate Tom Steyer,
called Thursday's resolution vote a
``\href{https://thehill.com/homenews/house/460793-dems-impeachment-message-leads-to-plenty-of-head-scratching}{pivotal
moment}.'' The speaker's camp characterized it as non-news. At her
Thursday news conference, Ms. Pelosi
\href{https://abcnews.go.com/Politics/pelosi-exasperated-repeated-impeachment-questions/story?id=65564443}{bristled}
when reporters pressed her on whether an impeachment investigation was
underway. The conference was ``gathering facts'' as it had been doing
for months and would make a decision ``when we're ready,'' she said.
``That's all I have to say about this subject.''

Complicating matters, in attempting to wrest documents and testimony
from
\href{https://www.nytimes.com/2019/04/25/us/politics/trump-white-house-democrats.html}{a
White House committed to stonewalling}, Democrats have argued in court
filings that they are already engaged in an impeachment inquiry. (Some
\href{https://beta.washingtonpost.com/opinions/2019/04/24/how-trump-is-making-his-own-impeachment-more-likely/}{legal
experts} contend that impeachment proceedings --- versus ordinary
investigations --- could strengthen Democrats' hand in such scuffles.)
So even as the leadership and other skeptics insist there's nothing
unusual going on, Democrats' court filings cite an existing impeachment
inquiry.

Republicans have waded into the mix, arguing that impeachment
investigations of past presidents required an authorization vote by the
full House. Democrats counter that the rules have been changed such that
the committee already possesses the investigatory powers that
authorization once conferred, making a vote unnecessary.

You can see why people might be confused.

But the muddled messages are creating
\href{https://www.politico.com/story/2019/09/10/impeachment-democrats-trump-1488401}{their
own problems} and threatening to undermine the push for presidential
accountability. The contradictory statements make Democrats look divided
and conflicted, complicating efforts to build public confidence in their
oversight powers. Representative Tom McClintock, a Republican, has
\href{https://www.cnn.com/videos/tv/2019/09/12/lead-manu-raju-live-jake-tapper.cnn}{mocked}
the Democrats' strategy as, ``You can have your impeachment and deny it,
too.''

More concretely, the Department of Justice is using Democrats' ambiguity
to argue that the administration need not hand over information sought
by congressional investigators. ``Most prominently, the speaker of the
House has been emphatic that the investigation is not a true impeachment
proceeding,'' the department contended in a court brief filed Friday.

The Democratic leadership should try to find a way forward that, at the
very least, doesn't leave members contradicting one another and further
embolden Mr. Trump. Consider having members defer on the question to Mr.
Nadler's committee, which can reply, truthfully, that the panel is
uncovering the facts and will decide how to proceed based on those
facts.

As the Judiciary Committee's hearings begin, fresh attention will fall
on its investigation. This exercise is about more than politics; it is
about safeguarding the health of our democracy. Democrats need to
clarify to the public --- and to themselves --- where they are headed.

\emph{The Times is committed to publishing}
\href{https://www.nytimes.com/2019/01/31/opinion/letters/letters-to-editor-new-york-times-women.html}{\emph{a
diversity of letters}} \emph{to the editor. We'd like to hear what you
think about this or any of our articles. Here are some}
\href{https://help.nytimes.com/hc/en-us/articles/115014925288-How-to-submit-a-letter-to-the-editor}{\emph{tips}}\emph{.
And here's our email:}
\href{mailto:letters@nytimes.com}{\emph{letters@nytimes.com}}\emph{.}

\emph{Follow The New York Times Opinion section on}
\href{https://www.facebook.com/nytopinion}{\emph{Facebook}}\emph{,}
\href{http://twitter.com/NYTOpinion}{\emph{Twitter (@NYTopinion)}}
\emph{and}
\href{https://www.instagram.com/nytopinion/}{\emph{Instagram}}\emph{.}

Advertisement

\protect\hyperlink{after-bottom}{Continue reading the main story}

\hypertarget{site-index}{%
\subsection{Site Index}\label{site-index}}

\hypertarget{site-information-navigation}{%
\subsection{Site Information
Navigation}\label{site-information-navigation}}

\begin{itemize}
\tightlist
\item
  \href{https://help.nytimes.com/hc/en-us/articles/115014792127-Copyright-notice}{©~2020~The
  New York Times Company}
\end{itemize}

\begin{itemize}
\tightlist
\item
  \href{https://www.nytco.com/}{NYTCo}
\item
  \href{https://help.nytimes.com/hc/en-us/articles/115015385887-Contact-Us}{Contact
  Us}
\item
  \href{https://www.nytco.com/careers/}{Work with us}
\item
  \href{https://nytmediakit.com/}{Advertise}
\item
  \href{http://www.tbrandstudio.com/}{T Brand Studio}
\item
  \href{https://www.nytimes.com/privacy/cookie-policy\#how-do-i-manage-trackers}{Your
  Ad Choices}
\item
  \href{https://www.nytimes.com/privacy}{Privacy}
\item
  \href{https://help.nytimes.com/hc/en-us/articles/115014893428-Terms-of-service}{Terms
  of Service}
\item
  \href{https://help.nytimes.com/hc/en-us/articles/115014893968-Terms-of-sale}{Terms
  of Sale}
\item
  \href{https://spiderbites.nytimes.com}{Site Map}
\item
  \href{https://help.nytimes.com/hc/en-us}{Help}
\item
  \href{https://www.nytimes.com/subscription?campaignId=37WXW}{Subscriptions}
\end{itemize}
