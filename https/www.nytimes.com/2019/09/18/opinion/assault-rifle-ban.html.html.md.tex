Sections

SEARCH

\protect\hyperlink{site-content}{Skip to
content}\protect\hyperlink{site-index}{Skip to site index}

\href{https://myaccount.nytimes.com/auth/login?response_type=cookie\&client_id=vi}{}

\href{https://www.nytimes.com/section/todayspaper}{Today's Paper}

\href{/section/opinion}{Opinion}\textbar{}Why an Assault Weapons Ban
Hits Such a Nerve With Many Conservatives

\href{https://nyti.ms/32VlJy9}{https://nyti.ms/32VlJy9}

\begin{itemize}
\item
\item
\item
\item
\item
\item
\end{itemize}

Advertisement

\protect\hyperlink{after-top}{Continue reading the main story}

\href{/section/opinion}{Opinion}

Supported by

\protect\hyperlink{after-sponsor}{Continue reading the main story}

\hypertarget{why-an-assault-weapons-ban-hits-such-a-nerve-with-many-conservatives-}{%
\section{Why an Assault Weapons Ban Hits Such a Nerve With Many
Conservatives
}\label{why-an-assault-weapons-ban-hits-such-a-nerve-with-many-conservatives-}}

The premise of Trumpist populism is that the political preferences of a
shrinking minority of citizens matter more than democracy.

\href{https://www.nytimes.com/column/will-wilkinson}{\includegraphics{https://static01.nyt.com/images/2018/04/18/opinion/will-wilkinson/will-wilkinson-thumbLarge.png}}

By \href{https://www.nytimes.com/column/will-wilkinson}{Will Wilkinson}

Contributing Opinion Writer

\begin{itemize}
\item
  Sept. 18, 2019
\item
  \begin{itemize}
  \item
  \item
  \item
  \item
  \item
  \item
  \end{itemize}
\end{itemize}

\includegraphics{https://static01.nyt.com/images/2019/09/18/opinion/18wilkinson/18wilkinson-articleLarge.jpg?quality=75\&auto=webp\&disable=upscale}

``Hell, yes, we're going to take your AR-15, your AK-47,''
\href{https://www.nytimes.com/2019/09/12/us/orourke-debate-guns-take-ar15.html?smtyp=cur\&smid=tw-nytimes}{Beto
O'Rourke exclaimed} at last week's Democratic debate. The gathered crowd
was enthusiastic about the proposal, but other Texans were \ldots{} less
receptive.

One --- Briscoe Cain, a Republican state legislator from the Houston
exburbs --- responded with an ominous
\href{https://twitter.com/BetoORourke/status/1172359875093061632/photo/1}{tweet}:
``My AR is ready for you Robert Francis'' (addressing Mr. O'Rourke by
his given name).

Mr. O'Rourke first brought up the mandatory buyback idea shortly after
August's string of mass shootings. Several well-known conservative
commentators met the proposal with a series of warnings, exposing
chilling and increasingly open hostility to majoritarian democracy on
the right.

``So, this is --- what you are calling for is civil war,''
\href{https://www.mediamatters.org/tucker-carlson/tucker-carlson-says-gun-buyback-program-will-lead-civil-war}{Tucker
Carlson} of Fox News said of Mr. O'Rourke's comments. ``What you are
calling for is an incitement to violence.'' On ABC's ``The View,''
\href{https://twitter.com/justinbaragona/status/1168928087163248640?ref_src=twsrc\%5Etfw\%7Ctwcamp\%5Etweetembed\%7Ctwterm\%5E1168928087163248640\&ref_url=https\%3A\%2F\%2Fwww.motherjones.com\%2Fpolitics\%2F2019\%2F09\%2Fbriscoe-cain-beto-orourke\%2F}{Meghan
McCain maintained} that ``the AR-15 is by far the most popular gun in
America, by far. I was just in the middle of nowhere Wyoming. If you're
talking about taking people's guns from them, there's going to be a lot
of violence.'' On Twitter, the conservative writer
\href{https://twitter.com/EWErickson/status/1169339895002914816}{Erick
Erickson said}: ``I know people who keep AR-15's buried because they're
afraid one day the government might come for them. I know others who are
stockpiling them. It is not a stretch to say there'd be violence if the
gov't tried to confiscate them.''

Bear in mind a critical point: A buyback law could not take effect
without approval from majorities in both houses of Congress and
endorsement by the president. This is all but impossible without unified
Democratic control of government; in fact, because our electoral system
puts Democrats at a forbidding structural disadvantage, especially in
the Senate, Democrats would need to command overpowering supermajority
support to turn such a proposal into law.

In that light, all of these ominous ``there will be violence'' warnings
clearly imply that \emph{it simply doesn't matter} whether or not
mandatory buyback legislation is enacted by duly elected representatives
of the American people with an extraordinary popular mandate, because
the wildly outvoted minority would nevertheless be right to regard the
law as an intolerable injustice that warrants retaliatory violence. Just
ask them.

The likes of Erick Erickson jamming a cocked finger into his jacket
pocket and pointing it at democracy may not strike terror in your heart.
But the seditious principle behind these blustering, elliptical threats
is genuinely alarming.

Democracy is what we do to prevent political disagreement from turning
into violent conflict. But the premise of Trumpist populism is that the
legitimacy and authority of government is conditional on agreement with
the political preferences of a shrinking minority of citizens --- a
group mainly composed of white, Christian conservatives.

Who, you may sensibly ask, granted Tucker Carlson's target demographic
veto power over the legislative will of the American people? Nobody.
They got high on their own supply and anointed themselves the ``real
American'' sovereigns of the realm. But their relative numbers are
dwindling, and they live in fear of a future in which the law of the
land reliably tracks the will of the people. Therein lies the appeal of
a personal cache of AR-15s.

Weapons of mass death, and the submissive fear they engender, put teeth
on that shrinking minority's entitled claim to indefinite power. Without
the threat of violence, what have they really got? Votes? Sooner or
later, they won't have enough, and they know it.

Nearly every Republican policy priority lacks majority support. New
restrictions on abortion are
\href{https://www.pbs.org/newshour/politics/new-abortion-laws-are-too-extreme-for-most-americans-poll-shows}{unpopular}.
Slashing legal immigration levels is unpopular. The president's single
major legislative achievement, tax cuts for corporations and high
earners, is unpopular.

Public support for enhanced background checks stands at
\href{https://www.suffolk.edu/news-features/news/2019/09/09/12/58/suffolk-university-usa-today-poll-guns-sept-2019}{an
astonishing 90 percent}, and
\href{https://www.suffolk.edu/-/media/suffolk/documents/academics/research-at-suffolk/suprc/polls/national/2019/9_9_2019_marginals_pdftxt.pdf?la=en\&hash=643B7DFBB5697987D73D59B15DC83E5F7D584CDA}{60
percent}
(\href{https://www.politico.com/story/2019/08/07/poll-most-voters-support-assault-weapons-ban-1452586}{and
more}) support a ban on assault weapon sales. Yet Republican
legislatures block modest, popular gun control measures at every turn.
The security of the minority's self-ascribed right to make the rules has
become their platform's major plank, because unpopular rules don't stand
a chance without it. Float a rule that threatens their grip on power, no
matter how popular, and it's ``my AR is waiting for you, Robert
Francis.''

They'll tell you their thinly veiled threats are \emph{really} about
defending their constitutional rights. Don't believe it. The
conservative Supreme Court majority's 2008 decision in District of
Columbia v. Heller \href{https://www.oyez.org/cases/2007/07-290}{found}
an individual right to own guns for self-protection, but no civilian
needs a weapon
\href{https://www.nytimes.com/2019/08/13/us/dayton-shooter-video-timeline.html}{capable
of shooting 26 people in 32 seconds} to ward off burglars. The Second
Amendment
\href{https://beta.washingtonpost.com/news/morning-mix/wp/2018/02/22/does-the-second-amendment-really-protect-assault-weapons-four-courts-have-said-no/}{doesn't
grant the right to own one} any more than it grants the right to own a
surface-to-air missile.

They'll tell you their foreboding ``predictions'' of lethal resistance
are \emph{really} about preserving the means to protect the republic
against an overweening, rights-stomping state. Don't believe that,
either. It's \emph{really} about the imagined peril of a multicultural
majority running the show. Many countries that do more to protect their
citizens against gun violence are more, not less, free than we are.
\href{https://object.cato.org/sites/cato.org/files/human-freedom-index-files/human-freedom-index-2018-revised.pdf}{According
to the libertarian Cato Institute}, 16 countries enjoy a higher level of
overall freedom than the United States, and most of them ban or severely
restrict ownership of assault weapons. The freedom to have your head
blown off in an Applebee's, to flee in terror from the bang of a
backfiring engine, might not be freedom at all.

I'm not too proud to admit that in my misspent libertarian youth, I
embraced the idea that a well-armed populace is a bulwark against
tyranny. I imagined us a vast Switzerland, hived with rifles to defend
our inviolable rights against \ldots{} Michael Dukakis? What I slowly
came to see is that freedom is inseparable from political disagreement
and that holding to a trove of weapons as your last line of defense in a
losing debate makes normal ideological opposition look like nascent
tyranny and readies you to suppress it.

So it's no surprise that the most authoritarian American president in
living memory, elected by a paltry minority, is not threatened in the
least by citizen militias bristling with military firepower. He knows
they're on his side.

Democrats don't want to grind the rights of Republicans underfoot. They
want to feel safe and think it should be harder for unhinged lunatics to
turn Walmarts into abattoirs. But when minority-rule radicals hear
determined talk of mandatory assault rifle buybacks, they start to feel
surrounded. They hear the hammers clicking back, imagine themselves in
the majority's cross hairs.

That's why they're unmoved by the mounting heap of slaughtered
innocents, by schoolkids missing recess to rehearse being hunted. It's a
sacrifice they're willing to let other Americans make, because they
think democracy's coming for their power, and they're right.

Will Wilkinson
(\href{https://twitter.com/willwilkinson}{@willwilkinson}) is a
contributing Opinion writer and the vice president for research at the
Niskanen Center.

\emph{The Times is committed to publishing}
\href{https://www.nytimes.com/2019/01/31/opinion/letters/letters-to-editor-new-york-times-women.html}{\emph{a
diversity of letters}} \emph{to the editor. We'd like to hear what you
think about this or any of our articles. Here are some}
\href{https://help.nytimes.com/hc/en-us/articles/115014925288-How-to-submit-a-letter-to-the-editor}{\emph{tips}}\emph{.
And here's our email:}
\href{mailto:letters@nytimes.com}{\emph{letters@nytimes.com}}\emph{.}

\emph{Follow The New York Times Opinion section on}
\href{https://www.facebook.com/nytopinion}{\emph{Facebook}}\emph{,}
\href{http://twitter.com/NYTOpinion}{\emph{Twitter (@NYTopinion)}}
\emph{and}
\href{https://www.instagram.com/nytopinion/}{\emph{Instagram}}\emph{.}

Advertisement

\protect\hyperlink{after-bottom}{Continue reading the main story}

\hypertarget{site-index}{%
\subsection{Site Index}\label{site-index}}

\hypertarget{site-information-navigation}{%
\subsection{Site Information
Navigation}\label{site-information-navigation}}

\begin{itemize}
\tightlist
\item
  \href{https://help.nytimes.com/hc/en-us/articles/115014792127-Copyright-notice}{©~2020~The
  New York Times Company}
\end{itemize}

\begin{itemize}
\tightlist
\item
  \href{https://www.nytco.com/}{NYTCo}
\item
  \href{https://help.nytimes.com/hc/en-us/articles/115015385887-Contact-Us}{Contact
  Us}
\item
  \href{https://www.nytco.com/careers/}{Work with us}
\item
  \href{https://nytmediakit.com/}{Advertise}
\item
  \href{http://www.tbrandstudio.com/}{T Brand Studio}
\item
  \href{https://www.nytimes.com/privacy/cookie-policy\#how-do-i-manage-trackers}{Your
  Ad Choices}
\item
  \href{https://www.nytimes.com/privacy}{Privacy}
\item
  \href{https://help.nytimes.com/hc/en-us/articles/115014893428-Terms-of-service}{Terms
  of Service}
\item
  \href{https://help.nytimes.com/hc/en-us/articles/115014893968-Terms-of-sale}{Terms
  of Sale}
\item
  \href{https://spiderbites.nytimes.com}{Site Map}
\item
  \href{https://help.nytimes.com/hc/en-us}{Help}
\item
  \href{https://www.nytimes.com/subscription?campaignId=37WXW}{Subscriptions}
\end{itemize}
