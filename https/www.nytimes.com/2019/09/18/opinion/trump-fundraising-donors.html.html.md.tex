Sections

SEARCH

\protect\hyperlink{site-content}{Skip to
content}\protect\hyperlink{site-index}{Skip to site index}

\href{https://myaccount.nytimes.com/auth/login?response_type=cookie\&client_id=vi}{}

\href{https://www.nytimes.com/section/todayspaper}{Today's Paper}

\href{/section/opinion}{Opinion}\textbar{}The Changing Shape of the
Parties Is Changing Where They Get Their Money

\href{https://nyti.ms/2V1QdeO}{https://nyti.ms/2V1QdeO}

\begin{itemize}
\item
\item
\item
\item
\item
\item
\end{itemize}

Advertisement

\protect\hyperlink{after-top}{Continue reading the main story}

\href{/section/opinion}{Opinion}

Supported by

\protect\hyperlink{after-sponsor}{Continue reading the main story}

\hypertarget{the-changing-shape-of-the-parties-is-changing-where-they-get-their-money}{%
\section{The Changing Shape of the Parties Is Changing Where They Get
Their
Money}\label{the-changing-shape-of-the-parties-is-changing-where-they-get-their-money}}

Trump leads among small donors. Democrats now get plenty of support from
the wealthy, with predictable consequences.

\href{https://www.nytimes.com/by/thomas-b-edsall}{\includegraphics{https://static01.nyt.com/images/2018/04/02/opinion/thomas-b-edsall/thomas-b-edsall-thumbLarge-v2.png}}

By \href{https://www.nytimes.com/by/thomas-b-edsall}{Thomas B. Edsall}

Mr. Edsall contributes a weekly column from Washington, D.C. on
politics, demographics and inequality.

\begin{itemize}
\item
  Sept. 18, 2019
\item
  \begin{itemize}
  \item
  \item
  \item
  \item
  \item
  \item
  \end{itemize}
\end{itemize}

\includegraphics{https://static01.nyt.com/images/2019/09/18/opinion/18edsall1/merlin_159095160_4c805c19-92d3-473e-a57a-34fe6c3f1f4c-articleLarge.jpg?quality=75\&auto=webp\&disable=upscale}

Money is the mother's milk of politics, as the old saying goes, and the
slow motion realignment of our two major political parties has changed
who raises more money from the rich and who raises more from small
donors.

A pair of major developments give us a hint about how future trends will
develop on the partisan battleground.

First: Heading into the 2020 election, President Trump is on track to
far surpass President Barack Obama's record in collecting small donor
contributions --- those under \$200 --- lending weight to his claim of
populist legitimacy.

Second: Democratic candidates and their party committees are making
inroads in gathering contributions from the wealthiest of the wealthy,
the Forbes 400, a once solid Republican constituency. Democrats are also
pulling ahead in contributions from highly educated professionals ---
doctors, lawyers, tech executives, software engineers, architects,
scientists, teachers and so on.

These
\href{https://books.google.com/books?hl=en\&lr=\&id=_MeCr31C9S8C\&oi=fnd\&pg=PR7\&dq=Mosco+and+McKercher+knowledge+worker\&ots=FvIgNUtEdr\&sig=QOCB7sbOf2Y7PT-f4SQZIsvcrh8\#v=onepage\&q=Mosco\%20and\%20McKercher\%20knowledge\%20worker\&f=false}{knowledge
class} donors, deeply hostile to Trump, propelled the fund-raising
success of
\href{https://www.opensecrets.org/overview/index.php?display=T\&type=A\&cycle=2018}{Democratic
House candidates in 2018} --- \$1 billion to the Republicans' \$661
million.

While there are advantages for Democrats in gaining support from
previously Republican-leaning donors, this success carries costs. In
winning over the high-tech industry, the party has acquired a
constituency at odds with competing Democratic interest groups,
especially organized labor and consumer protection proponents. Picking
up rich backers also reinforces the image of a party dominated by
elites.

In their paper,
``\href{https://papers.ssrn.com/sol3/papers.cfm?abstract_id=2668780}{Increasing
Inequality in Wealth and the Political Expenditures of Billionaires},''
Adam Bonica and
\href{http://as.nyu.edu/content/nyu-as/as/faculty/howard-l-rosenthal.html}{Howard
Rosenthal}, political scientists at Stanford and N.Y.U., track the
partisan contribution patterns of the Forbes 400 from the 1981-82
election cycle through the 2011-12 cycle.

For that three-decade period, the level of giving to Republicans and
Republican Party committees by members of the Forbes 400 followed a
steady downward trajectory, falling from 68 percent to 59 percent.

\includegraphics{https://static01.nyt.com/images/2019/09/18/opinion/18edsall2/merlin_152987637_1c6cada6-29ea-4ba9-81b0-4225d1e2ff14-articleLarge.jpg?quality=75\&auto=webp\&disable=upscale}

This downward trajectory coincided with the steady transformation of the
sources of wealth for those on the Forbes list. In 1982, when the list
was \href{https://www.verdict.co.uk/forbes-100th-birthday/}{first
published}, solidly Republican manufacturers and energy producers
dominated --- 89 of the 400 richest Americans having made their fortunes
in oil.

\href{https://www.forbes.com/forbes-400/\#1dcfc44d7e2f}{By 2018}, 59 of
the Forbes 400 had made their fortunes in technology, including six of
the top ten: Jeff Bezos, No. 1; Bill Gates No. 2; Mark Zuckerberg, No.
4; Larry Ellison, No. 5; Larry Page, number 6; and Sergey Brin number 9.
Eighty-eight more made their money in the financial sector. In contrast
to the 1982 Forbes list, only 14 on the 2018 list made their money in
manufacturing and 24 in energy.

``The 400 have trended steadily to the left,'' conclude Bonica and
Rosenthal.

The two authors write that from 1989 to 2017, members of the Forbes 400
have fared much better under Democratic presidents than Republican
presidents: The 400 ``did very well under the two Democrats, Clinton and
Obama. They did not do well under either Bush.''

Bonica and Rosenthal's analysis may prove troubling for those seeking to
slow or reverse increasing wealth and income inequality. As the Forbes
400 moves toward the Democratic Party, they write, ``Inequality in
campaign contributions in the American plutocracy has grown hand in hand
with the growth in economic inequality.''

They go on to raise another basic question: Does increased support for
Democrats among the affluent and the rich undermine efforts to stem the
growth of inequality?

\begin{quote}
The historic rise in inequality in recent decades has not ushered in an
era of Republican fund-raising dominance. On the contrary, Democrats
have made substantial gains against Republicans in recent decades while
inequality was on the rise.
\end{quote}

In a
\href{https://scholars.org/contribution/rising-economic-inequality-and-campaign-contributions-very-wealthy-americans}{separate
essay}, published on the \href{https://scholars.org/}{Scholars' Strategy
Network}, which discusses the implications of their work, Bonica and
Rosenthal wrote:

\begin{quote}
The superrich control resources that parties and politicians require
and, as a result, are courted. Politicians have incentives to pay
attention to the policy concerns that animate wealthy donors on left and
right alike --- and this dynamic influences public discussion and
policymaking.
\end{quote}

The continued concentration of money at the top, they write, translates
into more political power:

\begin{quote}
The ideas, values, and preferences of wealthy donors distort the focus
of U.S. democracy more than individuals' desires to grow their already
vast fortunes. Rather than worry about individual corruption, citizens
and leaders should worry about the many ways money in politics can
amplify the voices of the privileged few over those of the majority. As
wealth concentration grows, so will uneven political influence.
\end{quote}

Bonica has turned tracking campaign contributions by wealth and
occupation into a specialty.

He provided The Times with data extending from 1980 to 2016 covering the
contribution patterns of donors who gave the largest amount of money in
each election cycle. (Roughly a quarter of Bonica's list overlaps with
the Forbes 400 list.)

In the 1979-80 presidential election cycle, 71 percent of the top 400
donors gave to Republicans and to right-of-center political action
committees, while 29 percent gave to Democrats and left-of-center PACS,
a 42-point difference. In the 2015-16 presidential cycle, 54 percent
gave to Republicans and right-leaning PACs, and 46 percent gave to
Democrats, an 8-point difference.

In the case of contribution patterns of those in different occupations,
Bonica emailed in response to my inquiry:

\begin{quote}
The medical profession has perhaps experienced the largest generational
realignment. Physicians who graduated medical school before the 1990s
tend to favor Republicans, but younger cohorts have trended sharply to
the left.
\end{quote}

Bonica's data shows that doctors who graduated in the 30 years from 1960
to 1990 consistently gave more to Republicans than to Democrats,
generally in the 54-55 percent range.

Starting with those graduating in the 1990, however, the share of
contributions going to Republicans began to decline, dropping below 50
percent for those graduating in 1996 and falling to the low 30s for the
youngest cohort.

As further evidence of this trend, Bonica cited a 2017 survey of 1,660
medical students published in the journal
\href{https://journals.lww.com/academicmedicine/Fulltext/2019/09000/Looking_to_the_Future__Medical_Students__Views_on.31.aspx?casa_token=2x2FBy3QFVsAAAAA:RSTId-aUSYb0UkBscoLt9UTcx-UjUFORxL40UIIsWE3brErjCF5ku_eq5QVzSZXXsJRtkvdFN_giZ9UiZf6boyQa\#pdf-link}{Academic
Medicine}. The survey reported that 89.1 percent said they supported
Obamacare. In the survey, 77.7 identified themselves as liberal, 12.2
percent moderate and 7.2 percent conservative.

Bonica's study of lawyers, conducted with
\href{https://scholar.harvard.edu/msen/home}{Maya Sen}, a political
scientist at Harvard, also demonstrates a strong pro-Democratic trend in
campaign contributions, although attorneys have leaned Democratic for
decades.

On an ideological scale --- with plus numbers indicating right-of-center
and minus numbers indicating left-of-center --- Bonica found that
lawyers who graduated from nonelite schools shifted from roughly evenly
split between left and right in the 1950s to minus .6, or liberal, by
2012. Lawyers from elite schools (Harvard, Stanford, Yale etc.) were
liberal leaning in the 1950s (minus .25) and became rock solid liberals
by the current decade (minus .9).

Three different scholars ---
\href{https://dbroock.people.stanford.edu/}{David E. Broockman} and
\href{https://web.stanford.edu/~neilm/}{Neil Malhotra}, professors at
Stanford's Graduate School of Business, and Gregory Ferenstein, an
independent journalist who writes about Silicon Valley --- have a
different take. Their paper,
``\href{https://onlinelibrary.wiley.com/doi/epdf/10.1111/ajps.12408}{Predispositions
and the Political Behavior of American Economic Elites: Evidence from
Technology Entrepreneurs},'' explores some of the political consequences
of the ascendance of high-tech.

Technology entrepreneurs, despite their Democratic leanings, are
ambivalent on key elements of the Democratic agenda, according to
Broockman and his co-authors. They are reliably orthodox liberals on
some issues, not so reliable on others.

On matters of globalization, trade and immigration, this Silicon Valley
constituency is firmly pro-globalization. Eighty seven percent support
free trade agreements and 56 percent are ``in favor of increasing levels
of immigration,'' which is ``15 points higher than Democratic'' rank and
file, the paper says.

On social issues, the authors found that ``technology entrepreneurs are
again very liberal,'' including near universal (96 percent) support of
same-sex marriage, 82 percent support of gun control and 67 percent
opposition to the death penalty.

Perhaps most significant and most surprising, surveys of high tech
executives conducted by Broockman and colleagues show that tech
entrepreneurs ``strongly support redistribution and taxation.'' For
example, Broockman et al. continue, ``nearly all technology
entrepreneurs support increasing taxes on those making over \$250,000 or
\$1,000,000 per year (with 76 and 83 percent expressing some support for
each, respectively).'' Seventy five percent support programs
specifically targeted toward the poor, including 59 percent in support
of increased spending for the poor. Some 82 percent indicated ``support
for universal health care even if it means raising taxes.''

While high tech executives share the views of liberal elites generally
on the issues described above, there are significant areas of conflict.

``Technology entrepreneurs do not share conventional Democratic views on
the regulation of product and labor markets,'' the authors write.
``Technology entrepreneurs are indeed more conservative even than
Republican citizens and most similar to Republican donors.''

On specific issues, almost all (82 percent) tech executives believe

\begin{quote}
that it is too difficult to fire workers and that the government should
make it easier to do so. However, majorities of Democratic donors and
citizens believe the government should make it harder to fire workers (a
50 percentage point difference from technology entrepreneurs).
\end{quote}

In the case of organized labor, three quarters (74 percent) of tech
executives ``say they would like to see labor unions' influence
decrease, versus only 18 percent of Democratic donors and 33 percent of
Democratic citizens.''

In their conclusion, the three authors address how the growing influence
of the tech industry in Democratic politics will affect the party's
approach to social spending and the reduction of inequality.

On one hand, they write, ``technology entrepreneurs seem poised to
support Democratic candidates --- and therefore redistributive policies
that should reduce inequality --- financially.''

On the other, they point out that these entrepreneurs

\begin{quote}
generally stand opposed to many government interventions in markets ---
such as government support for labor unions, worker protections and
consumer protections --- that have long been central to the Democratic
Party's ideological answer to inequality and supported by traditional
Democratic constituencies.
\end{quote}

The result, they suggest, is that

\begin{quote}
as Democratic elected officials receive increasing financial support
from technology entrepreneurs and attempt to court further support from
them'' intraparty conflicts over ``regulating product and labor markets
may take center stage.
\end{quote}

Altogether, the developments at the high-end of campaign finance are a
mixed bag for the Democratic Party, expanding the sources of political
money while simultaneously risking internal divisions.

More worrisome for the Democratic Party and its candidates is Donald
Trump's exceptional success in raising campaign money in small dollar
amounts, which suggests that his racial and anti-immigrant rhetoric
continues to motivate supporters.

Federal Election Commission data on Barack Obama's 2012 campaign and
Hillary Clinton's 2016 bid, along with an analysis of Trump's
fund-raising in the 2020 campaign by the
\href{https://www.opensecrets.org/}{Center for Responsive Politics},
shows the following.

By the end of his re-election campaign, Obama
\href{https://www.fec.gov/data/candidate/P80003338/}{raised} a total of
\$549.4 million, of which \$234.4 million, or 42.7 percent, came in
contributions of less than \$200.

By the end of her 2016 campaign, Clinton
\href{https://www.fec.gov/data/candidate/P00003392/}{raised} a total of
\$405.7 million, of which \$105.6 million, or 26.0 percent, came in low
dollar amount,

By the end of June 2019, at a much earlier stage in his campaign, Trump
had raised a total of \$124.8 million, of which \$87.3 million, or 70.0
percent, is made up of donations under \$200.

\href{https://as.tufts.edu/politicalscience/people/faculty/schaffner}{Brian
Schaffner}, a political scientist at Tufts, wrote in an email that
``Trump's appeal is more to ideologues and emotional Republican
contributors rather than to strategic and traditional Republican large
dollar donors.''

He argues that

\begin{quote}
the fact that Trump raises such a large share from small dollar donors
is due less to Trump's improvement among small donors than it is to the
difficulty he has raising money from large donors. This is really a
story about how the traditional large donors in the Republican Party
didn't want to give to Trump in 2016 and even so far in 2020 they
continue to be reluctant to contribute to him.
\end{quote}

\href{https://polsci.umass.edu/people/ray-la-raja}{Raymond J. La Raja},
a political scientist at the University of Massachusetts-Amherst, also
emailed me:

\begin{quote}
It is not too surprising that Trump has outpaced others, even Obama, in
raising money from small donors. Individual donors --- big and small ---
tend to be much more polarized compared to the rest of the electorate.
They give because of strong ideological preferences and passions. People
like Trump ignite those passions.
\end{quote}

Bonica notes that ``there is a strong association between ideological
extremity and total funds raised from small donors at the presidential
level.'' Bonica's calculations of the ideological positioning of the
candidates shows that Trump is ``the most extreme conservative,'' while
Bernie Sanders, who ``raised 58 percent of his campaign dollars from
small donors'' in 2016, stands out as the most liberal candidate,
``which might mirror what we see in Trump from the left.''

Trump's success in raising small dollar contributions is not necessarily
a harbinger of his prospects in November 2020. It does, however, raise a
question about the contemporary role of the two major political parties.

Traditionally, one of the core strengths of the Democratic Party has
been that voters trust it more than the Republican Party to protect and
advance the interests of the middle class. In recent years, however,
that advantage has been eroding.

The NBC/WSJ poll
\href{http://wsj.com/public/resources/documents/181259NBCWSJOctober2018PollFinal.pdf?mod=article_inline\&mod=article_inline}{has
repeatedly asked voters} ``which party do you think would do a better
job looking out for the middle class?''

In the 1990s, an average of eight polls showed the Democrats with a
22.25 point advantage, 43.0 to 21.75. The question was dropped only to
be resumed in December 2011. From 2011 to September 2014, the Democratic
advantage fell to 19.5 points, 44.0 to 24.5.

Since then, in six surveys conducted from June 2015 to October 2018 ---
the Trump era --- the Democratic advantage continued to erode to 13.1
points, 41.3 to 28.2. In the two most recent surveys, the Democratic
advantage fell to 10 points, 41 to 31, less than half of what it was in
the 1990s.

The Democrats may or may not regain the presidency in 2020, but they
could well lose their invaluable credential as the party of the middle
class.

\emph{The Times is committed to publishing}
\href{https://www.nytimes.com/2019/01/31/opinion/letters/letters-to-editor-new-york-times-women.html}{\emph{a
diversity of letters}} \emph{to the editor. We'd like to hear what you
think about this or any of our articles. Here are some}
\href{https://help.nytimes.com/hc/en-us/articles/115014925288-How-to-submit-a-letter-to-the-editor}{\emph{tips}}\emph{.
And here's our email:}
\href{mailto:letters@nytimes.com}{\emph{letters@nytimes.com}}\emph{.}

\emph{Follow The New York Times Opinion section on}
\href{https://www.facebook.com/nytopinion}{\emph{Facebook}}\emph{,}
\href{http://twitter.com/NYTOpinion}{\emph{Twitter (@NYTopinion)}}
\emph{and}
\href{https://www.instagram.com/nytopinion/}{\emph{Instagram}}\emph{.}

Advertisement

\protect\hyperlink{after-bottom}{Continue reading the main story}

\hypertarget{site-index}{%
\subsection{Site Index}\label{site-index}}

\hypertarget{site-information-navigation}{%
\subsection{Site Information
Navigation}\label{site-information-navigation}}

\begin{itemize}
\tightlist
\item
  \href{https://help.nytimes.com/hc/en-us/articles/115014792127-Copyright-notice}{©~2020~The
  New York Times Company}
\end{itemize}

\begin{itemize}
\tightlist
\item
  \href{https://www.nytco.com/}{NYTCo}
\item
  \href{https://help.nytimes.com/hc/en-us/articles/115015385887-Contact-Us}{Contact
  Us}
\item
  \href{https://www.nytco.com/careers/}{Work with us}
\item
  \href{https://nytmediakit.com/}{Advertise}
\item
  \href{http://www.tbrandstudio.com/}{T Brand Studio}
\item
  \href{https://www.nytimes.com/privacy/cookie-policy\#how-do-i-manage-trackers}{Your
  Ad Choices}
\item
  \href{https://www.nytimes.com/privacy}{Privacy}
\item
  \href{https://help.nytimes.com/hc/en-us/articles/115014893428-Terms-of-service}{Terms
  of Service}
\item
  \href{https://help.nytimes.com/hc/en-us/articles/115014893968-Terms-of-sale}{Terms
  of Sale}
\item
  \href{https://spiderbites.nytimes.com}{Site Map}
\item
  \href{https://help.nytimes.com/hc/en-us}{Help}
\item
  \href{https://www.nytimes.com/subscription?campaignId=37WXW}{Subscriptions}
\end{itemize}
