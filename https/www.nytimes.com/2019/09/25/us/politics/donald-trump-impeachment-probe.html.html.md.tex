Sections

SEARCH

\protect\hyperlink{site-content}{Skip to
content}\protect\hyperlink{site-index}{Skip to site index}

\href{https://www.nytimes.com/section/politics}{Politics}

\href{https://myaccount.nytimes.com/auth/login?response_type=cookie\&client_id=vi}{}

\href{https://www.nytimes.com/section/todayspaper}{Today's Paper}

\href{/section/politics}{Politics}\textbar{}Trump Pressed Ukraine's
President to Investigate Democrats as `a Favor'

\url{https://nyti.ms/2nd7ncB}

\begin{itemize}
\item
\item
\item
\item
\item
\item
\end{itemize}

Advertisement

\protect\hyperlink{after-top}{Continue reading the main story}

Supported by

\protect\hyperlink{after-sponsor}{Continue reading the main story}

\hypertarget{trump-pressed-ukraines-president-to-investigate-democrats-as-a-favor}{%
\section{Trump Pressed Ukraine's President to Investigate Democrats as
`a
Favor'}\label{trump-pressed-ukraines-president-to-investigate-democrats-as-a-favor}}

\includegraphics{https://static01.nyt.com/images/2019/09/25/multimedia/25dc-impeach-sub/merlin_161492556_f7e5a557-5e3e-403c-8f7b-8ba151b2ed58-articleLarge.jpg?quality=75\&auto=webp\&disable=upscale}

By \href{https://www.nytimes.com/by/peter-baker}{Peter Baker}

\begin{itemize}
\item
  Published Sept. 25, 2019Updated Oct. 24, 2019
\item
  \begin{itemize}
  \item
  \item
  \item
  \item
  \item
  \item
  \end{itemize}
\end{itemize}

President Trump repeatedly pressured Ukraine's leader to investigate
leading Democrats as ``a favor'' to him during a telephone call last
summer in which the two discussed the former Soviet republic's need for
more American financial aid to counter Russian aggression.

In
\href{https://www.nytimes.com/interactive/2019/09/25/us/politics/trump-ukraine-transcript.html?module=inline}{a
reconstruction of the call} released Wednesday by the White House, Mr.
Trump urged President
\href{https://www.nytimes.com/2019/09/25/world/europe/ukraine-trump-whistleblower-zelensky.html?module=inline}{Volodymyr
Zelensky} to work with Attorney General William P. Barr and Rudolph W.
Giuliani, the president's personal lawyer, on corruption investigations
connected to former Vice President Joseph R. Biden Jr. and other
Democrats.

Although there was no explicit
\href{https://www.nytimes.com/2019/10/18/us/politics/quid-pro-quo-mean.html}{quid
pro quo} in the conversation, Mr. Trump raised the matter immediately
after Mr. Zelensky spoke of his country's need for more help from the
United States. The call came only days after Mr. Trump blocked \$391
million in aid to Ukraine, a decision that perplexed national security
officials at the time and that he has given conflicting explanations for
in recent days.

The aid freeze did not come up during the call, and Mr. Zelensky was not
yet aware of it. Instead, he thanked Mr. Trump for previous American
aid, including Javelin anti-tank weapons, and suggested he would need
more as part of Ukraine's five-year-old war with Russian-backed
separatists in the country's east.

{[}\href{https://www.nytimes.com/2019/09/25/us/politics/donald-trump-impeachment-probe.html}{\emph{Our
Washington reporters answer readers' questions on the impeachment
inquiry, and what may come next.}}{]}

``I would like you to do us a favor, though,'' Mr. Trump responded,
shifting to his interest in investigating Democrats and urging that he
work with Mr. Barr and Mr. Giuliani.

``Whatever you can do, it's very important that you do it if that's
possible,'' Mr. Trump said.

The July 25 call has become a major flash point in what is rapidly
shaping up as a divisive battle between the president and House
Democrats over impeachment that will consume Washington for weeks or
months. The conflicting interpretations of the call's meaning began to
define the contours of a debate that would seek to determine whether the
president committed high crimes and misdemeanors.

In a series of public appearances on Wednesday that veered from
bristling with anger to uncharacteristically subdued, Mr. Trump insisted
that he did nothing wrong and was once again the victim of ``a total
hoax.'' Mr. Zelensky, who by an odd coincidence was in New York for a
previously scheduled meeting with Mr. Trump, backed him up by saying
during a session with reporters that he did not feel pushed by the
president.

``It's a joke,'' Mr. Trump said. ``Impeachment for that?''

But House Democrats denounced Mr. Trump for seeking foreign help to tear
down Mr. Biden, a leading rival for his job, and said the quid pro quo
was implied and clear, comparing him with a mob boss who makes veiled
hints to extort money from his victims.

``The president has tried to make lawlessness a virtue in America and
now is exporting it abroad,'' Speaker Nancy Pelosi said.

The White House released the reconstructed transcript of the call in the
morning in hopes of undercutting suspicions about the president's
actions, but it failed to convince Democrats. By the end of the day, the
administration similarly sent Congress the original complaint filed by
an unidentified intelligence official that triggered the furor that in
just a matter of days has put the future of Mr. Trump's presidency at
risk.

\includegraphics{https://static01.nyt.com/images/2019/09/25/us/politics/25dc-impeach-2/merlin_161494746_ca4293b3-b2d5-4a33-a493-bf6545756b61-articleLarge.jpg?quality=75\&auto=webp\&disable=upscale}

The complaint reportedly calls into question a range of actions by the
president beyond just the phone conversation. Democrats and at least one
Republican who reviewed it on Wednesday said it contained disturbing
allegations, and, while still classified, it will be the central issue
on Thursday morning when Joseph Maguire, the acting director of national
intelligence, testifies before Congress.

The administration dropped its resistance to providing the complaint to
lawmakers in the face of a vote planned by House Democratic leaders
condemning its handling of the matter. By backing down, Mr. Trump made
it possible for Republicans to go along with the resolution, which all
but two did later in the day.

For Mr. Trump, keeping Republicans in his corner is more important than
winning over Democrats, most of whom White House aides consider
unmovable at this point. As of Wednesday, 218 House members have
publicly advocated impeachment or at least an inquiry, reaching a
majority for the first time after more than 70 declared their support
since Monday.

Ms. Pelosi does not seem ready to test her members' resolve, though,
planning to move forward with an inquiry without a vote on the floor to
authorize it, as was done in the past two presidential impeachments. And
even if the House did ultimately impeach Mr. Trump, it would require a
two-thirds vote by the Senate to convict and remove him from office,
meaning at least 20 Republican senators would have to decide he was
guilty.

Few Republicans broke with Mr. Trump on Wednesday. Senator Mitt Romney
of Utah called the record of Mr. Trump's phone call ``deeply
troubling,'' but most others who spoke publicly said it revealed no
impeachable action.

``From a quid pro quo aspect, there's nothing there,'' said Senator
Lindsey Graham, Republican of South Carolina, a Trump ally who served as
a House prosecutor during the impeachment trial of President Bill
Clinton in 1999.

Democrats said no direct quid pro quo was necessary to conclude that the
president overstepped his bounds. But even if it was, they said Mr.
Trump's meaning was hard to miss and the timing of the request to
Ukraine coming just after he put the aid on hold was damning.

``There was only one message that that president of Ukraine got from
that call and that was: `This is what I need, I know what you need,'''
said Representative Adam B. Schiff, Democrat of California and the
chairman of the House Intelligence Committee. ``Like any mafia boss, the
president didn't need to say, `That's a nice country you have --- it
would be a shame if something happened to it.'''

Mr. Biden said that the House should ``hold Donald Trump to account for
his abuse of power,'' although he did not directly call for impeachment.

``It is a tragedy for this country that our president put personal
politics above his sacred oath,'' Mr. Biden said. ``He has put his own
political interests over our national security interest, which is
bolstering Ukraine against Russian pressure.''

For Mr. Trump, the sudden turn of events has recast the remaining year
of his term before next year's election, seemingly all but dooming
chances for bipartisan legislation. He castigated Democrats for focusing
on this ``nonsense'' instead of gun control or trade.

And he expressed surprise that impeachment was now back on the table
after the threat seemed to fade following the report on Russian election
interference by the special counsel, Robert S. Mueller III. ``I thought
we won,'' the president said. ``I thought it was dead --- it was dead.''

He blamed Ms. Pelosi, who until this week had been reluctant to pursue
impeachment, which so far does not have the support of most Americans in
polls. ``She's lost her way,'' Mr. Trump said. ``She's been taken over
by the radical left.''

Image

Former Vice President Joseph R. Biden Jr. on Tuesday in Delaware. He
said that the House should ``hold Donald Trump to account for his abuse
of power.''Credit...Mark Makela for The New York Times

He also tried to turn the tables on Democrats, arguing that Mr. Biden
was the one who was really corrupt and even citing a letter written by
three Democratic senators last year to Ukraine's prosecutor urging
cooperation with Mr. Mueller. That letter, however, was written out of
what the senators said was concern that Ukraine would be intimidated
from cooperating by Mr. Trump's wrath.

The call between Mr. Trump and Mr. Zelensky took place just a day after
Mr. Mueller testified before Congress, and the issue was clearly still
on Mr. Trump's mind. Mr. Mueller reported that he did not find
sufficient evidence to prove a criminal conspiracy between Mr. Trump's
campaign and Russia, although he identified actions by Mr. Trump that
could be construed as obstruction of justice.

Feeling that he had survived the special counsel inquiry, Mr. Trump
apparently wanted to turn the tables and prove that it was illegitimate
to begin with. In his discussion with Mr. Zelensky, he pressed the
Ukrainian leader to use Mr. Barr's help to investigate a company
involved in the beginnings of the Russia inquiry.

Mr. Trump also pressed Mr. Zelensky to open an investigation of Mr.
Biden and his younger son, Hunter Biden, who sat on the board of a
Ukrainian energy company, asserting that the former vice president
forced the dismissal of a Ukrainian prosecutor to benefit the company's
owner. Neither claim has been borne out by evidence, but both held the
potential to benefit the president politically.

Mr. Zelensky told Mr. Trump that he would have the country's new top
prosecutor examine the matters he raised.

``The next prosecutor general will be 100 percent my person, my
candidate,'' Mr. Zelensky assured the president. ``He or she will look
into the situation.''

Mr. Trump did not directly condition any aid or support on Ukraine
following through, but he did start the call noting how generous he
believed he had been. ``The United States has been very, very good to
Ukraine,'' he said. ``I wouldn't say that it's reciprocal necessarily.''

Sitting side by side with Mr. Trump in their first face-to-face meeting
on Wednesday, Mr. Zelensky told reporters that he wanted to stay out of
United States politics but provided a benign interpretation of the call.

``We had, I think, a good phone call,'' Mr. Zelensky said. ``It was
normal; we spoke about many things. So, I think, and you read it, that
nobody pushed --- pushed me.''

``In other words, no pressure,'' Mr. Trump chimed in. ``And by the
way,'' he added, addressing a reporter, ``you know there was no
pressure.''

The meeting on the sideline of the United Nations General Assembly could
hardly have come at a more charged moment in Ukrainian-American
relations. Mr. Zelensky, a former comedian with no prior political
experience,
\href{https://www.nytimes.com/2019/04/21/world/europe/Volodymyr-Zelensky-ukraine-elections.html}{was
elected this year} to take over a country torn by Russian military
intervention and desperately dependent on help from the United States
and Europe.

Even as he flattered Mr. Trump, the Ukrainian leader made a point of
saying he did not actually order the sought-after investigation.

``We have independent country and independent general security, and I
can't push anyone,'' Mr. Zelensky said in halting English, referring to
the prosecutor general. ``So I didn't call somebody or the new general
security. I didn't ask him; I didn't push him.''

Advertisement

\protect\hyperlink{after-bottom}{Continue reading the main story}

\hypertarget{site-index}{%
\subsection{Site Index}\label{site-index}}

\hypertarget{site-information-navigation}{%
\subsection{Site Information
Navigation}\label{site-information-navigation}}

\begin{itemize}
\tightlist
\item
  \href{https://help.nytimes.com/hc/en-us/articles/115014792127-Copyright-notice}{©~2020~The
  New York Times Company}
\end{itemize}

\begin{itemize}
\tightlist
\item
  \href{https://www.nytco.com/}{NYTCo}
\item
  \href{https://help.nytimes.com/hc/en-us/articles/115015385887-Contact-Us}{Contact
  Us}
\item
  \href{https://www.nytco.com/careers/}{Work with us}
\item
  \href{https://nytmediakit.com/}{Advertise}
\item
  \href{http://www.tbrandstudio.com/}{T Brand Studio}
\item
  \href{https://www.nytimes.com/privacy/cookie-policy\#how-do-i-manage-trackers}{Your
  Ad Choices}
\item
  \href{https://www.nytimes.com/privacy}{Privacy}
\item
  \href{https://help.nytimes.com/hc/en-us/articles/115014893428-Terms-of-service}{Terms
  of Service}
\item
  \href{https://help.nytimes.com/hc/en-us/articles/115014893968-Terms-of-sale}{Terms
  of Sale}
\item
  \href{https://spiderbites.nytimes.com}{Site Map}
\item
  \href{https://help.nytimes.com/hc/en-us}{Help}
\item
  \href{https://www.nytimes.com/subscription?campaignId=37WXW}{Subscriptions}
\end{itemize}
