Sections

SEARCH

\protect\hyperlink{site-content}{Skip to
content}\protect\hyperlink{site-index}{Skip to site index}

\href{https://www.nytimes.com/section/world/europe}{Europe}

\href{https://myaccount.nytimes.com/auth/login?response_type=cookie\&client_id=vi}{}

\href{https://www.nytimes.com/section/todayspaper}{Today's Paper}

\href{/section/world/europe}{Europe}\textbar{}Top Secret Russian Unit
Seeks to Destabilize Europe, Security Officials Say

\url{https://nyti.ms/2LYVP6R}

\begin{itemize}
\item
\item
\item
\item
\item
\item
\end{itemize}

Advertisement

\protect\hyperlink{after-top}{Continue reading the main story}

Supported by

\protect\hyperlink{after-sponsor}{Continue reading the main story}

\hypertarget{top-secret-russian-unit-seeks-to-destabilize-europe-security-officials-say}{%
\section{Top Secret Russian Unit Seeks to Destabilize Europe, Security
Officials
Say}\label{top-secret-russian-unit-seeks-to-destabilize-europe-security-officials-say}}

\includegraphics{https://static01.nyt.com/images/2019/10/07/world/xxunit1/merlin_162115362_3dc8d702-97ea-48b2-a17a-fbbb978c4477-articleLarge.jpg?quality=75\&auto=webp\&disable=upscale}

By \href{https://www.nytimes.com/by/michael-schwirtz}{Michael Schwirtz}

\begin{itemize}
\item
  Oct. 8, 2019
\item
  \begin{itemize}
  \item
  \item
  \item
  \item
  \item
  \item
  \end{itemize}
\end{itemize}

First came a destabilization campaign in Moldova, followed by the
poisoning of an arms dealer in Bulgaria and then a thwarted coup in
Montenegro. Last year, there was an
\href{https://www.nytimes.com/2018/09/09/world/europe/sergei-skripal-russian-spy-poisoning.html}{attempt
to assassinate a former Russian spy} in Britain using a nerve agent.
Though the operations bore the fingerprints of Russia's intelligence
services, the authorities initially saw them as isolated, unconnected
attacks.

Western security officials have now concluded that these operations, and
potentially many others, are part of a coordinated and ongoing campaign
to destabilize Europe, executed by an elite unit inside the Russian
intelligence system skilled in subversion, sabotage and assassination.

The group, known as Unit 29155, has operated for at least a decade, yet
Western officials only recently discovered it. Intelligence officials in
four Western countries say it is unclear how often the unit is mobilized
and warn that it is impossible to know when and where its operatives
will strike.

The purpose of Unit 29155, which has not been previously reported,
underscores the degree to which the Russian president, Vladimir V.
Putin, is
\href{https://www.nytimes.com/series/russias-dark-arts}{actively
fighting the West} with his brand of so-called hybrid warfare --- a
blend of propaganda, hacking attacks and disinformation --- as well as
open military confrontation.

``I think we had forgotten how organically ruthless the Russians could
be,'' said Peter Zwack, a retired military intelligence officer and
former defense attaché at the United States Embassy in Moscow, who said
he was not aware of the unit's existence.

In a text message, Dmitri S. Peskov, Mr. Putin's spokesman, directed
questions about the unit to the Russian Defense Ministry. The ministry
did not respond to requests for comment.

\includegraphics{https://static01.nyt.com/images/2019/10/08/world/08Unite2-sub/08Unite2-sub-articleLarge.jpg?quality=75\&auto=webp\&disable=upscale}

Hidden behind concrete walls at the headquarters of the 161st Special
Purpose Specialist Training Center in eastern Moscow, the unit sits
within the command hierarchy of the Russian military intelligence
agency,
\href{https://www.nytimes.com/2018/07/13/world/europe/what-is-russian-gru.html}{widely
known as the G.R.U.}

Though much about G.R.U. operations remains a mystery, Western
intelligence agencies have begun to get a clearer picture of its
underlying architecture. In the months before the 2016 presidential
election, American officials say two G.R.U. cyber units, known as 26165
and 74455, hacked into the servers of the Democratic National Committee
and the Clinton campaign, and then published embarrassing internal
communications.

\emph{{[}Our correspondent Matt Apuzzo reported on Russia's blueprint
for foreign disruption on}
\href{https://www.nytimes.com/2019/09/06/the-weekly/russia-estonia-election-cyber-attack.html}{\emph{``The
Weekly,'' The Times's TV show}}\emph{. Watch on FX and Hulu.{]}}

Last year, Robert S. Mueller III, the special counsel overseeing the
inquiry into Russian interference in the 2016 elections,
\href{https://www.nytimes.com/2018/07/13/us/politics/mueller-indictment-russian-intelligence-hacking.html}{indicted
more than a dozen officers} from those units, though all still remain at
large. The hacking teams mostly operate from Moscow, thousands of miles
from their targets.

By contrast, officers from Unit 29155 travel to and from European
countries. Some are decorated veterans of Russia's bloodiest wars,
including in Afghanistan, Chechnya and Ukraine. Its operations are so
secret, according to assessments by Western intelligence services, that
the unit's existence is most likely unknown even to other G.R.U.
operatives.

The unit appears to be a tight-knit community. A photograph taken in
2017 shows the unit's commander, Maj. Gen. Andrei V. Averyanov, at his
daughter's wedding in a gray suit and bow tie. He is posing with Col.
Anatoly V. Chepiga, one of two officers indicted in Britain over the
poisoning of a former spy, Sergei V. Skripal.

\includegraphics{https://static01.nyt.com/images/2018/09/07/world/europe/06Novichok5/06Novichok5-videoSixteenByNineJumbo1600-v2.jpg}

``This is a unit of the G.R.U. that has been active over the years
across Europe,'' said one European security official, who spoke on
condition of anonymity to describe classified intelligence matters.
``It's been a surprise that the Russians, the G.R.U., this unit, have
felt free to go ahead and carry out this extreme malign activity in
friendly countries. That's been a shock.''

To varying degrees, each of the four operations linked to the unit
attracted public attention, even as it took time for the authorities to
confirm that they were connected. Western intelligence agencies first
identified the unit after
\href{https://www.nytimes.com/2016/11/26/world/europe/finger-pointed-at-russians-in-alleged-coup-plot-in-montenegro.html?module=inline}{the
failed 2016 coup in Montenegro}, which involved a plot by two unit
officers to kill the country's prime minister and seize the Parliament
building.

But officials began to grasp the unit's specific agenda of disruption
only after the March 2018 poisoning of Mr. Skripal, a former G.R.U.
officer who had betrayed Russia by spying for the British. Mr. Skripal
and his daughter, Yulia, fell grievously ill after exposure to
\href{https://www.nytimes.com/2018/03/13/world/europe/uk-russia-spy-poisoning.html?module=inline}{a
highly toxic nerve agent}, but survived.

(Three other people were sickened, including a police officer and a man
who
\href{https://www.nytimes.com/2018/07/24/world/europe/russia-uk-poison-charlie-rowley.html}{found
a small bottle} that British officials believe was used to carry the
nerve agent and gave it to his girlfriend. The girlfriend, Dawn
Sturgess,
\href{https://www.nytimes.com/2018/07/08/world/europe/uk-dawn-sturgess-novichok-salisbury.html}{died
after spraying the nerve} agent on her skin, mistaking the bottle for
perfume.)

The poisoning led to a geopolitical standoff, with more than 20 nations,
including the United States, expelling 150 Russian diplomats in a show
of solidarity with Britain.

Ultimately, the British authorities exposed two suspects, who had
traveled under aliases but were later identified by the investigative
site Bellingcat as Colonel Chepiga and Alexander Mishkin. Six months
after the poisoning, British prosecutors
\href{https://www.nytimes.com/2018/09/05/world/europe/salisbury-novichok-poisoning.html}{charged
both men} with transporting the nerve agent to Mr. Skripal's home in
Salisbury, England, and smearing it on his front door.

Image

The Bulgarian arms dealer Emilian Gebrev survived a poisoning attack in
2015.Credit...Nikolay Doychinov/Agence France-Presse --- Getty Images

But the operation was more complex than officials revealed at the time.

Exactly a year before the poisoning, three Unit 29155 operatives
traveled to Britain, possibly for a practice run, two European officials
said. One was Mr. Mishkin. A second man used the alias Sergei Pavlov.
Intelligence officials believe the third operative, who used the alias
Sergei Fedotov, oversaw the mission.

Soon, officials established that two of these officers --- the men using
the names Fedotov and Pavlov --- had been part of a team that
\href{https://www.nytimes.com/2019/02/11/world/europe/bulgaria-poisoning-russia-skripal.html}{attempted
to poison the Bulgarian arms dealer Emilian Gebrev} in 2015. (The other
operatives, also known only by their aliases, according to European
intelligence officials, were Ivan Lebedev, Nikolai Kononikhin, Alexey
Nikitin and Danil Stepanov.)

The team would twice try to kill Mr. Gebrev, once in Sofia, the capital,
and again a month later at his home on the Black Sea.

Speaking to reporters in February at the Munich Security Conference,
Alex Younger, the chief of MI6, Britain's foreign intelligence service,
spoke out against the growing Russian threat and hinted at coordination,
without mentioning a specific unit.

``You can see there is a concerted program of activity --- and, yes, it
does often involve the same people,'' Mr. Younger said, pointing
specifically to the Skripal poisoning and the Montenegro coup attempt.
He added: ``We assess there is a standing threat from the G.R.U. and the
other Russian intelligence services and that very little is off
limits.''

The Kremlin sees Russia as being at war with a Western liberal order
that it views as an existential threat.

Image

Western intelligence agencies first identified the unit after the failed
2016 coup in Montenegro, which involved a plot to kill Prime Minister
Milo Djukanovic, center.Credit...Savo Prelevic/Agence France-Presse ---
Getty Images

At \href{https://www.vesti.ru/doc.html?id=3079058}{a ceremony in
November} for the G.R.U.'s centenary, Mr. Putin stood beneath a glowing
backdrop of the agency's logo --- a red carnation and an exploding
grenade --- and described it as ``legendary.'' A former intelligence
officer himself, Mr. Putin drew a direct line between the Red Army spies
who helped defeat the Nazis in World War II and officers of the G.R.U.,
whose ``unique capabilities'' are now deployed against a different kind
of enemy.

``Unfortunately, the potential for conflict is on the rise in the
world,'' Mr. Putin said during the ceremony. ``Provocations and outright
lies are being used and attempts are being made to disrupt strategic
parity.''

In 2006, Mr. Putin signed a law legalizing targeted killings abroad, the
same year a team of Russian assassins used a radioactive isotope
\href{https://www.nytimes.com/2006/12/03/world/europe/03russian.html}{to
murder Aleksander V. Litvinenko}, another former Russian spy, in London.

Unit 29155 is not the only group authorized to carry out such
operations, officials said. The British authorities have
\href{https://www.nytimes.com/2016/01/22/world/europe/alexander-litvinenko-poisoning-inquiry-britain.html}{attributed
Mr. Litvinenko's killing to the Federal Security Service}, the
intelligence agency once headed by Mr. Putin that often competes with
the G.R.U.

Although little is known about Unit 29155 itself, there are clues in
public Russian records that suggest links to the Kremlin's broader
hybrid strategy.

A 2012 directive from the Russian Defense Ministry assigned bonuses to
three units for ``special achievements in military service.'' One was
Unit 29155. Another was Unit 74455, which was involved in the 2016
election interference. The third was Unit 99450, whose officers are
believed to have been involved in the annexation of the Crimean
Peninsula in 2014.

Image

An election sign in Chisinau, Moldova, in 2014 showed representatives of
the anti-European Socialist Party with President Vladimir V. Putin of
Russia.Credit...Gleb Garanich/Reuters

A retired G.R.U. officer with knowledge of Unit 29155 said that it
specialized in preparing for ``diversionary'' missions, ``in groups or
individually --- bombings, murders, anything.''

``They were serious guys who served there,'' the retired officer said.
``They were officers who worked undercover and as international
agents.''

Photographs of the unit's dilapidated former headquarters, which has
since been abandoned, show myriad gun racks with labels for an
assortment of weapons, including Belgian FN-30 sniper rifles, German
G3A3s, Austrian Steyr AUGs and American M16s. There was also a form
outlining a training regimen, including exercises for hand-to-hand
combat. The retired G.R.U. officer confirmed the authenticity of the
photographs, which were
\href{https://podpolkovnikvvs.livejournal.com/222086.html}{published by
a Russian blogger}.

The current commander, General Averyanov, graduated in 1988 from the
Tashkent Military Academy in what was then the Soviet Republic of
Uzbekistan. It is likely that he would have fought in both the first and
second Chechen wars, and he was awarded a Hero of Russia medal, the
country's highest honor, in January 2015. The two officers charged with
the Skripal poisoning also received the same award.

Though an elite force, the unit appears to operate on a shoestring
budget. According to Russian records, General Averyanov lives in a
run-down Soviet-era building a few blocks from the unit's headquarters
and drives a 1996 VAZ 21053, a rattletrap Russia-made sedan. Operatives
often share cheap accommodation to economize while on the road. British
investigators say the suspects in the Skripal poisoning stayed in a
low-cost hotel in Bow, a downtrodden neighborhood in East London.

But European security officials are also perplexed by the apparent
sloppiness in the unit's operations. Mr. Skripal survived the
assassination attempt, as did Mr. Gebrev, the Bulgarian arms dealer. The
attempted coup in Montenegro drew an enormous amount of attention, but
ultimately failed. A year later, Montenegro joined NATO. It is possible,
security officials say, that they have yet to discover other, more
successful operations.

It is difficult to know if the messiness has bothered the Kremlin.
Perhaps, intelligence experts say, it is part of the point.

``That kind of intelligence operation has become part of the
psychological warfare,'' said Eerik-Niiles Kross, a former intelligence
chief in Estonia. ``It's not that they have become that much more
aggressive. They want to be felt. It's part of the game.''

Advertisement

\protect\hyperlink{after-bottom}{Continue reading the main story}

\hypertarget{site-index}{%
\subsection{Site Index}\label{site-index}}

\hypertarget{site-information-navigation}{%
\subsection{Site Information
Navigation}\label{site-information-navigation}}

\begin{itemize}
\tightlist
\item
  \href{https://help.nytimes.com/hc/en-us/articles/115014792127-Copyright-notice}{©~2020~The
  New York Times Company}
\end{itemize}

\begin{itemize}
\tightlist
\item
  \href{https://www.nytco.com/}{NYTCo}
\item
  \href{https://help.nytimes.com/hc/en-us/articles/115015385887-Contact-Us}{Contact
  Us}
\item
  \href{https://www.nytco.com/careers/}{Work with us}
\item
  \href{https://nytmediakit.com/}{Advertise}
\item
  \href{http://www.tbrandstudio.com/}{T Brand Studio}
\item
  \href{https://www.nytimes.com/privacy/cookie-policy\#how-do-i-manage-trackers}{Your
  Ad Choices}
\item
  \href{https://www.nytimes.com/privacy}{Privacy}
\item
  \href{https://help.nytimes.com/hc/en-us/articles/115014893428-Terms-of-service}{Terms
  of Service}
\item
  \href{https://help.nytimes.com/hc/en-us/articles/115014893968-Terms-of-sale}{Terms
  of Sale}
\item
  \href{https://spiderbites.nytimes.com}{Site Map}
\item
  \href{https://help.nytimes.com/hc/en-us}{Help}
\item
  \href{https://www.nytimes.com/subscription?campaignId=37WXW}{Subscriptions}
\end{itemize}
