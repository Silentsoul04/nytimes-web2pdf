Sections

SEARCH

\protect\hyperlink{site-content}{Skip to
content}\protect\hyperlink{site-index}{Skip to site index}

\href{https://www.nytimes.com/section/style}{Style}

\href{https://myaccount.nytimes.com/auth/login?response_type=cookie\&client_id=vi}{}

\href{https://www.nytimes.com/section/todayspaper}{Today's Paper}

\href{/section/style}{Style}\textbar{}Orville Peck, a Masked Gay Country
Star, Rides Into Brooklyn

\href{https://nyti.ms/328lM9z}{https://nyti.ms/328lM9z}

\begin{itemize}
\item
\item
\item
\item
\item
\end{itemize}

Advertisement

\protect\hyperlink{after-top}{Continue reading the main story}

Supported by

\protect\hyperlink{after-sponsor}{Continue reading the main story}

Encounters

\hypertarget{orville-peck-a-masked-gay-country-star-rides-into-brooklyn}{%
\section{Orville Peck, a Masked Gay Country Star, Rides Into
Brooklyn}\label{orville-peck-a-masked-gay-country-star-rides-into-brooklyn}}

``The masks exist as a point of discussion,'' Mr. Peck said.

\includegraphics{https://static01.nyt.com/images/2019/10/13/fashion/11UPCLOSE-PECK1/11UPCLOSE-PECK1-articleLarge-v2.jpg?quality=75\&auto=webp\&disable=upscale}

\href{https://www.nytimes.com/by/alex-hawgood}{\includegraphics{https://static01.nyt.com/images/2019/02/20/multimedia/author-alex-hawgood/author-alex-hawgood-thumbLarge.png}}

By \href{https://www.nytimes.com/by/alex-hawgood}{Alex Hawgood}

\begin{itemize}
\item
  Published Oct. 12, 2019Updated Oct. 16, 2019
\item
  \begin{itemize}
  \item
  \item
  \item
  \item
  \item
  \end{itemize}
\end{itemize}

The enigmatic country singer known as Orville Peck is riding high these
days. He is on the cover of
\href{https://www.gq-magazine.co.uk/culture/article/orville-peck-interview}{British
GQ Style}, wearing star-spangled boots and a black mask that obscures
his face.

The New Yorker, in a review of his debut album, ``Pony,'' called him the
``\href{https://www.newyorker.com/recommends/listen/orville-peck-the-masked-man-our-yee-haw-moment-deserves}{masked
man our yee-haw moment deserves}'' and described his voice as having
``the sexy, menacing melodrama of Roy Orbison.''

The
\href{https://www.latimes.com/entertainment-arts/music/story/2019-09-04/orville-peck-country-gay-mask-singer}{Los
Angeles Times} declared that he is ``as venerated as any pop diva'' by
the ``hipsters in West Hollywood,'' which seems to be a thinly veiled
reference to his large gay fan base.

\includegraphics{https://static01.nyt.com/images/2019/10/13/fashion/11UPCLOSE-PECK2/11UPCLOSE-PECK2-articleLarge-v3.jpg?quality=75\&auto=webp\&disable=upscale}

On a sweltering afternoon in July, Mr. Peck made his way to the
\href{http://kensingtonstables.com}{Kensington Stables} in Brooklyn, to
explore the wild West of Prospect Park by horseback. ``Honestly, there
is nothing I love more than being around a horse,'' he said.

Giving off what he called a ``cowboy-in-the-city kind of vibe,'' he wore
a sleeveless Western vest that showed off his toned and tattooed arms,
and a pair of dusty Wrangler jeans. And although his character always
wears a mask, he arrived without one, revealing his piercing blue eyes
and chiseled features. But he quickly put one on (a menacing black
leather mask with fringe) when a photographer arrived.

``The masks exist as a point of discussion for people to add their own
take on them,'' he said.

A young instructor named Alexus Lawson walked him to the barn and had
him wait by a tower of saddles. ``Nice quality leather,'' Mr. Peck said.

Image

Credit...Devin Yalkin for The New York Times

As it turned out, Mr. Peck was not the only show pony there. He was
introduced to Tonka, a 20-year-old black horse who has appeared on
``Today'' and ``Saturday Night Live.'' He felt Tonka's muscular body.
``He's so beautiful,'' Mr. Peck said. ``Look at those eyelashes.''

After saddling up, he followed Ms. Lawson and rode toward Ocean Parkway,
looking like a campy butch Lone Ranger on the traffic-clogged streets of
central Brooklyn. Despite his disguises, Mr. Peck has a cerebral,
salt-of-the-earth nature.

``I don't avoid questions about my life because I am trying to be
obtuse,'' he said. ``I use it as something to enhance the artistry of
what I do.''

It took some cajoling before he offered any personal tidbits, referring
to a childhood spent in ``Africa, North America and Europe,'' with a
father who is a glam-rock sound engineer and a mother who is an artist.
He identifies as gay and gave his age as 31.

Image

Credit...Devin Yalkin for The New York Times

The rest of his story is more ambiguous. He hinted that he was a child
model, as well as a theater actor in London. He created the masked
character Orville Peck two years ago, and
\href{https://www.instagram.com/p/BbSeU8QD57B/}{posts} on social media
showcasing his rich baritone and noir, spaghetti-western look were
noticed by music executives.

``Pony'' \href{https://orvillepeck.bandcamp.com/album/pony}{was
released} on Sub Pop records in March, just as the conservative
traditions of country music were being challenged by artists like Kacey
Musgraves and Lil Nas X. Mr. Peck's torch songs made him an outlaw in a
genre not known to embrace gay urban cowboys.

The four music videos from ``Pony'' drip with ``Brokeback Mountain''
homoeroticism, including the one for
``\href{https://www.youtube.com/watch?v=60MHmrtEuRY}{Hope to Die},'' in
which Mr. Peck, clad in fringed denim chaps, echoes the line-dance steps
from Madonna's
\href{https://www.youtube.com/watch?v=gLFWRDsx5AI}{cowgirl phase.}

``There has always been people of color making country music, and there
has always been queer cowboys and cowgirls, like
\href{https://www.nytimes.com/2018/03/22/style/kd-lang-ingenue-tour.html}{K.D.
Lang},'' he said.

His fans, like K.D. Lang's, are far-flung. Mr. Peck is on an
international tour that stops at such diverse places as Austin, Tex;
Dublin; Kansas City, Mo.; London; Paris; Toronto; and Zurich. (His Oct.
15 show at the Music Hall of Williamsburg in Brooklyn is sold out.)

Rising stardom, however, has its downsides, especially for a masked
singer. Online sleuths have recently identified him as
\href{http://www.brooklynvegan.com/orville-peck-is-daniel-pitout-of-nu-sensae-eating-out-lp-tour-coming-soon/}{Daniel
Pitout}, a performance artist and former drummer in the Canadian punk
band Nü Sensae.

(After confirming Mr. Peck's identity, a publicist for his label
released a statement from Mr. Peck that read, in part: ``I understand
there is a temptation to try and unmask what I do, but to do so would be
to miss the point entirely.'' He added: ``All I ask is that people
respect my work and, more importantly, my fans enough to maintain this
crucial part of my expression as an artist.'')

Image

Credit...Devin Yalkin for The New York Times

After crossing the traffic circle, Mr. Peck made his way into Prospect
Park. The click-clack of Tonka's hooves on the asphalt alerted
bystanders to the masked horseman in their midst.

A group of schoolchildren shrieked with excitement. An old man sitting
on a bench looked up from his newspaper. The tassels guarding Mr. Peck's
jaw line began to sway in sync with Tonka's tail. ``They both work as a
fly swatter,'' he said.

An urge overcame him and he began to sing
``\href{https://www.nytimes.com/2019/05/10/arts/music/old-town-road-lil-nas-x.html}{Old
Town Road}'' (``I got the horses in the back''), in a deep range that
suggested a subversive Chris Isaak.

The impromptu performance, however, was cut short because he had to call
his label publicist. ``Hello, I'm on a horse right now,'' he said,
pantomiming a phone with his thumb and pinkie finger, before giving
Tonka a gentle kick to keep moving. ``I guess no one would think that
would be too strange coming from me,'' he added, riding off into the
afternoon sun.

Advertisement

\protect\hyperlink{after-bottom}{Continue reading the main story}

\hypertarget{site-index}{%
\subsection{Site Index}\label{site-index}}

\hypertarget{site-information-navigation}{%
\subsection{Site Information
Navigation}\label{site-information-navigation}}

\begin{itemize}
\tightlist
\item
  \href{https://help.nytimes.com/hc/en-us/articles/115014792127-Copyright-notice}{©~2020~The
  New York Times Company}
\end{itemize}

\begin{itemize}
\tightlist
\item
  \href{https://www.nytco.com/}{NYTCo}
\item
  \href{https://help.nytimes.com/hc/en-us/articles/115015385887-Contact-Us}{Contact
  Us}
\item
  \href{https://www.nytco.com/careers/}{Work with us}
\item
  \href{https://nytmediakit.com/}{Advertise}
\item
  \href{http://www.tbrandstudio.com/}{T Brand Studio}
\item
  \href{https://www.nytimes.com/privacy/cookie-policy\#how-do-i-manage-trackers}{Your
  Ad Choices}
\item
  \href{https://www.nytimes.com/privacy}{Privacy}
\item
  \href{https://help.nytimes.com/hc/en-us/articles/115014893428-Terms-of-service}{Terms
  of Service}
\item
  \href{https://help.nytimes.com/hc/en-us/articles/115014893968-Terms-of-sale}{Terms
  of Sale}
\item
  \href{https://spiderbites.nytimes.com}{Site Map}
\item
  \href{https://help.nytimes.com/hc/en-us}{Help}
\item
  \href{https://www.nytimes.com/subscription?campaignId=37WXW}{Subscriptions}
\end{itemize}
