Sections

SEARCH

\protect\hyperlink{site-content}{Skip to
content}\protect\hyperlink{site-index}{Skip to site index}

\href{https://www.nytimes.com/section/politics}{Politics}

\href{https://myaccount.nytimes.com/auth/login?response_type=cookie\&client_id=vi}{}

\href{https://www.nytimes.com/section/todayspaper}{Today's Paper}

\href{/section/politics}{Politics}\textbar{}U.S. Indicts Turkish Bank on
Charges of Evading Iran Sanctions

\url{https://nyti.ms/32kU7lU}

\begin{itemize}
\item
\item
\item
\item
\item
\item
\end{itemize}

Advertisement

\protect\hyperlink{after-top}{Continue reading the main story}

Supported by

\protect\hyperlink{after-sponsor}{Continue reading the main story}

\hypertarget{us-indicts-turkish-bank-on-charges-of-evading-iran-sanctions}{%
\section{U.S. Indicts Turkish Bank on Charges of Evading Iran
Sanctions}\label{us-indicts-turkish-bank-on-charges-of-evading-iran-sanctions}}

The charges against Halkbank were announced as President Trump, having
allowed Turkish-backed forces to invade Syria, is looking to take a
tougher stand against Turkey.

\includegraphics{https://static01.nyt.com/images/2019/10/15/us/politics/15dc-turkeybank/15dc-turkeybank-articleLarge-v2.jpg?quality=75\&auto=webp\&disable=upscale}

\href{https://www.nytimes.com/by/eric-lipton}{\includegraphics{https://static01.nyt.com/images/2018/12/06/multimedia/author-eric-lipton/author-eric-lipton-thumbLarge.png}}

By \href{https://www.nytimes.com/by/eric-lipton}{Eric Lipton}

\begin{itemize}
\item
  Oct. 15, 2019
\item
  \begin{itemize}
  \item
  \item
  \item
  \item
  \item
  \item
  \end{itemize}
\end{itemize}

WASHINGTON --- The Justice Department on Tuesday sharply escalated
economic pressure on Turkey by filing fraud and money-laundering charges
against the country's second-largest state-owned bank, accusing it of
helping Iran evade United States sanctions.

The charges against the institution, Halkbank, came as the
administration sought ways to project that it was taking a tough line
with Turkey after President Trump
\href{https://www.nytimes.com/2019/10/07/us/politics/trump-turkey-syria.html}{effectively
signaled this month} that the United States would not stand in the way
of Turkey's desire to send forces into northern Syria.

Mr. Trump's willingness to allow the military action
\href{https://www.nytimes.com/2019/10/11/us/politics/turkey-sanctions-syria-kurds-trump.html}{has
thrown} the region into chaos and ignited an intense bipartisan backlash
against him at home. As the criticism has mounted, the White House has
emphasized the steps it is taking to restrain Turkey's offensive,
including a round of
\href{https://www.nytimes.com/2019/10/14/us/politics/trump-turkey-tariffs.html}{sanctions
announced on Monday.}

President Recep Tayyip Erdogan of Turkey had repeatedly raised the
Halkbank case with Mr. Trump over the past year, urging the United
States not to take further action, saying that to do so would unfairly
expose Turkey to severe financial risks. One of the bank's top
executives was convicted on related charges last year, and the Justice
Department has been reviewing since then whether to pursue the case
further as Turkish officials and lawyers pressed the government not to
indict the bank.

\href{https://www.documentcloud.org/documents/6475078-2019-10-15-U-S-v-Halkbank-Indictment.html}{The
charges} appeared to catch at least some advisers to Turkey's government
off guard. They were filed by prosecutors in the Southern District of
New York, which has been investigating the bank's role in what has been
called the largest Iran sanctions violation in United States history, as
billions of dollars' worth of gold and cash were illegally transferred
to Iran in exchange for oil and gas.

Justice Department officials said high-ranking government officials in
Turkey ``participated in and protected this scheme,'' with some
receiving bribes worth tens of millions of dollars and helping to hide
the conspiracy from the scrutiny of regulators in the United States.

``This is one of the most serious Iran sanctions violations we have
seen, and no business should profit from evading our laws or risking our
national security,'' said John C. Demers, the assistant attorney general
for national security.

Lawyers and lobbyists representing the bank, including Brian D. Ballard,
a friend of Mr. Trump's and the vice chairman of his inauguration, have
been trying for more than a year to persuade the Trump administration
not to file charges against the bank, or at least to understand that
doing so could threaten the economy of a NATO ally.

Turkish officials had directly made other appeals to Secretary of State
Mike Pompeo and Treasury Secretary Steve Mnuchin. The lobbying campaign
led some sanctions experts in Washington to question if the case might
have been delayed or dropped.

After Mr. Trump came under intense criticism for choosing to stand aside
as Turkey pursued its plan to assert control over a section of northern
Syria, he began striking a tougher tone toward Mr. Erdogan, focusing in
particular on the threat of harming Turkey's economy if it put United
States military personnel at risk or engaged in atrocities against Kurds
in the region.

``I am fully prepared to swiftly destroy Turkey's economy if Turkish
leaders continue down this dangerous and destructive path,'' Mr. Trump
said in a
\href{https://www.whitehouse.gov/briefings-statements/statement-president-donald-j-trump-regarding-turkeys-actions-northeast-syria/}{statement
Monday}, shortly before signing an executive order to impose the first
set of sanctions.

Representatives for the Turkish government --- who in interviews early
Tuesday did not give any hint that they knew the criminal charges were
imminent --- said late in the day that they suspected a link between the
new prosecution of the bank and the invasion of Syria.

``The timing is beyond any reasonable coincidence,'' said one individual
who has been working with the bank, but spoke on the condition of
anonymity to discuss the matter.

The Justice Department and the White House did not respond to questions
about whether the decision was influenced by Turkey's decision to send
troops in Syria.

Mr. Ballard, along with Robert Wexler, a former House Democrat from
Florida, and James P. Rubin, a State Department official during the
Clinton administration, had each been working at times over the last two
years to lobby on the matter, Justice Department filings show. They had
reached out in 2018 to the office of Vice President Mike Pence and the
State Department, among others.

Rudolph W. Giuliani, the former New York mayor and adviser to Mr. Trump,
also was
\href{https://www.nytimes.com/2019/10/10/us/politics/giuliani-trump-rex-tillerson.html}{involved
in the matter} in 2016 and 2017, trying to secure the release of one
suspect in the case, in a possible prisoner swap with a pastor whom
Turkey was holding on espionage charges that the United States claimed
were fabricated.

Andrew Hruska, a former federal prosecutor in New York now with
\href{https://www.kslaw.com/people/andrew-hruska}{the law firm King and
Spalding}, had also been working on the matter, communicating directly
with the Justice and Treasury Departments, on behalf of the bank.

Mr. Erdogan
\href{http://www.hurriyetdailynews.com/erdogan-says-trump-promised-to-instruct-us-ministers-on-halkbank-case-138545}{brought
the case up} with President Trump in November 2018, and his son-in-law,
Berat Albayrak, the country's finance minister,
\href{https://www.reuters.com/article/us-turkey-usa-halkbank-minister/turkish-finance-minister-upbeat-over-halkbank-sanctions-busting-case-idUSKCN1NA1FF}{following
up} a few days later with Mr. Mnuchin, pushing him to closely follow the
case.

Lawyers for the bank did not dispute that money was illegally moved
through Halkbank to Iran starting around 2012 and continuing through
2016.

But they argued that the moves were largely orchestrated by an
Iranian-Turkish gold trader, named Reza Zarrab, who had hired Mr.
Giuliani to try to secure his release.

Turkish officials argued that Mr. Zarrab, who then decided to plead
guilty to charges and become a witness for the prosecution, had lied to
American prosecutors. The Turkish officials said Mr. Zarrab accused the
bank and government officials in Turkey of conspiring in the effort as
part of an attempt to reduce any time he would spend in prison, after he
was arrested by American authorities in 2016.

In January 2018, in part because of Mr. Zarrab's testimony, a Halkbank
executive named Mehmet Hakan Atilla was convicted of violating sanctions
as part of the case. At his
\href{https://www.justice.gov/usao-sdny/pr/turkish-banker-mehmet-hakan-atilla-sentenced-32-months-conspiring-violate-us-sanctions}{sentencing
in May 2018}, a federal judge said that while Mr. Atilla had
``unquestionably furthered'' the scheme, he was ``somewhat of a cog in
the wheel'' and not ``a mastermind.''

These assertions reflected claims made by federal prosecutors that the
wrongdoing had reached high into the Turkish government.

But until Tuesday, there had been no public follow-up by the Justice
Department, nor any action by the Treasury Department, which separately
has the power to impose sanctions on the bank or impose a fine.

The bank was formally charged on Tuesday with conspiracy to defraud the
United States, conspiracy to violate sanctions, bank fraud, conspiracy
to commit bank fraud, money laundering and conspiracy to commit money
laundering.

Representatives for the bank said that they feared the charges alone
might lead other global banks to limit doing business with Halkbank, and
if a multibillion-dollar penalty results, it could threaten the overall
viability of the institution.

Advertisement

\protect\hyperlink{after-bottom}{Continue reading the main story}

\hypertarget{site-index}{%
\subsection{Site Index}\label{site-index}}

\hypertarget{site-information-navigation}{%
\subsection{Site Information
Navigation}\label{site-information-navigation}}

\begin{itemize}
\tightlist
\item
  \href{https://help.nytimes.com/hc/en-us/articles/115014792127-Copyright-notice}{©~2020~The
  New York Times Company}
\end{itemize}

\begin{itemize}
\tightlist
\item
  \href{https://www.nytco.com/}{NYTCo}
\item
  \href{https://help.nytimes.com/hc/en-us/articles/115015385887-Contact-Us}{Contact
  Us}
\item
  \href{https://www.nytco.com/careers/}{Work with us}
\item
  \href{https://nytmediakit.com/}{Advertise}
\item
  \href{http://www.tbrandstudio.com/}{T Brand Studio}
\item
  \href{https://www.nytimes.com/privacy/cookie-policy\#how-do-i-manage-trackers}{Your
  Ad Choices}
\item
  \href{https://www.nytimes.com/privacy}{Privacy}
\item
  \href{https://help.nytimes.com/hc/en-us/articles/115014893428-Terms-of-service}{Terms
  of Service}
\item
  \href{https://help.nytimes.com/hc/en-us/articles/115014893968-Terms-of-sale}{Terms
  of Sale}
\item
  \href{https://spiderbites.nytimes.com}{Site Map}
\item
  \href{https://help.nytimes.com/hc/en-us}{Help}
\item
  \href{https://www.nytimes.com/subscription?campaignId=37WXW}{Subscriptions}
\end{itemize}
