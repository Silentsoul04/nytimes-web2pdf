Sections

SEARCH

\protect\hyperlink{site-content}{Skip to
content}\protect\hyperlink{site-index}{Skip to site index}

\href{https://www.nytimes.com/section/business}{Business}

\href{https://myaccount.nytimes.com/auth/login?response_type=cookie\&client_id=vi}{}

\href{https://www.nytimes.com/section/todayspaper}{Today's Paper}

\href{/section/business}{Business}\textbar{}Trump Officials Battle Over
Plan to Keep Technology Out of Chinese Hands

\url{https://nyti.ms/31FYiYt}

\begin{itemize}
\item
\item
\item
\item
\item
\item
\end{itemize}

Advertisement

\protect\hyperlink{after-top}{Continue reading the main story}

Supported by

\protect\hyperlink{after-sponsor}{Continue reading the main story}

\hypertarget{trump-officials-battle-over-plan-to-keep-technology-out-of-chinese-hands}{%
\section{Trump Officials Battle Over Plan to Keep Technology Out of
Chinese
Hands}\label{trump-officials-battle-over-plan-to-keep-technology-out-of-chinese-hands}}

The administration is split over restrictions on exporting sensitive
technologies that are vital to protecting national security.

\includegraphics{https://static01.nyt.com/images/2019/10/22/business/22DC-CHINACONTROLS-01/merlin_163084770_4b035de3-a6be-4061-8f0f-5d00f28da969-articleLarge.jpg?quality=75\&auto=webp\&disable=upscale}

\href{https://www.nytimes.com/by/ana-swanson}{\includegraphics{https://static01.nyt.com/images/2018/12/10/multimedia/author-ana-swanson/author-ana-swanson-thumbLarge.png}}

By \href{https://www.nytimes.com/by/ana-swanson}{Ana Swanson}

\begin{itemize}
\item
  Oct. 23, 2019
\item
  \begin{itemize}
  \item
  \item
  \item
  \item
  \item
  \item
  \end{itemize}
\end{itemize}

\href{https://cn.nytimes.com/business/20191024/trump-technology-china-trade/}{阅读简体中文版}\href{https://cn.nytimes.com/business/20191024/trump-technology-china-trade/zh-hant/}{閱讀繁體中文版}

WASHINGTON --- The Trump administration is divided over how aggressively
to restrict China's access to United States technology as it looks for
ways to protect national security without undercutting American
industry.

President Trump and many of his top advisers have identified China's
technological ambitions as a national security threat and want to limit
the type of American technology that can be sold overseas. But a plan to
do just that has encountered stiff resistance from some in the
administration, who argue that imposing too many constraints could
backfire and undermine American industry.

The debate underscores the extent to which Mr. Trump's trade fight with
China has left many issues unresolved. The president
\href{https://www.nytimes.com/2019/10/11/business/economy/us-china-trade-deal.html}{announced
plans this month} to sign a ``Phase 1'' trade agreement that would
require China to buy more farm products and agree to some technology
protections in exchange for a pause in new American tariffs.

The agreement, which has yet to be signed, lowered tensions between the
two nations. But concerns about Beijing's economic ambitions remain,
posing an even greater challenge as the United States considers what
steps to take to ensure American companies dominate the next generation
of technologies.

For nearly a year, the Bureau of Industry and Security, a division in
the Commerce Department, has been working to identify emerging
technologies that, if shared, could pose a security threat to the United
States.

The restrictions aim to head off new security threats. For instance, 3-D
printers could create weapons on the battlefield, making it unnecessary
to ship arms. Artificial intelligence can decode encryptions that
previously could not be cracked. Robots could provide surveillance from
space, while organelles could build tissue for soldiers injured in war.

\includegraphics{https://static01.nyt.com/images/2019/10/23/business/23DC-CHINACONTROLS-02/merlin_141902412_b9827bbc-780e-4718-94f0-c4c5e47797c6-articleLarge.jpg?quality=75\&auto=webp\&disable=upscale}

Last year, Congress passed a law requiring new controls on emerging
technologies. But deciding which technologies should be regulated has
taken longer than anticipated and prompted an ugly conflict in the
administration.

Some administration officials, along with many in the business and
scientific communities, assert that too-tight restrictions risk pushing
research offshore, crippling the commerce that gave rise to America's
technological superiority in the first place. But China critics,
including many of Mr. Trump's political appointees, say Beijing is a
security threat that must be addressed.

``There's a clear fight in the administration between those who want to
have a broad response to China's technology acquisition and development
strategies and those who want to surgically limit China's access to very
specific items and essentially return to a business-as-usual approach to
China,'' said Michael R. Wessel, a member of the U.S.-China Economic and
Security Review Commission, who advocates more comprehensive controls.

Next month, the bureau is expected to announce an initial set of
restrictions on exporting some technologies, including quantum
computing, 3-D manufacturing and an algorithm that guides artificial
intelligence, an official from the bureau said. While those restrictions
are a start, they are not enough to satisfy the president's more hawkish
advisers.

Some analysts say the fight goes beyond any specific technology and
encompasses a broader debate echoing from the halls of Congress to the
White House about how to revise American policy to confront a rising
China. While many in Washington see Beijing as its biggest long-term
rival, China is also the United States' largest trading partner and
crucial to industries like agriculture and manufacturing.

``It's more than just a battle in the Commerce Department,'' said Derek
M. Scissors, a resident scholar at the American Enterprise Institute.
``This is industry pushing back against the Congress.''

Opponents of broad controls say trade and the technological development
it fosters actually give the United States' security advantages ---
including information and income that can be plowed back into further
research.

Business leaders and researchers say rules that are too expansive could
weigh on industries that depend on freely trading components or
knowledge around the world, like developers of driverless cars or
biomaterials. Such restrictions could encourage American companies to
move research facilities to countries without export controls.

``You can't do science with walls around it,'' said Toby Smith, the vice
president for policy at the Association of American Universities. ``If
security dominates the conversation, our scientific leadership may lose
out.''

Companies like Google, General Motors, Microsoft, Toyota and Raytheon
have urged the government to tailor its controls as narrowly as possible
to avoid disrupting their ability to compete around the globe.

In comment letters submitted in January, companies contended that many
emerging technologies, like machine learning and quantum computing, were
already well established in companies and research universities abroad
and that tight restrictions could ultimately jeopardize American
technological development and national security.

``Ultimately, it is far better for U.S. national and economic security
for foreign countries to use U.S. technology products than for the U.S.
to be forced to use theirs,'' Qualcomm said in its letter.

Facebook argued that restrictions could hurt the ability of American
companies to develop technologies and ``risk slowing innovation, and the
hiring and retention of top researchers in the United States.''

The export controls would apply beyond China to Russia and other
nations. But it is Beijing's efforts to harness advanced technologies
that have prompted a bipartisan outcry in Washington.

As part of its Made in China 2025 program, China has introduced plans to
dominate industries of the future, like driverless cars and biomedicine.
In some areas of advanced technology, it is now on a par with the United
States, and its weaponry is increasingly state of the art.

Some of these technologies have been obtained through domestic
development or legitimate investments. But others have been stolen or
coerced through cyberattacks, espionage or unfair economic practices,
American officials say.

Image

The F.B.I. director, Christopher A. Wray, in April at the Council on
Foreign Relations.Credit...Alex Wong/Getty Images

``Put plainly, China seems determined to steal its way up the economic
ladder, at our expense,'' the F.B.I. director, Christopher A. Wray, told
a crowd at the Council on Foreign Relations this year.

That growing suspicion has given rise to an array of policies to more
closely scrutinize the money and technologies that are flowing between
the United States and China. Washington has stepped up reviews of
Chinese investments that could be a security threat, and blacklisted
dozens of Chinese technology firms, including the telecom giant Huawei,
from buying American technology without government approval.

Mr. Trump has also imposed tariffs on more than \$360 billion of Chinese
goods, including semiconductors and aircraft parts. The moves, taken
together, seem aimed at unwinding some of the economic connections
between the United States and China --- a process many in Washington
refer to as ``decoupling.''

Those who advocate limiting economic ties with China say that a previous
policy of engagement has failed to contain the country's more
threatening ambitions, and that setting up barriers is the best way to
protect the United States. Critics say attempts to splinter the world's
two largest economies and their technologies could have devastating
consequences, not just for businesses but for the world as a whole.

At a bureau conference in early July, representatives from companies
like Northrop Grumman and Verizon listened eagerly for clues as to how
the government would define the new rules.

Officials acknowledged the difficulty of policing against national
security threats that could come from a variety of advanced
technologies, without disrupting the United States' leading position as
a center for research and development.

The government's effort to roll out the controls has been complicated by
the departure of several high-level officials, including Nazak
Nikakhtar, who oversaw the process until she withdrew her nomination as
under secretary for industry and security and returned to a different
post in August. Her withdrawal stemmed in part from the internal
administration fight, according to people familiar with the
circumstances.

An official from the Bureau of Industry and Security, who spoke on the
condition of anonymity to discuss sensitive deliberations, acknowledged
that the effort was taking longer than originally anticipated, mostly
because of an intense workload and lack of resources. But the official
said that the bureau was working hard to produce tailored regulations
that would strengthen security, and that it would publish proposals for
controls of specific technologies on a rolling basis beginning next
month.

The department has been tasked with creating two lists of technologies
that cannot be exported, or shared with foreign citizens, even on
American soil. The first batch, which many analysts expected to be
announced earlier this year, focuses on ``emerging,'' or new,
technologies. The second will suggest updates to the ``foundational''
technologies that are already widely used, such as semiconductors.

The current task is made trickier because the United States is no longer
the clear leader in many technologies. Europe leads in some types of 3-D
printing. China is surging ahead in gene editing. And the
democratization of technology creates the potential for someone in any
country to build a biological weapon or a 3-D printer in his or her
basement.

These advanced technologies are a great advantage to the United States
--- and also to American adversaries, said Riz Ramakdawala, a senior
aerospace engineer at the Defense Technology Security Administration.

``The problem comes down to: What's the right level of control?'' he
said.

Past efforts to regulate technologies provide a cautionary tale. In the
late 1990s, the United States placed tight restrictions on exporting
satellite technology in an effort to protect an industry deemed vital to
national security.

The effort backfired. Wary of restrictions that could cripple their
ability to ship products overseas, companies like Boeing, Maxar
Technologies and Lockheed Martin moved satellite manufacturing overseas.
According to a
\href{https://www.bis.doc.gov/index.php/documents/technology-evaluation/898-space-export-control-report/file}{report
by the Commerce Department}, companies said the controls had eroded
American competitiveness in the industry and led to \$1 billion to \$2
billion of lost opportunities from 2009 to 2012.

Advertisement

\protect\hyperlink{after-bottom}{Continue reading the main story}

\hypertarget{site-index}{%
\subsection{Site Index}\label{site-index}}

\hypertarget{site-information-navigation}{%
\subsection{Site Information
Navigation}\label{site-information-navigation}}

\begin{itemize}
\tightlist
\item
  \href{https://help.nytimes.com/hc/en-us/articles/115014792127-Copyright-notice}{©~2020~The
  New York Times Company}
\end{itemize}

\begin{itemize}
\tightlist
\item
  \href{https://www.nytco.com/}{NYTCo}
\item
  \href{https://help.nytimes.com/hc/en-us/articles/115015385887-Contact-Us}{Contact
  Us}
\item
  \href{https://www.nytco.com/careers/}{Work with us}
\item
  \href{https://nytmediakit.com/}{Advertise}
\item
  \href{http://www.tbrandstudio.com/}{T Brand Studio}
\item
  \href{https://www.nytimes.com/privacy/cookie-policy\#how-do-i-manage-trackers}{Your
  Ad Choices}
\item
  \href{https://www.nytimes.com/privacy}{Privacy}
\item
  \href{https://help.nytimes.com/hc/en-us/articles/115014893428-Terms-of-service}{Terms
  of Service}
\item
  \href{https://help.nytimes.com/hc/en-us/articles/115014893968-Terms-of-sale}{Terms
  of Sale}
\item
  \href{https://spiderbites.nytimes.com}{Site Map}
\item
  \href{https://help.nytimes.com/hc/en-us}{Help}
\item
  \href{https://www.nytimes.com/subscription?campaignId=37WXW}{Subscriptions}
\end{itemize}
