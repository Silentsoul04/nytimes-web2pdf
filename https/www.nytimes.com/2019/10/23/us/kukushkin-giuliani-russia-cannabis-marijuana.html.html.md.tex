Sections

SEARCH

\protect\hyperlink{site-content}{Skip to
content}\protect\hyperlink{site-index}{Skip to site index}

\href{https://www.nytimes.com/section/us}{U.S.}

\href{https://myaccount.nytimes.com/auth/login?response_type=cookie\&client_id=vi}{}

\href{https://www.nytimes.com/section/todayspaper}{Today's Paper}

\href{/section/us}{U.S.}\textbar{}As Russian Money Poured Into Cannabis,
Giuliani Allies Scrambled to Partake

\url{https://nyti.ms/2pJZUmG}

\begin{itemize}
\item
\item
\item
\item
\item
\end{itemize}

Advertisement

\protect\hyperlink{after-top}{Continue reading the main story}

Supported by

\protect\hyperlink{after-sponsor}{Continue reading the main story}

\hypertarget{as-russian-money-poured-into-cannabis-giuliani-allies-scrambled-to-partake}{%
\section{As Russian Money Poured Into Cannabis, Giuliani Allies
Scrambled to
Partake}\label{as-russian-money-poured-into-cannabis-giuliani-allies-scrambled-to-partake}}

Russian investors have flocked to the U.S. cannabis industry in recent
years. One venture involving associates of Rudy Giuliani drew the
scrutiny of federal investigators.

\includegraphics{https://static01.nyt.com/images/2019/10/18/us/00Russia-Cannabis01/merlin_159576600_29bfe2d7-ae68-4d7e-9f06-6b6906535062-articleLarge.jpg?quality=75\&auto=webp\&disable=upscale}

By \href{https://www.nytimes.com/by/mike-baker}{Mike Baker} and
\href{https://www.nytimes.com/by/william-k-rashbaum}{William K.
Rashbaum}

\begin{itemize}
\item
  Published Oct. 23, 2019Updated Oct. 24, 2019
\item
  \begin{itemize}
  \item
  \item
  \item
  \item
  \item
  \end{itemize}
\end{itemize}

SAN FRANCISCO --- At a restaurant meeting in California a few years ago,
Brad Hirsch and one of his law clients gathered over a meal with two
potential business partners: Andrey Kukushkin and Andrey Muraviev, an
investor who had flown in from Russia.

Mr. Kukushkin and the Russian financier were hoping Mr. Hirsch could
help them build a stake in the state's burgeoning cannabis market, Mr.
Hirsch said, and he helped them set up a real estate business that would
cater to marijuana operators. Over the span of just a few years, Mr.
Kukushkin would join or develop cannabis companies around San Francisco,
Sacramento, Los Angeles and Las Vegas, establishing a foothold in
everything from real estate and cultivation to retail and delivery.

There was a reason that people like Mr. Kukushkin, who was born in
Ukraine and later worked at a Russian investment bank, had a unique
opportunity to get in on the ground floor. Federal law still treats
cannabis as an illegal substance, and traditional banks have been wary
of getting involved. Wealthy financiers have moved in to fill the void
--- including a growing cast of investors from Russia and former Soviet
Union countries who have helped shape the industry's growth.

One of the nation's largest cannabis companies, Curaleaf, is led by one
of Russia's most influential financiers and backed by another, allowing
the company to pursue rapid expansion and hefty acquisitions. Investment
firms have taken their own stakes: A San Francisco-based venture capital
fund run by the Russian tech entrepreneur Pavel Cherkashin, backed
largely by investors from Russia and the former Soviet Union, has put
\$2 million into Pure Spectrum, a Colorado-based business marketing CBD
products.

``I think there is a strong fear of missing out back in Russia,'' Mr.
Cherkashin said. ``It's one of the most promising and rapidly developing
markets.''

Mr. Kukushkin and some of his business partners appear to have gone a
step further, funneling political contributions to candidates in Nevada
and elsewhere in a way that has
\href{https://www.nytimes.com/2019/10/10/us/politics/lev-parnas-igor-fruman-arrested-giuliani.html?module=inline}{drawn
the scrutiny of federal prosecutors}. Earlier this month, a federal
\href{https://int.nyt.com/data/documenthelper/1886-indictment-giuliani-associates/a5dd834e350dc3b860c2/optimized/full.pdf\#page=1}{grand
jury indicted} four men, including Mr. Kukushkin, in a scheme to use
money from an unnamed Russian to support politicians who could
potentially help them obtain retail marijuana licenses around the
country.

The indictment attracted widespread attention because two of the men
charged are associates of President Trump's personal lawyer, Rudolph
Giuliani, and worked with Mr. Giuliani in the past to collect
potentially damaging information about targets of interest to Mr. Trump
in Ukraine.

They were accused in a separate scheme to conceal the source of a
\$325,000 donation to a pro-Trump super PAC, as well as other political
contributions.

But when it came to Russian money flowing into the United States,
prosecutors focused on its role in the Nevada marijuana business formed
by Mr. Kukushkin and the others. The case illustrates how Mr. Giuliani's
allies were operating not just to advance the president's political
interests, but to build a political network of their own that would give
them entree into one of the country's more promising new industries.

\hypertarget{big-investors}{%
\subsection{Big Investors}\label{big-investors}}

The reluctance of traditional banks to touch marijuana financing has
attracted private investors not from just Russia, but from China, Japan,
South America and from around the United States. Mr. Kukushkin,
according to the indictment, said he was trying to disguise the source
of the Nevada venture's money because of the financier's ``Russian roots
and current political paranoia about it.''

Mr. Kukushkin's lawyer, Gerald B. Lefcourt, declined to comment on the
case.

Other investors with Russian backgrounds have been public about their
involvement in the cannabis industry, and law enforcement officials do
not appear to have raised questions.

Curaleaf, based in Massachusetts, is led by Boris Jordan, a businessman
born in the United States who went on to build the investment bank
Renaissance Capital in Russia, where he now leads the Sputnik Group,
which has a major private equity division. The company's other major
individual investor was Andrei Blokh, a Moscow businessman.

In May, Curaleaf announced a \$950 million deal to acquire the
Oregon-based Cura Partners in one of the industry's largest deals ever.
In July, it followed up with an \$875 million deal to acquire the
Illinois-based Grassroots Cannabis.

Vedomosti, a Russian business publication,
\href{https://www.vedomosti.ru/business/articles/2019/05/22/802199-na-rinok}{reported
earlier this year} that it had talked with eight investment funds of
Russian origin that were either considering cannabis investments or had
already pursued them.

Some states, including Oregon and Maine, tried to reap the benefits of a
cannabis industry by requiring that companies be locally controlled. But
that has been a struggle as the industry has pushed for open markets in
order to get access to funding, said Andrew Freedman, who helped the
lead the development of Colorado's legal cannabis market.

``A lot of these states are trying to keep the money and the ownership
interest within the four corners of the state,'' Mr. Freedman said. ``It
simply isn't happening.''

Federal prosecutors said the Russian money backing the business of Mr.
Kukushkin and others was helping lay the groundwork of a multistate
operation. The Russian partner, according to two people familiar with
the case, was Andrey Muraviev --- the man Mr. Kukushkin had brought to
the meeting that day with Mr. Hirsch.

\hypertarget{from-russia-to-california}{%
\subsection{From Russia to California}\label{from-russia-to-california}}

\includegraphics{https://static01.nyt.com/images/2019/10/18/us/00Russia-Cannabis04/merlin_162871440_3b76aa06-748d-483b-9734-91a07b73c05d-articleLarge.jpg?quality=75\&auto=webp\&disable=upscale}

Born in the Ukrainian port city of Odessa when it was still part of the
Soviet Union, Mr. Kukushkin earned degrees at Odessa National
Polytechnic University in engineering and finance in 1992, then worked
in the Russian finance industry.

Even before entering the cannabis world, Mr. Kukushkin, 46, was living a
comfortable life, with photos on Russian social media showing him
vacationing at the elite French resort of Chamonix. Mr. Kukushkin listed
himself as living in Ukraine, Russia and San Francisco.

In 2013, Mr. Kukushkin took up residence in a 1,400 square-foot condo a
few blocks from San Francisco's financial district.

Mr. Muraviev, meanwhile, was born in Russia and partially educated in
San Francisco. He led a cement company in Russia before starting the
investment company Parus Capital.

Together, their first foray into the cannabis industry appears to have
been in 2015.

Records show that Mr. Kukushkin helped Mr. Muraviev steer a \$1 million
investment into a California cannabis management company, Venture Rebel,
which helped run a San Francisco cannabis shop known as MediThrive.
Details of the venture, outlined in a lawsuit filed in San Francisco
County Superior Court, were first reported by
\href{https://www.mcclatchydc.com/news/politics-government/article236028683.html}{McClatchy
newspapers} and
\href{https://www.motherjones.com/politics/2019/10/whos-the-secret-russian-in-the-indictment-of-giulianis-pals-we-found-some-clues/}{Mother
Jones}.

Mr. Kukushkin and Mr. Muraviev expanded next into the Sacramento area,
joining up with Mr. Hirsch, the lawyer who met with them in San
Francisco, and his client, Garib Karapetyan. Mr. Hirsch said he lost his
enthusiasm for the partnership when Mr. Kukushkin began to do things
like demanding new terms in the 11th hour of negotiations.

In regulatory applications, another Kukushkin company, Oasis Venture,
proposed a large cannabis cultivation site east of San Francisco,
including a greenhouse that would have 22,000 square feet of cannabis
canopy and a processing facility on an estate with garage space for 12
vehicles and panoramic views of the nearby Alameda County valleys and
foothills.

In the last few months, Mr. Kukushkin has been pursuing a dispensary
license in the Los Angeles area.

Sean Maddocks, a legal consultant who has helped in that effort, said
Mr. Kukushkin approached him last year looking for guidance on where he
could get additional licenses.

He said Mr. Kukushkin never discussed anything like campaign
contributions or any improper effort to get a license.

\hypertarget{political-contributions}{%
\subsection{Political Contributions}\label{political-contributions}}

Federal prosecutors have seized on the issue of campaign contributions
in the Nevada case, and that is where Mr. Giuliani's associates from
Florida entered the marijuana case: Lev Parnas, a native of Ukraine, and
Igor Fruman, originally from Belarus --- both now American citizens who
have long lived in Florida --- along with David Correia, another South
Florida resident.

In the summer of 2018, according to federal prosecutors, those three men
teamed up with Mr. Kukushkin to develop a multistate cannabis business
strategy.

Two people familiar with the details of the federal case said the
financier was to be Mr. Muraviev, who did not respond to emails or phone
messages. Both Mr. Fruman's lawyer, Todd Blanche, and Mr. Correia's
lawyer, Jeffrey Marcus, declined to comment. The lawyer for Mr. Parnas,
Edward B. MacMahon Jr., did not immediately respond to a request for
comment.

Mr. Correia drafted a document that considered between \$1 million and
\$2 million of potential political donations to help win marijuana
retail licenses in Nevada and elsewhere, according to the indictment.
Though donations from foreign nationals to American political campaigns
are illegal, the indictment says that the Russian arranged two wire
transfers totaling \$1 million to Mr. Fruman that were intended at least
in part for political candidates.

Near the end of October 2018, the group apparently realized that the
deadline for getting a license in Nevada had already passed --- ``unless
we change the rules,'' Mr. Kukushkin said, according to the indictment.
They talked about needing the support of a Nevada state candidate.

The indictment does not name the candidate being discussed, but a week
later, records show, Mr. Fruman donated the maximum amount, \$10,000, to
both Adam Laxalt and Wesley Duncan. Both Republicans, Mr. Laxalt was the
state's attorney general running for governor, while Mr. Duncan was
running to succeed Mr. Laxalt.

Mr. Duncan and Mr. Laxalt both said through spokespeople that they were
unaware of any illegal activity and were returning the donations.

If the contributions and support were intended to produce a result, they
failed. Mr. Duncan and Mr. Laxalt both lost their elections.

Mike Baker reported from San Francisco and William K. Rashbaum reported
from New York.

Advertisement

\protect\hyperlink{after-bottom}{Continue reading the main story}

\hypertarget{site-index}{%
\subsection{Site Index}\label{site-index}}

\hypertarget{site-information-navigation}{%
\subsection{Site Information
Navigation}\label{site-information-navigation}}

\begin{itemize}
\tightlist
\item
  \href{https://help.nytimes.com/hc/en-us/articles/115014792127-Copyright-notice}{©~2020~The
  New York Times Company}
\end{itemize}

\begin{itemize}
\tightlist
\item
  \href{https://www.nytco.com/}{NYTCo}
\item
  \href{https://help.nytimes.com/hc/en-us/articles/115015385887-Contact-Us}{Contact
  Us}
\item
  \href{https://www.nytco.com/careers/}{Work with us}
\item
  \href{https://nytmediakit.com/}{Advertise}
\item
  \href{http://www.tbrandstudio.com/}{T Brand Studio}
\item
  \href{https://www.nytimes.com/privacy/cookie-policy\#how-do-i-manage-trackers}{Your
  Ad Choices}
\item
  \href{https://www.nytimes.com/privacy}{Privacy}
\item
  \href{https://help.nytimes.com/hc/en-us/articles/115014893428-Terms-of-service}{Terms
  of Service}
\item
  \href{https://help.nytimes.com/hc/en-us/articles/115014893968-Terms-of-sale}{Terms
  of Sale}
\item
  \href{https://spiderbites.nytimes.com}{Site Map}
\item
  \href{https://help.nytimes.com/hc/en-us}{Help}
\item
  \href{https://www.nytimes.com/subscription?campaignId=37WXW}{Subscriptions}
\end{itemize}
