Sections

SEARCH

\protect\hyperlink{site-content}{Skip to
content}\protect\hyperlink{site-index}{Skip to site index}

\href{https://www.nytimes.com/section/arts/television}{Television}

\href{https://myaccount.nytimes.com/auth/login?response_type=cookie\&client_id=vi}{}

\href{https://www.nytimes.com/section/todayspaper}{Today's Paper}

\href{/section/arts/television}{Television}\textbar{}`His Dark
Materials': Witches, Quests and Talking Animals

\href{https://nyti.ms/333QX6A}{https://nyti.ms/333QX6A}

\begin{itemize}
\item
\item
\item
\item
\item
\item
\end{itemize}

Advertisement

\protect\hyperlink{after-top}{Continue reading the main story}

Supported by

\protect\hyperlink{after-sponsor}{Continue reading the main story}

\hypertarget{his-dark-materials-witches-quests-and-talking-animals}{%
\section{`His Dark Materials': Witches, Quests and Talking
Animals}\label{his-dark-materials-witches-quests-and-talking-animals}}

After years in development, this adaptation of Philip Pullman's
best-selling trilogy arrives as HBO's next fantasy epic.

\includegraphics{https://static01.nyt.com/images/2019/11/03/arts/3dark-materials1/3dark-materials1-articleLarge-v2.jpg?quality=75\&auto=webp\&disable=upscale}

By \href{https://www.nytimes.com/by/roslyn-sulcas}{Roslyn Sulcas}

\begin{itemize}
\item
  Oct. 31, 2019
\item
  \begin{itemize}
  \item
  \item
  \item
  \item
  \item
  \item
  \end{itemize}
\end{itemize}

CARDIFF, Wales --- Lin-Manuel Miranda sat patiently in a shabby,
capacious hot-air balloon, as a technician checked the lighting on a
stuffed bird poised overhead. A puppeteer stood nearby, holding a
stand-in for his ``daemon'' (in this case a hare), the animal that
represents an externalization of each individual's soul in the
fantastical world of Philip Pullman's ``His Dark Materials.''

``It's like a buddy movie,'' Miranda said with a laugh, nodding at the
puppet.

But this isn't exactly a buddy movie. The
\href{https://www.youtube.com/watch?v=APduGe1eLVI}{new HBO and BBC One
series}, based on Pullman's best-selling fantasy trilogy, features
hot-air balloons, talking animals, witches, armored polar bears,
multiple worlds, a young heroine on a quest and a world-class villain
who happens to be an alluringly beautiful woman. The details of
Pullman's elaborate, wide-ranging narrative have been painstakingly
recreated in this first season of ``His Dark Materials,'' which will
have its U.S. premiere on Monday, featuring:
\href{https://www.nytimes.com/2018/08/29/style/ruth-wilson-the-affair-little-stranger.html?searchResultPosition=4}{Ruth
Wilson (``The Affair'')} as Mrs. Coulter; James McAvoy (``X-Men'') as
Lord Asriel;
\href{https://www.nytimes.com/2015/02/18/theater/review-in-hamilton-lin-manuel-miranda-forges-democracy-through-rap.html?searchResultPosition=6}{Miranda
(``Hamilton'')} as Lee Scoresby; and Dafne Keen as the 12-year-old Lyra,
the protagonist of the story. (HBO and BBC also committed to a second
season, which is currently being filmed.)

Pullman's elaborate, thrilling narrative, nominally aimed at young
adults but read far more widely, has sold around 18 million books
worldwide, been translated into 40 languages and turned into
well-received stage and radio plays.

But it has never been successfully adapted for the screen. The lone
attempt, a
\href{https://www.nytimes.com/2007/12/07/movies/07comp.html?searchResultPosition=7}{2007
film adaptation of ``The Golden Compass}'' (the American title of the
first book in the series), did poorly at the box office. It also aroused
the ire of some religious groups,
\href{https://www.theguardian.com/uk/2007/oct/14/religion.books}{which
accused Pullman} of waging a stealth war against organized religion
through his depiction of the Magisterium, the authoritarian ruling power
that controls most of his heroine's world. (The name is the word the
Roman Catholic church uses for its official teaching authority.)

\includegraphics{https://static01.nyt.com/images/2019/11/03/arts/03dark-materials2/merlin_162858369_467face4-7932-4705-be27-f88aba279329-articleLarge.jpg?quality=75\&auto=webp\&disable=upscale}

None of that dissuaded the British producer Jane Tranter from acquiring,
in late 2015, the rights to the trilogy, which she called ``a national
treasure.'' Tranter, whose company Bad Wolf is producing the series with
New Line Cinema for the BBC and HBO, believed that the novels'
complexity made them better suited to a TV adaptation, she said.

But that doesn't mean it was easy. It took two-and-a-half years for the
creative team --- including
\href{https://www.nytimes.com/2019/09/04/theater/jack-thorne-his-dark-materials.html?searchResultPosition=1}{the
writer Jack Thorne}, the director Tom Hooper (who directed the first two
episodes) **** and the special-effects studio Framestore --- to figure
out how to structure the series, bring the daemons (pronounced
``demons'') to life and create the different environments that the
heroine journeys through.

Making things more tense was the fact that while the BBC signed on
early, HBO didn't commit until filming was about to begin. ``It was a
massive gamble, we had nowhere near the entire budget,'' Tranter said.
(No one would give a budget number, but the show is
\href{https://www.telegraph.co.uk/news/2019/07/20/bbc-takes-gamble-philip-pullmans-dark-materials-expensive-british/}{widely
rumored to be very expensive} because of its extensive special effects
and variety of locations.)

For HBO, ``His Dark Materials'' was an opportunity to broaden ``what
could be considered an HBO show,'' said Casey Bloys, the head of
programming. One thing it is not designed to be, he insisted, was a
replacement for the network's last big fantasy adaptation, the
\href{https://www.nytimes.com/interactive/2019/arts/game-of-thrones.html?module=inline}{recently
departed ``Game of Thrones}.''

``Let's put this to rest: There is no next `Game of Thrones,''' Bloys
said. What HBO liked about ``Dark Materials,'' he said, was that it was
family friendly but had big, complex themes.

It certainly does. Pullman's series --- consisting of ``The Golden
Compass,'' ``The Subtle Knife'' and ``The Amber Spyglass'' ---
essentially reimagines
\href{http://www.bbc.com/culture/story/20170419-why-paradise-lost-is-one-of-the-worlds-most-important-poems}{Milton's
epic poem, ``Paradise Lost}.'' (The title ``His Dark Materials'' comes
from a line in the poem.) The story posits a world that is saved rather
than ruined by original sin, represented in the novels by a mysterious
substance called Dust. Along the way it explores questions about good
and evil, religion and morality, the relationship of the soul to the
body and the nature of consciousness itself.

But you don't have to know any of that to enjoy the cracking pace of the
narrative.

Image

Ruth Wilson as Mrs. Coulter, with a ``daemon''.Credit...HBO

The story centers on Lyra Belacqua, an orphan living at Jordan College
in a parallel-world Oxford, in which zeppelins float and the gargoyles
represent animal daemons. When her closest friend disappears, Lyra
embarks on a quest to find him --- one that ultimately involves
traveling to London with the mysterious Mrs. Coulter (Wilson),
discovering a government sanctioned child-abduction program and heading
into the frozen north, where witches and armored bears hold sway.

``She is a wounded child, because she has no parents, and a brave young
girl,'' Keen said in a telephone interview. ``It is an amazing part to
play.''

Thorne, the prolific writer of the play
``\href{https://www.nytimes.com/2016/08/02/books/harry-potter-and-the-cursed-child-review.html?searchResultPosition=6}{Harry
Potter and the Cursed Child},'' among many other recent works for stage
and screen, said in a phone interview that he loved Pullman's choice of
a 12-year-old heroine to carry the story. (At the London premiere of the
series, he likened the character to the climate campaigner
\href{https://www.nytimes.com/2019/09/24/climate/greta-thunberg-un.html}{Greta
Thunberg}.)

A self-described workaholic, Thorne wrote 46 drafts of the first
episode. ``My job was to make sure that the storytelling was at the
right pace,'' he said. ``When you have all these possibilities of CGI,
you have to be careful. We need complexity as well as bells and
whistles.''

There are plenty of those in ``His Dark Materials,'' most notably in the
creation of the daemons, which had to be added through visual effects
technology in postproduction. (That they are extensions of each person
and cannot be separated without terrible physical and emotional pain is
an essential part of the plotline.) To help the actors when filming
those scenes, the production employed puppeteers.

``There is a nightmare version of this kind of work where you are acting
with a tennis ball held at the right-eye line, but this was amazing,''
Miranda said during a break on set. ``It really is another actor in the
scene, and it adds all kinds of fun and subtext.''

Eliot Gibbons, the production's workshop manager, said that his team of
puppeteers tried to create figures with character and personality, even
if they weren't fully realized. He picked up and manipulated a
startlingly charismatic wooden cat, which had one arm and was attached
to what looked like a Slinky.

``We worked closely with visual effects to make sure they were happy
with the size and movement,'' he said. ``You have birds and lizards,
hares and monkeys, and snakes coming in and out of clothes. Each one
would have about seven iterations, from eyes on a stick to more
sophisticated versions.''

Image

Wilson with the lead puppeteer Brian Fisher on the set.~``The animals
represent a side to each character,'' she said.Credit...Alex Bailey/HBO

Image

Armored polar bears are among the mystical beasts that appear in the
show.Credit...HBO

And then there were the mammoth polar bears, represented during the
shoot by a person wearing a huge puppetry rig, so that the animal's
movement had ``irregularity and breath,'' Gibbons said. The stand-ins
formed the visual foundation for the animation, which technicians would
then painstakingly layer into the scenes.

The animals are expressive and often cute, but their roles go beyond
simply charming younger viewers. Wilson, who worked closely with her
puppeteer, Brian Fisher, said the daemons help illuminate
flesh-and-blood characters like her Mrs. Coulter, a seductive villainess
of complex moral stature and motive.

``The animals represent a side to each character,'' she said. ``Why does
Mrs. Coulter have a monkey? Why can she separate from her daemon when no
one else can? Why does it have no name? I have my own theories about why
she is capable of doing horrific things and believing it's the right
thing to do: She is silencing her soul, and there is an abusive
relationship between the two. It has to be about self-harm.''

Mrs. Coulter is a powerful female figure in a male world, Wilson noted,
adding that she modeled the character's image on the
\href{https://www.biography.com/actor/hedy-lamarr}{1930s and 1940s
actress} \href{https://www.biography.com/actor/hedy-lamarr}{Hedy
Lamarr}, who was also a gifted inventor. ``Like her, Mrs. Coulter is
incredibly intelligent, but understands how to gain power through image
and sexuality,'' Wilson said.

Thorne expanded Mrs. Coulter's role in the show. ``Her relationship with
Lyra felt so key to everything,'' he said.

It was one of several changes he made --- others included bringing
forward parts of ``The Subtle Knife,'' the second book, and
incorporating a small amount of material from Pullman's 2017 prequel
novel,
\href{https://www.nytimes.com/2017/10/18/books/review-la-belle-sauvage-philip-pullman-book-of-dust.html}{``La
Belle Sauvage.''} At every step, Thorne said, he consulted Pullman.

``It was a blessing, not a mixed blessing, to be able to work with
him,'' he said. ``We heard the stories from the gaps between the
chapters.''

Miranda, who plays the Texan aeronaut Lee Scoresby, said that he had
loved the books since reading them in the early 2000s.

``If anyone is up to adapting Philip Pullman's world, it is Jack
Thorne,'' he said, adding that he had been thrilled to play Lee. ``In a
book full of larger than life characters, with bears and witches and
angels, he stands out. He is the Han Solo of the crew, in it for
himself, then slowly reveals himself as a hero.'' Miranda had
corresponded with Pullman, he added, but not yet met him.

``I just live in his dreams,'' he said. ``It's a nice place to live.''

Advertisement

\protect\hyperlink{after-bottom}{Continue reading the main story}

\hypertarget{site-index}{%
\subsection{Site Index}\label{site-index}}

\hypertarget{site-information-navigation}{%
\subsection{Site Information
Navigation}\label{site-information-navigation}}

\begin{itemize}
\tightlist
\item
  \href{https://help.nytimes.com/hc/en-us/articles/115014792127-Copyright-notice}{©~2020~The
  New York Times Company}
\end{itemize}

\begin{itemize}
\tightlist
\item
  \href{https://www.nytco.com/}{NYTCo}
\item
  \href{https://help.nytimes.com/hc/en-us/articles/115015385887-Contact-Us}{Contact
  Us}
\item
  \href{https://www.nytco.com/careers/}{Work with us}
\item
  \href{https://nytmediakit.com/}{Advertise}
\item
  \href{http://www.tbrandstudio.com/}{T Brand Studio}
\item
  \href{https://www.nytimes.com/privacy/cookie-policy\#how-do-i-manage-trackers}{Your
  Ad Choices}
\item
  \href{https://www.nytimes.com/privacy}{Privacy}
\item
  \href{https://help.nytimes.com/hc/en-us/articles/115014893428-Terms-of-service}{Terms
  of Service}
\item
  \href{https://help.nytimes.com/hc/en-us/articles/115014893968-Terms-of-sale}{Terms
  of Sale}
\item
  \href{https://spiderbites.nytimes.com}{Site Map}
\item
  \href{https://help.nytimes.com/hc/en-us}{Help}
\item
  \href{https://www.nytimes.com/subscription?campaignId=37WXW}{Subscriptions}
\end{itemize}
