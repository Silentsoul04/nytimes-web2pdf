Sections

SEARCH

\protect\hyperlink{site-content}{Skip to
content}\protect\hyperlink{site-index}{Skip to site index}

\href{https://www.nytimes.com/section/world/asia}{Asia Pacific}

\href{https://myaccount.nytimes.com/auth/login?response_type=cookie\&client_id=vi}{}

\href{https://www.nytimes.com/section/todayspaper}{Today's Paper}

\href{/section/world/asia}{Asia Pacific}\textbar{}China Says It Will
Roll Out `National Security' Steps for Hong Kong

\url{https://nyti.ms/2q4yxEe}

\begin{itemize}
\item
\item
\item
\item
\item
\end{itemize}

Advertisement

\protect\hyperlink{after-top}{Continue reading the main story}

Supported by

\protect\hyperlink{after-sponsor}{Continue reading the main story}

\hypertarget{china-says-it-will-roll-out-national-security-steps-for-hong-kong}{%
\section{China Says It Will Roll Out `National Security' Steps for Hong
Kong}\label{china-says-it-will-roll-out-national-security-steps-for-hong-kong}}

Communist Party leaders announced the move after months of protests in
the city, but gave no details. Here is an explanation of that measure
and others they approved.

\includegraphics{https://static01.nyt.com/images/2019/10/31/world/31china-xi/merlin_163388292_6c988391-cc4b-4818-8e3e-73e1f04680be-articleLarge.jpg?quality=75\&auto=webp\&disable=upscale}

\href{https://www.nytimes.com/by/chris-buckley}{\includegraphics{https://static01.nyt.com/images/2018/10/08/multimedia/author-chris-buckley/author-chris-buckley-thumbLarge.png}}

By \href{https://www.nytimes.com/by/chris-buckley}{Chris Buckley}

\begin{itemize}
\item
  Published Oct. 31, 2019Updated Nov. 6, 2019
\item
  \begin{itemize}
  \item
  \item
  \item
  \item
  \item
  \end{itemize}
\end{itemize}

\href{https://cn.nytimes.com/china/20191101/hong-kong-protests-china/}{阅读简体中文版}\href{https://cn.nytimes.com/china/20191101/hong-kong-protests-china/zh-hant/}{閱讀繁體中文版}

BEIJING --- China will roll out new steps to
``\href{https://www.nytimes.com/2019/11/06/world/asia/hong-kong-protests-china-national-security.html}{safeguard
national security}'' in Hong Kong after months of
\href{https://www.nytimes.com/2019/10/31/world/asia/hong-kong-halloween.html}{antigovernment
protests} that have destabilized the semiautonomous city, the Chinese
Communist Party leadership announced on Thursday.

The vague yet potentially far-reaching proposal for Hong Kong was
announced at the end of a four-day meeting of the party's Central
Committee, which brings together about 370 senior officials to decide
the direction of party policy around once a year.

The official summary of the meeting,
\href{http://www.chinanews.com/gn/2019/10-31/8994802.shtml}{released by
Xinhua}, the official Chinese news agency, contained few details of that
and other proposals intended to defend the authority of the Communist
Party and its leader, Xi Jinping, and to improve decision making.
Details may come out in documents and speeches released days or weeks
later.

Here are key points from the summary.

\hypertarget{hints-of-a-new-plan-to-quell-the-hong-kong-protests}{%
\subsection{Hints of a new plan to quell the Hong Kong
protests.}\label{hints-of-a-new-plan-to-quell-the-hong-kong-protests}}

The most eye-catching language was about Hong Kong, where for some 21
weeks
\href{https://www.nytimes.com/2019/10/31/world/asia/hong-kong-halloween.html}{protesters
have challenged the Beijing-backed government}, demanded democracy and
denounced China's growing hold over the city, a former British colony
that maintains its own laws and freedoms.

Hong Kong and Macau, a former Portuguese colony, are both run as
``special administrative regions'' under Chinese sovereignty. China
would ``build and improve a legal system and enforcement mechanism to
defend national security in the special administrative regions,'' the
meeting summary said.

The vague language leaves plenty of guesswork about what the Chinese
leaders may have in mind. Some pro-Beijing hard-liners in
\href{https://www.nytimes.com/2019/10/06/world/asia/hong-kong-protests-face-mask-ban.html}{Hong
Kong have suggested} the time may have come for the Chinese authorities
to impose new security legislation on the territory, which Britain
returned to Chinese sovereignty in 1997.

Article 18 of the Basic Law, the mini-constitution that defines Hong
Kong's status, gives Beijing broad authority, which it has never
exercised, to act on a perceived threat in Hong Kong to national threat
or unity.

The Basic Law also requires that Hong Kong pass its own national
security laws, but it has not done so, especially after protests in 2003
prompted the territory's government to abandon proposed legislation.

\hypertarget{a-fresh-focus-on-clearly-increasing-risks}{%
\subsection{A fresh focus on `clearly increasing'
risks.}\label{a-fresh-focus-on-clearly-increasing-risks}}

The Communist Party leadership met as
\href{https://www.nytimes.com/2019/10/30/us/politics/us-china-trade-deal.html}{China
is grappling with a trade war} with the Trump administration and a
\href{https://www.nytimes.com/2019/10/17/business/china-economic-growth.html}{marked
slowdown in China's economic growth}.

The official summary did not mention those issues --- top-level party
documents often stick to broad statements --- but one clause suggested
that Mr. Xi and his colleagues feel that the risks have risen. The
summary said the leadership had withstood ``a complex situation of
clearly increasing domestic and external risks and challenges.''

Over the past year, Mr. Xi has
\href{https://www.nytimes.com/2019/02/25/world/asia/china-xi-warnings.html}{repeatedly
warned} Communist Party officials to steel themselves for ``struggle''
and hazards such as possible economic turbulence, rising debt levels
linked to local governments, technological competition and sparks of
social discontent spread across the internet.

The wording from the latest leadership meeting suggests that Mr. Xi sees
no easing in those risks.

\hypertarget{extending-the-communist-partys-leadership-even-further}{%
\subsection{Extending the Communist Party's leadership even
further.}\label{extending-the-communist-partys-leadership-even-further}}

The Central Committee echoed Mr. Xi's frequent demands that the
``Communist Party leads everything,'' and that the authority of central
leaders, like himself, be fiercely protected. And it hinted that there
may be more changes to bolster Mr. Xi and the party, while also trying
to improve coordination in policymaking.

Since coming to power in 2012, Mr. Xi has created party policy groups
and
\href{https://www.nytimes.com/2017/11/29/world/asia/china-xi-jinping-anticorruption.html}{investigation
bodies} that enhance the power of central leadership, above all himself.

Last year, he swept away a term limit on his presidency, opening the way
to an indefinite stay in power. But investors, experts and officials
have complained that the flurry of change has
\href{https://www.nytimes.com/2017/03/04/world/asia/china-xi-jinping-economic-reform.html}{created
caution and confusion} among policymakers, and held back
\href{https://www.nytimes.com/2017/03/27/business/chinese-economy-reform-critical-report.html}{promised
reforms}.

The latest meeting promised both stronger centralized leadership and
better policy decisions.

``Improve the leadership system for managing the overall situation and
coordinating all sides,'' the summary said. ``Improve every institution
for firmly defending the authority of the party central leadership and
centralized, unified leadership.''

Such sweeping language could, for example, open the way to changes in
how party's policy-setting commissions operate.

Some details may become clearer if the party issues the decision on
improving the ``national governance system'' that the committee
approved. Precedent suggests that the decision may be published in the
coming days or weeks.

Keith Bradsher contributed reporting from Wuhan, China. Amber Wang
contributed research from Beijing.

Advertisement

\protect\hyperlink{after-bottom}{Continue reading the main story}

\hypertarget{site-index}{%
\subsection{Site Index}\label{site-index}}

\hypertarget{site-information-navigation}{%
\subsection{Site Information
Navigation}\label{site-information-navigation}}

\begin{itemize}
\tightlist
\item
  \href{https://help.nytimes.com/hc/en-us/articles/115014792127-Copyright-notice}{©~2020~The
  New York Times Company}
\end{itemize}

\begin{itemize}
\tightlist
\item
  \href{https://www.nytco.com/}{NYTCo}
\item
  \href{https://help.nytimes.com/hc/en-us/articles/115015385887-Contact-Us}{Contact
  Us}
\item
  \href{https://www.nytco.com/careers/}{Work with us}
\item
  \href{https://nytmediakit.com/}{Advertise}
\item
  \href{http://www.tbrandstudio.com/}{T Brand Studio}
\item
  \href{https://www.nytimes.com/privacy/cookie-policy\#how-do-i-manage-trackers}{Your
  Ad Choices}
\item
  \href{https://www.nytimes.com/privacy}{Privacy}
\item
  \href{https://help.nytimes.com/hc/en-us/articles/115014893428-Terms-of-service}{Terms
  of Service}
\item
  \href{https://help.nytimes.com/hc/en-us/articles/115014893968-Terms-of-sale}{Terms
  of Sale}
\item
  \href{https://spiderbites.nytimes.com}{Site Map}
\item
  \href{https://help.nytimes.com/hc/en-us}{Help}
\item
  \href{https://www.nytimes.com/subscription?campaignId=37WXW}{Subscriptions}
\end{itemize}
