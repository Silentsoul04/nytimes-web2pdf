Sections

SEARCH

\protect\hyperlink{site-content}{Skip to
content}\protect\hyperlink{site-index}{Skip to site index}

\href{https://www.nytimes.com/section/politics}{Politics}

\href{https://myaccount.nytimes.com/auth/login?response_type=cookie\&client_id=vi}{}

\href{https://www.nytimes.com/section/todayspaper}{Today's Paper}

\href{/section/politics}{Politics}\textbar{}Trump's Twitter War Room
Aims Its Punches at Decorated Colonel

\url{https://nyti.ms/2Nr2UOq}

\begin{itemize}
\item
\item
\item
\item
\item
\item
\end{itemize}

Advertisement

\protect\hyperlink{after-top}{Continue reading the main story}

Supported by

\protect\hyperlink{after-sponsor}{Continue reading the main story}

\hypertarget{trumps-twitter-war-room-aims-its-punches-at-decorated-colonel}{%
\section{Trump's Twitter War Room Aims Its Punches at Decorated
Colonel}\label{trumps-twitter-war-room-aims-its-punches-at-decorated-colonel}}

\includegraphics{https://static01.nyt.com/images/2019/11/06/multimedia/06vindman/merlin_163517688_abd15f6d-12cf-40a3-93ad-2851e1b01c68-articleLarge.jpg?quality=75\&auto=webp\&disable=upscale}

By \href{https://www.nytimes.com/by/mike-mcintire}{Mike McIntire} and
\href{https://www.nytimes.com/by/nicholas-confessore}{Nicholas
Confessore}

\begin{itemize}
\item
  Nov. 6, 2019
\item
  \begin{itemize}
  \item
  \item
  \item
  \item
  \item
  \item
  \end{itemize}
\end{itemize}

Days after a decorated Army lieutenant colonel offered damaging
testimony about President Trump's conduct on a July phone call with
Ukraine's leader, Mr. Trump stood on the South Lawn and issued a vague
but ominous warning.

``You'll be seeing very soon what comes out,'' Mr. Trump said on
Saturday, referring to the officer,
\href{https://www.nytimes.com/2019/10/29/us/politics/who-is-alexander-vindman.html}{Lt.
Col. Alexander S. Vindman}.

Mr. Trump was not more specific. But an attack on Colonel Vindman's
character and motives was already making its way from the dark corners
of Mr. Trump's social media following to the front lines of the
impeachment battle.

One day earlier, the right-wing commentator Jack Posobiec had retweeted
a lengthy thread by a Florida man --- a fan of QAnon,
\href{https://www.nytimes.com/interactive/2019/11/02/us/politics/trump-twitter-disinformation.html}{a
fringe conspiracy} about the ``deep state'' --- claiming to have
witnessed Colonel Vindman ``bash America'' in conversation with Russian
officers during a joint military exercise in Germany in 2013.

That accusation was unsubstantiated and has been rejected by some of the
colonel's colleagues. Even so, Mr. Posobiec's post was retweeted by Mr.
Trump's son and chief defender, Donald Trump Jr., driving it through
conservative social media circles and onto pro-Trump websites, whose
stories the younger Mr. Trump promoted to his four million followers.

``Anyone who's been watching for the past three years is not at all
surprised that this would be their `star witness,''' Donald Jr. posted
about Colonel Vindman, who had testified that he was concerned about the
United States' linking of military aid to Ukraine with an investigation
of Mr. Trump's political rival.

While the White House has scrambled to mount an organized response to
the House impeachment inquiry --- there is no consistent message from
Mr. Trump's team and little formal guidance to surrogates --- Twitter
has become the Trump war room. The president and his supporters,
including his family,
\href{https://www.nytimes.com/interactive/2019/11/02/us/politics/trump-twitter-presidency.html}{have
used Twitter} to frame his defense, torch his Democratic inquisitors and
try to undermine public officials, like Colonel Vindman, who have
testified against him.

It is hard to discern how the six-year-old comments attributed to the
officer affect the veracity of his testimony on Capitol Hill, which
aligns with that of numerous other witnesses. But by questioning the
colonel's loyalties, partisans who are spreading the story uncritically
to millions of Americans leave the impression he is somehow not to be
believed.

The attack emerged late on Halloween night, when a retired Army officer,
Jim Hickman, claimed he had overheard Colonel Vindman --- a major at the
time who was chatting with Russian soldiers during a military exercise
--- laugh ``about Americans not being educated or worldly'' and talking
up ``Obama \& globalism to the point of uncomfortable.'' Mr. Hickman
said he took the major aside and reprimanded him.

Through his lawyer, Michael Volkov, Colonel Vindman declined to comment.

Mr. Hickman, a former lieutenant colonel whose service record indicates
he served in Afghanistan and earned a Purple Heart, at some point took
an interest in QAnon. A review of his past tweets found more than 100 in
which he recirculated or commented on QAnon-related theories, including
hoaxes about Satanism and pedophilia, and until recently he had the
hashtag \#Q in his profile. Reached for comment, Mr. Hickman said he did
not believe in QAnon but found it ``interesting.''

``I do think it's actually been pretty accurate on predicting a lot of
things,'' he said.

He has also tweeted strident pro-Trump, anti-Democratic themes, writing,
``It's incredible how evil the Democrat party is.'' A week before going
public with his story about Colonel Vindman, he retweeted a Trump
supporter urging: ``STOP IMPEACHMENT! STOP THIS COUP!''

In a Twitter thread, Mr. Hickman, who said he was disabled from combat
injuries and living in Florida, said he had helped manage joint
exercises in Germany involving United States and Russian soldiers. He
met Colonel Vindman there in 2013, he said.

Colonel Vindman referred to himself as a patriot during closed-door
testimony in the House last month, and said he had reported concerns
about the president and his inner circle's conduct out of a ``sense of
duty.'' The colonel received a Purple Heart after being injured by an
improvised explosive device in Iraq. He now serves on the National
Security Council.

Several officials have publicly defended the colonel since his testimony
emerged. General Joseph F. Dunford Jr., the former chairman of the Joint
Chiefs of Staff, has
\href{https://thehill.com/homenews/administration/468262-dunford-on-vindman-a-professional-competent-patriotic-and-loyal}{called}
the colonel ``a professional, competent, patriotic and loyal officer.''
Michael McFaul, the former ambassador to Russia,
\href{https://twitter.com/McFaul/status/1191983177725861892}{has said}
he had worked with the colonel ``and interacted with him in front of
Russian officers. He never once said anything near what this `retired
Army officer' claims.''

Mr. Hickman appears to have first shared his story in a private ``DM
room'' on Twitter, where people can send direct messages to one another.
He said in a tweet that he had forgotten about the encounter with
Colonel Vindman, but that his Army friends ``reminded me of what
happened and it all came back,'' adding ``Damn TBI,'' a reference to
traumatic brain injury.

As the tale gained attention on Twitter, and received pushback from some
who questioned it, a new Twitter account popped up with the name Thomas
Lasch, tweeting that he had worked with Mr. Hickman and remembered the
2013 episode.

Mark Hertling, a retired general who was suspicious of the pair and
contacted them through direct messaging, later tweeted: ``They are who
they say they are.'' But he added that ``LTC Hickman and I agreed to
disagree on LTC Vindman and many other things.''

In an interview, General Hertling, who commanded the United States Army
in Europe, said that a number of things about Mr. Hickman's
recollections did not add up, including his claim of hearing what
Colonel Vindman, who was born in Ukraine, said to Russian soldiers.

``Vindman would've been speaking to Russian soldiers in Russian, not
English,'' he said. ``Russians, when they come to these exercises, they
don't speak English --- they take pride in it.''

General Hertling added: ``I asked Hickman about that, and he said,
`Well, they were going back and forth between Russian and English.'''

An effort to reach Mr. Lasch was unsuccessful. At his home in Homosassa,
Fla., Mr. Hickman said, ``All I want is the truth to get out.''

Within a day of Colonel Vindman's testimony, conservative media figures
on Fox News and elsewhere, as well as Republican surrogates like
\href{https://www.nytimes.com/2019/11/06/us/politics/ukraine-giuliani-charles-gucciardo.html}{Rudolph
W. Giuliani}, raised questions about whether the Ukrainian-born colonel
had ``dual loyalties.'' Some even pushed the innuendo that he could be
some sort of spy for Ukraine.

On Monday, Mr. Hickman's tweets were picked up by a pro-Trump web
magazine, American Greatness, in an article that was promoted by Mark
Levin, a talk radio host and vociferous Trump defender. On Wednesday,
Pete Hegseth, a host on the Fox morning show ``Fox \& Friends'' ---
arguably the president's favorite cable news show --- also tweeted the
article.

``Guys I've served with --- and trust with my life --- served with LTC
Vindman and saw the same thing,'' Mr. Hegseth said. ``He's been a
partisan from the beginning.''

But those who know and have worked with him have provided a different
account. They said that Colonel Vindman, then a military attaché, was
assigned to meet with Russians and gather whatever intelligence he
could.

He spoke to the Russians in Russian, did not denigrate the United States
and reported everything he heard, according to a person briefed on the
episode, speaking on the condition of anonymity because the colonel had
not publicly testified. Colonel Vindman did not have dealings with Mr.
Hickman in relation to his work during the exercise, the person said,
and was not reprimanded for it.

Peter B. Zwack, a retired brigadier general who was Colonel Vindman's
commanding officer during the joint exercise, said he was skeptical of
Mr. Hickman's account.

``If there was something egregious that occurred, believe me, we would
have had our ears rapped in Moscow,'' said General Zwack, who served as
the United States' senior defense official and attaché to Russia.

``The bottom line is, where there are Russians in an exercise in and
among our units and people, we have an attaché that coordinates with
them,'' the general said. ``It's all just a part of an attaché's job.''

Attachés are expected to overtly collect information on what is
happening in the country in which they are posted, as well as collect
information from unsuspecting foreign officials through casual
conversation. The allegations made by Mr. Hickman may simply describe
Colonel Vindman playing his assigned role.

In trying to undermine the impeachment inquiry, the president and his
allies have repeatedly called witnesses ``Never Trumpers,'' spread
rumors and conspiracy theories and described the process as a coup.

In more than 55 tweets over 47 days, The New York Times found, Mr. Trump
claimed falsely that elements of the original whistle-blower's account
had fallen apart or proven incorrect. The president has also encouraged
reporters to reveal the whistle-blower's identity, without doing so
himself.

His son has been less discerning. On Wednesday morning, Donald Trump Jr.
tweeted out an article purporting to name the whistle-blower.

He later denied coordinating with the White House.

Advertisement

\protect\hyperlink{after-bottom}{Continue reading the main story}

\hypertarget{site-index}{%
\subsection{Site Index}\label{site-index}}

\hypertarget{site-information-navigation}{%
\subsection{Site Information
Navigation}\label{site-information-navigation}}

\begin{itemize}
\tightlist
\item
  \href{https://help.nytimes.com/hc/en-us/articles/115014792127-Copyright-notice}{©~2020~The
  New York Times Company}
\end{itemize}

\begin{itemize}
\tightlist
\item
  \href{https://www.nytco.com/}{NYTCo}
\item
  \href{https://help.nytimes.com/hc/en-us/articles/115015385887-Contact-Us}{Contact
  Us}
\item
  \href{https://www.nytco.com/careers/}{Work with us}
\item
  \href{https://nytmediakit.com/}{Advertise}
\item
  \href{http://www.tbrandstudio.com/}{T Brand Studio}
\item
  \href{https://www.nytimes.com/privacy/cookie-policy\#how-do-i-manage-trackers}{Your
  Ad Choices}
\item
  \href{https://www.nytimes.com/privacy}{Privacy}
\item
  \href{https://help.nytimes.com/hc/en-us/articles/115014893428-Terms-of-service}{Terms
  of Service}
\item
  \href{https://help.nytimes.com/hc/en-us/articles/115014893968-Terms-of-sale}{Terms
  of Sale}
\item
  \href{https://spiderbites.nytimes.com}{Site Map}
\item
  \href{https://help.nytimes.com/hc/en-us}{Help}
\item
  \href{https://www.nytimes.com/subscription?campaignId=37WXW}{Subscriptions}
\end{itemize}
