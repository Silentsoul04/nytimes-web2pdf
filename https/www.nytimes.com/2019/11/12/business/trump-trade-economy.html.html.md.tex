Sections

SEARCH

\protect\hyperlink{site-content}{Skip to
content}\protect\hyperlink{site-index}{Skip to site index}

\href{https://www.nytimes.com/section/business}{Business}

\href{https://myaccount.nytimes.com/auth/login?response_type=cookie\&client_id=vi}{}

\href{https://www.nytimes.com/section/todayspaper}{Today's Paper}

\href{/section/business}{Business}\textbar{}Trump Renews Tariff Threat
Against China and Touts U.S. Economic `Boom'

\url{https://nyti.ms/2KhDqAV}

\begin{itemize}
\item
\item
\item
\item
\item
\item
\end{itemize}

Advertisement

\protect\hyperlink{after-top}{Continue reading the main story}

Supported by

\protect\hyperlink{after-sponsor}{Continue reading the main story}

\hypertarget{trump-renews-tariff-threat-against-china-and-touts-us-economic-boom}{%
\section{Trump Renews Tariff Threat Against China and Touts U.S.
Economic
`Boom'}\label{trump-renews-tariff-threat-against-china-and-touts-us-economic-boom}}

President Trump, in an economic address in New York, said his policies
have fueled America's growth but gave little indication that a
breakthrough in China trade talks was imminent.

\includegraphics{https://static01.nyt.com/images/2019/11/12/business/12DC-TRUMPTRADE-02/12DC-TRUMPTRADE-02-videoSixteenByNine3000.jpg}

By \href{https://www.nytimes.com/by/ana-swanson}{Ana Swanson},
\href{https://www.nytimes.com/by/maggie-haberman}{Maggie Haberman} and
\href{https://www.nytimes.com/by/jeanna-smialek}{Jeanna Smialek}

\begin{itemize}
\item
  Nov. 12, 2019
\item
  \begin{itemize}
  \item
  \item
  \item
  \item
  \item
  \item
  \end{itemize}
\end{itemize}

\href{https://cn.nytimes.com/business/20191113/trump-trade-economy/}{阅读简体中文版}\href{https://cn.nytimes.com/business/20191113/trump-trade-economy/zh-hant/}{閱讀繁體中文版}

WASHINGTON --- President Trump on Tuesday said the United States was
``close'' to an interim deal with China and that an agreement ``could
happen soon,'' even as he renewed his threat of additional tariffs.

Mr. Trump's remarks at the Economic Club of New York were likely to
\href{https://www.nytimes.com/2019/10/16/business/china-trade-deal-economy.html}{stoke
still more uncertainty} about how quickly the 19-month trade war will be
resolved. He used the speech to claim credit for an economic ``boom''
while downplaying any negative impact on the American economy from the
tariffs that both sides have placed on each other's goods.

If Beijing doesn't accede to America's trade terms, he said, ``we're
going to substantially raise those tariffs.''

``We will only accept a deal if it's good for the United States,'' he
added.

Washington and Beijing are trying to reach a ``Phase 1'' trade agreement
that would resolve some of the administration's concerns about China's
economic practices and help defuse tensions that have slowed the global
economy.

But Mr. Trump's comments provided little clarity about whether the two
sides were any closer to a deal. Since he announced on Oct. 11 that the
United States had
\href{https://www.nytimes.com/2019/10/11/business/economy/us-china-trade-deal.html}{reached
a preliminary agreement}with the Chinese, the administration has
vacillated between optimistic and pessimistic statements, and the
countries now appear further from signing an agreement than he initially
suggested.

Mr. Trump spent portions of the speech defending his tariffs to a group
that has been among the most skeptical of the import taxes --- New York
business leaders. The president defended his approach, accusing China of
cheating the United States for years and unfairly targeting American
farmers with its tariffs.

He denied that his trade policy had created uncertainty for the economy,
adding that the real cost ``would be if we did nothing.''

The Trump administration has sent mixed signals about whether it would
agree to roll back any of its existing tariffs on China as part of an
initial agreement. Beijing has said easing tariffs is essential to any
agreement and
\href{https://www.nytimes.com/2019/11/07/business/china-trade-trump.html}{administration
officials said last week}they expected tariff relief as part of Phase 1.
But other administration officials have
\href{https://www.nytimes.com/2019/11/08/us/politics/trump-china-tariffs.html}{disputed
those statements} and Mr. Trump himself has further confused things by
saying the United States
\href{https://www.nytimes.com/2019/11/08/us/politics/trump-china-tariffs.html}{has
not agreed to anything --- yet}.

In an interview later on Tuesday, Larry Kudlow, the director of the
National Economic Council, said that it was ``possible'' that tariffs
could be adjusted as part of the deal, but that neither side would agree
to any tariff adjustments until the entire deal was finalized.

``Since there's no formal agreement we can't say,'' he said.

Mr. Kudlow added that China and the United States had made progress on
strengthening China's protections on intellectual property and its
valuation of its currency, but that some other issues might slip into a
second-phase agreement.

In a speech that was heavy on politics, Mr. Trump credited a program of
tax cuts, deregulation and tough trade policy for creating 7 million new
jobs so far in his term and encouraging companies to set up shop in the
United States. If Republicans took back the House and retained control
of the Senate and the White House in 2020, America's prosperity would
continue, the president said.

He predicted the business crowd would vote for him next November.

``The truth is look, you have no choice, because the people we are
running against are crazy,'' he said to laughter. ``O.K.? They are
crazy.''

Speaking to an audience in Manhattan that included his son-in-law Jared
Kushner and Mr. Kudlow, Mr. Trump said he had delivered on his promises
to reverse what he described as a decline in American industry.

``We have ended the war on American workers, we have stopped the assault
on American industry, and we have launched an economic boom the likes of
which we have never seen before,'' Mr. Trump said, to the sound of forks
clinking on plates.

Mr. Trump castigated the Democrats' ``outrageous hoaxes and delusional
witch-hunts'' and talked about the importance of the rule of law, the
day before the first public hearings of an impeachment inquiry that
threatens his presidency.

``We must protect the constitutional rule of law in our country at all
cost,'' Mr. Trump said to applause. He boasted of his record of
appointing more than 160 federal judges, with another 30 or so expected
to clear the Senate in the next two months.

At other points, Mr. Trump criticized President Barack Obama, and he
repeatedly overstated his own economic record.

Mr. Trump has presided over a strong labor market, where unemployment
has fallen to a half-century low. The share of people working or looking
for work has increased, and wages are growing --- although not as
quickly as they did before the 2007 to 2009 recession.

Job growth has slowed somewhat recently, but that is in line with
economists' expectations, given that the United States' economic
expansion is in a record-breaking 11th year. Job growth averaged about
217,000 per month during Mr. Obama's second term, and has averaged about
189,000 in Mr. Trump's tenure through October.

Yet sectors of the economy, particularly manufacturing, have fallen
short of the strong performance that Mr. Trump spoke of on Tuesday.

``Factories and businesses will always find a home,'' the president
said. ``It is up to us to decide whether that home will be in a foreign
country, or right here in our country, our beloved U.S.A., and that's
where we want them to stay and be and move to.''

Manufacturing has slowed sharply this year, with factory activity
contracting in August, September and October, according to data from the
Institute for Supply Management.

Businesses have been slow to buy equipment despite Mr. Trump's corporate
tax cuts, which provided favorable tax treatment for new purchases. Many
businesses anecdotally cite uncertainty related to the administration's
drawn-out trade wars as a cause for their hesitancy. Productivity, which
looked to be trending upward, posted a decline in the third quarter,
data showed last week.

Against that backdrop, and as fiscal stimulus from tax cuts and higher
spending caps fades, economists are projecting slower growth ahead.
Economists in the Blue Chip Economic Indicators survey expect 1.8
percent growth in 2020, down from a 2.3 percent projection in 2019.

In his speech, Mr. Trump blamed any economic woes on the Federal
Reserve, berating central bank officials for keeping interest rates too
high and putting the United States at a competitive disadvantage to
other countries.

After quoting stock gains, he said that ``you could have added another
25 percent'' to the stock market's climb if the Fed had ``worked with
us.''

``But we all make mistakes, don't we?'' he asked.

Mr. Trump made no specific mention of
\href{https://www.nytimes.com/2019/11/11/business/trump-auto-tariffs.html}{a
Wednesday deadline} to decide whether to impose tariffs on automobiles
and auto parts that has particularly hung over trade with Europe.
Advisers have said the tariffs appear unlikely, but Mr. Trump will make
the ultimate decision.

But the president repeated the accusation that the European Union had
imposed ``terrible'' trade barriers on the United States that were ``in
many ways worse than China.'' And he warned that he would impose tariffs
on nations that ``mistreat'' the United States.

The speech sought to burnish his economic record, but the president
often strayed into political territory, hitting ``far-left'' politicians
in Washington, who he said are trying to control the economy from the
``top down.'' That is part of his refrain against the entire Democratic
Party with his general election opponent yet to be determined.

Mr. Trump's preoccupation with popularity --- and comparing himself with
Mr. Obama --- was also on display. At one point he said he had recently
seen a poll that Mr. Obama is more popular than Mr. Trump is in Germany.

``I said, guess what? He should be,'' Mr. Trump said. ``The day I'm more
popular than him, you know I'm not doing my job.''

At another point, Mr. Trump suggested that every foreign leader he meets
with in Washington praises the United States and said he has an
open-door policy to talk to any official who wants to visit the White
House.

``Anybody that wants to come in,'' he said, ``dictators, it's okay.''

He will host Turkey's leader, Recep Tayyip Erdogan, at the White House
on Wednesday.

Maggie Haberman reported from New York.

Advertisement

\protect\hyperlink{after-bottom}{Continue reading the main story}

\hypertarget{site-index}{%
\subsection{Site Index}\label{site-index}}

\hypertarget{site-information-navigation}{%
\subsection{Site Information
Navigation}\label{site-information-navigation}}

\begin{itemize}
\tightlist
\item
  \href{https://help.nytimes.com/hc/en-us/articles/115014792127-Copyright-notice}{©~2020~The
  New York Times Company}
\end{itemize}

\begin{itemize}
\tightlist
\item
  \href{https://www.nytco.com/}{NYTCo}
\item
  \href{https://help.nytimes.com/hc/en-us/articles/115015385887-Contact-Us}{Contact
  Us}
\item
  \href{https://www.nytco.com/careers/}{Work with us}
\item
  \href{https://nytmediakit.com/}{Advertise}
\item
  \href{http://www.tbrandstudio.com/}{T Brand Studio}
\item
  \href{https://www.nytimes.com/privacy/cookie-policy\#how-do-i-manage-trackers}{Your
  Ad Choices}
\item
  \href{https://www.nytimes.com/privacy}{Privacy}
\item
  \href{https://help.nytimes.com/hc/en-us/articles/115014893428-Terms-of-service}{Terms
  of Service}
\item
  \href{https://help.nytimes.com/hc/en-us/articles/115014893968-Terms-of-sale}{Terms
  of Sale}
\item
  \href{https://spiderbites.nytimes.com}{Site Map}
\item
  \href{https://help.nytimes.com/hc/en-us}{Help}
\item
  \href{https://www.nytimes.com/subscription?campaignId=37WXW}{Subscriptions}
\end{itemize}
