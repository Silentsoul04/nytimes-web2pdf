Sections

SEARCH

\protect\hyperlink{site-content}{Skip to
content}\protect\hyperlink{site-index}{Skip to site index}

\href{https://www.nytimes.com/section/business}{Business}

\href{https://myaccount.nytimes.com/auth/login?response_type=cookie\&client_id=vi}{}

\href{https://www.nytimes.com/section/todayspaper}{Today's Paper}

\href{/section/business}{Business}\textbar{}U.S. Said to Extend Reprieve
for Huawei

\url{https://nyti.ms/376z6yp}

\begin{itemize}
\item
\item
\item
\item
\item
\end{itemize}

Advertisement

\protect\hyperlink{after-top}{Continue reading the main story}

Supported by

\protect\hyperlink{after-sponsor}{Continue reading the main story}

\hypertarget{us-said-to-extend-reprieve-for-huawei}{%
\section{U.S. Said to Extend Reprieve for
Huawei}\label{us-said-to-extend-reprieve-for-huawei}}

The Trump administration is set to extend a license that will allow
American companies to continue doing business with the Chinese telecom
giant Huawei.

\includegraphics{https://static01.nyt.com/images/2019/11/15/business/15dc-huawei1/15dc-huawei1-articleLarge.jpg?quality=75\&auto=webp\&disable=upscale}

\href{https://www.nytimes.com/by/ana-swanson}{\includegraphics{https://static01.nyt.com/images/2018/12/10/multimedia/author-ana-swanson/author-ana-swanson-thumbLarge.png}}

By \href{https://www.nytimes.com/by/ana-swanson}{Ana Swanson}

\begin{itemize}
\item
  Nov. 15, 2019
\item
  \begin{itemize}
  \item
  \item
  \item
  \item
  \item
  \end{itemize}
\end{itemize}

WASHINGTON --- The Trump administration is set to once again
\href{https://www.nytimes.com/2019/08/19/us/politics/huawei-trump.html}{extend
a license} that will allow American companies to continue doing business
with Huawei, the Chinese telecom giant, people familiar with the
deliberations said.

The fate of Huawei has hung in the balance for many months, as the Trump
administration has deliberated over how to treat a company many American
officials consider a national security risk, but the Chinese government
views as central to its technology ambitions. While the company's future
is not technically a part of trade talks between the two countries,
President Trump has brought Huawei up as a potential bargaining chip in
a long-running trade war.

In May, the Commerce Department placed Huawei, which constructs advanced
5G networks that will be central to the next generation of wireless
communication, on a blacklist that banned the firm from buying American
products without government approval.

The ban posed problems for rural telecommunications companies in the
United States, which rely on Huawei for parts and equipment as well as
American companies that depend on selling to the Chinese firm. To give
them time to adjust to the new order, the Commerce Department issued a
general reprieve that allowed companies to continue to do business with
Huawei for a short time.

That reprieve is set to expire on Monday, but the administration is
expected to extend it for a period of time. It would be the third time
an extension is granted. However, the decision could change given
continuing trade discussions between the United States and China.

Chinese negotiators have repeatedly pressed their American counterparts
to lift sanctions on Huawei, and are likely to view a temporary reprieve
as a good will gesture in the trade talks, which are at a critical
point.

Washington and Beijing are trying to reach a ``Phase 1'' trade agreement
that would resolve some of the administration's concerns about China's
economic practices. Mr. Trump announced last month that the United
States had
\href{https://www.nytimes.com/2019/10/11/business/economy/us-china-trade-deal.html?module=inline}{reached
a preliminary agreement}with the Chinese. But in recent weeks, the
countries now appear further from signing an agreement than he initially
suggested.

The administration is also under pressure from companies that sell
components to Huawei, and the telecommunications companies that buy
Huawei equipment. In an interview on Monday with Fox Business Network,
the commerce secretary, Wilbur Ross, said past reprieves were intended
to allow rural telecom companies in the United States ``to continue to
function.''

``They unfortunately are very dependent on Huawei for 3G and 4G,'' he
said. ``There are enough problems with telephone service in the rural
communities; we don't want to knock them out. So, one of the main
purposes of the temporary general licenses is to let those rural guys
continue to operate.''

Huawei declined to comment. The news of the license's extension
\href{https://www.politico.com/news/2019/11/14/huawei-trade-waiver-070982}{was
first reported} by Politico.

The Trump administration is also separately considering product-specific
licenses that would allow select companies to supply nonsensitive goods
to Huawei, despite the blanket ban.

People familiar with the matter said Mr. Trump
\href{https://www.nytimes.com/2019/10/09/us/politics/trump-huawei-trade.html}{had
given the green light in early October} for those licenses to be
approved. But progress toward a Phase 1 trade deal between the United
States and China stalled in subsequent weeks, and those licenses were
not issued.

Their timing is now the subject of internal debate in the Trump
administration, and likely contingent on China and the United States
signing
\href{https://www.nytimes.com/2019/11/12/business/trump-trade-economy.html}{a
Phase 1 trade deal that is still uncertain}.

The Federal Communications Commission is expected to vote next week on a
measure that would ban companies from spending federal subsidies on
equipment made by Huawei or ZTE, another Chinese telecom company.
Carriers serving rural areas, which have bought Huawei gear because it
is cheaper than non-Chinese alternatives, have
\href{https://www.nytimes.com/2019/05/25/technology/huawei-rural-wireless-service.html?module=inline}{said
they are worried} that the proposal will harm their businesses.

Ajit Pai, the commission's chairman, said in a statement last month that
as providers put into place the next generation of wireless technology,
the agency could not ignore the chance ``that the Chinese government
will seek to exploit network vulnerabilities in order to engage in
espionage.''

When asked last month about Huawei's relationship to the Phase 1 trade
deal he had announced with China, Mr. Trump said, ``We haven't discussed
Huawei, relative to this deal.'' A few minutes later, he added: ``Well,
we're going to see what happens. We're going to be discussing that in
Phase 2.''

David McCabe contributed reporting.

Advertisement

\protect\hyperlink{after-bottom}{Continue reading the main story}

\hypertarget{site-index}{%
\subsection{Site Index}\label{site-index}}

\hypertarget{site-information-navigation}{%
\subsection{Site Information
Navigation}\label{site-information-navigation}}

\begin{itemize}
\tightlist
\item
  \href{https://help.nytimes.com/hc/en-us/articles/115014792127-Copyright-notice}{©~2020~The
  New York Times Company}
\end{itemize}

\begin{itemize}
\tightlist
\item
  \href{https://www.nytco.com/}{NYTCo}
\item
  \href{https://help.nytimes.com/hc/en-us/articles/115015385887-Contact-Us}{Contact
  Us}
\item
  \href{https://www.nytco.com/careers/}{Work with us}
\item
  \href{https://nytmediakit.com/}{Advertise}
\item
  \href{http://www.tbrandstudio.com/}{T Brand Studio}
\item
  \href{https://www.nytimes.com/privacy/cookie-policy\#how-do-i-manage-trackers}{Your
  Ad Choices}
\item
  \href{https://www.nytimes.com/privacy}{Privacy}
\item
  \href{https://help.nytimes.com/hc/en-us/articles/115014893428-Terms-of-service}{Terms
  of Service}
\item
  \href{https://help.nytimes.com/hc/en-us/articles/115014893968-Terms-of-sale}{Terms
  of Sale}
\item
  \href{https://spiderbites.nytimes.com}{Site Map}
\item
  \href{https://help.nytimes.com/hc/en-us}{Help}
\item
  \href{https://www.nytimes.com/subscription?campaignId=37WXW}{Subscriptions}
\end{itemize}
