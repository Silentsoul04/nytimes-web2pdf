An Artist Who Makes Paintings Without a Canvas

\href{https://nyti.ms/2KAHg8z}{https://nyti.ms/2KAHg8z}

\begin{itemize}
\item
\item
\item
\item
\item
\end{itemize}

\includegraphics{https://static01.nyt.com/images/2019/11/19/t-magazine/19tmag-jackson-slide-USMT/19tmag-jackson-slide-USMT-articleLarge.jpg?quality=75\&auto=webp\&disable=upscale}

Sections

\protect\hyperlink{site-content}{Skip to
content}\protect\hyperlink{site-index}{Skip to site index}

Artist's Questionnaire

\hypertarget{an-artist-who-makes-paintings-without-a-canvas}{%
\section{An Artist Who Makes Paintings Without a
Canvas}\label{an-artist-who-makes-paintings-without-a-canvas}}

Ahead of a new solo show, Suzanne Jackson talks about her creative
routine, her love of jazz music and the worst studio she ever had.

The painter Suzanne Jackson at her home in Savannah, Ga.Credit...Peter
Frank Edwards

Supported by

\protect\hyperlink{after-sponsor}{Continue reading the main story}

By Julia Felsenthal

\begin{itemize}
\item
  Nov. 19, 2019
\item
  \begin{itemize}
  \item
  \item
  \item
  \item
  \item
  \end{itemize}
\end{itemize}

\href{http://www.suzannefjackson.net/}{Suzanne Jackson} isn't fond of
the term ``overachiever,'' though you wouldn't be wrong to call her one.
At 75, she has had a long, storied, multi-hyphenate career as a painter,
poet, dancer, teacher, curator and theater designer. She's not a fan of
the word ``career'' either: ``It's my life's work,'' she says. ``There's
just a lot of things to be interested in.'' Jackson traces that attitude
back to her childhood in 1940s and '50s pre-statehood Alaska, where a
certain pioneering spirit prevailed. ``We just \emph{did} things,'' she
remembers. She went to college at San Francisco State University at 17,
studying art, drama and dance; toured South America as a ballerina; and
in 1967, **** moved to Los Angeles, where she tooled around town in a
Buick Hearse, took drawing classes from
\href{https://www.nytimes.com/2018/09/28/t-magazine/art/charles-white-moma-retrospective.html}{Charles
White}, began showing her paintings at the influential Ankrum gallery
and, in 1968, opened Gallery 32, the community-minded space she ran out
of her studio near MacArthur Park for two years. There, she hosted
exhibitions by emerging black artists like
\href{https://www.nytimes.com/2018/02/09/t-magazine/art/steve-cannon-david-hammons.html}{David
Hammons} and
\href{https://www.nytimes.com/2019/09/04/arts/design/betye-saar.html}{Betye
Saar}, as well as a fund-raiser for the Black Panthers. As a single
mother in her 40s with her son in tow, she earned a graduate degree in
theater design at Yale, worked on productions with the Kennedy Center
and the Berkeley Repertory Theater and eventually settled in Georgia in
the mid-90s to teach at the Savannah College of Art and Design. She has
now lived and worked in the city for two decades, in a rambling,
three-story, double-wide 1890s house that she owns in the historic
Starland district.

Though she has made paintings since she was a child, and exhibited since
the late 1960s, Jackson has had something of a banner year. In June, she
mounted a career survey at Savannah's
\href{https://www.telfair.org/}{Telfair Museums}; in September, she
\href{https://joanmitchellfoundation.org/artist-programs/artist-grants/painter-sculptors/2019/suzanne-jackson}{won
a grant} from the Joan Mitchell Foundation; and this month, she opens a
\href{http://www.ortuzarprojects.com/exhibitions/suzanne-jackson/works?view=slider}{solo
show} at Ortuzar Projects in Lower Manhattan, an exhibition that focuses
on her boundary-pushing recent work: otherworldly, dimensional paintings
composed entirely of acrylic paint --- with no canvas beneath ---
embedded with bits of household detritus and personal ephemera.

Image

Jackson's recent works, like ``Turtle, Phoenix, Pleasures'' (2019), are
made from pure, layered acrylic, with no canvas underneath. Here, she's
embedded bag netting and wood into the paint.Credit...Peter Frank
Edwards

Image

Jackson often begins a piece on a plastic covered table, then peels the
acrylic off and hangs it to dry.Credit...Peter Frank Edwards

Jackson first began using acrylics during her years in Los Angeles,
after her car was broken into and her oil paints stolen. She initially
**** deployed it almost like watercolor, setting down layer upon layer
of washy pigment to build up dreamy images of black figures commingling
with birds, flowers and hearts. In more recent decades, her art has
become abstract and more driven by materials. During the years she
worked as a theater designer, Jackson began incorporating discarded
bogus paper --- the sheeting used to protect a stage while painting sets
--- into her increasingly textural surfaces. Then it was leftover deer
netting from her garden, ballet netting from her costume designs and
produce bags and wood salvaged from renovating her house. Eventually she
figured out that she could put paint down directly onto a table covered
in plastic, then peel it up and hang the drying film as her canvas,
allowing her to paint acrylic straight onto acrylic. The result, which
looks delicate but is not --- ``you can kick it, stomp on it, it's not
going to be harmed,'' says Jackson --- blurs the line between painting
and sculpture. Upcycling remains central to her process. She even peels
the paint from her hands and stores the dried flakes for future use. In
this way, her synthetic medium ``becomes organic,'' she explains,
``because I'm reintegrating paint that would go into nature and destroy
it.''

When we speak over the phone in late October, Jackson has just come
inside after trying to help a butterfly with a broken wing that got
caught in her screen door. She bemoans the gentrification that's
changing her leafy neighborhood, the interlopers who seem intent on
cutting down trees and installing newfangled businesses, like a
shipping-container food court that, she says, ``looks like a prison.''
(Not all the neighbors are so terrible: At the brewery across the
street, the proprietors named a beer in her honor --- ``Ms. Suzanne,'' a
Guinness-like concoction that's best served in a wine glass.) Her own
home sits on three lots, her backyard lush with pomegranate trees, grape
vines, woodpeckers, turtles, feral cats, possums, raccoons and snakes.
``Everyone wants it, but they're not going to have it,'' Jackson says of
the property. She chuckles. ``That's just the way it is.'' Sitting in
one of her studio rooms on the west side of the building, her 8-year-old
Siamese, Lexi, on her lap, she answers T's artist's questionnaire.

Image

Jackson's work is inspired by her love of the natural world. Here,
``Baby Bogus'' (2005) hangs above a table full of dried leaves and
feathers.Credit...Peter Frank Edwards

Image

The recent piece ``Temporarily Untitled, Veils'' (2019) is made of
acrylic paint with a wire armature.Credit...Peter Frank Edwards

\textbf{What is your day like? How much do you sleep? What is your work
schedule?}

Last night, I didn't go to bed until 2. Sometimes I can wake up at 4. I
listen to NPR until 8 in the morning. My bedroom is on the second floor,
so I come downstairs and through the studio to see what I'm doing. And I
may end up climbing a ladder to work on something before I get around to
the kitchen to have a cup of coffee or some breakfast. Then I have to
feed all the cats. I come back into the studio and fiddle around some
more before I get dressed. I work less now at night. My neighborhood is
very noisy at night, almost like a circus with all these clubs and
parties.

\textbf{How many hours of creative work do you do in a day?}

All day long. If I'm not physically putting paint on something, I'm
writing or reading, thinking about it. And even when I'm supposed to be
sleeping, I'm thinking about what I'm going to do, how I'm going to
accomplish a structural idea, what should the title be, what's next. And
then, because I don't have assistants, I'm also having to think about
the calendar, the schedule of things going on. Sometimes I think I
really \emph{should} have an assistant, but I make my work so that I can
get up on the ladder and take it down. I believe my hand should be in
the work, and not somebody else's, unless I want to share the credit
with them.

\textbf{What is the worst studio you ever had?}

Physically the worst studio might be my very first one, for \$40 a month
on Temple Street in Los Angeles. It was like a storefront on the front
of a beautiful Victorian house. I had to put a parachute over my bed
because there were holes in the floor from the house upstairs, and the
kids would throw little pebbles down. I just remember my mother and
father sitting there, my dad in his suit, my mother in her nice little
dress, like, ``What has happened to our daughter?'' To me, that wasn't a
bad studio. It was my first studio, and it was really wonderful.

\textbf{What is the first work you ever sold, and for how much?}

That was a piece that I sold at the Laguna Beach art museum in 1968 ****
for \$300. I think it was called ``Gypsy Girl.'' It was a watercolor. I
have a feeling that when they had the fires in Laguna Beach, that
painting could have been destroyed. I don't even know who bought it.

\textbf{When you start a new piece, where do you begin? What is the
first step?}

Sometimes I'm working on three or four things at a time. One or two
pieces may be drawing, and then I'm working on something else. Sometimes
it's just, put down the brush stroke or a big palette knife on
something, then see what happens. Walk away and come back. Try to be
focused, and then try to be unfocused.

Image

Jackson pictured beneath ``Nine, Billie, Mingus, Monk's''
(2003).Credit...Peter Frank Edwards

Image

A detail shot of Jackson's studio includes her drawing materials. Going
to school for theater design made her a better draftswoman. ``That's
another element I've been really fascinated by,'' she says, ``how to
achieve this beautiful line.''Credit...Peter Frank Edwards

\textbf{How do you know when you're done with a piece?}

With these new pieces that are pure acrylic, I watch to see where the
stress is. Maybe there are places where the paint is thinner, or the
weight of the paint may pull. Sometimes I'll go back, but usually the
piece just tells me that it doesn't want to be touched anymore.

\textbf{What music do you play when you're making art?}

Lots of jazz. I used to start with
\href{https://www.nytimes.com/topic/person/yoyo-ma}{Yo-Yo Ma} in the
morning, and then it would evolve into jazz, and then maybe by 3 in the
morning it would be
\href{https://www.nytimes.com/topic/person/jimi-hendrix}{Jimi Hendrix}.
I met {[}the Savannah radio veteran{]} Ike Carter, and in 2013, we
started this group, bringing in music that we like for a radio program
{[}``Listen Hear,'' hosted by Savannah State University Radio{]}. He's
kind of a maestro of blues and African-American classical music.

\textbf{Is there a meal you eat on repeat when you're working?}

My lazy food is a veggie burger. I like really grainy Ezekiel or spelt
bread, and then I put on tomatoes and lettuce. I grew up with Miracle
Whip instead of mayo, and the strongest grainy mustard. That's my fast
food. With a glass of wine, maybe.

\textbf{Are you bingeing on any shows right now?}

I love
``\href{https://www.nytimes.com/watching/recommendations/poldark}{Poldark}.''
That's my Sunday night splurge.

\textbf{What is the weirdest object in your studio?}

Probably me? This is an odd building. On the third floor on the east
side there's a sign that says ``This Room is Haunt'' {[}sic{]}. One day,
this man came by in a truck and said, ``I used to live in that house
when I was a little boy, and we wrote a sign because we thought that
room was haunted.'' So that's left over.

\textbf{What's the last thing that made you cry?}

I was at the dentist the morning that the Joan Mitchell Foundation
called. When I got home, there was an email from them. I'm so used to
getting rejection notes. When I opened it up, I couldn't believe it. I
think I just said, ``Oh my God,'' and broke out in tears. Before that,
the last time I really cried was when my son passed away unexpectedly
{[}in 2016, of congestive heart failure{]}. Then I couldn't cry very
long because all his friends were there, and I had to console them. But
I broke out in tears when I realized I had actually, for the first time
in my life, won a real grant with money attached. I just sat there
crying all by myself.

\textbf{What do you bulk buy with most frequency?}

As soon as I got my grant money, I ordered five gallon buckets of
acrylic medium from Nova Color in California. I also buy 60 pounds of
cat food on a regular schedule. And bird feed. I have subscriptions for
that, and I have a subscription for wine. That's my other thing.

\textbf{Do you exercise, other than climbing up and down ladders?}

Well, I have a lot of steps to run up and down. I \emph{was} a dancer,
and I was really toned. I think the last time I danced was when I taught
modern dance in 1994. People don't seem to dance anymore, even social
dancing. They just stand and hoot and holler. They don't move. I can't
forgive myself for not being as toned as I was when I was younger, when
I weighed 105 pounds and hadn't had a baby. It's part of being a woman:
You grow up and you have a body. I've become less unforgiving about it,
but I'm still always trying to breathe in. Sometimes in the morning when
I wake up I try to stretch in bed, which is cheating.

\textbf{What are you reading right now?}

Aberjhani, who wrote a poem for my Telfair catalog, has a book called
``\href{https://www.author-poet-aberjhani.info/dreams-of-the-immortal-city-savannah}{Dreams
of the Immortal City Savannah}\emph{.''} That's the one I've read
recently. I just bought this big volume of George Herriman's ``Krazy
Kat.'' I said I was not going to buy any more books at my age, but I had
to have that one, because that's how I learned to read. And I think the
social commentary may have subconsciously influenced my life.

\textbf{What's your favorite artwork by someone else}?

Oh, that's so hard! I love Mary Lovelace O'Neal's paintings.
\href{https://www.nytimes.com/2013/08/18/arts/design/ruth-asawa-an-artist-who-wove-wire-dies-at-87.html}{Ruth
Asawa} is another of my favorites. We were on the California Arts
Council together.
\href{https://www.lehmannmaupin.com/artists/mary-corse}{Mary Corse}. Her
work is so subtle. \href{http://sengasenga.com/}{Senga Nengudi}. It's
really hard. You know the drawing that really affected me when I first
saw it? The one by Charles White of a woman with books spread out on the
table. Of course he was my teacher. I love
\href{https://www.moma.org/artists/670}{Lee Bontecou}'s work. And
\href{https://americanart.si.edu/artist/augusta-savage-4269}{Augusta
Savage}. There are just so many lovely things in the world, and people
who help you to think and see. You can't just choose one.

\emph{This interview has been condensed and edited.}

``Suzanne Jackson: News!'' is on view from Nov. 21, 2019, through Jan.
25, 2020, at Ortuzar Projects, 9 White Street, New York,
\href{http://www.ortuzarprojects.com/exhibitions/suzanne-jackson}{ortuzarprojects.com}.

Advertisement

\protect\hyperlink{after-bottom}{Continue reading the main story}

\hypertarget{site-index}{%
\subsection{Site Index}\label{site-index}}

\hypertarget{site-information-navigation}{%
\subsection{Site Information
Navigation}\label{site-information-navigation}}

\begin{itemize}
\tightlist
\item
  \href{https://help.nytimes.com/hc/en-us/articles/115014792127-Copyright-notice}{©~2020~The
  New York Times Company}
\end{itemize}

\begin{itemize}
\tightlist
\item
  \href{https://www.nytco.com/}{NYTCo}
\item
  \href{https://help.nytimes.com/hc/en-us/articles/115015385887-Contact-Us}{Contact
  Us}
\item
  \href{https://www.nytco.com/careers/}{Work with us}
\item
  \href{https://nytmediakit.com/}{Advertise}
\item
  \href{http://www.tbrandstudio.com/}{T Brand Studio}
\item
  \href{https://www.nytimes.com/privacy/cookie-policy\#how-do-i-manage-trackers}{Your
  Ad Choices}
\item
  \href{https://www.nytimes.com/privacy}{Privacy}
\item
  \href{https://help.nytimes.com/hc/en-us/articles/115014893428-Terms-of-service}{Terms
  of Service}
\item
  \href{https://help.nytimes.com/hc/en-us/articles/115014893968-Terms-of-sale}{Terms
  of Sale}
\item
  \href{https://spiderbites.nytimes.com}{Site Map}
\item
  \href{https://help.nytimes.com/hc/en-us}{Help}
\item
  \href{https://www.nytimes.com/subscription?campaignId=37WXW}{Subscriptions}
\end{itemize}
