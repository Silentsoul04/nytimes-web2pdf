Sections

SEARCH

\protect\hyperlink{site-content}{Skip to
content}\protect\hyperlink{site-index}{Skip to site index}

\href{https://www.nytimes.com/section/technology}{Technology}

\href{https://myaccount.nytimes.com/auth/login?response_type=cookie\&client_id=vi}{}

\href{https://www.nytimes.com/section/todayspaper}{Today's Paper}

\href{/section/technology}{Technology}\textbar{}Building a World Where
Data Privacy Exists Online

\url{https://nyti.ms/2NOwa1G}

\begin{itemize}
\item
\item
\item
\item
\item
\end{itemize}

Advertisement

\protect\hyperlink{after-top}{Continue reading the main story}

Supported by

\protect\hyperlink{after-sponsor}{Continue reading the main story}

\hypertarget{building-a-world-where-data-privacy-exists-online}{%
\section{Building a World Where Data Privacy Exists
Online}\label{building-a-world-where-data-privacy-exists-online}}

Dawn Song, an expert in computer security and trustworthy artificial
intelligence, is working on making that vision a reality.

\includegraphics{https://static01.nyt.com/images/2019/11/20/multimedia/20sp-women-song-1/merlin_163691466_e9950b00-e8da-4d87-a704-514a4c31dfd1-articleLarge.jpg?quality=75\&auto=webp\&disable=upscale}

\href{https://www.nytimes.com/by/craig-s-smith}{\includegraphics{https://static01.nyt.com/images/2017/05/25/world/craig-s-smith/craig-s-smith-thumbLarge.jpg}}

By \href{https://www.nytimes.com/by/craig-s-smith}{Craig S. Smith}

\begin{itemize}
\item
  Published Nov. 10, 2019Updated Nov. 19, 2019
\item
  \begin{itemize}
  \item
  \item
  \item
  \item
  \item
  \end{itemize}
\end{itemize}

\emph{This article is part of our}
\href{https://www.nytimes.com/spotlight/women-and-leadership}{\emph{Women
and Leadership special section}}\emph{, which focuses on approaches
taken by women, minorities or other disadvantaged groups challenging
traditional ways of thinking.}

\href{https://www.nytimes.com/interactive/2019/09/23/opinion/data-privacy-jaron-lanier.html}{Data
is valuable} --- something that companies like Facebook, Google and
Amazon realized far earlier than most consumers did. But computer
scientists have been working on alternative models, even as the public
has grown weary of having their data used and abused.

Dawn Song, a professor at the University of California, Berkeley, and
one of the world's foremost experts in computer security and trustworthy
artificial intelligence, envisions a new paradigm in which people
control their data and are compensated for its use by corporations.
While there have been many proposals for such a system, Professor Song
is one actually building the platform to make it a reality.

``As we talk about data as the new oil, it's particularly important to
develop technologies that can utilize data in a privacy-preserving
way,'' Professor Song said recently from her San Francisco office with
an expansive view of the bay.

It is an unlikely trajectory for Professor Song, who grew up in Dalian,
China, a seaport in the northeastern province of Liaoning. She is the
daughter of two local civil servants with no background in computers.
And while she was an exceptional student in high school, she dreamed of
being a National Geographic-style nature photographer. One of her
teachers, a mentor, gently dissuaded her.

Her mother wanted her to study business and filled out an application on
her behalf for a well-known business school. Then, shortly before the
national college entrance exams, her mentor intervened again, convincing
her mother that a brighter future lay ahead for her daughter in science.
Professor Song applied instead to Tsinghua University, China's top
science university, to study physics. She went on to study physics at
Cornell University but transferred to Carnegie Mellon University, where
she received an M.S. in computer science before settling at Berkeley to
finally finish her Ph.D. in computer science. By then, she was focused
on computer security.

Professor Song drew attention while still a graduate student at Berkeley
with pioneering work that showed a machine-learning algorithm can infer
what someone is typing from the timing of their keystrokes picked up by
eavesdropping on a network. Since then, she has been at the forefront of
trustworthy A.I., including improving the resilience of machine-learning
models themselves, the recursive blocks of computer code that learn to
recognize patterns in the data they consume.

\includegraphics{https://static01.nyt.com/images/2019/11/20/multimedia/20sp-women-song-3/merlin_163691523_af6129e7-4d70-47ec-823c-c6aa537f5196-articleLarge.jpg?quality=75\&auto=webp\&disable=upscale}

Machine-learning models, as amazing as they are at identifying
everything from tumors in X-ray images to words in slurred speech,
remain disturbingly easy to fool. Professor Song and her students were
the first ones to demonstrate that computer-vision systems could be
fooled into identifying a stop sign as a 40-miles-per-hour speed limit
sign simply by applying a few innocuous stickers to the sign. Examples
of these altered traffic signs have been on exhibit at London's Science
Museum.

``Her work on the stop sign was among the first to craft adversarial
examples in the physical domain rather than just manipulating image
pixels on a computer,'' said Battista Biggio, an assistant professor at
Italy's University of Cagliari and one of the first people to study the
vulnerabilities of such systems.

Professor Song, who has taught at Berkeley for a dozen years, has been
working to develop techniques and systems that not only can provide
security to computer systems, but also privacy. She envisions a world of
secure networks where individuals control their personal data and even
derive income from it. She compares the world today to a time in human
history when people did not have a clear notion of property rights. Once
those rights were institutionalized and protected, she notes, it helped
revolutionize economies.

She recently started a company, \href{https://www.oasislabs.com/}{Oasis
Labs}, that is building a platform that can give people the ability to
control their data and audit how it is used. She believes that once data
is viewed as property, it can propel the global economy in ways unseen
before. ``New business models can be built on this,'' she said.

Data, of course, is not like a physical object. If a person gives a
friend an apple, then someone else cannot have that apple. But data is
different, with a property that scientists call nonrivalry. People can
give (or sell) as many copies as they want.

Most people give away their data, signing it over to companies by
clicking ``accept,'' not even bothering to read the fine print. Either
people online accept the terms and participate in the digital world or
they unplug --- something that is not really an option for anyone
operating in the global economy. Fortunes were built on that data,
enriching\href{https://www.entrepreneur.com/article/319952}{a handful of
entrepreneurs}.

``Our data has never been more at risk, and our need for new kinds of
robust privacy solutions has never been greater,'' said Guy Zyskind,
co-founder and chief executive of \href{https://enigma.co/}{Enigma},
another company building a decentralized private computation protocol.

When people go online, data is collected and stored on centralized
servers that are vulnerable to attack. But Professor Song and her
colleagues believe that by marrying specialized computer chips and
blockchain technology, they can build a system that provides greater
scalability and privacy protection.

Image

Professor Song with Bennet Yee, left, and Rebekah Kim, right, both
software engineers at Oasis Labs in San Francisco.Credit...Jason Henry
for The New York Times

Some computer chips --- those in most cellphones, for example ---
already incorporate a secure zone, called a trusted execution
environment, that protects software from most kinds of attack. Professor
Song's group is working on enhancing the security of those zones by
building an open-source secure enclave,
\href{https://keystone-enclave.org/}{Keystone}. Within the secure
enclave, bits of computer code, called smart contracts, allow data
owners to control who has access to their data and how it is used.

``You can actually have the integrity that the blockchain ledger
provides and also you can have privacy or confidentiality for the smart
contract execution that's provided by the secure enclave,'' said
Professor Song, who speaks rapidly as if rushing to keep pace with her
thoughts. ``No central server ever sees the data.''

Oasis Labs has been building a platform to support enterprises and
developers. They have begun a pilot with
\href{https://nebula.org/}{Nebula Genomics}, a direct-to-consumer
gene-sequencing company, that offers whole genome sequencing reports on
ancestry, wellness, and genetic traits with weekly updates. Using Oasis
Labs' privacy-preserving tools, Nebula customers will retain full
control and ownership over their genomic data, while enabling Nebula to
run specific analysis on the data without exposing the underlying
information.

Another application, called \href{https://kara.cloud/\#/}{Kara}, a
collaboration with Dr. Robert Chang at the Stanford University School of
Medicine, gives eye patients the option to share retina scans and other
medical data with researchers who use the data to train machine-learning
models to recognize disease.

Part of the Kara project is studying what kind of incentives patients
will find meaningful in return for contributing their data for medical
research.

``Her approach is unique from other data aggregators,'' Dr. Chang said.
``This project is really asking the important question --- who really
owns the data?''

Someday, Professor Song believes, people will have an individual revenue
stream from their data. It may not be significant on a monthly or even
annual basis, but the fees that accumulate over the course of a lifetime
from companies using personal data could contribute to retirement
savings, for example. Or revenue from groups of people could be used to
fund particular causes. The unlocking of data, meanwhile, could lead to
improved services for consumers.

``Today, companies are taking users' data and essentially using it as a
product; they monetize it,'' Professor Song said. ``The world can be
very different if this is turned around and users maintain control of
the data and get revenue from it.''

Craig S. Smith is a former correspondent for The Times and now hosts the
podcast \href{https://www.eye-on.ai}{Eye on A.I.}

Advertisement

\protect\hyperlink{after-bottom}{Continue reading the main story}

\hypertarget{site-index}{%
\subsection{Site Index}\label{site-index}}

\hypertarget{site-information-navigation}{%
\subsection{Site Information
Navigation}\label{site-information-navigation}}

\begin{itemize}
\tightlist
\item
  \href{https://help.nytimes.com/hc/en-us/articles/115014792127-Copyright-notice}{©~2020~The
  New York Times Company}
\end{itemize}

\begin{itemize}
\tightlist
\item
  \href{https://www.nytco.com/}{NYTCo}
\item
  \href{https://help.nytimes.com/hc/en-us/articles/115015385887-Contact-Us}{Contact
  Us}
\item
  \href{https://www.nytco.com/careers/}{Work with us}
\item
  \href{https://nytmediakit.com/}{Advertise}
\item
  \href{http://www.tbrandstudio.com/}{T Brand Studio}
\item
  \href{https://www.nytimes.com/privacy/cookie-policy\#how-do-i-manage-trackers}{Your
  Ad Choices}
\item
  \href{https://www.nytimes.com/privacy}{Privacy}
\item
  \href{https://help.nytimes.com/hc/en-us/articles/115014893428-Terms-of-service}{Terms
  of Service}
\item
  \href{https://help.nytimes.com/hc/en-us/articles/115014893968-Terms-of-sale}{Terms
  of Sale}
\item
  \href{https://spiderbites.nytimes.com}{Site Map}
\item
  \href{https://help.nytimes.com/hc/en-us}{Help}
\item
  \href{https://www.nytimes.com/subscription?campaignId=37WXW}{Subscriptions}
\end{itemize}
