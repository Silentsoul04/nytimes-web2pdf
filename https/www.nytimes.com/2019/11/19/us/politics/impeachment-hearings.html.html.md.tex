Sections

SEARCH

\protect\hyperlink{site-content}{Skip to
content}\protect\hyperlink{site-index}{Skip to site index}

\href{https://www.nytimes.com/section/politics}{Politics}

\href{https://myaccount.nytimes.com/auth/login?response_type=cookie\&client_id=vi}{}

\href{https://www.nytimes.com/section/todayspaper}{Today's Paper}

\href{/section/politics}{Politics}\textbar{}Key Moments from the
Impeachment Inquiry Hearing: Vindman, Williams, Morrison and Volker
Testify

\url{https://nyti.ms/2O0bp2W}

\begin{itemize}
\item
\item
\item
\item
\item
\item
\end{itemize}

Advertisement

\protect\hyperlink{after-top}{Continue reading the main story}

Supported by

\protect\hyperlink{after-sponsor}{Continue reading the main story}

\hypertarget{key-moments-from-the-impeachment-inquiry-hearing-vindman-williams-morrison-and-volker-testify}{%
\section{Key Moments from the Impeachment Inquiry Hearing: Vindman,
Williams, Morrison and Volker
Testify}\label{key-moments-from-the-impeachment-inquiry-hearing-vindman-williams-morrison-and-volker-testify}}

The White House pushed back on a top aide to the vice president who
testified that Mr. Trump's conversation with Ukraine's president was
inappropriate.

\href{https://www.nytimes.com/by/peter-baker}{\includegraphics{https://static01.nyt.com/images/2018/06/13/multimedia/peter-baker/peter-baker-thumbLarge-v2.png}}\href{https://www.nytimes.com/by/michael-d-shear}{\includegraphics{https://static01.nyt.com/images/2018/06/13/multimedia/author-michael-d-shear/author-michael-d-shear-thumbLarge-v2.png}}

By \href{https://www.nytimes.com/by/peter-baker}{Peter Baker} and
\href{https://www.nytimes.com/by/michael-d-shear}{Michael D. Shear}

\begin{itemize}
\item
  Published Nov. 19, 2019Updated Nov. 21, 2019
\item
  \begin{itemize}
  \item
  \item
  \item
  \item
  \item
  \item
  \end{itemize}
\end{itemize}

\includegraphics{https://static01.nyt.com/images/2019/11/19/us/politics/19dc-impeach-hilights-sub/19dc-impeach-hilights-sub-videoSixteenByNine3000-v4.jpg}

\hypertarget{heres-what-you-need-to-know}{%
\subsubsection{Here's what you need to
know:}\label{heres-what-you-need-to-know}}

\begin{itemize}
\tightlist
\item
  \protect\hyperlink{link-1d1e3b4a}{Pence aides pushed back against
  Jennifer Williams after she told lawmakers she deemed Trump call
  ``unusual.''}
\item
  \protect\hyperlink{link-59501d94}{Former officials testified that
  focus on ``conspiracy theories'' detracted from national security.}
\item
  \protect\hyperlink{link-4b9b1be1}{``I did not know of a linkage.''
  Trump's former Ukraine envoy said he was unaware that security aid was
  tied to investigations of Democrats.}
\item
  \protect\hyperlink{link-4bbceac4}{Volker said he cringes when referred
  to as one of ``three amigos'' interfering in Ukraine policy.}
\item
  \protect\hyperlink{link-21f4e4b5}{Democrats expressed outrage at the
  attacks on Vindman by the White House and Republicans.}
\item
  \protect\hyperlink{link-696ff6ae}{The top White House Ukraine expert
  called Trump's call with Zelensky ``inappropriate'' and ``improper.''}
\item
  \protect\hyperlink{link-59b176a9}{Vindman and Williams testified that
  not a single national security official supported freezing Ukraine's
  security aid.}
\end{itemize}

\href{https://www.nytimes.com/2019/11/21/us/politics/impeachment-hearing.html}{\emph{Follow
our live coverage of David Holmes and Fiona Hill's testimony in the
Trump impeachment hearings}}.

\hypertarget{pence-aides-pushed-back-against-jennifer-williams-after-she-told-lawmakers-she-deemed-trump-call-unusual}{%
\subsection{Pence aides pushed back against Jennifer Williams after she
told lawmakers she deemed Trump call
``unusual.''}\label{pence-aides-pushed-back-against-jennifer-williams-after-she-told-lawmakers-she-deemed-trump-call-unusual}}

\includegraphics{https://static01.nyt.com/images/2019/11/19/nyregion/19dc-impeachbriefing24/merlin_164686602_0783266c-b6f6-4967-9791-356423e9f9d3-articleLarge.jpg?quality=75\&auto=webp\&disable=upscale}

Vice President Mike Pence's two senior most aides pushed back against
their colleague,
\href{https://www.nytimes.com/2019/11/19/us/politics/jennifer-williams.html}{Jennifer
Williams}, on Tuesday after she testified that she considered
\href{https://www.nytimes.com/2019/11/20/us/politics/impeachment-hearings.html}{President
Trump's} phone call with
\href{https://www.nytimes.com/2019/11/20/us/politics/impeachment-hearings.html}{Ukraine's
president} ``unusual'' because of its focus on domestic politics.

``I heard nothing wrong or improper on the call,'' Lt. Gen. Keith
Kellogg, the vice president's national security adviser, said in a
written statement released after her testimony. ``I had and have no
concerns. Ms. Williams was also on the call, and as she testified, she
never reported any personal or professional concerns to me, her direct
supervisor, regarding the call.

``In fact,'' he added, ``she never reported any personal or professional
concerns to any other member of the vice president's staff, including
our chief of staff and the vice president.''

Marc Short, the vice president's chief of staff, went on Fox News to
make the same point. ``She said she found the call unusual yet she never
raised any concerns with her supervisor General Kellogg, she never
raised any concerns with the chief of staff, she never raised any
concerns with the vice president,'' he said.

He added: ``We have
\href{https://www.nytimes.com/2019/11/20/us/politics/impeachment-hearings.html}{impeachment}
in pursuit of a crime.''

Mr. Trump attacked Ms. Williams on Twitter on Sunday,
\href{https://twitter.com/realDonaldTrump/status/1196155347117002752?s=20}{writing}that
she should read the transcript of the call and then ``meet with the
other Never Trumpers, who I don't know \& mostly never even heard of, \&
work out a better presidential attack!''

\hypertarget{former-officials-testified-that-focus-on-conspiracy-theories-detracted-from-national-security}{%
\subsection{Former officials testified that focus on ``conspiracy
theories'' detracted from national
security.}\label{former-officials-testified-that-focus-on-conspiracy-theories-detracted-from-national-security}}

Image

Kurt D. Volker arriving to testify on Tuesday.Credit...Erin Schaff/The
New York Times

The two former officials testifying on the afternoon panel were both
originally on the Republican witness list in hopes that their accounts
would provide testimony that would be more useful to President Trump's
defense. But while neither was as damning as the morning witnesses, both
highlighted how unusual the president's actions were.

``I don't think that raising 2016 elections or Vice President Biden or
these things I consider to be conspiracy theories that have been
circulated by the Ukrainians'' were ``things that we should be pursuing
as part of our national security strategy with Ukraine,'' Kurt D.
Volker, the president's former special envoy for Ukraine, told the House
Intelligence Committee.

``We should be supporting Ukraine's democracy, reforms, its own fight
against corruption domestically and the struggle against Russia and
defense capabilities and these are at the heart of what we should be
doing and I don't think pursuing these things serves a national
interest,'' he added.

Timothy Morrison, the former senior director for Europe and Ukraine at
the National Security Council, said he did not think the president's
July 25 call with President Volodymyr Zelensky of Ukraine was inherently
wrong or illegal, but feared it would ignite a political storm if it
became public.

``I feared at the time of the call on July 25 how its disclosure would
play in Washington's climate,'' he said. ``My fears have been realized.
I understand the gravity of these proceedings, but beg you not to lose
sight of the military conflict underway in Ukraine today.''

During later questioning, Daniel S. Goldman, the Democratic counsel,
asked: ``But you would agree, right, that asking a foreign government to
investigate a domestic political rival is inappropriate, would you
not?''

``It is not what we recommend the president discuss,'' Mr. Morrison
replied curtly.

\hypertarget{i-did-not-know-of-a-linkage-trumps-former-ukraine-envoy-said-he-was-unaware-that-security-aid-was-tied-to-investigations-of-democrats}{%
\subsection{``I did not know of a linkage.'' Trump's former Ukraine
envoy said he was unaware that security aid was tied to investigations
of
Democrats.}\label{i-did-not-know-of-a-linkage-trumps-former-ukraine-envoy-said-he-was-unaware-that-security-aid-was-tied-to-investigations-of-democrats}}

Image

Mr. Volker testifying in front of the House Intelligence Committee on
Tuesday.Credit...Anna Moneymaker/The New York Times

Mr. Volker portrayed himself as left out of key moments and unaware that
others working for Mr. Trump were linking the release of American
security aid to Ukraine committing to investigations of Democrats.

Opening the second panel of the day, Mr. Volker sought to reconcile his
original closed-door testimony with the accounts of other witnesses who
came after him. ``I have learned many things that I did not know at the
time of the events in question,'' he said in his opening statement.

Among other things, he said that at the time he worked with Rudolph W.
Giuliani, the president's personal attorney, to seek assurances from
Ukraine about investigations he was pushing, he did not understand those
investigations to include former Vice President Joseph R. Biden Jr. as a
target nor did he know that they would be tied to release of the frozen
security aid.

``I did not know of any linkage between the hold on security assistance
and Ukraine pursuing investigations,'' Mr. Volker said. ``No one had
ever said that to me --- and I never conveyed such a linkage to the
Ukrainians.'' He recalled telling the Ukrainians ``the opposite,'' that
they did not need to do anything to get the hold lifted and that it
would be taken care of. ``I did not know others were conveying a
different message to them around the same time,'' he said.

Mr. Volker sought to clarify why his testimony about the now-famous July
10 meeting at the White House differed from those of
\href{https://www.nytimes.com/2019/11/21/us/politics/impeachment-hearing.html}{Fiona
Hill}, then the senior director for Europe and Russia at the National
Security Council, and
\href{https://www.nytimes.com/2019/11/19/us/alexander-vindman.html}{Lt.
Col. Alexander S. Vindman}, her Ukraine policy deputy.

Ms. Hill and Colonel Vindman testified that John R. Bolton, then the
national security adviser, ended the meeting abruptly when Gordon D.
Sondland, the ambassador to the European Union, brought up the
investigations and that some in the room took the conversation
downstairs where it turned heated. Mr. Volker mentioned none of that in
his original testimony.

``As I remember, the meeting was essentially over when Ambassador
Sondland made a general comment about investigations,'' he said on
Tuesday. ``I think all of us thought it was inappropriate. The
conversation did not continue and the meeting concluded. Later on, in
the Ward Room, I may have been engaged in a side conversation or had
already left the complex, because I do not recall further discussion
regarding investigations or Burisma.''

More generally, he said he did not interpret the word Burisma to be
tantamount to Mr. Biden. ``In hindsight, I now understand that others
saw the idea of investigating possible corruption involving the
Ukrainian company Burisma as equivalent to investigating former Vice
President Biden. I saw them as very different --- the former being
appropriate and unremarkable, the latter being unacceptable. In
retrospect, I should have seen that connection differently, and had I
done so, I would have raised my own objections.''

\hypertarget{volker-said-he-cringes-when-referred-to-as-one-of-three-amigos-interfering-in-ukraine-policy}{%
\subsection{Volker said he cringes when referred to as one of ``three
amigos'' interfering in Ukraine
policy.}\label{volker-said-he-cringes-when-referred-to-as-one-of-three-amigos-interfering-in-ukraine-policy}}

Image

Mr. Volker, center, said he was not part of a shadow foreign
policy.Credit...Jason Andrew for The New York Times

Mr. Volker expressed annoyance at being lumped together with Mr.
Sondland and Energy Secretary Rick Perry as ``three amigos,'' as if they
were somehow indistinguishable, and he rejected the notion that he was
part of an irregular foreign policy channel.

The term ``three amigos'' has come to characterize how the usual foreign
policy process was warped by Mr. Trump's interest in obtaining damaging
information about Democrats from Ukraine. It originated from an
interview Mr. Sondland gave to Ukrainian television when he said ``we
have what are called the three amigos,'' naming Mr. Volker, Mr. Perry
and himself.

Mr. Volker in his testimony objected to the name and the implication.
``I've never used that term and frankly cringe when I hear it,'' he
said. In his mind, he said, he associated the phrase with his mentor,
Senator John McCain, Republican of Arizona, who died last year, and two
allies who supported a troop surge in Iraq in 2006 and 2007, Senator
Lindsey Graham, Republican of South Carolina, and Senator Joseph I.
Lieberman, independent of Connecticut, who has since left the Senate.

Mr. Volker said he was not part of a shadow foreign policy because he
was the officially designated diplomat assigned to help resolve
Ukraine's war with Russia. ``My role was not some irregular channel, but
the official channel,'' he said, noting that he reported to Rex W.
Tillerson, the secretary of state who appointed him, and Mike Pompeo,
his successor, and coordinated with diplomats and White House officials.

\hypertarget{democrats-expressed-outrage-at-the-attacks-on-vindman-by-the-white-house-and-republicans}{%
\subsection{Democrats expressed outrage at the attacks on Vindman by the
White House and
Republicans.}\label{democrats-expressed-outrage-at-the-attacks-on-vindman-by-the-white-house-and-republicans}}

Image

Representative Jim Jordan of Ohio cited critical comments about Colonel
Vindman's judgment from two other impeachment witnesses.Credit...Jason
Andrew for The New York Times

Democratic lawmakers responded angrily to attacks on
\href{https://www.nytimes.com/2019/11/19/us/alexander-vindman.html}{Colonel
Vindman}, who testified during the morning session, as the White House
and Republicans sought to discredit the colonel in real time during his
appearance before the committee.

``There's been a lot of insinuations and there's been a lot suggestions,
maybe, that your service is somehow not to be trusted,'' said
Representative Sean Patrick Maloney, Democrat of New York. He accused
Republicans of trying to ``air out some allegations with no basis and
proof, but they just want to get them out there and hope maybe some of
those strands of spaghetti I guess will stick on the wall if they keep
throwing them.''

His angry remarks came after the official, taxpayer-funded Twitter
account of the White House posted a critical quote about Colonel Vindman
from Mr. Morrison, his former boss at the National Security Council, who
testified later in the day on a separate panel.

Earlier, Representative Jim Jordan of Ohio had cited that comment as
well as criticism from Ms. Hill, Colonel Vindman's former boss at the
National Security Council.

``Any idea why they have those impressions?'' Mr. Jordan inquired.
Colonel Vindman, who apparently came prepared for the criticism, pulled
out a copy the performance evaluation Ms. Hill wrote about him in July
and read aloud from it.

``Alex is a top one percent military officer and the best army officer I
have worked with in my 15 years of government service,'' Colonel Vindman
said, quoting Ms. Hill. ``He is brilliant, unflappable, and exercises
excellent judgment.''

Republicans also questioned the loyalty of Colonel Vindman, an American
citizen and decorated Army combat veteran who was born in Ukraine, by
asking him about three instances when Oleksandr Danylyuk, the director
of Ukraine's national security council, had approached to offer him the
job of defense minister in Kyiv.

Under questioning by the committee's Republican counsel, Colonel Vindman
confirmed the offers and testified that he repeatedly declined,
dismissing the idea out of hand and reporting the approaches to his
superiors and to counterintelligence officials.

The line of questioning seemed to be designed, at least in part, to feed
doubts about Colonel Vindman's commitment to the United States, the
subject of
\href{https://www.nytimes.com/2019/11/06/us/politics/trump-vindman-twitter.html}{a
wave of character attacks on him} by Mr. Trump's allies. Fox News
quickly picked up on the tactic, sending out a news alert moments after
Mr. Castor finished: ``Vindman says Ukrainian official offered him the
job of Ukrainian defense minister.''

Mr. Maloney said he was particularly outraged by questions from a
Republican lawmaker questioning why Colonel Vindman wore his Army dress
uniform to the hearing.

``That dress uniform includes a breast plate that has a combat
infantryman badge on it and a purple ribbon,'' Mr. Maloney said. ``It
seems if there is someone who should wear that uniform, it's someone who
has a breast plate on it.''

\hypertarget{the-top-white-house-ukraine-expert-called-trumps-call-with-zelensky-inappropriate-and-improper}{%
\subsection{The top White House Ukraine expert called Trump's call with
Zelensky ``inappropriate'' and
``improper.''}\label{the-top-white-house-ukraine-expert-called-trumps-call-with-zelensky-inappropriate-and-improper}}

Image

Lt. Col. Alexander S. Vindman preparing to testify Tuesday before
Congress.Credit...Erin Schaff/The New York Times

Two senior national security officials at the White House challenged Mr.
Trump's description of his call with the Ukraine president as
``perfect,'' testifying on Tuesday about how concerned they were as they
listened in real time to Mr. Trump appealing for an investigation of Mr.
Biden.

\href{https://www.nytimes.com/2019/11/19/us/alexander-vindman.html}{Colonel
Vindman} testified that he was so disturbed by the call that he reported
it to the council's top lawyer.

``I couldn't believe what I was hearing,'' he said under questioning
about his first thoughts when he heard Mr. Trump's mention of
investigations into Mr. Biden and an unproven theory that it was
Ukraine, not Russia, that interfered in the 2016 election. ``It was
probably an element of shock, that maybe in certain regards, my worst
fear of how our Ukraine policy could play out was playing out, and how
this was likely to have significant implications for U.S. national
security.''

Earlier, Colonel Vindman explained why he felt it was his ``duty'' to
report his concerns to John Eisenberg, the top lawyer at the National
Security Council. ``It is improper for the president of the United
States to demand a foreign government investigate a U.S. citizen and
political opponent.''

\href{https://www.nytimes.com/2019/11/19/us/politics/jennifer-williams.html}{Ms;Williams},
a national security aide to Mr. Pence, said she found the president's
call unusual because it included discussion of a ``domestic political
matter.''

Their testimony kicked off three days of hearings featuring nine
diplomats and national security officials as Democrats on the House
Intelligence Committee continue to build their case that Mr. Trump
abused his power by trying to enlist Ukraine to publicly commit to
investigations that would discredit Mr. Biden, a leading political
rival, and other Democrats.

In a cabinet meeting as the hearing unfolded, Mr. Trump praised his
allies and dismissed the hearings as a ``kangaroo court,'' saying,
``Republicans are absolutely killing it, because it's a big scam.''

\hypertarget{vindman-and-williams-testified-that-not-a-single-national-security-official-supported-freezing-ukraines-security-aid}{%
\subsection{Vindman and Williams testified that not a single national
security official supported freezing Ukraine's security
aid.}\label{vindman-and-williams-testified-that-not-a-single-national-security-official-supported-freezing-ukraines-security-aid}}

Image

The hearing was shown on television at a bar near the
Capitol.Credit...Samuel Corum for The New York Times

Colonel Vindman and Ms. Williams both testified that they were never
aware of any other national security officials in the United States
government who supported the decision to withhold nearly \$400 million
in security aid for Ukraine, which both said was directed by the White
House chief of staff, Mick Mulvaney.

Both witnesses said withholding the military assistance from Ukraine was
damaging to relations between the two countries and to Ukraine's ability
to confront Russian aggression. Representative Mike Quigley of Illinois
asked Colonel Vindman whether anyone else supported the decision to
freeze the aid.

``No one from the national security?'' Mr. Quigley asked.

``None,'' Colonel Vindman said.

``No one from the State Department?''

``Correct.''

``No one from the Department of Defense?

``Correct.''

Ms. Williams testified that President Volodymyr Zelensky of Ukraine told
Vice President Mike Pence during a Sept. 1 meeting that continuing to
withhold the aid would indicate that United States support for Ukraine
was wavering, giving Russia a boost in the ongoing conflict between the
two countries.

``Any signal or sign that U.S. support was wavering would be construed
by Russia as potentially an opportunity for them to strengthen their own
hand in Ukraine,'' Ms. Williams said, relating what Mr. Zelensky told
Mr. Pence.

\hypertarget{nunes-tried-to-make-biden-not-trump-the-target}{%
\subsection{Nunes tried to make Biden, not Trump, the
target.}\label{nunes-tried-to-make-biden-not-trump-the-target}}

Image

Representative Devin Nunes, the ranking member, delivering his opening
remarks.Credit...Jason Andrew for The New York Times

Representative Devin Nunes of California, the top Republican on the
committee, sought to turn the focus away from Mr. Trump to Mr. Biden,
leading the witnesses through a series of questions intended to suggest
that the former vice president had intervened in Ukraine's domestic
affairs to benefit his son, Hunter Biden, despite the lack of evidence.

Mr. Biden, as vice president, pressured Ukrainian officials to fire a
prosecutor who was seen as tolerating corruption in keeping with the
policy of the United States, European allies and international financial
organizations at the time. But Mr. Nunes suggested that Mr. Biden was
acting to benefit his son, who was on the board of Burisma, a Ukrainian
energy company that had been investigated for corruption.

``Did you know that Joe Biden called Ukrainian President Poroshenko at
least three times in February 2016 after the president and owner of
Burisma's home was raided on February 2 by the state prosecutor's
office?'' Mr. Nunes asked, referring to Petro O. Poroshenko, then the
president.

``Not at the time,'' Ms. Williams answered. She added: ``I've become
aware of that through this proceeding.''

Mr. Nunes asked a series of similar questions and then repeated them for
Colonel Vindman. Neither witness was working on the issue at the time,
so neither could offer information to about it. But Mr. Nunes used the
opportunity to introduce his allegations, anyway. He also tried
repeatedly to extract information from Colonel Vindman about the
identity of the whistle-blower who filed a complaint about Mr. Trump's
dealings with Ukraine, drawing objections from the colonel's lawyer.

At one point, things turned testy when Mr. Nunes addressed Colonel
Vindman as ``Mr. Vindman.''

``Ranking member, it's Lieutenant Colonel Vindman, please,'' he shot
back.

Advertisement

\protect\hyperlink{after-bottom}{Continue reading the main story}

\hypertarget{site-index}{%
\subsection{Site Index}\label{site-index}}

\hypertarget{site-information-navigation}{%
\subsection{Site Information
Navigation}\label{site-information-navigation}}

\begin{itemize}
\tightlist
\item
  \href{https://help.nytimes.com/hc/en-us/articles/115014792127-Copyright-notice}{©~2020~The
  New York Times Company}
\end{itemize}

\begin{itemize}
\tightlist
\item
  \href{https://www.nytco.com/}{NYTCo}
\item
  \href{https://help.nytimes.com/hc/en-us/articles/115015385887-Contact-Us}{Contact
  Us}
\item
  \href{https://www.nytco.com/careers/}{Work with us}
\item
  \href{https://nytmediakit.com/}{Advertise}
\item
  \href{http://www.tbrandstudio.com/}{T Brand Studio}
\item
  \href{https://www.nytimes.com/privacy/cookie-policy\#how-do-i-manage-trackers}{Your
  Ad Choices}
\item
  \href{https://www.nytimes.com/privacy}{Privacy}
\item
  \href{https://help.nytimes.com/hc/en-us/articles/115014893428-Terms-of-service}{Terms
  of Service}
\item
  \href{https://help.nytimes.com/hc/en-us/articles/115014893968-Terms-of-sale}{Terms
  of Sale}
\item
  \href{https://spiderbites.nytimes.com}{Site Map}
\item
  \href{https://help.nytimes.com/hc/en-us}{Help}
\item
  \href{https://www.nytimes.com/subscription?campaignId=37WXW}{Subscriptions}
\end{itemize}
