Sections

SEARCH

\protect\hyperlink{site-content}{Skip to
content}\protect\hyperlink{site-index}{Skip to site index}

\href{https://www.nytimes.com/section/technology}{Technology}

\href{https://myaccount.nytimes.com/auth/login?response_type=cookie\&client_id=vi}{}

\href{https://www.nytimes.com/section/todayspaper}{Today's Paper}

\href{/section/technology}{Technology}\textbar{}Internet Companies
Prepare to Fight the `Deepfake' Future

\href{https://nyti.ms/2OeECau}{https://nyti.ms/2OeECau}

\begin{itemize}
\item
\item
\item
\item
\item
\end{itemize}

\begin{itemize}
\item
  \href{https://www.nytimes.com/2020/08/07/us/elections/biden-vs-trump.html?action=click\&pgtype=Article\&state=default\&region=TOP_BANNER\&context=storylines_menu}{Election
  Updates}
\item
  \href{https://www.nytimes.com/interactive/2020/08/08/us/elections/results-hawaii-primary-elections.html?action=click\&pgtype=Article\&state=default\&region=TOP_BANNER\&context=storylines_menu}{Hawaii
  Results}
\item
  \href{https://www.nytimes.com/article/biden-vice-president-2020.html?action=click\&pgtype=Article\&state=default\&region=TOP_BANNER\&context=storylines_menu}{Biden's
  V.P. Search}
\item
  \href{https://www.nytimes.com/interactive/2019/us/politics/2020-presidential-candidates.html?action=click\&pgtype=Article\&state=default\&region=TOP_BANNER\&context=storylines_menu}{The
  Candidates}
\item
  \href{https://www.nytimes.com/newsletters/politics?action=click\&pgtype=Article\&state=default\&region=TOP_BANNER\&context=storylines_menu}{Politics
  Newsletter}
\end{itemize}

Advertisement

\protect\hyperlink{after-top}{Continue reading the main story}

Supported by

\protect\hyperlink{after-sponsor}{Continue reading the main story}

\hypertarget{internet-companies-prepare-to-fight-the-deepfake-future}{%
\section{Internet Companies Prepare to Fight the `Deepfake'
Future}\label{internet-companies-prepare-to-fight-the-deepfake-future}}

Researchers are creating tools to find A.I.-generated fake videos before
they become impossible to detect. Some experts fear it is a losing
battle.

\includegraphics{https://static01.nyt.com/images/2019/11/24/business/24DEEPFAKES-01/24DEEPFAKES-01-articleLarge.jpg?quality=75\&auto=webp\&disable=upscale}

\href{https://www.nytimes.com/by/cade-metz}{\includegraphics{https://static01.nyt.com/images/2018/11/26/multimedia/author-cade-metz/author-cade-metz-thumbLarge.png}}

By \href{https://www.nytimes.com/by/cade-metz}{Cade Metz}

\begin{itemize}
\item
  Nov. 24, 2019
\item
  \begin{itemize}
  \item
  \item
  \item
  \item
  \item
  \end{itemize}
\end{itemize}

SAN FRANCISCO --- Several months ago, Google hired dozens of actors to
sit at a table, stand in a hallway and walk down a street while talking
into a video camera.

Then the company's researchers, using a new kind of artificial
intelligence software,
\href{https://ai.googleblog.com/2019/09/contributing-data-to-deepfake-detection.html}{swapped
the faces of the actors}. People who had been walking were suddenly at a
table. The actors who had been in a hallway looked like they were on a
street. Men's faces were put on women's bodies. Women's faces were put
on men's bodies. In time, the researchers had created hundreds of
so-called deepfake videos.

By creating these digitally manipulated videos, Google's scientists
believe they are learning how to spot deepfakes, which researchers and
lawmakers worry could become a new, insidious method for spreading
disinformation in the lead-up to the 2020 presidential election.

For internet companies like Google, finding the tools to spot deepfakes
has gained urgency. If someone wants to spread a fake video far and
wide, Google's YouTube or Facebook's social media platforms would be
great places to do it.

Imagine a fake Senator Elizabeth Warren, virtually indistinguishable
from the real thing, getting into a fistfight in a doctored video. Or a
fake President Trump doing the same. The technology capable of that
trickery is edging closer to reality.

``Even with current technology, it is hard for some people to tell what
is real and what is not,'' said Subbarao Kambhampati, a professor of
computer science at Arizona State University.

\hypertarget{on-the-weekly-ai-engineers-create-a-deepfake-video}{%
\subsubsection{On `The Weekly,' A.I. Engineers Create a Deepfake
Video}\label{on-the-weekly-ai-engineers-create-a-deepfake-video}}

{[}\emph{Sunday at 10 p.m. on FX and Streaming Monday on Hulu.}{]}

Video

transcript

Back

bars

0:00/2:01

-0:00

transcript

\begin{itemize}
\tightlist
\item
  {[}HIGH-PITCHED NOTE{]} ``You know when a person is working on
  something and it's good, but it's not perfect? And he just tries for
  perfection? That's me in a nutshell.'' {[}MUFFLED SPEECH{]} ``I just
  want to recreate humans.'' ``O.K. But why?'' ``I don't know. I mean,
  it's that feeling you get when you achieve something big. (ECHOING)
  ``It's really interesting. You hear these words coming out in your
  voice, but you never said them.'' ``Let's try again.'' ``We've been
  working to make a convincing total deepfake. The bar we're setting is
  very high.'' ``So you can see, it's not perfect.'' ``We're trying to
  make it so the population would totally believe this video.'' ``Give
  this guy an Oscar.'' {[}LAUGHTER{]} ``There are definitely people
  doing it at Google, Samsung, Microsoft. The technology moves super
  fast.'' ``Somebody else will beat you to it if you wait a year.''
  ``Someone else will. And that will hurt.'' ``O.K., let's try again.''
  ``Just make it natural, right?'' ``It's hard to be natural.'' ``It's
  hard to be natural when you're faking it.'' ``O.K.'' ``What are you up
  to these days?'' ``Today, I'm announcing my candidacy for the
  presidency of the United States.'' {[}LAUGHTER{]} ``And I would like
  to announce my very special running mate, the most famous chimp in the
  world, Bubbles Jackson. Are we good?'' ``People do not realize how
  close this is to happen. Fingers crossed. It's going to happen, like,
  in the upcoming months. Yeah, the world is going to change.'' ``I
  squint my eyes.'' ``Yeah.'' ``Look, this is how we got into the mess
  we're in today with technology, right? A bunch of idealistic young
  people thinking, we're going to change the world.'' ``It's weird to
  see his face on it.'' {[}LAUGHTER{]} ``I wondered what you would say
  to these engineers.'' ``I would say, I hope you're putting as much
  thought into how we deal with the consequences of this as you are into
  the realization of it. This is a Pandora's box you're opening.''
  {[}THEME MUSIC{]}
\end{itemize}

Deepfakes --- a term that generally describes videos doctored with
cutting-edge artificial intelligence --- have already challenged our
assumptions about what is real and what is not.

In recent months, video evidence was at the center of prominent
incidents in Brazil, Gabon in Central Africa and China. Each was colored
by the same question: Is the video real? The Gabonese president, for
example, was out of the country for medical care and his government
released a so-called proof-of-life video. Opponents claimed it had been
faked. Experts call that confusion ``the liar's dividend.''

``You can already see a material effect that deepfakes have had,'' said
Nick Dufour, one of the Google engineers overseeing the company's
deepfake research. ``They have allowed people to claim that video
evidence that would otherwise be very convincing is a fake.''

For decades, computer software has allowed people to manipulate photos
and videos or create fake images from scratch. But it has been a slow,
painstaking process usually reserved for experts trained in the vagaries
of software like Adobe Photoshop or After Effects.

Now, artificial intelligence technologies are streamlining the process,
reducing the cost, time and skill needed to doctor digital images. These
A.I. systems learn on their own how to build fake images by analyzing
thousands of real images. That means they can handle a portion of the
workload that once fell to trained technicians. And that means people
can create far more fake stuff than they used to.

The technologies used to create deepfakes is still fairly new and the
results are often easy to notice. But the technology is evolving. While
the tools used to detect these bogus videos are also evolving, some
researchers worry that they won't be able to keep pace.

Google recently
\href{https://ai.googleblog.com/2019/09/contributing-data-to-deepfake-detection.html}{said}
that any academic or corporate researcher could download its collection
of synthetic videos and use them to build tools for identifying
deepfakes. The video collection is essentially a syllabus of digital
trickery for computers. By analyzing all of those images, A.I. systems
learn how to watch for fakes. Facebook recently
\href{https://ai.facebook.com/blog/deepfake-detection-challenge/}{did
something similar}, using actors to build fake videos and then releasing
them to outside researchers.

Engineers at a Canadian company called Dessa, which specializes in
artificial intelligence, recently tested a deepfake detector that was
built using Google's synthetic videos. It could identify the Google
videos with almost perfect accuracy. But when they tested their detector
on deepfake videos plucked from across the internet, it failed more than
40 percent of the time.

\includegraphics{https://static01.nyt.com/images/2019/11/24/business/24DEEPFAKES-02/24DEEPFAKES-02-articleLarge.jpg?quality=75\&auto=webp\&disable=upscale}

They eventually fixed the problem, but only after rebuilding their
detector with help from videos found ``in the wild,'' not created with
paid actors --- proving that a detector is only as good as the data used
to train it.

Their tests showed that the fight against deepfakes and other forms of
online disinformation will require nearly constant reinvention. Several
hundred synthetic videos are not enough to solve the problem, because
they don't necessarily share the characteristics of fake videos being
distributed today, much less in the years to come.

``Unlike other problems, this one is constantly changing,'' said Ragavan
Thurairatnam, Dessa's founder and head of machine learning.

In December 2017, someone calling themselves ``deepfakes'' started using
A.I. technologies to graft
\href{https://www.nytimes.com/2018/03/04/technology/fake-videos-deepfakes.html}{the
heads of celebrities onto nude bodies in pornographic videos}. As the
practice spread across services like Twitter, Reddit and PornHub, the
term deepfake entered the popular lexicon. Soon, it was synonymous with
any fake video posted to the internet.

The technology has improved at a rate that surprises A.I. experts, and
there is little reason to believe it will slow. Deepfakes should benefit
from one of the few tech industry axioms that have held up over the
years: Computers always get more powerful and there is always more data.
That makes the so-called machine-learning software that helps create
deepfakes more effective.

``It is getting easier, and it will continue to get easier. There is no
doubt about it,'' said Matthias Niessner, a professor of computer
science at the Technical University of Munich who is working with Google
on its deepfake research. ``That trend will continue for years.''

The question is: Which side will improve more quickly?

Researchers like Dr. Niessner are working to build systems that can
automatically identify and remove deepfakes. This is the other side of
the same coin. Like deepfake creators, deepfake detectors learn their
skills by analyzing images.

Detectors can also improve by leaps and bounds. But that requires a
constant stream of new data representing the latest deepfake techniques
used around the internet, Dr. Niessner and other researchers said.
Collecting and sharing the right data can be difficult. Relevant
examples are scarce, and for privacy and copyright reasons, companies
cannot always share data with outside researchers.

Though activists and artists occasionally release deepfakes as a way of
showing
\href{https://www.nytimes.com/2019/06/11/technology/fake-zuckerberg-video-facebook.html}{how
these videos could shift the political discourse online,} these
techniques are not widely used to spread disinformation. They are mostly
used to spread humor or fake pornography, according to Facebook, Google
and others who track the progress of deepfakes.

Right now, deepfake videos have subtle imperfections that can be readily
detected by automated systems, if not by the naked eye. But some
researchers argue that the improved technology will be powerful enough
to create fake images without these tiny defects. Companies like Google
and Facebook hope they will have reliable detectors in place before that
happens.

``In the short term, detection will be reasonably effective,'' said Mr.
Kambhampati, the Arizona State professor. ``In the longer term, I think
it will be impossible to distinguish between the real pictures and the
fake pictures.''

\hypertarget{our-2020-election-guide}{%
\section{Our 2020 Election Guide}\label{our-2020-election-guide}}

Updated Aug. 8, 2020

\begin{itemize}
\item
  \begin{center}\rule{0.5\linewidth}{\linethickness}\end{center}

  \hypertarget{the-latest}{%
  \subsection{The Latest}\label{the-latest}}

  \begin{itemize}
  \tightlist
  \item
    With 160 lawsuits filed over voting rules and President Trump's
    baseless claims of fraud, Election Day in America
    \href{https://www.nytimes.com/2020/08/08/us/politics/voting-nov-3-election.html?action=click\&pgtype=Article\&state=default\&region=BELOW_MAIN_CONTENT\&context=storylines_guide}{could
    become Election Month}.
  \end{itemize}
\item
  \begin{center}\rule{0.5\linewidth}{\linethickness}\end{center}

  \hypertarget{bidens-vp-search}{%
  \subsection{Biden's V.P. Search}\label{bidens-vp-search}}

  \begin{itemize}
  \tightlist
  \item
    \href{https://www.nytimes.com/article/biden-vice-president-2020.html?action=click\&pgtype=Article\&state=default\&region=BELOW_MAIN_CONTENT\&context=storylines_guide}{Here
    are 13 women} who have been under consideration to be Joe Biden's
    running mate, and why each might be chosen --- and might not be.
  \end{itemize}
\item
  \begin{center}\rule{0.5\linewidth}{\linethickness}\end{center}

  \hypertarget{keep-up-with-our-coverage}{%
  \subsection{Keep Up With Our
  Coverage}\label{keep-up-with-our-coverage}}

  \begin{itemize}
  \tightlist
  \item
    Get an
    \href{https://www.nytimes.com/newsletters/politics?action=click\&pgtype=Article\&state=default\&region=BELOW_MAIN_CONTENT\&context=storylines_guide}{email}
    recapping the day's news
  \end{itemize}

  \begin{itemize}
  \tightlist
  \item
    Download our mobile app on
    \href{https://apps.apple.com/us/app/nytimes/id284862083?ls=1\&mat_click_id=5c79ae7455014fd1bd66b5610c05b8f2-20191112-16948\&referrer=mat_click_id\%3D5c79ae7455014fd1bd66b5610c05b8f2-20191112-16948\%26link_click_id\%3D722930677036718082}{iOS}
    and
    \href{http://a.localytics.com/android?id=com.nytimes.android\&referrer=utm_source\%3Dother_nyt_mobile_web\%26utm_medium\%3DWeb\%2520page\%26utm_term\%3DGeneral\%2520Mobile\%2520Page\%26utm_campaign\%3DNYT\%2520Mobile\%2520General\%2520Page}{Android}
    and turn on Breaking News and Politics alerts
  \end{itemize}
\end{itemize}

Advertisement

\protect\hyperlink{after-bottom}{Continue reading the main story}

\hypertarget{site-index}{%
\subsection{Site Index}\label{site-index}}

\hypertarget{site-information-navigation}{%
\subsection{Site Information
Navigation}\label{site-information-navigation}}

\begin{itemize}
\tightlist
\item
  \href{https://help.nytimes.com/hc/en-us/articles/115014792127-Copyright-notice}{©~2020~The
  New York Times Company}
\end{itemize}

\begin{itemize}
\tightlist
\item
  \href{https://www.nytco.com/}{NYTCo}
\item
  \href{https://help.nytimes.com/hc/en-us/articles/115015385887-Contact-Us}{Contact
  Us}
\item
  \href{https://www.nytco.com/careers/}{Work with us}
\item
  \href{https://nytmediakit.com/}{Advertise}
\item
  \href{http://www.tbrandstudio.com/}{T Brand Studio}
\item
  \href{https://www.nytimes.com/privacy/cookie-policy\#how-do-i-manage-trackers}{Your
  Ad Choices}
\item
  \href{https://www.nytimes.com/privacy}{Privacy}
\item
  \href{https://help.nytimes.com/hc/en-us/articles/115014893428-Terms-of-service}{Terms
  of Service}
\item
  \href{https://help.nytimes.com/hc/en-us/articles/115014893968-Terms-of-sale}{Terms
  of Sale}
\item
  \href{https://spiderbites.nytimes.com}{Site Map}
\item
  \href{https://help.nytimes.com/hc/en-us}{Help}
\item
  \href{https://www.nytimes.com/subscription?campaignId=37WXW}{Subscriptions}
\end{itemize}
