Sections

SEARCH

\protect\hyperlink{site-content}{Skip to
content}\protect\hyperlink{site-index}{Skip to site index}

\href{https://myaccount.nytimes.com/auth/login?response_type=cookie\&client_id=vi}{}

\href{https://www.nytimes.com/section/todayspaper}{Today's Paper}

A Design Dealer Who Lives Among His Wares

\href{https://nyti.ms/2LmwNhi}{https://nyti.ms/2LmwNhi}

\begin{itemize}
\item
\item
\item
\item
\item
\end{itemize}

Advertisement

\protect\hyperlink{after-top}{Continue reading the main story}

Supported by

\protect\hyperlink{after-sponsor}{Continue reading the main story}

\hypertarget{a-design-dealer-who-lives-among-his-wares}{%
\section{A Design Dealer Who Lives Among His
Wares}\label{a-design-dealer-who-lives-among-his-wares}}

Rhett Baruch has transformed his compact Los Angeles apartment into a
showroom for the colorful, unconventional items that he sells.

\includegraphics{https://static01.nyt.com/images/2019/10/16/t-magazine/design/16tmag-baruch-slide-Q8JV/16tmag-baruch-slide-Q8JV-articleLarge.jpg?quality=75\&auto=webp\&disable=upscale}

By \href{https://www.nytimes.com/by/max-berlinger}{Max Berlinger}

\begin{itemize}
\item
  Published Dec. 4, 2019Updated Dec. 5, 2019
\item
  \begin{itemize}
  \item
  \item
  \item
  \item
  \item
  \end{itemize}
\end{itemize}

A humdrum exterior belies the artfully wacky world behind the doors of
\href{http://rhettbaruch.com/}{Rhett Baruch}'s home in Los Angeles.
While the building --- a stuccoed two-story structure set on the border
of Koreatown and Westlake --- is fairly unremarkable, Baruch has
transformed his 1,200-square-foot second-floor apartment into an
ever-evolving series of colorful vignettes; a burl-wood console in the
style of the Dutch de Stijl designer Gerrit Rietveld might be paired
with a lumpy ceramic étagère by the Los Angeles-based artist
\href{https://www.emiliecarroll.com/}{Emilie Carroll}, or the sensual
curves of a Flemming Lassen chair could be offset by an amoeba-like neon
sculpture (called ``Blooops'') from
\href{https://www.alinahayes.com/}{Alina Hayes}. A collector and dealer
of contemporary art and design, Baruch uses the space as a de facto
showroom, a place to display beguiling curios and whimsical home décor
for the benefit of clients, though of course he also takes personal
pleasure in living among eccentric and beautiful things.

Baruch's aesthetic is a bold rejoinder to the more neutral, reserved
interiors that have prevailed in recent years (think fiddle-leaf figs in
ceramic pots nestled next to muted Eames chairs), perhaps because he is
something of an outsider. Originally from Arizona, where he held odd
jobs like working in call centers, Baruch, 34, has no formal training in
design. Soon after moving to Los Angeles three years ago, he received
word from the Miami-based vintage collector
\href{https://www.instagram.com/gillianbryce/}{Gillian Bryce}, who'd
seen posts of some of the décor pieces Baruch had collected on
Instagram. The two became fast friends --- Bryce assumed the role of
Baruch's mentor, really, though they'll meet in person for the first
time at this week's Art Basel fair in Miami Beach. She nudged him toward
turning his off-the-cuff posts of quirky found objects into a proper
design business.

Image

In the living room, a Carlo Scarpa bowl balances on the arm of a Stefan
White chair. An Amazonia vase by Gaetano Pesce sits in front of a
leather coat tree by the American fashion brand Coach and a ceramic
chain basket by the artist Taylor Kibby. On the wall is a painting by
the artist Zach Storm.Credit...Chris Mottalini

Image

The focal point of Baruch's kitchen is a French papier-mâché mirror from
the 1990s that he bought at auction, offset here by an LGS Studio
studded bowl, Elyse Graham resin vessels and glassware from Asp \&
Hand.Credit...Chris Mottalini

Baruch did just that last year, when he styled
\href{https://notsogeneral.la/}{Not So General}, the Los Angeles-based
gallery, showroom and design studio that belongs to Paul Davidge.
Davidge's endorsement not only helped legitimize Baruch's commercial
endeavors, it introduced him to a wider swath of the local design
community; he now counts among his clients interior designers such as
Louisa Pierce and Emily Ward of \href{https://pierceandward.com/}{Pierce
\& Ward}, Kelly Wearstler and the artist
\href{https://www.nytimes.com/2018/07/19/arts/design/geronimo-pier-17-south-street-seaport.html}{Jihan
Zencirli}.

Baruch describes his business as a ``constant evolution,'' as works come
and go, and so his home, too, is always in flux. This setup has its
benefits: It saves him the cost of renting a showroom and helps him
avoid any logistical issues, since Baruch recently did the unthinkable
for an Angeleno --- he gave up his car (discovering the city on foot has
been one of his recent joys and sources of inspiration, he says).
There's also something charming about coming to his personal space ---
and prospective buyers can imagine how they, too, could style some of
the more outré pieces --- and about the slight chaos that accompanies
Baruch, thanks to his pinging phone and the stream of delivery men
picking up or dropping off packages. To Baruch, it's all an extension of
the mercantilist tradition that's been around since the marketplaces of
Rome. ``What I do is the most traditional form of commerce,'' he says.
``It's taking a thing that's been made by a person in this world ---
finding it or having it made --- and moving it to the next home and
taking money. That's it.''

Image

A bedroom corner serves as a colorful reading nook with a chair by
Taidgh O'Neill and a side table by Jackie Rines.Credit...Chris Mottalini

Image

Next to Baruch's bed is a mixed-media valet by Kayla Thompson and a
ceramic sculpture by Alina Hayes. On the bed is a blanket by Another
Human, and above it hangs a mixed-media piece from Joseph
Stashkevetch.Credit...Chris Mottalini

On some days, ``I wake up and I don't necessarily know what's going to
come to me,'' Baruch notes. ``Suddenly a Knoll table pops up that's a
great deal, so I'll rearrange some things.'' Some of his most vibrant
items **** have unconventional provenances to match --- take, for
example, the six-foot-tall burned-orange leather coat tree in the shape
of a cactus, a castoff prop from the fashion brand Coach.

While contemporary design makes up the majority of Baruch's business,
lately he's introduced older works into the mix, like a green bowl by
the Italian designer Carlo Scarpa for Venini that dates back to 1936. He
pays particular attention to texture, evidenced by pieces like a drippy
black-and-white Amazonian vessel by the Italian designer Gaetano Pesce
that appears to be hard clay but is in fact a lightweight, rubbery
resin, or a vase by the Californian designer
\href{http://taylorkibby.com/}{Taylor Kibby} made of patinated ceramic
chain-link that folds in on itself and can be reshaped at will. Baruch
sources directly from local Los Angeles makers --- like Kibby and the
artist \href{http://www.zachstorm.com/}{Zach Storm} --- in part to help
**** foster a sense of community, trying to find locals with whom he
feels an affinity. ``It's not worth it to buy things that don't matter,
that aren't made by someone, that aren't rare or unique,'' he says.

Image

In the living room, a FontanaArte glass vessel sits atop the fireplace,
and a Faye Toogood Roly Poly chair contrasts with a ceramic chain basket
by Taylor Kibby.Credit...Chris Mottalini

Image

Baruch and his dog, Leon.Credit...Chris Mottalini

Although he is surrounded all the time by the pieces he sells, much of
Baruch's business happens online, through furniture marketplaces like
\href{https://www.chairish.com/shop/rhett}{Chairish}, and his days are
often spent staging scenes to photograph and post on Instagram (he hopes
to expand his services to include styling, decorating and consulting).
And while it may be easy for him to fall in love with a stocky Faye
Toogood chair or a knobbly Jackie Rines side table or even with a
particularly delightful room arrangement, he's found it's best not to:
Everything might be different within a day or so, if not sooner. He
likes to think of himself as a fast-moving matchmaker trying to pair
good things with their fated owners. ``Like a medium,'' Baruch says.
``I'm the catalyst for people to own beautiful things.''

Advertisement

\protect\hyperlink{after-bottom}{Continue reading the main story}

\hypertarget{site-index}{%
\subsection{Site Index}\label{site-index}}

\hypertarget{site-information-navigation}{%
\subsection{Site Information
Navigation}\label{site-information-navigation}}

\begin{itemize}
\tightlist
\item
  \href{https://help.nytimes.com/hc/en-us/articles/115014792127-Copyright-notice}{©~2020~The
  New York Times Company}
\end{itemize}

\begin{itemize}
\tightlist
\item
  \href{https://www.nytco.com/}{NYTCo}
\item
  \href{https://help.nytimes.com/hc/en-us/articles/115015385887-Contact-Us}{Contact
  Us}
\item
  \href{https://www.nytco.com/careers/}{Work with us}
\item
  \href{https://nytmediakit.com/}{Advertise}
\item
  \href{http://www.tbrandstudio.com/}{T Brand Studio}
\item
  \href{https://www.nytimes.com/privacy/cookie-policy\#how-do-i-manage-trackers}{Your
  Ad Choices}
\item
  \href{https://www.nytimes.com/privacy}{Privacy}
\item
  \href{https://help.nytimes.com/hc/en-us/articles/115014893428-Terms-of-service}{Terms
  of Service}
\item
  \href{https://help.nytimes.com/hc/en-us/articles/115014893968-Terms-of-sale}{Terms
  of Sale}
\item
  \href{https://spiderbites.nytimes.com}{Site Map}
\item
  \href{https://help.nytimes.com/hc/en-us}{Help}
\item
  \href{https://www.nytimes.com/subscription?campaignId=37WXW}{Subscriptions}
\end{itemize}
