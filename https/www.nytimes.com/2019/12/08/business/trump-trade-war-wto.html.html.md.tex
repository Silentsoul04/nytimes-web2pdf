Sections

SEARCH

\protect\hyperlink{site-content}{Skip to
content}\protect\hyperlink{site-index}{Skip to site index}

\href{https://www.nytimes.com/section/business}{Business}

\href{https://myaccount.nytimes.com/auth/login?response_type=cookie\&client_id=vi}{}

\href{https://www.nytimes.com/section/todayspaper}{Today's Paper}

\href{/section/business}{Business}\textbar{}Trump Cripples W.T.O. as
Trade War Rages

\url{https://nyti.ms/2qCcVzC}

\begin{itemize}
\item
\item
\item
\item
\item
\end{itemize}

Advertisement

\protect\hyperlink{after-top}{Continue reading the main story}

Supported by

\protect\hyperlink{after-sponsor}{Continue reading the main story}

\hypertarget{trump-cripples-wto-as-trade-war-rages}{%
\section{Trump Cripples W.T.O. as Trade War
Rages}\label{trump-cripples-wto-as-trade-war-rages}}

A U.S. offensive against the World Trade Organization will effectively
shutter the group's system for settling disputes, at a time it's most
needed.

\includegraphics{https://static01.nyt.com/images/2019/12/08/business/00DC-WTO-01/merlin_158492361_342fe44e-b0de-4db3-b9b1-e8ff8997a154-articleLarge.jpg?quality=75\&auto=webp\&disable=upscale}

\href{https://www.nytimes.com/by/ana-swanson}{\includegraphics{https://static01.nyt.com/images/2018/12/10/multimedia/author-ana-swanson/author-ana-swanson-thumbLarge.png}}

By \href{https://www.nytimes.com/by/ana-swanson}{Ana Swanson}

\begin{itemize}
\item
  Dec. 8, 2019
\item
  \begin{itemize}
  \item
  \item
  \item
  \item
  \item
  \end{itemize}
\end{itemize}

WASHINGTON --- The United States has spent two years chipping away at
the World Trade Organization, criticizing it as unfair, starving it of
personnel and disregarding its authority, as President Trump seeks to
upend the global trade system.

This week, the Trump administration is expected to go one step further
and effectively cripple the organization's system for enforcing its
rules --- even as Mr. Trump's widening trade war has thrown global
commerce into disarray and another tariff increase on Chinese goods set
for next weekend could send markets reeling.

Over the past two years, Washington has blocked the W.T.O. from
appointing new members to a crucial panel that hears appeals in trade
disputes. Only three members are left on the seven-member body, the
minimum needed to hear a case, and two members' terms expire on Tuesday.
With the administration blocking any new replacements, there will be no
official resolution for many international trade disputes.

The loss of the world's primary trade referee could turn the typically
deliberate process of resolving international disputes into a
free-for-all, paving the way for an outbreak of tit-for-tat tariff wars.

It could also signal the demise of the 24-year-old World Trade
Organization itself, since the system for settling disputes has long
been its most effective part.

``The W.T.O. is facing its deepest crisis since its creation,'' Phil
Hogan, the European trade commissioner, told members of the European
Parliament this year. If the rules governing international trade can no
longer be enforced, ``we'd have the law of the jungle.''

Mr. Trump has already embraced that scenario, wielding America's
economic power to press for better trade terms. He has sidestepped
W.T.O. rules by imposing metal tariffs on allies like Canada, Europe and
Japan, and by adding punishing levies to Chinese goods, prompting
appeals to the global body for relief.

The president and his top advisers have long viewed the W.T.O. as an
impediment to Mr. Trump's promise to put ``America First.'' They say the
organization, which insists that all of its members receive equal
treatment, has prevented the United States from protecting its workers
and exerting its influence as the world's most powerful economy. They
have also criticized the W.T.O. for emboldening China --- whose economy
boomed after it became a member in 2001 --- while doing little to curb
Beijing's unfair trade practices.

His advisers point to the W.T.O.'s inability to confront China as a
reason for Mr. Trump's trade war with Beijing.

\includegraphics{https://static01.nyt.com/images/2019/12/09/business/09dc-wto-print-2/merlin_163876272_6979f423-a164-42f7-9b33-d183596030c9-articleLarge.jpg?quality=75\&auto=webp\&disable=upscale}

``It's absolutely critical that the United States has the ability to
make its own trade policy,'' said Stephen P. Vaughn, a partner at King
\& Spalding, who left a high-level post at the Office of the United
States Trade Representative in May. ``This ability becomes even more
important given the challenges that we now face from China.''

The World Trade Organization was founded by American and European
officials more than two decades ago as a way to open global markets,
regulate commerce and promote peace and stability. One of its chief
responsibilities was to write trade agreements among its members, and
provide an orderly way to settle disputes.

But the W.T.O. almost immediately fell short when it came to writing
trade pacts, as it found it nearly impossible to achieve consensus
between disparate members like the United States, China, Afghanistan and
India.

China's entry into the organization --- 18 years ago this week, on Dec.
11, 2001 --- put further stress on the system. The addition of China's
more than one billion people to the global marketplace created a huge
opportunity for companies, and a shock for workers in the United States
and elsewhere who were forced to compete.

The W.T.O.'s rules were not written with an economy like China's in
mind, and critics say the organization has failed to adequately police
Beijing for using a mix of private enterprise and state support to
dominate global industries.

The Trump administration has criticized the body's decision to allow
China to claim
\href{https://www.nytimes.com/2019/07/26/us/politics/trump-wto-china.html}{a
special status for developing countries} given that it is now the
world's second largest economy. And it has condemned the W.T.O. for
doing little to stop China from
\href{https://www.nytimes.com/2019/05/12/business/china-trump-trade-subsidies.html}{subsidizing
its products} --- instead
\href{https://www.nytimes.com/2019/11/01/business/wto-china-us-trade.html}{cracking
down on American measures} that are meant to block those cheap goods at
the border.

While the W.T.O.'s ability to facilitate trade negotiations was largely
paralyzed, its other arm, which settles trade disputes, has been much
more active, reviewing dozens of cases a year.

Unlike other international organizations, whose rules have no way of
being enforced, the W.T.O. may dole out punishments along with its
verdicts. When one country is found to have suffered from another's
trade practices, the W.T.O. may allow the aggrieved country to recoup
losses through retaliatory tariffs.

The United States has long won the majority of cases it brings to the
W.T.O., though Mr. Trump incorrectly argues to the contrary. In October,
the W.T.O. gave the United States permission to add
\href{https://www.nytimes.com/2019/10/02/us/politics/airbus-tariffs-wto.html}{tariffs
on up to \$7.5 billion of European products} annually, after deciding
that Europe had illegally subsidized its largest plane maker, Airbus.

``We never won with the W.T.O., or essentially never won,'' Mr. Trump
said Oct. 16 as he met with the Italian president. ``And now we're
winning a lot. We're winning a lot because they know if we're not
treated fairly, we're leaving.''

But the United States has also lost cases, and the Trump administration
is facing numerous challenges to the president's aggressive use of
tariffs to punish trading partners. Japan, Canada, China, the European
Union and other governments are relying on the system to determine
whether Mr. Trump's tariffs on steel and aluminum violated global trade
rules. However, many of those governments --- including the European
Union, Mexico and Canada --- have not waited for a ruling before
imposing retaliatory tariffs on American goods.

Supporters have credited the dispute settlement system with bringing the
rule of law to an international trading system that formerly allowed
strong countries to dominate weak ones.

But critics say the system exerts too much control, especially at the
final stage when the seven-member appellate body makes a binding
determination. American officials, including in the Obama
administration, have accused the appellate body of judicial activism,
saying it is overstepping its authority in creating new rules.

Image

The United States trade representative, Robert Lighthizer, has spent two
years attacking the W.T.O.'s system for appeals.Credit...Sarah
Silbiger/The New York Times

Robert Lighthizer, the United States trade representative, has argued
that the body's decisions constrain America's ability to protect its
workers and has insisted it be overhauled. In March, he told lawmakers
on the Senate Finance Committee that the W.T.O. had migrated ``from a
negotiation forum to a litigation forum,'' a transformation that had
stifled new trade agreements and undermined some countries' commitment
to the organization.

Over the past two years, Mr. Lighthizer has overseen a targeted
offensive against the appellate body, in what he says is a push for
change. The United States has blocked the appointment of new appellate
body members, which requires the consensus of all governments.

Officials in other countries share some of America's concerns,
particularly related to China, but they disagree with the Trump
administration's methods. They argue the United States and other
countries should fix the problems and strengthen the global trading
system, not abandon it.

The prospective weakening of global trade rules has worried smaller and
poorer nations, who may find themselves at the mercy of the United
States. It has also rankled the European Union, a strong believer in the
multilateral system whose economy is heavily dependent on trade.

``They are not perfect, because they were born in a certain context, but
they've served us well,'' Cecilia Malmstrom, the former European Union
trade commissioner, said of the global trading rules in an interview in
September. ``And if they're not perfect, let's work to improve them.
Let's not just abolish them.''

Trump administration officials say proposals to overhaul the W.T.O. have
fallen short of what is needed. Dennis Shea, the American representative
to the World Trade Organization, said
\href{https://geneva.usmission.gov/2019/12/06/ambassador-sheas-statement-at-the-wto-trade-negotiating-committee-heads-of-delegation-meeting/}{last
week} that the United States had engaged constructively, but had ``have
yet to see the same level of engagement'' from other countries.

W.T.O. members have been discussing ways to deal with the appellate
body's disappearance, like setting up their own informal appeal process,
regardless of the verdict. Many are hopeful that the body can be
restored once the Trump administration leaves office, whether that is in
2021 or 2025.

Roberto Azevêdo, the W.T.O. director-general, said last week that the
suspension of appeals was a serious challenge but that it did ``not mean
the end of the multilateral trading system.''

But Ujal Singh Bhatia, one of the appellate body members whose term ends
Tuesday, said that by making the dispute settlement system potentially
non-functional, the United States' moves had cast doubts on the
effectiveness of the organization over all.

``Why would people come to the W.T.O. to negotiate rules if they are not
sure the rules can be enforced?'' Mr. Bhatia asked.

Jack Ewing contributed reporting from Frankfurt.

Advertisement

\protect\hyperlink{after-bottom}{Continue reading the main story}

\hypertarget{site-index}{%
\subsection{Site Index}\label{site-index}}

\hypertarget{site-information-navigation}{%
\subsection{Site Information
Navigation}\label{site-information-navigation}}

\begin{itemize}
\tightlist
\item
  \href{https://help.nytimes.com/hc/en-us/articles/115014792127-Copyright-notice}{©~2020~The
  New York Times Company}
\end{itemize}

\begin{itemize}
\tightlist
\item
  \href{https://www.nytco.com/}{NYTCo}
\item
  \href{https://help.nytimes.com/hc/en-us/articles/115015385887-Contact-Us}{Contact
  Us}
\item
  \href{https://www.nytco.com/careers/}{Work with us}
\item
  \href{https://nytmediakit.com/}{Advertise}
\item
  \href{http://www.tbrandstudio.com/}{T Brand Studio}
\item
  \href{https://www.nytimes.com/privacy/cookie-policy\#how-do-i-manage-trackers}{Your
  Ad Choices}
\item
  \href{https://www.nytimes.com/privacy}{Privacy}
\item
  \href{https://help.nytimes.com/hc/en-us/articles/115014893428-Terms-of-service}{Terms
  of Service}
\item
  \href{https://help.nytimes.com/hc/en-us/articles/115014893968-Terms-of-sale}{Terms
  of Sale}
\item
  \href{https://spiderbites.nytimes.com}{Site Map}
\item
  \href{https://help.nytimes.com/hc/en-us}{Help}
\item
  \href{https://www.nytimes.com/subscription?campaignId=37WXW}{Subscriptions}
\end{itemize}
