Sections

SEARCH

\protect\hyperlink{site-content}{Skip to
content}\protect\hyperlink{site-index}{Skip to site index}

\href{https://www.nytimes.com/section/world/asia}{Asia Pacific}

\href{https://myaccount.nytimes.com/auth/login?response_type=cookie\&client_id=vi}{}

\href{https://www.nytimes.com/section/todayspaper}{Today's Paper}

\href{/section/world/asia}{Asia Pacific}\textbar{}Indian Parliament
Passes Divisive Citizenship Bill, Moving It Closer to Law

\url{https://nyti.ms/2t8Z1G7}

\begin{itemize}
\item
\item
\item
\item
\item
\end{itemize}

Advertisement

\protect\hyperlink{after-top}{Continue reading the main story}

Supported by

\protect\hyperlink{after-sponsor}{Continue reading the main story}

\hypertarget{indian-parliament-passes-divisive-citizenship-bill-moving-it-closer-to-law}{%
\section{Indian Parliament Passes Divisive Citizenship Bill, Moving It
Closer to
Law}\label{indian-parliament-passes-divisive-citizenship-bill-moving-it-closer-to-law}}

The government says it is trying to protect religious minorities from
being persecuted in Muslim countries. Indian Muslims say the bill is
discriminatory, and protests are spreading.

\includegraphics{https://static01.nyt.com/images/2019/12/11/world/11india-citizenship-1/merlin_165782520_8ca55e78-d761-437e-991f-c1ad8d56c4e4-articleLarge.jpg?quality=75\&auto=webp\&disable=upscale}

\href{https://www.nytimes.com/by/jeffrey-gettleman}{\includegraphics{https://static01.nyt.com/images/2018/10/10/multimedia/author-jeffrey-gettleman/author-jeffrey-gettleman-thumbLarge.png}}\href{https://www.nytimes.com/by/suhasini-raj}{\includegraphics{https://static01.nyt.com/images/2019/11/22/reader-center/author-Suhasini-Raj/author-Suhasini-Raj-thumbLarge.png}}

By \href{https://www.nytimes.com/by/jeffrey-gettleman}{Jeffrey
Gettleman} and \href{https://www.nytimes.com/by/suhasini-raj}{Suhasini
Raj}

\begin{itemize}
\item
  Published Dec. 11, 2019Updated Feb. 26, 2020
\item
  \begin{itemize}
  \item
  \item
  \item
  \item
  \item
  \end{itemize}
\end{itemize}

NEW DELHI --- The upper house of the Indian Parliament passed a
contentious citizenship bill on Wednesday, bringing a religiously
polarizing measure one step closer to law as new protests erupted across
the country.

The measure, called the
\href{https://www.nytimes.com/2019/12/09/world/asia/india-muslims-citizenship-narendra-modi.html}{Citizenship
Amendment Bill}, uses religion as a criterion for determining whether
illegal migrants in India can be fast-tracked for citizenship. The bill
favors members of all South Asia's major religions except Islam, and
leaders of India's 200-million-strong Muslim community have called it
blatant discrimination.

The Rajya Sabha, India's version of a senate, approved the bill in a
125-to-105 vote. It next goes to the president's desk, where it is
expected to be signed into law in the coming days.

Opponents of the legislation in India and international rights groups
have called the bill a major blow to India's long-held commitment to a
secular democracy. Officials in Prime Minister Narendra Modi's
government have insisted that the legislation would protect human
rights.

\emph{{[}Update:}
\href{http://www.nytimes.com/2020/02/26/world/asia/india-hindu-muslim-violence-modi.html}{\emph{Modi
urges calm as New Delhi violence rages for third day}}\emph{.{]}}

``The bill provides expedited consideration for Indian citizenship to
persecuted religious minorities already in India from certain contiguous
countries,'' said Raveesh Kumar, a spokesman for the Foreign Ministry.
``It seeks to address their current difficulties and meet their basic
human rights. Such an initiative should be welcomed, not criticized by
those who are genuinely committed to religious freedom.''

The bill is a central piece of a far-reaching agenda by Mr. Modi and his
Bharatiya Janata Party, which has long espoused a Hindu-centric
worldview.

India is around 80 percent Hindu, with a large Muslim minority, and many
of Mr. Modi's supporters believe India should emphasize its Hindu
identity as much as possible and become more of an overtly Hindu nation.
India's founding leaders, Mohandas K. Gandhi and Jawaharlal Nehru,
resisted this, insisting on keeping India a secular state and carving
out special protections for minorities, including Muslims.

\includegraphics{https://static01.nyt.com/images/2019/12/11/world/11india-citizenship-3/merlin_165779391_4739c2a1-a80f-472f-a2f9-8d13b662eea4-articleLarge.jpg?quality=75\&auto=webp\&disable=upscale}

Mr. Modi is popular and powerful, and his political party is one of the
most dominant forces that India has produced in decades. Still, it has
plenty of detractors, and protests over the bill have grown bigger and
more violent throughout the week.

Late Wednesday, the Indian government shut down the internet in many
districts in the state of Assam and called up the army in other areas to
quell the chaos. The police were already battling demonstrators with
water cannons and tear gas.

Protests have broken out in Aligarh, Bhopal, Deoband, Jaipur and several
cities in the northeast, where the bill could have the largest effect by
allowing Hindu migrants to settle legally in areas where locals have
expressed anti-migrant sentiments.

The Indian government has promised to protect indigenous rights in these
areas, but that has failed to quash suspicions or resentments.

More than 1,000 protesters gathered in the heart of Assam's commercial
capital, Guwahati, yelling: ``Go Back Modi!'' In other places, angry men
set fire to tires and stomped on effigies of Mr. Modi.

As the protests spread, many people debated what kind of country India
should be.

``The idea of India that emerged from the independence movement,'' said
a
\href{https://docs.google.com/document/d/1dmuimPt4jESfWqqdz361N4aQu0Je0tiuoy0yp9nVp4w/edit}{letter
signed by more than 1,000 Indian intellectuals}, ``is that of a country
that aspires to treat people of all faiths equally.''

But this bill, the intellectuals said, is ``a radical break with this
history'' and will ``greatly strain the pluralistic fabric of the
country."

The citizenship legislation, which passed through the lower house of
Parliament on Monday, follows hand in hand with a
\href{https://www.nytimes.com/2019/08/17/world/asia/india-muslims-narendra-modi.html}{divisive
citizenship test} conducted this summer in one of India's states and
possibly soon to be expanded nationwide.

All residents of the state of Assam, along the Bangladesh border, had to
produce documentary proof that they or their ancestors had lived in
India since 1971. Around two million of Assam's population of 33 million
--- a mix of Hindus and Muslims ---
\href{https://www.nytimes.com/2019/08/31/world/asia/india-muslim-citizen-list.html}{failed
to pass the test}, and these people now risk being rendered stateless.
Huge new prisons are being built to incarcerate anyone determined to be
an illegal immigrant.

\href{https://timesofindia.indiatimes.com/india/ex-armyman-working-as-cop-in-assam-declared-foreigner/articleshow/69569675.cms}{Some
of those who have been arrested} have lived in India for generations.

The citizenship bill would allow Hindus, Christians, Sikhs, Buddhists,
Parsees or Jains who have migrated from Bangladesh, Pakistan or
Afghanistan a clear path to naturalization in India.

Image

India's home minister, Amit Shah, center, has vowed two things: to
impose the citizenship tests nationwide and to protect Hindus knocked
off the citizenship rolls.Credit...Prakash Singh/Agence France-Presse
--- Getty Images

Migrants who are Muslim --- which might include people who have lived in
India for generations but were unable to produce an old property deed or
birth certificate to prove it --- would not be afforded the same
protection.

The bill excludes Muslim members of religious minorities from
neighboring countries, such as the
\href{https://www.nytimes.com/2017/10/11/world/asia/rohingya-myanmar-atrocities.html}{Rohingya
who have been persecuted ruthlessly} in neighboring Myanmar.

International organizations have seized on that in criticizing the
legislation.

The
\href{https://www.uscirf.gov/news-room/press-releases-statements/uscirf-raises-serious-concerns-and-eyes-sanctions}{United
States Commission on International Religious Freedom}, a federal body,
called the measure a ``dangerous turn in the wrong direction'' and said
that the United States should consider sanctions against India if the
bill passes.

Indian officials and other supporters of the bill cite attacks on Hindus
in Bangladesh, Pakistan and Afghanistan, all of which are predominantly
Muslim, and the shrinking Hindu populations in those countries.

Kanwal Sibal, a former foreign secretary and a supporter of Mr. Modi's
foreign policy, said the bill ``certainly expresses a pro-Hindu, Sikh,
Buddhist sentiment for objective reasons as they are a beleaguered
community with no other option.''

But, he said, the citizenship bill has ``nothing to do with the Muslims
of India. It relates to foreign Muslims who have infiltrated into India
over the years."

The big question many Muslims in India are now asking is: Who will be
considered an Indian citizen? And who will be considered an illegal
foreigner?

Many of the people who failed to pass the citizenship test in Assam had
lived in India all their lives, felt deeply Indian and were despondent
to be stricken from the citizenship rolls. Some even killed themselves,
including a 14-year-old girl from a small village.

Where all of this is going is hardly clear. Even if India rounds up
thousands of people the government considers illegal migrants and puts
them into the newly built prisons, it will not be easy to deport them.

Bangladesh, Pakistan and Afghanistan will be unlikely to accept India's
determination of citizenship, especially if it seems that most of those
being slated for removal are Muslims.

India's home minister, Amit Shah, considered the second most powerful
man in the country, has vowed two things: to bring the citizenship tests
nationwide and to protect Hindus knocked off the citizenship rolls.

Harsh Mander, a well-known human rights defender, said in
\href{https://indianexpress.com/article/opinion/columns/this-land-is-mine-citizenship-amendment-bill-6160570/}{a
newspaper column} that these measures posed ``the gravest threat to
India's secular democratic constitution since India became a republic.''

Mr. Mander, who said he was born a Sikh, said that if the bill passed he
would declare himself a Muslim to stand in solidarity with ``my
undocumented Muslim sisters and brothers."

Advertisement

\protect\hyperlink{after-bottom}{Continue reading the main story}

\hypertarget{site-index}{%
\subsection{Site Index}\label{site-index}}

\hypertarget{site-information-navigation}{%
\subsection{Site Information
Navigation}\label{site-information-navigation}}

\begin{itemize}
\tightlist
\item
  \href{https://help.nytimes.com/hc/en-us/articles/115014792127-Copyright-notice}{©~2020~The
  New York Times Company}
\end{itemize}

\begin{itemize}
\tightlist
\item
  \href{https://www.nytco.com/}{NYTCo}
\item
  \href{https://help.nytimes.com/hc/en-us/articles/115015385887-Contact-Us}{Contact
  Us}
\item
  \href{https://www.nytco.com/careers/}{Work with us}
\item
  \href{https://nytmediakit.com/}{Advertise}
\item
  \href{http://www.tbrandstudio.com/}{T Brand Studio}
\item
  \href{https://www.nytimes.com/privacy/cookie-policy\#how-do-i-manage-trackers}{Your
  Ad Choices}
\item
  \href{https://www.nytimes.com/privacy}{Privacy}
\item
  \href{https://help.nytimes.com/hc/en-us/articles/115014893428-Terms-of-service}{Terms
  of Service}
\item
  \href{https://help.nytimes.com/hc/en-us/articles/115014893968-Terms-of-sale}{Terms
  of Sale}
\item
  \href{https://spiderbites.nytimes.com}{Site Map}
\item
  \href{https://help.nytimes.com/hc/en-us}{Help}
\item
  \href{https://www.nytimes.com/subscription?campaignId=37WXW}{Subscriptions}
\end{itemize}
