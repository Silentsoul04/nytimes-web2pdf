Sections

SEARCH

\protect\hyperlink{site-content}{Skip to
content}\protect\hyperlink{site-index}{Skip to site index}

\href{https://www.nytimes.com/section/us}{U.S.}

\href{https://myaccount.nytimes.com/auth/login?response_type=cookie\&client_id=vi}{}

\href{https://www.nytimes.com/section/todayspaper}{Today's Paper}

\href{/section/us}{U.S.}\textbar{}Boston Marathon Bomber's Death
Sentence May Depend on What His Jurors Tweeted

\url{https://nyti.ms/36ud9Ih}

\begin{itemize}
\item
\item
\item
\item
\item
\end{itemize}

Advertisement

\protect\hyperlink{after-top}{Continue reading the main story}

Supported by

\protect\hyperlink{after-sponsor}{Continue reading the main story}

\hypertarget{boston-marathon-bombers-death-sentence-may-depend-on-what-his-jurors-tweeted}{%
\section{Boston Marathon Bomber's Death Sentence May Depend on What His
Jurors
Tweeted}\label{boston-marathon-bombers-death-sentence-may-depend-on-what-his-jurors-tweeted}}

At an appeals hearing in Boston, lawyers for Dzhokhar Tsarnaev,
condemned to death in the bombing, argued that trial errors corrupted
the case.

\includegraphics{https://static01.nyt.com/images/2019/12/12/us/12marathonbombing1/12marathonbombing1-articleLarge.jpg?quality=75\&auto=webp\&disable=upscale}

\href{https://www.nytimes.com/by/ellen-barry}{\includegraphics{https://static01.nyt.com/images/2018/10/08/multimedia/author-ellen-barry/author-ellen-barry-thumbLarge.png}}\href{https://www.nytimes.com/by/kate-taylor}{\includegraphics{https://static01.nyt.com/images/2018/02/20/multimedia/author-kate-taylor/author-kate-taylor-thumbLarge.jpg}}

By \href{https://www.nytimes.com/by/ellen-barry}{Ellen Barry} and
\href{https://www.nytimes.com/by/kate-taylor}{Kate Taylor}

\begin{itemize}
\item
  Published Dec. 12, 2019Updated Dec. 13, 2019
\item
  \begin{itemize}
  \item
  \item
  \item
  \item
  \item
  \end{itemize}
\end{itemize}

BOSTON --- During the days after bombs exploded at the Boston Marathon
in 2013, killing and maiming people who had gathered to cheer on
runners, a restaurant manager from Dorchester joined the chorus of
heartbreak and outrage on Twitter.

``:-( RIP little man,'' wrote the woman, who used the handle
HerLadyship, of an 8-year-old boy from her neighborhood who had been
killed. She sympathized with Twitter friends forced to ``shelter in
place'' during the manhunt, but also said ``it's worse having to work
knowing your family is locked down!''

When a 19-year-old named Dzhokhar Tsarnaev was arrested after the
attack, HerLadyship retweeted a post from someone who praised ``all of
the law enforcement professionals who went through hell to bring in that
piece of garbage.''

More than six years later, those casual posts --- by the woman who
became the forewoman of Mr. Tsarnaev's jury --- could become the basis
for reversing his death sentence.

At an appeal hearing on Thursday, a panel of federal judges raised sharp
questions about whether Judge George A. O'Toole, who presided over Mr.
Tsarnaev's trial in 2015, had adequately screened jurors for bias.

They zeroed in on a moment when Judge O'Toole learned that two sitting
jurors had failed to disclose tweets and Facebook posts about Mr.
Tsarnaev, and opted not to question them in detail about it or remove
them from the jury.

``It's just very puzzling,'' said Judge William J. Kayatta Jr., in
remarks to a government prosecutor. ``You have a defendant who is
clearly guilty of this heinous crime and you then stretch and don't try
to follow the rules that we've laid down for a trial.''

The line of questioning was echoed by Judge Ojetta Rogeriee Thompson,
who asked why the judge did not screen jurors in detail about where they
received information about Mr. Tsarnaev.

``Why isn't this the kind of case that would require probing that kind
of information,'' she said, ``such that not doing so, even if it's not
an abuse of law, is simply an abuse of discretion because the
circumstances simply require it?''

Their questions offered a grain of hope to Mr. Tsarnaev's defense team,
which has argued that it was impossible to select an impartial jury in
Boston, a city that had been steeped in powerful emotions over the
bombings and deluged with pretrial publicity.

The federal death sentence was a rare event in Massachusetts, which has
no death penalty for state crimes.

Mr. Tsarnaev and his older brother set down two pressure-cooker bombs
packed with nails and BBs in a crowd that had gathered to cheer on
marathon runners on April 13, 2013.

The bombs killed three people and injured 260 more, many of them
grievously. Seventeen people lost limbs. A fourth person, a law
enforcement officer, was killed a few days later as the brothers were
fleeing.

Ahead of the trial, two years later, Judge O'Toole
\href{https://www.nytimes.com/2014/09/25/us/trial-of-marathon-bombing-suspect-to-remain-in-boston-judge-rules.html?searchResultPosition=6}{denied
three motions} for a change of venue, arguing that he could easily
select impartial jurors from a pool of five million people.

The defense has argued that two jurors had clearly demonstrated
prejudice against Mr. Tsarnaev.

Before the trial, Juror 286, a restaurant manager who was chosen as
forewoman of the jury, had tweeted or retweeted 22 posts about the
bombing, including the one that described Mr. Tsarnaev as ``a piece of
garbage,'' Mr. Tsarnaev's lawyers say. Efforts by The New York Times to
reach the forewoman were unsuccessful.

Juror 138, a man who worked for the Peabody Water Department, had posted
on Facebook that he was in the jury pool for the case. One friend wrote
that ``if you're really on jury duty, this guys got no shot in hell.''
Another wrote, ``play the part so u get on the jury then send him to
jail where he will be taken care of.''

On the day of sentencing, the defense team's brief said, Juror 138 said
on Twitter that Mr. Tsarnaev was ``scum'' and ``trash,'' and that he
belonged in a ``dungeon where he will be forgotten about until his time
comes.'' The former juror declined a request for comment from The Times.

Thursday's arguments represent the first step in a process that will
probably last for years. The arguments are constrained to the trial
record, and must establish that the judge or prosecutors erred in some
way that is significant enough to merit a reversal, a high bar since
judges are typically granted broad discretion.

Discussions of the legal questions returned again and again to social
media, and the increasing difficulty of sealing off jurors from
prejudicial information.

Daniel Habib, a lawyer for Mr. Tsarnaev, said the screening rules were
from an era before social media, and were even more necessary now, when
potential jurors are immersed in ``all manner of opinion and fact and
suggestion and innuendo about this case.''

``It wasn't just that they were reading The Boston Globe,'' Mr. Habib
said. ``They were hearing from their family and their friends and
complete strangers on Twitter, on Facebook, on Instagram, and any other
social media they participated in, thoughts about the case and beliefs
about Tsarnaev.''

Sources of pretrial publicity, he said, ``have multiplied, and are less
checked.''

Prosecutors have pushed back against the defense arguments, saying that
Juror 138 ``never endorsed'' the ``flippant and joking remarks'' left by
his friends on Facebook, and that Juror 286 may have failed to disclose
her Twitter posts because she misunderstood the instruction.

George Vien, a former federal prosecutor who now works at the law firm
Donnelly, Conroy \& Gelhaar, said he did not believe that the evidence
of prejudice was strong.

``I think everyone should stay off social media, but I don't find it
compelling,'' he said. ``If you look at the overwhelming evidence
against him, and all the aggravating factors, it's easy to understand
why an impartial jury would have come down on the side of the death
penalty.''

One of the jurors, Kelley A. McCarthy, said in an interview that she
believed that the jury had been open-minded and had not been biased by
media coverage about the bombings.

``That was one thing the judge was very careful about, that there wasn't
that bias,'' she said. ``I think we did it fairly, rationally.''

She said she was surprised to learn of social media posts by the
forewoman, and said that the forewoman had not had a significant
influence on other members of the jury.

``She was not a huge influencer on that trial,'' she said. ``I mean
every single person on the trial had their own free will, and she wasn't
somebody who was persuading anybody, other than what the facts were.''

Liz Norden, the mother of two men who lost legs in the bombing, said a
reversal of the verdict or sentence would be a bitter disappointment.

``I think he got a fair trial,'' she said. ``His defense attorney said
he did it. He said he did it. There was no place in the world where you
could take it where it would be different. He terrorized a nation.''

Ms. Norden, who is 57, said seeing Mr. Tsarnaev executed is ``my only
hope in life now.''

``For the human part of me, it's a sad situation to wish death on
anyone,'' she said. ``That's not in my makeup. I'm not an eye for an
eye. But in this instance, I felt from Day 1 if he was found guilty,
that's what I wanted. That's what I thought justice would be.''

She added, ``I personally hope I live long enough to see it.''

Alain Delaquérière contributed research.

Advertisement

\protect\hyperlink{after-bottom}{Continue reading the main story}

\hypertarget{site-index}{%
\subsection{Site Index}\label{site-index}}

\hypertarget{site-information-navigation}{%
\subsection{Site Information
Navigation}\label{site-information-navigation}}

\begin{itemize}
\tightlist
\item
  \href{https://help.nytimes.com/hc/en-us/articles/115014792127-Copyright-notice}{©~2020~The
  New York Times Company}
\end{itemize}

\begin{itemize}
\tightlist
\item
  \href{https://www.nytco.com/}{NYTCo}
\item
  \href{https://help.nytimes.com/hc/en-us/articles/115015385887-Contact-Us}{Contact
  Us}
\item
  \href{https://www.nytco.com/careers/}{Work with us}
\item
  \href{https://nytmediakit.com/}{Advertise}
\item
  \href{http://www.tbrandstudio.com/}{T Brand Studio}
\item
  \href{https://www.nytimes.com/privacy/cookie-policy\#how-do-i-manage-trackers}{Your
  Ad Choices}
\item
  \href{https://www.nytimes.com/privacy}{Privacy}
\item
  \href{https://help.nytimes.com/hc/en-us/articles/115014893428-Terms-of-service}{Terms
  of Service}
\item
  \href{https://help.nytimes.com/hc/en-us/articles/115014893968-Terms-of-sale}{Terms
  of Sale}
\item
  \href{https://spiderbites.nytimes.com}{Site Map}
\item
  \href{https://help.nytimes.com/hc/en-us}{Help}
\item
  \href{https://www.nytimes.com/subscription?campaignId=37WXW}{Subscriptions}
\end{itemize}
