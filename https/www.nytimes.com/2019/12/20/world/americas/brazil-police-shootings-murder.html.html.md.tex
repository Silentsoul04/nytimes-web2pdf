Sections

SEARCH

\protect\hyperlink{site-content}{Skip to
content}\protect\hyperlink{site-index}{Skip to site index}

\href{https://www.nytimes.com/section/world/americas}{Americas}

\href{https://myaccount.nytimes.com/auth/login?response_type=cookie\&client_id=vi}{}

\href{https://www.nytimes.com/section/todayspaper}{Today's Paper}

\href{/section/world/americas}{Americas}\textbar{}Where the Police Wear
Masks, and the Bodies Pile Up Fast

\url{https://nyti.ms/2PEGrhK}

\begin{itemize}
\item
\item
\item
\item
\item
\item
\end{itemize}

Advertisement

\protect\hyperlink{after-top}{Continue reading the main story}

Supported by

\protect\hyperlink{after-sponsor}{Continue reading the main story}

\hypertarget{where-the-police-wear-masks-and-the-bodies-pile-up-fast}{%
\section{Where the Police Wear Masks, and the Bodies Pile Up
Fast}\label{where-the-police-wear-masks-and-the-bodies-pile-up-fast}}

The police killed an average of 17 people every day in Brazil last year,
and rogue officers are killing even more off duty. ``I'm a hero to my
people,'' one militia leader said.

\includegraphics{https://static01.nyt.com/images/2019/12/18/world/Brazil/Brazil-articleLarge.jpg?quality=75\&auto=webp\&disable=upscale}

By \href{https://www.nytimes.com/by/azam-ahmed}{Azam Ahmed}

\begin{itemize}
\item
  Published Dec. 20, 2019Updated Dec. 21, 2019
\item
  \begin{itemize}
  \item
  \item
  \item
  \item
  \item
  \item
  \end{itemize}
\end{itemize}

\href{https://www.nytimes.com/es/2019/12/26/espanol/america-latina/policias-brasil-ejecuciones.html}{Leer
en español}

BELÉM, Brazil --- The masked gunmen pulled up to Wanda's Bar at 3:49
p.m. on May 19 and began firing the moment they left their vehicles. Two
people, including Wanda herself, died on the patio.

Inside, the gunmen worked in silence: two in front, shooting unarmed
patrons at the bar and in the main room, while a third followed behind
with a gun in each hand, firing a single shot into the head of anyone
still moving.

When the massacre ended, 11 people lay dead, slumped over the bar,
draped across chairs or huddled on the floor. Only two people survived,
one by hiding under a friend's lifeless body, case files show.

Once again, masked gunmen had struck in the Brazilian city of Belém, as
they have for nearly a decade, stalking the streets in open defiance of
the law. Robbing, extorting and killing without compunction.

Yet they did not belong to one of the many gangs that traffic drugs or
guns in Brazil, leaving a trail of corpses.

They were cops.

\includegraphics{https://static01.nyt.com/images/2019/12/18/world/Brazil-02/Brazil-02-articleLarge.jpg?quality=75\&auto=webp\&disable=upscale}

The killings drew national attention to the police militias that have
long plagued Belém, a dilapidated port city on the Amazon River. Part
death squad, part criminal enterprise, their ranks are filled with
retired and off-duty police officers who kill at will, often with total
impunity.

VENEZUELA

FRENCH

GUIANA

Atlantic

Ocean

Belém

PARÁ

STATE

Amazon R.

BRAZIL

BOLIVIA

Rio de Janeiro

ARGENTINA

500 miles

By The New York Times

In fact, the slaughter at Wanda's Bar was not unique because off-duty
police officers gunned down civilians without cause. Such killings are
routine. What made this case stand out beyond its brutality was the
government's response: It decided to prosecute.

Of the seven people charged with the crime, four were off-duty police
officers --- including the three suspected gunmen.

``We've discovered a cancer inside the police,'' said Armando Brasil,
one of the prosecutors. ``Now, we are seeing just how far it has
spread.''

The militias operate in the shadows of a
\href{https://www.nytimes.com/2019/05/26/world/americas/brazil-rio-police-kill.html}{severe
crackdown} on crime by the Brazilian government, which has openly
declared war on the gangs, thieves and drug dealers afflicting the
nation. Killings by the police have soared in recent years, as a force
long known for its deadliness has managed to outdo itself.

The number of people officially killed by the police reached a five-year
high last year, rising to 6,220 --- an average of 17 people each day,
according to the Brazilian Public Security Forum, which compiles
government data. Police killings may exceed that this year,
\href{https://www.nytimes.com/2018/11/01/world/americas/bolsonaro-police-kill-criminals.html}{coaxed
on by President Jair Bolsonaro} and his contention that criminals should
``die like cockroaches.''

Image

Residents at the scene where a young man was shot twice and killed by an
assailant in Belém.Credit...Tyler Hicks/The New York Times

The deaths have stirred a familiar debate in Brazil. Human rights
advocates denounce the heavy-handed approach as both inhumane and
ineffective, while proponents say it is the only way to confront a crime
wave that has put the entire nation at risk.

But even police officers acknowledge that the official statistics are
only part of the picture.

There is a parallel form of police violence, masked from the public and
carried out by illegal militias that draw their ranks from officers with
little patience or respect for due process, according to interviews with
militia members here in Belém.

By their own admission, groups of off-duty and retired officers
regularly commit extrajudicial killings, targeting people they consider
criminals, robbers and cop killers without so much as an arrest warrant.

``We're going after criminals who hurt innocent people,'' said one
militia commander who, like others, asked that his name be withheld
because he confessed to extrajudicial killings.

In their telling, militia members are delivering a public service,
eliminating threats to society who, they fear, may never get convicted
or will simply participate in sprawling criminal networks from prison,
\href{https://www.nytimes.com/2006/05/30/world/americas/30brazil.html}{as
often happens in Brazil}.

``I've killed more than 80 criminals in my time as a police officer,''
said another militia leader. ``I'm a hero to my people. They love me.''

Latin America is in the midst of a homicide crisis. More killings take
place in the region's five most violent nations than in every major war
zone combined, according to the Igarapé Institute, which tracks violence
worldwide.

The usual suspects are often to blame:
\href{https://www.nytimes.com/2019/12/14/world/americas/sicario-mexico-drug-cartels.html}{the
cartels and gangs}, the surfeit of guns,
\href{https://www.nytimes.com/2019/08/25/world/americas/one-handgun-9-murders-how-american-firearms-cause-carnage-abroad.html}{frequently
from the United States},
\href{https://www.nytimes.com/2019/08/18/world/americas/guatemala-violence-women-asylum.html}{the
paralyzed legal systems}.

But violence by the state is another important factor in the bloodshed
--- driven by an abiding belief that nations must fight force with
ruthless force to find peace.

Image

A young man shot twice and killed by an assailant in
Belém.Credit...Tyler Hicks/The New York Times

In Brazil, El Salvador, Mexico and other countries, the use of deadly
force by the authorities --- and the acceptance, or even applause, by
the population for that approach --- is so widespread that even the
public statistics point to an abundance of extrajudicial killings,
researchers say.

In many dangerous places, even when gangs and organized crime are very
well armed, it is not surprising that criminals die in greater numbers
than the police or military they are fighting, researchers say.

But when that ratio is highly skewed --- and 10 or more suspected
criminals die for every police officer or soldier killed --- researchers
often view that as a clear indication of excessive force by the
authorities.

In El Salvador, where the government is battling the gangs, the ratio is
staggering --- almost 102 to 1 --- according to the Lethal Force
Monitor, a research group that tracks the rates across several Latin
American countries. In other words, for every policeman killed in El
Salvador, nearly 102 suspected criminals die --- 10 times the level
researchers consider suspiciously high.

In Brazil, the number is also striking: 57 suspected criminals die for
every police officer killed, the analysts found.

``We believe that homicides are not a problem, they're a solution,''
said Bruno Paes Manso, a researcher at the University of São Paulo,
describing the public acceptance of killings by the police.

``There is a strong belief that violence promotes order,'' he added.
``And the militias thrive off this feeling.''

Image

The burial of Vinicius Santos Lobo, 18, who was killed by an
unidentified man in Belém.Credit...Tyler Hicks/The New York Times

But extrajudicial killings are often much more than an extreme step by
overzealous officers in cities like Belém and Rio de Janeiro, and some
militia members are candid about their criminal motivations.

To line their pockets, some militia members say they bill businesses for
security services, taking in hefty sums with mafia-style promises to
keep the peace, or they charge local residents for the right to engage
in basic commerce, like selling cooking gas or pizzas.

The militias also extort criminals and kill those who don't pay,
operations that hardly differ from the ones they are supposedly
confronting.

``It became explicit for me,'' said a third militia member. ``It became
organized crime.''

Today in Belém, there are hundreds of militia members operating in more
than a dozen different factions, often with help from on-duty police
officers, according to officials and militia members themselves. And
until recently, officials say, the government rarely prosecuted or
investigated them aggressively.

The government of Pará State, where Belém is the capital, says most
police officers ``do not deviate from their duties,'' but acknowledges
that others do. It says it has arrested about 50 officers this year in
operations ``to dismantle criminal organizations involving public
security agents.''

The prosecutor investigating the massacre at Wanda's Bar, Mr. Brasil,
has linked the militias to at least 100 murders in the state in the last
three years, but he thinks the actual number is much higher.

``They've killed way more than that,'' said Mr. Brasil, who has
bodyguards because he is going after the militias. ``It's well into the
hundreds.''

Image

Members of the elite ROTAM police force after their unit killed a
suspected drug dealer in Belém.Credit...Tyler Hicks/The New York Times

\hypertarget{i-felt-like-an-instrument-of-justice}{%
\subsection{`I felt like an instrument of
justice'}\label{i-felt-like-an-instrument-of-justice}}

He took his first life in 2010, a few years out of the police academy,
after a gang called the Red Command killed his colleague.

He and other officers shed their uniforms, put on masks and killed a
dozen people they deemed responsible or connected in some way, he said.

After that, every time an officer was killed, he said, he and his fellow
officers killed at least 10 suspected gang members in response. If
violence was the language of the streets, their message would be the
loudest.

Residents took notice, he said, and in 2012 a father in his neighborhood
asked for help. A man had raped his daughter and was still walking free.

He asked if the officer would kill the man, to end his family's
nightmare. When it was done and the suspect was dead, the officer said,
the father wept with gratitude and offered money.

He refused at first, then accepted it.

``It was the first time I felt like a hero,'' said the officer. ``I felt
like an instrument of justice.''

From there, it was a short jump to becoming a contract killer, the
officer said. Each step away from the law grew easier. Soon, the
self-declared principles that marked the start of his militia activity
were gone.

By 2014, the officer said, he was robbing drug dealers, kidnapping and
torturing them when they resisted. His hatred of criminals justified
just about anything, even killing innocent civilians accidentally. He
said he came to embody the thing he hated most.

By that time, he said, militias were operating all over Belém. Some were
strictly about killing known criminals. Others were about making money.

Then in 2014, one of the most powerful militia members in Belém, Antônio
Figueiredo, was gunned down in the street. The militias took his death
personally, three members said, and decided to respond.

On the night of Nov. 4, 2014, they retaliated, killing at least 10
people. But the revenge was reckless, sweeping up innocents as masked
officers unleashed their rage.

The officer said he joined a team on motorcycles that went to the Terra
Firme neighborhood, an area of mud streets and open sewage canals. He
said he watched as a fellow officer dismounted, raised his weapon and
fired at a teenager in a baseball cap.

The teenager, Eduardo Chaves, 16, was the first person gunned down in
the massacre that night. At the time, his family said, he was leaving
church with his grandparents and girlfriend. It was shortly after 9 p.m.

The masked officer shot Eduardo five times, killing him, while the
others watched.

Image

Eduardo Chaves, 16, was gunned down on this corner in 2014, by the wall
where the man with the umbrella is walking.Credit...Tyler Hicks/The New
York Times

``He was a kid,'' the officer said. ``I knew he was innocent and I knew
things were getting out of control. But I was so full of anger I didn't
say anything.''

``By that point, I was already hard-core,'' he said. ``I didn't feel
anything.''

The boy's relatives said they ran to the scene and found his body in the
mud. His grandmother, Maria Auxiliadora Neves, said she wept as she
collected his silver necklace, his cellphone and the few dollars he had
saved to buy his girlfriend a pair of sandals.

In the aftermath, Mrs. Neves began to speak out about his murder, a risk
even the police warned her against. She became an activist, calling
attention to police shootings across Belém.

And then, it happened to her family again.

On New Year's Day, 2016, Danilo de Campos Galucio, another of her
grandsons, was shot, this time by men in an unmarked car, she said.
Investigators call that a telltale sign of a militia shooting.

The bullet passed through several organs and left him debilitated, at
15. He spent the next four years in and out of the hospital undergoing
surgeries. Bedridden and depressed, he tried to kill himself twice.

This September, he died at 19, having succumbed to medical complications
related to the shooting.

``I never paid attention to this before because it never affected me,''
his grandmother said, referring to the killings by militias, which she
once assumed were justified. ``I don't want revenge. I want justice.''

Image

Maria Auxiliadora Neves became an activist, calling attention to police
shootings across Belém.Credit...Tyler Hicks/The New York Times

\hypertarget{the-deadly-toll-of-one-day-in-beluxe9m}{%
\subsection{The deadly toll of one day in
Belém}\label{the-deadly-toll-of-one-day-in-beluxe9m}}

Officially, the police here in Pará State killed 626 people last year
--- a dozen each week.

That's more than 150 times the number of
\href{https://www.nytimes.com/2019/10/25/nyregion/police-involved-shooting-brooklyn.html}{deadly
police shootings} in all of New York City last year, even though they
are roughly the same size.

In Belém, the state capital, the people killed by the police are
disproportionately poor people of color, as they are elsewhere in
Brazil. Nationwide, researchers say, 75 percent of the people shot and
killed by the police are black.

Those factors --- the frequency of official police shootings and the
marginalized status of the people shot --- add to an atmosphere in which
death by the police seems common, almost inevitable, experts say, laying
the groundwork for the militias to operate with relative ease.

Over the course of a week, The New York Times tracked seven police
shootings in Belém, with nine casualties. This is a snapshot of just one
day.

On Nov. 16, three young men tried to rob a clothing store. But the
building belonged to a police officer, a member of the elite ROTAM
force, known for its military culture and hyperviolence.

The officer, who was home at the time, saw the men enter the store on
his security cameras and took them on himself, according to the police.
As they left the store, he opened fire, shooting two of the men --- one
in the hand and the other in the head.

The officer stood outside, shirtless, clutching a revolver with a streak
of blood smeared on his abdomen as the young man he shot in the head was
rushed to the hospital. He survived.

Image

A police officer who is a member of the elite ROTAM~ force described the
attempted robbery of a clothing store last month. Credit...Tyler
Hicks/The New York Times

Less than an hour later, an image of the young man's face appeared on a
WhatsApp group shared by militiamen, police officers and sympathizers.
In case he evaded justice somehow, they would all know who he was,
according to a person included in the group.

That evening, two men stole an S.U.V. and exchanged gunfire with the
police as they tried to escape. The officers fired three shots into the
vehicle. When it stopped, one of the men was taken into custody,
witnesses said, adding that he appeared injured but could walk.

An hour later, when he arrived at the hospital, he was dead, with a
gunshot wound to the heart, a photo of the body showed.

``I don't know whether they executed him, and I don't want to know,''
his sister said on condition of anonymity, fearful of reprisals from the
police. ``The police here do what they want.''

Later that night, Ramon Silva Oliveira, 18, was also killed. He and a
friend were coming home from a party, sharing a motorcycle, when the
police tried to stop them, the family said.

Ramon, they said, was young, black and had a large tattoo, which
officers here openly admit arouses suspicion. But he was no gang member,
his family said. He had applied to join the military and, in the
meantime, was looking for work. He played soccer well. Medals hung from
the walls of his room like ornaments.

But that night, his friend, who was driving the motorcycle, decided to
keep going. The police fired at the two young men, striking Ramon and
forcing the motorcycle to fall over. He died almost immediately.

``I don't know whether the gunshot wound killed him or the fall,'' said
his mother, Marlene Silva de Oliveira, folded in grief. ``I didn't have
the heart to go and look at his body.''

The family held a wake for him the following evening, next to a plot of
grass where children played soccer.

Image

A family wake for Ramon Silva Oliveira, 18, who was killed in a police
shooting in the Marituba neighborhood of Belém.Credit...Tyler Hicks/The
New York Times

\hypertarget{allow-us-to-kill-anyone}{%
\subsection{`Allow us to kill anyone'}\label{allow-us-to-kill-anyone}}

The arrests began days after the massacre at Wanda's Bar. Using
surveillance footage from street cameras, investigators found the
gunmen's car at a local repair shop.

The owner was trying to get some work done to the car, to disguise it.
Soon enough, the authorities arrested four police officers --- two
hailed from the elite ROTAM force --- and three others suspected in the
crime.

Tying the murders to the police was relatively straightforward. Forensic
analysts found numerous .40 caliber shells at the scene, a bullet
available only to the military police, a prosecutor said.

But a judge in one of the cases thinks the evidence is relatively weak,
partly because prosecutors have failed to uncover a motive.

In the meantime, the bar is closed, a mausoleum to the events of May 19,
and residents remain terrified. Some of the accused lived nearby --- and
their friends still do.

The fear is so palpable that not a single family member of the deceased
agreed to be interviewed. Some have moved, others changed phones and
those still around refused to answer their doors or respond to messages.

But a close family friend of the bar's owner, Maria Ivanilza Pinheiro
Monteiro, known widely as Wanda, contended that everyone in the bar was
innocent. They were all friends, partying, and the bar itself was a
haunt for lots of militia guys, he said on the condition of anonymity
for fear for his life.

That's why the motive is so elusive, he said. The bar had been around
for 15 years. They all knew the militias, or were even friends with
them. Some of the people killed in the attack actually supported what
the militias did, he said, thinking it was the only way to clean up the
community.

In fact, the friend still felt that way. He believed that rogue cops
were the best way to rid Belém of crime. Even with many of his friends
now dead, he still clung to the belief that the militias were a
``necessary evil.''

``They make life easier for the good people,'' he said. ``Overall, I
still think they are a force for good.''

The militia men interviewed for this article all felt the killings at
Wanda's Bar were inexcusable, but they defended the militias in general.
To them, violence was the only solution, and the only question was how
to wield it.

``There's a way to fix this,'' said one of the militia leaders. ``The
governor should call the good cops and let us go and allow us to kill
anyone. Only the bad people, the criminals, those who prey on the
weak.''

``That will finish the violence once and for all,'' he said.

Yan Boechat contributed reporting from Belém.

Advertisement

\protect\hyperlink{after-bottom}{Continue reading the main story}

\hypertarget{site-index}{%
\subsection{Site Index}\label{site-index}}

\hypertarget{site-information-navigation}{%
\subsection{Site Information
Navigation}\label{site-information-navigation}}

\begin{itemize}
\tightlist
\item
  \href{https://help.nytimes.com/hc/en-us/articles/115014792127-Copyright-notice}{©~2020~The
  New York Times Company}
\end{itemize}

\begin{itemize}
\tightlist
\item
  \href{https://www.nytco.com/}{NYTCo}
\item
  \href{https://help.nytimes.com/hc/en-us/articles/115015385887-Contact-Us}{Contact
  Us}
\item
  \href{https://www.nytco.com/careers/}{Work with us}
\item
  \href{https://nytmediakit.com/}{Advertise}
\item
  \href{http://www.tbrandstudio.com/}{T Brand Studio}
\item
  \href{https://www.nytimes.com/privacy/cookie-policy\#how-do-i-manage-trackers}{Your
  Ad Choices}
\item
  \href{https://www.nytimes.com/privacy}{Privacy}
\item
  \href{https://help.nytimes.com/hc/en-us/articles/115014893428-Terms-of-service}{Terms
  of Service}
\item
  \href{https://help.nytimes.com/hc/en-us/articles/115014893968-Terms-of-sale}{Terms
  of Sale}
\item
  \href{https://spiderbites.nytimes.com}{Site Map}
\item
  \href{https://help.nytimes.com/hc/en-us}{Help}
\item
  \href{https://www.nytimes.com/subscription?campaignId=37WXW}{Subscriptions}
\end{itemize}
