Sections

SEARCH

\protect\hyperlink{site-content}{Skip to
content}\protect\hyperlink{site-index}{Skip to site index}

\href{/section/world/middleeast}{Middle East}\textbar{}Where Doctors Are
Criminals

\href{https://nyti.ms/2Z9HWYE}{https://nyti.ms/2Z9HWYE}

\begin{itemize}
\item
\item
\item
\item
\item
\item
\end{itemize}

\includegraphics{https://static01.nyt.com/images/2019/12/20/world/00syria-medicalprofile-4/merlin_165020136_d7b68879-0263-49a3-91cd-4e23c12dd66e-articleLarge.jpg?quality=75\&auto=webp\&disable=upscale}

\hypertarget{where-doctors-are-criminals}{%
\section{Where Doctors Are
Criminals}\label{where-doctors-are-criminals}}

The Syrian government considers some health workers enemies of the
state. We talked to four of them about why they risked their lives
anyway.

Dr. Ahmed in a park in Gaziantep, Turkey in Nov. 2019. He was detained
by the Syrian military and subjected to beatings, electric shocks and
mock executions.Credit...The New York Times

Supported by

\protect\hyperlink{after-sponsor}{Continue reading the main story}

There was the medical student who volunteered in eastern Aleppo even
after his classmates were tortured and killed as a warning. There was
the pharmacist who smuggled drugs past government checkpoints to cancer
patients who needed them. There was the pediatrics medic who relied on
expired medicines taken from an abandoned factory.

Each took enormous risks to provide medical care to areas in Syria
aligned against President Bashar al-Assad. Some were imprisoned and
tortured, evidence of how the nearly 9-year-old conflict in Syria has
normalized the criminalization of medical care.

Physicians for Human Rights, which has documented the collapse of
Syria's health care system, said in a recently
\href{https://phr.org/our-work/resources/my-only-crime-was-that-i-was-a-doctor/}{released
study} that Mr. al-Assad has successfully made medical assistance given
to his enemies a terrorist act.

The study is based on interviews with 21 formerly detained Syrian health
care workers who have fled the country. None wished to be identified by
name, fearing retribution against their families or themselves if they
ever returned.

The New York Times independently interviewed three of them. It also
interviewed an emergency medic of an underground hospital, the subject
of \href{https://www.nationalgeographic.com/films/the-cave/\#/}{``The
Cave,''} an acclaimed 2019 documentary, who was so overcome by bombings
she abandoned her aspirations to be a pediatrician.

\begin{center}\rule{0.5\linewidth}{\linethickness}\end{center}

\hypertarget{the-medical-student-navigating-a-deadly-dystopian-odyssey}{%
\subsection{The Medical Student: Navigating a Deadly Dystopian
Odyssey}\label{the-medical-student-navigating-a-deadly-dystopian-odyssey}}

The passport clerk said it would take five minutes to answer some
questions. ``It lasted about 110 days.''

\includegraphics{https://static01.nyt.com/images/2019/12/20/world/00syria-medicalprofile-1/merlin_165020034_121d9916-f622-4cec-99d3-e0be94c0bfab-articleLarge.jpg?quality=75\&auto=webp\&disable=upscale}

By Jack Ewing and Karam Shoumali

FRANKFURT, Germany --- The Syrian medical students were well aware of
the risks when they crossed over to rebel-held districts of Aleppo in
2013. The previous year, two other students had been arrested trying to
smuggle bandages and painkillers through a checkpoint. A week later the
security services told other students to collect the corpses, which had
holes in their foreheads, tongues and eyes from a power drill.

``That was a message for all the medical students,'' said a former
student who asked that his name not be used because of fear of
retaliation. ```If you do something against us, this is the result.'''

Still, he and a friend decided to go. ``There were no doctors at all in
eastern Aleppo. The aerial strikes were really intense. It was a
catastrophe.''

Today the former student is working at a hospital in rural Germany where
the hills are carpeted with vineyards. Two years after arriving he
speaks fluent German and is studying for an exam that will give him
status equal to a German-educated doctor. He lives with his young family
in a quiet village. He told his story in a compact living room furnished
with two soft brown couches and a large-screen television.

After making contact with other students already working in east Aleppo,
the student and a friend crossed over, pretending to visit relatives. As
a third-year medical student, he had few skills, but doctors there
taught him basics like inserting an IV needle or stitching a wound.

In the beginning, medical supplies were so scarce that surgeons
conducted an appendectomy on a young boy without anesthetics. ``That was
terrible.''

Later the situation improved as outside aid groups provided supplies and
training. A British doctor taught the Syrian surgeons how to repair a
severed artery --- essential in a war zone. The medical students visited
Turkey to learn how to treat victims of chemical warfare.

Despite their inexperience, the students admitted patients and provided
emergency treatment because the doctors were always busy operating. The
wounded were classified by color code: white for survival without
treatment, black for hopeless, yellow or red for those in between.

Some cases haunt him. A family trying to escape Aleppo by car came under
fire, killing the father and fatally wounding the two children, one cut
almost in half. ``The mother said to me, `Please don't help me, help my
children.'''

He lied to the mother that the children were fine, and the doctors
treated her. She was the only survivor.

Another time a government missile struck a marketplace and ignited cans
of fuel for sale. About 10 people came in severely burned. He and other
students pushed tubes down their throats to administer liquids and
medicine, but as far as he knows only one person survived.

After a couple of months he crossed back to west Aleppo to take his
exams. The head doctor at the hospital told him he was crazy --- the
student had been filmed by a French television crew. Undaunted, he
passed his exams and returned to east Aleppo.

Asked why he had gone back, he told a story about a mother brought in by
her children after a bombing attack in the middle of the night. She was
covered in blood and classified as ``black'' --- a hopeless case. But
the doctors revived the woman with blood transfusions and liquids, as
her children, aged 3 and 6, were curled asleep beside her. The children
awoke, overjoyed. Cases like that, he said, ``were a motivation for me
to go back.''

After about another six months, around January 2014, he left east Aleppo
again to take more exams. He applied for a passport, because it was
getting harder to cross into Turkey and he wanted more training there.

That was a mistake.

His name was on a list at the passport office. The clerk said the
student needed to answer some questions that would take five minutes. He
said ``it lasted about 110 days.''

The first night he was held with eight people in a cell measuring one
meter by two meters. He was interrogated repeatedly and accused of
providing treatment to rebels, but he was not tortured.

That changed after he was transferred to another facility in Aleppo,
which he described as a large house, operated by state security.

For the next 96 days he was detained with 35 men in a cell about as big
as his living room in Germany, or about three meters by three meters.
There wasn't room for anyone to lie down. The prisoners sat in rows,
their legs wrapped around the person in front. The first three days he
couldn't sleep. The prisoners wore only their underwear and were allowed
two bathroom trips daily. The guards counted down as the prisoners
relieved themselves.

Occasionally prisoners were hauled out. The others could hear the
screams from beatings in nearby rooms. The youngest prisoner was 14,
arrested for demonstrating. The oldest was 76, a teacher who developed a
foot infection after a beating and died.

Eventually the medical student's turn came. A muscled guard made him lie
down on the floor, hands bound. He was blindfolded and beaten with a
braided electric cord. He said the first blow was unbearable. The
beating lasted an hour.

The next day he was beaten again until he was bleeding, with broken
teeth lying on the floor. The guard wanted him to confess to giving
medical treatment to rebels.

After 96 days he and 50 other prisoners were loaded onto a bus with
blacked-out windows. Guards told them they were en route to the desert
to be shot.

``I said, `O.K., this is it. This is the end.''' But it turned out they
were en route to Damascus, where conditions improved dramatically.

He was held in a less crowded cell with a toilet. He got a haircut and
was allowed to wash and shave. The meals included eggs, vegetables and
fruit.

It turned out his parents had bribed officials the equivalent of about
\$1,650 to win his release. After 10 days he was freed.

Despite his trauma, he went back to east Aleppo. The city by then was
under constant attack, and his contacts behind the lines told him the
situation was catastrophic. The student's father tried to stop him. ``He
said, `Are you the only doctor? Please don't go.' I went anyway.''

After a short time in east Aleppo he left, finished his medical studies
and married a doctor colleague. Demoralized by the fall of Aleppo, in
2016 they became refugees bound for Germany.

That was another odyssey, including a crossing in an overcrowded
inflatable boat from Turkey to the Greek island of Chios on New Year's
Eve and the sale of his wedding ring to pay for train fare from Warsaw
to Berlin.

Now he works 7 a.m. to 7 p.m. in the hospital now, but isn't
complaining. ``People here are very nice.''

\begin{center}\rule{0.5\linewidth}{\linethickness}\end{center}

\hypertarget{the-pharmacist-smuggling-medicine-more-dangerous-than-running-guns}{%
\subsection{The Pharmacist: Smuggling Medicine More Dangerous Than
Running
Guns}\label{the-pharmacist-smuggling-medicine-more-dangerous-than-running-guns}}

``If they find a weapon in your car it will be easier for you than if
they find bags of blood, for example, or anesthesia drugs.''

Image

A pharmacist who was detained and interrogated by intelligence services
for providing medical supplies during protests in Syria.Credit...The New
York Times

By Carlotta Gall

GAZIANTEP, Turkey --- Soon after the Syria demonstrations began in
February 2011, the government started using lethal force against the
protesters, and medical personnel were pulled in to help.

A pharmacist from Damascus began handing out basic first aid supplies
because his pharmacy was in one of the suburbs where the protests first
took hold.

``People came for bandages and cotton,'' he said. ``People tried to
organize themselves. They tried to set up field hospitals, in houses, to
do some managing,'' he said. ``Some doctors tried to do that. I knew a
lot of them. A lot of them were my friends. A lot of them were
arrested.''

Wary of Syria's feared intelligence service, protesters cared for the
injured in secret, fetching medical personnel to treat them in private
homes or safe houses, not trusting the public hospitals where the police
and intelligence agents could detain wounded patients.

The pharmacist began organizing networks of medical workers. He had
experience from his student days when he had raised money to help
orphans and the sick. He began collecting drugs and medical supplies
from friends, relatives and organizations and getting them delivered.

By 2012 the protests had spread countrywide and escalated into an armed
uprising. The government had sealed off opposition-held areas including
the eastern Ghouta suburb of Damascus, the southwest city of Dara'a and
districts of the western city of Homs, preventing food and medical
deliveries by enforcing a blockade.

``The regime was preventing any help for them,'' the pharmacist said.
``The regime claimed that those people were part of the opposition
parties and militias - children, women or men, without discrimination.''

As a pharmacist, he supplied drugs to public hospitals, so he had access
to drug supplies and he carried a health ministry card, which allowed
him to drive through government checkpoints unhindered.

``Sometimes I was trying to deliver very critical drugs,'' he said. ``We
are talking about cancer, cancer affects all people, anyone can have
this disease,'' he went on. ``In the besieged areas it was a very
important intervention from my side to deliver those drugs.''

He knew an oncologist in eastern Ghouta who had chosen to stay within
the besieged suburb, and he sought ways to keep supplying drugs and
medical supplies to her hospital.

The pharmacist paid government militias to take drugs and medical
supplies across government lines, and delivered supplies near tunnels in
eastern Ghouta that the rebels had dug.

The dangers to people like him were clear, the pharmacist said. Under
President Bashar al-Assad ---
\href{https://www.biography.com/dictator/bashar-al-assad}{who was a
doctor himself, specializing in ophthalmology} --- the Syrian government
arrested medical professionals who showed any sympathy for the popular
uprising.

``If they find a weapon in your car it will be easier for you than if
they find bags of blood, for example, or anesthesia drugs,'' he said.
''Working in medicine was a very critical issue because the regime hated
us more than the people, more than the revolutionaries.''

Moreover, he said, the government mistrusted medical professionals
because they were educated and capable of independent thinking.

``They hate the educated people because we are trying to do some
organizing that is not in their way,'' he said. ``They are trying to
make all people think in the same way, what Assad needs and what Assad
wants, not against him.''

Fear of arrest did not deter him, he said. ``It is our choice, our
life,'' he said. ``As a human, we have this belief, and we have our
belief in God, as a Muslim. And it is our families and our people who
are being affected.''

One of the supply networks he had formed with a friend consisted of 10
doctors and medical personnel. They used basic security, operating in
cells, using code names. Only the leader, his friend, knew who the other
10 members were. But it turned out one member was a government
informant.

They worked for two years, longer than many medical activists, but in
July 2014, agents of the Syrian intelligence service detained the group
leader, who led them to the pharmacist.

Plainclothes intelligence officers surrounded the pharmacist outside his
office as he was getting into his car. He spotted his friend sitting in
one of their cars. They took the pharmacist home and seized his
computer, cash, and car, and ordered him to call his wife to tell her to
come home.

As they hauled him away, he recalled, the couple exchanged glances. ``I
looked at her --- it was a very sad moment,'' he said.

He was interrogated and tortured with beatings for 60 days in the 215th
branch of the Intelligence Service in Damascus.

``My interrogator asked me directly: `Where is your gun? Why are you
helping terrorists?''' The interrogator dismissed his protests that he
was a government-approved pharmacist supplying public hospitals. They
showed him a fellow member of his network who had been arrested. The
man's back had been broken after he was bent backward in a form of
torture that inmates call the German chair.

The pharmacist's ordeal reinforced to him the Syrian government's
weaponization of medical care in war.

``My interrogator told me, `We hate you more than the fighters. Why?
Because you will treat people, you will treat fighters,''' he said.

He was held in a cell so cramped that inmates had to take turns to rest.
One sat with knees bent while another stood. Disease was rife that
prisoners sometimes died in the cell.

A family of three, father, son and grandfather, died one after the
other. The father died after interrogation, and the 18-year-old son, who
had been arrested trying to buy bread at the local bakery, was so
traumatized that started biting cellmates. ``He died in the night, and
the guards did not remove his body for a whole day.''

The pharmacist ended up signing blank papers and his interrogators
filled in his confession, inventing details that he had stored weapons
in a mountain cave, had treated fighters and knew the leaders of Al
Qaeda and other militias.

If the government had really believed such accusations his captors would
never have let him out alive, he said. ``They know I am not like that,''
he said. Instead, they took a bribe of \$10,000 from the pharmacist's
family to gain his release. A few months later he paid \$2,000 for him
and his wife to be smuggled out of Damascus and into Turkey.

He lives in a modern apartment block in the city of Gaziantep, not far
from the Syrian border in southern Turkey, and works for a
nongovernmental agency, providing humanitarian assistance to vulnerable
Syrians.

The pharmacist said he remains opposed to the Syrian government and its
enforcers. ``We are fighting them, not with weapons but with ideas,
concerns and also humanitarian work.''

\begin{center}\rule{0.5\linewidth}{\linethickness}\end{center}

\hypertarget{the-pediatrician-war-destroyed-her-dream}{%
\subsection{The Pediatrician: War Destroyed Her
Dream}\label{the-pediatrician-war-destroyed-her-dream}}

``They focus on hospitals because if they destroy the hospitals, people
would give up.''

Image

Dr. Amani Ballour in her apartment in Gaziantep, Turkey in Nov. 2019.
Dr. Ballour managed an underground hospital in the besieged suburb of
Ghouta, Syria.Credit...The New York Times

By Carlotta Gall

For some, the war destroyed their dreams.

Amani Ballour's ambition was to be a pediatrician. She lived in Ghouta,
a large suburb east of Damascus, and was in her fifth year of a medical
degree in Damascus when demonstrations began in 2011. She recalls a
building sense of terror.

Police began checking student's IDs at the university entrance and she
watched in fear as fellow medical students were beaten and detained, and
people were hauled off buses.

``That started very early,'' she said. ``Everyone in Syria saw that.''

A slim pale-faced figure in a head scarf and long coat, Ms. Ballour, 32,
recounted her ordeal with the calm efficiency of a medical professional
as she sat in the sparse one-room apartment she shares with her husband,
a civil engineer, Hamza el Hiraki, 37.

When demonstrations broke out in her own suburb, the risk of detention
grew. ``They started to do the same thing,'' she said, ``I felt very
afraid.''

Then one day, Nov. 25, 2011, her brother and brother-in-law, both
mechanics who were traveling by bus on their way to fix a water pump,
were detained.

``My brother did not participate in the demonstrations,'' she said,
``but they took their IDs and because they were from Ghouta they were
arrested.''

``They disappeared from that time, nine years ago,'' she said. ``Till
now we don't know.''

Ms. Ballour was still traveling to the university by bus, and was
already helping to treat wounded protesters in a small clinic in Ghouta.

``It was dangerous for doctors,'' she said. ``If you helped injured
people they would arrest you, so I had to decide if I wanted to stay in
Ghouta, or stay in Damascus and study. I decided to stay in Ghouta.''

People who knew she was studying pediatrics began bringing their
children to her. She handled respiratory and intestinal infections and
referred serious cases to specialists in Damascus.

When the government imposed a siege on the suburb, conditions worsened.
At the beginning of 2013, a woman came to her with newborn twins. They
were in good health, but she had no milk. And with no milk powder
available, the babies died within weeks.

The medics relied on expired medicines they found in an old
pharmaceutical factory, but by 2014 even those were exhausted. ``We did
not think it would last that long,'' she recalled of the siege. ``By
2014 we had nothing. I saw a lot of children die with infections, and
some died of pneumonia.''

The numbers of wounded escalated sharply when the government began
aerial bombardment in 2012. When a hospital she worked in was destroyed
by fire, Ms. Ballour began assisting a surgeon, Dr. Salim Namour in a
hospital that was dug underground to protect against airstrikes. Their
work in The Cave is now the subject of
\href{https://www.nationalgeographic.com/films/the-cave/\#/}{a
documentary film}.

They trained volunteers to assist in the operating theater, and on the
wards and Ms. Ballour was able to focus on pediatric cases. Eventually
the staff voted for her to become the hospital's manager.

``He bombed it six or seven times but he could not injure anyone, he
could not reach the basement,'' she said, not needing to mention
President Bashar al-Assad by name. Only when Russia intervened in Syria
in 2015 and Russian jets joined the fight, were they able to pierce
underground, she said.

``A missile entered the basement,'' she said. ``They destroyed a part of
the hospital and they killed three of my colleagues.'' Ms. Ballour had
just walked out of their room into the corridor and narrowly escaped.

``They focus on hospitals because if they destroy the hospitals, people
would give up,'' she said.

``A doctor represents hope for the patient,'' her husband, Mr. el-Hiraki
chipped in.

Nothing prepared her for the devastating sarin gas attack of 2013.
``This was the most difficult thing I saw. I had never seen something
like that before, hundreds of dead bodies.''

When she reached the hospital that night the whole square in front of
the hospital was covered in bodies. ``There was no blood,'' she said. We
did not know what it was but people were shouting, `Chemical,
chemical!'''

``I saw a lot of people, most of them children and they were
suffocating,'' she said. ``Some of them were dead and some were dying.''

The patients were foaming at the mouth, had pinpoint pupils and were in
seizure. As medics tried to suck out the foam, more foam kept building.
They gave every patient an injection of atropine but it was not enough
and they had no oxygen.

``People asked me to help their children but they were dead,'' she said.
With seconds to save those still alive, she brushed off a woman whose
three children were her patients. ``I could not even look at them. They
were dead and I had to help others. There was no time and I did not
sympathize with her. And when I remember that I feel bad.''

``That night 1,400 died, most of them were children,'' she said**.**

The incident made Ms. Ballour and her surgeon colleague, Dr. Salim,
targets of the Assad government because they were important witnesses to
one of the war's worst atrocities.

They were eventually evacuated in to Idlib, the last opposition-held
province in northwestern Syria. But there, they received a warning that
they were on a government hit list because of their knowledge of the
sarin attack and were forced to move to Turkey.

In Ghouta medical colleagues who chose to stay behind were arrested,
including a former military doctor, Dr. Motaz. He died in prison.

For Ms. Ballour, the last days of the siege, when she saw so many
children killed and maimed, many of them her own patients, finally broke
her. She is working on a new project for Syrian women but gave up her
dreams of being a pediatrician.

``I cannot describe it, there are no words, but I could not work,'' she
said. ``I will not be a pediatrician any more. I could not work with the
children. Every child reminds me of another child.''

\begin{center}\rule{0.5\linewidth}{\linethickness}\end{center}

\hypertarget{the-surgeon-electrocuted-in-ankle-deep-water}{%
\subsection{The Surgeon: Electrocuted in Ankle-Deep
Water}\label{the-surgeon-electrocuted-in-ankle-deep-water}}

``Maybe I could have done more. This feeling of guilt never left us.''

Image

Dr. Ahmed in a park in Gaziantep, Turkey in Nov. 2019. He was detained
by the Syrian military and subjected to beatings, electric shocks and
mock executions.Credit...The New York Times

By Carlotta Gall

GAZIANTEP, Turkey --- Dr. Ahmed was training to be an orthopedic surgeon
at a government training hospital on the outskirts of Damascus when he
became involved in coordinating first-aid points for injured protesters
in 2011. With a group of 10 friends in the suburb of Dummar, Dr. Ahmed
helped to move wounded protesters to a couple's private house where he
would bring his instruments and medication, and provide first aid.

They kept the medical work secret, but at the same time were actively
supporting the demonstrations on social media.

``We were expressing our opinions in public on Facebook. I was using my
real name,'' he said. ``That was to encourage people to express their
opinion. So I never used a fake name which was crazy in that time.''

In August 2011, intelligence officials came to the hospital where he
worked and detained him. Unknown to Dr. Ahmed, the whole group was taken
into custody at the same time.

He endured a month of interrogation and torture of beatings, electric
shocks and mock executions. He was beaten with rubber, wooden and steel
cables, and electrocuted in ankle-deep in water.

``It was like someone threw me into a wall. I lost consciousness and
then I woke up on the watery floor.''

Three times his torturers told him to prepare for his death by hanging,
marching him out in the morning, and then after hours of waiting, giving
him a reprieve.

Although his interrogators did not know about his secret medical work,
they always took exception to his status as a doctor and his education.

``They said you studied in government schools, and it was for free, and
the health service is free. So now you are receiving training from the
government and receiving a salary, and now you want to bring down this
government. You are a cheat,'' he said.

``They always were torturing me double because I was a doctor.''

Eventually he confessed, was charged with multiple crimes including
trying to overthrow the government, and released. Despite a grueling
four months detention, he immediately returned to his activism.

``A lot had changed,'' he said. ``The Free Syrian Army had formed, the
international community was with us. I felt, `O.K., we have hope.' And
the regime increased its violence so I felt it was our responsibility
and we should not stop.''

He created a network of safe houses to treat the wounded, both civilians
and those who took up arms and joined the Free Syrian Army. ``We helped
all of them. At that time there was no Al Qaeda or ISIS, so we felt the
F.S.A. were part of us.''

He set up a safe house in a luxury villa just yards from one of the
Syrian government's main military bases. They devised a network to ferry
serious casualties to Lebanon. He returned to his post in the government
hospital, working by day as a government orthopedic surgeon and by night
for the opposition. ``Like Dr. Jekyll and Mr. Hyde,'' he said, laughing.
``Most of us had these two lives.''

At one first-aid point he amputated the arm of a Free Syrian Army
fighter without equipment. He used a simple razor blade and cut the
bones with garden shears. ``It worked,'' he said, ``but a couple of
hours after we finished, they said that the Army was very close to the
center and we have to evacuate.'' The doctors could drive out because
they had passes, but they had to leave the patient. ``He told us, go,
and we left and we don't know what happened to him. It was one of the
most difficult moments of my life.''

In 2013, he received a warning that he was about to be arrested and fled
Damascus for the rebel-held area of Idlib. It was just in time, as
government officials came looking for him at the hospital the next day.

He joined a small rural hospital, and in 2014 encountered one of the
most dramatic surgeries of his life. A car bomb exploded in the market
and caused dozens of casualties. He treated a 10-year-old boy who had an
open leg fracture, but then discovered his femoral artery was ruptured.
As the blood spurted out, he told his assistant to put his hand on the
wound and called a surgeon friend in Germany.

``He said `O.K. I will send you a YouTube link, watch it and then go to
the operating theater and call me through Skype and I will tell you what
to do,''' he recalled. Dr. Ahmed watched the video, and then his friend
talked him through the operation, taking a piece of vein from the boy's
other leg, mending the rupture and watching the color return to the
boy's foot. ``Till now, I have never felt happiness like this in my
life,'' he said.

He now works in Gaziantep in southern Turkey, meeting for an interview
in a cafe because his wife wants no more activism in their lives.

He no longer practices medicine and describes feeling survivors' guilt.
``Maybe I could have done more. This feeling of guilt never left us,''
he said. His new mission is to help train and support medical personnel
in northwestern Syria, where there is a lack of doctors.

``I came here to bridge the gap as much as I can, and I think I did good
work in that.''

\emph{Produced by Rick Gladstone and Malachy Browne.}

Advertisement

\protect\hyperlink{after-bottom}{Continue reading the main story}

\hypertarget{site-index}{%
\subsection{Site Index}\label{site-index}}

\hypertarget{site-information-navigation}{%
\subsection{Site Information
Navigation}\label{site-information-navigation}}

\begin{itemize}
\tightlist
\item
  \href{https://help.nytimes.com/hc/en-us/articles/115014792127-Copyright-notice}{©~2020~The
  New York Times Company}
\end{itemize}

\begin{itemize}
\tightlist
\item
  \href{https://www.nytco.com/}{NYTCo}
\item
  \href{https://help.nytimes.com/hc/en-us/articles/115015385887-Contact-Us}{Contact
  Us}
\item
  \href{https://www.nytco.com/careers/}{Work with us}
\item
  \href{https://nytmediakit.com/}{Advertise}
\item
  \href{http://www.tbrandstudio.com/}{T Brand Studio}
\item
  \href{https://www.nytimes.com/privacy/cookie-policy\#how-do-i-manage-trackers}{Your
  Ad Choices}
\item
  \href{https://www.nytimes.com/privacy}{Privacy}
\item
  \href{https://help.nytimes.com/hc/en-us/articles/115014893428-Terms-of-service}{Terms
  of Service}
\item
  \href{https://help.nytimes.com/hc/en-us/articles/115014893968-Terms-of-sale}{Terms
  of Sale}
\item
  \href{https://spiderbites.nytimes.com}{Site Map}
\item
  \href{https://help.nytimes.com/hc/en-us}{Help}
\item
  \href{https://www.nytimes.com/subscription?campaignId=37WXW}{Subscriptions}
\end{itemize}
