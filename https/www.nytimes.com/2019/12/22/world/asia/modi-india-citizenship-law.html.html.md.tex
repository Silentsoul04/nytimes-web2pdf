Sections

SEARCH

\protect\hyperlink{site-content}{Skip to
content}\protect\hyperlink{site-index}{Skip to site index}

\href{https://www.nytimes.com/section/world/asia}{Asia Pacific}

\href{https://myaccount.nytimes.com/auth/login?response_type=cookie\&client_id=vi}{}

\href{https://www.nytimes.com/section/todayspaper}{Today's Paper}

\href{/section/world/asia}{Asia Pacific}\textbar{}Modi Defends Indian
Citizenship Law Amid Violent Protests

\url{https://nyti.ms/2Zg0YMS}

\begin{itemize}
\item
\item
\item
\item
\item
\end{itemize}

Advertisement

\protect\hyperlink{after-top}{Continue reading the main story}

Supported by

\protect\hyperlink{after-sponsor}{Continue reading the main story}

\hypertarget{modi-defends-indian-citizenship-law-amid-violent-protests}{%
\section{Modi Defends Indian Citizenship Law Amid Violent
Protests}\label{modi-defends-indian-citizenship-law-amid-violent-protests}}

Prime Minister Narendra Modi stridently backed a law establishing a
religious test for migrants that has led to deadly protests.

\includegraphics{https://static01.nyt.com/images/2019/12/22/world/22india-modi/merlin_166287444_3f05ee1d-2311-4a89-8d21-a822e8181e4e-articleLarge.jpg?quality=75\&auto=webp\&disable=upscale}

\href{https://www.nytimes.com/by/kai-schultz}{\includegraphics{https://static01.nyt.com/images/2019/11/22/reader-center/author-kai-schultz/author-kai-schultz-thumbLarge.png}}

By \href{https://www.nytimes.com/by/kai-schultz}{Kai Schultz}

\begin{itemize}
\item
  Published Dec. 22, 2019Updated Feb. 24, 2020
\item
  \begin{itemize}
  \item
  \item
  \item
  \item
  \item
  \end{itemize}
\end{itemize}

NEW DELHI --- Prime Minister Narendra Modi of India delivered on Sunday
a strident defense of a contentious
\href{https://www.nytimes.com/2019/12/11/world/asia/india-muslims-citizenship-narendra-modi.html}{citizenship
law} that has fueled deadly protests, accusing opposition politicians of
``spreading lies'' and demonstrators of trying to destroy the country
through vandalism and bloodshed.

During an often combative speech in New Delhi, Mr. Modi signaled that he
would not scrap the law, which
\href{https://www.nytimes.com/2019/12/09/world/asia/india-muslims-citizenship-narendra-modi.html}{favors
every major South Asian faith other than Islam}.

Critics argue that the law is glaring evidence that the government plans
to turn India into a Hindu-centric state and marginalize the country's
200 million minority Muslims. Mr. Modi, in his speech, dismissed the
notion that the law was discriminatory.

``Respect the Parliament!'' Mr. Modi said to thousands of supporters.
``Respect the Constitution! Respect the people elected by the people! I
challenge the ones who are spreading lies. If there is a smell of
discrimination in anything I have done, then put me in front of the
country.''

Over the past two weeks,
\href{https://www.nytimes.com/2019/12/16/world/asia/india-citizenship-protests.html}{hundreds
of thousands of Indians have taken to the streets} in opposition of the
Citizenship Amendment Act, which the Indian Parliament approved this
month. The protests have drawn people of all faiths, concerned that the
law undermines India's foundation as a secular nation. Around two dozen
people have been killed in the increasingly violent protests, and
hundreds have been arrested.

The demonstrations are the most significant challenge to Mr. Modi's
leadership since his Bharatiya Janata Party rose to power in 2014. The
authorities have been criticized for detaining demonstrators ---
including children --- without legal recourse,
\href{https://www.nytimes.com/2019/12/17/world/asia/india-internet-modi-protests.html}{shutting
down internet and phone services}, and firing live ammunition into
crowds.

\emph{{[}Read:}
\href{https://www.nytimes.com/2020/02/24/world/asia/trump-india-modi.html?}{\emph{`America
loves India,' Trump declares at rally with Modi.}}\emph{{]}}

Under the government of Mr. Modi, Muslims and others have been
\href{https://www.nytimes.com/2019/12/16/world/asia/india-citizenship-protests.html}{fearful
about the rise of Hindu nationalism}. Muslims have been lynched by Hindu
mobs. The government stripped the country's only Muslim-majority state,
\href{https://www.nytimes.com/2019/08/05/world/asia/india-pakistan-kashmir-jammu.html}{Jammu
and Kashmir}, of its autonomy. It instituted a citizenship test in
Assam, which it plans to roll out nationally.

To critics, Mr. Modi is pushing an authoritarian agenda that threatens
to erode the country's secular foundation, shrink space for religious
minorities and move the country closer to a Hindu nation.

\includegraphics{https://static01.nyt.com/images/2019/12/22/world/22india-modi2/merlin_166290789_1e6b8a3d-5e8f-4b2c-ad68-bec859aa7a98-articleLarge.jpg?quality=75\&auto=webp\&disable=upscale}

Long a dream of Hindu nationalists, the Citizenship Amendment Act
establishes a religious test for migrants who want to become citizens.
Muslim Indians worry that the government could use it to render many of
them stateless.

In his speech, Mr. Modi argued that the law was meant only to extend
citizenship to religious minorities fleeing persecution in three
Muslim-majority countries: Afghanistan, Bangladesh and Pakistan. He said
that it would not be used against Indian citizens, pointing to
development projects as a sign that the government extends public
services without regard for religion.

``If we haven't asked your religion for previous policies, why would we
ask your religion for this policy?'' Mr. Modi said. ``We never asked
their religion. We only saw the poverty of the poor and gave them a
home.''

``I want to clarify once again that the C.A.A. is not going to take away
anybody's citizenship,'' he added, referring to the law. ``It is about
giving citizenship to those facing discrimination.''

But many protesters worry that the law would be used in tandem with a
citizenship check to discriminate against Muslims.

The check began this year in the northeastern state of Assam, which
borders Bangladesh and has a high migrant population. The state's 33
million residents had to prove, with documentary evidence, that their
families were Indian citizens. Approximately two million people --- many
of them Muslims ---
\href{https://www.nytimes.com/2019/08/31/world/asia/india-muslim-citizen-list.html}{were
left off the state's citizenship rolls} after that exercise.

The new citizenship law would most likely protect Hindus and people of
other religions who failed such a test. Muslims, though, would be
excluded.

The Indian home minister, Amit Shah, has vowed in speeches to expand the
checks used in Assam to other states and then use the citizenship law to
purge India of ``infiltrators'' and ``termites.'' The authorities have
\href{https://www.ndtv.com/india-news/assam-detention-centre-inside-indias-1st-detention-centre-for-illegal-immigrants-after-nrc-school-ho-2099626}{already
started building detention centers} for those who cannot prove their
roots.

Indian Muslims, who were relatively quiet as Hindu nationalism reached
new heights under Mr. Modi's government, have finally erupted in anger.
They have been joined by Indians concerned about the threat to the
secular state, with protests spreading across the country.

Over the weekend, the demonstrations took another deadly turn. Residents
across Uttar Pradesh, India's most populous state and a stronghold of
Mr. Modi's party, said that the police broke into the homes of Muslims,
took away hundreds of young men, vandalized property and beat people
with sticks in the streets. Curfews and internet blackouts were
widespread in the state, and the government instructed universities
\href{https://theprint.in/india/education/modi-govt-asks-iits-iims-varsities-to-track-student-social-media-posts-amid-caa-protests/338552/?fbclid=IwAR1kVcA0_t9XN_mvyBlEc8_BPdsfZQ8-cmNFvCQwcFNN-l35LQZ2uha1Q_o}{to
track students' social media posts}.

Image

Police officers in Lucknow, India, on Sunday. The authorities have been
criticized for their response to the protests.Credit...Rajesh Kumar
Singh/Associated Press

``The police came in the night, destroyed cars parked on the roadside,
and broke gates to our homes,'' said Mohammad Rashid, who lives in the
city of Kanpur, where at least two people have died. ``We are being
treated like animals.''

The police maintained until recently that no bullets have been fired at
demonstrators, but videos posted on social media have challenged that
claim.

\href{https://www.ndtv.com/india-news/citizenship-amendment-act-protests-video-suggests-up-cop-opened-fire-in-kanpur-contrary-to-no-police-2152566}{In
one video}, a police officer wearing a safety jacket and a helmet fires
what looks like a revolver in a street where protesters had gathered.
Moments later, a person shouts in the background, ``Remove the cameras!
Let them shoot!''

On Sunday, Mr. Modi sought to quell consternation over the protests,
characterizing them as fueled by his political enemies and ``urban
Naxalites'' intent on destroying the government at any cost.

He beseeched the crowd to resist their ``evil game.'' He accused
protesters of attacking school buses, targeting the police (whom he
called ``martyrs'') and spreading rumors about the fate of Indian
Muslims. He made no mention of protesters who had been killed or
injured. He denied that detention centers existed.

Image

Supporters of Mr. Modi at his rally in New Delhi on
Sunday.Credit...Prakash Singh/Agence France-Presse --- Getty Images

The law ``has nothing to do with Muslims who are made out of the soil of
India, whose ancestors are the sons of Mother India,'' he said, adding:
``No Indian Muslims are being sent to detention camps.''

Evoking Mahatma Gandhi, the Indian independence fighter who championed
nonviolent protest, Mr. Modi suggested that standing against the law
``with stones'' was a betrayal to India and part of a ``conspiracy to
malign the country around the world.''

``They have an illicit intention of destroying the country,'' he said of
the demonstrators. ``When you see bricks and sticks in the hands of the
protesters, I feel pain, as does the rest of India.''

Sameer Yasir contributed reporting.

Advertisement

\protect\hyperlink{after-bottom}{Continue reading the main story}

\hypertarget{site-index}{%
\subsection{Site Index}\label{site-index}}

\hypertarget{site-information-navigation}{%
\subsection{Site Information
Navigation}\label{site-information-navigation}}

\begin{itemize}
\tightlist
\item
  \href{https://help.nytimes.com/hc/en-us/articles/115014792127-Copyright-notice}{©~2020~The
  New York Times Company}
\end{itemize}

\begin{itemize}
\tightlist
\item
  \href{https://www.nytco.com/}{NYTCo}
\item
  \href{https://help.nytimes.com/hc/en-us/articles/115015385887-Contact-Us}{Contact
  Us}
\item
  \href{https://www.nytco.com/careers/}{Work with us}
\item
  \href{https://nytmediakit.com/}{Advertise}
\item
  \href{http://www.tbrandstudio.com/}{T Brand Studio}
\item
  \href{https://www.nytimes.com/privacy/cookie-policy\#how-do-i-manage-trackers}{Your
  Ad Choices}
\item
  \href{https://www.nytimes.com/privacy}{Privacy}
\item
  \href{https://help.nytimes.com/hc/en-us/articles/115014893428-Terms-of-service}{Terms
  of Service}
\item
  \href{https://help.nytimes.com/hc/en-us/articles/115014893968-Terms-of-sale}{Terms
  of Sale}
\item
  \href{https://spiderbites.nytimes.com}{Site Map}
\item
  \href{https://help.nytimes.com/hc/en-us}{Help}
\item
  \href{https://www.nytimes.com/subscription?campaignId=37WXW}{Subscriptions}
\end{itemize}
