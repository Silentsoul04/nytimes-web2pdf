Sections

SEARCH

\protect\hyperlink{site-content}{Skip to
content}\protect\hyperlink{site-index}{Skip to site index}

\href{https://www.nytimes.com/section/world/europe}{Europe}

\href{https://myaccount.nytimes.com/auth/login?response_type=cookie\&client_id=vi}{}

\href{https://www.nytimes.com/section/todayspaper}{Today's Paper}

\href{/section/world/europe}{Europe}\textbar{}How a Poisoning in
Bulgaria Exposed Russian Assassins in Europe

\url{https://nyti.ms/2Zfr2rs}

\begin{itemize}
\item
\item
\item
\item
\item
\end{itemize}

Advertisement

\protect\hyperlink{after-top}{Continue reading the main story}

Supported by

\protect\hyperlink{after-sponsor}{Continue reading the main story}

\hypertarget{how-a-poisoning-in-bulgaria-exposed-russian-assassins-in-europe}{%
\section{How a Poisoning in Bulgaria Exposed Russian Assassins in
Europe}\label{how-a-poisoning-in-bulgaria-exposed-russian-assassins-in-europe}}

For years, members of a secret team, Unit 29155, operated without
Western security officials having any idea about their activities. But
an attack on an arms dealer in Sofia helped blow their cover.

\includegraphics{https://static01.nyt.com/images/2019/12/22/world/22bulgaria1/merlin_150422367_7e8be145-064f-479a-b88a-e1f4d6ad3880-articleLarge.jpg?quality=75\&auto=webp\&disable=upscale}

\href{https://www.nytimes.com/by/michael-schwirtz}{\includegraphics{https://static01.nyt.com/images/2018/02/20/multimedia/author-michael-schwirtz/author-michael-schwirtz-thumbLarge-v2.jpg}}

By \href{https://www.nytimes.com/by/michael-schwirtz}{Michael Schwirtz}

\begin{itemize}
\item
  Dec. 22, 2019
\item
  \begin{itemize}
  \item
  \item
  \item
  \item
  \item
  \end{itemize}
\end{itemize}

SOFIA, Bulgaria --- The Russian assassin used an alias, Sergei Fedotov,
and slipped into Bulgaria unnoticed, checking into a hotel in Sofia near
the office of a local arms manufacturer who had been selling ammunition
to Ukraine.

He led a team of three men.

Within days, one man sneaked into a locked parking garage, smeared
poison on the handle of the arms manufacturer's car, then left,
undetected, except for blurry images captured by surveillance video.

Shortly after, the arms manufacturer, Emilian Gebrev, was meeting with
business partners at a rooftop restaurant when he began to hallucinate
and vomit.

The poisoning left Mr. Gebrev, now 65, hospitalized for a month. His son
was poisoned, and so was another top executive at his company. When Mr.
Gebrev was discharged, the assassins poisoned him and his son again, at
their summer home on the Black Sea. They all survived, though Mr.
Gebrev's business has yet to recover fully.

The assassination attempts in 2015 were remarkable not only for their
brazenness and persistence, but also because security and intelligence
officials in the West initially did not notice. Bulgarian prosecutors
looked at the case, failed to unearth any evidence and closed it.

Now Western security and intelligence officials say the Bulgaria
poisonings were a critical clue that helped expose a campaign by the
Kremlin and its sprawling web of intelligence operatives to eliminate
Russia's enemies abroad and destabilize the West.

\includegraphics{https://static01.nyt.com/images/2019/12/22/world/22bulgaria2/merlin_165799545_171d5056-5129-4ca7-9e84-25e083e8c6ad-articleLarge.jpg?quality=75\&auto=webp\&disable=upscale}

``With Bulgaria, there was an `aha' moment,'' said one European security
official, who spoke on condition of anonymity to discuss classified
intelligence matters. ``We looked at it and thought, damn, everything
aligned.''

Entering his third decade in power, President Vladimir V. Putin of
Russia is pushing hard to re-establish Russia as a world power. Russia
cannot compete economically or militarily with the United States and
China, so Mr. Putin is waging an asymmetric shadow war. Russian
mercenaries are fighting in
\href{https://www.nytimes.com/2018/05/24/world/middleeast/american-commandos-russian-mercenaries-syria.html}{Syria},
\href{https://www.nytimes.com/2019/11/05/world/middleeast/russia-libya-mercenaries.html}{Libya}
and
\href{https://www.nytimes.com/2014/05/28/world/europe/ukraine.html?searchResultPosition=2}{Ukraine}.
Russian hackers are sowing discord through disinformation and working to
undermine elections.

Russian assassins have also been busy.

In October,
\href{https://www.nytimes.com/2019/10/08/world/europe/unit-29155-russia-gru.html}{The
New York Times revealed} that a specialized group of Russian
intelligence operatives --- Unit 29155 --- had for years been assigned
to carry out killings and political disruption campaigns in Europe.
Intelligence and security officials say the unit is responsible for the
\href{https://www.nytimes.com/2018/03/05/world/europe/russian-spy-falls-ill-in-britain-again.html}{assassination
attempt last year against Sergei V. Skripal}, a Russian former spy in
Britain; a failed operation in 2016 to
\href{https://www.nytimes.com/2016/11/26/world/europe/finger-pointed-at-russians-in-alleged-coup-plot-in-montenegro.html}{provoke
a military coup in Montenegro}; and a campaign to destabilize Moldova.

Western intelligence agencies now know the name of the unit's commanding
officer, Maj. Gen. Andrei V. Averyanov, and the location of its
headquarters in Moscow. Based on interviews with officials in Europe and
the United States, it is also now clear that the assassination attempts
against Mr. Gebrev served as a kind of Rosetta Stone that helped Western
intelligence agencies to discover Unit 29155 --- and to decipher the
kind of threat it presented.

Image

Maj. Gen. Andrei V. AveryanovCredit...Agence France-Presse --- Getty
Images

Since the original Times story, more information has come to light,
including the true identities of some of the unit's members and other
possible activities in Spain and France. This month,
\href{https://www.nytimes.com/2019/12/04/world/europe/germany-assassination-russia.html}{Germany
expelled two Russian diplomats} as punishment for
\href{https://www.nytimes.com/2019/09/26/world/europe/berlin-murder-russia.html}{the
daylight assassination in Berlin} of a former Chechen rebel commander,
though it is unclear whether operatives from 29155 were involved.

Security and intelligence officials are still working to understand how
and why the unit is assigned certain targets. Even now, investigators
have not determined the precise motive in the Gebrev case. Most likely,
intelligence officials say, Mr. Gebrev was a target because of the way
his business rankled the Kremlin: his arms sales to Ukraine, his
company's intrusion into markets long dominated by Russia, and his
efforts to purchase a weapons factory coveted by a Russian oligarch.

Mr. Gebrev says he also believes that local business rivals or
politicians might somehow be involved.

``I have been thrown to the wolves,'' Mr. Gebrev said in an interview.
``But why and how, I'm still asking myself.''

\hypertarget{a-visit-to-the-afterlife}{%
\subsection{A Visit to the Afterlife}\label{a-visit-to-the-afterlife}}

The poison took effect slowly.

Mr. Gebrev first realized something was wrong on the evening of April
27, 2015, when his right eye suddenly turned ``as red as the red on the
Russian flag.'' It felt, he said, as if someone had dumped a bucket of
sand into his pupil.

The next evening, Mr. Gebrev went to his favorite restaurant on the 19th
floor of the Hotel Marinela, a luxury hangout in Sofia, the Bulgarian
capital, where the clientele can pose for selfies with the peacocks
wandering freely around the bar. At dinner, Mr. Gebrev began to vomit
violently and was rushed to a military hospital. There, he began to see
explosions of vivid colors. Then, his field of vision suddenly turned to
black and white.

As his hallucinations intensified, he imagined angry, fantastical
creatures that threatened to drag him away.

``I visited the afterlife three times, by my estimate,'' he said in one
of a series of interviews conducted over the past half year. ``The
doctors said they almost lost me.''

A day later, the company's production manager, Valentin Tahchiev, was
hospitalized, too. Days after that, Mr. Gebrev's son, Hristo, who was
being groomed to lead Mr. Gebrev's company, Emco, was also rushed to
intensive care.

``When they get rid of me and my son, the company will be destroyed,''
Mr. Gebrev said later. ``Who would sign contracts? Who has the rights?''

For the next month, as Mr. Gebrev recuperated in the hospital, the
Bulgarian authorities made little progress on the case. In a former
Soviet satellite country with a long history of contract killings, the
Bulgarian news media barely paid attention. The prosecutor general
suggested that Mr. Gebrev had been sickened by tainted arugula.
Eventually, though, officials concluded that all three men had been
poisoned.

Image

``I visited the afterlife three times by my estimate,'' Mr. Gebrev,
left, said. ``The doctors said they almost lost me.''Credit...Nikolay
Doychinov/Agence France-Presse --- Getty Images

In late May, Mr. Gebrev was released from the hospital and joined his
son at the family vacation home on the Black Sea. There, the two men
were poisoned again. This time, the symptoms were less dramatic and they
drove themselves back to Sofia and checked into the same hospital for
about two weeks.

Despite two poisonings, Bulgarian prosecutors failed to unearth any
leads or evidence. Bulgarian intelligence agencies never reported
detecting a Russian assassination team in the country, and possibly
never realized it had been there.

``Anytime it's linked to something with Russia, Bulgarian intelligence
is very impotent,'' said Rosen Plevneliev, who was Bulgaria's president
at the time of the poisonings. ``Bulgarian intelligence is not willing
to counter Russian intelligence and hybrid warfare.''

When the hospital failed to determine the substance used in the
poisoning, Mr. Gebrev enlisted a Finnish laboratory, Verifin, which
detected two chemicals in his urine, including diethyl phosphonate,
which is found in pesticides. The other chemical could not be
identified.

By the following summer, the Bulgarian authorities had dropped the case.
They apparently had no idea that Unit 29155 even existed. Neither did
intelligence and security officials in the rest of Europe.

Yet as Mr. Gebrev's case remained colder than cold, members of Unit
29155 were very busy, according to partial travel records reviewed by
The Times. From 2016 to 2018, operatives made at least two dozen trips
from Moscow to different European countries.

Their operation in Bulgaria most likely would never have been detected.

Then there was another poisoning.

\hypertarget{an-unexpected-breakthrough}{%
\subsection{An Unexpected
Breakthrough}\label{an-unexpected-breakthrough}}

In March 2018, a former Russian spy named Sergei V. Skripal was poisoned
by a lethal nerve agent in the English town of Salisbury. He began
ranting at a restaurant and fell into a coma before clawing his way back
to life. It was the first recorded use of a chemical weapon in Europe
since World War II, and it touched off a frantic investigation to
determine the extent of the threat.

British prosecutors
\href{https://www.nytimes.com/2018/09/05/world/europe/russia-uk-novichok-skripal.html}{attributed
the attack to assassins} working for Russia's military intelligence
agency, known widely as the G.R.U. Working with European allies, the
British authorities analyzed travel records of known Russian operatives.
One stood out, a man using a Russian passport with the name of Sergei
Fedotov.

Image

Military personnel investigating the poisoning of Sergei V. Skripal in
Salisbury, England, last year.Credit...Chris J Ratcliffe/Getty Images

For five years, he had traveled extensively in Europe, visiting Serbia,
Spain and Switzerland. He was in London a few days before Mr. Skripal
was poisoned, leaving shortly after that attack, and British authorities
have now identified him as the commander of the team that poisoned Mr.
Skripal.

It also turned out that he had been in Bulgaria in 2015, making three
visits: in February; in April, when Mr. Gebrev was first poisoned; and
again in late May, coinciding with the second poisoning.

Investigators from the Britain-based open-source news outlet Bellingcat
\href{https://www.bellingcat.com/news/uk-and-europe/2019/02/14/third-suspect-in-skripal-poisoning-identified-as-denis-sergeev-high-ranking-gru-officer/}{have
identified the man} using the Fedotov alias as Denis V. Sergeev, a
high-ranking G.R.U. officer and a veteran of Russia's wars in the North
Caucasus. The British authorities confirmed the accuracy of the report.

The revelation that he was connected to the poisonings in both England
and Bulgaria was critical in helping Western officials conclude that
these were not one-off Russian attacks but rather part of a coordinated
campaign run by Unit 29155.

In recent weeks, another operation possibly involving the man known as
Mr. Fedotov has emerged in Spain. The highest criminal court there is
investigating whether Mr. Fedotov and other Russian operatives might
have had some involvement in the protests that destabilized Catalonia in
October 2017. Travel records show that he arrived in Barcelona a few
days before the region held an independence referendum that month.

Image

The headquarters of Russia's military intelligence agency, known widely
as the G.R.U., in Moscow.Credit...Natalia Kolesnikova/Agence
France-Presse --- Getty Images

During his visits to Bulgaria two years earlier, Mr. Fedotov was joined
by other officers. For the April 2015 poisoning, it was two men using
the aliases Georgi Gorshkov and Sergei Pavlov, according to two European
security officials, who requested anonymity to discuss sensitive
intelligence matters. The man using the Pavlov identity also visited
London a year before the Skripal poisoning, possibly in preparation for
the attack.

Armed with new evidence provided by the British, the Bulgarian
prosecutor general, Sotir Tsatsarov, reopened the case in October 2018.
Almost immediately, investigators discovered fresh clues. Before the
initial poisoning, Mr. Fedotov and two other operatives from Unit 29155
had checked into the Hill Hotel, in the same complex where Mr. Gebrev
has his office. They insisted, prosecutors now say, on rooms with views
of the entrance to an underground parking garage where Emco executives
kept their cars.

In the garage, prosecutors discovered grainy surveillance video that
showed a well-dressed figure approaching Mr. Gebrev's gray Nissan, as
well as the cars owned by Mr. Gebrev's son and by the production
manager. The figure appears to smear something on the handles of all
three cars. Western intelligence officials have surmised that the
substance was a poison.

The surveillance video was described to The Times by two security
officials familiar with its contents, but who had not watched it
themselves. They requested anonymity to discuss a live investigation.
This month, the office of Mr. Tsatsarov, the prosecutor general,
confirmed the existence of the video --- but said that its poor quality
prevented investigators from identifying the well-dressed figure. Mr.
Tsatsarov, whose term ended on Wednesday, has sent the video for
analysis by the Federal Bureau of Investigation.

Travel information shared with The Times shows that all three assassins
with Unit 29155 left Bulgaria on April 28, as Mr. Gebrev lay in the
hospital imagining monsters trying to tear him apart.

\hypertarget{bad-arugula}{%
\subsection{Bad Arugula}\label{bad-arugula}}

There is little doubt that Mr. Gebrev's profession --- the manufacture
and sale of munitions and light weapons --- places him in a risky field,
especially in Bulgaria.

In recent years, the Kremlin has grown increasingly alarmed as smaller
countries have nibbled away at Russia's dominance in the arms industry.
At \href{http://kremlin.ru/events/president/news/60812}{a meeting in
June} with high-ranking security officials, Mr. Putin warned that
Russia's position in the industry was threatened.

``New factors, complicating our work with our partners in military and
technical cooperation --- including competitive fights and increasingly
aggressive use of unscrupulous methods of political blackmail, and
sanctions --- demand attention and an adequate response,'' Mr. Putin
said. ``We need to do everything we can to preserve Russia's leading
position in the world arms market.''

Bulgaria now sells more than 1.2 billion euros, about \$1.3 billion, in
weapons annually, a relatively modest figure for the sector, but a sum
that has not gone unnoticed by Moscow. Tihomir Bezlov, a security
analyst, says he believes that is what made Mr. Gebrev a target.

``This is really big trouble for Russia,'' said Mr. Bezlov, of the
Center for the Study of Democracy in Sofia. ``We don't produce planes
and tanks, but in this area of light weapons, this is serious
competition.''

Image

An annual meeting of Russia supporters in the central Bulgarian town of
Kazanlak in 2014.Credit...Nikolay Doychinov/Agence France-Presse ---
Getty Images

Mr. Gebrev's business grew out of the collapse of communism. When a
scramble ensued for control of weapons factories, the new Bulgarian
government blocked Russian buyers and doled out export licenses to men
like Mr. Gebrev.

He has since moved into areas long dominated by Russia, including the
Indian market, where he describes himself as ``a niche player.''

``While Russia is exporting ammo worth billions of euros, we are
exporting for millions or hundreds of millions,'' Mr. Gebrev said. ``But
never mind, we're winning tenders and they're dreaming and thinking that
the markets belong to them.''

Emco, Mr. Gebrev's company, also made sales to Ukraine, Russia's enemy.

At the outset of
\href{https://www.nytimes.com/2014/03/19/world/europe/ukraine.html}{Ukraine's
war with Russian-backed separatists} in 2014, Emco signed a contract
with the Ukrainian government to supply artillery ammunition, according
to Sergii Bondarchuk, a former head of one of Ukraine's state-controlled
arms companies.

After Russian protests, the Bulgarian government canceled the contract
in 2015, Mr. Bondarchuk said. In a statement to The Times on Wednesday,
Emco made no mention of Russian pressure but said it had unilaterally
halted sales to Ukraine in November 2014. Mr. Gebrev described his
contracts in Ukraine as ``peanuts.''

Image

Members of a Ukrainian border guard unit in Luhansk, Ukraine, in
2014.Credit...Sergey Ponomarev for The New York Times

Mr. Gebrev was also entangled with another project that might have
displeased Moscow. Shortly before he was poisoned, Mr. Gebrev tried to
buy Dunarit, a large arms production plant in Bulgaria coveted by a
Kremlin-backed oligarch. A secret memo written around the same time (and
since made public by Bulgarian prosecutors) detailed a Russian plan to
transfer Dunarit to the oligarch, Konstantin Malofeev.

The United States and the European Union have imposed sanctions on Mr.
Malofeev for funding Russian-backed separatists in eastern Ukraine.

Today, Mr. Gebrev has recovered physically, though his business is still
ailing. In August 2017, the Bulgarian Economic Ministry temporarily
revoked his export license. The ministry is headed by Emil Karanikolov,
who was nominated to his post by the far-right Ataka party, which has
long faced scrutiny over its close ties with Moscow.

Unlike many wealthy businessmen in Bulgaria, Mr. Gebrev has no bodyguard
and prefers to drive himself. But he remains jumpy. Last fall, a
surveillance camera at his home captured infrared images of a spectral
figure with a mask snooping around outside.

``I would be the happiest man on earth if the poisoning didn't take
place and I felt sick because I had eaten some bad arugula,'' he said
later. ``I don't see myself as so important that someone would try to
kill me.''

Boryana Dzhambazova contributed reporting and research.

Advertisement

\protect\hyperlink{after-bottom}{Continue reading the main story}

\hypertarget{site-index}{%
\subsection{Site Index}\label{site-index}}

\hypertarget{site-information-navigation}{%
\subsection{Site Information
Navigation}\label{site-information-navigation}}

\begin{itemize}
\tightlist
\item
  \href{https://help.nytimes.com/hc/en-us/articles/115014792127-Copyright-notice}{©~2020~The
  New York Times Company}
\end{itemize}

\begin{itemize}
\tightlist
\item
  \href{https://www.nytco.com/}{NYTCo}
\item
  \href{https://help.nytimes.com/hc/en-us/articles/115015385887-Contact-Us}{Contact
  Us}
\item
  \href{https://www.nytco.com/careers/}{Work with us}
\item
  \href{https://nytmediakit.com/}{Advertise}
\item
  \href{http://www.tbrandstudio.com/}{T Brand Studio}
\item
  \href{https://www.nytimes.com/privacy/cookie-policy\#how-do-i-manage-trackers}{Your
  Ad Choices}
\item
  \href{https://www.nytimes.com/privacy}{Privacy}
\item
  \href{https://help.nytimes.com/hc/en-us/articles/115014893428-Terms-of-service}{Terms
  of Service}
\item
  \href{https://help.nytimes.com/hc/en-us/articles/115014893968-Terms-of-sale}{Terms
  of Sale}
\item
  \href{https://spiderbites.nytimes.com}{Site Map}
\item
  \href{https://help.nytimes.com/hc/en-us}{Help}
\item
  \href{https://www.nytimes.com/subscription?campaignId=37WXW}{Subscriptions}
\end{itemize}
