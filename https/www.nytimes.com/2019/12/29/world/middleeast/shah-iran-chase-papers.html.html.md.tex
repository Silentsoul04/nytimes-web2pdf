Sections

SEARCH

\protect\hyperlink{site-content}{Skip to
content}\protect\hyperlink{site-index}{Skip to site index}

\href{https://www.nytimes.com/section/world/middleeast}{Middle East}

\href{https://myaccount.nytimes.com/auth/login?response_type=cookie\&client_id=vi}{}

\href{https://www.nytimes.com/section/todayspaper}{Today's Paper}

\href{/section/world/middleeast}{Middle East}\textbar{}How a Chase Bank
Chairman Helped the Deposed Shah of Iran Enter the U.S.

\href{https://nyti.ms/2rGRgXC}{https://nyti.ms/2rGRgXC}

\begin{itemize}
\item
\item
\item
\item
\item
\end{itemize}

Advertisement

\protect\hyperlink{after-top}{Continue reading the main story}

Supported by

\protect\hyperlink{after-sponsor}{Continue reading the main story}

\hypertarget{how-a-chase-bank-chairman-helped-the-deposed-shah-of-iran-enter-the-us}{%
\section{How a Chase Bank Chairman Helped the Deposed Shah of Iran Enter
the
U.S.}\label{how-a-chase-bank-chairman-helped-the-deposed-shah-of-iran-enter-the-us}}

The fateful decision in 1979 to admit Mohammed Reza Pahlavi prompted the
seizure of the American Embassy in Tehran and helped doom the Carter
presidency.

\includegraphics{https://static01.nyt.com/images/2019/12/29/business/29IRAN-SHAH-01/00IRAN-SHAH-01-articleLarge.jpg?quality=75\&auto=webp\&disable=upscale}

\href{https://www.nytimes.com/by/david-d-kirkpatrick}{\includegraphics{https://static01.nyt.com/images/2018/10/15/multimedia/author-david-d-kirkpatrick/author-david-d-kirkpatrick-thumbLarge-v2.png}}

By \href{https://www.nytimes.com/by/david-d-kirkpatrick}{David D.
Kirkpatrick}

\begin{itemize}
\item
  Dec. 29, 2019
\item
  \begin{itemize}
  \item
  \item
  \item
  \item
  \item
  \end{itemize}
\end{itemize}

One late fall evening 40 years ago, a worn-out white Gulfstream II jet
descended over Fort Lauderdale, Fla., carrying a regal but sickly
passenger almost no one was expecting.

Crowded aboard were a Republican political operative, a retinue of
Iranian military officers, four smelly and hyperactive dogs and Mohammed
Reza Pahlavi, the newly deposed shah of Iran.

Yet as the jet touched down, the only one waiting to receive the deposed
monarch was a senior executive of Chase Manhattan Bank, which had not
only lobbied the White House to admit the former shah but had arranged
visas for his entourage, searched out private schools and mansions for
his family and helped arrange the Gulfstream to deliver him.

``The Eagle has landed,'' Joseph V. Reed Jr., the chief of staff to the
bank's chairman, David Rockefeller, declared in a celebratory meeting at
the bank the next morning.

Less than two weeks later, on Nov. 4, 1979, vowing revenge for the
admission of the shah to the United States, revolutionary Iranian
students seized the American Embassy in Tehran and then held more than
50 Americans --- and Washington --- hostage for 444 days.

The shah, Washington's closest ally in the Persian Gulf, had fled Tehran
in January 1979 in the face of a burgeoning uprising against his 38
years of iron-fisted rule. Liberals, leftists and religious
conservatives were rallying against him. Strikes and demonstrations had
shut down Tehran, and his security forces were losing control.

The shah sought refuge in America. But President Jimmy Carter, hoping to
forge ties to the new government rising out of the chaos and concerned
about the security of the United States Embassy in Tehran, refused him
entry for the first 10 months of his exile. Even then, the White House
only begrudgingly let him in for medical treatment.

Now, a newly disclosed secret history from the offices of Mr.
Rockefeller shows in vivid detail how Chase Manhattan Bank and its
well-connected chairman worked behind the scenes to persuade the Carter
administration to admit the shah, one of the bank's most profitable
clients.

\includegraphics{https://static01.nyt.com/images/2019/12/27/world/00iran-shah-02/00iran-shah-02-articleLarge.jpg?quality=75\&auto=webp\&disable=upscale}

For Mr. Carter, for the United States and for the Middle East it was an
incendiary decision.

The ensuing hostage crisis enabled Ayatollah Ruhollah Khomeini to
consolidate his theocratic rule, started a four-decade conflict between
Washington and Tehran that is still roiling the region and helped Ronald
Reagan take the White House. To American policymakers, Iran became
\href{https://www.commentarymagazine.com/articles/dictatorships-double-standards/}{a
parable about the political perils in the fall of a friendly strongman}.

Although Mr. Carter
\href{https://www.nytimes.com/1981/05/17/magazine/why-carter-admitted-the-shah.html}{complained
publicly at the time about the pressure campaign}, the full,
behind-the-scenes story --- laid out in the recently disclosed documents
--- has never been told.

Mr. Rockefeller's team called the campaign Project Eagle, after the code
name used for the shah. Exploiting clubby networks of power stretching
deep into the White House, Mr. Rockefeller mobilized a phalanx of elder
statesmen.

Image

A handwritten letter from the shah to David Rockefeller.Credit...Verner
Reed (1937-2016) papers, via Yale

They included Henry A. Kissinger, the former secretary of state and the
chairman of a Chase advisory board;
\href{https://www.nytimes.com/1989/03/12/obituaries/john-j-mccloy-lawyer-and-diplomat-is-dead-at-93.html}{John
J. McCloy}, the former commissioner of occupied Germany after World War
II and an adviser to eight presidents as well as a future Chase
chairman; a Chase executive and former C.I.A. agent,
\href{https://www.nytimes.com/1990/06/01/obituaries/a-b-roosevelt-a-cia-veteran-and-banking-official-dies-at-72.html}{Archibald
B. Roosevelt Jr.}, whose cousin, the C.I.A. agent
\href{https://www.nytimes.com/2000/06/11/us/kermit-roosevelt-leader-of-cia-coup-in-iran-dies-at-84.html}{Kermit
Roosevelt Jr.}, had orchestrated a 1953 coup to keep the shah in power;
and
\href{https://www.nytimes.com/2002/10/23/obituaries/richard-m-helms-dies-at-89-dashing-exchief-of-the-cia.html}{Richard
M. Helms}, a former director of the C.I.A. and former ambassador to
Iran.

Charles Francis, a veteran of corporate public affairs who worked for
Chase at the time, brought the documents to the attention of The Times.

``Today's corporate campaigns are demolition derbies compared to this
operation,'' he said. ``It was smooth, smooth, smooth and almost
entirely invisible.''

Records of Project Eagle were donated to Yale by Mr. Reed, the
campaign's director. But he deemed the material so potentially
embarrassing to his patron that Mr. Reed,
\href{https://www.nytimes.com/2016/10/05/us/joseph-verner-reed-jr-protocol-chief-who-presided-over-colorful-gaffe-dies-at-78.html}{who
died in 2016}, stipulated that the records remain sealed until Mr.
Rockefeller's death. Mr. Rockefeller
\href{https://www.nytimes.com/2017/03/20/business/david-rockefeller-dead-chase-manhattan-banker.html}{died
in 2017 at the age of 101}.

Some of the information may embarrass others as well.
\href{https://www.commentarymagazine.com/articles/dictatorships-double-standards/}{Hawkish
critics} have often faulted Mr. Carter as worrying too much about human
rights and thus failing to prop up the shah.

But the papers reveal that the president's special envoy to Iran had
actually urged the country's generals to use as much deadly force as
needed to suppress the revolt, advising them about how to carry out a
military takeover to keep the shah in power.

A spokeswoman for Mr. Carter did not respond to requests for comment. A
spokesman for Mr. Carter at the time of the crisis was not immediately
available.

After the hostages were taken, the Carter administration worked
desperately to try to free the captives, and on April 24, 1980,
authorized a rescue mission that collapsed in disaster: A helicopter
crash in the desert killed eight service members, whose charred bodies
were gleefully exhibited by Iranian officials.

The hostage crisis doomed Mr. Carter's presidency. And the team around
Mr. Rockefeller, a lifelong Republican with a dim view of Mr. Carter's
dovish foreign policy, collaborated closely with the Reagan campaign in
its efforts to pre-empt and discourage what it derisively labeled an
``October surprise'' --- a pre-election release of the American
hostages, the papers show.

The Chase team helped the Reagan campaign gather and spread rumors about
possible payoffs to win the release, a propaganda effort that Carter
administration officials have said impeded talks to free the captives.

``I had given my all'' to thwarting any effort by the Carter officials
``to pull off the long-suspected `October surprise,''' Mr. Reed wrote in
a letter to his family after the election, apparently referring to the
Chase effort to track and discourage a hostage release deal. He was
later named Mr. Reagan's ambassador to Morocco.

Mr. Rockefeller then personally lobbied the incoming administration to
ensure that its Iran policies protected the bank's financial interests.

The records indicate that Mr. Rockefeller hoped for the restoration of a
version of the deposed government.

Image

Iranian students climbing the wall of the United States Embassy in
Tehran in 1979.Credit...Irna, via Agence France-Presse

At the start of the Iranian upheaval, the papers show, Mr. Kissinger
advised Mr. Rockefeller that the probable conclusion would be ``a sort
of Bonapartist counterrevolution that rallies the pro-Western elements
together with what was left of the army.''

Mr. Kissinger, in a recent email, acknowledged that the prediction
``reflects my thinking at the time'' but said ``it was a judgment, not a
policy proposal.''

But Mr. Rockefeller evidently continued to advocate for some form of
restoration long after the shah fled Tehran.

As late as December 1980, Mr. Rockefeller personally urged the incoming
Reagan administration to encourage a counterrevolution by stopping ``rug
merchant type bargaining'' for the hostages and instead taking military
action to punish Iran if the hostages were not released. He suggested
occupying three Iranian-controlled islands in the Persian Gulf.

``The most likely outcome of this situation is an eventual replacement
of the present fanatic Shiite Muslim government, either by a military
one or a combination of the military with the civilian democratic
leaders,'' Mr. Rockefeller argued, according to his talking points for
meetings with the Reagan transition team.

Image

A portrait of Ayatollah Ruhollah Khomeini, who became Iran's supreme
leader, being hoisted at Tehran University in 1979.Credit...Abbas/Magnum
Photos

An heir to his family's oil fortune, Mr. Rockefeller styled himself a
corporate statesman and personally knew many White House officials,
including Mr. Carter. He had known the shah since 1962, socializing with
him in New York, Tehran and St. Moritz, Switzerland.

As Tehran's coffers swelled with oil revenues in the 1970s, Chase formed
a joint venture with an Iranian state bank and earned big fees advising
the national oil company.

By 1979, the bank had syndicated more than \$1.7 billion in loans for
Iranian public projects (the equivalent of about \$5.8 billion today).
The Chase balance sheet held more than \$360 million in loans to Iran
and more than \$500 million in Iranian deposits.

Mr. Rockefeller often insisted that his concern for the shah was purely
about Washington's ``prestige and credibility.'' It was about ``the
abandonment of a friend when he needed us most,'' he wrote in his
memoirs.

His only advocacy for the shah, Mr. Rockefeller wrote, had been in a
brief aside to Mr. Carter during an unrelated White House meeting in
April 1979.

``I did nothing more, publicly or privately, to influence the
administration's thinking.''

Yet the Project Eagle papers show that Mr. Rockefeller received detailed
updates on the risks to Chase's holdings, and that even his aside to Mr.
Carter in April had been planned out the previous day with Mr. Reed, Mr.
McCloy and Mr. Kissinger.

Over lunch at the Knickerbocker Club in New York, Mr. Carter's
\href{https://www.nytimes.com/1979/12/08/archives/shah-says-that-us-aided-in-overthrow-memoirs-say-an-american.html}{special
envoy to Tehran, Gen. Robert E. Huyser}, told the Project Eagle team
that he had urged Iran's top military leaders to kill as many
demonstrators as necessary to keep the shah in power.

Image

President Carter and his wife,~ Rosalynn, hosting the shah and empress
at a state dinner at the White House in 1977.Credit... Corbis, via Getty
Images

If shooting over the heads of demonstrators failed to disperse them,
``move to focusing on the chests,'' General Huyser said he told the
Iranian generals, according to minutes of the lunch. ``I got stern and
noisy with the military,'' he added, but in the end, the top general was
``gutless.''

Mr. Rockefeller had his own special envoy to try to help the shah:
\href{https://www.nytimes.com/1979/08/14/archives/the-shah-gets-a-publicrelations-man-and-city-hall-loses-an-official.html}{Robert
F. Armao}, a Republican operative and public relations consultant who
had worked for Mr. Rockefeller's brother Nelson, the former governor of
New York and former vice president.

Mr. Armao became one of the shah's closest advisers, and after
\href{https://www.nytimes.com/1979/01/27/archives/rockefeller-is-dead-at-70-vice-president-under-ford-and-governor.html}{Nelson
Rockefeller died at the start of 1979}, he reported to the Project Eagle
team at Chase nearly every day for more than two years.

``Everybody had the hope that there would be a repeat of the 1953
events,'' Mr. Armao recalled recently, referring to the American-backed
coup that restored the shah the first time he fled.

When the shah's rule became untenable at the start of 1979, the State
Department first turned to David Rockefeller for help relocating the
Iranian monarch in the United States.

``Not large enough for my very special client,'' Mr. Reed wrote to a
Greenwich, Conn., broker who had offered two estates priced at around
\$2 million each --- about \$7.4 million today.

But while the shah tarried in Egypt and Morocco, an Iranian mob
\href{https://www.nytimes.com/1979/02/15/archives/armed-iranians-rush-us-embassy-khomeinis-forces-free-staff-of-100-a.html}{briefly
seized the American Embassy in February}. Diplomats warned that
admitting the shah risked another assault, and Mr. Carter changed his
mind about offering haven.

Mr. Rockefeller refused to deliver this bad news to the shah, afraid
that it would hurt the bank by alienating a prized client.

``The risks were too high relating to the CMB position in Iran,'' he
responded, referring to Chase Manhattan Bank, according to the records.

Instead, Mr. Rockefeller scrambled to find accommodations elsewhere ---
first in the Bahamas, and then in Mexico --- while strategizing with Mr.
Kissinger, Mr. McCloy and others about how to persuade the White House
to let in the shah.

During a three-day push in April, Mr. Kissinger made a personal appeal
to the national security adviser,
\href{https://www.nytimes.com/2017/05/26/us/zbigniew-brzezinski-dead-national-security-adviser-to-carter.html}{Zbigniew
Brzezinski}, and a follow-up phone call to Mr. Carter. Mr. Rockefeller
buttonholed the president at the White House.

And in a speech, Mr. Kissinger publicly accused the Carter
administration of forcing a loyal ally to sail the world in search of
refuge, ``like a flying Dutchman looking for a port of call'' --- the
seed of what became a ``who lost Iran'' campaign theme for the
Republicans.

Mr. McCloy flooded the White House with lengthy letters to senior
officials, often arguing about the danger of demoralizing other
``friendly sovereigns.'' ``Dear Zbig,'' he addressed his old friend Mr.
Brzezinski.

Finally, in October, Mr. Reed sent his personal doctor to Cuernavaca,
Mexico, ``to take a `look-see''' at the shah.

Image

The deposed shah in the Bahamas with his family in March
1979.~Credit...Phil Sandlin/Associated Press

He had been hiding a cancer diagnosis. The doctor,
\href{https://www.nytimes.com/1993/09/26/obituaries/benjamin-h-kean-shah-s-physician-dies-at-81.html}{Benjamin
H. Kean}, determined that the shah needed sophisticated treatment within
a few weeks --- in Mexico, if necessary, Dr. Kean later said he had
concluded.

But when Mr. Reed put the doctor in touch with State Department
officials, they came away with a different prognosis: that the shah was
``at the point of death'' and that only a New York hospital ``was
capable of possibly saving his life,'' as Mr. Carter
\href{https://www.nytimes.com/1981/05/17/magazine/why-carter-admitted-the-shah.html}{described
it at the time to The Times}.

With that opening, the Chase team began preparing the flight to Fort
Lauderdale.

``When I told the Customs man who the principal was, he almost
fainted,'' the waiting executive, Eugene Swanzey, reported the next
morning.

The plane's bathroom was malfunctioning. The shah and his wife hunted in
vain for a missing videocassette to finish a movie. And their four dogs
--- a poodle, a collie, a cocker spaniel and a Great Dane --- jumped on
everyone. The Great Dane ``hadn't been washed in weeks,'' Mr. Swanzey
said. ``The aroma was just terrible.''

When Mr. Reed met the plane on its final arrival in New York, he
recalled the next day, the shah seemed to be thinking, ```At last I am
getting into competent hands.'''

But as he checked the shah into New York Hospital, Mr. Reed was
circumspect.

``I am the unidentified American,'' he told the inquisitive staff.

Mr. Reed, Mr. Rockefeller and Mr. Kissinger met again three days after
the hostages were taken.

Image

Henry Kissinger publicly accused the Carter administration of forcing
the shah, a loyal ally, to sail the world in search of
refuge.Credit...William E. Sauro/The New York Times

``Noted was the feeling of indignation as being high and nothing useful
to say,'' read the minutes.

The White House said the shah had to depart as soon as possible, but
Project Eagle continued.

``The ideal place for the Eagle to land,'' Mr. Reed wrote to Mr. Armao
on Nov. 9, forwarding a brochure for
\href{https://leadingestates.com/estates/hammersley-hill-pawling-new-york/}{a
350-acre Hudson Valley estate}.

A week later, Mr. Rockefeller personally urged Mr. Carter in a phone
call to direct the secretary of state to meet with the shah about ``the
current situation.'' Mr. Carter did not and the shah soon departed, for
Panama, then Egypt.

Only after the death of the shah, on July 27, 1980, nine months after
his landing in Fort Lauderdale, did the Project Eagle team shift to new
objectives. One was protecting Mr. Rockefeller from blame for the
crisis.

Over roast loin of veal and vintage wine at the exclusive River Club in
New York, Mr. Rockefeller and nine others on the team gathered on Aug.
19. Amid discussion of a laudatory biography of the shah by a Berkeley
professor that the team had commissioned, some warned that a Rockefeller
link to the embassy seizure would be hard to escape.

Why was the shah admitted? ``Medical treatment/DR recommended,'' one
said, using Mr. Rockefeller's initials, according to minutes of the
dinner. ``This association cannot be ignored.''

But Mr. Kissinger was reassuring. Congress would never hold an
investigation during an election campaign.

``I don't think we are in trouble any more, David,'' Mr. Kissinger told
him.

The hostages were released on Inauguration Day, Jan. 20, 1981, and a few
days later Mr. Carter's departing White House counsel called Mr.
Rockefeller to inquire about how the release deal affected Chase bank.

``Worked out very well,'' Mr. Rockefeller told him, according to his
records. ``Far better than we had feared.''

Image

Released American hostages being greeted as they arrived in Washington
on Jan. 27, 1981.Credit...Associated Press

Advertisement

\protect\hyperlink{after-bottom}{Continue reading the main story}

\hypertarget{site-index}{%
\subsection{Site Index}\label{site-index}}

\hypertarget{site-information-navigation}{%
\subsection{Site Information
Navigation}\label{site-information-navigation}}

\begin{itemize}
\tightlist
\item
  \href{https://help.nytimes.com/hc/en-us/articles/115014792127-Copyright-notice}{©~2020~The
  New York Times Company}
\end{itemize}

\begin{itemize}
\tightlist
\item
  \href{https://www.nytco.com/}{NYTCo}
\item
  \href{https://help.nytimes.com/hc/en-us/articles/115015385887-Contact-Us}{Contact
  Us}
\item
  \href{https://www.nytco.com/careers/}{Work with us}
\item
  \href{https://nytmediakit.com/}{Advertise}
\item
  \href{http://www.tbrandstudio.com/}{T Brand Studio}
\item
  \href{https://www.nytimes.com/privacy/cookie-policy\#how-do-i-manage-trackers}{Your
  Ad Choices}
\item
  \href{https://www.nytimes.com/privacy}{Privacy}
\item
  \href{https://help.nytimes.com/hc/en-us/articles/115014893428-Terms-of-service}{Terms
  of Service}
\item
  \href{https://help.nytimes.com/hc/en-us/articles/115014893968-Terms-of-sale}{Terms
  of Sale}
\item
  \href{https://spiderbites.nytimes.com}{Site Map}
\item
  \href{https://help.nytimes.com/hc/en-us}{Help}
\item
  \href{https://www.nytimes.com/subscription?campaignId=37WXW}{Subscriptions}
\end{itemize}
