Sections

SEARCH

\protect\hyperlink{site-content}{Skip to
content}\protect\hyperlink{site-index}{Skip to site index}

\href{https://www.nytimes.com/section/style/self-care/}{Self-Care}

\href{https://myaccount.nytimes.com/auth/login?response_type=cookie\&client_id=vi}{}

\href{https://www.nytimes.com/section/todayspaper}{Today's Paper}

\href{/section/style/self-care/}{Self-Care}\textbar{}What Is the Keto
Diet and Does It Work?

\url{https://nyti.ms/2rQUjfS}

\begin{itemize}
\item
\item
\item
\item
\item
\end{itemize}

Advertisement

\protect\hyperlink{after-top}{Continue reading the main story}

Supported by

\protect\hyperlink{after-sponsor}{Continue reading the main story}

Scam or Not

\hypertarget{what-is-the-keto-diet-and-does-it-work}{%
\section{What Is the Keto Diet and Does It
Work?}\label{what-is-the-keto-diet-and-does-it-work}}

Yes. But we don't know how effective, or for how long.

\includegraphics{https://static01.nyt.com/images/2020/01/02/fashion/02scams-keto-1/02scams-keto-1-articleLarge.jpg?quality=75\&auto=webp\&disable=upscale}

By Dawn MacKeen

\begin{itemize}
\item
  Published Jan. 2, 2020Updated Jan. 6, 2020
\item
  \begin{itemize}
  \item
  \item
  \item
  \item
  \item
  \end{itemize}
\end{itemize}

A recent
\href{https://www.lpollockpr.com/in-the-news/nutrition-experts-forecast-2020-will-usher-in-the-ultimate-food-revolution/}{survey}
of registered dietitians named the low-carbohydrate keto diet yet again
as the most popular diet in the United States. Powering this diet is
fat, and loads of it --- all the way up to a hefty 90 percent of one's
daily calories.

Its fans (and marketers) feed social media with before and after photos,
crediting the diet for life-altering weight loss or other effects. They
swirl butter into their coffee, load up on cheese and eat lonely burgers
without their bestie: the bun. Staples like whole grains, legumes, fruit
and starchy vegetables are being largely pushed off the plate as
devotees strive for ketosis --- when the body begins to burn fat instead
of glucose as its primary energy source.

``America is in a state of carbophobia,'' said Whitney Linsenmeyer, a
spokeswoman for the Academy of Nutrition and Dietetics.

The diet is hailed for dropping pounds, burning more calories, reducing
hunger, managing **** diabetes, treating drug resistant epilepsy,
improving blood pressure and lowering cholesterol, as well as
triglycerides, the major storage form of fat in the body. People have
reported improved concentration, too. ``We see pretty dramatic
benefits,'' said Dr. William Yancy, director of the Duke Diet and
Fitness Center.

First, a word: Choosing an eating plan or an approach to eating is very
personal. Everyone's body, tastes and background are unique. The best
approach to food intake is one in which you are healthy and nurtured and
which matches your social and cultural preference. If you want guidance,
it's recommended you consult with a registered dietitian.

\begin{center}\rule{0.5\linewidth}{\linethickness}\end{center}

\hypertarget{what-is-the-ketogenic-diet}{%
\subsection{What is the ketogenic
diet?}\label{what-is-the-ketogenic-diet}}

A ``typical''
\href{https://www.hsph.harvard.edu/nutritionsource/healthy-weight/diet-reviews/ketogenic-diet/}{ketogenic
diet} consists of at least 70 percent of calories derived from fat, less
than 10 percent from carbs and less than 20 percent from protein. The
ketogenic diet, long used to treat epilepsy in children, calls for 90
percent of daily calories to come from fat, with the amount of protein
or carbs varying as long as it's 4 grams of fat for every combined 1
gram of carb and protein, according to the American Epilepsy Society.
That can mean chowing down on a lot of cheese, butter, eggs, nuts,
salmon, bacon, olive oil and non-starchy vegetables such as broccoli,
cauliflower, greens and spinach. For the arithmetic-challenged, apps and
online programs can do the math for you. (No matter what, the keto diet
is vastly different than the USDA dietary
\href{https://health.gov/dietaryguidelines/2015/resources/2015-2020_Dietary_Guidelines.pdf}{recommendations}
of 45 to 65 percent of one's total calories to be carbohydrates, 20 to
35 percent from fat and 10 to 35 percent from protein.)

The goal of the ketogenic diet is to enter a state of ketosis through
fat metabolism. In a ketogenic state, the body uses primarily fat for
energy instead of carbohydrates; with low levels of carbohydrate, fats
can be converted into ketones to fuel the body.

For ketosis, a typical adult must stay below 20 to 50 grams of net
carbohydrates --- measured as total carbs minus fiber --- each day.
Crossing that threshold is easy: a thick slice of bread adds 21
carbohydrates, a medium apple 25 and a cup of milk 12. ``It's very
restrictive,'' said Carla Prado, an associate professor and director of
the University of Alberta's Human Nutrition Research Unit. It's not just
bread and soda that are on the outs but high-sugar fruit and starchy
veggies like potatoes, as well as too much protein. Also, dieters have
to be on high alert for hidden **** carbs, often invisible to the eye,
yet coating that seemingly keto-friendly fried cheese.

\begin{center}\rule{0.5\linewidth}{\linethickness}\end{center}

\hypertarget{can-i-lose-weight-on-the-keto-diet}{%
\subsection{Can I lose weight on the keto
diet?}\label{can-i-lose-weight-on-the-keto-diet}}

Yes. Certainly in the short-term, it appears that way. For the first two
to six months, there's evidence that a very low-carbohydrate diet can
help you lose more weight than the standard high-carbohydrate, low-fat
diet, according to a new literature
\href{https://www.lipidjournal.com/article/S1933-2874(19)30267-3/pdf}{review}
of low-carb diets by the National Lipid Association.

``By 12 months, that advantage is essentially gone,'' said Carol F.
Kirkpatrick, director of Idaho State University's Wellness Center, and
lead author of the new literature review.

After that, weight loss seems to equalize between those two popular diet
regimens. She said keto was best used to kick-start a diet, before
transitioning to a carb intake that you can adhere to for the longer
term.

\begin{center}\rule{0.5\linewidth}{\linethickness}\end{center}

\hypertarget{how-long-does-it-take-to-see-results-on-the-keto-diet}{%
\subsection{How long does it take to see results on the keto
diet?}\label{how-long-does-it-take-to-see-results-on-the-keto-diet}}

For some, it's the promised land of diets. Instead of cringing through
carrot sticks, they can fill up guilt-free on chorizo with scrambled
eggs. Indeed,
\href{https://onlinelibrary.wiley.com/doi/full/10.1111/obr.12230}{some
evidence} suggests that people feel less hungry while in ketosis, and
have fewer cravings.

``That's why it's become so popular for the general population,'' said
Dr. Mackenzie C. Cervenka, medical director of Johns Hopkins Hospital's
Adult Epilepsy Diet Center. ``Because once you are in ketosis, it's easy
to follow.'' Usually, it takes between one to four days to enter the
state, doctors say, but it depends on many factors like activity level:
a runner, for example, may sprint there faster than a couch potato.

The keto diet appears to deliver fast results: The first pounds may seem
to slip off. That can be seductive but it's likely water weight. Then,
dietitians say, it's back to energy in minus energy out. You can
absolutely gain weight on any diet if you're consuming 5,000 calories a
day, according to Dr. Linsenmeyer, who is also director of Saint Louis
University's Didactic Program in Dietetics.

``It's not like it is going to magically alter your metabolism to where
calories don't matter anymore,'' she said. And when resuming the carbs,
that water weight returns.

\begin{center}\rule{0.5\linewidth}{\linethickness}\end{center}

\hypertarget{is-this-a-scam}{%
\subsection{Is This A Scam?}\label{is-this-a-scam}}

\hypertarget{is-}{%
\subsubsection{Is ...}\label{is-}}

Celery Juice

,

Kombucha

,

Activated Charcoal

,

CBD

,

Turmeric

,

Fish Oil

,

Chlorophyll

,

Intermittent Fasting

,

The Keto Diet

,

Probiotics

,

Collagen

,

Coffee

,

\hypertarget{a-scam}{%
\subsubsection{A Scam?}\label{a-scam}}

Facts about wellness.

Will these trends change your life --- or

take your money?

\begin{center}\rule{0.5\linewidth}{\linethickness}\end{center}

\hypertarget{but-can-the-ketogenic-diet-help-to-burn-more-calories}{%
\subsection{But can the ketogenic diet help to burn more
calories?}\label{but-can-the-ketogenic-diet-help-to-burn-more-calories}}

There is some evidence that it can. The research is limited and
conflicting here too. It may be a very small effect, and not meaningful
for weight control. That's what one
\href{https://www.ncbi.nlm.nih.gov/pmc/articles/PMC4962163/}{study}
found. In it, 17 obese or overweight volunteers moved into metabolic
wards for two months and had every last spoonful of food monitored.
(This recounting of the science uses definitional terms like ``obese''
to be clear about the subjects of research studies.) For the first
month, they consumed a high-carb diet; for the second, they had a
ketogenic one, with both plans equal in calories.

``We fed them every morsel of food that they ate,'' said Kevin Hall,
integrative physiology section chief for the National Institute of
Diabetes and Digestive and Kidney Diseases' Laboratory of Biological
Modeling. ``There were no cheat days.'' In the end, though the
participants' insulin levels did decrease while eating the bunless
burger, the subjects didn't lose more fat than when they had bread. The
study was limited, though, by having a small sample size, and not having
a comparison group that wasn't on the back-to-back regimens.

For some, a low-carb diet can be appealing. That doesn't mean that diet
is superior, according to a study that followed 609 overweight adults on
either a low-carb or a low-fat diet for a year. In the end, both groups
shed almost the same amount on average --- about 12 to 13 pounds,
according to the randomized clinical
\href{https://jamanetwork.com/journals/jama/fullarticle/2673150?resultClick=1}{trial}
that examined a low-carb diet less restrictive than the keto. The
take-home message? ``You can succeed on both,'' said Christopher
Gardner, the lead author and a professor of medicine and nutrition
scientist at Stanford Prevention Research Center.

\begin{center}\rule{0.5\linewidth}{\linethickness}\end{center}

\hypertarget{does-the-ketogenic-diet-offer-long-term-benefits}{%
\subsection{Does the ketogenic diet offer long-term
benefits?}\label{does-the-ketogenic-diet-offer-long-term-benefits}}

It's not known yet. ``If you tell people to go on this diet forever and
for a longer term, there is no evidence,'' said Dr. Prado, of the
University of Alberta who co-authored a
\href{https://jandonline.org/article/S2212-2672(17)30115-6/fulltext}{narrative
review} on the ketogenic diet as a possible therapy for cancer.

The diet does help children with epilepsy: Nearly a third to two-thirds
of patients experience
\href{https://www.epilepsybehavior.com/article/S1525-5050(11)00112-0/fulltext}{50
percent} fewer seizures after six months on the regimen. (Even back in
\href{https://www.ncbi.nlm.nih.gov/pmc/articles/PMC6123874/}{400 B.C.}
people fasted to treat epilepsy. And the ketogenic diet itself is nearly
a century old, having been popular to help with seizures until the
discovery of an anticonvulsant drug.) There are case studies on how
\href{https://www.clinicalnutritionjournal.com/article/S0261-5614(17)31399-7/fulltext}{10
patients} with a rare condition fared on the diet for a decade, but most
well-designed studies in this field have not extended beyond two years.

\begin{center}\rule{0.5\linewidth}{\linethickness}\end{center}

\hypertarget{does-a-low-carb-diet-help-people-with-diabetes}{%
\subsection{Does a low-carb diet help people with
diabetes?}\label{does-a-low-carb-diet-help-people-with-diabetes}}

Yes. ``Carbohydrate is the biggest driver of blood sugar,'' said Duke's
Dr. Yancy, who sees a lot of promise in the diet helping those with
diabetes.

A new
\href{https://jamanetwork.com/journals/jamainternalmedicine/article-abstract/2753678?utm_campaign=articlePDF\%26utm_medium\%3DarticlePDFlink\%26utm_source\%3DarticlePDF\%26utm_content\%3Djamainternmed.2019.4802}{randomized
clinical trial} enrolled 263 adults with Type 2 diabetes into group
medical visits, with half receiving medication adjustment for better
blood sugar control, and the others undergoing weight management
counseling using a low-carb diet. (All participants of the study had a
BMI that fell within the range of overweight or obese.) Both groups
experienced lowered average blood sugar levels at the end of 48 weeks,
according to findings in the Journal of the American Medical Association
Internal Medicine. However, the weight management group on the low-carb
diet slimmed down more, required less medication and had fewer
problematic low blood sugar episodes.

For those with Type 2 diabetes, a low-carb diet seems to improve average
blood sugar levels better in the first year than the high-carbohydrate,
low-fat diet. After that time period, the review by the National Lipid
Association found that difference almost disappears --- but with a very
important benefit: the low-carb participants were able to use less
medication. ``People like that because they don't like to be on diabetes
medicines,'' Dr. Yancy said.

\begin{center}\rule{0.5\linewidth}{\linethickness}\end{center}

\hypertarget{is-there-a-healthy-way-to-eat-more-fat}{%
\subsection{Is there a healthy way to eat more
fat?}\label{is-there-a-healthy-way-to-eat-more-fat}}

When Dr. Cervenka of Johns Hopkins Hospital starts her patients with
epilepsy on a low-carbohydrate diet, she doesn't rule out saturated fats
from animal products. She wants them to get used to the new way of
eating. But if cholesterol levels climb and stay that way, she advises
them to shift to foods and oils with mono- and polyunsaturated fats like
avocados or olive oil.

While the diet's effect on LDL (``bad'' cholesterol) appears to be
mixed, the National Lipid Association's review found that a very
low-carbohydrate diet does seem to improve HDL (commonly known as the
good cholesterol). Beyond a year, it seems these benefits don't last,
much like in weight loss. Only lowered triglyceride levels seem to have
any staying power. Other findings: The evidence on blood pressure is
inconsistent, and the reports of improved mental clarity are not
supported by controlled studies.

\begin{center}\rule{0.5\linewidth}{\linethickness}\end{center}

\hypertarget{whats-the-effect-of-all-that-fatty-meat-on-your-health}{%
\subsection{What's the effect of all that fatty meat on your
health?}\label{whats-the-effect-of-all-that-fatty-meat-on-your-health}}

And what happens, for example, after cutting down fruits, legumes and
whole grains --- all food that
\href{https://www.ncbi.nlm.nih.gov/pubmed/?term=Mozaffarian+D.+Dietary+and+policy+priorities+for+cardiovascular+disease\%2C+diabetes\%2C+and+obesity\%3A+a+comprehensive+review.+Circulation}{studies}
point to reducing cardiometabolic risk?

Dr. Neil J. Stone, a preventive cardiologist at Northwestern
University's Feinberg School of Medicine, worries about this, having
seen the bad cholesterol levels of some of his patients on the keto diet
increase drastically. (It doesn't happen to all but it does happen to
some.) ``Any diet that raises major risk factors for coronary heart
disease puts patients at risk over the long term,'' he said.

(There's also much
\href{https://www.atherosclerosis-journal.com/article/S0021-9150(19)31589-8/fulltext}{debate}
about LDL particles and whether the type that's increasing with the keto
diet, larger LDL particles, doesn't increase heart disease risk.)

An
\href{https://ahajournals.org/doi/10.1161/CIR.0000000000000510}{advisory}
Dr. Stone co-authored by the American Heart Association stated that
lowering dietary saturated fat, like fatty meats and high-fat dairy, can
be beneficial. And swapping it for unsaturated fats like safflower oil
or olive oil may reduce the risk of cardiovascular disease. But before
going on any diet, he recommends you ask yourself: What are your goals?
Are they short-term or long-term? Can you get there without taking as
many risks?

There are many ways to interpret the keto diet. Some people will eat a
salad with chicken, dressed in olive oil, while others will feast on
stacks of bacon washed down by diet soda, the kind of diet known as
``dirty keto.'' That's eating anything, including processed foods, as
long as your carbs are low enough and your fat high enough to achieve
ketosis. The best diet is one that works for you, but if you want to try
this, they recommend avoiding trans fats like margarine, limiting
saturated fat by consuming lean cuts of beef and skinless chicken breast
and incorporating fatty fish like salmon into your diet. Reach for foods
high in unsaturated fats like avocado, nuts, seeds and olive oil.

Stanford's Dr. Gardner also says he sees one common misconception about
keto: eating too much protein. Most amino acids in protein foods can be
converted into glucose in the body, undermining efforts to keep carb
intake low. ``It drives me nuts that people don't get it,'' he said when
he sees people eat, for instance, steak after steak.

\begin{center}\rule{0.5\linewidth}{\linethickness}\end{center}

\hypertarget{are-there-side-effects-of-the-keto-diet}{%
\subsection{Are there side effects of the keto
diet?}\label{are-there-side-effects-of-the-keto-diet}}

At first some can experience some stomach issues and GI distress.
``Ninety percent of calories from fat is probably going to be a shock to
the system,'' said Dr. Linsenmeyer.

It's crucial, doctors say, to consult with a dietitian or physician,
have cholesterol levels regularly checked, and replenish the fluids and
sodium lost by increased urination and the severe restriction of
carbohydrates. If not, within two to four days of beginning the diet,
that depletion can bring on the ``keto flu'' --- symptoms like
dizziness, poor sleep and fatigue in some people.

``Carbohydrates have a lot of nutrients that can help us maintain our
body function,'' said Dr. Prado. On the diet, some people experience
``keto breath,'' a halitosis likely caused by the production of acetone,
which is one of the ketone bodies.

Possible
\href{https://www.epilepsybehavior.com/article/S1525-5050(11)00112-0/abstract}{side
effects} for patients with epilepsy starting the diet include
constipation from reduced fiber intake, vomiting, fatigue, hypoglycemia,
worsening reflux and increased frequency of seizures. The National Lipid
Association review urges that patients with lipid disorders (like high
cholesterol or triglycerides), a history of atherosclerotic
cardiovascular disease (such as having a heart attack or stroke), heart
failure and kidney and liver disease take caution if considering the
diet. People on blood thinners should take extra care.

\begin{center}\rule{0.5\linewidth}{\linethickness}\end{center}

\hypertarget{advice-we-can-all-agree-on-eat-healthily-there-is-no-quick-fix}{%
\subsection{Advice we can all agree on: Eat healthily. There is no quick
fix.}\label{advice-we-can-all-agree-on-eat-healthily-there-is-no-quick-fix}}

Advice from the battling diet camps can be confusing. But Dr. Hall of
the National Institutes of Health **** said there is a middle ground:
``Can we get beyond this low-fat, low-carb diet wars, and look to where
people have this common piece of advice?'' He said some versions of both
the low-fat and keto diets can be healthier than the standard American
diet, which is known as
\href{https://www.nytimes.com/2019/04/26/books/review/self-help-diet-weight-good-health.html}{SAD}for
a reason. Low in
\href{https://health.gov/dietaryguidelines/2015/guidelines/chapter-1/}{vegetables
and fruit}, it's filled with prepackaged foods with additives, added
sugars and unrecognizable ingredients.

Keto isn't the only way to lose weight or change your life, obviously.
Dietitians say it is not essential to cut back on as many foods, since a
moderate low-carb diet may still hold
\href{https://doi.org/10.1038/s41430-017-0019-4}{benefits} for diabetes
or weight loss.

One thing is certain: Any meaningful change starts with behavior. Are
you at a right point to make a change in your life? Dr. Yancy suggests
asking friends and family to support you, confer with a doctor,
incorporate physical activity and begin to think of it not as a
temporary measure but more of a lifestyle change.

Whichever eating plan one chooses for 2020, Dr. Hall said certain
recommendations are nearly universal: cut down on refined carbs and
\href{https://www.cell.com/cell-metabolism/fulltext/S1550-4131(19)30248-7?_returnURL=https\%3A\%2F\%2Flinkinghub.elsevier.com\%2Fretrieve\%2Fpii\%2FS1550413119302487\%3Fshowall\%3Dtrue}{ultra-processed
foods}, and consume more whole foods, particularly non-starchy
vegetables, such as broccoli, asparagus and spinach.

``It may be the `optimal diet' lies somewhere between what has been
proposed historically --- meaning the high-carbohydrate, low-fat diet
--- and the ketogenic diet,'' said Dr. Cervenka.

Advertisement

\protect\hyperlink{after-bottom}{Continue reading the main story}

\hypertarget{site-index}{%
\subsection{Site Index}\label{site-index}}

\hypertarget{site-information-navigation}{%
\subsection{Site Information
Navigation}\label{site-information-navigation}}

\begin{itemize}
\tightlist
\item
  \href{https://help.nytimes.com/hc/en-us/articles/115014792127-Copyright-notice}{©~2020~The
  New York Times Company}
\end{itemize}

\begin{itemize}
\tightlist
\item
  \href{https://www.nytco.com/}{NYTCo}
\item
  \href{https://help.nytimes.com/hc/en-us/articles/115015385887-Contact-Us}{Contact
  Us}
\item
  \href{https://www.nytco.com/careers/}{Work with us}
\item
  \href{https://nytmediakit.com/}{Advertise}
\item
  \href{http://www.tbrandstudio.com/}{T Brand Studio}
\item
  \href{https://www.nytimes.com/privacy/cookie-policy\#how-do-i-manage-trackers}{Your
  Ad Choices}
\item
  \href{https://www.nytimes.com/privacy}{Privacy}
\item
  \href{https://help.nytimes.com/hc/en-us/articles/115014893428-Terms-of-service}{Terms
  of Service}
\item
  \href{https://help.nytimes.com/hc/en-us/articles/115014893968-Terms-of-sale}{Terms
  of Sale}
\item
  \href{https://spiderbites.nytimes.com}{Site Map}
\item
  \href{https://help.nytimes.com/hc/en-us}{Help}
\item
  \href{https://www.nytimes.com/subscription?campaignId=37WXW}{Subscriptions}
\end{itemize}
