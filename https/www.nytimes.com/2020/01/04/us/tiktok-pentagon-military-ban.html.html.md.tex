Sections

SEARCH

\protect\hyperlink{site-content}{Skip to
content}\protect\hyperlink{site-index}{Skip to site index}

\href{https://www.nytimes.com/section/us}{U.S.}

\href{https://myaccount.nytimes.com/auth/login?response_type=cookie\&client_id=vi}{}

\href{https://www.nytimes.com/section/todayspaper}{Today's Paper}

\href{/section/us}{U.S.}\textbar{}U.S. Military Branches Block Access to
TikTok App Amid Pentagon Warning

\url{https://nyti.ms/2Qois6N}

\begin{itemize}
\item
\item
\item
\item
\item
\end{itemize}

Advertisement

\protect\hyperlink{after-top}{Continue reading the main story}

Supported by

\protect\hyperlink{after-sponsor}{Continue reading the main story}

\hypertarget{us-military-branches-block-access-to-tiktok-app-amid-pentagon-warning}{%
\section{U.S. Military Branches Block Access to TikTok App Amid Pentagon
Warning}\label{us-military-branches-block-access-to-tiktok-app-amid-pentagon-warning}}

The Department of Defense advised military personnel to remove the
Chinese-owned social media application from government-issued and
personal smartphones.

\includegraphics{https://static01.nyt.com/images/2020/01/04/us/04xp-tiktok/merlin_163685841_df847d09-1fbb-4439-a40c-5ebee5c2c8f6-articleLarge.jpg?quality=75\&auto=webp\&disable=upscale}

\href{https://www.nytimes.com/by/neil-vigdor}{\includegraphics{https://static01.nyt.com/images/2019/07/25/reader-center/author-neil-vigdor/author-neil-vigdor-thumbLarge.png}}

By \href{https://www.nytimes.com/by/neil-vigdor}{Neil Vigdor}

\begin{itemize}
\item
  Jan. 4, 2020
\item
  \begin{itemize}
  \item
  \item
  \item
  \item
  \item
  \end{itemize}
\end{itemize}

\href{https://cn.nytimes.com/usa/20200106/tiktok-pentagon-military-ban/}{阅读简体中文版}\href{https://cn.nytimes.com/usa/20200106/tiktok-pentagon-military-ban/zh-hant/}{閱讀繁體中文版}

The warning from the Pentagon was unequivocal: Military personnel should
delete TikTok from all smartphones.

Now, a number of United States military branches are heeding that
advice, issued last month by the Defense Department, and have banned the
popular Chinese-owned social media app on government-issued smartphones.

Some have even strongly discouraged members of the armed forces from
keeping TikTok on their personal electronic devices.

The vigilance coincides with heightened scrutiny of the short-form
video-sharing platform by Congress and a
\href{https://www.nytimes.com/2019/11/01/technology/tiktok-national-security-review.html}{national
security review of TikTok}, which is
\href{https://sensortower.com/blog/q3-2019-data-digest}{among the top
downloaded smartphone apps worldwide}.

``Marine Corps Forces Cyberspace Command has blocked TikTok from
government-issued mobile devices,'' Capt. Christopher Harrison, a United
States Marine Corps spokesman, said Friday in an email. ``This decision
is consistent with our efforts to proactively address existing and
emerging threats as we secure and defend our network. This block only
applies to government-issued mobile devices.''

In a Dec. 16 message to the various military branches, the Pentagon said
there was a ``potential risk associated with using the TikTok app,'' and
it advised employees to take several precautions to safeguard their
personal information. It said the easiest solution to prevent ``unwanted
actors'' from getting access to that information was to remove the app.

``Doing so will not prevent already potentially compromised information
from propagating, but it could keep additional information from being
collected,'' the Pentagon's message said.

The United States Army banned TikTok from military-issued smartphones in
response to last month's warning, Lt. Col. Robin Ochoa, an Army
spokeswoman, said Friday in an email.

``Those who have a government issued device are requested to remove the
application,'' she said.

Josh Gartner, a spokesman for ByteDance, the Chinese parent company of
TikTok, declined to comment about the Pentagon warning and the response
of several military branches.

This was not the first time that the Defense Department had been
compelled to urge members of the military to remove a popular app from
their phones.

In 2016, the Defense Department banned Pokémon Go, the augmented-reality
game, from military smartphones. But in that case, military officials
cited concerns over productivity and the potential distraction hazards
of pursuing the virtual Pokémon while driving or walking. The Canadian
military also
\href{https://www.nytimes.com/2020/01/01/world/canada/pokemon-go-canada-military.html}{grappled
with Pokémon Go}.

The concerns over TikTok center on cybersecurity and spying by the
Chinese government.

In a November blog post on TikTok's website, the general manager of
TikTok US, Vanessa Pappas,
\href{https://newsroom.tiktok.com/en-us/explaining-tiktoks-approach-in-the-us}{wrote
that data security was a priority} and that the company wanted to be as
transparent as possible for stakeholders in the United States.

The blog post came as the United States government opened a national
security review of a Chinese company's acquisition of the American
company that became TikTok.

``As we have said before, and recently confirmed through an independent
security audit, we store all US user data in the United States, with
backup redundancy in Singapore,'' Ms. Pappas wrote. ``TikTok's data
centers are located entirely outside of China.''

In October, Senators Chuck Schumer and Tom Cotton, Democrat of New York
and Republican of Arkansas,
\href{https://www.democrats.senate.gov/imo/media/doc/10232019\%20TikTok\%20Letter\%20-\%20FINAL\%20PDF.pdf}{sent
a letter to the acting director of national intelligence}, Joseph
Maguire, calling for an assessment of national security risks posed by
TikTok and other China-based content platforms.

The senators said Chinese companies must comply with a ``vague
patchwork'' of intelligence, national security and cybersecurity laws
that have no mechanism for appealing decisions of the Chinese Communist
government.

``Questions have also been raised regarding the potential for censorship
or manipulation of certain content,'' the senators' letter said.

``TikTok reportedly censors materials deemed politically sensitive to
the Chinese Communist Party, including content related to the recent
Hong Kong protests, as well as references to Tiananmen Square, Tibetan
and Taiwanese independence, and the treatment of Uighurs. The platform
is also a potential target of foreign influence campaigns like those
carried out during the 2016 election on U.S.-based social media
platforms.''

Members of the United States Air Force are not allowed to install
unauthorized apps on their military-issued phones, an Air Force
spokeswoman said Friday in an email. The spokeswoman did not specify
whether TikTok was one of those applications and did not immediately
respond to a follow-up inquiry.

``The threats posed by social media are not unique to TikTok (though
they may certainly be greater on that platform), and DoD personnel must
be cautious when making any public or social media post,'' the Air Force
spokeswoman said. ``All DoD personnel take annual cyber-awareness
training that covers the threats that social media can pose, as well as
annual operations security training that covers the broader issue of
safeguarding information.''

Chief Warrant Officer Barry Lane, a United States Coast Guard spokesman,
said in an email Friday, ``TikTok is not an application currently used
on any official Coast Guard device.''

He said Coast Guard members go through security and cyberawareness
training ever year.

``This training includes best practices to safeguard sensitive and
personal information on social media platforms,'' he said.

The United States Navy did not immediately respond to requests for
comment.

Advertisement

\protect\hyperlink{after-bottom}{Continue reading the main story}

\hypertarget{site-index}{%
\subsection{Site Index}\label{site-index}}

\hypertarget{site-information-navigation}{%
\subsection{Site Information
Navigation}\label{site-information-navigation}}

\begin{itemize}
\tightlist
\item
  \href{https://help.nytimes.com/hc/en-us/articles/115014792127-Copyright-notice}{©~2020~The
  New York Times Company}
\end{itemize}

\begin{itemize}
\tightlist
\item
  \href{https://www.nytco.com/}{NYTCo}
\item
  \href{https://help.nytimes.com/hc/en-us/articles/115015385887-Contact-Us}{Contact
  Us}
\item
  \href{https://www.nytco.com/careers/}{Work with us}
\item
  \href{https://nytmediakit.com/}{Advertise}
\item
  \href{http://www.tbrandstudio.com/}{T Brand Studio}
\item
  \href{https://www.nytimes.com/privacy/cookie-policy\#how-do-i-manage-trackers}{Your
  Ad Choices}
\item
  \href{https://www.nytimes.com/privacy}{Privacy}
\item
  \href{https://help.nytimes.com/hc/en-us/articles/115014893428-Terms-of-service}{Terms
  of Service}
\item
  \href{https://help.nytimes.com/hc/en-us/articles/115014893968-Terms-of-sale}{Terms
  of Sale}
\item
  \href{https://spiderbites.nytimes.com}{Site Map}
\item
  \href{https://help.nytimes.com/hc/en-us}{Help}
\item
  \href{https://www.nytimes.com/subscription?campaignId=37WXW}{Subscriptions}
\end{itemize}
