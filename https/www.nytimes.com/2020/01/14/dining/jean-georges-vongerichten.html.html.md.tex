Sections

SEARCH

\protect\hyperlink{site-content}{Skip to
content}\protect\hyperlink{site-index}{Skip to site index}

\href{https://www.nytimes.com/section/food}{Food}

\href{https://myaccount.nytimes.com/auth/login?response_type=cookie\&client_id=vi}{}

\href{https://www.nytimes.com/section/todayspaper}{Today's Paper}

\href{/section/food}{Food}\textbar{}How Jean-Georges Vongerichten Went
From `No Good' Kid to 4-Star Chef

\url{https://nyti.ms/2uN9hog}

\begin{itemize}
\item
\item
\item
\item
\item
\item
\end{itemize}

Advertisement

\protect\hyperlink{after-top}{Continue reading the main story}

Supported by

\protect\hyperlink{after-sponsor}{Continue reading the main story}

\hypertarget{how-jean-georges-vongerichten-went-from-no-good-kid-to-4-star-chef}{%
\section{How Jean-Georges Vongerichten Went From `No Good' Kid to 4-Star
Chef}\label{how-jean-georges-vongerichten-went-from-no-good-kid-to-4-star-chef}}

The globally prolific chef started out as a small-town truant and
troublemaker. Then he got to work.

\includegraphics{https://static01.nyt.com/images/2020/01/15/dining/14jeangeorges1/14jeangeorges1-articleLarge.jpg?quality=75\&auto=webp\&disable=upscale}

By Alan Richman

\begin{itemize}
\item
  Jan. 14, 2020
\item
  \begin{itemize}
  \item
  \item
  \item
  \item
  \item
  \item
  \end{itemize}
\end{itemize}

This is how it began, the career of one of the most versatile, ingenious
and adventurous chefs in the history of American cuisine. Jean-Georges
Vongerichten can pinpoint the day, the place, the words.

His family had taken him to
\href{https://www.auberge-de-l-ill.com/fr/}{Auberge de l'Ill}, a
restaurant with three Michelin stars in Alsace, the French region where
they lived, to celebrate his 16th birthday. To the table came the
renowned chef
\href{https://www.nytimes.com/2008/05/13/world/europe/13haeberlin.html}{Paul
Haeberlin}, and the boy's father requested a favor.

``My father liked to talk,'' Mr. Vongerichten recalled. ``He already had
three glasses of wine, and he said to the chef: `My son is no good. Do
you need somebody to wash dishes? He will do it.' ''

\includegraphics{https://static01.nyt.com/images/2020/01/15/dining/14jeangeorges2/14jeangeorges2-articleLarge.jpg?quality=75\&auto=webp\&disable=upscale}

One might assume that an individual who has mastered gastronomy as
thoroughly as Mr. Vongerichten would have been fascinated by food from
birth, gone from crib to kitchen at a crawl, yanked his mother's apron
strings, licked pots, cried out for crème brûlée, been fascinated by the
rituals of the family table.

Not so young Jean-Georges, whose primary relationship with food was
showing up for meals on time. Every day, his mother and grandmother
prepared lunch for the family, 12 in all, and employees of his father's
company.

``We were feeding 35 people for lunch,'' Mr. Vongerichten said. ``There
was a lot of food on the table at 12:30. By 12:45 it was gone. And we
never went to restaurants. The family was too big.''

In his mother's cooking lurked a hint of the creativity that would
emerge in the son. She made lavish use of an Alsatian vinegar, called
Melfor, infused with honey as well as plant and fruit extracts. Today,
complex, amplified flavors appear in almost all of Mr. Vongerichten's
dishes.

Image

Mr. Vongerichten with his mother, Jeanine, and father, George, in the
late 1980s.Credit...

No chef working in America is quite his equal. Many, like him, have
created multiple restaurants. A very few, like him, are considered among
the best. Even fewer are both. Mr. Vongerichten is able to do more, and
he is able to do it better, his style augmented by extraordinary
open-mindedness, a willingness to embrace the local culture wherever he
might be.

``He has a radar sense of trends, giving customers tastes they crave
before other chefs realize they are desired,'' said Eric Ripert, the
chef and co-owner of
\href{https://www.nytimes.com/2012/05/23/dining/reviews/le-bernardin-in-midtown-manhattan.html}{Le
Bernardin}, in New York. ``And he has an uncanny eye for little-known
cuisines.''

Image

Mr. Vongerichten, flanked by his business partner, Phil Suarez, right,
and the chef Eric Ripert.~Credit...Patrick McMullan/Getty Images

Mr. Vongerichten met his business partner, the entrepreneur Phil Suarez,
almost 40 years ago when the chef was cooking at
\href{https://www.nytimes.com/1988/04/22/arts/restaurants-067888.html}{Lafayette},
a restaurant of matchless creativity **** in the Drake Hotel in New
York. Mr. Suarez kept coming in for lunch, bringing celebrity guests
like Michael Jackson. Each time, he would hand Mr. Vongerichten his
business card.

``Finally,'' Mr. Vongerichten recalled, ``I said to him, `Phil, I have
25 of your cards.' ''

The two now operate 38 restaurants around the world. Few are exact
duplicates in size, style or cuisine. Their smallest is JG Tokyo, a
ground-floor establishment with 14 counter seats and Mr. Vongerichten's
food served in the style of a sushi bar --- presented piece by piece,
and eaten with chopsticks. The largest (and the highest, more than 1,000
feet above the ground) opened last August in Philadelphia, in a new Four
Seasons Hotel: Jean-Georges Philadelphia, a fine-dining restaurant,
seats 120, while the JG SkyHigh bar and lounge accommodates 92.

Scheduled to open this year is a 50,000-square-foot food hall in New
York's seaport district; the partners opened a seafood restaurant,
\href{https://www.nytimes.com/2019/07/30/dining/the-fulton-review-pete-wells-jean-georges.html}{the
Fulton}, there in May.

Image

The Fulton, a seafood restaurant that Mr. Vongerichten opened in May in
the seaport district of New York.Credit...Ellen Silverman for The New
York Times

Mr. Suarez said he gets several calls a week from all over the country
asking about partnerships. ``A guy who wants to be the lead restaurateur
in a town wants to set the town on fire with a Jean-Georges
restaurant,'' he said. ``They are all flattery and enthusiasm. We're
lucky enough, after almost 40 years together, to be at that point.''

At 62, Mr. Vongerichten is fit and tireless, as
\href{https://www.nytimes.com/2019/10/17/magazine/jean-georges-restaurants.html}{hard-working
as ever}. He recently published a memoir,
``\href{https://wwnorton.com/books/9780393608489}{JGV: A Life in 12
Recipes}.'' He works out daily, keeps his weight steady at 170 pounds,
cooks in Prada shoes. Walking with him through the Union Square
Greenmarket, where he shops, meets with his chefs and offers advice to
anyone who asks, is like stepping into a Las Vegas casino with Frank
Sinatra.

He has three grandchildren, and loves everything about them except being
called Grandfather.

``I don't think he likes that name because I don't think he likes to
age,'' said his daughter Louise Ulukaya Vongerichten, one of three
children. ``He is young in his mind.''

Retirement, Mr. Vongerichten said, ``sounds like a disease.''

Image

Philippe Vongerichten, general manager of Jean-Georges, demonstrating
his proficiency at carving pineapples.Credit...Brian Harkin for The New
York Times

His youngest brother,
\href{https://ny.eater.com/2011/7/15/6669497/philippe-vongerichten-looks-back-on-14-years-of-jean-georges}{Philippe
Vongerichten}, general manager of the flagship
\href{https://www.jean-georgesrestaurant.com/jean-georges/}{Jean-Georges
restaurant} in New York (which The New York Times has
\href{https://www.nytimes.com/2014/04/09/dining/restaurant-review-jean-georges-on-the-upper-west-side.html}{awarded
four stars}) is asked if the chef was as terrible a child as he claims
to have been.

His answer is immediate: ``He was.''

The family business was coal. Jean-Georges's great-great-grandfather,
who was Dutch, came upon land alongside a canal not far from Strasbourg,
the Alsatian capital, and claimed it, much as American settlers did in
the Old West. ``It is where my grandfather was born, my father was born
and I was born,'' Mr. Vongerichten said.

The coal was transported from mines in the north of France on barges
pulled by horses. Eventually, the family business shifted to heating
oil, but when Mr. Vongerichten was a boy, it was coal.

``I was always black from head to toe from the coal dust in our
backyard,'' he said. ``We were living and breathing coal. The topic of
every meal was coal. No way I was going to join my dad in that
company.''

Image

A few of those who dined at the Vongerichten family table each day.
Jean-Georges stands between a cousin (seated) and his grandmother
(behind him) in 1965.

Image

Mr. Vongerichten, right, with his brother Christian, who as a boy loved
to wear his older brother's clothes.

Philippe, one of four children, shared a large bedroom with Jean-Georges
and their middle brother, Christian. ``We had one big closet with three
drawers,'' Philippe said. ``Open the drawers and Jean-Georges's was
perfectly neat, everything arranged by colors. Christian was two years
younger than Jean-Georges, and sometimes he would take a pair of
Jean-Georges's socks or even his underwear. It would make him crazy.''

Mr. Vongerichten became an altar boy --- ``but a bad altar boy,'' he
confessed. ``I would steal my father's cigars when I was 8 or 9.'' At
14, he stole a motorbike. The police knew him well: They came to the
Vongerichten home and found it in the garage.

His parents sent him to parochial school. He skipped classes, paid no
attention to studies. They tried a trade school for engineering, still
hoping he would take over the family business.

``The school called my parents and said, `He's been here for a month, we
saw him eight times,' '' Mr. Vongerichten said. At Christmas, school
officials told them their son had to leave. ``They had paid in advance
for two years. My father was so mad. I was happy. I was hating it with a
passion.''

Philippe said his brother is still much the same: When he wants
something, he makes certain he gets it. The two have worked together at
Jean-Georges for more than 20 years.

``We have never raised our voices at each other,'' he said. ``We do have
disagreements about certain things.'' Philippe paused and laughed. ``He
always wins.''

Image

Mentor and apprentice: Paul Haeberlin, left, with Mr. Vongerichten in
1988.Credit...

At that 16th birthday dinner in 1973, Chef Haeberlin showed interest. As
a matter of fact, he told Mr. Vongerichten's father, the restaurant was
looking for an apprentice.

In those days, France's great restaurants accepted one apprentice each
year to serve an unpaid term of three years; that way there was always a
one-year, a two-year and a three-year apprentice. On his first day, Mr.
Vongerichten washed dishes from 8 a.m. to 10 p.m. without complaint.

He moved into a room above the restaurant, and he loved it. He skinned
hares, plucked chickens. He ran to the pond when a guest ordered truite
au bleu --- the trout turned blue only when it was cooked alive.
``Everything came to the restaurant whole,'' he said. ``It was a little
medieval.''

After the apprenticeship came mandatory army service. Haeberlin offered
to help get him a job in the Élysée Palace, cooking for the president of
the republic. ``The three-star restaurants are like a mafia,'' Mr.
Vongerichten said. ``You never have to write a résumé.''

He turned the offer down, realizing he would do little except peel
shallots. He yearned to see the world, and asked to be stationed aboard
a boat. He was assigned to cook for the captain and three officers on
anti-submarine patrol. ``It was a sardine can, but it had a wonderful
wine cellar,'' he said. ``The only thing I learned was how to drink.''

He recited the ports of call: Hamburg, Copenhagen and Lisbon among them.
When the boat pulled into Casablanca, Morocco, he discovered cumin, and
prepared carrots with cumin for the officers, his first experience with
the spices that would later define his career. The seasonings of his
childhood had been meager: white and black pepper, and at Christmas,
mace, cinnamon and ginger.

``After this,'' he said, ``I cannot be in my village any longer. It is
not the same.''

He returned briefly to Auberge de l'Ill, but after three months he left
for the south of France, landing a position at L'Oasis, near Cannes,
where the chef
\href{https://www.nytimes.com/1986/12/24/garden/from-outhier-dishes-with-hint-of-orient.html}{Louis
Outhier} had received three Michelin stars.

``It was the opposite of Auberge de l'Ill,'' Mr. Vongerichten said.
``You couldn't prep anything. If you needed parsley, you chopped it at
that moment. Every dish, every sauce, à la minute. He didn't turn on the
stoves until 11:30 in the morning.''

Mr. Vongerichten loved the Riviera, the rosemary, the olives, the
markets. There he met the chef
\href{https://www.nytimes.com/2018/01/20/obituaries/paul-bocuse-dead.html}{Paul
Bocuse}, who predicted that Mr. Vongerichten would one day work for him.
He did, in 1979, but remained only nine months, unhappy with Bocuse's
traditional style of cooking.

Image

The chef Paul Bocuse, left, predicted that Mr. Vongerichten would one
day work for him. He did, but only for nine months.Credit...

Image

In Mr. Bocuse's kitchen in Lyon, France. Mr. Vongerichten is third from
the right.Credit...

``I felt I was going backward,'' Mr. Vongerichten said. ``Bocuse was not
happy when I left.'' The chef addressed him with an epithet, he said.
``It was his last words to me, but he was a little bit joking.''

Mr. Vongerichten was still in his early 20s, and restless. He left for
Munich, to work under the acclaimed Austrian chef
\href{https://www.myriadrestaurantgroup.com/news-item/batard-honors-chef-eckart-witzigmann/}{Eckart
Witzigmann}. Linguistically, the move was effortless; Mr. Vongerichten
had grown up speaking Alsatian, a dialect more German than French. The
most difficult adjustment was joining the staff for a beer after dinner
service.

``You know Germany,'' he said. ``The glasses are so large, after one
beer you are done.''

After six months, he got a phone call from Mr. Outhier that changed his
life: The chef wanted Mr. Vongerichten, then 23 and with no experience
running a kitchen, to be chef de cuisine of a restaurant he was opening
in the Oriental Hotel in Bangkok.

He broke the news to Mr. Witzigmann, expecting him to insist that he
stay. Instead, the chef told him he had no choice but to go.

Mr. Vongerichten said he wasn't ready, but Mr. Witzigmann promoted him
to sous-chef for a week. ``After two or three days, I am bossing
everyone around,'' Mr. Vongerichten said. ``I realize I can do this. I
take the job.''

Returning to L'Oasis for a brief refresher course, he cooked for Mr.
Outhier, perfecting the chef's recipes, taking copious notes. The boy
who had never paid attention in class had turned into a scribe. Today,
there are more than 50,000 recipes in Mr. Vongerichten's business
computer.

Image

Mr. Vongerichten working at L'Oasis, near Cannes, France.Credit...

Image

The chef there was Louis Outhier, who later sent him to
Bangkok.Credit...

On the flight to Bangkok in 1979, he was surprised to see another young
cook from L'Oasis accompanying him: Mr. Outhier had sent a backup, in
case one of them failed.

They came to like each other, and went out together after service. Mr.
Vongerichten would end his nights at 2 a.m., but his friend partied all
night. ``I couldn't keep up with this guy,'' he said. ``He went berserk.
He wasn't making it. He couldn't handle Bangkok.''

Mr. Vongerichten, on the other hand, always came home. ``Not always
alone,'' he said, ``but I was coming home.''

He loved the city, fascinated by all he saw. ``Everything was different
--- the people, the language, the religion, the food.'' But the tastes
he encountered were not permitted at the hotel, where his job was to
cook the French food of Mr. Outhier.

Mr. Vongerichten sent ingredients like bok choy and lemongrass to France
for the chef to try. ``I was doing duck with Armagnac in Bangkok while
Outhier was doing duck with spicy sesame sauce and bok choy on the
Riviera.''

Image

Mr. Vongerichten (sans toque) at Lafayette, a grand restaurant in the
Drake Hotel in New York, in 1988Credit...Ruby Washington/The New York
Times

Over the next decade, Mr. Vongerichten opened 10 restaurants for Mr.
Outhier, including ones in Singapore and Hong Kong, before moving to the
United States. There, he opened the restaurant Le Marquis de Lafayette
in Boston in 1985, and then Lafayette in New York in 1986.

He lived on the ground floor of the Drake Hotel, his salary \$35,000 a
year. He moved up vertically and financially with each additional star
that Lafayette received from The New York Times. After achieving
\href{https://www.nytimes.com/1988/04/22/arts/restaurants-067888.html}{four
stars in 1988}, he was living on the top floor and earning \$108,000 a
year, more than the hotel's general manager.

Mr. Vongerichten had met his first wife, Muriel Vongerichten, on the
Riviera, and later brought her to Bangkok, where they married. Not long
after settling in New York, she moved back to France with their two
young children, Cédric and Louise, and the couple later divorced.

Ms. Vongerichten did not want to talk about the marriage for this
article, Cédric Vongerichten said. She returns regularly to New York, he
said, and the couple's relationship is amicable.

``My mother has no grudges,'' said Cédric, currently the chef of
\href{https://www.nytimes.com/2019/04/23/dining/wayan-restaurant-review.html}{Wayan}
and three of his father's restaurants:
\href{https://www.nytimes.com/2005/09/07/dining/reviews/showmanship-yields-to-elegance.html}{Perry
St.} in New York and two in Jakarta, Indonesia. ``Anyway,'' he added,
``she has to come if she wants to see her grandkids.''

Image

Cédric Vongerichten at Wayan, his French-Indonesian restaurant in
NoLIta.Credit...Daniel Krieger for The New York Times

Jean-Georges had no interest in moving back to France. ``I felt New York
was the town where I could make it happen, and going back was like
starting from scratch,'' he said. ``So instead of working 10 or 12 hours
a day, I started working 14 hours a day. I was good for nothing but
cooking.''

That same year, 1990, he ran the New York City Marathon. As a birthday
present, a friend bought him a number and persuaded him to participate
by pointing out that they would start just behind the elite runners.

Mr. Vongerichten had never run, never worked out, never gone to a gym.
He purchased his shorts, shirt and running shoes the night before the
race. His rationale for believing he possessed the stamina to complete
it: ``I am standing 14 hours a day in my kitchen.''

``There are 19,000 people running,'' he recalled. ``I think 18,000
passed me.'' He ran 16 miles, then walked the rest of the way, urged on
by a woman who had been following him for miles and insisted he not give
up. They crossed the finish line together, and have remained friends.

Years later, he took his daughter Louise to Aspen, Colo., on a skiing
vacation. She was 10 or 11 years old and had never had a lesson. He took
her to the top of the mountain, and down they went, side by side.

``My dad wants people to go fast, learn, go forward,'' she said. ``He
didn't want to wait for me.''

He remained restless. After he left Lafayette in 1991, he and Mr. Suarez
opened a small restaurant on 64th Street they named
\href{https://www.jojorestaurantnyc.com/}{Jo Jo}, after his childhood
nickname. (More than one family member has said he was actually called
Jo Jo la Terreur --- the Terror.)

Jo Jo was
\href{https://www.nytimes.com/1991/07/12/arts/restaurants-737091.html}{a
sensation}. For two years, Mr. Vongerichten was in the kitchen every
day. But, he said, ``I was bored after three months. I thought, `O.K.,
what is next'?''

More opportunities came along, and he rarely said no. Among them were
his French-Asian restaurant
\href{https://www.nytimes.com/2006/08/16/dining/reviews/16rest.html}{Vong},
and
\href{https://www.nytimes.com/2004/03/24/dining/restaurants-fancy-street-food-but-what-a-street.html}{Spice
Market}, which he and Mr. Suarez announced in 2006 that they were
selling to Starwood Hotels \& Resorts, for an amount Mr. Vongerichten
would not disclose.

He learned from his failures. After creating a thriving steakhouse at
the Bellagio Hotel in Las Vegas, in 2004 he opened
\href{https://www.nytimes.com/2004/07/14/dining/restaurants-elaborate-dishes-assembly-required.html}{V
Steakhouse} in the Time Warner Center. It featured gold-leaf columns,
velvet chairs and rhubarb ketchup. From that mistake came his first
lesson: ``Don't try to reinvent the steakhouse. It's an American
staple.''

Another misstep was 66, an upscale Chinese restaurant he opened in
TriBeCa in 2003. When speaking of it, his voice rises in frustration,
unusual for him. He loved the food. But his fried rice with fresh crab
meat went for \$15, while four blocks away, in Chinatown, the same dish
with canned crab meat was \$3.50. Bad idea, he admits.

He reconceived the space, unsuccessfully, as a Japanese restaurant.
Another lesson: ``If I cannot do the food myself, don't do it. I do not
know how to use a wok. I cannot do sushi.''

Image

With his wife, Marja Vongerichten, in 2011.Credit...Everett Collection
Inc/Alamy Stock Photo

He remarried in 2004. His new wife, Marja Vongerichten, had been working
in the reservation office at Jean-Georges. She got the idea he might be
interested in her when he said to a manager: ``Why is she stuck down
here? Bring her upstairs.''

And what made her fall for him? ``His
\href{https://cooking.nytimes.com/recipes/1014719-molten-chocolate-cake}{molten
chocolate cake},'' she said.

They have a daughter, Chloe, and a house in Waccabuc, a hamlet in
northern Westchester County. She says efforts to get him to slow down
have shown promise. ``The first time I ever saw him veg out on a couch
was after we got the house,'' Marja said. ``He'll lay down with the
remote and watch mindless movies.''

He is not handy. ``He can't change light bulbs,'' she said, but he has
demonstrated inordinate interest in their leaf blower. ``It can be
raining and he's out with the leaf blower. He sees any leaf on the
property, he blows it to oblivion.''

Finding a connection between the rebellious child he once was and the
take-charge adult he is today isn't easily done.

His daughter Louise has an example that illustrates both: Her father
brought the family, about a dozen in all, to the Cannes Film Festival in
2016. He learned to his dismay that none of them, himself included, had
been invited to a gala at the Hotel du Cap-Eden-Roc.

Since Mr. Vongerichten was staying there, he came up with a plan: He led
them all on a serpentine route through back corridors to the hotel
kitchen, where they grabbed trays of canapés and marched into the
ballroom, posing as waiters.

``He was very proud of himself,'' she said. ``He had done something a
little illegal.''

Image

Retirement, Mr. Vongerichten said, ``sounds like a
disease.''Credit...Sasha Maslov for The New York Times

Mr. Vongerichten's best friend, Hervé Descottes, believes that the
chef's story is a classic tale of an ugly duckling emerging as a swan.
The president of Mr. Vongerichten's company, Lois Freedman, credits the
chef's rise in part to his mysterious ability to dream about food. ``He
sees flavors in his mind,'' she said. ``He told me that.''

Asked about the reason for his metamorphosis, Mr. Vongerichten replied,
``I have no idea.'' Then he suggested possibilities: fine mentoring,
simple good fortune, a miracle. None was quite right.

Finally, he smiled, having found an answer that satisfied him: ``I
always broke the rules.''

\emph{Follow} \href{https://twitter.com/nytfood}{\emph{NYT Food on
Twitter}} \emph{and}
\href{https://www.instagram.com/nytcooking/}{\emph{NYT Cooking on
Instagram}}\emph{,}
\href{https://www.facebook.com/nytcooking/}{\emph{Facebook}}\emph{,}
\href{https://www.youtube.com/nytcooking}{\emph{YouTube}} \emph{and}
\href{https://www.pinterest.com/nytcooking/}{\emph{Pinterest}}\emph{.}
\href{https://www.nytimes.com/newsletters/cooking}{\emph{Get regular
updates from NYT Cooking, with recipe suggestions, cooking tips and
shopping advice}}\emph{.}

Advertisement

\protect\hyperlink{after-bottom}{Continue reading the main story}

\hypertarget{site-index}{%
\subsection{Site Index}\label{site-index}}

\hypertarget{site-information-navigation}{%
\subsection{Site Information
Navigation}\label{site-information-navigation}}

\begin{itemize}
\tightlist
\item
  \href{https://help.nytimes.com/hc/en-us/articles/115014792127-Copyright-notice}{©~2020~The
  New York Times Company}
\end{itemize}

\begin{itemize}
\tightlist
\item
  \href{https://www.nytco.com/}{NYTCo}
\item
  \href{https://help.nytimes.com/hc/en-us/articles/115015385887-Contact-Us}{Contact
  Us}
\item
  \href{https://www.nytco.com/careers/}{Work with us}
\item
  \href{https://nytmediakit.com/}{Advertise}
\item
  \href{http://www.tbrandstudio.com/}{T Brand Studio}
\item
  \href{https://www.nytimes.com/privacy/cookie-policy\#how-do-i-manage-trackers}{Your
  Ad Choices}
\item
  \href{https://www.nytimes.com/privacy}{Privacy}
\item
  \href{https://help.nytimes.com/hc/en-us/articles/115014893428-Terms-of-service}{Terms
  of Service}
\item
  \href{https://help.nytimes.com/hc/en-us/articles/115014893968-Terms-of-sale}{Terms
  of Sale}
\item
  \href{https://spiderbites.nytimes.com}{Site Map}
\item
  \href{https://help.nytimes.com/hc/en-us}{Help}
\item
  \href{https://www.nytimes.com/subscription?campaignId=37WXW}{Subscriptions}
\end{itemize}
