Sections

SEARCH

\protect\hyperlink{site-content}{Skip to
content}\protect\hyperlink{site-index}{Skip to site index}

\href{https://www.nytimes.com/section/books}{Books}

\href{https://myaccount.nytimes.com/auth/login?response_type=cookie\&client_id=vi}{}

\href{https://www.nytimes.com/section/todayspaper}{Today's Paper}

\href{/section/books}{Books}\textbar{}Roger Scruton, a Provocative
Public Intellectual, Dies at 75

\url{https://nyti.ms/2NxxIwC}

\begin{itemize}
\item
\item
\item
\item
\item
\end{itemize}

Advertisement

\protect\hyperlink{after-top}{Continue reading the main story}

Supported by

\protect\hyperlink{after-sponsor}{Continue reading the main story}

\hypertarget{roger-scruton-a-provocative-public-intellectual-dies-at-75}{%
\section{Roger Scruton, a Provocative Public Intellectual, Dies at
75}\label{roger-scruton-a-provocative-public-intellectual-dies-at-75}}

A philosopher, author and columnist, he was an outspoken hero to
conservatives in Britain and recently at the center of, in his words, a
``hate storm.''

\includegraphics{https://static01.nyt.com/images/2020/01/18/obituaries/18Scruton-obit1/merlin_102917293_6abc2cf9-7765-476b-806a-66e02386d42b-articleLarge.jpg?quality=75\&auto=webp\&disable=upscale}

By \href{https://www.nytimes.com/by/alan-cowell}{Alan Cowell}

\begin{itemize}
\item
  Published Jan. 16, 2020Updated Jan. 17, 2020
\item
  \begin{itemize}
  \item
  \item
  \item
  \item
  \item
  \end{itemize}
\end{itemize}

LONDON --- Roger Scruton, a prominent British philosopher and public
intellectual whose espousal of conservative causes and contentious views
elicited both plaudits and opprobrium, which he likened to ``falling to
the bottom in my own country,'' died on Sunday. He was 75.

His family announced the death \href{https://www.roger-scruton.com/}{on
his website} without providing other details. Mr. Scruton, who lived for
many years on a farm in Wiltshire, in southwest England, was said to
have been treated for lung cancer in recent months.

In the course of a long academic career, which included spells in the
United States, Mr. Scruton wrote more than 50 books, ranging over topics
like art, aesthetics, architecture, music, philosophy and sexual
behavior. On the defining issue of the new century in Britain, he said,
he voted in favor of leaving the European Union, the so-called Brexit
that propelled the Conservative Party's landslide victory in elections
in December.

He also wrote four novels in addition to newspaper and magazine columns,
in which he mused on wine, politics and horseback hunting, which he
pursued enthusiastically until his final birthday. As a musician, he
composed operas. He qualified as a barrister, too, but did not practice
law.

In the Cold War years of the late 1970s and '80s, he transcended the
frontiers of formal Western academia by traveling beyond the Iron
Curtain --- to Poland, Hungary and Czechoslovakia --- to deliver
clandestine lectures and smuggle samizdat works disguised as blank CDs
to Soviet bloc students. In later years he was awarded medals in
recognition of that role.

He was knighted in Britain in 2016. After his death, Prime Minister
\href{https://twitter.com/BorisJohnson/status/1216674269721219072?ref_src=twsrc\%5Etfw\%7Ctwcamp\%5Etweetembed\%7Ctwterm\%5E1216674269721219072\&ref_url=https\%3A\%2F\%2Fwww.bbc.com\%2Fnews\%2Fuk-51084248}{Boris
Johnson tweeted}, ``We have lost the greatest modern conservative
thinker --- who not only had the guts to say what he thought but said it
beautifully.''

Toward the end of his life, Mr. Scruton concluded that he had been
treated unfairly in his own land, subjected to what he termed a ``hate
storm'' inspired by critics who had accused him of Islamophobia,
anti-Semitism and disparagement of Chinese people --- allegations that
Mr. Scruton called ``fantastic and fabricated.''

The immediate cause of the furor was
\href{https://www.newstatesman.com/politics/uk/2019/04/roger-scruton-interview-full-transcript}{an
article about him} in April in the left-wing magazine New Statesman.
Based on an interview with him, the article, which a New Statesman
editor said on social media contained ``a series of outrageous
remarks,'' prompted an uproar. Mr. Scruton was said to have belittled
the term Islamophobia, spoken stereotypically of Chinese people and
evoked a ``Soros empire in Hungary,'' referring to the financier George
Soros, who is Jewish.

Within hours of its publication Mr. Scruton was sacked from an
unsalaried position he had held as the head of a government-appointed
body that advised on modern architecture,
\href{https://www.gov.uk/government/groups/building-better-building-beautiful-commission}{the
Building Better, Building Beautiful Commission}.

But he was reappointed after the magazine acknowledged that his views
``were not accurately represented in the tweets'' that had been
published along with the article. The magazine apologized.

The episode recalled Mr. Scruton's longstanding reputation as an
iconoclast. Peter Stothard, who had been his editor at The Times of
London in the 1980s, when Mr. Scruton wrote a column for the paper on
art and politics, was quoted as saying that ``there was no one I ever
commissioned to write whose articles provoked more rage'' than Mr.
Scruton's.

Critics also assailed his views on homosexuality and gender issues.
\href{https://www.newstatesman.com/politics/uk/2019/04/roger-scruton-interview-full-transcript}{In
his interview} with New Statesman, he said that homosexuality was
``different'' but denied that he was homophobic. He described the
21st-century debate on gender and identity as ``a kind of theatrical
obsession which is being imposed on children whether or not they
understand it.''

Mr. Scruton dated his conversion to the conservative cause to the Paris
student riots of 1968, when, at 24, he observed young people, including
his friends, clashing with the police in the Latin Quarter. ``What I saw
was an unruly mob of self-indulgent middle-class hooligans,'' he said in
an
\href{https://www.theguardian.com/books/2000/oct/28/politics}{interview
with The Guardian} in 2000.

``When I asked my friends what they wanted, what were they trying to
achieve, all I got back was this ludicrous Marxist gobbledygook,'' he
continued. ``I was disgusted by it, and thought there must be a way back
to the defense of western civilization against these things. That's when
I became a conservative. I knew I wanted to conserve things rather than
pull them down.''

Roger Vernon Scruton was born in Buslingthorpe, a village in
Lincolnshire, in eastern England, on Feb. 7, 1944, the son of John and
Beryl (Claris) Scruton. His father was a teacher, his mother a
homemaker. The couple also had two daughters.

Roger was educated at a grammar school in High Wycombe, West London, and
won a scholarship to Jesus College at Cambridge University, where he
studied philosophy. He met his future first wife, Danielle Laffitte, a
teacher, while traveling in France. They married in 1973, the same year
he was awarded his doctorate. They divorced in 1979.

From 1971 to 1992 he taught at Birkbeck College in London, where, he
said, he was the only conservative on the teaching staff.

In later years he was sometimes depicted as providing the intellectual
spine to Thatcherism in Britain, although he said he did not share Prime
Minister Margaret Thatcher's devotion to the free market.

In 1982, Mr. Scruton helped found a conservative journal, The Salisbury
Review, which stirred controversy in 1984 by publishing an article by a
headmaster in the north of England who raised questions about the value
of multicultural education.

Image

Mr. Scruton published a torrent of books on a range of subjects,
including ``The Aesthetics of Architecture'' (1979).Credit...Princeton
University Press

Mr. Scruton published a torrent of books, including ``Art and
Imagination: A Study in the Philosophy of the Mind'' (1974), ``The
Aesthetics of Architecture'' (1979) and ``Sexual Desire'' (1986). His
novels included ``Notes From Underground'' (2014), based on his
experiences behind the Iron Curtain.

In 1992 he became a professor of philosophy at Boston University; he
returned to Britain in 1995. In 1996 he married Sophie Jeffreys, an
architectural historian, with whom he had two children, Sam and Lucy.
They all survive him.

The episode revolving around the New Statesman article, in the last year
of his life, left Mr. Scruton feeling bruised.

In a column in the conservative magazine Spectator, under the headline
``Roger Scruton: My 2019,'' he wrote, ``During this year much was taken
from me --- my reputation, my standing as a public intellectual, my
position in the Conservative movement, my peace of mind, my health.''

But, he went on, ``Falling to the bottom in my own country, I have been
raised to the top elsewhere, and looking back over the sequence of
events I can only be glad that I have lived long enough to see this
happen.''

``Coming close to death you begin to know what life means,'' he added,
``and what it means is gratitude.''

Advertisement

\protect\hyperlink{after-bottom}{Continue reading the main story}

\hypertarget{site-index}{%
\subsection{Site Index}\label{site-index}}

\hypertarget{site-information-navigation}{%
\subsection{Site Information
Navigation}\label{site-information-navigation}}

\begin{itemize}
\tightlist
\item
  \href{https://help.nytimes.com/hc/en-us/articles/115014792127-Copyright-notice}{©~2020~The
  New York Times Company}
\end{itemize}

\begin{itemize}
\tightlist
\item
  \href{https://www.nytco.com/}{NYTCo}
\item
  \href{https://help.nytimes.com/hc/en-us/articles/115015385887-Contact-Us}{Contact
  Us}
\item
  \href{https://www.nytco.com/careers/}{Work with us}
\item
  \href{https://nytmediakit.com/}{Advertise}
\item
  \href{http://www.tbrandstudio.com/}{T Brand Studio}
\item
  \href{https://www.nytimes.com/privacy/cookie-policy\#how-do-i-manage-trackers}{Your
  Ad Choices}
\item
  \href{https://www.nytimes.com/privacy}{Privacy}
\item
  \href{https://help.nytimes.com/hc/en-us/articles/115014893428-Terms-of-service}{Terms
  of Service}
\item
  \href{https://help.nytimes.com/hc/en-us/articles/115014893968-Terms-of-sale}{Terms
  of Sale}
\item
  \href{https://spiderbites.nytimes.com}{Site Map}
\item
  \href{https://help.nytimes.com/hc/en-us}{Help}
\item
  \href{https://www.nytimes.com/subscription?campaignId=37WXW}{Subscriptions}
\end{itemize}
