Sections

SEARCH

\protect\hyperlink{site-content}{Skip to
content}\protect\hyperlink{site-index}{Skip to site index}

\href{https://www.nytimes.com/section/world/europe}{Europe}

\href{https://myaccount.nytimes.com/auth/login?response_type=cookie\&client_id=vi}{}

\href{https://www.nytimes.com/section/todayspaper}{Today's Paper}

\href{/section/world/europe}{Europe}\textbar{}In Huawei Battle, China
Threatens Germany `Where It Hurts': Automakers

\url{https://nyti.ms/3ab5wJD}

\begin{itemize}
\item
\item
\item
\item
\item
\item
\end{itemize}

Advertisement

\protect\hyperlink{after-top}{Continue reading the main story}

Supported by

\protect\hyperlink{after-sponsor}{Continue reading the main story}

\hypertarget{in-huawei-battle-china-threatens-germany-where-it-hurts-automakers}{%
\section{In Huawei Battle, China Threatens Germany `Where It Hurts':
Automakers}\label{in-huawei-battle-china-threatens-germany-where-it-hurts-automakers}}

VW, Daimler and BMW sell more cars in China than anywhere else and many
already cooperate with Huawei --- a dependency Beijing is not shy to
exploit.

\includegraphics{https://static01.nyt.com/images/2019/12/19/world/germany-china1/merlin_141033957_5ae92733-07b5-4088-b715-8b3f3ae09d08-articleLarge.jpg?quality=75\&auto=webp\&disable=upscale}

\href{https://www.nytimes.com/by/katrin-bennhold}{\includegraphics{https://static01.nyt.com/images/2018/07/13/multimedia/author-katrin-bennhold/author-katrin-bennhold-thumbLarge.png}}\href{https://www.nytimes.com/by/jack-ewing}{\includegraphics{https://static01.nyt.com/images/2018/07/18/multimedia/author-jack-ewing/author-jack-ewing-thumbLarge.png}}

By \href{https://www.nytimes.com/by/katrin-bennhold}{Katrin Bennhold}
and \href{https://www.nytimes.com/by/jack-ewing}{Jack Ewing}

\begin{itemize}
\item
  Jan. 16, 2020
\item
  \begin{itemize}
  \item
  \item
  \item
  \item
  \item
  \item
  \end{itemize}
\end{itemize}

\href{https://cn.nytimes.com/world/20200117/huawei-germany-china-5g-automakers/}{阅读简体中文版}\href{https://cn.nytimes.com/world/20200117/huawei-germany-china-5g-automakers/zh-hant/}{閱讀繁體中文版}

BERLIN --- Chancellor Angela Merkel of Germany and Premier Li Keqiang of
China settled into the back seat of a
\href{https://www.instagram.com/p/BlDUI7QgqUR/}{driverless Volkswagen
van}, fastened their seatbelts and went for a spin around a disused
airport landing strip in central Berlin.

``There is nothing like seeing in practice what's possible,'' Ms. Merkel
beamed when they returned.

That was July 2018, when economic cooperation between the two countries
looked limitless --- combining Germany's powerful auto industry and
China's technology giant, Huawei.

Eighteen months later, Germany is embroiled in a tortured debate over
whether to allow Huawei to help build its 5G next generation mobile
network. But with German automakers, including Audi and Daimler, already
working closely with Huawei, it may be China who sits in the driver's
seat.

Whatever Germany decides will shape its relations with China for years
and reverberate across the Continent. It will send a powerful political
signal on how united, or fractured, Europe will be in the digital age of
rivalry between Washington and Beijing.

Germany, like all of Europe, is under tremendous pressure to ostracize
Huawei by the American government, which fears that the Chinese company
is a Trojan horse that would allow China's government to spy on or
control European and American communication networks. The pressure
remains even after President Trump signed an initial trade deal with
China on Wednesday.

``The West should have a joint solution to 5G because we view the world
the same way,'' Richard Grenell, the United States ambassador to
Germany, said on Thursday in an email.

But for Germany that decision is especially fraught. Relations with the
Trump administration are infused with threats of tariffs against German
automakers and mounting distrust that Europeans have come to believe may
permanently reshape, if not rupture, a once ironclad trans-Atlantic
alliance.

China, on the other hand, is elbowing its way onto the European stage as
a new strategic player and an increasingly indispensable economic
partner. By far the largest market in the world, it has become the
biggest source of growth for Germany's main carmakers and the key to
their dominance of the luxury car market.

It is a position that China has not been shy to weaponize.

``If Germany were to make a decision that led to Huawei's exclusion from
the German market, there will be consequences,'' Wu Ken, China's
ambassador to Germany warned last month. ``The Chinese government will
not stand idly by.''

Konstantin von Notz, a lawmaker and member of the digital affairs
committee in the German Parliament, put it this way: ``The Chinese have
made clear that they will retaliate where it hurts: The car industry.''

For months, German lawmakers have danced around the issue of whether to
effectively exclude Huawei from the bidding process. The issue is
expected to be debated in Parliament again in the coming weeks. As a
decision approaches, Ms. Merkel has found herself caught between worried
German automakers, who accompanied her on a dozen junkets to Beijing,
and her own wary intelligence community.

\includegraphics{https://static01.nyt.com/images/2019/12/19/world/germany-china2/merlin_163483965_9cef9a60-3880-4622-a1ed-180882bd6603-articleLarge.jpg?quality=75\&auto=webp\&disable=upscale}

Ms. Merkel, steward of the pro-business Christian Democratic Party, met
on Thursday with lawmakers from her party and urged them to find a
solution that would settle the dispute.

Ms. Merkel is opposed to banning the Chinese company.

``It is not about individual companies, but rather security standards,''
the chancellor said in November. ``It is about the certification we will
carry out. That should be our guiding benchmark.''

But a rebellion is brewing in Germany's foreign policy and intelligence
community --- scared of American threats to limit intelligence sharing
--- and even among some of the chancellor's own lawmakers, who want to
submit a proposal to Parliament with tougher security criteria that
would, in effect, keep Huawei out.

Ms. Merkel's critics say the current certification process, which merely
demands that companies sign a pledge not to spy, is inherently flawed
because it relies on trust.

At her party's annual conference in November, the chancellor's Christian
Democrats disinvited Huawei as a corporate sponsor and passed a motion
demanding that only companies ``which demonstrably fulfill a clearly
defined catalog of safety requirements'' should be allowed to bid. One
key requirement would be to rule out state interference.

The motion did not name Huawei or China but the implication was clear.

``Under Chinese law companies are obliged to cooperate with the Chinese
Secret Service,'' said Norbert Röttgen, a conservative lawmaker who
co-wrote the motion against Ms. Merkel's Huawei policy. ``When you deal
with Huawei you also have to accept that you might be dealing with the
Chinese Communist Party.''

Cars that can steer themselves may make driving safer but they also open
up opportunities for government surveillance and control.

``Car companies gather loads of personal data from the drivers of their
cars, and they face an enormous risk of an angry public outraged to find
their data used by the Chinese Communist Party,'' said Mr. Grenell, the
United States ambassador.

Beyond fears of spying and sabotage, lawmakers warned that if Germany
allowed Huawei to bid it would not just alienate Washington but risk
undermining a badly needed united European front.

``Our only hope is to stick together as Europeans,'' Mr. Röttgen said.
That, he said, was also an argument for giving the 5G contract to
European companies like Nokia or Ericsson.

Analysts say Nokia and Ericsson, which have won 5G contracts in Denmark
and elsewhere, have the competence to build the 5G network, but it would
take longer and cost more --- not least because Huawei is already a huge
part of the existing networks in Germany. Switching will be messy and
costly.

Still, Mr. Röttgen said, given the scale of the new bid, if it went to
Huawei, Europe risked permanently falling behind.

``If you let Huawei build a big chunk of the 5G network after a while
you won't understand your own system,'' he said. ``It would be a maximal
loss of control and sovereignty.''

``Strategically it is a crystal clear case,'' Mr. Röttgen said.

Others, however, say that giving the bid to Huawei may not be such a bad
idea.

``If we ban Huawei, the German car industry will be pushed out of the
Chinese market --- and this in a situation where the American president
is also threatening to punish German carmakers,'' said Sigmar Gabriel, a
former German foreign minister and vice chancellor.

``Just because we have an American president who doesn't like alliances,
we give all that up?" he said. ``Why would we? Especially since he does
exactly what the Chinese do and threatens the German car industry.''

Image

German automakers like Audi already work closely with
Huawei.Credit...Lukas Barth/EPA, via Shutterstock

German automakers like Volkswagen, Daimler and BMW have continued to
record sales gains in China and to take share from rivals like Ford,
even as the overall market has slumped.

``See, last year, 28 million cars were sold in China, 7 million of those
were German,'' Mr. Wu, China's ambassador to Germany, added in his
remarks in December, making what many in Germany interpreted as a veiled
threat.

As Germany's automakers have become more deeply dependent on China, they
also have become more beholden to the Chinese government.

Chinese consumer preferences, and Chinese government policies,
increasingly determine what models the carmakers build and what kind of
technology they develop.

China also has become the stage where German carmakers develop and test
new technology, often with Huawei.

Audi, the luxury car unit of Volkswagen, announced a
``\href{https://www.audi-mediacenter.com/de/pressemitteilungen/audi-und-huawei-unterzeichnen-absichtserklaerung-zur-strategischen-kooperation-10427}{strategic
cooperation}'' with Huawei on developing autonomous driving technology
during Mr. Li's visit to Berlin last year. Daimler, 9.9 percent owned by
Chinese investor Li Shufu, uses Huawei high-performance computing. BMW
and others partner with Huawei on research and development.

No car company is more closely entwined with China than Volkswagen. The
company has been operating in China since the early 1980s, when the
Communist government began opening to the West.

Today Volkswagen earns almost half its sales revenue in China and has 14
percent of the Chinese car market.

``If we were to pull out'' of China, Herbert Diess, the chief executive
of Volkswagen, told the Wolfsburger Nachrichten newspaper in December,
``a day later 10,000 of our 20,000 development engineers in Germany
would be out of work.''

German carmakers deny that their dependence on Chinese sales has turned
them into advocates of Chinese interests.

``We don't want political developments to spill over into product
development,'' Bernhard Mattes, president of the German Association of
the Automotive Industry, said in an interview in Berlin.

But Mr. Mattes conceded, ``We are not operating in a politics-free
space, that is clear.''

Image

President Xi Jinping of China with Ms. Merkel at a G20 summit meeting in
2017. Ms. Merkel, steward of the pro-business Christian Democratic
Party, is opposed to banning Huawei.Credit...Wolfgang Rattay/Reuters

Huawei has understood as much. Its German headquarters are in Bavaria,
alongside BMW and Audi and many other companies deeply embedded in
China. The company has been a generous sponsor of all mainstream
parties, including Bavaria's governing conservatives.

Markus Söder, Bavaria's conservative leader, has publicly defended
Huawei's right to bid, while also lashing out at the United States.

``To say up front that I rule it out because another partner in the
world doesn't like it,'' he said, is ``a bit of a problem.''

Christopher F. Schuetze contributed reporting.

Advertisement

\protect\hyperlink{after-bottom}{Continue reading the main story}

\hypertarget{site-index}{%
\subsection{Site Index}\label{site-index}}

\hypertarget{site-information-navigation}{%
\subsection{Site Information
Navigation}\label{site-information-navigation}}

\begin{itemize}
\tightlist
\item
  \href{https://help.nytimes.com/hc/en-us/articles/115014792127-Copyright-notice}{©~2020~The
  New York Times Company}
\end{itemize}

\begin{itemize}
\tightlist
\item
  \href{https://www.nytco.com/}{NYTCo}
\item
  \href{https://help.nytimes.com/hc/en-us/articles/115015385887-Contact-Us}{Contact
  Us}
\item
  \href{https://www.nytco.com/careers/}{Work with us}
\item
  \href{https://nytmediakit.com/}{Advertise}
\item
  \href{http://www.tbrandstudio.com/}{T Brand Studio}
\item
  \href{https://www.nytimes.com/privacy/cookie-policy\#how-do-i-manage-trackers}{Your
  Ad Choices}
\item
  \href{https://www.nytimes.com/privacy}{Privacy}
\item
  \href{https://help.nytimes.com/hc/en-us/articles/115014893428-Terms-of-service}{Terms
  of Service}
\item
  \href{https://help.nytimes.com/hc/en-us/articles/115014893968-Terms-of-sale}{Terms
  of Sale}
\item
  \href{https://spiderbites.nytimes.com}{Site Map}
\item
  \href{https://help.nytimes.com/hc/en-us}{Help}
\item
  \href{https://www.nytimes.com/subscription?campaignId=37WXW}{Subscriptions}
\end{itemize}
