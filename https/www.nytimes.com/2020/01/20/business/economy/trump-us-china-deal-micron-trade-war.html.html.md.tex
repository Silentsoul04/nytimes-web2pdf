Sections

SEARCH

\protect\hyperlink{site-content}{Skip to
content}\protect\hyperlink{site-index}{Skip to site index}

\href{https://www.nytimes.com/section/business/economy}{Economy}

\href{https://myaccount.nytimes.com/auth/login?response_type=cookie\&client_id=vi}{}

\href{https://www.nytimes.com/section/todayspaper}{Today's Paper}

\href{/section/business/economy}{Economy}\textbar{}Trump's China Deal
Creates Collateral Damage for Tech Firms

\url{https://nyti.ms/2G5x6tZ}

\begin{itemize}
\item
\item
\item
\item
\item
\end{itemize}

Advertisement

\protect\hyperlink{after-top}{Continue reading the main story}

Supported by

\protect\hyperlink{after-sponsor}{Continue reading the main story}

\hypertarget{trumps-china-deal-creates-collateral-damage-for-tech-firms}{%
\section{Trump's China Deal Creates Collateral Damage for Tech
Firms}\label{trumps-china-deal-creates-collateral-damage-for-tech-firms}}

Micron secured some gains from the China deal but it may end up
suffering bigger losses from the broader U.S.-China battle.

\includegraphics{https://static01.nyt.com/images/2020/01/17/business/17DC-CHINACHIPS-lede-sub/merlin_167187393_2bb960c4-12d3-4e04-9e73-acd681726771-articleLarge.jpg?quality=75\&auto=webp\&disable=upscale}

\href{https://www.nytimes.com/by/ana-swanson}{\includegraphics{https://static01.nyt.com/images/2018/12/10/multimedia/author-ana-swanson/author-ana-swanson-thumbLarge.png}}\href{https://www.nytimes.com/by/cecilia-kang}{\includegraphics{https://static01.nyt.com/images/2019/01/29/multimedia/author-cecilia-kang/author-cecilia-kang-thumbLarge.png}}

By \href{https://www.nytimes.com/by/ana-swanson}{Ana Swanson} and
\href{https://www.nytimes.com/by/cecilia-kang}{Cecilia Kang}

\begin{itemize}
\item
  Jan. 20, 2020
\item
  \begin{itemize}
  \item
  \item
  \item
  \item
  \item
  \end{itemize}
\end{itemize}

WASHINGTON --- Among the corporate titans recognized last week by
President Trump during a White House signing ceremony for his China
trade deal was Sanjay Mehrotra, the chief executive of Micron
Technology, whose Idaho semiconductor company is at the heart of Mr.
Trump's trade war.

Micron, which makes memory chips for computers and smartphones, is
precisely the kind of advanced technology company that the Trump
administration views as crucial to maintaining a competitive edge over
China. After Micron rebuffed
\href{https://www.nytimes.com/2015/07/15/business/international/micron-technology-is-said-to-be-takeover-target-of-chinese-company.html}{a
2015 takeover attempt by a Chinese state-owned company}, it watched with
disbelief as its innovations were
\href{https://www.nytimes.com/2018/06/22/technology/china-micron-chips-theft.html}{stolen
and copied by a Chinese competitor} and its business was blocked from
China.

China's treatment of American companies like Micron fed Mr. Trump's
decision to unleash a punishing trade war with the world's
second-largest economy, a fight he said would halt Beijing's use of
unfair practices to undermine the United States. But that two-year
conflagration may wind up being more damaging to American technology
companies.

The initial trade deal announced last week should make operating in
China easier for companies like Micron. The deal contains provisions
meant to protect American technology and trade secrets and allow
companies to challenge China on accusations of theft, including older
cases like Micron's that precede the agreement.

But Mr. Trump's aggressive trade approach has also accelerated a
technology arms race between the two countries, putting American
companies like Micron at risk as the two nations try to
\href{https://www.nytimes.com/2019/05/14/business/us-china-tariffs.html}{decouple
their economies}. In an effort to reduce its reliance on American
components, China has expedited efforts to produce its own
semiconductors, driverless cars, artificial intelligence and other
technologies. Those efforts, along with the Trump administration's
\href{https://www.nytimes.com/2019/10/23/business/trump-technology-china-trade.html}{desire
to restrict the sales of American tech products} to China, could hurt
the very companies Mr. Trump set out to protect.

``Let's be clear, the trade war has been very bad for the semiconductor
industry in several ways,'' said Robert D. Atkinson, president of the
Information Technology and Innovation Foundation, a think tank funded by
the tech industry. ``It's like China woke up and said, `We've relied too
much on the United States.'''

The trade deal does nothing to curtail China's use of subsidies,
industrial plans and state-owned companies, which have helped it build
formidable industries in steel, wind turbines and solar panels. Those
state-directed efforts, which put many American manufacturers out of
business, are now being harnessed for high-tech industries.

The Trump administration is constructing its own walls around American
technology, reducing access to the lucrative Chinese market out of
security concerns. It is restricting exports of sensitive technologies,
\href{https://www.nytimes.com/2019/10/07/us/politics/us-to-blacklist-28-chinese-entities-over-abuses-in-xinjiang.html}{barring
sales to certain Chinese companies} and
\href{https://www.nytimes.com/2019/07/21/us/politics/china-investment-trade-war.html}{blocking
Chinese entities from investing} in the United States.

The administration is considering further restricting sales to Huawei,
the Chinese telecom company that relies on components from Micron and
other American suppliers. And the China trade deal
\href{https://www.nytimes.com/2019/12/15/business/economy/us-china-trade-deal.html}{leaves
tariffs on more than \$360 billion} in Chinese goods in place as Mr.
Trump tries to push American companies to bring manufacturing back home.

Semiconductor sales to China, which represent more than half the global
chip demand, have fallen, and semiconductor stocks have been
\href{https://www.nytimes.com/2019/05/23/business/dealbook/semiconductor-stocks-trade-war.html}{whipsawed
by the trade war}.

\includegraphics{https://static01.nyt.com/images/2020/01/17/business/17DC-CHINACHIPS-02/17DC-CHINACHIPS-02-articleLarge.jpg?quality=75\&auto=webp\&disable=upscale}

Mr. Trump and his supporters say that conflict is no longer avoidable,
and that the president's unconventional approach is necessary to take on
a growing threat from China. Officials across the administration look
with suspicion on Chinese industrial plans, including Made in China
2025, which
\href{https://www.nytimes.com/2017/03/07/business/china-trade-manufacturing-europe.html}{called
for \$300 billion in financing and other support} for 10 advanced
industries, including semiconductors.

American officials worry that gaining an advantage in semiconductors
would give China both a commercial and military edge.

Chips, which serve as the tiny sensors, brains and memories of all
high-tech devices, are crucial to next-generation telecom networks,
supercomputers, artificial intelligence and driverless cars, as well as
military ships, satellites and aircraft. They are also one of the United
States' largest exports, along with airplanes, oil and cars.

While China's ability to make chips is still far behind the United
States', the Chinese government, its state-owned enterprises, and
provincial and private equity funds have been pumping billions of
dollars into the industry, particularly the kind of memory chips that
Micron makes. In areas where Chinese companies cannot develop or buy
technology, companies say, some will simply steal their intellectual
property.

For the Trump administration, which was looking for a fight with China,
Micron's story proved a formative one. As officials prepared an
\href{https://ustr.gov/issue-areas/enforcement/section-301-investigations/section-301-china/investigation}{investigation}
into Chinese intellectual property theft that would ultimately spiral
into the trade war, Micron provided a ``camera ready'' case that fit
everything the administration was looking for, one industry executive
said.

In 2015, Micron was the target of
\href{https://www.nytimes.com/2015/07/15/business/international/micron-technology-is-said-to-be-takeover-target-of-chinese-company.html}{a
\$23 billion takeover attempt} by a Chinese state-owned company, but the
overture was withdrawn over United States national security concerns. In
2016, another Chinese state-owned company, Fujian Jinhua Integrated
Circuit,\href{https://www.nytimes.com/2018/06/22/technology/china-micron-chips-theft.html}{allegedly
worked in concert} with a Taiwanese company to steal the American
company's designs and market them as their own.

According to Taiwanese authorities, Fujian Jinhua used Micron's
proprietary designs to build an enormous \$5.7 billion microchip factory
in China. In 2018, the Department of Justice
\href{https://www.reuters.com/article/us-usa-justice-china-espionage/u-s-indicts-chinese-taiwan-firms-for-targeting-micron-trade-secrets-idUSKCN1N65R2}{charged
the Chinese company and others} with stealing trade secrets from Micron,
and the Commerce Department
\href{https://www.nytimes.com/2018/10/29/us/politics/fujian-jinhua-china-sales.html}{blacklisted
it} for national security concerns.

The same year, a Chinese court
\href{https://www.reuters.com/article/us-micron-tech-stocks/china-court-bans-micron-chip-sales-in-patent-case-taiwans-umc-idUSKBN1JT2DL}{temporarily
blocked} Micron from selling some products in China, after Fujian Jinhua
and another company accused Micron of patent infringement.

Image

President Trump has pointed to accusations of technology theft from
Micron as a rationale for the trade war with China.Credit...Aaron Wojack
for The New York Times

Through 2017 and 2018, Micron employees met repeatedly with
administration officials, sometimes with the National Security Council
and National Economic Council. The company's case was discussed in
internal planning meetings attended by Robert Lighthizer, the United
States trade representative, and Peter Navarro, a top Trump trade
adviser.

In July of last year, Mr. Trump met at the White House with Mr. Mehrotra
of Micron, as well as the chiefs of Intel, Google and Broadcom, to
discuss the trade clash with China and the administration's policies
toward Huawei.

Two months later, in an address to the United Nations, Mr. Trump
described the Micron theft as a rationale for the trade war.

``To advance the Chinese government's five-year economic plan, a company
owned by the Chinese state allegedly stole Micron's designs, valued at
up to \$8.7 billion,'' the president said. ``Soon, the Chinese company
obtains patents for nearly an identical product, and Micron was banned
from selling its own goods in China. But we are seeking justice.''

``For years, these abuses were tolerated, ignored or even encouraged,''
Mr. Trump added. ``But as far as America is concerned, those days are
over.''

Chip makers initially supported the Trump administration's willingness
to take on China. Companies had long grumbled about intellectual
property theft and unfair treatment in the Chinese market, but they had
little recourse: Going public about their troubles could spook investors
and invite Chinese retaliation.

Then, in April 2018, the administration
\href{https://www.nytimes.com/2018/04/03/us/politics/white-house-chinese-imports-tariffs.html}{announced
\$50 billion in tariffs} that would directly hit semiconductor companies
by raising prices for imported equipment and materials. A chip finished
in China would be subject to a 25 percent tariff, even if its components
had been made in America.

The tariffs caught the industry by surprise. The Semiconductor Industry
Association, a trade group, pushed back,
\href{https://www.semiconductors.org/wp-content/uploads/2018/08/Final-_SIA_Submission_on_301_Tariffs.pdf}{telling
the United States trade representative in July 2018} that the tariffs
would ``undermine U.S. technological leadership, cost jobs, and
adversely impact U.S. consumers of semiconductor products and the U.S.
semiconductor producers.''

Some industry executives grew more nervous as Mr. Trump escalated his
trade fight and the prospect of an economic rupture between the United
States and China became more real. Chinese customers shifted their
purchases to suppliers in South Korea, Taiwan and elsewhere.

Mr. Trump's trade pact did ink some victories --- it includes greater
protections for companies like Micron, including preliminary injunctions
and expanded legal recourse for theft of trade secrets. It also contains
new promises from China to refrain from pressuring American businesses
to transfer their technology to Chinese companies, and it allows
American companies to sue individuals, including former employees and
hackers.

Semiconductor companies said they would press the administration to make
more gains in the next phase of negotiations, including subsidies, which
Mr. Trump said he plans to address. Just getting China to acknowledge
and agree to forgo unfair practices was progress, they said.

In a statement, Micron said it applauded the deal. ``We look forward to
additional discussions between the countries on significant issues that
are important to Micron and the semiconductor industry, such as
intellectual property protection and subsidies,'' said Jon Hoganson,
Micron's managing director of global government affairs.

But the fight has spilled over into more damaging areas. Last May, the
Commerce Department placed Huawei, which makes handsets and telecom
equipment,
\href{https://www.nytimes.com/2019/05/16/technology/huawei-ban-president-trump.html}{on
a national security blacklist} that bans it from buying some American
products. Other Chinese technology companies
\href{https://www.nytimes.com/2019/10/07/us/politics/us-to-blacklist-28-chinese-entities-over-abuses-in-xinjiang.html}{were
added to the list}, and the government began planning which types of
advanced technologies
\href{https://www.nytimes.com/2019/10/23/business/trump-technology-china-trade.html}{it
would no longer allow companies to export overseas}.

Micron had so far experienced limited effect from Mr. Trump's tariffs
since it does not ship the products it makes in China to the United
States. But Huawei's blacklisting was potentially devastating --- 13
percent of Micron's chip sales are to the Chinese company.

In its fourth-quarter earnings call with investors last September,
Micron warned that the clash could damage its bottom line.

``We see ongoing uncertainty surrounding U.S.-China trade negotiations.
If the Entity List restrictions against Huawei continue and we are
unable to get licenses, we could see a worsening decline in our sales to
Huawei over the coming quarters,'' Mr. Mehrotra said. Micron's stock
sank 11 percent after his remarks.

Micron, Intel and other companies with global operations initially
\href{https://www.nytimes.com/2019/06/25/technology/huawei-trump-ban-technology.html}{found
a way to keep selling} to Huawei since the rule did not restrict
products containing less than 25 percent of certain types of American
content. But the Commerce Department is considering lowering that
threshold and expanding the number of goods subject to the ban,
according to five people with knowledge of the plan.

Like other Chinese companies, Huawei has worked to curtail its
dependence on America. By substituting parts from Japan and other
countries, the company has recently produced handsets and telecom
equipment that do not contain any American components.

Its internal semiconductor unit, HiSilicon, has also developed
replacements for advanced chips that Huawei once bought from American
companies. Huawei said its 2019 sales topped \$120 billion, representing
18 percent growth over the year before --- less than its initial target,
but not by much.

Image

Micron, which makes memory chips for computers and smartphones, is the
kind of company the Trump administration views as crucial to maintaining
a competitive edge over China.Credit...Tomohiro Ohsumi/Bloomberg

American companies say they are sympathetic to the administration's
complaints about China. But they must compete globally, and they are not
willing to forgo access to China, the hub of the global electronics
supply chain and probably one of the world's fastest growing markets for
decades to come.

Jim McGregor, the chairman of Greater China for APCO Worldwide, said the
trade war and other restrictions were already shaping investment
decisions by American technology companies. When deciding where to put
their money next, many companies have quietly been looking to invest
outside the United States to secure access to China.

``You've got to be there, no matter what the president says,'' he said.

Raymond Zhong contributed reporting from Beijing.

Advertisement

\protect\hyperlink{after-bottom}{Continue reading the main story}

\hypertarget{site-index}{%
\subsection{Site Index}\label{site-index}}

\hypertarget{site-information-navigation}{%
\subsection{Site Information
Navigation}\label{site-information-navigation}}

\begin{itemize}
\tightlist
\item
  \href{https://help.nytimes.com/hc/en-us/articles/115014792127-Copyright-notice}{©~2020~The
  New York Times Company}
\end{itemize}

\begin{itemize}
\tightlist
\item
  \href{https://www.nytco.com/}{NYTCo}
\item
  \href{https://help.nytimes.com/hc/en-us/articles/115015385887-Contact-Us}{Contact
  Us}
\item
  \href{https://www.nytco.com/careers/}{Work with us}
\item
  \href{https://nytmediakit.com/}{Advertise}
\item
  \href{http://www.tbrandstudio.com/}{T Brand Studio}
\item
  \href{https://www.nytimes.com/privacy/cookie-policy\#how-do-i-manage-trackers}{Your
  Ad Choices}
\item
  \href{https://www.nytimes.com/privacy}{Privacy}
\item
  \href{https://help.nytimes.com/hc/en-us/articles/115014893428-Terms-of-service}{Terms
  of Service}
\item
  \href{https://help.nytimes.com/hc/en-us/articles/115014893968-Terms-of-sale}{Terms
  of Sale}
\item
  \href{https://spiderbites.nytimes.com}{Site Map}
\item
  \href{https://help.nytimes.com/hc/en-us}{Help}
\item
  \href{https://www.nytimes.com/subscription?campaignId=37WXW}{Subscriptions}
\end{itemize}
