Sections

SEARCH

\protect\hyperlink{site-content}{Skip to
content}\protect\hyperlink{site-index}{Skip to site index}

\href{https://myaccount.nytimes.com/auth/login?response_type=cookie\&client_id=vi}{}

\href{https://www.nytimes.com/section/todayspaper}{Today's Paper}

\href{/section/opinion}{Opinion}\textbar{}We're Banning Facial
Recognition. We're Missing the Point.

\url{https://nyti.ms/2G76rNq}

\begin{itemize}
\item
\item
\item
\item
\item
\end{itemize}

Advertisement

\protect\hyperlink{after-top}{Continue reading the main story}

\href{/section/opinion}{Opinion}

Supported by

\protect\hyperlink{after-sponsor}{Continue reading the main story}

\hypertarget{were-banning-facial-recognition-were-missing-the-point}{%
\section{We're Banning Facial Recognition. We're Missing the
Point.}\label{were-banning-facial-recognition-were-missing-the-point}}

The whole point of modern surveillance is to treat people differently,
and facial recognition technologies are only a small part of that.

By Bruce Schneier

Mr. Schneier is a fellow at the Harvard Kennedy School.

\begin{itemize}
\item
  Jan. 20, 2020
\item
  \begin{itemize}
  \item
  \item
  \item
  \item
  \item
  \end{itemize}
\end{itemize}

\includegraphics{https://static01.nyt.com/images/2020/01/20/opinion/20schneier-privacy/20schneier-privacy-articleLarge.jpg?quality=75\&auto=webp\&disable=upscale}

Communities across the United States are starting to ban facial
recognition technologies. In May of last year,
\href{https://www.nytimes.com/2019/05/14/us/facial-recognition-ban-san-francisco.html}{San
Francisco} banned facial recognition; the neighboring city of
\href{https://www.vice.com/en_us/article/zmpaex/oakland-becomes-third-us-city-to-ban-facial-recognition-xz}{Oakland}
soon followed, as did
\href{https://www.bostonglobe.com/metro/2019/06/27/somerville-city-council-passes-facial-recognition-ban/SfaqQ7mG3DGulXonBHSCYK/story.html}{Somerville}
and
\href{https://www.boston.com/news/local-news/2019/12/12/brookline-facial-recognition}{Brookline}
in Massachusetts (a
\href{https://www.aclum.org/en/news/massachusetts-voters-strongly-support-pausing-use-unregulated-face-recognition-technology}{statewide
ban} may follow). In December,
\href{https://www.fastcompany.com/90440198/san-diegos-massive-7-year-experiment-with-facial-recognition-technology-appears-to-be-a-flop}{San
Diego} suspended a facial recognition program in advance of a new
statewide law, which declared it illegal, coming into effect. Forty
major music festivals
\href{https://www.vice.com/en_us/article/ywakpj/40-major-music-festivals-have-pledged-not-to-use-facial-recognition-technology}{pledged}
not to use the technology, and
\href{https://www.banfacialrecognition.com/}{activists} **** are calling
for a nationwide ban. Many Democratic presidential candidates
\href{https://www.vox.com/policy-and-politics/2019/12/3/20965470/2020-presidential-candidates-facial-recognition}{support
at least a partial ban} on the technology.

These efforts are well intentioned, but facial recognition bans are the
wrong way to fight against modern surveillance. Focusing on one
particular identification method misconstrues the nature of the
surveillance society we're in the process of building. Ubiquitous mass
surveillance is increasingly the norm. In countries like China, a
surveillance infrastructure is being built by the government for social
control. In countries like the United States, it's being built by
corporations in order to influence our buying behavior, and is
incidentally used by the government.

In all cases, modern mass surveillance has three broad components:
identification, correlation and discrimination. Let's take them in turn.

Facial recognition is a technology that can be used to identify people
without their knowledge or consent. It relies on the prevalence of
cameras, which are becoming both more powerful and smaller, and machine
learning technologies that can match the output of these cameras with
images from a database of existing photos.

But that's just one identification technology among many. People can be
identified at a distance by their
\href{https://www.technologyreview.com/s/613891/the-pentagon-has-a-laser-that-can-identify-people-from-a-distanceby-their-heartbeat/}{heart
beat} or by their
\href{http://www.watrix.ai/en/gait-recognition/}{gait}, using a
laser-based system. Cameras are so good that they can read
\href{https://www.technologyreview.com/s/422400/fingerprints-go-the-distance/}{fingerprints}
and
\href{https://www.theatlantic.com/technology/archive/2015/05/long-range-iris-scanning-is-here/393065/}{iris}\textbf{\href{https://www.theatlantic.com/technology/archive/2015/05/long-range-iris-scanning-is-here/393065/}{}}\href{https://www.theatlantic.com/technology/archive/2015/05/long-range-iris-scanning-is-here/393065/}{patterns}
from meters away. And even without any of these technologies, we can
always be identified because our smartphones
\href{https://www.howtogeek.com/196998/your-devices-broadcast-unique-numbers-and-theyre-being-used-to-track-you/}{broadcast}
unique numbers called MAC addresses. Other things identify us as well:
our phone numbers, our credit card numbers, the license plates on our
cars. China, for example,
\href{https://www.nytimes.com/2019/12/17/technology/china-surveillance.html}{uses
multiple identification technologies} to support its surveillance state.

Once we are identified, the data about who we are and what we are doing
can be correlated with other data collected at other times. This might
be movement data, which can be used to ``follow'' us as we move
throughout our day. It can be purchasing data, internet browsing data,
or data about who we talk to via email or text. It might be data about
our income, ethnicity, lifestyle, profession and interests. There is an
entire industry of data brokers who make a living analyzing and
\href{https://www.eff.org/wp/behind-the-one-way-mirror}{augmenting data}
about who we are --- using surveillance data collected by all sorts of
companies and then sold without our knowledge or consent.

There is a huge --- and almost entirely unregulated --- data broker
industry in the United States that trades on our information. This is
how large internet companies like Google and Facebook make their money.
It's not just that they know who we are, it's that they correlate what
they know about us to create profiles about who we are and what our
interests are. This is why many companies
\href{https://www.vice.com/en_us/article/43kxzq/dmvs-selling-data-private-investigators-making-millions-of-dollars}{buy
license plate data} from states. It's also why companies
\href{https://www.nytimes.com/2019/11/11/business/google-ascension-health-data.html}{like
Google} are buying health records, and part of the reason Google
\href{https://www.forbes.com/sites/brucelee/2019/11/02/google-to-buy-fitbit-for-21-billion-what-about-privacy-concerns/}{bought
the company Fitbit}, along with all of its data.

The whole purpose of this process is for companies --- and governments
--- to treat individuals differently. We are shown different ads on the
internet and receive different offers for credit cards.
\href{https://www.bloomberg.com/opinion/articles/2018-08-10/google-s-targeted-ads-are-coming-to-a-billboard-near-you}{Smart
billboards} display different advertisements based on who we are. In the
future, we might be treated differently when we walk into a store, just
as we currently are when we visit websites.

The point is that it doesn't matter which technology is used to identify
people. That there currently is no comprehensive database of heart beats
or gaits doesn't make the technologies that gather them any less
effective. And most of the time, it doesn't matter if identification
isn't tied to a real name. What's important is that we can be
consistently identified over time. We might be completely anonymous in a
\href{https://privacy.net/stop-cookies-tracking/}{system that uses
unique cookies} to track us as we browse the internet, but the same
process of correlation and discrimination still occurs. It's the same
with faces; we can be tracked as we move around a store or shopping
mall, even if that tracking isn't tied to a specific name. And that
anonymity is fragile: If we ever order something online with a credit
card, or purchase something with a credit card in a store, then suddenly
our real names are attached to what was anonymous tracking information.

Regulating this system means addressing all three steps of the process.
A ban on facial recognition won't make any difference if, in response,
surveillance systems switch to identifying people by smartphone MAC
addresses. The problem is that we are being identified without our
knowledge or consent, and society needs rules about when that is
permissible.

Similarly, we need rules about how our data can be combined with other
data, and then bought and sold without our knowledge or consent. The
data broker industry is almost entirely unregulated; there's only one
law --- passed in
\href{https://www.fastcompany.com/90302036/over-120-data-brokers-inch-out-of-the-shadows-under-landmark-vermont-law}{Vermont}
in 2018 --- that requires data brokers to register and explain in broad
terms what kind of data they collect. The large internet surveillance
companies like Facebook and Google collect dossiers on us more detailed
than those of any police state of the previous century. Reasonable laws
would prevent the worst of their abuses.

Finally, we need better rules about when and how it is permissible for
companies to discriminate. Discrimination based on protected
characteristics like race and gender is already illegal, but those rules
are ineffectual against the current technologies of surveillance and
control. When people can be identified and their data correlated at a
speed and scale previously unseen, we need new rules.

Today, facial recognition technologies are receiving the brunt of the
tech backlash, but focusing on them misses the point. We need to have a
serious conversation about all the technologies of identification,
correlation and discrimination, and decide how much we as a society want
to be spied on by governments and corporations --- and what sorts of
influence we want them to have over our lives.

Bruce Schneier is a fellow at the Harvard Kennedy School and the author,
most recently, of ``Click Here to Kill Everybody: Security and Survival
in a Hyper-Connected World.''

\emph{Like other media companies, The Times collects data on its
visitors when they read stories like this one. For more detail please
see}
\href{https://help.nytimes.com/hc/en-us/articles/115014892108-Privacy-policy?module=inline}{\emph{our
privacy policy}} \emph{and}
\href{https://www.nytimes.com/2019/04/10/opinion/sulzberger-new-york-times-privacy.html?rref=collection\%2Fspotlightcollection\%2Fprivacy-project-does-privacy-matter\&action=click\&contentCollection=opinion\&region=stream\&module=stream_unit\&version=latest\&contentPlacement=8\&pgtype=collection}{\emph{our
publisher's description}} \emph{of The Times's practices and continued
steps to increase transparency and protections.}

\emph{Follow}
\href{https://twitter.com/privacyproject}{\emph{@privacyproject}}
\emph{on Twitter and The New York Times Opinion Section on}
\href{https://www.facebook.com/nytopinion}{\emph{Facebook}}
\emph{and}\href{https://www.instagram.com/nytopinion/}{\emph{Instagram}}\emph{.}

\hypertarget{glossary-replacer}{%
\subsection{glossary replacer}\label{glossary-replacer}}

Advertisement

\protect\hyperlink{after-bottom}{Continue reading the main story}

\hypertarget{site-index}{%
\subsection{Site Index}\label{site-index}}

\hypertarget{site-information-navigation}{%
\subsection{Site Information
Navigation}\label{site-information-navigation}}

\begin{itemize}
\tightlist
\item
  \href{https://help.nytimes.com/hc/en-us/articles/115014792127-Copyright-notice}{©~2020~The
  New York Times Company}
\end{itemize}

\begin{itemize}
\tightlist
\item
  \href{https://www.nytco.com/}{NYTCo}
\item
  \href{https://help.nytimes.com/hc/en-us/articles/115015385887-Contact-Us}{Contact
  Us}
\item
  \href{https://www.nytco.com/careers/}{Work with us}
\item
  \href{https://nytmediakit.com/}{Advertise}
\item
  \href{http://www.tbrandstudio.com/}{T Brand Studio}
\item
  \href{https://www.nytimes.com/privacy/cookie-policy\#how-do-i-manage-trackers}{Your
  Ad Choices}
\item
  \href{https://www.nytimes.com/privacy}{Privacy}
\item
  \href{https://help.nytimes.com/hc/en-us/articles/115014893428-Terms-of-service}{Terms
  of Service}
\item
  \href{https://help.nytimes.com/hc/en-us/articles/115014893968-Terms-of-sale}{Terms
  of Sale}
\item
  \href{https://spiderbites.nytimes.com}{Site Map}
\item
  \href{https://help.nytimes.com/hc/en-us}{Help}
\item
  \href{https://www.nytimes.com/subscription?campaignId=37WXW}{Subscriptions}
\end{itemize}
