Sections

SEARCH

\protect\hyperlink{site-content}{Skip to
content}\protect\hyperlink{site-index}{Skip to site index}

\href{https://myaccount.nytimes.com/auth/login?response_type=cookie\&client_id=vi}{}

\href{https://www.nytimes.com/section/todayspaper}{Today's Paper}

\href{/section/opinion}{Opinion}\textbar{}We Need a Law to Save Us From
Dystopia

\url{https://nyti.ms/2NO5LRc}

\begin{itemize}
\item
\item
\item
\item
\item
\end{itemize}

Advertisement

\protect\hyperlink{after-top}{Continue reading the main story}

\href{/section/opinion}{Opinion}

Supported by

\protect\hyperlink{after-sponsor}{Continue reading the main story}

\hypertarget{we-need-a-law-to-save-us-from-dystopia}{%
\section{We Need a Law to Save Us From
Dystopia}\label{we-need-a-law-to-save-us-from-dystopia}}

It's not too late. And it better be comprehensive.

\href{https://www.nytimes.com/by/charlie-warzel}{\includegraphics{https://static01.nyt.com/images/2019/03/15/opinion/charlie-warzel/charlie-warzel-thumbLarge-v3.png}}

By \href{https://www.nytimes.com/by/charlie-warzel}{Charlie Warzel}

Mr. Warzel is an Opinion writer at large.

\begin{itemize}
\item
  Jan. 21, 2020
\item
  \begin{itemize}
  \item
  \item
  \item
  \item
  \item
  \end{itemize}
\end{itemize}

\includegraphics{https://static01.nyt.com/images/2020/01/21/opinion/21warzelWeb/merlin_167287026_e9863609-1a6a-4104-b9cb-f3ebcef5c25b-articleLarge.jpg?quality=75\&auto=webp\&disable=upscale}

Over the long weekend, my newsroom colleague Kashmir Hill had
\href{https://www.nytimes.com/2020/01/18/technology/clearview-privacy-facial-recognition.html}{a
blockbuster article about a facial recognition company} ``that might end
privacy as we know it.'' It charts the rise of Clearview AI, a company
that scrapes images from social networks like Facebook, YouTube, Venmo
and millions of other sites to create a repository of billions of
images. Using Clearview's app, law enforcement and government agencies
can upload a photo of a person and the database will return matches to
more photos and links to where the pictures came from.

You should read the whole article but one part that's really stayed with
me comes from one of Clearview's early investors, David Scalzo:

\begin{quote}
``I've come to the conclusion that because information constantly
increases, there's never going to be privacy,'' Mr. Scalzo said. ``Laws
have to determine what's legal, but you can't ban technology. Sure, that
might lead to a dystopian future or something, but you can't ban it.''
\end{quote}

Mr. Scalzo's quotation is helpful because he's saying the quiet part out
loud. His reasoning is alarming: Privacy is dead and nothing should halt
the march of technological progress --- not even the possibility of
dystopia.

Clearview's founder, Hoan Ton-That, also seemed caught off guard when
asked to imagine the negative externalities of his tech. ``There's
always going to be a community of bad people who will misuse it,'' he
told The Times. And when faced with the bigger question~--- How do you
feel about effectively eroding the ability to be anonymous in a crowd?
--- Mr. Ton-That was hesitant. ``I have to think about that,'' he said.
``Our belief is that this is the best use of the technology.''

Mr. Ton-That and Mr. Scalzo give a master class in what the writer Rose
Eveleth calls
``\href{https://www.vox.com/the-highlight/2019/10/1/20887003/tech-technology-evolution-natural-inevitable-ethics}{the
myth of inevitable technological progress}.'' Technologists decide to
compare the creep of new tools to evolution~--- a natural process. Of
course, this isn't true. Tech doesn't evolve naturally; it's the result
of calculated decisions by people motivated by any number of factors:
ambition, greed, curiosity or even boredom.

A better, more depressing explanation comes from Al Gidari, a professor
specializing in privacy issues at Stanford Law School, who argues that
companies like Clearview will proliferate because ``there is no monopoly
on math.'' He then utters the most ominous line in the article: ``Absent
a very strong federal privacy law, we're all screwed.''

Professor Gidari is right. But what does a very strong federal privacy
law look like? It's hard to know. In reporting out
\href{https://www.nytimes.com/interactive/2019/12/19/opinion/location-tracking-cell-phone.html}{our
series on location data}, I kept running into examples of companies
using loopholes to skirt privacy laws like Europe's General Data
Protection Regulation.

Location data is declared technically ``anonymous'' even though it's
easily tied to advertiser IDs or deanonymized. And so the companies
didn't have to classify it as ``personally identifiable information''
and don't have to provide it to citizens who request it. In other cases,
I've heard anecdotes about companies simply claiming falsely that they
complied with privacy laws because they knew full well that regulators
wouldn't scrutinize them.

Without teeth and tough enforcement, a federal privacy law won't stop
techno-evolutionists like Mr. Ton-That and Mr. Scalzo. Which means the
law would have to be comprehensive. This brings me to a
\href{https://www.nytimes.com/2020/01/20/opinion/facial-recognition-ban-privacy.html}{Privacy
Project essay on Monday} by Bruce Schneier, a Harvard Kennedy School
fellow and computer security professional. In it, he discusses the
scourge of facial recognition but suggests that simply banning it isn't
enough. Instead, he argues, mass surveillance has three important
components: identification, correlation and discrimination. Any privacy
law needs to tackle all three.

Here's Mr. Schneier's key point:

\begin{quote}
A ban on facial recognition won't make any difference if, in response,
surveillance systems switch to identifying people by smartphone MAC
addresses. The problem is that we are being identified without our
knowledge or consent, and society needs rules about when that is
permissible.
\end{quote}

Think of it this way: Facial recognition apps like Clearview AI are an
engine. But for the engine to run, it needs some kind of fuel, which is
provided by other areas of the surveillance economy. In Clearview's
case, it's the social networks like Facebook, which with lax privacy
settings and default-to-public profiles, allowed its users' photos to be
scraped against the platforms' own terms of service. None of these
companies operated in a vacuum and, as Mr. Schneier notes, even random
bits of information can tie anonymous data back to your true identity.

Perhaps most important, Mr. Schneier argues for ``better rules about
when and how it is permissible for companies to discriminate.'' This is
absolutely crucial as it is more about societal norms than any
particular line of code or piece of technology. Setting clear rules
about when technology can single us out and treat us differently based
on unique identification requires that we all pause and do the difficult
work of imagining the world we want to build. It means not hiding behind
the false premise that privacy-obliterating technology is inevitable, as
Mr. Ton-That and Mr. Scalzo have chosen to do.

We don't have to resign ourselves to dystopia. It's not inevitable. But
it will require something. We have to envision the world we want and
lobby for it. We have to demand that lawmakers create a framework that
allows technology to operate inside those constraints. And if we don't,
Professor Gidari is likely right. We're all screwed.

\hypertarget{policing-the-porch}{%
\subsection{Policing the Porch}\label{policing-the-porch}}

I absolutely
\href{https://www.nytimes.com/2020/01/19/style/ring-video-doorbell-home-security.html}{loved
this article on Ring doorbell cameras} by my newsroom colleague John
Herrman. In it, he describes how Ring is changing our relationships to
the places we live in ways that feel subtle now, but may soon feel
drastic. ``Ring is something like a home-security counterpart to the
work email account on your personal phone, or the scheduling app buzzing
you about a shift, ensuring you can never truly clock out,'' he writes.
``Home surveillance means you're never quite home, but you're never
completely away from home, either.''

Here Mr. Herrman gets at something important. Privacy-eroding technology
warps our attention by allowing us to pay attention to things (like our
front steps) that we previously couldn't. In Ring's case there's a
personal effect, which is that you're constantly tethered to your home.
But there's also a public element to Ring's surveillance, as the article
notes:

\begin{quote}
The presence of a camera at the door creates peculiar new forms of
interaction. The father who half-seriously interrogated his daughter's
date --- in a video publicized by the company and later covered by
national news outlets --- was at work when his phone buzzed. He
conducted his grilling remotely, using the doorbell's voice function.
Ring cameras themselves are
\href{https://chicago.cbslocal.com/2019/12/19/ring-home-camera-thefts/}{now
being stolen}, leaving their owners with a final few seconds of footage
--- a hand, a face, a mask --- before losing their connections.
\end{quote}

Already we're seeing ways that Ring is commandeering and warping our
attention. Ring's social network, Neighbors, feeds a steady stream of
porch videos onto the internet. They fuel local news coverage, sometimes
depicting funny or silly moments but, more often than not, showing a
petty thief absconding with a box. The clips go viral, and sometimes
they're picked up by national news outlets.

In this way, Ring makes the hyperlocal national. Our attention is
directed to a porch in a small town we've never heard of, watching
footage of somebody we don't know that's been recorded by somebody who
isn't home. And yet maybe we feel violated just watching those clips.
The world feels less safe. We think about our own porches and the
packages that may or may not be waiting for us. Who's on our stoop? Who
might show up?

The type of constant surveillance offered by technology like Ring has
many obvious downsides. It can fuel our paranoia or our worst,
discriminatory instincts. But, just as important, it can begin to chip
away at our sense of place. As Mr. Herrman writes, it places us in a
strange tech-fueled purgatory. We're never quite in the moment, but
we're never quite removed from it, either.

\hypertarget{what-im-reading}{%
\subsection{What I'm Reading:}\label{what-im-reading}}

\href{https://www.ft.com/content/4ade8884-1b40-11ea-97df-cc63de1d73f4}{Can
we ever trust Google with our health data?}

\href{https://www.lawfareblog.com/apple-vs-fbi-pensacola-isnt-san-bernardino}{Apple
vs. F.B.I.: Pensacola Isn't San Bernardino}

\href{https://www.wsj.com/articles/thousands-of-chinese-students-data-exposed-on-internet-11579283410?mod=e2tw}{Thousands
of Chinese Students' Data Exposed on Internet}

\href{https://www.politico.eu/pro/eu-considers-temporary-ban-on-facial-recognition-in-public-spaces/}{E.U.
Considers Temporary Ban on Facial Recognition in Public Spaces}

\emph{Like other media companies, The Times collects data on its
visitors when they read articles like this one. For more detail please
see}
\href{https://help.nytimes.com/hc/en-us/articles/115014892108-Privacy-policy?module=inline}{\emph{our
privacy policy}} \emph{and}
\href{https://www.nytimes.com/2019/04/10/opinion/sulzberger-new-york-times-privacy.html?rref=collection\%2Fspotlightcollection\%2Fprivacy-project-does-privacy-matter\&action=click\&contentCollection=opinion\&region=stream\&module=stream_unit\&version=latest\&contentPlacement=8\&pgtype=collection}{\emph{our
publisher's description}} \emph{of The Times's practices and continued
steps to increase transparency and protections.}

\emph{Follow}
\href{https://twitter.com/privacyproject}{\emph{@privacyproject}}
\emph{on Twitter and The New York Times Opinion Section on}
\href{https://www.facebook.com/nytopinion}{\emph{Facebook}}
\emph{and}\href{https://www.instagram.com/nytopinion/}{\emph{Instagram}}\emph{.}

\hypertarget{glossary-replacer}{%
\subsection{glossary replacer}\label{glossary-replacer}}

Advertisement

\protect\hyperlink{after-bottom}{Continue reading the main story}

\hypertarget{site-index}{%
\subsection{Site Index}\label{site-index}}

\hypertarget{site-information-navigation}{%
\subsection{Site Information
Navigation}\label{site-information-navigation}}

\begin{itemize}
\tightlist
\item
  \href{https://help.nytimes.com/hc/en-us/articles/115014792127-Copyright-notice}{©~2020~The
  New York Times Company}
\end{itemize}

\begin{itemize}
\tightlist
\item
  \href{https://www.nytco.com/}{NYTCo}
\item
  \href{https://help.nytimes.com/hc/en-us/articles/115015385887-Contact-Us}{Contact
  Us}
\item
  \href{https://www.nytco.com/careers/}{Work with us}
\item
  \href{https://nytmediakit.com/}{Advertise}
\item
  \href{http://www.tbrandstudio.com/}{T Brand Studio}
\item
  \href{https://www.nytimes.com/privacy/cookie-policy\#how-do-i-manage-trackers}{Your
  Ad Choices}
\item
  \href{https://www.nytimes.com/privacy}{Privacy}
\item
  \href{https://help.nytimes.com/hc/en-us/articles/115014893428-Terms-of-service}{Terms
  of Service}
\item
  \href{https://help.nytimes.com/hc/en-us/articles/115014893968-Terms-of-sale}{Terms
  of Sale}
\item
  \href{https://spiderbites.nytimes.com}{Site Map}
\item
  \href{https://help.nytimes.com/hc/en-us}{Help}
\item
  \href{https://www.nytimes.com/subscription?campaignId=37WXW}{Subscriptions}
\end{itemize}
