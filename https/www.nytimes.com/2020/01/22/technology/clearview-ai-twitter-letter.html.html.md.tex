Sections

SEARCH

\protect\hyperlink{site-content}{Skip to
content}\protect\hyperlink{site-index}{Skip to site index}

\href{https://www.nytimes.com/section/technology}{Technology}

\href{https://myaccount.nytimes.com/auth/login?response_type=cookie\&client_id=vi}{}

\href{https://www.nytimes.com/section/todayspaper}{Today's Paper}

\href{/section/technology}{Technology}\textbar{}Twitter Tells Facial
Recognition Trailblazer to Stop Using Site's Photos

\url{https://nyti.ms/2RIvPy5}

\begin{itemize}
\item
\item
\item
\item
\item
\end{itemize}

Advertisement

\protect\hyperlink{after-top}{Continue reading the main story}

Supported by

\protect\hyperlink{after-sponsor}{Continue reading the main story}

\hypertarget{twitter-tells-facial-recognition-trailblazer-to-stop-using-sites-photos}{%
\section{Twitter Tells Facial Recognition Trailblazer to Stop Using
Site's
Photos}\label{twitter-tells-facial-recognition-trailblazer-to-stop-using-sites-photos}}

Twitter said Clearview AI, whose app is spreading in law enforcement,
was violating its policies. Lawmakers also expressed privacy concerns.

\includegraphics{https://static01.nyt.com/images/2020/01/22/business/22clearview/merlin_167287032_e019a47e-a331-475a-996f-022fb7f11282-articleLarge.jpg?quality=75\&auto=webp\&disable=upscale}

By \href{https://www.nytimes.com/by/kashmir-hill}{Kashmir Hill}

\begin{itemize}
\item
  Jan. 22, 2020
\item
  \begin{itemize}
  \item
  \item
  \item
  \item
  \item
  \end{itemize}
\end{itemize}

A mysterious company that has licensed its powerful facial recognition
technology to hundreds of law enforcement agencies is facing attacks
from Capitol Hill and from at least one Silicon Valley giant.

Twitter sent a letter this week to the small start-up company, Clearview
AI, demanding that it stop taking photos and any other data from the
social media website ``for any reason'' and delete any data that it
previously collected, a Twitter spokeswoman said. The cease-and-desist
letter, sent on Tuesday, accused Clearview of violating Twitter's
policies.

\href{https://www.nytimes.com/2020/01/18/technology/clearview-privacy-facial-recognition.html}{The
New York Times reported} last week that Clearview had amassed a database
of more than three billion photos from social media sites --- including
Facebook, YouTube, Twitter and Venmo --- and elsewhere on the internet.
The vast database powers an app that can match people to their online
photos and link back to the sites the images came from.

The app is used by more than 600 law enforcement agencies, ranging from
local police departments to the F.B.I. and the Department of Homeland
Security. Law enforcement officials told The Times that the app had
helped them identify suspects in many criminal cases.

Clearview's database of photos dwarfs those previously used by law
enforcement agencies. Other technology companies capable of building
such a tool, like Google, have decided not to because of concerns about
the potential for abuse.

Tor Ekeland, a lawyer for Clearview, confirmed that it had received
Twitter's letter and said the company ``will respond appropriately.'' He
declined to comment further.

The Times article set off angry protests from Democratic lawmakers and
privacy watchdogs, who said it was paving the way for universal facial
recognition technology that would effectively end people's ability to
remain anonymous while in public.

On Wednesday, Senator Edward J. Markey, Democrat of Massachusetts, also
sent
\href{https://int.nyt.com/data/documenthelper/6718-sen-markey-letter-to-clearview/33422997119c3d43033d/optimized/full.pdf\#page=1}{a
letter} to Clearview, addressed to its co-founder and chief executive,
Hoan Ton-That. ``Widespread use of your technology could facilitate
dangerous behavior and could effectively destroy individuals' ability to
go about their daily lives anonymously,'' Mr. Markey wrote.

The senator's letter poses 14 questions to the company and asks that it
respond by Feb. 12. Mr. Markey wants Clearview to provide a list of all
law enforcement and intelligence agencies, as well as private entities,
that use the app. He also asked about the collection of children's
information by the company and how it vets its product for accuracy and
security.

``In the absence of a rigorously enforced consumer privacy law,
technology companies will continue to develop and market products that
pose existential threats to our fundamental privacy rights,'' Mr. Markey
said in a statement.

Mr. Ekeland said Clearview was reviewing Mr. Markey's letter and ``will
respond accordingly.''

Senator Ron Wyden, Democrat of Oregon, said on Twitter that he was
concerned that Americans' personal photos were being included in a
corporate database without their knowledge. He also said it was
``extremely troubling'' that Clearview had contacted police officers who
were talking to the media, apparently after monitoring the activity of
police officers who uploaded a photo of a Times reporter to the
Clearview app.

Officials in Mr. Wyden's office will meet soon with Mr. Ton-That in
Washington, said Keith Chu, Mr. Wyden's spokesman.

Mr. Ekeland said: ``Senator Wyden's office reached out to us in
December, and we are in the process of scheduling a meeting. We look
forward to it.''

An aide to Senator Bernie Sanders's presidential campaign, Josh Orton,
also condemned Clearview, saying that its practices were ``disgusting''
and that Mr. Sanders, if elected president, would bar law enforcement
from using facial recognition software.

In an interview with The Times this month, Mr. Ton-That defended
Clearview's technology as a valuable resource for law enforcement. ``Our
belief is that this is the best use of the technology,'' he said. He
added that the company had no plans to release its app for use by the
public, though some private companies use it.

Mr. Ton-That acknowledged that Clearview had amassed its database of
photos by ``scraping'' them from publicly available websites like
Facebook and Twitter. The social media companies said such activity
would violate their terms of service, and Facebook said it was reviewing
the situation with Clearview and ``will take appropriate action if we
find they are violating our rules.''

It isn't clear what power Twitter and other social media sites have to
force Clearview to remove images from its database. In the past,
companies have sued websites that scrape information, accusing them of
violating the Computer Fraud and Abuse Act, an anti-hacking law. But in
September, a federal appeals court in California ruled
\href{https://www.eff.org/deeplinks/2019/09/victory-ruling-hiq-v-linkedin-protects-scraping-public-data}{against
LinkedIn} in such a case, establishing a precedent that the scraping of
public data most likely doesn't violate the law.

The case ``eviscerated the legal argument that Facebook used to use on
scammers and spammers,'' said Alex Stamos, director of the Stanford
Internet Observatory and a former chief information security officer at
Facebook.

When asked whether Facebook had sent a cease-and-desist letter to
Clearview, a Facebook spokesman said the company had ``no updates to
share at this time.''

A Venmo spokesman, Justin Higgs, said on Wednesday, ``Scraping Venmo is
a violation of our terms of service and we actively work to limit and
block activity that violates these policies.''

YouTube didn't respond to a request for comment on Wednesday.

One of Clearview's early investors was Peter Thiel, a venture capitalist
who backed Facebook and sits on Facebook's board of directors. Jeremiah
Hall, a spokesman for Mr. Thiel, previously told The Times that Mr.
Thiel's ``only contribution'' to Clearview was a \$200,000 investment
that was converted into equity, and that ``he is not involved in the
company.''

Advertisement

\protect\hyperlink{after-bottom}{Continue reading the main story}

\hypertarget{site-index}{%
\subsection{Site Index}\label{site-index}}

\hypertarget{site-information-navigation}{%
\subsection{Site Information
Navigation}\label{site-information-navigation}}

\begin{itemize}
\tightlist
\item
  \href{https://help.nytimes.com/hc/en-us/articles/115014792127-Copyright-notice}{©~2020~The
  New York Times Company}
\end{itemize}

\begin{itemize}
\tightlist
\item
  \href{https://www.nytco.com/}{NYTCo}
\item
  \href{https://help.nytimes.com/hc/en-us/articles/115015385887-Contact-Us}{Contact
  Us}
\item
  \href{https://www.nytco.com/careers/}{Work with us}
\item
  \href{https://nytmediakit.com/}{Advertise}
\item
  \href{http://www.tbrandstudio.com/}{T Brand Studio}
\item
  \href{https://www.nytimes.com/privacy/cookie-policy\#how-do-i-manage-trackers}{Your
  Ad Choices}
\item
  \href{https://www.nytimes.com/privacy}{Privacy}
\item
  \href{https://help.nytimes.com/hc/en-us/articles/115014893428-Terms-of-service}{Terms
  of Service}
\item
  \href{https://help.nytimes.com/hc/en-us/articles/115014893968-Terms-of-sale}{Terms
  of Sale}
\item
  \href{https://spiderbites.nytimes.com}{Site Map}
\item
  \href{https://help.nytimes.com/hc/en-us}{Help}
\item
  \href{https://www.nytimes.com/subscription?campaignId=37WXW}{Subscriptions}
\end{itemize}
