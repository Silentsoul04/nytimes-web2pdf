Sections

SEARCH

\protect\hyperlink{site-content}{Skip to
content}\protect\hyperlink{site-index}{Skip to site index}

\href{https://www.nytimes.com/section/world/asia}{Asia Pacific}

\href{https://myaccount.nytimes.com/auth/login?response_type=cookie\&client_id=vi}{}

\href{https://www.nytimes.com/section/todayspaper}{Today's Paper}

\href{/section/world/asia}{Asia Pacific}\textbar{}Scale of China's Wuhan
Shutdown Is Believed to Be Without Precedent

\url{https://nyti.ms/36pDxCU}

\begin{itemize}
\item
\item
\item
\item
\item
\item
\end{itemize}

\href{https://www.nytimes.com/news-event/coronavirus?action=click\&pgtype=Article\&state=default\&region=TOP_BANNER\&context=storylines_menu}{The
Coronavirus Outbreak}

\begin{itemize}
\tightlist
\item
  live\href{https://www.nytimes.com/2020/08/02/world/coronavirus-updates.html?action=click\&pgtype=Article\&state=default\&region=TOP_BANNER\&context=storylines_menu}{Latest
  Updates}
\item
  \href{https://www.nytimes.com/interactive/2020/us/coronavirus-us-cases.html?action=click\&pgtype=Article\&state=default\&region=TOP_BANNER\&context=storylines_menu}{Maps
  and Cases}
\item
  \href{https://www.nytimes.com/interactive/2020/science/coronavirus-vaccine-tracker.html?action=click\&pgtype=Article\&state=default\&region=TOP_BANNER\&context=storylines_menu}{Vaccine
  Tracker}
\item
  \href{https://www.nytimes.com/interactive/2020/07/29/us/schools-reopening-coronavirus.html?action=click\&pgtype=Article\&state=default\&region=TOP_BANNER\&context=storylines_menu}{What
  School May Look Like}
\item
  \href{https://www.nytimes.com/live/2020/07/31/business/stock-market-today-coronavirus?action=click\&pgtype=Article\&state=default\&region=TOP_BANNER\&context=storylines_menu}{Economy}
\end{itemize}

Advertisement

\protect\hyperlink{after-top}{Continue reading the main story}

Supported by

\protect\hyperlink{after-sponsor}{Continue reading the main story}

\hypertarget{scale-of-chinas-wuhan-shutdown-is-believed-to-be-without-precedent}{%
\section{Scale of China's Wuhan Shutdown Is Believed to Be Without
Precedent}\label{scale-of-chinas-wuhan-shutdown-is-believed-to-be-without-precedent}}

In sealing off a city of 11 million people, China is trying to halt a
coronavirus outbreak using a tactic with a complicated history of
ethical concerns.

\includegraphics{https://static01.nyt.com/images/2020/01/22/multimedia/22xp-cordon/merlin_167592177_faab52d5-95c1-48a0-b533-934d06fbed8d-articleLarge.jpg?quality=75\&auto=webp\&disable=upscale}

By \href{https://www.nytimes.com/by/michael-levenson}{Michael Levenson}

\begin{itemize}
\item
  Jan. 22, 2020
\item
  \begin{itemize}
  \item
  \item
  \item
  \item
  \item
  \item
  \end{itemize}
\end{itemize}

\href{https://cn.nytimes.com/china/20200123/coronavirus-quarantines-history/}{阅读简体中文版}\href{https://cn.nytimes.com/china/20200123/coronavirus-quarantines-history/zh-hant/}{閱讀繁體中文版}

In closing off Wuhan, a city of more than 11 million people, China
deployed on Thursday morning a centuries-old public health tactic to
prevent the spread of infectious disease --- this time, a mysterious
respiratory infection caused by a coronavirus.

Experts said the stunning scale of the shutdown,
\href{https://www.nytimes.com/2020/01/22/world/asia/china-coronavirus-travel.html}{isolating
a major urban transit hub larger than New York City}, was without
precedent.

``It's an unbelievable undertaking,'' said Dr. Howard Markel, a
professor of the history of medicine at the University of Michigan,
adding that he had never heard of so many people being cordoned off as a
disease-prevention measure.

Still, ``people are going to get out,'' he said. ``It's going to be
leaky.''

By Thursday evening, China said it planned to extend the shutdown even
further. Officials said they would impose travel restrictions on at
least four other nearby cities --- Huanggang, Ezhou, Chibi and Zhijiang
--- affecting millions more residents.

By limiting the movements of millions of people in an attempt to protect
public health, China is engaging in a balancing act with a long and
complicated history fraught with social, political and ethical concerns.

James G. Hodge Jr., director of the Center for Public Health Law and
Policy at Arizona State University, said the shutdown would almost
certainly lead to human rights violations and would be patently
unconstitutional in the United States.

\hypertarget{latest-updates-global-coronavirus-outbreak}{%
\section{\texorpdfstring{\href{https://www.nytimes.com/2020/08/01/world/coronavirus-covid-19.html?action=click\&pgtype=Article\&state=default\&region=MAIN_CONTENT_1\&context=storylines_live_updates}{Latest
Updates: Global Coronavirus
Outbreak}}{Latest Updates: Global Coronavirus Outbreak}}\label{latest-updates-global-coronavirus-outbreak}}

Updated 2020-08-02T17:52:35.962Z

\begin{itemize}
\tightlist
\item
  \href{https://www.nytimes.com/2020/08/01/world/coronavirus-covid-19.html?action=click\&pgtype=Article\&state=default\&region=MAIN_CONTENT_1\&context=storylines_live_updates\#link-34047410}{The
  U.S. reels as July cases more than double the total of any other
  month.}
\item
  \href{https://www.nytimes.com/2020/08/01/world/coronavirus-covid-19.html?action=click\&pgtype=Article\&state=default\&region=MAIN_CONTENT_1\&context=storylines_live_updates\#link-780ec966}{Top
  U.S. officials work to break an impasse over the federal jobless
  benefit.}
\item
  \href{https://www.nytimes.com/2020/08/01/world/coronavirus-covid-19.html?action=click\&pgtype=Article\&state=default\&region=MAIN_CONTENT_1\&context=storylines_live_updates\#link-2bc8948}{Its
  outbreak untamed, Melbourne goes into even greater lockdown.}
\end{itemize}

\href{https://www.nytimes.com/2020/08/01/world/coronavirus-covid-19.html?action=click\&pgtype=Article\&state=default\&region=MAIN_CONTENT_1\&context=storylines_live_updates}{See
more updates}

More live coverage:
\href{https://www.nytimes.com/live/2020/07/31/business/stock-market-today-coronavirus?action=click\&pgtype=Article\&state=default\&region=MAIN_CONTENT_1\&context=storylines_live_updates}{Markets}

``It could very easily backfire,'' he said, adding that the restrictions
could prevent healthy people from fleeing the city, perhaps exposing
them to greater risk of infection. ``In general, this is risky
business.''

\hypertarget{chinas-expanding-measures}{%
\subsection{China's expanding
measures}\label{chinas-expanding-measures}}

To combat the spread of the virus, which first appeared at the end of
December and has killed at least 17 people and sickened more than 600,
the Chinese government said it would
\href{https://www.nytimes.com/2020/01/22/world/asia/china-coronavirus-travel.html}{cancel
planes and trains leaving Wuhan} beginning Thursday, and suspend buses,
subways and ferries within it.

In Huanggang, a city of seven million about 30 miles east of Wuhan,
residents would not be allowed to leave the city without special
permission, according to a government statement. In Ezhou, which has
about one million residents, all rail stations were to be closed.

The practice of isolating people and goods to halt the spread of disease
dates at least to the 14th century, when ships arriving in Venice during
the plague epidemic were required to anchor off the coast for 40 days.
The isolation period gave rise to the term quarantine, from the Italian
quaranta giorni, meaning 40 days, according to the
\href{https://www.cdc.gov/quarantine/historyquarantine.html}{Centers for
Disease Control and Prevention}.

Professor Hodge said quarantines could be effective if they selectively
isolate only those who have been infected or are suspected of infection.
The response in Wuhan, with the establishment of a
``\href{https://www.nytimes.com/2014/08/13/science/using-a-tactic-unseen-in-a-century-countries-cordon-off-ebola-racked-areas.html}{cordon
sanitaire}''-type boundary, goes much further than that.

``Quarantine would be saying `You can't leave your own home, can't go to
school, work or church,''' he said. But the Chinese authorities ``have
drawn a line around this city and said, `No one in and no one out.' That
type of thing is obviously an excessive response.''

\hypertarget{history-and-the-darker-side-of-quarantine}{%
\subsection{History and `the darker side of
quarantine'}\label{history-and-the-darker-side-of-quarantine}}

In recent years, governments have imposed other large-scale measures to
prevent the spread of infectious diseases.

\href{https://www.nytimes.com/news-event/coronavirus?action=click\&pgtype=Article\&state=default\&region=MAIN_CONTENT_3\&context=storylines_faq}{}

\hypertarget{the-coronavirus-outbreak-}{%
\subsubsection{The Coronavirus Outbreak
›}\label{the-coronavirus-outbreak-}}

\hypertarget{frequently-asked-questions}{%
\paragraph{Frequently Asked
Questions}\label{frequently-asked-questions}}

Updated July 27, 2020

\begin{itemize}
\item ~
  \hypertarget{should-i-refinance-my-mortgage}{%
  \paragraph{Should I refinance my
  mortgage?}\label{should-i-refinance-my-mortgage}}

  \begin{itemize}
  \tightlist
  \item
    \href{https://www.nytimes.com/article/coronavirus-money-unemployment.html?action=click\&pgtype=Article\&state=default\&region=MAIN_CONTENT_3\&context=storylines_faq}{It
    could be a good idea,} because mortgage rates have
    \href{https://www.nytimes.com/2020/07/16/business/mortgage-rates-below-3-percent.html?action=click\&pgtype=Article\&state=default\&region=MAIN_CONTENT_3\&context=storylines_faq}{never
    been lower.} Refinancing requests have pushed mortgage applications
    to some of the highest levels since 2008, so be prepared to get in
    line. But defaults are also up, so if you're thinking about buying a
    home, be aware that some lenders have tightened their standards.
  \end{itemize}
\item ~
  \hypertarget{what-is-school-going-to-look-like-in-september}{%
  \paragraph{What is school going to look like in
  September?}\label{what-is-school-going-to-look-like-in-september}}

  \begin{itemize}
  \tightlist
  \item
    It is unlikely that many schools will return to a normal schedule
    this fall, requiring the grind of
    \href{https://www.nytimes.com/2020/06/05/us/coronavirus-education-lost-learning.html?action=click\&pgtype=Article\&state=default\&region=MAIN_CONTENT_3\&context=storylines_faq}{online
    learning},
    \href{https://www.nytimes.com/2020/05/29/us/coronavirus-child-care-centers.html?action=click\&pgtype=Article\&state=default\&region=MAIN_CONTENT_3\&context=storylines_faq}{makeshift
    child care} and
    \href{https://www.nytimes.com/2020/06/03/business/economy/coronavirus-working-women.html?action=click\&pgtype=Article\&state=default\&region=MAIN_CONTENT_3\&context=storylines_faq}{stunted
    workdays} to continue. California's two largest public school
    districts --- Los Angeles and San Diego --- said on July 13, that
    \href{https://www.nytimes.com/2020/07/13/us/lausd-san-diego-school-reopening.html?action=click\&pgtype=Article\&state=default\&region=MAIN_CONTENT_3\&context=storylines_faq}{instruction
    will be remote-only in the fall}, citing concerns that surging
    coronavirus infections in their areas pose too dire a risk for
    students and teachers. Together, the two districts enroll some
    825,000 students. They are the largest in the country so far to
    abandon plans for even a partial physical return to classrooms when
    they reopen in August. For other districts, the solution won't be an
    all-or-nothing approach.
    \href{https://bioethics.jhu.edu/research-and-outreach/projects/eschool-initiative/school-policy-tracker/}{Many
    systems}, including the nation's largest, New York City, are
    devising
    \href{https://www.nytimes.com/2020/06/26/us/coronavirus-schools-reopen-fall.html?action=click\&pgtype=Article\&state=default\&region=MAIN_CONTENT_3\&context=storylines_faq}{hybrid
    plans} that involve spending some days in classrooms and other days
    online. There's no national policy on this yet, so check with your
    municipal school system regularly to see what is happening in your
    community.
  \end{itemize}
\item ~
  \hypertarget{is-the-coronavirus-airborne}{%
  \paragraph{Is the coronavirus
  airborne?}\label{is-the-coronavirus-airborne}}

  \begin{itemize}
  \tightlist
  \item
    The coronavirus
    \href{https://www.nytimes.com/2020/07/04/health/239-experts-with-one-big-claim-the-coronavirus-is-airborne.html?action=click\&pgtype=Article\&state=default\&region=MAIN_CONTENT_3\&context=storylines_faq}{can
    stay aloft for hours in tiny droplets in stagnant air}, infecting
    people as they inhale, mounting scientific evidence suggests. This
    risk is highest in crowded indoor spaces with poor ventilation, and
    may help explain super-spreading events reported in meatpacking
    plants, churches and restaurants.
    \href{https://www.nytimes.com/2020/07/06/health/coronavirus-airborne-aerosols.html?action=click\&pgtype=Article\&state=default\&region=MAIN_CONTENT_3\&context=storylines_faq}{It's
    unclear how often the virus is spread} via these tiny droplets, or
    aerosols, compared with larger droplets that are expelled when a
    sick person coughs or sneezes, or transmitted through contact with
    contaminated surfaces, said Linsey Marr, an aerosol expert at
    Virginia Tech. Aerosols are released even when a person without
    symptoms exhales, talks or sings, according to Dr. Marr and more
    than 200 other experts, who
    \href{https://academic.oup.com/cid/article/doi/10.1093/cid/ciaa939/5867798}{have
    outlined the evidence in an open letter to the World Health
    Organization}.
  \end{itemize}
\item ~
  \hypertarget{what-are-the-symptoms-of-coronavirus}{%
  \paragraph{What are the symptoms of
  coronavirus?}\label{what-are-the-symptoms-of-coronavirus}}

  \begin{itemize}
  \tightlist
  \item
    Common symptoms
    \href{https://www.nytimes.com/article/symptoms-coronavirus.html?action=click\&pgtype=Article\&state=default\&region=MAIN_CONTENT_3\&context=storylines_faq}{include
    fever, a dry cough, fatigue and difficulty breathing or shortness of
    breath.} Some of these symptoms overlap with those of the flu,
    making detection difficult, but runny noses and stuffy sinuses are
    less common.
    \href{https://www.nytimes.com/2020/04/27/health/coronavirus-symptoms-cdc.html?action=click\&pgtype=Article\&state=default\&region=MAIN_CONTENT_3\&context=storylines_faq}{The
    C.D.C. has also} added chills, muscle pain, sore throat, headache
    and a new loss of the sense of taste or smell as symptoms to look
    out for. Most people fall ill five to seven days after exposure, but
    symptoms may appear in as few as two days or as many as 14 days.
  \end{itemize}
\item ~
  \hypertarget{does-asymptomatic-transmission-of-covid-19-happen}{%
  \paragraph{Does asymptomatic transmission of Covid-19
  happen?}\label{does-asymptomatic-transmission-of-covid-19-happen}}

  \begin{itemize}
  \tightlist
  \item
    So far, the evidence seems to show it does. A widely cited
    \href{https://www.nature.com/articles/s41591-020-0869-5}{paper}
    published in April suggests that people are most infectious about
    two days before the onset of coronavirus symptoms and estimated that
    44 percent of new infections were a result of transmission from
    people who were not yet showing symptoms. Recently, a top expert at
    the World Health Organization stated that transmission of the
    coronavirus by people who did not have symptoms was ``very rare,''
    \href{https://www.nytimes.com/2020/06/09/world/coronavirus-updates.html?action=click\&pgtype=Article\&state=default\&region=MAIN_CONTENT_3\&context=storylines_faq\#link-1f302e21}{but
    she later walked back that statement.}
  \end{itemize}
\end{itemize}

Sierra Leone, a country of about seven million people, said
\href{https://www.nytimes.com/2014/09/07/world/africa/sierra-leone-to-impose-widespread-ebola-quarantine.html}{``everybody''
was expected to stay indoors} for three days in September 2014, as 7,000
teams of health and community workers went door to door to find hidden
Ebola patients.

Earlier that year, Liberian officials placed West Point, a sprawling
slum in Monrovia where 60,000 to 120,000 people were crammed into
shacks,
\href{https://www.nytimes.com/2014/08/29/world/africa/in-liberias-capital-an-ebola-outbreak-like-no-other.html}{under
an Ebola quarantine}. The order led to deadly clashes with soldiers and
may have helped to spread the disease, experts said, forcing people to
crowd together for basic humanitarian aid.

During the SARS outbreak of 2003, Canadian health officials asked anyone
in Ontario who had even one symptom of the respiratory infection to
\href{https://www.nytimes.com/2003/04/18/world/fearing-sars-ontario-urges-wider-quarantines.html}{stay
home for a few days} out of fear that the disease might spread during
the Easter holiday weekend.

In Beijing, at least 4,000 residents who had been exposed to the virus
were
\href{https://www.nytimes.com/2003/04/25/international/asia/4000-quarantined-in-beijing-as-suspected-sars-cases-climb.html}{kept
in isolation}, and 300 college students who had had contact with
infected people were sequestered in a military camp for two weeks.

Historians have noted that quarantines have often targeted marginalized
populations.

During the plague epidemic of the 14th century, European city-states
posted armed guards on roads and access points to keep out merchants,
people with leprosy and minority groups such as Jews, according to
Eugenia Tognotti, a researcher in Italy who has
\href{https://www.ncbi.nlm.nih.gov/pmc/articles/PMC3559034/pdf/12-0312.pdf}{written
on the history of quarantine}.

And during a wave of cholera outbreaks in Europe in the 1830s, Naples
restricted the movement of prostitutes and beggars, who were thought to
be carriers of the contagion, she wrote.

Russian Jews brought typhus fever into the Lower East Side of Manhattan
in 1892, Dr. Markel said. It was not only infected people who were
rounded up and quarantined on an island off the Bronx, however, but also
their neighbors and others whom they had simply greeted on the street.

``That's the darker side of quarantine --- its misuse as a social tool
rather than its scientific use as a medical tool,'' Dr. Markel said.

Mihir Zaveri contributed reporting.

Advertisement

\protect\hyperlink{after-bottom}{Continue reading the main story}

\hypertarget{site-index}{%
\subsection{Site Index}\label{site-index}}

\hypertarget{site-information-navigation}{%
\subsection{Site Information
Navigation}\label{site-information-navigation}}

\begin{itemize}
\tightlist
\item
  \href{https://help.nytimes.com/hc/en-us/articles/115014792127-Copyright-notice}{©~2020~The
  New York Times Company}
\end{itemize}

\begin{itemize}
\tightlist
\item
  \href{https://www.nytco.com/}{NYTCo}
\item
  \href{https://help.nytimes.com/hc/en-us/articles/115015385887-Contact-Us}{Contact
  Us}
\item
  \href{https://www.nytco.com/careers/}{Work with us}
\item
  \href{https://nytmediakit.com/}{Advertise}
\item
  \href{http://www.tbrandstudio.com/}{T Brand Studio}
\item
  \href{https://www.nytimes.com/privacy/cookie-policy\#how-do-i-manage-trackers}{Your
  Ad Choices}
\item
  \href{https://www.nytimes.com/privacy}{Privacy}
\item
  \href{https://help.nytimes.com/hc/en-us/articles/115014893428-Terms-of-service}{Terms
  of Service}
\item
  \href{https://help.nytimes.com/hc/en-us/articles/115014893968-Terms-of-sale}{Terms
  of Sale}
\item
  \href{https://spiderbites.nytimes.com}{Site Map}
\item
  \href{https://help.nytimes.com/hc/en-us}{Help}
\item
  \href{https://www.nytimes.com/subscription?campaignId=37WXW}{Subscriptions}
\end{itemize}
