Sections

SEARCH

\protect\hyperlink{site-content}{Skip to
content}\protect\hyperlink{site-index}{Skip to site index}

\href{https://www.nytimes.com/section/business/economy}{Economy}

\href{https://myaccount.nytimes.com/auth/login?response_type=cookie\&client_id=vi}{}

\href{https://www.nytimes.com/section/todayspaper}{Today's Paper}

\href{/section/business/economy}{Economy}\textbar{}Tougher Huawei
Restrictions Stall After Defense Department Objects

\url{https://nyti.ms/2RoJLyz}

\begin{itemize}
\item
\item
\item
\item
\item
\end{itemize}

Advertisement

\protect\hyperlink{after-top}{Continue reading the main story}

Supported by

\protect\hyperlink{after-sponsor}{Continue reading the main story}

\hypertarget{tougher-huawei-restrictions-stall-after-defense-department-objects}{%
\section{Tougher Huawei Restrictions Stall After Defense Department
Objects}\label{tougher-huawei-restrictions-stall-after-defense-department-objects}}

Proposed changes to further limit American shipments to Huawei have been
delayed amid arguments they could backfire.

\includegraphics{https://static01.nyt.com/images/2020/01/25/business/23DC-TRADE-print/merlin_165832227_7caea52e-5119-4d6e-92a8-4a9a4fec9a4c-articleLarge.jpg?quality=75\&auto=webp\&disable=upscale}

\href{https://www.nytimes.com/by/ana-swanson}{\includegraphics{https://static01.nyt.com/images/2018/12/10/multimedia/author-ana-swanson/author-ana-swanson-thumbLarge.png}}

By \href{https://www.nytimes.com/by/ana-swanson}{Ana Swanson}

\begin{itemize}
\item
  Published Jan. 24, 2020Updated July 14, 2020
\item
  \begin{itemize}
  \item
  \item
  \item
  \item
  \item
  \end{itemize}
\end{itemize}

WASHINGTON --- The Trump administration has temporarily shelved a
proposed rule change that would further restrict American sales to
\href{https://www.nytimes.com/2020/07/14/business/huawei-uk-5g.html}{Huawei},
the Chinese telecom giant, after some officials in the Defense
Department and other agencies argued that the measure, which was
intended to protect national security, could actually undermine it,
according to people familiar with the matter.

The rule change, which multiple government agencies were reviewing,
would close a loophole that allowed technology companies like Intel and
Micron
\href{https://www.nytimes.com/2019/06/25/technology/huawei-trump-ban-technology.html}{to
continue shipping chips, software and other products to Huawei} despite
a ban that prevented the Chinese company from buying some American
products.

Some government officials have objected to the tougher restrictions,
arguing they could discourage the use of American components abroad,
weakening American firms and the country's technological
competitiveness.

The rule has been withdrawn from the Office of Management and Budget,
effectively putting the tighter limits on hold. The change, along with
other China technology issues, will be discussed in a meeting of
President Trump's top advisers, though a date has yet to be set, one of
the people said.

The measure is the latest in a series of steps the Trump administration
has taken to combat what it describes as a pressing security threat:
China's acquisition of advanced technologies that could give the country
both a commercial and a military edge. Many of those efforts have
focused on
\href{https://www.nytimes.com/2020/05/15/business/economy/commerce-department-huawei.html}{Huawei},
which sells global telecom equipment that American officials fear will
give Beijing new channels for control and surveillance. Huawei says that
its networks are secure and that it does not spy for the Chinese
government.

Tensions between the United States and China have eased since the
countries
\href{https://www.nytimes.com/2020/01/15/business/economy/china-trade-deal.html}{concluded
a Phase 1 trade deal}. But the fate of Huawei, and the American
companies that supply it, continues to hang in the balance. Last May,
the Trump administration placed Huawei on a United States blacklist and
\href{https://www.nytimes.com/2019/07/30/technology/huawei-trump.html}{moved
to cut off shipments} of certain goods, software and technology to the
Chinese firm. In order to keep selling certain products to Huawei,
companies had to apply for --- and obtain --- a special license.

The restrictions threatened to cut off lucrative sales for a number of
American tech companies that supplied components to Huawei, including
Intel, Micron and Google. Some firms, eager to continue selling to
Huawei, took advantage of
\href{https://www.nytimes.com/2019/06/25/technology/huawei-trump-ban-technology.html}{a
loophole that allowed them to sell products made outside the United
States to Huawei} without a government license, as long as the products
contained less than 25 percent of certain types of sensitive American
content.

The proposed measure, which applies only to Huawei, would lower that
threshold to 10 percent from 25 percent. It would also expand the rule
so that all types of American content would count toward that 10 percent
threshold.

Such a change would expand the rule's reach beyond sensitive types of
technology to include American software, chips and other components that
are widely available and that Huawei could easily purchase from
Taiwanese, Korean and Japanese manufacturers instead.

The exceptions to the existing rules have allowed Huawei to continue
buying many of the components it needs to make its telecom networks and
smartphones from American suppliers. That has allowed Huawei --- the
third-largest purchaser of chips globally after Apple and Samsung --- to
\href{https://www.nytimes.com/2019/12/30/business/huawei-revenue-growth.html}{continue
growing and increase its revenue}, defying expectations within the tech
industry and in Washington. Huawei said its sales in 2019 topped \$120
billion, which was 18 percent growth over the year before --- less than
its initial target, but not by much.

At the World Economic Forum in Davos, Switzerland, on Tuesday, Huawei's
chief executive, Ren Zhengfei, said he expected the United States to
continue escalating its campaign against Huawei, but was ``confident we
can survive even further attacks.''

Some trade experts say the Trump administration should have anticipated
that business with Huawei would continue, since American controls on
exports are designed to target only sensitive material and technologies.

But some administration officials, including Commerce Secretary Wilbur
Ross, have been surprised that placing Huawei on the entity list, which
designates companies that the United States considers a security or
foreign policy threat, did not halt more business with the company.

In
\href{https://www.bloomberg.com/news/videos/2020-01-23/wilbur-ross-says-u-s-not-cutting-huawei-off-from-exports-video}{an
interview in Davos} on Thursday, Mr. Ross said Huawei had been
encouraging American companies to flout federal laws, which had
attracted the Commerce Department's attention. He added that revisions
to the rules were ``works in progress that will come out in the near
term.''

The rule, which was being considered by officials at the Commerce,
Defense, Treasury, State and Energy Departments, was designed to take
effect before industry had a chance to comment on it.

The Commerce Department has also been weighing a separate rule change
that would expand its jurisdiction over items manufactured overseas with
American technology. People familiar with the planning said that
policymakers were potentially considering a far more expansive measure,
but that the rule was still in the drafting stage.

\includegraphics{https://static01.nyt.com/images/2020/01/23/business/23DC-TRADE-02sub/merlin_167250342_1fcd4ca3-fd20-4e18-8f5e-f6d604449f18-articleLarge.jpg?quality=75\&auto=webp\&disable=upscale}

The proposed measures have not been made public, and their exact scope
is unclear. But reports of their existence have generated panic among
companies and parts of the defense industry, said current and former
government officials.

American tech companies have complained that the changes would backfire,
\href{https://www.nytimes.com/2020/01/20/business/economy/trump-us-china-deal-micron-trade-war.html}{eroding
the country's technological advantages} rather than protecting them.
Those changes could be
\href{https://www.nytimes.com/2020/01/20/business/economy/trump-us-china-deal-micron-trade-war.html}{particularly
devastating for some segments of the semiconductor industry}, where
Huawei can switch to purchasing products from South Korea, Japan, Taiwan
or elsewhere.

In a Dec. 5 letter to Mr. Ross, which was viewed by The New York Times,
a collection of industry groups, including the Semiconductor Industry
Association and the National Association of Manufacturers, wrote that
the changes could reduce innovation and competitiveness in American
industry, cause customers abroad to stop purchasing American technology
and accelerate the offshoring of manufacturing and research.

``While we fully understand the paramount importance of maintaining our
national security, we believe these actions would have serious negative
consequences for U.S. economic leadership and, ultimately, U.S. national
security,'' the letter said.

Within government, the battle lines are blurred.
\href{https://www.nytimes.com/2019/10/23/business/trump-technology-china-trade.html}{The
Commerce Department is split}over how aggressively the government should
regulate industry. Some defense officials have concerns about how the
rule change will affect key military suppliers. Other senior defense
officials believe the national security case for cutting Huawei off from
American components overrides other concerns.

In a letter to Defense Secretary Mark Esper on Friday, three Republican
senators expressed concerns about easier treatment of Huawei.

``Huawei is an arm of the Chinese Communist Party and should be treated
as such,'' Senators Ben Sasse of Nebraska, Tom Cotton of Arkansas and
Marco Rubio of Florida wrote. ``We are concerned that the Defense
Department is not appropriately weighing the risks.''

Officials said that the disagreement could ultimately be resolved in the
next few weeks and that the rule could still move forward.

A Pentagon spokeswoman, Sue Gough, said the department was aware of
Commerce's proposed rule change, but ``will not prematurely discuss
ongoing interagency collaboration.''

The Trump administration has been trying, with limited success, to
discourage other governments like Britain, Germany and India from
allowing Huawei to construct the next generation of wireless networks.

Mr. Trump's advisers warn that allowing Chinese companies to build 5G
networks could compromise intelligence sharing between the United States
and its allies. But
\href{https://www.nytimes.com/aponline/2020/01/14/business/bc-eu-britain-huawei.html}{foreign
officials}say the United States has not provided compelling evidence
that Huawei poses a threat.

The American crackdown has prompted Huawei to try to reduce its
dependence on the United States.

It recently produced handsets and telecom equipment that do not contain
any American components. The company has found substitutes for some
parts from suppliers in other countries, including Japan, and its
in-house semiconductor unit, HiSilicon, has developed replacements for
some advanced chips.

The proposed rule change could accelerate those efforts and persuade
companies like Taiwan Semiconductor Manufacturing Corporation, which
uses many American parts, to halt purchases from the United States, at
least temporarily, Paul Triolo, practice head of geo-technology for
Eurasia Group, wrote in a note to clients.

Industry executives say other Chinese companies are concluding that
American partners are also unreliable suppliers, given the
administration's crackdown. Manufacturers of computers,
air-conditioners, medical devices and other products are canceling their
contracts with American firms and turning to European and Japanese
products, they said.

``We want U.S.-origin technology to be consumed, we want that to be the
industry standard,'' said Scott Jones, a nonresident fellow with the
Stimson Center. ``We don't want it to be designed out.''

But Clyde Prestowitz, the president of the Economic Strategy Institute,
said the short-term costs to American companies would be worth it.

``We are engaged in a non-shooting but completely serious tech war.
Keeping the most advanced chips and chip making equipment out of China
will slow them down,'' Mr. Prestowitz said. ``So, the whining of the
corporate C.E.O.s is really completely short sighted.''

Julian E. Barnes contributed reporting from Washington, and Raymond
Zhong from Beijing.

Advertisement

\protect\hyperlink{after-bottom}{Continue reading the main story}

\hypertarget{site-index}{%
\subsection{Site Index}\label{site-index}}

\hypertarget{site-information-navigation}{%
\subsection{Site Information
Navigation}\label{site-information-navigation}}

\begin{itemize}
\tightlist
\item
  \href{https://help.nytimes.com/hc/en-us/articles/115014792127-Copyright-notice}{©~2020~The
  New York Times Company}
\end{itemize}

\begin{itemize}
\tightlist
\item
  \href{https://www.nytco.com/}{NYTCo}
\item
  \href{https://help.nytimes.com/hc/en-us/articles/115015385887-Contact-Us}{Contact
  Us}
\item
  \href{https://www.nytco.com/careers/}{Work with us}
\item
  \href{https://nytmediakit.com/}{Advertise}
\item
  \href{http://www.tbrandstudio.com/}{T Brand Studio}
\item
  \href{https://www.nytimes.com/privacy/cookie-policy\#how-do-i-manage-trackers}{Your
  Ad Choices}
\item
  \href{https://www.nytimes.com/privacy}{Privacy}
\item
  \href{https://help.nytimes.com/hc/en-us/articles/115014893428-Terms-of-service}{Terms
  of Service}
\item
  \href{https://help.nytimes.com/hc/en-us/articles/115014893968-Terms-of-sale}{Terms
  of Sale}
\item
  \href{https://spiderbites.nytimes.com}{Site Map}
\item
  \href{https://help.nytimes.com/hc/en-us}{Help}
\item
  \href{https://www.nytimes.com/subscription?campaignId=37WXW}{Subscriptions}
\end{itemize}
