Sections

SEARCH

\protect\hyperlink{site-content}{Skip to
content}\protect\hyperlink{site-index}{Skip to site index}

\href{https://www.nytimes.com/section/world/asia}{Asia Pacific}

\href{https://myaccount.nytimes.com/auth/login?response_type=cookie\&client_id=vi}{}

\href{https://www.nytimes.com/section/todayspaper}{Today's Paper}

\href{/section/world/asia}{Asia Pacific}\textbar{}As Coronavirus Fears
Intensify, Effectiveness of Quarantines Is Questioned

\url{https://nyti.ms/37u7Jhp}

\begin{itemize}
\item
\item
\item
\item
\item
\end{itemize}

\href{https://www.nytimes.com/news-event/coronavirus?action=click\&pgtype=Article\&state=default\&region=TOP_BANNER\&context=storylines_menu}{The
Coronavirus Outbreak}

\begin{itemize}
\tightlist
\item
  live\href{https://www.nytimes.com/2020/08/02/world/coronavirus-updates.html?action=click\&pgtype=Article\&state=default\&region=TOP_BANNER\&context=storylines_menu}{Latest
  Updates}
\item
  \href{https://www.nytimes.com/interactive/2020/us/coronavirus-us-cases.html?action=click\&pgtype=Article\&state=default\&region=TOP_BANNER\&context=storylines_menu}{Maps
  and Cases}
\item
  \href{https://www.nytimes.com/interactive/2020/science/coronavirus-vaccine-tracker.html?action=click\&pgtype=Article\&state=default\&region=TOP_BANNER\&context=storylines_menu}{Vaccine
  Tracker}
\item
  \href{https://www.nytimes.com/interactive/2020/07/29/us/schools-reopening-coronavirus.html?action=click\&pgtype=Article\&state=default\&region=TOP_BANNER\&context=storylines_menu}{What
  School May Look Like}
\item
  \href{https://www.nytimes.com/live/2020/07/31/business/stock-market-today-coronavirus?action=click\&pgtype=Article\&state=default\&region=TOP_BANNER\&context=storylines_menu}{Economy}
\end{itemize}

Advertisement

\protect\hyperlink{after-top}{Continue reading the main story}

Supported by

\protect\hyperlink{after-sponsor}{Continue reading the main story}

\hypertarget{as-coronavirus-fears-intensify-effectiveness-of-quarantines-is-questioned}{%
\section{As Coronavirus Fears Intensify, Effectiveness of Quarantines Is
Questioned}\label{as-coronavirus-fears-intensify-effectiveness-of-quarantines-is-questioned}}

Amid news the coronavirus is spreading at an accelerating rate, concern
is growing that China's lockdown of cities may not only have come too
late but could even make the situation worse.

\includegraphics{https://static01.nyt.com/images/2020/01/26/world/26china-virus1/26china-virus1-articleLarge.jpg?quality=75\&auto=webp\&disable=upscale}

By \href{https://www.nytimes.com/by/chris-buckley}{Chris Buckley},
\href{https://www.nytimes.com/by/raymond-zhong}{Raymond Zhong},
\href{https://www.nytimes.com/by/denise-grady}{Denise Grady} and
\href{https://www.nytimes.com/by/roni-caryn-rabin}{Roni Caryn Rabin}

\begin{itemize}
\item
  Published Jan. 26, 2020Updated July 3, 2020
\item
  \begin{itemize}
  \item
  \item
  \item
  \item
  \item
  \end{itemize}
\end{itemize}

WUHAN, China --- A top Chinese health official warned on Sunday that the
\href{https://www.nytimes.com/2020/01/27/world/asia/china-coronavirus.html}{spread
of the dangerous new coronavirus}, already extraordinarily rapid, is
accelerating further, deepening global fears about an illness that has
sickened more than 2,700 people worldwide and killed at least 80 people
in China.

The grim diagnosis came amid concerns that China's efforts to contain
the spread of the disease, despite a
\href{https://www.nytimes.com/2020/01/23/world/asia/china-coronavirus-outbreak.html}{lockdown
of unprecedented scope} affecting 56 million people, may not only have
come too late but could even make the situation worse, including by
exacerbating shortages of medical supplies.

Adding to the
\href{https://www.nytimes.com/2020/02/19/world/asia/china-coronavirus.html}{growing
global alarm}, people who are carrying the virus but not showing
symptoms may still be able to infect others, according to the Chinese
official, Ma Xiaowei, the director of China's National Health
Commission. Such asymptomatic transmissions would make the disease much
more difficult to control, as seemingly healthy people travel and
interact with others.

``The epidemic is now entering a more serious and complex period,'' Mr.
Ma said during a Sunday news conference in Beijing. ``It looks like it
will continue for some time, and the number of cases may increase.''

China's attempts to curb the disease's spread --- essentially cordoning
off the major cities in the province of
\href{https://www.nytimes.com/2020/07/03/world/asia/china-floods-rain.html}{Hubei},
including its capital, Wuhan, a city of 11 million people --- are a
``public health experiment, the scale of which has not been done
before,'' said Dr. William Schaffner, an infectious disease expert at
Vanderbilt University. ``Logistically, it's stunning, and it was done so
quickly.''

Whether the lockdowns will succeed in stemming the spread of the virus
is a matter of debate by experts in public health and epidemiology. Some
said the lockdowns would help, at least in theory.

``Anything that is done that increases social distancing can help
decrease the spread of the virus,'' said Dr. Thomas R. Frieden, a former
director of the Centers for Disease Control and Prevention. ``If you do
it right, it's not impossible it will have positive impact.''

But doing it right at this scale has never been tried before anywhere in
the world.

``To put a ring around cities of this size and population is
unprecedented,'' said Dr. Howard Markel, a professor of the history of
medicine at the University of Michigan and author of the book
``Quarantine.''

Maintaining the lockdown will pose tremendous challenges, starting with
the provision of food, fuel and medical care to millions of people.
``It's enormously difficult to do effectively, and also difficult to
assess the effectiveness,'' said Dr. Schaffner.

\includegraphics{https://static01.nyt.com/images/2020/01/26/world/26china-virus6/merlin_167781783_40ce385e-53e4-46a8-ac0e-21ada49c5f8d-articleLarge.jpg?quality=75\&auto=webp\&disable=upscale}

Other experts were skeptical that the travel restrictions would prove at
all effective because they had probably come too late and the barriers
would prove too permeable. Five million people had left Wuhan before
travel out of the city was restricted, said the city's mayor, Zhou
Xianwang. It was a stunning disclosure that intensified questions about
the government's delayed response.

``You can't board up a germ. A novel infection will spread,'' said
Lawrence O. Gostin, a law professor at Georgetown University and
director of the World Health Organization Collaborating Center on
National and Global Health Law. ``It will get out; it always does.''

\hypertarget{latest-updates-global-coronavirus-outbreak}{%
\section{\texorpdfstring{\href{https://www.nytimes.com/2020/08/01/world/coronavirus-covid-19.html?action=click\&pgtype=Article\&state=default\&region=MAIN_CONTENT_1\&context=storylines_live_updates}{Latest
Updates: Global Coronavirus
Outbreak}}{Latest Updates: Global Coronavirus Outbreak}}\label{latest-updates-global-coronavirus-outbreak}}

Updated 2020-08-02T17:52:35.962Z

\begin{itemize}
\tightlist
\item
  \href{https://www.nytimes.com/2020/08/01/world/coronavirus-covid-19.html?action=click\&pgtype=Article\&state=default\&region=MAIN_CONTENT_1\&context=storylines_live_updates\#link-34047410}{The
  U.S. reels as July cases more than double the total of any other
  month.}
\item
  \href{https://www.nytimes.com/2020/08/01/world/coronavirus-covid-19.html?action=click\&pgtype=Article\&state=default\&region=MAIN_CONTENT_1\&context=storylines_live_updates\#link-780ec966}{Top
  U.S. officials work to break an impasse over the federal jobless
  benefit.}
\item
  \href{https://www.nytimes.com/2020/08/01/world/coronavirus-covid-19.html?action=click\&pgtype=Article\&state=default\&region=MAIN_CONTENT_1\&context=storylines_live_updates\#link-2bc8948}{Its
  outbreak untamed, Melbourne goes into even greater lockdown.}
\end{itemize}

\href{https://www.nytimes.com/2020/08/01/world/coronavirus-covid-19.html?action=click\&pgtype=Article\&state=default\&region=MAIN_CONTENT_1\&context=storylines_live_updates}{See
more updates}

More live coverage:
\href{https://www.nytimes.com/live/2020/07/31/business/stock-market-today-coronavirus?action=click\&pgtype=Article\&state=default\&region=MAIN_CONTENT_1\&context=storylines_live_updates}{Markets}

In China, it was a weekend of grim new warnings about the
little-understood virus and a rising tally of infections and deaths. The
official number of confirmed infections across China jumped
significantly within a span of 24 hours, building to 2,744 by Monday
from around 1,975 the day before.

\href{https://www.nytimes.com/interactive/2020/01/21/world/asia/china-coronavirus-maps.html}{}

\includegraphics{https://static01.nyt.com/images/2020/01/31/us/china-wuhan-coronavirus-promo-1579641872730/china-wuhan-coronavirus-promo-1579641872730-articleLarge-v21.jpg}

\hypertarget{wuhan-coronavirus-map-tracking-the-spread-of-the-outbreak}{%
\subsection{Wuhan Coronavirus Map: Tracking the Spread of the
Outbreak}\label{wuhan-coronavirus-map-tracking-the-spread-of-the-outbreak}}

The virus has sickened tens of thousands of people in China and a number
of other countries.

Among the most recent announced fatalities from the coronavirus was an
88-year-old man in Shanghai --- the first death to be reported in the
commercial hub, and one likely to fuel anxieties about the disease's
spread.

New cases cropped up in Hong Kong, Taiwan and Orange County, Calif.,
bringing to five the number of confirmed cases in the United States. The
virus had already been found in Thailand, France, Japan, South Korea,
Australia and beyond.

\href{https://www.imperial.ac.uk/mrc-global-infectious-disease-analysis/news--wuhan-coronavirus/}{Epidemiologists
at Imperial College London}estimated that each case infected an average
of between 1.5 and 3.5 other people in the early stages of the crisis.
For seasonal flu, it's about 1.3.

That number could drop as the authorities take more stringent measures
to halt the spread. But if it holds up, the number of infected could
rise sharply.

In Wuhan, the city at the center of the outbreak, the streets were
eerily quiet as the authorities had ordered people not to drive, forcing
some to walk to hospitals. Mr. Zhou, the mayor, said that health
officials were likely to confirm an additional 1,000 cases of the
illness in the city. He said that the estimate was based on the
assumption that around half of the city's nearly 3,000 suspected cases
of the coronavirus would eventually test positive.

China's top leader, Xi Jinping, has promised drastic measures to contain
the virus.

In a signal of the gravity of the crisis, and its likely disruption to
China's short-term growth, the
\href{https://news.sina.com.cn/c/2020-01-27/doc-iihnzahk6491235.shtml}{government
announced on Monday} that the annual weeklong Lunar New Year Holiday
would be extended. For now, at least, many workers will get another
three days off, and go back to work on Feb. 3.

Even before that notice, Suzhou, a big manufacturing hub in eastern
China, declared that factories there should not start back at work any
earlier than Feb. 8.

The national government on Sunday also banned the wildlife trade until
the epidemic passes. The outbreak had drawn fresh attention
\href{https://www.nytimes.com/2020/01/25/world/asia/china-markets-coronavirus-sars.html}{to
China's animal markets}, where the sale of exotic creatures has been
linked to epidemiological risks.

In Hong Kong --- which was badly hit by the SARS coronavirus in 2003,
with nearly 300 deaths, more than any city in the world --- worries
about the spread of infectious diseases run deep. On Sunday, the
government said it would bar residents of Hubei Province, which includes
Wuhan, and people who had been to the province in the past 14 days from
entering Hong Kong until further notice.

Image

A nearly empty street, normally busy with tourists, in Beijing on
Sunday.Credit...Kevin Frayer/Getty Images

Six cases of the new coronavirus have been confirmed in the city,
already hobbled by months of antigovernment protests.

Health officials in the United States, in what could turn out to be a
positive development in stemming the disease, said there was no ``clear
evidence'' that asymptomatic transmissions of the disease were
happening.

``We at the Centers for Disease Control and Prevention don't have any
clear evidence of patients' being infectious before symptom onset,'' Dr.
Nancy Messonnier of the National Center for Immunization and Respiratory
Diseases said at a news briefing on Sunday. ``We are actively
investigating that possibility.''

Some global health experts said China's focus, and resources, going
forward should not be devoted to closing off cities.

Michael T. Osterholm, director of the Center for Infectious Disease
Research and Policy at the University of Minnesota, thought China's
approach to the crisis could easily ``backfire,'' comparing it to the
so-called cordons sanitaires that were imposed to seal off swaths of
West Africa during the 2014-2016 Ebola epidemic. Those cordons left
people starving and spurred violent uprisings. Others routinely found
ways to sneak around or through the boundaries.

``It was a disaster,'' Dr. Osterholm said.

Dr. Tom Inglesby, an infectious diseases specialist and director of the
Johns Hopkins Center for Health Security, also expressed concern.

\href{https://www.nytimes.com/news-event/coronavirus?action=click\&pgtype=Article\&state=default\&region=MAIN_CONTENT_3\&context=storylines_faq}{}

\hypertarget{the-coronavirus-outbreak-}{%
\subsubsection{The Coronavirus Outbreak
›}\label{the-coronavirus-outbreak-}}

\hypertarget{frequently-asked-questions}{%
\paragraph{Frequently Asked
Questions}\label{frequently-asked-questions}}

Updated July 27, 2020

\begin{itemize}
\item ~
  \hypertarget{should-i-refinance-my-mortgage}{%
  \paragraph{Should I refinance my
  mortgage?}\label{should-i-refinance-my-mortgage}}

  \begin{itemize}
  \tightlist
  \item
    \href{https://www.nytimes.com/article/coronavirus-money-unemployment.html?action=click\&pgtype=Article\&state=default\&region=MAIN_CONTENT_3\&context=storylines_faq}{It
    could be a good idea,} because mortgage rates have
    \href{https://www.nytimes.com/2020/07/16/business/mortgage-rates-below-3-percent.html?action=click\&pgtype=Article\&state=default\&region=MAIN_CONTENT_3\&context=storylines_faq}{never
    been lower.} Refinancing requests have pushed mortgage applications
    to some of the highest levels since 2008, so be prepared to get in
    line. But defaults are also up, so if you're thinking about buying a
    home, be aware that some lenders have tightened their standards.
  \end{itemize}
\item ~
  \hypertarget{what-is-school-going-to-look-like-in-september}{%
  \paragraph{What is school going to look like in
  September?}\label{what-is-school-going-to-look-like-in-september}}

  \begin{itemize}
  \tightlist
  \item
    It is unlikely that many schools will return to a normal schedule
    this fall, requiring the grind of
    \href{https://www.nytimes.com/2020/06/05/us/coronavirus-education-lost-learning.html?action=click\&pgtype=Article\&state=default\&region=MAIN_CONTENT_3\&context=storylines_faq}{online
    learning},
    \href{https://www.nytimes.com/2020/05/29/us/coronavirus-child-care-centers.html?action=click\&pgtype=Article\&state=default\&region=MAIN_CONTENT_3\&context=storylines_faq}{makeshift
    child care} and
    \href{https://www.nytimes.com/2020/06/03/business/economy/coronavirus-working-women.html?action=click\&pgtype=Article\&state=default\&region=MAIN_CONTENT_3\&context=storylines_faq}{stunted
    workdays} to continue. California's two largest public school
    districts --- Los Angeles and San Diego --- said on July 13, that
    \href{https://www.nytimes.com/2020/07/13/us/lausd-san-diego-school-reopening.html?action=click\&pgtype=Article\&state=default\&region=MAIN_CONTENT_3\&context=storylines_faq}{instruction
    will be remote-only in the fall}, citing concerns that surging
    coronavirus infections in their areas pose too dire a risk for
    students and teachers. Together, the two districts enroll some
    825,000 students. They are the largest in the country so far to
    abandon plans for even a partial physical return to classrooms when
    they reopen in August. For other districts, the solution won't be an
    all-or-nothing approach.
    \href{https://bioethics.jhu.edu/research-and-outreach/projects/eschool-initiative/school-policy-tracker/}{Many
    systems}, including the nation's largest, New York City, are
    devising
    \href{https://www.nytimes.com/2020/06/26/us/coronavirus-schools-reopen-fall.html?action=click\&pgtype=Article\&state=default\&region=MAIN_CONTENT_3\&context=storylines_faq}{hybrid
    plans} that involve spending some days in classrooms and other days
    online. There's no national policy on this yet, so check with your
    municipal school system regularly to see what is happening in your
    community.
  \end{itemize}
\item ~
  \hypertarget{is-the-coronavirus-airborne}{%
  \paragraph{Is the coronavirus
  airborne?}\label{is-the-coronavirus-airborne}}

  \begin{itemize}
  \tightlist
  \item
    The coronavirus
    \href{https://www.nytimes.com/2020/07/04/health/239-experts-with-one-big-claim-the-coronavirus-is-airborne.html?action=click\&pgtype=Article\&state=default\&region=MAIN_CONTENT_3\&context=storylines_faq}{can
    stay aloft for hours in tiny droplets in stagnant air}, infecting
    people as they inhale, mounting scientific evidence suggests. This
    risk is highest in crowded indoor spaces with poor ventilation, and
    may help explain super-spreading events reported in meatpacking
    plants, churches and restaurants.
    \href{https://www.nytimes.com/2020/07/06/health/coronavirus-airborne-aerosols.html?action=click\&pgtype=Article\&state=default\&region=MAIN_CONTENT_3\&context=storylines_faq}{It's
    unclear how often the virus is spread} via these tiny droplets, or
    aerosols, compared with larger droplets that are expelled when a
    sick person coughs or sneezes, or transmitted through contact with
    contaminated surfaces, said Linsey Marr, an aerosol expert at
    Virginia Tech. Aerosols are released even when a person without
    symptoms exhales, talks or sings, according to Dr. Marr and more
    than 200 other experts, who
    \href{https://academic.oup.com/cid/article/doi/10.1093/cid/ciaa939/5867798}{have
    outlined the evidence in an open letter to the World Health
    Organization}.
  \end{itemize}
\item ~
  \hypertarget{what-are-the-symptoms-of-coronavirus}{%
  \paragraph{What are the symptoms of
  coronavirus?}\label{what-are-the-symptoms-of-coronavirus}}

  \begin{itemize}
  \tightlist
  \item
    Common symptoms
    \href{https://www.nytimes.com/article/symptoms-coronavirus.html?action=click\&pgtype=Article\&state=default\&region=MAIN_CONTENT_3\&context=storylines_faq}{include
    fever, a dry cough, fatigue and difficulty breathing or shortness of
    breath.} Some of these symptoms overlap with those of the flu,
    making detection difficult, but runny noses and stuffy sinuses are
    less common.
    \href{https://www.nytimes.com/2020/04/27/health/coronavirus-symptoms-cdc.html?action=click\&pgtype=Article\&state=default\&region=MAIN_CONTENT_3\&context=storylines_faq}{The
    C.D.C. has also} added chills, muscle pain, sore throat, headache
    and a new loss of the sense of taste or smell as symptoms to look
    out for. Most people fall ill five to seven days after exposure, but
    symptoms may appear in as few as two days or as many as 14 days.
  \end{itemize}
\item ~
  \hypertarget{does-asymptomatic-transmission-of-covid-19-happen}{%
  \paragraph{Does asymptomatic transmission of Covid-19
  happen?}\label{does-asymptomatic-transmission-of-covid-19-happen}}

  \begin{itemize}
  \tightlist
  \item
    So far, the evidence seems to show it does. A widely cited
    \href{https://www.nature.com/articles/s41591-020-0869-5}{paper}
    published in April suggests that people are most infectious about
    two days before the onset of coronavirus symptoms and estimated that
    44 percent of new infections were a result of transmission from
    people who were not yet showing symptoms. Recently, a top expert at
    the World Health Organization stated that transmission of the
    coronavirus by people who did not have symptoms was ``very rare,''
    \href{https://www.nytimes.com/2020/06/09/world/coronavirus-updates.html?action=click\&pgtype=Article\&state=default\&region=MAIN_CONTENT_3\&context=storylines_faq\#link-1f302e21}{but
    she later walked back that statement.}
  \end{itemize}
\end{itemize}

``If you continue to quarantine more and more places in China, you're
going to start to really break normal societal interaction, normal
movement of goods and people and medical supplies and food and
medicine,'' Dr. Inglesby said. ``At a macro level, it seems to me that
it's more likely to be harmful than helpful in controlling the
epidemic.''

Instead, Dr. Inglesby, Dr. Osterholm and other health experts suggested
China should concentrate on traditional public health measures that have
stopped other outbreaks, like identifying and monitoring contacts and
making sure medical care is available to everyone.

Even as the highest echelons of China's government mobilize to fight the
illness, much of the task of preventing contagion still falls on local
officials, who can be unsure of how to respond to crises and uneven
about following through on policies.

On Sunday in Wuhan, for example, police officers were flummoxed by new
restrictions on driving within the city limits.

First, the city authorities said that most cars
\href{https://mp.weixin.qq.com/s?src=11\&timestamp=1580035588\&ver=2120\&signature=SxuKbIMYjrZ4tw2dH2Sf-41vvxx9SlwFgJfynxcr6wjvcfvE4FyGoNh50zGorYJ7JeaFWmsNZELIGwOekOTIi98jPNkDdjwEKebM*JE3TAIdS17Ax5-oi0gQ14Fa6mYW\&new=1}{should
stay off the roads}, and that a fleet of 6,000 taxis
\href{http://www.chinanews.com/gn/2020/01-26/9069778.shtml}{would be on
call} to deliver food and medicine. Then, the authorities said drivers
would be
\href{https://baijiahao.baidu.com/s?id=1656737257191494058\&wfr=spider\&for=pc}{notified
by text message} if they had to stay off the roads. Nobody seemed to
receive the text messages on Sunday.

``My understanding,'' one police officer said, ``is that you can drive
in your district if you don't get a text message telling you that you
can't. But you should check that with the transport authorities.''

In the end, most drivers stayed off the streets. But as the day went on,
more ventured out, and the police did not seem to do much about it.

For some residents, it was another exasperating fumble by Wuhan
officials, who many believe have mishandled the epidemic.

Image

A street in Wuhan on Sunday. The government announced a ban on
non-essential vehicles in the downtown area to try to contain the
coronavirus.Credit...China News Service, via Reuters

Health experts said the government's ability to keep the trust of the
public was a key element in any successful quarantine, and never easy to
do.

Dr. Inglesby said that previous, much smaller scale lockdown efforts ---
including closing off the Amoy Gardens housing complex in Hong Kong
during the SARS outbreak 17 years ago --- show that residents may become
fearful and lose confidence in the government.

``You need people to willingly present themselves for diagnosis,'' he
said. ``If they don't understand what the government's doing or they
feel in some way their bond with the government has been broken, that's
another key process that's being interrupted by the quarantine.''

For now, in Wuhan, the restrictions seem to be mostly accepted with the
same stoic fortitude that many residents showed over the past several
days as the city imposed bans on outbound travel for all but a select
few.

That mood could shift, however, if, for example, food prices rise.

``Now is not the time for recriminations,'' said Li Xiandu, a retired
business manager. ``The local government wasn't forthcoming with
information and didn't take vigorous enough measures. But we need to get
through this first, and then we can assign blame.''

While the government has pledged to build at least two new hospitals
with thousands of beds in Wuhan --- and to do so
in\href{https://www.nytimes.com/2020/01/25/world/asia/china-coronavirus.html}{just
a few days} --- the city's existing hospitals remain intensely crowded,
a condition that does not bode well for stopping the disease.

``If you wanted to create the perfect mixing vessel for a coronavirus,''
Dr. Osterholm said, ``you'd create the emergency rooms in Wuhan right
now.''

Chris Buckley reported from Wuhan, China, Raymond Zhong from Shanghai,
and Denise Grady and Roni Caryn Rabin from New York. Sheri Fink
contributed reporting from New York, and Claire Fu and Wang Yiwei
contributed research.

Advertisement

\protect\hyperlink{after-bottom}{Continue reading the main story}

\hypertarget{site-index}{%
\subsection{Site Index}\label{site-index}}

\hypertarget{site-information-navigation}{%
\subsection{Site Information
Navigation}\label{site-information-navigation}}

\begin{itemize}
\tightlist
\item
  \href{https://help.nytimes.com/hc/en-us/articles/115014792127-Copyright-notice}{©~2020~The
  New York Times Company}
\end{itemize}

\begin{itemize}
\tightlist
\item
  \href{https://www.nytco.com/}{NYTCo}
\item
  \href{https://help.nytimes.com/hc/en-us/articles/115015385887-Contact-Us}{Contact
  Us}
\item
  \href{https://www.nytco.com/careers/}{Work with us}
\item
  \href{https://nytmediakit.com/}{Advertise}
\item
  \href{http://www.tbrandstudio.com/}{T Brand Studio}
\item
  \href{https://www.nytimes.com/privacy/cookie-policy\#how-do-i-manage-trackers}{Your
  Ad Choices}
\item
  \href{https://www.nytimes.com/privacy}{Privacy}
\item
  \href{https://help.nytimes.com/hc/en-us/articles/115014893428-Terms-of-service}{Terms
  of Service}
\item
  \href{https://help.nytimes.com/hc/en-us/articles/115014893968-Terms-of-sale}{Terms
  of Sale}
\item
  \href{https://spiderbites.nytimes.com}{Site Map}
\item
  \href{https://help.nytimes.com/hc/en-us}{Help}
\item
  \href{https://www.nytimes.com/subscription?campaignId=37WXW}{Subscriptions}
\end{itemize}
