Sections

SEARCH

\protect\hyperlink{site-content}{Skip to
content}\protect\hyperlink{site-index}{Skip to site index}

\href{https://www.nytimes.com/section/world/asia}{Asia Pacific}

\href{https://myaccount.nytimes.com/auth/login?response_type=cookie\&client_id=vi}{}

\href{https://www.nytimes.com/section/todayspaper}{Today's Paper}

\href{/section/world/asia}{Asia Pacific}\textbar{}India Restores Some
Internet Access in Kashmir After Long Shutdown

\href{https://nyti.ms/36qZm4P}{https://nyti.ms/36qZm4P}

\begin{itemize}
\item
\item
\item
\item
\item
\end{itemize}

Advertisement

\protect\hyperlink{after-top}{Continue reading the main story}

Supported by

\protect\hyperlink{after-sponsor}{Continue reading the main story}

\hypertarget{india-restores-some-internet-access-in-kashmir-after-long-shutdown}{%
\section{India Restores Some Internet Access in Kashmir After Long
Shutdown}\label{india-restores-some-internet-access-in-kashmir-after-long-shutdown}}

The announcement applied only to 301 websites, and many Kashmiris said
they were still in an information black hole.

\includegraphics{https://static01.nyt.com/images/2020/01/26/world/26kashmir/merlin_167831301_c7d95d5b-6cdd-43df-acce-360966649f77-articleLarge.jpg?quality=75\&auto=webp\&disable=upscale}

\href{https://www.nytimes.com/by/kai-schultz}{\includegraphics{https://static01.nyt.com/images/2019/11/22/reader-center/author-kai-schultz/author-kai-schultz-thumbLarge.png}}\href{https://www.nytimes.com/by/sameer-yasir}{\includegraphics{https://static01.nyt.com/images/2019/11/22/reader-center/author-sameer-yasir/author-sameer-yasir-thumbLarge.png}}

By \href{https://www.nytimes.com/by/kai-schultz}{Kai Schultz} and
\href{https://www.nytimes.com/by/sameer-yasir}{Sameer Yasir}

\begin{itemize}
\item
  Jan. 26, 2020
\item
  \begin{itemize}
  \item
  \item
  \item
  \item
  \item
  \end{itemize}
\end{itemize}

NEW DELHI --- Months after imposing a sweeping communications blackout
in Kashmir, the Indian government on Saturday
\href{https://thewire.in/government/2g-data-services-to-be-restored-throughout-jammu-kashmir-union-territory}{unblocked
several hundred websites} in the disputed Himalayan region, bringing a
tentative end to the world's longest internet shutdown in a democracy.

The announcement comes nearly half a year after India's government, led
by Prime Minister Narendra Modi,
\href{https://www.nytimes.com/2019/08/05/world/asia/india-pakistan-kashmir-jammu.html}{revoked
Kashmir's semiautonomous status}. Bracing for subsequent unrest, the
authorities suspended landline and cellular connections, cut internet
services and dispatched tens of thousands of troops to the area.

Since then, the Kashmir Valley, home to as many as eight million people,
has endured a punishing information blackout.

Foreign journalists and diplomats were blocked from visiting the
predominantly Muslim territory, which is contested between India and
Pakistan. The government arrested scores of Kashmiris, including former
heads of state,
\href{https://www.nytimes.com/2019/08/23/world/asia/kashmir-arrests-india.html}{without
disclosing charges}. Ordinary Kashmiris struggled
\href{https://www.nytimes.com/2019/10/07/world/asia/kashmir-doctors-phone.html}{to
procure medicines} and contact loved ones.

The lifting of restrictions on Saturday applied only to 301
``whitelisted'' websites. Among them were entertainment platforms like
Netflix and Amazon and some international news outlets, including The
New York Times. Many Indian publications remained blocked, along with
all social media. Mobile data access was also restored, though it was
limited to 2G connections.

``It is very slow --- and a good joke,'' said Sajeel Majid, 35, a
restaurant owner in Srinagar, the summer capital of Kashmir. ``India
wants to deceive the world by saying we have restored internet, but we
can't even access email with 2G speed.''

Though some Kashmiris said the partial restoration of internet services
could bring some semblance of normalcy to the region, they pointed out
that shops remained largely shut and troops were still posted
everywhere. Over the last week, around half a dozen Kashmiri militants
were killed in gun battles with Indian forces, who have been accused of
torturing civilians and using excessive force against protesters.

\href{https://twitter.com/NEETAS11/status/1220765770876981248/photo/1}{In
a statement}, the government of Jammu and Kashmir said continued
internet restrictions were necessary to prevent the ``propagation of
terror activities'' and the ``circulation of inflammatory material.''
Officials said they would approve more websites in the coming days.

India has increasingly come under scrutiny, both domestically and
abroad, for cutting off the internet, a tactic more commonly associated
with dictatorships than democracies. The country
\href{https://www.nytimes.com/2019/12/17/world/asia/india-internet-modi-protests.html}{tops
the world} in the number of internet shutdowns, with 134 last year,
according to SFLC.in, a legal advocacy group in New Delhi that
\href{https://internetshutdowns.in/}{tracks such restrictions}.

This month, the Supreme Court ruled that internet access was integral to
an individual's right to freedom of speech and expression. Judges said
the government's methods in Kashmir were an ``arbitrary exercise of
power,'' though they stopped short of declaring them illegal.

Hours after the government lifted some restrictions, Kashmiris said web
pages --- including approved ones --- were
\href{https://www.thehindu.com/news/national/low-speed-mobile-internet-in-kashmir-temporarily-suspended-to-resume-after-republic-day-celebrations/article30654435.ece}{again
blocked} for Republic Day festivities on Sunday, which commemorate the
day when India's Constitution went into effect.

``It is just a game to tell people we have restored internet services,
but on the ground it doesn't work and is of no use,'' said Adnan Bhat,
19, a student in Srinagar.

Advertisement

\protect\hyperlink{after-bottom}{Continue reading the main story}

\hypertarget{site-index}{%
\subsection{Site Index}\label{site-index}}

\hypertarget{site-information-navigation}{%
\subsection{Site Information
Navigation}\label{site-information-navigation}}

\begin{itemize}
\tightlist
\item
  \href{https://help.nytimes.com/hc/en-us/articles/115014792127-Copyright-notice}{©~2020~The
  New York Times Company}
\end{itemize}

\begin{itemize}
\tightlist
\item
  \href{https://www.nytco.com/}{NYTCo}
\item
  \href{https://help.nytimes.com/hc/en-us/articles/115015385887-Contact-Us}{Contact
  Us}
\item
  \href{https://www.nytco.com/careers/}{Work with us}
\item
  \href{https://nytmediakit.com/}{Advertise}
\item
  \href{http://www.tbrandstudio.com/}{T Brand Studio}
\item
  \href{https://www.nytimes.com/privacy/cookie-policy\#how-do-i-manage-trackers}{Your
  Ad Choices}
\item
  \href{https://www.nytimes.com/privacy}{Privacy}
\item
  \href{https://help.nytimes.com/hc/en-us/articles/115014893428-Terms-of-service}{Terms
  of Service}
\item
  \href{https://help.nytimes.com/hc/en-us/articles/115014893968-Terms-of-sale}{Terms
  of Sale}
\item
  \href{https://spiderbites.nytimes.com}{Site Map}
\item
  \href{https://help.nytimes.com/hc/en-us}{Help}
\item
  \href{https://www.nytimes.com/subscription?campaignId=37WXW}{Subscriptions}
\end{itemize}
