Sections

SEARCH

\protect\hyperlink{site-content}{Skip to
content}\protect\hyperlink{site-index}{Skip to site index}

\href{https://www.nytimes.com/section/health}{Health}

\href{https://myaccount.nytimes.com/auth/login?response_type=cookie\&client_id=vi}{}

\href{https://www.nytimes.com/section/todayspaper}{Today's Paper}

\href{/section/health}{Health}\textbar{}Researchers Are Racing to Make a
Coronavirus Vaccine. Will It Help?

\url{https://nyti.ms/2RAOsp9}

\begin{itemize}
\item
\item
\item
\item
\item
\item
\end{itemize}

\href{https://www.nytimes.com/news-event/coronavirus?action=click\&pgtype=Article\&state=default\&region=TOP_BANNER\&context=storylines_menu}{The
Coronavirus Outbreak}

\begin{itemize}
\tightlist
\item
  live\href{https://www.nytimes.com/2020/08/02/world/coronavirus-updates.html?action=click\&pgtype=Article\&state=default\&region=TOP_BANNER\&context=storylines_menu}{Latest
  Updates}
\item
  \href{https://www.nytimes.com/interactive/2020/us/coronavirus-us-cases.html?action=click\&pgtype=Article\&state=default\&region=TOP_BANNER\&context=storylines_menu}{Maps
  and Cases}
\item
  \href{https://www.nytimes.com/interactive/2020/science/coronavirus-vaccine-tracker.html?action=click\&pgtype=Article\&state=default\&region=TOP_BANNER\&context=storylines_menu}{Vaccine
  Tracker}
\item
  \href{https://www.nytimes.com/interactive/2020/07/29/us/schools-reopening-coronavirus.html?action=click\&pgtype=Article\&state=default\&region=TOP_BANNER\&context=storylines_menu}{What
  School May Look Like}
\item
  \href{https://www.nytimes.com/live/2020/07/31/business/stock-market-today-coronavirus?action=click\&pgtype=Article\&state=default\&region=TOP_BANNER\&context=storylines_menu}{Economy}
\end{itemize}

Advertisement

\protect\hyperlink{after-top}{Continue reading the main story}

Supported by

\protect\hyperlink{after-sponsor}{Continue reading the main story}

\hypertarget{researchers-are-racing-to-make-a-coronavirus-vaccine-will-it-help}{%
\section{Researchers Are Racing to Make a Coronavirus Vaccine. Will It
Help?}\label{researchers-are-racing-to-make-a-coronavirus-vaccine-will-it-help}}

New technology and better coordination have sped up development. But a
coronavirus vaccine is still months --- and most likely years --- away.

\includegraphics{https://static01.nyt.com/images/2020/01/28/science/28VIRUS-VACCINE1/28VIRUS-VACCINE1-articleLarge.jpg?quality=75\&auto=webp\&disable=upscale}

By \href{https://www.nytimes.com/by/knvul-sheikh}{Knvul Sheikh} and
\href{https://www.nytimes.com/by/katie-thomas}{Katie Thomas}

\begin{itemize}
\item
  Published Jan. 28, 2020Updated June 10, 2020
\item
  \begin{itemize}
  \item
  \item
  \item
  \item
  \item
  \item
  \end{itemize}
\end{itemize}

In the early days of January, as cases of a strange, pneumonia-like
illness were reported in China, researchers at the National Institutes
of Health in Maryland readied themselves to hunt for a
\href{https://www.nytimes.com/2020/04/27/world/europe/coronavirus-vaccine-update-oxford.html}{vaccine}
to prevent the new disease.

They had clues that a
\href{https://www.nytimes.com/2020/04/27/world/europe/coronavirus-vaccine-update-oxford.html}{coronavirus},
similar to ones that caused the SARS outbreak in 2003 and MERS in 2012,
was the culprit. Dr. Barney Graham, deputy director of the Vaccine
Research Center at the N.I.H, urged government scientists in China to
share the genetic makeup of the virus so his team could begin its race
to develop a vaccine.

On Friday, Jan. 10, the Chinese scientists posted the information on
\href{https://www.ncbi.nlm.nih.gov/nuccore/MN908947}{a public database}.
The next morning, Dr. Graham's team was in the lab. And within hours,
they had pinpointed the letters of the genetic code that could be used
to make a vaccine.

Scientists in Australia and at least three companies --- Johnson \&
Johnson,
\href{https://www.nytimes.com/2020/05/18/health/coronavirus-vaccine-moderna.html}{Moderna
Therapeutics} and Inovio Pharmaceuticals --- are also working on vaccine
candidates to stop the spread of the disease, which has infected about
6,000 people and killed more than 130.

``Everybody is trying to move as quickly as possible,'' said Jacqueline
Shea, the chief operating officer at Inovio.

Inovio received a grant of up to \$9 million to develop a coronavirus
vaccine from the Coalition for Epidemic Preparedness Innovations, a
group whose aim is to speed vaccines to market. Moderna, which is
working with Dr. Graham's team at the N.I.H.,
\href{https://cepi.net/news_cepi/cepi-to-fund-three-programmes-to-develop-vaccines-against-the-novel-coronavirus-ncov-2019/}{received
a similar grant}, as did researchers at the University of Queensland in
Australia.

Historically, vaccines have been one of the greatest public health tools
to prevent disease. But even as new technology, advancements in genomics
and improved global coordination have allowed researchers to move at
unprecedented speed, vaccine development remains an expensive and risky
process. It takes months and even years because the vaccines must
undergo extensive testing in animals and humans. In the best case, it
takes at least a year --- and most likely longer --- for any vaccine to
become available to the public.

``They may not help in the very early stages of an outbreak, but if
we're able to develop vaccines in time, they will be an asset later,''
said Richard Hatchett, the chief executive of the epidemic preparedness
coalition.

\emph{{[}}\href{https://www.nytimes.com/interactive/2020/science/coronavirus-vaccine-tracker.html}{\emph{Follow
our Live Coronavirus Vaccine Tracker}}\emph{.{]}}

\includegraphics{https://static01.nyt.com/images/2020/01/28/science/28VIRUS-VACCINE2/28VIRUS-VACCINE2-articleLarge.jpg?quality=75\&auto=webp\&disable=upscale}

With each new outbreak, scientists typically have to start from scratch.
After the SARS outbreak in 2003, it took researchers about 20 months
from the release of the viral genome to get a vaccine ready for human
trials. By the time an epidemic caused by the Zika virus occurred in
2015, researchers had brought the timeline down to six months. Now, they
hope the joint efforts will cut that time in half.

\hypertarget{latest-updates-global-coronavirus-outbreak}{%
\section{\texorpdfstring{\href{https://www.nytimes.com/2020/08/01/world/coronavirus-covid-19.html?action=click\&pgtype=Article\&state=default\&region=MAIN_CONTENT_1\&context=storylines_live_updates}{Latest
Updates: Global Coronavirus
Outbreak}}{Latest Updates: Global Coronavirus Outbreak}}\label{latest-updates-global-coronavirus-outbreak}}

Updated 2020-08-02T17:52:35.962Z

\begin{itemize}
\tightlist
\item
  \href{https://www.nytimes.com/2020/08/01/world/coronavirus-covid-19.html?action=click\&pgtype=Article\&state=default\&region=MAIN_CONTENT_1\&context=storylines_live_updates\#link-34047410}{The
  U.S. reels as July cases more than double the total of any other
  month.}
\item
  \href{https://www.nytimes.com/2020/08/01/world/coronavirus-covid-19.html?action=click\&pgtype=Article\&state=default\&region=MAIN_CONTENT_1\&context=storylines_live_updates\#link-780ec966}{Top
  U.S. officials work to break an impasse over the federal jobless
  benefit.}
\item
  \href{https://www.nytimes.com/2020/08/01/world/coronavirus-covid-19.html?action=click\&pgtype=Article\&state=default\&region=MAIN_CONTENT_1\&context=storylines_live_updates\#link-2bc8948}{Its
  outbreak untamed, Melbourne goes into even greater lockdown.}
\end{itemize}

\href{https://www.nytimes.com/2020/08/01/world/coronavirus-covid-19.html?action=click\&pgtype=Article\&state=default\&region=MAIN_CONTENT_1\&context=storylines_live_updates}{See
more updates}

More live coverage:
\href{https://www.nytimes.com/live/2020/07/31/business/stock-market-today-coronavirus?action=click\&pgtype=Article\&state=default\&region=MAIN_CONTENT_1\&context=storylines_live_updates}{Markets}

The morning after the Chinese scientists published their data earlier
this month, Dr. Graham's team got to work checking the sequence and
comparing it with what they already had for SARS and MERS. They wanted
to focus on the spike protein, which forms the crown of the coronavirus
and recognizes receptors, or entry points, on a host cell.

``If you can block the spike protein from binding to a cell, then you've
effectively prevented an infection,'' said Kizzmekia Corbett, the
scientific lead for Dr. Graham's coronavirus team.

Dr. Corbett and others had studied the spike proteins on SARS and MERS
viruses in detail, using them to develop experimental vaccines. The
vaccines never made it to market because SARS was successfully contained
with public health measures before the vaccine was ready and
preliminary\href{https://www.thelancet.com/journals/laninf/article/PIIS1473-3099(19)30266-X/fulltext}{human
trials for the MERS vaccine showed success} last year.

But the scientists had a method for developing vaccines that could help
them fast-track production for the new coronavirus. They used the
template for the SARS vaccine and swapped in just enough genetic code
that would make it work for the new virus. ``I call it plug and play,''
Dr. Corbett said.

Within a few hours, Dr. Corbett was able to prepare the modified
sequence that the researchers needed. On Tuesday, Jan. 14, the team held
a conference call to discuss the next steps with collaborators in labs
across the country, and sent off the sequence to Moderna.

Scientists at the company plan to use the genetic information to create
synthetic messenger RNA, which carries instructions for cells'
protein-making machinery. The technology will help induce high levels of
antibodies that can identify the spike protein and fight off an
infection.

Once Moderna manufactures the messenger RNA in a few weeks, the N.I.H.
will run more tests, Dr. Corbett said. Collaborators in academic labs
will then test the vaccine in mice infected with the virus and check
blood samples from the animals to see how well the experimental vaccine
worked.

Image

Dr. Anthony Fauci, director of the National Institute of Allergy and
Infectious Diseases, during a coronavirus update by U.S. public health
officials on Tuesday.Credit...Amanda Voisard/Reuters

Dr. Anthony Fauci, director of the National Institute of Allergy and
Infectious Diseases at the N.I.H., who oversees Dr. Graham's team, said
he expected the vaccine research to move quickly.

``If we don't run into any unforeseen obstacles, we'll be able to get a
Phase 1 trial going within the next three months, which will be record
speed,'' he said, referring to early human trials that test for safety.

\href{https://www.nytimes.com/news-event/coronavirus?action=click\&pgtype=Article\&state=default\&region=MAIN_CONTENT_3\&context=storylines_faq}{}

\hypertarget{the-coronavirus-outbreak-}{%
\subsubsection{The Coronavirus Outbreak
›}\label{the-coronavirus-outbreak-}}

\hypertarget{frequently-asked-questions}{%
\paragraph{Frequently Asked
Questions}\label{frequently-asked-questions}}

Updated July 27, 2020

\begin{itemize}
\item ~
  \hypertarget{should-i-refinance-my-mortgage}{%
  \paragraph{Should I refinance my
  mortgage?}\label{should-i-refinance-my-mortgage}}

  \begin{itemize}
  \tightlist
  \item
    \href{https://www.nytimes.com/article/coronavirus-money-unemployment.html?action=click\&pgtype=Article\&state=default\&region=MAIN_CONTENT_3\&context=storylines_faq}{It
    could be a good idea,} because mortgage rates have
    \href{https://www.nytimes.com/2020/07/16/business/mortgage-rates-below-3-percent.html?action=click\&pgtype=Article\&state=default\&region=MAIN_CONTENT_3\&context=storylines_faq}{never
    been lower.} Refinancing requests have pushed mortgage applications
    to some of the highest levels since 2008, so be prepared to get in
    line. But defaults are also up, so if you're thinking about buying a
    home, be aware that some lenders have tightened their standards.
  \end{itemize}
\item ~
  \hypertarget{what-is-school-going-to-look-like-in-september}{%
  \paragraph{What is school going to look like in
  September?}\label{what-is-school-going-to-look-like-in-september}}

  \begin{itemize}
  \tightlist
  \item
    It is unlikely that many schools will return to a normal schedule
    this fall, requiring the grind of
    \href{https://www.nytimes.com/2020/06/05/us/coronavirus-education-lost-learning.html?action=click\&pgtype=Article\&state=default\&region=MAIN_CONTENT_3\&context=storylines_faq}{online
    learning},
    \href{https://www.nytimes.com/2020/05/29/us/coronavirus-child-care-centers.html?action=click\&pgtype=Article\&state=default\&region=MAIN_CONTENT_3\&context=storylines_faq}{makeshift
    child care} and
    \href{https://www.nytimes.com/2020/06/03/business/economy/coronavirus-working-women.html?action=click\&pgtype=Article\&state=default\&region=MAIN_CONTENT_3\&context=storylines_faq}{stunted
    workdays} to continue. California's two largest public school
    districts --- Los Angeles and San Diego --- said on July 13, that
    \href{https://www.nytimes.com/2020/07/13/us/lausd-san-diego-school-reopening.html?action=click\&pgtype=Article\&state=default\&region=MAIN_CONTENT_3\&context=storylines_faq}{instruction
    will be remote-only in the fall}, citing concerns that surging
    coronavirus infections in their areas pose too dire a risk for
    students and teachers. Together, the two districts enroll some
    825,000 students. They are the largest in the country so far to
    abandon plans for even a partial physical return to classrooms when
    they reopen in August. For other districts, the solution won't be an
    all-or-nothing approach.
    \href{https://bioethics.jhu.edu/research-and-outreach/projects/eschool-initiative/school-policy-tracker/}{Many
    systems}, including the nation's largest, New York City, are
    devising
    \href{https://www.nytimes.com/2020/06/26/us/coronavirus-schools-reopen-fall.html?action=click\&pgtype=Article\&state=default\&region=MAIN_CONTENT_3\&context=storylines_faq}{hybrid
    plans} that involve spending some days in classrooms and other days
    online. There's no national policy on this yet, so check with your
    municipal school system regularly to see what is happening in your
    community.
  \end{itemize}
\item ~
  \hypertarget{is-the-coronavirus-airborne}{%
  \paragraph{Is the coronavirus
  airborne?}\label{is-the-coronavirus-airborne}}

  \begin{itemize}
  \tightlist
  \item
    The coronavirus
    \href{https://www.nytimes.com/2020/07/04/health/239-experts-with-one-big-claim-the-coronavirus-is-airborne.html?action=click\&pgtype=Article\&state=default\&region=MAIN_CONTENT_3\&context=storylines_faq}{can
    stay aloft for hours in tiny droplets in stagnant air}, infecting
    people as they inhale, mounting scientific evidence suggests. This
    risk is highest in crowded indoor spaces with poor ventilation, and
    may help explain super-spreading events reported in meatpacking
    plants, churches and restaurants.
    \href{https://www.nytimes.com/2020/07/06/health/coronavirus-airborne-aerosols.html?action=click\&pgtype=Article\&state=default\&region=MAIN_CONTENT_3\&context=storylines_faq}{It's
    unclear how often the virus is spread} via these tiny droplets, or
    aerosols, compared with larger droplets that are expelled when a
    sick person coughs or sneezes, or transmitted through contact with
    contaminated surfaces, said Linsey Marr, an aerosol expert at
    Virginia Tech. Aerosols are released even when a person without
    symptoms exhales, talks or sings, according to Dr. Marr and more
    than 200 other experts, who
    \href{https://academic.oup.com/cid/article/doi/10.1093/cid/ciaa939/5867798}{have
    outlined the evidence in an open letter to the World Health
    Organization}.
  \end{itemize}
\item ~
  \hypertarget{what-are-the-symptoms-of-coronavirus}{%
  \paragraph{What are the symptoms of
  coronavirus?}\label{what-are-the-symptoms-of-coronavirus}}

  \begin{itemize}
  \tightlist
  \item
    Common symptoms
    \href{https://www.nytimes.com/article/symptoms-coronavirus.html?action=click\&pgtype=Article\&state=default\&region=MAIN_CONTENT_3\&context=storylines_faq}{include
    fever, a dry cough, fatigue and difficulty breathing or shortness of
    breath.} Some of these symptoms overlap with those of the flu,
    making detection difficult, but runny noses and stuffy sinuses are
    less common.
    \href{https://www.nytimes.com/2020/04/27/health/coronavirus-symptoms-cdc.html?action=click\&pgtype=Article\&state=default\&region=MAIN_CONTENT_3\&context=storylines_faq}{The
    C.D.C. has also} added chills, muscle pain, sore throat, headache
    and a new loss of the sense of taste or smell as symptoms to look
    out for. Most people fall ill five to seven days after exposure, but
    symptoms may appear in as few as two days or as many as 14 days.
  \end{itemize}
\item ~
  \hypertarget{does-asymptomatic-transmission-of-covid-19-happen}{%
  \paragraph{Does asymptomatic transmission of Covid-19
  happen?}\label{does-asymptomatic-transmission-of-covid-19-happen}}

  \begin{itemize}
  \tightlist
  \item
    So far, the evidence seems to show it does. A widely cited
    \href{https://www.nature.com/articles/s41591-020-0869-5}{paper}
    published in April suggests that people are most infectious about
    two days before the onset of coronavirus symptoms and estimated that
    44 percent of new infections were a result of transmission from
    people who were not yet showing symptoms. Recently, a top expert at
    the World Health Organization stated that transmission of the
    coronavirus by people who did not have symptoms was ``very rare,''
    \href{https://www.nytimes.com/2020/06/09/world/coronavirus-updates.html?action=click\&pgtype=Article\&state=default\&region=MAIN_CONTENT_3\&context=storylines_faq\#link-1f302e21}{but
    she later walked back that statement.}
  \end{itemize}
\end{itemize}

Other researchers are using different methods to develop their vaccines.

Inovio, which is also developing a vaccine for MERS, uses a DNA-based
technology. Johnson \& Johnson delivers vaccines through adenoviruses
--- which can cause coldlike symptoms but have been made harmless. And
researchers at the University of Queensland are testing particles that
mimic the structure of a virus.

``We don't know which vaccine approach will be successful at this stage,
so we have to try everything in our arsenal,'' said Dr. Gregory Poland,
a vaccine expert at the Mayo Clinic in Rochester, Minn.

In interviews, company executives said that partnerships with
governments and philanthropic foundations were essential to developing
vaccines for outbreaks because there
\href{https://www.nytimes.com/2014/10/24/health/without-lucrative-market-potential-ebola-vaccine-was-shelved-for-years.html}{are
so many uncertainties}.

Dr. Paul Stoffels, Johnson \& Johnson's chief scientific officer,
estimated it could take eight to 12 months before his company's vaccines
reach human clinical trials. By then, it is possible the coronavirus
outbreak will have been contained. Testing of Johnson \& Johnson's Zika
vaccine is currently halted, he said, because new outbreaks of the
disease have slowed.

``You have to be brave and you have to be a solid company to do this,
because there is no real incentive to do this, no financial incentive,''
he said.

Stéphane Bancel, the chief executive of Moderna, said vaccines were
necessary, even if an outbreak wanes, because it could always return.
``I think it's important to be prepared,'' he said.

Experts believe that the frequency of outbreaks will only increase
because of
\href{https://www.nytimes.com/interactive/2019/06/10/climate/dengue-mosquito-spread-map.html}{climate
change}, urbanization and global travel, among other factors.

``We probably need to start thinking about putting in a special
infrastructure for coronavirus infections the same way we have for the
flu,'' said Dr. Peter Hotez, who is co-director of the Texas Children's
Hospital Center for Vaccine Development and was involved in the
production of a SARS vaccine that may be repurposed for the new
coronavirus. The detection and monitoring of infections, as well as the
development of vaccines, will put an insurance policy in place for
future outbreaks, he said.

``We're just starting to realize that the power of vaccines goes way
beyond public health,'' he said. ``They are also critical to the global
economy and global security.''

\textbf{\emph{{[}}\href{http://on.fb.me/1paTQ1h}{\emph{Like the Science
Times page on Facebook.}}} ****** \emph{\textbar{} Sign up for the}
\textbf{\href{http://nyti.ms/1MbHaRU}{\emph{Science Times
newsletter.}}\emph{{]}}}

Advertisement

\protect\hyperlink{after-bottom}{Continue reading the main story}

\hypertarget{site-index}{%
\subsection{Site Index}\label{site-index}}

\hypertarget{site-information-navigation}{%
\subsection{Site Information
Navigation}\label{site-information-navigation}}

\begin{itemize}
\tightlist
\item
  \href{https://help.nytimes.com/hc/en-us/articles/115014792127-Copyright-notice}{©~2020~The
  New York Times Company}
\end{itemize}

\begin{itemize}
\tightlist
\item
  \href{https://www.nytco.com/}{NYTCo}
\item
  \href{https://help.nytimes.com/hc/en-us/articles/115015385887-Contact-Us}{Contact
  Us}
\item
  \href{https://www.nytco.com/careers/}{Work with us}
\item
  \href{https://nytmediakit.com/}{Advertise}
\item
  \href{http://www.tbrandstudio.com/}{T Brand Studio}
\item
  \href{https://www.nytimes.com/privacy/cookie-policy\#how-do-i-manage-trackers}{Your
  Ad Choices}
\item
  \href{https://www.nytimes.com/privacy}{Privacy}
\item
  \href{https://help.nytimes.com/hc/en-us/articles/115014893428-Terms-of-service}{Terms
  of Service}
\item
  \href{https://help.nytimes.com/hc/en-us/articles/115014893968-Terms-of-sale}{Terms
  of Sale}
\item
  \href{https://spiderbites.nytimes.com}{Site Map}
\item
  \href{https://help.nytimes.com/hc/en-us}{Help}
\item
  \href{https://www.nytimes.com/subscription?campaignId=37WXW}{Subscriptions}
\end{itemize}
