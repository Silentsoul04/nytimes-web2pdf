Sections

SEARCH

\protect\hyperlink{site-content}{Skip to
content}\protect\hyperlink{site-index}{Skip to site index}

\href{https://www.nytimes.com/section/technology}{Technology}

\href{https://myaccount.nytimes.com/auth/login?response_type=cookie\&client_id=vi}{}

\href{https://www.nytimes.com/section/todayspaper}{Today's Paper}

\href{/section/technology}{Technology}\textbar{}Britain Defies Trump
Plea to Ban Huawei From 5G Network

\url{https://nyti.ms/3aRsky2}

\begin{itemize}
\item
\item
\item
\item
\item
\item
\end{itemize}

Advertisement

\protect\hyperlink{after-top}{Continue reading the main story}

Supported by

\protect\hyperlink{after-sponsor}{Continue reading the main story}

\hypertarget{britain-defies-trump-plea-to-ban-huawei-from-5g-network}{%
\section{Britain Defies Trump Plea to Ban Huawei From 5G
Network}\label{britain-defies-trump-plea-to-ban-huawei-from-5g-network}}

The move shows how an American effort against the Chinese wireless
equipment company has stumbled.

\includegraphics{https://static01.nyt.com/images/2020/01/27/business/00ukhuawei-02/00ukhuawei-02-articleLarge-v2.jpg?quality=75\&auto=webp\&disable=upscale}

By \href{https://www.nytimes.com/by/adam-satariano}{Adam Satariano}

\begin{itemize}
\item
  Published Jan. 28, 2020Updated Jan. 29, 2020
\item
  \begin{itemize}
  \item
  \item
  \item
  \item
  \item
  \item
  \end{itemize}
\end{itemize}

LONDON --- Britain said on Tuesday that it would not ban equipment made
by the Chinese technology giant Huawei from being used in its new
high-speed 5G wireless network, the starkest sign yet that an American
campaign against the telecommunications company is faltering.

Despite more than
\href{https://www.nytimes.com/2019/05/15/business/huawei-ban-trump.html}{a
year of intense lobbying by the Trump administration}, which has accused
Huawei of having ties to China's Communist Party that pose a national
security threat, the
\href{https://www.gov.uk/government/news/new-plans-to-safeguard-countrys-telecoms-network-and-pave-way-for-fast-reliable-and-secure-connectivity}{British
government announced} it would allow the company to provide equipment in
some portions of a next-generation network to be built in the coming
years.

The British decision was crucial
\href{https://www.nytimes.com/2020/01/16/world/europe/huawei-germany-china-5g-automakers.html}{in
a broader fight for tech supremacy} between the United States and China.
Britain, a key American ally, is the most important country so far to
reject White House warnings that Huawei is an instrument of Beijing.
Britain's membership in the ``five eyes'' intelligence-sharing group of
countries, which also includes Australia, Canada and New Zealand, gave
the outcome an added significance.

Many countries have been caught between the United States and China in
their tech cold war. American officials have threatened to withhold
intelligence if countries do not ban Huawei, while Chinese
representatives have warned of
\href{https://www.nytimes.com/2019/12/20/technology/faroe-islands-huawei-china-us.html}{economic
retaliation} if they do.

``This is a U.K.-specific solution for U.K.-specific reasons and the
decision deals with the challenges we face right now,'' said Nicky
Morgan, the secretary for digital, culture, media and sport, the
government agency that oversaw the decision.

``It not only paves the way for secure and resilient networks, with our
sovereignty over data protected, but it also builds on our strategy to
develop a diversity of suppliers,'' she said.

The rules were announced on Tuesday after Prime Minister Boris Johnson
met with his National Security Council. The decision did not mention
Huawei by name, instead referring more broadly to ``high-risk vendors''
that ``pose greater security and resilience risks to U.K. telecoms
networks.'' Such vendors will be limited to certain parts of the
wireless infrastructure, such as antennas and base stations, that are
not seen as posing a threat to the integrity of the system.

No single high-risk vendor will be allowed to exceed a 35 percent market
share of the network,
\href{https://www.gov.uk/government/news/new-plans-to-safeguard-countrys-telecoms-network-and-pave-way-for-fast-reliable-and-secure-connectivity}{the
rules said}, an effort to encourage new competition that could benefit
companies including Ericsson, Nokia and Samsung.

A Trump administration official said the United States was
``disappointed'' by Mr. Johnson's decision.

``We look forward to working with the U.K. on a way forward that results
in the exclusion of untrusted vendor components from 5G networks,'' the
official said. ``We continue to urge all countries to carefully assess
the long-term national security and economic impacts of allowing
untrusted vendors access to important 5G network infrastructure.''

Huawei has long denied that it is beholden to the Chinese government.

``Huawei is reassured by the U.K. government's confirmation that we can
continue working with our customers to keep the 5G rollout on track,''
Victor Zhang, Huawei's vice president, said in a statement. ``This
evidence-based decision will result in a more advanced, more secure and
more cost-effective telecoms infrastructure that is fit for the
future.''

The crown jewel of China's tech sector, Huawei is the largest provider
of equipment to build systems based on fifth-generation wireless
technology,
\href{https://www.nytimes.com/2018/12/31/technology/personaltech/5g-what-you-need-to-know.html}{known
as 5G}. That technology is seen as essential infrastructure in an
increasingly digitized global economy. The networks will provide
substantially faster download speeds, as well as new commercial
applications in industries such as transportation, manufacturing and
health care.

Huawei's prominence has made it a target of the United States. Meng
Wanzhou, Huawei's chief financial officer and the daughter of the
company's founder, is
\href{https://www.nytimes.com/2020/01/20/world/canada/meng-wanzhou-huawei-detention-vancouver.html}{fighting
an extradition order} in Canada stemming from an American indictment on
fraud charges.

The Trump administration's global effort against Huawei has had some
success. In 2018, Australia imposed a ban on Huawei gear, and Japan put
restrictions on purchasing Huawei equipment for government use.

But in Europe, the White House has had more difficulty.
\href{http://www.nytimes.com/2020/01/29/world/europe/eu-huawei-5g.html}{The
European Union} has
\href{https://www.nytimes.com/2019/10/09/world/europe/eu-huawei-report.html}{warned
of national security risks} related to 5G, but it has not called out
China or Huawei by name or recommended an outright ban. In France, the
government said it
\href{https://mobile.reuters.com/article/amp/idUSKBN1XZ1U9}{did not
believe a ban was necessary}. Chancellor Angela Merkel of Germany has
shared similar views, though a final decision has not been made, and
some in the government are calling for a harder line.

Perhaps no country was lobbied by the United States and China as hard as
Britain, delaying the country's decision-making about building its new
5G network. President Trump, Secretary of State Mike Pompeo and Treasury
Secretary Steven Mnuchin have all warned Britain in recent weeks. An
American delegation visited London this month to make a last-minute case
against Huawei. Mr. Pompeo is scheduled to visit Britain this week.

Huawei began working in Britain more than 15 years ago and now employs
1,600 people in the country, helping it gain acceptance and a foothold
to expand to other parts of Europe. The European market, which also
includes the Middle East and Africa, is now Huawei's
\href{https://www.huawei.com/uk/press-events/annual-report/2018}{largest}
outside China.

Britain's acceptance of Huawei will influence the decisions of other
countries facing American pressure, said Eric Sayers, a senior adjunct
fellow at the Center for a New American Security, a Washington think
tank.

``Now allies and partners can more easily conclude, `If it's safe for
London, it's safe for us,''' Mr. Sayers said.

British officials have said the risk Huawei presents can be managed
through oversight and by limiting its access to more critical areas of
the network that handle sensitive data. Banning the company would delay
the construction of its 5G network and cost billions to replace old
equipment.

Under the new rules, Huawei would be limited to providing antennas and
other equipment that send data directly to consumer devices, and kept
out of areas considered the nerve center of the network, such as servers
that route traffic within the system.

Britain has always kept Huawei out of those parts of its
telecommunications networks that handle sensitive data to limit the
vulnerability to espionage or eavesdropping. In 2010, British officials
set up a lab where Huawei's equipment could be reviewed for security
flaws. The lab has identified security vulnerabilities in the equipment,
but officials have said the problems were not a result of interference
from the Chinese government and could be managed.

``High-risk vendors have never been --- and never will be --- in our
most sensitive networks,'' said Ciaran Martin, the chief executive of
the National Cyber Security Center, which oversees the lab.

American officials disagree that the risks can be contained because
software plays a bigger role in 5G networks, with constantly updating
code making it harder to maintain complete oversight.

\includegraphics{https://static01.nyt.com/images/2020/01/27/business/00ukhuawei-01/merlin_165832251_7a4a40e6-d88e-4cd2-8390-f883945c7a93-articleLarge.jpg?quality=75\&auto=webp\&disable=upscale}

``Digital technology is being upgraded regularly, and a level of risk
with present-day technology that is manageable today may or may not be
so four or five years down the line,'' said Steve Tsang, director of the
China Institute at SOAS University of London.

The decision over whether to use Huawei equipment in Britain's 5G
network would usually be a technical one made by agencies that oversee
cybersecurity and the nation's digital infrastructure. But it became a
political dilemma that spanned two administrations --- first for Theresa
May when she was British prime minister, and now for Mr. Johnson.

British officials and executives at wireless companies have said the
United States did not share smoking-gun evidence that would justify a
ban of the Chinese company. American officials emphasized the
vulnerabilities it could create within a national communications network
in the event of a future confrontation with China.

Under the rules announced on Tuesday, high-risk firms would be excluded
from providing technology at sensitive geographic locations, such as
nuclear sites and military bases. Companies like Vodafone and BT are
likely to have to find alternative suppliers for some pieces of their
networks to comply with new limits on Huawei equipment.

``There is definitely a potential security risk,'' said Alan Woodward, a
cybersecurity expert and visiting professor at the University of Surrey.
``Is it manageable? That is the big question out there.''

Britain is in a precarious position as it negotiates an exit from the
European Union. The country must forge new stand-alone trade deals in
the aftermath. Maintaining close ties to Washington is vital for
Britain's security and economy, but it also needs to foster ties with
China, which is a significant investor in the country and a growing
buyer of British goods.

``Post-Brexit Britain will increasingly have to rely on China even more
than we already do,'' said Anthony Glees, professor emeritus at the
University of Buckingham, where he was head of the Centre for Security
and Intelligence Studies.

Even with the British decision, American critics of Huawei say the
United States could still slow Huawei's march by blocking American firms
from providing needed chips or by helping rivals. The Trump
administration is expected to continue pressing Germany and France to
keep Huawei out of the network. Other big countries, like India, are
also yet to make a final decision on their networks.

``I don't believe all is lost,'' said Michael Rogers, a former chairman
of the House Intelligence Committee who now leads a group called 5G
Action Now. ``It just means we have to redouble our efforts in places
like Poland and Germany and keep our Canadian friends from polluting
their networks.''

Julian E. Barnes and David McCabe contributed reporting from Washington.

Advertisement

\protect\hyperlink{after-bottom}{Continue reading the main story}

\hypertarget{site-index}{%
\subsection{Site Index}\label{site-index}}

\hypertarget{site-information-navigation}{%
\subsection{Site Information
Navigation}\label{site-information-navigation}}

\begin{itemize}
\tightlist
\item
  \href{https://help.nytimes.com/hc/en-us/articles/115014792127-Copyright-notice}{©~2020~The
  New York Times Company}
\end{itemize}

\begin{itemize}
\tightlist
\item
  \href{https://www.nytco.com/}{NYTCo}
\item
  \href{https://help.nytimes.com/hc/en-us/articles/115015385887-Contact-Us}{Contact
  Us}
\item
  \href{https://www.nytco.com/careers/}{Work with us}
\item
  \href{https://nytmediakit.com/}{Advertise}
\item
  \href{http://www.tbrandstudio.com/}{T Brand Studio}
\item
  \href{https://www.nytimes.com/privacy/cookie-policy\#how-do-i-manage-trackers}{Your
  Ad Choices}
\item
  \href{https://www.nytimes.com/privacy}{Privacy}
\item
  \href{https://help.nytimes.com/hc/en-us/articles/115014893428-Terms-of-service}{Terms
  of Service}
\item
  \href{https://help.nytimes.com/hc/en-us/articles/115014893968-Terms-of-sale}{Terms
  of Sale}
\item
  \href{https://spiderbites.nytimes.com}{Site Map}
\item
  \href{https://help.nytimes.com/hc/en-us}{Help}
\item
  \href{https://www.nytimes.com/subscription?campaignId=37WXW}{Subscriptions}
\end{itemize}
