Sections

SEARCH

\protect\hyperlink{site-content}{Skip to
content}\protect\hyperlink{site-index}{Skip to site index}

\href{https://www.nytimes.com/section/us}{U.S.}

\href{https://myaccount.nytimes.com/auth/login?response_type=cookie\&client_id=vi}{}

\href{https://www.nytimes.com/section/todayspaper}{Today's Paper}

\href{/section/us}{U.S.}\textbar{}U.S. Accuses Harvard Scientist of
Concealing Chinese Funding

\url{https://nyti.ms/38FHqVA}

\begin{itemize}
\item
\item
\item
\item
\item
\end{itemize}

Advertisement

\protect\hyperlink{after-top}{Continue reading the main story}

Supported by

\protect\hyperlink{after-sponsor}{Continue reading the main story}

\hypertarget{us-accuses-harvard-scientist-of-concealing-chinese-funding}{%
\section{U.S. Accuses Harvard Scientist of Concealing Chinese
Funding}\label{us-accuses-harvard-scientist-of-concealing-chinese-funding}}

Prosecutors say Charles M. Lieber, the chair of Harvard's chemistry
department, lied about contacts with a Chinese state-run initiative that
seeks to draw foreign-educated talent.

\includegraphics{https://static01.nyt.com/images/2020/01/28/us/28harvardscientist/28harvardscientist-articleLarge.jpg?quality=75\&auto=webp\&disable=upscale}

\href{https://www.nytimes.com/by/ellen-barry}{\includegraphics{https://static01.nyt.com/images/2018/10/08/multimedia/author-ellen-barry/author-ellen-barry-thumbLarge.png}}

By \href{https://www.nytimes.com/by/ellen-barry}{Ellen Barry}

\begin{itemize}
\item
  Jan. 28, 2020
\item
  \begin{itemize}
  \item
  \item
  \item
  \item
  \item
  \end{itemize}
\end{itemize}

BOSTON --- Early Tuesday morning, F.B.I. agents arrived at two of the
most protected corners of Harvard University's academic cloister, raking
through a gabled house in the suburb of Lexington and a neoclassical
brick building in Cambridge.

By afternoon, one of Harvard's scientific luminaries was in handcuffs,
charged with making a false statement to federal authorities about his
financial relationship with the Chinese government, and especially his
participation in its Thousand Talents program, a campaign to attract
foreign-educated scientists to China.

The arrest of Charles M. Lieber, the chair of Harvard's department of
chemistry and chemical biology, signaled a new, aggressive phase in the
Justice Department's campaign
\href{https://www.nytimes.com/2019/11/04/health/china-nih-scientists.html}{to
root out scientists who are stealing research} from American
laboratories.

For months, news has been trickling out about the prosecution of
scientists, mainly Chinese graduate students and researchers working in
American laboratories. But Dr. Lieber represents a different kind of
target, a star researcher who had risen to the highest reaches of the
American academic hierarchy.

Dr. Lieber, a leader in the field of nanoscale electronics, has not been
accused of sharing sensitive information with Chinese officials, but
rather of hiding --- from
\href{https://www.nytimes.com/2020/07/15/us/steven-pinker-harvard.html}{Harvard},
from the National Institutes of Health and from the Defense Department
--- the amount of money that Chinese funders were paying him.

Dr. Lieber's lawyer, Peter Levitt, made no comment after a preliminary
hearing in federal court in Boston on Tuesday.

His arrest sent shock waves through research circles.

``This is a very, very highly esteemed, highly regarded investigator
working at Harvard, a major U.S. institution, at the highest rank he
could have, so, all the success you can have in this sphere,'' said Dr.
Ross McKinney Jr., chief scientific officer of the Association of
American Medical Colleges. ``It's like, when you've got it all, why do
you want more?''

Dr. McKinney described anxiety among his colleagues that scientists will
be scrutinized over legitimate sources of international funding.

``We worry that, slowly but surely, we're going to have something of a
McCarthyish purity testing,'' he said. ``He's being criminally charged.
This is a big deal. He could end up in jail.''

Dr. Lieber, 60, was charged with one count of making a false or
misleading statement, which carries a maximum sentence of five years in
prison. He appeared in court on Tuesday wearing the outfit he had put on
to head to his office at Harvard: a Brooks Brothers polo shirt, cargo
pants and hiking boots. He appeared subdued as he flipped through the
charge sheet. Mr. Levitt, his lawyer, said it was his first opportunity
to read the charge against him.

Harvard said Dr. Lieber had been placed on indefinite administrative
leave.

``The charges brought by the U.S. government against Professor Lieber
are extremely serious,'' said Jonathan Swain, a spokesman for the
university. ``Harvard is cooperating with federal authorities, including
the National Institutes of Health, and is initiating its own review of
the alleged misconduct.''

Dr. Lieber was one of three scientists to be charged with crimes on
Tuesday.

Zaosong Zheng, a Harvard-affiliated cancer researcher was
\href{https://www.nytimes.com/2019/12/31/us/chinese-scientist-cancer-research-investigation.html}{caught
leaving the country with 21 vials of cells stolen from a laboratory} at
Beth Israel Deaconess Hospital in Boston, according to the authorities.
They said he had admitted that he had planned to turbocharge his career
by publishing the research in China under his own name. He was charged
with smuggling goods from the United States and with making false
statements, and was being held without bail in Massachusetts after a
judge determined that he was a flight risk. His lawyer has not responded
to a request for comment.

The third was Yanqing Ye, who had been conducting research at Boston
University's department of physics, chemistry and biomedical engineering
until last spring, when she returned to China. Prosecutors said she hid
the fact that she was a lieutenant in the People's Liberation Army, and
continued to carry out assignments from Chinese military officers while
at B.U.

Ms. Yanqing was charged with visa fraud, making false statements, acting
as an agent of a foreign government and conspiracy. She was in China and
was not arrested.

Prosecutors made it clear that the charges announced on Tuesday were
part of a bigger crackdown on researchers working with the Chinese
government.

``No country poses a greater, more severe or long-term threat to our
national security and economic prosperity than China,'' said Joseph
Bonavolonta, special agent in charge of the F.B.I.'s Boston field
office. ``China's communist government's goal, simply put, is to replace
the U.S. as the world superpower, and they are breaking the law to get
there.''

He called Massachusetts, with its cluster of elite universities and
research institutions, ``a target-rich environment.''

Charging documents in the case describe Dr. Lieber's growing commitments
in China, and efforts to hide them from his employers in the United
States.

In 2011, the documents say, he signed an agreement to become a
``strategic scientist'' at Wuhan University of Technology in China,
entitling him to a \$50,000 monthly salary, \$150,000 in annual in
living expenses and more than \$1.5 million for a second laboratory in
Wuhan. In 2013, he celebrated the founding of a joint laboratory, the
\href{http://english.whut.edu.cn/wn/201301/t20130104_91316.htmlhttp://english.whut.edu.cn/wn/201301/t20130104_91316.html}{WUT-Harvard
Joint Nano Key Laboratory}.

The authorities said that he was informed in 2012 that he had been
selected to participate in the Thousand Talents plan, the China-run
program.

In 2015, Harvard officials discovered that Dr. Lieber was leading a
laboratory at Wuhan University, and informed him that the use of
Harvard's name and logo was a violation of university policy. Dr. Lieber
then distanced himself from the project, but continued to receive
payment, prosecutors said.

Then in 2017 he was named a university professor, Harvard's highest
faculty rank, one of only 26 professors to hold that status. The same
year, he earned the N.I.H.
\href{https://commonfund.nih.gov/pioneer}{Director's Pioneer Award} for
inventing
\href{https://commonfund.nih.gov/pioneer/AwardRecipients17}{syringe-injectable
mesh electronics} that can integrate with the brain.

Investigators from the Defense Department --- which had extended \$8
million in grants to Dr. Lieber --- began questioning him in 2018 about
secondary sources of income, prosecutors said.

Dr. Lieber told them that he was aware of China's Thousand Talents
program, but had never been invited to participate, prosecution
documents say. Two days after that conversation, the documents say, Dr.
Lieber asked a laboratory associate to help him identify web pages in
which he was named as the head of the Chinese lab.

``I lost a lot of sleep worrying all of these things last night and want
to start taking steps to correct sooner than later,'' he wrote in an
email to a research colleague that was cited by prosecutors. ``I will be
careful about what I discuss with Harvard University, and none of this
will be shared with government investigators at this time.''

Last year, Harvard was required to submit a detailed report about Dr.
Lieber to N.I.H., which had provided \$10 million in grants for his
research projects. He told university officials that he had ``no formal
association'' with the Wuhan University of Technology, prosecutors said,
and that he ``is not and has never been'' a participant in the Thousand
Talents program.

The campaign to scrutinize scientists' foreign funding is a relatively
new one.

Late in 2018, Jeff Sessions, then the attorney general,
\href{https://www.justice.gov/opa/speech/attorney-general-jeff-sessions-announces-new-initiative-combat-chinese-economic-espionage}{announced}
that the United States was ``standing up to the deliberate, systematic
and calculated threats posed, in particular, by the communist regime in
China.''

As a result, researchers are adjusting to a higher level of scrutiny
about foreign funding than they faced in the past, said Derek Adams, a
former federal prosecutor who specialized in civil fraud.

``The problem here, in my view, is that in 2018 there was a material
change in the way the F.B.I. and the agencies were approaching this
issue,'' said Mr. Adams, now a partner in the law firm Feldesman Tucker
Leifer Fidell.

In many cases, he said, ``they're looking at conduct that occurred many
years ago. For an individual that may have had an obligation to
disclose, it may not have been front at center at that time.''

Frank Wu, a law professor and former president of the Committee of 100,
an organization of prominent Chinese-Americans, has criticized the
recent prosecutions as ``potentially devastating to American science,
because the number of people who have some connection to China is so
vast.'' Until recently, he said, such collaborations were considered
healthy. ``These rules are new rules,'' he said.

Advertisement

\protect\hyperlink{after-bottom}{Continue reading the main story}

\hypertarget{site-index}{%
\subsection{Site Index}\label{site-index}}

\hypertarget{site-information-navigation}{%
\subsection{Site Information
Navigation}\label{site-information-navigation}}

\begin{itemize}
\tightlist
\item
  \href{https://help.nytimes.com/hc/en-us/articles/115014792127-Copyright-notice}{©~2020~The
  New York Times Company}
\end{itemize}

\begin{itemize}
\tightlist
\item
  \href{https://www.nytco.com/}{NYTCo}
\item
  \href{https://help.nytimes.com/hc/en-us/articles/115015385887-Contact-Us}{Contact
  Us}
\item
  \href{https://www.nytco.com/careers/}{Work with us}
\item
  \href{https://nytmediakit.com/}{Advertise}
\item
  \href{http://www.tbrandstudio.com/}{T Brand Studio}
\item
  \href{https://www.nytimes.com/privacy/cookie-policy\#how-do-i-manage-trackers}{Your
  Ad Choices}
\item
  \href{https://www.nytimes.com/privacy}{Privacy}
\item
  \href{https://help.nytimes.com/hc/en-us/articles/115014893428-Terms-of-service}{Terms
  of Service}
\item
  \href{https://help.nytimes.com/hc/en-us/articles/115014893968-Terms-of-sale}{Terms
  of Sale}
\item
  \href{https://spiderbites.nytimes.com}{Site Map}
\item
  \href{https://help.nytimes.com/hc/en-us}{Help}
\item
  \href{https://www.nytimes.com/subscription?campaignId=37WXW}{Subscriptions}
\end{itemize}
