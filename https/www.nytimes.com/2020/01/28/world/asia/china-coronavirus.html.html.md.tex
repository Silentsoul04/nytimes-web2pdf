Sections

SEARCH

\protect\hyperlink{site-content}{Skip to
content}\protect\hyperlink{site-index}{Skip to site index}

\href{https://www.nytimes.com/section/world/asia}{Asia Pacific}

\href{https://myaccount.nytimes.com/auth/login?response_type=cookie\&client_id=vi}{}

\href{https://www.nytimes.com/section/todayspaper}{Today's Paper}

\href{/section/world/asia}{Asia Pacific}\textbar{}Coronavirus: Death
Toll Climbs, and So Does the Number of Infections

\url{https://nyti.ms/2RzxbfT}

\begin{itemize}
\item
\item
\item
\item
\item
\item
\end{itemize}

\href{https://www.nytimes.com/news-event/coronavirus?action=click\&pgtype=Article\&state=default\&region=TOP_BANNER\&context=storylines_menu}{The
Coronavirus Outbreak}

\begin{itemize}
\tightlist
\item
  live\href{https://www.nytimes.com/2020/08/02/world/coronavirus-updates.html?action=click\&pgtype=Article\&state=default\&region=TOP_BANNER\&context=storylines_menu}{Latest
  Updates}
\item
  \href{https://www.nytimes.com/interactive/2020/us/coronavirus-us-cases.html?action=click\&pgtype=Article\&state=default\&region=TOP_BANNER\&context=storylines_menu}{Maps
  and Cases}
\item
  \href{https://www.nytimes.com/interactive/2020/science/coronavirus-vaccine-tracker.html?action=click\&pgtype=Article\&state=default\&region=TOP_BANNER\&context=storylines_menu}{Vaccine
  Tracker}
\item
  \href{https://www.nytimes.com/interactive/2020/07/29/us/schools-reopening-coronavirus.html?action=click\&pgtype=Article\&state=default\&region=TOP_BANNER\&context=storylines_menu}{What
  School May Look Like}
\item
  \href{https://www.nytimes.com/live/2020/07/31/business/stock-market-today-coronavirus?action=click\&pgtype=Article\&state=default\&region=TOP_BANNER\&context=storylines_menu}{Economy}
\end{itemize}

Advertisement

\protect\hyperlink{after-top}{Continue reading the main story}

Supported by

\protect\hyperlink{after-sponsor}{Continue reading the main story}

\hypertarget{coronavirus-death-toll-climbs-and-so-does-the-number-of-infections}{%
\section{Coronavirus: Death Toll Climbs, and So Does the Number of
Infections}\label{coronavirus-death-toll-climbs-and-so-does-the-number-of-infections}}

The number of known cases of the new virus rose by nearly a third
overnight. A shortage of test kits has led experts to warn that the real
number may be higher.

By The New York Times

\begin{itemize}
\item
  Published Jan. 28, 2020Updated March 9, 2020
\item
  \begin{itemize}
  \item
  \item
  \item
  \item
  \item
  \item
  \end{itemize}
\end{itemize}

\includegraphics{https://static01.nyt.com/images/2020/01/28/world/28china-briefing95/merlin_167975481_9403edbf-0973-4a81-b87d-b3f87ce79479-articleLarge.jpg?quality=75\&auto=webp\&disable=upscale}

\hypertarget{death-toll-climbs-above-130-and-the-number-of-infections-jumps-as-well}{%
\subsection{Death toll climbs above 130, and the number of infections
jumps as
well.}\label{death-toll-climbs-above-130-and-the-number-of-infections-jumps-as-well}}

As the outbreak of the mysterious new
\href{http://www.nytimes.com/2020/03/09/world/coronavirus-news.html}{coronavirus}
rapidly spreads, the Chinese authorities said on Wednesday that the
official count of known cases jumped again overnight, with the death
toll now exceeding 130.

◆
\href{https://www.nytimes.com/2020/02/11/world/asia/coronavirus-indonesia-bali.html}{China}
said on Wednesday that 132 people had died from the
\href{https://www.nytimes.com/2020/02/11/world/asia/coronavirus-indonesia-bali.html}{virus},
which is believed to have originated in the
\href{https://www.nytimes.com/interactive/2020/world/asia/china-wuhan-coronavirus-maps.html}{central
city of Wuhan and is spreading across the country}. The previous count,
on Tuesday, was 106.

◆ The number of confirmed cases increased to 5,974 on Wednesday, up from
4,515 on Tuesday, according to the National Health Commission.

◆ Most of the confirmed cases have been in the central Chinese province
of Hubei, where several cities, including Wuhan, the epicenter of the
outbreak, have been placed under what amounts to a lockdown.

◆ Thailand has reported 14 cases of infection; Hong Kong has eight; the
United States, Taiwan, Australia and Macau have five each; Singapore,
South Korea and Malaysia each have reported four; Japan has seven;
France has four; Canada has three; Vietnam has two; and Nepal, Cambodia
and Germany each have one. There have been no deaths outside China.

\href{https://www.nytimes.com/interactive/2020/01/21/world/asia/china-coronavirus-maps.html}{}

\includegraphics{https://static01.nyt.com/images/2020/01/31/us/china-wuhan-coronavirus-promo-1579641872730/china-wuhan-coronavirus-promo-1579641872730-articleLarge-v21.jpg}

\hypertarget{wuhan-coronavirus-map-tracking-the-spread-of-the-outbreak}{%
\subsection{Wuhan Coronavirus Map: Tracking the Spread of the
Outbreak}\label{wuhan-coronavirus-map-tracking-the-spread-of-the-outbreak}}

The virus has sickened tens of thousands of people in China and a number
of other countries.

\hypertarget{the-united-states-is-expanding-screenings-for-coronavirus-at-airports-and-borders}{%
\subsection{The United States is expanding screenings for coronavirus at
airports and
borders.}\label{the-united-states-is-expanding-screenings-for-coronavirus-at-airports-and-borders}}

Image

``Americans should know this is a potentially very serious public health
threat, but at this point Americans should not worry for their own
safety,'' said Alex M. Azar, secretary of health and human services,
second from left, at a news briefing in Washington. ``This is a very
fast-moving, constantly changing situation.''Credit...Shawn Thew/EPA,
via Shutterstock

The United States is
\href{https://www.nytimes.com/2020/01/28/health/airports-screening-coronavirus.html}{expanding
the screening of travelers} arriving from Wuhan --- to 20 airports and
land crossings, from five airports, federal officials said on Tuesday.

``Right now, there is no spread of this virus in our communities at
home,'' said Dr. Robert Redfield, director of the Centers for Disease
Control and Prevention, at a news briefing in Washington.

``The coming days and weeks are likely to bring more cases including the
possibility of person-to-person spread,'' he said. ``Our goal is to
contain this virus and prevent sustained spread of the virus in our
country.''

\emph{{[}A plane carrying Americans is leaving from Wuhan. If you know
anyone on board, or anyone trying to leave Wuhan,} ****** \emph{we would
like to hear from you for a coming article. Please contact Miriam Jordan
at}
\href{mailto:miriam.jordan@nytimes.com}{\emph{miriam.jordan@nytimes.com}}
\emph{to share your story.{]}}

Officials also announced that after repeated offers of assistance,
Chinese authorities agreed on Tuesday to allow in teams of international
experts, coordinated by the World Health Organization, to help with
research and containment.

``Americans should know this is a potentially very serious public health
threat, but at this point Americans should not worry for their own
safety,'' said Alex M. Azar, secretary of health and human services, at
the briefing.

\hypertarget{scientists-race-to-make-a-coronavirus-vaccine}{%
\subsection{Scientists race to make a coronavirus
vaccine.}\label{scientists-race-to-make-a-coronavirus-vaccine}}

Scientists in the United States, Australia and at least three companies
are working on
v\href{https://www.nytimes.com/2020/01/28/health/coronavirus-vaccine.html}{accine
candidates to stop the spread of the coronavirus.}

Government scientists as well as those working at Johnson \& Johnson,
Moderna Therapeutics and Inovio Pharmaceuticals are all working quickly
to develop a vaccine.

The hunt began Jan. 10, when Chinese scientists posted the genetic
makeup of the virus on
\href{https://www.ncbi.nlm.nih.gov/nuccore/MN908947}{a public database}.
The next morning, researchers at the National Institutes of Health's
Vaccine Research Center in Maryland went to work. Within hours, they had
pinpointed the letters of the genetic code that could be used to make a
vaccine.

Historically, vaccines have been one of the greatest public health tools
to prevent disease. But even as technology, genomics and global
coordination have improved, allowing researchers to move at top speed,
vaccine development remains an expensive and risky process.

\hypertarget{how-the-coronavirus-is-affecting-earnings-season}{%
\subsection{How the coronavirus is affecting earnings
season.}\label{how-the-coronavirus-is-affecting-earnings-season}}

The number of infections and deaths related to the outbreak of a
mysterious virus in China keeps rising, but economic analysts have
counseled caution, saying it's too soon to ring alarms about the impact
on the global economy.

Even so, individually, some American companies with a sizable presence
in China are feeling the strain.
\href{https://s22.q4cdn.com/869488222/files/doc_financials/2020/Q1/Q1-FY20-Earnings-Release.pdf}{Starbucks},
for example, announced on Tuesday that it was temporarily shuttering
half of its locations there.

``The magnitude of the impact will depend on the duration of store
closures as we work with local authorities to manage the situation and
protect our partners and customers,'' Pat Grismer, chief financial
officer, said during an earnings call.

Other companies that closed stores there included McDonald's and Yum
China, the country's largest restaurant company, which operates the KFC,
Pizza Hut and Taco Bell brands in China and also controls its own
brands.

\hypertarget{latest-updates-global-coronavirus-outbreak}{%
\section{\texorpdfstring{\href{https://www.nytimes.com/2020/08/01/world/coronavirus-covid-19.html?action=click\&pgtype=Article\&state=default\&region=MAIN_CONTENT_1\&context=storylines_live_updates}{Latest
Updates: Global Coronavirus
Outbreak}}{Latest Updates: Global Coronavirus Outbreak}}\label{latest-updates-global-coronavirus-outbreak}}

Updated 2020-08-02T17:52:35.962Z

\begin{itemize}
\tightlist
\item
  \href{https://www.nytimes.com/2020/08/01/world/coronavirus-covid-19.html?action=click\&pgtype=Article\&state=default\&region=MAIN_CONTENT_1\&context=storylines_live_updates\#link-34047410}{The
  U.S. reels as July cases more than double the total of any other
  month.}
\item
  \href{https://www.nytimes.com/2020/08/01/world/coronavirus-covid-19.html?action=click\&pgtype=Article\&state=default\&region=MAIN_CONTENT_1\&context=storylines_live_updates\#link-780ec966}{Top
  U.S. officials work to break an impasse over the federal jobless
  benefit.}
\item
  \href{https://www.nytimes.com/2020/08/01/world/coronavirus-covid-19.html?action=click\&pgtype=Article\&state=default\&region=MAIN_CONTENT_1\&context=storylines_live_updates\#link-2bc8948}{Its
  outbreak untamed, Melbourne goes into even greater lockdown.}
\end{itemize}

\href{https://www.nytimes.com/2020/08/01/world/coronavirus-covid-19.html?action=click\&pgtype=Article\&state=default\&region=MAIN_CONTENT_1\&context=storylines_live_updates}{See
more updates}

More live coverage:
\href{https://www.nytimes.com/live/2020/07/31/business/stock-market-today-coronavirus?action=click\&pgtype=Article\&state=default\&region=MAIN_CONTENT_1\&context=storylines_live_updates}{Markets}

Travel restrictions in China and screening at airports in other
countries have also disrupted plans. United Airlines announced today
that it was suspending some flights. American Airlines stock fell more
than 5 percent on Tuesday.

Hotels and resorts with properties in the affected areas, which include
Macau, a special administrative region and gambling mecca, like Wynn
Resorts, Las Vegas Sands and MGM Resorts International also saw the
value of their shares sink. Marriott, Hyatt and Hilton which have
several properties in China also saw their stock prices slide.

Other brands that are popular in China, like Estee Lauder, Nike and
Tapestry, which sells Coach, Kate Spade and Stuart Weitzman, are likely
to see a dent in earnings, bank analysts said.

Anxiety about sales spread to Chinese tech companies including Alibaba,
JD.com, and Baidu.

China is the world's second largest economy.

\hypertarget{japan-has-evacuated-some-of-its-citizens}{%
\subsection{Japan has evacuated some of its
citizens.}\label{japan-has-evacuated-some-of-its-citizens}}

A charter plane hired by the Japanese government to evacuate citizens
from Wuhan landed at a Tokyo airport Wednesday morning.

Outside, tour buses --- with drivers in sanitary masks --- and some
ambulances waited to accept the 206 passengers. Japanese news media
reported that one of the passengers had a fever and one was coughing.

In a news conference late Tuesday, Japanese authorities said that people
who showed symptoms would be transferred directly to a hospital, while
those who appeared healthy would be tested for the virus and then asked
to quarantine themselves for two weeks.

The flight left Tokyo Tuesday night carrying medical supplies requested
by the Chinese government to help fight the viral outbreak that has put
Wuhan and other cities on lockdown.

At another news conference, two of the passengers expressed relief at
being back in Japan.

Authorities plan to send a second plane Wednesday night to Wuhan, where
around 450 Japanese citizens still await evacuation.

\hypertarget{united-airlines-suspends-some-flights-from-the-united-states-to-china}{%
\subsection{United Airlines suspends some flights from the United States
to
China.}\label{united-airlines-suspends-some-flights-from-the-united-states-to-china}}

United Airlines said on Tuesday that it will reduce flights between the
United States and Beijing, Hong Kong and Shanghai because of a
``significant decline in demand.''

United said in a statement that it would cut flights starting Feb. 1,
through Feb. 8. The suspension will affect 24 flights.

``We will continue to monitor the situation as it develops and will
adjust our schedule as needed,'' United said.

Delta Air Lines and American Airlines have not yet cut flights to China,
but have said they are closely monitoring the situation there as the
coronavirus spreads.

\hypertarget{france-confirms-fourth-case-of-the-virus}{%
\subsection{France confirms fourth case of the
virus.}\label{france-confirms-fourth-case-of-the-virus}}

Image

The lobby of University Hospital of Pellegrin in Bordeaux on Monday,
where one patient was hospitalized after experiencing coronavirus-like
symptoms.Credit...Nicolas Tucat/Agence France-Presse --- Getty Images

A fourth case of the new coronavirus was confirmed in France, Jérôme
Salomon, an official in the health ministry, announced on Tuesday.

The patient is a tourist from Hubei province, Mr. Salomon said. He was
in serious condition in the intensive care unit of a Paris hospital, and
the authorities were seeking out anyone who had been in close contact
with the man, who is in his eighties.

The previously identified cases affected a 48-year-old man in Bordeaux
and a 31-year-old man and 30-year-old woman in Paris.

France is also working with China to repatriate French citizens in
Wuhan, with a first flight scheduled for later in the week. Agnès Buzyn,
France's health minister, said on Tuesday that about 500 to 1,000 French
citizens could qualify.

Those who return without any symptoms will be quarantined for 14 days,
and those who present possible symptoms will be hospitalized.

Other European countries have also asked France to help bring back some
of their own citizens on those flights, Ms. Buzyn said.

\hypertarget{germany-and-japan-say-the-virus-has-spread-in-their-countries}{%
\subsection{Germany and Japan say the virus has spread in their
countries.}\label{germany-and-japan-say-the-virus-has-spread-in-their-countries}}

Image

Commuters near Shinjuku Station in Tokyo on Monday. Japan has announced
that it has diagnosed six people with coronavirus.Credit...Jae C.
Hong/Associated Press

Health officials on Tuesday reported what appear to be the first known
cases of human-to-human transmission of the virus in Europe ---
specifically, in Germany --- and in Japan. Another case was recently
reported in Vietnam.

The cases show that countries across the world are now faced with the
task of limiting the spread of the disease on their own soil, not just
seeking to identify and quarantine ailing patients who had traveled from
China.

Japan's Ministry of Health, Labor and Welfare said the first Japanese
national --- and the sixth person in Japan overall --- to be diagnosed
with the coronavirus was a man in his 60s. He had never been to Wuhan,
but he had worked as a bus driver earlier this month for two different
group tours from that city, officials said.

The man began experiencing symptoms on Jan. 14, was hospitalized on
Saturday and was confirmed to have the coronavirus on Tuesday.

The infected German, whose case was also confirmed on Tuesday, is a
33-year-old man from Bavaria who had been in contact with a Chinese
woman in Germany, officials said. The woman was diagnosed with the virus
after flying home to China. The man was in good condition, German
officials said.

``It was to be expected that the virus would come to Germany,'' Jens
Spahn, Germany's health minister, said in a statement on Tuesday. ``But
the Bavarian case shows us that we are well prepared.''

The World Health Organization said on Friday that there appeared to have
been
\href{https://www.who.int/docs/default-source/coronaviruse/situation-reports/20200124-sitrep-4-2019-ncov.pdf}{a
case of human-to-human transmission in Vietnam}, where a person who had
never been to China, but who had a relative who had visited Wuhan, was
confirmed to have the virus.

\hypertarget{hong-kong-puts-significant-limits-on-travel-from-the-mainland-}{%
\subsection{Hong Kong puts significant limits on travel from the
mainland.
}\label{hong-kong-puts-significant-limits-on-travel-from-the-mainland-}}

Image

Hong Kong's Chief Executive, Carrie Lam, center, during a press
conference addressing the coronavirus outbreak in Hong Kong on
Tuesday.Credit...Jerome Favre/EPA, via Shutterstock

Hong Kong on Tuesday put in place a broad series of restrictions aimed
at controlling the spread of the coronavirus by limiting the number of
mainland Chinese travelers entering the territory, one of Asia's busiest
travel and financial hubs.

The restrictions --- which included the suspension of high-speed and
other train services between Hong Kong and the mainland, a 50 percent
reduction in the number of flights --- and a ban on tourism visas for
many travelers --- were announced by Carrie Lam, the city's chief
executive.

The regulations, which apply to some plane, rail, bus and ferry
arrivals, will begin on Thursday. They follow days of rising pressure
from health care workers, epidemiologists and even pro-Beijing
politicians who have traditionally supported Mrs. Lam's government.

Hong Kong has so far recorded eight confirmed cases of the virus.

Tibet, the only region in China that has yet to report any cases, has
temporarily closed all tourist sites, state news media reported. Major
Chinese cities, including Shanghai and Beijing, have suspended
long-distance bus services.

The medical faculty of the Chinese University of Hong Kong called for
more restrictions on border checkpoints as the virus spreads across
China.

Workers from Hong Kong's Hospital Authority have planned a strike for
next week to demand a law requiring the wearing of masks in public and
banning all visitors from entering the city through the mainland.

\hypertarget{ethiopia-and-ivory-coast-test-suspected-cases}{%
\subsection{Ethiopia and Ivory Coast test suspected
cases.}\label{ethiopia-and-ivory-coast-test-suspected-cases}}

Image

Kenyatta National Hospital in Nairobi, Kenya, quarantined a student who
arrived from China's Guangzhou airport\\
with coronavirus-like symptoms.Credit...Daniel Irungu/EPA, via
Shutterstock

Numerous African countries are shoring up coronavirus screening efforts
at major airports, and samples from at least five potentially infected
patients were being tested.

Ethiopia's state minister of health
\href{https://twitter.com/lia_tadesse/status/1222146999371300865}{confirmed}
on Tuesday that four potential cases were isolated in the capital, Addis
Ababa, pending laboratory tests. On Monday, officials in Ivory Coast
said they were testing a suspected case related to a female student who
had traveled from Beijing to the capital, Abidjan.

The epidemic comes as travel between China and African states has
increased at a rapid pace. As Beijing has ramped up its diplomatic,
economic and political support for African states, Chinese firms and
migrants have been setting up shop in cities from Nairobi to
Johannesburg.

Data from the China Africa Research Initiative at Johns Hopkins
University
\href{http://www.sais-cari.org/data-chinese-workers-in-africa}{put the
number of official Chinese workers} as of 2017 at over 202,000. African
entrepreneurs have moved to Chinese cities, while African students now
\href{https://theconversation.com/china-tops-us-and-uk-as-destination-for-anglophone-african-students-78967}{make
up a large percentage of the foreign student body} in China.

\href{https://www.nytimes.com/news-event/coronavirus?action=click\&pgtype=Article\&state=default\&region=MAIN_CONTENT_3\&context=storylines_faq}{}

\hypertarget{the-coronavirus-outbreak-}{%
\subsubsection{The Coronavirus Outbreak
›}\label{the-coronavirus-outbreak-}}

\hypertarget{frequently-asked-questions}{%
\paragraph{Frequently Asked
Questions}\label{frequently-asked-questions}}

Updated July 27, 2020

\begin{itemize}
\item ~
  \hypertarget{should-i-refinance-my-mortgage}{%
  \paragraph{Should I refinance my
  mortgage?}\label{should-i-refinance-my-mortgage}}

  \begin{itemize}
  \tightlist
  \item
    \href{https://www.nytimes.com/article/coronavirus-money-unemployment.html?action=click\&pgtype=Article\&state=default\&region=MAIN_CONTENT_3\&context=storylines_faq}{It
    could be a good idea,} because mortgage rates have
    \href{https://www.nytimes.com/2020/07/16/business/mortgage-rates-below-3-percent.html?action=click\&pgtype=Article\&state=default\&region=MAIN_CONTENT_3\&context=storylines_faq}{never
    been lower.} Refinancing requests have pushed mortgage applications
    to some of the highest levels since 2008, so be prepared to get in
    line. But defaults are also up, so if you're thinking about buying a
    home, be aware that some lenders have tightened their standards.
  \end{itemize}
\item ~
  \hypertarget{what-is-school-going-to-look-like-in-september}{%
  \paragraph{What is school going to look like in
  September?}\label{what-is-school-going-to-look-like-in-september}}

  \begin{itemize}
  \tightlist
  \item
    It is unlikely that many schools will return to a normal schedule
    this fall, requiring the grind of
    \href{https://www.nytimes.com/2020/06/05/us/coronavirus-education-lost-learning.html?action=click\&pgtype=Article\&state=default\&region=MAIN_CONTENT_3\&context=storylines_faq}{online
    learning},
    \href{https://www.nytimes.com/2020/05/29/us/coronavirus-child-care-centers.html?action=click\&pgtype=Article\&state=default\&region=MAIN_CONTENT_3\&context=storylines_faq}{makeshift
    child care} and
    \href{https://www.nytimes.com/2020/06/03/business/economy/coronavirus-working-women.html?action=click\&pgtype=Article\&state=default\&region=MAIN_CONTENT_3\&context=storylines_faq}{stunted
    workdays} to continue. California's two largest public school
    districts --- Los Angeles and San Diego --- said on July 13, that
    \href{https://www.nytimes.com/2020/07/13/us/lausd-san-diego-school-reopening.html?action=click\&pgtype=Article\&state=default\&region=MAIN_CONTENT_3\&context=storylines_faq}{instruction
    will be remote-only in the fall}, citing concerns that surging
    coronavirus infections in their areas pose too dire a risk for
    students and teachers. Together, the two districts enroll some
    825,000 students. They are the largest in the country so far to
    abandon plans for even a partial physical return to classrooms when
    they reopen in August. For other districts, the solution won't be an
    all-or-nothing approach.
    \href{https://bioethics.jhu.edu/research-and-outreach/projects/eschool-initiative/school-policy-tracker/}{Many
    systems}, including the nation's largest, New York City, are
    devising
    \href{https://www.nytimes.com/2020/06/26/us/coronavirus-schools-reopen-fall.html?action=click\&pgtype=Article\&state=default\&region=MAIN_CONTENT_3\&context=storylines_faq}{hybrid
    plans} that involve spending some days in classrooms and other days
    online. There's no national policy on this yet, so check with your
    municipal school system regularly to see what is happening in your
    community.
  \end{itemize}
\item ~
  \hypertarget{is-the-coronavirus-airborne}{%
  \paragraph{Is the coronavirus
  airborne?}\label{is-the-coronavirus-airborne}}

  \begin{itemize}
  \tightlist
  \item
    The coronavirus
    \href{https://www.nytimes.com/2020/07/04/health/239-experts-with-one-big-claim-the-coronavirus-is-airborne.html?action=click\&pgtype=Article\&state=default\&region=MAIN_CONTENT_3\&context=storylines_faq}{can
    stay aloft for hours in tiny droplets in stagnant air}, infecting
    people as they inhale, mounting scientific evidence suggests. This
    risk is highest in crowded indoor spaces with poor ventilation, and
    may help explain super-spreading events reported in meatpacking
    plants, churches and restaurants.
    \href{https://www.nytimes.com/2020/07/06/health/coronavirus-airborne-aerosols.html?action=click\&pgtype=Article\&state=default\&region=MAIN_CONTENT_3\&context=storylines_faq}{It's
    unclear how often the virus is spread} via these tiny droplets, or
    aerosols, compared with larger droplets that are expelled when a
    sick person coughs or sneezes, or transmitted through contact with
    contaminated surfaces, said Linsey Marr, an aerosol expert at
    Virginia Tech. Aerosols are released even when a person without
    symptoms exhales, talks or sings, according to Dr. Marr and more
    than 200 other experts, who
    \href{https://academic.oup.com/cid/article/doi/10.1093/cid/ciaa939/5867798}{have
    outlined the evidence in an open letter to the World Health
    Organization}.
  \end{itemize}
\item ~
  \hypertarget{what-are-the-symptoms-of-coronavirus}{%
  \paragraph{What are the symptoms of
  coronavirus?}\label{what-are-the-symptoms-of-coronavirus}}

  \begin{itemize}
  \tightlist
  \item
    Common symptoms
    \href{https://www.nytimes.com/article/symptoms-coronavirus.html?action=click\&pgtype=Article\&state=default\&region=MAIN_CONTENT_3\&context=storylines_faq}{include
    fever, a dry cough, fatigue and difficulty breathing or shortness of
    breath.} Some of these symptoms overlap with those of the flu,
    making detection difficult, but runny noses and stuffy sinuses are
    less common.
    \href{https://www.nytimes.com/2020/04/27/health/coronavirus-symptoms-cdc.html?action=click\&pgtype=Article\&state=default\&region=MAIN_CONTENT_3\&context=storylines_faq}{The
    C.D.C. has also} added chills, muscle pain, sore throat, headache
    and a new loss of the sense of taste or smell as symptoms to look
    out for. Most people fall ill five to seven days after exposure, but
    symptoms may appear in as few as two days or as many as 14 days.
  \end{itemize}
\item ~
  \hypertarget{does-asymptomatic-transmission-of-covid-19-happen}{%
  \paragraph{Does asymptomatic transmission of Covid-19
  happen?}\label{does-asymptomatic-transmission-of-covid-19-happen}}

  \begin{itemize}
  \tightlist
  \item
    So far, the evidence seems to show it does. A widely cited
    \href{https://www.nature.com/articles/s41591-020-0869-5}{paper}
    published in April suggests that people are most infectious about
    two days before the onset of coronavirus symptoms and estimated that
    44 percent of new infections were a result of transmission from
    people who were not yet showing symptoms. Recently, a top expert at
    the World Health Organization stated that transmission of the
    coronavirus by people who did not have symptoms was ``very rare,''
    \href{https://www.nytimes.com/2020/06/09/world/coronavirus-updates.html?action=click\&pgtype=Article\&state=default\&region=MAIN_CONTENT_3\&context=storylines_faq\#link-1f302e21}{but
    she later walked back that statement.}
  \end{itemize}
\end{itemize}

On Tuesday, Kenya Airways announced that the health authorities had
quarantined a passenger who traveled from Guangzhou.

Because of the Chinese New Year celebration, ``a good number of African
students living in Wuhan or Hubei traveled home before the extent of the
virus became clear,'' said Hannah Ryder, chief executive of the
Beijing-headquartered consultancy Development Reimagined.

``It's unclear how exposed they may have been and if governments have
the resources to check on them,'' she said.

\hypertarget{world-health-organization-buries-updated-global-risk-assessment-in-a-footnote}{%
\subsection{World Health Organization buries updated global risk
assessment in a
footnote.}\label{world-health-organization-buries-updated-global-risk-assessment-in-a-footnote}}

Image

Tedros Adhanom Ghebreyesus, left, the director-general of the World
Health Organization, with Wang Yi, China's foreign minister, in Beijing,
on Tuesday.Credit...Pool photo by Naohiko Hatta

The World Health Organization revised its global risk assessment for the
coronavirus outbreak from ``moderate'' to ``high,'' but concealed the
change in a footnote buried in a report published on Monday.

The change to the report, which coincided with a visit to China by the
organization's director-general, risked confusing the public about the
severity of the outbreak, which has killed more than 100 people in China
and been found in at least 14 countries.

In a statement, the organization said the director-general, Tedros
Adhanom Ghebreyesus, and Chinese officials ``discussed measures to
protect the health of Chinese and foreigners in outbreak areas,
including possible alternatives to evacuation of foreigners if there are
ways to accommodate them and protect their health.''

Chinese state-run media reported that Dr. Tedros met with President Xi
Jinping of China and spoke highly of Chinese efforts. Mr. Xi urged the
health organization to assess the epidemic in an ``objective, fair, calm
and rational manner.''

In Hubei, medical workers have complained about a desperate need for
resources to treat thousands of patients who have at times overwhelmed
hospitals.

The group, which is a United Nations body, was criticized when it
refused twice in recent days to declare the outbreak a global emergency,
despite its spread.

\hypertarget{shortage-of-test-kits-in-china-prompts-concern-that-cases-have-been-underreported}{%
\subsection{Shortage of test kits in China prompts concern that cases
have been
underreported.}\label{shortage-of-test-kits-in-china-prompts-concern-that-cases-have-been-underreported}}

Image

A condominium security guard checking temperatures in Beijing on
Monday.Credit...Nicolas Asfouri/Agence France-Presse --- Getty Images

A shortage of medical kits in China needed to quickly diagnose the
coronavirus has slowed the country's ability to respond to the outbreak
and fueled concerns that the number of cases has been underreported.

China's Medical Products Administration said on Sunday that it had
approved four new virus detection kits, including one that sequences the
genetic makeup of the disease.

But China's three leading medical device manufacturers said they did not
have the capacity to quickly produce the products, according to state
news media reports.

Residents in Wuhan who arrived at hospitals to seek testing were told
that medical workers did not have the kits needed to confirm a
diagnosis.

``For any new emerging virus, most local hospitals or public health
laboratories will not able to make a diagnosis'' said Yuen Kwok-yung,
the chairman of the infectious diseases department at Hong Kong
University. ``Thus many cases will not be investigated at all if they
are mild.''

A woman in Wuhan told The South China Morning Post that her uncle
learned he had viral pneumonia after a CT scan, but that the doctor
could not confirm it was the new virus because no testing kits were
available.

China may have to rely on outside technical support as front line
responders battle to contain the virus's spread, experts said. The Bill
and Melinda Gates Foundation said Sunday that it would commit \$5
million to help China respond to the crisis, including ``efforts to
identify and confirm cases.''

\hypertarget{us-health-officials-recommend-avoiding-china-and-businesses-follow-suit}{%
\subsection{U.S. health officials recommend avoiding China, and
businesses follow
suit.}\label{us-health-officials-recommend-avoiding-china-and-businesses-follow-suit}}

Image

Wuhan, China, the epicenter of the outbreak, on Tuesday.Credit...Arek
Rataj/Associated Press

As the outbreak continues to spread, global companies have begun to
limit their workers' travel to mainland China, and China's biggest
companies have urged employees to work from home.

On Monday, health officials in the United States urged travelers to
\href{https://wwwnc.cdc.gov/travel/notices/warning/novel-coronavirus-china}{avoid
any nonessential travel to China}, and many companies cited that as
justification for internal travel bans. The
\href{https://wwwnc.cdc.gov/travel/notices/warning/novel-coronavirus-china}{new
guidance}, from the Centers for Disease Control and Prevention, warned
that transportation in and out of Hubei Province, the center of the
outbreak, is restricted, and that there is ``limited access to adequate
medical care in affected areas.''

Companies with large operations or interest in China, like General
Motors, Honeywell, Bloomberg and Facebook, have all warned employees not
to travel within mainland China in a flurry of emails in recent days.

Honeywell, which has offices and operations across China, said it had
restricted travel to certain regions, without specifying them. A
spokesperson for General Motors said the company had issued a global ban
on travel to China, under which only ``business-critical'' travel would
be allowed.

Bloomberg told its employees in Hong Kong and mainland China to work
remotely until further notice, and it barred other employees from
traveling to either place for the next 30 days, according to an email
seen by The New York Times. Facebook said it asked all employees to
suspend nonessential travel and asked those who had recently been in
China to work from home for a period of time.

The authorities in China have extended the Lunar New Year holiday to
Feb. 3, and some of China's biggest cities have gone further, telling
businesses not to open until the next week. The country's biggest
technology companies, including Alibaba, Tencent, Bytedance, Sina,
Maimai, Netease and Didi, told employees to work from home from Feb. 3
to Feb. 10.

Netease, an internet and entertainment platform, asked employees
returning from another city within China to quarantine themselves for 14
days.

\hypertarget{worries-rise-about-the-outbreaks-economic-impact}{%
\subsection{Worries rise about the outbreak's economic
impact.}\label{worries-rise-about-the-outbreaks-economic-impact}}

Image

Shoppers at a mall in Bangkok on Monday. Thailand is one of more than a
dozen countries in which the new coronavirus has been
reported.Credit...Rungroj Yongrit/EPA, via Shutterstock

After sharp losses around the world on Monday, investors on Tuesday
continued to assess the
\href{https://www.nytimes.com/2020/01/27/business/coronavirus-china-economic-impact.html}{long-term
economic effects} of the coronavirus epidemic.

The verdict was mixed. Investors abandoned stocks in Asia, while markets
in Europe steadied. In the United States, the S\&P 500 was up more than
1 percent on Tuesday.

Many of Asia's stock markets were closed for the Lunar New Year holiday,
but those that were open --- Japan's and South Korea's --- fell as
futures trading in China slumped. Money poured into safe-haven assets
like gold and pushed up the value of the United States dollar.

Hong Kong's stock market will reopen on Wednesday. In China, where
authorities have extended the New Year holiday by a week, the major
exchanges in Shenzhen and Shanghai said they would remain closed until
Feb. 3.

``The coronavirus is the No. 1 threat to financial markets currently as
global investors are becoming jittery on the uncertainty,'' said Nigel
Gre, the founder of the investment group deVere Group.

\hypertarget{back-in-new-york-from-wuhan--and-into-a-self-imposed-quarantine}{%
\subsection{Back in New York from Wuhan \ldots{} and into a self-imposed
quarantine.}\label{back-in-new-york-from-wuhan--and-into-a-self-imposed-quarantine}}

Image

Downtown Flushing, Queens, last week.Credit...An Rong Xu for The New
York Times

Some of the last passengers who arrived at Kennedy International Airport
before direct flights from Wuhan were canceled have
\href{https://www.nytimes.com/2020/01/27/nyregion/new-york-city-coronavirus.html}{quarantined
themselves at home.}

Scott Liu, 56, who leads an association for immigrants from Hubei, said
he confined himself to his house in Queens. He said he and his fellow
passengers on the Wuhan flight learned of the lockdown in that city
mid-flight.

He said he has not felt sick, but is taking precautions because he knows
symptoms take time to appear. His friends have dropped off on his
doorstep traditional Lunar New Year dishes like lotus root and pork rib
soup, salted fish and dumplings. Last year, they were host to a big New
Year celebration at a banquet hall in Flushing.

This year, Mr. Liu said, ``all the events here are canceled.''

``Everybody is in a state of panic,'' he added.

Reporting was contributed by Chris Buckley, Russell Goldman, Elaine Yu,
Raymond Zhong, Austin Ramzy, Sui-Lee Wee, Alexandra Stevenson, Cao Li,
Eimi Yamamitsu, Tiffany May, Joseph Goldstein, Jeffrey E. Singer, Peter
S. Goodman, Roni Caryn Rabin, Motoko Rich, Paul Mozur, Christopher F.
Schuetze, Abdi Latif Dahir, Simon Marks, Ben Dooley, Eimi Yamamitsu,
Patricia Cohen and Aurelien Breeden. Jin Wu, Zoe Mou, Albee Zhang, Amber
Wang, Yiwei Wang and Claire Fu contributed research.

Advertisement

\protect\hyperlink{after-bottom}{Continue reading the main story}

\hypertarget{site-index}{%
\subsection{Site Index}\label{site-index}}

\hypertarget{site-information-navigation}{%
\subsection{Site Information
Navigation}\label{site-information-navigation}}

\begin{itemize}
\tightlist
\item
  \href{https://help.nytimes.com/hc/en-us/articles/115014792127-Copyright-notice}{©~2020~The
  New York Times Company}
\end{itemize}

\begin{itemize}
\tightlist
\item
  \href{https://www.nytco.com/}{NYTCo}
\item
  \href{https://help.nytimes.com/hc/en-us/articles/115015385887-Contact-Us}{Contact
  Us}
\item
  \href{https://www.nytco.com/careers/}{Work with us}
\item
  \href{https://nytmediakit.com/}{Advertise}
\item
  \href{http://www.tbrandstudio.com/}{T Brand Studio}
\item
  \href{https://www.nytimes.com/privacy/cookie-policy\#how-do-i-manage-trackers}{Your
  Ad Choices}
\item
  \href{https://www.nytimes.com/privacy}{Privacy}
\item
  \href{https://help.nytimes.com/hc/en-us/articles/115014893428-Terms-of-service}{Terms
  of Service}
\item
  \href{https://help.nytimes.com/hc/en-us/articles/115014893968-Terms-of-sale}{Terms
  of Sale}
\item
  \href{https://spiderbites.nytimes.com}{Site Map}
\item
  \href{https://help.nytimes.com/hc/en-us}{Help}
\item
  \href{https://www.nytimes.com/subscription?campaignId=37WXW}{Subscriptions}
\end{itemize}
