Sections

SEARCH

\protect\hyperlink{site-content}{Skip to
content}\protect\hyperlink{site-index}{Skip to site index}

\href{https://www.nytimes.com/section/health}{Health}

\href{https://myaccount.nytimes.com/auth/login?response_type=cookie\&client_id=vi}{}

\href{https://www.nytimes.com/section/todayspaper}{Today's Paper}

\href{/section/health}{Health}\textbar{}Mask Hoarders May Raise Risk of
a Coronavirus Outbreak in the U.S.

\url{https://nyti.ms/36yotTf}

\begin{itemize}
\item
\item
\item
\item
\item
\end{itemize}

\href{https://www.nytimes.com/news-event/coronavirus?action=click\&pgtype=Article\&state=default\&region=TOP_BANNER\&context=storylines_menu}{The
Coronavirus Outbreak}

\begin{itemize}
\tightlist
\item
  live\href{https://www.nytimes.com/2020/08/02/world/coronavirus-updates.html?action=click\&pgtype=Article\&state=default\&region=TOP_BANNER\&context=storylines_menu}{Latest
  Updates}
\item
  \href{https://www.nytimes.com/interactive/2020/us/coronavirus-us-cases.html?action=click\&pgtype=Article\&state=default\&region=TOP_BANNER\&context=storylines_menu}{Maps
  and Cases}
\item
  \href{https://www.nytimes.com/interactive/2020/science/coronavirus-vaccine-tracker.html?action=click\&pgtype=Article\&state=default\&region=TOP_BANNER\&context=storylines_menu}{Vaccine
  Tracker}
\item
  \href{https://www.nytimes.com/interactive/2020/07/29/us/schools-reopening-coronavirus.html?action=click\&pgtype=Article\&state=default\&region=TOP_BANNER\&context=storylines_menu}{What
  School May Look Like}
\item
  \href{https://www.nytimes.com/live/2020/07/31/business/stock-market-today-coronavirus?action=click\&pgtype=Article\&state=default\&region=TOP_BANNER\&context=storylines_menu}{Economy}
\end{itemize}

Advertisement

\protect\hyperlink{after-top}{Continue reading the main story}

Supported by

\protect\hyperlink{after-sponsor}{Continue reading the main story}

Global health

\hypertarget{mask-hoarders-may-raise-risk-of-a-coronavirus-outbreak-in-the-us}{%
\section{Mask Hoarders May Raise Risk of a Coronavirus Outbreak in the
U.S.}\label{mask-hoarders-may-raise-risk-of-a-coronavirus-outbreak-in-the-us}}

Stores are selling out of masks, and health care workers risk infection
if they cannot get the protective gear.

\includegraphics{https://static01.nyt.com/images/2020/01/29/science/29VIRUS-MASKS4/29VIRUS-MASKS4-articleLarge.jpg?quality=75\&auto=webp\&disable=upscale}

\href{https://www.nytimes.com/by/donald-g-mcneil-jr}{\includegraphics{https://static01.nyt.com/images/2018/06/13/multimedia/author-donald-g-mcneil-jr/author-donald-g-mcneil-jr-thumbLarge-v4.png}}

By \href{https://www.nytimes.com/by/donald-g-mcneil-jr}{Donald G. McNeil
Jr.}

\begin{itemize}
\item
  Published Jan. 29, 2020Updated March 22, 2020
\item
  \begin{itemize}
  \item
  \item
  \item
  \item
  \item
  \end{itemize}
\end{itemize}

Even though there are only five cases of
\href{https://www.nytimes.com/2020/01/30/podcasts/the-daily/coronavirus.html}{Wuhan
coronavirus} in the United States, the
\href{https://www.cnn.com/2020/01/28/health/coronavirus-us-masks-prevention-trnd/index.html}{mask
hoarding}
\href{https://finance.yahoo.com/news/china-virus-outbreak-triggers-global-082240405.html}{has
begun}.

Some pharmacies
\href{https://gothamist.com/news/chinese-new-yorkers-worried-about-coronavirus-are-wearing-masks-do-they-work}{report
being entirely sold out} of masks. Some
\href{https://www.businessinsider.com/amazon-sold-out-sellers-warn-counterfeit-masks-coronavirus-2020-1}{popular
sellers on Amazon} say deliveries will be delayed for weeks.

Although masks actually
\href{https://www.nytimes.com/2020/01/28/opinion/coronavirus-prevention-tips.html?action=click\&module=Opinion\&pgtype=Homepage}{do
little to protect healthy people}, the prospect of shortages created by
panic buying worries some public health experts.

Masks are thought to slow the spread of disease when they are worn by
sick people in crowded places like emergency rooms, offices, subways and
buses. By containing coughs and sneezes, masks stop virus-laden droplets
from being spewed into the air and onto nearby surfaces.

But hoarding by those who are well means that hospitals, clinics and
doctors' offices could run short.
\href{https://www.nytimes.com/2020/02/14/world/asia/china-coronavirus-doctors.html}{Doctors
and nurses treating patients for respiratory infections} should wear
masks and replace them often --- as soon as they become soggy, the
Centers for Disease Control and Prevention says.

The C.D.C. is now reaching out to manufacturers to head off the
possibility of shortages, especially in hospitals, an agency official
said.

``We see panic ordering and buying that doesn't reflect the actual
need,'' said Dr. Anita Patel, the senior adviser for pandemic medical
care in the agency's influenza coordination unit. ``We're talking to
manufacturers. They understand the situation, and I'm confident that
they are being responsible. The health care industry is their biggest
customer.''

On Tuesday, Alex M. Azar II, the secretary of health and human services,
said it was ``unnecessary'' for Americans to buy masks now.

``In the U.S., the risk to any individual American is extremely low,''
he said.

Some experts want the government to step in and educate the public about
the dangers of hoarding.

``I worked through the 2009 H1N1 flu epidemic at Yale Hospital, and we
ran out of N-95 masks --- and being in a high-risk situation without
enough masks is not a good feeling,'' said Dr. Peter Rabinowitz,
co-director of the University of Washington MetaCenter for Pandemic
Preparedness and Global Health Security in Seattle.

\hypertarget{latest-updates-global-coronavirus-outbreak}{%
\section{\texorpdfstring{\href{https://www.nytimes.com/2020/08/01/world/coronavirus-covid-19.html?action=click\&pgtype=Article\&state=default\&region=MAIN_CONTENT_1\&context=storylines_live_updates}{Latest
Updates: Global Coronavirus
Outbreak}}{Latest Updates: Global Coronavirus Outbreak}}\label{latest-updates-global-coronavirus-outbreak}}

Updated 2020-08-02T17:52:35.962Z

\begin{itemize}
\tightlist
\item
  \href{https://www.nytimes.com/2020/08/01/world/coronavirus-covid-19.html?action=click\&pgtype=Article\&state=default\&region=MAIN_CONTENT_1\&context=storylines_live_updates\#link-34047410}{The
  U.S. reels as July cases more than double the total of any other
  month.}
\item
  \href{https://www.nytimes.com/2020/08/01/world/coronavirus-covid-19.html?action=click\&pgtype=Article\&state=default\&region=MAIN_CONTENT_1\&context=storylines_live_updates\#link-780ec966}{Top
  U.S. officials work to break an impasse over the federal jobless
  benefit.}
\item
  \href{https://www.nytimes.com/2020/08/01/world/coronavirus-covid-19.html?action=click\&pgtype=Article\&state=default\&region=MAIN_CONTENT_1\&context=storylines_live_updates\#link-2bc8948}{Its
  outbreak untamed, Melbourne goes into even greater lockdown.}
\end{itemize}

\href{https://www.nytimes.com/2020/08/01/world/coronavirus-covid-19.html?action=click\&pgtype=Article\&state=default\&region=MAIN_CONTENT_1\&context=storylines_live_updates}{See
more updates}

More live coverage:
\href{https://www.nytimes.com/live/2020/07/31/business/stock-market-today-coronavirus?action=click\&pgtype=Article\&state=default\&region=MAIN_CONTENT_1\&context=storylines_live_updates}{Markets}

``There's no rational reason why everyone needs to run out and get
masks,'' he added. ``Public health officials should be talking about
this.''

\includegraphics{https://static01.nyt.com/images/2020/01/29/science/29VIRUS-MASKS2/29VIRUS-MASKS2-articleLarge.jpg?quality=75\&auto=webp\&disable=upscale}

\includegraphics{https://static01.nyt.com/images/2017/01/29/podcasts/the-daily-album-art/the-daily-album-art-articleInline-v2.jpg?quality=75\&auto=webp\&disable=upscale}

\hypertarget{listen-to-the-daily-a-viruss-journey-across-china}{%
\subsubsection{Listen to `The Daily': A Virus's Journey Across
China}\label{listen-to-the-daily-a-viruss-journey-across-china}}

Government officials thought they could control the disease and cover up
its deadly wake. They were wrong.

transcript

Back to The Daily

bars

0:00/24:27

-24:27

transcript

\hypertarget{listen-to-the-daily-a-viruss-journey-across-china-1}{%
\subsection{Listen to `The Daily': A Virus's Journey Across
China}\label{listen-to-the-daily-a-viruss-journey-across-china-1}}

\hypertarget{hosted-by-michael-barbaro-produced-by-neena-pathak-annie-brown-and-eric-krupke-with-help-from-kelly-prime-and-edited-by-mj-davis-lin-and-mike-benoist}{%
\subsubsection{Hosted by Michael Barbaro; produced by Neena Pathak,
Annie Brown and Eric Krupke; with help from Kelly Prime; and edited by
M.J. Davis Lin and Mike
Benoist}\label{hosted-by-michael-barbaro-produced-by-neena-pathak-annie-brown-and-eric-krupke-with-help-from-kelly-prime-and-edited-by-mj-davis-lin-and-mike-benoist}}

\hypertarget{government-officials-thought-they-could-control-the-disease-and-cover-up-its-deadly-wake-they-were-wrong}{%
\paragraph{Government officials thought they could control the disease
and cover up its deadly wake. They were
wrong.}\label{government-officials-thought-they-could-control-the-disease-and-cover-up-its-deadly-wake-they-were-wrong}}

\begin{itemize}
\item
  michael barbaro\\
  From The New York Times, I'm Michael Barbaro. This is ``The Daily.''
\item
  {[}music{]}\\
  Today: China says it has made lifesaving reforms since the last time
  it was the source of a public health crisis 17 years ago. So why is
  the deadly coronavirus spreading so rapidly across the country? My
  colleague, Javier Hernández, reports from the center of the outbreak.

  It's Thursday, January 30.

  Javier, how did you first hear about this outbreak?
\item
  javier hernández\\
  Well, we started hearing reports in early January that there was this
  mysterious virus that was affecting Wuhan, which is a city in central
  China.
\item
  archived recording\\
  Staff at Wuhan Hospital are working around the clock to identify a
  mystery virus.
\end{itemize}

javier hernández

People were falling ill to this kind of pneumonia-like virus, which
scientists were calling a coronavirus.

\begin{itemize}
\tightlist
\item
  archived recording\\
  Experts believe this is a new kind of coronavirus, which typically
  causes symptoms of the common cold, but in some rare cases, can lead
  to pneumonia.
\end{itemize}

javier hernández

It's a very frightening virus that spreads from animals to humans. It
was believed to be behind some very serious respiratory illnesses, so it
sounded pretty scary. But the official narrative was that this strain
couldn't pass from human to human. It would only go from animals to
humans.

\begin{itemize}
\tightlist
\item
  archived recording\\
  The potentially deadly coronavirus is thought to have spread from
  animals into humans at a live produce market in Wuhan.
\end{itemize}

javier hernández

And at that point, they were saying that it was all originating at a
single meat market in central China. This meat market sold wild animals,
things like bamboo rats, badgers, wolves even. And the assumption was
that people had come into contact with these animals and picked up this
virus.

michael barbaro

And it sounds like because it was one meat market, and because this
virus could only pass from animals to humans, that this is a pretty
small problem.

javier hernández

At this point, the reports were that only about 130 people had been
infected and that, of those, only about four had died.

\begin{itemize}
\tightlist
\item
  archived recording\\
  Chinese officials say the outbreak is under control.
\end{itemize}

javier hernández

And so the prognosis was that the outbreak was controllable, that it was
treatable and that things would be O.K.

michael barbaro

And Javier, what are you thinking when you hear the Chinese government
talking this way, saying that everything is going to be O.K.?

javier hernández

We were a bit skeptical. We knew that the Chinese government had a
history of downplaying outbreaks like this. And as we saw more and more
reports in the international media of this virus spreading, we began to
question whether the officials were being totally upfront about what was
happening.

michael barbaro

And what do you mean when you say a history of downplaying situations
like this?

javier hernández

Well before this virus, there was the outbreak of SARS 17 years ago. And
it turned into this global health crisis that infected more than 8,000
people. It killed more than 800 people. And a big part of the reason
that it spread so violently was because the Chinese government didn't
tell the world. And there was this period of months and months when it
was spreading very rapidly in China, and the world just didn't know
about it. And for a lot of Chinese today, that experience with SARS
makes them very skeptical of anything that the government says when it
comes to public health.

michael barbaro

And I imagine a reporter like you would be similarly skeptical.

javier hernández

Right. We really wanted to make sure that we got this one right.

We wanted to double check that this was, in fact, a small scale outbreak
as the government was portraying it. And we wanted to talk to people who
were getting sick with this virus to get a sense of what it was like for
them --- how severe it was, whether they were getting the care they
needed.

michael barbaro

And so what do you do?

javier hernández

So I live in Beijing. So I rushed to our bureau here and grabbed masks,
enough to last me a couple days, and then headed to the airport and
boarded a flight to Wuhan. And I just remember everyone on board, almost
everyone, was wearing a mask. I was just struck by that. You never see
people wear masks on this scale. And so it felt like there was this
sense of anxiety already in the air even before we landed.

{[}music{]}

Once I got there, I rushed off to meet an ambulance doctor who had been
posting on social media about this outbreak. And we had contacted him
because he was challenging the official narrative about what was
happening. He was saying he didn't believe the official numbers. He
thought that this could be another SARS-like outbreak.

michael barbaro

And what was happening to his posts once they went online?

javier hernández

They lasted for a while, and then they would be taken down, just
disappeared. It was clear that the government was cleaning and trying to
scrub the internet of any critical questions about official data. And so
we wanted to talk to him about what he was seeing on the ground.

michael barbaro

And what did he tell you?

javier hernández

He told me that he felt like people weren't taking enough precautions,
that the government wasn't being urgent enough. And he worked in a
hospital and had a lot of interaction with doctors and other medical
workers. So he felt like his job could be at stake if he were to come
forward publicly.

michael barbaro

So the numbers the Chinese government is providing are relatively small.
This ambulance doctor you're meeting with is saying the problem is
bigger. So what do you do to try to reconcile those two pieces of
information?

javier hernández

We wanted to find the people whose relatives had fallen ill to this
virus. And so we were looking online for people who were reporting
symptoms of fever, cough, basically anything that was associated with
this kind of virus. And we stumbled on a guy from Shanghai. He was a
40-year-old architect, and he told me that his stepmother had died just
a couple of days earlier from a mysterious pneumonia-like illness. And
so I found him, sent him a message, and we agreed to meet at a cafe.
{[}CHATTER{]}

michael barbaro

And what is the story of what happened to his stepmom?

javier hernández

He tells me that she was a very healthy, normal 65-year-old woman who, a
couple of weeks earlier, had started to feel like she was having a flu.
So she had some coughing. She had kind of a sore throat, but it didn't
seem like anything that bad. But it suddenly began to worsen. She was
having a fever. She needed a respirator to breathe. And all very
suddenly she was sent to a contagious disease ward, and the family was
told that she would likely die.

michael barbaro

And what did she or her stepson understand about what was behind her
illness?

javier hernández

They didn't really understand much. For days and days, according to Mr.
Wei, the family had tried to get her tested for this coronavirus, this
mysterious virus that was spreading across Wuhan. And the doctors and
other medical workers refused.

He pulled out his phone and showed me her death certificate, and all it
said was that she died of severe pneumonia.

michael barbaro

But why wouldn't they test her, and what does it tell you that they
wouldn't test her?

javier hernández

It became clear to me in that moment that there seemed to be something
going on here. Were these hospitals just not prepared? Did they not have
enough testing kits? Were they running out of tests? Had they tested so
many people that they didn't have any tests left? Were they just not
counting people anymore because they weren't even diagnosing them? And
if she wasn't diagnosed with this illness, then maybe she wasn't even
included in the official count. And one other thing really stuck with me
from my interview with Mr. Wei. And that was that the hospital,
according to him, told them that his mother's body had to be burned
immediately, had to be cremated immediately. {[}CHATTER{]}

michael barbaro

And what does that tell you? What does that mean to you that they're
burning these bodies?

javier hernández

It began to put in my head the sense that the hospital workers were
beginning to think that this illness was contagious. And the fact that
they wanted that body burned immediately began to suggest that perhaps
they thought it posed a threat to other people.

Her death got us thinking about whether there were other people like
her. And when you went online and search social media, you could quickly
find others who were reporting similar experiences --- people saying
that they went to hospitals, they brought their sick relatives and were
simply turned away. We began to kind of add things up, and it seemed
like the numbers weren't right.

michael barbaro

So at this point, it sounds like your skepticism is growing, and it's
sounding pretty warranted. So what do you do next?

javier hernández

So I head back to my hotel and start to try to make sense of all of
this. And as I'm sitting there in my room, I begin to see these reports
emerging on social media ---

\begin{itemize}
\tightlist
\item
  archived recording\\
  For the first time since the mysterious pneumonia-like illness broke
  out in China, the country's health authorities admitted the disease
  could be contagious.
\end{itemize}

javier hernández

--- that one of China's top health experts has acknowledged that this
virus is now spreading from person to person, which had never been the
case before.

\begin{itemize}
\item
  archived recording (zhong nanshan)\\
  {[}SPEAKING CHINESE{]}
\item
  archived recording (translator)\\
  We considered risks of this before, but now evidence has confirmed
  that it is contagious among humans.
\end{itemize}

javier hernández

And he's an 83-year-old guy that they dug out of retirement essentially.
He was a renowned expert during the SARS crisis. And suddenly, he's all
over social media, telling people that this crisis is much worse than
had been previously known.

michael barbaro

So this doctor is very much validating what you have been hearing.

javier hernández

Yes. He's saying that there was even a case where one patient was able
to infect 14 medical workers.

michael barbaro

Wow.

javier hernández

And I begin to see the panic kind of spread across social media.

\begin{itemize}
\item
  archived recording 1\\
  Say I have coronavirus. Could I give it to you right now?
\item
  archived recording 2\\
  It's possible, if you're coughing and sneezing. If you have a fever,
  it can be transmitted.
\item
  archived recording 3\\
  But the spread of Corona virus is accelerating.
\end{itemize}

javier hernández

This is the moment when the world begins to wake up.

\begin{itemize}
\item
  archived recording 1\\
  Meanwhile, new cases are being confirmed outside of China as well.
\item
  archived recording 2\\
  Thailand has reported the first case of the Wuhan coronavirus found
  outside of China.
\item
  archived recording 3\\
  --- because now they're talking Thailand, Singapore, possibly
  Scotland. The U.K. is bracing for it to go in there. We have airports
  here in the United States very concerned about it ---
\item
  archived recording 4\\
  That are doing surveillance.
\item
  archived recording 5\\
  --- here in New York, Chicago and L.A. Every day it seems growing
  exponentially.
\item
  archived recording 6\\
  It is. It's growing.
\end{itemize}

javier hernández

The numbers after that point just keep spiraling and spiraling.

\begin{itemize}
\item
  archived recording 1\\
  The number of people who have died ---
\item
  archived recording 2\\
  The death toll climbing once again overnight.
\item
  archived recording 3\\
  The death toll from a deadly viral outbreak of the new coronavirus ---
\item
  archived recording 4\\
  26 deaths so far from the ---
\item
  archived recording 5\\
  --- has now passed 40.
\item
  archived recording 6\\
  {[}SPEAKING CHINESE{]}
\item
  archived recording 7\\
  It's not just Chinese people who are worried. Everyone is worried.
  They're all afraid of dying.
\end{itemize}

javier hernández

You just hear people wondering whether the virus is ever going to be
contained.

michael barbaro

We'll be right back.

And Javier, what are you thinking and feeling at this moment? Because
you're in this city. You're interacting with people who are taking care
of those who are sickened and infected by this illness.

javier hernández

Yeah.

michael barbaro

Are you anxious?

javier hernández

Well I'm beginning to feel like this is much worse than I thought. And I
think a sense of paranoia can easily settle in sometimes in these
situations. I had masks. I was washing my hands all the time. But I
couldn't help but think about every button in the elevator that I
touched, every surface and every cough or sniffle that I saw around me.
The virus could be anywhere, so I had to be careful.

michael barbaro

I mean, understandably.

javier hernández

So I finished up my reporting, and I grabbed a flight back to Beijing.

\begin{itemize}
\tightlist
\item
  archived recording\\
  Just one day before China's massive Lunar New Year holiday, as
  hundreds of millions crisscross the country to celebrate with
  families, an unprecedented act.
\end{itemize}

javier hernández

And then the very next day, I hear ---

\begin{itemize}
\tightlist
\item
  archived recording\\
  Today, the entire city of Wuhan, population 11 million, is on
  lockdown.
\end{itemize}

javier hernández

The Chinese government has placed this entire city under lockdown.

michael barbaro

Wow.

\begin{itemize}
\tightlist
\item
  archived recording\\
  Trains, flights, buses, and ferries have been canceled. Even public
  transport within Wuhan has ground to a halt.
\end{itemize}

michael barbaro

So the city you have just left has basically told everybody else who
remains that they're not going anywhere.

javier hernández

Right. This city is being closed off in a way that China has never done
before --- or even any other major modern city, really, hasn't done it
in recent times.

\begin{itemize}
\tightlist
\item
  archived recording\\
  Across China tonight, an expanding lockdown to contain an epidemic.
\end{itemize}

javier hernández

And it was quickly becoming clear to the government that this wasn't
just a local problem.

\begin{itemize}
\tightlist
\item
  archived recording\\
  Travel bans in over a dozen cities affecting 35 million people.
\end{itemize}

javier hernández

They quickly expanded it to not just Wuhan, but to other cities, so that
there were tens of millions of people who were essentially forced to
stay at home and not allowed to go out. They've just put in place the
biggest lockdown that we've ever seen and what experts are saying is the
biggest experiment in public health that they've ever seen.

\begin{itemize}
\item
  archived recording 1\\
  Despite their efforts, authorities say the virus has not been stopped.
\item
  archived recording 2\\
  Much depends on whether the measures set in place by the Chinese
  government will prove effective.
\end{itemize}

michael barbaro

It's hard to imagine most any other country being able to mount that
kind of a response. I mean, I'm just trying to fathom an American city
somehow being locked down.

javier hernández

So this is what it looks like when China's authoritarian system is in
full force. There's no choice for people to leave. Many people are stuck
there. They are going to hospitals that are overcrowded, but they can't
get the health care they need. Doctors are complaining about a lack of
medical supplies and critical items like masks and goggles. And you get
the sense that people are kind of stuck with what they have, and that's
the bargain they've made by living in this system. They have no choice
but to follow the government's orders. They can't push back. They can't
swim against the current here. Everyone's essentially forced to comply
with this mass lockdown.

michael barbaro

You know, from everything we've learned about China and its
authoritarian government, it seems like it controls every aspect of its
citizens' lives and it surveils every aspect of citizens' lives. And
you've just walked us through how they were able to essentially suddenly
drop a wall around Wuhan. So knowing all that, shouldn't China's
government have been able to identify and control this outbreak before
it got out of hand?

javier hernández

You would think that, because the government is so well known for being
able to pull off these massive displays of citizen mobilization --- it's
been known to be able to pull off these technological feats. But at the
same time, there's a flip side of this coin, which is that China's
authoritarian culture, in many ways, set the stage for this crisis.

michael barbaro

What do you mean?

javier hernández

Well for decades, China has built this system, this ruthless system in
which if you are an official in the Communist Party, you are expected to
be almost perfect. If anything goes bad, you are the one who is going to
take responsibility. You are the one who is going to fall. And this has
created an incentive system where local officials fear saying anything
about bad news. They worry that if they are found to have done something
wrong, that they will lose their power. So in a situation like this, the
incentive is to cover up. It's to conceal. It's to delay. It's to try to
get a handle on these problems on your own, hoping that perhaps nobody
will ever hear about it.

michael barbaro

So by the time something like, say, a medical crisis gets really big, it
may be too late for the local officials who have been trying to contain
it themselves and keep it from Beijing.

javier hernández

Exactly. These kinds of dynamics played a huge role in the scale of the
SARS outbreak. It was clear in this case that local officials knew
exactly what was going on. They knew that people were dying of this
illness. But for months and months, they didn't want to report it up the
chain. Instead, they tried to cover it up. They tried to see if they
could perhaps deal with it secretly, and maybe nobody would ever find
out about it. They hoped that Beijing would know about it. But
eventually it broke.

michael barbaro

So did China learn from that experience with SARS, that dynamic that led
to that breakdown, that cover-up?

javier hernández

China said it would make a lot of changes to its system after the SARS
crisis. It said that it would expand its disease reporting system so
that these kinds of reports from the local provinces would come to the
central government in a more timely fashion. They promised to be more
transparent in the release of data and other things. But what they
didn't change was this authoritarian culture, where people fear bringing
bad news. And so that has actually gotten worse under President Xi.

michael barbaro

Why?

javier hernández

He has made himself out to be the most powerful leader since Mao. He is
somebody who's always speaking about this great ascendant moment for
China in which China is going to be this superpower. And anything that
goes against Xi's vision of this harmonious, resurgent China is going to
be seen as a problem. And the people who are creating that problem will
pay the consequences. And when I was there in Wuhan, I could sense the
fear just rippling across all parts of the society. There were people
like the ambulance doctor, who was afraid of challenging the official
statistics. There were hospitals that seemed to be paralyzed, that
didn't want to test patients for fear of knowing the results. They
didn't want to be seen as speaking out or telling the truth or bringing
this unsavory story into public view.

michael barbaro

So that had trickled down all the way to the frontline health care
workers, who are supposed to be treating this and sounding the alarm.

javier hernández

Right. They're fearful of being seen as responsible for this crisis.
They don't want to stand out. And when you think about where this virus
might be headed next --- to other provinces, to other cities --- you
have to wonder if these same dynamics would be playing out again. If
people will stay silent, if they will not report official cases, because
they fear for their jobs and they fear for their livelihoods.

michael barbaro

So it seems like whatever reforms the Chinese government made after the
failures of SARS, that in the end, none of them really matter, because
it hasn't changed the problem that's at the root of this.

javier hernández

At the root of all this is a sense of fear that is both the Communist
Party's strength, but it's also a huge weakness. They can mobilize
entire cities to be on lockdown. They can convince people to stay
indoors. They can scrub social media of information that counters their
narrative. That's all very effective when you're trying to get the
public behind your policies. But at the end of the day, the fear is also
a huge vulnerability. It creates this system where people are unwilling
to speak out, unwilling to bring problems that are really important and
that affect people's lives.

michael barbaro

Right. I mean, for the Chinese leadership, that works until it doesn't,
right? Until you very much need for people to stop being afraid and to
speak out about something like a public health crisis.

javier hernández

Exactly.

And so when you look at the culture, you wonder whether China can
actually contain these viruses, whether we will continue to live in a
world where the internal politics of the party are going to put lives
around the world in danger.

michael barbaro

Javier, thank you.

javier hernández

Thanks, Michael.

michael barbaro

On Wednesday, the total number of people infected by the coronavirus in
mainland China surpassed those infected with the SARS virus during that
epidemic. As of Wednesday evening, the virus had infected more than
6,000 people in mainland China. Later today, the World Health
Organization will convene its emergency committee to determine whether
the outbreak amounts to a public health emergency of international
concern, its most severe classification.

We'll be right back.

Here's what else you need to know today. The Times reports that the
White House has sent former national security adviser, John Bolton, a
letter, warning him not to publish a book in which he recounts speaking
to President Trump about the quid pro quo with Ukraine at the heart of
the impeachment trial. In the letter, the White House claims that the
book contains significant amounts of classified information that could
harm U.S. national security. The letter was sent before The Times
published a story describing the book's contents, a story that has
further fueled calls for Bolton to testify in the Senate trial. Senators
are expected to vote on whether to call witnesses like Bolton later
today.

{[}music{]}

That's it for ``The Daily.'' I'm Michael Barbaro. See you tomorrow.

During the West African Ebola outbreak in 2014, when a few cases turned
up in the United States, some hospitals were unable to get waterproof
Tyvek suits ``because there was a run on them,'' said Dr. Amesh Adalja,
a scholar at the Center for Health Security at the Johns Hopkins
Bloomberg School of Public Health in Baltimore.

Panicked people engage in irrational impulse buying, he said: ``People
like to buy duct tape during emergencies --- it's psychologically
soothing.''

But artificial shortages can harm those who really need the goods.

``I think public health authorities like the H.H.S. or the C.D.C. or the
Surgeon General should be talking about the repercussions of a run on
masks,'' Dr. Adalja said.

Kristen Nordlund, a C.D.C. spokeswoman, said she would ask
\href{https://www.cdc.gov/about/leadership/leaders/ncird.html}{Dr. Nancy
Messonnier}, director of the National Center for Immunization and
Respiratory Diseases to address that issue in her next public briefing.

\href{https://www.nytimes.com/2020/03/22/business/coronavirus-n95-masks-target.html}{Masks}
are part of the medical equipment stored in the National Strategic
Stockpile, which is distributed in government-controlled warehouses
around the country. But experts said they did not know how many were in
storage or how long the supply was projected to last in an epidemic.

On Tuesday, major pharmacy chains said they were seeing spot shortages
but were not yet restricting sales. The CVS chain had stores running out
of masks and was resupplying them ``as quickly as possible,'' said
Stephanie Cunha, a company spokeswoman.

Walgreens and Duane Reade pharmacies saw greater demand for
\href{https://www.nytimes.com/2020/02/29/health/coronavirus-n95-face-masks.html}{face
masks} and hand sanitizer in many stores, said Alexandra Brown, a
spokeswoman for the Walgreens Boots Alliance, which owns both chains.

\href{https://www.nytimes.com/news-event/coronavirus?action=click\&pgtype=Article\&state=default\&region=MAIN_CONTENT_3\&context=storylines_faq}{}

\hypertarget{the-coronavirus-outbreak-}{%
\subsubsection{The Coronavirus Outbreak
›}\label{the-coronavirus-outbreak-}}

\hypertarget{frequently-asked-questions}{%
\paragraph{Frequently Asked
Questions}\label{frequently-asked-questions}}

Updated July 27, 2020

\begin{itemize}
\item ~
  \hypertarget{should-i-refinance-my-mortgage}{%
  \paragraph{Should I refinance my
  mortgage?}\label{should-i-refinance-my-mortgage}}

  \begin{itemize}
  \tightlist
  \item
    \href{https://www.nytimes.com/article/coronavirus-money-unemployment.html?action=click\&pgtype=Article\&state=default\&region=MAIN_CONTENT_3\&context=storylines_faq}{It
    could be a good idea,} because mortgage rates have
    \href{https://www.nytimes.com/2020/07/16/business/mortgage-rates-below-3-percent.html?action=click\&pgtype=Article\&state=default\&region=MAIN_CONTENT_3\&context=storylines_faq}{never
    been lower.} Refinancing requests have pushed mortgage applications
    to some of the highest levels since 2008, so be prepared to get in
    line. But defaults are also up, so if you're thinking about buying a
    home, be aware that some lenders have tightened their standards.
  \end{itemize}
\item ~
  \hypertarget{what-is-school-going-to-look-like-in-september}{%
  \paragraph{What is school going to look like in
  September?}\label{what-is-school-going-to-look-like-in-september}}

  \begin{itemize}
  \tightlist
  \item
    It is unlikely that many schools will return to a normal schedule
    this fall, requiring the grind of
    \href{https://www.nytimes.com/2020/06/05/us/coronavirus-education-lost-learning.html?action=click\&pgtype=Article\&state=default\&region=MAIN_CONTENT_3\&context=storylines_faq}{online
    learning},
    \href{https://www.nytimes.com/2020/05/29/us/coronavirus-child-care-centers.html?action=click\&pgtype=Article\&state=default\&region=MAIN_CONTENT_3\&context=storylines_faq}{makeshift
    child care} and
    \href{https://www.nytimes.com/2020/06/03/business/economy/coronavirus-working-women.html?action=click\&pgtype=Article\&state=default\&region=MAIN_CONTENT_3\&context=storylines_faq}{stunted
    workdays} to continue. California's two largest public school
    districts --- Los Angeles and San Diego --- said on July 13, that
    \href{https://www.nytimes.com/2020/07/13/us/lausd-san-diego-school-reopening.html?action=click\&pgtype=Article\&state=default\&region=MAIN_CONTENT_3\&context=storylines_faq}{instruction
    will be remote-only in the fall}, citing concerns that surging
    coronavirus infections in their areas pose too dire a risk for
    students and teachers. Together, the two districts enroll some
    825,000 students. They are the largest in the country so far to
    abandon plans for even a partial physical return to classrooms when
    they reopen in August. For other districts, the solution won't be an
    all-or-nothing approach.
    \href{https://bioethics.jhu.edu/research-and-outreach/projects/eschool-initiative/school-policy-tracker/}{Many
    systems}, including the nation's largest, New York City, are
    devising
    \href{https://www.nytimes.com/2020/06/26/us/coronavirus-schools-reopen-fall.html?action=click\&pgtype=Article\&state=default\&region=MAIN_CONTENT_3\&context=storylines_faq}{hybrid
    plans} that involve spending some days in classrooms and other days
    online. There's no national policy on this yet, so check with your
    municipal school system regularly to see what is happening in your
    community.
  \end{itemize}
\item ~
  \hypertarget{is-the-coronavirus-airborne}{%
  \paragraph{Is the coronavirus
  airborne?}\label{is-the-coronavirus-airborne}}

  \begin{itemize}
  \tightlist
  \item
    The coronavirus
    \href{https://www.nytimes.com/2020/07/04/health/239-experts-with-one-big-claim-the-coronavirus-is-airborne.html?action=click\&pgtype=Article\&state=default\&region=MAIN_CONTENT_3\&context=storylines_faq}{can
    stay aloft for hours in tiny droplets in stagnant air}, infecting
    people as they inhale, mounting scientific evidence suggests. This
    risk is highest in crowded indoor spaces with poor ventilation, and
    may help explain super-spreading events reported in meatpacking
    plants, churches and restaurants.
    \href{https://www.nytimes.com/2020/07/06/health/coronavirus-airborne-aerosols.html?action=click\&pgtype=Article\&state=default\&region=MAIN_CONTENT_3\&context=storylines_faq}{It's
    unclear how often the virus is spread} via these tiny droplets, or
    aerosols, compared with larger droplets that are expelled when a
    sick person coughs or sneezes, or transmitted through contact with
    contaminated surfaces, said Linsey Marr, an aerosol expert at
    Virginia Tech. Aerosols are released even when a person without
    symptoms exhales, talks or sings, according to Dr. Marr and more
    than 200 other experts, who
    \href{https://academic.oup.com/cid/article/doi/10.1093/cid/ciaa939/5867798}{have
    outlined the evidence in an open letter to the World Health
    Organization}.
  \end{itemize}
\item ~
  \hypertarget{what-are-the-symptoms-of-coronavirus}{%
  \paragraph{What are the symptoms of
  coronavirus?}\label{what-are-the-symptoms-of-coronavirus}}

  \begin{itemize}
  \tightlist
  \item
    Common symptoms
    \href{https://www.nytimes.com/article/symptoms-coronavirus.html?action=click\&pgtype=Article\&state=default\&region=MAIN_CONTENT_3\&context=storylines_faq}{include
    fever, a dry cough, fatigue and difficulty breathing or shortness of
    breath.} Some of these symptoms overlap with those of the flu,
    making detection difficult, but runny noses and stuffy sinuses are
    less common.
    \href{https://www.nytimes.com/2020/04/27/health/coronavirus-symptoms-cdc.html?action=click\&pgtype=Article\&state=default\&region=MAIN_CONTENT_3\&context=storylines_faq}{The
    C.D.C. has also} added chills, muscle pain, sore throat, headache
    and a new loss of the sense of taste or smell as symptoms to look
    out for. Most people fall ill five to seven days after exposure, but
    symptoms may appear in as few as two days or as many as 14 days.
  \end{itemize}
\item ~
  \hypertarget{does-asymptomatic-transmission-of-covid-19-happen}{%
  \paragraph{Does asymptomatic transmission of Covid-19
  happen?}\label{does-asymptomatic-transmission-of-covid-19-happen}}

  \begin{itemize}
  \tightlist
  \item
    So far, the evidence seems to show it does. A widely cited
    \href{https://www.nature.com/articles/s41591-020-0869-5}{paper}
    published in April suggests that people are most infectious about
    two days before the onset of coronavirus symptoms and estimated that
    44 percent of new infections were a result of transmission from
    people who were not yet showing symptoms. Recently, a top expert at
    the World Health Organization stated that transmission of the
    coronavirus by people who did not have symptoms was ``very rare,''
    \href{https://www.nytimes.com/2020/06/09/world/coronavirus-updates.html?action=click\&pgtype=Article\&state=default\&region=MAIN_CONTENT_3\&context=storylines_faq\#link-1f302e21}{but
    she later walked back that statement.}
  \end{itemize}
\end{itemize}

The company is moving supplies ``to meet the needs of our customers,''
she added.

Masks are not very protective when worn by healthy people, experts say.
People often pull them aside for a variety of reasons.

Vapor from breath can leave masks soggy, for example. Users may want to
talk on their cellphones, or put their fingers under their masks to
scratch their noses. Frequent handwashing is considered more protective.

Yet there is strong evidence that masks protect health workers.

Image

A shelf emptied of face masks at a Duane Reade store in Flushing,
Queens, on Wednesday.Credit...Chang W. Lee/The New York Times

Trials in Canadian hospitals during
\href{https://www.ncbi.nlm.nih.gov/pmc/articles/PMC3322898/}{the 2003
SARS epidemic}and
\href{https://jamanetwork.com/journals/jama/fullarticle/184819}{during
flu season} showed that nurses who wore a mask were less likely to get
flu.

Nurses who wore N-95 masks --- which are thicker, fit tighter to the
face and are designed to filter out 95 percent of all particles --- were
particularly protective for those who did dangerous procedures like
intubating SARS patients.

\textbf{\emph{{[}}\href{http://on.fb.me/1paTQ1h}{\emph{Like the Science
Times page on Facebook.}}} ****** \emph{\textbar{} Sign up for the}
\textbf{\href{http://nyti.ms/1MbHaRU}{\emph{Science Times
newsletter.}}\emph{{]}}}

There is less data proving that masks keep patients from spreading
germs. But this common sense notion has led many hospitals to adopt
rules saying that any emergency room patient with flu symptoms is
immediately handed a mask and asked to put it on.

Dr. Mark Loeb, an infectious disease specialist at McMaster University
in Hamilton, Ontario, who led the SARS and flu trials, said one
\href{https://bmjopen.bmj.com/content/6/12/e012330.long}{relatively
small study of hospitalized flu patients in Beijing} found that
mask-wearers were less likely to infect their family members. But the
effect was so small that it was considered statistically insignificant.

Another obstacle to mask-wearing is psychological.

People who are sick and should be wearing a mask are often reluctant to
do so, because it makes them stand out in the crowd. Unless these
patients are ordered to wear one --- as they might be in a hospital
emergency room --- people are reluctant to do so.

``When masks aren't common in a culture, it does raise eyebrows,'' Dr.
Adalja said.

The exceptions are in some Asian communities, where it is common for
people to wear masks to protect themselves against germs and pollution,
or because it is considered impolite to not wear a mask if one is
coughing or sneezing.

Experts, including Dr. Patel of the C.D.C., said they knew of no studies
of the psychology of mask usage or how reluctance could be overcome.

Advertisement

\protect\hyperlink{after-bottom}{Continue reading the main story}

\hypertarget{site-index}{%
\subsection{Site Index}\label{site-index}}

\hypertarget{site-information-navigation}{%
\subsection{Site Information
Navigation}\label{site-information-navigation}}

\begin{itemize}
\tightlist
\item
  \href{https://help.nytimes.com/hc/en-us/articles/115014792127-Copyright-notice}{©~2020~The
  New York Times Company}
\end{itemize}

\begin{itemize}
\tightlist
\item
  \href{https://www.nytco.com/}{NYTCo}
\item
  \href{https://help.nytimes.com/hc/en-us/articles/115015385887-Contact-Us}{Contact
  Us}
\item
  \href{https://www.nytco.com/careers/}{Work with us}
\item
  \href{https://nytmediakit.com/}{Advertise}
\item
  \href{http://www.tbrandstudio.com/}{T Brand Studio}
\item
  \href{https://www.nytimes.com/privacy/cookie-policy\#how-do-i-manage-trackers}{Your
  Ad Choices}
\item
  \href{https://www.nytimes.com/privacy}{Privacy}
\item
  \href{https://help.nytimes.com/hc/en-us/articles/115014893428-Terms-of-service}{Terms
  of Service}
\item
  \href{https://help.nytimes.com/hc/en-us/articles/115014893968-Terms-of-sale}{Terms
  of Sale}
\item
  \href{https://spiderbites.nytimes.com}{Site Map}
\item
  \href{https://help.nytimes.com/hc/en-us}{Help}
\item
  \href{https://www.nytimes.com/subscription?campaignId=37WXW}{Subscriptions}
\end{itemize}
