Sections

SEARCH

\protect\hyperlink{site-content}{Skip to
content}\protect\hyperlink{site-index}{Skip to site index}

\href{https://www.nytimes.com/section/world/asia}{Asia Pacific}

\href{https://myaccount.nytimes.com/auth/login?response_type=cookie\&client_id=vi}{}

\href{https://www.nytimes.com/section/todayspaper}{Today's Paper}

\href{/section/world/asia}{Asia Pacific}\textbar{}Philippines Reports
First Coronavirus Death Outside China

\url{https://nyti.ms/2GP7Ng0}

\begin{itemize}
\item
\item
\item
\item
\item
\end{itemize}

\href{https://www.nytimes.com/news-event/coronavirus?action=click\&pgtype=Article\&state=default\&region=TOP_BANNER\&context=storylines_menu}{The
Coronavirus Outbreak}

\begin{itemize}
\tightlist
\item
  live\href{https://www.nytimes.com/2020/08/04/world/coronavirus-covid-19.html?action=click\&pgtype=Article\&state=default\&region=TOP_BANNER\&context=storylines_menu}{Latest
  Updates}
\item
  \href{https://www.nytimes.com/interactive/2020/us/coronavirus-us-cases.html?action=click\&pgtype=Article\&state=default\&region=TOP_BANNER\&context=storylines_menu}{Maps
  and Cases}
\item
  \href{https://www.nytimes.com/interactive/2020/science/coronavirus-vaccine-tracker.html?action=click\&pgtype=Article\&state=default\&region=TOP_BANNER\&context=storylines_menu}{Vaccine
  Tracker}
\item
  \href{https://www.nytimes.com/2020/08/02/us/covid-college-reopening.html?action=click\&pgtype=Article\&state=default\&region=TOP_BANNER\&context=storylines_menu}{College
  Reopening}
\item
  \href{https://www.nytimes.com/live/2020/08/03/business/stock-market-today-coronavirus?action=click\&pgtype=Article\&state=default\&region=TOP_BANNER\&context=storylines_menu}{Economy}
\end{itemize}

Advertisement

\protect\hyperlink{after-top}{Continue reading the main story}

Supported by

\protect\hyperlink{after-sponsor}{Continue reading the main story}

\hypertarget{philippines-reports-first-coronavirus-death-outside-china}{%
\section{Philippines Reports First Coronavirus Death Outside
China}\label{philippines-reports-first-coronavirus-death-outside-china}}

A 44-year-old man who traveled from Wuhan, China, the center of the
outbreak, died in the Philippines, officials said.

\includegraphics{https://static01.nyt.com/images/2020/02/02/world/02china-virus1/merlin_168131283_0e7d58f9-57f3-4219-bed2-465b44954559-articleLarge.jpg?quality=75\&auto=webp\&disable=upscale}

\href{https://www.nytimes.com/by/austin-ramzy}{\includegraphics{https://static01.nyt.com/images/2018/10/15/multimedia/author-austin-ramzy/author-austin-ramzy-thumbLarge.png}}\href{https://www.nytimes.com/by/tiffany-may}{\includegraphics{https://static01.nyt.com/images/2019/12/04/reader-center/author-tiffany-may/author-tiffany-may-thumbLarge.png}}

By \href{https://www.nytimes.com/by/austin-ramzy}{Austin Ramzy} and
\href{https://www.nytimes.com/by/tiffany-may}{Tiffany May}

\begin{itemize}
\item
  Feb. 2, 2020
\item
  \begin{itemize}
  \item
  \item
  \item
  \item
  \item
  \end{itemize}
\end{itemize}

HONG KONG --- A 44-year-old man in the Philippines has died of the
coronavirus, the first known fatality outside China, health officials
said on Sunday, as the number of deaths jumped to more than 360 and
other countries expanded
\href{https://www.nytimes.com/2020/02/01/world/asia/china-coronavirus-us-australia.html}{travel
restrictions} in an effort to slow
\href{https://www.nytimes.com/interactive/2020/world/asia/china-coronavirus-contain.html}{the
spread of the outbreak}.

The man, whose name was not released, was a resident of Wuhan, China,
the center of the outbreak. He died on Saturday after developing what
officials called ``severe pneumonia.''

``This is the first known death of someone with 2019-nCoV outside of
China,'' the World Health Organization's office in the Philippines
\href{https://www.facebook.com/whophilippines/photos/a.514585071988103/2664930260286896/?type=3\&__xts__\%5B0\%5D=68.ARANVVlwrEFTML5lekIxBZNMqIbfrFgdiCrfveU-KkAfadCSrp4IBIqFDS02YQzuLERpt-CdtaoSjeg71-s8JN5syTzk8bnheLsSOOS0h5V3GR86fa3688BBhxMwuyaGe-IwatUH1sYDoUo-4BALQjKnvQDnewFAioKiQ3megPQRw7f-BcCbW1mChINqsYlEaUjMlMQYIVsHmdoCh-zZhGr5JfIXqmffJTWMtS66jfl-CoSoVu_pfaoBxBhAUenU1TCHQ5RXl6eqHI67HS_tT0S_DaW-TuMaO8ds0RKn7FBHgqn4EqkbQxJ_v0a-15TntvvUQV1UFa692Oqpkesg75PWmQ\&__tn__=-R}{said
in a statement}, using the technical shorthand for the coronavirus.

Philippine health officials said the man had arrived in the country on
Jan. 21 with a 38-year-old woman. She remains under observation.

Hours before the announcement, the Philippines said it was temporarily
barring non-Filipino travelers arriving from mainland China, Hong Kong
and Macau.

In Hong Kong, a semiautonomous Chinese territory, a new union
representing health care workers vowed to strike on Monday to force the
city's government to ban travel from mainland China. Fears in the city
escalated after another case was confirmed and officials could not rule
out the possibility that the patient, an 80-year-old man, had been
infected within Hong Kong.

\includegraphics{https://static01.nyt.com/images/2020/02/02/world/02china-virus2/merlin_168254769_2e715035-7273-4741-9796-300d356994ca-articleLarge.jpg?quality=75\&auto=webp\&disable=upscale}

``We are very worried,'' Chuang Shuk-kwan, a health official, said at a
news conference on Sunday. ``Everyone should prepare mentally for the
possibility that the disease is spreading within the community.''

Hours later, the government announced its 15th case, which appeared to
confirm local transmission of the virus. A 72-year-old woman who had
largely stayed at home was infected, likely via her son. He returned
from a trip to Wuhan on Jan. 23.

\hypertarget{latest-updates-global-coronavirus-outbreak}{%
\section{\texorpdfstring{\href{https://www.nytimes.com/2020/08/04/world/coronavirus-covid-19.html?action=click\&pgtype=Article\&state=default\&region=MAIN_CONTENT_1\&context=storylines_live_updates}{Latest
Updates: Global Coronavirus
Outbreak}}{Latest Updates: Global Coronavirus Outbreak}}\label{latest-updates-global-coronavirus-outbreak}}

Updated 2020-08-04T09:15:14.275Z

\begin{itemize}
\tightlist
\item
  \href{https://www.nytimes.com/2020/08/04/world/coronavirus-covid-19.html?action=click\&pgtype=Article\&state=default\&region=MAIN_CONTENT_1\&context=storylines_live_updates\#link-6b644638}{`Long
  days, long nights': Washington prepares for a prolonged fight over
  virus relief.}
\item
  \href{https://www.nytimes.com/2020/08/04/world/coronavirus-covid-19.html?action=click\&pgtype=Article\&state=default\&region=MAIN_CONTENT_1\&context=storylines_live_updates\#link-7af9fca0}{Israel's
  rocky reopening of its schools may be a lesson for the U.S.}
\item
  \href{https://www.nytimes.com/2020/08/04/world/coronavirus-covid-19.html?action=click\&pgtype=Article\&state=default\&region=MAIN_CONTENT_1\&context=storylines_live_updates\#link-33bf9168}{Hurricane
  Isaias arrives in North Carolina as officials along the East Coast
  scramble.}
\end{itemize}

\href{https://www.nytimes.com/2020/08/04/world/coronavirus-covid-19.html?action=click\&pgtype=Article\&state=default\&region=MAIN_CONTENT_1\&context=storylines_live_updates}{See
more updates}

More live coverage:
\href{https://www.nytimes.com/live/2020/08/03/business/stock-market-today-coronavirus?action=click\&pgtype=Article\&state=default\&region=MAIN_CONTENT_1\&context=storylines_live_updates}{Markets}

By Sunday, with infections standing at more than 14,000 worldwide,
nations continued to expand travel restrictions and bar visitors from
China.

New Zealand said on Sunday that it would deny entry to visitors
departing from or traveling through mainland China for two weeks
starting on Monday. Citizens and residents of New Zealand will be
allowed entry from China but will be required to quarantine themselves
for 14 days,
\href{https://www.tvnz.co.nz/one-news/new-zealand/new-zealand-restrict-travellers-china-coronavirus-spreads-worldwide}{Prime
Minister Jacinda Ardern said}.

``Ultimately, this is a public health decision,'' she said, adding that
the restrictions were precautionary measures to keep New Zealand
virus-free and to contain the worldwide outbreak.

In Israel, Prime Minister Benjamin Netanyahu convened a meeting of
ministers in Jerusalem on Sunday to make national preparations ``because
we estimate that the virus will arrive,'' his office said in a
statement. Along with preparations ``to isolate those who have been
infected and treat them'' for two weeks at home, the country had
temporarily closed ``land crossings, seaports and airports to arrivals
from China.''

Israeli citizens who visited China will be allowed to return,
\href{https://twitter.com/IsraeliPM/status/1223971839099187200}{the
statement said.}

In addition, about 1,700 Chinese construction workers whose work permits
had expired and who were supposed to return to China this weekend would
be granted extended stays in Israel. They are being allowed to keep
working rather than being replaced by newly recruited laborers who could
potentially arrive with the virus, according to Israeli news reports.

The United States and Australia have also expanded travel restrictions,
temporarily barring noncitizens who recently traveled to China.

Image

A passenger receiving a temperature check before flying to Wuhan, China,
from Haneda Airport in Tokyo.Credit...Tomohiro Ohsumi/Getty Images

South Korea said on Sunday that it would deny entry to any ****
foreigners who have traveled in the past 14 days to Wuhan and
surrounding Hubei Province, the area at the center of the outbreak. In a
move to help stop the spread of the virus in Wuhan, the Chinese
authorities have fulfilled a promise to build a new 1,000-bed specialty
hospital in the city within 10 days. About 1,400 military medics are to
begin working there on Monday.

South Korea's travel restrictions will take effect on Tuesday, Prime
Minister Chung Sye-kyun said, as the number of people testing positive
for the coronavirus in South Korea increased to 15.

Any South Korean returning home who has been in Hubei Province within
the past two weeks will be subject to 14 days of self-quarantine and
monitoring, the government said. It also said it would bar South Koreans
from visiting China as tourists.

Japan will bar noncitizens who traveled recently to Hubei. Taiwan is
denying entry to Chinese nationals from Guangdong, a southern coastal
province that has been battered by the virus, and travelers who recently
visited the area.

Vietnam recently barred almost all flights to and from mainland China,
Hong Kong, Macau and Taiwan until May 1, according to the United States
Federal Aviation Administration. But Vietnam then eased its ban,
allowing flights from Hong Kong, Macau and Taiwan to continue while
keeping the prohibitions in place for mainland China, the aviation
authorities said.

Image

A train stop in Taipei, Taiwan's capital, on Thursday.~Taiwan has 10
confirmed cases of the virus.Credit...Sam Yeh/Agence France-Presse ---
Getty Images

Taiwan, which is self-ruled but which China claims is part of its
territory, complained on Sunday that it was being punished with flight
restrictions because the World Health Organization considers it part of
China.

Italy included Taiwan in a ban on flights from China, a move that it
announced after the W.H.O.
\href{https://www.nytimes.com/2020/01/30/health/coronavirus-world-health-organization.html}{declared
the coronavirus outbreak a global health emergency}. While Vietnam
backtracked, Italy's ban remains, Taiwan's foreign minister told
reporters.

\href{https://www.nytimes.com/news-event/coronavirus?action=click\&pgtype=Article\&state=default\&region=MAIN_CONTENT_3\&context=storylines_faq}{}

\hypertarget{the-coronavirus-outbreak-}{%
\subsubsection{The Coronavirus Outbreak
›}\label{the-coronavirus-outbreak-}}

\hypertarget{frequently-asked-questions}{%
\paragraph{Frequently Asked
Questions}\label{frequently-asked-questions}}

Updated August 3, 2020

\begin{itemize}
\item ~
  \hypertarget{im-a-small-business-owner-can-i-get-relief}{%
  \paragraph{I'm a small-business owner. Can I get
  relief?}\label{im-a-small-business-owner-can-i-get-relief}}

  \begin{itemize}
  \tightlist
  \item
    The
    \href{https://www.nytimes.com/article/small-business-loans-stimulus-grants-freelancers-coronavirus.html?action=click\&pgtype=Article\&state=default\&region=MAIN_CONTENT_3\&context=storylines_faq}{stimulus
    bills enacted in March} offer help for the millions of American
    small businesses. Those eligible for aid are businesses and
    nonprofit organizations with fewer than 500 workers, including sole
    proprietorships, independent contractors and freelancers. Some
    larger companies in some industries are also eligible. The help
    being offered, which is being managed by the Small Business
    Administration, includes the Paycheck Protection Program and the
    Economic Injury Disaster Loan program. But lots of folks have
    \href{https://www.nytimes.com/interactive/2020/05/07/business/small-business-loans-coronavirus.html?action=click\&pgtype=Article\&state=default\&region=MAIN_CONTENT_3\&context=storylines_faq}{not
    yet seen payouts.} Even those who have received help are confused:
    The rules are draconian, and some are stuck sitting on
    \href{https://www.nytimes.com/2020/05/02/business/economy/loans-coronavirus-small-business.html?action=click\&pgtype=Article\&state=default\&region=MAIN_CONTENT_3\&context=storylines_faq}{money
    they don't know how to use.} Many small-business owners are getting
    less than they expected or
    \href{https://www.nytimes.com/2020/06/10/business/Small-business-loans-ppp.html?action=click\&pgtype=Article\&state=default\&region=MAIN_CONTENT_3\&context=storylines_faq}{not
    hearing anything at all.}
  \end{itemize}
\item ~
  \hypertarget{what-are-my-rights-if-i-am-worried-about-going-back-to-work}{%
  \paragraph{What are my rights if I am worried about going back to
  work?}\label{what-are-my-rights-if-i-am-worried-about-going-back-to-work}}

  \begin{itemize}
  \tightlist
  \item
    Employers have to provide
    \href{https://www.osha.gov/SLTC/covid-19/standards.html}{a safe
    workplace} with policies that protect everyone equally.
    \href{https://www.nytimes.com/article/coronavirus-money-unemployment.html?action=click\&pgtype=Article\&state=default\&region=MAIN_CONTENT_3\&context=storylines_faq}{And
    if one of your co-workers tests positive for the coronavirus, the
    C.D.C.} has said that
    \href{https://www.cdc.gov/coronavirus/2019-ncov/community/guidance-business-response.html}{employers
    should tell their employees} -\/- without giving you the sick
    employee's name -\/- that they may have been exposed to the virus.
  \end{itemize}
\item ~
  \hypertarget{should-i-refinance-my-mortgage}{%
  \paragraph{Should I refinance my
  mortgage?}\label{should-i-refinance-my-mortgage}}

  \begin{itemize}
  \tightlist
  \item
    \href{https://www.nytimes.com/article/coronavirus-money-unemployment.html?action=click\&pgtype=Article\&state=default\&region=MAIN_CONTENT_3\&context=storylines_faq}{It
    could be a good idea,} because mortgage rates have
    \href{https://www.nytimes.com/2020/07/16/business/mortgage-rates-below-3-percent.html?action=click\&pgtype=Article\&state=default\&region=MAIN_CONTENT_3\&context=storylines_faq}{never
    been lower.} Refinancing requests have pushed mortgage applications
    to some of the highest levels since 2008, so be prepared to get in
    line. But defaults are also up, so if you're thinking about buying a
    home, be aware that some lenders have tightened their standards.
  \end{itemize}
\item ~
  \hypertarget{what-is-school-going-to-look-like-in-september}{%
  \paragraph{What is school going to look like in
  September?}\label{what-is-school-going-to-look-like-in-september}}

  \begin{itemize}
  \tightlist
  \item
    It is unlikely that many schools will return to a normal schedule
    this fall, requiring the grind of
    \href{https://www.nytimes.com/2020/06/05/us/coronavirus-education-lost-learning.html?action=click\&pgtype=Article\&state=default\&region=MAIN_CONTENT_3\&context=storylines_faq}{online
    learning},
    \href{https://www.nytimes.com/2020/05/29/us/coronavirus-child-care-centers.html?action=click\&pgtype=Article\&state=default\&region=MAIN_CONTENT_3\&context=storylines_faq}{makeshift
    child care} and
    \href{https://www.nytimes.com/2020/06/03/business/economy/coronavirus-working-women.html?action=click\&pgtype=Article\&state=default\&region=MAIN_CONTENT_3\&context=storylines_faq}{stunted
    workdays} to continue. California's two largest public school
    districts --- Los Angeles and San Diego --- said on July 13, that
    \href{https://www.nytimes.com/2020/07/13/us/lausd-san-diego-school-reopening.html?action=click\&pgtype=Article\&state=default\&region=MAIN_CONTENT_3\&context=storylines_faq}{instruction
    will be remote-only in the fall}, citing concerns that surging
    coronavirus infections in their areas pose too dire a risk for
    students and teachers. Together, the two districts enroll some
    825,000 students. They are the largest in the country so far to
    abandon plans for even a partial physical return to classrooms when
    they reopen in August. For other districts, the solution won't be an
    all-or-nothing approach.
    \href{https://bioethics.jhu.edu/research-and-outreach/projects/eschool-initiative/school-policy-tracker/}{Many
    systems}, including the nation's largest, New York City, are
    devising
    \href{https://www.nytimes.com/2020/06/26/us/coronavirus-schools-reopen-fall.html?action=click\&pgtype=Article\&state=default\&region=MAIN_CONTENT_3\&context=storylines_faq}{hybrid
    plans} that involve spending some days in classrooms and other days
    online. There's no national policy on this yet, so check with your
    municipal school system regularly to see what is happening in your
    community.
  \end{itemize}
\item ~
  \hypertarget{is-the-coronavirus-airborne}{%
  \paragraph{Is the coronavirus
  airborne?}\label{is-the-coronavirus-airborne}}

  \begin{itemize}
  \tightlist
  \item
    The coronavirus
    \href{https://www.nytimes.com/2020/07/04/health/239-experts-with-one-big-claim-the-coronavirus-is-airborne.html?action=click\&pgtype=Article\&state=default\&region=MAIN_CONTENT_3\&context=storylines_faq}{can
    stay aloft for hours in tiny droplets in stagnant air}, infecting
    people as they inhale, mounting scientific evidence suggests. This
    risk is highest in crowded indoor spaces with poor ventilation, and
    may help explain super-spreading events reported in meatpacking
    plants, churches and restaurants.
    \href{https://www.nytimes.com/2020/07/06/health/coronavirus-airborne-aerosols.html?action=click\&pgtype=Article\&state=default\&region=MAIN_CONTENT_3\&context=storylines_faq}{It's
    unclear how often the virus is spread} via these tiny droplets, or
    aerosols, compared with larger droplets that are expelled when a
    sick person coughs or sneezes, or transmitted through contact with
    contaminated surfaces, said Linsey Marr, an aerosol expert at
    Virginia Tech. Aerosols are released even when a person without
    symptoms exhales, talks or sings, according to Dr. Marr and more
    than 200 other experts, who
    \href{https://academic.oup.com/cid/article/doi/10.1093/cid/ciaa939/5867798}{have
    outlined the evidence in an open letter to the World Health
    Organization}.
  \end{itemize}
\end{itemize}

Taiwan has 10 confirmed cases, versus more than 17,000 in mainland
China, said the foreign minister, Joseph Wu.

``The number of confirmed cases of coronavirus in Taiwan is not higher
than in most countries affected,'' he said. ``Other than China, no other
country, no other country has had its flight banned by Italy.''

China has long sought to limit Taiwan's diplomatic relations and
recognition at international bodies such as the W.H.O. Taiwan previously
participated as an observer at the World Health Assembly, the group's
governing body. But it has since been excluded as Beijing has increased
pressure on Taiwan under President Tsai Ing-wen, who is skeptical about
closer ties with Beijing.

``It is not fair to the 23 million people in Taiwan, and it is not fair
to other people who might otherwise obtain support from Taiwan if we
were not excluded,'' Mr. Wu said.

In Hong Kong, some residents are pushing for tougher restrictions on
arrivals from mainland China. As many as 9,000 medical workers have
pledged to strike beginning on Monday, a threat that alarms the
territory's officials as they are struggling to contain the outbreak.

Image

Residents protesting on Sunday at a Hong Kong private housing estate
near a hotel that will be used to quarantine patients.Credit...Jerome
Favre/EPA, via Shutterstock

The workers are demanding that Hong Kong close all checkpoints to
visitors from mainland China, saying they represent a threat to health
care workers. The workers plan to paralyze nonemergency and emergency
services at hospitals, said the Hospital Authority Employees Alliance, a
union formed during the city's antigovernment protest movement.

``We believe such actions are our last resort,'' the alliance wrote in a
\href{https://telegra.ph/Healthcare-Workers-Together-We-Stand-Strike-to-Protect-Hong-Kong-02-01}{statement}
on Saturday night.

Under the plan, nonessential hospital staff members who belong to the
union would not go to work on Monday. If the government did not close
the border and heed their other demands by 9 p.m. local time, union
members handling emergency services would also strike, the union said.

Matthew Cheung, Hong Kong's No. 2 official, appealed to medical workers
to reconsider, comparing them to guardians of the public.

``At this critical moment, I believe the general public would count on
medical personnel to fight against the epidemic together, in the spirit
of professionalism,'' he wrote in
\href{https://www.cso.gov.hk/eng/blog/blog20200202.htm}{a blog post} on
Sunday.

Image

A traveler at the arrival hall of Hong Kong's high-speed rail station,
whose routes connect to mainland China. Residents want to bar visitors
from the mainland.Credit...Anthony Kwan/Getty Images

Government officials in Hong Kong say that the number of visitors from
the mainland and other countries had decreased significantly after they
closed several border points and rail stations and cut flight arrivals
by half.

But several border locations remain open, and many medical workers fear
being overwhelmed by a flood of visitors seeking treatment in Hong
Kong's well-regarded health care system.

They have also voiced frustrations about patients from mainland China
hiding their travel and medical history, potentially endangering other
patients.

Reporting was contributed by Chris Buckley from Wuhan, China; Alexandra
Stevenson and Sui-Lee Wee from Hong Kong; Choe Sang-Hun from South
Korea; Jason Gutierrez from Manila; and David Halbfinger and Isabel
Kershner from Jerusalem.

Advertisement

\protect\hyperlink{after-bottom}{Continue reading the main story}

\hypertarget{site-index}{%
\subsection{Site Index}\label{site-index}}

\hypertarget{site-information-navigation}{%
\subsection{Site Information
Navigation}\label{site-information-navigation}}

\begin{itemize}
\tightlist
\item
  \href{https://help.nytimes.com/hc/en-us/articles/115014792127-Copyright-notice}{©~2020~The
  New York Times Company}
\end{itemize}

\begin{itemize}
\tightlist
\item
  \href{https://www.nytco.com/}{NYTCo}
\item
  \href{https://help.nytimes.com/hc/en-us/articles/115015385887-Contact-Us}{Contact
  Us}
\item
  \href{https://www.nytco.com/careers/}{Work with us}
\item
  \href{https://nytmediakit.com/}{Advertise}
\item
  \href{http://www.tbrandstudio.com/}{T Brand Studio}
\item
  \href{https://www.nytimes.com/privacy/cookie-policy\#how-do-i-manage-trackers}{Your
  Ad Choices}
\item
  \href{https://www.nytimes.com/privacy}{Privacy}
\item
  \href{https://help.nytimes.com/hc/en-us/articles/115014893428-Terms-of-service}{Terms
  of Service}
\item
  \href{https://help.nytimes.com/hc/en-us/articles/115014893968-Terms-of-sale}{Terms
  of Sale}
\item
  \href{https://spiderbites.nytimes.com}{Site Map}
\item
  \href{https://help.nytimes.com/hc/en-us}{Help}
\item
  \href{https://www.nytimes.com/subscription?campaignId=37WXW}{Subscriptions}
\end{itemize}
