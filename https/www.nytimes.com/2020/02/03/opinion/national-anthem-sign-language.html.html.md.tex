Sections

SEARCH

\protect\hyperlink{site-content}{Skip to
content}\protect\hyperlink{site-index}{Skip to site index}

\href{https://myaccount.nytimes.com/auth/login?response_type=cookie\&client_id=vi}{}

\href{https://www.nytimes.com/section/todayspaper}{Today's Paper}

\href{/section/opinion}{Opinion}\textbar{}I Performed at the Super Bowl.
You Might Have Missed Me.

\href{https://nyti.ms/2ufpwdQ}{https://nyti.ms/2ufpwdQ}

\begin{itemize}
\item
\item
\item
\item
\item
\item
\end{itemize}

Advertisement

\protect\hyperlink{after-top}{Continue reading the main story}

\href{/section/opinion}{Opinion}

Supported by

\protect\hyperlink{after-sponsor}{Continue reading the main story}

disability

\hypertarget{i-performed-at-the-super-bowl-you-might-have-missed-me}{%
\section{I Performed at the Super Bowl. You Might Have Missed
Me.}\label{i-performed-at-the-super-bowl-you-might-have-missed-me}}

It was an act of both patriotism and protest. And it brought both joy
and frustration.

By Christine Sun Kim

Ms. Kim is an artist and performer.

\begin{itemize}
\item
  Feb. 3, 2020
\item
  \begin{itemize}
  \item
  \item
  \item
  \item
  \item
  \item
  \end{itemize}
\end{itemize}

\includegraphics{https://static01.nyt.com/images/2020/02/03/opinion/03disNew/merlin_168292356_19b99c76-742e-4fe4-9150-d3029a7deba6-articleLarge.jpg?quality=75\&auto=webp\&disable=upscale}

As a child of immigrants, a grandchild of refugees, a
\href{https://www.nad.org/resources/american-sign-language/community-and-culture-frequently-asked-questions/}{Deaf}
woman of color, an artist and a mother, I was proud to perform the
national anthem and ``America the Beautiful'' in American Sign Language
at the opening of the Super Bowl on Sunday. I accepted the invitation to
represent the National Association of the Deaf in partnership with the
National Football League because I wanted to express my patriotism and
honor the country that I am proud to be from --- a country that, at its
core, believes in equal rights for all citizens, including those with
disabilities. So while Yolanda Adams and Demi Lovato sang on the 50-yard
line at Hard Rock Stadium in Miami, I signed along on the 40.

Unfortunately, while the performance was broadcast in real time on the
jumbo screens in the stadium, those watching on their televisions,
computers and phones got a seriously truncated version.

While Fox Sports announced the signed performance of the two songs on
Twitter, it did not actually show it. On the television broadcast, I was
visible for only a few seconds. On what was supposed to be a ``bonus
feed'' dedicated to my full performance on the Fox Sports website,
\href{https://twitter.com/mhairiforrest/status/1224168857973092354?s=21}{the
cameras cut away} to show close-ups of the players roughly midway
through each song.

Why have a sign language performance that is not accessible to anyone
who would like to see it? It's 2020: We've had the technology to do so
for decades. And people noticed. The deaf actress Marlee Matlin, who has
signed the anthem at three Super Bowls (in 1993, 2007 and 2016),
tweeted:

\begin{quote}
To ALL networks: next time at
\href{https://twitter.com/hashtag/SuperBowl?src=hash\&ref_src=twsrc\%5Etfw}{\#SuperBowl}
when the signer like
\href{https://twitter.com/chrisunkim?ref_src=twsrc\%5Etfw}{@chrisunkim}
is performing
\href{https://twitter.com/hashtag/NationalAnthem?src=hash\&ref_src=twsrc\%5Etfw}{\#NationalAnthem}
\& America the Beautiful they should be on the broadcast the WHOLE time,
not a few seconds. SHOW the beauty of ASL either in a bubble, split
screen or next to singer It's time!
\href{https://t.co/C9gKykgZAI}{pic.twitter.com/C9gKykgZAI}

--- Marlee Matlin (@MarleeMatlin)
\href{https://twitter.com/MarleeMatlin/status/1224153599506186240?ref_src=twsrc\%5Etfw}{February
3, 2020}
\end{quote}

To be honest, it was a huge disappointment --- a missed opportunity in
the struggle for media inclusiveness on a large scale. Though thrilled
and excited to be on the field serving the deaf community, I was angry
and exasperated.

Still, my pride in being chosen for this performance was genuine.
Because of the passage of the Americans with Disabilities Act in 1990, I
have been afforded the rights and access that have allowed me to live a
life on par with my fellow citizens. I am able to watch TV with
captions, make phone calls through an online video platform, and have
interpreters provided for my education, among many other privileges. I
have traveled to many countries and witnessed firsthand the lack of
equality for deaf people all over the world, making me appreciate being
an American even more. I realize that being a citizen of this country is
not something to be taken for granted.

It is a little-known fact that the N.F.L. has been providing some access
to deaf people for more than three decades. Recently, that has extended
to game-day commercials. In 2008, Alexis Kashar, a deaf civil rights
attorney, raised the issue of Super Bowl commercials not being fully
captioned with the N.F.L. Ms. Kashar, who at that time was the chair of
the National Association of the Deaf's civil rights committee, brought
in her colleague Howard A. Rosenblum, the organization's chief
executive, to petition for more accessibility. The N.F.L. subsequently
encouraged advertisers to address the issue, even though they were not
required by law to add captions. Within a few years almost all Super
Bowl commercials were captioned. The N.A.D. continues to ensure the
commercials are captioned to this day.

\includegraphics{https://static01.nyt.com/images/2020/02/03/opinion/03kimWeb/merlin_168291918_239accfa-ee46-441d-9cde-ce1c0b7394c8-articleLarge.jpg?quality=75\&auto=webp\&disable=upscale}

Even so, I accepted the invitation with some hesitation. Being deaf in
America has always been political, and I needed to process some internal
conflict before accepting. I knew that some musical artists had refused
opportunities to perform at the game last year in support of Colin
Kaepernick. I wondered whether I should do the same.

I thought of the deaf people who had been mistreated or killed by
police. I thought about Robert Kim, a deaf man who was beaten and
tasered in 2012 by a police officer in a St. Louis suburb; and Daniel
Kevin Harris, who was shot and killed in 2016 by a North Carolina state
trooper; and two Oklahoma men, Pearl Pearson, who was badly beaten by
police officers in 2014, and Magdiel Sanchez, who was shot and killed by
police in front of his home in Oklahoma City in 2017.

These incidents are part of a pattern, as Amiel Fields-Meyer noted in
\href{https://www.theatlantic.com/politics/archive/2017/09/the-steadily-problematic-interactions-between-deaf-americans-and-police/541083/}{a
2017 article for The Atlantic}. While the Americans with Disabilities
Act requires the government to provide ``effective communication'' and
``reasonable accommodation,'' Fields-Meyer wrote, ``When it comes to the
deaf community, officers' gaps in cultural competency have led to strain
and miscommunication --- and, at times, the use of deadly force.''

Of course, deaf people face many other obstacles and injustices: limited
access to mental health service, health care and sign language itself,
inadequate online accessibility, lack of employment opportunities and
more. And as usual, these inequalities and the burden of their effects
fall disproportionately on people of color.

But in the end, I accepted the invitation. Our rights can easily
disappear if we do not continue to show up in places like the Super
Bowl.

I had hoped to provide a public service for deaf viewers, and believed
that my appearance might raise awareness of the systemic barriers and
the stigmas attached to our deafness --- and move some people to action.
I hope that despite the failure of Fox to make the performance
accessible to all, it did do that.

What kind of action? My parents took action by learning to sign for my
\href{https://www.nad.org/resources/american-sign-language/community-and-culture-frequently-asked-questions/}{Deaf}
sister and me. This made us feel seen and respected. Today, I'm teaching
my hearing child to sign as well. Respecting all languages and
identities can only lead to better laws and a higher quality of life.
Signing the anthem was a way to celebrate my language.

The N.A.D. has posted video of my entire performances of
``\href{https://www.youtube.com/watch?v=EmXFzQo4BFw\&feature=youtu.be}{America
the Beautiful}'' and the
\href{https://www.youtube.com/watch?v=c2TCT5HYlHQ\&feature=youtu.be}{national
anthem} on YouTube.

I wholeheartedly support the efforts to promote accessibility and
equality that have been undertaken by the N.F.L. and the N.A.D. I also
stand in recognition and support of those who have used the N.F.L.'s
platform toward wider goals of social justice. I see my performance as
part of that expression.

Christine Sun Kim (@chrisunkim) is a sound artist living in Berlin whose
work has been shown or performed at many museums and galleries,
including the Berlin and Shanghai Biennials and the Museum of Modern Art
in New York.

\emph{Disability is a series of essays, art and opinion by and about
people living with disabilities.}

\emph{\textbf{Now in print:}}
\emph{``}\href{https://www.aboutusbook.com/}{\emph{About Us: Essays From
the Disability Series of The New York Times}}\emph{,'' edited by Peter
Catapano and Rosemarie Garland-Thomson, published by Liveright.}

\emph{The Times is committed to publishing}
\href{https://www.nytimes.com/2019/01/31/opinion/letters/letters-to-editor-new-york-times-women.html}{\emph{a
diversity of letters}} \emph{to the editor. We'd like to hear what you
think about this or any of our articles. Here are some}
\href{https://help.nytimes.com/hc/en-us/articles/115014925288-How-to-submit-a-letter-to-the-editor}{\emph{tips}}\emph{.
And here's our email:}
\href{mailto:letters@nytimes.com}{\emph{letters@nytimes.com}}\emph{.}

\emph{Follow The New York Times Opinion section on}
\href{https://www.facebook.com/nytopinion}{\emph{Facebook}}\emph{,}
\href{http://twitter.com/NYTOpinion}{\emph{Twitter (@NYTopinion)}}
\emph{and}
\href{https://www.instagram.com/nytopinion/}{\emph{Instagram}}\emph{.}

Advertisement

\protect\hyperlink{after-bottom}{Continue reading the main story}

\hypertarget{site-index}{%
\subsection{Site Index}\label{site-index}}

\hypertarget{site-information-navigation}{%
\subsection{Site Information
Navigation}\label{site-information-navigation}}

\begin{itemize}
\tightlist
\item
  \href{https://help.nytimes.com/hc/en-us/articles/115014792127-Copyright-notice}{©~2020~The
  New York Times Company}
\end{itemize}

\begin{itemize}
\tightlist
\item
  \href{https://www.nytco.com/}{NYTCo}
\item
  \href{https://help.nytimes.com/hc/en-us/articles/115015385887-Contact-Us}{Contact
  Us}
\item
  \href{https://www.nytco.com/careers/}{Work with us}
\item
  \href{https://nytmediakit.com/}{Advertise}
\item
  \href{http://www.tbrandstudio.com/}{T Brand Studio}
\item
  \href{https://www.nytimes.com/privacy/cookie-policy\#how-do-i-manage-trackers}{Your
  Ad Choices}
\item
  \href{https://www.nytimes.com/privacy}{Privacy}
\item
  \href{https://help.nytimes.com/hc/en-us/articles/115014893428-Terms-of-service}{Terms
  of Service}
\item
  \href{https://help.nytimes.com/hc/en-us/articles/115014893968-Terms-of-sale}{Terms
  of Sale}
\item
  \href{https://spiderbites.nytimes.com}{Site Map}
\item
  \href{https://help.nytimes.com/hc/en-us}{Help}
\item
  \href{https://www.nytimes.com/subscription?campaignId=37WXW}{Subscriptions}
\end{itemize}
