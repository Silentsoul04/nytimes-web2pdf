Sections

SEARCH

\protect\hyperlink{site-content}{Skip to
content}\protect\hyperlink{site-index}{Skip to site index}

\href{https://myaccount.nytimes.com/auth/login?response_type=cookie\&client_id=vi}{}

\href{https://www.nytimes.com/section/todayspaper}{Today's Paper}

\href{/section/opinion}{Opinion}\textbar{}To Sway Swing Voters, Try
Empathy

\href{https://nyti.ms/2SgGzUJ}{https://nyti.ms/2SgGzUJ}

\begin{itemize}
\item
\item
\item
\item
\item
\end{itemize}

Advertisement

\protect\hyperlink{after-top}{Continue reading the main story}

\href{/section/opinion}{Opinion}

Supported by

\protect\hyperlink{after-sponsor}{Continue reading the main story}

FIXES

\hypertarget{to-sway-swing-voters-try-empathy}{%
\section{To Sway Swing Voters, Try
Empathy}\label{to-sway-swing-voters-try-empathy}}

A Brooklyn organization trains canvassers to engage with prospective
voters about their hopes and disappointments, not just give them a
candidate's talking points.

By Michael Massing

Mr. Massing writes about politics, the media, and social issues.

\begin{itemize}
\item
  Feb. 4, 2020
\item
  \begin{itemize}
  \item
  \item
  \item
  \item
  \item
  \end{itemize}
\end{itemize}

\includegraphics{https://static01.nyt.com/images/2020/02/04/opinion/04FIXESMassing/merlin_168178077_0498721c-5700-423b-a3fb-73ef84e982f1-articleLarge.jpg?quality=75\&auto=webp\&disable=upscale}

It's a truism that America has been gripped by tribalism, polarization
and rage. But what if it were possible to have a civil conversation with
an unlike-minded stranger? To find common ground and even persuade that
person to think differently?

In early December in Doylestown, Pa., a group of canvassers trained by a
liberal grass-roots organization tried to do just that.

Thirty of them set out with trepidation from the student center at
Delaware Valley University to knock on doors in Bucks County --- a swing
district that Hillary Clinton won by only 2,700 votes in 2016. (Mr.
Trump won Pennsylvania by just 44,300 votes.) Each canvasser had a list
of several dozen registered but infrequent voters to approach and
encourage to vote Democratic.

But these were not traditional canvassers. They were not working for a
particular candidate, nor did they have a set of fixed talking points to
hurriedly deliver. Instead, they hoped to engage people in a 10- to
20-minute conversation that would forge a connection based on shared
values. Normally, canvassers seek to identify their party's base and
mobilize its members. These canvassers were trying instead to reach
across the political and cultural divide.

\includegraphics{https://static01.nyt.com/images/2020/02/04/opinion/04FIXESMassing2/merlin_168178071_465e617c-0fd2-432a-bce2-95ba36afcba8-articleLarge.jpg?quality=75\&auto=webp\&disable=upscale}

Their technique is known as deep canvassing. It stresses active
listening and empathetic dialogue, rather than facts and arguments. A
leading advocate for it is \href{https://www.ctctogether.org/}{Changing
the Conversation Together}, a shoestring operation in Brooklyn. In early
November, the organization held a two-and-a-half-hour training workshop
for 20 volunteers. It was led by its director, Adam Barbanel-Fried. A
bearded 43-year-old who worked for years as a community organizer using
the principles of Saul Alinsky, Mr. Barbanel-Fried began by describing
the group's success in the 2018 congressional race in New York's 11th
District. Made up of Staten Island and a slice of Brooklyn, it was the
only New York City district to back Mr. Trump in 2016.

The organization recruited and trained nearly 300 canvassers to work in
Staten Island in support of the Democratic candidate,
\href{https://www.nytimes.com/2018/11/07/nyregion/what-max-rose-can-teach-democrats-about-beating-republicans.html}{Max
Rose}. They had more than 1,900 conversations --- Mr. Rose won the
district as a whole and took Staten Island by 1,800 votes. A
postelection survey by the organization found that 65 percent of those
who were canvassed reported voting Democratic, compared to 45 percent in
the same neighborhoods who were not canvassed.

One of the
\href{https://science.sciencemag.org/content/352/6282/220.full}{few
studies of the efficacy of deep canvassing} appeared in Science magazine
in 2016. The authors studied 56 canvassers who went door to door in
South Florida targeting prejudice against transgender individuals. They
found that a single 10-minute conversation that encouraged seeing the
perspective of others ``substantially reduced transphobic,'' with the
effects lasting at least three months.

Now Changing the Conversation Together is trying to use this approach at
the national level. But it aspires to more than putting a Democrat in
the White House. ``We want to form a national corps of deep canvassers
that embraces compassion and inclusion,'' Mr. Barbanel-Fried said.
``There's a whole world of
\href{https://www.nytimes.com/2020/06/29/us/politics/trump-swing-voters.html}{voters
in the middle} longing for connection.''

Storytelling is the key to achieving that connection, he said. Each
volunteer is expected to tell voters a story about a person he or she
loves --- and listen to the voter tell a similar story. In a
role-playing exercise, Mr. Barbanel-Fried talked about his 93-year-old
father, who read history books to him when he was young. ``He really
taught me that history doesn't just happen and isn't just a random
series of dates you have to memorize, but that they're a series of
choices that people make,'' he said. ``This year, when I vote, I'm
thinking about my father.''

After the volunteers divided into subgroups to practice their own
stories, each was given a two-page script. Canvassers were to begin by
asking voters what they would say to President Trump if given the
opportunity. Then, after acknowledging that they usually vote
Democratic, they were to ask voters to rate their likely voting
preference on a scale of zero (for unwaveringly Republican) to 10
(steadfastly Democratic).

Then came the personal story. ``Try to use the word LOVE,'' the script
advised. After the voter told his or her own story, the canvasser was to
note how the voter's values seemed to conflict with those of the
president, who, they would say, ``appeals to the worst human
tendencies.'' At the end, the canvasser was to ask the voter to again
rate her or his preference on the zero to 10 scale.

Some volunteers said they thought that talking about love was corny or
too personal. Mr. Barbanel-Fried insisted that it was critical to
connecting. There is a group of Trump loyalists whose votes can't be
affected, he said, ``but there are people whose values we share, and
we're trying to show them that there's a cognitive dissonance in their
lives'' between their love of people and their support for Mr. Trump.
``We need to lead with love, not hate.'' He ended by encouraging people
to sign up for the December canvass in Bucks County.

Each of the 30 people, including some local residents, who did show up
was given a list of people from across the political spectrum with
spotty voting histories. I accompanied Cindi Sternfeld, a 58-year-old
psychotherapist living in nearby Lambertville, N.J. She told me that she
had been canvassing since she was 18 and had found deep canvassing more
effective than the traditional kind. ``I like canvassing that advances
the discussion,'' she said. ``Being angry is not the answer because it
pushes people away.''

At her designated neighborhood, she found stately three- and
four-bedroom homes on spacious lots. After a string of unopened doors,
Ms. Sternfeld spotted one of her target voters standing in his driveway,
preparing to put up Christmas lights with his daughter. Tall and lanky,
with closely-cropped hair, James genially returned our greeting.

``I'm usually Republican, but I go by the candidate,'' he said. In the
last presidential election he wanted change, and Mr. Trump seemed more
likely to deliver it. He said he had a Trump sign in his garage. A
neighbor had a sign saying, ``Hate Has No Home Here.'' James asked his
neighbor if she hated Mr. Trump, and she said yes. ``Then why do you
have that sign?'' he asked. Because of such encounters, James said, he
would rate himself a two or three on the zero to 10 scale.

Ms. Sternfeld then told a story about her father --- a special-education
teacher who worked with older high school students with cognitive
disabilities and who spoke to them as the grown men they were. Years
later, when she became an educator, she said, she learned that people
with cognitive disabilities are not often shown such respect. ``My dad
is gone,'' she said, but his values of respect and dignity ``still
motivate me.''

``My dad's gone, too,'' James said. ``And I was brought up the same way.
I have a 45-year-old cousin with Down syndrome. I get fired up when
people use the `R' word.''

Ms. Sternfeld referred to their shared values, and James said that he
could see ``right off the bat'' that he could talk with her.

After about 20 minutes, in which James described several friendships
damaged by his support for Mr. Trump, Ms. Sternfeld asked James to rate
himself again. ``I'd say I'm a 5 now. I'm not adverse to voting
Democratic. I could go either way.'' He said he appreciated our having
taken the time to talk. ``You didn't push me in one direction,'' he
said. ``It was a conversation.''

Overall, in two hours of canvassing, Ms. Sternfeld had meaningful
conversations with five people, including a 26-year-old woman terrified
of Vice President Mike Pence's views on women's issues, a 37-year-old
man so disgusted by President Trump that he asked for more information
about how to get involved in canvassing, and a vehemently pro-Trump
contractor who derided his laborers as greedy and ungrateful. For Ms.
Sternfeld, however, James stood out. ``He's my favorite person ever,''
she said. Back at the center, Mr. Barbanel-Fried urged the group to
spread the word about the next Bucks County canvass on Jan. 26.

More than 90 people showed up, and they knocked on more than 700 doors.
Voters were not the only ones changed by the experience. One canvasser
said that when she saw a pickup truck outside one house, she assumed its
residents were Republicans. ``They invited me in,'' she said. ``The man
looked like the ultimate Trump supporter, but he was a Never Trumper.''
The lesson, she said, was not to judge so quickly.

Can such a labor-intensive effort be brought up to scale? Currently, Mr.
Barbanel-Fried is Changing the Conversation Together's sole full-time
staff member, a reflection of the difficulties he has had raising money.
``We're outside mainstream Democratic thinking,'' he lamented.

He said he hopes to have hundreds of trainers preparing thousands of
deep canvassers to build an electorate bound by an emotional connection.
That, he thinks, could change the national conversation on matters like
health care and the environment.

``It's very helpful to just talk human to human,'' says Jennifer Jarret,
an activist in Doylestown who helped the organization's efforts in Bucks
County. ``We might not see things the same way, but if we can get past
the idea of our neighbors as `others,' that alone could help us as a
community and a nation.''

Michael Massing is a former executive editor of the Columbia Journalism
Review and the author, most recently, of ``Fatal Discord: Erasmus,
Luther, and the Fight for the Western Mind.''

\emph{To receive email alerts for Fixes columns, sign up}
\href{http://eepurl.com/ABIxL}{\emph{here.}}

\emph{The Times is committed to publishing}
\href{https://www.nytimes.com/2019/01/31/opinion/letters/letters-to-editor-new-york-times-women.html}{\emph{a
diversity of letters}} \emph{to the editor. We'd like to hear what you
think about this or any of our articles. Here are some}
\href{https://help.nytimes.com/hc/en-us/articles/115014925288-How-to-submit-a-letter-to-the-editor}{\emph{tips}}\emph{.
And here's our email:}
\href{mailto:letters@nytimes.com}{\emph{letters@nytimes.com}}\emph{.}

\emph{Follow The New York Times Opinion section on}
\href{https://www.facebook.com/nytopinion}{\emph{Facebook}}\emph{,}
\href{http://twitter.com/NYTOpinion}{\emph{Twitter (@NYTopinion)}}
\emph{and}
\href{https://www.instagram.com/nytopinion/}{\emph{Instagram}}\emph{.}

Advertisement

\protect\hyperlink{after-bottom}{Continue reading the main story}

\hypertarget{site-index}{%
\subsection{Site Index}\label{site-index}}

\hypertarget{site-information-navigation}{%
\subsection{Site Information
Navigation}\label{site-information-navigation}}

\begin{itemize}
\tightlist
\item
  \href{https://help.nytimes.com/hc/en-us/articles/115014792127-Copyright-notice}{©~2020~The
  New York Times Company}
\end{itemize}

\begin{itemize}
\tightlist
\item
  \href{https://www.nytco.com/}{NYTCo}
\item
  \href{https://help.nytimes.com/hc/en-us/articles/115015385887-Contact-Us}{Contact
  Us}
\item
  \href{https://www.nytco.com/careers/}{Work with us}
\item
  \href{https://nytmediakit.com/}{Advertise}
\item
  \href{http://www.tbrandstudio.com/}{T Brand Studio}
\item
  \href{https://www.nytimes.com/privacy/cookie-policy\#how-do-i-manage-trackers}{Your
  Ad Choices}
\item
  \href{https://www.nytimes.com/privacy}{Privacy}
\item
  \href{https://help.nytimes.com/hc/en-us/articles/115014893428-Terms-of-service}{Terms
  of Service}
\item
  \href{https://help.nytimes.com/hc/en-us/articles/115014893968-Terms-of-sale}{Terms
  of Sale}
\item
  \href{https://spiderbites.nytimes.com}{Site Map}
\item
  \href{https://help.nytimes.com/hc/en-us}{Help}
\item
  \href{https://www.nytimes.com/subscription?campaignId=37WXW}{Subscriptions}
\end{itemize}
