Sections

SEARCH

\protect\hyperlink{site-content}{Skip to
content}\protect\hyperlink{site-index}{Skip to site index}

\href{https://www.nytimes.com/section/technology}{Technology}

\href{https://myaccount.nytimes.com/auth/login?response_type=cookie\&client_id=vi}{}

\href{https://www.nytimes.com/section/todayspaper}{Today's Paper}

\href{/section/technology}{Technology}\textbar{}Twitter Moves to Target
Fake Videos and Photos

\url{https://nyti.ms/2txOQvk}

\begin{itemize}
\item
\item
\item
\item
\item
\end{itemize}

Advertisement

\protect\hyperlink{after-top}{Continue reading the main story}

Supported by

\protect\hyperlink{after-sponsor}{Continue reading the main story}

\hypertarget{twitter-moves-to-target-fake-videos-and-photos}{%
\section{Twitter Moves to Target Fake Videos and
Photos}\label{twitter-moves-to-target-fake-videos-and-photos}}

While stopping short of an outright ban, the social media company plans
to label or take down material that appears to have been digitally
manipulated.

\includegraphics{https://static01.nyt.com/images/2020/02/04/business/04twitter/merlin_161161551_08cc2cb7-ceaf-46b8-a650-38d39bc36f6c-articleLarge.jpg?quality=75\&auto=webp\&disable=upscale}

By \href{https://www.nytimes.com/by/davey-alba}{Davey Alba} and
\href{https://www.nytimes.com/by/kate-conger}{Kate Conger}

\begin{itemize}
\item
  Feb. 4, 2020
\item
  \begin{itemize}
  \item
  \item
  \item
  \item
  \item
  \end{itemize}
\end{itemize}

Twitter, bowing to pressure from its users, said on Tuesday that it
would more aggressively scrutinize fake or altered photos and videos.

Starting in March, the company said, it will add labels or take down
tweets carrying manipulated images and videos. The move, while short of
an outright ban, was announced one day after
\href{https://www.nytimes.com/2020/02/03/technology/youtube-misinformation-election.html}{YouTube
also said it planned to remove misleading election-related content on
its site}.

Twitter's new policy highlights a balancing act --- between allowing
parody and removing disinformation --- that social media companies face
as they try to more aggressively police the content posted to their
platforms.

To determine whether a tweet should be removed or labeled, Twitter
\href{https://blog.twitter.com/en_us/topics/company/2020/new-approach-to-synthetic-and-manipulated-media.html}{said
in a blog post}, it will apply several tests: Is the media included with
a tweet significantly altered or fabricated to mislead? Is it shared in
a deceptive manner? In those cases, the tweet will probably get a label.

But if a tweet is ``likely to impact public safety or cause serious
harm,'' it will be taken down. Twitter said it might also show a warning
to people before they engaged with a tweet carrying manipulated content,
or limit that tweet's reach.

``Our approach does not focus on the specific technologies used to
manipulate or fabricate media,'' said Yoel Roth, Twitter's head of site
integrity. ``Whether you're using advanced machine learning tools or
just slowing down a video using a 99-cent app on your phone, our focus
under this policy is to look at the outcome, not how it was achieved.''

The company developed its rules after surveying more than 6,500 users,
civil groups and academics, said Del Harvey, Twitter's vice president
for trust and safety.

They found that about 70 percent of surveyed Twitter users believed it
was unacceptable for the company to take no action against manipulated
content. More than 90 percent said such content should be removed or
placed behind a warning label saying the video or image had been
altered.

``Things that distort or distract from what's happening threaten the
integrity of information on Twitter,'' Ms. Harvey said.

Like other social networks that have tried to crack down on bogus
content, Twitter will be under pressure to consistently apply its new
rules.

Samantha Bradshaw, a researcher at the Oxford Internet Institute, said
that defining harm was not always clear, especially in the context of
social media. ``And it would be difficult to automate these responses on
a global scale,'' she said.

Last year, an
\href{https://www.nytimes.com/2019/05/24/us/politics/pelosi-doctored-video.html}{altered
video of Speaker Nancy Pelosi} that made it appear that she was slurring
her words spread across the internet. Another heavily
\href{https://www.nytimes.com/2020/01/07/us/politics/biden-video-disinformation-spread.html}{edited
clip of former Vice President Joseph R. Biden Jr}. falsely made it seem
as though he had made racist remarks.

Mr. Roth said the manipulated videos of Ms. Pelosi and Mr. Biden would
get a label under Twitter's new policy. Depending on what a tweet
sharing the video said and if it caused harm, Mr. Roth said, Twitter
could take the tweet down.

Last year, Twitter said it would add warning labels to
\href{https://www.nytimes.com/2019/06/27/technology/twitter-politicans-labels-abuse.html}{hide
messages from major political figures} who broke the company's rules for
harassment or abuse. Normally, those tweets would be taken down, but the
company argued that they would be newsworthy enough to remain on the
platform. As of Tuesday, Twitter had not yet used the labels.

In January,
\href{https://www.nytimes.com/2020/01/07/technology/facebook-says-it-will-ban-deepfakes.html}{Facebook
banned ``deepfake'' videos} from its platform. But the company said the
videos of Ms. Pelosi and Mr. Biden would not be removed under the policy
because they had been edited with video editing software, not artificial
intelligence.

YouTube banned misleading political content on Monday as part of a new
policy ahead of the presidential election in November.

Advertisement

\protect\hyperlink{after-bottom}{Continue reading the main story}

\hypertarget{site-index}{%
\subsection{Site Index}\label{site-index}}

\hypertarget{site-information-navigation}{%
\subsection{Site Information
Navigation}\label{site-information-navigation}}

\begin{itemize}
\tightlist
\item
  \href{https://help.nytimes.com/hc/en-us/articles/115014792127-Copyright-notice}{©~2020~The
  New York Times Company}
\end{itemize}

\begin{itemize}
\tightlist
\item
  \href{https://www.nytco.com/}{NYTCo}
\item
  \href{https://help.nytimes.com/hc/en-us/articles/115015385887-Contact-Us}{Contact
  Us}
\item
  \href{https://www.nytco.com/careers/}{Work with us}
\item
  \href{https://nytmediakit.com/}{Advertise}
\item
  \href{http://www.tbrandstudio.com/}{T Brand Studio}
\item
  \href{https://www.nytimes.com/privacy/cookie-policy\#how-do-i-manage-trackers}{Your
  Ad Choices}
\item
  \href{https://www.nytimes.com/privacy}{Privacy}
\item
  \href{https://help.nytimes.com/hc/en-us/articles/115014893428-Terms-of-service}{Terms
  of Service}
\item
  \href{https://help.nytimes.com/hc/en-us/articles/115014893968-Terms-of-sale}{Terms
  of Sale}
\item
  \href{https://spiderbites.nytimes.com}{Site Map}
\item
  \href{https://help.nytimes.com/hc/en-us}{Help}
\item
  \href{https://www.nytimes.com/subscription?campaignId=37WXW}{Subscriptions}
\end{itemize}
