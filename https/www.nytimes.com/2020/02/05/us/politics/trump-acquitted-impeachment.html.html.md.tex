Sections

SEARCH

\protect\hyperlink{site-content}{Skip to
content}\protect\hyperlink{site-index}{Skip to site index}

\href{https://www.nytimes.com/section/politics}{Politics}

\href{https://myaccount.nytimes.com/auth/login?response_type=cookie\&client_id=vi}{}

\href{https://www.nytimes.com/section/todayspaper}{Today's Paper}

\href{/section/politics}{Politics}\textbar{}Trump Acquitted of Two
Impeachment Charges in Near Party-Line Vote

\url{https://nyti.ms/382Saxs}

\begin{itemize}
\item
\item
\item
\item
\item
\item
\end{itemize}

\begin{itemize}
\item
  \href{https://www.nytimes.com/2020/07/31/us/elections/biden-vs-trump.html?action=click\&pgtype=Article\&state=default\&region=TOP_BANNER\&context=storylines_menu}{Election
  Updates}
\item
  \href{https://www.nytimes.com/article/biden-vice-president-2020.html?action=click\&pgtype=Article\&state=default\&region=TOP_BANNER\&context=storylines_menu}{Biden's
  V.P. Search}
\item
  \href{https://www.nytimes.com/interactive/2020/07/24/us/politics/trump-biden-campaign-donors.html?action=click\&pgtype=Article\&state=default\&region=TOP_BANNER\&context=storylines_menu}{Map
  of Donations}
\item
  \href{https://www.nytimes.com/interactive/2020/us/elections/delegate-count-primary-results.html?action=click\&pgtype=Article\&state=default\&region=TOP_BANNER\&context=storylines_menu}{Delegate
  Count}
\item
  \href{https://www.nytimes.com/interactive/2019/us/politics/2020-presidential-candidates.html?action=click\&pgtype=Article\&state=default\&region=TOP_BANNER\&context=storylines_menu}{The
  Candidates}
\item
  \href{https://www.nytimes.com/newsletters/politics?action=click\&pgtype=Article\&state=default\&region=TOP_BANNER\&context=storylines_menu}{Politics
  Newsletter}
\end{itemize}

Advertisement

\protect\hyperlink{after-top}{Continue reading the main story}

Supported by

\protect\hyperlink{after-sponsor}{Continue reading the main story}

\hypertarget{trump-acquitted-of-two-impeachment-charges-in-near-party-line-vote}{%
\section{Trump Acquitted of Two Impeachment Charges in Near Party-Line
Vote}\label{trump-acquitted-of-two-impeachment-charges-in-near-party-line-vote}}

As Republicans rallied behind President Trump, Senator Mitt Romney of
Utah, the party's 2012 presidential nominee, joined Democrats in voting
to convict, the only senator to cross party lines.

\includegraphics{https://static01.nyt.com/images/2020/02/05/us/politics/05dc-impeach1-sub/05dc-impeach1-sub-articleLarge-v2.jpg?quality=75\&auto=webp\&disable=upscale}

\href{https://www.nytimes.com/by/nicholas-fandos}{\includegraphics{https://static01.nyt.com/images/2018/11/06/multimedia/author-nicholas-fandos/author-nicholas-fandos-thumbLarge-v2.png}}

By \href{https://www.nytimes.com/by/nicholas-fandos}{Nicholas Fandos}

\begin{itemize}
\item
  Feb. 5, 2020
\item
  \begin{itemize}
  \item
  \item
  \item
  \item
  \item
  \item
  \end{itemize}
\end{itemize}

WASHINGTON --- After five months of hearings, investigations and
revelations about President Trump's dealings with Ukraine, a divided
United States Senate acquitted him on Wednesday of charges that he
abused his power and obstructed Congress to aid his own re-election,
bringing an acrimonious impeachment trial to its expected end.

In a pair of votes whose outcome was never in doubt, the Senate fell
well short of the two-thirds margin that would have been needed to
remove the 45th president. The verdicts came down --- after three weeks
of debate --- almost entirely along party lines, with every Democrat
voting ``guilty'' on both charges and Republicans uniformly voting ``not
guilty'' on the obstruction of Congress charge.

Only one Republican,
\href{https://www.nytimes.com/2020/02/06/podcasts/the-daily/mitt-romney.html}{Senator
Mitt Romney} of Utah, broke with his party to judge Mr. Trump guilty of
abuse of power.

It was the third impeachment trial of a president and the third
acquittal in American history, and it ended the way it began: with
Republicans and Democrats at odds. They disagreed over Mr. Trump's
conduct and his fitness for office, even as some members of his own
party conceded the basic allegations that undergirded the charges, that
he sought to pressure Ukraine to smear his political rivals.

But in a sign of the widening partisan divide testing the country and
its institutions, the verdict did not promise finality, which members of
both parties conceded would come only after the November election.

The president himself did not directly address his acquittal, but
shortly afterward,
\href{https://twitter.com/realDonaldTrump/status/1225179058000089090}{he
announced on Twitter} that he would make a public statement on Thursday
at the White House about what he called ``our Country's VICTORY on the
Impeachment Hoax.'' He then
\href{https://twitter.com/realDonaldTrump/status/1225203837226700800}{tweeted
an attack ad} against Mr. Romney that called the senator a ``Democrat
secret asset.''

\href{https://www.nytimes.com/interactive/2020/02/05/us/politics/impeachment-vote-results.html}{}

\includegraphics{https://static01.nyt.com/images/2020/02/05/us/politics/impeachment-vote-results-promo-final/impeachment-vote-results-promo-final-articleLarge.png}

\hypertarget{trump-impeachment-results-how-democrats-and-republicans-voted}{%
\subsection{Trump Impeachment Results: How Democrats and Republicans
Voted}\label{trump-impeachment-results-how-democrats-and-republicans-voted}}

See how each senator will vote on whether to convict and remove
President Trump from office.

At the Capitol earlier in the day, Chief Justice John G. Roberts Jr.,
who presided over the trial, put the question to senators shortly after
4 p.m.: ``Senators how say you? Is the respondent, Donald John Trump,
president of the United States guilty or not guilty?''

Seated at their
\href{https://www.nytimes.com/2020/01/17/us/politics/senate-impeachment-trial-furniture.html}{mahogany
desks}, senators stood one by one to answer ``guilty'' or ``not guilty''
to each of the two articles of impeachment.

``It is, therefore, ordered and adjudged that the said Donald John Trump
be, and he is hereby, acquitted of the charges in said articles,''
declared Chief Justice Roberts after the second article was defeated.

\hypertarget{latest-updates-2020-election}{%
\section{\texorpdfstring{\href{https://www.nytimes.com/2020/07/31/us/elections/biden-vs-trump.html?action=click\&pgtype=Article\&state=default\&region=MAIN_CONTENT_1\&context=storylines_live_updates}{Latest
Updates: 2020
Election}}{Latest Updates: 2020 Election}}\label{latest-updates-2020-election}}

Updated 2020-08-01T01:26:45.732Z

\begin{itemize}
\tightlist
\item
  \href{https://www.nytimes.com/2020/07/31/us/elections/biden-vs-trump.html?action=click\&pgtype=Article\&state=default\&region=MAIN_CONTENT_1\&context=storylines_live_updates\#link-29fdff45}{Kamala
  Harris, a top vice-presidential contender, confronts double
  standards.}
\item
  \href{https://www.nytimes.com/2020/07/31/us/elections/biden-vs-trump.html?action=click\&pgtype=Article\&state=default\&region=MAIN_CONTENT_1\&context=storylines_live_updates\#link-13ec3d9c}{Karen
  Bass and Susan Rice are rising on Biden's vice-presidential
  shortlist.}
\item
  \href{https://www.nytimes.com/2020/07/31/us/elections/biden-vs-trump.html?action=click\&pgtype=Article\&state=default\&region=MAIN_CONTENT_1\&context=storylines_live_updates\#link-49e9a016}{Trump
  says Russian bounties to kill U.S. troops `never took place.'}
\end{itemize}

\href{https://www.nytimes.com/2020/07/31/us/elections/biden-vs-trump.html?action=click\&pgtype=Article\&state=default\&region=MAIN_CONTENT_1\&context=storylines_live_updates}{See
more updates}

Democratic leaders immediately insisted the result was illegitimate, the
product of a self-interested cover-up by Republicans, and promised to
continue their investigations of Mr. Trump.

``The verdict of this kangaroo court will be meaningless,'' Senator
Chuck Schumer of New York, the Democratic leader, said moments before
the vote. ``By refusing the facts --- by refusing witnesses and
documents --- the Republican majority has placed a giant asterisk, the
asterisk of a sham trial, next to the acquittal of President Trump,
written in permanent ink.''

\includegraphics{https://static01.nyt.com/images/2017/01/29/podcasts/the-daily-album-art/the-daily-album-art-articleInline-v2.jpg?quality=75\&auto=webp\&disable=upscale}

\hypertarget{listen-to-the-daily-mitt-romneys-lonely-vote}{%
\subsubsection{Listen to `The Daily': Mitt Romney's Lonely
Vote}\label{listen-to-the-daily-mitt-romneys-lonely-vote}}

We spoke with the only senator in history who has voted to convict a
president of his own party about the thinking behind his decision.

transcript

Back to The Daily

bars

0:00/31:11

-31:11

transcript

\hypertarget{listen-to-the-daily-mitt-romneys-lonely-vote-1}{%
\subsection{Listen to `The Daily': Mitt Romney's Lonely
Vote}\label{listen-to-the-daily-mitt-romneys-lonely-vote-1}}

\hypertarget{hosted-by-michael-barbaro-produced-by-rachel-quester-eric-krupke-neena-pathak-and-jonathan-wolfe-and-edited-by-mj-davis-lin-and-lisa-tobin}{%
\subsubsection{Hosted by Michael Barbaro, produced by Rachel Quester,
Eric Krupke, Neena Pathak and Jonathan Wolfe, and edited by M.J. Davis
Lin and Lisa
Tobin}\label{hosted-by-michael-barbaro-produced-by-rachel-quester-eric-krupke-neena-pathak-and-jonathan-wolfe-and-edited-by-mj-davis-lin-and-lisa-tobin}}

\hypertarget{we-spoke-with-the-only-senator-in-history-who-has-voted-to-convict-a-president-of-his-own-party-about-the-thinking-behind-his-decision}{%
\paragraph{We spoke with the only senator in history who has voted to
convict a president of his own party about the thinking behind his
decision.}\label{we-spoke-with-the-only-senator-in-history-who-has-voted-to-convict-a-president-of-his-own-party-about-the-thinking-behind-his-decision}}

\begin{itemize}
\item
  michael barbaro\\
  From The New York Times, I'm Michael Barbaro. This is ``The Daily.''

  Today: President Donald Trump is acquitted of both articles of
  impeachment. Just one senator crossed party lines to vote to convict
  him. A conversation with Mitt Romney about that decision.

  It's Thursday, February 6.

  Mark Leibovich, tell me about these conversations that you've been
  having with Senator Mitt Romney.
\item
  mark leibovich\\
  Well, Mitt Romney has always fascinated me as kind of a wild card in
  his political life as a moderate governor of Massachusetts, as a,
  quote, ``severely conservative'' presidential candidate in 2012. Then,
  as a critic of Donald Trump when Donald Trump took over the party.
  Then, as a potential Donald Trump cabinet member when he talked to him
  about being secretary of state in 2016. Then, as a Senate candidate,
  someone who I wouldn't say embraced Donald Trump, but someone who
  certainly didn't push him away. And then as a senator, someone who has
  been fairly unshy at times about defying Donald Trump, being critical
  of him.
\item
  michael barbaro\\
  Right. Somebody who is seen as ideologically malleable. Someone who's
  seen as inconsistent.
\item
  mark leibovich\\
  Correct. And in this impeachment proceeding, he has been the ultimate
  wild card. No one knew what he was going to do really. People had
  ideas back and forth. But you never know what you're going to get with
  Mitt Romney. So I had been asking his office --- and I'm certainly not
  the only reporter who had been asking --- whether I could hang around
  with him a little bit, whether I could actually go through this
  process with him. Which I figured was a bit of a long shot because
  he's been in such demand. And to my surprise last week, right as we
  were leading up to the big vote on witnesses, Mitt Romney agreed to
  sit down with me.
\item
  michael barbaro\\
  Wow.
\item
  mark leibovich\\
  Thank you again. {[}INAUDIBLE{]}
\end{itemize}

mark leibovich

We went up to his hideaway office, which was like a kind of ---

michael barbaro

Hideaway office?

mark leibovich

It's like a remote office. Every senator gets one.

\begin{itemize}
\item
  mark leibovich\\
  I mean, here's the question. Have you every taken a nap in here?
\item
  mitt romney\\
  I have not.
\end{itemize}

mark leibovich

And Mitt Romney's was filled with M\&M's.

\begin{itemize}
\tightlist
\item
  mark leibovich\\
  Oh my god, peanut M\&M's.
\end{itemize}

mark leibovich

This was like a little break in the proceedings. So we didn't have a lot
of time.

\begin{itemize}
\tightlist
\item
  mark leibovich\\
  I feel your body language. So I'm going to be real quick here.
\end{itemize}

mark leibovich

And I was amazed at how open he was about the kinds of things he was
thinking about.

\begin{itemize}
\item
  mark leibovich\\
  Hello.
\item
  mitt romney\\
  Hi.
\item
  mark leibovich\\
  It's good to see you.
\item
  mitt romney\\
  Good to see you.
\item
  mark leibovich\\
  So just sitting there, you've sat there, now, what, a week and a half?
\item
  mitt romney\\
  No, it's been a couple of months. {[}LAUGHS{]}
\item
  mark leibovich\\
  Well, I know it feels like a a couple of months.
\item
  mitt romney\\
  Yes, probably. What did we say, nine days? Is that it?
\item
  aide\\
  Yeah.
\item
  mark leibovich\\
  Does the experience of sitting in the chamber, listening to this day
  in and day out, intensify the burden at all? Does it make it seem
  weightier? Does it make it seem less weighty? What's it been like for
  you?
\item
  mitt romney\\
  I think it was most weighty having the chief justice come in the first
  time. And I think there was a sense of how important this is and how
  historic it is. We have to do what we feel is right in the case. But I
  think we also have to think more broadly as to what are the
  implications nationally and what are the implications for the
  institution of the Senate.
\item
  mark leibovich\\
  What about the presidency?
\item
  mitt romney\\
  And the presidency, both. I mean, that's ---
\item
  mark leibovich\\
  I mean, put aside impeachment. What about just the nature of right and
  wrong? Like what a chief executive of this country should do. I mean,
  isn't that on trial to some degree?
\item
  mitt romney\\
  Well, I think if there were a president that really was going to be
  removed under the crime or misdemeanor standard, he or she would have
  done something wrong. By definition that would ---
\item
  {[}interposing voices{]}
\item
  mark leibovich\\
  Yes, one would hope.
\item
  mitt romney\\
  But by and large, most matters of right and wrong associated with the
  president are going to be determined by the electorate in the upcoming
  election. Impeachment is not to judge right and wrong alone. It is to
  judge right and wrong in the context of, has there been a high crime
  and misdemeanor?
\end{itemize}

mark leibovich

And one of the last things he brought up to me was a sense of obligation
he felt to the United States Constitution.

\begin{itemize}
\tightlist
\item
  mitt romney\\
  This is --- it's a constitutional issue. I feel a sense of deep
  responsibility to abide by the Constitution and to determine absent
  that pulls from the right and pulls from the left. What is the right
  thing to do? What does the Constitution demand?
\end{itemize}

mark leibovich

So it was very clear that this was weighing on him.

\begin{itemize}
\tightlist
\item
  {[}chatter{]}
\end{itemize}

mark leibovich

And I got sort of ushered out of his office. And he had to go back to
the floor to continue the trial.

\begin{itemize}
\item
  mark leibovich\\
  O.K., wait. Do we go this way or this way?
\item
  mitt romney\\
  I'll show you.
\item
  speaker\\
  We'll go ---
\end{itemize}

mark leibovich

We're walking out. He goes back to the floor. I took a right turn in a
hallway. And there was Amy Klobuchar, who was back in Washington for
about 36 hours to do her impeachment stuff, taking a break from her
presidential campaign. And I said, Senator, can I ask you a few
questions about Mitt Romney?

\begin{itemize}
\item
  mark leibovich\\
  Mitt Romney --- have you thought of this role here just as kind of a
  fellow center --- not centrist, but someone who is a bit of a wild
  card in all this?
\item
  amy klobuchar\\
  Um, uh, uh ---
\end{itemize}

mark leibovich

And the name Mitt Romney sort of stopped her in her tracks a little bit.

\begin{itemize}
\tightlist
\item
  amy klobuchar\\
  Hold on. I'm sorry. Yes, that's a good question. So I hope he plays
  that leadership role that I think John McCain would have played if he
  was here. I thought like every hour about if Senator McCain was here.
\end{itemize}

mark leibovich

And she used this as a prompt to talk about John McCain. The very
important role that John McCain, the maverick, played in the United
States Senate, as someone who could recruit other potentially dissident
Republicans to his side and create a counterforce within the prevailing
force that is the Republican Party in the Senate.

michael barbaro

So she's raising the specter that Mitt Romney could be a John McCain in
this moment, i.e., could buck the party and perhaps even take other
Republicans with him in siding with Democrats in impeachment.

mark leibovich

Correct. She saw him as someone who was genuinely agonizing, someone who
could be a leader and someone that no one really had a grip on at this
point.

\begin{itemize}
\item
  mark leibovich\\
  Has he demonstrated at all that he could have that like capability to
  actually play a McCain-like ---
\item
  amy klobuchar\\
  Well, at least he's been willing to tell the truth here about the need
  to have the witness, which I've appreciated. This is his moment to
  shine and hopefully he can bring some people with him.
\item
  mark leibovich\\
  Thank you.
\end{itemize}

mark leibovich

So after my conversation with Amy Klobuchar, I left the Hill. I said to
Romney's office, I said, I'd love to keep in touch with him. We did not
talk about his ultimate decision. He didn't want to talk about it yet.
He made that pretty clear. But also, I wanted to get a window into his
decision-making process. And we had about five days until this was all
going to come to a head. So to my surprise and delight, they said, come
on back on Wednesday morning.

michael barbaro

Wow, the day of ---

mark leibovich

The day of the decision. The day of the final vote.

\begin{itemize}
\item
  mark leibovich\\
  Senator ---
\item
  mitt romney\\
  Long time no see.
\item
  mark leibovich\\
  I know.
\end{itemize}

michael barbaro

O.K. So on Wednesday morning, what happens? What do you do?

mark leibovich

Wednesday morning, 10:30. I went into Mitt Romney's office.

\begin{itemize}
\item
  mark leibovich\\
  Let's do this. We have a table, table there.
\item
  mitt romney\\
  We have a table there.
\end{itemize}

mark leibovich

I felt myself getting quite nervous, which is interesting because I
almost --- you always sort of know what you're going to get in an
interview. I just didn't know how this was going to go. This was a real
jump ball. I mean ---

michael barbaro

You called it a wild card.

mark leibovich

It was a wild card. You could call it whatever you want. But I just
didn't know how this was going to go. He looked exactly like he always
looks, which is fresh as a daisy. He hadn't been sleeping. But you
couldn't tell. He looked ready to go. He didn't look like he was in a
mood for any kind of small talk. I wasn't either.

\begin{itemize}
\item
  mark leibovich\\
  Anyway, so anyone care about what you have to say today?
\item
  mitt romney\\
  Probably.
\item
  mark leibovich\\
  Probably. All right, let's get right to it. Do you have --- do you
  know how you're going to vote?
\item
  mitt romney\\
  Yeah, yeah, I'm going to vote yes on the first article. No on the
  second.
\end{itemize}

mark leibovich

He said that he would vote yes to convict the president on charges of
abuse of power. And he would vote no on the second impeachment article,
which is obstruction of Congress.

\begin{itemize}
\item
  mitt romney\\
  There's no question the president fought providing documents and
  witnesses. But he did so through the use of the law. And the House, in
  my view, should have gone to the courts to get that resolved. By not
  going to the courts, they ---
\item
  mark leibovich\\
  They forfeited.
\item
  mitt romney\\
  --- I don't know how they make the case.
\end{itemize}

mark leibovich

And as soon as he started talking, I could see the emotion in his face.
I could hear him, not choke up, but certainly there was a strain in his
voice that I had not heard before.

\begin{itemize}
\tightlist
\item
  mitt romney\\
  On the first article, I think the case was made. And I believe that
  attempting to corrupt an election to maintain power is about as an
  egregious an assault on the Constitution as can be made. And for that
  reason, it is a high crime and misdemeanor. And I have no choice under
  the oath I took but to express that conclusion.
\end{itemize}

mark leibovich

He said, basically, I think he'd gone back and forth.

\begin{itemize}
\item
  mitt romney\\
  There's an old Protestant hymn that we sing in our church called ``Do
  what is right, let the consequence follow.'' I'm sure I'm doing the
  right thing. I don't know that I can weigh the consequence at this
  stage. But it's going to be substantial.
\item
  mark leibovich\\
  Yeah I mean, this is interesting. Wow, I mean ---
\item
  mitt romney\\
  But you know I'm a very religious person. And when you swear an oath
  before God to apply impartial justice, that's what I believe I have to
  do. And by the way, I believe other senators do the same thing. I'm
  not the only one voting my conscience. But not voting my conscience,
  in order for me to have a better political and personal benefit, would
  subject my own conscience to its censure. So I just --- I don't have a
  choice there. This for me is fundamental to my oath to God and
  fundamental to how our country must work, which is that people have to
  be seen as honest in fulfilling that oath that they take.
\item
  mark leibovich\\
  When did you make this decision?
\item
  mitt romney\\
  I tried to keep from forming a final decision as I listened to both
  sides ---
\item
  mark leibovich\\
  As a juror.
\item
  mitt romney\\
  --- and I went all the way --- as a juror. And as I was going, of
  course, sometimes I'd be swayed one way. Sometimes I'd be swayed the
  other. I reached my conclusion really after the last day of questions
  and answers.
\item
  mark leibovich\\
  So that would have been Friday --- or Thursday?
\item
  mitt romney\\
  Yeah, Thursday. But I can tell you that throughout this entire period,
  there's not been a morning I've gotten up after 4 a.m. Just obviously
  thinking about how important this, with the consequences. But then
  also analyzing, going back and looking at the testimony, reading the
  briefs of the two sides, going back through Federalist Papers. And
  then applying logic to it.
\item
  mark leibovich\\
  What was it like for you to sit in --- or stand, in some cases ---
  through what was a very tribal gathering last night at the State of
  the Union. And I was watching you quite a bit because I was up in the
  gallery.

  It was quite a speech. And you knew what you --- the current you were
  going to be swimming up against in 24 hours. What was that like?
\item
  mitt romney\\
  I think people have a very hard time understanding how you just don't
  vote with the team, and how you can make a decision of this
  significance, unless you're just doing it with the team. And it's
  like, well, then think back to a jury you may have been on. And ask
  did you just go with whether it was a male or a female, or a black or
  a white or Hispanic, or non-Hispanic? Or did you try and apply a
  partial justice? Did you take your oath seriously? And you take your
  oath seriously. I agree with most of the things the president has
  done. The policies he put in with regards to the economy are very
  close to the policies I campaigned on four years before. I agree with
  those things. The fact the economy is doing as well as it is in part
  because of those policies. So he's going to take a bow for those
  policies, I --- I'm with him. So I'm with the pre --- and by the way,
  I think he's going to get re-elected. I think if Bernie or Elizabeth
  is the nominee on the Democratic side, he'll get elected in a
  landslide. I will still vote for the policies I agree with. I'll stand
  and applaud when he says things that are right. But then he did one
  thing we know of that was a very seriously wrong thing. And not to
  call it grievously wrong would be to violate my oath, violate my
  conscience, subject me to the censure of history.
\item
  mark leibovich\\
  What kind of consequences do you think you'll endure for this?
\item
  mitt romney\\
  Unimaginable. I don't know what they'll be. There's some I know. I
  know there will be consequence. And I just have to recognize that. And
  do what you think is right. And ---
\end{itemize}

mark leibovich

What was interesting to me --- and this is one of those things that
doesn't pick up so much on tape --- but you see his facial expressions.
I mean, Mitt Romney is a very smooth character in his own ways. His face
got red. He had a bit of dread in his eyes. It was as if he knew that a
chandelier was about to drop on his head.

michael barbaro

Wow.

\begin{itemize}
\item
  mitt romney\\
  The reason I wanted witnesses --- and that was the area that there was
  a greatest discussion within our caucus, was we don't get witnesses.
  The reason I wanted to hear from John Bolton is that I hoped beyond
  measure that he would say something that would provide reasonable
  doubt so I wouldn't have to vote to convict.
\item
  mark leibovich\\
  So you were looking for reasons not to vote for ---
\item
  mitt romney\\
  Look, my personal and political and team affiliation made me very much
  not want to have to convict. I mean, I want to be with my colleagues
  in the Senate. I don't want to be the skunk at the garden party. I
  don't want to have the disdain of Republicans across the country. I
  was at the grocery store this weekend, and a guy went by me and said
  ``traitor.''
\item
  mark leibovich\\
  Where was this? In Utah?
\item
  mitt romney\\
  No, it was in Florida. I was down at one of Ann's competitions.
  Another person yelled from their cars as I was taking my groceries out
  of the car. Yelled from his car, ``Stick with the team!'' And so I
  recognized this is going to be a whole different level. But ---
\end{itemize}

michael barbaro

Why do you think Mitt Romney did this interview with you? I mean, he
could have decided to show up and vote to convict the president. Why sit
down with The New York Times and talk about all the agony and explain
himself?

mark leibovich

Well, I mean, one thing I've always been interested in with Mitt Romney
is that he has always been, not in a self-absorbed way, but he's always
been very aware of his own political narrative. He has been aware of how
he was viewed, maybe as a political opportunist. Him maybe doing things
out of expediency rather than principle. And I think ultimately, one of
the things that this Senate chapter has done for him and his career is
it has enabled him to maybe rewrite the ending. Maybe recast himself as
someone who did feel as if he was doing the right thing at the expense
of whatever the expedient decision at the moment would have been.

\begin{itemize}
\item
  mark leibovich\\
  --- any number of issues. Does any of that weigh on you? You're in
  your 70s now. This is probably your last job. Maybe this is an
  important enough issue that I could really take a stand, and I mean,
  and just do the right thing. I mean, does this --- do you ever think
  about these decisions in light of other decisions you have made when
  you had more politically to lose?
\item
  mitt romney\\
  I haven't given the full analysis to my whole political history. But I
  will with time, particularly I'm sure in retirement.

  My guess is that I was influenced in some cases by political benefit.
  And I regret that. And probably not to the extent to which my
  opponents tried to characterize it. But looking back, there's an item
  or two where I said I wish I had said that differently or taken a
  different position rather. I don't mean to make it just seem like just
  a couple of words. No, take the position that I --- and as is often
  the case, I have found in business, in particular, but also in
  politics, that when something is in your personal best interest, the
  ability of the mind to rationalize that that's the right thing is
  really quite extraordinary. And I'm talking about myself. And I've
  seen it in others. I've seen it in myself.
\item
  mark leibovich\\
  Especially in politics generally.
\item
  mitt romney\\
  And by the way, and you could swear on a Bible that you are doing
  exactly what is right. But and that's because our mind has the
  capacity to do that. In this case, I worked very hard to prevent my
  personal feelings and my personal desire from influencing a decision
  that was going to be an important decision and the most difficult
  decision I'd ever make.
\end{itemize}

mark leibovich

I think history is important to Mitt Romney. It's important to him for a
lot of reasons. I mean, part of it is his ego. I mean, people in the
U.S. Senate want to think that everything they do is actually relevant
to history. But I think when you're Mitt Romney --- when you've lived a
lot of history, when you've been the nominee of a Republican Party, when
you've run twice for president and lost, when you've held a number of
offices --- things like how you make a decision that will mark you
forever are important in the historical context. And you could argue,
you could be a cynic and say, oh, well, they're just full of themselves,
they care about how history will view them. Who does that? But I
actually think it was important and a very kind of formative part of the
process of coming to this decision for Mitt Romney.

michael barbaro

Was to just talk about it.

mark leibovich

Was just to talk about it.

\begin{itemize}
\item
  mark leibovich\\
  Right, you cannot send that off {[}INAUDIBLE{]}. Thank you.
\item
  mitt romney\\
  Thanks.
\item
  mark leibovich\\
  Appreciate it. O.K., all done.
\end{itemize}

michael barbaro

As you were leaving the office, he's just told you what he's going to
do. What are you thinking?

mark leibovich

Well, I mean, part of it is just pure, straight-ahead, opportunistic
reporter think, which is God, I hope they don't call me and say he
changed his mind. Because, you know, he's going to go on the floor in a
few hours and shock the world. I mean, not to put too fine a point on
it. But this is a very momentous decision that would be a major headline
at the end of a process that people had assumed was over, right? I mean,
it wasn't a major twist. But it was certainly a twist in something that
would be remembered here.

michael barbaro

But beyond your own journalistic ---

mark leibovich

Beyond my own ---

michael barbaro

--- self-absorption.

mark leibovich

--- self-absorption. My thought was, I hope he knows what he is in for.

{[}music{]}

michael barbaro

We'll be right back.

So Mark, what happens after your interview with Romney on Wednesday
morning?

mark leibovich

O.K. So Wednesday morning becomes Wednesday afternoon. It was probably
about 12:20 p.m. in Washington. I walked out of the office. I headed
back to the New York Times, Washington bureau. And I knew he was
scheduled to speak at 2 o'clock. And he took the mic.

\begin{itemize}
\tightlist
\item
  archived recording (mitt romney)\\
  Thank you, Mr. President. The Constitution is at the foundation of our
  Republic's success.
\end{itemize}

mark leibovich

And of course, you want to actually be watching this because, one, you
don't know if what you just learned is going to hold, whether he changed
his mind or not.

\begin{itemize}
\tightlist
\item
  archived recording (mitt romney)\\
  The allegations made in the articles of impeachment are very serious.
\end{itemize}

mark leibovich

But the other thing is, how does this look and feel when he's actually
delivering it to the world? And one of things I was struck by is that he
looked really nervous. He looked a lot more nervous on the floor than he
did with me.

\begin{itemize}
\tightlist
\item
  archived recording (mitt romney)\\
  I take an oath before God as enormously consequential.
\end{itemize}

mark leibovich

And he got emotional at a couple of points.

\begin{itemize}
\tightlist
\item
  archived recording (mitt romney)\\
  I knew from the outset that being tasked with judging the president,
  the leader of my own party, would be the most difficult decision I
  have ever faced. I was not wrong. The people will judge us for how
  well and faithfully we fulfill our duty.
\end{itemize}

mark leibovich

And it took him a while to get through this.

michael barbaro

Yeah, I was watching it. He was flipping the pages ---

mark leibovich

Absolutely, yeah.

michael barbaro

--- and speaking with all sorts of pregnant pauses.

\begin{itemize}
\tightlist
\item
  archived recording (mitt romney)\\
  The grave question the Constitution tasked senators to answer is
  whether the president committed an act so extreme and egregious that
  it rises to the level of a high crime and misdemeanor. Yes, he did.
\end{itemize}

mark leibovich

Absolutely, yeah. And I don't think he was doing that for any kind of
dramatic reason. I think it was just a genuinely hard speech to get
through. And at the end of the speech, Mitt Romney invoked his children
and his grandchildren.

\begin{itemize}
\tightlist
\item
  archived recording (mitt romney)\\
  With my vote, I will tell my children and their children that I did my
  duty to the best of my ability.
\end{itemize}

mark leibovich

This is something he does fairly regularly. But it also, you know he's
playing for keeps here.

\begin{itemize}
\tightlist
\item
  archived recording (mitt romney)\\
  I will only be one name among many, no more no less, to future
  generations of Americans who look at the record of this trial. We are
  all footnotes at best in the annals of history. But in the most
  powerful nation on earth, the nation conceived in liberty and justice,
  that distinction is enough for any citizen.
\end{itemize}

mark leibovich

But I think maybe there's some false modesty at work here too. I mean,
he's not a footnote. He is a dissenting voice. And the Republican Party
has not had many of those at all through this process.

\begin{itemize}
\tightlist
\item
  mitt romney\\
  Thank you, Mr. President. I yield the floor.
\end{itemize}

mark leibovich

And then he walked off into history.

michael barbaro

So, Mark, after Romney's speech, the full Senate formally reconvenes.

\begin{itemize}
\item
  archived recording (john roberts)\\
  The majority leader is recognized.
\item
  archived recording (mitch mcconnell)\\
  Mr. Chief Justice, the Senate is not ready to vote on the articles of
  impeachment.
\end{itemize}

michael barbaro

And becomes a court of impeachment.

\begin{itemize}
\tightlist
\item
  archived recording (john roberts)\\
  Each senator, when his or her name is called, will stand in his or her
  place and vote ``guilty'' or ``not guilty,'' as required by rule 23 of
  the Senate rules on impeachment.
\end{itemize}

michael barbaro

So what happens?

mark leibovich

Chief Justice John Roberts gives final instructions to the jury, or the
U.S. Senate. The formal articles of impeachment are read aloud.

\begin{itemize}
\tightlist
\item
  archived recording (john roberts)\\
  The question is on the first article of impeachment. Senators, how say
  you? Is the respondent Donald John Trump guilty or not guilty?
\end{itemize}

mark leibovich

And he calls a vote.

\begin{itemize}
\item
  archived recording (john roberts)\\
  A roll call vote is required. The clerk will call the roll.
\item
  archived recording (clerk)\\
  Mr. Alexander.
\item
  archived recording (lamar alexander)\\
  Not guilty.
\item
  archived recording (clerk)\\
  Mr. Alexander, not guilty. Miss Baldwin.
\item
  archived recording (tammy baldwin)\\
  Guilty.
\item
  archived recording (clerk)\\
  Miss Baldwin, guilty.
\end{itemize}

mark leibovich

On the first article of impeachment, presidential abuse of power ---

\begin{itemize}
\item
  archived recording (clerk)\\
  Mr. Romney.
\item
  archived recording (mitt romney)\\
  Guilty.
\item
  archived recording\\
  Mr. Romney, guilty.
\end{itemize}

mark leibovich

Mitt Romney votes guilty. He votes to convict.

\begin{itemize}
\tightlist
\item
  archived recording (john roberts)\\
  Two-thirds of the senators present not having pronounced him guilty.
  The Senate ajudges that the respondent, Donald John Trump, president
  of the United States, is not guilty as charged in the first article of
  impeachment.
\end{itemize}

mark leibovich

On the first article of impeachment, abuse of presidential power, the
president was acquitted by a final count of 52 noes, 48 yeses.

\begin{itemize}
\tightlist
\item
  archived recording (john roberts)\\
  Two-thirds of the senators present not having pronounced him guilty.
  The Senate ajudges that respondent Donald John Trump, president of the
  United States, is not guilty as charged in the second article of
  impeachment.
\end{itemize}

mark leibovich

On the second article of impeachment, which is obstruction of Congress,
Mitt Romney voted to acquit the president. The president was acquitted
by 53 noes and 47 yeses.

\begin{itemize}
\tightlist
\item
  archived recording (john roberts)\\
  Without objection, the motion is agreed to. The Senate sitting as a
  court of impeachment stands adjourned. Sine die.
\end{itemize}

mark leibovich

And in the end, Mitt Romney was the only U.S. senator, and the first
senator in U.S. history, to vote to convict a president of his own party
of an impeachable offense.

michael barbaro

You could have hinted at this, Mark. But there is something really
intriguing about choosing this moment for Mitt Romney to take a stand.
His career --- I covered it very closely, covered 2012 campaign --- is
littered with examples of moments where it seemed he wanted to be on
both sides of an issue. Or he evolved in ways that felt opportunistic.
Yet, at this moment, he becomes a senator of conscience. And he's not
malleable. But it's a moment where his vote to convict the president on
one of two counts has no impact whatsoever on the process. And when you
think back to people like John McCain, as Senator Amy Klobuchar did, he
stood for conscience at moments that had huge consequence.

mark leibovich

The decisive vote in the Affordable Care Act.

michael barbaro

Was the deciding vote on Affordable Care Act, for example. In this case,
Romney is the lone dissenting voice in a case that he can have no
influence over. So what do you make of that?

mark leibovich

Here's why that's important. One, Donald Trump has craved some kind of
way to say this is just a partisan witch hunt. Every Republican voted to
support me. This denies him that opportunity. The other part in the
context of Mitt Romney's career is, again, as you mentioned, this is not
something Mitt Romney has traditionally done. Now you could argue the
counter-factual. If he was up for re-election in Utah next year, would
he vote differently? If he was thinking about running for president and
going for the Republican nomination in 2020, would he vote differently?
I think at this point, he has lived a long career. He has had a long,
long life. And he would say at this point that he is answering to
different forces.

michael barbaro

Regardless of whether it changes any of the dynamics of this Congress
and the Republican Party and this president?

mark leibovich

Yeah, I don't think it will change any dynamics at all, except that Mitt
Romney's life is going to get a lot more uncomfortable for reasons that
I think he can handle, given how he weighed this decision.

michael barbaro

Well to that point, what has been the reaction for Romney in the hours
since he went on the floor, gave that speech and then cast a vote to
convict the president?

mark leibovich

I would say quite unpleasant.

Everything from the president of the United States's son Don Jr. calling
for his expulsion from the Senate. His own niece Romney McDaniel, the
chair of the Republican National Committee, publicly rebuking him
basically on Twitter.

michael barbaro

His own niece?

mark leibovich

His own niece, yes, calls for recall elections in Utah. Things like
that. Now this is a window into the kinds of things that are in store
for someone who dissents from President Trump.

michael barbaro

And maybe what his colleagues in the Senate deliberately avoided by not
---

mark leibovich

Absolutely.

michael barbaro

--- doing what he did.

mark leibovich

It is a faith that they have voted to avoid. I mean, it's obviously ---
there are a lot of things at work when you decide to make a vote like
this. But the noise is an absolutely undeniable part of the experience
of voting against the interests of the person who leads your party.
Donald Trump.

michael barbaro

Thank you, Mark.

mark leibovich

Thank you, Michael.

michael barbaro

On Tuesday night, President Trump himself began attacking Mitt Romney on
Twitter, promoting a video that calls Romney slippery and stealthy, and
without any evidence, claims that Romney is, quote, ``a secret asset of
the Democratic Party.''

We'll be right back.

Here's what else you need to know today. On Wednesday, Democratic
officials in Iowa released more results from Monday's caucuses, which
left the position of the candidates unchanged and the final outcome of
the vote uncertain. With 97 percent of the results in, Pete Buttigieg
maintained a narrow lead over Bernie Sanders that verged on a tie.
Meanwhile, in New Hampshire ---

\begin{itemize}
\tightlist
\item
  archived recording (joe biden)\\
  Donald Trump is desperate to pin the socialist label of socialist,
  socialist, socialist on our party. We can't let him do that. But if
  Senator Sanders is a nominee for the party, every Democrat will have
  to carry the label Senator Sanders has chosen for himself.
\end{itemize}

michael barbaro

Former Vice President Joe Biden, who stands at a distant fourth place in
Iowa, attacked both Sanders and Buttigieg as flawed candidates for the
Democratic nomination.

\begin{itemize}
\tightlist
\item
  archived recording (joe biden)\\
  I have great respect for Mayor Pete and his service of this nation.
  But I do believe it's a risk, to be just straight up with you, for
  this party to nominate someone who's never held a office higher than
  mayor of a town of 100,000 people in Indiana. I do believe it's a
  risk.
\end{itemize}

michael barbaro

That's it for ``The Daily.'' I'm Michael Barbaro. See you tomorrow.

\includegraphics{https://static01.nyt.com/images/2020/02/05/us/politics/05-video-romney/05-video-romney-videoSixteenByNine3000-v2.jpg}

The president's Republican allies excoriated Democrats for a proceeding
they said had damaged the country and its institutions in the name of
saving them.

``We will reject this incoherent case that comes nowhere near justifying
the first presidential removal in history,'' said Senator Mitch
McConnell of Kentucky, the majority leader.

Yet at a news conference after the vote, Mr. McConnell declined several
times to answer reporters who asked whether he considered Mr. Trump's
actions appropriate.

``This decision has been made,'' Mr. McConnell said curtly. ``As far as
I'm concerned, it's in the rearview mirror.''

As expected, the tally in favor of conviction on each article fell far
below the 67-vote threshold necessary for removal. The first charge was
abuse of power, accusing Mr. Trump of a scheme to use the levers of
government to coerce Ukraine to do his political bidding. It did not
even garner a majority vote, failing 48 to 52, with Mr. Romney voting
with the Democrats. The second article, charging Mr. Trump with
obstruction of Congress for an across-the-board blockade of House
subpoenas and oversight requests, failed 47 to 53, strictly on party
lines.

Like this one, the trials of Presidents Andrew Johnson and Bill Clinton
also ended in acquittal --- a reflection of the Constitution's high
burden for removing a chief executive.

But in a stinging rebuke of the country's leader aimed at history, Mr.
Romney, the 2012 Republican presidential nominee, said that Mr. Trump's
pressure campaign on Ukraine was ``the most abusive and destructive
violation of one's oath of office that I can imagine.'' Though he voted
against the second article, Mr. Romney became emotional on the Senate
floor in the hours before the verdict on Wednesday as he described why
he deemed Mr. Trump guilty of abuse of power, calling it a matter of
conscience. He was the first senator ever to vote to remove a president
of his own party.

\includegraphics{https://static01.nyt.com/images/2020/02/05/us/politics/05dc-impeach3-sub2/merlin_168432996_e4e87896-0bf4-4e7a-8fd7-a777d286af65-articleLarge.jpg?quality=75\&auto=webp\&disable=upscale}

``I am sure to hear abuse from the president and his supporters,'' Mr.
Romney said. ``Does anyone seriously believe I would consent to these
consequences other than from an inescapable conviction that my oath
before God demanded it of me?''

Mr. Romney's defection, which he announced a couple of hours before the
final vote, was a stark reflection of the sweeping transformation of the
Republican Party over the past eight years into one that is now
dominated almost entirely by Mr. Trump. And it deprived the president of
the monolithic Republican support he had eagerly anticipated.

Still, the White House declared victory.

``Today, the sham impeachment attempt concocted by Democrats ended in
the full vindication and exoneration of President Donald J. Trump,''
said Stephanie Grisham, the White House press secretary. ``As we have
said all along, he is not guilty.''

Mr. Trump's re-election campaign moved quickly to capitalize on the
moment, distributing a fund-raising email declaring, ``Sorry haters, I'm
not going anywhere.''

Several Republican senators ultimately acknowledged the heart of the
House case --- that Mr. Trump undertook a concerted pressure campaign on
Ukraine to secure politically beneficial investigations into his rivals,
including former Vice President Joseph R. Biden Jr., using nearly \$400
million in military aid as leverage.

But most argued that the conduct was not sufficiently dangerous to
warrant the Senate removing a president from office for the first time
in history --- and certainly not with an election so near. Others
dismissed Democrats' arguments altogether, insisting their case was
merely an attempt to dress up hatred for Mr. Trump and his policies as a
constitutional case.

Image

``It is simply a matter of right and wrong,'' said Senator Doug Jones,
Democrat of Alabama.Credit...Erin Schaff/The New York Times

Senators Susan Collins of Maine and Lisa Murkowski of Alaska, two
Republican swing votes who have tilted against the president in the
past, both voted against conviction and removal. And two Democrats from
traditionally conservative-leaning states, Senators Joe Manchin III of
West Virginia and Kyrsten Sinema of Arizona, voted to convict Mr. Trump,
denying him the bipartisan acquittal he coveted.

Although the verdict was never in doubt, Democrats lobbied to expand the
scope of the Senate trial to include witnesses and documents that the
president refused to provide during the House inquiry, working to
pressure vulnerable Republicans facing challenging re-election contests,
like Ms. Collins, to join them or risk being portrayed as beholden to
Mr. Trump. All but two Republicans refused, making the trial the first
impeachment proceeding in American history to reach a verdict without
calling witnesses.

As they closed their case, the seven Democratic House managers who
prosecuted it warned that Mr. Trump would emerge only emboldened in his
monarchical tendencies and that those who appeased him would be judged
harshly by history. Republicans, they said, had chosen to leave the
president's future up to voters despite evidence that he had tried to
cheat in the election, and would continue to do so.

Seldom used in American history, impeachment is the Constitution's most
extreme mechanism for checking a corrupt or out of control officeholder.
In unsheathing it, Democrats took on political risk that could backfire
in November on their presidential nominee or their incumbents in
Congress, including moderates in conservative districts and states where
Mr. Trump is popular.

At least one Democrat, Senator Doug Jones of Alabama, glancingly
acknowledged that his vote to convict would most likely contribute to
his loss this fall in deeply conservative Alabama.

``There will be so many who will simply look at what I am doing today
and say it is a profile in courage,'' Mr. Jones said before the vote.
``It is not. It is simply a matter of right and wrong.''

For now, the impeachment of Mr. Trump appears to have evenly divided the
nation. Public opinion polls suggest that even though a growing number
of Americans agreed that the president most likely abused his office and
acted improperly, more than a slight majority never agreed that he
should be removed from office.

If Mr. Trump's standing among the public has been hurt by the trial, it
is not yet evident. To the contrary, the latest Gallup poll, released on
Tuesday, showed that 49 percent of Americans approved of his job
performance --- the highest figure since he took office three years ago.
The same survey showed that Republicans' image has improved markedly,
with 51 percent viewing them favorably compared with 43 percent in
September.

The possibility of impeachment has hung over Mr. Trump's presidency
virtually since it began, but Speaker Nancy Pelosi, Democrat of
California, initially resisted it. After Robert S. Mueller III, the
special counsel who investigated Russia's election interference in 2016
and possible collaboration with the Trump campaign, found 10 instances
of potential obstruction of justice by Mr. Trump, she said impeachment
was too divisive a remedy to pursue.

Image

Speaker Nancy Pelosi meeting with the House impeachment managers ahead
of the Senate vote on Wednesday in the Capitol.Credit...Erin Schaff/The
New York Times

Her calculations changed in September with the emergence of
\href{https://www.nytimes.com/interactive/2019/09/26/us/politics/whistle-blower-complaint.html}{an
anonymous C.I.A. whistle-blower} that accused the president of
marshaling the powers of government to press Ukraine to investigate Mr.
Biden and a theory that Democrats had colluded with Ukraine in the 2016
election.
\href{https://www.nytimes.com/2019/09/24/us/politics/democrats-impeachment-trump.html}{In
authorizing the impeachment inquiry}, Ms. Pelosi tasked the House
Intelligence Committee to investigate the scheme and build a case.

Mr. Trump issued
\href{https://www.nytimes.com/2019/10/08/us/politics/sondland-trump-ukraine-impeach.html}{a
blanket directive to all government agencies} not to comply with the
inquiry, robbing investigators of key witnesses and facts that could
have filled out their case, and ultimately giving rise to the
obstruction of Congress charge.

Still, more than a dozen administration officials came forward, offering
testimony in private and then in scintillating public hearings that
confirmed and expanded on the whistle-blower complaint. On Dec. 18,
\href{https://www.nytimes.com/2019/12/18/us/politics/trump-impeached.html}{the
House impeached Mr. Trump on both counts}.

To protect his Senate majority as much as the presidency, Mr. McConnell
promised a swift acquittal --- and he delivered it. It was just 20 days
from the time the articles of impeachment were first read on the Senate
floor to Wednesday's vote. By comparison, the Clinton trial in 1999
lasted five weeks, and in 1868, the Senate took the better part of three
months to try Johnson.

The final shift in defenses by all but the most conservative of Mr.
Trump's allies came last week, when
\href{https://www.nytimes.com/2020/01/26/us/politics/trump-bolton-book-ukraine.html}{The
New York Times reported the first in a series of articles} revealing
that in August, Mr. Trump told John R. Bolton, the former national
security adviser, that he would not release the military aid for Ukraine
until the country helped out with investigations into Mr. Biden and
other Democrats.

Impeachment was seriously contemplated for a president only once in the
first two centuries of the American republic; it now has been so three
times since the 1970s, and two of the past four presidents have been
impeached.

Reporting was contributed by Emily Cochrane, Catie Edmondson, Patricia
Mazzei, Michael D. Shear and Sheryl Gay Stolberg.

\hypertarget{our-2020-election-guide}{%
\section{Our 2020 Election Guide}\label{our-2020-election-guide}}

Updated July 31, 2020

\begin{itemize}
\item
  \begin{center}\rule{0.5\linewidth}{\linethickness}\end{center}

  \hypertarget{the-latest}{%
  \subsection{The Latest}\label{the-latest}}

  \begin{itemize}
  \tightlist
  \item
    President Trump's assault on the Postal Service is intersecting with
    his attacks on mail-in voting.
    \href{https://www.nytimes.com/2020/07/31/us/politics/trump-usps-mail-delays.html?action=click\&pgtype=Article\&state=default\&region=BELOW_MAIN_CONTENT\&context=storylines_guide}{Voting
    rights groups say it is a recipe for disaster.}
  \end{itemize}
\item
  \begin{center}\rule{0.5\linewidth}{\linethickness}\end{center}

  \hypertarget{bidens-vp-search}{%
  \subsection{Biden's V.P. Search}\label{bidens-vp-search}}

  \begin{itemize}
  \tightlist
  \item
    \href{https://www.nytimes.com/article/biden-vice-president-2020.html?action=click\&pgtype=Article\&state=default\&region=BELOW_MAIN_CONTENT\&context=storylines_guide}{Here
    are 13 women} who have been under consideration to be Joe Biden's
    running mate, and why each might be chosen --- and might not be.
  \end{itemize}
\item
  \begin{center}\rule{0.5\linewidth}{\linethickness}\end{center}

  \hypertarget{keep-up-with-our-coverage}{%
  \subsection{Keep Up With Our
  Coverage}\label{keep-up-with-our-coverage}}

  \begin{itemize}
  \tightlist
  \item
    Get an
    \href{https://www.nytimes.com/newsletters/politics?action=click\&pgtype=Article\&state=default\&region=BELOW_MAIN_CONTENT\&context=storylines_guide}{email}
    recapping the day's news
  \end{itemize}

  \begin{itemize}
  \tightlist
  \item
    Download our mobile app on
    \href{https://apps.apple.com/us/app/nytimes/id284862083?ls=1\&mat_click_id=5c79ae7455014fd1bd66b5610c05b8f2-20191112-16948\&referrer=mat_click_id\%3D5c79ae7455014fd1bd66b5610c05b8f2-20191112-16948\%26link_click_id\%3D722930677036718082}{iOS}
    and
    \href{http://a.localytics.com/android?id=com.nytimes.android\&referrer=utm_source\%3Dother_nyt_mobile_web\%26utm_medium\%3DWeb\%2520page\%26utm_term\%3DGeneral\%2520Mobile\%2520Page\%26utm_campaign\%3DNYT\%2520Mobile\%2520General\%2520Page}{Android}
    and turn on Breaking News and Politics alerts
  \end{itemize}
\end{itemize}

Advertisement

\protect\hyperlink{after-bottom}{Continue reading the main story}

\hypertarget{site-index}{%
\subsection{Site Index}\label{site-index}}

\hypertarget{site-information-navigation}{%
\subsection{Site Information
Navigation}\label{site-information-navigation}}

\begin{itemize}
\tightlist
\item
  \href{https://help.nytimes.com/hc/en-us/articles/115014792127-Copyright-notice}{©~2020~The
  New York Times Company}
\end{itemize}

\begin{itemize}
\tightlist
\item
  \href{https://www.nytco.com/}{NYTCo}
\item
  \href{https://help.nytimes.com/hc/en-us/articles/115015385887-Contact-Us}{Contact
  Us}
\item
  \href{https://www.nytco.com/careers/}{Work with us}
\item
  \href{https://nytmediakit.com/}{Advertise}
\item
  \href{http://www.tbrandstudio.com/}{T Brand Studio}
\item
  \href{https://www.nytimes.com/privacy/cookie-policy\#how-do-i-manage-trackers}{Your
  Ad Choices}
\item
  \href{https://www.nytimes.com/privacy}{Privacy}
\item
  \href{https://help.nytimes.com/hc/en-us/articles/115014893428-Terms-of-service}{Terms
  of Service}
\item
  \href{https://help.nytimes.com/hc/en-us/articles/115014893968-Terms-of-sale}{Terms
  of Sale}
\item
  \href{https://spiderbites.nytimes.com}{Site Map}
\item
  \href{https://help.nytimes.com/hc/en-us}{Help}
\item
  \href{https://www.nytimes.com/subscription?campaignId=37WXW}{Subscriptions}
\end{itemize}
