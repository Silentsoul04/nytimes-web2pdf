Sections

SEARCH

\protect\hyperlink{site-content}{Skip to
content}\protect\hyperlink{site-index}{Skip to site index}

\href{https://www.nytimes.com/section/politics}{Politics}

\href{https://myaccount.nytimes.com/auth/login?response_type=cookie\&client_id=vi}{}

\href{https://www.nytimes.com/section/todayspaper}{Today's Paper}

\href{/section/politics}{Politics}\textbar{}Trump Hails Acquittal and
Lashes Out at His `Evil' and `Corrupt' Opponents

\url{https://nyti.ms/2tB92MT}

\begin{itemize}
\item
\item
\item
\item
\item
\item
\end{itemize}

\begin{itemize}
\item
  \href{https://www.nytimes.com/2020/07/31/us/elections/biden-vs-trump.html?action=click\&pgtype=Article\&state=default\&region=TOP_BANNER\&context=storylines_menu}{Election
  Updates}
\item
  \href{https://www.nytimes.com/article/biden-vice-president-2020.html?action=click\&pgtype=Article\&state=default\&region=TOP_BANNER\&context=storylines_menu}{Biden's
  V.P. Search}
\item
  \href{https://www.nytimes.com/interactive/2020/07/24/us/politics/trump-biden-campaign-donors.html?action=click\&pgtype=Article\&state=default\&region=TOP_BANNER\&context=storylines_menu}{Map
  of Donations}
\item
  \href{https://www.nytimes.com/interactive/2020/us/elections/delegate-count-primary-results.html?action=click\&pgtype=Article\&state=default\&region=TOP_BANNER\&context=storylines_menu}{Delegate
  Count}
\item
  \href{https://www.nytimes.com/interactive/2019/us/politics/2020-presidential-candidates.html?action=click\&pgtype=Article\&state=default\&region=TOP_BANNER\&context=storylines_menu}{The
  Candidates}
\item
  \href{https://www.nytimes.com/newsletters/politics?action=click\&pgtype=Article\&state=default\&region=TOP_BANNER\&context=storylines_menu}{Politics
  Newsletter}
\end{itemize}

Advertisement

\protect\hyperlink{after-top}{Continue reading the main story}

Supported by

\protect\hyperlink{after-sponsor}{Continue reading the main story}

\hypertarget{trump-hails-acquittal-and-lashes-out-at-his-evil-and-corrupt-opponents}{%
\section{Trump Hails Acquittal and Lashes Out at His `Evil' and
`Corrupt'
Opponents}\label{trump-hails-acquittal-and-lashes-out-at-his-evil-and-corrupt-opponents}}

As President Trump renewed his attacks on Speaker Nancy Pelosi and
Democrats, his press secretary threatened payback against them.

\includegraphics{https://static01.nyt.com/images/2020/02/06/us/politics/06dc-trumpspeech-sub/06dc-trumpspeech-sub-videoSixteenByNine3000-v2.jpg}

\href{https://www.nytimes.com/by/peter-baker}{\includegraphics{https://static01.nyt.com/images/2018/06/13/multimedia/peter-baker/peter-baker-thumbLarge-v2.png}}

By \href{https://www.nytimes.com/by/peter-baker}{Peter Baker}

\begin{itemize}
\item
  Published Feb. 6, 2020Updated Feb. 14, 2020
\item
  \begin{itemize}
  \item
  \item
  \item
  \item
  \item
  \item
  \end{itemize}
\end{itemize}

WASHINGTON ---
\href{https://www.nytimes.com/2020/02/14/podcasts/the-daily/trump-acquittal.html?action=click\&module=Briefings\&pgtype=Homepage}{President
Trump} and his Republican allies focused on exacting payback against his
political opponents on Thursday after
\href{https://www.nytimes.com/2020/02/05/us/politics/trump-acquitted-impeachment.html}{his
acquittal in his Senate impeachment trial}, signaling that the conflict
that has consumed Washington for months may only escalate rather than
recede.

Choosing retaliation over reconciliation, Mr. Trump lashed out at
Democrats and the one Republican senator who voted for conviction. He
\href{https://www.nytimes.com/2020/02/06/us/politics/trump-prayer-impeachment.html}{turned
a prayer breakfast into a launching pad} for political attacks and then
staged a long, rambling venting session at the White House where he
denounced ``evil'' and ``crooked'' lawmakers and the ``top scum'' at the
F.B.I. for trying to take him down.

Mr. Trump's team indicated that his desire to turn the tables on his
foes may go beyond just tough language. The White House press secretary
declared that Democrats
``\href{https://www.foxnews.com/media/stephanie-grisham-dems-must-be-held-accountable-for-corrupt-impeachment}{should
pay for}'' impeaching the president, and the Trump administration worked
to
\href{https://www.nytimes.com/2020/02/06/us/politics/hunter-biden-investigation.html}{facilitate
a Senate Republican investigation} of Hunter Biden, the son of former
Vice President Joseph R. Biden Jr., the goal of Mr. Trump that was at
the heart of his impeachment.

``It was evil,'' Mr. Trump said of the investigations that led to his
Senate trial in an hourlong stream-of-consciousness address to
supporters in the East Room of the White House, tossing aside the more
calibrated text prepared by his staff. ``It was corrupt. It was dirty
cops. It was leakers and liars, and this should never ever happen to
another president, ever. I don't know that other presidents would have
been able to take it.''

Democrats showed little sign of backing down either. House Democrats
have already said they are likely to resume their investigation into Mr.
Trump's efforts to pressure Ukraine to incriminate the Bidens, while a
Senate Democrat on Thursday called for an inquiry into whether the
administration covered up related information by improperly classifying
it.

Speaker Nancy Pelosi, who sat just feet from Mr. Trump as he questioned
her faith during the annual National Prayer Breakfast, later pushed back
against his implication that she was disingenuous for saying she prayed
for him. Some of his remarks, she said, were ``particularly without
class'' and ``so inappropriate at a prayer breakfast.''

She also suggested that Mr. Trump appeared to be on medication during
his State of the Union address on Tuesday. ``He looked to me like he was
a little sedated,'' she told reporters. ``Looked that way last year,
too.''

Mr. Trump's vituperative performance on Thursday was the diametrical
opposite of how President Bill Clinton responded to
\href{https://www.washingtonpost.com/politics/clinton-impeachment/senate-acquits-president-clinton/}{his
own acquittal} after a Senate impeachment trial in 1999. On the day he
was cleared of charges of perjury and obstruction of justice,
\href{https://www.nytimes.com/1999/02/13/us/the-president-s-acquittal-clinton-statement.html}{Mr.
Clinton appeared alone} in the Rose Garden, said he was ``profoundly
sorry'' and called for ``reconciliation and renewal.''

His Republican opponents at the time were just as eager to move on,
feeling burned after losing seats in midterm elections and watching not
one but two of their House speakers step down. One important difference
is that Mr. Clinton was in his second term, while Mr. Trump is seeking
re-election in a campaign framed in part by the impeachment debate.

For Mr. Trump, the Senate's rejection of the two articles of impeachment
against him on Wednesday was marred by the fact that Mitt Romney,
Republican of Utah, was the
\href{https://www.nytimes.com/2020/02/05/us/politics/romney-trump-impeachment.html}{only
senator to break rank}, joining every Democrat in
\href{https://www.nytimes.com/2020/02/06/podcasts/the-daily/mitt-romney.html}{voting
to convict Mr. Trump} for
\href{https://www.nytimes.com/interactive/2020/01/22/us/politics/impeachment-articles-arguments.html}{abuse
of power}.

\hypertarget{latest-updates-2020-election}{%
\section{\texorpdfstring{\href{https://www.nytimes.com/2020/07/31/us/elections/biden-vs-trump.html?action=click\&pgtype=Article\&state=default\&region=MAIN_CONTENT_1\&context=storylines_live_updates}{Latest
Updates: 2020
Election}}{Latest Updates: 2020 Election}}\label{latest-updates-2020-election}}

Updated 2020-08-01T01:26:45.732Z

\begin{itemize}
\tightlist
\item
  \href{https://www.nytimes.com/2020/07/31/us/elections/biden-vs-trump.html?action=click\&pgtype=Article\&state=default\&region=MAIN_CONTENT_1\&context=storylines_live_updates\#link-29fdff45}{Kamala
  Harris, a top vice-presidential contender, confronts double
  standards.}
\item
  \href{https://www.nytimes.com/2020/07/31/us/elections/biden-vs-trump.html?action=click\&pgtype=Article\&state=default\&region=MAIN_CONTENT_1\&context=storylines_live_updates\#link-13ec3d9c}{Karen
  Bass and Susan Rice are rising on Biden's vice-presidential
  shortlist.}
\item
  \href{https://www.nytimes.com/2020/07/31/us/elections/biden-vs-trump.html?action=click\&pgtype=Article\&state=default\&region=MAIN_CONTENT_1\&context=storylines_live_updates\#link-49e9a016}{Trump
  says Russian bounties to kill U.S. troops `never took place.'}
\end{itemize}

\href{https://www.nytimes.com/2020/07/31/us/elections/biden-vs-trump.html?action=click\&pgtype=Article\&state=default\&region=MAIN_CONTENT_1\&context=storylines_live_updates}{See
more updates}

Angry at Mr. Romney's defection, Mr. Trump waited a day to appear in
person with supporters in the East Room in a ceremony that veered
between celebration and confrontation.

Mr. Trump held up a copy of The Washington Post to show its banner
headline, ``Trump Acquitted,'' then reviewed the long litany of
investigations against him over the last three years, dismissing them as
partisan efforts to stop him from serving as president.

``We first went through Russia, Russia, Russia,'' he said, mocking the
investigations into the Kremlin's interference in the 2016 presidential
election on his behalf and ties between his campaign and Moscow. ``It
was all bullshit,'' he said, the first time he or any president has been
known to use that profanity in a formal event on camera in the East
Room, according to \href{https://factba.se/}{Factba.se}, a research
service.

\includegraphics{https://static01.nyt.com/images/2017/01/29/podcasts/the-daily-album-art/the-daily-album-art-articleInline-v2.jpg?quality=75\&auto=webp\&disable=upscale}

\hypertarget{listen-to-the-daily-the-post-acquittal-presidency}{%
\subsubsection{Listen to `The Daily': The Post-Acquittal
Presidency}\label{listen-to-the-daily-the-post-acquittal-presidency}}

How has President Trump's acquittal in the Senate impeachment trial
emboldened him in the Oval Office?

transcript

Back to The Daily

bars

0:00/28:08

-28:08

transcript

\hypertarget{listen-to-the-daily-the-post-acquittal-presidency-1}{%
\subsection{Listen to `The Daily': The Post-Acquittal
Presidency}\label{listen-to-the-daily-the-post-acquittal-presidency-1}}

\hypertarget{hosted-by-michael-barbaro-produced-by-eric-krupke-adizah-eghan-and-jonathan-wolfe-and-edited-by-lisa-chow}{%
\subsubsection{Hosted by Michael Barbaro, produced by Eric Krupke,
Adizah Eghan, and Jonathan Wolfe, and edited by Lisa
Chow}\label{hosted-by-michael-barbaro-produced-by-eric-krupke-adizah-eghan-and-jonathan-wolfe-and-edited-by-lisa-chow}}

\hypertarget{how-has-president-trumps-acquittal-in-the-senate-impeachment-trial-emboldened-him-in-the-oval-office}{%
\paragraph{How has President Trump's acquittal in the Senate impeachment
trial emboldened him in the Oval
Office?}\label{how-has-president-trumps-acquittal-in-the-senate-impeachment-trial-emboldened-him-in-the-oval-office}}

\begin{itemize}
\item
  {[}music{]}
\item
  michael barbaro\\
  From The New York Times, I'm Michael Barbaro. This is ``The Daily.''

  Today: President Trump has undertaken a campaign of retribution
  against those who crossed him during the impeachment inquiry and
  favors for those who have tried to protect him. Peter Baker on the
  post-acquittal presidency.

  It's Friday, February 14.
\item
  archived recording\\
  {[}HORNS{]} Ladies and gentlemen, the President of the United States.
\end{itemize}

michael barbaro

Peter, I want to begin with retribution. How does that start?

\begin{itemize}
\tightlist
\item
  archived recording (donald trump)\\
  Well, thank you very much, everybody. Wow.
\end{itemize}

peter baker

The day after his acquittal in the Senate, the president gathers people
in the East Room of the White House for an event. It's not quite a press
conference. It's not quite a speech. It's really kind of a mix, a mix of
a celebration of his acquittal but a venting session of his grievances.

\begin{itemize}
\tightlist
\item
  archived recording (donald trump)\\
  I want to start by thanking some of --- and I call them friends,
  because you develop friendships and relationships when you're in
  battle and war.
\end{itemize}

peter baker

And he wants to thank the people who stood behind him, names them in the
audience.

\begin{itemize}
\tightlist
\item
  archived recording (donald trump)\\
  Mitch McConnell, I want to tell you, you did a fantastic job.
  {[}APPLAUSE{]}
\end{itemize}

peter baker

Mitch McConnell, the Republican leader who did more than anybody to
secure his acquittal in the trial. And he mentions Jim Jordan ---

\begin{itemize}
\tightlist
\item
  archived recording (donald trump)\\
  When I first got to know Jim, I said, huh, he never wears a jacket.
  What the hell's going on? He's obviously very proud of his body.
  {[}LAUGHTER{]}
\end{itemize}

peter baker

--- and other members of the House, the Freedom Caucus, the conservative
Republicans who always stood by him in the most aggressive and assertive
and staunch way. And then, of course, he turns to his enemies.

The people he blames for his ordeal, the people he thinks have treated
him so unfairly, have plotted against him, been disloyal or what have
you. And he names ones that you would expect, of course.

\begin{itemize}
\tightlist
\item
  archived recording (donald trump)\\
  Nancy Pelosi is a horrible person.
\end{itemize}

peter baker

Nancy Pelosi, he says she's a horrible person.

\begin{itemize}
\tightlist
\item
  archived recording (donald trump)\\
  A corrupt politician named Adam Schiff made up my statement to the
  Ukrainian president. He brought it out of thin air --- just made it
  up. They say he's a screenwriter, a failed screenwriter.
\end{itemize}

peter baker

He names, of course, Adam Schiff, the lead House prosecutor.

\begin{itemize}
\tightlist
\item
  archived recording (donald trump)\\
  And then you have some that used religion as a crutch. They never used
  it before.
\end{itemize}

peter baker

He names Mitt Romney, the Republican, the only Republican senator to
vote for conviction.

\begin{itemize}
\tightlist
\item
  archived recording (donald trump)\\
  But, you know, it's a failed presidential candidate, so things can
  happen when you fail so badly running for president.
\end{itemize}

peter baker

These two now, of course, are really at odds. And you see the visceral
anger in the president in this moment.

And he mentions Colonel Alexander Vindman, a member of his own staff, a
detailee from the Pentagon working on Ukraine issues, and his twin
brother Yevgeny Vindman, who also works at the N.S.C. staff. He says it
almost in passing.

\begin{itemize}
\tightlist
\item
  archived recording (donald trump)\\
  Lieutenant Colonel Vindman and his twin brother, right?
\end{itemize}

peter baker

And he says it with such dripping disdain in his voice. You'd get the
sense immediately, of course, that this is somebody who's really angry
at the president, and he's got his attention.

michael barbaro

And remind us what puts Vindman in this list of enemies.

peter baker

Colonel Vindman was one of the members of the White House staff, the
National Security Council staff who were subpoenaed by the House to
testify in the impeachment inquiry. He didn't come forward voluntarily.
He was required to by law to give his testimony to the committee. And
during his testimony, he told about being on the famous July 25 call
between the president and President Zelensky of Ukraine when the
president asked him to investigate Joe Biden and the Democrats. And
Colonel Vindman told the committee that he thought that was
inappropriate, and he reported it to his superiors at the N.S.C. And for
that, he has been on the target list of President Trump and his allies
ever since. Painted as disloyal, painted as even treasonous to the
country. His patriotism questioned even though he's a decorated veteran
of the Iraq War, injured in battle, and really, a kind of a symbol to
both sides of sort of where this fight has evolved.

\begin{itemize}
\tightlist
\item
  archived recording (donald trump)\\
  Our country is just respected again, and it's an honor to be with the
  people in this room. Thank you very much, everybody. Thank you. Thank
  you very much. Thank you.
\end{itemize}

peter baker

And so he comes to the end of this sort of rambling, meandering talk
that goes on for an hour and two minutes. And you get the sense that
this is not the end and that there's more to come.

\begin{itemize}
\tightlist
\item
  archived recording\\
  Well, President Trump has begun his revenge in the wake of his
  impeachment trial.
\end{itemize}

peter baker

Colonel Vindman, the same witness he had just talked about so
dismissively at the East Room event finds himself escorted out of the
White House by security guards and told his services are no longer
needed ---

michael barbaro

Wow.

peter baker

--- exiled back to the Pentagon from which he came. Not just him ---

\begin{itemize}
\tightlist
\item
  archived recording\\
  Escorted out of his job and off the White House grounds, as was his
  twin brother, who was also assigned to the N.S.C.
\end{itemize}

peter baker

His brother Yevgeny Vindman --- who didn't do anything, had nothing to
do with the impeachment hearings at all, except to show up and sit
behind his brother just as a matter of family support --- also dismissed
from his post at the National Security Council, marched out at the same
time by security and sent back to the Pentagon.

\begin{itemize}
\tightlist
\item
  archived recording\\
  Today, Vindman's lawyer issued a statement saying, quote, ``The truth
  has cost him his job, his career and his privacy.''
\end{itemize}

peter baker

You can understand why a president might not want somebody on his staff
who had testified an impeachment hearing against him. But it was handled
in a way that was meant to maximize the public message, right? I'll tell
you what I mean by that. The N.S.C. is currently undergoing a
downsizing. And in fact, the plan was to move Colonel Vindman out as
part of that, or at least to use that as the cover to to say, it's not
about reprisal. It's not about his role in impeachment. It's just part
of this overall restructuring. And that's frankly how other presidents
might have handled a situation like that.

michael barbaro

Come up with a rationale.

peter baker

Come up with a rationale, come up with a public face-saving kind of
storyline, a narrative, at least, that even though people would see
through it, would at least have the veneer of looking professional
rather than vindictive. That was not what the president wanted. He made
sure they did this separate from that reorganization. They did not
explain it as part of that reorganization. And they did not deny when we
called them that day that this was what it looked like, which was, of
course, an act of retribution.

michael barbaro

OK, so what happens next?

peter baker

Well, we thought that was the story for the day, these two brothers
being marched out of there.

michael barbaro

Right.

peter baker

And then we discover as the evening arrives that it's not over.

\begin{itemize}
\tightlist
\item
  archived recording\\
  Now we're getting word that the U.S. ambassador to the European Union,
  Gordon Sondland, he is out as well.
\end{itemize}

peter baker

Gordon Sondland, you may remember him. He was the ambassador to the
European Union, who had been deeply involved in the Ukraine pressure
campaign, on the phone with the president and required to testify,
became a key witness in the House hearings. He said that they were
operating on the order of the president himself. He said that it was
clearly a quid pro quo, and he said that everyone was in the loop.
Suddenly, it turns out he's out as well. Now, as with Vindman, there was
a way to do this that would have minimized the public kerfuffle. Gordon
Sondland actually was ready to leave. He had told his superiors at the
State Department that he was ready to step down on his own. And he got
word that Friday you have to resign today, they told him. But he says,
no. I don't want to resign on the same day that you're pushing out these
Vindmans as if I'm part of some sort of purge.

michael barbaro

Wow.

peter baker

If you want me today, you're going to have to fire me. And they called
back and said, OK, you're fired.

michael barbaro

So at this point, it's clear that this is a vindictive purge of anyone
who did anything that put the president in a negative light during the
impeachment process. And what is the reaction to that, that very clear
and deliberate message from the president inside Washington?

peter baker

Certainly among Democrats, even among a few Republicans who say what's
the message you're sending? If you respond to a subpoena, as ordered by
the law, and you give your testimony, you shouldn't be punished for
doing that. The president's view is, why should I have people I can't
trust working for me? It's my right as the president to have a staff
that serves my interests that I believe is loyal. And he's made clear
that loyalty is a number one when it comes to this president. There's no
other quality that matters more to him.

michael barbaro

And, Peter, as somebody who's covered many White Houses, is he right
about that? Is it ultimately a presidential prerogative to decide if
someone testified against you, that, you know, you no longer want them
around, you don't want them in those jobs anymore, especially
presidential appointments?

peter baker

It's a good question, right? Because it does feel like it would be
untenable to have testified and provided damaging testimony against the
president, and then come to work every day afterwards. You would think,
in fact, you might not want to necessarily do that. But the question
isn't what's the right place then for that person to work. The question
is what the message the president is trying to send by what he's doing,
right?

michael barbaro

Right.

peter baker

This president has made a point of making sure everybody knows these
people are out, and they're out because of him and because he will not
tolerate disloyalty.

michael barbaro

OK, so that is the campaign of retribution so far, post acquittal, which
brings us to the campaign of protection for the president's allies.

peter baker

Right. It's not enough just to go after his perceived enemies. Now it's
time to do something to protect his friends. And for him, this is going
to start with a colorful character and longtime friend and adviser named
Roger Stone, who's about to go to prison.

michael barbaro

We'll be right back.

So, Peter, before we get to how the president is trying to protect Roger
Stone, remind us who Roger Stone is.

peter baker

Roger Stone has been in American politics going back decades.

He is somebody who calls himself a dirty trickster.

\begin{itemize}
\tightlist
\item
  archived recording (roger stone)\\
  I'm certainly guilty of bluffing and posturing and punking the
  Democrats. Unless they pass some law against {[}BLEEP{]} and I missed
  it, I'm engaging in tradecraft. It's politics.
\end{itemize}

peter baker

He's a self-proclaimed fan of Richard Nixon. Even to this day, he has a
Richard Nixon tattoo.

michael barbaro

Right.

peter baker

He's somebody who's involved early on in some of the Reagan and Dole
campaigns, but over the years kind of drifted off into the side, really
kind of more of a fringe character, a conspiracy theorist, a
provocateur.

\begin{itemize}
\item
  archived recording 1\\
  In 1980, Stone began a lobbying firm with Paul Manafort that
  unapologetically catered to human rights abusers.
\item
  archived recording 2\\
  He has these maxims on how he conducts his political strategy. One of
  his rules is never turn down an opportunity to have sex or be on
  television.
\item
  archived recording 3\\
  We've seen a lot of colorful characters in the world of political
  consulting, none more colorful than Roger Stone. And that is the most
  charitable adjective you can apply to the single weirdest man possibly
  in the history of political consulting.
\end{itemize}

peter baker

He'd been friends for years with Donald Trump. And like Roger Stone,
Trump comes from the outside, right? He was not part of the Republican
establishment. But suddenly, he's powering forward toward a presidential
bid. And he brings with him people like Roger Stone, who had not been in
the center of American politics now for years.

michael barbaro

Right. And my recollection is that it's during that campaign that Roger
Stone gets into very significant trouble.

peter baker

Right. He becomes wrapped up in the whole story about the Russian
hacking of the Democratic emails.

\begin{itemize}
\tightlist
\item
  archived recording\\
  Hillary Clinton's campaign dealing with more email problems. The email
  account of campaign chairman John Podesta was hacked and many of the
  emails released.
\end{itemize}

peter baker

Things he said gave the impression that he might have known about it in
advance.

michael barbaro

Right.

\begin{itemize}
\item
  archived recording\\
  So were you surprised when John Podesta's emails came out, as you
  seemed to predict ahead of time?
\item
  archived recording (roger stone)\\
  I was interested, like the rest of the country.
\item
  archived recording\\
  Were you surprised?
\item
  archived recording (roger stone)\\
  No, I wouldn't say that I was surprised.
\end{itemize}

peter baker

And that puts him right in the heart of this. Is he a link between the
Trump campaign and Russia through perhaps WikiLeaks, which is the cutout
that the Russians used to get these emails out. And so, once the
president wins and comes into office, his friend Roger Stone finds
himself under investigation for what he knew and when he knew it. And
then Congress jumps in. They call Stone to testify at the House
Intelligence Committee. And this is where he really gets into trouble.

\begin{itemize}
\tightlist
\item
  archived recording (roger stone)\\
  We had a very frank exchange. I answered all of the questions. I made
  the case that the accusation that I knew about John Podesta's email
  hack in advance was false, that I knew about the content and source of
  the WikiLeaks disclosures regarding Hillary Clinton was false.
\end{itemize}

peter baker

He starts telling things that are demonstrably not true. And he
ultimately ends up getting charged with lying to Congress. He also tries
to get an associate of his to not tell the truth, threatens him even,
threatens to kill his dog.

michael barbaro

Whoa.

peter baker

And he was put on trial. And last fall Roger Stone was convicted of
seven crimes, seven felonies, including lying to Congress and witness
intimidation.

michael barbaro

And these are conditions on very serious charges of obstructing a
congressional investigation into Russian meddling in the 2016 election.
That's right. I remember thinking when that happened, like, whoa. This
is the big leagues for Roger Stone.

peter baker

Exactly. And the question is, why is he lying? Why is he obstructing? Is
he trying to protect the president? This is how this all fits together,
right? This goes back to the whole Russian interference. This goes back
to the Mueller probe. This goes back to the things that have dominated
this presidency for three years and frustrated this president for three
years. So he sees Roger Stone's conviction as an illegitimate shot at
him, at himself, the president. A way of trying to take him down because
they couldn't take him down any other way.

michael barbaro

OK. So Peter, how does the president try to protect Stone after this
conviction?

peter baker

So even as he's in the middle of this campaign of retribution against
the Vindman brothers and Gordon Sondland, he is increasingly aware that
the sentencing for Roger Stone is coming up. And then, when Monday comes
around and the prosecutors present their recommendation for a sentence
to the court, the prosecutors ask for seven to nine years behind bars.
That's the normal sentence that would be required under the sentencing
guidelines passed by Congress for crimes of the type that Roger Stone
was convicted of. So they didn't go outside of those guidelines. They
simply said we want to sentence him to what the guidelines say. That
doesn't mean the judge would go along with it, but that was their
recommendation. Well, that set the president off.

\begin{itemize}
\tightlist
\item
  archived recording\\
  The president expressed his outrage on Twitter, calling it a very
  unfair situation, adding, ``Cannot allow this miscarriage of
  justice!''
\end{itemize}

peter baker

In the middle of the night, he starts sending out tweets, angry tweets.
How can this happen? Nine years, this is outrageous. And they're going
after him. How come they don't go after my enemies but they go after
him? And that just sort of sets the town ablaze.

\begin{itemize}
\tightlist
\item
  archived recording\\
  Controversy in the nation's capital now over a sentencing
  recommendation for President Trump's longtime friend Roger Stone.
\end{itemize}

peter baker

Here's a president weighing in directly on a court case involving a
friend of his. This is something that we have not seen really since
Watergate. Presidents don't, especially publicly, weigh in on
prosecutions of people that they are personally connected to, at least
except in the venue of issuing pardons at some point, which they
sometimes do. So this has shocked a lot of people. But what really
shocked a lot of people in Washington was when they woke up a few hours
later on Tuesday and they saw not only these tweets, but they saw that
the attorney general of the United States, Bill Barr, had essentially
overruled the career prosecutors.

\begin{itemize}
\tightlist
\item
  archived recording\\
  Breaking news involving President Trump. A stunning reversal in the
  sentencing recommendation for Trump confidant Roger Stone.
\end{itemize}

peter baker

And said, no, we're not going to ask for a sentence this heavy. We're
going to ask for something lighter.

michael barbaro

So not seven to nine years, something less.

peter baker

Not seven to nine years, something less. It doesn't specify what, but
something below what the guidelines would normally call for. And so this
is causing a huge furor in the U.S. attorney's office in Washington.

\begin{itemize}
\tightlist
\item
  archived recording\\
  What is going on? President Trump knows how to get away with stuff
  when we're not watching.
\end{itemize}

peter baker

The four career prosecutors who worked on the Stone case, all four of
them, quit.

\begin{itemize}
\tightlist
\item
  archived recording\\
  We're following some truly stunning, breaking news, still developing
  by the minute this hour. Federal prosecutors in the Roger Stone
  criminal case have resigned this afternoon.
\end{itemize}

peter baker

One after the other. One, two, three, four, just like that.

\begin{itemize}
\tightlist
\item
  archived recording\\
  This does not happen. Prosecutors don't resign just days before they
  go to sentencing after a case that they've worked so hard on.
\end{itemize}

peter baker

One of them actually quits his job altogether, leaves the Justice
Department as a whole.

\begin{itemize}
\tightlist
\item
  archived recording\\
  In protest.
\end{itemize}

peter baker

Well, they don't say it, but that's the obvious conclusion. Yes, they're
protesting the overruling of their recommendation. And I think that they
felt like they had an ethical obligation. If they had told the court
this is the sentence we think is appropriate, and then suddenly a day
later the same department is coming and saying, no, we don't --- how is
that tenable for them to continue on that case?

michael barbaro

And, Peter, given what has just happened --- the firing of Vindman,
Sondland, Vindman's brother --- what is the reaction to this
intervention? Not just the retribution, but this protection?

peter baker

Well, in effect, the Democrats are saying we told you so, right?

\begin{itemize}
\tightlist
\item
  archived recording (chuck schumer)\\
  No serious person believes President Trump has learned any lesson. He
  doesn't learn any lessons. He does just what he wants, what suits his
  ego at the moment.
\end{itemize}

peter baker

Senator Schumer, the Democratic leader in the upper chamber goes to the
floor and gives a pretty passionate speech in which he says that the
natural consequence of acquitting the president on the Ukraine matter
means that he feels completely unleashed and empowered to do whatever he
thinks is right for his own political interests.

\begin{itemize}
\tightlist
\item
  archived recording (chuck schumer)\\
  We are witnessing a crisis in the rule of law in America, unlike one
  we have ever seen before. It's a crisis of President Trump's making,
  but it was enabled and emboldened by every Senate Republican.
\end{itemize}

peter baker

Even amongst some Republicans, you're seeing you know some discomfort,
particularly among moderate Republicans who tried to give the president
the benefit of the doubt by standing with him in the impeachment trial.
A couple of them had said even, well, maybe he'll have learned a lesson
from all of this and he'll be more measured, he'll be more restrained in
the future and that that would be a good thing. Well, what you're
hearing a lot of people saying is that doesn't seem to be the case. And
I think that the question going forward is going to be, is it just a
burst of energy and lashing out in the days after the acquittal, or is
this the beginning of a month's long recalibration of his
administration? What is he going to do going forward?

michael barbaro

Right. Is this the post-aquittal presidency, one in which enemies are
punished and allies are at all costs protected?

peter baker

Right, exactly. And that the instruments of government are to serve the
president's interests, not just the public's interests.

michael barbaro

Peter, what you have described here is what an old school political
terms might be called a strategy of carrots and sticks, but on steroids,
right? You protect those who have done right by you, and you punish
those who have somehow wronged you. And in the case of the president,
that ``steroided up'' strategy clearly worked when it came to
impeachment. And we talked to you. We talked to many of our colleagues
about the fact that there was genuine fear of crossing this president,
and that that influenced how the Senate voted in the impeachment trial.
So if this strategy is working --- and by all accounts, it is working
--- why shouldn't the president keep it up?

peter baker

Well, it's a great question. I think one of the things we've learned
about the last three years is that the norms, the standards, the lines
that we used to think of that constrained a president were more
aspirational and conceptual than they were legal. You go look. You go
back far enough, you're going to find plenty of presidents who punished
their enemies and protected their friends. But in the post-Watergate
period in particular, when we put in new guardrails, we put a new laws,
we put in new systems, we thought that that had been minimized at the
very least, right? That, yeah, you're going to probably give an
appointment to somebody who's been good to you. And you're going to
maybe take away a grant from the state of somebody who crossed you on a
vote. These things happen. They happen under any presidency. This is
that, as you put it, on steroids. And it's overt. It's right out there
in the open. He wants everybody to know what he's doing. He wants
everybody to understand. You are loyal to this president or you should
get out. And that's true of people in government. That's true of people
even in Congress. He's made very clear that the Republican Party has no
room for anybody who is not on his side. You're either in his camp or
you're not.

michael barbaro

And, of course, there's a larger context here, which is we're in the
middle of a presidential election. And I wonder how this behavior by the
president fits into his re-election strategy.

peter baker

You say that all this fits into a broader approach by this presidential
politics. It's not about unifying. It's about dividing. It's about us
versus them. And this is what the appeal is to his constituents. It is:
I am fighting for you. And they are trying to stop me. It's the deep
state, it's the Democrats, it's the fake news media. They're all trying
to stop me, and by extension, you. And that's why you should stick with
me in this election this fall. So this idea that Washington is all
alarmed by retributions and protections of friends because it violates
norms doesn't hurt his appeal to many of his voters out there, because
it's part of this larger argument that he's making. And the larger
argument is I am a force of disruption. I am a force that is shaking
things up. And the reason why you're seeing things in the news that are
bad about me is because they're fighting back. And you should stay with
me because it's not just me. It's about you too.

michael barbaro

Peter, thank you.

peter baker

OK, thank you.

michael barbaro

On Thursday, in an interview with ABC News, Attorney General Bill Barr
said that the president's interference in cases like Roger Stone's was
making it all but impossible for him to run the Department of Justice.

\begin{itemize}
\tightlist
\item
  archived recording (william barr)\\
  To have public statements and tweets made about the department, about
  people in the department, our men and women here, about cases pending
  in the department and about judges before whom we have cases, make it
  impossible for me to do my job and to assure the courts and the
  prosecutors and the department that we're doing our work with
  integrity.
\end{itemize}

michael barbaro

But Barr did not directly criticize the president, and confirmed in the
interview that he had overruled prosecutors to recommend a more lenient
sentence for Stone.

We'll be right back.

Here's what else you need to know today.

\begin{itemize}
\tightlist
\item
  archived recording\\
  Are there any senators in the chamber wishing to change their vote? If
  not, the yeas are 55, the nays are 45. The joint resolution as amended
  is passed.
\end{itemize}

michael barbaro

On Thursday, a bipartisan majority in the Senate passed a resolution
requiring President Trump to seek authorization from Congress before
taking further military action against Iran. The legislation, which was
already passed by the House, is an unusual move to restrain presidential
power and reflected the growing unease within Congress over Trump's
approach to Iraq, which many fear could lead to all-out war. It follows
Trump's decision six weeks ago to kill Qassim Suleimani, a top Iranian
military commander, without the authorization of Congress.

``The Daily'' is made by Theo Balcomb, Andy Mills, Lisa Tobin, Rachel
Quester, Lynsea Garrison, Annie Brown, Clare Toeniskoetter, Paige
Cowett, Michael Simon Johnson, Brad Fisher, Larissa Anderson, Wendy
Dorr, Chris Wood, Jessica Cheung, Alexandra Leigh Young, Jonathan Wolfe,
Lisa Chow, Eric Krupke, Marc Georges, Luke Vander Ploeg, Adizah Eghan,
Kelly Prime, Julia Longoria, Donna Summer, Jazmin Aguilera, M.J. Davis
Lin, Austin Mitchell, Sayre Quevedo, Neena Pathak, Dan Jimison. Dave
Shaw, Sydney Harper, Daniel Guillematte, Hans Buetow and Robert Jimison.
Our theme music is by Jim Brunberg and Ben Landsverk of Wonderly.
Special thanks to Sam Dolnick, Mikayla Bouchard, Stella Tan, Lauren
Jackson, Julia Simon, Mahima Chablani and Nora Keller. That's it for
``The Daily.'' I'm Michael Barbaro. See you on Tuesday after the
holiday.

The talk included a greatest-hits string of attacks on some of Mr.
Trump's top villains, including the former F.B.I. director
\href{https://www.nytimes.com/2017/05/09/us/politics/james-comey-fired-fbi.html}{James
B. Comey} (``that sleazebag''), his onetime deputy
\href{https://www.nytimes.com/2019/10/01/us/politics/andrew-mccabe-judge-prosecutors.html}{Andrew
G. McCabe}, the former F.B.I. officials
\href{https://www.nytimes.com/2019/12/02/us/trump-lisa-page.html}{Lisa
Page} and
\href{https://www.nytimes.com/2019/08/06/us/politics/peter-strzok-lawsuit.html}{Peter
Strzok} (``two lowlifes''), the former British intelligence officer
Christopher Steele, as well as Hunter Biden, Hillary Clinton and Barack
Obama.

He called Ms. Pelosi ``a horrible person,'' Mr. Romney ``a failed
presidential candidate'' who used ``religion as a crutch'' and
Representative Adam B. Schiff, Democrat of California and the lead House
manager, a ``corrupt politician.''

The president thanked his lawyers and congressional Republicans,
praising them one by one for their support. In particular, he
highlighted Senator Mitch McConnell of Kentucky, the majority leader and
his most important defender in the Senate. ``You did a fantastic job,''
Mr. Trump told him.

He called out more than a dozen other Republican defenders, including
Representatives Kevin McCarthy of California, the minority leader; Jim
Jordan of Ohio; Mark Meadows of North Carolina; and Elise Stefanik of
New York. Noticeably absent, and unmentioned by the president, were
\href{https://www.nytimes.com/interactive/2020/01/15/magazine/rudy-giuliani.html}{Rudolph
W. Giuliani}, his personal lawyer at the center of the Ukraine pressure
campaign, and Senator Lindsey Graham of South Carolina, one of his most
outspoken allies.

``This is sort of a day of celebration, because we went through hell,''
Mr. Trump said. ``But I'm sure they'll try and cook up other things,''
he added of the Democrats, ``because instead of wanting to heal our
country and fix our country, all they want to do --- in my opinion, it's
almost like they want to destroy our country. We can't let it happen.''

In the wake of Mr. Trump's acquittal, Republican senators pressed their
inquiries into Hunter Biden's finances, seeking to prove that the
president was right to insist that Ukraine investigate him and the
former vice president.

A spokeswoman for Senator Ron Wyden of Oregon, the top Democrat on the
Senate Finance Committee, said the Treasury Department had readily
complied with a request by the Republican majority for documents related
to Hunter Biden's business dealings in Ukraine, contrasted with the
administration's refusal to provide papers for the House impeachment
inquiry.

For their part, Democrats were still seeking investigations, too.
Senator Christopher S. Murphy of Connecticut asked the Government
Accountability Office to review whether the Trump administration misused
classification power to hide information about the president's Ukraine
pressure campaign. And House Democrats have already said they will
probably subpoena John R. Bolton, the former national security adviser,
to ask about Ukraine.

Stephanie Grisham, the White House press secretary, said Democrats
should be made to answer for what she called a dishonest attack on Mr.
Trump. ``Maybe people should pay for that,''
\href{https://www.foxnews.com/media/stephanie-grisham-dems-must-be-held-accountable-for-corrupt-impeachment}{she
said on Fox News}. Asked to elaborate, she equated Mr. Trump with the
United States. ``People should be held accountable for anything they do
to hurt this country and this president,'' she said.

Mr. Trump's speech at the National Prayer Breakfast was as overtly
political as any president has delivered at the annual event,
traditionally a bipartisan affair marked by talk of faith and common
ground. He triumphantly held up newspapers reporting his acquittal,
cited rising stock markets, boasted about his approval rating and urged
the audience to vote in the fall.

Mr. Trump's speech followed a keynote address by Arthur Brooks, a
Harvard professor and prominent conservative thinker, who called on
Americans to ``love your enemies.'' At one point, Mr. Brooks asked the
audience, ``How many of you love somebody with whom you disagree
politically?'' Hands around the room shot up. ``I'm going to round that
off to 100 percent,'' he said. But Mr. Trump did not raise his hand.

``Contempt is ripping our country apart,'' Mr. Brooks continued. ``We're
like a couple on the rocks in this country.'' Without directly
mentioning Mr. Trump, Mr. Brooks added: ``Ask God to take political
contempt from your heart. And sometimes when it's too hard, ask God to
help you fake it.''

Mr. Trump made no effort to fake it. ``Arthur, I don't know if I agree
with you,'' he said when he took the microphone. ``I don't know if
Arthur is going to like what I'm going to say.''

He then launched into his grievances. ``As everybody knows, my family,
our great country and your president have been put through a terrible
ordeal by some very dishonest and corrupt people,'' he said.

Without naming them, Mr. Trump singled out Mr. Romney and Ms. Pelosi.
``I don't like people who use their faith as justification for doing
what they know is wrong,'' Mr. Trump said of Mr. Romney. Then, referring
to Ms. Pelosi, he said, ``Nor do I like people who say, `I pray for
you,' when they know that's not so.''

It was the first time the speaker and the president had appeared
together since
\href{https://www.nytimes.com/2020/02/05/us/politics/trump-state-of-the-union.html}{the
State of the Union address}, when Mr. Trump refused to shake Ms.
Pelosi's hand before his speech and
\href{https://www.nytimes.com/2020/02/05/us/politics/trump-pelosi.html}{she
ripped up her copy of his speech} after he gave it. When Ms. Pelosi gave
a short talk at Thursday's breakfast about the poor and persecuted, Mr.
Trump refused to look at her, glowering with undisguised antipathy.

By the end of his own speech at the prayer breakfast, Mr. Trump
recognized that his message did not fit the love-your-enemies theme. ``I
apologize. I'm trying to learn,'' he said. ``It's not easy. It's not
easy. When they impeach you for nothing, then you're supposed to like
them? It's not easy, folks. I do my best.''

At a news conference later at the Capitol, Ms. Pelosi dismissed Mr.
Trump's comments. ``I don't know if the president understands about
prayer,'' she told reporters, but said she prays ``hard for him because
he's so off the track of our Constitution, our values.''

``He really needs our prayers,'' she added. ``He can say whatever he
wants. But I do pray for him.''

Reporting was contributed by Sheryl Gay Stolberg, Alan Rappeport,
Michael Crowley and Maggie Haberman.

\hypertarget{our-2020-election-guide}{%
\section{Our 2020 Election Guide}\label{our-2020-election-guide}}

Updated July 31, 2020

\begin{itemize}
\item
  \begin{center}\rule{0.5\linewidth}{\linethickness}\end{center}

  \hypertarget{the-latest}{%
  \subsection{The Latest}\label{the-latest}}

  \begin{itemize}
  \tightlist
  \item
    President Trump's assault on the Postal Service is intersecting with
    his attacks on mail-in voting.
    \href{https://www.nytimes.com/2020/07/31/us/politics/trump-usps-mail-delays.html?action=click\&pgtype=Article\&state=default\&region=BELOW_MAIN_CONTENT\&context=storylines_guide}{Voting
    rights groups say it is a recipe for disaster.}
  \end{itemize}
\item
  \begin{center}\rule{0.5\linewidth}{\linethickness}\end{center}

  \hypertarget{bidens-vp-search}{%
  \subsection{Biden's V.P. Search}\label{bidens-vp-search}}

  \begin{itemize}
  \tightlist
  \item
    \href{https://www.nytimes.com/article/biden-vice-president-2020.html?action=click\&pgtype=Article\&state=default\&region=BELOW_MAIN_CONTENT\&context=storylines_guide}{Here
    are 13 women} who have been under consideration to be Joe Biden's
    running mate, and why each might be chosen --- and might not be.
  \end{itemize}
\item
  \begin{center}\rule{0.5\linewidth}{\linethickness}\end{center}

  \hypertarget{keep-up-with-our-coverage}{%
  \subsection{Keep Up With Our
  Coverage}\label{keep-up-with-our-coverage}}

  \begin{itemize}
  \tightlist
  \item
    Get an
    \href{https://www.nytimes.com/newsletters/politics?action=click\&pgtype=Article\&state=default\&region=BELOW_MAIN_CONTENT\&context=storylines_guide}{email}
    recapping the day's news
  \end{itemize}

  \begin{itemize}
  \tightlist
  \item
    Download our mobile app on
    \href{https://apps.apple.com/us/app/nytimes/id284862083?ls=1\&mat_click_id=5c79ae7455014fd1bd66b5610c05b8f2-20191112-16948\&referrer=mat_click_id\%3D5c79ae7455014fd1bd66b5610c05b8f2-20191112-16948\%26link_click_id\%3D722930677036718082}{iOS}
    and
    \href{http://a.localytics.com/android?id=com.nytimes.android\&referrer=utm_source\%3Dother_nyt_mobile_web\%26utm_medium\%3DWeb\%2520page\%26utm_term\%3DGeneral\%2520Mobile\%2520Page\%26utm_campaign\%3DNYT\%2520Mobile\%2520General\%2520Page}{Android}
    and turn on Breaking News and Politics alerts
  \end{itemize}
\end{itemize}

Advertisement

\protect\hyperlink{after-bottom}{Continue reading the main story}

\hypertarget{site-index}{%
\subsection{Site Index}\label{site-index}}

\hypertarget{site-information-navigation}{%
\subsection{Site Information
Navigation}\label{site-information-navigation}}

\begin{itemize}
\tightlist
\item
  \href{https://help.nytimes.com/hc/en-us/articles/115014792127-Copyright-notice}{©~2020~The
  New York Times Company}
\end{itemize}

\begin{itemize}
\tightlist
\item
  \href{https://www.nytco.com/}{NYTCo}
\item
  \href{https://help.nytimes.com/hc/en-us/articles/115015385887-Contact-Us}{Contact
  Us}
\item
  \href{https://www.nytco.com/careers/}{Work with us}
\item
  \href{https://nytmediakit.com/}{Advertise}
\item
  \href{http://www.tbrandstudio.com/}{T Brand Studio}
\item
  \href{https://www.nytimes.com/privacy/cookie-policy\#how-do-i-manage-trackers}{Your
  Ad Choices}
\item
  \href{https://www.nytimes.com/privacy}{Privacy}
\item
  \href{https://help.nytimes.com/hc/en-us/articles/115014893428-Terms-of-service}{Terms
  of Service}
\item
  \href{https://help.nytimes.com/hc/en-us/articles/115014893968-Terms-of-sale}{Terms
  of Sale}
\item
  \href{https://spiderbites.nytimes.com}{Site Map}
\item
  \href{https://help.nytimes.com/hc/en-us}{Help}
\item
  \href{https://www.nytimes.com/subscription?campaignId=37WXW}{Subscriptions}
\end{itemize}
