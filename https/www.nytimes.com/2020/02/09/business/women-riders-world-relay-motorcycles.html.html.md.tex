Sections

SEARCH

\protect\hyperlink{site-content}{Skip to
content}\protect\hyperlink{site-index}{Skip to site index}

\href{/section/business}{Business}\textbar{}3,500 Women, 6 Continents
and a Year of Riding High

\href{https://nyti.ms/3bnvqdE}{https://nyti.ms/3bnvqdE}

\begin{itemize}
\item
\item
\item
\item
\item
\end{itemize}

\includegraphics{https://static01.nyt.com/images/2020/02/07/business/07wheels-women-relay1/merlin_168428499_39d90424-bc5f-4003-8e6a-7423eabd846a-articleLarge.jpg?quality=75\&auto=webp\&disable=upscale}

Wheels

\hypertarget{3500-women-6-continents-and-a-year-of-riding-high}{%
\section{3,500 Women, 6 Continents and a Year of Riding
High}\label{3500-women-6-continents-and-a-year-of-riding-high}}

Through sleet and monsoons, cities and villages, the Women Riders World
Relay has inspired motorcyclists ``to show themselves and each other
what they are capable of.''

Sarah Abou-Saleh during the Women Riders World Relay's first day in
Dubai.Credit...Anna Nielsen for The New York Times

Supported by

\protect\hyperlink{after-sponsor}{Continue reading the main story}

By Susan Carpenter

\begin{itemize}
\item
  Published Feb. 9, 2020Updated Feb. 10, 2020
\item
  \begin{itemize}
  \item
  \item
  \item
  \item
  \item
  \end{itemize}
\end{itemize}

The night Hayley Bell threw a leg over her KTM motorcycle and pointed
the front tire toward Scotland, it was pitch black and sleeting ---
exactly the sort of miserable weather most bikers would avoid. But she
isn't like most bikers. Ms. Bell, a 28-year-old from Northern England,
was on a mission.

It was Feb. 26, 2019, and she was wheeling through the dark for eight
hours straight, hauling a few weeks' worth of clothes and a wooden baton
that has become a kind of talisman for the yearlong event she pioneered
to bring attention to female motorcyclists: the
\href{https://womenridersworldrelay.com/}{Women Riders World Relay}.

It's exactly as it sounds.

More than 3,500 women from 79 countries have spent a year
circumnavigating the globe on two wheels, logging some 63,000 miles.
Some of them rode a few hours, others spent days or months, and a lot of
them didn't even speak the same language. But together, they broke new
ground and forged personal connections as the baton was passed from
rider to rider on a journey that spanned six continents.

The women were most recently in Dubai as the event was wrapping up, and
\href{https://www.instagram.com/p/B7p99oIBxLi/}{a final celebration} is
set for Saturday in London.

\includegraphics{https://static01.nyt.com/images/2020/02/07/business/07wheels-women-relay2/merlin_168428496_460012e9-a1de-427c-9209-bbc180866709-articleLarge.jpg?quality=75\&auto=webp\&disable=upscale}

Image

The GPS-outfitted baton is a symbol of the relay. Lara T. Saab gave it a
kiss before the ride in Dubai.Credit...Anna Nielsen for The New York
Times

``There was no `Shall we do a little trip 'round the U.K.?''' said Ms.
Bell, who was inspired by an affliction common to adventurous women with
office jobs: boredom. Forget that, they said, ``let's just do a world
relay.''

``I was at work one day, and I just wanted to travel with women who
enjoyed motor-biking and not shopping,'' she added. ``I wanted that
adrenaline excursion with females.''

Ms. Bell has been riding for five years, but she struggled to find other
women as passionate about motorcycling as she is. So she posted her
\href{https://www.facebook.com/groups/2149024862007828/}{bold idea on
Facebook}.

``I sort of got dragged into this thing,'' Liza Miller said. ``It's one
of these things that you don't really realize how much time you're
committing, but once you're in, you're glad to be there.''

Ms. Miller, a native of Santa Cruz, Calif., offered to help organize the
United States leg when the relay was just a tantalizing question mark
thrown into the vast expanse of the web four days earlier.

``There was no structure. There was no plan,'' Ms. Miller said. But the
audacity of the idea drew her in.

``Also, that women riders are overlooked, but not just that,'' she said.
``Women riders don't have the same confidence that male riders do. I
thought this would really inspire and encourage women to show themselves
and each other what they are capable of.''

Image

Hayley Bell, the relay's founder, speaking in Dubai at the end of
January.~Credit...Anna Nielsen for The New York Times

Ms. Miller, who said she ``lives, eats, sleeps and breathes
motorcycles,'' runs the Re-Cycle Garage in Santa Cruz and hosts the
``\href{http://motorcyclesandmisfits.com/}{Motorcycles and Misfits}''
podcast. But for the last 18 months, she has been using Google Translate
to communicate with other female riders from all over the world, and
Google Street View to help plot the routes, from Albania to Indonesia to
Zimbabwe.

``The big secret is that we're still building the world right ahead of
everybody as they're riding around the world,'' she said. ``We are
staying one step ahead of them.''

Ms. Miller, 53, has been riding motorcycles since she was 12 and
considers herself proficient from the littlest dirt bikes up to sport
bikes and heavyweight cruisers.

To prove her point, for the 18-day United States leg in October, she
rode one of the biggest bikes on the market: an Indian Motorcycle
Roadmaster, which tips the scales at 930 pounds.

It wasn't hers. Recognizing the growing importance of women to the
American motorsports industry, Indian Motorcycle sponsored the United
States portion of the relay, providing bikes to the lead riders and
meals at dealerships.

``A global relay ride is a huge undertaking for anyone, and the fact
that it's a group of female riders just makes it all the more exciting
for us,'' said Indian Motorcycle's customer growth manager, Joey
Lindahl.

Image

Dana Adam, from Yemen, getting ready for the ride.Credit...Anna Nielsen
for The New York Times

Indian has a long history with female motorcyclists. In 1916, the Van
Buren sisters were one of the first all-women teams to ride motorcycles
across the United States. Both rode Indian bikes.

Back then, a woman riding a motorcycle was a novelty. Today, one in five
bikers is a woman, according to the Motorcycle Industry Council. Its
2018 Motorcycle Owner Survey found that women ride for a lot of the same
reasons as men: because it's fun, gives them a sense of freedom, helps
them relax and makes it possible to enjoy nature.

But it's also about connecting with like-minded women.

``Anytime you can meet another woman who rides and share a lot of common
experiences, it grows from there,'' said Andria Yu, communications
director for the Motorcycle Industry Council. ``You see someone else do
it, and if they're kind of like you, then you think you can do it,
too.''

Increasingly, the women are meeting through Instagram, Facebook and
other social media sites, Ms. Yu said.

Facebook was how Guliafshan Tariq, from Lahore, Pakistan, got involved
in the Women Riders World Relay, or ``Wer Wer,'' as its participants
call it.

Image

Iman al-Gharabally on the second day of the rally in Dubai.Credit...Anna
Nielsen for The New York Times

Image

Widad Neiroukh at the last stop of the relay in Burj Park.Credit...Anna
Nielsen for The New York Times

``When I heard about WRWR, it excited me, because people across the
globe don't know that Pakistan is now becoming better and it has a lot
to offer,'' said Ms. Tariq, 27, who has been riding motorcycles for six
years.

She is the rare female motorcyclist in her country, she said. ``I wanted
to show the world the soft image of my country and wanted to depict the
strong face of Muslim Pakistani female bikers on an international
platform.''

Ms. Tariq's is one of the better documented legs on the relay's website.
Photos and professionally shot video show her and a small group of women
wheeling their bikes past ancient monuments, most of them wearing
helmets while riding and some donning head scarves when they aren't.

Image

From its sleety start in Scotland, the relay made it to the highways of
Dubai in late January.Credit...Anna Nielsen for The New York Times

Her trip wasn't without incident, however. Ms. Tariq was originally
supposed to take the baton from a rider in Iran, until the relay's
organizers learned that Islamic clerics in the country had issued legal
rulings, or fatwas, against women riding motorcycles in front of men.

So the previous rider in Turkey had to ship the baton with a delivery
service. But because the baton is outfitted with a GPS tracker, customs
agents confiscated it as a possible terrorist device.

It took so many days for the baton to be released that it threw off Ms.
Tariq's schedule for the Pakistani leg of the relay. And then, because
of political troubles between Pakistan and India, she wasn't able to get
a visa to ride across the border to pass the baton to the next rider.
She had to give it to a Dutch woman to cross into India and hand it off.

Still, Ms. Tariq said participating had been worth the trouble. At least
the weather cooperated.

In Laos, the relay's sole rider, Nilamon Binthavone, braved a monsoon.
Other riders have crashed, stalled, skidded and fixed their bikes on the
fly. They've cried, they've laughed. They've had an adventure, and
they've proved their point. Yes, women do ride.

Advertisement

\protect\hyperlink{after-bottom}{Continue reading the main story}

\hypertarget{site-index}{%
\subsection{Site Index}\label{site-index}}

\hypertarget{site-information-navigation}{%
\subsection{Site Information
Navigation}\label{site-information-navigation}}

\begin{itemize}
\tightlist
\item
  \href{https://help.nytimes.com/hc/en-us/articles/115014792127-Copyright-notice}{©~2020~The
  New York Times Company}
\end{itemize}

\begin{itemize}
\tightlist
\item
  \href{https://www.nytco.com/}{NYTCo}
\item
  \href{https://help.nytimes.com/hc/en-us/articles/115015385887-Contact-Us}{Contact
  Us}
\item
  \href{https://www.nytco.com/careers/}{Work with us}
\item
  \href{https://nytmediakit.com/}{Advertise}
\item
  \href{http://www.tbrandstudio.com/}{T Brand Studio}
\item
  \href{https://www.nytimes.com/privacy/cookie-policy\#how-do-i-manage-trackers}{Your
  Ad Choices}
\item
  \href{https://www.nytimes.com/privacy}{Privacy}
\item
  \href{https://help.nytimes.com/hc/en-us/articles/115014893428-Terms-of-service}{Terms
  of Service}
\item
  \href{https://help.nytimes.com/hc/en-us/articles/115014893968-Terms-of-sale}{Terms
  of Sale}
\item
  \href{https://spiderbites.nytimes.com}{Site Map}
\item
  \href{https://help.nytimes.com/hc/en-us}{Help}
\item
  \href{https://www.nytimes.com/subscription?campaignId=37WXW}{Subscriptions}
\end{itemize}
