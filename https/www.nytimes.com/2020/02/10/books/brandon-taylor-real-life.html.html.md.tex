Sections

SEARCH

\protect\hyperlink{site-content}{Skip to
content}\protect\hyperlink{site-index}{Skip to site index}

\href{https://www.nytimes.com/section/books}{Books}

\href{https://myaccount.nytimes.com/auth/login?response_type=cookie\&client_id=vi}{}

\href{https://www.nytimes.com/section/todayspaper}{Today's Paper}

\href{/section/books}{Books}\textbar{}For a Scientist Turned Novelist,
an Experiment Pays Off

\url{https://nyti.ms/2UKQqFr}

\begin{itemize}
\item
\item
\item
\item
\item
\item
\end{itemize}

Advertisement

\protect\hyperlink{after-top}{Continue reading the main story}

Supported by

\protect\hyperlink{after-sponsor}{Continue reading the main story}

\hypertarget{for-a-scientist-turned-novelist-an-experiment-pays-off}{%
\section{For a Scientist Turned Novelist, an Experiment Pays
Off}\label{for-a-scientist-turned-novelist-an-experiment-pays-off}}

``Real Life'' follows a pivotal weekend in the life of a black gay
student in the Midwest, something Brandon Taylor said was an effort to
write himself into the campus-life genre he loves reading.

\includegraphics{https://static01.nyt.com/images/2020/02/10/books/10Taylor1/merlin_167715483_7ec949aa-aa37-47d0-ad95-b4b45fff1b89-articleLarge.jpg?quality=75\&auto=webp\&disable=upscale}

By MJ Franklin

\begin{itemize}
\item
  Published Feb. 10, 2020Updated Feb. 18, 2020
\item
  \begin{itemize}
  \item
  \item
  \item
  \item
  \item
  \item
  \end{itemize}
\end{itemize}

When he set out to write a novel, Brandon Taylor, a former doctoral
student in biochemistry at the University of Wisconsin, approached it
like a scientist.

``I have this very technical approach to almost everything,'' he said
during a video interview from Iowa, where he now lives. ``If there is a
problem, I first determine the parameters of the problem, and then I try
to lay out a very systematic way of doing it.''

He started with a series of lists: Reasons he had failed to write a
novel (too concerned with inventing everything, problems with setting
and time frame). Things he considered himself good at (tone, dialogue).
Scenes he wanted in the book (a tennis match, a dinner party). He gave
himself rules, setting a goal to write 10,000 words a day. ``It began in
this very mercenary place,'' he said, ``but it moved to a place of
genuine artistic interest.''

The result is ``Real Life,'' which Riverhead is publishing next week, a
novel that merges two versions of him: Brandon Taylor the writer and
Brandon Taylor the scientist.

When he was a boy growing up in a small community outside Montgomery,
Ala., Taylor, now 30, dreamed of a career in medicine. ``My entire life,
I wanted to be a neurosurgeon,'' he said. ``Because if you're a black
boy from the South who is good at science, everyone is like, `Oh, Ben
Carson, you should be a neurosurgeon.'''

\emph{{[}}
\href{https://www.nytimes.com/2020/02/18/books/review/brandon-taylor-real-life.html}{\emph{Read
Jeremy O. Harris's review of ``Real Life.''}} \emph{{]}}

For just as long, he has been writing. ``As a kid, I was always writing
little stories, or trying to, but I never considered myself a good
writer,'' he said. It hasn't always been easy for him to reconcile these
two aspirations. When he signed up for his first creative writing class,
he remembers thinking, ``They're all English majors, and I study
chemistry.''

But it was Taylor's life as a scientist that enabled him to write ``Real
Life.''

\includegraphics{https://static01.nyt.com/images/2020/02/14/books/10Taylor2/10Taylor2-articleLarge.jpg?quality=75\&auto=webp\&disable=upscale}

He began working on it while he was in his graduate biochemistry
program. He spent most of his days in the lab, working on his
experiments on nematode worms, so he wrote mainly at night. It took him
five weeks to finish a manuscript. At one point, he threw it in the
trash after two agents rejected it. ``It felt like the universe was
telling me that I wasn't good enough, and that my work wasn't worth
sharing with the world,'' he said.

His roommate Antonio Byrd, a fellow Ph.D. student, fished it out. ``I
told him, I'm keeping this draft in my bedroom until you come to your
senses,'' Byrd said.

Taylor also deleted the manuscript files from his computer, attempting
to scrub the book from his life. A few weeks later, he found out he had
received a fellowship from the Tin House summer writing workshop.
Encouraged, he went back to his novel, recovering it from one of his
rejected queries. ``That kind of seems like a sign, too,'' he said.

Throughout his undergraduate years at Auburn University at Montgomery
and graduate school in Wisconsin, he felt he had to choose between
science or writing, and science often won. But when he received an
acceptance letter from the Iowa Writers' Workshop, he decided that, this
time, writing would win. ``I could survive not having science, but I
couldn't survive not having writing,'' he said.

``Real Life'' follows one pivotal weekend in the life of Wallace, a
black gay biochemistry Ph.D. student in the Midwest. Grappling with the
death of his father, a nascent romance with a straight friend, the
potential failure of his scientific work and a general sense that he
doesn't fit into the predominantly white cohort of his university
campus, Wallace must figure out whether he wants to continue on his path
as a student or chart a different course.

\emph{{[} This book was one of our most anticipated titles of February.}
\href{https://www.nytimes.com/2020/01/29/books/new-february-books.html?smid=nytcore-ios-share}{\emph{See
the full list}}\emph{. {]}}

Taylor knows that Wallace sounds a lot like him. Both are black gay
scientists. Both are migrants to the Midwest by way of Alabama. Both
have had confusing trysts with straight men. (``My life, in some ways,
is just a series of inappropriate encounters with heterosexual men,''
Taylor joked.) And both have stood on the precipice of a scientific
career and had to ask whether to walk back or leap.

Image

``What I wanted to do was to take this genre and this milieu that I
really respond to as a reader and to sort of write myself into it,''
Brandon Taylor said.Credit...Vivian Le for The New York Times

But Wallace --- whose name is based on Mrs. Wallis from Ann Patchett's
novel ``Commonwealth*,''* Taylor said --- is not Taylor. Instead,
Wallace is an amalgam of Taylor's own experiences as well as those of
other queer black people on college campuses, he said.

``We wanted to see us in a story, and we didn't have that,'' said
Christopher Sprott, a friend and former roommate of Taylor's who is also
black and queer.

The academic setting is one that Taylor gravitates toward as a reader
--- some of his favorite novels include
``\href{https://www.nytimes.com/2017/02/28/books/review-elif-batuman-idiot.html}{The
Idiot},'' by Elif Batuman;
``\href{https://www.nytimes.com/2011/10/16/books/review/the-marriage-plot-by-jeffrey-eugenides-book-review.html}{The
Marriage Plot},'' by Jeffrey Eugenides;
``\href{https://www.nytimes.com/2013/04/11/books/harvard-square-by-andre-aciman.html}{Harvard
Square},'' by André Aciman; and
``\href{https://www.nytimes.com/2015/09/13/books/review/lauren-groffs-fates-and-furies.html}{Fates
and Furies},'' by Lauren Groff --- but he rarely sees people like
himself when he reads them. He hopes ``Real Life'' changes that. ``What
I wanted to do was to take this genre and this milieu that I really
respond to as a reader and to sort of write myself into it,'' Taylor
said.

He channeled this desire into his first published piece of writing, the
story ``Cold River,'' which appeared in 2015 in Jonathan, a literary
journal published by Sibling Rivalry Press. He wrote the story as an
undergraduate student, after he had gone to a bookstore in Montgomery
but couldn't find the queer books he was looking for. When he asked the
clerk if they had them, he said, ``the guy was like, `We're a family
store, we don't stock that kind of stuff here.'''

Taylor considers himself primarily a short-story writer, but the desire
to see people like him represented in literature led him to make his
book debut with a novel. ``I had this feeling no one was going to take
me seriously until I write this novel,'' he said. ``I'm going to write a
novel so that people will let me write short stories in peace.''

Stories are on the way. His next book is a collection, ``Filthy
Animals,'' which will also be published with Riverhead.

But now that Taylor has made space for himself in the world of novels,
maybe he'll stick around, he said. ``Over the summer, I was like `Oh,
maybe I will write another novel.'''

\emph{\textbf{Correction: Feb. 10, 2020}}\\
\emph{An earlier version of this article misstated the publisher of the
literary journal Jonathan. It is Sibling Rivalry Press, not Lambda
Literary.}

\emph{Follow New York Times Books on}
\href{https://www.facebook.com/nytbooks/}{\emph{Facebook}}\emph{,}
\href{https://twitter.com/nytimesbooks}{\emph{Twitter}} \emph{and}
\href{https://www.instagram.com/nytbooks/}{\emph{Instagram}}\emph{, sign
up for}
\href{https://www.nytimes.com/newsletters/books-review}{\emph{our
newsletter}} \emph{or}
\href{https://www.nytimes.com/interactive/2017/books/books-calendar.html}{\emph{our
literary calendar}}\emph{. And listen to us on the}
\href{https://www.nytimes.com/column/book-review-podcast}{\emph{Book
Review podcast}}\emph{.}

Advertisement

\protect\hyperlink{after-bottom}{Continue reading the main story}

\hypertarget{site-index}{%
\subsection{Site Index}\label{site-index}}

\hypertarget{site-information-navigation}{%
\subsection{Site Information
Navigation}\label{site-information-navigation}}

\begin{itemize}
\tightlist
\item
  \href{https://help.nytimes.com/hc/en-us/articles/115014792127-Copyright-notice}{©~2020~The
  New York Times Company}
\end{itemize}

\begin{itemize}
\tightlist
\item
  \href{https://www.nytco.com/}{NYTCo}
\item
  \href{https://help.nytimes.com/hc/en-us/articles/115015385887-Contact-Us}{Contact
  Us}
\item
  \href{https://www.nytco.com/careers/}{Work with us}
\item
  \href{https://nytmediakit.com/}{Advertise}
\item
  \href{http://www.tbrandstudio.com/}{T Brand Studio}
\item
  \href{https://www.nytimes.com/privacy/cookie-policy\#how-do-i-manage-trackers}{Your
  Ad Choices}
\item
  \href{https://www.nytimes.com/privacy}{Privacy}
\item
  \href{https://help.nytimes.com/hc/en-us/articles/115014893428-Terms-of-service}{Terms
  of Service}
\item
  \href{https://help.nytimes.com/hc/en-us/articles/115014893968-Terms-of-sale}{Terms
  of Sale}
\item
  \href{https://spiderbites.nytimes.com}{Site Map}
\item
  \href{https://help.nytimes.com/hc/en-us}{Help}
\item
  \href{https://www.nytimes.com/subscription?campaignId=37WXW}{Subscriptions}
\end{itemize}
