Sections

SEARCH

\protect\hyperlink{site-content}{Skip to
content}\protect\hyperlink{site-index}{Skip to site index}

\href{https://www.nytimes.com/section/politics}{Politics}

\href{https://myaccount.nytimes.com/auth/login?response_type=cookie\&client_id=vi}{}

\href{https://www.nytimes.com/section/todayspaper}{Today's Paper}

\href{/section/politics}{Politics}\textbar{}Prosecutors Quit Roger Stone
Case After Justice Dept. Intervenes on Sentencing

\url{https://nyti.ms/2uAGbsk}

\begin{itemize}
\item
\item
\item
\item
\item
\item
\end{itemize}

Advertisement

\protect\hyperlink{after-top}{Continue reading the main story}

Supported by

\protect\hyperlink{after-sponsor}{Continue reading the main story}

\hypertarget{prosecutors-quit-roger-stone-case-after-justice-dept-intervenes-on-sentencing}{%
\section{Prosecutors Quit Roger Stone Case After Justice Dept.
Intervenes on
Sentencing}\label{prosecutors-quit-roger-stone-case-after-justice-dept-intervenes-on-sentencing}}

President Trump had complained that the recommendation of seven to nine
years in prison for his former adviser and longtime friend was a
``miscarriage of justice.''

\includegraphics{https://static01.nyt.com/images/2020/02/11/us/politics/11dc-stone/merlin_164454759_5990d84e-14f8-44b5-9863-a7152991e947-articleLarge.jpg?quality=75\&auto=webp\&disable=upscale}

\href{https://www.nytimes.com/by/katie-benner}{\includegraphics{https://static01.nyt.com/images/2018/02/16/multimedia/author-katie-benner/author-katie-benner-thumbLarge-v2.png}}\href{https://www.nytimes.com/by/sharon-lafraniere}{\includegraphics{https://static01.nyt.com/images/2018/07/12/multimedia/author-sharon-lafraniere/author-sharon-lafraniere-thumbLarge.png}}\href{https://www.nytimes.com/by/adam-goldman}{\includegraphics{https://static01.nyt.com/images/2018/07/12/multimedia/author-adam-goldman/author-adam-goldman-thumbLarge.png}}

By \href{https://www.nytimes.com/by/katie-benner}{Katie Benner},
\href{https://www.nytimes.com/by/sharon-lafraniere}{Sharon LaFraniere}
and \href{https://www.nytimes.com/by/adam-goldman}{Adam Goldman}

\begin{itemize}
\item
  Published Feb. 11, 2020Updated July 8, 2020
\item
  \begin{itemize}
  \item
  \item
  \item
  \item
  \item
  \item
  \end{itemize}
\end{itemize}

WASHINGTON --- Four prosecutors abruptly withdrew on Tuesday from the
case of
\href{https://www.nytimes.com/2020/02/12/us/politics/trump-stone.html}{President
Trump's} longtime friend
\href{https://www.nytimes.com/2020/07/08/technology/roger-stone-facebook.html}{Roger
J. Stone Jr.} after senior Justice Department officials intervened to
recommend a more lenient sentence for crimes he committed in a bid to
protect the president.

In an extraordinary decision overruling career lawyers, the
\href{https://www.nytimes.com/2020/02/12/us/politics/trump-stone.html}{Justice
Department} recommended an unspecified term of incarceration for Mr.
Stone instead of the prosecutors' request of a punishment of seven to
nine years. The move coincided with Mr. Trump's declaration on Twitter
early Tuesday that the government was treating Mr. Stone too harshly.

The development immediately prompted questions about whether the Justice
Department was
\href{https://www.nytimes.com/2020/02/12/us/politics/trump-barr-stone.html}{bending
to White House pressure}. The gulf between the prosecutors and their
Justice Department superiors burst into public view the week before Mr.
Stone was to be sentenced for
\href{https://www.nytimes.com/2019/11/15/us/politics/roger-stone-trial-guilty.html}{trying
to sabotage} a congressional investigation that had posed a threat to
the president.

\emph{{[}}\href{https://www.nytimes.com/2020/02/11/us/roger-stone-federal-prosecutors.html}{\emph{What
we know about the four prosecutors who quit the Roger Stone
case.}}\emph{{]}}

The prosecutors --- one of whom resigned from the department --- were
said to be furious over the reversal of
\href{https://www.nytimes.com/2020/02/10/us/roger-stone-prison-sentence.html}{their
sentencing request}, filed in federal court late Monday. The Stone case
was one of the most high-profile criminal prosecutions arising from the
nearly two-year investigation of Russian interference in the 2016
election by the special counsel, Robert S. Mueller III.

The development added to the sense of turmoil in Washington that has
followed
\href{https://www.nytimes.com/2020/02/05/us/politics/trump-acquitted-impeachment.html}{Mr.
Trump's acquittal} by the Senate six days ago on charges of abuse of
power and obstructing Congress. With the impeachment case behind him,
Mr. Trump
\href{https://www.nytimes.com/2020/02/07/us/politics/alexander-vindman-gordon-sondland-fired.html}{fired
an ambassador} while his national security adviser dismissed an aide.
Both had testified against the president in the impeachment hearings.

To some, the surprising reversal in the politically sensitive Stone case
underscored questions about Attorney General William P. Barr's
willingness to protect the department's independence from any political
influence by Mr. Trump. Critics have accused Mr. Barr of
\href{https://www.nytimes.com/2019/04/18/us/politics/william-barr-robert-mueller-donald-trump.html}{seeming
to side with the president} over law enforcement, including his
criticism of the origins of the F.B.I.'s investigation into whether the
Trump campaign conspired with Russia in 2016. That is now the subject of
\href{https://www.nytimes.com/2019/10/24/us/politics/john-durham-criminal-investigation.html}{a
criminal inquiry} that Mr. Barr is overseeing.

A friend of Mr. Trump for decades, Mr. Stone, 67, was convicted in
November of obstructing an inquiry by the House Intelligence Committee
into Russian interference in the 2016 election, lying to investigators
under oath and trying to block the testimony of a witness who would have
exposed his lies.

In
\href{https://twitter.com/realDonaldTrump/status/1227122206783811585}{a
message on Twitter} early Tuesday, Mr. Trump criticized the sentencing
recommendation of seven to nine years as ``horrible and very unfair.''
As he did after the jury's guilty verdict, he attacked federal law
enforcement officials, saying ``the real crimes were on the other
side.''

``Cannot allow this miscarriage of justice!'' Mr. Trump added. He later
denied to reporters that he tried to influence the case in any way, but
described the Justice Department's initial sentencing request as a
disgrace.

The president
\href{https://twitter.com/realDonaldTrump/status/1227423392078409728}{assailed
the prosecutors directly}, asking on Twitter who were the lawyers ``who
cut and ran after being exposed for recommending a ridiculous 9 year
prison sentence'' for Mr. Stone, who he said ``got caught up in an
investigation that was illegal, the Mueller Scam.''

In yet another Twitter message, he attacked Judge Amy Berman Jackson of
the Federal District Court in Washington, who is presiding over Mr.
Stone's case. He asked whether she had ordered solitary confinement for
Mr. Trump's former campaign chairman, Paul J. Manafort. The president
said Mr. Manafort suffered worse treatment than ``even mobster Al Capone
had to endure.''

Judge Jackson handled one of two criminal cases that resulted in a
prison term of seven and a half years for Mr. Manafort for financial
fraud and other crimes. But prison and jail officials, not Judge
Jackson, determined his conditions of confinement.

In a new court filing on Tuesday, Timothy Shea, the interim head of the
United States attorney's office in Washington, wrote that the Justice
Department believed that Mr. Stone should be imprisoned but that a term
of seven to nine years would be excessive.

``Ultimately, the government defers to the court as to what specific
sentence is appropriate under the facts and circumstances of this
case,'' Mr. Shea stated in the filing, which was also signed by a
prosecutor newly assigned to the case, John Crabb Jr. The new filing
also noted that Mr. Stone is a senior citizen with no prior criminal
record.

Three of the four prosecutors who conducted the investigation and trial
of Mr. Stone withdrew from the case, while a fourth resigned from the
Justice Department entirely. Some former senior officials said the case
showed that the department was in an increasingly precarious position
under Mr. Trump.

Michael R. Bromwich, who served as the department's inspector general
under President Bill Clinton, advised prosecutors to report all
instances of improper political influence to the agency's watchdog.

``This is not what you signed up for. The four prosecutors who bailed on
the Stone case have shown the way,''
\href{https://twitter.com/mrbromwich/status/1227411988030132225}{he
wrote on Twitter}. He
\href{https://twitter.com/mrbromwich/status/1227298957627396096}{described
the political pressure} from the White House as ``truly a cancer on our
system of justice.''

Mary McCord, who led the Justice Department's national security division
at the end of the Obama administration and the start of the Trump era,
predicted that the department would be beset with questions about
whether officials had bowed to political pressure from the president.

``The department has to seriously consider what impact a reversal that
appears to be in response to the president's displeasure will have on
its credibility and reputation in the courts,'' she said.

Justice Department officials did not discuss the case with anyone at the
White House, including the president, said Kerri Kupec, a department
spokeswoman, adding that they were not reacting to any directive from
Mr. Trump or to his criticism on Twitter. Mr. Trump also told reporters
later in the day that he did not discuss the case with the department.

As is customary in prominent prosecutions, the line prosecutors on the
Stone case discussed their proposed sentencing recommendation with
senior officials. But they apparently came to no clear agreement before
the document was filed in court, an outcome that one Justice Department
official blamed on a breakdown in management.

Among those involved were Mr. Shea, who took over last week as the
United States attorney in Washington; his chief of staff, David Metcalf;
the deputy attorney general, Jeffrey A. Rosen; and officials in Mr.
Barr's office, according to people familiar with the discussions who
spoke on the condition of anonymity.

Both Mr. Shea and officials in Mr. Rosen's office argued that a prison
term of seven to nine years was too harsh but they did not push for any
specific punishment, one Justice Department official said.

Officials in the offices of Mr. Barr and Mr. Rosen decided to override
the prosecutors' recommendation after they filed it in court on Monday
night, officials said.

The line prosecutors were even more upset because they were told that
they would be reversed only after Fox News had reported it late Tuesday
morning, according to people familiar with the situation. Other
prosecutors were also distressed, including those working on the case of
Mr. Trump's former national security adviser Michael T. Flynn, who faces
his own sentencing after
\href{https://www.nytimes.com/2017/12/01/us/politics/michael-flynn-guilty-russia-investigation.html}{pleading
guilty to lying to investigators} in the Russia inquiry.

At least one senior department official expressed surprise at the
decision by all four prosecutors to pull out of the case. Two of them
--- Adam C. Jed and Aaron Zelinsky --- began working on the case as
members of the special counsel's team. Michael J. Marando also resigned
from the case, as did Jonathan Kravis, who left the Justice Department
altogether.

In 2018, three career lawyers withdrew from an Affordable Care Act case
after
\href{https://www.nytimes.com/2018/10/19/us/politics/jeff-sessions-justice-department.html}{it
became entangled} in the heated politics of the Trump administration,
and one resigned in protest.

But David Laufman, a former chief of the Justice Department's
counterintelligence unit, said he could not recall another criminal case
in which an entire team of prosecutors had resigned en masse, apparently
to protest improper political interference.

``This is a `break glass in case of fire' moment,'' he said. ``We have
now seen the political leadership of the department, presumably acting
on the president's desires, reaching down into a criminal case to
withdraw a reasoned sentencing recommendation to the court.''

The prosecutors' withdrawals suggest that they not only disagreed with
officials at the department's headquarters, but were concerned about
compromising their own ethics, said Greg Brower, a former prosecutor and
senior F.B.I. official.

Until now, the Stone case had been viewed as one of the more important
successes of the special counsel investigation. Mr. Stone put up a weak
defense, and the jury deliberated only seven hours before convicting him
on all counts. In what some saw as a last-minute plea for salvation
before the verdict came in, Mr. Stone expressed hope through a proxy
that the president would pardon him.

If the president intervened to reverse the decision of career
prosecutors, it would be ``a blatant abuse of power,'' said
Representative Adam B. Schiff, Democrat of California, who oversees the
House committee that Mr. Stone was convicted of lying to.

``Doing so would send an unmistakable message that President Trump will
protect those who lie to Congress to cover up his own misconduct and
that the attorney general will join him in that effort,'' Mr. Schiff
said in a statement.

Grant Smith, a lawyer for Mr. Stone, said the defense team was ``looking
forward to reviewing'' the department's revised position. Judge Jackson
is scheduled to sentence Mr. Stone on Feb. 20.

In their initial sentencing memorandum, federal prosecutors said that
Mr. Stone deserved a stiff sentence because he threatened a witness with
bodily harm, deceived congressional investigators and carried out an
extensive, deliberate, illegal scheme that included repeatedly lying
under oath and forging documents.

Even after he was charged in a felony indictment, the prosecutors said,
Mr. Stone continued to try to manipulate the administration of justice
by threatening Judge Jackson in a social media post and violating her
gag orders.

Those and other aggravating factors justified a prison term of up to
nine years under federal sentencing guidelines, the prosecutors said.
While the guidelines are advisory, federal judges typically consider
them carefully.

Defense lawyers characterized the prosecutors' arguments as overblown.
Mr. Stone not only never intended to harm the witness, they said, but he
also never created any real obstacle for investigators. While the
witness, a New York radio host named Randy Credico, refused to testify
before the House Intelligence Committee, they pointed out, he was later
repeatedly interviewed by the F.B.I., appeared before the federal grand
jury and testified against Mr. Stone during his trial.

In a letter asking Judge Jackson to spare Mr. Stone a prison term, Mr.
Credico said that while he stood by his testimony, he never believed Mr.
Stone would carry out his threat to injure him or his beloved dog. ``I
chalked up his bellicose tirades to `Stone being Stone.' All bark and no
bite,'' Mr. Credico wrote.

Mr. Stone's defense team also said that his violations of Judge
Jackson's orders should not count against him because the criminal
proceedings had exacerbated his ``longstanding battle with anxiety'' and
that he had corrected that problem through therapy. They requested he be
sentenced to less than 15 months in prison --- the least serious
punishment under the guidelines for his crimes.

The decision to seek a more lenient punishment for Mr. Stone came less
than two weeks after prosecutors backed off on their sentencing
recommendation for Mr. Flynn. Prosecutors had
\href{https://www.nytimes.com/2020/01/07/us/politics/flynn-prosecutors-sentencing.html}{initially
sought} up to six months in prison,
\href{https://www.washingtonpost.com/local/legal-issues/prosecutors-appear-to-back-away-from-prison-recommendation-for-michael-flynn/2020/01/29/f861f3e0-42c5-11ea-b5fc-eefa848cde99_story.html}{then
said} they would not oppose probation instead of prison time.

The intervention by senior Justice Department officials in Mr. Stone's
case serves as the first big test for Mr. Shea, who assumed charge of
the United States attorney's office in Washington only
\href{https://www.justice.gov/opa/pr/attorney-general-william-p-barr-appoints-timothy-shea-interim-us-attorney-district-columbia}{last
Monday}.

A longtime trusted adviser to Mr. Barr and former senior counselor to
him, Mr. Shea now oversees some of the department's most politically
fraught cases, including two inquiries focusing on two former law
enforcement officials whom Mr. Trump has cast as political enemies. The
former F.B.I. director James B. Comey is said to be
\href{https://www.nytimes.com/2020/01/16/us/politics/leak-investigation-james-comey.html}{the
focus of investigators} in an unusual inquiry into years-old leaks to
the news media. Mr. Comey's former deputy Andrew G. McCabe faces
allegations that he misled investigators in an administrative inquiry.
That case
\href{https://www.nytimes.com/2019/10/01/us/politics/andrew-mccabe-judge-prosecutors.html}{has
languished}.

Mr. Shea replaced Jessie K. Liu, who stepped down after two years as
United States attorney after the president nominated her as the Treasury
Department's under secretary for terrorism and financial crimes. But
some Republicans questioned her conservative credentials and her loyalty
to the Trump administration.

On Tuesday, the White House withdrew her nomination, a person familiar
with the matter said, two days before
\href{https://www.banking.senate.gov/hearings/02/03/2020/nomination-hearing}{her
scheduled confirmation hearing.}

Advertisement

\protect\hyperlink{after-bottom}{Continue reading the main story}

\hypertarget{site-index}{%
\subsection{Site Index}\label{site-index}}

\hypertarget{site-information-navigation}{%
\subsection{Site Information
Navigation}\label{site-information-navigation}}

\begin{itemize}
\tightlist
\item
  \href{https://help.nytimes.com/hc/en-us/articles/115014792127-Copyright-notice}{©~2020~The
  New York Times Company}
\end{itemize}

\begin{itemize}
\tightlist
\item
  \href{https://www.nytco.com/}{NYTCo}
\item
  \href{https://help.nytimes.com/hc/en-us/articles/115015385887-Contact-Us}{Contact
  Us}
\item
  \href{https://www.nytco.com/careers/}{Work with us}
\item
  \href{https://nytmediakit.com/}{Advertise}
\item
  \href{http://www.tbrandstudio.com/}{T Brand Studio}
\item
  \href{https://www.nytimes.com/privacy/cookie-policy\#how-do-i-manage-trackers}{Your
  Ad Choices}
\item
  \href{https://www.nytimes.com/privacy}{Privacy}
\item
  \href{https://help.nytimes.com/hc/en-us/articles/115014893428-Terms-of-service}{Terms
  of Service}
\item
  \href{https://help.nytimes.com/hc/en-us/articles/115014893968-Terms-of-sale}{Terms
  of Sale}
\item
  \href{https://spiderbites.nytimes.com}{Site Map}
\item
  \href{https://help.nytimes.com/hc/en-us}{Help}
\item
  \href{https://www.nytimes.com/subscription?campaignId=37WXW}{Subscriptions}
\end{itemize}
