Sections

SEARCH

\protect\hyperlink{site-content}{Skip to
content}\protect\hyperlink{site-index}{Skip to site index}

\href{https://www.nytimes.com/section/world/middleeast}{Middle East}

\href{https://myaccount.nytimes.com/auth/login?response_type=cookie\&client_id=vi}{}

\href{https://www.nytimes.com/section/todayspaper}{Today's Paper}

\href{/section/world/middleeast}{Middle East}\textbar{}Her Dream of
Becoming a Doctor Turned Into a Nightmare, and a Movie

\href{https://nyti.ms/2UL2tCC}{https://nyti.ms/2UL2tCC}

\begin{itemize}
\item
\item
\item
\item
\item
\end{itemize}

Advertisement

\protect\hyperlink{after-top}{Continue reading the main story}

Supported by

\protect\hyperlink{after-sponsor}{Continue reading the main story}

In her words

\hypertarget{her-dream-of-becoming-a-doctor-turned-into-a-nightmare-and-a-movie}{%
\section{Her Dream of Becoming a Doctor Turned Into a Nightmare, and a
Movie}\label{her-dream-of-becoming-a-doctor-turned-into-a-nightmare-and-a-movie}}

Dr. Amani Ballour was the manager of an underground hospital in Syria.
Now her life is the subject of an Oscar-nominated documentary.

\includegraphics{https://static01.nyt.com/images/2020/02/11/multimedia/00ihw-amani1/00ihw-amani1-articleLarge-v3.jpg?quality=75\&auto=webp\&disable=upscale}

\href{https://www.nytimes.com/by/alisha-haridasani-gupta}{\includegraphics{https://static01.nyt.com/images/2018/09/10/multimedia/author-alisha-haridasani-gupta/author-alisha-haridasani-gupta-thumbLarge-v3.png}}

By \href{https://www.nytimes.com/by/alisha-haridasani-gupta}{Alisha
Haridasani Gupta}

\begin{itemize}
\item
  Published Feb. 11, 2020Updated Feb. 13, 2020
\item
  \begin{itemize}
  \item
  \item
  \item
  \item
  \item
  \end{itemize}
\end{itemize}

\begin{center}\rule{0.5\linewidth}{\linethickness}\end{center}

\hypertarget{i-just-wanted-to-study-and-be-a-doctor-and-have-a-clinic-that-was-my-dream}{%
\subsection{``I just wanted to study and be a doctor and have a clinic.
That was my
dream.''}\label{i-just-wanted-to-study-and-be-a-doctor-and-have-a-clinic-that-was-my-dream}}

\emph{--- Dr. Amani Ballour, the first and only woman to manage a
hospital in rebel-held Syria}

\begin{center}\rule{0.5\linewidth}{\linethickness}\end{center}

Dr. Amani Ballour's first ever patient was a 12-year-old boy who had
been shot in the head.

She was 24 at the time, studying medicine at Damascus University in
Syria, just as protests against President Bashar al-Assad were reaching
a boiling point in early 2012. The boy had been near one of the
demonstrations when government forces swooped in to quash the rally,
shooting at random and turning the young bystander into a victim.

His parents, worried that the authorities would arrest them, decided not
to take him to a hospital and instead turned to Ballour, who was their
neighbor at the time.

But by the time they got to her, he was dead, Ballour recalled in an
interview at The Times. ``I could do nothing for him.''

From that moment, her life became inextricably linked with the fate of
Syria, a country that for nine years would be leveled by a brutal war,
leaving
\href{https://www.hrw.org/world-report/2019/country-chapters/syria}{an
estimated 500,000 men, women and children dead}, as of 2018, and
displacing
\href{https://data2.unhcr.org/en/situations/syria\#_ga=2.119637504.960870915.1580872645-354387927.1580872645}{more
than five million more}.

``I just wanted to study and be a doctor and have a clinic. That was my
dream,'' Ballour said. ``But lots of things changed in my life.''

Almost immediately after she graduated from the university, Ballour
started volunteering at a hospital in eastern Ghouta --- a rebel
stronghold near Damascus --- as one of
\href{https://www.nytimes.com/2019/12/20/world/middleeast/syria-medical-criminalization.html}{the
few doctors in the area}. The hospital, which was under construction,
was intended to be a large, six-story medical hub, but the work was
abandoned as government forces
\href{https://www.cnn.com/2018/04/11/middleeast/inside-eastern-ghouta-syria-intl/index.html}{ramped
up attacks and seized the area}, forcing the team of 13 doctors to move
operations into the winding, subterranean space that made up the
foundation of the unfinished building. Soon enough, the new, underground
clinic came to be known as the cave.

In 2016, after four years on the job, Ballour was promoted to manager,
making her the first and only woman in charge of a hospital in
rebel-occupied Syria. Her work in that role formed the gravitational
center of a new, aptly-named documentary, ``The Cave,'' from National
Geographic Documentary Films that was nominated for an Oscar this year.

As the conflict dragged on, the patients, some as young as a few days
old, continued to pour in by the thousands, injured from the battles,
weakened from the war, some with shrapnel wounds, others with missing
limbs, and many coughing and suffocating from
\href{https://www.nytimes.com/2018/04/08/world/middleeast/syria-chemical-attack-ghouta.html}{chemical
attacks} that had been repeatedly condemned by the rest of the world.

Ballour, as manager, would supervise their treatments, jumping in to
perform emergency surgeries herself and making house calls for patients
too sick to come to her, all the while remaining
\href{https://www.facebook.com/amer.almohibany/videos/1924624087851286/?t=104}{composed
and human}. It was also on her to find ways to keep the hospital a safe
haven, ordering fortifications above ground when needed and scrounging
for resources. With Ghouta under siege by government forces, the
hospital survived on financial aid from medical nonprofits, while
supplies of medicine, food and milk for children were all cut off and
had to be smuggled in, Ballour said.

Every day, Ballour and her largely-female team of doctors and nurses
grappled with the kinds of challenges that medical professionals in
other parts of the world rarely face: What to prescribe a sick,
malnourished baby when there's no food? How do you perform a surgery
without anesthetics? How do you keep a hospital running smoothly to the
deafening soundtrack of bombings above ground and wailing parents below?

The most difficult thing, Ballour recalled, was choosing which ones to
help with the few resources they had. ``All of them have the same
symptoms; all of them are suffocating,'' she said. ``But I had to
choose: I will work with this child and the other will die.''

``I always think about that. I feel I am guilty.''

Ballour, the second youngest of two brothers and three sisters, was
always ``stubborn.'' Her eldest sister was married off at the age of 13,
but when Ballour entered her teenage years, she insisted on finishing
her education and going to college.

``I wanted to do something different,'' she said. ``Before I started
studying medicine, I wanted to be an engineer.''

Her family refused to support her because in their eyes engineering was
a man's job. So she switched plans, focusing instead on becoming a
pediatrician, which seemed more palatable for her family, she explained.

Later in her career, despite having proved herself as both a skilled
doctor and a strong leader, she nevertheless constantly bumped up
against the deep-rooted limitations of being a woman in a religiously
conservative society.

In one scene in the documentary, which was directed by Feras Fayyad and
offers a fly-on-the-wall glimpse of the hospital's daily operations, a
man looking for medicine for his sick wife asks Ballour if he can speak
to the manager.

``I am the manager,'' she responds.

When she informs him that the hospital has no more medicine, he pushes
back. ``Find me someone who can help. A male manager who can do a better
job,'' he says.

``Are hospitals with male managers able to get you the medicine?''
Ballour responds, calmly, quietly.

``Yes,'' he insists, ``women should stay home, not work.''

It is only after one of her male colleagues steps in to defend her that
the man relents and leaves.

\includegraphics{https://static01.nyt.com/images/2020/02/11/multimedia/00ihw-amani2/merlin_168661704_2078b4d8-8ccd-4dc3-b667-50c09c349aef-articleLarge.jpg?quality=75\&auto=webp\&disable=upscale}

But Ballour said she had noticed a slight shift in perception. ``Some
men said, `the hospital was very good, you were right, you did a great
job.'''

``That's why I believe we can change the community,'' she added.

In 2018, Assad's Russian-backed forces intensified their attacks on
Ghouta and, according to Ballour, offered the remaining residents a
choice: Leave in buses or stay back and be killed. So she and her team
made the difficult decision to close down the cave and leave, moving
first to Idlib in northern Syria and then crossing into Turkey, where
Ballour has been for the last two years.

Since she fled, her circumstances have improved and she has found
fleeting moments of levity. In the summer of 2018, she fell in love with
a man named Hamza whom she had been in touch with digitally while she
was in Syria but met in person only when she arrived in Turkey. Within a
month, they got married. ``I liked him,'' said Ballour, blushing and
cracking a rare giggle as she spoke of their love story. ``We have the
same things we care about. I felt comfortable when I met him, as if I
knew him for a long time.''

In January, she was awarded the Council of Europe's
\href{https://www.coe.int/en/web/human-rights-rule-of-law/-/dr-amani-ballour-a-paediatrician-from-syria-has-been-awarded-the-council-of-europe-s-raoul-wallenberg-prize}{Raoul
Wallenberg Prize} for her humanitarian efforts. Last weekend, she
attended the Academy Awards in Los Angeles, Calif. And, today, she and
her husband are applying for asylum in Canada, hoping to start afresh
and move forward.

But the memories of the war continue to haunt her and make it difficult
for her to work with children again. ``When I see sick children, they
remind me of my children in Ghouta,'' she said referring to all the
children who came through the cave and she considered her own.

``I can't forget them.''

\begin{center}\rule{0.5\linewidth}{\linethickness}\end{center}

\hypertarget{what-else-is-happening}{%
\subsection{What else is happening}\label{what-else-is-happening}}

\emph{Here are three articles from The Times you may have missed.}

Image

Sarah Abou-Saleh on the first day of the worldwide women's motorbike
relay in Dubai.Credit...Anna Nielsen for The New York Times

\begin{itemize}
\item
  \textbf{``I just wanted to travel with women who enjoyed motor-biking
  and not shopping.''} More than 3,500 women from 79 countries have
  spent a year circumnavigating the globe on two wheels, logging some
  63,000 miles.
  {[}\href{https://www.nytimes.com/2020/02/09/business/women-riders-world-relay-motorcycles.html}{Read
  the story}.{]}
\item
  \textbf{``I would never put myself in that position.''} One of The
  Times reporters who broke the story of Harvey Weinstein's alleged
  abuse more than two years ago spoke with Donna Rotunno, the lawyer
  behind Mr. Weinstein's legal strategy.
  {[}\href{https://www.nytimes.com/2020/02/07/podcasts/the-daily/weinstein-trial.html}{Listen
  to the podcast}.{]}
\item
  \textbf{``Leftover women.''} The Chinese government, in an attempt to
  pressure single women above the age of 27 to get married, labels them
  \emph{sheng nu} or ``leftover women.'' We follow one woman as she
  navigates the social pressure to find a husband.
  {[}\href{https://www.nytimes.com/2020/02/11/opinion/china-marriage-leftover-women.html}{Watch
  the Op-Doc}.{]}
\end{itemize}

\begin{center}\rule{0.5\linewidth}{\linethickness}\end{center}

\hypertarget{icymi-natalie-portmans-cloak}{%
\subsection{ICYMI: Natalie Portman's
Cloak}\label{icymi-natalie-portmans-cloak}}

Image

Credit...Robyn Beck/Agence France-Presse --- Getty Images

Many stars chose to walk the red carpet at last weekend's Academy Awards
in the safe, glamorous looks of Old Hollywood. But the handful who chose
to, as
\href{https://www.nytimes.com/2020/02/10/style/oscars-red-carpet-fashion.html}{our
chief fashion critic Vanessa Friedman put it}, ``throw down the gauntlet
and change things up,'' stood out.

Natalie Portman (above) wore a Dior gown and a black cape with the names
of the female directors who had been snubbed by the Academy embroidered
in gold, including Lorene Scafaria (``Hustlers), Lulu Wang (``The
Farewell'') and Greta Gerwig (``Little Women'').

\begin{center}\rule{0.5\linewidth}{\linethickness}\end{center}

\emph{In Her Words is written by Alisha Haridasani Gupta and edited by
Francesca Donner. Our art director is Catherine Gilmore-Barnes, and our
photo editor is Sandra Stevenson.}

\emph{Did someone forward you this email?}
\href{https://www.nytimes.com/newsletters/in-her-words}{\emph{Sign up
here}} \emph{to get future installments. You can also}
\href{https://www.instagram.com/nytgender/}{\emph{follow us on
Instagram}} \emph{or email us at}
\href{mailto:inherwords@nytimes.com}{\emph{inherwords@nytimes.com}}\emph{.}

Advertisement

\protect\hyperlink{after-bottom}{Continue reading the main story}

\hypertarget{site-index}{%
\subsection{Site Index}\label{site-index}}

\hypertarget{site-information-navigation}{%
\subsection{Site Information
Navigation}\label{site-information-navigation}}

\begin{itemize}
\tightlist
\item
  \href{https://help.nytimes.com/hc/en-us/articles/115014792127-Copyright-notice}{©~2020~The
  New York Times Company}
\end{itemize}

\begin{itemize}
\tightlist
\item
  \href{https://www.nytco.com/}{NYTCo}
\item
  \href{https://help.nytimes.com/hc/en-us/articles/115015385887-Contact-Us}{Contact
  Us}
\item
  \href{https://www.nytco.com/careers/}{Work with us}
\item
  \href{https://nytmediakit.com/}{Advertise}
\item
  \href{http://www.tbrandstudio.com/}{T Brand Studio}
\item
  \href{https://www.nytimes.com/privacy/cookie-policy\#how-do-i-manage-trackers}{Your
  Ad Choices}
\item
  \href{https://www.nytimes.com/privacy}{Privacy}
\item
  \href{https://help.nytimes.com/hc/en-us/articles/115014893428-Terms-of-service}{Terms
  of Service}
\item
  \href{https://help.nytimes.com/hc/en-us/articles/115014893968-Terms-of-sale}{Terms
  of Sale}
\item
  \href{https://spiderbites.nytimes.com}{Site Map}
\item
  \href{https://help.nytimes.com/hc/en-us}{Help}
\item
  \href{https://www.nytimes.com/subscription?campaignId=37WXW}{Subscriptions}
\end{itemize}
