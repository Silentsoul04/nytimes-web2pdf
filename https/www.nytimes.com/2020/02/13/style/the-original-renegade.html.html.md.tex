Sections

SEARCH

\protect\hyperlink{site-content}{Skip to
content}\protect\hyperlink{site-index}{Skip to site index}

\href{https://www.nytimes.com/section/style}{Style}

\href{https://myaccount.nytimes.com/auth/login?response_type=cookie\&client_id=vi}{}

\href{https://www.nytimes.com/section/todayspaper}{Today's Paper}

\href{/section/style}{Style}\textbar{}The Original Renegade

\url{https://nyti.ms/2SKejKF}

\begin{itemize}
\item
\item
\item
\item
\item
\item
\end{itemize}

Advertisement

\protect\hyperlink{after-top}{Continue reading the main story}

Supported by

\protect\hyperlink{after-sponsor}{Continue reading the main story}

\hypertarget{the-original-renegade}{%
\section{The Original Renegade}\label{the-original-renegade}}

A 14-year-old in Atlanta created one of the biggest dances on the
internet. But nobody really knows that.

\includegraphics{https://static01.nyt.com/images/2020/02/13/fashion/13RENEGADE1/13RENEGADE1-articleLarge-v2.jpg?quality=75\&auto=webp\&disable=upscale}

\href{https://www.nytimes.com/by/taylor-lorenz}{\includegraphics{https://static01.nyt.com/images/2020/03/18/reader-center/author-taylor-lorenz/author-taylor-lorenz-thumbLarge.png}}

By \href{https://www.nytimes.com/by/taylor-lorenz}{Taylor Lorenz}

\begin{itemize}
\item
  Published Feb. 13, 2020Updated July 16, 2020
\item
  \begin{itemize}
  \item
  \item
  \item
  \item
  \item
  \item
  \end{itemize}
\end{itemize}

FAYETTEVILLE, Ga. --- Jalaiah Harmon is coming up in a dance world
completely reshaped by the internet.

She trains in all the traditional ways, taking classes in hip-hop,
ballet, lyrical, jazz, tumbling and tap after school at a dance studio
near her home in the Atlanta suburbs. She is also building a career
online, studying viral dances, collaborating with peers and posting
original choreography.

Recently, a sequence of hers turned into one of the most viral dances
online: \href{https://www.youtube.com/watch?v=L2046dlkjQQ}{the
Renegade}.

\begin{quote}
\end{quote}

There's basically nothing bigger right now. Teenagers are doing the
dance in the halls of high
\href{https://www.tiktok.com/@zoey_rene1213/video/6783451327117724934}{schools,
at pep rallies} and
\href{https://www.youtube.com/watch?v=L2046dlkjQQ}{across the internet}.
Lizzo, Kourtney Kardashian, David Dobrik and members of the K-pop band
\href{https://twitter.com/bangchannies/status/1225950060569075712}{Stray
Kids} have all performed it. Charli D'Amelio, TikTok's biggest homegrown
star, with nearly 26 million followers on the platform, has been
affectionately deemed the dance's ``C.E.O.'' for popularizing it.

\begin{center}\rule{0.5\linewidth}{\linethickness}\end{center}

Some of the latest from
\href{https://www.nytimes.com/by/taylor-lorenz}{Taylor Lorenz}:

\begin{itemize}
\item
  \href{https://www.nytimes.com/2020/07/10/style/tiktok-ban-us-users-influencers-taylor-lorenz.html}{TikTok
  Users React to Threat to Ban App in U.S.}
\item
  \href{https://www.nytimes.com/2020/07/09/style/tiktok-stars-race-to-land-reality-shows.html}{TikTok
  Stars Race to Land Reality Shows}
\item
  \href{https://www.nytimes.com/2020/06/21/style/tiktok-trump-rally-tulsa.html}{TikTok
  Teens and K-Pop Stans Say They Sunk Trump Rally}
\item
  \href{https://www.nytimes.com/2020/07/16/style/taylor-lorenz-internet-culture-reporting.html}{How
  We Report on Internet Culture and the Teens Who Rule It}
\end{itemize}

\begin{center}\rule{0.5\linewidth}{\linethickness}\end{center}

But the one person who hasn't been able to capitalize on the attention
is Jalaiah, the Renegade's 14-year-old creator.

``I was happy when I saw my dance all over,'' she said. ``But I wanted
credit for it.''

\includegraphics{https://static01.nyt.com/images/2020/02/16/fashion/13RENEGADE5/13RENEGADE5-articleLarge-v3.jpg?quality=75\&auto=webp\&disable=upscale}

\hypertarget{the-viral-dance-iearchy}{%
\subsection{The Viral Dance-iearchy}\label{the-viral-dance-iearchy}}

TikTok, one of the biggest video apps in the world, has become
synonymous with dance culture. Yet many of its most popular dances,
including the Renegade, Holy Moly Donut Shop, the Mmmxneil and Cookie
Shop have come from young black creators on myriad smaller apps.

Most of these dancers identify as Dubsmashers. This means, in essence,
that they use the Dubsmash app and other short-form social video apps,
like Funimate, ‎Likee and Triller, to document choreography to songs
they love. They then post (or cross-post) the videos to Instagram, where
they can reach a wider audience. If it's popular there, it's only a
matter of time before the dance is co-opted by the TikTok masses.

``TikTok is like a mainstream Dubsmash,'' said Kayla Nicole Jones, 18, a
YouTube star and music artist. ``They take from Dubsmash and they run
off with the sauce.''

Polow da Don, a producer, songwriter and rapper who has worked with
Usher and Missy Elliott, said: ``Dubsmash catches things at the roots
when they're culturally relevant. TikTok is the suburban kids that take
things on when it's already the style and bring it to their community.''

Though Jalaiah is very much a suburban kid herself --- she lives in a
picturesque home on a quiet street outside of Atlanta --- she is part of
the young, cutting-edge dance community online that more mainstream
influencers co-opt.

The Renegade dance followed this exact path. On Sept. 25, 2019, Jalaiah
came home from school and asked a friend she had met through Instagram,
\href{https://www.instagram.com/chillingwk/}{Kaliyah Davis}, 12, if she
wanted to create a post together. Jalaiah listened to the beats in the
song ``Lottery'' by the Atlanta rapper K-Camp and then choreographed a
difficult sequence to its chorus, incorporating other viral moves like
the \href{https://www.youtube.com/watch?v=6CPtOe3GVwk}{wave} and the
\href{https://www.youtube.com/watch?v=ZPNfN63WgXw}{whoa}.

She filmed herself \href{https://www.instagram.com/p/B22za3xD1Fh/}{and
posted it}, first to Funimate (where she has more than 1,700 followers)
and then to her more than 20,000 followers on
\href{https://www.instagram.com/_.xoxlaii/}{Instagram} (with a
side-by-side shot of Kaliyah and her performing it together).

``I posted on Instagram and it got about 13,000 views, and people
started doing it over and over again,'' Jalaiah said. In October, a user
named \href{https://www.tiktok.com/@global.jones}{@global.jones} brought
it to TikTok, changing up some of the moves at the end, and the dance
spread like wildfire. Before long, Charli D'Amelio had posted a video of
herself doing it, as did many other TikTok influencers. None gave
Jalaiah credit.

After long days in the ninth grade and between dance classes, Jalaiah
tried to get the word out. She hopped in the comments of several videos,
asking influencers to tag her. For the most part she was ridiculed or
ignored.

She even set up her own \href{https://www.tiktok.com/@_.xoxlaii}{TikTok}
account and
\href{https://www.tiktok.com/@_.xoxlaii/video/6785619884479876358}{created
a video of herself} in front of a green screen, Googling the question
``who created the Renegade dance?'' in an attempt to set the record
straight. ``I was upset,'' she said. ``It wasn't fair.''

Image

Jalaiah teaching her friends the Renegade at the Sky Dance Academy
studio, where she takes classes.Credit...Jill Frank for The New York
Times

Image

Her repertoire includes power tumbling!Credit...Jill Frank for The New
York Times

To be robbed of credit on TikTok is to be robbed of real opportunities.
In 2020, virality means income: Creators of popular dances, like the
\href{https://www.abc.net.au/news/2018-12-19/floss-dance-creator-backpack-kids-sues-fortnite/10633962}{Backpack
Kid} or
\href{https://abcnews.go.com/Entertainment/shiggy-feelings-challenge-changed-life/story?id=59782945}{Shiggy},
often amass large online followings and become influencers themselves.
That, in turn, opens the door to brand deals, media opportunities and,
most important for Jalaiah, introductions to those in the professional
dance and choreography community.

Obtaining credit isn't easy, though. As the writer Rebecca Jennings
\href{https://www.vox.com/the-goods/2020/2/4/21112444/renegade-tiktok-song-dance}{noted
in Vox} in an article about the online dance world's thorny ethics:
``Dances are virtually impossible to legally claim as one's own.''

But credit and attention are valuable even without legal ownership. ``I
think I could have gotten money for it, promos for it, I could have
gotten famous off it, get noticed,'' Jalaiah said. ``I don't think any
of that stuff has happened for me because no one knows I made the
dance.''

\hypertarget{scares-of-the-share-economy}{%
\subsection{Scares of the Share
Economy}\label{scares-of-the-share-economy}}

Image

Stefanie Harmon, Jalaiah's mother, learned the true extent of Jalaiah's
online success only recently. ``She told me, `Mommy, I made a dance and
it went viral,''' Ms. Harmon said.Credit...Jill Frank for The New York
Times

Cross-platform sharing --- of dances, of memes, of information --- is
how things are made on the internet. Popular tweets go viral on
Instagram, videos made on Instagram make their way onto YouTube. But in
recent years, several large Instagram meme accounts have faced backlash
for sharing jokes that went viral without crediting the creator.

TikTok was introduced in the United States only a year and a half ago.
Norms, particularly around credit, are still being established. But for
Dubsmashers and those in the Instagram dance community, it's common
courtesy to tag the handles of dance creators and musicians, and use
hashtags to track the evolution of a dance.

It has set up a culture clash between the two influencer communities.
``On TikTok they don't give people credit,'' said Raemoni Johnson, a
15-year-old Dubsmasher. ``They just do the video and they don't tag
us.'' (This acrimony is exacerbated by the fact that TikTok does not
make it easy to find the creator of a dance.)

On Jan. 17, tensions boiled over after Barrie Segal, the head of content
at Dubsmash, posted a series of videos asking Charli D'Amelio to give a
dance credit to \href{https://www.instagram.com/thereald1.nayah/}{D1
Nayah}, a popular Dubsmash dancer with more than one million followers
on Instagram, for her Donut Shop dance.
\href{https://www.instagram.com/p/B7fThb4nSrH/}{TikTok Room}, a gossip
account on Instagram, picked up the controversy, and spurred a sea of
comments.

``Why is it so hard to give black creators their credit,'' said
\href{https://www.instagram.com/chickenlegbiz/}{one Instagram
commenter}, referring to the mostly white TikTokers who have taken
dances from Dubsmashers and posted them without credit. ``Instead of
using dubsmash, use tiktok and then ppl would credit you maybe,'' a
TikToker fan \href{https://www.instagram.com/annielebob/}{said}.

``I'm not an argumentative person on social media --- I don't want beef
or anything like that,'' said Jhacari Blunt, an 18-year-old Dubsmasher
who has had some of his dances co-opted by TikTokers. ``But it's like,
we all know where that dance came from.''

Image

``We're all inspired by other people,'' Jalaiah said. ``We make up a
dance and it grows.''Credit...Jill Frank for The New York Times

At this point, if a TikToker doesn't initially know who did a dance,
commenters will usually tag the original creator's handle. Charli
D'Amelio and other stars have started giving dance credits and tagging
creators in their captions.

And the creators who are flooding into TikTok from Instagram and
Dubsmash are leading the way by example. ``We have 1.7 million followers
and we always give credit whether the person has zero followers or
not,'' said Yoni Wicker, 14, one half of the
\href{https://www.instagram.com/thewickertwinz/?hl=en}{TheWickerTwinz}.
``We know how important it is. That person who made that dance, they
might be a fan of ours. Us tagging them makes their day.''

\hypertarget{onward-and-upward}{%
\subsection{Onward and Upward}\label{onward-and-upward}}

Stefanie Harmon, Jalaiah's mother, learned the true extent of Jalaiah's
online success only recently. ``She told me, `Mommy, I made a dance and
it went viral,''' Ms. Harmon said.

``She wasn't kicking and screaming about the fact that she wasn't
getting credit,'' she added, ``but I could tell it had affected her. I
said, `Why do you care whether you're not getting credit? Just make
another one.'''

Jalaiah continues to post a steady stream of dance videos to Funimate,
Dubsmash, and Instagram. She said she doesn't harbor any hard feelings
against Charli D'Amelio for popularizing the Renegade without naming
her. Instead, she hopes she can collaborate with her one day.

Charli D'Amelio, through a publicist, said that she was ``so glad to
know'' who created the dance. ``I know it's so associated with me,'' she
said, ``but I'm so happy to give Jalaiah credit and I'd love to
collaborate with her.''

``We're all inspired by other people,'' Jalaiah said. ``We make up a
dance and it grows.''

Off the internet, she continues to compete in dance competitions with
her studio and hopes to one day take classes at
\href{https://www.dance411.com/}{Dance 411}, a prestigious dance school
in Atlanta. Ultimately, it's the art form that she loves. ``It makes me
happy to dance,'' she said.

Advertisement

\protect\hyperlink{after-bottom}{Continue reading the main story}

\hypertarget{site-index}{%
\subsection{Site Index}\label{site-index}}

\hypertarget{site-information-navigation}{%
\subsection{Site Information
Navigation}\label{site-information-navigation}}

\begin{itemize}
\tightlist
\item
  \href{https://help.nytimes.com/hc/en-us/articles/115014792127-Copyright-notice}{©~2020~The
  New York Times Company}
\end{itemize}

\begin{itemize}
\tightlist
\item
  \href{https://www.nytco.com/}{NYTCo}
\item
  \href{https://help.nytimes.com/hc/en-us/articles/115015385887-Contact-Us}{Contact
  Us}
\item
  \href{https://www.nytco.com/careers/}{Work with us}
\item
  \href{https://nytmediakit.com/}{Advertise}
\item
  \href{http://www.tbrandstudio.com/}{T Brand Studio}
\item
  \href{https://www.nytimes.com/privacy/cookie-policy\#how-do-i-manage-trackers}{Your
  Ad Choices}
\item
  \href{https://www.nytimes.com/privacy}{Privacy}
\item
  \href{https://help.nytimes.com/hc/en-us/articles/115014893428-Terms-of-service}{Terms
  of Service}
\item
  \href{https://help.nytimes.com/hc/en-us/articles/115014893968-Terms-of-sale}{Terms
  of Sale}
\item
  \href{https://spiderbites.nytimes.com}{Site Map}
\item
  \href{https://help.nytimes.com/hc/en-us}{Help}
\item
  \href{https://www.nytimes.com/subscription?campaignId=37WXW}{Subscriptions}
\end{itemize}
