Sections

SEARCH

\protect\hyperlink{site-content}{Skip to
content}\protect\hyperlink{site-index}{Skip to site index}

\href{/section/us}{U.S.}\textbar{}An Old and Contested Solution to Boost
Reading Scores: Phonics

\url{https://nyti.ms/39CHxBW}

\begin{itemize}
\item
\item
\item
\item
\item
\item
\end{itemize}

\includegraphics{https://static01.nyt.com/images/2020/02/05/us/00reading-01alt/00reading-01alt-articleLarge.jpg?quality=75\&auto=webp\&disable=upscale}

\hypertarget{an-old-and-contested-solution-to-boost-reading-scores-phonics}{%
\section{An Old and Contested Solution to Boost Reading Scores:
Phonics}\label{an-old-and-contested-solution-to-boost-reading-scores-phonics}}

As test scores lag, there's a growing debate between proponents of the
``science of reading,'' which emphasizes phonics, and traditional
educators who prefer to instill a love of literature.

Ari Cotton, a first grader, reading a book at Garrison Elementary School
in Washington, D.C., one of only two jurisdictions in the country that
saw rising scores on national reading tests last year.Credit...Lexey
Swall for The New York Times

Supported by

\protect\hyperlink{after-sponsor}{Continue reading the main story}

\href{https://www.nytimes.com/by/dana-goldstein}{\includegraphics{https://static01.nyt.com/images/2018/06/12/multimedia/author-dana-goldstein/author-dana-goldstein-thumbLarge.png}}

By \href{https://www.nytimes.com/by/dana-goldstein}{Dana Goldstein}

\begin{itemize}
\item
  Feb. 15, 2020
\item
  \begin{itemize}
  \item
  \item
  \item
  \item
  \item
  \item
  \end{itemize}
\end{itemize}

WASHINGTON --- ``Bit!'' Ayana Smith called out as she paced the alphabet
rug in front of her kindergarten students at Garrison Elementary School.

``Buh! Ih! Tuh!'' the class responded in unison, making karate chop
motions as they enunciated the sound of each letter. In a 10-minute
lesson, the students chopped up and correctly spelled a string of words:

Top. ``Tuh! Ah! Puh!''

Wig. ``Wuh! Ih! Guh!''

Ship. ``Shuh! Ih! Puh!''

Ms. Smith's sounding-out exercises might seem like a common-sense way to
teach reading. But for decades, many teachers have embraced a different
approach, convinced that exposing students to the likes of Dr. Seuss and
Maya Angelou is more important than drilling them on phonics.

Lagging student performance and
\href{https://www.apmreports.org/emily-hanford}{newly relevant
research}, though, have prompted some educators to reconsider the ABCs
of reading instruction. Their effort gained new urgency after
\href{https://www.nytimes.com/2019/10/30/us/reading-scores-national-exam.html}{national
test scores last year} showed that only a third of American students
were proficient in reading, with widening gaps between good readers and
bad ones.

Now members of this vocal minority, proponents of what they call the
``science of reading,'' congregate on social media and swap lesson plans
intended to avoid creating ``curriculum casualties'' --- students who
have not been effectively taught to read and who will continue to
struggle into adulthood, unable to comprehend medical forms, news
stories or job listings.

The bible for these educators is a body of research produced by
linguists, psychologists and cognitive scientists. Their findings have
pushed some states and school districts to make big changes in how
teachers are trained and students are taught.

The ``science of reading'' stands in contrast to the ``balanced
literacy'' theory that
\href{https://www.edweek.org/ew/articles/2020/01/22/preservice-teachers-are-getting-mixed-messages-on.html}{many
teachers are exposed to in schools of education}. That theory holds that
students can learn to read through exposure to a wide range of books
that appeal to them, without too much emphasis on technically complex
texts or sounding out words.

Eye-tracking studies and brain scans now show that the opposite is true,
according to many scientists. Learning to read, they say, is the work of
deliberately practicing how to quickly connect the letters on the page
to the sounds we hear each day.

The evidence ``is about as close to conclusive as research on complex
human behavior can get,''
\href{https://www.amazon.com/Language-Speed-Sight-Can\%C2\%92t-About/dp/0465019323}{writes
Mark Seidenberg}, a cognitive neuroscientist and reading expert at the
University of Wisconsin, Madison.

Phonics has gone in and out of style for decades, and the current
conflict over how to teach reading is only the latest in a tug-of-war
that dates to the 19th century. A major push for phonics instruction
under President George W. Bush, through a federal program called Reading
First,
\href{https://www.mdrc.org/sites/default/files/understanding_reading_first.pdf}{did
not produce} widespread achievement gains, raising questions about
whether the current efforts can succeed.

Phonics boosters say they now know more about what works, and that
phonics alone isn't the answer. Alongside bigger doses of sounding out,
they want struggling students to grapple with more advanced books, so
they won't get stuck in a cycle of low expectations and boredom. Some
schools are devoting more time to social studies and science, subjects
that help build vocabulary and knowledge in ways that can make students
stronger readers.

States have passed laws
\href{http://dese.ade.arkansas.gov/divisions/learning-services/r.i.s.e.-arkansas/its-all-about-meaning}{requiring}
that schools use phonics-centric curriculums and screen students more
aggressively for reading problems --- or even
\href{https://www.ncsl.org/documents/legisbriefs/2018/june/LBJune2018_A_Look_at_Third_Grade_Reading_Retention_Policies_goID32459.pdf}{hold
back those} who struggle most. In January, Education Secretary Betsy
DeVos castigated
\href{https://twitter.com/BetsyDeVosED/status/1221912011451224064}{colleges
of education for teaching what she described as ``junk science''} about
reading.

But the education establishment is
\href{https://ncte.org/statement/the-act-of-reading/}{pushing back},
worried that too many lessons like Ms. Smith's could be stultifying ---
a poor substitute for a teacher reading aloud from a book of Shel
Silverstein poems, or guiding children through lushly illustrated
stories by Ezra Jack Keats. They blame low student performance on such
factors as inexperienced teachers, school funding inequities and homes
that lack books or time for parents to read to their children.

The guardians of balanced literacy acknowledge that phonics has a place.
But they trust their own classroom experience over brain scans or
laboratory experiments, and say they have seen many children overcome
reading problems without sound-it-out drills. They value children
picking books that interest them and worry that pushing students into
harder texts could turn them off reading entirely.

Karen K. Wixson, an author of a
\href{https://www.literacyworldwide.org/docs/default-source/where-we-stand/ila-children-experiencing-reading-difficulties.pdf}{recent
report} warning that too much phonics can harm children, called the new
push ``incredibly, scarily naïve.''

\includegraphics{https://static01.nyt.com/images/2020/02/05/us/00reading-01/merlin_166428825_8d0f0bcb-45fd-4af6-bffc-d96216880dce-articleLarge.jpg?quality=75\&auto=webp\&disable=upscale}

\hypertarget{a-growing-demand-for-phonics}{%
\subsection{A Growing Demand for
Phonics}\label{a-growing-demand-for-phonics}}

In Ms. Smith's classroom in Washington, Madisyn Hall-Jones, 5,
demonstrated her progress by reading aloud a short story about picking
apples that she had written and illustrated herself.

``It's not rote,'' the school's principal, Brigham Kiplinger, said of
the phonics-driven curriculum. ``It's joyful.''

Washington is one of only two jurisdictions, along with Mississippi, to
\href{https://www.nationsreportcard.gov/mathematics/supportive_files/2019_infographic.pdf}{increase}
average reading scores on National Assessment of Educational Progress
tests between 2017 and 2019. Both did so despite high-poverty student
populations, and both are requiring more phonics.

``For us, this is social justice work,'' Mr. Kiplinger said. The
majority of students at Garrison Elementary come from low-income
families. If parents express concerns about the new curriculum, he
invites them to visit a classroom like Ms. Smith's and see the
difference.

Parents in suburban St. Louis are looking for similar results. More than
a third of kindergarten to third-grade students in the highly regarded
Lindbergh school district tested as ``at risk'' for dyslexia last
spring, after Missouri instituted mandatory screening. Angry district
residents sent an
\href{https://docs.google.com/document/d/17PZ1w1TEjcSZnlhCCnyAld6rF3X3pdNjTpOhPE4YuD8/edit?fbclid=IwAR25mkkUcWVAogUBwTQ3cl6VWXjdRwW9wi5Igk6J2SdgFGXCMImSMR2d5eY}{open
letter} to the school board in November, demanding that the district
embrace the science of reading.

The district said it had added a new phonics sequence in the early
elementary grades and retrained some teachers. But it stands by its
broader balanced literacy approach, which it said gives teachers the
autonomy to tailor instruction to students at all levels.

That's not enough for parents like Diane Dragan. An attorney who has
three children with dyslexia, Ms. Dragan noted that well-off parents in
her area regularly pay thousands of dollars to have their children
taught intensive phonics at private tutoring centers.

``The irony to me is that the public-school teacher who teaches balanced
literacy during the day moonlights to do science-based tutoring for kids
who fail to learn to read,'' Ms. Dragan said.

In Mississippi, all prospective elementary schoolteachers are now
required to pass a test in the foundations of reading, including
phonics. The state has also dispatched literacy coaches to struggling
schools.

\href{https://fordhaminstitute.org/national/commentary/mississippi-rising-partial-explanation-its-naep-improvement-it-holds-students}{More
controversially}, it passed a law in 2013 requiring third graders to be
held back if they score poorly on an end-of-year reading exam; last
year,
\href{https://www.mdek12.org/sites/default/files/Offices/MDE/SSE/lbpa_summary_2018.pdf}{about
10 percent} of them were retained, for
\href{https://hechingerreport.org/mississippi-made-the-biggest-leap-in-national-test-scores-this-year-is-this-controversial-law-the-reason-why/}{reading
difficulties or other reasons}.

Some reading experts have called Mississippi's recent gains into
question, arguing that by retaining so many of the lowest-scoring third
graders, the state had stigmatized students and manufactured a
higher-performing pool of test takers. But Shannon D. Whitehead, the
principal of McNeal Elementary School in Canton, Miss., supported the
state's decision to get tough.

Her school put in place a phonics sequence that continues through fifth
grade, and started assigning more challenging literature, including
Langston Hughes poems. It hosts early-morning, after-school and Saturday
tutoring sessions for students at risk of failing state tests. Scores
have improved modestly.

As painful as it can be to tell a child they have to repeat a year, Dr.
Whitehead said, ``in order for us to ensure that our students are able
to compete globally, we have to have an accountability system.''

Image

Kate Drake and Frank Brier, who are both in first grade, reading a book
together at Garrison Elementary School.Credit...Lexey Swall for The New
York Times

\hypertarget{a-curriculum-guru-embraces-some-change}{%
\subsection{A Curriculum Guru Embraces (Some)
Change}\label{a-curriculum-guru-embraces-some-change}}

One of the most popular reading curriculums in the country --- used in
about 20 percent of schools, including the Lindbergh district near St.
Louis --- was developed by Lucy Calkins, a professor at Teachers
College, Columbia University. She is widely admired for her emphasis on
helping students develop a love of reading and writing.

But her curriculum, which follows the balanced literacy model, has come
under
\href{https://achievethecore.org/page/3240/comparing-reading-research-to-program-design-an-examination-of-teachers-college-units-of-study}{increasing
fire} from critics who say it devotes too little time to phonics
practice and gives teachers and students too much choice over what books
to read, allowing them to avoid more challenging texts. Earlier this
month, the public schools in Oakland, Calif., told staff members that
the district would move away from Professor Calkins's materials after
the city's N.A.A.C.P. chapter and parent activists
\href{https://www.change.org/p/city-of-oakland-literacy-for-all-it-is-time-to-ensure-every-child-becomes-a-powerful-lifelong-reader?recruiter=1034689940\&recruited_by_id=81ef6730-38c3-11ea-8956-efb177e9a8f1\&utm_source=share_petition\&utm_medium=copylink\&utm_campaign=petition_dashboard\&use_react=false}{demanded
the use} of ``research-proven'' strategies.

In an interview, Professor Calkins decried what she called a feeling of
``animosity and mistrust'' between the camps in the reading wars. She
acknowledged that many teachers needed more training on how to teach
phonics effectively, and said she was working with schools in her
network to provide that.

But she pushed back against a key argument of many phonics activists ---
that there is no downside to all of the children in a classroom getting
the type of repetitive practice in letter-sound relationships that
struggling readers need.

``There's not a chance we're going to want to hold an entire class to
the pace of the 5 percent that have dyslexia,'' she said. ``Other
children need opportunities for comprehension, for writing instruction
and for analytic thinking.''

Wiley Blevins, a phonics expert who considers himself to be in the
center of the reading wars, acknowledged that phonics instruction is
often implemented badly. Texts created to help students practice
sound-letter combinations can be boring and even nonsensical, he said.

Ideally, students in early elementary school would spend about half of
their reading and writing time on phonics, he said, using quality
materials. If this happened consistently, by third grade, most students
would not need explicit phonics anymore.

Even some leading researchers in the science of reading, including
Professor Seidenberg, acknowledge that studies do not yet point toward
specific curriculum materials that will be most effective at teaching
phonics.

``The science that you need to know it is good,'' he said. ``The science
on how to teach it effectively is not.''

Ms. Smith, the Washington kindergarten teacher, has embraced her
school's new focus on phonics, which she said had engaged both
low-achieving and high-achieving students.

She reads to her class each day from beloved children's literature, like
the ``Elephant and Piggie'' series by Mo Willems. But it is the simple
phonics texts, she said, that have done the most to build the students'
confidence, because over time, they are able to accurately read them
aloud to their classmates.

``They will get to the end of the sentence and see a period,'' she said,
``and their face will light up.''

Advertisement

\protect\hyperlink{after-bottom}{Continue reading the main story}

\hypertarget{site-index}{%
\subsection{Site Index}\label{site-index}}

\hypertarget{site-information-navigation}{%
\subsection{Site Information
Navigation}\label{site-information-navigation}}

\begin{itemize}
\tightlist
\item
  \href{https://help.nytimes.com/hc/en-us/articles/115014792127-Copyright-notice}{©~2020~The
  New York Times Company}
\end{itemize}

\begin{itemize}
\tightlist
\item
  \href{https://www.nytco.com/}{NYTCo}
\item
  \href{https://help.nytimes.com/hc/en-us/articles/115015385887-Contact-Us}{Contact
  Us}
\item
  \href{https://www.nytco.com/careers/}{Work with us}
\item
  \href{https://nytmediakit.com/}{Advertise}
\item
  \href{http://www.tbrandstudio.com/}{T Brand Studio}
\item
  \href{https://www.nytimes.com/privacy/cookie-policy\#how-do-i-manage-trackers}{Your
  Ad Choices}
\item
  \href{https://www.nytimes.com/privacy}{Privacy}
\item
  \href{https://help.nytimes.com/hc/en-us/articles/115014893428-Terms-of-service}{Terms
  of Service}
\item
  \href{https://help.nytimes.com/hc/en-us/articles/115014893968-Terms-of-sale}{Terms
  of Sale}
\item
  \href{https://spiderbites.nytimes.com}{Site Map}
\item
  \href{https://help.nytimes.com/hc/en-us}{Help}
\item
  \href{https://www.nytimes.com/subscription?campaignId=37WXW}{Subscriptions}
\end{itemize}
