Sections

SEARCH

\protect\hyperlink{site-content}{Skip to
content}\protect\hyperlink{site-index}{Skip to site index}

\href{https://www.nytimes.com/section/business}{Business}

\href{https://myaccount.nytimes.com/auth/login?response_type=cookie\&client_id=vi}{}

\href{https://www.nytimes.com/section/todayspaper}{Today's Paper}

\href{/section/business}{Business}\textbar{}Pardon Closes the Book on
Michael Milken's Case but Can't Rewrite It

\url{https://nyti.ms/39KsAO7}

\begin{itemize}
\item
\item
\item
\item
\item
\end{itemize}

Advertisement

\protect\hyperlink{after-top}{Continue reading the main story}

Supported by

\protect\hyperlink{after-sponsor}{Continue reading the main story}

Common Sense

\hypertarget{pardon-closes-the-book-on-michael-milkens-case-but-cant-rewrite-it}{%
\section{Pardon Closes the Book on Michael Milken's Case but Can't
Rewrite
It}\label{pardon-closes-the-book-on-michael-milkens-case-but-cant-rewrite-it}}

An alternative narrative portrays Michael Milken, a symbol of 1980s
greed, as a maverick crushed by the establishment. Even money won't make
that a reality.

\includegraphics{https://static01.nyt.com/images/2020/02/18/business/18stewart2/18stewart2-articleLarge.jpg?quality=75\&auto=webp\&disable=upscale}

\href{https://www.nytimes.com/by/james-b-stewart}{\includegraphics{https://static01.nyt.com/images/2018/06/14/multimedia/author-james-b-stewart/author-james-b-stewart-thumbLarge.jpg}}

By \href{https://www.nytimes.com/by/james-b-stewart}{James B. Stewart}

\begin{itemize}
\item
  Published Feb. 18, 2020Updated Feb. 21, 2020
\item
  \begin{itemize}
  \item
  \item
  \item
  \item
  \item
  \end{itemize}
\end{itemize}

By pardoning Michael R. Milken, a potent symbol of the ``greed is good''
1980s and arguably the most significant white-collar criminal of his
generation, President Trump has sent two powerful messages: When it
comes to justice, money counts. And white-collar crime doesn't really
matter.

So much for the rule of law, already under siege by the Trump
administration, and the notion that no one, no matter how rich or
powerful, is above it.

Lest history be entirely rewritten, it's worth considering what Judge
Kimba M. Wood
\href{https://www.nytimes.com/1990/11/22/business/milken-sentence-excerpts-judge-wood-s-explanation-milken-sentencing.html}{told
Mr. Milken} at his sentencing on Nov. 21, 1990, on charges including
conspiracy and fraud:

``When a man of your power in the financial world, at the head of the
most important department of one of the most important investment
banking houses in this country, repeatedly conspires to violate, and
violates, securities and tax laws in order to achieve more power and
wealth for himself and his wealthy clients, and commits financial crimes
that are particularly hard to detect, a significant prison term is
required in order to deter others.''

She added that Mr. Milken, who was a senior executive at Drexel Burnham
Lambert, had committed ``serious crimes warranting serious punishment
and the discomfort and opprobrium of being removed from society.''

Mr. Milken, advised by a team of the country's most experienced and
expensive lawyers, pleaded guilty rather than face a trial on even more
expansive charges. Contrary to subsequent myth, he was not charged
because he championed junk bonds. He was not charged because the
savings-and-loan industry all but collapsed (though Mr. Milken's
junk-bond dealings played a direct role in the collapse of some
institutions). He was not charged because of the resulting recession,
which cost millions of people their jobs. Rather, he was charged so that
``our financial markets in which so many people who are not rich invest
their savings'' can be ``free of secret manipulation,'' Judge Wood said
at his sentencing.

Mr. Milken fainted outside the courtroom after she
\href{https://www.nytimes.com/1990/11/22/business/the-milken-sentence-milken-gets-10-years-for-wall-st-crimes.html}{imposed
a 10-year prison term}.

Mr. Milken's transgressions didn't end with his guilty plea and
imprisonment. Released from federal custody
\href{https://www.nytimes.com/1993/03/04/business/milken-freed-but-burdened.html}{two
years into his term} and with a diagnosis of prostate cancer, he faced a
lifetime ban on deal making. That didn't stop him from negotiating CNN's
\$7.5 billion sale to Time Warner in 1996 on behalf of his old friend
and client Ted Turner, for which Mr. Milken collected a \$50 million
fee, and working for another friend and client, the billionaire Ronald
O. Perelman. In 1998, Mr. Milken
\href{https://www.nytimes.com/1998/02/27/business/milken-settles-sec-complaint-for-47-million.html}{agreed
to pay} \$47 million to settle a Securities and Exchange Commission
complaint that he had violated the ban --- he neither admitted nor
denied the allegations --- and the government dropped a criminal
investigation into his activities after his release.

Mr. Milken's wealthy and powerful friends have been clamoring for a
pardon for years on his behalf, but the prospect seemed remote until Mr.
Trump's election. Even Bill Clinton, who as president found a
\href{https://www.nytimes.com/2001/01/24/us/influential-backers-helped-commodities-trader-win-pardon.html}{justification
to pardon} the notorious commodities trader and tax evader Marc Rich,
balked at granting Mr. Milken a pardon.

\includegraphics{https://static01.nyt.com/images/2020/02/18/business/18stewart4/18stewart4-articleLarge.jpg?quality=75\&auto=webp\&disable=upscale}

Until Mr. Trump's move was announced Tuesday, I had hoped to have
written the last about Mr. Milken. He was a major character in my book
``Den of Thieves,'' which chronicles the rise and fall of Mr. Milken and
his co-conspirator Ivan F. Boesky, the takeover speculator and model for
the Gordon Gekko character in the ``Wall Street'' movies. (As someone
who incriminated Mr. Milken and cooperated with the government, Mr.
Boesky seems to have little chance of a pardon of his own from Mr.
Trump.)

After the book was published in 1991, one of Mr. Milken's former
lawyers, Michael Armstrong, sued me, my research assistant and my
publisher, claiming \$35 million in damages in a case financed by Mr.
Milken and his brother. (We won a
\href{https://www.nytimes.com/1999/09/24/business/milken-supported-libel-suit-against-a-writer-is-dismissed.html}{resounding
victory}, albeit after nearly a decade of costly litigation.) I returned
to the subject of Mr. Milken in an article for The New Yorker about his
post-prison deal making while that case was pending.

But since then, Mr. Milken appears to have focused on nurturing his vast
wealth (estimated to be in the billions of dollars even after he paid
his \$600 million fine) and devoting himself to reputation-enhancing
charitable pursuits, ably chronicled by other reporters. The 1998 S.E.C.
complaint and the threat of a return to prison seem to have worked, and
so far as I'm aware, Mr. Milken has avoided the siren call of deal
making for others. He deserves credit for his impressive record of good
works.

While none of that warrants a presidential pardon, it's not hard to
fathom why Mr. Milken's saga would resonate with Mr. Trump.

Like the president, Mr. Milken studied business at the Wharton School of
the University of Pennsylvania but was largely shunned by New York's
elite.

Mr. Milken's early clients were corporate raiders who, like Mr. Trump,
were disdained by establishment firms like Goldman Sachs and Morgan
Stanley. Mr. Milken and his junk-bond-fueled takeovers were seen as
disruptive forces, threats to a complacent status quo on Wall Street and
in corporate America, just as Mr. Trump has upended Washington.

And of course Mr. Milken underwent years of distracting investigations
and related bad publicity. He was even represented for a time by Mr.
Trump's celebrity lawyer Alan Dershowitz (who at one point attacked me
in an advertisement in The New York Times). In one of his many startling
about-faces, the Trump lawyer Rudolph W. Giuliani went from being Mr.
Milken's principal accuser and the architect of his plea deal as U.S.
attorney to a Milken champion and advocate for a pardon.

Seen as an underdog, even a very wealthy and well-connected one, Mr.
Milken has long inspired a counternarrative that he was a victim of a
media and Wall Street establishment jealous of his wealth and success.
However unfounded in fact, that version of reality has now gotten a
presidential stamp of approval.

Advertisement

\protect\hyperlink{after-bottom}{Continue reading the main story}

\hypertarget{site-index}{%
\subsection{Site Index}\label{site-index}}

\hypertarget{site-information-navigation}{%
\subsection{Site Information
Navigation}\label{site-information-navigation}}

\begin{itemize}
\tightlist
\item
  \href{https://help.nytimes.com/hc/en-us/articles/115014792127-Copyright-notice}{©~2020~The
  New York Times Company}
\end{itemize}

\begin{itemize}
\tightlist
\item
  \href{https://www.nytco.com/}{NYTCo}
\item
  \href{https://help.nytimes.com/hc/en-us/articles/115015385887-Contact-Us}{Contact
  Us}
\item
  \href{https://www.nytco.com/careers/}{Work with us}
\item
  \href{https://nytmediakit.com/}{Advertise}
\item
  \href{http://www.tbrandstudio.com/}{T Brand Studio}
\item
  \href{https://www.nytimes.com/privacy/cookie-policy\#how-do-i-manage-trackers}{Your
  Ad Choices}
\item
  \href{https://www.nytimes.com/privacy}{Privacy}
\item
  \href{https://help.nytimes.com/hc/en-us/articles/115014893428-Terms-of-service}{Terms
  of Service}
\item
  \href{https://help.nytimes.com/hc/en-us/articles/115014893968-Terms-of-sale}{Terms
  of Sale}
\item
  \href{https://spiderbites.nytimes.com}{Site Map}
\item
  \href{https://help.nytimes.com/hc/en-us}{Help}
\item
  \href{https://www.nytimes.com/subscription?campaignId=37WXW}{Subscriptions}
\end{itemize}
