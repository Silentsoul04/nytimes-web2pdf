Sections

SEARCH

\protect\hyperlink{site-content}{Skip to
content}\protect\hyperlink{site-index}{Skip to site index}

\href{https://www.nytimes.com/section/politics}{Politics}

\href{https://myaccount.nytimes.com/auth/login?response_type=cookie\&client_id=vi}{}

\href{https://www.nytimes.com/section/todayspaper}{Today's Paper}

\href{/section/politics}{Politics}\textbar{}Trump Grants Clemency to
Blagojevich, Milken and Kerik

\url{https://nyti.ms/2V32xgU}

\begin{itemize}
\item
\item
\item
\item
\item
\item
\end{itemize}

Advertisement

\protect\hyperlink{after-top}{Continue reading the main story}

Supported by

\protect\hyperlink{after-sponsor}{Continue reading the main story}

\hypertarget{trump-grants-clemency-to-blagojevich-milken-and-kerik}{%
\section{Trump Grants Clemency to Blagojevich, Milken and
Kerik}\label{trump-grants-clemency-to-blagojevich-milken-and-kerik}}

The president also pardoned or commuted the sentences of eight others on
Tuesday, including Edward DeBartolo, a former owner of the San Francisco
49ers.

\includegraphics{https://static01.nyt.com/images/2020/02/22/us/18dc-pardon-sub/18dc-pardon-sub-articleLarge-v2.jpg?quality=75\&auto=webp\&disable=upscale}

\href{https://www.nytimes.com/by/michael-d-shear}{\includegraphics{https://static01.nyt.com/images/2018/06/13/multimedia/author-michael-d-shear/author-michael-d-shear-thumbLarge-v2.png}}\href{https://www.nytimes.com/by/maggie-haberman}{\includegraphics{https://static01.nyt.com/images/2018/07/12/multimedia/author-maggie-haberman/author-maggie-haberman-thumbLarge.png}}

By \href{https://www.nytimes.com/by/michael-d-shear}{Michael D. Shear}
and \href{https://www.nytimes.com/by/maggie-haberman}{Maggie Haberman}

\begin{itemize}
\item
  Published Feb. 18, 2020Updated Feb. 19, 2020
\item
  \begin{itemize}
  \item
  \item
  \item
  \item
  \item
  \item
  \end{itemize}
\end{itemize}

WASHINGTON --- President Trump, citing what he said was advice from
friends and business associates, granted clemency on Tuesday to a who's
who of white-collar criminals from politics, sports and business who
were convicted on charges involving fraud, corruption and lies ---
including the financier
\href{https://www.nytimes.com/2020/02/18/business/michael-milken-case-lessons.html}{Michael
R. Milken}.

The president pardoned Mr. Milken, the so-called junk bond king of the
1980s, as well as the former New York City police commissioner Bernard
B. Kerik and Edward J. DeBartolo Jr., a former owner of the San
Francisco 49ers. He also commuted the sentence of
\href{https://www.nytimes.com/2020/02/19/us/rod-blagojevich-chicago.html}{Rod
R. Blagojevich}, a former Democratic governor of Illinois.

Their political and finance schemes made them household names, and three
received prison terms while Mr. DeBartolo paid a \$1 million fine.

Mr. Trump also pardoned David Safavian, the **** top federal procurement
official under President George W. Bush, who had been sentenced in 2009
to a year in prison for lying about his ties to the lobbyist Jack
Abramoff and obstructing the sprawling investigation into Mr. Abramoff's
efforts to win federal business. The president also granted clemency to
six other people.

Mr. Trump has repeatedly stated his commitment to a criminal justice
overhaul and addressing the excessive sentences given to minorities. At
the urging of Kim Kardashian West in 2018, he pardoned Alice Marie
Johnson, a 63-year-old African-American woman serving a life sentence
for a nonviolent drug conviction. Ms. Johnson was the
\href{https://www.nytimes.com/2020/02/06/us/politics/alice-johnson-trump-super-bowl-ad.html}{centerpiece
of a TV ad} the Trump campaign ran this month during the Super Bowl.

But the president's announcements on Tuesday were mostly aimed at wiping
clean the slates of rich, powerful and well-connected white men. And
they came after years of sophisticated public relations campaigns aimed
at persuading Mr. Trump to exercise the authority given to him under the
Constitution.

Patti Blagojevich, the wife of the former Illinois governor, frequently
appeared on Fox News calling for Mr. Trump to commute her husband's
sentence. Mr. Kerik, a regular on Fox News, appeared on the network as
recently as Monday night. Mr. Milken has sought to rebrand himself as a
philanthropist in recent years as allies campaigned on his behalf for a
pardon.

In conversations with his advisers, Mr. Trump has also raised the
prospect of commuting the sentence of Roger J. Stone Jr., his longtime
adviser, who was convicted in November of seven felony charges,
including tampering with a witness and lying under oath in order to
obstruct a congressional inquiry into whether the Trump campaign
conspired with Russia to influence the 2016 election.

Asked about a pardon for Mr. Stone on Tuesday, Mr. Trump insisted that
``I haven't given it any thought.''

Democrats pounced on the president's announcements.

``Today, Trump granted clemency to tax cheats, Wall Street crooks,
billionaires and corrupt government officials,'' said Senator Bernie
Sanders, independent of Vermont, the leading Democratic candidate for
president. ``Meanwhile, thousands of poor and working-class kids sit in
jail for nonviolent drug convictions. This is what a broken and racist
criminal justice system looks like.''

Representative Bill Pascrell Jr., Democrat of New Jersey, said in a
statement that the president abused the pardon power by using it to
reward friends and repair the reputations of felons who do not deserve
it.

``The pardoning of these disgraced figures should be treated as another
national scandal by a lawless executive,'' he said.

But Mr. Trump defended his grants of clemency on Tuesday.

He was particularly critical of the 14-year prison sentence for Mr.
Blagojevich, who was convicted of trying when he was governor of
Illinois to essentially sell the Senate seat vacated by Barack Obama
when he became president. Mr. Blagojevich also once appeared on the
reality TV series ``The Celebrity Apprentice,'' which Mr. Trump hosted.

``That was a tremendously powerful, ridiculous sentence, in my
opinion,'' Mr. Trump said after announcing that Mr. Blagojevich would go
free after serving eight years. The president alleged that the former
governor was a victim of the same forces that investigated him for
years, citing James B. Comey, the former F.B.I. director, and Patrick
Fitzgerald, the U.S. attorney in Chicago who prosecuted Mr. Blagojevich.

``It was a prosecution by the same people --- Comey, Fitzpatrick, the
same group,'' Mr. Trump told reporters, misstating Mr. Fitzgerald's
name.

Mr. Trump gave no indication that he relied on the usual vetting process
that guides presidents making use of their constitutional authority to
wipe away criminal convictions or commute prison sentences.

Traditionally, the Justice Department's pardons office would make
recommendations about pardons and commutations to the deputy attorney
general, who would weigh in and then pass the department's final
determinations to the White House. Instead, Mr. Trump told reporters
that he followed ``recommendations'' in making his decisions.

Those recommendations, according to a White House statement, came from
the president's longtime friends, business executives, celebrities,
campaign donors, sports figures and political allies.

In pardoning Mr. Kerik, who pleaded guilty to tax fraud and lying to the
government, Mr. Trump said he heard from more than a dozen people,
including Rudolph W. Giuliani, the former New York mayor and Mr. Trump's
personal lawyer; Geraldo Rivera, a Fox TV personality; and Eddie
Gallagher, a former Navy SEAL and accused war criminal whose
\href{https://www.nytimes.com/2019/11/21/us/trump-seals-eddie-gallagher.html}{demotion
was overturned} by Mr. Trump last year.

Mr. Kerik had a pardon application pending and Mr. Blagojevich had a
commutation application pending, but a source close to the pardons
office did not believe that the pardon attorney had given either of
those applications full-throated support.

Mr. Milken, whose dealings contributed to the collapse of the
savings-and-loan industry, fought for decades to reverse his conviction
for securities fraud. Richard LeFrak, a billionaire real estate magnate
and longtime friend; Sheldon G. Adelson, a prominent Republican donor;
and Nelson Peltz, a billionaire investor who hosted a \$10 million
fund-raiser for the president's 2020 campaign on Saturday, were among
those who suggested that Mr. Trump pardon him.

Mr. Milken did not have a pardon or commutation application pending at
the Justice Department's pardons office, meaning that the president made
that decision entirely without official Justice Department input. Two
previous applications had been denied and closed.

The football greats Jerry Rice and Joe Montana --- as well as the
singer-songwriter Paul Anka --- urged the president to pardon Mr.
DeBartolo, who
\href{https://www.nytimes.com/1998/10/07/us/owner-of-nfl-team-ties-ex-governor-to-extortion.html}{pleaded
guilty in 1998} to concealing an extortion attempt. Mr. DeBartolo
avoided prison but was fined \$1 million and was
\href{http://www.nytimes.com/1999/03/17/sports/pro-football-nfl-picks-los-angeles-with-conditions.html}{suspended
for a year} by the N.F.L. He later handed over the 49ers to his sister
Denise DeBartolo York.

\includegraphics{https://static01.nyt.com/images/2020/02/20/us/politics/18dc-pardon-video/18dc-pardon-video-videoSixteenByNine3000.jpg}

``You have to see the recommendations,'' Mr. Trump said on Tuesday
before boarding Air Force One for a four-day trip to the West Coast,
where he was scheduled to hold three campaign rallies. ``I rely on
recommendations.''

Previous presidents have often waited until the final moments of their
presidencies to wield the pardon power on behalf of their friends.
President Bill Clinton pardoned Marc Rich, a hedge fund manager and
financier who was convicted of tax evasion and other crimes, on Jan. 20,
2001, Mr. Clinton's last day in office.

Others, including Mr. Bush and Mr. Obama, largely reserved their
clemency authority for people convicted of nonviolent, low-level drug
crimes and other offenses who were identified as part of a rigorous
process run by a team of Justice Department lawyers.

Mr. Trump, however, has shrugged off those traditions and the
controversy that sometimes comes with the use of the pardon power. He
issued a ``full and unconditional pardon'' to Joe Arpaio, the Arizona
sheriff and immigration hard-liner convicted of contempt of court, in
August 2017.

Less than a year later, he did the same for I. Lewis Libby Jr., a former
aide to Mr. Bush who was convicted of obstructing justice and perjury.

In addition to helping erase the convictions of the well connected and
powerful, Mr. Trump also pardoned on Tuesday a tech executive who
pleaded guilty to conspiracy, the owner of a construction company who
underpaid his taxes and a woman convicted of stealing cars. The
president also commuted the sentences of a woman convicted of drug
distribution, another woman who was part of a marijuana smuggling ring
and a minority-owner of a health care company who was sentenced to 35
years for a scheme to defraud the government.

Their relative anonymity was a sharp contrast to the prominence of the
four men highlighted by the president.

Mr. Milken, who was credited in the 1980s with using junk bonds to
finance big debt-laden corporate buyouts, pleaded guilty to securities
reporting violations and tax offenses, and the Securities and Exchange
Commission banned him for life. The investigation came to highlight that
decade's corporate excesses on Wall Street.

In the years since his conviction, Mr. Milken has emerged as a major
cancer philanthropist and is the founder of the Milken Institute, a
nonpartisan think tank that holds a popular conference in Los Angeles
that convenes the world's most powerful people in government, industry
and finance.

Mr. Kerik, a police detective, served as Mr. Giuliani's bodyguard and
chauffeur during the 1993 mayoral race and later served in a series of
high-ranking positions in New York City's Department of Correction.
Eventually, Mr. Giuliani named Mr. Kerik correction commissioner in 1997
and police commissioner in 2000.

In 2004, Mr. Kerik's bid to become homeland security secretary in the
Bush cabinet collapsed amid scandals. In June 2006,
\href{https://www.nytimes.com/2006/07/01/nyregion/01kerik.html}{he
pleaded guilty in State Supreme Court in the Bronx to two
misdemeanors}tied to renovations done on his apartment. Four years
later, Mr. Kerik pleaded guilty to tax fraud and making false
statements.

\includegraphics{https://static01.nyt.com/images/2020/02/18/us/politics/18dc-pardon2/merlin_169098033_5b923415-48fb-4d7e-bef9-edb3c7b56425-articleLarge.jpg?quality=75\&auto=webp\&disable=upscale}

Mr. DeBartolo presided over the golden era of the 49ers when the team
won five Super Bowl championships under the coach Bill Walsh, with
legendary players like Mr. Montana, Steve Young, Ronnie Lott and Mr.
Rice. Mr. DeBartolo was elected to the Pro Football Hall of Fame in 2016
despite his conviction.

But in the late 1990s, Mr. DeBartolo was an investor in the Hollywood
Casino Corp., a Dallas company seeking permission for a riverboat casino
in Louisiana. On March 12, 1997, he met Edwin W. Edwards, the
influential former governor of Louisiana, for lunch in California and
handed over \$400,000 that Mr. Edwards had demanded for his help in
securing a license. The next day, the Gaming Board granted it. A month
later, federal agents raided Mr. Edwards's house and office, seizing the
\$400,000.

``Why do it? It actually was just plain stupidity, and I should have
just walked away from it,''
\href{https://www.cleveland.com/browns/2013/04/youngstowns_eddie_debartolo_a.html}{Mr.
DeBartolo told NFL Films} for a biographical documentary in 2012. ``I
was as much to blame because I was old enough to know better and too
stupid to do anything about it.''

Michael D. Shear reported from Washington, and Maggie Haberman from New
York. Reporting was contributed by Peter Baker and Katie Benner from
Washington, and Ed Shanahan, Matthew Goldstein and Jesse Drucker from
New York.

Advertisement

\protect\hyperlink{after-bottom}{Continue reading the main story}

\hypertarget{site-index}{%
\subsection{Site Index}\label{site-index}}

\hypertarget{site-information-navigation}{%
\subsection{Site Information
Navigation}\label{site-information-navigation}}

\begin{itemize}
\tightlist
\item
  \href{https://help.nytimes.com/hc/en-us/articles/115014792127-Copyright-notice}{©~2020~The
  New York Times Company}
\end{itemize}

\begin{itemize}
\tightlist
\item
  \href{https://www.nytco.com/}{NYTCo}
\item
  \href{https://help.nytimes.com/hc/en-us/articles/115015385887-Contact-Us}{Contact
  Us}
\item
  \href{https://www.nytco.com/careers/}{Work with us}
\item
  \href{https://nytmediakit.com/}{Advertise}
\item
  \href{http://www.tbrandstudio.com/}{T Brand Studio}
\item
  \href{https://www.nytimes.com/privacy/cookie-policy\#how-do-i-manage-trackers}{Your
  Ad Choices}
\item
  \href{https://www.nytimes.com/privacy}{Privacy}
\item
  \href{https://help.nytimes.com/hc/en-us/articles/115014893428-Terms-of-service}{Terms
  of Service}
\item
  \href{https://help.nytimes.com/hc/en-us/articles/115014893968-Terms-of-sale}{Terms
  of Sale}
\item
  \href{https://spiderbites.nytimes.com}{Site Map}
\item
  \href{https://help.nytimes.com/hc/en-us}{Help}
\item
  \href{https://www.nytimes.com/subscription?campaignId=37WXW}{Subscriptions}
\end{itemize}
