Sections

SEARCH

\protect\hyperlink{site-content}{Skip to
content}\protect\hyperlink{site-index}{Skip to site index}

\href{https://www.nytimes.com/section/world/asia}{Asia Pacific}

\href{https://myaccount.nytimes.com/auth/login?response_type=cookie\&client_id=vi}{}

\href{https://www.nytimes.com/section/todayspaper}{Today's Paper}

\href{/section/world/asia}{Asia Pacific}\textbar{}Shadowy Church Is at
Center of Coronavirus Outbreak in South Korea

\url{https://nyti.ms/2vSsFka}

\begin{itemize}
\item
\item
\item
\item
\item
\item
\end{itemize}

\href{https://www.nytimes.com/news-event/coronavirus?action=click\&pgtype=Article\&state=default\&region=TOP_BANNER\&context=storylines_menu}{The
Coronavirus Outbreak}

\begin{itemize}
\tightlist
\item
  live\href{https://www.nytimes.com/2020/08/02/world/coronavirus-updates.html?action=click\&pgtype=Article\&state=default\&region=TOP_BANNER\&context=storylines_menu}{Latest
  Updates}
\item
  \href{https://www.nytimes.com/interactive/2020/us/coronavirus-us-cases.html?action=click\&pgtype=Article\&state=default\&region=TOP_BANNER\&context=storylines_menu}{Maps
  and Cases}
\item
  \href{https://www.nytimes.com/interactive/2020/science/coronavirus-vaccine-tracker.html?action=click\&pgtype=Article\&state=default\&region=TOP_BANNER\&context=storylines_menu}{Vaccine
  Tracker}
\item
  \href{https://www.nytimes.com/interactive/2020/07/29/us/schools-reopening-coronavirus.html?action=click\&pgtype=Article\&state=default\&region=TOP_BANNER\&context=storylines_menu}{What
  School May Look Like}
\item
  \href{https://www.nytimes.com/live/2020/07/31/business/stock-market-today-coronavirus?action=click\&pgtype=Article\&state=default\&region=TOP_BANNER\&context=storylines_menu}{Economy}
\end{itemize}

Advertisement

\protect\hyperlink{after-top}{Continue reading the main story}

Supported by

\protect\hyperlink{after-sponsor}{Continue reading the main story}

\hypertarget{shadowy-church-is-at-center-of-coronavirus-outbreak-in-south-korea}{%
\section{Shadowy Church Is at Center of Coronavirus Outbreak in South
Korea}\label{shadowy-church-is-at-center-of-coronavirus-outbreak-in-south-korea}}

As the country's infection numbers soar, most cases have been connected
to the Shincheonji Church of Jesus, which mainstream churches consider a
cult.

\includegraphics{https://static01.nyt.com/images/2020/02/21/world/21skorea-virus-1/merlin_169260762_6ee1bed5-1fea-4189-b0ea-30a9c09a412e-articleLarge.jpg?quality=75\&auto=webp\&disable=upscale}

\href{https://www.nytimes.com/by/choe-sang-hun}{\includegraphics{https://static01.nyt.com/images/2018/07/18/multimedia/author-choe-sang-hun/author-choe-sang-hun-thumbLarge.png}}

By \href{https://www.nytimes.com/by/choe-sang-hun}{Choe Sang-Hun}

\begin{itemize}
\item
  Published Feb. 21, 2020Updated May 22, 2020
\item
  \begin{itemize}
  \item
  \item
  \item
  \item
  \item
  \item
  \end{itemize}
\end{itemize}

\href{http://www.nytimes.com/2020/03/04/world/coronavirus-news.html}{\emph{Read
live updates on the coronavirus outbreak here.}}

SEOUL, South Korea --- At meetings of the secretive Shincheonji Church
of Jesus, worshipers sit packed together on the floor, wearing no
glasses --- or face masks. They come to church even when sick, former
members say. After services, they split up into groups for Bible study,
or to go out into the streets and proselytize.

After the first coronavirus infection was reported among its members,
they were told to lie about being followers, though the church later
said that was not its policy.

Now, health officials are zeroing in on the church's practices as they
seek to contain South Korea's alarming coronavirus outbreak, in which
members of Shincheonji, along with their relatives and others who got
the virus from them, account for more than half of the confirmed
infections. On Saturday, the number of cases in the country soared to
346 --- second only to mainland China, if the outbreak on
\href{https://www.nytimes.com/2020/02/19/world/asia/japan-cruise-ship-coronavirus.html}{the
Diamond Princess cruise ship} is excluded from Japan's count.

More than 1,250 other church members have reported potential symptoms,
health officials said, raising the possibility that the nation's
caseload
\href{http://www.nytimes.com/2020/02/25/world/asia/daegu-south-korea-coronavirus.html}{could
soon skyrocket further}. In response, the government is shutting down
thousands of day-care facilities, nursing homes and community centers,
even banning the outdoor political rallies that are a feature of life in
downtown Seoul.

\emph{{[}Read:}
\href{http://www.nytimes.com/2020/02/27/world/asia/coronavirus-south-korea.html}{\emph{South
Korean leader said coronavirus would `disappear.' It was a costly
error}}\emph{.{]}}

As of Saturday, more than 700 members of Shincheonji, which mainstream
South Korean churches consider a cult, still could not be reached,
according to health officials, who were frantically hoping to screen
them for signs of infection.

\includegraphics{https://static01.nyt.com/images/2020/02/21/world/21skorea-virus-3/merlin_169259415_819f4c49-db00-40ab-814d-051fc557b70c-articleLarge.jpg?quality=75\&auto=webp\&disable=upscale}

``Shincheonji members know of their bad image and they usually hide
their affiliation from nonchurch members, even from their parents,''
said Hwang Eui-jong, a pastor who has researched the church. ``No wonder
many of them are unreachable. They must be huddled together somewhere,
praying that this will eventually go away.''

The snowballing outbreak among the church's followers is testing South
Korea's health care system, which successfully tamed a deadly outbreak
of Middle East respiratory syndrome in 2015. Experts on South Korean
religious sects and former members of the church said its practices made
its members unusually vulnerable to contagious diseases.

``Unlike other churches, Shincheonji makes its members sit on the floor
tightly together during services, in neat, militarylike ranks and
files,'' said Lee Ho-yeon, who left the church in 2015. ``We were not
supposed to have anything on our faces, like glasses or masks. We were
trained to sing our hymns loudly.''

``We were taught not to be afraid of illness,'' Ms. Lee said. ``We were
taught not to care about such worldly things like jobs, ambition or
passion. Everything was focused on proselytizing, even when we were
sick.''

\hypertarget{latest-updates-global-coronavirus-outbreak}{%
\section{\texorpdfstring{\href{https://www.nytimes.com/2020/08/01/world/coronavirus-covid-19.html?action=click\&pgtype=Article\&state=default\&region=MAIN_CONTENT_1\&context=storylines_live_updates}{Latest
Updates: Global Coronavirus
Outbreak}}{Latest Updates: Global Coronavirus Outbreak}}\label{latest-updates-global-coronavirus-outbreak}}

Updated 2020-08-02T17:52:35.962Z

\begin{itemize}
\tightlist
\item
  \href{https://www.nytimes.com/2020/08/01/world/coronavirus-covid-19.html?action=click\&pgtype=Article\&state=default\&region=MAIN_CONTENT_1\&context=storylines_live_updates\#link-34047410}{The
  U.S. reels as July cases more than double the total of any other
  month.}
\item
  \href{https://www.nytimes.com/2020/08/01/world/coronavirus-covid-19.html?action=click\&pgtype=Article\&state=default\&region=MAIN_CONTENT_1\&context=storylines_live_updates\#link-780ec966}{Top
  U.S. officials work to break an impasse over the federal jobless
  benefit.}
\item
  \href{https://www.nytimes.com/2020/08/01/world/coronavirus-covid-19.html?action=click\&pgtype=Article\&state=default\&region=MAIN_CONTENT_1\&context=storylines_live_updates\#link-2bc8948}{Its
  outbreak untamed, Melbourne goes into even greater lockdown.}
\end{itemize}

\href{https://www.nytimes.com/2020/08/01/world/coronavirus-covid-19.html?action=click\&pgtype=Article\&state=default\&region=MAIN_CONTENT_1\&context=storylines_live_updates}{See
more updates}

More live coverage:
\href{https://www.nytimes.com/live/2020/07/31/business/stock-market-today-coronavirus?action=click\&pgtype=Article\&state=default\&region=MAIN_CONTENT_1\&context=storylines_live_updates}{Markets}

The outbreak has struck hardest at Shincheonji's church in Daegu, a city
of about 2.5 million in the country's southeast, where a 61-year-old
woman known as Patient No. 31 is believed to be a link between many of
the cases. The restrictions on public gatherings have been implemented
more forcefully in Daegu than elsewhere in the country.

\emph{{[}Update:}
\href{http://www.nytimes.com/2020/02/27/world/asia/coronavirus-treament-recovery.html}{\emph{Most
coronavirus cases are mild. That's good and bad news}}\emph{.{]}}

Patient No. 31 checked into a small Daegu hospital on Feb. 7, after a
minor traffic accident. The next day, she complained of a sore throat.
The day after that --- a Sunday --- she attended a Shincheonji church
service, health officials say.

She developed a fever the next day, one that lingered, and she stayed in
the hospital. Still, she slipped out the following Sunday to go to
church again. At least 1,000 Shincheonji members attended one of those
two **** Sunday services, officials said.

At least twice, doctors recommended that the woman transfer to a bigger
hospital to be tested for the coronavirus, but she refused, health
officials said. She insisted that she had not visited China in recent
months, nor had she met anyone known to have the virus.

Finally, on Monday, she felt sick enough to check into a government-run
clinic for a coronavirus test. On Tuesday, she was confirmed to be
infected.

``Her behavior is not surprising to people familiar with the church,''
said Chung Yun-seok, an expert on religious cults who runs the website
Christian Portal News. ``To them, getting sick is a sin because it
prevents them from doing God's work.''

Image

People suspected to have the coronavirus talked with medical staff
workers in Daegu on Friday.Credit...Kim Hyun-Tae/Yonhap, via Associated
Press

The church dismissed criticism of its practices on Friday, calling it
``slandering based on the prejudices among the established churches.''
It said its members sat close together on the floor because local
authorities would not give it permits to build bigger churches.

Health officials were still trying to figure out how Patient No. 31
contracted the disease. Mr. Hwang noted that the church had been
proselytizing among ethnic Koreans in northeastern China, many of whom
it invited to South Korea.

Jung Eun-kyeong, director of the Korea Centers for Disease Control and
Prevention, said the authorities were investigating reports that
Shincheonji had operations in Hubei, the Chinese province that includes
Wuhan, where the virus emerged. The South Korean news agency Newsis
reported on Friday that Shincheonji had opened a church in Wuhan last
year, and that references to it had been removed from the church's
website. Church officials could not immediately be reached for comment.

South Korean officials learned that Patient No. 31 had visited Cheongdo,
a county near Daegu, in early February. As of Saturday, 108 patients and
medical staff at a Cheongdo hospital had tested positive for the
coronavirus; two of them died this week.

Cheongdo is the birthplace of Lee Man-hee, the 88-year-old self-styled
messiah who founded Shincheonji, and followers regularly go on
pilgrimages there and do volunteer work. Church members are also
believed to have attended the funeral of Mr. Lee's brother in Cheongdo
in early February.

On Friday, Newsis quoted Patient No. 31 as saying that she had not
visited the hospital or attended the funeral, but that she had used a
public bathhouse while in Cheongdo.

``We need a thorough investigation of the people who attended the church
services and the funeral,'' President Moon Jae-in said on Friday while
presiding over an emergency meeting on the outbreak.

After the case of Patient No. 31 was first reported, social media
messages went out to Shincheonji members telling them to continue their
evangelical work in small groups. The messages also told members that if
officials asked, they should deny that they belonged to the church or
went to its services.

\href{https://www.nytimes.com/news-event/coronavirus?action=click\&pgtype=Article\&state=default\&region=MAIN_CONTENT_3\&context=storylines_faq}{}

\hypertarget{the-coronavirus-outbreak-}{%
\subsubsection{The Coronavirus Outbreak
›}\label{the-coronavirus-outbreak-}}

\hypertarget{frequently-asked-questions}{%
\paragraph{Frequently Asked
Questions}\label{frequently-asked-questions}}

Updated July 27, 2020

\begin{itemize}
\item ~
  \hypertarget{should-i-refinance-my-mortgage}{%
  \paragraph{Should I refinance my
  mortgage?}\label{should-i-refinance-my-mortgage}}

  \begin{itemize}
  \tightlist
  \item
    \href{https://www.nytimes.com/article/coronavirus-money-unemployment.html?action=click\&pgtype=Article\&state=default\&region=MAIN_CONTENT_3\&context=storylines_faq}{It
    could be a good idea,} because mortgage rates have
    \href{https://www.nytimes.com/2020/07/16/business/mortgage-rates-below-3-percent.html?action=click\&pgtype=Article\&state=default\&region=MAIN_CONTENT_3\&context=storylines_faq}{never
    been lower.} Refinancing requests have pushed mortgage applications
    to some of the highest levels since 2008, so be prepared to get in
    line. But defaults are also up, so if you're thinking about buying a
    home, be aware that some lenders have tightened their standards.
  \end{itemize}
\item ~
  \hypertarget{what-is-school-going-to-look-like-in-september}{%
  \paragraph{What is school going to look like in
  September?}\label{what-is-school-going-to-look-like-in-september}}

  \begin{itemize}
  \tightlist
  \item
    It is unlikely that many schools will return to a normal schedule
    this fall, requiring the grind of
    \href{https://www.nytimes.com/2020/06/05/us/coronavirus-education-lost-learning.html?action=click\&pgtype=Article\&state=default\&region=MAIN_CONTENT_3\&context=storylines_faq}{online
    learning},
    \href{https://www.nytimes.com/2020/05/29/us/coronavirus-child-care-centers.html?action=click\&pgtype=Article\&state=default\&region=MAIN_CONTENT_3\&context=storylines_faq}{makeshift
    child care} and
    \href{https://www.nytimes.com/2020/06/03/business/economy/coronavirus-working-women.html?action=click\&pgtype=Article\&state=default\&region=MAIN_CONTENT_3\&context=storylines_faq}{stunted
    workdays} to continue. California's two largest public school
    districts --- Los Angeles and San Diego --- said on July 13, that
    \href{https://www.nytimes.com/2020/07/13/us/lausd-san-diego-school-reopening.html?action=click\&pgtype=Article\&state=default\&region=MAIN_CONTENT_3\&context=storylines_faq}{instruction
    will be remote-only in the fall}, citing concerns that surging
    coronavirus infections in their areas pose too dire a risk for
    students and teachers. Together, the two districts enroll some
    825,000 students. They are the largest in the country so far to
    abandon plans for even a partial physical return to classrooms when
    they reopen in August. For other districts, the solution won't be an
    all-or-nothing approach.
    \href{https://bioethics.jhu.edu/research-and-outreach/projects/eschool-initiative/school-policy-tracker/}{Many
    systems}, including the nation's largest, New York City, are
    devising
    \href{https://www.nytimes.com/2020/06/26/us/coronavirus-schools-reopen-fall.html?action=click\&pgtype=Article\&state=default\&region=MAIN_CONTENT_3\&context=storylines_faq}{hybrid
    plans} that involve spending some days in classrooms and other days
    online. There's no national policy on this yet, so check with your
    municipal school system regularly to see what is happening in your
    community.
  \end{itemize}
\item ~
  \hypertarget{is-the-coronavirus-airborne}{%
  \paragraph{Is the coronavirus
  airborne?}\label{is-the-coronavirus-airborne}}

  \begin{itemize}
  \tightlist
  \item
    The coronavirus
    \href{https://www.nytimes.com/2020/07/04/health/239-experts-with-one-big-claim-the-coronavirus-is-airborne.html?action=click\&pgtype=Article\&state=default\&region=MAIN_CONTENT_3\&context=storylines_faq}{can
    stay aloft for hours in tiny droplets in stagnant air}, infecting
    people as they inhale, mounting scientific evidence suggests. This
    risk is highest in crowded indoor spaces with poor ventilation, and
    may help explain super-spreading events reported in meatpacking
    plants, churches and restaurants.
    \href{https://www.nytimes.com/2020/07/06/health/coronavirus-airborne-aerosols.html?action=click\&pgtype=Article\&state=default\&region=MAIN_CONTENT_3\&context=storylines_faq}{It's
    unclear how often the virus is spread} via these tiny droplets, or
    aerosols, compared with larger droplets that are expelled when a
    sick person coughs or sneezes, or transmitted through contact with
    contaminated surfaces, said Linsey Marr, an aerosol expert at
    Virginia Tech. Aerosols are released even when a person without
    symptoms exhales, talks or sings, according to Dr. Marr and more
    than 200 other experts, who
    \href{https://academic.oup.com/cid/article/doi/10.1093/cid/ciaa939/5867798}{have
    outlined the evidence in an open letter to the World Health
    Organization}.
  \end{itemize}
\item ~
  \hypertarget{what-are-the-symptoms-of-coronavirus}{%
  \paragraph{What are the symptoms of
  coronavirus?}\label{what-are-the-symptoms-of-coronavirus}}

  \begin{itemize}
  \tightlist
  \item
    Common symptoms
    \href{https://www.nytimes.com/article/symptoms-coronavirus.html?action=click\&pgtype=Article\&state=default\&region=MAIN_CONTENT_3\&context=storylines_faq}{include
    fever, a dry cough, fatigue and difficulty breathing or shortness of
    breath.} Some of these symptoms overlap with those of the flu,
    making detection difficult, but runny noses and stuffy sinuses are
    less common.
    \href{https://www.nytimes.com/2020/04/27/health/coronavirus-symptoms-cdc.html?action=click\&pgtype=Article\&state=default\&region=MAIN_CONTENT_3\&context=storylines_faq}{The
    C.D.C. has also} added chills, muscle pain, sore throat, headache
    and a new loss of the sense of taste or smell as symptoms to look
    out for. Most people fall ill five to seven days after exposure, but
    symptoms may appear in as few as two days or as many as 14 days.
  \end{itemize}
\item ~
  \hypertarget{does-asymptomatic-transmission-of-covid-19-happen}{%
  \paragraph{Does asymptomatic transmission of Covid-19
  happen?}\label{does-asymptomatic-transmission-of-covid-19-happen}}

  \begin{itemize}
  \tightlist
  \item
    So far, the evidence seems to show it does. A widely cited
    \href{https://www.nature.com/articles/s41591-020-0869-5}{paper}
    published in April suggests that people are most infectious about
    two days before the onset of coronavirus symptoms and estimated that
    44 percent of new infections were a result of transmission from
    people who were not yet showing symptoms. Recently, a top expert at
    the World Health Organization stated that transmission of the
    coronavirus by people who did not have symptoms was ``very rare,''
    \href{https://www.nytimes.com/2020/06/09/world/coronavirus-updates.html?action=click\&pgtype=Article\&state=default\&region=MAIN_CONTENT_3\&context=storylines_faq\#link-1f302e21}{but
    she later walked back that statement.}
  \end{itemize}
\end{itemize}

But the church later said those messages did not reflect its official
policy, and that it had disciplined the person who sent them out.

On Friday, Mr. Lee urged his members to ``follow the government's
instructions,'' asking them to avoid gatherings and take their
proselytizing online.

``This disease outbreak is the work of the devil, which is hellbent on
stopping the rapid growth of the Shincheonji,'' he said in a message to
his followers.

South Korea has long been fertile ground for unorthodox religious
groups, some of which have amassed enormous wealth and influence. After
an overloaded ferry sank in 2014, killing more than 300 people, South
Koreans were shocked to learn that the ferry company was
\href{https://www.nytimes.com/2014/07/27/world/asia/in-ferry-deaths-a-south-korean-tycoons-downfall.html}{controlled
by a religious leader} who had been shunned as a heretic by mainstream
churches.

Image

A suspected coronavirus patient was brought to a hospital in Daegu on
Wednesday.Credit...Yonhap, via Associated Press

Shincheonji claims 150,000 members and has 12 congregations in South
Korea. It also has many smaller operations, which present themselves as
cafes or churches of other denominations and are used for proselytizing,
said Mr. Chung.

Shincheonji has long been criticized for its aggressive evangelical
work. Many mainstream churches post signs warning undercover Shincheonji
missionaries not to try to infiltrate their congregations.

Members of Shincheonji have recently targeted young South Koreans,
offering them free tarot readings, personality tests and
foreign-language classes, according to Mr. Hwang.

Moon Yoo-ja, 60, who spent years trying to ``rescue'' her daughter from
the church, accused Shincheonji of ruining many families.

``Once they fall into the trap of the church, they often abandon school
and jobs,'' Ms. Moon said. ``Some housewives packed up and joined the
church, abandoning their husbands and children.''

Hwang Gui-hag, editor in chief of the Seoul-based Law Times, which
specializes in religious news, cautioned against focusing too much on
Shincheonji's practices, some of which he said could be found in other
South Korean churches.

``This is essentially not a religious issue, but a medical and health
issue,'' Mr. Hwang said. ``If we pay too much attention to religion, we
miss the point. How would you explain the huge outbreak in Wuhan, China,
which is not really caused by any church?''

Advertisement

\protect\hyperlink{after-bottom}{Continue reading the main story}

\hypertarget{site-index}{%
\subsection{Site Index}\label{site-index}}

\hypertarget{site-information-navigation}{%
\subsection{Site Information
Navigation}\label{site-information-navigation}}

\begin{itemize}
\tightlist
\item
  \href{https://help.nytimes.com/hc/en-us/articles/115014792127-Copyright-notice}{©~2020~The
  New York Times Company}
\end{itemize}

\begin{itemize}
\tightlist
\item
  \href{https://www.nytco.com/}{NYTCo}
\item
  \href{https://help.nytimes.com/hc/en-us/articles/115015385887-Contact-Us}{Contact
  Us}
\item
  \href{https://www.nytco.com/careers/}{Work with us}
\item
  \href{https://nytmediakit.com/}{Advertise}
\item
  \href{http://www.tbrandstudio.com/}{T Brand Studio}
\item
  \href{https://www.nytimes.com/privacy/cookie-policy\#how-do-i-manage-trackers}{Your
  Ad Choices}
\item
  \href{https://www.nytimes.com/privacy}{Privacy}
\item
  \href{https://help.nytimes.com/hc/en-us/articles/115014893428-Terms-of-service}{Terms
  of Service}
\item
  \href{https://help.nytimes.com/hc/en-us/articles/115014893968-Terms-of-sale}{Terms
  of Sale}
\item
  \href{https://spiderbites.nytimes.com}{Site Map}
\item
  \href{https://help.nytimes.com/hc/en-us}{Help}
\item
  \href{https://www.nytimes.com/subscription?campaignId=37WXW}{Subscriptions}
\end{itemize}
