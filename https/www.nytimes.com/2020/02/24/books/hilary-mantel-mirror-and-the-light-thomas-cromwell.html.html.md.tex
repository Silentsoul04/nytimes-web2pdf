Sections

SEARCH

\protect\hyperlink{site-content}{Skip to
content}\protect\hyperlink{site-index}{Skip to site index}

\href{https://www.nytimes.com/section/books}{Books}

\href{https://myaccount.nytimes.com/auth/login?response_type=cookie\&client_id=vi}{}

\href{https://www.nytimes.com/section/todayspaper}{Today's Paper}

\href{/section/books}{Books}\textbar{}For Hilary Mantel, There's No Time
Like the Past

\url{https://nyti.ms/37UQzc4}

\begin{itemize}
\item
\item
\item
\item
\item
\item
\end{itemize}

Advertisement

\protect\hyperlink{after-top}{Continue reading the main story}

Supported by

\protect\hyperlink{after-sponsor}{Continue reading the main story}

\hypertarget{for-hilary-mantel-theres-no-time-like-the-past}{%
\section{For Hilary Mantel, There's No Time Like the
Past}\label{for-hilary-mantel-theres-no-time-like-the-past}}

``Wolf Hall'' and ``Bring Up the Bodies,'' the first books in her Thomas
Cromwell trilogy, have sold millions. Now the two-time Booker Prize
winner is finishing the job with ``The Mirror and the Light.''

\includegraphics{https://static01.nyt.com/images/2020/03/08/books/29Mantel3/29Mantel3-articleLarge-v3.jpg?quality=75\&auto=webp\&disable=upscale}

By \href{https://www.nytimes.com/by/alexandra-alter}{Alexandra Alter}

\begin{itemize}
\item
  Published Feb. 24, 2020Updated March 7, 2020
\item
  \begin{itemize}
  \item
  \item
  \item
  \item
  \item
  \item
  \end{itemize}
\end{itemize}

BUDLEIGH SALTERTON, England --- Hilary Mantel has a recurring anxiety
dream that takes place in a library. She finds a book with some scrap of
historical information she's been seeking, but when she tries to read
it, the words disintegrate before her eyes.

``And then when you wake up,'' she said, ``you've got the rhythm of a
sentence in your head, but you don't know what the sentence was.''

As deflated as she feels upon waking, the dreams have been instructive,
Mantel said.

``There's always going to be something slightly beyond your
comprehension, but you must go reaching for it,'' she told me last
month. ``If you thought the record was the whole story, the dream is
teaching you how fragile the record is.''

To an unusual degree for a novelist, Mantel feels bound by facts. That
approach has made her latest project --- a nearly 1,800-page trilogy
about the 16th-century lawyer and fixer Thomas Cromwell --- more
complicated than anything she's undertaken in her four decades of
writing.

The trilogy, which began in 2009 with
``\href{https://www.nytimes.com/2009/10/05/books/05maslin.html}{Wolf
Hall},'' traces Cromwell's unlikely rise, from his origins as a
blacksmith's son to the court of King Henry VIII. It concludes with
Mantel's next book, ``The Mirror and the Light,'' an account of the last
four years of Cromwell's life, as he amasses more wealth, influence and
power but loses the king's favor and later, his head.

The Cromwell series has turned Mantel into a literary celebrity and
something of a national icon. The first two books collectively sold more
than five million copies and have been translated into more than 30
languages. Both
``\href{https://www.nytimes.com/2009/11/01/books/review/Benfey-t.html}{Wolf
Hall}'' and its 2012 sequel,
``\href{https://www.nytimes.com/2012/05/02/books/bring-up-the-bodies-a-wolf-hall-sequel-by-hilary-mantel.html}{Bring
Up the Bodies},''
\href{https://artsbeat.blogs.nytimes.com/2012/10/16/hilary-mantel-wins-a-second-booker-prize/}{won
the Booker Prize}, making Mantel the first woman to win twice, and the
first author ever to win for a sequel. The books were adapted into an
award-winning pair of plays by the Royal Shakespeare Company and a BBC
mini-series. In 2015, Prince Charles anointed Mantel with the title of
Dame Commander, Order of the British Empire, the equivalent of
knighthood, prompting some in the press to sneeringly draw comparisons
between the modern-day royals and the louche, back-stabbing behavior of
the Tudors.

\includegraphics{https://static01.nyt.com/images/2020/03/09/books/29Mantel-sub/29Mantel-sub-articleLarge.jpg?quality=75\&auto=webp\&disable=upscale}

Throughout her rise to prominence, Mantel has remained aloof. She's
never been part of the London literary establishment and seems to prefer
the company of her long-dead characters to the demands of being a public
figure. For the past decade, she and her husband Gerald McEwen, a
retired geologist, have lived in Budleigh Salterton, an idyllic village
on the coast of Devon.

She's far from shy, though. A staunch iconoclast, Mantel has
occasionally stirred controversy with her heterodox attitudes about
British royalty and politics. In 2013, the tabloids pounced on comments
she made during a
lecture\href{https://www.theguardian.com/uk/2013/feb/19/kate-duchess-cambridge-hilary-mantel}{in
which she called} the Duchess of Cambridge ``a shop-window mannequin''
with no personality. A year later, she angered conservative British
politicians and set off another media maelstrom when she published a
short story that imagined the planned assassination of Margaret Thatcher
by an I.R.A. sniper.

``She was imprisoned in her own home for a week while the press went
absolutely bonkers,'' said her literary agent Bill Hamilton, who called
the episode ``incredibly funny, if inconvenient for her.''

More recently, Mantel has been hounded by the British press over the
delayed publication of ``The Mirror and the Light,'' which is due out
next month but was originally planned for release in 2018. The lag set
off speculation that Mantel suffered from writer's block, or was
distracted by the stage and television adaptations, or was
procrastinating because she couldn't bear to kill Cromwell. Expectations
for the novel, which were high to begin with, are now stratospheric, and
Mantel felt pressure to deliver a worthy ending.

``The reason it took so long is that it's difficult, and that is a
totally sufficient explanation,'' Mantel said, sounding bewildered and
slightly irritated. ``But that's not an explanation that has any news
value, so people are looking for a dramatic story of the whole process
breaking down.''

Writing ``The Mirror and the Light,'' which at nearly 800 pages is the
longest and most intricately plotted book in the trilogy, was at times a
grueling undertaking. In the final months of writing, Mantel, who is now
67 and has endured chronic pain and illness throughout her adult life,
kept herself on a punishing schedule. She didn't realize what a toll the
project had taken on her until she was done.

\emph{{[}}
\href{https://www.nytimes.com/2020/03/03/books/review-mirror-light-hilary-mantel.html}{\emph{Read
our review of ``The Mirror and the Light.''}} \emph{{]}}

Now that she's finished the grim final chapter of Cromwell's story,
Mantel says she's done with historical fiction and plans to focus on
writing plays, an entirely new medium for her. She's abandoning the
genre in part because she feels she doesn't have the stamina to take on
a big research project, and because she can't imagine finding another
historical figure as appealing as Cromwell.

``I'm not going to meet another Thomas Cromwell, if you think how long
he's been around in my consciousness,'' she said.

◇ ◇ ◇

\hypertarget{theyre-more-real-and-solid-to-you-than-actual-people}{%
\subsection{`They're more real and solid to you than actual
people.'}\label{theyre-more-real-and-solid-to-you-than-actual-people}}

Mantel and I met over two wet, windy days in Budleigh Salterton, where
she lives with McEwen. Their apartment looks out on a stretch of rocky
beach, and the choppy waves were gray and a dull red, stained from the
eroding sandstone cliffs.

Apart from a few knickknacks --- a stuffed lion and dog perched in a
window seat, as if guarding the premises --- and a robust library full
of classics by Jane Austen, T.S. Eliot, Gustave Flaubert and Fyodor
Dostoyevsky, their apartment felt like a secular shrine to Tudor
England, with shelves of books on Cromwell and his contemporaries, and
titles about medieval fashion, food and metallurgy. Hanging in the
hallway was a photograph of Mantel standing in front of the famous Hans
Holbein oil painting of Cromwell: stout, beady-eyed, vaguely
threatening. (In ``Wolf Hall,'' when Cromwell worries that the portrait
makes him look like a murderer, his son replies, ``Did you not know?'')

With her pale skin, wispy graying blond hair and wide, arresting
light-blue eyes that are ringed with a deeper blue, Mantel has an almost
ethereal appearance. She moves deliberately, a habit she acquired after
living for decades with chronic pain, and seems to glide rather than
walk. She spoke slowly and so softly at times that I worried my recorder
wouldn't pick up her voice over the rumble of the waves and rain.

Talking about the book feels surreal after years in near isolation, she
said. ``I've been like someone in a religious order who's taken a vow of
silence. It's strange, because all that time I was listening to the
past, and now I'm almost talking for a living, and it feels very
frivolous and empty compared to the stillness that there used to be in
every day.''

Though I expected to find her in mourning, it became clear as Mantel
began to talk about Cromwell that for her, he isn't really gone. She
writes and speaks about him in the present tense. After finishing the
final novel, she began working on a stage adaptation of ``The Mirror and
the Light,'' so Cromwell is still very much in her head.

Image

Hilary Mantel, center, at the April 9, 2015, opening night of ``Wolf
Hall'' in New York.Credit...John Lamparski/WireImage, via Getty Images

``She talks with him as if he's a living presence,'' said
\href{https://www.nytimes.com/2015/03/22/theater/ben-miles-takes-on-wolf-hall-onstage.html}{Ben
Miles}, who played Cromwell in
t\href{https://www.nytimes.com/2014/01/05/theater/wolf-hall-and-bring-up-the-bodies-head-for-the-stage.html}{he
2014 Royal Shakespeare Company}stage adaptation and is expected to
resume the role when ``The Mirror and the Light'' has its premiere.
``She seems to know him intimately but is always striving to understand
him.''

During rehearsals, Miles and Mantel acted as each other's muses. He
asked her questions about Cromwell's childhood, family life and
religious beliefs, and her detailed answers informed his performance. In
turn, his queries and insights into the character helped to shape the
third book, sometimes sending her on a different trajectory than she'd
been planning and leading her to an even deeper investigation of
Cromwell's psyche.

They've become such close collaborators that when Mantel decided to
adapt ``The Mirror and the Light'' herself, rather than handing it off
to a playwright, she chose Miles to co-write it with her.

Mantel has never written for the theater before, and she is taking an
unorthodox approach, using her source material to develop something
almost entirely new. ``If you're an adapter, you feel so bound to the
original text, but I don't have to put in a single word from the book if
I don't want to,'' she said. ``Most of what I've written now is
completely fresh. It's not obliged to the book.''

Mantel's work on the play has also kept Cromwell and his contemporaries
vivid in her imagination. Even when she's not at her desk writing, she
can still hear them chattering away.

``Once those voices begin, it's like having the radio on in the
background for 15 years. It never actually fades. It runs continuously
with whatever else you're doing, and that means you're never off duty to
the book, you never stop working on it. You fall asleep with it, you
wake up with it,'' she said. ``There's a point where you're living with
these people and only with them. They're more real and solid to you than
actual people in your life.''

◇ ◇ ◇

\hypertarget{i-am-used-to-seeing-things-that-arent-there}{%
\subsection{`I am used to ``seeing'' things that aren't
there.'}\label{i-am-used-to-seeing-things-that-arent-there}}

Mantel in many ways is perfectly suited to the task of excavating and
reanimating the past. Ever since she was a child, she's been prone to
visions of ghosts and spirits. ``I am used to `seeing' things that
aren't there,'' she writes in her memoir,
``\href{https://www.nytimes.com/2003/10/05/books/unsuited-to-everything.html}{Giving
Up the Ghost}.''

Growing up in an Irish Catholic family in Hadfield, a village in
Derbyshire, Mantel was obsessed with myths, folklore and the
supernatural. Before she was old enough to read, she insisted that
relatives read to her tales from King Arthur and the Knights of the
Round Table. ``I had a head stuffed full of chivalric epigrams, and the
self-confidence that comes from a thorough knowledge of horsemanship and
swordplay,'' she writes.

At 18, she went to the London School of Economics to study law, with the
hope of becoming a barrister, but she couldn't afford to continue with
professional training. By then, she'd met McEwen. They married when they
were 20 and moved to Manchester, where he found a teaching position and
she worked various jobs and started writing.

Around that time, Mantel's health began to deteriorate. A doctor
dismissed her symptoms as a bid for attention and referred her to a
psychiatrist. The psychiatrist gave her tranquilizers and an
antipsychotic drug and told her to stop writing.

Years later, when Mantel and McEwen were living in Botswana, she
researched her symptoms and diagnosed herself with endometriosis.
Doctors confirmed her suspicions, and when she was 27, she had surgery
to remove her uterus and ovaries. The pain didn't abate, and Mantel
suffered from complications that still afflict her: her weight
increased, her legs swelled, she felt exhausted and alien to herself.

Her illness made a normal day job impossible: ``It narrowed my options
in life, and it narrowed them to writing,'' she said.

Mantel finished her first book, a novel about the French Revolution
titled ``A Place of Greater Safety,'' in 1979, and sent it to publishers
and agents, but no one wanted a 700-plus page historical novel by an
unknown writer. She wrote a second book, a brisk, darkly comic
contemporary novel, ``Every Day Is Mother's Day,'' which became a
critical success when it was published in 1985.

Over the next two decades, she published seven other novels and
developed a cult following. Though her books vary in their subject
matter, style and tone, they are bound by recurring themes: her
fascination with transformation and the unseen realm, with myths and
archetypes.

When she was writing her novel
``\href{https://www.nytimes.com/2005/05/15/books/review/beyond-black-demons-revealed.html}{Beyond
Black},'' about a medium who channels the voices of the dead, Mantel
realized she was creating a road map for the Cromwell trilogy. ``I was
thinking, this isn't just about a medium,'' she said, ``it's about how
to induce the necessary frame of mind to let the past enact itself.''

◇ ◇ ◇

\hypertarget{the-real-story-is-better-than-anything-i-can-come-up-with}{%
\subsection{`The real story is better than anything I can come up
with.'}\label{the-real-story-is-better-than-anything-i-can-come-up-with}}

When she began writing ``Wolf Hall'' in 2005, Mantel was still
relatively obscure. She was also entering a saturated marketplace for
Tudor historical fiction, territory that had already been mined by
novelists like Philippa Gregory, Antonia Fraser and Alison Weir.

Mantel had been fascinated by Cromwell for decades, ever since she
learned, while she was attending a convent-run high school in Cheshire,
about Cromwell's role in dissolving the country's monasteries. In her
research, she found he was often reduced to a thuggish caricature. ``I
realized that some imaginative work is due on this man,'' she said.

She deployed the same methods she used for ``A Place of Greater
Safety,'' gathering as much historical evidence as she could find, then
using the facts to stitch together a narrative. Whenever she hit a
roadblock, she would write another section of the story.

Image

Claire Foy as Anne Boleyn and Damian Lewis as King Henry VIII in the
television adaptation of ``Wolf Hall.''~Credit...Ed Miller/Playground \&
Company Pictures for Masterpiece, BBC

In her office, in an apartment up the hill from her home, Mantel showed
me the card catalog she used to keep track of Cromwell's whereabouts, so
that she didn't mistakenly put him in the wrong place at the wrong time.
A card I pulled out at random read, ``31 July 1536, TC could be at
Cookham or Sunninghill.''

Even though as a novelist, she has license to invent, Mantel dreads the
thought of contradicting an available historical fact. ``If you started
out with the attitude that the truth is optional, I couldn't take any
pleasure in it at all,'' she said. ``I know that the real story is
better than anything I can come up with.''

By bringing a historian's rigor to her fiction, Mantel has had a
profound impact on history itself. Before ``Wolf Hall,'' Cromwell was
often cast as a cartoonish villain who persecuted the pious and helped a
lustful king dispatch of unwanted wives. Mantel rehabilitated Cromwell,
depicting him as a strategist and visionary, and convincing some
scholars to re-evaluate his place in history.

``Hilary has reset the historical patterns through the way in which
she's reimagined the man,'' said Diarmaid MacCulloch, an Oxford theology
professor who published a new
\href{https://www.theguardian.com/books/2018/sep/22/thomas-cromwell-life-diarmaid-maccolloch-review}{Cromwell
biography} in 2018. ``It's fiction which is extraordinarily probable,
and it's remarkably like the Cromwell I'd been excavating myself.''

Image

``All that time I was listening to the past, and now I'm almost talking
for a living,'' Mantel said, ``and it feels very frivolous and empty
compared to the stillness that there used to be in every
day.''Credit...Ellie Smith for The New York Times

There was never any question how Cromwell's story would end. Not long
after she wrote the opening of ``Wolf Hall'' --- a young Thomas Cromwell
lies bleeding on the cobblestones, beaten by his abusive father --- she
wrote about his beheading.

``All I had to do was fill in the middle,'' Mantel said, then laughed.
``There wasn't a day when I woke up and thought, `Today I have to kill
Cromwell,' because I'd already killed him and brought him back to life
so many times.''

As Mantel spoke about Cromwell and how he endures for her, it reminded
me of a moment in ``The Mirror and the Light'' when Cromwell realizes
that he's losing the king's confidence, and thinks of his beloved
master, Cardinal Wolsey, who still speaks to him from the grave.

``The dead are more faithful than the living,'' Cromwell thinks. ``For
better or worse, they do not leave you. They last out the longest
night.''

\emph{Follow New York Times Books on}
\href{https://www.facebook.com/nytbooks/}{\emph{Facebook}}\emph{,}
\href{https://twitter.com/nytimesbooks}{\emph{Twitter}} \emph{and}
\href{https://www.instagram.com/nytbooks/}{\emph{Instagram}}\emph{, sign
up for}
\href{https://www.nytimes.com/newsletters/books-review}{\emph{our
newsletter}} \emph{or}
\href{https://www.nytimes.com/interactive/2017/books/books-calendar.html}{\emph{our
literary calendar}}\emph{. And listen to us on the}
\href{https://www.nytimes.com/column/book-review-podcast}{\emph{Book
Review podcast}}\emph{.}

Advertisement

\protect\hyperlink{after-bottom}{Continue reading the main story}

\hypertarget{site-index}{%
\subsection{Site Index}\label{site-index}}

\hypertarget{site-information-navigation}{%
\subsection{Site Information
Navigation}\label{site-information-navigation}}

\begin{itemize}
\tightlist
\item
  \href{https://help.nytimes.com/hc/en-us/articles/115014792127-Copyright-notice}{©~2020~The
  New York Times Company}
\end{itemize}

\begin{itemize}
\tightlist
\item
  \href{https://www.nytco.com/}{NYTCo}
\item
  \href{https://help.nytimes.com/hc/en-us/articles/115015385887-Contact-Us}{Contact
  Us}
\item
  \href{https://www.nytco.com/careers/}{Work with us}
\item
  \href{https://nytmediakit.com/}{Advertise}
\item
  \href{http://www.tbrandstudio.com/}{T Brand Studio}
\item
  \href{https://www.nytimes.com/privacy/cookie-policy\#how-do-i-manage-trackers}{Your
  Ad Choices}
\item
  \href{https://www.nytimes.com/privacy}{Privacy}
\item
  \href{https://help.nytimes.com/hc/en-us/articles/115014893428-Terms-of-service}{Terms
  of Service}
\item
  \href{https://help.nytimes.com/hc/en-us/articles/115014893968-Terms-of-sale}{Terms
  of Sale}
\item
  \href{https://spiderbites.nytimes.com}{Site Map}
\item
  \href{https://help.nytimes.com/hc/en-us}{Help}
\item
  \href{https://www.nytimes.com/subscription?campaignId=37WXW}{Subscriptions}
\end{itemize}
