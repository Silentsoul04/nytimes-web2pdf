Sections

SEARCH

\protect\hyperlink{site-content}{Skip to
content}\protect\hyperlink{site-index}{Skip to site index}

\href{https://www.nytimes.com/section/world/africa}{Africa}

\href{https://myaccount.nytimes.com/auth/login?response_type=cookie\&client_id=vi}{}

\href{https://www.nytimes.com/section/todayspaper}{Today's Paper}

\href{/section/world/africa}{Africa}\textbar{}Nigeria Responds to First
Coronavirus Case in Sub-Saharan Africa

\url{https://nyti.ms/2I5I97c}

\begin{itemize}
\item
\item
\item
\item
\item
\end{itemize}

\href{https://www.nytimes.com/news-event/coronavirus?action=click\&pgtype=Article\&state=default\&region=TOP_BANNER\&context=storylines_menu}{The
Coronavirus Outbreak}

\begin{itemize}
\tightlist
\item
  live\href{https://www.nytimes.com/2020/08/01/world/coronavirus-covid-19.html?action=click\&pgtype=Article\&state=default\&region=TOP_BANNER\&context=storylines_menu}{Latest
  Updates}
\item
  \href{https://www.nytimes.com/interactive/2020/us/coronavirus-us-cases.html?action=click\&pgtype=Article\&state=default\&region=TOP_BANNER\&context=storylines_menu}{Maps
  and Cases}
\item
  \href{https://www.nytimes.com/interactive/2020/science/coronavirus-vaccine-tracker.html?action=click\&pgtype=Article\&state=default\&region=TOP_BANNER\&context=storylines_menu}{Vaccine
  Tracker}
\item
  \href{https://www.nytimes.com/interactive/2020/07/29/us/schools-reopening-coronavirus.html?action=click\&pgtype=Article\&state=default\&region=TOP_BANNER\&context=storylines_menu}{What
  School May Look Like}
\item
  \href{https://www.nytimes.com/live/2020/07/31/business/stock-market-today-coronavirus?action=click\&pgtype=Article\&state=default\&region=TOP_BANNER\&context=storylines_menu}{Economy}
\end{itemize}

Advertisement

\protect\hyperlink{after-top}{Continue reading the main story}

Supported by

\protect\hyperlink{after-sponsor}{Continue reading the main story}

\hypertarget{nigeria-responds-to-first-coronavirus-case-in-sub-saharan-africa}{%
\section{Nigeria Responds to First Coronavirus Case in Sub-Saharan
Africa}\label{nigeria-responds-to-first-coronavirus-case-in-sub-saharan-africa}}

The illness was diagnosed in an Italian who landed in Lagos, the
country's largest city and one of the continent's biggest metropolises.
Nigeria acted quickly, but there are fears that the virus could sweep
the region.

\includegraphics{https://static01.nyt.com/images/2020/02/28/world/28virus-africa/merlin_169694988_e354b3c5-45b0-4857-ac0c-453ef3c09756-articleLarge.jpg?quality=75\&auto=webp\&disable=upscale}

By \href{https://www.nytimes.com/by/ruth-maclean}{Ruth Maclean} and
\href{https://www.nytimes.com/by/abdi-latif-dahir}{Abdi Latif Dahir}

\begin{itemize}
\item
  Published Feb. 28, 2020Updated July 23, 2020
\item
  \begin{itemize}
  \item
  \item
  \item
  \item
  \item
  \end{itemize}
\end{itemize}

DAKAR, Senegal --- An Italian contractor who flew into
\href{https://www.nytimes.com/2020/07/23/world/africa/aid-workers-executed-nigeria.html}{Nigeria}
from Milan became sub-Saharan Africa's first confirmed coronavirus
patient on Friday, stoking concern that an outbreak on the continent
could overwhelm already strained health care systems.

The patient, a young man, had no symptoms when he landed this week in
Lagos, traveled 60 miles north to the cement factory where he works and
developed a fever there, a Nigerian health commissioner said to
reporters.

The appearance of the first case in Lagos --- a city of 20 million
people and the economic capital of Africa's most populous nation --- set
off fear, rumors and panicked buying of hand sanitizer and face masks.
It also posed a test for the Nigerian medical system, which like others
in Africa, has been preparing for the virus to arrive for weeks.

``I feel scared,'' said Karo Otitifore, an elementary schoolteacher
waiting at a bus stop in Yaba, the Lagos suburb where the Italian
patient was being treated. ``I try to sit tight, squeeze my whole body
so that I won't have to have too much body contact with people.''

Public health officials have been warning that
\href{https://www.nytimes.com/2020/02/06/world/africa/africa-coronavirus-china.html}{the
coronavirus could be devastating in Africa}, a continent with relatively
weak health care and disease surveillance systems. So the Nigerian
health system --- which had gained a reputation for efficiently
containing cases during the Ebola epidemic in 2014 --- responded quickly
to the appearance of a suspected case of coronavirus.

The young Italian was sent back to Lagos, where he is being treated in a
hospital facility that had been set aside to handle coronavirus, while a
team of epidemiologists traces his contacts.

Dr. Ngozi Erondu, an associate fellow in the Global Health Program at
Chatham House, an international research group in London, said Nigeria's
confirmation of the coronavirus case in just two days was ``a very
positive reflection of the disease surveillance and laboratory capacity
in Nigeria.''

\hypertarget{latest-updates-global-coronavirus-outbreak}{%
\section{\texorpdfstring{\href{https://www.nytimes.com/2020/08/01/world/coronavirus-covid-19.html?action=click\&pgtype=Article\&state=default\&region=MAIN_CONTENT_1\&context=storylines_live_updates}{Latest
Updates: Global Coronavirus
Outbreak}}{Latest Updates: Global Coronavirus Outbreak}}\label{latest-updates-global-coronavirus-outbreak}}

Updated 2020-08-02T07:42:09.613Z

\begin{itemize}
\tightlist
\item
  \href{https://www.nytimes.com/2020/08/01/world/coronavirus-covid-19.html?action=click\&pgtype=Article\&state=default\&region=MAIN_CONTENT_1\&context=storylines_live_updates\#link-34047410}{The
  U.S. reels as July cases more than double the total of any other
  month.}
\item
  \href{https://www.nytimes.com/2020/08/01/world/coronavirus-covid-19.html?action=click\&pgtype=Article\&state=default\&region=MAIN_CONTENT_1\&context=storylines_live_updates\#link-780ec966}{Top
  U.S. officials work to break an impasse over the federal jobless
  benefit.}
\item
  \href{https://www.nytimes.com/2020/08/01/world/coronavirus-covid-19.html?action=click\&pgtype=Article\&state=default\&region=MAIN_CONTENT_1\&context=storylines_live_updates\#link-2bc8948}{Its
  outbreak untamed, Melbourne goes into even greater lockdown.}
\end{itemize}

\href{https://www.nytimes.com/2020/08/01/world/coronavirus-covid-19.html?action=click\&pgtype=Article\&state=default\&region=MAIN_CONTENT_1\&context=storylines_live_updates}{See
more updates}

More live coverage:
\href{https://www.nytimes.com/live/2020/07/31/business/stock-market-today-coronavirus?action=click\&pgtype=Article\&state=default\&region=MAIN_CONTENT_1\&context=storylines_live_updates}{Markets}

But it will be key, she said, for all African countries to enhance
scrutiny of travelers, especially those coming from countries with
documented outbreaks.

The coronavirus, which emerged in central China late last year, has
spread to almost 50 countries, sickening about 84,000 people and killing
nearly 3,000 --- mostly in mainland China. But in recent weeks, the
virus has spread across the globe with outbreaks in Iran, Japan, South
Korea and Italy, especially in Milan.

To date, 26 African countries
\href{http://www.africacdc.org/press-centre/news/123-outbreak-update-on-the-ongoing-novel-coronavirus-global-epidemic-issue-6-25-feb-2020}{have
reported suspected coronavirus cases}, according to the Africa Centers
for Disease Control and Prevention. Only three countries on the
continent --- Algeria and Egypt in North Africa --- and now Nigeria in
the sub-Saharan region, have announced confirmed cases. The case in
Algeria was an Italian
\href{https://www.afro.who.int/news/second-covid-19-case-confirmed-africa}{who
arrived in the country} on Feb. 17, according to the World Health
Organization.

Global health experts had anticipated that the virus would most likely
spread to Africa from China, which has increased its ties with Africa
enormously over the past two decades.

To combat the potential spread of the deadly outbreak, airlines
including Egypt Air, Kenya Airways and South African Airways have
suspended their flights to China.

Ethiopian Airlines, which operates the largest number of flights between
China and Africa, has refused to follow suit despite widespread
criticism from political and business leaders across the continent.

\includegraphics{https://static01.nyt.com/images/2020/02/28/world/28virus-africa2/merlin_168418659_300b6cee-3811-4be6-b201-c09fd174e3a9-articleLarge.jpg?quality=75\&auto=webp\&disable=upscale}

The Kenyan government
\href{https://twitter.com/dailynation/status/1233224185444552704}{faced
criticism} this week for allowing China Southern Airlines to
\href{https://twitter.com/MOH_Kenya/status/1232735617446010881}{resume
flights} from Guangzhou Province in China to Nairobi, the Kenyan
capital. The government said that all 239 passengers on the first plane
after the service restarted had been screened onboard, cleared and
advised to self-quarantine for 14 days. But on social media, many
expressed outrage about the flights, calling officials negligent.

On Friday, after numerous lawsuits were filed against the government,
Kenya's high court temporarily suspended flights from China for ten
days.

The
\href{https://www.nytimes.com/topic/subject/the-ebola-outbreak-in-west-africa}{Ebola
outbreak} that ravaged Liberia, Sierra Leone and Guinea five years ago
is still fresh in the minds of West Africans. Many are fearful that
their governments are not much better prepared to detect, respond to and
contain outbreaks than they were then.

A continentwide Center for Disease Control has since been established
and has been coordinating efforts across the continent. The Africa
C.D.C., in collaboration with the W.H.O., has worked with countries to
improve their surveillance and testing processes.

\href{https://www.nytimes.com/news-event/coronavirus?action=click\&pgtype=Article\&state=default\&region=MAIN_CONTENT_3\&context=storylines_faq}{}

\hypertarget{the-coronavirus-outbreak-}{%
\subsubsection{The Coronavirus Outbreak
›}\label{the-coronavirus-outbreak-}}

\hypertarget{frequently-asked-questions}{%
\paragraph{Frequently Asked
Questions}\label{frequently-asked-questions}}

Updated July 27, 2020

\begin{itemize}
\item ~
  \hypertarget{should-i-refinance-my-mortgage}{%
  \paragraph{Should I refinance my
  mortgage?}\label{should-i-refinance-my-mortgage}}

  \begin{itemize}
  \tightlist
  \item
    \href{https://www.nytimes.com/article/coronavirus-money-unemployment.html?action=click\&pgtype=Article\&state=default\&region=MAIN_CONTENT_3\&context=storylines_faq}{It
    could be a good idea,} because mortgage rates have
    \href{https://www.nytimes.com/2020/07/16/business/mortgage-rates-below-3-percent.html?action=click\&pgtype=Article\&state=default\&region=MAIN_CONTENT_3\&context=storylines_faq}{never
    been lower.} Refinancing requests have pushed mortgage applications
    to some of the highest levels since 2008, so be prepared to get in
    line. But defaults are also up, so if you're thinking about buying a
    home, be aware that some lenders have tightened their standards.
  \end{itemize}
\item ~
  \hypertarget{what-is-school-going-to-look-like-in-september}{%
  \paragraph{What is school going to look like in
  September?}\label{what-is-school-going-to-look-like-in-september}}

  \begin{itemize}
  \tightlist
  \item
    It is unlikely that many schools will return to a normal schedule
    this fall, requiring the grind of
    \href{https://www.nytimes.com/2020/06/05/us/coronavirus-education-lost-learning.html?action=click\&pgtype=Article\&state=default\&region=MAIN_CONTENT_3\&context=storylines_faq}{online
    learning},
    \href{https://www.nytimes.com/2020/05/29/us/coronavirus-child-care-centers.html?action=click\&pgtype=Article\&state=default\&region=MAIN_CONTENT_3\&context=storylines_faq}{makeshift
    child care} and
    \href{https://www.nytimes.com/2020/06/03/business/economy/coronavirus-working-women.html?action=click\&pgtype=Article\&state=default\&region=MAIN_CONTENT_3\&context=storylines_faq}{stunted
    workdays} to continue. California's two largest public school
    districts --- Los Angeles and San Diego --- said on July 13, that
    \href{https://www.nytimes.com/2020/07/13/us/lausd-san-diego-school-reopening.html?action=click\&pgtype=Article\&state=default\&region=MAIN_CONTENT_3\&context=storylines_faq}{instruction
    will be remote-only in the fall}, citing concerns that surging
    coronavirus infections in their areas pose too dire a risk for
    students and teachers. Together, the two districts enroll some
    825,000 students. They are the largest in the country so far to
    abandon plans for even a partial physical return to classrooms when
    they reopen in August. For other districts, the solution won't be an
    all-or-nothing approach.
    \href{https://bioethics.jhu.edu/research-and-outreach/projects/eschool-initiative/school-policy-tracker/}{Many
    systems}, including the nation's largest, New York City, are
    devising
    \href{https://www.nytimes.com/2020/06/26/us/coronavirus-schools-reopen-fall.html?action=click\&pgtype=Article\&state=default\&region=MAIN_CONTENT_3\&context=storylines_faq}{hybrid
    plans} that involve spending some days in classrooms and other days
    online. There's no national policy on this yet, so check with your
    municipal school system regularly to see what is happening in your
    community.
  \end{itemize}
\item ~
  \hypertarget{is-the-coronavirus-airborne}{%
  \paragraph{Is the coronavirus
  airborne?}\label{is-the-coronavirus-airborne}}

  \begin{itemize}
  \tightlist
  \item
    The coronavirus
    \href{https://www.nytimes.com/2020/07/04/health/239-experts-with-one-big-claim-the-coronavirus-is-airborne.html?action=click\&pgtype=Article\&state=default\&region=MAIN_CONTENT_3\&context=storylines_faq}{can
    stay aloft for hours in tiny droplets in stagnant air}, infecting
    people as they inhale, mounting scientific evidence suggests. This
    risk is highest in crowded indoor spaces with poor ventilation, and
    may help explain super-spreading events reported in meatpacking
    plants, churches and restaurants.
    \href{https://www.nytimes.com/2020/07/06/health/coronavirus-airborne-aerosols.html?action=click\&pgtype=Article\&state=default\&region=MAIN_CONTENT_3\&context=storylines_faq}{It's
    unclear how often the virus is spread} via these tiny droplets, or
    aerosols, compared with larger droplets that are expelled when a
    sick person coughs or sneezes, or transmitted through contact with
    contaminated surfaces, said Linsey Marr, an aerosol expert at
    Virginia Tech. Aerosols are released even when a person without
    symptoms exhales, talks or sings, according to Dr. Marr and more
    than 200 other experts, who
    \href{https://academic.oup.com/cid/article/doi/10.1093/cid/ciaa939/5867798}{have
    outlined the evidence in an open letter to the World Health
    Organization}.
  \end{itemize}
\item ~
  \hypertarget{what-are-the-symptoms-of-coronavirus}{%
  \paragraph{What are the symptoms of
  coronavirus?}\label{what-are-the-symptoms-of-coronavirus}}

  \begin{itemize}
  \tightlist
  \item
    Common symptoms
    \href{https://www.nytimes.com/article/symptoms-coronavirus.html?action=click\&pgtype=Article\&state=default\&region=MAIN_CONTENT_3\&context=storylines_faq}{include
    fever, a dry cough, fatigue and difficulty breathing or shortness of
    breath.} Some of these symptoms overlap with those of the flu,
    making detection difficult, but runny noses and stuffy sinuses are
    less common.
    \href{https://www.nytimes.com/2020/04/27/health/coronavirus-symptoms-cdc.html?action=click\&pgtype=Article\&state=default\&region=MAIN_CONTENT_3\&context=storylines_faq}{The
    C.D.C. has also} added chills, muscle pain, sore throat, headache
    and a new loss of the sense of taste or smell as symptoms to look
    out for. Most people fall ill five to seven days after exposure, but
    symptoms may appear in as few as two days or as many as 14 days.
  \end{itemize}
\item ~
  \hypertarget{does-asymptomatic-transmission-of-covid-19-happen}{%
  \paragraph{Does asymptomatic transmission of Covid-19
  happen?}\label{does-asymptomatic-transmission-of-covid-19-happen}}

  \begin{itemize}
  \tightlist
  \item
    So far, the evidence seems to show it does. A widely cited
    \href{https://www.nature.com/articles/s41591-020-0869-5}{paper}
    published in April suggests that people are most infectious about
    two days before the onset of coronavirus symptoms and estimated that
    44 percent of new infections were a result of transmission from
    people who were not yet showing symptoms. Recently, a top expert at
    the World Health Organization stated that transmission of the
    coronavirus by people who did not have symptoms was ``very rare,''
    \href{https://www.nytimes.com/2020/06/09/world/coronavirus-updates.html?action=click\&pgtype=Article\&state=default\&region=MAIN_CONTENT_3\&context=storylines_faq\#link-1f302e21}{but
    she later walked back that statement.}
  \end{itemize}
\end{itemize}

Currently, 26 laboratories
\href{https://www.afro.who.int/news/who-pledges-support-african-countries-joint-coronavirus-disease-preparedness-and-response}{are
able to test} for the coronavirus on the continent, up from just two in
early February, according to the W.H.O. The Africa C.D.C. has shipped
one thousand test kits to Nigeria.

Nigerians expressed some confidence in their government's ability to
contain coronavirus on Friday, stemming from their experience with the
Ebola epidemic, which experts say appears to be far more deadly than the
coronavirus.

A Liberian-American who brought Ebola to Lagos in 2014 was prevented
from leaving a clinic, his contacts were traced and 900 people were
monitored. He and a doctor and six other people died --- a number many
epidemiologists considered mercifully low given that more than 11,000
people died in the West African outbreak, which lasted from 2014 to
2016.

However, Lagos is a city that contains both extreme wealth and poverty,
and if the virus gets into its poorest areas, the lack of adequate
housing and sanitation among people living in cramped conditions could
cause it to spread at breakneck speed, a public health expert said.

In Lagos on Friday, people across the city were hurriedly buying hand
sanitizer, face masks and in the shaky hope it would make a difference,
vitamin C supplements.

``Somebody came in and just bought all our vitamin C. People have been
constantly asking for it,'' said Eniola Okunnuga, a pharmacist at one of
the biggest pharmacies in Lagos.

Unfounded rumors spread rapidly that bathing in water laced with ginger
and garlic would protect against the virus, particularly on the
messaging service WhatsApp. But sound advice also circulated, about
washing hands and keeping away from people with coughs.

Mr. Otitifore, the schoolteacher, scoffed at old wives' tales about
ginger and garlic, but also repeated some rumors he had given credence
to. Drinking a lot of water would wash the virus down so that stomach
acid could attack it, he said.

He waited at a crowded bus stop for a bus crammed with passengers. A
recent ban --- unrelated to coronavirus --- on the city's fleet of
motorcycle taxis and auto-rickshaws meant that many Lagosians are in
even closer contact than usual, raising the risk of exposure should the
virus spread.

Oluwatosin Adeshokan contributed reporting from Lagos, Nigeria, and
Simon Marks from Addis Ababa, Ethiopia.

Advertisement

\protect\hyperlink{after-bottom}{Continue reading the main story}

\hypertarget{site-index}{%
\subsection{Site Index}\label{site-index}}

\hypertarget{site-information-navigation}{%
\subsection{Site Information
Navigation}\label{site-information-navigation}}

\begin{itemize}
\tightlist
\item
  \href{https://help.nytimes.com/hc/en-us/articles/115014792127-Copyright-notice}{©~2020~The
  New York Times Company}
\end{itemize}

\begin{itemize}
\tightlist
\item
  \href{https://www.nytco.com/}{NYTCo}
\item
  \href{https://help.nytimes.com/hc/en-us/articles/115015385887-Contact-Us}{Contact
  Us}
\item
  \href{https://www.nytco.com/careers/}{Work with us}
\item
  \href{https://nytmediakit.com/}{Advertise}
\item
  \href{http://www.tbrandstudio.com/}{T Brand Studio}
\item
  \href{https://www.nytimes.com/privacy/cookie-policy\#how-do-i-manage-trackers}{Your
  Ad Choices}
\item
  \href{https://www.nytimes.com/privacy}{Privacy}
\item
  \href{https://help.nytimes.com/hc/en-us/articles/115014893428-Terms-of-service}{Terms
  of Service}
\item
  \href{https://help.nytimes.com/hc/en-us/articles/115014893968-Terms-of-sale}{Terms
  of Sale}
\item
  \href{https://spiderbites.nytimes.com}{Site Map}
\item
  \href{https://help.nytimes.com/hc/en-us}{Help}
\item
  \href{https://www.nytimes.com/subscription?campaignId=37WXW}{Subscriptions}
\end{itemize}
