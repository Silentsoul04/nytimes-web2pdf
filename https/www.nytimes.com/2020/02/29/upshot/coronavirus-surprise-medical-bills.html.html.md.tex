Sections

SEARCH

\protect\hyperlink{site-content}{Skip to
content}\protect\hyperlink{site-index}{Skip to site index}

\href{https://myaccount.nytimes.com/auth/login?response_type=cookie\&client_id=vi}{}

\href{https://www.nytimes.com/section/todayspaper}{Today's Paper}

\href{/section/upshot}{The Upshot}\textbar{}Kept at the Hospital on
Coronavirus Fears, Now Facing Large Medical Bills

\url{https://nyti.ms/39cGQzt}

\begin{itemize}
\item
\item
\item
\item
\item
\item
\end{itemize}

\href{https://www.nytimes.com/news-event/coronavirus?action=click\&pgtype=Article\&state=default\&region=TOP_BANNER\&context=storylines_menu}{The
Coronavirus Outbreak}

\begin{itemize}
\tightlist
\item
  live\href{https://www.nytimes.com/2020/08/04/world/coronavirus-cases.html?action=click\&pgtype=Article\&state=default\&region=TOP_BANNER\&context=storylines_menu}{Latest
  Updates}
\item
  \href{https://www.nytimes.com/interactive/2020/us/coronavirus-us-cases.html?action=click\&pgtype=Article\&state=default\&region=TOP_BANNER\&context=storylines_menu}{Maps
  and Cases}
\item
  \href{https://www.nytimes.com/interactive/2020/science/coronavirus-vaccine-tracker.html?action=click\&pgtype=Article\&state=default\&region=TOP_BANNER\&context=storylines_menu}{Vaccine
  Tracker}
\item
  \href{https://www.nytimes.com/2020/08/02/us/covid-college-reopening.html?action=click\&pgtype=Article\&state=default\&region=TOP_BANNER\&context=storylines_menu}{College
  Reopening}
\item
  \href{https://www.nytimes.com/live/2020/08/04/business/stock-market-today-coronavirus?action=click\&pgtype=Article\&state=default\&region=TOP_BANNER\&context=storylines_menu}{Economy}
\end{itemize}

Advertisement

\protect\hyperlink{after-top}{Continue reading the main story}

Upshot

Supported by

\protect\hyperlink{after-sponsor}{Continue reading the main story}

\hypertarget{kept-at-the-hospital-on-coronavirus-fears-now-facing-large-medical-bills}{%
\section{Kept at the Hospital on Coronavirus Fears, Now Facing Large
Medical
Bills}\label{kept-at-the-hospital-on-coronavirus-fears-now-facing-large-medical-bills}}

Care was mandated by the government, but it's not clear who has to pay.

By \href{https://www.nytimes.com/by/sarah-kliff}{Sarah Kliff}

\begin{itemize}
\item
  Published Feb. 29, 2020Updated March 10, 2020
\item
  \begin{itemize}
  \item
  \item
  \item
  \item
  \item
  \item
  \end{itemize}
\end{itemize}

\includegraphics{https://static01.nyt.com/images/2020/02/29/upshot/29up-virus-bills/29up-virus-bills-articleLarge.jpg?quality=75\&auto=webp\&disable=upscale}

Frank Wucinski and his 3-year-old daughter, Annabel, are among the
dozens of Americans the government has flown back to the country from
Wuhan, China, and put under quarantine to check for signs of
coronavirus.

Now they are among what could become a growing number of families hit
with surprise medical bills related to government-mandated actions.

Mr. Wucinski, a Pennsylvania native who has lived in China for years,
accepted the U.S. government's offer to evacuate from Wuhan with Annabel
in early February as the new coronavirus spread. His wife, who is not an
American citizen and remains in China, developed pneumonia that doctors
think resulted from Covid-19, the disease caused by the respiratory
virus. Her father, whom she helped care for, was infected and recently
died.

The first stop for Mr. Wucinski and Annabel was a two-week quarantine at
Marine Corps Station Miramar near San Diego. During that time, they had
\href{https://www.cbs8.com/article/news/health/coronavirus/girl-back-in-hospital-mcas-miramar-radys/509-c2a9e22e-0323-4152-be3b-3cf7720ebc3a}{two
mandatory stays} in an isolation unit at a nearby children's hospital.
The first started upon arrival in the United States, and the second was
a few days later, after an official heard Annabel coughing.

``The hospital staff were very nice, they brought us a lot of toys,''
Mr. Wucinski said. ``Each time it was three or four days. I love my
daughter to death, but being in the same room that long, she is not a
great conversationalist.''

Both have repeatedly tested negative for the virus.

After their release from quarantine, Mr. Wucinski and his daughter went
to stay with his mother in Harrisburg, Pa. That's where they found a
pile of medical bills waiting: \$3,918 in charges from hospital doctors,
radiologists and an ambulance company.

``I assumed it was all being paid for,'' Mr. Wucinski said. ``We didn't
have a choice. When the bills showed up, it was just a pit in my
stomach, like, `How do I pay for this?'''

\hypertarget{latest-updates-global-coronavirus-outbreak}{%
\section{\texorpdfstring{\href{https://www.nytimes.com/2020/08/04/world/coronavirus-cases.html?action=click\&pgtype=Article\&state=default\&region=MAIN_CONTENT_1\&context=storylines_live_updates}{Latest
Updates: Global Coronavirus
Outbreak}}{Latest Updates: Global Coronavirus Outbreak}}\label{latest-updates-global-coronavirus-outbreak}}

Updated 2020-08-05T04:01:36.184Z

\begin{itemize}
\tightlist
\item
  \href{https://www.nytimes.com/2020/08/04/world/coronavirus-cases.html?action=click\&pgtype=Article\&state=default\&region=MAIN_CONTENT_1\&context=storylines_live_updates\#link-762df92}{As
  talks drag on, McConnell signals openness to jobless aid extension,
  and negotiators agree on a deadline.}
\item
  \href{https://www.nytimes.com/2020/08/04/world/coronavirus-cases.html?action=click\&pgtype=Article\&state=default\&region=MAIN_CONTENT_1\&context=storylines_live_updates\#link-1228a480}{Novavax
  sees encouraging results from two studies of its experimental
  vaccine.}
\item
  \href{https://www.nytimes.com/2020/08/04/world/coronavirus-cases.html?action=click\&pgtype=Article\&state=default\&region=MAIN_CONTENT_1\&context=storylines_live_updates\#link-794484ed}{Mississippians
  must now wear masks in public, governor says.}
\end{itemize}

\href{https://www.nytimes.com/2020/08/04/world/coronavirus-cases.html?action=click\&pgtype=Article\&state=default\&region=MAIN_CONTENT_1\&context=storylines_live_updates}{See
more updates}

More live coverage:
\href{https://www.nytimes.com/live/2020/08/04/business/stock-market-today-coronavirus?action=click\&pgtype=Article\&state=default\&region=MAIN_CONTENT_1\&context=storylines_live_updates}{Markets}

Mr. Wucinski's employer, a standardized testing company, provided health
benefits when he lived in China but does not offer coverage in the
United States.

Patients in the United States
\href{https://www.nytimes.com/2020/02/19/opinion/surprise-medical-bill.html}{regularly
confront}surprise medical bills that are hard to decode. Mr. Wucinski's
case suggests that those held in mandatory isolation for suspected
coronavirus may be no exception.

The federal government has the authority to
\href{https://www.cdc.gov/quarantine/historyquarantine.html}{quarantine
and isolate patients} if officials believe them to be a public health
threat. These powers, which date back to cholera outbreaks among ship
passengers in the late 19th century, are rarely used. They don't say
anything about who pays when the isolation happens in a nongovernmental
medical facility --- or when they're brought there by a private
ambulance company.

``There is no uniform practice,'' said Lawrence Gostin, a professor of
global health law at Georgetown University. ``They do have the powers,
but they've almost never used them in modern times.''

Few patients have been held in mandatory isolation, but the number is
likely to grow if the coronavirus continues to spread across the United
States. Eleven cases were
\href{https://www.click2houston.com/news/texas/2020/02/29/11-cases-of-coronavirus-confirmed-in-san-antonio/?utm_source=twitter\&utm_medium=social\&utm_campaign=snd\&utm_content=kprc2}{confirmed}
in San Antonio on Friday evening. Earlier in the day, public health
authorities identified
\href{https://www.nytimes.com/2020/02/28/us/coronavirus-solano-county.html?action=click\&module=RelatedLinks\&pgtype=Article}{a
second case in California} and
\href{https://www.nytimes.com/aponline/2020/02/28/us/ap-us-virus-outbreak-oregon-4th-ld-writethru.html?searchResultPosition=9}{a
first in Oregon in which} patients who had not traveled to an affected
country became infected.

A Centers for Disease Control and Prevention spokesman declined to
comment on whether it would pay the bills of patients kept in mandatory
isolation.

Mr. Gostin worries that high charges for mandatory isolation could make
patients wary of seeking needed medical treatment.

``The most important rule of public health is to gain the cooperation of
the population,'' he said. ``There are legal, moral and public health
reasons not to charge the patients.''

These hospital stays could prove expensive. The International Federation
of Health Plans
\href{https://s3.amazonaws.com/assets.fiercemarkets.net/public/005-LifeSciences/ifhpreport.pdf}{estimates}
that the average day in a U.S. hospital costs \$4,293, compared with
\$1,308 in Australia and \$481 in Spain. The hospital stays may be
especially costly for patients without health insurance or for those who
have large deductibles, which they must pay before their health benefits
kick in.

Mr. Wucinski recalls other patients asking about how medical bills would
be handled during daily town hall meetings for those quarantined at the
Marine Corps station. He felt the answers weren't clear.

\href{https://www.nytimes.com/news-event/coronavirus?action=click\&pgtype=Article\&state=default\&region=MAIN_CONTENT_3\&context=storylines_faq}{}

\hypertarget{the-coronavirus-outbreak-}{%
\subsubsection{The Coronavirus Outbreak
›}\label{the-coronavirus-outbreak-}}

\hypertarget{frequently-asked-questions}{%
\paragraph{Frequently Asked
Questions}\label{frequently-asked-questions}}

Updated August 4, 2020

\begin{itemize}
\item ~
  \hypertarget{i-have-antibodies-am-i-now-immune}{%
  \paragraph{I have antibodies. Am I now
  immune?}\label{i-have-antibodies-am-i-now-immune}}

  \begin{itemize}
  \tightlist
  \item
    As of right
    now,\href{https://www.nytimes.com/2020/07/22/health/covid-antibodies-herd-immunity.html?action=click\&pgtype=Article\&state=default\&region=MAIN_CONTENT_3\&context=storylines_faq}{that
    seems likely, for at least several months.} There have been
    frightening accounts of people suffering what seems to be a second
    bout of Covid-19. But experts say these patients may have a
    drawn-out course of infection, with the virus taking a slow toll
    weeks to months after initial exposure. People infected with the
    coronavirus typically
    \href{https://www.nature.com/articles/s41586-020-2456-9}{produce}
    immune molecules called antibodies, which are
    \href{https://www.nytimes.com/2020/05/07/health/coronavirus-antibody-prevalence.html?action=click\&pgtype=Article\&state=default\&region=MAIN_CONTENT_3\&context=storylines_faq}{protective
    proteins made in response to an
    infection}\href{https://www.nytimes.com/2020/05/07/health/coronavirus-antibody-prevalence.html?action=click\&pgtype=Article\&state=default\&region=MAIN_CONTENT_3\&context=storylines_faq}{.
    These antibodies may} last in the body
    \href{https://www.nature.com/articles/s41591-020-0965-6}{only two to
    three months}, which may seem worrisome, but that's perfectly normal
    after an acute infection subsides, said Dr. Michael Mina, an
    immunologist at Harvard University. It may be possible to get the
    coronavirus again, but it's highly unlikely that it would be
    possible in a short window of time from initial infection or make
    people sicker the second time.
  \end{itemize}
\item ~
  \hypertarget{im-a-small-business-owner-can-i-get-relief}{%
  \paragraph{I'm a small-business owner. Can I get
  relief?}\label{im-a-small-business-owner-can-i-get-relief}}

  \begin{itemize}
  \tightlist
  \item
    The
    \href{https://www.nytimes.com/article/small-business-loans-stimulus-grants-freelancers-coronavirus.html?action=click\&pgtype=Article\&state=default\&region=MAIN_CONTENT_3\&context=storylines_faq}{stimulus
    bills enacted in March} offer help for the millions of American
    small businesses. Those eligible for aid are businesses and
    nonprofit organizations with fewer than 500 workers, including sole
    proprietorships, independent contractors and freelancers. Some
    larger companies in some industries are also eligible. The help
    being offered, which is being managed by the Small Business
    Administration, includes the Paycheck Protection Program and the
    Economic Injury Disaster Loan program. But lots of folks have
    \href{https://www.nytimes.com/interactive/2020/05/07/business/small-business-loans-coronavirus.html?action=click\&pgtype=Article\&state=default\&region=MAIN_CONTENT_3\&context=storylines_faq}{not
    yet seen payouts.} Even those who have received help are confused:
    The rules are draconian, and some are stuck sitting on
    \href{https://www.nytimes.com/2020/05/02/business/economy/loans-coronavirus-small-business.html?action=click\&pgtype=Article\&state=default\&region=MAIN_CONTENT_3\&context=storylines_faq}{money
    they don't know how to use.} Many small-business owners are getting
    less than they expected or
    \href{https://www.nytimes.com/2020/06/10/business/Small-business-loans-ppp.html?action=click\&pgtype=Article\&state=default\&region=MAIN_CONTENT_3\&context=storylines_faq}{not
    hearing anything at all.}
  \end{itemize}
\item ~
  \hypertarget{what-are-my-rights-if-i-am-worried-about-going-back-to-work}{%
  \paragraph{What are my rights if I am worried about going back to
  work?}\label{what-are-my-rights-if-i-am-worried-about-going-back-to-work}}

  \begin{itemize}
  \tightlist
  \item
    Employers have to provide
    \href{https://www.osha.gov/SLTC/covid-19/standards.html}{a safe
    workplace} with policies that protect everyone equally.
    \href{https://www.nytimes.com/article/coronavirus-money-unemployment.html?action=click\&pgtype=Article\&state=default\&region=MAIN_CONTENT_3\&context=storylines_faq}{And
    if one of your co-workers tests positive for the coronavirus, the
    C.D.C.} has said that
    \href{https://www.cdc.gov/coronavirus/2019-ncov/community/guidance-business-response.html}{employers
    should tell their employees} -\/- without giving you the sick
    employee's name -\/- that they may have been exposed to the virus.
  \end{itemize}
\item ~
  \hypertarget{should-i-refinance-my-mortgage}{%
  \paragraph{Should I refinance my
  mortgage?}\label{should-i-refinance-my-mortgage}}

  \begin{itemize}
  \tightlist
  \item
    \href{https://www.nytimes.com/article/coronavirus-money-unemployment.html?action=click\&pgtype=Article\&state=default\&region=MAIN_CONTENT_3\&context=storylines_faq}{It
    could be a good idea,} because mortgage rates have
    \href{https://www.nytimes.com/2020/07/16/business/mortgage-rates-below-3-percent.html?action=click\&pgtype=Article\&state=default\&region=MAIN_CONTENT_3\&context=storylines_faq}{never
    been lower.} Refinancing requests have pushed mortgage applications
    to some of the highest levels since 2008, so be prepared to get in
    line. But defaults are also up, so if you're thinking about buying a
    home, be aware that some lenders have tightened their standards.
  \end{itemize}
\item ~
  \hypertarget{what-is-school-going-to-look-like-in-september}{%
  \paragraph{What is school going to look like in
  September?}\label{what-is-school-going-to-look-like-in-september}}

  \begin{itemize}
  \tightlist
  \item
    It is unlikely that many schools will return to a normal schedule
    this fall, requiring the grind of
    \href{https://www.nytimes.com/2020/06/05/us/coronavirus-education-lost-learning.html?action=click\&pgtype=Article\&state=default\&region=MAIN_CONTENT_3\&context=storylines_faq}{online
    learning},
    \href{https://www.nytimes.com/2020/05/29/us/coronavirus-child-care-centers.html?action=click\&pgtype=Article\&state=default\&region=MAIN_CONTENT_3\&context=storylines_faq}{makeshift
    child care} and
    \href{https://www.nytimes.com/2020/06/03/business/economy/coronavirus-working-women.html?action=click\&pgtype=Article\&state=default\&region=MAIN_CONTENT_3\&context=storylines_faq}{stunted
    workdays} to continue. California's two largest public school
    districts --- Los Angeles and San Diego --- said on July 13, that
    \href{https://www.nytimes.com/2020/07/13/us/lausd-san-diego-school-reopening.html?action=click\&pgtype=Article\&state=default\&region=MAIN_CONTENT_3\&context=storylines_faq}{instruction
    will be remote-only in the fall}, citing concerns that surging
    coronavirus infections in their areas pose too dire a risk for
    students and teachers. Together, the two districts enroll some
    825,000 students. They are the largest in the country so far to
    abandon plans for even a partial physical return to classrooms when
    they reopen in August. For other districts, the solution won't be an
    all-or-nothing approach.
    \href{https://bioethics.jhu.edu/research-and-outreach/projects/eschool-initiative/school-policy-tracker/}{Many
    systems}, including the nation's largest, New York City, are
    devising
    \href{https://www.nytimes.com/2020/06/26/us/coronavirus-schools-reopen-fall.html?action=click\&pgtype=Article\&state=default\&region=MAIN_CONTENT_3\&context=storylines_faq}{hybrid
    plans} that involve spending some days in classrooms and other days
    online. There's no national policy on this yet, so check with your
    municipal school system regularly to see what is happening in your
    community.
  \end{itemize}
\end{itemize}

He did receive a document upon leaving quarantine directing him to
contact a government email address with any medical bills. He sent an
email on Feb. 24 detailing the charges and asking what would be done.

``My question is why are we being charged for these stays, if they were
mandatory and we had no choice in the matter?'' Mr. Wucinski wrote in
his message.

He has not received a response, he said. When contacted by The New York
Times, a Rady Children's Hospital spokesman said the physicians' bill
had been sent in error and that the family would not be held responsible
for the charges.

``We're in the process of assessing how the error occurred,'' the
spokesman, Benjamin Metcalf, said. ``We are working with government
agencies regarding billing for these cases.''

But the hospital bill represented only a fraction of those the family
received.

The ambulance company that transported the Wucinskis, American Medical
Response, charged the family \$2,598 for taking them to the hospital. A
company representative declined to comment on the bill ``due to patient
privacy concerns,'' but said the company would look into the case.

An additional \$90 in charges came from radiologists who read the
patients' X-ray scans and do not work for the hospital. Having such
doctors, who may be outside a patient's insurance networks, provide
services to hospital patients is one of the
\href{https://www.nytimes.com/2014/09/21/us/drive-by-doctoring-surprise-medical-bills.html}{major
causes} of surprise medical bills.

The radiologists' employer, San Diego Medical Imaging Group, did not
respond to a request for comment.

Mr. Wucinski is looking into whether he or his daughter might qualify
for Medicaid, the public insurance program that covers low-income
Americans. Some news outlets have
\href{https://nypost.com/2020/02/28/man-cleared-after-coronavirus-quarantine-cant-stop-coughing-in-tv-interview/}{seized
on} the fact that he coughed enough on a recent television interview to
require water.

Mr. Wucinski recently noticed that his daughter is blinking a lot and
hopes to get the issue examined by a doctor, but is worried about the
charges they may face. He has wondered whether the stress of the past
month and the separation from her mother has played a role.

``I should take her to the doctor this week for a checkup, but we don't
have insurance, so it's just going to have to be cash,'' he said.

Advertisement

\protect\hyperlink{after-bottom}{Continue reading the main story}

\hypertarget{site-index}{%
\subsection{Site Index}\label{site-index}}

\hypertarget{site-information-navigation}{%
\subsection{Site Information
Navigation}\label{site-information-navigation}}

\begin{itemize}
\tightlist
\item
  \href{https://help.nytimes.com/hc/en-us/articles/115014792127-Copyright-notice}{©~2020~The
  New York Times Company}
\end{itemize}

\begin{itemize}
\tightlist
\item
  \href{https://www.nytco.com/}{NYTCo}
\item
  \href{https://help.nytimes.com/hc/en-us/articles/115015385887-Contact-Us}{Contact
  Us}
\item
  \href{https://www.nytco.com/careers/}{Work with us}
\item
  \href{https://nytmediakit.com/}{Advertise}
\item
  \href{http://www.tbrandstudio.com/}{T Brand Studio}
\item
  \href{https://www.nytimes.com/privacy/cookie-policy\#how-do-i-manage-trackers}{Your
  Ad Choices}
\item
  \href{https://www.nytimes.com/privacy}{Privacy}
\item
  \href{https://help.nytimes.com/hc/en-us/articles/115014893428-Terms-of-service}{Terms
  of Service}
\item
  \href{https://help.nytimes.com/hc/en-us/articles/115014893968-Terms-of-sale}{Terms
  of Sale}
\item
  \href{https://spiderbites.nytimes.com}{Site Map}
\item
  \href{https://help.nytimes.com/hc/en-us}{Help}
\item
  \href{https://www.nytimes.com/subscription?campaignId=37WXW}{Subscriptions}
\end{itemize}
