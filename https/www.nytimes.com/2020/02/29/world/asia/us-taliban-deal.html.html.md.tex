Sections

SEARCH

\protect\hyperlink{site-content}{Skip to
content}\protect\hyperlink{site-index}{Skip to site index}

\href{https://www.nytimes.com/section/world/asia}{Asia Pacific}

\href{https://myaccount.nytimes.com/auth/login?response_type=cookie\&client_id=vi}{}

\href{https://www.nytimes.com/section/todayspaper}{Today's Paper}

\href{/section/world/asia}{Asia Pacific}\textbar{}Taliban and U.S.
Strike Deal to Withdraw American Troops From Afghanistan

\url{https://nyti.ms/2VtE6JX}

\begin{itemize}
\item
\item
\item
\item
\item
\item
\end{itemize}

Advertisement

\protect\hyperlink{after-top}{Continue reading the main story}

Supported by

\protect\hyperlink{after-sponsor}{Continue reading the main story}

\hypertarget{taliban-and-us-strike-deal-to-withdraw-american-troops-from-afghanistan}{%
\section{Taliban and U.S. Strike Deal to Withdraw American Troops From
Afghanistan}\label{taliban-and-us-strike-deal-to-withdraw-american-troops-from-afghanistan}}

After more than a year of talks, the agreement lays out the beginning of
the end of the United States' longest war. But many obstacles remain.

\includegraphics{https://static01.nyt.com/images/2020/02/29/world/29taliban-peace-1/merlin_126327182_8ad16e92-e9af-4e60-9193-f359e0edba67-articleLarge.jpg?quality=75\&auto=webp\&disable=upscale}

\href{https://www.nytimes.com/by/mujib-mashal}{\includegraphics{https://static01.nyt.com/images/2018/10/15/multimedia/author-mujib-mashal/author-mujib-mashal-thumbLarge.png}}

By \href{https://www.nytimes.com/by/mujib-mashal}{Mujib Mashal}

\begin{itemize}
\item
  Feb. 29, 2020
\item
  \begin{itemize}
  \item
  \item
  \item
  \item
  \item
  \item
  \end{itemize}
\end{itemize}

DOHA, Qatar --- The United States signed a deal with the Taliban on
Saturday that sets the stage to end America's longest war --- the nearly
two-decade-old conflict in Afghanistan that began after the Sept. 11
attacks, killed tens of thousands of people, vexed three White House
administrations and left mistrust and uncertainty on all sides.

The agreement lays out a timetable for the final withdrawal of United
States troops from Afghanistan, the impoverished Central Asian country
once unfamiliar to many Americans that now symbolizes endless conflict,
foreign entanglements and an incubator of terrorist plots.

The war in Afghanistan in some ways echoes the American experience in
Vietnam. In both, a superpower bet heavily on brute strength and the
lives of its young, then walked away with seemingly little to show.

American efforts to instill a democratic system in the country, and to
improve opportunities for women and minorities, are at risk if the
Taliban, which banned girls from schools and women from public life,
become dominant again. Corruption is still rampant, the country's
institutions are feeble, and the economy is heavily dependent on
American and other international aid.

The agreement signed in Doha, Qatar, which followed more than a year of
stop-and-start negotiations and conspicuously excluded the
American-backed Afghanistan government, is not a final peace deal, is
filled with ambiguity, and could still unravel.

But it is seen as a step toward negotiating a more sweeping agreement
that some hope could eventually end the insurgency of the Taliban, the
militant movement that once ruled Afghanistan under a severe Islamic
code.

The war cost \$2 trillion and took the lives of more than 3,500 American
and coalition troops and tens of thousands of Afghans since the U.S.
invasion in aftermath of the Sept. 11 attacks, which were plotted by Al
Qaeda leaders under the protection of the Taliban.

The withdrawal of American troops --- about 12,000 are still in
Afghanistan --- is dependent on the Taliban's fulfillment of major
commitments that have been obstacles for years, including its severance
of ties with international terrorist groups such as Al Qaeda.

The agreement also hinges on more difficult negotiations to come between
the Taliban and the Afghan government over the country's future.
Officials hope those talks will produce a power-sharing arrangement and
lasting cease-fire, but both ideas have been anathema to the Taliban in
the past.

\includegraphics{https://static01.nyt.com/images/2020/03/29/world/29taliban-peace-1/merlin_169765173_aa1e72ed-a34e-403b-9044-9a77781170cb-articleLarge.jpg?quality=75\&auto=webp\&disable=upscale}

``I really believe the Taliban wants to do something to show that we're
not all wasting time,'' President Trump said in Washington hours after
the agreement had been signed. ``If bad things happen, we'll go back.''

In rambling remarks on Afghanistan at a news conference on the
coronavirus epidemic, Mr. Trump also seemed to suggest that the Taliban
might be America's newfound allies.

``I'll be meeting personally with Taliban leaders in the not-too-distant
future, and will be very much hoping that they will be doing what they
say,'' the president said. ``They will be killing terrorists. They will
be killing some very bad people. They will keep that fight going.''

Secretary of State Mike Pompeo, who was in Doha for the signing
ceremony, seemed more cautious in his assessment of the Taliban's future
behavior.

``The agreement will mean nothing --- and today's good feelings will not
last --- if we don't take concrete action on commitments stated and
promises made,'' Mr. Pompeo said.

The Trump administration has framed the deal as the long-awaited promise
made to war-weary Americans, for whom
\href{https://www.nytimes.com/2020/02/29/world/asia/afghanistan-war-photos-pictures.html}{the
Afghan war has defined a generation of loss and trauma} but has yielded
no victory.

At the height of the war, more than 100,000 American troops occupied
Afghanistan, as did tens of thousands from about 40 nations in the
United States-led NATO coalition.

The war has gone on so long --- the first allied warplane and cruise
missiles struck on Oct. 7, 2001, and American boots hit the ground in
numbers on Oct. 19 --- that many young Afghan soldiers and their
coalition partners have no memory of its onset.

Retaliation against Al Qaeda and its allies among the Taliban was the
catalyst that drove the American invasion. But it has been a dawning
sense of futility, perhaps best demonstrated in the American acceptance
of relatively small concessions from Taliban in the agreement, that has
driven
\href{https://www.nytimes.com/2020/02/29/world/asia/trump-taliban.html?}{efforts
of successive administrations} to find a way out.

Image

Waiters and customers at a shop in Kabul were glued to the TV to watch
the historic deal signing on Saturday.Credit...Kiana Hayeri for The New
York Times

Even in the description of Al Qaeda in the agreement, the Taliban
refused to accept the word ``terrorist.'' The language focuses on the
Taliban's commitment to prevent future attacks, rather than any regrets
over the past.

From the start of the talks, late in 2018, Afghan officials were
troubled that the Taliban had blocked them from participating. They
worried that Mr. Trump would abruptly withdraw troops without securing
conditions they saw as crucial, including a
\href{https://www.nytimes.com/2020/02/21/world/asia/afghanistan-cease-fire-peace-talks.html}{reduction
in violence} and a Taliban promise to negotiate in good faith with the
government.

The chief American envoy, Zalmay Khalilzad, signed on behalf of the
United States. Mullah Abdul Ghani Baradar, a current Taliban deputy and
a figure from the original Taliban government, signed for the Taliban.
The two shook hands as the room erupted in cheers.

Some Taliban members in attendance chanted ``Allahu akbar,'' or ``God is
great,'' a cry of victory.

More than 1,200 miles away during the signing, another senior American
official, Defense Secretary Mark T. Esper, was with Afghan officials in
Kabul to ease the Afghan government's concerns. Joined by NATO Secretary
General Jens Stoltenberg, they issued a declaration asserting the United
States' commitment to helping sustain the Afghan military.

Mr. Esper emphasized that if the Taliban violated pledges, ``the United
States would not hesitate to nullify the agreement.''

President Ashraf Ghani of Afghanistan called for a moment of silence for
war's victims and said, ``Today can be a day of overcoming the past.''

The best-case prospect laid out by the deal signed on Saturday could go
far beyond America's disengagement. It raised the hope of ending a
conflict that began more than 20 years before the United States
invasion, when the Soviet Union's forces invaded the country and the
United States began supporting the guerrilla resistance against them.

But behind the hope lies a web of contradictions, and a large degree of
ambiguity that has Afghans worried.

The United States, which struggled to help secure better rights for
women and minorities and instill a democratic system and institutions in
Afghanistan, has struck a deal with
\href{https://www.nytimes.com/2019/06/28/world/asia/taliban-peace-talks-constitution.html}{an
insurgency that has never clearly renounced its desire} for a government
and justice system rooted in a severe interpretation of Islam.

Though the Taliban get their primary wish under this agreement --- the
withdrawal of American troops --- they have remained vague in
commitments to protect the civil rights that they had brutally repressed
when in power.

Image

President Trump addressed U.S. troops at Bagram Air Field in Afghanistan
in November.Credit...Erin Schaff/The New York Times

Among the Taliban, bringing the world's strongest military power to the
humbling point of withdrawal has widely been seen as a victory. A day
before the signing ceremony at the Doha Sheraton hotel, the Taliban's
multimedia chief described it as a historic landmark for proclaiming
``the defeat of the arrogance of the White House in the face of the
white turban.''

But at the signing ceremony, Mr. Pompeo warned the Taliban to moderate
their celebration.

``I know there will be a temptation to declare victory,'' he said. ``But
victory for Afghans can only be achieved if they can live in peace and
prosper.''

The deal's conditional schedule for the withdrawal of the remaining
American troops specifies that in the first phase, nearly 5,000 are to
leave Afghanistan in 135 days. The withdrawal of the rest, to be
completed within 14 months of the signing, will depend on the Taliban
keeping its end of the bargain.

The insurgents pledged to keep
\href{https://www.nytimes.com/2019/03/07/world/asia/taliban-peace-talks-afghanistan.html}{international
terrorist networks} such as Al Qaeda from using Afghanistan as a base
for attacks. And the United States pledged to work toward the gradual
removal of Taliban leaders from both American and United Nations
sanctions blacklists.

But the deal leaves an awkward reality for the Trump administration: It
has signed an agreement with a movement in which an officially listed
terrorist group, the Haqqani Network, known for its campaign of suicide
bombings, is integral to the leadership. The network's leader,
Sirajuddin Haqqani, is the Taliban's deputy leader and military
commander.

The United States also committed to seek the release of 5,000 Taliban
prisoners, held by the Afghan government, and 1,000 members of
government security forces from the Taliban side by March 10 --- less
than two weeks away --- before the Afghan and Taliban sides are expected
to start direct negotiations.

While American diplomats had pushed for a cease-fire, they settled for
what they called a ``reduction in violence'' and
\href{https://www.nytimes.com/2020/02/27/world/asia/afghanistan-taliban-peace.html}{tested
it over seven days} before the signing. Officials said attacks had
dropped by as much as 80 percent.

Image

The aftermath of a Taliban bombing in Kabul, in July.Credit...Jim
Huylebroek for The New York Times

With the signing of the deal, the U.S. and Taliban sides clearly stated
their commitment to not attack each other. Just how much the Taliban
will hold fire on Afghan security forces before a cease-fire is reached
in Afghan negotiations remains a point of uncertainty and worry.

In recent years, the brunt of fighting has been borne by Afghan soldiers
and police officers, many of them American-trained. But even some of
them came to see U.S. troops as invaders, turning their guns on their
American and NATO partners. More than 150 American and NATO troops have
been killed in such ``green-on-blue'' attacks, including two American
service members
\href{https://www.nytimes.com/2020/02/17/world/asia/afghanistan-peace-american-deaths.html}{gunned
down this month}.

Mr. Khalilzad, the veteran diplomat leading the American peace efforts
and himself a native of Afghanistan, long insisted that the United
States was not seeking a withdrawal agreement, but ``a peace agreement
that enables withdrawal.''

The Taliban's willingness to enter negotiations with other Afghans,
including the government, over a political settlement has offered both
hope and fear to the Afghan people.

The hope is that some kind of lasting peace can be reached. The fear is
that the most difficult work lies ahead, and that the Taliban will be
emboldened by the American withdrawal announcement to challenge a
\href{https://www.nytimes.com/2020/02/18/world/asia/afghanistan-election-ashraf-ghani.html}{bitterly
divided government} in Kabul.

Much of the peace negotiations happened in a year of record violence
from both sides. In just the last quarter of 2019, the Taliban carried
out 8,204 attacks, the highest for that period over the past decade. The
United States
dropped\href{https://www.nytimes.com/2020/01/31/world/asia/afghanistan-violence-taliban.html}{7,423
bombs and missiles} during the year, a record since the Air Force began
recording the data in 2006.

In the past five years, more than
\href{https://www.nytimes.com/2018/09/21/world/asia/afghanistan-security-casualties-taliban.html}{50,000
members of the Afghan security forces} have been killed, and tens of
thousands wounded. The Taliban's losses are harder to verify, but their
casualty rate is believed to be comparable. Out of about 3,550 NATO
coalition deaths in Afghanistan, nearly 2,400 have been Americans.

Mr. Khalilzad, the chief U.S. negotiator, struck an optimistic but
somber tone.

``Today is a day for hope,'' he said. ``Today is a day to remember. We
must remember the lessons of history, and the darkness of conflict.''

Image

American personnel over Helmand Province, in southern Afghanistan, in
September.Credit...Jim Huylebroek for The New York Times

Thomas Gibbons-Neff contributed reporting from Kabul, Afghanistan, and
Lara Jakes from Washington.

Advertisement

\protect\hyperlink{after-bottom}{Continue reading the main story}

\hypertarget{site-index}{%
\subsection{Site Index}\label{site-index}}

\hypertarget{site-information-navigation}{%
\subsection{Site Information
Navigation}\label{site-information-navigation}}

\begin{itemize}
\tightlist
\item
  \href{https://help.nytimes.com/hc/en-us/articles/115014792127-Copyright-notice}{©~2020~The
  New York Times Company}
\end{itemize}

\begin{itemize}
\tightlist
\item
  \href{https://www.nytco.com/}{NYTCo}
\item
  \href{https://help.nytimes.com/hc/en-us/articles/115015385887-Contact-Us}{Contact
  Us}
\item
  \href{https://www.nytco.com/careers/}{Work with us}
\item
  \href{https://nytmediakit.com/}{Advertise}
\item
  \href{http://www.tbrandstudio.com/}{T Brand Studio}
\item
  \href{https://www.nytimes.com/privacy/cookie-policy\#how-do-i-manage-trackers}{Your
  Ad Choices}
\item
  \href{https://www.nytimes.com/privacy}{Privacy}
\item
  \href{https://help.nytimes.com/hc/en-us/articles/115014893428-Terms-of-service}{Terms
  of Service}
\item
  \href{https://help.nytimes.com/hc/en-us/articles/115014893968-Terms-of-sale}{Terms
  of Sale}
\item
  \href{https://spiderbites.nytimes.com}{Site Map}
\item
  \href{https://help.nytimes.com/hc/en-us}{Help}
\item
  \href{https://www.nytimes.com/subscription?campaignId=37WXW}{Subscriptions}
\end{itemize}
