Sections

SEARCH

\protect\hyperlink{site-content}{Skip to
content}\protect\hyperlink{site-index}{Skip to site index}

\href{https://www.nytimes.com/section/nyregion}{New York}

\href{https://myaccount.nytimes.com/auth/login?response_type=cookie\&client_id=vi}{}

\href{https://www.nytimes.com/section/todayspaper}{Today's Paper}

\href{/section/nyregion}{New York}\textbar{}Coronavirus in N.Y.C.: Why
Closing Public Schools Is a `Last Resort'

\url{https://nyti.ms/32Xd5QN}

\begin{itemize}
\item
\item
\item
\item
\item
\end{itemize}

\hypertarget{schools-reopening}{%
\subsubsection{\texorpdfstring{\href{https://www.nytimes.com/spotlight/schools-reopening?name=styln-coronavirus-schools-reopening\&region=TOP_BANNER\&variant=undefined\&block=storyline_menu_recirc\&action=click\&pgtype=Article\&impression_id=c34505d0-e108-11ea-b2b2-f779a84a62c4}{Schools
Reopening}}{Schools Reopening}}\label{schools-reopening}}

\begin{itemize}
\tightlist
\item
  \href{https://www.nytimes.com/2020/08/17/us/k-12-schools-reopening.html?name=styln-coronavirus-schools-reopening\&region=TOP_BANNER\&variant=undefined\&block=storyline_menu_recirc\&action=click\&pgtype=Article\&impression_id=c34505d1-e108-11ea-b2b2-f779a84a62c4}{State
  of Play for K-12}
\item
  \href{https://www.nytimes.com/2020/08/15/us/covid-college-tuition.html?name=styln-coronavirus-schools-reopening\&region=TOP_BANNER\&variant=undefined\&block=storyline_menu_recirc\&action=click\&pgtype=Article\&impression_id=c3452ce0-e108-11ea-b2b2-f779a84a62c4}{College
  Costs}
\item
  \href{https://www.nytimes.com/2020/08/14/us/covid-schools-learning-pods.html?name=styln-coronavirus-schools-reopening\&region=TOP_BANNER\&variant=undefined\&block=storyline_menu_recirc\&action=click\&pgtype=Article\&impression_id=c3452ce1-e108-11ea-b2b2-f779a84a62c4}{Priced
  Out of Learning Pods}
\item
  \href{https://www.nytimes.com/2020/08/14/nyregion/school-reopening-nyc.html?name=styln-coronavirus-schools-reopening\&region=TOP_BANNER\&variant=undefined\&block=storyline_menu_recirc\&action=click\&pgtype=Article\&impression_id=c3452ce2-e108-11ea-b2b2-f779a84a62c4}{N.Y.C.
  Schools Not Ready}
\item
  \href{https://www.nytimes.com/2020/08/05/parenting/parents-distance-learning.html?name=styln-coronavirus-schools-reopening\&region=TOP_BANNER\&variant=undefined\&block=storyline_menu_recirc\&action=click\&pgtype=Article\&impression_id=c3452ce3-e108-11ea-b2b2-f779a84a62c4}{Prepare
  for Distance Learning}
\end{itemize}

Advertisement

\protect\hyperlink{after-top}{Continue reading the main story}

Supported by

\protect\hyperlink{after-sponsor}{Continue reading the main story}

\hypertarget{coronavirus-in-nyc-why-closing-public-schools-is-a-last-resort}{%
\section{Coronavirus in N.Y.C.: Why Closing Public Schools Is a `Last
Resort'}\label{coronavirus-in-nyc-why-closing-public-schools-is-a-last-resort}}

The city's schools will probably stay open because they double as social
service centers for hundreds of thousands of poor students.

\includegraphics{https://static01.nyt.com/images/2020/03/06/nyregion/06nyvirus-schools01/merlin_169594782_9fc9f9f5-95f1-4405-93ca-1a37b00ca818-articleLarge.jpg?quality=75\&auto=webp\&disable=upscale}

\href{https://www.nytimes.com/by/eliza-shapiro}{\includegraphics{https://static01.nyt.com/images/2018/12/28/multimedia/author-eliza-shapiro/author-eliza-shapiro-thumbLarge.png}}

By \href{https://www.nytimes.com/by/eliza-shapiro}{Eliza Shapiro}

\begin{itemize}
\item
  Published March 7, 2020Updated April 16, 2020
\item
  \begin{itemize}
  \item
  \item
  \item
  \item
  \item
  \end{itemize}
\end{itemize}

\emph{{[}Update: On Monday, several suburban schools and private
universities}
\href{https://www.nytimes.com/2020/03/09/nyregion/coronavirus-new-york.html}{\emph{announced
closings}}\emph{.{]}}

New York City has the largest public school system in the United States,
a vast district with about 750,000 children who are poor,
\href{https://www.nytimes.com/interactive/2019/11/19/nyregion/student-homelessness-nyc.html}{including
around 114,000 who are homeless}.

For such students,
\href{https://www.nytimes.com/2020/04/16/nyregion/special-education-coronavirus-nyc.html}{school}
may be the only place they can get three hot meals a day and medical
care, and even wash their dirty laundry.

That is why the city's public schools will probably stay open even if
the
\href{https://www.nytimes.com/2020/03/07/nyregion/coronavirus-new-york-queens.html}{new
coronavirus becomes more widespread in New York}. Richard A. Carranza,
the schools chancellor, said earlier this week that he considered
long-term closings an ``extreme'' measure and a ``last resort.''

There are no plans to shut schools down, and Mayor Bill de Blasio said
on Friday that none of the city's 1.1 million public school students had
shown any symptoms of the virus. The federal Centers for Disease Control
and Prevention have advised that, so far,
\href{https://www.cdc.gov/coronavirus/2019-ncov/specific-groups/children-faq.html}{children
have been less likely than adults to become infected}.

Even a single snow day can seriously disrupt the lives of New York's
most vulnerable children and their parents and other relatives, whose
jobs often do not provide paid time off, said Aaron Pallas, a professor
of education at Columbia University's Teachers College.

``Kids will need to be supervised,'' Professor Pallas said. ``And there
are complex interactions here that affect the well-being of families.''

Large-scale
\href{https://www.nytimes.com/2020/04/16/nyregion/special-education-coronavirus-nyc.html}{school}
closings might mean, for example, that subway conductors and bus drivers
must stay home with their children, or that nurses at public hospitals
would not be able to come to work, potentially slowing essential city
services.

\emph{{[}Read more:}
\href{https://www.nytimes.com/2020/03/08/nyregion/coronavirus-nyc.html}{\emph{New
York in the Age of Coronavirus}}\emph{{]}}

\href{https://www.nytimes.com/2020/03/04/world/coronavirus-schools-closed.html}{Although
millions of students around the world} have already had their schools
close because of the virus, such a move would present a major challenge
for a district where many children do not have internet access at home,
making remote learning nearly impossible.

Nicole Manning, a ninth-grade math teacher at Herbert H. Lehman High
School in the Bronx, estimated that up to half of her students did not
have internet access at home.

``We can't do distance learning,'' she said. ``It wouldn't be fair.''

Valerie Green-Thomas, a teachers' coach at Middle School 390 in the
South Bronx, said she would be concerned that students would not have
access to crucial medical help at the school's on-site clinic if there
were widespread closings.

``We have a lot of underserved kids,'' Ms. Green-Thomas said.

The situation has been starkly different thus far at some of the city's
elite private schools, where the student bodies tend to be much whiter
and wealthier than they are in public schools.

Spence, an all-girls school on Manhattan's Upper East Side, closed on
Friday for a ``comprehensive sanitization of the entire campus,''
according to a notice posted on its website. It was unclear whether the
school had a link to one of New York State's confirmed coronavirus
cases. School representatives did not respond to requests for comment.

Collegiate, a private all-boys school on the Upper West Side, was also
closed on Friday for a similar purpose. An email to families from the
school's headmaster did not indicate any connections to a confirmed
case, but said that a parent of one student might have been exposed to
the virus.

Private schools can decide to close independently, but public schools
must follow guidance from the
\href{https://www.nytimes.com/2020/03/08/nyregion/coronavirus-newyork.html}{city
and state education departments}.

In interviews, public-school teachers across the city exuded calm and
said that they believed school was a safe place for children to be given
the current circumstances. It appeared that most parents agreed: Student
attendance rates were as high if not higher this past week than they
were a year ago at this time, Mr. de Blasio said.

Teachers said that, at this point, they were much more concerned about
racism and xenophobia directed at Asian students because of the virus's
origins in China than they were with the virus itself.

\href{https://www.nytimes.com/spotlight/schools-reopening?action=click\&pgtype=Article\&state=default\&region=MAIN_CONTENT_3\&context=storylines_keepup}{}

\hypertarget{schools-reopening-}{%
\subsubsection{Schools Reopening ›}\label{schools-reopening-}}

\hypertarget{back-to-school}{%
\paragraph{Back to School}\label{back-to-school}}

Updated Aug. 17, 2020

The latest on how schools are navigating an uncertain season.

\begin{itemize}
\item
  \begin{itemize}
  \tightlist
  \item
    Universities across the country are facing
    \href{https://www.nytimes.com/2020/08/15/us/covid-college-tuition.html?action=click\&pgtype=Article\&state=default\&region=MAIN_CONTENT_3\&context=storylines_keepup}{a
    rising demand for tuition rebates} as students ask if college is
    becoming ``glorified Skype.''
  \item
    In Los Angeles, the nation's second-largest school district has
    \href{https://www.nytimes.com/2020/08/16/us/los-angeles-schools-virus-testing.html?action=click\&pgtype=Article\&state=default\&region=MAIN_CONTENT_3\&context=storylines_keepup}{perhaps
    the most ambitious plan in the country} to test for the coronavirus.
  \item
    Families
    \href{https://www.nytimes.com/2020/08/14/us/covid-schools-learning-pods.html?action=click\&pgtype=Article\&state=default\&region=MAIN_CONTENT_3\&context=storylines_keepup}{priced
    out of ``learning pods'' are seeking alternatives}.
  \item
    How are campus newspapers covering back to school?
    \href{https://www.nytimes.com/2020/08/17/us/student-newspaper-schools-reopening.html?action=click\&pgtype=Article\&state=default\&region=MAIN_CONTENT_3\&context=storylines_keepup}{We
    want to hear from student journalists}.
  \end{itemize}
\end{itemize}

Ms. Manning is used to nasty stomach bugs and seasonal flus spreading
through her school like wildfire.

``We have good hygiene, and we don't really do much different,'' she
said, adding that students were being asked to be especially vigilant
about wiping down their calculators and desks, and about washing their
hands.

``I'm a rational person, I'm a math person,'' Ms. Manning said, noting
that the small number of confirmed cases in New York City had not yet
been a cause for alarm.

But she also said that she was spending much of her time ``squelching
rumors'' about where the virus comes from and how people contract it.
``I don't really put up with nonsense,'' she said.

Lynn Shon, a science teacher at Middle School 88 in Brooklyn, which has
a large Asian-American population, said that ``with crisis often comes
opportunity.'' After one her students blurted out in class that ``bat
soup'' in China was the source of the virus, and indicated that she was
disgusted by the idea, Ms. Shon, who is Asian-American, was distraught.

Later that day, she set to work making a
\href{https://docs.google.com/presentation/d/1HGQ3_xUlLwSstRVsn9XvQzfs9ucib55yf6IrheZhQaM/edit\#slide=id.g70fd9b34e6_0_198}{presentation
about the virus that she could share with her class and other teachers}.

``It's very obvious that the students want to understand this,'' she
said. ``Not every child has an adult that's able to talk about it.''

Younger students have also been coming to school confused and sometimes
fearful, said Deirdre Levy, a third-grade special education teacher at
Public School 9 in Brooklyn.

Ms. Levy, who is of Filipino ancestry, recently held a morning meeting
for her students and asked them what they thought coronavirus was. She
heard responses about bats and wild animals, and spent the morning
reassuring students that it was safe to come to school and to go about
their normal lives.

``When it comes to the coronavirus,'' she said, ``it's better to
educate.''

Advertisement

\protect\hyperlink{after-bottom}{Continue reading the main story}

\hypertarget{site-index}{%
\subsection{Site Index}\label{site-index}}

\hypertarget{site-information-navigation}{%
\subsection{Site Information
Navigation}\label{site-information-navigation}}

\begin{itemize}
\tightlist
\item
  \href{https://help.nytimes.com/hc/en-us/articles/115014792127-Copyright-notice}{©~2020~The
  New York Times Company}
\end{itemize}

\begin{itemize}
\tightlist
\item
  \href{https://www.nytco.com/}{NYTCo}
\item
  \href{https://help.nytimes.com/hc/en-us/articles/115015385887-Contact-Us}{Contact
  Us}
\item
  \href{https://www.nytco.com/careers/}{Work with us}
\item
  \href{https://nytmediakit.com/}{Advertise}
\item
  \href{http://www.tbrandstudio.com/}{T Brand Studio}
\item
  \href{https://www.nytimes.com/privacy/cookie-policy\#how-do-i-manage-trackers}{Your
  Ad Choices}
\item
  \href{https://www.nytimes.com/privacy}{Privacy}
\item
  \href{https://help.nytimes.com/hc/en-us/articles/115014893428-Terms-of-service}{Terms
  of Service}
\item
  \href{https://help.nytimes.com/hc/en-us/articles/115014893968-Terms-of-sale}{Terms
  of Sale}
\item
  \href{https://spiderbites.nytimes.com}{Site Map}
\item
  \href{https://help.nytimes.com/hc/en-us}{Help}
\item
  \href{https://www.nytimes.com/subscription?campaignId=37WXW}{Subscriptions}
\end{itemize}
