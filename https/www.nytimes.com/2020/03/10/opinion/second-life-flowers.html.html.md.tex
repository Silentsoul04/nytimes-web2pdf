Sections

SEARCH

\protect\hyperlink{site-content}{Skip to
content}\protect\hyperlink{site-index}{Skip to site index}

\href{https://myaccount.nytimes.com/auth/login?response_type=cookie\&client_id=vi}{}

\href{https://www.nytimes.com/section/todayspaper}{Today's Paper}

\href{/section/opinion}{Opinion}\textbar{}A Second Life for Flowers

\href{https://nyti.ms/2TDh2XG}{https://nyti.ms/2TDh2XG}

\begin{itemize}
\item
\item
\item
\item
\item
\item
\end{itemize}

Advertisement

\protect\hyperlink{after-top}{Continue reading the main story}

\href{/section/opinion}{Opinion}

Supported by

\protect\hyperlink{after-sponsor}{Continue reading the main story}

FIXES

\hypertarget{a-second-life-for-flowers}{%
\section{A Second Life for Flowers}\label{a-second-life-for-flowers}}

As a form of therapy, arranging gently used blooms is enriching the
lives of older and marginalized people.

By Miriam Zoila Pérez

The writer addresses issues of race, health and gender.

\begin{itemize}
\item
  March 10, 2020
\item
  \begin{itemize}
  \item
  \item
  \item
  \item
  \item
  \item
  \end{itemize}
\end{itemize}

\includegraphics{https://static01.nyt.com/images/2020/03/10/opinion/10FIXESZoilaPerez3/10FIXESZoilaPerez3-articleLarge.jpg?quality=75\&auto=webp\&disable=upscale}

The flowers come from all sorts of places: weddings, farmer's markets,
the online florist company UrbanStems, even a first lady's luncheon. In
the truck she calls her ``Bloom Mobile 1.0,'' Kaifa Anderson-Hall
recovers flowers from venues all over the Washington, D.C., area that
would otherwise be discarded and gives them a purpose. After being
preserved, often for weeks in refrigerators she keeps in the basement of
a four-unit apartment building she and her husband own, those flowers
take on their second job --- in therapeutic activities with seniors,
homeless women and people with disabilities around the city.

Part education, part art therapy and part wellness activity, Ms.
Anderson-Hall's nonprofit,
\href{http://www.plantsandbloomsreimagined.org/p/home.html}{Plants and
Blooms Reimagined}, represents an enticing intersection of a number of
pressing social issues: conservation; combating isolation for elders,
people with disabilities and other marginalized communities; and
wellness. Ms. Anderson-Hall is a horticultural therapist, a practice
based on the idea that working with plants, both indoors and outside,
can have therapeutic benefits.

Ms. Anderson-Hall grew up in a Washington public housing complex where,
she said, children weren't allowed to play in the grass, chased off by
property managers who wanted to preserve the aesthetic around the
buildings. But she also lived just a few blocks from the 400-plus-acre
U.S. National Arboretum. ``We spent many days being free in this green
space,'' Ms. Anderson-Hall recalled. But it wasn't until she joined a
school program in fifth grade that she was formally invited into the
arboretum, or more specifically, the Washington Youth Garden within it.
There, she learned to grow food and manage her own 4-foot-by-6-foot
garden plot, and decades later, she served as the director of that youth
garden for six years.

Ms. Anderson-Hall became a social worker. Her positive experiences as a
young person led her to creating and maintaining community and school
garden spaces around Washington. But four years ago she had a
realization: The people who most needed access to those spaces were also
the least likely to seek them out. That realization was inspired by a
chance sighting of a vehicle emblazoned with ``Children's Blood Mobile''
passing on the street. But instead of a ``D,'' she envisioned an ``M,''
making it a Bloom Mobile. ``It was very clear that my mission was to
take this experience to where people are --- those who can't get out for
myriad reasons.''

So rather than cultivating gardens and other green spaces, Ms.
Anderson-Hall's social work shifted its focus toward bringing nature to
people who are marginalized, isolated and in need of support.

\includegraphics{https://static01.nyt.com/images/2020/03/10/opinion/10FIXESZoilaPerez2/merlin_169912014_def83ea1-d4d9-4f61-b879-232e47e4f9d4-articleLarge.jpg?quality=75\&auto=webp\&disable=upscale}

A short time later, Ms. Anderson-Hall connected with Sidra Forman, a
local floral designer. Ms. Forman introduced her to a little-noticed
resource that could facilitate this next phase of her work: 30 to 40
events daily in and around Washington where gorgeous and expensive
flowers are arranged and displayed for a few hours before being
discarded --- well before reaching their peak. By working with Ms.
Anderson-Hall, floral designers like Ms. Forman could offer their
clients the added bonus of feeling good about what happens to the
flowers once an event is over.

The reuse of flowers from weddings and other events has been growing in
recent years. There are services and nonprofit organizations around the
country that will help turn event flowers into bouquets for hospitals,
nursing homes and the like. \href{https://repeatroses.com/}{Repeat
Roses}, founded in 2014, will even ensure that the donated bouquets get
composted once they reach the end of their glory. And Ms. Anderson-Hall
has taken one important element of this work a step further: Her
donations become the material for therapeutic workshops in floral
arrangement for the predominantly low-income and African-American people
she works with. ``We're a throwaway society,'' Ms. Anderson-Hall said.
``I want them to recognize there is still beauty in something that is
gently loved.''

The \href{https://www.ahta.org/}{American Horticultural Therapy
Association} is the institutional home for people like Ms. Anderson-Hall
whose work focuses on the therapeutic benefits of working with plants.
The organization has almost 500 members, says Matthew Wichrowski, who
leads its research team. He's also a senior horticultural therapist and
clinical assistant professor at N.Y.U. Langone Health, where research is
conducted about the therapeutic benefit of work like Ms.
Anderson-Hall's. In 2005, Mr. Wichrowski led the writing of a study that
found that among patients in a cardiac rehabilitation unit, those who
participated in a horticulture therapy activity showed a positive impact
on their mood. Another
\href{https://journals.sagepub.com/doi/pdf/10.1177/147470490500300109}{2005
study} conducted by Rutgers University showed that flowers presented to
older participants ``elicited positive mood reports and improved
episodic memory.''

In a \href{https://www.ncbi.nlm.nih.gov/pubmed/21273226}{2009 literature
review} in the Scandinavian Journal of Public Health, researchers
examined 38 studies published between 1980 and 2009 of nature-assisted
therapy, which is defined as therapy that ``involves plants, natural
materials and/or outdoor environment.'' The review found that
``significant improvements were found for varied outcomes in diverse
diagnoses, spanning from obesity to schizophrenia.''

While there is additional research that supports the health benefits of
exposure to plants and nature, it's still a limited body of research,
and Mr. Wichrowski acknowledges that more is needed to further
legitimize the field. He said that much of horticultural therapy ``is
intuitive, but until you prove it in a scientific fashion, it's not
accepted by the medical community.''

But Ms. Anderson-Hall said she already sees the positive impacts in each
of her workshops. For example, with an older woman at Pleasant Homes, a
rental community in Maryland that offers programming for residents, Ms.
Anderson-Hall said she has seen improvement in her participant's
dexterity in the time they've been working together. This participant
walks quite slowly and has a significant curve in her upper spine. When
Ms. Anderson-Hall helps her with the bouquet being arranged, she brings
the flower up so that her helper has to lift her head. Ms. Anderson-Hall
also encourages her to use the heavier scissors when she's trimming the
stems of her flowers, which Ms. Anderson-Hall says helps with her hand
strength.

While those specific improvements could come from any kind of arts and
crafts activity, Ms. Anderson-Hall feels that working with live flowers
is at the center of her work's effectiveness.

Image

Lucille Watson at the Model Cities center.Credit...Justin T. Gellerson
for The New York Times

``All of the senses are activated when you're working with live plant
materials,'' she said. This is particularly important in her work at the
Seabury Center for the Blind \& Visually Impaired, where her
participants use touch and smell to identify and arrange their bouquets.
She also relies heavily on metaphors drawn from the living plants as
tools in her workshops. ``You can move through so many more metaphors,''
she said, ``when working with live plant material in terms of the cycle
of life that you can't do with plant material that isn't alive.''

Ms. Anderson-Hall also does a monthly workshop at N Street Village, a
day center for homeless women. There, she has seen the emotional
benefits of working with flowers. A regular participant in a workshop in
June said the workshop calms her: ``It relaxes me. Just holding the
flowers takes me away from my situation. I've always loved flowers.''
She has been receiving services from N Street Village for two years and
lives in one of its shelters. She isn't allowed to take the flowers she
arranges in the workshop to the shelter, but they stay at the day center
and decorate the tables where the participants eat their meals.

Despite Ms. Anderson-Hall's certainty that her work makes a significant
impact, she has faced challenges getting support for her work, like
finding partners for her workshops. She's also still working toward
making that Bloom Mobile dream a reality --- her current vehicle doesn't
have space for the mobile workshops she envisioned. ``The greatest
barrier is probably minimizing and undervaluing the impact of flowers,''
she said. ``They are just so present to so many people's experience that
it's probably hard to really put a true value on their meaning and
impact.''

Miriam Zoila Pérez is the author of ``The Radical Doula Guide: A
Political Primer for Full Spectrum Pregnancy and Childbirth Support,'' a
freelance journalist and the creator of the
\href{https://houseplantparenthood.com/}{Houseplant Parenthood} website.

\emph{To receive email alerts for Fixes columns, sign up}
\href{http://eepurl.com/ABIxL}{\emph{here.}}

\emph{The Times is committed to publishing}
\href{https://www.nytimes.com/2019/01/31/opinion/letters/letters-to-editor-new-york-times-women.html}{\emph{a
diversity of letters}} \emph{to the editor. We'd like to hear what you
think about this or any of our articles. Here are some}
\href{https://help.nytimes.com/hc/en-us/articles/115014925288-How-to-submit-a-letter-to-the-editor}{\emph{tips}}\emph{.
And here's our email:}
\href{mailto:letters@nytimes.com}{\emph{letters@nytimes.com}}\emph{.}

\emph{Follow The New York Times Opinion section on}
\href{https://www.facebook.com/nytopinion}{\emph{Facebook}}\emph{,}
\href{http://twitter.com/NYTOpinion}{\emph{Twitter (@NYTopinion)}}
\emph{and}
\href{https://www.instagram.com/nytopinion/}{\emph{Instagram}}\emph{.}

Advertisement

\protect\hyperlink{after-bottom}{Continue reading the main story}

\hypertarget{site-index}{%
\subsection{Site Index}\label{site-index}}

\hypertarget{site-information-navigation}{%
\subsection{Site Information
Navigation}\label{site-information-navigation}}

\begin{itemize}
\tightlist
\item
  \href{https://help.nytimes.com/hc/en-us/articles/115014792127-Copyright-notice}{©~2020~The
  New York Times Company}
\end{itemize}

\begin{itemize}
\tightlist
\item
  \href{https://www.nytco.com/}{NYTCo}
\item
  \href{https://help.nytimes.com/hc/en-us/articles/115015385887-Contact-Us}{Contact
  Us}
\item
  \href{https://www.nytco.com/careers/}{Work with us}
\item
  \href{https://nytmediakit.com/}{Advertise}
\item
  \href{http://www.tbrandstudio.com/}{T Brand Studio}
\item
  \href{https://www.nytimes.com/privacy/cookie-policy\#how-do-i-manage-trackers}{Your
  Ad Choices}
\item
  \href{https://www.nytimes.com/privacy}{Privacy}
\item
  \href{https://help.nytimes.com/hc/en-us/articles/115014893428-Terms-of-service}{Terms
  of Service}
\item
  \href{https://help.nytimes.com/hc/en-us/articles/115014893968-Terms-of-sale}{Terms
  of Sale}
\item
  \href{https://spiderbites.nytimes.com}{Site Map}
\item
  \href{https://help.nytimes.com/hc/en-us}{Help}
\item
  \href{https://www.nytimes.com/subscription?campaignId=37WXW}{Subscriptions}
\end{itemize}
