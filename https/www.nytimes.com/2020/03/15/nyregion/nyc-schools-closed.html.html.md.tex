Sections

SEARCH

\protect\hyperlink{site-content}{Skip to
content}\protect\hyperlink{site-index}{Skip to site index}

\href{https://www.nytimes.com/section/nyregion}{New York}

\href{https://myaccount.nytimes.com/auth/login?response_type=cookie\&client_id=vi}{}

\href{https://www.nytimes.com/section/todayspaper}{Today's Paper}

\href{/section/nyregion}{New York}\textbar{}New York City Public Schools
to Close to Slow Spread of Coronavirus

\url{https://nyti.ms/2U9Enzu}

\begin{itemize}
\item
\item
\item
\item
\item
\end{itemize}

\hypertarget{schools-reopening}{%
\subsubsection{\texorpdfstring{\href{https://www.nytimes.com/spotlight/schools-reopening?name=styln-coronavirus-schools-reopening\&region=TOP_BANNER\&variant=undefined\&block=storyline_menu_recirc\&action=click\&pgtype=Article\&impression_id=1f77e400-e0fd-11ea-8666-1b5a8ddc794b}{Schools
Reopening}}{Schools Reopening}}\label{schools-reopening}}

\begin{itemize}
\tightlist
\item
  \href{https://www.nytimes.com/2020/08/17/us/k-12-schools-reopening.html?name=styln-coronavirus-schools-reopening\&region=TOP_BANNER\&variant=undefined\&block=storyline_menu_recirc\&action=click\&pgtype=Article\&impression_id=1f77e401-e0fd-11ea-8666-1b5a8ddc794b}{State
  of Play for K-12}
\item
  \href{https://www.nytimes.com/2020/08/15/us/covid-college-tuition.html?name=styln-coronavirus-schools-reopening\&region=TOP_BANNER\&variant=undefined\&block=storyline_menu_recirc\&action=click\&pgtype=Article\&impression_id=1f77e402-e0fd-11ea-8666-1b5a8ddc794b}{College
  Costs}
\item
  \href{https://www.nytimes.com/2020/08/14/us/covid-schools-learning-pods.html?name=styln-coronavirus-schools-reopening\&region=TOP_BANNER\&variant=undefined\&block=storyline_menu_recirc\&action=click\&pgtype=Article\&impression_id=1f780b10-e0fd-11ea-8666-1b5a8ddc794b}{Priced
  Out of Learning Pods}
\item
  \href{https://www.nytimes.com/2020/08/14/nyregion/school-reopening-nyc.html?name=styln-coronavirus-schools-reopening\&region=TOP_BANNER\&variant=undefined\&block=storyline_menu_recirc\&action=click\&pgtype=Article\&impression_id=1f780b11-e0fd-11ea-8666-1b5a8ddc794b}{N.Y.C.
  Schools Not Ready}
\item
  \href{https://www.nytimes.com/2020/08/05/parenting/parents-distance-learning.html?name=styln-coronavirus-schools-reopening\&region=TOP_BANNER\&variant=undefined\&block=storyline_menu_recirc\&action=click\&pgtype=Article\&impression_id=1f780b12-e0fd-11ea-8666-1b5a8ddc794b}{Prepare
  for Distance Learning}
\end{itemize}

Advertisement

\protect\hyperlink{after-top}{Continue reading the main story}

Supported by

\protect\hyperlink{after-sponsor}{Continue reading the main story}

\hypertarget{new-york-city-public-schools-to-close-to-slow-spread-of-coronavirus}{%
\section{New York City Public Schools to Close to Slow Spread of
Coronavirus}\label{new-york-city-public-schools-to-close-to-slow-spread-of-coronavirus}}

Following days of pressure, Mayor Bill de Blasio announced plans to
close the nation's largest public school system.

\includegraphics{https://static01.nyt.com/images/2020/03/15/nyregion/00NYVIRUS-SCHOOLSCLOSING11/merlin_170345649_375b0f86-81d2-4e7d-9db3-9be737bd924b-articleLarge.jpg?quality=75\&auto=webp\&disable=upscale}

\href{https://www.nytimes.com/by/eliza-shapiro}{\includegraphics{https://static01.nyt.com/images/2018/12/28/multimedia/author-eliza-shapiro/author-eliza-shapiro-thumbLarge.png}}

By \href{https://www.nytimes.com/by/eliza-shapiro}{Eliza Shapiro}

\begin{itemize}
\item
  March 15, 2020
\item
  \begin{itemize}
  \item
  \item
  \item
  \item
  \item
  \end{itemize}
\end{itemize}

New York City's public school system, the nation's largest, will begin
shutting down this week, by far the most far-reaching and disruptive
measure the city has taken yet to slow the spread of the coronavirus.

The city's vast system of 1,800 schools now faces its most serious
challenge in decades, as it embarks on a mass closure that could
potentially last through the end of the school year.

``This is not something in a million years I could have imagined having
to do,'' Mayor Bill de Blasio, appearing visibly distraught, said on
Sunday, adding that it was an ``extraordinarily painful'' moment for
city schools.

The closures will alter the lives and routines of 1.1 million children,
75,000 teachers and well over 1 million parents, and will no doubt
prompt broader upheaval in a moment of profound anxiety for New Yorkers.

Mr. de Blasio said that the schools will be closed on Monday for all
students and staff, but teachers will be asked to report to work later
in the week for training on remote learning.

By March 23, the city will move to remote learning, and the system will
be closed except for several dozen school buildings throughout the city,
which will be used as ``learning centers'' to support the children of
essential city workers like health care employees.

Schools will be closed until at least April 20, after the upcoming
spring break, but could stay closed for significantly longer, Mr. de
Blasio said.

City school buildings will remain open this week for children to pick up
food, and then the city will find alternative sites for students who
need food to receive it. Students who do not have computers at home will
be lent laptops, and the city will work on helping students who do not
have internet access get online.

All public schools in Westchester and Long Island will also close, Gov.
Andrew M. Cuomo said Sunday.

The mayor and Mr. Cuomo had resisted closing the city's schools~even as
other states shuttered their public schools, and urban school districts,
like Los Angeles and Seattle, did the same.

\href{https://www.nytimes.com/2020/03/13/nyregion/coronavirus-nyc-schools.html}{But
in recent days}, a growing chorus of local politicians, public health
experts, parents and educators have ramped up the pressure on the city
to shut down schools. By Sunday afternoon, even Mr. Cuomo said the city
schools should close within 24 hours, as soon as the city came up with a
plan for child care and food.

Student attendance has plummeted as nervous parents have kept their
children at home. Teachers concerned about the virus organized ``sick
outs'' and flooded Twitter and 311 with pleas to shut down schools. And
as the outbreak continued, each day seemed to bring another major set of
school closures in cities and states with smaller outbreaks than New
York's.

\href{https://www.nytimes.com/2020/03/07/nyregion/nyc-schools-coronavirus.html}{New
York City's school system stands apart from every other} in the country
for its sheer size and particularly vulnerable student population,
including enough homeless children --- 114,000 --- to fill an entire
small city school district.

Even if only half of New York City's students reported to school, the
district would still be larger than any in the country except for Los
Angeles Unified, which announced on Friday that its schools would close
for at least two weeks. Gov. Philip D. Murphy of New Jersey said on
Sunday that state's public school system would close imminently.

Public health experts agree that effective closures would have to last
for the length of the virus, which could take months. And students
getting together in their homes or other places outside of school could
diminish the effectiveness of closures, experts said. Mr. de Blasio was
blunt about the prospect of keeping children apart in the weeks and
months ahead.

\href{https://www.nytimes.com/spotlight/schools-reopening?action=click\&pgtype=Article\&state=default\&region=MAIN_CONTENT_3\&context=storylines_keepup}{}

\hypertarget{schools-reopening-}{%
\subsubsection{Schools Reopening ›}\label{schools-reopening-}}

\hypertarget{back-to-school}{%
\paragraph{Back to School}\label{back-to-school}}

Updated Aug. 17, 2020

The latest on how schools are navigating an uncertain season.

\begin{itemize}
\item
  \begin{itemize}
  \tightlist
  \item
    Universities across the country are facing
    \href{https://www.nytimes.com/2020/08/15/us/covid-college-tuition.html?action=click\&pgtype=Article\&state=default\&region=MAIN_CONTENT_3\&context=storylines_keepup}{a
    rising demand for tuition rebates} as students ask if college is
    becoming ``glorified Skype.''
  \item
    In Los Angeles, the nation's second-largest school district has
    \href{https://www.nytimes.com/2020/08/16/us/los-angeles-schools-virus-testing.html?action=click\&pgtype=Article\&state=default\&region=MAIN_CONTENT_3\&context=storylines_keepup}{perhaps
    the most ambitious plan in the country} to test for the coronavirus.
  \item
    Families
    \href{https://www.nytimes.com/2020/08/14/us/covid-schools-learning-pods.html?action=click\&pgtype=Article\&state=default\&region=MAIN_CONTENT_3\&context=storylines_keepup}{priced
    out of ``learning pods'' are seeking alternatives}.
  \item
    How are campus newspapers covering back to school?
    \href{https://www.nytimes.com/2020/08/17/us/student-newspaper-schools-reopening.html?action=click\&pgtype=Article\&state=default\&region=MAIN_CONTENT_3\&context=storylines_keepup}{We
    want to hear from student journalists}.
  \end{itemize}
\end{itemize}

``We're not going to convince teenagers not to gather,'' he said.

The long-term effects of closure are extremely daunting, and will
unquestionably lead to extensive learning loss for scores of students.

Many students could fall months behind on instruction, a worst-case
scenario for children who are struggling to read or just beginning to
make improvements in school. About 20 percent of children have special
needs, some of which are advanced, and many of those students get
services at school they cannot get at home. About three-quarters of city
students are eligible for free or reduced price meals at school.

Annual standardized math and English exams scheduled for this spring
will almost certainly have to be delayed, education officials said.

And there are hundreds of thousands of public school parents who do some
of the city's most essential work: staffing public hospitals, driving
city buses and subways and caring for older people.

The large-scale, indefinite school closures leave New York City in
uncharted territory. A handful of city schools closed during the 2009
H1N1 flu epidemic, and the city closed its school system for a week
after Hurricane Sandy in 2013, with some schools remaining shut for
longer.

On Sunday evening, New York City parents reacted to the news with a mix
of confusion and concern.

``Are we going to be home schooling? It's something I always wanted to
try,'' asked Belinda Blum, who lives in Crown Heights, Brooklyn, and has
two daughters in public school. Ms. Blum said she wanted to make sure
that her daughters still had structure in their day, even if they were
learning from home.

Gary Purdy, who has two children in the public school system, said he
didn't know how the city would decide to reopen schools.

``Are you going to test every single school, every single person in the
city?'' Mr. Purdy, also a Brooklyn resident, asked.

In Harlem, Alicia Taylor said she had already made a backup plan: She
was preparing to take her children to North Carolina, where they could
be close to family and have room to spread out.

She said she wasn't sure how her children would be educated remotely.
``What the hell is everyone going to do?'' she asked.

Ashley Southall and Matthew Sedacca contributed reporting.

Advertisement

\protect\hyperlink{after-bottom}{Continue reading the main story}

\hypertarget{site-index}{%
\subsection{Site Index}\label{site-index}}

\hypertarget{site-information-navigation}{%
\subsection{Site Information
Navigation}\label{site-information-navigation}}

\begin{itemize}
\tightlist
\item
  \href{https://help.nytimes.com/hc/en-us/articles/115014792127-Copyright-notice}{©~2020~The
  New York Times Company}
\end{itemize}

\begin{itemize}
\tightlist
\item
  \href{https://www.nytco.com/}{NYTCo}
\item
  \href{https://help.nytimes.com/hc/en-us/articles/115015385887-Contact-Us}{Contact
  Us}
\item
  \href{https://www.nytco.com/careers/}{Work with us}
\item
  \href{https://nytmediakit.com/}{Advertise}
\item
  \href{http://www.tbrandstudio.com/}{T Brand Studio}
\item
  \href{https://www.nytimes.com/privacy/cookie-policy\#how-do-i-manage-trackers}{Your
  Ad Choices}
\item
  \href{https://www.nytimes.com/privacy}{Privacy}
\item
  \href{https://help.nytimes.com/hc/en-us/articles/115014893428-Terms-of-service}{Terms
  of Service}
\item
  \href{https://help.nytimes.com/hc/en-us/articles/115014893968-Terms-of-sale}{Terms
  of Sale}
\item
  \href{https://spiderbites.nytimes.com}{Site Map}
\item
  \href{https://help.nytimes.com/hc/en-us}{Help}
\item
  \href{https://www.nytimes.com/subscription?campaignId=37WXW}{Subscriptions}
\end{itemize}
