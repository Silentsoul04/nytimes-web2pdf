Sections

SEARCH

\protect\hyperlink{site-content}{Skip to
content}\protect\hyperlink{site-index}{Skip to site index}

\href{https://www.nytimes.com/section/politics}{Politics}

\href{https://myaccount.nytimes.com/auth/login?response_type=cookie\&client_id=vi}{}

\href{https://www.nytimes.com/section/todayspaper}{Today's Paper}

\href{/section/politics}{Politics}\textbar{}Biden and Sanders Fight Over
Policy and Records in Head-to-Head Debate

\url{https://nyti.ms/2w9bUBN}

\begin{itemize}
\item
\item
\item
\item
\item
\item
\end{itemize}

\begin{itemize}
\item
  \href{https://www.nytimes.com/2020/07/31/us/elections/biden-vs-trump.html?action=click\&pgtype=Article\&state=default\&region=TOP_BANNER\&context=storylines_menu}{Election
  Updates}
\item
  \href{https://www.nytimes.com/article/biden-vice-president-2020.html?action=click\&pgtype=Article\&state=default\&region=TOP_BANNER\&context=storylines_menu}{Biden's
  V.P. Search}
\item
  \href{https://www.nytimes.com/interactive/2020/07/24/us/politics/trump-biden-campaign-donors.html?action=click\&pgtype=Article\&state=default\&region=TOP_BANNER\&context=storylines_menu}{Map
  of Donations}
\item
  \href{https://www.nytimes.com/interactive/2020/us/elections/delegate-count-primary-results.html?action=click\&pgtype=Article\&state=default\&region=TOP_BANNER\&context=storylines_menu}{Delegate
  Count}
\item
  \href{https://www.nytimes.com/interactive/2019/us/politics/2020-presidential-candidates.html?action=click\&pgtype=Article\&state=default\&region=TOP_BANNER\&context=storylines_menu}{The
  Candidates}
\item
  \href{https://www.nytimes.com/newsletters/politics?action=click\&pgtype=Article\&state=default\&region=TOP_BANNER\&context=storylines_menu}{Politics
  Newsletter}
\end{itemize}

Advertisement

\protect\hyperlink{after-top}{Continue reading the main story}

Supported by

\protect\hyperlink{after-sponsor}{Continue reading the main story}

\hypertarget{biden-and-sanders-fight-over-policy-and-records-in-head-to-head-debate}{%
\section{Biden and Sanders Fight Over Policy and Records in Head-to-Head
Debate}\label{biden-and-sanders-fight-over-policy-and-records-in-head-to-head-debate}}

With the coronavirus as a backdrop, and with no audience in attendance,
the two Democrats were at times feisty discussing their differences over
health care. Mr. Biden also committed to choosing a woman as a running
mate.

\includegraphics{https://static01.nyt.com/images/2020/03/15/us/politics/15debate-ledeall-top/15debate-ledeall-top-videoSixteenByNine3000-v2.jpg}

\href{https://www.nytimes.com/by/alexander-burns}{\includegraphics{https://static01.nyt.com/images/2018/09/25/multimedia/author-alexander-burns/author-alexander-burns-thumbLarge-v2.png}}\href{https://www.nytimes.com/by/jonathan-martin}{\includegraphics{https://static01.nyt.com/images/2018/11/06/multimedia/author-jonathan-martin/author-jonathan-martin-thumbLarge.png}}

By \href{https://www.nytimes.com/by/alexander-burns}{Alexander Burns}
and \href{https://www.nytimes.com/by/jonathan-martin}{Jonathan Martin}

\begin{itemize}
\item
  March 15, 2020
\item
  \begin{itemize}
  \item
  \item
  \item
  \item
  \item
  \item
  \end{itemize}
\end{itemize}

Joseph R. Biden Jr. and Senator Bernie Sanders called for vastly more
aggressive government action to battle the coronavirus but split over
some of the details along familiar ideological lines on Sunday night, as
the two Democrats tangled over the right to lead their party into a
campaign overshadowed by the pandemic inflicting havoc on the country's
economy and its social fabric.

In their first one-on-one debate of the primary race, Mr. Sanders, a
democratic socialist from Vermont, demanded sweeping economic reform and
the creation of a single-payer health care system to address crises like
the virus. Mr. Biden said he would call up the military to help and
enact a ``multi-multi-billion dollar program'' of disease containment
and economic rescue, and said that there were more issues at hand that
could not wait on reinventing the health care system.

Mr. Biden, the former vice president, said managing the coronavirus was
``like a war,'' while Mr. Sanders said it had exposed ``the
dysfunctionality'' of the country's patchwork health care system.

Yet on matters beyond the virus, the two men careened from making
gestures toward party cohesion and personal comity to clashing over
their divergent policy agendas and records in office.

Mr. Biden repeatedly sought to play down criticism from Mr. Sanders and
to play a unifying role. He offered lengthy praise for a proposal by
Senator Elizabeth Warren of Massachusetts, his progressive former rival,
to reform the bankruptcy code, and in a striking commitment he made an
ironclad pledge to name a woman as his running mate. Mr. Sanders said
that ``in all likelihood'' he, too, would choose a female running mate.

But Mr. Sanders persistently challenged Mr. Biden over his positions on
core Democratic issues, leaving the former vice president struggling to
explain --- and in some cases misrepresenting --- his past record on
matters like funding Social Security and the war in Iraq. And Mr. Biden
also went on offense, especially over Mr. Sanders's history of praising
leftist governments in Latin America.

Still, with Democrats single-mindedly focused on defeating President
Trump, and with the country as a whole dismayed by a growing pandemic,
both candidates offered assurances of mutual support for the general
election. ``It's much bigger than either of us,'' Mr. Biden said, vowing
to campaign for Mr. Sanders if he were the nominee.

``If I lose this thing, Joe wins --- Joe, I will be there for you,'' Mr.
Sanders promised.

Their sparring represented what may be the last vestiges of a
once-uncertain contest that --- after his commanding victories over the
past two weeks --- now clearly favors Mr. Biden. Indeed, the growing
threat of the coronavirus, combined with the former vice president's
significant advantage in delegates, lowered the debate stakes and might
have rendered it most memorable for Mr. Biden's promise to choose a
woman as his running mate.

He has been widely expected to do so, but his pledge was sure to draw
attention, fuel speculation about who the woman may be and generally
tilt attention toward the general election and away from the primary
race.

\includegraphics{https://static01.nyt.com/images/2020/03/15/us/politics/15debate-ledeall2-sub/15debate-ledeall2-sub-articleLarge.jpg?quality=75\&auto=webp\&disable=upscale}

The specter of the coronavirus pervaded their encounter from their first
moments onstage: Mr. Biden and Mr. Sanders declined to shake hands at
the start and stood six feet apart from each other at a television
studio in Washington, D.C., following the guidelines for social
distancing prescribed by public health authorities. In deference to the
same regulations, the debate took place without a live audience.

\hypertarget{latest-updates-2020-election}{%
\section{\texorpdfstring{\href{https://www.nytimes.com/2020/07/31/us/elections/biden-vs-trump.html?action=click\&pgtype=Article\&state=default\&region=MAIN_CONTENT_1\&context=storylines_live_updates}{Latest
Updates: 2020
Election}}{Latest Updates: 2020 Election}}\label{latest-updates-2020-election}}

Updated 2020-08-01T01:26:45.732Z

\begin{itemize}
\tightlist
\item
  \href{https://www.nytimes.com/2020/07/31/us/elections/biden-vs-trump.html?action=click\&pgtype=Article\&state=default\&region=MAIN_CONTENT_1\&context=storylines_live_updates\#link-29fdff45}{Kamala
  Harris, a top vice-presidential contender, confronts double
  standards.}
\item
  \href{https://www.nytimes.com/2020/07/31/us/elections/biden-vs-trump.html?action=click\&pgtype=Article\&state=default\&region=MAIN_CONTENT_1\&context=storylines_live_updates\#link-13ec3d9c}{Karen
  Bass and Susan Rice are rising on Biden's vice-presidential
  shortlist.}
\item
  \href{https://www.nytimes.com/2020/07/31/us/elections/biden-vs-trump.html?action=click\&pgtype=Article\&state=default\&region=MAIN_CONTENT_1\&context=storylines_live_updates\#link-49e9a016}{Trump
  says Russian bounties to kill U.S. troops `never took place.'}
\end{itemize}

\href{https://www.nytimes.com/2020/07/31/us/elections/biden-vs-trump.html?action=click\&pgtype=Article\&state=default\&region=MAIN_CONTENT_1\&context=storylines_live_updates}{See
more updates}

Mr. Biden and Mr. Sanders approached each other at first with caution,
splitting over matters of policy but largely declining to go on the
attack in sharp terms, before clashing in a feistier manner as the event
wore on. Their early restraint was notable, because the debate was their
first encounter since Mr. Biden reclaimed his once-wobbly status as the
Democratic front-runner and Mr. Sanders lost the surging momentum he
captured in the race for a time last month.

Where the two men agreed was in deploring Mr. Trump's approach to the
coronavirus, and in demanding a more far-reaching government strategy to
contain the outbreak and patch up the economic wreckage it is causing.
Mr. Biden argued that every resource of the government should be
mobilized to limit short-term damage, while Mr. Sanders said that
countering it would require a more drastic overhaul of economic and
health care systems.

``This is like we are being attacked from abroad,'' Mr. Biden said.
``This is something that is of great consequence. This is like a war.''

Mr. Sanders called the coronavirus crisis an ``unprecedented moment in
American history,'' and said it drew attention to the fact that the
country lacked ``a system that is prepared to provide health care for
all people.''

In an early barb directed at Mr. Biden, Mr. Sanders said it would take a
direct confrontation with the insurance and pharmaceutical industries to
remedy the situation, including enacting his proposal for a ``Medicare
for all''-style system that he has championed.

``Do we have the guts to take on the health care industry, some of which
is funding the vice president's campaign?'' Mr. Sanders asked.

But Mr. Biden pushed back aggressively on the notion that Mr. Sanders's
signature proposal could mitigate the virus, invoking Europe's
hardest-hit country.

``With all due respect to Medicare for all, you have a single-payer
system now in Italy,'' he said, arguing that such a system ``would not
solve the problem.''

Beyond their familiar disagreements on health care, Mr. Sanders leveled
a larger critique of Mr. Biden's approach to leadership, accusing him of
compromising too readily with Republicans and corporate interests. He
also challenged the former vice president over decades' worth of votes
on abortion, gay rights, foreign wars, bankruptcy regulation and
retirement-security programs.

At times, the debate became a kind of tense colloquy between longtime
colleagues, as Mr. Sanders prodded Mr. Biden to account for his past
positions.

In the most heated moments, the two candidates scolded each other by
name --- interjecting ``Joe!'' and ``Bernie!'' --- and showed visible
exasperation. Mr. Biden laughed at one of Mr. Sanders's rebukes and
muttered ``Give me a break'' when the Vermont senator urged him to
disavow a super PAC supporting his campaign.

``I won't give you a break on this one, Joe,'' Mr. Sanders said.
``You've condemned super PACs. You've got a super PAC. It's running
negative ads.''

Indeed, Mr. Sanders repeatedly forced Mr. Biden onto the defensive,
leading the former vice president to play down or misstate portions of
his own record. Challenged by Mr. Sanders for his role championing
bankruptcy legislation backed by the credit card industry, Mr. Biden
said he did not help write the legislation and ``made it clear to the
industry that I did not like the bill'' --- though Mr. Biden was among
its most vocal Democratic supporters.

Pressed by Mr. Sanders about his past comments in the Senate suggesting
it could be necessary to rein in Social Security and other popular
entitlement programs, Mr. Biden gave a halting series of answers that
prompted Mr. Sanders to urge him to ``be straight with the American
people.''

Mr. Biden was not shy about defending other elements of his record from
his liberal rival: When Mr. Sanders dismissed the 2008 bailout of the
financial sector as a gift to Wall Street executives, Mr. Biden chided
him and argued that if the banks had not been stabilized the country
would have plunged into ``a great depression.''

``All those people Bernie said he cares about would have been in deep
trouble,'' Mr. Biden said.

Mr. Biden confronted Mr. Sanders with some of his past votes as well,
including his opposition to some gun-control legislation.

Image

Mr. Biden and Mr. Sanders bumped elbows instead of shaking hands at the
start of the debate.Credit...Erin Schaff/The New York Times

Two days before Florida's crucial primary, Mr. Sanders came prepared
with an answer about recent comments that have set off a wave of
criticism --- his praise for some elements of Fidel Castro's rule in
Cuba. He vowed that as president he would ``put the flag down'' for
``democracy and human rights.''

But when he was pressed by Mr. Biden and the moderators, Mr. Sanders
returned to his longstanding position that it was possible to condemn
authoritarians while also praising positive elements of their
governments. He held up China, which he said had made significant
strides in reducing poverty, a comment that prompted Mr. Biden to assail
him for praising ``a dictatorship.''

Even as they feuded, the candidates stopped well short of the kind of
scorched-earth attacks that have characterized the climactic debates in
past nomination fights, including both the Democratic and Republican
primary campaigns in 2016.

After one extended attack on him, Mr. Biden good-naturedly noted that he
had tried to give Mr. Sanders ``credit for some things'' but that he was
``making it harder for me,'' causing the Vermont senator to smile.

Both candidates highlighted their own new habits: online gatherings
instead of in-person campaign rallies and no more handshaking ---~and
lots of handwashing. ``I'm using a lot of soap and hand sanitizers,''
Mr. Sanders said. Added Mr. Biden, ``I wash my hands God knows how many
times a day with hot water and soap.''

Mr. Sanders addressed a broader electoral issue in an interview on CNN
after the debate, when asked if the primaries in four states on Tuesday
should go on. ``That is a very good question,'' he said, adding that he
was ``not sure it makes a lot of sense'' to have older people who are
susceptible to the virus gathered at polling places.

The debate showed Mr. Sanders in fighting form, and perhaps sounding
further away from conceding than Mr. Biden's camp had hoped. After the
debate, Anita Dunn, Mr. Biden's chief strategist, likened Mr. Sanders to
a political demonstrator, saying the former vice president had spent the
debate ``graciously dealing with the kind of protester who often shows
up at campaign events, on live television.''

The forum on Sunday was originally set to be held in Phoenix, before a
live audience. But in a series of incremental announcements over the
past week, CNN and the Democratic National Committee declared that there
would be no live audience; that the debate would be moved to Washington;
and that one of the planned moderators, Jorge Ramos, would be replaced
because he might have been exposed to the coronavirus.

The unusual circumstances reflected a larger freeze in the presidential
campaign: Mr. Biden and Mr. Sanders have all but halted public campaign
activity since last week's primaries.

Both candidates are in their late 70s and could face the risk of
contracting the virus themselves from prolonged exposure to large
numbers of voters. Aides to Mr. Biden, 77, and Mr. Sanders, 78, have
said that neither man has been tested for the virus or shown any
symptoms of the disease.

Amid the slowdown in campaigning, Mr. Biden and Mr. Sanders have been
trying to match their core themes to the moment of crisis. There is now
a general view among all of the political campaigns --- including Mr.
Trump's --- that the coronavirus
\href{https://www.nytimes.com/2020/03/12/us/politics/trump-vs-biden.html}{could
well redraw the existing contours of the presidential race.} The crisis
might undermine Mr. Trump's plans to run on promises of continued
prosperity and instead focus public attention above all on the difficult
process of managing a contagion and rebuilding a shaken economy.

\hypertarget{our-2020-election-guide}{%
\section{Our 2020 Election Guide}\label{our-2020-election-guide}}

Updated July 31, 2020

\begin{itemize}
\item
  \begin{center}\rule{0.5\linewidth}{\linethickness}\end{center}

  \hypertarget{the-latest}{%
  \subsection{The Latest}\label{the-latest}}

  \begin{itemize}
  \tightlist
  \item
    President Trump's assault on the Postal Service is intersecting with
    his attacks on mail-in voting.
    \href{https://www.nytimes.com/2020/07/31/us/politics/trump-usps-mail-delays.html?action=click\&pgtype=Article\&state=default\&region=BELOW_MAIN_CONTENT\&context=storylines_guide}{Voting
    rights groups say it is a recipe for disaster.}
  \end{itemize}
\item
  \begin{center}\rule{0.5\linewidth}{\linethickness}\end{center}

  \hypertarget{bidens-vp-search}{%
  \subsection{Biden's V.P. Search}\label{bidens-vp-search}}

  \begin{itemize}
  \tightlist
  \item
    \href{https://www.nytimes.com/article/biden-vice-president-2020.html?action=click\&pgtype=Article\&state=default\&region=BELOW_MAIN_CONTENT\&context=storylines_guide}{Here
    are 13 women} who have been under consideration to be Joe Biden's
    running mate, and why each might be chosen --- and might not be.
  \end{itemize}
\item
  \begin{center}\rule{0.5\linewidth}{\linethickness}\end{center}

  \hypertarget{keep-up-with-our-coverage}{%
  \subsection{Keep Up With Our
  Coverage}\label{keep-up-with-our-coverage}}

  \begin{itemize}
  \tightlist
  \item
    Get an
    \href{https://www.nytimes.com/newsletters/politics?action=click\&pgtype=Article\&state=default\&region=BELOW_MAIN_CONTENT\&context=storylines_guide}{email}
    recapping the day's news
  \end{itemize}

  \begin{itemize}
  \tightlist
  \item
    Download our mobile app on
    \href{https://apps.apple.com/us/app/nytimes/id284862083?ls=1\&mat_click_id=5c79ae7455014fd1bd66b5610c05b8f2-20191112-16948\&referrer=mat_click_id\%3D5c79ae7455014fd1bd66b5610c05b8f2-20191112-16948\%26link_click_id\%3D722930677036718082}{iOS}
    and
    \href{http://a.localytics.com/android?id=com.nytimes.android\&referrer=utm_source\%3Dother_nyt_mobile_web\%26utm_medium\%3DWeb\%2520page\%26utm_term\%3DGeneral\%2520Mobile\%2520Page\%26utm_campaign\%3DNYT\%2520Mobile\%2520General\%2520Page}{Android}
    and turn on Breaking News and Politics alerts
  \end{itemize}
\end{itemize}

Advertisement

\protect\hyperlink{after-bottom}{Continue reading the main story}

\hypertarget{site-index}{%
\subsection{Site Index}\label{site-index}}

\hypertarget{site-information-navigation}{%
\subsection{Site Information
Navigation}\label{site-information-navigation}}

\begin{itemize}
\tightlist
\item
  \href{https://help.nytimes.com/hc/en-us/articles/115014792127-Copyright-notice}{©~2020~The
  New York Times Company}
\end{itemize}

\begin{itemize}
\tightlist
\item
  \href{https://www.nytco.com/}{NYTCo}
\item
  \href{https://help.nytimes.com/hc/en-us/articles/115015385887-Contact-Us}{Contact
  Us}
\item
  \href{https://www.nytco.com/careers/}{Work with us}
\item
  \href{https://nytmediakit.com/}{Advertise}
\item
  \href{http://www.tbrandstudio.com/}{T Brand Studio}
\item
  \href{https://www.nytimes.com/privacy/cookie-policy\#how-do-i-manage-trackers}{Your
  Ad Choices}
\item
  \href{https://www.nytimes.com/privacy}{Privacy}
\item
  \href{https://help.nytimes.com/hc/en-us/articles/115014893428-Terms-of-service}{Terms
  of Service}
\item
  \href{https://help.nytimes.com/hc/en-us/articles/115014893968-Terms-of-sale}{Terms
  of Sale}
\item
  \href{https://spiderbites.nytimes.com}{Site Map}
\item
  \href{https://help.nytimes.com/hc/en-us}{Help}
\item
  \href{https://www.nytimes.com/subscription?campaignId=37WXW}{Subscriptions}
\end{itemize}
