Sections

SEARCH

\protect\hyperlink{site-content}{Skip to
content}\protect\hyperlink{site-index}{Skip to site index}

\href{https://www.nytimes.com/section/nyregion}{New York}

\href{https://myaccount.nytimes.com/auth/login?response_type=cookie\&client_id=vi}{}

\href{https://www.nytimes.com/section/todayspaper}{Today's Paper}

\href{/section/nyregion}{New York}\textbar{}Drastic `Shelter in Place'
May Be Next for N.Y.C. to Combat Coronavirus

\url{https://nyti.ms/391FMNM}

\begin{itemize}
\item
\item
\item
\item
\item
\end{itemize}

\href{https://www.nytimes.com/news-event/coronavirus?action=click\&pgtype=Article\&state=default\&region=TOP_BANNER\&context=storylines_menu}{The
Coronavirus Outbreak}

\begin{itemize}
\tightlist
\item
  live\href{https://www.nytimes.com/2020/08/04/world/coronavirus-cases.html?action=click\&pgtype=Article\&state=default\&region=TOP_BANNER\&context=storylines_menu}{Latest
  Updates}
\item
  \href{https://www.nytimes.com/interactive/2020/us/coronavirus-us-cases.html?action=click\&pgtype=Article\&state=default\&region=TOP_BANNER\&context=storylines_menu}{Maps
  and Cases}
\item
  \href{https://www.nytimes.com/interactive/2020/science/coronavirus-vaccine-tracker.html?action=click\&pgtype=Article\&state=default\&region=TOP_BANNER\&context=storylines_menu}{Vaccine
  Tracker}
\item
  \href{https://www.nytimes.com/2020/08/02/us/covid-college-reopening.html?action=click\&pgtype=Article\&state=default\&region=TOP_BANNER\&context=storylines_menu}{College
  Reopening}
\item
  \href{https://www.nytimes.com/live/2020/08/04/business/stock-market-today-coronavirus?action=click\&pgtype=Article\&state=default\&region=TOP_BANNER\&context=storylines_menu}{Economy}
\end{itemize}

Advertisement

\protect\hyperlink{after-top}{Continue reading the main story}

Supported by

\protect\hyperlink{after-sponsor}{Continue reading the main story}

\hypertarget{drastic-shelter-in-place-may-be-next-for-nyc-to-combat-coronavirus}{%
\section{Drastic `Shelter in Place' May Be Next for N.Y.C. to Combat
Coronavirus}\label{drastic-shelter-in-place-may-be-next-for-nyc-to-combat-coronavirus}}

Life in New York City, a colossus of 8.6 million people and an economic
engine for the country, is grinding to a shocking halt.

\includegraphics{https://static01.nyt.com/images/2020/03/17/nyregion/17nyvirus-ledeall1/merlin_170647176_6e1b9868-36a9-4359-b636-47aae2487df8-articleLarge.jpg?quality=75\&auto=webp\&disable=upscale}

\href{https://www.nytimes.com/by/andy-newman}{\includegraphics{https://static01.nyt.com/images/2018/02/16/multimedia/author-andy-newman/author-andy-newman-thumbLarge.jpg}}

By \href{https://www.nytimes.com/by/andy-newman}{Andy Newman}

\begin{itemize}
\item
  March 17, 2020
\item
  \begin{itemize}
  \item
  \item
  \item
  \item
  \item
  \end{itemize}
\end{itemize}

New York City, a colossus of 8.6 million people and an economic engine
for the country, ground to a shocking halt on Tuesday because of the
\href{https://www.nytimes.com/2020/03/18/podcasts/the-daily/cuomo-new-york-coronavirus.html?action=click\&module=Briefings\&pgtype=Homepage}{coronavirus
outbreak} and the restrictions on public life put in place to stem its
spread.

The city's mayor signaled that the shutdown could go even further with
the possibility of an order to ``shelter in place'' --- a **** decision
he said ``should be made in the next 48 hours.''

``If that moment came, there are tremendously substantial challenges
that would have to be met,'' Mayor Bill de Blasio said in an
\href{https://www.nytimes.com/video/nyregion/100000007039735/nyc-shelter-in-place-de-blasio.html}{emotional
address at City Hall}. ``And I don't take this lightly at all.

``What is going to happen with folks who have no money?'' he continued.
``How are they going to get food? How are they going to get medicines?
How are we going to ensure in a dynamic like that, that supplies are
sufficient for our population?''

As officials grappled with an epidemic that has stricken more than 800
city residents and killed at least seven,
\href{https://www.nytimes.com/2020/03/17/nyregion/coronavirus-new-york-update.html}{the
toll on the life of the city was becoming}apparent.

Times Square emptied out. Macy's closed. The Statue of Liberty was
cordoned off. The Empire State Building was shuttered. Restaurants and
bars, the ones that had not closed entirely, stood nearly empty and
tried to survive on takeout and delivery orders alone.

``I'd like to see them try keeping New Yorkers off the street,'' said
Rafael Morales, 52, a super at a co-op building on the Upper West Side
of Manhattan.

New York's desperation was also made clear when the Metropolitan
Transportation Authority, which runs the subway system, buses and two
commuter railroads, said on Tuesday that it
was\href{https://www.nytimes.com/2020/03/17/nyregion/coronavirus-nyc-subway-federal-aid-.html}{seeking
a \$4 billion federal bailout}.

Ridership has plummeted by as much as 90 percent on the region's trains
and 60 percent on the subway --- rendering the normally jampacked
underground practically unrecognizable.

\hypertarget{latest-updates-global-coronavirus-outbreak}{%
\section{\texorpdfstring{\href{https://www.nytimes.com/2020/08/04/world/coronavirus-cases.html?action=click\&pgtype=Article\&state=default\&region=MAIN_CONTENT_1\&context=storylines_live_updates}{Latest
Updates: Global Coronavirus
Outbreak}}{Latest Updates: Global Coronavirus Outbreak}}\label{latest-updates-global-coronavirus-outbreak}}

Updated 2020-08-05T06:48:23.151Z

\begin{itemize}
\tightlist
\item
  \href{https://www.nytimes.com/2020/08/04/world/coronavirus-cases.html?action=click\&pgtype=Article\&state=default\&region=MAIN_CONTENT_1\&context=storylines_live_updates\#link-762df92}{As
  talks drag on, McConnell signals openness to jobless aid extension,
  and negotiators agree on a deadline.}
\item
  \href{https://www.nytimes.com/2020/08/04/world/coronavirus-cases.html?action=click\&pgtype=Article\&state=default\&region=MAIN_CONTENT_1\&context=storylines_live_updates\#link-1228a480}{Novavax
  sees encouraging results from two studies of its experimental
  vaccine.}
\item
  \href{https://www.nytimes.com/2020/08/04/world/coronavirus-cases.html?action=click\&pgtype=Article\&state=default\&region=MAIN_CONTENT_1\&context=storylines_live_updates\#link-794484ed}{Mississippians
  must now wear masks in public, governor says.}
\end{itemize}

\href{https://www.nytimes.com/2020/08/04/world/coronavirus-cases.html?action=click\&pgtype=Article\&state=default\&region=MAIN_CONTENT_1\&context=storylines_live_updates}{See
more updates}

More live coverage:
\href{https://www.nytimes.com/live/2020/08/04/business/stock-market-today-coronavirus?action=click\&pgtype=Article\&state=default\&region=MAIN_CONTENT_1\&context=storylines_live_updates}{Markets}

Officials have grasped for comparisons to other catastrophes. Mr. de
Blasio said the economic fallout from the shutdown as a result of the
virus could rival that of the Great Depression and the health impact
that of the 1918 influenza epidemic that killed over 20,000 in the city.

But even as New Yorkers were struggling with the vast shutdown, the
mayor and the governor, Andrew M. Cuomo, fell into a familiar pattern:
battling with each other over control of the city.

\includegraphics{https://static01.nyt.com/images/2020/03/17/nyregion/17nyvirus-ledeall2/merlin_170615367_20df85b0-4433-4825-bfd8-0130b6856772-articleLarge.jpg?quality=75\&auto=webp\&disable=upscale}

As the mayor conducted his news conference on Tuesday, Mr. Cuomo's
office sent out a news release proclaiming that any kind of mass
quarantine order would need state approval and that none was imminent.
The governor then doubled down on that message.

``There is not going to be any quarantine, no one is going to lock you
in your home, no one is going to tell you, you can't leave the city,''
the governor said in an interview on NY1. ``That's not going to
happen.''

Some New Yorkers greeted the possibility of being put on virtual
lockdown with grim resignation.

Joseph Montes, who was skateboarding down Fourth Avenue in Brooklyn back
to a homeless shelter Tuesday night from his job as a tattoo artist in
the Bronx, was appalled at the prospect.

``That's totally crazy,'' said Mr. Montes, 27, who was wearing a face
mask with a jack-o-lantern mouth.

``People need to be outside, to breathe fresh air,'' he said. ``I've
been inside for a long time. It messes with you, makes you feel like a
prisoner.''

The public disagreement between the mayor and the governor was nothing
new, but the immediacy and rawness of it illustrated the tension and
uncertainty of the situation, and how public officials have been
struggling to respond to it.

It's uncertain what a ``shelter in place'' order would mean for New York
City. Mayor de Blasio said it could limit movement to people with
essential jobs like police officers, firefighters and health care
workers.

A ``shelter in place'' order
\href{https://www.nytimes.com/2020/03/17/us/shelter-in-place-order-bay-area.html}{enacted
on Tuesday} in California's Bay Area requires people to largely stay at
home except for essential activities and forbids people who do not live
in the same house from gathering anywhere. Going outside, for example,
to a park, is still allowed as long as people maintain a six-foot
distance from others.

Asked the difference between sheltering in place and quarantine, Mr. de
Blasio said ``I don't want to be the guy'' to define the distinction and
said he would decide on the matter in consultation with the governor.

New York City has quickly become an epicenter of the pandemic in this
country: New cases jumped by 75 percent from Monday to Tuesday, to 814,
underscoring the need for even more drastic measures.

The mayor, clearly agonizing over his course of action, said that
options were running out, and that new restrictions would bring new
pain.

``Folks have to understand that, right now, with so many New Yorkers
losing employment, losing paychecks, dealing with all sorts of stresses
and strains, I'm hearing constantly from people who are tremendously
worried about how they're going to make ends meet,'' the mayor said.

\href{https://www.nytimes.com/news-event/coronavirus?action=click\&pgtype=Article\&state=default\&region=MAIN_CONTENT_3\&context=storylines_faq}{}

\hypertarget{the-coronavirus-outbreak-}{%
\subsubsection{The Coronavirus Outbreak
›}\label{the-coronavirus-outbreak-}}

\hypertarget{frequently-asked-questions}{%
\paragraph{Frequently Asked
Questions}\label{frequently-asked-questions}}

Updated August 4, 2020

\begin{itemize}
\item ~
  \hypertarget{i-have-antibodies-am-i-now-immune}{%
  \paragraph{I have antibodies. Am I now
  immune?}\label{i-have-antibodies-am-i-now-immune}}

  \begin{itemize}
  \tightlist
  \item
    As of right
    now,\href{https://www.nytimes.com/2020/07/22/health/covid-antibodies-herd-immunity.html?action=click\&pgtype=Article\&state=default\&region=MAIN_CONTENT_3\&context=storylines_faq}{that
    seems likely, for at least several months.} There have been
    frightening accounts of people suffering what seems to be a second
    bout of Covid-19. But experts say these patients may have a
    drawn-out course of infection, with the virus taking a slow toll
    weeks to months after initial exposure. People infected with the
    coronavirus typically
    \href{https://www.nature.com/articles/s41586-020-2456-9}{produce}
    immune molecules called antibodies, which are
    \href{https://www.nytimes.com/2020/05/07/health/coronavirus-antibody-prevalence.html?action=click\&pgtype=Article\&state=default\&region=MAIN_CONTENT_3\&context=storylines_faq}{protective
    proteins made in response to an
    infection}\href{https://www.nytimes.com/2020/05/07/health/coronavirus-antibody-prevalence.html?action=click\&pgtype=Article\&state=default\&region=MAIN_CONTENT_3\&context=storylines_faq}{.
    These antibodies may} last in the body
    \href{https://www.nature.com/articles/s41591-020-0965-6}{only two to
    three months}, which may seem worrisome, but that's perfectly normal
    after an acute infection subsides, said Dr. Michael Mina, an
    immunologist at Harvard University. It may be possible to get the
    coronavirus again, but it's highly unlikely that it would be
    possible in a short window of time from initial infection or make
    people sicker the second time.
  \end{itemize}
\item ~
  \hypertarget{im-a-small-business-owner-can-i-get-relief}{%
  \paragraph{I'm a small-business owner. Can I get
  relief?}\label{im-a-small-business-owner-can-i-get-relief}}

  \begin{itemize}
  \tightlist
  \item
    The
    \href{https://www.nytimes.com/article/small-business-loans-stimulus-grants-freelancers-coronavirus.html?action=click\&pgtype=Article\&state=default\&region=MAIN_CONTENT_3\&context=storylines_faq}{stimulus
    bills enacted in March} offer help for the millions of American
    small businesses. Those eligible for aid are businesses and
    nonprofit organizations with fewer than 500 workers, including sole
    proprietorships, independent contractors and freelancers. Some
    larger companies in some industries are also eligible. The help
    being offered, which is being managed by the Small Business
    Administration, includes the Paycheck Protection Program and the
    Economic Injury Disaster Loan program. But lots of folks have
    \href{https://www.nytimes.com/interactive/2020/05/07/business/small-business-loans-coronavirus.html?action=click\&pgtype=Article\&state=default\&region=MAIN_CONTENT_3\&context=storylines_faq}{not
    yet seen payouts.} Even those who have received help are confused:
    The rules are draconian, and some are stuck sitting on
    \href{https://www.nytimes.com/2020/05/02/business/economy/loans-coronavirus-small-business.html?action=click\&pgtype=Article\&state=default\&region=MAIN_CONTENT_3\&context=storylines_faq}{money
    they don't know how to use.} Many small-business owners are getting
    less than they expected or
    \href{https://www.nytimes.com/2020/06/10/business/Small-business-loans-ppp.html?action=click\&pgtype=Article\&state=default\&region=MAIN_CONTENT_3\&context=storylines_faq}{not
    hearing anything at all.}
  \end{itemize}
\item ~
  \hypertarget{what-are-my-rights-if-i-am-worried-about-going-back-to-work}{%
  \paragraph{What are my rights if I am worried about going back to
  work?}\label{what-are-my-rights-if-i-am-worried-about-going-back-to-work}}

  \begin{itemize}
  \tightlist
  \item
    Employers have to provide
    \href{https://www.osha.gov/SLTC/covid-19/standards.html}{a safe
    workplace} with policies that protect everyone equally.
    \href{https://www.nytimes.com/article/coronavirus-money-unemployment.html?action=click\&pgtype=Article\&state=default\&region=MAIN_CONTENT_3\&context=storylines_faq}{And
    if one of your co-workers tests positive for the coronavirus, the
    C.D.C.} has said that
    \href{https://www.cdc.gov/coronavirus/2019-ncov/community/guidance-business-response.html}{employers
    should tell their employees} -\/- without giving you the sick
    employee's name -\/- that they may have been exposed to the virus.
  \end{itemize}
\item ~
  \hypertarget{should-i-refinance-my-mortgage}{%
  \paragraph{Should I refinance my
  mortgage?}\label{should-i-refinance-my-mortgage}}

  \begin{itemize}
  \tightlist
  \item
    \href{https://www.nytimes.com/article/coronavirus-money-unemployment.html?action=click\&pgtype=Article\&state=default\&region=MAIN_CONTENT_3\&context=storylines_faq}{It
    could be a good idea,} because mortgage rates have
    \href{https://www.nytimes.com/2020/07/16/business/mortgage-rates-below-3-percent.html?action=click\&pgtype=Article\&state=default\&region=MAIN_CONTENT_3\&context=storylines_faq}{never
    been lower.} Refinancing requests have pushed mortgage applications
    to some of the highest levels since 2008, so be prepared to get in
    line. But defaults are also up, so if you're thinking about buying a
    home, be aware that some lenders have tightened their standards.
  \end{itemize}
\item ~
  \hypertarget{what-is-school-going-to-look-like-in-september}{%
  \paragraph{What is school going to look like in
  September?}\label{what-is-school-going-to-look-like-in-september}}

  \begin{itemize}
  \tightlist
  \item
    It is unlikely that many schools will return to a normal schedule
    this fall, requiring the grind of
    \href{https://www.nytimes.com/2020/06/05/us/coronavirus-education-lost-learning.html?action=click\&pgtype=Article\&state=default\&region=MAIN_CONTENT_3\&context=storylines_faq}{online
    learning},
    \href{https://www.nytimes.com/2020/05/29/us/coronavirus-child-care-centers.html?action=click\&pgtype=Article\&state=default\&region=MAIN_CONTENT_3\&context=storylines_faq}{makeshift
    child care} and
    \href{https://www.nytimes.com/2020/06/03/business/economy/coronavirus-working-women.html?action=click\&pgtype=Article\&state=default\&region=MAIN_CONTENT_3\&context=storylines_faq}{stunted
    workdays} to continue. California's two largest public school
    districts --- Los Angeles and San Diego --- said on July 13, that
    \href{https://www.nytimes.com/2020/07/13/us/lausd-san-diego-school-reopening.html?action=click\&pgtype=Article\&state=default\&region=MAIN_CONTENT_3\&context=storylines_faq}{instruction
    will be remote-only in the fall}, citing concerns that surging
    coronavirus infections in their areas pose too dire a risk for
    students and teachers. Together, the two districts enroll some
    825,000 students. They are the largest in the country so far to
    abandon plans for even a partial physical return to classrooms when
    they reopen in August. For other districts, the solution won't be an
    all-or-nothing approach.
    \href{https://bioethics.jhu.edu/research-and-outreach/projects/eschool-initiative/school-policy-tracker/}{Many
    systems}, including the nation's largest, New York City, are
    devising
    \href{https://www.nytimes.com/2020/06/26/us/coronavirus-schools-reopen-fall.html?action=click\&pgtype=Article\&state=default\&region=MAIN_CONTENT_3\&context=storylines_faq}{hybrid
    plans} that involve spending some days in classrooms and other days
    online. There's no national policy on this yet, so check with your
    municipal school system regularly to see what is happening in your
    community.
  \end{itemize}
\end{itemize}

``I think New Yorkers should be prepared right now for the possibility
of a `shelter in place' order,'' Mr. de Blasio said.

As it is, the day after broad shutdowns were put in place on Monday
across New York, New Jersey and Connecticut, the region was quietly
staggering.

After the state waived the seven-day waiting period for filing for
unemployment benefits and the resulting surge crashed the Labor
Department's website, officials said the number of applicants was like
nothing seen since the aftermath of the Sept. 11 attacks of 2001. The
city comptroller said that New York could lose more than \$3 billion in
tax revenue.
\href{https://www.nytimes.com/2020/03/16/nyregion/Coronavirus-nyc-economy-.html}{One
estimate} put lost wages in the tourism industry alone at \$1 billion
per month.

``The economy was teetering to begin with,'' Mr. Cuomo said on Monday
night. ``This is a deep, deep economic hole. You'll have businesses
close that never reopen.''

\includegraphics{https://static01.nyt.com/images/2017/01/29/podcasts/the-daily-album-art/the-daily-album-art-articleInline-v2.jpg?quality=75\&auto=webp\&disable=upscale}

\hypertarget{listen-to-the-daily-gov-andrew-cuomo-its-making-sure-we-live-through-this}{%
\subsubsection{Listen to `The Daily': Gov. Andrew Cuomo: `It's Making
Sure We Live Through
This.'}\label{listen-to-the-daily-gov-andrew-cuomo-its-making-sure-we-live-through-this}}

We sat down with the person in charge of New York State's response to
the coronavirus crisis.

transcript

Back to The Daily

bars

0:00/33:01

-33:01

transcript

\hypertarget{listen-to-the-daily-gov-andrew-cuomo-its-making-sure-we-live-through-this-1}{%
\subsection{Listen to `The Daily': Gov. Andrew Cuomo: `It's Making Sure
We Live Through
This.'}\label{listen-to-the-daily-gov-andrew-cuomo-its-making-sure-we-live-through-this-1}}

\hypertarget{hosted-by-michael-barbaro-produced-by-austin-mitchell-adizah-eghan-and-lynsea-garrison-with-help-from-jessica-cheung-and-edited-by-lisa-tobin}{%
\subsubsection{Hosted by Michael Barbaro; produced by Austin Mitchell,
Adizah Eghan and Lynsea Garrison; with help from Jessica Cheung; and
edited by Lisa
Tobin}\label{hosted-by-michael-barbaro-produced-by-austin-mitchell-adizah-eghan-and-lynsea-garrison-with-help-from-jessica-cheung-and-edited-by-lisa-tobin}}

\hypertarget{we-sat-down-with-the-person-in-charge-of-new-york-states-response-to-the-coronavirus-crisis}{%
\paragraph{We sat down with the person in charge of New York State's
response to the coronavirus
crisis.}\label{we-sat-down-with-the-person-in-charge-of-new-york-states-response-to-the-coronavirus-crisis}}

\begin{itemize}
\item
  michael barbaro\\
  I am now disinfecting this microphone for the governor. And the
  windscreen. Because that's how we roll these days.
\item
  lynsea garrison\\
  Yep.
\item
  michael barbaro\\
  I travel everywhere now with Lysol wipes.
\item
  lynsea garrison\\
  Yeah. Do you want to disinfect this? Since you maybe, depending on our
  sitting arrangement, might be holding that.
\item
  michael barbaro\\
  Yep.
\item
  lynsea garrison\\
  OK.
\item
  michael barbaro\\
  OK, let's go.
\item
  speaker\\
  We're just going to hold on one second.
\item
  michael barbaro\\
  Oh, we're going to hold ---
\item
  lynsea garrison\\
  Oh, sure.
\item
  andrew cuomo\\
  Look at you!
\item
  michael barbaro\\
  How are you?
\item
  speaker\\
  Hey, Michael!
\item
  andrew cuomo\\
  Ageless.
\item
  michael barbaro\\
  Are we allowed to shake hands?
\item
  lisa tobin\\
  No.
\item
  andrew cuomo\\
  Oh, you're right, you're right, you're right.
\item
  speaker\\
  Absolutely not allowed.
\item
  michael barbaro\\
  Ritual is very hard. Governor, this is Lynsea Garrison, Lisa Tobin,
  Governor Cuomo.
\item
  andrew cuomo\\
  {[}LAUGHS{]}
\item
  michael barbaro\\
  From The New York Times, I'm Michael Barbaro. This is ``The Daily.''
\item
  {[}music{]}
\item
  archived recording 1\\
  Tonight, a scramble to contain the spread of the coronavirus in New
  York. In New York City tonight, about 1,000 people are now under
  self-quarantine.
\item
  archived recording 2\\
  Governor Andrew Cuomo declared a state of emergency this weekend to
  help fund the medical response to the outbreak.
\end{itemize}

michael barbaro

As one of the earliest states with confirmed cases of the coronavirus,
and with the most confirmed cases so far, New York State has begun to
aggressively move to control its spread.

\begin{itemize}
\tightlist
\item
  archived recording\\
  Governor Cuomo signing an executive order closing all schools
  statewide for the next two weeks. Now this means ---
\end{itemize}

michael barbaro

Taking a series of increasingly drastic steps over the past few days.

\begin{itemize}
\item
  archived recording 1\\
  In New York, Governor Cuomo is advising nonessential businesses to
  close each night at 8 p.m.
\item
  archived recording 2\\
  Mayor de Blasio warned New York City residents to be prepared for a
  possible shelter-in-place order in the next 48 hours. Governor Cuomo
  shifts his emphasis to the health care system ---
\end{itemize}

michael barbaro

Today: A conversation with Governor Andrew Cuomo.

It's Wednesday, March 18.

So I want to thank you for letting us ---

andrew cuomo

I'm just examining ---

michael barbaro

Examining the microphone? That's a windscreen. It'll keep your ---

andrew cuomo

Keep the wind down.

michael barbaro

The wind down.

andrew cuomo

Because it's windy.

michael barbaro

So Governor, I want to thank you for letting us in in the middle of an
extraordinary crisis, and tell you how much we appreciate it. I want to
start this conversation by asking you where New York is in this
pandemic? It's Tuesday afternoon, around 3 o'clock. How many New Yorkers
do we understand have the coronavirus at this point?

andrew cuomo

We have, right now, over 1,000 cases. It's a little misleading, because
we're talking about these tests as if it's taking a random sample,
right? But it's not. The test results are purely symbolic of how many
tests you're taking. We are now taking more tests than most states, and
we're finding more positives, which would make sense, also. Because we
are the dense state, and this is a function of density at the end of the
day. You're getting on a subway train, you're getting on a bus, you're
in a crowded restaurant, you're in a crowded office space. And this
transfers in the crowds. So that it would be here first is not
surprising. That it would communicate most easily here is not
surprising. And that we would have the sophisticated health system that
would detect it here first is not surprising.

michael barbaro

So if these are the front lines of this epidemic, and I've heard you
describe this as a kind of war that we're in right now, what stage of
the war are we at in a place like New York?

andrew cuomo

We are seeing the enemy on the horizon, and they are approaching very
quickly, and we don't have our defenses in place.

michael barbaro

We don't.

andrew cuomo

We don't. Testing was the first level of defense, right? The testing was
slow nationwide. We're now ramping up in this state because the federal
government, I think, made a wise decision. We were the first to ask for
it. I asked the president for it directly. Basically said, decentralize
the testing, leave it to the states. We have 200 laboratories in this
state. I said, decentralize it, let the states do it.

michael barbaro

But you weren't allowed at first.

andrew cuomo

Right. The federal government was controlling it, and you were running
all the national tests through the C.D.C., which was then sending them
to Atlanta. So we're now ramping up on testing, that's why our numbers
are high. But testing is no longer going to keep the genie in the
bottle, right? The genie is out of the bottle now. Where this all comes
down to is, when they talk about flattening the curve, flattening the
curve, they're trying to slow the advance of the enemy until we can get
enough of our defenses in place. What are the defenses? A health care
system that can handle the injured, to torture the metaphor. And we're
not there. If you look at the speed, the increase in the rate, the spike
in the increase of the number of cases, we're looking at a possibility
of an apex being about 45 days away.

michael barbaro

The peak of this pandemic here?

andrew cuomo

The peak. That's one projection --- 45 days. Needing 110,000 hospital
beds. In this state, you have 50,000 hospital beds. Needing 37,000
intensive care unit beds, and having 3,000 I.C.U. beds.

michael barbaro

Needing 37,000, having three.

andrew cuomo

Yes.

michael barbaro

That's a pretty extraordinary gap.

andrew cuomo

Yes. Because the injured here are going to be predominantly senior
citizens, compromised immune systems, underlying illness. And those
people need I.C.U.s. When they come into the hospital, they don't need a
normal bed and moderate health care. They need an I.C.U.

michael barbaro

So I want to talk about your leadership in this war, to similarly
torture the metaphor. The work you've done in the last few days to
flatten the curve. Because you've made some extraordinary decisions in
the past 72 hours or so. Efforts to essentially start shutting down
systematically elements of our life here in New York. So help me
understand the information that you've been receiving, the calculations
that you've been weighing, and the very real trade-offs that you
understood would have to be made.

andrew cuomo

I'm watching the increase in cases. And you take one measure, and you
see what the effect was. You take another measure, and you see what the
effect was. And nothing was having an effect. Nothing we were doing.

michael barbaro

What steps did you take that were not effective?

andrew cuomo

The testing was supposed to be step one. That was supposed to slow the
spread. That didn't work. OK, the enemy keeps coming. You start moderate
social distancing. Businesses, voluntary basis, work from home. That
didn't make any difference. The numbers have kept going up regardless of
everything we did. When you keep seeing those numbers increase, your
efforts have to become more and more dramatic. Yesterday, we went to the
point of closing bars, restaurants, gyms and schools, with the
precaution of providing child care for essential workers, especially
nurses, health care workers. The next level of efforts to control
density, control the spread, would be to start closing --- mandatory
closing --- of businesses.

michael barbaro

Let me focus in on that decision. Bars, restaurants. Because that is
billions of dollars in lost revenue. It's tens of thousands of people
out of work. On my way here, I got a text from a friend who said he had
just laid off 90 employees. And he was crying the whole time he had to
do it. So let's talk about how you made that decision because of the
impact that that is immediately going to have --- it's a huge part of
the economy in the state. And so how did you get to that decision?

andrew cuomo

Michael, you are past the point of monetizing these decisions.

michael barbaro

What do you mean?

andrew cuomo

You are at a point of deciding how many people are going to live, how
many people are going to die? That's where you are. Closing restaurants
reduces the spread of the disease. The disease transfers very quickly,
not just in the cough and the droplets, et cetera. There are some
studies that say that disease can live --- the virus can live up to two
or three days on a surface.

michael barbaro

Like a table at a restaurant.

andrew cuomo

Just think about that. Like a table in a restaurant, like a sink, like a
handrail, in a bus. Two or three days. It's why this virus is so
vicious. And we know the trajectory right now overwhelms the hospital
system. Three or four-fold. It's not even close. People will die because
they can't get the health care service they need.

michael barbaro

So you're reducing the number of people who die because they can't get
into a hospital bed, for every restaurant you close and every
transmission you prevent in closing that restaurant.

andrew cuomo

Yes.

michael barbaro

That's the thing. It's just pure numbers.

andrew cuomo

Yes. And it's not even just New York. The whole nation is past the point
of, let's try to save money, right? You look at the Dow Jones market,
you look at all the businesses that are closing. This is now a national
phenomenon that this economy is going to be very badly hurt. The
recovery of this economy is going to be an economic feat never seen
before. You're going to have to go back to the Great Depression to come
up with a revival plan for the economy like we're seeing now. You're
going to see mortgage foreclosures, you're going to see bankruptcies,
you're going to see massive unemployment claims all across the board.

michael barbaro

I don't see you sugarcoating this at all.

andrew cuomo

No. This is going to be --- our state finances are decimated, right? How
does the state work? A state is just a percentage of every other
business.

michael barbaro

Right.

andrew cuomo

Those businesses are all closed. Or their revenues have been cut by 50,
60, 70 percent. But I think the good thing, as a nation, is we said, so
what? So what? What value on a human life? If I can save here 5,000
lives, 10,000 lives, I don't care what it costs, Michael. That's what
I'm going to do.

michael barbaro

I wonder what you want to say to somebody who has just lost their job,
because there are now a lot of them, who may not be able to pay their
rent, who may not be able to pay their mortgage, who may lose their
housing, and who are really scared because of these economic
consequences. What do you want them to hear you say?

andrew cuomo

I would say first, I hope no one in your family, or no one you know,
dies because of this. Because that's what we're trying to accomplish. I
hope no one in your family dies. Second, we all understand the economic
consequences. It's not just you, it is everyone. And by the way, take
solace in that fact. Because maybe if it was just you, you could be
forgotten and left on the side of the road. It's not just you. It's
everyone, and it's everywhere. The Italians have an old saying that the
rich man is the man who has health, right? If you have your health, you
can figure anything else out. And it's true. We'll figure out the
economy. I went through 9/11. Oh, downtown Manhattan is devastated, we
have to rebuild, how do we do this? We're alive, first of all. And if we
are alive, we'll figure out the rest. We'll figure out the money. It's
making sure we live through this.

{[}music{]}

michael barbaro

We'll be right back.

Governor, I want to understand how you're thinking about something else,
which is hospitals, supplies and readiness. You've started to signal
that there's a major shortage of I.C.U. units. What about respirators?
What does the picture start to look like in a couple of weeks, and are
we ready for it?

andrew cuomo

We are not ready for it, certainly, today. The picture looks like you
have tens of thousands of people coming to the hospital. These are
respiratory illnesses. They can't breathe. They need an I.C.U. bed with
a ventilator. OK, buy more ventilators. OK, you can't.

michael barbaro

You can't.

andrew cuomo

Because the entire world is trying to buy ventilators.

michael barbaro

So you've tried to buy ventilators?

andrew cuomo

We try every which way to buy ventilators. We're trying to go to China,
which is now over it, trying to buy their ventilators.

michael barbaro

Wow.

andrew cuomo

I mean, it is a global competition to buy ventilators. The federal
government has an emergency medical stockpile. I reached out to the
president. Federal cooperation is everything, Michael, because it's
whatever the federal government has in that stockpile is going to be our
main access.

michael barbaro

Did you ask to tap into the stockpile?

andrew cuomo

Yes.

michael barbaro

Then what did the president say?

andrew cuomo

Yes.

michael barbaro

If I can ask.

andrew cuomo

He has said he will be very helpful. We're looking at the Army Corps of
Engineers to try to build additional hospital beds, convert dormitories,
et cetera. Because you're overwhelming the capacity of the health care
system by two or three times. You need backup staff, backup nurses,
backup doctors, more space, more equipment, more gloves, more food, more
everything.

michael barbaro

Is there a version where hospitals can handle this influx? Or is it just
a matter of how short they fall?

andrew cuomo

There is no way they can handle this.

michael barbaro

So then, do you accept that some incredibly difficult decisions are
going to have to be made inside hospitals in the coming days? Decisions
of who lives and who dies, who gets a bed, who doesn't, who gets a
respirator, who doesn't get a respirator. Who to prioritize. Is that
something doctors should be deciding, or is that something government
should be playing a role in?

andrew cuomo

It will be a question of triage. Remember, most of these people will
have serious underlying conditions already. And in some ways, it will
become self-selecting, depending on how ill you were when you came in.

michael barbaro

Right, but when the decision has to be made, do I put the 85-year-old
with underlying conditions in the I.C.U., who might have a 50-50 chance,
or do I put a 45-year-old in the I.C.U. who's come in with respiratory
problems who has a 60 percent chance? We just talked to a doctor in
Italy who had to make these choices. Do you want to be the one issuing
protocols? Do you want the president to be issuing those protocols? Who
should be guiding those kind of awful decisions?

andrew cuomo

Well, I pray we don't get there. But if we get there, it should be a
medical decision, unless God intervenes and God makes the determination
first.

michael barbaro

What is the ideal role of the federal government right now, in your
mind?

andrew cuomo

Right now, crank up the Army Corps of Engineers, which does have
building capacity. Add to hospital capacity in the states that need it
--- New York would be at the top of the list. That's what they do,
right? They build the infrastructure for war. They go into a country
where nothing exists, they cut down trees, they build roads, they build
camps. Because the states don't have the capacity or the resources. I
don't have a workforce. Mobilized FEMA, the Federal Emergency Management
Agency, which has tremendous potential when it works well, right? FEMA
did Hurricane Katrina, which was FEMA not doing a good job. FEMA can be
extraordinarily good when it's staffed and funded. So we need them fully
deployed here.

michael barbaro

And are they doing that?

andrew cuomo

The president has now --- I believe yesterday, the president's tone was
100 percent serious. He showed more sobriety on this issue than he has
shown. I spoke with him twice today already. I know he has his team
working. I was on the phone with him late last night, early this
morning. So I believe he is fully committed, and he understands the role
and he understands the severity. And that is good news.

michael barbaro

Let me ask you directly, what do you think of President Trump's
leadership in this moment? It began with some skepticism about the
severity of the situation. It has changed, like you just signaled. Is
the president your partner here?

andrew cuomo

Yeah. Let me say this. I have had a tumultuous relationship with this
president. I have opposed many of his policies, vociferously. You could
probably say there has been no governor in the country who has been as
aggressive in his opposition to the president as I have. Both
ideologically and practically. And I probably have sued the president
more than any governor in the United States. So having said that, I said
to the president again this morning, look, forget everything. Forget
Democrats, forget Republicans. We're Americans, and that always came
first. And that's where we are. I put out my hand in partnership. I need
your help. I'm grateful for your help. I'll be a committed partner.
Let's get this done. Let's save lives.

michael barbaro

Did he say anything to make you feel like that was to be reciprocated?

andrew cuomo

Yes. Yes. He said --- yes, exactly. This country has gotten itself into
this hyperpartisan hype. This ideological intensity. And I understand
why. It has been for me too, in truth. But then something happens and it
changes your whole perspective. Right? You can be fighting in your
family, with your siblings, and I'm not going to go to your birthday
party, and I'm not going --- and then the parent dies, and you say to
each other, what have we been doing? What a waste of time.

michael barbaro

You think we're at a moment that may transcend?

andrew cuomo

You're talking about Americans dying here. That's what you're talking
about. Americans dying. Forget everything else. Life is as life and
death, and that changes your perspective. We can have the arguments
another day. It also changes, by the way, your perspective on
government. Think about this. When was the last time this country
actually needed government? Needed it. Needed it to be competent and
qualified and needed leaders to be real leaders. Not celebrity leaders,
not good-looking, handsome, charismatic leaders. I like this one. This
one's sexy. This one's funny. It's a totally different lens. No, this
thing called government is very serious. This is serious business. You
have to know what you're doing, you have to know how to mobilize ---
what is this Army Corps of Engineers, and FEMA, and how do you build a
hospital in 45 days, and how do you do triage, and how do you make all
these things happen in state local relations, and passing emergency
appropriations, and how do you get emergency funding for purchasing, and
emergency orders? Wow, I didn't even know government did that.

michael barbaro

Right. This is what government is actually for.

andrew cuomo

Yes.

michael barbaro

And every so often, we have a moment that demonstrates why government
exists.

andrew cuomo

Yes. And it doesn't matter until it matters.

michael barbaro

Right. Is there more coming, Governor? What kind of measures should your
constituents --- should all New Yorkers, and maybe even people beyond
New York --- be getting ready to take on? As we walked into this room,
we got word, for example, that it looks like New York is going to
essentially order a shelter-in-place condition, which means, basically,
you can't leave your house. What more is coming?

andrew cuomo

Yes. That is not going to happen, shelter in place. For New York City,
or any city or county to take an emergency action, the state has to
approve it. And I wouldn't approve shelter in place. That scares people,
right? Quarantine in place, you can't leave your home. The fear, the
panic, is a bigger problem than the virus.

michael barbaro

It is?

andrew cuomo

Yes. And I shut that down immediately. The density control measures
would be more --- we're going to close businesses.

michael barbaro

You're going to potentially close all businesses?

andrew cuomo

Potentially. Potentially. Italy took the most drastic density control,
that only essential businesses --- grocery stores, first responders,
pharmacies, et cetera. But I am against quarantines, you must stay in
your home. You can come out of your house, just don't be in a crowded
situation, don't cause more density, don't sneeze in someone's face
within 6 feet.

michael barbaro

Right.

andrew cuomo

Go walk in the park. I mean that as a nice thing. That's a positive
suggestion, you go walk in the park.

michael barbaro

No, I try to take an evening walk. I appreciate that.

andrew cuomo

The old neighborhood, they used to say, go take a walk in the park. That
was a bad thing.

michael barbaro

In Queens.

andrew cuomo

In Queens.

michael barbaro

If we're at a moment where it's too late to look back and say, if only
we had done this, if only we had done that, and instead, we're at a
moment where, even if the government steps up in every way we want it
to, everyone now has to do their part as well. What's your message to
them?

andrew cuomo

First of all, welcome to life. If I had only done this, if I had only
done this, if I had only done this. That's life, my brother. That's all
of us. I forget that. You're here, now. What do you do now? And that's
all that matters. The enemy has not advanced to a point where they are
in the foxhole, right? We still have some time.

michael barbaro

Not much.

andrew cuomo

Not much. But what we do between now and then matters gravely. Do
everything you can. Do everything you can to flatten that curve. Yes,
your friend who owns the restaurant, I'm sure is very angry at me.

michael barbaro

Mm hmm.

andrew cuomo

But you know what? I did it because I believe it was necessary to save
lives. We're going to have to take more actions like that to reduce the
density and flatten the curve. Do everything you can to build more
hospital beds in 45 days. Well, it's impossible. Yeah, well, I'm going
to try my damnedest to show you it's not impossible. Do everything that
you can humanly possibly do. Extend your imagination in a way you never
thought. Extend your ambition beyond yourself. Because it's not about
you, it's about us. It's about the collective. It's about society. Don't
expose yourself to other people. Don't indulge yourself. Yeah, I know
you really want to go out and go shopping and then --- yeah, I know you
do. But don't think of just yourself. Save as many lives as you can.

{[}music{]}

andrew cuomo

Be responsible, be civic-minded, be kind, be considerate, think of one
another. Yes, we're going to have an inconvenient period for a few
months. We are. Deal with it. And deal with it gracefully. And deal with
it with kindness and intelligence.

michael barbaro

Governor, I really appreciate your time. Thank you for having us in.

andrew cuomo

Thank you.

michael barbaro

And good luck getting through all this.

andrew cuomo

Thank you.

michael barbaro

We're going to walk in this office, but keep our space.

lisa tobin

Can I just ask a quick question? If it says New York City tells 8
million people to be prepared to shelter in place, that is not going to
happen?

andrew cuomo

No.

lisa tobin

But it's playing on the television right now.

andrew cuomo

Yeah, I know. I know.

lisa tobin

What are you going to do?

andrew cuomo

Yeah, I don't know anyone at CNN. Yeah. But see how scary that is?

michael barbaro

Your brother is an anchor on CNN.

andrew cuomo

That was a joke. Bada boom. Bada boom.

michael barbaro

Bada boom.

andrew cuomo

I normally hold up a little sign, saying joke coming.

lisa tobin

No, but --- I'm sorry to interrupt, but in all seriousness, if that's on
CNN ---

speaker

We already put a statement out that said that we were not considering
it.

lisa tobin

OK.

speaker

So it'll be clarified, hopefully in the next five minutes.

lisa tobin

OK.

andrew cuomo

But that's why the fear, why the panic? Because you watch things like
that all day. And everybody --- somebody says something, and then it's
on the screen right away. Oh my god, I'm going to be locked in my home.
I better go to the store and buy stuff. And now the stores ---

michael barbaro

We'll be right back.

Here's what else you need to know today. On Tuesday, dozens of countries
moved to close their borders, to slow the spread of the coronavirus.

\begin{itemize}
\tightlist
\item
  archived recording (ursula von der leyen)\\
  The less travel, the more we can contain the virus. Therefore, I
  propose to the heads of state and government to introduce temporary
  restriction on nonessential travel to the European Union.
\end{itemize}

michael barbaro

The European Union voted to shut off at least 26 of its nations to
nearly all outside visitors for at least 30 days and perhaps longer.
While Russia will bar entry to most foreigners, starting today. In the
United States, where thousands of businesses have stopped operating over
the past few days, the Trump administration said it was racing to
stimulate the economy, to stave off a deep recession.

\begin{itemize}
\tightlist
\item
  archived recording (steve mnuchin)\\
  We're looking at sending checks to Americans immediately. And what
  we've heard from hardworking Americans, many companies have now shut
  down, whether it's bars or restaurants. Americans need cash now, and
  the president wants to get cash now.
\end{itemize}

michael barbaro

During a news conference, Treasury Secretary Steve Mnuchin outlined a
plan under negotiation with Congress for direct payments to Americans.

\begin{itemize}
\item
  archived recording (steve mnuchin)\\
  And I mean now, in the next two weeks.
\item
  archived recording\\
  How much?
\item
  archived recording (steve mnuchin)\\
  There's some numbers out there. They may be a little bit bigger than
  what's in the press.
\end{itemize}

michael barbaro

Meanwhile, on Tuesday night, West Virginia became the 50th state to
report an infection. And Joe Biden won all three primaries on Tuesday
--- in Florida, Illinois and Arizona, dashing Bernie Sanders's hope of a
comeback and solidifying Biden's lead.

\begin{itemize}
\tightlist
\item
  archived recording (joe biden)\\
  Now, it's the moments like these when we realize that we need to put
  politics aside and work together as Americans.
\end{itemize}

michael barbaro

On Tuesday night, Biden devoted much of his victory speech to the
pandemic.

\begin{itemize}
\tightlist
\item
  archived recording (joe biden)\\
  The coronavirus doesn't care if you're a Democrat or Republican. It
  will not discriminate based on national origin, race, gender or your
  zip code. It will touch people in positions of power, as well as the
  most vulnerable people in our society. We're all in this together.
  This is a moment for each of us.
\end{itemize}

{[}music{]}

michael barbaro

That's it for ``The Daily.'' I'm Michael Barbaro. See you tomorrow.

Image

At McSorley's Old Ale House in the East Village, no one could drink at
the bar but customers could buy beer to go.Credit...Chang W. Lee/The New
York Times

It has not come to that at the Pasha Turkish restaurant on the Upper
West Side. ``Every customer that comes in, we try to reassure them that
we're all in this together,'' said Rhea Alexis Stuart, the general
manager.

But she's worried about her waiters and waitresses: ``What do you say to
somebody who most of their salary is tips?''

In Greenpoint, Brooklyn, Tim Murray, co-owner of the Broken Land bar,
was also concerned. He said the bar was giving all of its profits to
employees who were now out of work because of virus-related
restrictions. ``We had to lay everybody off on Monday,'' he said. ``It
really broke our hearts. It's the worst.''

Still, the virus continued its march. More than 1,500 people in New York
State had tested positive as of Tuesday --- a jump of more than 500
since Monday. The few that have been identified include
\href{https://www.nytimes.com/2020/03/17/sports/brooklyn-nets-coronavirus.html}{four
basketball players from the Brooklyn Nets}; several city and state
lawmakers; and the head of the New York Police Department's transit
bureau. Twelve people with the virus have died in the state.

People who traveled to New York City took the virus home with them: 19
people who attended a conference of group therapists in Midtown
Manhattan in the first week of March have tested positive, the American
Group Psychotherapy Association said on Tuesday. They came from six
states and three countries, from locations as far-flung as Nebraska and
Singapore.

Officials in New York have warned that the virus is threatening to
overwhelm the health care system within a matter of weeks. Mr. Cuomo
said that the contagion is expected to keep rising until it peaks around
the beginning of May.

At that point, he said, the state is expected to need at least 55,000
hospital beds and 18,000 beds in intensive-care units, possibly double
those figures. The state currently has only 53,000 hospital beds and
3,000 intensive-care beds --- and 80 percent of the intensive-care beds
are already occupied. The governor has urged that the Army Corps of
Engineers be deployed to create makeshift medical wards out of
dormitories and other buildings.

The state is many
\href{https://www.nytimes.com/2020/03/17/nyregion/ny-coronavirus-ventilators.html}{thousands
short of the number of ventilators} it would need to help the sickest
people breathe if the virus behaves as expected.

``The numbers are daunting,'' Mr. Cuomo said. ``What are we doing?
Everything we can.''

In one encouraging development on Tuesday, Mr. de Blasio announced that
the city had reached an agreement with a large testing company,
BioReference Laboratories, that will let the city's public hospitals and
clinics test up to 5,000 people a day. They are currently testing only
hundreds. But the results of the tests, the mayor said, would inevitably
bring much bad news.

``We are certainly going to have thousands of cases next week,'' Mr. de
Blasio said. ``It is not that long before we hit 10,000 cases. That is a
true statement.''

Reporting was contributed by Annie Correal, Luis Ferré-Sadurní, Joseph
Goldstein, Corey Kilgannon, Patrick McGeehan, Jeffery C. Mays, Jesse
McKinley Liam Stack and Neil Vigdor.

Advertisement

\protect\hyperlink{after-bottom}{Continue reading the main story}

\hypertarget{site-index}{%
\subsection{Site Index}\label{site-index}}

\hypertarget{site-information-navigation}{%
\subsection{Site Information
Navigation}\label{site-information-navigation}}

\begin{itemize}
\tightlist
\item
  \href{https://help.nytimes.com/hc/en-us/articles/115014792127-Copyright-notice}{©~2020~The
  New York Times Company}
\end{itemize}

\begin{itemize}
\tightlist
\item
  \href{https://www.nytco.com/}{NYTCo}
\item
  \href{https://help.nytimes.com/hc/en-us/articles/115015385887-Contact-Us}{Contact
  Us}
\item
  \href{https://www.nytco.com/careers/}{Work with us}
\item
  \href{https://nytmediakit.com/}{Advertise}
\item
  \href{http://www.tbrandstudio.com/}{T Brand Studio}
\item
  \href{https://www.nytimes.com/privacy/cookie-policy\#how-do-i-manage-trackers}{Your
  Ad Choices}
\item
  \href{https://www.nytimes.com/privacy}{Privacy}
\item
  \href{https://help.nytimes.com/hc/en-us/articles/115014893428-Terms-of-service}{Terms
  of Service}
\item
  \href{https://help.nytimes.com/hc/en-us/articles/115014893968-Terms-of-sale}{Terms
  of Sale}
\item
  \href{https://spiderbites.nytimes.com}{Site Map}
\item
  \href{https://help.nytimes.com/hc/en-us}{Help}
\item
  \href{https://www.nytimes.com/subscription?campaignId=37WXW}{Subscriptions}
\end{itemize}
