Sections

SEARCH

\protect\hyperlink{site-content}{Skip to
content}\protect\hyperlink{site-index}{Skip to site index}

\href{https://www.nytimes.com/section/world/asia}{Asia Pacific}

\href{https://myaccount.nytimes.com/auth/login?response_type=cookie\&client_id=vi}{}

\href{https://www.nytimes.com/section/todayspaper}{Today's Paper}

\href{/section/world/asia}{Asia Pacific}\textbar{}China Defends
Expulsion of American Journalists, Accusing U.S. of Prejudice

\url{https://nyti.ms/33vNQWa}

\begin{itemize}
\item
\item
\item
\item
\item
\item
\end{itemize}

Advertisement

\protect\hyperlink{after-top}{Continue reading the main story}

Supported by

\protect\hyperlink{after-sponsor}{Continue reading the main story}

\hypertarget{china-defends-expulsion-of-american-journalists-accusing-us-of-prejudice}{%
\section{China Defends Expulsion of American Journalists, Accusing U.S.
of
Prejudice}\label{china-defends-expulsion-of-american-journalists-accusing-us-of-prejudice}}

An official said the expulsions were needed to defend China's media
against American suppression. Chinese state media outlets criticized
American newspapers for coverage that they described as biased.

\includegraphics{https://static01.nyt.com/images/2020/03/18/world/18china-journalists-1/merlin_170660475_b0e4c2a8-4431-470a-83fa-37ffe065ead5-articleLarge.jpg?quality=75\&auto=webp\&disable=upscale}

\href{https://www.nytimes.com/by/alexandra-stevenson}{\includegraphics{https://static01.nyt.com/images/2018/02/20/multimedia/author-alexandra-stevenson/author-alexandra-stevenson-thumbLarge.jpg}}\href{https://www.nytimes.com/by/austin-ramzy}{\includegraphics{https://static01.nyt.com/images/2018/10/15/multimedia/author-austin-ramzy/author-austin-ramzy-thumbLarge.png}}

By \href{https://www.nytimes.com/by/alexandra-stevenson}{Alexandra
Stevenson} and \href{https://www.nytimes.com/by/austin-ramzy}{Austin
Ramzy}

\begin{itemize}
\item
  Published March 18, 2020Updated March 19, 2020
\item
  \begin{itemize}
  \item
  \item
  \item
  \item
  \item
  \item
  \end{itemize}
\end{itemize}

\href{https://cn.nytimes.com/china/20200319/china-expels-journalists/}{阅读简体中文版}\href{https://cn.nytimes.com/china/20200319/china-expels-journalists/zh-han}{閱讀繁體中文版}

HONG KONG --- An increasingly rancorous rivalry between the United
States and China entered a new phase on Wednesday as Beijing accused the
Trump administration of starting a diplomatic clash that led it to
\href{https://www.nytimes.com/2020/03/17/business/media/china-expels-american-journalists.html}{expel
almost all American journalists} from three newspapers.

The Chinese government cast its expulsion of the journalists from The
New York Times, The Wall Street Journal and The Washington Post as
necessary to defend Beijing against what it perceived as an ideological
campaign by the United States to impose its values on China. Around a
dozen reporters could be required to leave, in a move that Beijing said
was reciprocation for the United States' forcing out of about 60 Chinese
reporters, who worked for propaganda outlets, this month.

``The United States cannot proceed from ideological prejudice, use its
own standards and likes and dislikes to judge the media of other
countries, let alone suppress the Chinese media unreasonably,'' Geng
Shuang, a Foreign Ministry spokesman, said at a news conference in
Beijing on Wednesday.

Beijing has said that the expulsions were a response to the Trump
administration's
\href{https://www.nytimes.com/2020/03/02/world/asia/china-journalists-diplomats-expulsion.html}{decision
to limit the number} of Chinese citizens from five state-controlled
media outlets who could work in the United States to 100. On Wednesday,
the Chinese government indicated that it was prepared to take more
measures if needed.

``We urge the United States to immediately change its course, correct
mistakes, and stop political suppression and unreasonable restrictions
on Chinese media,'' Mr. Geng said. ``If the United States insists on
taking its own course, compounding mistakes, China will be forced to
take further countermeasures.''

\emph{{[}Analysis:}
\href{http://www.nytimes.com/2020/03/19/world/asia/coronavirus-china-united-states.html}{\emph{The
world feared China over coronavirus. Now the tables have
turned}}\emph{.{]}}

The expulsions, not seen to such an extent in recent history, point to
the governing Communist Party's growing resolve to strike back in all
aspects of what is quickly becoming a bare-knuckled competition with the
United States. Over the past year, tensions have escalated over issues
ranging from trade deficits to technological capacity and military
dominance, with bruising effect on American and Chinese companies,
business executives, and even university students and academics.

The dispute over media access underlines how this new era of great power
rivalry has extended into the marketplace of ideas. It not only signals
a more muscular approach to foreign policy in China, but also accords
with the party's tightening grip over information under Xi Jinping, the
country's authoritarian leader.

The expulsions ``will definitely have a big influence'' on relations
between the two countries, said Zhan Jiang, a retired journalism
professor at Beijing Foreign Studies University. ``We've never really
seen anything like this in the past 40 years. This shows the relations
between the two sides have fallen into a deadlock, with neither side
retreating.''

Under Mr. Xi, the news media has come under an increasingly tight grip
and foreign reporters who displease the authorities have been punished
with visa denials. In recent weeks, as the coronavirus spread through
China, the government has
\href{https://www.nytimes.com/2020/03/14/business/media/coronavirus-china-journalists.html}{cracked
down on domestic} and foreign reporting, muzzling medical professionals
and
\href{https://www.nytimes.com/2020/03/16/business/china-coronavirus-internet-police.html}{censoring
and removing reports} and commentaries online that have challenged the
official narrative.

On Tuesday, China went even further, requiring all American journalists
for the three newspapers whose credentials expire by the end of the year
to turn in their press cards within 10 days. It said they would not be
allowed to continue working as journalists in China.

In an unusual move, it said the Americans were also forbidden to work as
journalists in Macau or Hong Kong, two semiautonomous Chinese
territories that have traditionally had greater protections for press
freedom than the mainland.

The American news outlets criticized the Chinese government's decision.
On Wednesday, American officials were discussing what measures to take
in response, including further retaliatory actions against official
Chinese interests in the United States. Secretary of State Mike Pompeo
on Tuesday called the expulsions ``unfortunate'' and said he hoped China
would reconsider.

Mr. Pompeo maintained that there was a fundamental difference between
the expelled American reporters, who are employed by independent media
outlets, and the expelled Chinese journalists, who work for a state
propaganda machine.

In its official rhetoric, the Chinese government has cast its decision
as a matter of diplomacy. But its own comments and reports in state-run
news outlets indicated that Beijing, which often accuses the Western
media of bias, also takes issue with the three American news outlets'
reporting on China.

``We reject ideological bias against China, reject fake news made in the
name of press freedom, reject breaches of ethics in journalism,''
\href{https://twitter.com/SpokespersonCHN/status/1240067426684788736}{tweeted
Hua Chunying}, a Chinese Foreign Ministry spokeswoman.

An article from The Global Times, a stridently nationalistic tabloid
controlled by the party, criticized the The New York Times's coverage of
the
\href{https://www.nytimes.com/2020/03/07/world/asia/china-coronavirus-cost.html}{coronavirus
outbreak}, the
\href{https://www.nytimes.com/2019/12/07/world/asia/hong-kong-protests-us-chamber-commerce.html}{monthslong
antigovernment protests} in Hong Kong, and the Chinese authorities'
\href{https://www.nytimes.com/interactive/2019/11/16/world/asia/china-xinjiang-documents.html}{internment
of ethnic-minority Muslims} in far western Xinjiang. The paper's
\href{https://www.nytimes.com/2020/02/01/world/asia/china-coronavirus.html}{coverage
of the epidemic}, the article said, was ``aiming to attack China's
political system and smear China's efforts'' to contain the virus.

It said reports by The Times and other outlets about
\href{https://www.nytimes.com/2019/11/24/world/asia/leak-chinas-internment-camps.html}{the
government's policies in the Xinjiang region} had ``smeared and attacked
China without basis.''

\includegraphics{https://static01.nyt.com/images/2020/03/18/world/18china-journalists-2/merlin_170661780_f23523b8-198d-4f08-a9ca-6346476ec866-articleLarge.jpg?quality=75\&auto=webp\&disable=upscale}

China also said it was requiring the three outlets as well as Time
magazine and Voice of America to disclose details of their staff, assets
and operations in China.

The action would affect at least 13 American journalists, but the number
could be higher, the Foreign Correspondents' Club of China said in a
statement on Wednesday.

The organization said the expulsion ``diminishes us in number and in
spirit, though not in our commitment to vigorously cover China. There
are no winners in the use of journalists as diplomatic pawns by the
world's two pre-eminent economic powers.''

The Global Times echoed this in an editorial on Wednesday that accused
Washington of starting the tit-for-tat. ``As a Chinese media, we regret
that the conflict between China and the United States has escalated due
to political differences,'' it said. Both Chinese and American
journalists, it added, were ``implicated by political frictions between
China and the United States.''

The cycle of retaliation began when China
\href{https://www.nytimes.com/2020/02/19/business/media/china-wall-street-journal.html}{expelled
three Journal reporters} over a headline last month, ``China Is the Real
Sick Man of Asia,'' on an opinion column about the country's coronavirus
response efforts.

In recent days, The People's Daily, a Communist Party mouthpiece, and
China Daily, a state-run newspaper, posted messages that circulated
widely on China's Twitter-like forum Weibo targeting The Times for what
they called ``double standards'' in its tweets about the lockdown
imposed in China and in Italy to curb the spread of the virus.

Image

Outside The People's Daily and Global Times in Beijing last
year.~Credit...Giulia Marchi for The New York Times

Some questions remained unanswered, including how and whether the Hong
Kong government would take further steps to enforce Beijing's expulsion.
Hong Kong operates under a political formula known as ``one country, two
systems'' that promises the Chinese territory a high degree of autonomy,
including independent courts, a free news media and extensive
protections of civil liberties.

Many global news organizations use Hong Kong as headquarters for the
Asia region. Under the Basic Law, Hong Kong's mini-constitution, the
region has jurisdiction over immigration matters. If Hong Kong refused
to allow the journalists to work in the city, it would be seen by some
critics as the
\href{https://www.nytimes.com/2018/10/08/world/asia/victor-mallet-hong-kong-financial-times.html}{latest
sign of eroding freedoms in the territory}.

On Wednesday, Mr. Geng said the action taken by Beijing was diplomatic
in nature, and thus fell under the central government's authority, not
that of Hong Kong.

But some of Hong Kong's pro-democracy lawmakers rejected the argument
and threatened to request a judicial review if any of the American
reporters were turned away at Hong Kong's border.

``Carrie Lam doesn't have to follow exactly what Beijing says,'' said
Dennis Kwok, a pro-democracy lawmaker who represents Hong Kong's legal
sector, referring to Hong Kong's leader. ``If she has any integrity
left, she can say this is Hong Kong and we have freedom of the press.''

Claudia Mo, another lawmaker, said Beijing was using the situation as an
opportunity to ``shut down the free flow of information.''

``Rule of law is quite dead in Hong Kong, we knew,'' Ms. Mo said at a
news conference in Hong Kong.

``Free flow of information, they're telling us, forget about it.''

Edward Wong contributed reporting from Washington, and Tiffany May from
Hong Kong. Claire Fu contributed research from Beijing.

Advertisement

\protect\hyperlink{after-bottom}{Continue reading the main story}

\hypertarget{site-index}{%
\subsection{Site Index}\label{site-index}}

\hypertarget{site-information-navigation}{%
\subsection{Site Information
Navigation}\label{site-information-navigation}}

\begin{itemize}
\tightlist
\item
  \href{https://help.nytimes.com/hc/en-us/articles/115014792127-Copyright-notice}{©~2020~The
  New York Times Company}
\end{itemize}

\begin{itemize}
\tightlist
\item
  \href{https://www.nytco.com/}{NYTCo}
\item
  \href{https://help.nytimes.com/hc/en-us/articles/115015385887-Contact-Us}{Contact
  Us}
\item
  \href{https://www.nytco.com/careers/}{Work with us}
\item
  \href{https://nytmediakit.com/}{Advertise}
\item
  \href{http://www.tbrandstudio.com/}{T Brand Studio}
\item
  \href{https://www.nytimes.com/privacy/cookie-policy\#how-do-i-manage-trackers}{Your
  Ad Choices}
\item
  \href{https://www.nytimes.com/privacy}{Privacy}
\item
  \href{https://help.nytimes.com/hc/en-us/articles/115014893428-Terms-of-service}{Terms
  of Service}
\item
  \href{https://help.nytimes.com/hc/en-us/articles/115014893968-Terms-of-sale}{Terms
  of Sale}
\item
  \href{https://spiderbites.nytimes.com}{Site Map}
\item
  \href{https://help.nytimes.com/hc/en-us}{Help}
\item
  \href{https://www.nytimes.com/subscription?campaignId=37WXW}{Subscriptions}
\end{itemize}
