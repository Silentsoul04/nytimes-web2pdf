Sections

SEARCH

\protect\hyperlink{site-content}{Skip to
content}\protect\hyperlink{site-index}{Skip to site index}

\href{https://www.nytimes.com/section/politics}{Politics}

\href{https://myaccount.nytimes.com/auth/login?response_type=cookie\&client_id=vi}{}

\href{https://www.nytimes.com/section/todayspaper}{Today's Paper}

\href{/section/politics}{Politics}\textbar{}Trump Resists Pressure to
Use Wartime Law to Mobilize Industry in Virus Response

\url{https://nyti.ms/2U7IJZ9}

\begin{itemize}
\item
\item
\item
\item
\item
\item
\end{itemize}

\href{https://www.nytimes.com/news-event/coronavirus?action=click\&pgtype=Article\&state=default\&region=TOP_BANNER\&context=storylines_menu}{The
Coronavirus Outbreak}

\begin{itemize}
\tightlist
\item
  live\href{https://www.nytimes.com/2020/08/03/world/coronavirus-covid-19.html?action=click\&pgtype=Article\&state=default\&region=TOP_BANNER\&context=storylines_menu}{Latest
  Updates}
\item
  \href{https://www.nytimes.com/interactive/2020/us/coronavirus-us-cases.html?action=click\&pgtype=Article\&state=default\&region=TOP_BANNER\&context=storylines_menu}{Maps
  and Cases}
\item
  \href{https://www.nytimes.com/interactive/2020/science/coronavirus-vaccine-tracker.html?action=click\&pgtype=Article\&state=default\&region=TOP_BANNER\&context=storylines_menu}{Vaccine
  Tracker}
\item
  \href{https://www.nytimes.com/2020/08/02/us/covid-college-reopening.html?action=click\&pgtype=Article\&state=default\&region=TOP_BANNER\&context=storylines_menu}{College
  Reopening}
\item
  \href{https://www.nytimes.com/live/2020/08/03/business/stock-market-today-coronavirus?action=click\&pgtype=Article\&state=default\&region=TOP_BANNER\&context=storylines_menu}{Economy}
\end{itemize}

Advertisement

\protect\hyperlink{after-top}{Continue reading the main story}

Supported by

\protect\hyperlink{after-sponsor}{Continue reading the main story}

\hypertarget{trump-resists-pressure-to-use-wartime-law-to-mobilize-industry-in-virus-response}{%
\section{Trump Resists Pressure to Use Wartime Law to Mobilize Industry
in Virus
Response}\label{trump-resists-pressure-to-use-wartime-law-to-mobilize-industry-in-virus-response}}

The president insisted he has used the Defense Production Act, but said
at a briefing that ``we are literally being besieged'' by companies
``that want to do the work and help our country.''

\includegraphics{https://static01.nyt.com/images/2020/03/20/us/politics/20dc-trump/20dc-trump-articleLarge-v2.jpg?quality=75\&auto=webp\&disable=upscale}

\href{https://www.nytimes.com/by/katie-rogers}{\includegraphics{https://static01.nyt.com/images/2018/06/12/multimedia/author-katie-rogers/author-katie-rogers-thumbLarge-v2.png}}\href{https://www.nytimes.com/by/maggie-haberman}{\includegraphics{https://static01.nyt.com/images/2018/07/12/multimedia/author-maggie-haberman/author-maggie-haberman-thumbLarge.png}}\href{https://www.nytimes.com/by/ana-swanson}{\includegraphics{https://static01.nyt.com/images/2018/12/10/multimedia/author-ana-swanson/author-ana-swanson-thumbLarge.png}}

By \href{https://www.nytimes.com/by/katie-rogers}{Katie Rogers},
\href{https://www.nytimes.com/by/maggie-haberman}{Maggie Haberman} and
\href{https://www.nytimes.com/by/ana-swanson}{Ana Swanson}

\begin{itemize}
\item
  March 20, 2020
\item
  \begin{itemize}
  \item
  \item
  \item
  \item
  \item
  \item
  \end{itemize}
\end{itemize}

WASHINGTON --- President Trump and his advisers have resisted calls from
congressional Democrats and a growing number of governors to use a
federal law that would mobilize industry and provide badly needed
resources against the
\href{https://www.nytimes.com/2020/03/20/world/coronavirus-news.html}{coronavirus}
spread, days after the president said he would consider using that
authority.

Mr. Trump has given conflicting signals about the Defense Production Act
since he first said on Wednesday that he was prepared to invoke the law,
which was passed by Congress at the outset of the Korean War and grants
presidents extraordinary powers to force American industries to ensure
the availability of critical equipment.

The next day, he suggested that obtaining medical equipment should be up
to individual governors because ``we're not a shipping clerk.'' But on
Friday, he reversed himself, asserting that he had used the law to spur
the production of ``millions of masks,'' without offering evidence or
specifics about who was manufacturing them or when they would reach
health workers.

And Senator Chuck Schumer of New York, the Democratic leader, said that
he was left with the impression after talking with Mr. Trump that he had
decided to move to put the act into effect. He said ``a commitment on
the phone was a good start,'' but that the president now needed to push
the government ``to move full steam ahead.''

But Mr. Trump's confusing statements played out in the middle of a
growing health crisis that within days has abruptly and indefinitely
altered the course of American life.

With the
\href{https://www.nytimes.com/interactive/2020/world/coronavirus-maps.html\#us}{number
of coronavirus cases in the United States surging above 17,000} --- over
40 percent of those concentrated in New York --- front-line health care
workers have reported a dire shortage of masks, surgical gowns and eye
gear to protect them from the virus. State lawmakers have also implored
the president to help them get the supplies they need.

Business leaders have said invoking the defense law is not necessary.
During his appearance with the members of his coronavirus task force on
Friday, Mr. Trump supported that idea and said that private companies,
including General Motors, had volunteered to produce supplies without
any prompting from the government.

``We are literally being besieged in a beautiful way by companies that
want to do the work and help our country,'' Mr. Trump said. ``We have
not had a problem with that at all.''

Some of the president's advisers have privately said that they share the
longstanding opposition of conservatives to government intervention and
oppose using the law, and the president again signaled his own
ambivalence about it.

\hypertarget{latest-updates-global-coronavirus-outbreak}{%
\section{\texorpdfstring{\href{https://www.nytimes.com/2020/08/03/world/coronavirus-covid-19.html?action=click\&pgtype=Article\&state=default\&region=MAIN_CONTENT_1\&context=storylines_live_updates}{Latest
Updates: Global Coronavirus
Outbreak}}{Latest Updates: Global Coronavirus Outbreak}}\label{latest-updates-global-coronavirus-outbreak}}

Updated 2020-08-04T01:26:47.893Z

\begin{itemize}
\tightlist
\item
  \href{https://www.nytimes.com/2020/08/03/world/coronavirus-covid-19.html?action=click\&pgtype=Article\&state=default\&region=MAIN_CONTENT_1\&context=storylines_live_updates\#link-4e40df05}{Birx,
  criticized by Trump, is defended by Fauci on the pandemic's spread.}
\item
  \href{https://www.nytimes.com/2020/08/03/world/coronavirus-covid-19.html?action=click\&pgtype=Article\&state=default\&region=MAIN_CONTENT_1\&context=storylines_live_updates\#link-15e7f995}{Trump
  derides Democrats as lawmakers and administration officials try to
  break stimulus impasse.}
\item
  \href{https://www.nytimes.com/2020/08/03/world/coronavirus-covid-19.html?action=click\&pgtype=Article\&state=default\&region=MAIN_CONTENT_1\&context=storylines_live_updates\#link-4c85ed64}{As
  some students and teachers go back to school in the U.S., they're
  bringing the virus with them.}
\end{itemize}

\href{https://www.nytimes.com/2020/08/03/world/coronavirus-covid-19.html?action=click\&pgtype=Article\&state=default\&region=MAIN_CONTENT_1\&context=storylines_live_updates}{See
more updates}

More live coverage:
\href{https://www.nytimes.com/live/2020/08/03/business/stock-market-today-coronavirus?action=click\&pgtype=Article\&state=default\&region=MAIN_CONTENT_1\&context=storylines_live_updates}{Markets}

``When we need something, we'll order something,'' Mr. Trump said of the
act. ``As you know two days ago, I invoked the act. It is a big step. I
am not sure if it is done before. When we need something, we'll use
it.''

Asked Friday night about specific ways the Defense Production Act has
been used, the White House said in a statement that the president was
``currently using it to drive the private sector's response to this
crisis,'' and that he had invoked the act ``to ensure that the necessary
authorities will be available to prioritize production of items under
government contracts and to allocate scarce items where they are needed
most.''

In signing the executive order on Wednesday to put the act into effect,
Mr. Trump said the purpose was to expedite distribution of ``health and
medical resources needed to respond to the spread of Covid-19, including
personal protective equipment and ventilators,'' and that Alex M. Azar
II, the secretary of health and human services, could order production
and distribution of supplies, if necessary.

But the president did not say if masks and ventilators in anything near
the necessary quantity have actually been delivered to the workers who
need them.

During Friday's briefing, Mr. Trump grew increasingly confrontational
with reporters as they pressed him on the details of the Defense
Production Act, and he snapped at Peter Alexander, a reporter for NBC
News, who asked him what he would say to Americans who were scared.

``I say that you're a terrible reporter, that's what I say,'' Mr. Trump
replied.

``It is a bad signal that you are putting out to the American people,''
the president continued. ``You want to get back to reporting instead of
sensationalism. Let's see if it works. I happen to feel good about it.
Who knows. I have been right a lot. Let's see what happens.''

As the coronavirus has spread, Mr. Trump has come under withering attack
from Democrats for the speed at which he has mobilized the government to
respond.

``We're talking about a president who is basically doing what Herbert
Hoover did at the beginning of the Depression and minimizing the danger
and refusing to use available federal action,'' Mayor Bill de Blasio of
New York said Friday in an interview with the radio station WNYC. ``And
people are going to die, and they shouldn't, they don't have to, if we
could get the support that we're asking for.''

Republicans have not been openly critical, but some governors have been
explicit in describing their difficulties in depending on the private
sector for medical supplies.

In a call held on Thursday with Mr. Trump, a group of governors stressed
to him that they were struggling to address the staggering demand for
equipment.

At one point, Gov. Kristi Noem, Republican of South Dakota, grew
frustrated as she expressed to the president and members of the
coronavirus task force that state officials had been working
unsuccessfully with private suppliers.

\href{https://www.nytimes.com/news-event/coronavirus?action=click\&pgtype=Article\&state=default\&region=MAIN_CONTENT_3\&context=storylines_faq}{}

\hypertarget{the-coronavirus-outbreak-}{%
\subsubsection{The Coronavirus Outbreak
›}\label{the-coronavirus-outbreak-}}

\hypertarget{frequently-asked-questions}{%
\paragraph{Frequently Asked
Questions}\label{frequently-asked-questions}}

Updated August 3, 2020

\begin{itemize}
\item ~
  \hypertarget{im-a-small-business-owner-can-i-get-relief}{%
  \paragraph{I'm a small-business owner. Can I get
  relief?}\label{im-a-small-business-owner-can-i-get-relief}}

  \begin{itemize}
  \tightlist
  \item
    The
    \href{https://www.nytimes.com/article/small-business-loans-stimulus-grants-freelancers-coronavirus.html?action=click\&pgtype=Article\&state=default\&region=MAIN_CONTENT_3\&context=storylines_faq}{stimulus
    bills enacted in March} offer help for the millions of American
    small businesses. Those eligible for aid are businesses and
    nonprofit organizations with fewer than 500 workers, including sole
    proprietorships, independent contractors and freelancers. Some
    larger companies in some industries are also eligible. The help
    being offered, which is being managed by the Small Business
    Administration, includes the Paycheck Protection Program and the
    Economic Injury Disaster Loan program. But lots of folks have
    \href{https://www.nytimes.com/interactive/2020/05/07/business/small-business-loans-coronavirus.html?action=click\&pgtype=Article\&state=default\&region=MAIN_CONTENT_3\&context=storylines_faq}{not
    yet seen payouts.} Even those who have received help are confused:
    The rules are draconian, and some are stuck sitting on
    \href{https://www.nytimes.com/2020/05/02/business/economy/loans-coronavirus-small-business.html?action=click\&pgtype=Article\&state=default\&region=MAIN_CONTENT_3\&context=storylines_faq}{money
    they don't know how to use.} Many small-business owners are getting
    less than they expected or
    \href{https://www.nytimes.com/2020/06/10/business/Small-business-loans-ppp.html?action=click\&pgtype=Article\&state=default\&region=MAIN_CONTENT_3\&context=storylines_faq}{not
    hearing anything at all.}
  \end{itemize}
\item ~
  \hypertarget{what-are-my-rights-if-i-am-worried-about-going-back-to-work}{%
  \paragraph{What are my rights if I am worried about going back to
  work?}\label{what-are-my-rights-if-i-am-worried-about-going-back-to-work}}

  \begin{itemize}
  \tightlist
  \item
    Employers have to provide
    \href{https://www.osha.gov/SLTC/covid-19/standards.html}{a safe
    workplace} with policies that protect everyone equally.
    \href{https://www.nytimes.com/article/coronavirus-money-unemployment.html?action=click\&pgtype=Article\&state=default\&region=MAIN_CONTENT_3\&context=storylines_faq}{And
    if one of your co-workers tests positive for the coronavirus, the
    C.D.C.} has said that
    \href{https://www.cdc.gov/coronavirus/2019-ncov/community/guidance-business-response.html}{employers
    should tell their employees} -\/- without giving you the sick
    employee's name -\/- that they may have been exposed to the virus.
  \end{itemize}
\item ~
  \hypertarget{should-i-refinance-my-mortgage}{%
  \paragraph{Should I refinance my
  mortgage?}\label{should-i-refinance-my-mortgage}}

  \begin{itemize}
  \tightlist
  \item
    \href{https://www.nytimes.com/article/coronavirus-money-unemployment.html?action=click\&pgtype=Article\&state=default\&region=MAIN_CONTENT_3\&context=storylines_faq}{It
    could be a good idea,} because mortgage rates have
    \href{https://www.nytimes.com/2020/07/16/business/mortgage-rates-below-3-percent.html?action=click\&pgtype=Article\&state=default\&region=MAIN_CONTENT_3\&context=storylines_faq}{never
    been lower.} Refinancing requests have pushed mortgage applications
    to some of the highest levels since 2008, so be prepared to get in
    line. But defaults are also up, so if you're thinking about buying a
    home, be aware that some lenders have tightened their standards.
  \end{itemize}
\item ~
  \hypertarget{what-is-school-going-to-look-like-in-september}{%
  \paragraph{What is school going to look like in
  September?}\label{what-is-school-going-to-look-like-in-september}}

  \begin{itemize}
  \tightlist
  \item
    It is unlikely that many schools will return to a normal schedule
    this fall, requiring the grind of
    \href{https://www.nytimes.com/2020/06/05/us/coronavirus-education-lost-learning.html?action=click\&pgtype=Article\&state=default\&region=MAIN_CONTENT_3\&context=storylines_faq}{online
    learning},
    \href{https://www.nytimes.com/2020/05/29/us/coronavirus-child-care-centers.html?action=click\&pgtype=Article\&state=default\&region=MAIN_CONTENT_3\&context=storylines_faq}{makeshift
    child care} and
    \href{https://www.nytimes.com/2020/06/03/business/economy/coronavirus-working-women.html?action=click\&pgtype=Article\&state=default\&region=MAIN_CONTENT_3\&context=storylines_faq}{stunted
    workdays} to continue. California's two largest public school
    districts --- Los Angeles and San Diego --- said on July 13, that
    \href{https://www.nytimes.com/2020/07/13/us/lausd-san-diego-school-reopening.html?action=click\&pgtype=Article\&state=default\&region=MAIN_CONTENT_3\&context=storylines_faq}{instruction
    will be remote-only in the fall}, citing concerns that surging
    coronavirus infections in their areas pose too dire a risk for
    students and teachers. Together, the two districts enroll some
    825,000 students. They are the largest in the country so far to
    abandon plans for even a partial physical return to classrooms when
    they reopen in August. For other districts, the solution won't be an
    all-or-nothing approach.
    \href{https://bioethics.jhu.edu/research-and-outreach/projects/eschool-initiative/school-policy-tracker/}{Many
    systems}, including the nation's largest, New York City, are
    devising
    \href{https://www.nytimes.com/2020/06/26/us/coronavirus-schools-reopen-fall.html?action=click\&pgtype=Article\&state=default\&region=MAIN_CONTENT_3\&context=storylines_faq}{hybrid
    plans} that involve spending some days in classrooms and other days
    online. There's no national policy on this yet, so check with your
    municipal school system regularly to see what is happening in your
    community.
  \end{itemize}
\item ~
  \hypertarget{is-the-coronavirus-airborne}{%
  \paragraph{Is the coronavirus
  airborne?}\label{is-the-coronavirus-airborne}}

  \begin{itemize}
  \tightlist
  \item
    The coronavirus
    \href{https://www.nytimes.com/2020/07/04/health/239-experts-with-one-big-claim-the-coronavirus-is-airborne.html?action=click\&pgtype=Article\&state=default\&region=MAIN_CONTENT_3\&context=storylines_faq}{can
    stay aloft for hours in tiny droplets in stagnant air}, infecting
    people as they inhale, mounting scientific evidence suggests. This
    risk is highest in crowded indoor spaces with poor ventilation, and
    may help explain super-spreading events reported in meatpacking
    plants, churches and restaurants.
    \href{https://www.nytimes.com/2020/07/06/health/coronavirus-airborne-aerosols.html?action=click\&pgtype=Article\&state=default\&region=MAIN_CONTENT_3\&context=storylines_faq}{It's
    unclear how often the virus is spread} via these tiny droplets, or
    aerosols, compared with larger droplets that are expelled when a
    sick person coughs or sneezes, or transmitted through contact with
    contaminated surfaces, said Linsey Marr, an aerosol expert at
    Virginia Tech. Aerosols are released even when a person without
    symptoms exhales, talks or sings, according to Dr. Marr and more
    than 200 other experts, who
    \href{https://academic.oup.com/cid/article/doi/10.1093/cid/ciaa939/5867798}{have
    outlined the evidence in an open letter to the World Health
    Organization}.
  \end{itemize}
\end{itemize}

``I need to understand how you're triaging supplies,'' Ms. Noem said.
``We, for two weeks, were requesting reagents for our public health lab
from C.D.C., who pushed us to private suppliers who kept canceling
orders on us. And we kept making requests, placing orders.''

She added, ``I don't want to be less of a priority because we're a
smaller state or less populated.''

Mr. Trump promised her that would ``never'' happen before Ms. Noem's
telephone line was disconnected.

When it was originally passed, the Defense Production Act granted
President Harry S. Truman the power to spur the production of aluminum,
titanium and other needed materials during wartime. Since then, it has
been used for both the prevention of terrorism and to prepare for
natural disasters.

The act would give the Trump administration the authority to override
companies' existing contracts and to direct supplies to hot spots like
New York City or Seattle. It could also help mobilize funds for
retooling factories, refitting pharmacy drive-throughs into testing
sites and ramping up production of an eventual vaccine.

But John Murphy, the senior vice president for international policy at
the U.S. Chamber of Commerce, said that corporate executives were
already working ``hand in glove'' with the government on production
challenges.

``American companies will do whatever it takes to support our country's
response to the pandemic and shore up the economy,'' he said. ``The
Defense Production Act was designed for defense industry products with a
single supplier, often with purely domestic production chains, and
invoking it may do more harm than good in sectors such as
pharmaceuticals and medical equipment.''

Companies that manufacture face masks, medical wipes and other supplies
say they are already operating around the clock to meet elevated demand.
Some factories that make similar products --- like surgical gowns,
diapers and incontinence products --- have already switched over to
manufacture the face masks or other protective gear that health care
workers need.

But people familiar with the administration's actions say it is still
trying to figure out how industry supply chains operate, which companies
could produce additional products and what kinds of subsidies it may
need to offer.

And without the Defense Production Act, the government will lack the
ability to channel these supplies to areas that need it most --- or to
persuade companies to act quickly and without regard for their profits.

As reported cases of the virus in the United States have soared, Mr.
Trump, who is known to recruit input from a variety of outside advisers,
has been getting conflicting advice. The proliferating number of private
sector voices with direct access to the president and his top advisers
--- notably his son-in-law and adviser, Jared Kushner --- has resulted
in a chaotic process.

The president's advisers say they see the role of the federal government
as a facilitator, as opposed to the chief producer or a national
governor. They have tried to encourage states to get by with what they
can, suggesting there will be support from the federal government but
that this should not be the first option.

In practice, the administration has been trying to use the provision to
jawbone companies into taking voluntary action while holding over them
the possibility that the federal government would intervene, according
to administration officials familiar with the state of play.

``We're actually encouraged that the partnership with the private sector
can meeting many of these needs,'' said Marc Short, Vice President Mike
Pence's chief of staff, on Friday morning in a discussion with
reporters.

Katie Rogers and Ana Swanson reported from Washington, and Maggie
Haberman from New York. Michael Gold contributed reporting from New
York, and Noah Weiland from Washington.

Advertisement

\protect\hyperlink{after-bottom}{Continue reading the main story}

\hypertarget{site-index}{%
\subsection{Site Index}\label{site-index}}

\hypertarget{site-information-navigation}{%
\subsection{Site Information
Navigation}\label{site-information-navigation}}

\begin{itemize}
\tightlist
\item
  \href{https://help.nytimes.com/hc/en-us/articles/115014792127-Copyright-notice}{©~2020~The
  New York Times Company}
\end{itemize}

\begin{itemize}
\tightlist
\item
  \href{https://www.nytco.com/}{NYTCo}
\item
  \href{https://help.nytimes.com/hc/en-us/articles/115015385887-Contact-Us}{Contact
  Us}
\item
  \href{https://www.nytco.com/careers/}{Work with us}
\item
  \href{https://nytmediakit.com/}{Advertise}
\item
  \href{http://www.tbrandstudio.com/}{T Brand Studio}
\item
  \href{https://www.nytimes.com/privacy/cookie-policy\#how-do-i-manage-trackers}{Your
  Ad Choices}
\item
  \href{https://www.nytimes.com/privacy}{Privacy}
\item
  \href{https://help.nytimes.com/hc/en-us/articles/115014893428-Terms-of-service}{Terms
  of Service}
\item
  \href{https://help.nytimes.com/hc/en-us/articles/115014893968-Terms-of-sale}{Terms
  of Sale}
\item
  \href{https://spiderbites.nytimes.com}{Site Map}
\item
  \href{https://help.nytimes.com/hc/en-us}{Help}
\item
  \href{https://www.nytimes.com/subscription?campaignId=37WXW}{Subscriptions}
\end{itemize}
