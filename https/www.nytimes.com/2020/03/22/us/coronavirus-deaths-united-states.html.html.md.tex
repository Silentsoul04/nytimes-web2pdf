Sections

SEARCH

\protect\hyperlink{site-content}{Skip to
content}\protect\hyperlink{site-index}{Skip to site index}

\href{https://www.nytimes.com/section/us}{U.S.}

\href{https://myaccount.nytimes.com/auth/login?response_type=cookie\&client_id=vi}{}

\href{https://www.nytimes.com/section/todayspaper}{Today's Paper}

\href{/section/us}{U.S.}\textbar{}She Was More Than a Statistic in a
Pandemic: `We Didn't Want Her to Get Lost'

\url{https://nyti.ms/3bfP2zq}

\begin{itemize}
\item
\item
\item
\item
\item
\end{itemize}

\href{https://www.nytimes.com/news-event/coronavirus?action=click\&pgtype=Article\&state=default\&region=TOP_BANNER\&context=storylines_menu}{The
Coronavirus Outbreak}

\begin{itemize}
\tightlist
\item
  live\href{https://www.nytimes.com/2020/08/03/world/coronavirus-covid-19.html?action=click\&pgtype=Article\&state=default\&region=TOP_BANNER\&context=storylines_menu}{Latest
  Updates}
\item
  \href{https://www.nytimes.com/interactive/2020/us/coronavirus-us-cases.html?action=click\&pgtype=Article\&state=default\&region=TOP_BANNER\&context=storylines_menu}{Maps
  and Cases}
\item
  \href{https://www.nytimes.com/interactive/2020/science/coronavirus-vaccine-tracker.html?action=click\&pgtype=Article\&state=default\&region=TOP_BANNER\&context=storylines_menu}{Vaccine
  Tracker}
\item
  \href{https://www.nytimes.com/2020/08/02/us/covid-college-reopening.html?action=click\&pgtype=Article\&state=default\&region=TOP_BANNER\&context=storylines_menu}{College
  Reopening}
\item
  \href{https://www.nytimes.com/live/2020/08/03/business/stock-market-today-coronavirus?action=click\&pgtype=Article\&state=default\&region=TOP_BANNER\&context=storylines_menu}{Economy}
\end{itemize}

Advertisement

\protect\hyperlink{after-top}{Continue reading the main story}

Supported by

\protect\hyperlink{after-sponsor}{Continue reading the main story}

THOSE WE'VE LOST

\hypertarget{she-was-more-than-a-statistic-in-a-pandemic-we-didnt-want-her-to-get-lost}{%
\section{She Was More Than a Statistic in a Pandemic: `We Didn't Want
Her to Get
Lost'}\label{she-was-more-than-a-statistic-in-a-pandemic-we-didnt-want-her-to-get-lost}}

Loretta Mendoza Dionisio, outgoing and unstoppable, died of the
coronavirus, a statistic in a growing U.S. count. But her life meant
much more.

\includegraphics{https://static01.nyt.com/images/2020/03/20/us/00virus-death01/00virus-death01-articleLarge.jpg?quality=75\&auto=webp\&disable=upscale}

\href{https://www.nytimes.com/by/ellen-barry}{\includegraphics{https://static01.nyt.com/images/2018/10/08/multimedia/author-ellen-barry/author-ellen-barry-thumbLarge.png}}

By \href{https://www.nytimes.com/by/ellen-barry}{Ellen Barry}

\begin{itemize}
\item
  Published March 22, 2020Updated April 16, 2020
\item
  \begin{itemize}
  \item
  \item
  \item
  \item
  \item
  \end{itemize}
\end{itemize}

\emph{This obituary is part of a series about}
\href{https://www.nytimes.com/series/people-who-have-died-of-the-coronavirus}{\emph{people
who have died in the coronavirus pandemic}}\emph{.}

Her name was Loretta, but they called her Lettie. She stood 4 feet 10
inches tall. She was outrageously friendly, the kind of person liable to
invite the sales clerk at T-Mobile to join the family for dinner. This
made her children cringe but was also something they loved. Pure Lettie.

She was tough. At work, she could stare down colleagues who were hairy,
blustery and taller than her by a foot or two. And it was true of her
husband, Roddy. He could not say no to her.

Roddy had not wanted to go on their February trip to the Philippines. He
was watching the early news about the coronavirus, and worried it would
put his wife, a cancer survivor, in danger. But she was adamant. There
was something she needed to finish.

On March 11, Loretta Dionisio became a data point.

At the
\href{https://www.dailybreeze.com/2020/03/11/la-county-reports-first-death-related-to-coronavirus-6-new-cases/}{news
conference where her death was announced}, the public health director in
Los Angeles County did not name her, in accordance with federal privacy
regulations.

The public health director referred only to a woman in her 60s with
``underlying health conditions'' who was stopping briefly in California
after travels in Asia, adding that ``shortly after being hospitalized,
she unfortunately passed.'' In the ongoing tally of fatalities
associated with the coronavirus, hers was the 37th death in the United
States, the first in Los Angeles County.

\emph{{[}Sign up}
\href{https://www.nytimes.com/newsletters/california-today}{\emph{for
California Today}}\emph{, our newsletter about California, for
updates.{]}}

Nearly two weeks later, Ms. Dionisio's family was still grappling with
the bureaucracy that surrounds infectious disease. She died far from her
home in Orlando, Fla., during a layover 2,500 miles away. Her son and
daughter, on the East Coast, have been unable to see their father, who
is in quarantine in California after giving their mother cardiopulmonary
resuscitation. For days after her death, he barely spoke.

And in the painful logistics of hygiene and quarantine, no funeral Mass
has been said for her.

``Through this whole ordeal, we didn't want her to get lost in the
story,'' said her son, Rembert Dionisio.

Janice Jenkins, a close friend of Ms. Dionisio's, said that the days
after her death had felt strange and disjointed, without the ceremonies
that mark the passing of someone dear.

``This whole thing is just like a hole in the ground that they're just
throwing bodies into,'' she said.

\hypertarget{a-rush-of-numbers}{%
\subsection{A rush of numbers}\label{a-rush-of-numbers}}

News of the pandemic is released in the form of data, illnesses and
deaths compiled by countries and counties. But sparks of humanity glow
here and there.

\hypertarget{latest-updates-global-coronavirus-outbreak}{%
\section{\texorpdfstring{\href{https://www.nytimes.com/2020/08/03/world/coronavirus-covid-19.html?action=click\&pgtype=Article\&state=default\&region=MAIN_CONTENT_1\&context=storylines_live_updates}{Latest
Updates: Global Coronavirus
Outbreak}}{Latest Updates: Global Coronavirus Outbreak}}\label{latest-updates-global-coronavirus-outbreak}}

Updated 2020-08-04T05:55:16.339Z

\begin{itemize}
\tightlist
\item
  \href{https://www.nytimes.com/2020/08/03/world/coronavirus-covid-19.html?action=click\&pgtype=Article\&state=default\&region=MAIN_CONTENT_1\&context=storylines_live_updates\#link-4547638f}{Fauci
  defends Birx after she is criticized by Trump.}
\item
  \href{https://www.nytimes.com/2020/08/03/world/coronavirus-covid-19.html?action=click\&pgtype=Article\&state=default\&region=MAIN_CONTENT_1\&context=storylines_live_updates\#link-15e7f995}{Trump
  derides Democrats as lawmakers and administration officials try to
  break stimulus impasse.}
\item
  \href{https://www.nytimes.com/2020/08/03/world/coronavirus-covid-19.html?action=click\&pgtype=Article\&state=default\&region=MAIN_CONTENT_1\&context=storylines_live_updates\#link-e5a2cda}{The
  deadline for 2020 census counting has been moved up by a month.}
\end{itemize}

\href{https://www.nytimes.com/2020/08/03/world/coronavirus-covid-19.html?action=click\&pgtype=Article\&state=default\&region=MAIN_CONTENT_1\&context=storylines_live_updates}{See
more updates}

More live coverage:
\href{https://www.nytimes.com/live/2020/08/03/business/stock-market-today-coronavirus?action=click\&pgtype=Article\&state=default\&region=MAIN_CONTENT_1\&context=storylines_live_updates}{Markets}

Consider John Brennan, a New Jersey man whose death was announced March
10. He had once
\href{https://www.northjersey.com/story/news/2020/03/10/nj-horse-trainer-john-brennan-dies-coronavirus/5015168002/}{trained
a winning racehorse named Sugar Trader}. ``I'm a minor leaguer, and I'm
in the big leagues,'' he said at the time. ``Unbelievable.''

Merle Dry, 55, who died on Wednesday in a hospital in Tulsa, Okla.,
trimmed the hedges at Oral Roberts University into
\href{http://oruoracle.com/lifestyle/behind-the-scenery/}{topiary birds
and curlicues.}

Jeffrey Ghazarian, 34, a cancer survivor who died on Thursday at a
hospital in Pasadena, Calif.,
\href{https://www.facebook.com/search/top/?q=Jeffrey\%20Ghazarian\&epa=SEARCH_BOX}{liked
to quote the movie ``Swingers,''}the speech that went: ``You're money,
baby. You're so money and you don't even know it.''

Gary Young, 66, a retired cabinet maker who died in Gilroy, Calif., on
Tuesday,
\href{https://www.mercurynews.com/2020/03/19/coronavirus-gilroy-family-forbidden-from-dying-dads-bedside-broke-my-heart-into-a-million-pieces/}{was
a talker,} sometimes lingering for half an hour with goodbyes as his
family waited in the car.

His daughter
\href{https://www.mercurynews.com/2020/03/19/coronavirus-gilroy-family-forbidden-from-dying-dads-bedside-broke-my-heart-into-a-million-pieces/}{told
The San Jose Mercury News} that she watched through a glass divider as
he died in an isolation ward, and a medical team in blue protective gear
turned off his heart monitor.

``It broke my heart into a million pieces,'' she said. ``I didn't want
him to feel alone.''

The list goes on. As of Sunday, 390 deaths had been tied to the
coronavirus in the United States. The average age of those who had died
was a few months over 77, according to the Centers for Disease Control
and Prevention. In the most vulnerable age brackets,
\href{https://www.nytimes.com/2020/02/20/health/coronavirus-men-women.html}{men
are nearly twice as likely to die} as women.

Because people over 80 are far more likely to die, the
\href{https://www.medrxiv.org/content/10.1101/2020.03.15.20036293v1}{deaths
may ultimately be clustered} in wealthier countries with higher life
expectancies, according to a study published last week by demographers
at Oxford University. The virus spreads faster in countries like Italy,
where there is a high level of contact between the young and the old.

\includegraphics{https://static01.nyt.com/images/2020/03/23/us/23virus-death-print/merlin_170575302_6a115112-fbfb-4ab8-aced-6c62eac1742a-articleLarge.jpg?quality=75\&auto=webp\&disable=upscale}

As the crisis has deepened, mourning rituals have fallen by the wayside.

In China, where more than 3,100 people have died, the national health
commission has banned funerals. Patients die in intensive care units
that do not allow visitors, and in the moments after a person's death,
health workers in hazmat suits enter a hospital room and take the body
away.

In Italy, where funerals serve as a central pillar of community life,
many of the dead are being buried by a lone priest, without mourners
present.
\href{https://www.nytimes.com/2020/03/16/world/europe/italy-coronavirus-funerals.html}{A
local cemetery in the province of Bergamo, at the center of the
outbreak, shut down this past week} for the first time since World War
II. The local newspaper, \href{https://www.ecodibergamo.it/}{L'Eco di
Bergamo}, ran 10 pages of obituaries.

``These are people who die alone and who are buried alone,'' the
newspaper's editor, Alberto Ceresoli, said.

\hypertarget{a-specific-person}{%
\subsection{A specific person}\label{a-specific-person}}

Ms. Dionisio, 68, was fond of emeralds (the real kind), serial killer
documentaries and the Home Shopping Network.

She had a passionate interest in food. Her brainstorming about lunch
plans, her co-workers would joke, sometimes began at 9:30 in the
morning. People not related to her called her ``mom.'' And she had such
a close, tender relationship with her husband that when he had to work
on Saturdays, she would drive an hour to work with him just so she could
sit by his side.

``I don't mind, I just want to be with him,'' she said, once, when a
co-worker asked her why.

It would be a mistake, however, to consider her a softy. When the family
needed someone to negotiate a deal, they sent Ms. Dionisio.

``She wasn't necessarily persuasive, she was just persistent,'' said her
son-in-law, Chris Connelly. ``She would say, `I want that car and I want
it for this price.' You would be there until 10 o'clock at night and the
manager would say, `What do I have to do to go home?'''

Loretta Mendoza was born in Pasay City, in the Philippines, to parents
who had fled by foot into the countryside to escape the Japanese
occupation during World War II, at times eating roots to survive. It was
an ordeal they rarely spoke of. In first grade, she was sent away to a
convent school, to be educated by nuns.

``She and I took care of each other,'' said her sister, Norma Quijano,
73, who is five years her senior. ``You know how the nuns are.''

Image

Loretta Dionisio, left, with her siblings, Barbara Poole, Norma Quijano
and Jesse Mendoza.

Ms. Dionisio's decision to travel to the Philippines in February was
bound up in that history.

She and her husband, Rodrigo, a classmate from art school in the
Philippines, left for the United States in the 1970s, and found work as
commercial artists. They eventually settled in Orlando and raised two
children, Rembert and Rowena.

But unfinished business from the Philippines nagged at Ms. Dionisio.

Her father, who had grown up in poverty, had spent years scrimping to
buy land for a coconut plantation in the coastal region of Camarines
Norte, promising his children it would support them in their old age.

\href{https://www.nytimes.com/news-event/coronavirus?action=click\&pgtype=Article\&state=default\&region=MAIN_CONTENT_3\&context=storylines_faq}{}

\hypertarget{the-coronavirus-outbreak-}{%
\subsubsection{The Coronavirus Outbreak
›}\label{the-coronavirus-outbreak-}}

\hypertarget{frequently-asked-questions}{%
\paragraph{Frequently Asked
Questions}\label{frequently-asked-questions}}

Updated August 3, 2020

\begin{itemize}
\item ~
  \hypertarget{im-a-small-business-owner-can-i-get-relief}{%
  \paragraph{I'm a small-business owner. Can I get
  relief?}\label{im-a-small-business-owner-can-i-get-relief}}

  \begin{itemize}
  \tightlist
  \item
    The
    \href{https://www.nytimes.com/article/small-business-loans-stimulus-grants-freelancers-coronavirus.html?action=click\&pgtype=Article\&state=default\&region=MAIN_CONTENT_3\&context=storylines_faq}{stimulus
    bills enacted in March} offer help for the millions of American
    small businesses. Those eligible for aid are businesses and
    nonprofit organizations with fewer than 500 workers, including sole
    proprietorships, independent contractors and freelancers. Some
    larger companies in some industries are also eligible. The help
    being offered, which is being managed by the Small Business
    Administration, includes the Paycheck Protection Program and the
    Economic Injury Disaster Loan program. But lots of folks have
    \href{https://www.nytimes.com/interactive/2020/05/07/business/small-business-loans-coronavirus.html?action=click\&pgtype=Article\&state=default\&region=MAIN_CONTENT_3\&context=storylines_faq}{not
    yet seen payouts.} Even those who have received help are confused:
    The rules are draconian, and some are stuck sitting on
    \href{https://www.nytimes.com/2020/05/02/business/economy/loans-coronavirus-small-business.html?action=click\&pgtype=Article\&state=default\&region=MAIN_CONTENT_3\&context=storylines_faq}{money
    they don't know how to use.} Many small-business owners are getting
    less than they expected or
    \href{https://www.nytimes.com/2020/06/10/business/Small-business-loans-ppp.html?action=click\&pgtype=Article\&state=default\&region=MAIN_CONTENT_3\&context=storylines_faq}{not
    hearing anything at all.}
  \end{itemize}
\item ~
  \hypertarget{what-are-my-rights-if-i-am-worried-about-going-back-to-work}{%
  \paragraph{What are my rights if I am worried about going back to
  work?}\label{what-are-my-rights-if-i-am-worried-about-going-back-to-work}}

  \begin{itemize}
  \tightlist
  \item
    Employers have to provide
    \href{https://www.osha.gov/SLTC/covid-19/standards.html}{a safe
    workplace} with policies that protect everyone equally.
    \href{https://www.nytimes.com/article/coronavirus-money-unemployment.html?action=click\&pgtype=Article\&state=default\&region=MAIN_CONTENT_3\&context=storylines_faq}{And
    if one of your co-workers tests positive for the coronavirus, the
    C.D.C.} has said that
    \href{https://www.cdc.gov/coronavirus/2019-ncov/community/guidance-business-response.html}{employers
    should tell their employees} -\/- without giving you the sick
    employee's name -\/- that they may have been exposed to the virus.
  \end{itemize}
\item ~
  \hypertarget{should-i-refinance-my-mortgage}{%
  \paragraph{Should I refinance my
  mortgage?}\label{should-i-refinance-my-mortgage}}

  \begin{itemize}
  \tightlist
  \item
    \href{https://www.nytimes.com/article/coronavirus-money-unemployment.html?action=click\&pgtype=Article\&state=default\&region=MAIN_CONTENT_3\&context=storylines_faq}{It
    could be a good idea,} because mortgage rates have
    \href{https://www.nytimes.com/2020/07/16/business/mortgage-rates-below-3-percent.html?action=click\&pgtype=Article\&state=default\&region=MAIN_CONTENT_3\&context=storylines_faq}{never
    been lower.} Refinancing requests have pushed mortgage applications
    to some of the highest levels since 2008, so be prepared to get in
    line. But defaults are also up, so if you're thinking about buying a
    home, be aware that some lenders have tightened their standards.
  \end{itemize}
\item ~
  \hypertarget{what-is-school-going-to-look-like-in-september}{%
  \paragraph{What is school going to look like in
  September?}\label{what-is-school-going-to-look-like-in-september}}

  \begin{itemize}
  \tightlist
  \item
    It is unlikely that many schools will return to a normal schedule
    this fall, requiring the grind of
    \href{https://www.nytimes.com/2020/06/05/us/coronavirus-education-lost-learning.html?action=click\&pgtype=Article\&state=default\&region=MAIN_CONTENT_3\&context=storylines_faq}{online
    learning},
    \href{https://www.nytimes.com/2020/05/29/us/coronavirus-child-care-centers.html?action=click\&pgtype=Article\&state=default\&region=MAIN_CONTENT_3\&context=storylines_faq}{makeshift
    child care} and
    \href{https://www.nytimes.com/2020/06/03/business/economy/coronavirus-working-women.html?action=click\&pgtype=Article\&state=default\&region=MAIN_CONTENT_3\&context=storylines_faq}{stunted
    workdays} to continue. California's two largest public school
    districts --- Los Angeles and San Diego --- said on July 13, that
    \href{https://www.nytimes.com/2020/07/13/us/lausd-san-diego-school-reopening.html?action=click\&pgtype=Article\&state=default\&region=MAIN_CONTENT_3\&context=storylines_faq}{instruction
    will be remote-only in the fall}, citing concerns that surging
    coronavirus infections in their areas pose too dire a risk for
    students and teachers. Together, the two districts enroll some
    825,000 students. They are the largest in the country so far to
    abandon plans for even a partial physical return to classrooms when
    they reopen in August. For other districts, the solution won't be an
    all-or-nothing approach.
    \href{https://bioethics.jhu.edu/research-and-outreach/projects/eschool-initiative/school-policy-tracker/}{Many
    systems}, including the nation's largest, New York City, are
    devising
    \href{https://www.nytimes.com/2020/06/26/us/coronavirus-schools-reopen-fall.html?action=click\&pgtype=Article\&state=default\&region=MAIN_CONTENT_3\&context=storylines_faq}{hybrid
    plans} that involve spending some days in classrooms and other days
    online. There's no national policy on this yet, so check with your
    municipal school system regularly to see what is happening in your
    community.
  \end{itemize}
\item ~
  \hypertarget{is-the-coronavirus-airborne}{%
  \paragraph{Is the coronavirus
  airborne?}\label{is-the-coronavirus-airborne}}

  \begin{itemize}
  \tightlist
  \item
    The coronavirus
    \href{https://www.nytimes.com/2020/07/04/health/239-experts-with-one-big-claim-the-coronavirus-is-airborne.html?action=click\&pgtype=Article\&state=default\&region=MAIN_CONTENT_3\&context=storylines_faq}{can
    stay aloft for hours in tiny droplets in stagnant air}, infecting
    people as they inhale, mounting scientific evidence suggests. This
    risk is highest in crowded indoor spaces with poor ventilation, and
    may help explain super-spreading events reported in meatpacking
    plants, churches and restaurants.
    \href{https://www.nytimes.com/2020/07/06/health/coronavirus-airborne-aerosols.html?action=click\&pgtype=Article\&state=default\&region=MAIN_CONTENT_3\&context=storylines_faq}{It's
    unclear how often the virus is spread} via these tiny droplets, or
    aerosols, compared with larger droplets that are expelled when a
    sick person coughs or sneezes, or transmitted through contact with
    contaminated surfaces, said Linsey Marr, an aerosol expert at
    Virginia Tech. Aerosols are released even when a person without
    symptoms exhales, talks or sings, according to Dr. Marr and more
    than 200 other experts, who
    \href{https://academic.oup.com/cid/article/doi/10.1093/cid/ciaa939/5867798}{have
    outlined the evidence in an open letter to the World Health
    Organization}.
  \end{itemize}
\end{itemize}

This was dubious --- the land's value had dwindled over the years to a
few thousand dollars, and the government had forced the family to
surrender the property when they immigrated to the United States. But
Ms. Dionisio, as a tribute to her deceased father, was intent on
collecting compensation for the plot from the Department of Agrarian
Reform. She and her sister had chipped away at this task for years, a
wrestling match with provincial land bureaucrats who demanded a long
list of notarized documents. This spring's trip was the one in which Ms.
Dionisio would collect the check.

``She said, `I have to finish this,''' Ms. Quijano said. ``She wanted to
settle it once and for all.''

News about the virus was already beginning to circulate, and an active
volcano had spewed ash particles into the air. People tried to talk Ms.
Dionisio, who had survived two bouts of cancer and suffered from
diabetes, out of traveling. Her husband was against making the trip, and
so was their daughter and her sister. Ms. Jenkins tried to dissuade her,
as well, but understood it was useless.

``It goes back to her never wanting to give up,'' Ms. Jenkins said. Ms.
Dionisio's daughter, Rowena Dionisio-Connelly, agreed: As a small girl,
Ms. Dionisio would run after a neighbor who teased her brother, who was
gay. ``She would pull out a wooden spoon and chase the neighbor with it
all the way to his doorstep,'' she said.

So the couple set off for a month of travel, sending back dispatches
from their journey. Increasingly, they were in the shadow of the virus.

\hypertarget{a-gut-punch}{%
\subsection{A gut punch}\label{a-gut-punch}}

``We've been wearing our masks to avoid the coronavirus,'' Mr. Dionisio
wrote in a text message to Ms. Jenkins, along with a photo of a carved
bench nestled in tropical greenery. As the trip went on, he wrote, more
and more of the tourist sights were being closed.

In early March, Ms. Dionisio called home to announce that she had
achieved her goal: Zipped inside her suitcase was a check from the
Filipino government, compensating the family for the loss of the coconut
plantation, said her sister, Ms. Quijano. She turned toward home, a trip
that would take the couple through Thailand and, briefly, South Korea.

There were, their children now realize, small signs that something was
wrong.

Image

Loretta and Rodrigo Dionisio, center, with, from left, their son-in-law,
Chris Connelly, daughter, Rowena Dionisio-Connelly, daughter-in-law,~
Cathrina Dionisio, and son, Rembert Dionisio.

At one point, Mr. Dionisio mentioned to Rowena that her mother had a
fever. In a phone call to her sister, passing on the triumphant news
about the check, Ms. Dionisio mentioned she wasn't feeling well.

Then they were on their way to safety. Mr. Dionisio told his daughter
that they had to wait, exhausted, for nine hours in the airport in
Seoul, where there was no free seat for them.

That is something Rowena's mind gets stuck on now.

``It's a gut punch, because I can see how they were,'' she said. ``My
dad was probably trying to make her comfortable on the windowsill, her
head was probably resting on his shoulder.''

When they landed in Los Angeles, her mother called from outside her
sister-in-law's house, saying she needed to rest after the 11-hour
flight.

``She was joking and laughing about not being able to get into the
house,'' she said. ``She said, `I'll call you later. I need to sleep.'
And then I never heard from her again.''

When Mr. Dionisio awoke from a deep, jet-lagged sleep hours later, he
could not wake his wife. Panicked, he performed CPR and called an
ambulance, which took her to a hospital with a weak pulse. Over the next
hours, she experienced four cardiac arrests, her family said. She was
declared dead at 2:57 a.m. on March 10.

After she tested positive for the virus on March 11, the family was
occupied with crisis management, five or six hours a day of phone calls
to public health officials, the crematory, hospital staff. Not only
their father, but also their aunt and uncle, and another aunt and
cousin, have been ordered to self-quarantine.

A memorial gathering, for now, is out of the question.

``We don't want to put any other family members in harm's way,'' Ms.
Dionisio's son, Rembert, said. ``That's what makes everything really
rough right now. It's almost taken away from what is happening with my
mother.''

His cousin, Paula, is thinking of setting up a conference call so that a
priest can say Mass.

The check for the coconut planation will be deposited, its balance given
to Ms. Dionisio's eldest sister.

Rowena has trouble speaking about her mother without breaking down.

``It's hard for me to to come to terms with the fact that Mom is gone,''
she said. ``I'm searching for her. Her smell. I want to touch her
hand.''

Sarah Mervosh contributed reporting from New York, Amy Qin from Beijing
and Jason Horowitz from Rome.

Kitty Bennett contributed research from New York.

Advertisement

\protect\hyperlink{after-bottom}{Continue reading the main story}

\hypertarget{site-index}{%
\subsection{Site Index}\label{site-index}}

\hypertarget{site-information-navigation}{%
\subsection{Site Information
Navigation}\label{site-information-navigation}}

\begin{itemize}
\tightlist
\item
  \href{https://help.nytimes.com/hc/en-us/articles/115014792127-Copyright-notice}{©~2020~The
  New York Times Company}
\end{itemize}

\begin{itemize}
\tightlist
\item
  \href{https://www.nytco.com/}{NYTCo}
\item
  \href{https://help.nytimes.com/hc/en-us/articles/115015385887-Contact-Us}{Contact
  Us}
\item
  \href{https://www.nytco.com/careers/}{Work with us}
\item
  \href{https://nytmediakit.com/}{Advertise}
\item
  \href{http://www.tbrandstudio.com/}{T Brand Studio}
\item
  \href{https://www.nytimes.com/privacy/cookie-policy\#how-do-i-manage-trackers}{Your
  Ad Choices}
\item
  \href{https://www.nytimes.com/privacy}{Privacy}
\item
  \href{https://help.nytimes.com/hc/en-us/articles/115014893428-Terms-of-service}{Terms
  of Service}
\item
  \href{https://help.nytimes.com/hc/en-us/articles/115014893968-Terms-of-sale}{Terms
  of Sale}
\item
  \href{https://spiderbites.nytimes.com}{Site Map}
\item
  \href{https://help.nytimes.com/hc/en-us}{Help}
\item
  \href{https://www.nytimes.com/subscription?campaignId=37WXW}{Subscriptions}
\end{itemize}
