Sections

SEARCH

\protect\hyperlink{site-content}{Skip to
content}\protect\hyperlink{site-index}{Skip to site index}

\href{https://www.nytimes.com/podcasts/the-daily}{The Daily}

\href{https://myaccount.nytimes.com/auth/login?response_type=cookie\&client_id=vi}{}

\href{https://www.nytimes.com/section/todayspaper}{Today's Paper}

\href{/podcasts/the-daily}{The Daily}\textbar{}The Pandemic and the
Primary

\url{https://nyti.ms/2Ue7YZL}

\begin{itemize}
\item
\item
\item
\item
\item
\item
\end{itemize}

Advertisement

\protect\hyperlink{after-top}{Continue reading the main story}

transcript

Back to The Daily

bars

0:00/30:07

-30:07

transcript

\hypertarget{the-pandemic-and-the-primary}{%
\subsection{The Pandemic and the
Primary}\label{the-pandemic-and-the-primary}}

\hypertarget{hosted-by-michael-barbaro-produced-by-jessica-cheung-eric-krupke-and-alexandra-leigh-young-with-help-from-stella-tan-and-robert-jimison-and-edited-by-larissa-anderson}{%
\subsubsection{Hosted by Michael Barbaro; produced by Jessica Cheung,
Eric Krupke and Alexandra Leigh Young; with help from Stella Tan and
Robert Jimison; and edited by Larissa
Anderson}\label{hosted-by-michael-barbaro-produced-by-jessica-cheung-eric-krupke-and-alexandra-leigh-young-with-help-from-stella-tan-and-robert-jimison-and-edited-by-larissa-anderson}}

\hypertarget{how-the-coronavirus-has-changed-the-stakes-of-the-presidential-race--and-created-a-crisis-candidate}{%
\paragraph{How the coronavirus has changed the stakes of the
presidential race --- and created a crisis
candidate.}\label{how-the-coronavirus-has-changed-the-stakes-of-the-presidential-race--and-created-a-crisis-candidate}}

Monday, March 23rd, 2020

\begin{itemize}
\item
  michael barbaro\\
  From The New York Times, I'm Michael Barbaro. This is ``The Daily.''
\item
  {[}music{]}\\
  Today: Two weeks ago, the biggest story in the country was the
  Democratic presidential primary. Now, with the coronavirus, it's been
  largely forgotten. Alex Burns on what happened when those two stories
  collided. It's Monday, March 23.
\item
  michael barbaro\\
  Hello?
\item
  alex burns\\
  Hello, Michael.
\item
  michael barbaro\\
  Hey. It's always nice to be heralded by a bing, you know?
\item
  alex burns\\
  Is that not your usual entrance into halls and rooms?
\item
  michael barbaro\\
  Usually it's trumpets. {[}LAUGHS{]} I feel like every day pretty much
  for two weeks, we would talk on the show. And then poof, we have this
  unplanned hiatus, and you go away. And so I kind of miss you a little
  bit.
\item
  alex burns\\
  {[}LAUGHS{]} It's a particularly painful kind of social isolation, for
  me at least.
\item
  michael barbaro\\
  So, bring us up to speed on the Democratic primary. How would you
  describe the current state of the race?
\item
  alex burns\\
  Well, it's pretty close to over at this point. Joe Biden has emerged
  as the overwhelming favorite to be the Democratic nominee. He clearly
  has a support from the majority of the party, wide lead in the
  delegate count. And Bernie Sanders has not conceded the race, but he's
  acknowledged that he is sort of reassessing his campaign. And that's
  often the first stage in the process of winding things down.
\item
  michael barbaro\\
  So in effect, it feels like what you just described is more or less
  where we were a couple of weeks ago. But with the benefit of some
  hindsight and some reporting on your part, I wonder if you could tell
  us how exactly that happened, because I don't think we've properly
  accounted for the whiplash and the speed with which the Sanders
  campaign came kind of crashing down.
\item
  alex burns\\
  No, I don't think we have. And I think really, you have to rewind the
  tape almost exactly a month ago to what was the high point.
\item
  archived recording (bernie sanders)\\
  And now, I'm delighted to bring you some pretty good news.
  {[}CHEERING{]} I think all of you know, we won the popular vote in
  Iowa. {[}CHEERING{]} We won the New Hampshire primary. {[}CHEERING{]}
  And according to three networks in the A.P., we have now won the
  Nevada caucus! {[}CHEERING{]}
\end{itemize}

alex burns

He wins the Nevada caucuses. And he wins them by just an enormous
margin.

\begin{itemize}
\tightlist
\item
  archived recording (crowd)\\
  Bernie! Bernie! Bernie!
\end{itemize}

alex burns

He crushes Joe Biden, Elizabeth Warren, Pete Buttigieg.

\begin{itemize}
\tightlist
\item
  archived recording (crowd)\\
  Bernie! Bernie! Bernie!
\end{itemize}

alex burns

He wins young voters. He wins older voters. He wins folks who have
participated in caucuses in the past and people who are participating
for the first time. It's in Nevada, where we see him go from winning
about a quarter of the vote to winning nearly half the vote.

\begin{itemize}
\tightlist
\item
  archived recording (bernie sanders)\\
  And no campaign has a grass-roots movement like we do, which is
  another reason why we're going to win this election. {[}CROWD
  CHEERING{]}
\end{itemize}

alex burns

And that sends a really powerful signal across the Democratic Party.

michael barbaro

And what does that signal?

alex burns

I think what most of us thought at the time was that it was sending a
signal that Bernie Sanders was broadening his appeal, and that he was
building a more diverse and more muscular political coalition than he
had been able to demonstrate so far. It's also clear now that another
message it sent to the rest of the Democratic Party was that Sanders was
becoming a real freight train in this race. And that if you were going
to stop him, you were going to need to do it real fast. So the moment
where Sanders is riding high like that, I think he's kind of faced with
a choice of either trying to more actively reassure the Democratic Party
that they can trust him to be the nominee --- and to make a more
explicit case about his own electability and to address himself more
clearly to moderate voters who have, you know, been beyond wary of his
campaign, just terrified of the idea of nominating him. Or he can stick
with the approach that got him there to begin with. And that's to run as
this anti-establishment progressive populist who is taking on his own
party in addition to taking on the Republican Party. And the question
then I think is, which version are we going to hear over the coming week
and the coming months from Bernie Sanders?

michael barbaro

And what happens?

alex burns

The day after he wins the Nevada caucuses, a ``60 Minutes'' interview
airs.

\begin{itemize}
\item
  archived recording (anderson cooper)\\
  Back in the 1980s, Sanders had some positive things to say about the
  former Soviet Union and the Sandinistas in Nicaragua.
\item
  archived recording (bernie sanders)\\
  And everybody was totally convinced ---
\item
  archived recording (anderson cooper)\\
  Here he is explaining why the Cuban people didn't rise up and help the
  U.S. overthrow Cuban leader, Fidel Castro.
\item
  archived recording (bernie sanders)\\
  He educated their kids, gave them health care ---
\end{itemize}

alex burns

The piece of it that really pops out to a lot of Democrats is when
Anderson Cooper asked Bernie Sanders about his past praise for the
Castro regime in Cuba.

\begin{itemize}
\tightlist
\item
  archived recording (bernie sanders)\\
  We're very opposed to the authoritarian nature of Cuba. But it's
  unfair to simply say, everything is bad. When Fidel Castro came into
  office, you know what he did? He had a massive literacy program. Is
  that a bad thing? Even though Fidel Castro ---
\end{itemize}

alex burns

And it just sends a shockwave through Democrats.

{[}music{]}

\begin{itemize}
\item
  archived recording\\
  Bernie Sanders has the lead in total votes in delegates. But his
  comments he made Sunday night on ``60 Minutes'' that are causing fresh
  panic for some Democrats.
\item
  archived recording 1\\
  It's absolutely inconceivable that any American as old as him, knowing
  everything we know about Fidel Castro and the people that he's
  murdered over the years, that anybody could support him in any way.
\item
  archived recording 2\\
  The blowback is emblematic of broader uncertainty about how nominating
  a self-described democratic socialist could impact Democrats' chances
  in the general election.
\item
  archived recording 3\\
  I like Bernie.
\item
  archived recording 4\\
  How do you feel about him praising the Soviet Union and Fidel Castro?
\item
  archived recording 5\\
  Yeah, I don't like that part.
\end{itemize}

alex burns

So to hear that from Sanders, and to hear him essentially be
unapologetic about it, I think, was a real sign to people that if you
thought this guy was going to start moving to the middle now, that is
not happening.

\begin{itemize}
\tightlist
\item
  archived recording\\
  His response infuriated democratic lawmakers from South Florida, a key
  swing state where public support for the Castro regime is a
  nonstarter.
\end{itemize}

alex burns

More specifically, and in a more localized, but really no less important
way, this is terrifying to Democrats in Florida.

\begin{itemize}
\item
  archived recording (reporter)\\
  Freshman Congresswoman Debbie Mucarsel-Powell, she called Sanders's
  comment quote, ``absolutely unacceptable.''
\item
  archived recording (debbie mucarsel-powell)\\
  He made more than a mistake. It's what he believes. And it's
  unacceptable to our community.
\item
  archived recording (reporter)\\
  And Congresswoman Donna Shalala, who suggested that Sanders talk to
  her constituents before quote, ``singing the praises of a murderous
  tyrant``, unquote.
\end{itemize}

alex burns

You see just a unified, almost unified, wall of criticism of Sanders
coming from Democrats in that state, saying, you are imperiling the
general election in one of the biggest swing states on the map.

{[}music{]}

After the ``60 Minutes'' interview, you then start to hear prominent
national Democrats say something that many of them haven't said so far,
which is, we just can't nominate this guy.

\begin{itemize}
\tightlist
\item
  archived recording (jim clyburn)\\
  Let me thank all of you for joining us here this morning.
\end{itemize}

alex burns

And that's the point where you see Joe Biden get a major, major
endorsement from Jim Clyburn, popular congressman from South Carolina,
highest ranking African-American member of Congress.

\begin{itemize}
\item
  archived recording (jim clyburn)\\
  Well, I want the public to know that I'm voting for Joe Biden. South
  Carolinans should be voting for Joe Biden. And here's why. I know Joe.
  We know Joe. But most importantly, Joe knows us.
\item
  archived recording\\
  That's right. That's right.
\item
  archived recording (jim clyburn)\\
  That's important.
\end{itemize}

alex burns

So we head into the South Carolina primary, which Joe Biden was always
favored to win.

\begin{itemize}
\tightlist
\item
  archived recording\\
  NBC News is now projecting that former Vice President Joe Biden has
  won a decisive victory in the South Carolina Democratic primary.
\end{itemize}

alex burns

And he wins it by 30 points.

\begin{itemize}
\tightlist
\item
  archived recording\\
  And he has done so by a substantial margin, potentially changing the
  dynamics of a race dominated so far by Vermont Senator Bernie Sanders.
\end{itemize}

alex burns

That is beyond what even the Biden campaign was expecting. Biden just
coalesces the overwhelming majority of Democrats who are not for Bernie
Sanders behind his campaign.

michael barbaro

And so why in that moment did we not see Biden's win in South Carolina,
which as you just said, was kind of mathematically quite significant as
the beginning of a turning point kind of comeback?

alex burns

So on the night of South Carolina, we can look at Biden's 30 point
margin, and say, wow, that was impressive. And this guy is clearly more
resilient than even some of his supporters, even some of his inner
circle believed he was. What we didn't know is that the next day ---

\begin{itemize}
\tightlist
\item
  archived recording (pete buttigieg)\\
  So tonight, I am making the difficult decision to suspend my campaign
  for the presidency.
\end{itemize}

alex burns

--- Pete Buttigieg would drop out of the race.

\begin{itemize}
\tightlist
\item
  archived recording (pete buttigieg)\\
  I will no longer ---
\end{itemize}

alex burns

And then on Monday morning, Amy Klobuchar would drop out of the race.

\begin{itemize}
\tightlist
\item
  archived recording (amy klobuchar)\\
  Today, I am ending my campaign and endorsing Joe Biden for president.
  {[}CROWD CHEERING{]}
\end{itemize}

alex burns

And by the end of Monday ---

\begin{itemize}
\tightlist
\item
  archived recording (pete buttigieg)\\
  That I'm delighted to endorse and support Joe Biden for president.
  {[}CROWD CHEERING{]}
\end{itemize}

alex burns

--- both of them would endorse Joe Biden.

\begin{itemize}
\tightlist
\item
  archived recording (beto o'rourke)\\
  I will be casting my ballot for Joe Biden. {[}CROWD CHEERING{]}
\end{itemize}

alex burns

And by the way, so would Beto O'Rourke, who dropped out of the race a
couple of months ealier. We saw a transformation of voter's preferences
within this field at a speed that I don't think it's an overstatement to
call it totally unprecedented.

michael barbaro

I want to understand this phenomenon. Let me just begin with those
endorsements that you just described. Why did Buttigieg, did Klobuchar
drop out and endorse him so quickly? What's your understanding now of
how that happened?

alex burns

There are a couple of things going on here. Pete Buttigieg was on track
to get totally waxed on Super Tuesday, which is just three days after
South Carolina. So he is staring at the possibility of not just defeat
and not just a setback, but something like political humiliation to go
in a month from essentially winning Iowa --- basically splitting the win
with Bernie Sanders --- to winning absolutely nothing on Super Tuesday.
And so there is a logic of self-interest that says, maybe you should
take your winnings and walk away from the table at this point. Amy
Klobuchar faces a somewhat different situation because she does look
like she will win her home state of Minnesota. But she is clear-eyed
enough at that point to recognize there's really no path forward for her
in the race beyond Super Tuesday. Beyond the self-interest, though,
these are two of the candidates who have been the bluntest and most
pointed all along about their concern for the implications of nominating
Sanders. They have been talking about the idea of nominating Sanders as
deeply, deeply politically risky. And there are people who can do the
math for themselves and see that after Nevada and South Carolina, there
are really only two candidates in this race who are putting up big
numbers on the national level in the way that it would take to go the
distance. And between those two candidates, there's no question about
whether they're closer to Biden or Sanders.

michael barbaro

OK, so at this point, post-South Carolina and post-Super Tuesday, Biden
is the front-runner. But there's a ton of primaries and delegates left.
And still theoretically, time for a Sanders comeback, right?

alex burns

Right, and it turns out to be very much a theoretical exercise, the
Sanders comeback. You see starting right after Super Tuesday, he points
the way to the next round of primaries, most importantly, Michigan.

\begin{itemize}
\tightlist
\item
  archived recording\\
  Well, the ``Joementum'' continues. Former Vice President Biden
  swooping to victory overnight in a pivotal primary contest.
\end{itemize}

alex burns

Sanders ends up totally flopping in Michigan. It's a blowout in the
state.

\begin{itemize}
\tightlist
\item
  archived recording\\
  Biden sweeping every county in Michigan, Missouri and Mississippi. He
  also won Idaho. The wins giving him a commanding 160 delegate lead
  over Sanders.
\end{itemize}

alex burns

And what happens, essentially the night that Biden wins in Michigan and
in a number of other important states, is that the campaign is
essentially frozen in place by a force that hits the campaign and hits
the entire country in a way that nobody could have anticipated. And of
course, that's the coronavirus.

{[}music{]}

michael barbaro

We'll be right back.

Alex, how exactly does the coronavirus epidemic --- eventually becomes a
pandemic --- how does that hurt Sanders and help Biden? That's not
entirely intuitive to me.

alex burns

Well, what it does is it essentially ends the active portion of the
campaign. On the night of the March 10 primaries, both of them are
supposed to hold election night events where they address a roaring
crowd of supporters. Both of those events get canceled. There have been
no campaign rallies since then. Bernie Sanders cannot hit the road and
gather tens of thousands of people in stadiums and deliver a forceful
plea to the Democratic Party to not go ahead and nominate Joe Biden. The
window to make that argument has essentially closed. What's also going
on is that the terms of debate go from being about ideological
differences and policy differences to the reality of a terrifying
national crisis. And what we see consistently in public polling for
months, and in exit polls taken around the March primaries, is that on
the question of which candidate you trust to handle a major crisis, Joe
Biden is overwhelmingly favored, not just over Bernie Sanders, but over
every alternative that democratic voters had in the race.

michael barbaro

So in a sense, the coronavirus doesn't just freeze the campaign and
freeze Joe Biden's advantages electorally, it amplifies them because
many democratic voters see him as a crisis-style leader.

alex burns

Exactly. Biden's biggest strengths from the beginning have involved his
experience, and his perceived steadiness, and the fact that voters
basically find him trustworthy, and reassuring.

michael barbaro

Well, that's interesting because another way of thinking about this, and
the impact that we're seeing already on Americans from this pandemic ---
the health care shortcomings, the thin financial cushion on which so
many Americans are living --- that's the stuff that Bernie Sanders has
been saying forever. So I could also imagine a version of this where the
pandemic strengthens Sanders's candidacy, not weakens it.

alex burns

I think that's really, really sharply put. But I do think people are
processing this differently than they would process, for instance, a
crash just of the financial sector. That if you saw an economic collapse
in which people felt like the government was racing to contain a
contagion from the financial industry, and that their lives were
basically safe, I suspect that we might be having a different political
debate right now. And that you would see Bernie Sanders holding these
enormous rallies and making exactly the case you just laid out. And I
think that things are so turbulent and unpredictable right now that we
can't totally rule out the possibility that maybe that happens at some
point once people see the scale of economic damage and the kind of
vividness that we certainly and unfortunately will. What we have right
now, though, is people who are experiencing a terrifying disruption in
their daily lives. They're experiencing it yes, as an economic crisis,
but also as a public health crisis, and something that probably feels to
a lot of people like a national security crisis. And while a lot of
Sanders's themes and ideas about the economy will probably be a bigger
part of the conversation in the coming months, I don't know that the
country has reached that point yet.

michael barbaro

So I want to turn now to the practical question of how the rest of the
Democratic primary unfolds. Because the situation we're in hasn't just
frozen the dynamics of the race, it also seems to have actually frozen
the mechanics of the campaign. Which feels pretty tricky because people
have to leave their homes and go vote in order for there to ever be a
nominee. So how is that going to work?

alex burns

Well, the short answer is we still don't really know. Almost every day
now, we hear from another state that is delaying its primary well into
May or even into June. Now, some of the relevance of those changes is
going to depend on what Bernie Sanders does next. If Sanders does stick
around, and if Biden is not able to functionally unify the Democratic
Party, with or without Bernie Sanders's support, then we could see this
really weird long period of dormancy in the campaign followed by a
sudden frenzy of activity again in the late spring when maybe the virus
will be more under control, and maybe people will start voting again.
Personally right now, I think that that is an unlikely scenario.

michael barbaro

Alex, what's your understanding of how Bernie Sanders is thinking of the
big and difficult question of how long to stay in the race if it doesn't
really seem as a practical path to the nomination? He's certainly
hearing lots of calls to step aside in a moment of crisis, kind of let
the party coalesce around a nominee and prepare itself for a general
election.

alex burns

I think there are a couple things about the mindset of the Sanders's
camp right now that are really worth emphasizing here. One is that this
is a group that not that long ago thought that they were on, not a glide
path, but a pretty convincing course to the nomination. And they saw it
fall away with astonishing speed. So there's a level, I think, still of
kind of shell shock, at feeling like they had this, or they were close
to having this. And then it was yanked away from them. That's a hard
thing for a campaign, and especially, for a candidate to process. I
think the conditions of the pandemic also make it harder for, well,
anybody involved in the race at this point to think through, what is the
right thing to do next? What we know about Bernie Sanders is that he
cares a great deal about his agenda. And we also know that, as a
personal matter, he likes Joe Biden. This is not the Sanders-Clinton
rivalry. He doesn't feel that the party really conspired to kneecap him
in this race in the way that he did, with some justification, in 2016.
So what you see here is a candidate, Sanders, who I think understands
what an underdog he is right now, and an opponent in Biden, who is a
negotiator. And I think that's why you're seeing Biden make such
explicit overtures to Sanders supporters.

\begin{itemize}
\tightlist
\item
  archived recording (joe biden)\\
  Tonight in keeping with the latest guidance from the CDC, I'm speaking
  to you from my home in Wilmington, Delaware ---
\end{itemize}

alex burns

That in the last two primary nights that we'll have for a while, he has
in his election night remarks, addressed himself to Sanders supporters
---

\begin{itemize}
\tightlist
\item
  archived recording (joe biden)\\
  So let me say, especially to the young voters who have been inspired
  by Senator Sanders, I hear you. I know what's at stake. I know what we
  have to do.
\end{itemize}

alex burns

--- saying that he admires their enthusiasm and their ideas.

\begin{itemize}
\tightlist
\item
  archived recording (joe biden)\\
  Senator Sanders and his supporters have brought a remarkable passion
  and tenacity to all of these issues. And together they have shifted
  the fundamental conversation in this country.
\end{itemize}

alex burns

He gave them credit for having fundamentally changed the framework of
American politics. And he, specifically addressing young people, said
---

\begin{itemize}
\tightlist
\item
  archived recording (joe biden)\\
  Senator Sanders and I may disagree on tactics. But we share a common
  vision for the need to provide affordable health care for all
  Americans, reduce income inequity that has risen so drastically, to
  tackling the existential threat of our time, climate change.
\end{itemize}

alex burns

--- he understands the gravity of the challenges that they feel in their
lives. When Biden takes those steps, it's a clear signal that he's
trying to show Bernie Sanders that he has respect for the movement that
he's built.

\begin{itemize}
\tightlist
\item
  archived recording (joe biden)\\
  We have to step up and care for one another. Thank you all. Thank you
  all for listening.
\end{itemize}

michael barbaro

Finally, Alex, if Joe Biden is becoming a kind of de facto nominee over
the next few weeks during this dormancy in the campaign, and if it
starts to feel like a general election is getting underway between Biden
and President Trump, I wonder what this really unique set of
circumstances --- which has meant so much for the Democratic primary ---
is going to mean for Biden's potential challenger, the sitting
president, Donald Trump.

alex burns

We know that the president is not going to be able to run for
re-election on a message that happy days are here again, and there is
nothing but prosperity as far as we can see. That message is gone. What
we don't know is what kind of story he will be able to tell about
managing this crisis. We just don't know what the conditions on the
ground are going to be like in a couple of months, let alone in the
general election. When this crisis hit, Joe Biden had a pretty solid
advantage over the president in general election polls. The map just
feels to me like it's really up for grabs right now, because we've never
conducted an election under these kinds of conditions. And even 2008,
the election in the middle of a financial crisis, we hadn't had the kind
of time that we are going to have now to process the meaning of the
setbacks that the country is currently experiencing.

{[}music{]}

michael barbaro

It's interesting you mentioned 2008 because it feels to me that that
race might be the proper analogy, a crisis. And as you've said
throughout this conversation, Democrats are starting to view Joe Biden
as the candidate of crisis. I'm sure Republicans view President Trump as
the candidate of crisis. And the question will become, once this crisis
is over, what the general electorate views as the candidate of the
crisis, who handled the crisis well and who would get us out of the
crisis best.

alex burns

And is there a candidate they blame for the crisis? If people ultimately
see the president as having let them down in this, that feels awfully
hard to escape. As it is, we can't say that that's how the country is
going to feel. But we can say that he was an unpopular president on the
day this crisis started. And that it's certainly not, based on what we
know now, changing that picture in his favor.

{[}music{]}

michael barbaro

Alex, thank you very much.

alex burns

Thank you.

michael barbaro

We'll be right back.

{[}music{]}

michael barbaro

Here's what else you need to know today. Over the weekend, global
efforts to contain the coronavirus by restricting people's movements
intensified. Australia ordered most public spaces closed. India said it
was shutting down all but essential services in its capital, Delhi.
Germany limited gatherings to no more than two people. Britain ordered
1.5 million people with serious medical problems to self-quarantine. And
Lebanon called in the army to endorse a lockdown.

\begin{itemize}
\tightlist
\item
  archived recording (mike dewine)\\
  We are certainly at war. In a time of war, we have to make sacrifices.
  And I certainly, in the last week or so, have asked the people of Ohio
  to make many sacrifices.
\end{itemize}

michael barbaro

In the United States, Ohio and Louisiana became the latest states to
instruct residents to stay at home as infections in each state surged.

\begin{itemize}
\tightlist
\item
  archived recording (mike dewine)\\
  Other states have referred to this as shelter in place. We prefer stay
  at home. Either one, it's pretty much the same thing.
\end{itemize}

michael barbaro

In Washington, negotiations over a \$2 trillion stimulus package
designed to protect businesses and workers hurt by the pandemic reached
an impasse in the Senate.

\begin{itemize}
\tightlist
\item
  archived recording (joe manchin iii)\\
  The proposal that Leader McConnell from the Republican side has put
  forth is absolutely totally worried about Wall Street at this time.
  I'm worried about the people in little rural West Virginia and all
  over Main Street. That's the people we're worried about.
\end{itemize}

michael barbaro

On Sunday, Senate Democrats blocked the stimulus bill, saying it favors
big business and does not contain enough protections for workers by
allowing companies to fire workers even after receiving federal
bailouts.

\begin{itemize}
\tightlist
\item
  archived recording (joe manchin iii)\\
  And Wall Street's going to do just fine. It's always rebounded real
  well. They've always come back strong.
\end{itemize}

michael barbaro

Several Senate Republicans failed to cast votes because they are
self-quarantining over fears that they may have been exposed to the
coronavirus. And at least one senator, Republican Rand Paul of Kentucky,
has now tested positive for the virus.

The Times is providing free access to our most important updates on the
pandemic. To read it, go to nytimes.com/coronavirus.

{[}music{]}

That's it for ``The Daily.'' I'm Michael Barbaro. See you tomorrow.

\href{https://www.nytimes.com/column/the-daily}{\includegraphics{https://static01.nyt.com/images/2017/01/29/podcasts/the-daily-album-art/the-daily-album-art-square320-v4.png}The
Daily}Subscribe:

\begin{itemize}
\tightlist
\item
  \href{https://itunes.apple.com/us/podcast/id1200361736}{Apple
  Podcasts}
\item
  \href{https://www.google.com/podcasts?feed=aHR0cHM6Ly9yc3MuYXJ0MTkuY29tL3RoZS1kYWlseQ\%3D\%3D}{Google
  Podcasts}
\end{itemize}

\hypertarget{the-pandemic-and-the-primary-1}{%
\section{The Pandemic and the
Primary}\label{the-pandemic-and-the-primary-1}}

\hypertarget{how-the-coronavirus-has-changed-the-stakes-of-the-presidential-race--and-created-a-crisis-candidate-1}{%
\subsection{How the coronavirus has changed the stakes of the
presidential race --- and created a crisis
candidate.}\label{how-the-coronavirus-has-changed-the-stakes-of-the-presidential-race--and-created-a-crisis-candidate-1}}

Hosted by Michael Barbaro; produced by Jessica Cheung, Eric Krupke and
Alexandra Leigh Young; with help from Stella Tan and Robert Jimison; and
edited by Larissa Anderson

Transcript

transcript

Back to The Daily

bars

0:00/30:07

-0:00

transcript

\hypertarget{the-pandemic-and-the-primary-2}{%
\subsection{The Pandemic and the
Primary}\label{the-pandemic-and-the-primary-2}}

\hypertarget{hosted-by-michael-barbaro-produced-by-jessica-cheung-eric-krupke-and-alexandra-leigh-young-with-help-from-stella-tan-and-robert-jimison-and-edited-by-larissa-anderson-1}{%
\subsubsection{Hosted by Michael Barbaro; produced by Jessica Cheung,
Eric Krupke and Alexandra Leigh Young; with help from Stella Tan and
Robert Jimison; and edited by Larissa
Anderson}\label{hosted-by-michael-barbaro-produced-by-jessica-cheung-eric-krupke-and-alexandra-leigh-young-with-help-from-stella-tan-and-robert-jimison-and-edited-by-larissa-anderson-1}}

\hypertarget{how-the-coronavirus-has-changed-the-stakes-of-the-presidential-race--and-created-a-crisis-candidate-2}{%
\paragraph{How the coronavirus has changed the stakes of the
presidential race --- and created a crisis
candidate.}\label{how-the-coronavirus-has-changed-the-stakes-of-the-presidential-race--and-created-a-crisis-candidate-2}}

Monday, March 23rd, 2020

\begin{itemize}
\item
  michael barbaro\\
  From The New York Times, I'm Michael Barbaro. This is ``The Daily.''
\item
  {[}music{]}\\
  Today: Two weeks ago, the biggest story in the country was the
  Democratic presidential primary. Now, with the coronavirus, it's been
  largely forgotten. Alex Burns on what happened when those two stories
  collided. It's Monday, March 23.
\item
  michael barbaro\\
  Hello?
\item
  alex burns\\
  Hello, Michael.
\item
  michael barbaro\\
  Hey. It's always nice to be heralded by a bing, you know?
\item
  alex burns\\
  Is that not your usual entrance into halls and rooms?
\item
  michael barbaro\\
  Usually it's trumpets. {[}LAUGHS{]} I feel like every day pretty much
  for two weeks, we would talk on the show. And then poof, we have this
  unplanned hiatus, and you go away. And so I kind of miss you a little
  bit.
\item
  alex burns\\
  {[}LAUGHS{]} It's a particularly painful kind of social isolation, for
  me at least.
\item
  michael barbaro\\
  So, bring us up to speed on the Democratic primary. How would you
  describe the current state of the race?
\item
  alex burns\\
  Well, it's pretty close to over at this point. Joe Biden has emerged
  as the overwhelming favorite to be the Democratic nominee. He clearly
  has a support from the majority of the party, wide lead in the
  delegate count. And Bernie Sanders has not conceded the race, but he's
  acknowledged that he is sort of reassessing his campaign. And that's
  often the first stage in the process of winding things down.
\item
  michael barbaro\\
  So in effect, it feels like what you just described is more or less
  where we were a couple of weeks ago. But with the benefit of some
  hindsight and some reporting on your part, I wonder if you could tell
  us how exactly that happened, because I don't think we've properly
  accounted for the whiplash and the speed with which the Sanders
  campaign came kind of crashing down.
\item
  alex burns\\
  No, I don't think we have. And I think really, you have to rewind the
  tape almost exactly a month ago to what was the high point.
\item
  archived recording (bernie sanders)\\
  And now, I'm delighted to bring you some pretty good news.
  {[}CHEERING{]} I think all of you know, we won the popular vote in
  Iowa. {[}CHEERING{]} We won the New Hampshire primary. {[}CHEERING{]}
  And according to three networks in the A.P., we have now won the
  Nevada caucus! {[}CHEERING{]}
\end{itemize}

alex burns

He wins the Nevada caucuses. And he wins them by just an enormous
margin.

\begin{itemize}
\tightlist
\item
  archived recording (crowd)\\
  Bernie! Bernie! Bernie!
\end{itemize}

alex burns

He crushes Joe Biden, Elizabeth Warren, Pete Buttigieg.

\begin{itemize}
\tightlist
\item
  archived recording (crowd)\\
  Bernie! Bernie! Bernie!
\end{itemize}

alex burns

He wins young voters. He wins older voters. He wins folks who have
participated in caucuses in the past and people who are participating
for the first time. It's in Nevada, where we see him go from winning
about a quarter of the vote to winning nearly half the vote.

\begin{itemize}
\tightlist
\item
  archived recording (bernie sanders)\\
  And no campaign has a grass-roots movement like we do, which is
  another reason why we're going to win this election. {[}CROWD
  CHEERING{]}
\end{itemize}

alex burns

And that sends a really powerful signal across the Democratic Party.

michael barbaro

And what does that signal?

alex burns

I think what most of us thought at the time was that it was sending a
signal that Bernie Sanders was broadening his appeal, and that he was
building a more diverse and more muscular political coalition than he
had been able to demonstrate so far. It's also clear now that another
message it sent to the rest of the Democratic Party was that Sanders was
becoming a real freight train in this race. And that if you were going
to stop him, you were going to need to do it real fast. So the moment
where Sanders is riding high like that, I think he's kind of faced with
a choice of either trying to more actively reassure the Democratic Party
that they can trust him to be the nominee --- and to make a more
explicit case about his own electability and to address himself more
clearly to moderate voters who have, you know, been beyond wary of his
campaign, just terrified of the idea of nominating him. Or he can stick
with the approach that got him there to begin with. And that's to run as
this anti-establishment progressive populist who is taking on his own
party in addition to taking on the Republican Party. And the question
then I think is, which version are we going to hear over the coming week
and the coming months from Bernie Sanders?

michael barbaro

And what happens?

alex burns

The day after he wins the Nevada caucuses, a ``60 Minutes'' interview
airs.

\begin{itemize}
\item
  archived recording (anderson cooper)\\
  Back in the 1980s, Sanders had some positive things to say about the
  former Soviet Union and the Sandinistas in Nicaragua.
\item
  archived recording (bernie sanders)\\
  And everybody was totally convinced ---
\item
  archived recording (anderson cooper)\\
  Here he is explaining why the Cuban people didn't rise up and help the
  U.S. overthrow Cuban leader, Fidel Castro.
\item
  archived recording (bernie sanders)\\
  He educated their kids, gave them health care ---
\end{itemize}

alex burns

The piece of it that really pops out to a lot of Democrats is when
Anderson Cooper asked Bernie Sanders about his past praise for the
Castro regime in Cuba.

\begin{itemize}
\tightlist
\item
  archived recording (bernie sanders)\\
  We're very opposed to the authoritarian nature of Cuba. But it's
  unfair to simply say, everything is bad. When Fidel Castro came into
  office, you know what he did? He had a massive literacy program. Is
  that a bad thing? Even though Fidel Castro ---
\end{itemize}

alex burns

And it just sends a shockwave through Democrats.

{[}music{]}

\begin{itemize}
\item
  archived recording\\
  Bernie Sanders has the lead in total votes in delegates. But his
  comments he made Sunday night on ``60 Minutes'' that are causing fresh
  panic for some Democrats.
\item
  archived recording 1\\
  It's absolutely inconceivable that any American as old as him, knowing
  everything we know about Fidel Castro and the people that he's
  murdered over the years, that anybody could support him in any way.
\item
  archived recording 2\\
  The blowback is emblematic of broader uncertainty about how nominating
  a self-described democratic socialist could impact Democrats' chances
  in the general election.
\item
  archived recording 3\\
  I like Bernie.
\item
  archived recording 4\\
  How do you feel about him praising the Soviet Union and Fidel Castro?
\item
  archived recording 5\\
  Yeah, I don't like that part.
\end{itemize}

alex burns

So to hear that from Sanders, and to hear him essentially be
unapologetic about it, I think, was a real sign to people that if you
thought this guy was going to start moving to the middle now, that is
not happening.

\begin{itemize}
\tightlist
\item
  archived recording\\
  His response infuriated democratic lawmakers from South Florida, a key
  swing state where public support for the Castro regime is a
  nonstarter.
\end{itemize}

alex burns

More specifically, and in a more localized, but really no less important
way, this is terrifying to Democrats in Florida.

\begin{itemize}
\item
  archived recording (reporter)\\
  Freshman Congresswoman Debbie Mucarsel-Powell, she called Sanders's
  comment quote, ``absolutely unacceptable.''
\item
  archived recording (debbie mucarsel-powell)\\
  He made more than a mistake. It's what he believes. And it's
  unacceptable to our community.
\item
  archived recording (reporter)\\
  And Congresswoman Donna Shalala, who suggested that Sanders talk to
  her constituents before quote, ``singing the praises of a murderous
  tyrant``, unquote.
\end{itemize}

alex burns

You see just a unified, almost unified, wall of criticism of Sanders
coming from Democrats in that state, saying, you are imperiling the
general election in one of the biggest swing states on the map.

{[}music{]}

After the ``60 Minutes'' interview, you then start to hear prominent
national Democrats say something that many of them haven't said so far,
which is, we just can't nominate this guy.

\begin{itemize}
\tightlist
\item
  archived recording (jim clyburn)\\
  Let me thank all of you for joining us here this morning.
\end{itemize}

alex burns

And that's the point where you see Joe Biden get a major, major
endorsement from Jim Clyburn, popular congressman from South Carolina,
highest ranking African-American member of Congress.

\begin{itemize}
\item
  archived recording (jim clyburn)\\
  Well, I want the public to know that I'm voting for Joe Biden. South
  Carolinans should be voting for Joe Biden. And here's why. I know Joe.
  We know Joe. But most importantly, Joe knows us.
\item
  archived recording\\
  That's right. That's right.
\item
  archived recording (jim clyburn)\\
  That's important.
\end{itemize}

alex burns

So we head into the South Carolina primary, which Joe Biden was always
favored to win.

\begin{itemize}
\tightlist
\item
  archived recording\\
  NBC News is now projecting that former Vice President Joe Biden has
  won a decisive victory in the South Carolina Democratic primary.
\end{itemize}

alex burns

And he wins it by 30 points.

\begin{itemize}
\tightlist
\item
  archived recording\\
  And he has done so by a substantial margin, potentially changing the
  dynamics of a race dominated so far by Vermont Senator Bernie Sanders.
\end{itemize}

alex burns

That is beyond what even the Biden campaign was expecting. Biden just
coalesces the overwhelming majority of Democrats who are not for Bernie
Sanders behind his campaign.

michael barbaro

And so why in that moment did we not see Biden's win in South Carolina,
which as you just said, was kind of mathematically quite significant as
the beginning of a turning point kind of comeback?

alex burns

So on the night of South Carolina, we can look at Biden's 30 point
margin, and say, wow, that was impressive. And this guy is clearly more
resilient than even some of his supporters, even some of his inner
circle believed he was. What we didn't know is that the next day ---

\begin{itemize}
\tightlist
\item
  archived recording (pete buttigieg)\\
  So tonight, I am making the difficult decision to suspend my campaign
  for the presidency.
\end{itemize}

alex burns

--- Pete Buttigieg would drop out of the race.

\begin{itemize}
\tightlist
\item
  archived recording (pete buttigieg)\\
  I will no longer ---
\end{itemize}

alex burns

And then on Monday morning, Amy Klobuchar would drop out of the race.

\begin{itemize}
\tightlist
\item
  archived recording (amy klobuchar)\\
  Today, I am ending my campaign and endorsing Joe Biden for president.
  {[}CROWD CHEERING{]}
\end{itemize}

alex burns

And by the end of Monday ---

\begin{itemize}
\tightlist
\item
  archived recording (pete buttigieg)\\
  That I'm delighted to endorse and support Joe Biden for president.
  {[}CROWD CHEERING{]}
\end{itemize}

alex burns

--- both of them would endorse Joe Biden.

\begin{itemize}
\tightlist
\item
  archived recording (beto o'rourke)\\
  I will be casting my ballot for Joe Biden. {[}CROWD CHEERING{]}
\end{itemize}

alex burns

And by the way, so would Beto O'Rourke, who dropped out of the race a
couple of months ealier. We saw a transformation of voter's preferences
within this field at a speed that I don't think it's an overstatement to
call it totally unprecedented.

michael barbaro

I want to understand this phenomenon. Let me just begin with those
endorsements that you just described. Why did Buttigieg, did Klobuchar
drop out and endorse him so quickly? What's your understanding now of
how that happened?

alex burns

There are a couple of things going on here. Pete Buttigieg was on track
to get totally waxed on Super Tuesday, which is just three days after
South Carolina. So he is staring at the possibility of not just defeat
and not just a setback, but something like political humiliation to go
in a month from essentially winning Iowa --- basically splitting the win
with Bernie Sanders --- to winning absolutely nothing on Super Tuesday.
And so there is a logic of self-interest that says, maybe you should
take your winnings and walk away from the table at this point. Amy
Klobuchar faces a somewhat different situation because she does look
like she will win her home state of Minnesota. But she is clear-eyed
enough at that point to recognize there's really no path forward for her
in the race beyond Super Tuesday. Beyond the self-interest, though,
these are two of the candidates who have been the bluntest and most
pointed all along about their concern for the implications of nominating
Sanders. They have been talking about the idea of nominating Sanders as
deeply, deeply politically risky. And there are people who can do the
math for themselves and see that after Nevada and South Carolina, there
are really only two candidates in this race who are putting up big
numbers on the national level in the way that it would take to go the
distance. And between those two candidates, there's no question about
whether they're closer to Biden or Sanders.

michael barbaro

OK, so at this point, post-South Carolina and post-Super Tuesday, Biden
is the front-runner. But there's a ton of primaries and delegates left.
And still theoretically, time for a Sanders comeback, right?

alex burns

Right, and it turns out to be very much a theoretical exercise, the
Sanders comeback. You see starting right after Super Tuesday, he points
the way to the next round of primaries, most importantly, Michigan.

\begin{itemize}
\tightlist
\item
  archived recording\\
  Well, the ``Joementum'' continues. Former Vice President Biden
  swooping to victory overnight in a pivotal primary contest.
\end{itemize}

alex burns

Sanders ends up totally flopping in Michigan. It's a blowout in the
state.

\begin{itemize}
\tightlist
\item
  archived recording\\
  Biden sweeping every county in Michigan, Missouri and Mississippi. He
  also won Idaho. The wins giving him a commanding 160 delegate lead
  over Sanders.
\end{itemize}

alex burns

And what happens, essentially the night that Biden wins in Michigan and
in a number of other important states, is that the campaign is
essentially frozen in place by a force that hits the campaign and hits
the entire country in a way that nobody could have anticipated. And of
course, that's the coronavirus.

{[}music{]}

michael barbaro

We'll be right back.

Alex, how exactly does the coronavirus epidemic --- eventually becomes a
pandemic --- how does that hurt Sanders and help Biden? That's not
entirely intuitive to me.

alex burns

Well, what it does is it essentially ends the active portion of the
campaign. On the night of the March 10 primaries, both of them are
supposed to hold election night events where they address a roaring
crowd of supporters. Both of those events get canceled. There have been
no campaign rallies since then. Bernie Sanders cannot hit the road and
gather tens of thousands of people in stadiums and deliver a forceful
plea to the Democratic Party to not go ahead and nominate Joe Biden. The
window to make that argument has essentially closed. What's also going
on is that the terms of debate go from being about ideological
differences and policy differences to the reality of a terrifying
national crisis. And what we see consistently in public polling for
months, and in exit polls taken around the March primaries, is that on
the question of which candidate you trust to handle a major crisis, Joe
Biden is overwhelmingly favored, not just over Bernie Sanders, but over
every alternative that democratic voters had in the race.

michael barbaro

So in a sense, the coronavirus doesn't just freeze the campaign and
freeze Joe Biden's advantages electorally, it amplifies them because
many democratic voters see him as a crisis-style leader.

alex burns

Exactly. Biden's biggest strengths from the beginning have involved his
experience, and his perceived steadiness, and the fact that voters
basically find him trustworthy, and reassuring.

michael barbaro

Well, that's interesting because another way of thinking about this, and
the impact that we're seeing already on Americans from this pandemic ---
the health care shortcomings, the thin financial cushion on which so
many Americans are living --- that's the stuff that Bernie Sanders has
been saying forever. So I could also imagine a version of this where the
pandemic strengthens Sanders's candidacy, not weakens it.

alex burns

I think that's really, really sharply put. But I do think people are
processing this differently than they would process, for instance, a
crash just of the financial sector. That if you saw an economic collapse
in which people felt like the government was racing to contain a
contagion from the financial industry, and that their lives were
basically safe, I suspect that we might be having a different political
debate right now. And that you would see Bernie Sanders holding these
enormous rallies and making exactly the case you just laid out. And I
think that things are so turbulent and unpredictable right now that we
can't totally rule out the possibility that maybe that happens at some
point once people see the scale of economic damage and the kind of
vividness that we certainly and unfortunately will. What we have right
now, though, is people who are experiencing a terrifying disruption in
their daily lives. They're experiencing it yes, as an economic crisis,
but also as a public health crisis, and something that probably feels to
a lot of people like a national security crisis. And while a lot of
Sanders's themes and ideas about the economy will probably be a bigger
part of the conversation in the coming months, I don't know that the
country has reached that point yet.

michael barbaro

So I want to turn now to the practical question of how the rest of the
Democratic primary unfolds. Because the situation we're in hasn't just
frozen the dynamics of the race, it also seems to have actually frozen
the mechanics of the campaign. Which feels pretty tricky because people
have to leave their homes and go vote in order for there to ever be a
nominee. So how is that going to work?

alex burns

Well, the short answer is we still don't really know. Almost every day
now, we hear from another state that is delaying its primary well into
May or even into June. Now, some of the relevance of those changes is
going to depend on what Bernie Sanders does next. If Sanders does stick
around, and if Biden is not able to functionally unify the Democratic
Party, with or without Bernie Sanders's support, then we could see this
really weird long period of dormancy in the campaign followed by a
sudden frenzy of activity again in the late spring when maybe the virus
will be more under control, and maybe people will start voting again.
Personally right now, I think that that is an unlikely scenario.

michael barbaro

Alex, what's your understanding of how Bernie Sanders is thinking of the
big and difficult question of how long to stay in the race if it doesn't
really seem as a practical path to the nomination? He's certainly
hearing lots of calls to step aside in a moment of crisis, kind of let
the party coalesce around a nominee and prepare itself for a general
election.

alex burns

I think there are a couple things about the mindset of the Sanders's
camp right now that are really worth emphasizing here. One is that this
is a group that not that long ago thought that they were on, not a glide
path, but a pretty convincing course to the nomination. And they saw it
fall away with astonishing speed. So there's a level, I think, still of
kind of shell shock, at feeling like they had this, or they were close
to having this. And then it was yanked away from them. That's a hard
thing for a campaign, and especially, for a candidate to process. I
think the conditions of the pandemic also make it harder for, well,
anybody involved in the race at this point to think through, what is the
right thing to do next? What we know about Bernie Sanders is that he
cares a great deal about his agenda. And we also know that, as a
personal matter, he likes Joe Biden. This is not the Sanders-Clinton
rivalry. He doesn't feel that the party really conspired to kneecap him
in this race in the way that he did, with some justification, in 2016.
So what you see here is a candidate, Sanders, who I think understands
what an underdog he is right now, and an opponent in Biden, who is a
negotiator. And I think that's why you're seeing Biden make such
explicit overtures to Sanders supporters.

\begin{itemize}
\tightlist
\item
  archived recording (joe biden)\\
  Tonight in keeping with the latest guidance from the CDC, I'm speaking
  to you from my home in Wilmington, Delaware ---
\end{itemize}

alex burns

That in the last two primary nights that we'll have for a while, he has
in his election night remarks, addressed himself to Sanders supporters
---

\begin{itemize}
\tightlist
\item
  archived recording (joe biden)\\
  So let me say, especially to the young voters who have been inspired
  by Senator Sanders, I hear you. I know what's at stake. I know what we
  have to do.
\end{itemize}

alex burns

--- saying that he admires their enthusiasm and their ideas.

\begin{itemize}
\tightlist
\item
  archived recording (joe biden)\\
  Senator Sanders and his supporters have brought a remarkable passion
  and tenacity to all of these issues. And together they have shifted
  the fundamental conversation in this country.
\end{itemize}

alex burns

He gave them credit for having fundamentally changed the framework of
American politics. And he, specifically addressing young people, said
---

\begin{itemize}
\tightlist
\item
  archived recording (joe biden)\\
  Senator Sanders and I may disagree on tactics. But we share a common
  vision for the need to provide affordable health care for all
  Americans, reduce income inequity that has risen so drastically, to
  tackling the existential threat of our time, climate change.
\end{itemize}

alex burns

--- he understands the gravity of the challenges that they feel in their
lives. When Biden takes those steps, it's a clear signal that he's
trying to show Bernie Sanders that he has respect for the movement that
he's built.

\begin{itemize}
\tightlist
\item
  archived recording (joe biden)\\
  We have to step up and care for one another. Thank you all. Thank you
  all for listening.
\end{itemize}

michael barbaro

Finally, Alex, if Joe Biden is becoming a kind of de facto nominee over
the next few weeks during this dormancy in the campaign, and if it
starts to feel like a general election is getting underway between Biden
and President Trump, I wonder what this really unique set of
circumstances --- which has meant so much for the Democratic primary ---
is going to mean for Biden's potential challenger, the sitting
president, Donald Trump.

alex burns

We know that the president is not going to be able to run for
re-election on a message that happy days are here again, and there is
nothing but prosperity as far as we can see. That message is gone. What
we don't know is what kind of story he will be able to tell about
managing this crisis. We just don't know what the conditions on the
ground are going to be like in a couple of months, let alone in the
general election. When this crisis hit, Joe Biden had a pretty solid
advantage over the president in general election polls. The map just
feels to me like it's really up for grabs right now, because we've never
conducted an election under these kinds of conditions. And even 2008,
the election in the middle of a financial crisis, we hadn't had the kind
of time that we are going to have now to process the meaning of the
setbacks that the country is currently experiencing.

{[}music{]}

michael barbaro

It's interesting you mentioned 2008 because it feels to me that that
race might be the proper analogy, a crisis. And as you've said
throughout this conversation, Democrats are starting to view Joe Biden
as the candidate of crisis. I'm sure Republicans view President Trump as
the candidate of crisis. And the question will become, once this crisis
is over, what the general electorate views as the candidate of the
crisis, who handled the crisis well and who would get us out of the
crisis best.

alex burns

And is there a candidate they blame for the crisis? If people ultimately
see the president as having let them down in this, that feels awfully
hard to escape. As it is, we can't say that that's how the country is
going to feel. But we can say that he was an unpopular president on the
day this crisis started. And that it's certainly not, based on what we
know now, changing that picture in his favor.

{[}music{]}

michael barbaro

Alex, thank you very much.

alex burns

Thank you.

michael barbaro

We'll be right back.

{[}music{]}

michael barbaro

Here's what else you need to know today. Over the weekend, global
efforts to contain the coronavirus by restricting people's movements
intensified. Australia ordered most public spaces closed. India said it
was shutting down all but essential services in its capital, Delhi.
Germany limited gatherings to no more than two people. Britain ordered
1.5 million people with serious medical problems to self-quarantine. And
Lebanon called in the army to endorse a lockdown.

\begin{itemize}
\tightlist
\item
  archived recording (mike dewine)\\
  We are certainly at war. In a time of war, we have to make sacrifices.
  And I certainly, in the last week or so, have asked the people of Ohio
  to make many sacrifices.
\end{itemize}

michael barbaro

In the United States, Ohio and Louisiana became the latest states to
instruct residents to stay at home as infections in each state surged.

\begin{itemize}
\tightlist
\item
  archived recording (mike dewine)\\
  Other states have referred to this as shelter in place. We prefer stay
  at home. Either one, it's pretty much the same thing.
\end{itemize}

michael barbaro

In Washington, negotiations over a \$2 trillion stimulus package
designed to protect businesses and workers hurt by the pandemic reached
an impasse in the Senate.

\begin{itemize}
\tightlist
\item
  archived recording (joe manchin iii)\\
  The proposal that Leader McConnell from the Republican side has put
  forth is absolutely totally worried about Wall Street at this time.
  I'm worried about the people in little rural West Virginia and all
  over Main Street. That's the people we're worried about.
\end{itemize}

michael barbaro

On Sunday, Senate Democrats blocked the stimulus bill, saying it favors
big business and does not contain enough protections for workers by
allowing companies to fire workers even after receiving federal
bailouts.

\begin{itemize}
\tightlist
\item
  archived recording (joe manchin iii)\\
  And Wall Street's going to do just fine. It's always rebounded real
  well. They've always come back strong.
\end{itemize}

michael barbaro

Several Senate Republicans failed to cast votes because they are
self-quarantining over fears that they may have been exposed to the
coronavirus. And at least one senator, Republican Rand Paul of Kentucky,
has now tested positive for the virus.

The Times is providing free access to our most important updates on the
pandemic. To read it, go to nytimes.com/coronavirus.

{[}music{]}

That's it for ``The Daily.'' I'm Michael Barbaro. See you tomorrow.

Previous

More episodes ofThe Daily

\href{https://www.nytimes.com/2020/07/31/podcasts/the-daily/vanessa-guillen-military-metoo.html?action=click\&module=audio-series-bar\&region=header\&pgtype=Article}{\includegraphics{https://static01.nyt.com/images/2020/07/12/us/politics/31daily/00dc-army-metoo-thumbLarge.jpg}}

July 31, 2020A \#MeToo Moment in the Military

\href{https://www.nytimes.com/2020/07/30/podcasts/the-daily/congress-facebook-amazon-google-apple.html?action=click\&module=audio-series-bar\&region=header\&pgtype=Article}{\includegraphics{https://static01.nyt.com/images/2020/07/30/reader-center/30daily/merlin_175077825_5ebc931b-baa1-489a-960c-34e4d845e997-thumbLarge.jpg}}

July 30, 2020The Big Tech Hearing

\href{https://www.nytimes.com/2020/07/29/podcasts/the-daily/china-trump-foreign-policy.html?action=click\&module=audio-series-bar\&region=header\&pgtype=Article}{\includegraphics{https://static01.nyt.com/images/2020/07/26/world/29daily/00china-us-clash1-thumbLarge.jpg}}

July 29, 2020~~•~ 28:40Confronting China

\href{https://www.nytimes.com/2020/07/28/podcasts/the-daily/unemployment-benefits-coronavirus.html?action=click\&module=audio-series-bar\&region=header\&pgtype=Article}{\includegraphics{https://static01.nyt.com/images/2020/07/23/business/28daily/23virus-uiexplain1-thumbLarge.jpg}}

July 28, 2020~~•~ 26:13Why \$600 Checks Are Tearing Republicans Apart

\href{https://www.nytimes.com/2020/07/27/podcasts/the-daily/new-york-hospitals-covid.html?action=click\&module=audio-series-bar\&region=header\&pgtype=Article}{\includegraphics{https://static01.nyt.com/images/2020/07/27/world/27daily-hospitals/27daily-hospitals-thumbLarge.jpg}}

July 27, 2020~~•~ 33:28The Mistakes New York Made

\href{https://www.nytimes.com/2020/07/26/podcasts/the-daily/the-accusation-the-sunday-read.html?action=click\&module=audio-series-bar\&region=header\&pgtype=Article}{\includegraphics{https://static01.nyt.com/images/2020/03/22/magazine/26audm-2/22mag-titleix-thumbLarge.jpg}}

July 26, 2020The Sunday Read: `The Accusation'

\href{https://www.nytimes.com/2020/07/24/podcasts/the-daily/mlb-baseball-season-coronavirus.html?action=click\&module=audio-series-bar\&region=header\&pgtype=Article}{\includegraphics{https://static01.nyt.com/images/2020/07/22/sports/24daily/22mlb-previewlede1-thumbLarge.jpg}}

July 24, 2020~~•~ 45:34The Battle for a Baseball Season

\href{https://www.nytimes.com/2020/07/23/podcasts/the-daily/portland-protests.html?action=click\&module=audio-series-bar\&region=header\&pgtype=Article}{\includegraphics{https://static01.nyt.com/images/2020/07/22/us/23daily-image/22portland-tactics02-thumbLarge.jpg}}

July 23, 2020~~•~ 30:04The Showdown in Portland

\href{https://www.nytimes.com/2020/07/22/podcasts/the-daily/school-reopenings-coronavirus.html?action=click\&module=audio-series-bar\&region=header\&pgtype=Article}{\includegraphics{https://static01.nyt.com/images/2020/07/12/science/22daily/00virus-schools-reopen01-thumbLarge.jpg}}

July 22, 2020~~•~ 27:24The Science of School Reopenings

\href{https://www.nytimes.com/2020/07/21/podcasts/the-daily/coronavirus-vaccine.html?action=click\&module=audio-series-bar\&region=header\&pgtype=Article}{\includegraphics{https://static01.nyt.com/images/2020/07/19/science/21daily/00VIRUS-VAX-DOUBTS1-thumbLarge.jpg}}

July 21, 2020~~•~ 29:14The Vaccine Trust Problem

\href{https://www.nytimes.com/2020/07/20/podcasts/the-daily/john-lewis.html?action=click\&module=audio-series-bar\&region=header\&pgtype=Article}{\includegraphics{https://static01.nyt.com/images/2020/01/07/obituaries/20thedaily_lewis/00Lewis-John13-thumbLarge.jpg}}

July 20, 2020~~•~ 38:56The Life and Legacy of John Lewis

\href{https://www.nytimes.com/2020/07/19/podcasts/the-daily/lottery-winner-scam.html?action=click\&module=audio-series-bar\&region=header\&pgtype=Article}{\includegraphics{https://static01.nyt.com/images/2018/05/05/magazine/31audm-image/05mag-lottery-image1-thumbLarge-v4.png}}

July 19, 2020~~•~ 45:27The Sunday Read: `The Man Who Cracked the
Lottery'

\href{https://www.nytimes.com/column/the-daily}{See All Episodes ofThe
Daily}

Next

March 23, 2020

\begin{itemize}
\item
\item
\item
\item
\item
\item
\end{itemize}

\emph{\textbf{Listen and subscribe to our podcast from your mobile
device:}}\\
\textbf{\href{https://itunes.apple.com/us/podcast/the-daily/id1200361736?mt=2}{\emph{Via
Apple Podcasts}}} \emph{\textbf{\textbar{}}}
\textbf{\href{https://open.spotify.com/show/3IM0lmZxpFAY7CwMuv9H4g?si=SfuMSC55R1qprFsRZU3_zw}{\emph{Via
Spotify}}} \emph{\textbf{\textbar{}}}
\textbf{\href{http://www.stitcher.com/podcast/the-new-york-times/the-daily-10}{\emph{Via
Stitcher}}}

Two weeks ago, the biggest story in the country was the race for the
Democratic presidential nomination. Now, with the dramatic onset of the
coronavirus crisis, the primary has largely gone off the radar. Today,
we talk to Alexander Burns, a political reporter at The New York Times,
about what happened when those two stories collided.

\includegraphics{https://static01.nyt.com/images/2020/03/15/us/politics/15debate-takeaways1/15debate-ledeall1-videoSixteenByNine3000.jpg}

\textbf{Background reading:}

\begin{itemize}
\item
  In
  \href{https://www.nytimes.com/2020/03/15/us/politics/biden-sanders-debate-recap.html}{a
  presidential debate without an in-person audience} earlier this month,
  former Vice President Joseph R. Biden Jr. and Senator Bernie Sanders
  clashed over how to handle the coronavirus crisis. With so much news,
  you may have missed the debate ---
  \href{https://www.nytimes.com/2020/03/16/us/politics/takeaways-march-democratic-debate.html}{here
  are six takeaways} to catch you up.
\item
  Mr. Sanders is now
  \href{https://www.nytimes.com/2020/03/21/us/politics/biden-sanders-coronavirus.html?searchResultPosition=9}{reassessing
  his campaign} as Mr. Biden plans for the nomination,
  \href{https://www.nytimes.com/2020/03/16/us/politics/joe-biden-vp-running-mate.html?searchResultPosition=10}{announcing
  that he will pick a woman} as his running mate should he be chosen as
  the candidate.
\end{itemize}

\emph{Tune in, and tell us what you think. Email us at}
\href{mailto:thedaily@nytimes.com}{\emph{thedaily@nytimes.com}}\emph{.
Follow Michael Barbaro on Twitter:}
\href{https://twitter.com/mikiebarb}{\emph{@mikiebarb}}\emph{. And if
you're interested in advertising with ``The Daily,'' write to us at}
\href{mailto:thedaily-ads@nytimes.com}{\emph{thedaily-ads@nytimes.com}}\emph{.}

Alexander Burns contributed reporting.

``The Daily'' is made by Theo Balcomb, Andy Mills, Lisa Tobin, Rachel
Quester, Lynsea Garrison, Annie Brown, Clare Toeniskoetter, Paige
Cowett, Michael Simon Johnson, Brad Fisher, Larissa Anderson, Wendy
Dorr, Chris Wood, Jessica Cheung, Alexandra Leigh Young, Jonathan Wolfe,
Lisa Chow, Eric Krupke, Marc Georges, Luke Vander Ploeg, Adizah Eghan,
Kelly Prime, Julia Longoria, Sindhu Gnanasambandan, Jazmín Aguilera,
M.J. Davis Lin, Austin Mitchell, Sayre Quevedo, Neena Pathak, Dan
Powell, Dave Shaw, Sydney Harper, Daniel Guillemette, Hans Buetow,
Robert Jimison and Mike Benoist. Our theme music is by Jim Brunberg and
Ben Landsverk of Wonderly. Special thanks to Sam Dolnick, Mikayla
Bouchard, Stella Tan, Lauren Jackson, Julia Simon, Mahima Chablani and
Nora Keller.

Advertisement

\protect\hyperlink{after-bottom}{Continue reading the main story}

\hypertarget{site-index}{%
\subsection{Site Index}\label{site-index}}

\hypertarget{site-information-navigation}{%
\subsection{Site Information
Navigation}\label{site-information-navigation}}

\begin{itemize}
\tightlist
\item
  \href{https://help.nytimes.com/hc/en-us/articles/115014792127-Copyright-notice}{©~2020~The
  New York Times Company}
\end{itemize}

\begin{itemize}
\tightlist
\item
  \href{https://www.nytco.com/}{NYTCo}
\item
  \href{https://help.nytimes.com/hc/en-us/articles/115015385887-Contact-Us}{Contact
  Us}
\item
  \href{https://www.nytco.com/careers/}{Work with us}
\item
  \href{https://nytmediakit.com/}{Advertise}
\item
  \href{http://www.tbrandstudio.com/}{T Brand Studio}
\item
  \href{https://www.nytimes.com/privacy/cookie-policy\#how-do-i-manage-trackers}{Your
  Ad Choices}
\item
  \href{https://www.nytimes.com/privacy}{Privacy}
\item
  \href{https://help.nytimes.com/hc/en-us/articles/115014893428-Terms-of-service}{Terms
  of Service}
\item
  \href{https://help.nytimes.com/hc/en-us/articles/115014893968-Terms-of-sale}{Terms
  of Sale}
\item
  \href{https://spiderbites.nytimes.com}{Site Map}
\item
  \href{https://help.nytimes.com/hc/en-us}{Help}
\item
  \href{https://www.nytimes.com/subscription?campaignId=37WXW}{Subscriptions}
\end{itemize}
