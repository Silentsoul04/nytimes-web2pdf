Sections

SEARCH

\protect\hyperlink{site-content}{Skip to
content}\protect\hyperlink{site-index}{Skip to site index}

\href{https://myaccount.nytimes.com/auth/login?response_type=cookie\&client_id=vi}{}

\href{https://www.nytimes.com/section/todayspaper}{Today's Paper}

\href{/section/opinion}{Opinion}\textbar{}Falling Ill, Testing Negative

\href{https://nyti.ms/2woenIU}{https://nyti.ms/2woenIU}

\begin{itemize}
\item
\item
\item
\item
\item
\item
\end{itemize}

Advertisement

\protect\hyperlink{after-top}{Continue reading the main story}

\href{/section/opinion}{Opinion}

Supported by

\protect\hyperlink{after-sponsor}{Continue reading the main story}

\hypertarget{falling-ill-testing-negative}{%
\section{Falling Ill, Testing
Negative}\label{falling-ill-testing-negative}}

I had the symptoms. But did I have the coronavirus?

\href{https://www.nytimes.com/by/ross-douthat}{\includegraphics{https://static01.nyt.com/images/2018/04/03/opinion/ross-douthat/ross-douthat-thumbLarge.png}}

By \href{https://www.nytimes.com/by/ross-douthat}{Ross Douthat}

Opinion Columnist

\begin{itemize}
\item
  March 24, 2020
\item
  \begin{itemize}
  \item
  \item
  \item
  \item
  \item
  \item
  \end{itemize}
\end{itemize}

\includegraphics{https://static01.nyt.com/images/2020/03/23/opinion/23douthatWeb/23douthatWeb-articleLarge.jpg?quality=75\&auto=webp\&disable=upscale}

As of this writing, the United States has tested approximately 313,000
people for the coronavirus, and more than 270,000 have tested negative.
I'm one of them. Here's my story, offered without a definite conclusion.

I traveled a lot in the weeks before America went into lockdown,
promoting a
\href{https://www.simonandschuster.com/books/The-Decadent-Society/Ross-Douthat/9781476785240}{book}
about (aah, irony) the decadence of the developed world. I was in New
York, Washington, Boston, Los Angeles --- then home to Connecticut, then
back to New York and Washington again.

I thought of myself as woke to the coronavirus: I had followed reports
from Wuhan via grainy Chinese videos and fringe alarmist Twitter, warned
skeptical relatives to stock up and prepare to bunker down, and filled
our basement shelves with rice and beans, paper towels, the works.

But I also felt, a bit idiotically, that if I was savvy enough I could
stay one step ahead of the virus --- giving up handshakes early,
carrying Purell everywhere, projecting from the early case numbers to
figure out how long I could safely travel, and when the virus would
explode and the country would shut down.

My shutdown prediction was correct: I got home and started canceling
future book events just before the lockdowns started. But the day after
my return I felt achy and strange, and the following morning I woke up
with a dry cough, tightness in my chest and pain across my lungs.

I went to the emergency room, where the doctors told me that my symptoms
and travel history made them presume I had the virus, but that I wasn't
sick enough for them to test. They told me to self-quarantine for two
weeks, or at least until drive-through testing centers opened, and stay
away from my (eight-months pregnant) wife and kids as much as possible.

Within the same day, though, two of our kids were sick as well, with
hacking chest coughs, mild fevers and congestion. My wife had a dry
cough and body aches. So we quarantined as a family. I tried to write to
everyone I'd encountered in the previous week to let them know I was a
suspected case. And we tried to figure out how to get a test.

Over the next several days my lung pain got worse, though I never ran a
fever. I would feel short of breath after reading to the kids and
lightheaded after getting up. Talking on the phone was like running a
race. I have had one serious illness in my life and innumerable colds
and flus; none of my symptoms resembled any of those past experiences.

Three days after the E.R. visit I managed to get a doctor's script to
test myself and my 4-year-old son (the youngest and sickest of our kids)
at the Waterbury Hospital, then the only open drive-through center in
the state. It was a surreal episode, a science-fiction scene dropped
down in a faded industrial town, with space-suited nurses and masked
doctors directing traffic while unmasked construction workers hung out
casually nearby. We rolled down our windows, they swabbed each of our
noses once, promised results in three days, and sent us on our way.

Then we waited. Family and neighbors delivered us groceries, leaving
them on the front porch like gifts from kindly elves. The kids had a few
bad nights, then started to improve. We tried to take walks in the
neighborhood (the E.R. doctors had recommended it), but quickly found
that the narrow sidewalks required us to constantly circle away from our
neighbors, which required shouted explanations that provoked bemusement
in some cases, fright in others. So we drove instead, looking for
deserted corners of state parks, an empty greensward near a monastery,
anywhere with grass and air and little chance of human contact.

Five days went by with no test results. My symptoms stabilized,
fluctuated and then ebbed a little; my wife's mostly went away. We had
friends in Minnesota who were having a similar experience: Their family
had been on a Disney cruise just before the lockdowns (they are True
Americans) and come back with an illness; it seemed like a flu for most
of them, but the husband, a man of very different physique and
temperament from me, had my symptoms --- shortness of breath, chest
tension, windedness.

Finally, we received my results; the sample had apparently been sent to
the wrong lab and the lab had called the wrong doctor's office to report
them. The test was negative. Trying to explain my symptoms, our doctor
speculated about flus that cause asthmatic attacks in otherwise healthy
people. But she also noted that plenty of infected people can have
negative tests from a single nose swab. (In one
\href{https://jamanetwork.com/journals/jama/fullarticle/2762997}{study}
of Chinese patients, the nose swab detected only about 60 percent of
coronavirus cases.)

The next day our friends in Minnesota got the husband's results. They
were negative as well.

My son's test was delayed --- another day, they said. That was three
days ago, and yesterday, as I was reading the final edit of this column,
our doctor called with the news that they were apparently unable to
complete his test because they needed to redo part of it, and they had
insufficient material from the initial swab.

So that apparently concludes our testing experience. For our family
quarantine, it's been almost the full 14 days. I feel better, though
there are still flashes of chest pain and discomfort. My wife seems fine
now. The kids have what amounts to the remains of a cold, nothing
frightening any longer. Whether we had it or not, we appear to be coming
through OK.

\emph{{[}}\href{https://www.nytimes.com/column/the-argument}{\emph{Listen
to ``The Argument'' podcast every Thursday morning, with Ross Douthat,
Michelle Goldberg and David Leonhardt.}}\emph{{]}}

So did we have it? There are three possibilities. The first is that on
my travels I acquired a different virus, one we shared throughout our
family, that happened to mimic some of the crucial symptoms of the
coronavirus during the exact moment the outbreak accelerated.

The second is that we all just had a normal flu, and there is some kind
of mass psychology during pandemics that makes people who fall sick with
other illnesses experience some kind of sympathetic symptomology that
mirrors the more dangerous disease.

The third possibility is that my negative results were wrong, and my
son's test would have been positive if the testing weren't incompetent.

From our family's perspective I hope it's the third case; it would mean
that we've been through the Thing Itself, hopefully acquired some kind
of immunity, and can breathe a little easier as we approach the birth of
our child.

From the country's perspective, on the other hand, it would be better if
we didn't have it, because it would be bad news for all our containment
efforts if false negatives were plentiful.

But since we can't know, my family will be exiting our ``do we have the
coronavirus?'' experience without answers, and entering back into the
same uncertainty as everybody else.

\emph{The Times is committed to publishing}
\href{https://www.nytimes.com/2019/01/31/opinion/letters/letters-to-editor-new-york-times-women.html}{\emph{a
diversity of letters}} \emph{to the editor. We'd like to hear what you
think about this or any of our articles. Here are some}
\href{https://help.nytimes.com/hc/en-us/articles/115014925288-How-to-submit-a-letter-to-the-editor}{\emph{tips}}\emph{.
And here's our email:}
\href{mailto:letters@nytimes.com}{\emph{letters@nytimes.com}}\emph{.}

\emph{Follow The New York Times Opinion section on}
\href{https://www.facebook.com/nytopinion}{\emph{Facebook}}\emph{,}
\href{http://twitter.com/NYTOpinion}{\emph{Twitter (@NYTOpinion)}}
\emph{and}
\href{https://www.instagram.com/nytopinion/}{\emph{Instagram}}\emph{,
join the Facebook political discussion group,}
\href{https://www.facebook.com/groups/votingwhilefemale/}{\emph{Voting
While Female}}\emph{.}

Advertisement

\protect\hyperlink{after-bottom}{Continue reading the main story}

\hypertarget{site-index}{%
\subsection{Site Index}\label{site-index}}

\hypertarget{site-information-navigation}{%
\subsection{Site Information
Navigation}\label{site-information-navigation}}

\begin{itemize}
\tightlist
\item
  \href{https://help.nytimes.com/hc/en-us/articles/115014792127-Copyright-notice}{©~2020~The
  New York Times Company}
\end{itemize}

\begin{itemize}
\tightlist
\item
  \href{https://www.nytco.com/}{NYTCo}
\item
  \href{https://help.nytimes.com/hc/en-us/articles/115015385887-Contact-Us}{Contact
  Us}
\item
  \href{https://www.nytco.com/careers/}{Work with us}
\item
  \href{https://nytmediakit.com/}{Advertise}
\item
  \href{http://www.tbrandstudio.com/}{T Brand Studio}
\item
  \href{https://www.nytimes.com/privacy/cookie-policy\#how-do-i-manage-trackers}{Your
  Ad Choices}
\item
  \href{https://www.nytimes.com/privacy}{Privacy}
\item
  \href{https://help.nytimes.com/hc/en-us/articles/115014893428-Terms-of-service}{Terms
  of Service}
\item
  \href{https://help.nytimes.com/hc/en-us/articles/115014893968-Terms-of-sale}{Terms
  of Sale}
\item
  \href{https://spiderbites.nytimes.com}{Site Map}
\item
  \href{https://help.nytimes.com/hc/en-us}{Help}
\item
  \href{https://www.nytimes.com/subscription?campaignId=37WXW}{Subscriptions}
\end{itemize}
