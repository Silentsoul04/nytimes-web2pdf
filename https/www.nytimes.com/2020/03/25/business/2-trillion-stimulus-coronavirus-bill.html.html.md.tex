Sections

SEARCH

\protect\hyperlink{site-content}{Skip to
content}\protect\hyperlink{site-index}{Skip to site index}

\href{https://www.nytimes.com/section/business}{Business}

\href{https://myaccount.nytimes.com/auth/login?response_type=cookie\&client_id=vi}{}

\href{https://www.nytimes.com/section/todayspaper}{Today's Paper}

\href{/section/business}{Business}\textbar{}A \$2 Trillion Lifeline Will
Help, but More May Be Needed

\url{https://nyti.ms/33KdmHb}

\begin{itemize}
\item
\item
\item
\item
\item
\item
\end{itemize}

\href{https://www.nytimes.com/news-event/coronavirus?action=click\&pgtype=Article\&state=default\&region=TOP_BANNER\&context=storylines_menu}{The
Coronavirus Outbreak}

\begin{itemize}
\tightlist
\item
  live\href{https://www.nytimes.com/2020/08/01/world/coronavirus-covid-19.html?action=click\&pgtype=Article\&state=default\&region=TOP_BANNER\&context=storylines_menu}{Latest
  Updates}
\item
  \href{https://www.nytimes.com/interactive/2020/us/coronavirus-us-cases.html?action=click\&pgtype=Article\&state=default\&region=TOP_BANNER\&context=storylines_menu}{Maps
  and Cases}
\item
  \href{https://www.nytimes.com/interactive/2020/science/coronavirus-vaccine-tracker.html?action=click\&pgtype=Article\&state=default\&region=TOP_BANNER\&context=storylines_menu}{Vaccine
  Tracker}
\item
  \href{https://www.nytimes.com/interactive/2020/07/29/us/schools-reopening-coronavirus.html?action=click\&pgtype=Article\&state=default\&region=TOP_BANNER\&context=storylines_menu}{What
  School May Look Like}
\item
  \href{https://www.nytimes.com/live/2020/07/31/business/stock-market-today-coronavirus?action=click\&pgtype=Article\&state=default\&region=TOP_BANNER\&context=storylines_menu}{Economy}
\end{itemize}

Advertisement

\protect\hyperlink{after-top}{Continue reading the main story}

Supported by

\protect\hyperlink{after-sponsor}{Continue reading the main story}

News analysis

\hypertarget{a-2-trillion-lifeline-will-help-but-more-may-be-needed}{%
\section{A \$2 Trillion Lifeline Will Help, but More May Be
Needed}\label{a-2-trillion-lifeline-will-help-but-more-may-be-needed}}

The bill moving through Congress is more than twice as large as the
stimulus package passed in 2009, but it will soothe a shutdown economy
for only a few months.

\includegraphics{https://static01.nyt.com/images/2020/03/25/world/25dc-virus-impact01/merlin_170909976_fe18c90e-b1ce-408b-89d2-d73670255267-articleLarge.jpg?quality=75\&auto=webp\&disable=upscale}

\href{https://www.nytimes.com/by/jim-tankersley}{\includegraphics{https://static01.nyt.com/images/2018/10/19/multimedia/author-jim-tankersley/author-jim-tankersley-thumbLarge.png}}

By \href{https://www.nytimes.com/by/jim-tankersley}{Jim Tankersley}

\begin{itemize}
\item
  Published March 25, 2020Updated April 15, 2020
\item
  \begin{itemize}
  \item
  \item
  \item
  \item
  \item
  \item
  \end{itemize}
\end{itemize}

WASHINGTON --- If you want to
\href{https://www.nytimes.com/2020/03/22/us/politics/coronavirus-economy-shutdown.html}{shut
down an economy} to fight a pandemic without driving millions of people
and businesses into bankruptcy, you need the
\href{https://www.nytimes.com/2020/04/15/business/coronavirus-stimulus-money.html}{government}
to cut some checks. The
\href{https://www.nytimes.com/2020/04/15/business/coronavirus-stimulus-money.html}{coronavirus}
response deal that
\href{https://www.nytimes.com/2020/03/25/us/politics/coronavirus-senate-deal.html}{the
Senate passed late Wednesday} will get a lot of checks into the mail,
but it will soothe only a few months of financial pain.

If the outbreak and the disruptions continue through summer, lawmakers
will need to spend even more.

The bill, a compromise between the Trump administration and Republican
and Democratic leaders in Congress, includes loans and grants for
corporations and small businesses, increased unemployment benefits for
workers laid off or working fewer hours amid the outbreak, and direct
payments to low- and middle-income individuals and families. Negotiators
estimate its cost at \$2 trillion.

Taken together, those measures form a novel, temporary expansion of the
federal government's role in the economy: It will be essentially paying
millions of Americans not to work, and thousands of businesses not to
shut down even if they have no customers, in order to slow the spread of
the pandemic. Its cost is more than double the roughly \$800 billion
stimulus package that Congress passed in 2009 to ease the Great
Recession. Yet it still may not be large enough, given the enormous
economic challenge the United States faces today.

The economy, which has been shuttered to control the spread of the
virus, does not need a jolt to get moving again. The government is just
trying to tide people and firms over until it is safe to
\href{https://www.nytimes.com/2020/03/24/business/economy/coronavirus-economy.html}{start
back up}.

Viewed through that particular set of circumstances, the deal was not
economic stimulus at all. It was a series of survival payments. And
those payments will only last a few months.

How quickly those payments find their way to households and businesses
will be critical. Prospects for swift passage dimmed on Wednesday
afternoon, when three Republican senators raised concerns over the
generosity of the enhanced unemployment benefits. In a best-case
scenario where Mr. Trump signed the law on Thursday, people close to the
negotiations said, dollars could flow to small businesses as soon as
next week. Many business have little time to spare: The typical small
business carries only enough cash to last for 12 days without new
revenues,
\href{https://institute.jpmorganchase.com/content/dam/jpmc/jpmorgan-chase-and-co/institute/pdf/institute-growth-vitality-cash-flows.pdf}{according
to research} from the JPMorgan Chase Institute.

\hypertarget{latest-updates-economy}{%
\section{\texorpdfstring{\href{https://www.nytimes.com/live/2020/07/31/business/stock-market-today-coronavirus?action=click\&pgtype=Article\&state=default\&region=MAIN_CONTENT_1\&context=storylines_live_updates}{Latest
Updates:
Economy}}{Latest Updates: Economy}}\label{latest-updates-economy}}

\href{https://www.nytimes.com/live/2020/07/31/business/stock-market-today-coronavirus?action=click\&pgtype=Article\&state=default\&region=MAIN_CONTENT_1\&context=storylines_live_updates\#kodaks-chief-executive-was-given-stock-options-then-the-share-price-spiked-1000-percent}{34h
ago}

\href{https://www.nytimes.com/live/2020/07/31/business/stock-market-today-coronavirus?action=click\&pgtype=Article\&state=default\&region=MAIN_CONTENT_1\&context=storylines_live_updates\#kodaks-chief-executive-was-given-stock-options-then-the-share-price-spiked-1000-percent}{Kodak's
chief executive was given stock options. Then the share price spiked
1,000 percent.}

\href{https://www.nytimes.com/live/2020/07/31/business/stock-market-today-coronavirus?action=click\&pgtype=Article\&state=default\&region=MAIN_CONTENT_1\&context=storylines_live_updates\#fitch-ratings-downgrades-its-outlook-on-us-debt}{37h
ago}

\href{https://www.nytimes.com/live/2020/07/31/business/stock-market-today-coronavirus?action=click\&pgtype=Article\&state=default\&region=MAIN_CONTENT_1\&context=storylines_live_updates\#fitch-ratings-downgrades-its-outlook-on-us-debt}{Fitch
Ratings downgrades its outlook on U.S. debt.}

\href{https://www.nytimes.com/live/2020/07/31/business/stock-market-today-coronavirus?action=click\&pgtype=Article\&state=default\&region=MAIN_CONTENT_1\&context=storylines_live_updates\#us-sanctions-more-chinese-officials-over-human-rights-violations-as-tensions-flare}{44h
ago}

\href{https://www.nytimes.com/live/2020/07/31/business/stock-market-today-coronavirus?action=click\&pgtype=Article\&state=default\&region=MAIN_CONTENT_1\&context=storylines_live_updates\#us-sanctions-more-chinese-officials-over-human-rights-violations-as-tensions-flare}{U.S.
sanctions more Chinese officials over human rights violations as
tensions flare}

\href{https://www.nytimes.com/live/2020/07/31/business/stock-market-today-coronavirus?action=click\&pgtype=Article\&state=default\&region=MAIN_CONTENT_1\&context=storylines_live_updates}{See
more updates}

More live coverage:
\href{https://www.nytimes.com/2020/08/01/world/coronavirus-covid-19.html?action=click\&pgtype=Article\&state=default\&region=MAIN_CONTENT_1\&context=storylines_live_updates}{Global}

``Already balance sheets are running red,'' a group of nearly 900
economists, including several Nobel Prize winners,
\href{http://www.columbia.edu/~wk2110/Corona/Statement.html}{wrote this
week} in a letter urging quick congressional action. ``Businesses that
fail during this necessary stoppage time will see the jobs that they
provided disappear. With them, much of the productive capacity of the
economy will be destroyed.''

The speed of payments to households will also depend in large part on
whether individuals have bank accounts: The Treasury Department is
expected to begin directly depositing checks within a few weeks of the
bill's passing, but mailed payments will take one or two weeks longer,
Republican Senate aides said Wednesday.

Mr. Trump said Tuesday that he hoped the economy will be
\href{https://www.nytimes.com/2020/03/24/us/politics/trump-coronavirus-easter.html}{``reopened''
by Easter}, in two and a half weeks. Public health experts and a
\href{https://economicstrategygroup.org/resource/economic-strategy-group-statement-covid19/}{wide
range} of
\href{http://www.columbia.edu/~wk2110/Corona/Statement.html}{economists
say} that is both unlikely and inadvisable. The country still lacks
widespread testing for the virus, and confirmed infections and deaths
\href{https://www.nytimes.com/interactive/2020/us/coronavirus-us-cases.html}{continue
to climb rapidly}.

The extraordinary measures that mayors and governors have taken to
restrict economic activity, which at their most extreme include shutting
down all nonessential businesses and ordering people to shelter in the
homes, are unlikely to show success in ``bending the curve'' of the
virus for at least another week. If they prove effective, and the
infection rate slows dramatically, activity could be back to normal ---
or at least something that reasonably resembles it --- within a few
months for many businesses and workers.

\includegraphics{https://static01.nyt.com/images/2020/03/25/world/25dc-virus-impact02/25dc-virus-impact02-articleLarge-v2.jpg?quality=75\&auto=webp\&disable=upscale}

If the measures do not prove effective, or if they are relaxed under
orders from Mr. Trump or defied en masse, experts warn the crisis could
stretch much longer, under the growing cloud of a recession. That's why
it's hard to say if the congressional deal will be enough to keep
families from going hungry and businesses from going under.

On Wednesday, Treasury Secretary Steven Mnuchin suggested that the
package Congress was expected to pass would be more than enough money to
get the economy over the hump.

``I would say we've anticipated three months,'' Mr. Mnuchin said,
referring to the amount of time the economy might need extra support.
``Hopefully we won't need this for three months. Hopefully this war will
be won quicker, but we expect that this is a significant amount of money
if needed to cover the economy.''

Still, economists hailed the emerging agreement as a good start --- one
that works on multiple fronts to keep money flowing through the parts of
the economy that have been suddenly rendered inactive.

``The response looks to be proportionate to the extent of the problem,''
said Justin Wolfers, a University of Michigan economist who has pushed
for a large fiscal response to sustain the economy through the virus
shutdown. But, he said, ``we have no idea what the extent of the problem
is.''

The bill includes \$350 billion in loans for small businesses to help
bridge their expenses for up to 10 weeks. Firms would not need to repay
up to eight weeks of the loans if they refrain from laying off
employees, or move by June to rehire employees they have already laid
off. Supporters of the measure say those loans, if rapidly deployed,
could help thousands of firms survive, at least temporarily.

``It is incredibly important that policymakers credibly convince
business owners that these conditional loans will indeed be forgiven and
that firms' owners will be treated equitably,'' said Stan Veuger, an
economist at the conservative American Enterprise Institute. But, he
said, ``I am skeptical that the size of the package is large enough to
cover the entire shutdown-slowdown period.''

The bill also includes \$500 billion in aid to airlines and other large
corporations that have been hurt by a cratering of consumer demand amid
the crisis. Much of the money would be used to backstop loans and other
assistance that the
\href{https://www.nytimes.com/2020/03/23/business/economy/coronavirus-fed-bond-buying.html}{Federal
Reserve said it plans to extend} to companies.

Those programs are in part meant to encourage companies to keep workers
on their payrolls. Even if workers are furloughed without pay, the
government will essentially step in and assume paying their salaries
while the workers continue to be covered by any health insurance
provided by their employers.

For workers who lose their jobs, the bill supplies expanded unemployment
benefits for up to four months. For many, those payments will match or
even exceed the wages they were earning before the outbreak.

Image

The bill also includes \$500 billion in aid to airlines and other large
corporations that have been hurt by a cratering of consumer demand.
Credit...Nick Oxford/Reuters

The bill also includes a \$1,200 payment for each adult --- and \$500
per child --- in households that earn up to \$75,000 per year for
individuals or \$150,000 for couples. The assistance phases out for
people who earn more.

Neither Republicans nor Democrats love the bill, which was the product
of frenzied negotiations punctuated by often bitter partisan anger. Some
liberal groups denounced it as a slush fund for corporations. Some
conservatives warned that the large amount of borrowed money it would
plow into the economy could stoke rampant inflation.

Business groups celebrated it as a late but necessary intervention, and
so did many lawmakers and policy advocates.

``Nothing is perfect around here,'' Senator Rob Portman, Republican of
Ohio, said in a Tuesday speech on the Senate floor. ``But if you make
perfect the enemy of the good, you're going to hurt more people, more
small businesses will shut, more people will be out on their own and
there will be more and more people who will be infected with this virus
who otherwise could have been saved.''

Jacob Leibenluft, a senior fellow at the liberal Center for American
Progress, said Congress ``will need much more over the coming months,
but the crucial thing the bill appears to do is begin providing relief
to families and communities through channels that can get it out
quickly, like expanded unemployment insurance, direct payments and state
aid.''

Policy experts and business lobbyists have been warning for days that
congressional failure to reach a deal was causing more companies to
shutter and workers to lose their jobs. Some said on Tuesday that
lawmakers needed to be ready to start work on another plan to avoid any
additional losses if the outbreak effects stretch into summer and fall.

``Much of the small business community is facing an extinction-level
event,'' said John Lettieri, the chief of the Economic Innovation Group
think tank in Washington, who pushed heavily for a package of small
business loans in the agreement. ``Will this bill help? Absolutely. But
the lending capacity needed to prevent mass closures and layoffs could
be four or five times larger than what is being provided.''

``Congress,'' Mr. Lettieri said, ``needs to be prepared now for how
quickly these resources are going to evaporate.''

Advertisement

\protect\hyperlink{after-bottom}{Continue reading the main story}

\hypertarget{site-index}{%
\subsection{Site Index}\label{site-index}}

\hypertarget{site-information-navigation}{%
\subsection{Site Information
Navigation}\label{site-information-navigation}}

\begin{itemize}
\tightlist
\item
  \href{https://help.nytimes.com/hc/en-us/articles/115014792127-Copyright-notice}{©~2020~The
  New York Times Company}
\end{itemize}

\begin{itemize}
\tightlist
\item
  \href{https://www.nytco.com/}{NYTCo}
\item
  \href{https://help.nytimes.com/hc/en-us/articles/115015385887-Contact-Us}{Contact
  Us}
\item
  \href{https://www.nytco.com/careers/}{Work with us}
\item
  \href{https://nytmediakit.com/}{Advertise}
\item
  \href{http://www.tbrandstudio.com/}{T Brand Studio}
\item
  \href{https://www.nytimes.com/privacy/cookie-policy\#how-do-i-manage-trackers}{Your
  Ad Choices}
\item
  \href{https://www.nytimes.com/privacy}{Privacy}
\item
  \href{https://help.nytimes.com/hc/en-us/articles/115014893428-Terms-of-service}{Terms
  of Service}
\item
  \href{https://help.nytimes.com/hc/en-us/articles/115014893968-Terms-of-sale}{Terms
  of Sale}
\item
  \href{https://spiderbites.nytimes.com}{Site Map}
\item
  \href{https://help.nytimes.com/hc/en-us}{Help}
\item
  \href{https://www.nytimes.com/subscription?campaignId=37WXW}{Subscriptions}
\end{itemize}
