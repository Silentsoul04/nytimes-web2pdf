Sections

SEARCH

\protect\hyperlink{site-content}{Skip to
content}\protect\hyperlink{site-index}{Skip to site index}

\href{https://myaccount.nytimes.com/auth/login?response_type=cookie\&client_id=vi}{}

\href{https://www.nytimes.com/section/todayspaper}{Today's Paper}

\href{/section/opinion}{Opinion}\textbar{}After the Lockdown, Fear and
Chaos in India

\url{https://nyti.ms/39cXXQQ}

\begin{itemize}
\item
\item
\item
\item
\item
\end{itemize}

Advertisement

\protect\hyperlink{after-top}{Continue reading the main story}

\href{/section/opinion}{Opinion}

Supported by

\protect\hyperlink{after-sponsor}{Continue reading the main story}

\hypertarget{after-the-lockdown-fear-and-chaos-in-india}{%
\section{After the Lockdown, Fear and Chaos in
India}\label{after-the-lockdown-fear-and-chaos-in-india}}

As panic about the virus escalates, the middle and upper classes are
hoarding food, supplies and medicines without a thought for the millions
of poor who stand to starve and die.

By Pragya Tiwari

Ms. Tiwari is a writer based in New Delhi.

\begin{itemize}
\item
  March 25, 2020
\item
  \begin{itemize}
  \item
  \item
  \item
  \item
  \item
  \end{itemize}
\end{itemize}

\includegraphics{https://static01.nyt.com/images/2020/03/26/opinion/26tiwari/26tiwari-articleLarge.jpg?quality=75\&auto=webp\&disable=upscale}

NEW DELHI --- On Tuesday evening, India's prime minister, Narendra Modi,
ordered a strict nationwide
\href{https://www.nytimes.com/2020/03/24/world/asia/india-coronavirus-lockdown.html}{lockdown
for the next 21 days}to battle the spread of the coronavirus.

The busy marketplace in my upscale South Delhi neighborhood is desolate
the next morning. Almost all shops are shuttered. The florist who
delivered exotic flowers to wealthy homes has abandoned his stock, and
the pungent smell of rotting flowers hangs heavy in the air. A pet store
has locked up and left the animals inside. Their muffled screams are
unbearable.

At the local chemist, two men are at each other's throats. A large
gray-haired man in a lawyer's robe is shouting expletives through his
mask as he towers over a short, scruffy domestic worker. The worker has
bought all the acetaminophen in the shop for his employers, and the
lawyer is having none of it. The scuffle between the two men seems like
an act of transgression --- not because it is violent but because it
involves freewheeling physical contact.

``Touch is curse,'' I was told by a man as he wheeled his stock of sweet
potatoes down deserted streets, defying the lockdown in the hope of
earning enough to buy food for his family. He offered free sweet
potatoes to an old man in a tattered mask sweeping the road. The
sweeper, wary of infection, turned his offer down.

As always, the poor are the worst affected. As work began to dry up,
\href{https://scroll.in/article/957166/coronavirus-lockdown-narendra-modi-has-cut-indias-poor-adrift}{thousands
of migrant laborers}were forced to head back to their villages. Some of
them probably brought the virus with them to the villages that have
little access to basic infrastructure, including running water.

Trains and buses were discontinued overnight to put a stop to this.
Those who could not make it in time and can no longer afford to live in
the cities where they work are now
\href{https://www.ndtv.com/india-news/36-hours-80-km-long-walk-home-for-labourers-amid-coronavirus-lockdown-2200450}{attempting
to walk for hundreds of miles} to get home. Many of them have no money
left for food, and the few who do cannot find food anywhere.

Groups of beggars are abandoning their stations at traffic lights and
trying to enter gated colonies. Six of them arrived at our neighborhood
temple and furiously knocked its doors, asking for food.

In our wealthy neighborhood, even the gods have shut themselves in. But
the old order that insulated the wealthier classes from the distress
that India's poor endure is no longer holding. Like the hungry, disease
is bound to come knocking sooner or later.

My parents are old and frail, and live by themselves in Calcutta.
Whenever there has been an emergency, I have flown down within hours to
take care of them. We have the means to mitigate the physical distance
between us, but for the first time in our lives, our privilege counts
for less. Airports across India are shut. Indian states have scrambled
to seal their borders. I can no longer reach my parents. Late last night
I found myself wondering who would cremate them if I were to lose them.

The only world that I have known is one in which anxieties are reined in
by hope. The privilege of India's middle class is defined by its
investment in India's growth story, its lust for consumption and its
quest for a legacy. But India's
\href{https://www.bloomberg.com/news/features/2020-01-29/india-s-worst-economic-slowdown-in-a-decade}{economy
has been floundering,}and the fall in trade coupled with the lockdown
will bring it to its knees. It is unclear how many will still have jobs
and businesses by the end of the year.

And it is not just the economy that is struggling. A nationwide lockdown
strikes at the very roots of civilization --- museums, theaters,
cinemas, bookshops, schools, universities, libraries, playgrounds, are
all out of bounds. Print-newspaper readership is steeply declining amid
fears that papers can be carriers of the disease.

With just
\href{https://theprint.in/opinion/current-rate-india-30000-covid-19-deaths-may-no-hospital-bed-june-data/385386/}{one
bed for every 2,000 people}, Indian hospitals will be unable to
accommodate patients by the end of June at the current rate of growth of
cases, and as early as the end of April if the infection rate goes up.
The huge gulf between the health infrastructure available to India's
poor and to its wealthier classes is closing fast.

As fear escalates, the middle and upper classes trapped in their homes
are surrendering to their worst instincts --- hoarding food, supplies
and medicines without sparing a thought for the millions of poor who
stand to starve and die.

People from the northeast of India are being
\href{https://www.indiatoday.in/india/story/coronavirus-outbreak-in-india-northeast-racial-targeting-delhi-1657276-2020-03-19}{threatened
and abused} for being racially similar to the Chinese. Houses where
people have been quarantined are being marked by authorities, and
neighbors are circulating photographs of notices pasted on their doors
to encourage ostracizing them.

Perhaps worst of all, doctors are being evicted from their homes for
fear they are carriers. In a country that has only one doctor for every
1,404 people, discouraging doctors from doing their jobs is nothing
short of self-destruction.

This comes at a time when courts have closed and liberal democracy is
threatened by the excessive power the state takes on during a national
emergency. Even liberals are demanding that the police be tough on
people breaking curfews as panic about the virus spreads. And police
patrols are unleashing indiscriminate violence in the name of enforcing
the lockdown.

As democratic freedoms recede, superstition is tightening its grip. On
Sunday, Indians across the country stood on their balconies banging
utensils. This was ostensibly an exercise to thank health workers, but
rumors began to circulate that the sound vibrations under the particular
alignment of stars at the time will dispel the virus from earth.

Elsewhere,
\href{https://www.moneycontrol.com/news/india/create-an-idol-of-corona-seek-forgiveness-hindu-mahasabha-chief-advises-xi-jinping-4951821.html}{idols
of coronavirus}are being erected and fed to ``satisfy its appetite.'' A
security guard in our building was apprehended by the police for trying
to reach his village for a religious rite performed for the dead. He was
convinced that his family would be spared by the contagion if the souls
of his ancestors were pacified through the ritual.

Mr. Modi's lockdown is the largest confinement in human history, with
1.3 billion people abruptly shut in. Neither the system nor the
citizenry is prepared for it. No financial package has been declared by
the central government to help the poor who live off daily wages. There
is no plan in place for what millions of Indian farmers are to do at the
coming harvest time.

The enforcement of the lockdown has been extremely ham-handed. On paper,
groceries and medicines have been declared essential items that will
remain available, but in reality,
\href{https://scroll.in/article/957222/coronavirus-on-day-1-of-nationwide-lockdown-complaints-about-supply-bottlenecks-from-across-india}{supply
chains have been crippled} by overzealous policemen.

Outside city limits,
\href{https://www.ndtv.com/india-news/coronavirus-india-lockdown-e-tailers-complain-police-beating-up-delivery-agents-2200587}{milk
and vegetables are being dumped} from stranded trucks, while in the
cities, groups of angry men assaulted vendors selling vegetables from
pushcarts. There are reports of policemen soliciting bribes to allow
necessary movement, and breakdown of law and order in parts of the
country.

If this continues, the trust in institutions, and in one another, that
glues a society together is bound to come unstuck. Never before has the
fate of all human beings been so desperately interlinked, and yet we
seem to have been ushered into an unrecognizable world where it might
just be every person for himself.

Pragya Tiwari is a writer.

\emph{The Times is committed to publishing}
\href{https://www.nytimes.com/2019/01/31/opinion/letters/letters-to-editor-new-york-times-women.html}{\emph{a
diversity of letters}} \emph{to the editor. We'd like to hear what you
think about this or any of our articles. Here are some}
\href{https://help.nytimes.com/hc/en-us/articles/115014925288-How-to-submit-a-letter-to-the-editor}{\emph{tips}}\emph{.
And here's our email:}
\href{mailto:letters@nytimes.com}{\emph{letters@nytimes.com}}\emph{.}

\emph{Follow The New York Times Opinion section on}
\href{https://www.facebook.com/nytopinion}{\emph{Facebook}}\emph{,}
\href{http://twitter.com/NYTOpinion}{\emph{Twitter (@NYTopinion)}}
\emph{and}
\href{https://www.instagram.com/nytopinion/}{\emph{Instagram}}\emph{.}

Advertisement

\protect\hyperlink{after-bottom}{Continue reading the main story}

\hypertarget{site-index}{%
\subsection{Site Index}\label{site-index}}

\hypertarget{site-information-navigation}{%
\subsection{Site Information
Navigation}\label{site-information-navigation}}

\begin{itemize}
\tightlist
\item
  \href{https://help.nytimes.com/hc/en-us/articles/115014792127-Copyright-notice}{©~2020~The
  New York Times Company}
\end{itemize}

\begin{itemize}
\tightlist
\item
  \href{https://www.nytco.com/}{NYTCo}
\item
  \href{https://help.nytimes.com/hc/en-us/articles/115015385887-Contact-Us}{Contact
  Us}
\item
  \href{https://www.nytco.com/careers/}{Work with us}
\item
  \href{https://nytmediakit.com/}{Advertise}
\item
  \href{http://www.tbrandstudio.com/}{T Brand Studio}
\item
  \href{https://www.nytimes.com/privacy/cookie-policy\#how-do-i-manage-trackers}{Your
  Ad Choices}
\item
  \href{https://www.nytimes.com/privacy}{Privacy}
\item
  \href{https://help.nytimes.com/hc/en-us/articles/115014893428-Terms-of-service}{Terms
  of Service}
\item
  \href{https://help.nytimes.com/hc/en-us/articles/115014893968-Terms-of-sale}{Terms
  of Sale}
\item
  \href{https://spiderbites.nytimes.com}{Site Map}
\item
  \href{https://help.nytimes.com/hc/en-us}{Help}
\item
  \href{https://www.nytimes.com/subscription?campaignId=37WXW}{Subscriptions}
\end{itemize}
