Sections

SEARCH

\protect\hyperlink{site-content}{Skip to
content}\protect\hyperlink{site-index}{Skip to site index}

\href{https://www.nytimes.com/section/politics}{Politics}

\href{https://myaccount.nytimes.com/auth/login?response_type=cookie\&client_id=vi}{}

\href{https://www.nytimes.com/section/todayspaper}{Today's Paper}

\href{/section/politics}{Politics}\textbar{}Senate Approves \$2 Trillion
Stimulus After Bipartisan Deal

\url{https://nyti.ms/3bq5j54}

\begin{itemize}
\item
\item
\item
\item
\item
\item
\end{itemize}

\href{https://www.nytimes.com/news-event/coronavirus?action=click\&pgtype=Article\&state=default\&region=TOP_BANNER\&context=storylines_menu}{The
Coronavirus Outbreak}

\begin{itemize}
\tightlist
\item
  live\href{https://www.nytimes.com/2020/08/01/world/coronavirus-covid-19.html?action=click\&pgtype=Article\&state=default\&region=TOP_BANNER\&context=storylines_menu}{Latest
  Updates}
\item
  \href{https://www.nytimes.com/interactive/2020/us/coronavirus-us-cases.html?action=click\&pgtype=Article\&state=default\&region=TOP_BANNER\&context=storylines_menu}{Maps
  and Cases}
\item
  \href{https://www.nytimes.com/interactive/2020/science/coronavirus-vaccine-tracker.html?action=click\&pgtype=Article\&state=default\&region=TOP_BANNER\&context=storylines_menu}{Vaccine
  Tracker}
\item
  \href{https://www.nytimes.com/interactive/2020/07/29/us/schools-reopening-coronavirus.html?action=click\&pgtype=Article\&state=default\&region=TOP_BANNER\&context=storylines_menu}{What
  School May Look Like}
\item
  \href{https://www.nytimes.com/live/2020/07/31/business/stock-market-today-coronavirus?action=click\&pgtype=Article\&state=default\&region=TOP_BANNER\&context=storylines_menu}{Economy}
\end{itemize}

Advertisement

\protect\hyperlink{after-top}{Continue reading the main story}

Supported by

\protect\hyperlink{after-sponsor}{Continue reading the main story}

\hypertarget{senate-approves-2-trillion-stimulus-after-bipartisan-deal}{%
\section{Senate Approves \$2 Trillion Stimulus After Bipartisan
Deal}\label{senate-approves-2-trillion-stimulus-after-bipartisan-deal}}

The plan would provide direct payments to taxpayers, jobless benefits
and a \$500 billion fund to assist distressed businesses, with oversight
requirements demanded by Democrats.

\includegraphics{https://static01.nyt.com/images/2020/03/25/us/politics/25dc-virus-cong-new1/merlin_170959563_fdb99bdd-3fc1-477c-b046-cf7bb723eb40-articleLarge.jpg?quality=75\&auto=webp\&disable=upscale}

\href{https://www.nytimes.com/by/emily-cochrane}{\includegraphics{https://static01.nyt.com/images/2018/11/28/multimedia/author-emily-cochrane/author-emily-cochrane-thumbLarge-v3.png}}\href{https://www.nytimes.com/by/nicholas-fandos}{\includegraphics{https://static01.nyt.com/images/2018/11/06/multimedia/author-nicholas-fandos/author-nicholas-fandos-thumbLarge-v2.png}}

By \href{https://www.nytimes.com/by/emily-cochrane}{Emily Cochrane} and
\href{https://www.nytimes.com/by/nicholas-fandos}{Nicholas Fandos}

\begin{itemize}
\item
  Published March 25, 2020Updated May 5, 2020
\item
  \begin{itemize}
  \item
  \item
  \item
  \item
  \item
  \item
  \end{itemize}
\end{itemize}

WASHINGTON --- The \$2 trillion economic stabilization package agreed to
by Congress and the Trump administration early Wednesday morning is the
largest of its kind in modern American history, intended to respond to
the
c\href{https://www.nytimes.com/article/where-is-my-stimulus-payment.html}{oronavirus}
pandemic and provide direct payments and jobless benefits for
individuals, money for states and a huge bailout fund for businesses.

The measure, which the Senate approved unanimously just before midnight
on Wednesday, amounts to a government aid plan unprecedented in its
sheer scope and size, touching on every facet of American life with the
goal of salvaging and ultimately reviving a battered economy.

Its cost is hundreds of billions of dollars more than Congress provides
for the entire United States federal budget for a single year, outside
of social safety net programs. Administration officials said they hoped
that its effect on a battered economy would be exponentially greater, as
much as \$4 trillion.

The legislation would send direct payments of \$1,200 to millions of
Americans, including those earning up to \$75,000, and an additional
\$500 per child. It would substantially expand jobless aid, providing an
additional 13 weeks and a four-month enhancement of benefits, and would
extend the payments for the first time to freelancers and gig workers.

The measure would also offer \$377 billion in federally guaranteed loans
to small businesses and establish a \$500 billion government lending
program for distressed companies reeling from the impact of the crisis,
including allowing the administration the ability to take equity stakes
in airlines that received aid to help compensate taxpayers. It would
also send \$100 billion to hospitals on the front lines of the pandemic.

``This is certainly, in terms of dollars, by far and away the biggest
ever, ever done,'' President Trump said at the White House, where he
veered from his usual partisan vitriol and praised Democrats for their
work on the agreement. ``That is a tremendous thing because a lot of
this money goes to jobs, jobs, jobs --- and families, families,
families.''

The deal is the product of a marathon set of negotiations among Senate
Republicans, Democrats and Mr. Trump's team that nearly fell apart as
Democrats insisted on stronger worker protections, more funds for
hospitals and state governments, and tougher oversight over new loan
programs intended to bail out distressed businesses.

Anticipation of the vote sent the markets higher for the second
consecutive day, with the S\&P 500 up a little more than 1 percent. But
investors appeared to grow jittery toward the end of trading as a group
of Republican senators delayed a final vote over concerns that the
jobless aid was so generous that it could lead to layoffs and discourage
people from working.

The last-minute snag revealed the tenuous nature of the bipartisan
compromise that was at the core of the measure, which emerged from an
extraordinary
\href{https://www.nytimes.com/2020/03/19/us/politics/1000-checks-coronavirus-stimulus.html}{five-day
stretch} of intense negotiations between lawmakers and White House
officials over how to deliver critical financial support to businesses
forced to shut their doors, American families and hospitals overwhelmed
by
\href{https://www.nytimes.com/interactive/2020/world/coronavirus-maps.html?action=click\&pgtype=Article\&state=default\&module=styln-coronavirus-markets\&variant=show\&region=TOP_BANNER\&context=storyline_menu?action=click\&pgtype=Article\&state=default\&module=styln-coronavirus-markets\&variant=show\&region=TOP_BANNER\&context=storyline_menu\#us}{the
spread of the novel coronavirus}. It has already killed more than 900
people and infected more than 68,000 in the United States.

The perils of the pandemic, which by Wednesday had spread within the
marble halls of the Capitol to infect lawmakers themselves, prompted
Republicans to put aside their usual antipathy for big government and
spearhead an effort to send cash to American families, while agreeing to
astonishingly large additions to the social safety net. Democrats, for
their part, dropped their routine opposition to showering tax cuts and
other benefits on big corporations --- all in the interest of getting a
deal.

\includegraphics{https://static01.nyt.com/images/2020/03/25/us/politics/25dc-virus-cong-new2/merlin_170960169_dbb53078-21a0-4eee-8d3b-5c8bbcb74487-articleLarge.jpg?quality=75\&auto=webp\&disable=upscale}

And even as they prepared to approve it, lawmakers were already
discussing the likelihood that they would soon have to consider yet
another package to respond to the pandemic and the toll it was taking on
the United States.

The
\href{https://www.nytimes.com/article/where-is-my-stimulus-payment.html}{stimulus}
package was intended to encourage companies to keep employees on their
payrolls even if their businesses have shuttered temporarily --- and it
increases aid to workers who are laid off anyway or have had their hours
and wages cut back.

\hypertarget{latest-updates-global-coronavirus-outbreak}{%
\section{\texorpdfstring{\href{https://www.nytimes.com/2020/08/01/world/coronavirus-covid-19.html?action=click\&pgtype=Article\&state=default\&region=MAIN_CONTENT_1\&context=storylines_live_updates}{Latest
Updates: Global Coronavirus
Outbreak}}{Latest Updates: Global Coronavirus Outbreak}}\label{latest-updates-global-coronavirus-outbreak}}

Updated 2020-08-02T07:42:09.613Z

\begin{itemize}
\tightlist
\item
  \href{https://www.nytimes.com/2020/08/01/world/coronavirus-covid-19.html?action=click\&pgtype=Article\&state=default\&region=MAIN_CONTENT_1\&context=storylines_live_updates\#link-34047410}{The
  U.S. reels as July cases more than double the total of any other
  month.}
\item
  \href{https://www.nytimes.com/2020/08/01/world/coronavirus-covid-19.html?action=click\&pgtype=Article\&state=default\&region=MAIN_CONTENT_1\&context=storylines_live_updates\#link-780ec966}{Top
  U.S. officials work to break an impasse over the federal jobless
  benefit.}
\item
  \href{https://www.nytimes.com/2020/08/01/world/coronavirus-covid-19.html?action=click\&pgtype=Article\&state=default\&region=MAIN_CONTENT_1\&context=storylines_live_updates\#link-2bc8948}{Its
  outbreak untamed, Melbourne goes into even greater lockdown.}
\end{itemize}

\href{https://www.nytimes.com/2020/08/01/world/coronavirus-covid-19.html?action=click\&pgtype=Article\&state=default\&region=MAIN_CONTENT_1\&context=storylines_live_updates}{See
more updates}

More live coverage:
\href{https://www.nytimes.com/live/2020/07/31/business/stock-market-today-coronavirus?action=click\&pgtype=Article\&state=default\&region=MAIN_CONTENT_1\&context=storylines_live_updates}{Markets}

Though the bill is more than double the size of the roughly \$800
billion stimulus package that Congress passed in 2009 to ease the Great
Recession, analysts and economists warned it may provide only a few
months of financial relief given the unknown breadth of the pandemic's
reach. With lawmakers besieged by an array of lobbyists and
special-interest groups, not to mention constituents and businesses
desperate for relief, the package more than doubled in size since
Senator Mitch McConnell, Republican of Kentucky and the majority leader,
first introduced legislation last Thursday.

``This is not a moment of celebration, but one of necessity,'' Senator
Chuck Schumer of New York, the Democratic leader, said before the vote.
``To all Americans, I say, `Help is on the way.'''

Still, some states said they needed far more government aid than it
planned to provide. Gov. Andrew M. Cuomo of New York, whose state is
battling by far the largest outbreak of the virus in the United States,
said Wednesday that the package was ``terrible'' for New York, and that
the \$3.1 billion earmarked to help the state with its budget gap was
not nearly enough.

``We need more federal help than this bill gives us,'' he said.

Speaker Nancy Pelosi of California endorsed the deal, and planned to
push it through the House on Friday by voice vote --- meaning that no
roll call would be taken --- given that the chamber is in recess and its
members are scattered across the country, some in places that have
imposed travel restrictions and quarantines.

``Members who want to come to the House floor to debate this bill will
be able to do so,'' Representative Steny H. Hoyer of Maryland, the No. 2
House Democrat, wrote in a letter to his colleagues. ``In addition, we
are working to ensure that those who are unable to return to Washington
may express their views on this legislation remotely.''

The Senate vote unfolded as Covid-19 took its toll on that chamber, as
well. Senator Rand Paul, Republican of Kentucky, was absent because he
has contracted the coronavirus, while two Utah Republicans, Senators
Mitt Romney and Mike Lee, remain in self-isolation out of an abundance
of caution after spending time with Mr. Paul. Senator John Thune of
South Dakota, the No. 2 Senate Republican, missed the vote because he
was not feeling well, a spokesman said, and had returned home out of an
abundance of caution.

The agreement came together after a furious final round of haggling
between administration officials led by Steven Mnuchin, the Treasury
secretary, and Mr. Schumer after Democrats twice blocked action on the
measure as they insisted on concessions.

Image

Treasury Secretary Steven Mnuchin, left, Eric Ueland, the White House
legis­lativ­e affai­rs direc­tor, and Representative Mark Meadows on
Tuesday at the Capitol.Credit...Anna Moneymaker/The New York Times

Once passed by both houses, the measure would be the third emergency
bill approved by Congress this month to address the pandemic. Mr. Trump
previously signed into law both \$8.3 billion in emergency aid and a
sweeping package providing paid leave, free coronavirus testing and
additional aid for families affected by the pandemic.

In the final measure, most Republicans agreed to Democrats' demands for
a substantial expansion of jobless benefits, including \$600 extra per
week on top of the usual amount provided by states.

On Wednesday afternoon, four Republican senators said they were
concerned the new benefits would be larger than some people's wages,
prompting employers to lay off workers and some employees to prefer
staying home and collecting unemployment payments.

``If this is not a drafting error, then this is the worst idea I have
seen in a long time,'' said Senator Lindsey Graham, Republican of South
Carolina. ``We need to create a sustainable system.''

\href{https://www.nytimes.com/news-event/coronavirus?action=click\&pgtype=Article\&state=default\&region=MAIN_CONTENT_3\&context=storylines_faq}{}

\hypertarget{the-coronavirus-outbreak-}{%
\subsubsection{The Coronavirus Outbreak
›}\label{the-coronavirus-outbreak-}}

\hypertarget{frequently-asked-questions}{%
\paragraph{Frequently Asked
Questions}\label{frequently-asked-questions}}

Updated July 27, 2020

\begin{itemize}
\item ~
  \hypertarget{should-i-refinance-my-mortgage}{%
  \paragraph{Should I refinance my
  mortgage?}\label{should-i-refinance-my-mortgage}}

  \begin{itemize}
  \tightlist
  \item
    \href{https://www.nytimes.com/article/coronavirus-money-unemployment.html?action=click\&pgtype=Article\&state=default\&region=MAIN_CONTENT_3\&context=storylines_faq}{It
    could be a good idea,} because mortgage rates have
    \href{https://www.nytimes.com/2020/07/16/business/mortgage-rates-below-3-percent.html?action=click\&pgtype=Article\&state=default\&region=MAIN_CONTENT_3\&context=storylines_faq}{never
    been lower.} Refinancing requests have pushed mortgage applications
    to some of the highest levels since 2008, so be prepared to get in
    line. But defaults are also up, so if you're thinking about buying a
    home, be aware that some lenders have tightened their standards.
  \end{itemize}
\item ~
  \hypertarget{what-is-school-going-to-look-like-in-september}{%
  \paragraph{What is school going to look like in
  September?}\label{what-is-school-going-to-look-like-in-september}}

  \begin{itemize}
  \tightlist
  \item
    It is unlikely that many schools will return to a normal schedule
    this fall, requiring the grind of
    \href{https://www.nytimes.com/2020/06/05/us/coronavirus-education-lost-learning.html?action=click\&pgtype=Article\&state=default\&region=MAIN_CONTENT_3\&context=storylines_faq}{online
    learning},
    \href{https://www.nytimes.com/2020/05/29/us/coronavirus-child-care-centers.html?action=click\&pgtype=Article\&state=default\&region=MAIN_CONTENT_3\&context=storylines_faq}{makeshift
    child care} and
    \href{https://www.nytimes.com/2020/06/03/business/economy/coronavirus-working-women.html?action=click\&pgtype=Article\&state=default\&region=MAIN_CONTENT_3\&context=storylines_faq}{stunted
    workdays} to continue. California's two largest public school
    districts --- Los Angeles and San Diego --- said on July 13, that
    \href{https://www.nytimes.com/2020/07/13/us/lausd-san-diego-school-reopening.html?action=click\&pgtype=Article\&state=default\&region=MAIN_CONTENT_3\&context=storylines_faq}{instruction
    will be remote-only in the fall}, citing concerns that surging
    coronavirus infections in their areas pose too dire a risk for
    students and teachers. Together, the two districts enroll some
    825,000 students. They are the largest in the country so far to
    abandon plans for even a partial physical return to classrooms when
    they reopen in August. For other districts, the solution won't be an
    all-or-nothing approach.
    \href{https://bioethics.jhu.edu/research-and-outreach/projects/eschool-initiative/school-policy-tracker/}{Many
    systems}, including the nation's largest, New York City, are
    devising
    \href{https://www.nytimes.com/2020/06/26/us/coronavirus-schools-reopen-fall.html?action=click\&pgtype=Article\&state=default\&region=MAIN_CONTENT_3\&context=storylines_faq}{hybrid
    plans} that involve spending some days in classrooms and other days
    online. There's no national policy on this yet, so check with your
    municipal school system regularly to see what is happening in your
    community.
  \end{itemize}
\item ~
  \hypertarget{is-the-coronavirus-airborne}{%
  \paragraph{Is the coronavirus
  airborne?}\label{is-the-coronavirus-airborne}}

  \begin{itemize}
  \tightlist
  \item
    The coronavirus
    \href{https://www.nytimes.com/2020/07/04/health/239-experts-with-one-big-claim-the-coronavirus-is-airborne.html?action=click\&pgtype=Article\&state=default\&region=MAIN_CONTENT_3\&context=storylines_faq}{can
    stay aloft for hours in tiny droplets in stagnant air}, infecting
    people as they inhale, mounting scientific evidence suggests. This
    risk is highest in crowded indoor spaces with poor ventilation, and
    may help explain super-spreading events reported in meatpacking
    plants, churches and restaurants.
    \href{https://www.nytimes.com/2020/07/06/health/coronavirus-airborne-aerosols.html?action=click\&pgtype=Article\&state=default\&region=MAIN_CONTENT_3\&context=storylines_faq}{It's
    unclear how often the virus is spread} via these tiny droplets, or
    aerosols, compared with larger droplets that are expelled when a
    sick person coughs or sneezes, or transmitted through contact with
    contaminated surfaces, said Linsey Marr, an aerosol expert at
    Virginia Tech. Aerosols are released even when a person without
    symptoms exhales, talks or sings, according to Dr. Marr and more
    than 200 other experts, who
    \href{https://academic.oup.com/cid/article/doi/10.1093/cid/ciaa939/5867798}{have
    outlined the evidence in an open letter to the World Health
    Organization}.
  \end{itemize}
\item ~
  \hypertarget{what-are-the-symptoms-of-coronavirus}{%
  \paragraph{What are the symptoms of
  coronavirus?}\label{what-are-the-symptoms-of-coronavirus}}

  \begin{itemize}
  \tightlist
  \item
    Common symptoms
    \href{https://www.nytimes.com/article/symptoms-coronavirus.html?action=click\&pgtype=Article\&state=default\&region=MAIN_CONTENT_3\&context=storylines_faq}{include
    fever, a dry cough, fatigue and difficulty breathing or shortness of
    breath.} Some of these symptoms overlap with those of the flu,
    making detection difficult, but runny noses and stuffy sinuses are
    less common.
    \href{https://www.nytimes.com/2020/04/27/health/coronavirus-symptoms-cdc.html?action=click\&pgtype=Article\&state=default\&region=MAIN_CONTENT_3\&context=storylines_faq}{The
    C.D.C. has also} added chills, muscle pain, sore throat, headache
    and a new loss of the sense of taste or smell as symptoms to look
    out for. Most people fall ill five to seven days after exposure, but
    symptoms may appear in as few as two days or as many as 14 days.
  \end{itemize}
\item ~
  \hypertarget{does-asymptomatic-transmission-of-covid-19-happen}{%
  \paragraph{Does asymptomatic transmission of Covid-19
  happen?}\label{does-asymptomatic-transmission-of-covid-19-happen}}

  \begin{itemize}
  \tightlist
  \item
    So far, the evidence seems to show it does. A widely cited
    \href{https://www.nature.com/articles/s41591-020-0869-5}{paper}
    published in April suggests that people are most infectious about
    two days before the onset of coronavirus symptoms and estimated that
    44 percent of new infections were a result of transmission from
    people who were not yet showing symptoms. Recently, a top expert at
    the World Health Organization stated that transmission of the
    coronavirus by people who did not have symptoms was ``very rare,''
    \href{https://www.nytimes.com/2020/06/09/world/coronavirus-updates.html?action=click\&pgtype=Article\&state=default\&region=MAIN_CONTENT_3\&context=storylines_faq\#link-1f302e21}{but
    she later walked back that statement.}
  \end{itemize}
\end{itemize}

Mr. Mnuchin said the extra payments were calculated as a way to ensure
that states could get money out quickly, saying that he did not believe
it would create any perverse incentives. Most Americans, he said, ``want
to keep their jobs.''

Still, the Republicans' threat to hold up the bill because of the issue
prompted Senator Bernie Sanders, independent of Vermont and a Democratic
presidential contender, to issue his own warning that he, too, would
seek to block the legislation for being too lenient on corporations.
Later, in a speech on the floor, Mr. Sanders said he would support the
bill despite his many reservations.

Encapsulating the sentiment of many lawmakers in both parties about the
hastily negotiated package, Senator Ben Sasse of Nebraska, one of the
Republicans who sought to cap the jobless aid, said while he disagreed
with Mr. Sanders, ``I appreciate his candor in admitting that this is
kind of a big crap sandwich.''

In the end, though, not a single senator voted ``no.''

The hardest-fought concessions were related to the \$500 billion aid
fund for distressed businesses, which would include \$425 billion for
the Federal Reserve to leverage for loans to help broad groups of
distressed companies and \$75 billion for industry-specific loans to
airlines and other hard-hit sectors.

Democrats insisted on stricter oversight, in the form of an inspector
general and a five-person panel appointed by Congress. Republicans also
agreed to require companies that accepted money through the fund to halt
any stock buybacks for as long as they were receiving government
assistance, plus an additional year.

The agreement also includes \$350 billion for lending programs for small
businesses, but only those that kept their payrolls steady through the
crisis. Small businesses that pledged to keep their workers would also
receive cash-flow assistance structured as federally guaranteed loans.
If the employer continued to pay workers for the duration of the crisis,
those loans would be forgiven.

Airlines stand to benefit from multiple provisions, according to Senator
Patrick J. Toomey, Republican of Pennsylvania. He pointed to an
additional \$25 billion in grants for them, with the potential for
equity to benefit taxpayers. He also said that \$17 billion is available
for direct loans to companies related to America's national security.

Image

The legislation includes provisions for small-business
loans.~Credit...Jose A. Alvarado Jr. for The New York Times

Democrats won a provision to block Trump family businesses --- or those
of other senior government officials --- from receiving loan money under
the programs, though the president's real estate empire could still
benefit from other parts of the bill.

Senators also directly targeted those on the front lines of responding
to the pandemic, allocating \$100 billion to hospitals, more than \$1
billion for virus-related research, and \$150 billion for state and
local governments to help them weather drop-offs in tax revenue and the
costs of fighting the pandemic.

Buried in thousands of pages of dense legal text were less visible steps
to mitigate the pandemic's effects on Americans lives and retool large
sections of the government to function remotely for the first time.

The bill, for example, would funnel \$3.5 billion to states to prop up
child care facilities and allows universities to keep paying students in
federal work-study jobs even if their academic terms have been cut
short.

It would allocate \$100 million for additional rural broadband and \$150
million for arts and humanities grants to bring cultural programming to
Americans stuck at home. It would increase funding for domestic violence
shelters and hotlines and set aside \$425 million to deal with mental
health and substance abuse disorders related to the pandemic. \$400
million would become available to protect and expand voting for the 2020
election cycle.

Image

Journalists were spaced out in the Senate studio before a news
conference on Wednesday.Credit...Anna Moneymaker/The New York Times

Under other provisions, Americans affected by the virus will soon be
able to temporarily withdraw up to \$100,000 penalty free from their
retirement accounts to use for virus-related expenses. And menstrual
products will become eligible for reimbursement under flexible spending
accounts.

After Wednesday's vote, Mr. McConnell said, the Senate is not scheduled
to return until April 20.

Carl Hulse and Sheryl Gay Stolberg contributed reporting from
Washington, and Nicole Hong from New York.

Advertisement

\protect\hyperlink{after-bottom}{Continue reading the main story}

\hypertarget{site-index}{%
\subsection{Site Index}\label{site-index}}

\hypertarget{site-information-navigation}{%
\subsection{Site Information
Navigation}\label{site-information-navigation}}

\begin{itemize}
\tightlist
\item
  \href{https://help.nytimes.com/hc/en-us/articles/115014792127-Copyright-notice}{©~2020~The
  New York Times Company}
\end{itemize}

\begin{itemize}
\tightlist
\item
  \href{https://www.nytco.com/}{NYTCo}
\item
  \href{https://help.nytimes.com/hc/en-us/articles/115015385887-Contact-Us}{Contact
  Us}
\item
  \href{https://www.nytco.com/careers/}{Work with us}
\item
  \href{https://nytmediakit.com/}{Advertise}
\item
  \href{http://www.tbrandstudio.com/}{T Brand Studio}
\item
  \href{https://www.nytimes.com/privacy/cookie-policy\#how-do-i-manage-trackers}{Your
  Ad Choices}
\item
  \href{https://www.nytimes.com/privacy}{Privacy}
\item
  \href{https://help.nytimes.com/hc/en-us/articles/115014893428-Terms-of-service}{Terms
  of Service}
\item
  \href{https://help.nytimes.com/hc/en-us/articles/115014893968-Terms-of-sale}{Terms
  of Sale}
\item
  \href{https://spiderbites.nytimes.com}{Site Map}
\item
  \href{https://help.nytimes.com/hc/en-us}{Help}
\item
  \href{https://www.nytimes.com/subscription?campaignId=37WXW}{Subscriptions}
\end{itemize}
