Sections

SEARCH

\protect\hyperlink{site-content}{Skip to
content}\protect\hyperlink{site-index}{Skip to site index}

\href{https://www.nytimes.com/section/politics}{Politics}

\href{https://myaccount.nytimes.com/auth/login?response_type=cookie\&client_id=vi}{}

\href{https://www.nytimes.com/section/todayspaper}{Today's Paper}

\href{/section/politics}{Politics}\textbar{}Fine Print of Stimulus Bill
Contains Special Deals for Industries

\url{https://nyti.ms/2WELVx3}

\begin{itemize}
\item
\item
\item
\item
\item
\end{itemize}

\href{https://www.nytimes.com/news-event/coronavirus?action=click\&pgtype=Article\&state=default\&region=TOP_BANNER\&context=storylines_menu}{The
Coronavirus Outbreak}

\begin{itemize}
\tightlist
\item
  live\href{https://www.nytimes.com/2020/08/01/world/coronavirus-covid-19.html?action=click\&pgtype=Article\&state=default\&region=TOP_BANNER\&context=storylines_menu}{Latest
  Updates}
\item
  \href{https://www.nytimes.com/interactive/2020/us/coronavirus-us-cases.html?action=click\&pgtype=Article\&state=default\&region=TOP_BANNER\&context=storylines_menu}{Maps
  and Cases}
\item
  \href{https://www.nytimes.com/interactive/2020/science/coronavirus-vaccine-tracker.html?action=click\&pgtype=Article\&state=default\&region=TOP_BANNER\&context=storylines_menu}{Vaccine
  Tracker}
\item
  \href{https://www.nytimes.com/interactive/2020/07/29/us/schools-reopening-coronavirus.html?action=click\&pgtype=Article\&state=default\&region=TOP_BANNER\&context=storylines_menu}{What
  School May Look Like}
\item
  \href{https://www.nytimes.com/live/2020/07/31/business/stock-market-today-coronavirus?action=click\&pgtype=Article\&state=default\&region=TOP_BANNER\&context=storylines_menu}{Economy}
\end{itemize}

Advertisement

\protect\hyperlink{after-top}{Continue reading the main story}

Supported by

\protect\hyperlink{after-sponsor}{Continue reading the main story}

\hypertarget{fine-print-of-stimulus-bill-contains-special-deals-for-industries}{%
\section{Fine Print of Stimulus Bill Contains Special Deals for
Industries}\label{fine-print-of-stimulus-bill-contains-special-deals-for-industries}}

Small banks, retailers and for-profit colleges got provisions they
wanted. So did Boeing. Among those who could potentially benefit:
President Trump's company.

\includegraphics{https://static01.nyt.com/images/2020/03/25/us/politics/25dc-virus-fineprint1/merlin_140689365_bc977b1a-eb5d-4a2e-b2f4-49722198cbdf-articleLarge.jpg?quality=75\&auto=webp\&disable=upscale}

\href{https://www.nytimes.com/by/eric-lipton}{\includegraphics{https://static01.nyt.com/images/2018/12/06/multimedia/author-eric-lipton/author-eric-lipton-thumbLarge.png}}\href{https://www.nytimes.com/by/kenneth-p-vogel}{\includegraphics{https://static01.nyt.com/images/2018/02/20/multimedia/author-kenneth-p-vogel/author-kenneth-p-vogel-thumbLarge-v3.png}}

By \href{https://www.nytimes.com/by/eric-lipton}{Eric Lipton} and
\href{https://www.nytimes.com/by/kenneth-p-vogel}{Kenneth P. Vogel}

\begin{itemize}
\item
  Published March 25, 2020Updated May 5, 2020
\item
  \begin{itemize}
  \item
  \item
  \item
  \item
  \item
  \end{itemize}
\end{itemize}

WASHINGTON --- Restaurants and retailers will get a tweak to federal tax
law they have been seeking for more than a year that could save them
\$15 billion. Community banks are being granted their long-held wish of
being freed to reduce the amount of capital they have to hold in
reserve.

And for-profit colleges will be able to keep federal loan money from
students who drop out because of the coronavirus.

Tucked into the largest bailout in United States history --- a \$2
trillion federal
\href{https://www.nytimes.com/article/where-is-my-stimulus-payment.html}{stimulus}package
\href{https://www.nytimes.com/2020/03/25/us/politics/coronavirus-senate-deal.html}{agreed
to by congressional leaders and the White House early Wednesday} in an
effort
\href{https://www.nytimes.com/2020/03/25/business/2-trillion-stimulus-coronavirus-bill.html}{to
reduce the economic devastation} of the coronavirus outbreak --- are a
range of provisions that stand to benefit specific industries and
interest groups.

Even the fine print in a near-final
\href{https://int.nyt.com/data/documenthelper/6843-revised-final-cares-10-p-m-3-2/ae6f534c4fc474f4ba33/optimized/full.pdf\#page=1}{880-page
version of the bill} has fine print. Democrats proudly announced that
they had won agreement on language to block President Trump, other
government officials and their families from receiving assistance from a
\$500 billion fund to be administered by the Treasury Department.

But it turns out that the provision might not preclude funds from going
to companies owned by the family of Mr. Trump's son-in-law and White
House adviser, Jared Kushner, while Mr. Trump's companies would not be
barred from benefiting from other elements of the bill intended to help
broad swaths of American business.

For example, certain hotel owners, even those employing thousands of
people, will be eligible for small-business loans, a provision that
could potentially benefit Mr. Trump's company to help to continue to pay
wages for his employees. The Trump Organization could also benefit from
the \$15 billion change to the tax code won by restaurants and
retailers.

The legislation,
\href{https://www.nytimes.com/2020/03/25/us/politics/coronavirus-senate-deal.html}{which
passed the Senate} and could be passed by the House and signed into law
by Mr. Trump by the end of the week, is intended primarily
\href{https://www.nytimes.com/2020/03/25/us/politics/whats-in-coronavirus-stimulus-bill.html}{to
put money in the hands of many households} and prop up especially
hard-hit industries like airlines.

But its sheer size and the rushed way it was put together made it an
irresistible target for lobbyists, who launched a
\href{https://www.nytimes.com/2020/03/20/us/politics/coronavirus-stimulus-lobbying.html}{frenzied
effort to insert into the must-pass legislation} provisions their
clients wanted, some of which had a timely rationale and others of which
were largely unconnected to the coronavirus crisis.

This lobbying push was unlike any other, as social distancing measures
intended to limit the spread of the virus among lawmakers and their
staffs left the
\href{https://www.nytimes.com/2020/03/19/us/politics/coronavirus-congress-voting.html}{Capitol
eerily quiet}. Lobbyists instead pressed their causes to staff members
and lawmakers over the phone, or via email.

``We went to McConnell's people, we went to Schumer's people, and
Pelosi's and McCarthy's people --- we pinged them all,'' said Rachelle
B. Bernstein,
\href{https://www.opensecrets.org/federal-lobbying/firms/lobbyists?cycle=2019\&id=D000000741}{a
lobbyist} and tax counsel at the National Retail Federation, which
pressed successfully for the \$15-billion-a-year change in federal tax
law.

\includegraphics{https://static01.nyt.com/images/2020/03/25/us/politics/25dc-virus-fineprint2/merlin_170909958_75d6ac5f-056b-4fdf-8222-b4ab3b63d5a0-articleLarge.jpg?quality=75\&auto=webp\&disable=upscale}

While some industries and companies are benefiting from provisions
tailored for them, others appear certain to get a piece of the pie
through more general components of the bill, from the \$454 billion
general purpose fund for businesses and state and local governments to
the \$50 billion earmarked for airlines and \$8 billion for air cargo
carriers.

The deal specifically sets aside \$17 billion for ``businesses critical
to maintaining national security'' --- a category seen as intended at
least partly for Boeing, the troubled aircraft manufacturer and Pentagon
contractor, whose name appears nowhere in the bill.

\hypertarget{latest-updates-global-coronavirus-outbreak}{%
\section{\texorpdfstring{\href{https://www.nytimes.com/2020/08/01/world/coronavirus-covid-19.html?action=click\&pgtype=Article\&state=default\&region=MAIN_CONTENT_1\&context=storylines_live_updates}{Latest
Updates: Global Coronavirus
Outbreak}}{Latest Updates: Global Coronavirus Outbreak}}\label{latest-updates-global-coronavirus-outbreak}}

Updated 2020-08-02T07:42:09.613Z

\begin{itemize}
\tightlist
\item
  \href{https://www.nytimes.com/2020/08/01/world/coronavirus-covid-19.html?action=click\&pgtype=Article\&state=default\&region=MAIN_CONTENT_1\&context=storylines_live_updates\#link-34047410}{The
  U.S. reels as July cases more than double the total of any other
  month.}
\item
  \href{https://www.nytimes.com/2020/08/01/world/coronavirus-covid-19.html?action=click\&pgtype=Article\&state=default\&region=MAIN_CONTENT_1\&context=storylines_live_updates\#link-780ec966}{Top
  U.S. officials work to break an impasse over the federal jobless
  benefit.}
\item
  \href{https://www.nytimes.com/2020/08/01/world/coronavirus-covid-19.html?action=click\&pgtype=Article\&state=default\&region=MAIN_CONTENT_1\&context=storylines_live_updates\#link-2bc8948}{Its
  outbreak untamed, Melbourne goes into even greater lockdown.}
\end{itemize}

\href{https://www.nytimes.com/2020/08/01/world/coronavirus-covid-19.html?action=click\&pgtype=Article\&state=default\&region=MAIN_CONTENT_1\&context=storylines_live_updates}{See
more updates}

More live coverage:
\href{https://www.nytimes.com/live/2020/07/31/business/stock-market-today-coronavirus?action=click\&pgtype=Article\&state=default\&region=MAIN_CONTENT_1\&context=storylines_live_updates}{Markets}

``We'll be helping Boeing,'' Mr. Trump said Tuesday evening. ``We'll be
helping the airlines, the cruise lines.''

As with any complex piece of legislation, this one will create winners
and losers.

Image

Boeing 737 Max airplanes parked at Tulsa International Airport. ``We'll
be helping Boeing,'' President Trump said.Credit...Nick Oxford/Reuters

Despite the effort by Democrats to limit access by top federal officials
and members of Congress **** to the bailout funds, the law would still
leave room for Mr. Trump to benefit. At least two of the provisions,
intended to help the hotel and restaurant industries, could potentially
provide financial help to the Trump Organization.

A spokesman for the Trump Organization did not respond to a request for
comment. Mr. Trump declined to respond to a question this week about
whether his family business intended to take advantage of any of the tax
breaks or other benefits included in the legislation.

``I don't know,'' Mr. Trump said at a news conference on Sunday. ``I
just don't know what the government assistance would be for what I have.
I have hotels.''

Many of these special-interest provisions would be impossible for a
casual reader of the legislation to identify. For example, on Page 15 of
the bill, there is a section with the title ``Business Concerns With
More Than 1 Physical Location.'' It says this change in federal law will
apply to companies that fit ``a North American Industry Classification
System code beginning with 72'' --- a reference that turns out to mean
the hotel and restaurant industry.

The provision says that if a company owns multiple hotels, even if the
overall hotel or restaurant chain has more than 500 employees --- the
limit to qualify for treatment as a small business --- it will still be
able to take advantage of the small-business benefits offered in the
rescue package.

That means loans from the federal government worth up to 2.5 times the
firm's monthly payroll that will not have to be repaid if the company
uses them to keep paying employees during any coronavirus shutdowns.

Representatives from the American Hotel \& Lodging Association reached
out to Republicans and Democrats to push them to insert the language,
arguing that it would allow the federal assistance to cover an
additional 33,000 hotels, with a total of about one million employees.

The large corporations that own the big brands --- like Marriott or
Hilton --- would not be eligible. But any individual hotel, including
from one of these brands, that has fewer than 500 employees would be.
Many hotels are owned by franchisees.

The provision could benefit the Trump Organization, which operates a
relatively small chain, with six hotels in the United States in cities
including New York, Washington and Chicago. Several Trump hotels are
members of the trade association.

A representative for the trade group said executives at the Trump
Organization were not involved in the lobbying effort. Representatives
of the Trump Organization did not respond to a question on Wednesday
about the provision.

Image

Retailers and restaurants sought a specific tweak to the tax
code.Credit...Jose A. Alvarado Jr. for The New York Times

The tweak to the tax code sought by the nation's retailers and grocers
could mean \$15 billion a year worth of tax savings for hotels,
restaurants, supermarkets and other retailers. Groups representing those
industries separately intervened with both Mr. Trump and leaders on
Capitol Hill to push lawmakers to include it in the final package.

\href{https://www.nytimes.com/news-event/coronavirus?action=click\&pgtype=Article\&state=default\&region=MAIN_CONTENT_3\&context=storylines_faq}{}

\hypertarget{the-coronavirus-outbreak-}{%
\subsubsection{The Coronavirus Outbreak
›}\label{the-coronavirus-outbreak-}}

\hypertarget{frequently-asked-questions}{%
\paragraph{Frequently Asked
Questions}\label{frequently-asked-questions}}

Updated July 27, 2020

\begin{itemize}
\item ~
  \hypertarget{should-i-refinance-my-mortgage}{%
  \paragraph{Should I refinance my
  mortgage?}\label{should-i-refinance-my-mortgage}}

  \begin{itemize}
  \tightlist
  \item
    \href{https://www.nytimes.com/article/coronavirus-money-unemployment.html?action=click\&pgtype=Article\&state=default\&region=MAIN_CONTENT_3\&context=storylines_faq}{It
    could be a good idea,} because mortgage rates have
    \href{https://www.nytimes.com/2020/07/16/business/mortgage-rates-below-3-percent.html?action=click\&pgtype=Article\&state=default\&region=MAIN_CONTENT_3\&context=storylines_faq}{never
    been lower.} Refinancing requests have pushed mortgage applications
    to some of the highest levels since 2008, so be prepared to get in
    line. But defaults are also up, so if you're thinking about buying a
    home, be aware that some lenders have tightened their standards.
  \end{itemize}
\item ~
  \hypertarget{what-is-school-going-to-look-like-in-september}{%
  \paragraph{What is school going to look like in
  September?}\label{what-is-school-going-to-look-like-in-september}}

  \begin{itemize}
  \tightlist
  \item
    It is unlikely that many schools will return to a normal schedule
    this fall, requiring the grind of
    \href{https://www.nytimes.com/2020/06/05/us/coronavirus-education-lost-learning.html?action=click\&pgtype=Article\&state=default\&region=MAIN_CONTENT_3\&context=storylines_faq}{online
    learning},
    \href{https://www.nytimes.com/2020/05/29/us/coronavirus-child-care-centers.html?action=click\&pgtype=Article\&state=default\&region=MAIN_CONTENT_3\&context=storylines_faq}{makeshift
    child care} and
    \href{https://www.nytimes.com/2020/06/03/business/economy/coronavirus-working-women.html?action=click\&pgtype=Article\&state=default\&region=MAIN_CONTENT_3\&context=storylines_faq}{stunted
    workdays} to continue. California's two largest public school
    districts --- Los Angeles and San Diego --- said on July 13, that
    \href{https://www.nytimes.com/2020/07/13/us/lausd-san-diego-school-reopening.html?action=click\&pgtype=Article\&state=default\&region=MAIN_CONTENT_3\&context=storylines_faq}{instruction
    will be remote-only in the fall}, citing concerns that surging
    coronavirus infections in their areas pose too dire a risk for
    students and teachers. Together, the two districts enroll some
    825,000 students. They are the largest in the country so far to
    abandon plans for even a partial physical return to classrooms when
    they reopen in August. For other districts, the solution won't be an
    all-or-nothing approach.
    \href{https://bioethics.jhu.edu/research-and-outreach/projects/eschool-initiative/school-policy-tracker/}{Many
    systems}, including the nation's largest, New York City, are
    devising
    \href{https://www.nytimes.com/2020/06/26/us/coronavirus-schools-reopen-fall.html?action=click\&pgtype=Article\&state=default\&region=MAIN_CONTENT_3\&context=storylines_faq}{hybrid
    plans} that involve spending some days in classrooms and other days
    online. There's no national policy on this yet, so check with your
    municipal school system regularly to see what is happening in your
    community.
  \end{itemize}
\item ~
  \hypertarget{is-the-coronavirus-airborne}{%
  \paragraph{Is the coronavirus
  airborne?}\label{is-the-coronavirus-airborne}}

  \begin{itemize}
  \tightlist
  \item
    The coronavirus
    \href{https://www.nytimes.com/2020/07/04/health/239-experts-with-one-big-claim-the-coronavirus-is-airborne.html?action=click\&pgtype=Article\&state=default\&region=MAIN_CONTENT_3\&context=storylines_faq}{can
    stay aloft for hours in tiny droplets in stagnant air}, infecting
    people as they inhale, mounting scientific evidence suggests. This
    risk is highest in crowded indoor spaces with poor ventilation, and
    may help explain super-spreading events reported in meatpacking
    plants, churches and restaurants.
    \href{https://www.nytimes.com/2020/07/06/health/coronavirus-airborne-aerosols.html?action=click\&pgtype=Article\&state=default\&region=MAIN_CONTENT_3\&context=storylines_faq}{It's
    unclear how often the virus is spread} via these tiny droplets, or
    aerosols, compared with larger droplets that are expelled when a
    sick person coughs or sneezes, or transmitted through contact with
    contaminated surfaces, said Linsey Marr, an aerosol expert at
    Virginia Tech. Aerosols are released even when a person without
    symptoms exhales, talks or sings, according to Dr. Marr and more
    than 200 other experts, who
    \href{https://academic.oup.com/cid/article/doi/10.1093/cid/ciaa939/5867798}{have
    outlined the evidence in an open letter to the World Health
    Organization}.
  \end{itemize}
\item ~
  \hypertarget{what-are-the-symptoms-of-coronavirus}{%
  \paragraph{What are the symptoms of
  coronavirus?}\label{what-are-the-symptoms-of-coronavirus}}

  \begin{itemize}
  \tightlist
  \item
    Common symptoms
    \href{https://www.nytimes.com/article/symptoms-coronavirus.html?action=click\&pgtype=Article\&state=default\&region=MAIN_CONTENT_3\&context=storylines_faq}{include
    fever, a dry cough, fatigue and difficulty breathing or shortness of
    breath.} Some of these symptoms overlap with those of the flu,
    making detection difficult, but runny noses and stuffy sinuses are
    less common.
    \href{https://www.nytimes.com/2020/04/27/health/coronavirus-symptoms-cdc.html?action=click\&pgtype=Article\&state=default\&region=MAIN_CONTENT_3\&context=storylines_faq}{The
    C.D.C. has also} added chills, muscle pain, sore throat, headache
    and a new loss of the sense of taste or smell as symptoms to look
    out for. Most people fall ill five to seven days after exposure, but
    symptoms may appear in as few as two days or as many as 14 days.
  \end{itemize}
\item ~
  \hypertarget{does-asymptomatic-transmission-of-covid-19-happen}{%
  \paragraph{Does asymptomatic transmission of Covid-19
  happen?}\label{does-asymptomatic-transmission-of-covid-19-happen}}

  \begin{itemize}
  \tightlist
  \item
    So far, the evidence seems to show it does. A widely cited
    \href{https://www.nature.com/articles/s41591-020-0869-5}{paper}
    published in April suggests that people are most infectious about
    two days before the onset of coronavirus symptoms and estimated that
    44 percent of new infections were a result of transmission from
    people who were not yet showing symptoms. Recently, a top expert at
    the World Health Organization stated that transmission of the
    coronavirus by people who did not have symptoms was ``very rare,''
    \href{https://www.nytimes.com/2020/06/09/world/coronavirus-updates.html?action=click\&pgtype=Article\&state=default\&region=MAIN_CONTENT_3\&context=storylines_faq\#link-1f302e21}{but
    she later walked back that statement.}
  \end{itemize}
\end{itemize}

The provision could potentially benefit Mr. Trump's companies, among
many others, by allowing them to immediately write off money spent on
renovations at hotels or restaurants, instead of having to take the
deduction over 37 years.

Industry lobbyists have been pushing for the change for more than a
year. They have called it a technical correction to the 2017 tax
legislation that Mr. Trump signed into law. The provision is simply
called ``Technical Amendments Regarding Qualified Improvement
Property.''

It would allow hotels, restaurants and retailers that have spent money
fixing up their properties in the last two years to accelerate the way
they write off those expenses, effectively giving them an immediate tax
refund, Ms. Bernstein, the industry lobbyist said. They could then use
the refund to help cover bills during the crisis. The special tax
benefit would be retroactive to 2018 and would last for at least three
more years, before it is gradually phased out.

``If you let us amend our returns, we will be getting billions back and
it will help us pay our employees and our rents and stay in business
until consumers can come back into our stores,'' Ms. Bernstein said,
echoing the argument she said she made to lawmakers on Capitol Hill.

Meanwhile, the provision inserted by Democrats to block the families of
government officials from receiving certain assistance might not exempt
the companies owned by the family of Mr. Kushner.

While the provision expressly bars such funds from going to companies
controlled by ``the spouse, child, son-in-law or daughter-in-law'' of
the president and other officials, in order for the prohibition to kick
in, the person in question would have to ``directly or indirectly'' own
or control **** 20 percent or more of a company. Mr. Kushner rarely owns
that much in his family firm's various real estate projects, according
to a person familiar with the family's business arrangements. The
ownership is usually divided between Mr. Kushner, his three siblings,
his two parents and various outside investors.

Mr. Kushner's representatives did not immediately respond to a request
for comment.

Image

Restaurants and retailers could accelerate the way they write off
expenses for renovations.Credit...Jordan Gale for The New York Times

Another business that could benefit from the bill is for-profit
colleges, which have been
\href{https://www.nytimes.com/2019/06/28/us/politics/betsy-devos-for-profit-colleges.html}{championed
by some Republicans}, but targeted as predatory by Democrats and
advocates for student borrowers.

A provision in the bill would allow all colleges to retain federal funds
allocated to help educate qualifying students, even if the students in
question dropped out because of coronavirus-related emergencies. While
the provision applies to all colleges, critics of for-profit colleges
contend that, because those schools tend to have higher dropout rates,
they would be able to retain more of the money they collect via federal
loans to their students than would traditional nonprofit colleges.

``What's happening now is causing a crisis for all sectors of higher ed,
and I understand the intent, but it would disproportionately help
for-profit schools because their dropout rates are higher than other
segments of higher ed,'' said Toby Merrill, the founder of the Project
on Predatory Student Lending.

Both parties jammed in provisions to help favored constituencies. The
deal included \$25 million for the John F. Kennedy Center for the
Performing Arts in Washington --- money that has
\href{https://www.foxnews.com/media/dana-perino-democrats-coronavirus-bill-kennedy-center}{generated
complaints} from some conservatives.

It also includes \$7.5 million for the Smithsonian Institution, as well
as \$75 million each for the National Endowment for the Arts and the
National Endowment for the Humanities to provide grants to arts
organizations, museums and libraries during the coronavirus outbreak.

The bill also contains a six-month extension of federal funding through
the end of November for abstinence-only education programs favored by
social conservatives who are a critical Republican voting bloc. The
extension is coupled with one for sex education programs that provide
information about birth control and safe sex, which are supported by
reproductive rights groups that tend to back Democrats.

Provisions sought by the nation's smaller banks, a powerful
constituency, were also included in the bill. One change would allow
those banks to have lower requirements for capital reserves, the buffer
that financial institutions are required to keep on hand to ensure they
remain solvent if they run into trouble.

The industry late last year had fought for the change, and lost. This
time the banks used the disaster to reopen the fight and get the lower
capital requirements they wanted, arguing that it would allow them to do
more lending to small businesses during the coronavirus emergency. The
change would remain in effect as long as the virus emergency continued,
although the industry is already discussing looking for ways to make it
permanent.

Jesse Drucker contributed reporting from New York.

Advertisement

\protect\hyperlink{after-bottom}{Continue reading the main story}

\hypertarget{site-index}{%
\subsection{Site Index}\label{site-index}}

\hypertarget{site-information-navigation}{%
\subsection{Site Information
Navigation}\label{site-information-navigation}}

\begin{itemize}
\tightlist
\item
  \href{https://help.nytimes.com/hc/en-us/articles/115014792127-Copyright-notice}{©~2020~The
  New York Times Company}
\end{itemize}

\begin{itemize}
\tightlist
\item
  \href{https://www.nytco.com/}{NYTCo}
\item
  \href{https://help.nytimes.com/hc/en-us/articles/115015385887-Contact-Us}{Contact
  Us}
\item
  \href{https://www.nytco.com/careers/}{Work with us}
\item
  \href{https://nytmediakit.com/}{Advertise}
\item
  \href{http://www.tbrandstudio.com/}{T Brand Studio}
\item
  \href{https://www.nytimes.com/privacy/cookie-policy\#how-do-i-manage-trackers}{Your
  Ad Choices}
\item
  \href{https://www.nytimes.com/privacy}{Privacy}
\item
  \href{https://help.nytimes.com/hc/en-us/articles/115014893428-Terms-of-service}{Terms
  of Service}
\item
  \href{https://help.nytimes.com/hc/en-us/articles/115014893968-Terms-of-sale}{Terms
  of Sale}
\item
  \href{https://spiderbites.nytimes.com}{Site Map}
\item
  \href{https://help.nytimes.com/hc/en-us}{Help}
\item
  \href{https://www.nytimes.com/subscription?campaignId=37WXW}{Subscriptions}
\end{itemize}
