Sections

SEARCH

\protect\hyperlink{site-content}{Skip to
content}\protect\hyperlink{site-index}{Skip to site index}

\href{https://www.nytimes.com/section/business}{Business}

\href{https://myaccount.nytimes.com/auth/login?response_type=cookie\&client_id=vi}{}

\href{https://www.nytimes.com/section/todayspaper}{Today's Paper}

\href{/section/business}{Business}\textbar{}Sol Kerzner, South African
Casino Tycoon, Is Dead at 84

\url{https://nyti.ms/2UKZ6d2}

\begin{itemize}
\item
\item
\item
\item
\item
\end{itemize}

Advertisement

\protect\hyperlink{after-top}{Continue reading the main story}

Supported by

\protect\hyperlink{after-sponsor}{Continue reading the main story}

\hypertarget{sol-kerzner-south-african-casino-tycoon-is-dead-at-84}{%
\section{Sol Kerzner, South African Casino Tycoon, Is Dead at
84}\label{sol-kerzner-south-african-casino-tycoon-is-dead-at-84}}

His Sun City resort let black and white people mingle in the apartheid
era, but some saw it as an emblem of all that was wrong with that racial
system.

\includegraphics{https://static01.nyt.com/images/2020/03/29/obituaries/29Kerzner-obit1/merlin_170991291_906108c3-1dbb-4f7c-af97-0d559ff36f0c-articleLarge.jpg?quality=75\&auto=webp\&disable=upscale}

By \href{https://www.nytimes.com/by/alan-cowell}{Alan Cowell}

\begin{itemize}
\item
  Published March 27, 2020Updated March 30, 2020
\item
  \begin{itemize}
  \item
  \item
  \item
  \item
  \item
  \end{itemize}
\end{itemize}

Sol Kerzner, a sharp-elbowed tycoon and developer of luxurious hotels,
casinos and resorts, who used the splintered geography of ethnic
division in his native South Africa to profit hugely by circumventing
the social and sexual strictures of apartheid, died on March 20 at his
home in Cape Town. He was 84.

The cause was cancer, his family
\href{https://www.prnewswire.com/in/news-releases/sol-kerzner-1935-2020-visionary-south-african-hotelier-left-an-indelible-mark-on-the-global-hospitality-industry-897064696.html}{said
in a statement}.

Mr. Kerzner's story was often depicted as a poor-boy-makes-good one,
from beginnings in a blue-collar neighborhood of Johannesburg to
membership in an international cabal of tycoons and celebrities like
Frank Sinatra and Liza Minnelli. His empire stretched from the United
States to China by way of the Bahamas, Morocco, Mauritius, Dubai and
elsewhere. In the 1990s he was labeled a South African version of Donald
J. Trump.

For all his international profile, though, his name was most closely
associated with\href{https://www.sun-city-south-africa.com/}{Sun City},
a gaudy hotel, casino and golf complex with a 6,000-seat arena, situated
about 90 miles from Johannesburg. Starting in 1975, Mr. Kerzner oversaw
its creation, hewn from raw bushlands and rising in a jumble of
architectural whimsy in what was then the nominally independent homeland
of Bophuthatswana.

The so-called homelands --- known derisively as bantustans --- formed a
pillar of apartheid, created to strip black South Africans of
citizenship and assign to them a nationality based on the ethnicity of
their notional new states. Bophuthatswana was intended for people of
Tswana descent.

The homelands won no international recognition and were reabsorbed into
South Africa after the 1994 all-race elections that brought Nelson
Mandela and his African National Congress to power.

In the late 1970s and '80s, however, even as South Africa nudged into
ever-sharpening conflict between its white minority leaders and its
black majority, Sun City seemed a creation of staggering chutzpah.
Bophuthatswana had no restrictions on gambling and did not share
apartheid's puritanism in matters of sex and race.

White South Africans could thus drive a couple of hours from
Johannesburg to play the tables, feed the slots or watch topless revues
at Sun City. And there, black and white people could do what was
forbidden in the rest of South Africa: mingle and frolic freely.

``It was a place all South Africans could enjoy irrespective of their
race,'' Mr. Kerzner told The Financial Times in 2010. Brash and flashy,
the complex became known as Sin City, and the flamboyant Mr. Kerzner,
often in the headlines because of his succession of romances, became
known as the Sun King. Some people called the complex South Africa's Las
Vegas.

\includegraphics{https://static01.nyt.com/images/2020/03/30/obituaries/27Kerzner2/27Kerzner2-articleLarge.jpg?quality=75\&auto=webp\&disable=upscale}

In 1992, he expanded Sun City with the addition of the 62-acre Lost
City,
\href{https://www.nytimes.com/1992/12/03/world/resort-not-too-african-for-rich-tourists.html}{described
by The New York Times} as ``the most audacious and most deafeningly
hyped theme resort in the Southern hemisphere, at least.''

By arguing that Sun City was a place apart from the encircling South
Africa, Mr. Kerzner lured an array of entertainers and sports
personalities to appear there, paying them huge fees. Jack Nicklaus,
Elton John and Freddie Mercury were just a few who showed up. There were
boxing and golf events with lucrative payoffs.

To antiapartheid protesters outside the country, however, Sun City
became an emblem of all that was wrong about the system of racial
separation. In 1985, a group of entertainers called Artists United
Against Apartheid produced a song called ``Sun City,'' whose chorus line
proclaimed, ``Ain't going to play Sun City.''

Asked by The Financial Times in 2010 how he dealt with criticism, Mr.
Kerzner replied: ``I don't. I have a saying: The dogs bark and the
caravan moves on.''

Nonetheless, he was hounded by charges --- later dropped in South Africa
--- that in 1986 he participated in a roughly \$900,000 bribe to the
leader of Transkei, another so-called homeland, in return for a monopoly
on its gambling industry. Mr. Kerzner was quoted as saying that the
money had been extortion demanded by its recipients.

In 1997, after Mr. Kerzner had applied for a license to operate a casino
in Atlantic City, N.J., the state's Casino Control Commission ruled that
while he had ``committed bribery under New Jersey law by a preponderance
of the evidence,'' the ``unsavory aspects'' of his dealings in Transkei
were an ``aberration that occurred a decade ago.'' Adding that he had
``convincingly demonstrated good character, honesty and integrity,''
\href{https://www.nytimes.com/1997/10/23/nyregion/south-african-gets-approval-for-a-casino-in-new-jersey.html}{it
granted him the license}.

There was controversy, too, about a contribution of two million rand ---
worth about \$500,000 at the time --- that he made to Mr. Mandela's
election campaign in 1994 at a time when Mr. Kerzner was still under
investigation in the Transkei affair.

Mr. Mandela easily won the presidency and invited Mr. Kerzner to arrange
the celebration of his inauguration. ``Theirs was a genuine friendship
that would endure until Mandela's own passing in 2013,'' the Kerzner
family statement said.

Mr. Mandela
\href{https://www.theguardian.com/business/2006/may/07/theobserver.observerbusiness10}{was
quoted} as saying that the Kerzners were ``an example of a family not
only interested in their own enrichment, but willing to give something
back to their own country.''

Image

Mr. Kerzner with Nelson Mandela and his wife, Graca Machel, in 2009 in
Cape Town. Behind them is Mr. Kerzner's wife at the time, Heather
Kerzner. ``Theirs was a genuine friendship,'' the Kerzner family said of
Mr. Mandela and Mr. Kerzner.Credit...Chris Jackson/Getty Images

Solomon Kerzner was born on Aug. 23, 1935, the son of Jewish immigrants
from Russia who ran a chain of kosher hotels and lived in a down-market
neighborhood of Johannesburg known as Bez Valley. He was the youngest of
four children and the only son.

He once said that his father, identified in one account as Morris
Kerzner, had been the greatest influence on his life, persuading him to
secure a degree in accounting at the University of the Witwatersrand in
Johannesburg. His father was also credited with persuading him to take
up boxing in his early teens after he had been bullied at school because
of his Jewish faith. By the time he graduated from college, he had
become its welterweight boxing champion.

Mr. Kerzner's first venture as a hotelier began in 1962, according to
the family, when he abandoned accounting and bought a modest inn, the
Astra, in the Indian Ocean port city of Durban, which became one of
South Africa's principal vacation resorts. He went on to much bigger
projects, opening what the family described as South Africa's first
five-star luxury hotel, at Umhlanga Rocks, north of Durban. He called it
the Beverly Hills.

In partnership with South African Breweries, Mr. Kerzner established a
company called Southern Sun Hotels, which was operating 30 luxury hotels
by 1983.

In 1994, he acquired the bankrupt Paradise Island Resort in the Bahamas
and converted it into a 2,300-room resort called Atlantis --- a brand
that he also used in Dubai in 2008, when he opened the \$1.5 billion,
1,500-room Atlantis, the Palm. His company said a fireworks display, the
centerpiece of a \$20 million launch party, had been the world's biggest
and had been visible from space.

Working with his son Howard, who was known as Butch, Mr. Kerzner built
his first casino in the United States in 1996, on an Indian reservation
in eastern Connecticut, naming it the
\href{https://mohegansun.com/}{Mohegan Sun.}

A decade later, the Kerzners took their company private for \$3.6
billion, including debt --- a move described by some analysts as a rare
misstep in light of the financial chaos that was about to roil the
global tourism industry.

By 2011, the company had renegotiated the debt, effectively becoming a
management company rather than an owner-operator,
\href{https://www.reuters.com/article/kerzner-bahamas/update-1-kerzner-transfers-atlantis-resort-to-brookfield-idUSN1E7AS18S20111129}{Reuters
reported}. He retired to his family's 25-acre estate near Cape Town in
2014, four years after he was knighted by Queen Elizabeth II for
services to the Bahamas.

His personal life, well chronicled by gossip columnists, was one of
extravagance and ostentation punctured by tragedy. In 2006, Butch
Kerzner was killed in a helicopter crash in the Dominican Republic. The
second of his four wives, Shirley Bestbier, committed suicide soon after
the birth of their second child.

His first marriage, to Maureen Adler, ended in divorce. His third
marriage, to Anneline Kriel, who, representing South Africa, was crowned
Miss World in 1974, lasted from 1980 to 1985. In 2000 he married Heather
Murphy, a model; they divorced in 2011.

He is survived by four children: Andrea and Beverley, from his first
marriage, and Brandon and Chantal, from his second.

Advertisement

\protect\hyperlink{after-bottom}{Continue reading the main story}

\hypertarget{site-index}{%
\subsection{Site Index}\label{site-index}}

\hypertarget{site-information-navigation}{%
\subsection{Site Information
Navigation}\label{site-information-navigation}}

\begin{itemize}
\tightlist
\item
  \href{https://help.nytimes.com/hc/en-us/articles/115014792127-Copyright-notice}{©~2020~The
  New York Times Company}
\end{itemize}

\begin{itemize}
\tightlist
\item
  \href{https://www.nytco.com/}{NYTCo}
\item
  \href{https://help.nytimes.com/hc/en-us/articles/115015385887-Contact-Us}{Contact
  Us}
\item
  \href{https://www.nytco.com/careers/}{Work with us}
\item
  \href{https://nytmediakit.com/}{Advertise}
\item
  \href{http://www.tbrandstudio.com/}{T Brand Studio}
\item
  \href{https://www.nytimes.com/privacy/cookie-policy\#how-do-i-manage-trackers}{Your
  Ad Choices}
\item
  \href{https://www.nytimes.com/privacy}{Privacy}
\item
  \href{https://help.nytimes.com/hc/en-us/articles/115014893428-Terms-of-service}{Terms
  of Service}
\item
  \href{https://help.nytimes.com/hc/en-us/articles/115014893968-Terms-of-sale}{Terms
  of Sale}
\item
  \href{https://spiderbites.nytimes.com}{Site Map}
\item
  \href{https://help.nytimes.com/hc/en-us}{Help}
\item
  \href{https://www.nytimes.com/subscription?campaignId=37WXW}{Subscriptions}
\end{itemize}
