Sections

SEARCH

\protect\hyperlink{site-content}{Skip to
content}\protect\hyperlink{site-index}{Skip to site index}

\href{https://www.nytimes.com/section/arts/music}{Music}

\href{https://myaccount.nytimes.com/auth/login?response_type=cookie\&client_id=vi}{}

\href{https://www.nytimes.com/section/todayspaper}{Today's Paper}

\href{/section/arts/music}{Music}\textbar{}Mike Longo, Jazz Pianist,
Composer and Educator, Dies at 83

\url{https://nyti.ms/2QU3m8T}

\begin{itemize}
\item
\item
\item
\item
\item
\end{itemize}

\href{https://www.nytimes.com/news-event/coronavirus?action=click\&pgtype=Article\&state=default\&region=TOP_BANNER\&context=storylines_menu}{The
Coronavirus Outbreak}

\begin{itemize}
\tightlist
\item
  live\href{https://www.nytimes.com/2020/08/03/world/coronavirus-covid-19.html?action=click\&pgtype=Article\&state=default\&region=TOP_BANNER\&context=storylines_menu}{Latest
  Updates}
\item
  \href{https://www.nytimes.com/interactive/2020/us/coronavirus-us-cases.html?action=click\&pgtype=Article\&state=default\&region=TOP_BANNER\&context=storylines_menu}{Maps
  and Cases}
\item
  \href{https://www.nytimes.com/interactive/2020/science/coronavirus-vaccine-tracker.html?action=click\&pgtype=Article\&state=default\&region=TOP_BANNER\&context=storylines_menu}{Vaccine
  Tracker}
\item
  \href{https://www.nytimes.com/2020/08/02/us/covid-college-reopening.html?action=click\&pgtype=Article\&state=default\&region=TOP_BANNER\&context=storylines_menu}{College
  Reopening}
\item
  \href{https://www.nytimes.com/live/2020/08/03/business/stock-market-today-coronavirus?action=click\&pgtype=Article\&state=default\&region=TOP_BANNER\&context=storylines_menu}{Economy}
\end{itemize}

Advertisement

\protect\hyperlink{after-top}{Continue reading the main story}

Supported by

\protect\hyperlink{after-sponsor}{Continue reading the main story}

THOSE we've lost

\hypertarget{mike-longo-jazz-pianist-composer-and-educator-dies-at-83}{%
\section{Mike Longo, Jazz Pianist, Composer and Educator, Dies at
83}\label{mike-longo-jazz-pianist-composer-and-educator-dies-at-83}}

Best known for his long association with Dizzy Gillespie, Mr. Longo, who
died of the coronavirus, also led a big band and promoted the work of
other musicians.

\includegraphics{https://static01.nyt.com/images/2020/04/01/obituaries/26Longo1-sub/26Longo1-sub-articleLarge.jpg?quality=75\&auto=webp\&disable=upscale}

By Steve Smith

\begin{itemize}
\item
  Published March 28, 2020Updated April 16, 2020
\item
  \begin{itemize}
  \item
  \item
  \item
  \item
  \item
  \end{itemize}
\end{itemize}

\emph{This obituary is part of a series about people who have died in
the coronavirus pandemic. Read about others}
\href{https://www.nytimes.com/series/people-who-have-died-of-the-coronavirus}{\emph{here}}\emph{.}

\href{https://www.mikelongojazz.com/}{Mike Longo}, a jazz pianist,
composer and educator best known for his long association with the
trumpeter Dizzy Gillespie, died on March 22 in Manhattan. He was 83.

The cause was the coronavirus, Dorothy Longo, his wife of 32 years,
said.

As a musician and a composer, said Matthew Snyder, who had studied
composition with Mr. Longo and played baritone saxophone with the big
band he led, the \href{https://www.youtube.com/watch?v=Xr-lHtW5wDA}{New
York State of the Art Jazz Ensemble}, Mr. Longo ``was simultaneously
very earthy and also had the highest possible level of harmony and
melodicism and complexity in his musical conception.''

As an educator, Mr. Longo wrote 10 books and produced four DVDs,
espousing concepts he had refined while working with Mr. Gillespie. He
also advocated tirelessly for other artists, engaging them for concerts
and releasing their recordings on
\href{https://jazztimes.com/archives/label-watch-consolidated-artists/}{CAP
(Consolidated Artists Productions)}, which he had established as a
publishing company in 1970 and a record label in 1981.

``He took on other artists because he wanted them to have a forum to
produce their own music and express their creativity,'' Ms. Longo said
in an email. ``CAP is an umbrella organization whereby musicians
produced and owned their own product, but if Mike chose to take them on,
because of his reputation, he was able to get airplay and
distribution.''

Born into a musical household, Mr. Longo played his first nightclub
date, with the alto saxophonist Cannonball Adderley, while still in high
school. After arriving in New York in 1960, he found work supporting
musicians like the trumpeter Red Allen and the tenor saxophonist Coleman
Hawkins at the Metropole, a Manhattan nightclub. A year later, he moved
to Toronto to study with the pianist Oscar Peterson.

Returning to New York in 1962, Mr. Longo became an in-demand accompanist
for singers including Nancy Wilson, Gloria Lynne and Joe Williams. In
1965 he led a house band at the New York nightclub Embers West, where he
performed with a wide range of luminaries. A year later, Mr. Gillespie
engaged him as his musical director and arranger, an association that
would endure until 1975, and informally until shortly before
\href{https://www.nytimes.com/1993/01/07/arts/dizzy-gillespie-who-sounded-some-of-modern-jazz-s-earliest-notes-dies-at-75.html}{Mr.
Gillespie's death in 1993}.

Mr. Longo went on to perform and record solo, in duos
and\href{https://www.youtube.com/watch?v=8bgr-fDptYQ}{trios}, and with
the New York State of the Art Jazz Ensemble, which he founded in 1998.

``Mike's book was roughly split between his arrangements of other tunes
and his original tunes,'' Mr. Snyder said of Mr. Longo's repertoire,
``and it was obvious it was all the same thing for him; even his
arrangements were recompositions.''

Image

Mr. Longo was still with Mr. Gillespie when he released the album
``Matrix'' in 1972. He would continue to perform and would record
prolifically as a bandleader, arranger and composer after leaving Mr.
Gillespie's band in 1975.

Michael Joseph Longo was born on March 19, 1937, in Cincinnati, to
Michael Anthony Longo and Elvira Margaret (Vitello) Longo. He began to
study piano with his mother, a homemaker who sang and played the piano
and the organ, at age 3, starting formal lessons a year later. The
family moved to Fort Lauderdale, Fla., where Mr. Longo's father
established a successful business supplying produce to stores and to
restaurants while also leading bands in which he played bass.

Mr. Longo's father hired Mr. Adderley, who was black, to play in his
band at a time when racial mixing was uncommon and potentially perilous.
Mr. Adderley in turn took young Mr. Longo under his wing, engaging him
for church performances and, on one occasion, an engagement at Porky's
Hideaway, a Fort Lauderdale jazz club.

Mr. Longo studied classical piano at Western Kentucky University,
graduating in 1959 with a B.A. in music. Offered a scholarship by the
jazz magazine DownBeat, he opted instead to pursue his education on the
road with a small combo, the Salt City Six, and then in New York. His
studies with Mr. Peterson in Toronto, Mr. Longo recalled in a 2006
interview with the website All About Jazz, taught him ``how to play
piano and how to be a jazz pianist --- textures, voicings, touch, time,
conception, tone on the instrument.''

Mr. Longo studied composition privately with
\href{https://www.nytimes.com/1972/11/26/archives/hall-overton-of-juilliard-dead-symphonic-and-jazz-composer.html}{Hall
Overton} from 1970 to 1972 and worked prolifically as a bandleader,
arranger and composer after leaving Mr. Gillespie's employ. But his
association with Mr. Gillespie would dominate much of his professional
career, even offering him the opportunity to compose an orchestral work,
``A World of Gillespie'' (1980), which Mr. Gillespie performed with the
Detroit Symphony Orchestra.

In addition to his wife, Mr. Longo is survived by a sister, Ellen.

Like Mr. Gillespie, Mr. Longo embraced the Baha'i faith, a religion that
espouses the unity of all people and finds truth in multiple faith
traditions. In 2004, he began leading weekly concerts at the
\href{http://bahainyc.org/}{New York City Baha'i Center} in Greenwich
Village. The last concert was on March 10.

\href{https://www.nytimes.com/interactive/2020/obituaries/people-died-coronavirus-obituaries.html?action=click\&pgtype=Article\&state=default\&region=BELOW_MAIN_CONTENT\&context=covid_obits_promo}{}

\hypertarget{those-weve-lost}{%
\section{Those We've Lost}\label{those-weve-lost}}

The coronavirus pandemic has taken an incalculable death toll. This
series is designed to put names and faces to the numbers.

Read more

\includegraphics{https://static01.nyt.com/images/2020/07/30/obituaries/30Pedro/30Pedro-square640.jpg}

\hypertarget{bernaldina-josuxe9-pedro}{%
\section{Bernaldina José Pedro}\label{bernaldina-josuxe9-pedro}}

d. Boa Vista, Brazil

Leader among the Indigenous Macuxi

\includegraphics{https://static01.nyt.com/images/2020/07/31/obituaries/31Swing/merlin_175167783_8913bc90-0d64-43f3-a655-1bb1bf1601c9-square640.jpg}

\hypertarget{john-eric-swing}{%
\section{John Eric Swing}\label{john-eric-swing}}

d. Fountain Valley, Calif.

Champion of Filipino-Americans

\includegraphics{https://static01.nyt.com/images/2020/07/27/obituaries/27Victor/merlin_175001436_38b11f8e-227a-4e2c-9821-7618af9b2524-square640.jpg}

\hypertarget{victor-victor}{%
\section{Victor Victor}\label{victor-victor}}

d. Santo Domingo, Dominican Republic

Beloved musician of the Dominican Republic

\includegraphics{https://static01.nyt.com/images/2020/07/31/obituaries/31Negron/merlin_175160169_516322ae-fd23-4969-b6b2-193ced371105-square640.jpg}

\hypertarget{dr-eddie-negruxf3n}{%
\section{Dr. Eddie Negrón}\label{dr-eddie-negruxf3n}}

d. Fort Walton Beach, Fla.

Internist on Florida's Emerald Coast

\includegraphics{https://static01.nyt.com/images/2020/07/30/obituaries/30Dobson/merlin_175115928_f6b9271c-8f05-4fe1-a38a-5ca4a58f8935-square640.jpg}

\hypertarget{dobby-dobson}{%
\section{Dobby Dobson}\label{dobby-dobson}}

d. Coral Springs, Fla.

Jamaican singer and songwriter

\includegraphics{https://static01.nyt.com/images/2020/08/01/obituaries/28Gonzalez/merlin_175002771_beb57888-3951-409a-ae13-03a94b2e962e-square640.jpg}

\hypertarget{waldemar-gonzalez}{%
\section{Waldemar Gonzalez}\label{waldemar-gonzalez}}

d. White Plains, N.Y.

Teacher and social worker

Advertisement

\protect\hyperlink{after-bottom}{Continue reading the main story}

\hypertarget{site-index}{%
\subsection{Site Index}\label{site-index}}

\hypertarget{site-information-navigation}{%
\subsection{Site Information
Navigation}\label{site-information-navigation}}

\begin{itemize}
\tightlist
\item
  \href{https://help.nytimes.com/hc/en-us/articles/115014792127-Copyright-notice}{©~2020~The
  New York Times Company}
\end{itemize}

\begin{itemize}
\tightlist
\item
  \href{https://www.nytco.com/}{NYTCo}
\item
  \href{https://help.nytimes.com/hc/en-us/articles/115015385887-Contact-Us}{Contact
  Us}
\item
  \href{https://www.nytco.com/careers/}{Work with us}
\item
  \href{https://nytmediakit.com/}{Advertise}
\item
  \href{http://www.tbrandstudio.com/}{T Brand Studio}
\item
  \href{https://www.nytimes.com/privacy/cookie-policy\#how-do-i-manage-trackers}{Your
  Ad Choices}
\item
  \href{https://www.nytimes.com/privacy}{Privacy}
\item
  \href{https://help.nytimes.com/hc/en-us/articles/115014893428-Terms-of-service}{Terms
  of Service}
\item
  \href{https://help.nytimes.com/hc/en-us/articles/115014893968-Terms-of-sale}{Terms
  of Sale}
\item
  \href{https://spiderbites.nytimes.com}{Site Map}
\item
  \href{https://help.nytimes.com/hc/en-us}{Help}
\item
  \href{https://www.nytimes.com/subscription?campaignId=37WXW}{Subscriptions}
\end{itemize}
