Sections

SEARCH

\protect\hyperlink{site-content}{Skip to
content}\protect\hyperlink{site-index}{Skip to site index}

\href{https://www.nytimes.com/section/arts/music}{Music}

\href{https://myaccount.nytimes.com/auth/login?response_type=cookie\&client_id=vi}{}

\href{https://www.nytimes.com/section/todayspaper}{Today's Paper}

\href{/section/arts/music}{Music}\textbar{}Wallace Roney, Jazz Trumpet
Virtuoso, Is Dead at 59

\url{https://nyti.ms/3aEv2GQ}

\begin{itemize}
\item
\item
\item
\item
\item
\item
\end{itemize}

\href{https://www.nytimes.com/news-event/coronavirus?action=click\&pgtype=Article\&state=default\&region=TOP_BANNER\&context=storylines_menu}{The
Coronavirus Outbreak}

\begin{itemize}
\tightlist
\item
  live\href{https://www.nytimes.com/2020/08/03/world/coronavirus-covid-19.html?action=click\&pgtype=Article\&state=default\&region=TOP_BANNER\&context=storylines_menu}{Latest
  Updates}
\item
  \href{https://www.nytimes.com/interactive/2020/us/coronavirus-us-cases.html?action=click\&pgtype=Article\&state=default\&region=TOP_BANNER\&context=storylines_menu}{Maps
  and Cases}
\item
  \href{https://www.nytimes.com/interactive/2020/science/coronavirus-vaccine-tracker.html?action=click\&pgtype=Article\&state=default\&region=TOP_BANNER\&context=storylines_menu}{Vaccine
  Tracker}
\item
  \href{https://www.nytimes.com/2020/08/02/us/covid-college-reopening.html?action=click\&pgtype=Article\&state=default\&region=TOP_BANNER\&context=storylines_menu}{College
  Reopening}
\item
  \href{https://www.nytimes.com/live/2020/08/03/business/stock-market-today-coronavirus?action=click\&pgtype=Article\&state=default\&region=TOP_BANNER\&context=storylines_menu}{Economy}
\end{itemize}

Advertisement

\protect\hyperlink{after-top}{Continue reading the main story}

Supported by

\protect\hyperlink{after-sponsor}{Continue reading the main story}

THOSE WE'VE LOST

\hypertarget{wallace-roney-jazz-trumpet-virtuoso-is-dead-at-59}{%
\section{Wallace Roney, Jazz Trumpet Virtuoso, Is Dead at
59}\label{wallace-roney-jazz-trumpet-virtuoso-is-dead-at-59}}

Initially dismissed by some as a Miles Davis imitator, Mr. Roney, who
has died of coronavirus complications, emerged as a major musician in
his own right.

\includegraphics{https://static01.nyt.com/images/2020/04/02/obituaries/31Roney2/31Roney2-articleLarge-v2.jpg?quality=75\&auto=webp\&disable=upscale}

\href{https://www.nytimes.com/by/giovanni-russonello}{\includegraphics{https://static01.nyt.com/images/2019/04/03/multimedia/author-giovanni-russonello/author-giovanni-russonello-thumbLarge.png}}

By \href{https://www.nytimes.com/by/giovanni-russonello}{Giovanni
Russonello}

\begin{itemize}
\item
  Published March 31, 2020Updated April 16, 2020
\item
  \begin{itemize}
  \item
  \item
  \item
  \item
  \item
  \item
  \end{itemize}
\end{itemize}

\emph{This obituary is part of a series about}
\href{https://www.nytimes.com/series/people-who-have-died-of-the-coronavirus}{\emph{people
who have died in the coronavirus pandemic}}\emph{.}

Wallace Roney, a virtuoso trumpeter whose term as Miles Davis's only
true protégé opened onto a prominent career in jazz, died on Tuesday in
Paterson, N.J. He was 59.

The cause was complications of the coronavirus, his fiancée, Dawn Jones,
said.

By the time he linked up with Davis, Mr. Roney was already a leading
voice in what came to be called the Young Lions movement,~a coterie of
young musicians devoted to bringing jazz back into line with its
midcentury sound. And he was already associated --- sometimes
distressingly so --- with Davis's legacy. Many dismissed him as a
musical clone: ravishingly talented but lacking the necessary distance
from his idol to claim creative agency.

Yet as his career went on, Mr. Roney managed to neutralize most of those
criticisms. His nuanced understanding of Davis's playing --- its
harmonic and rhythmic wirings as well as its smoldering tone ---~was
only part of a vast musical ken. His own style bespoke an investment in
the entire lineage of jazz trumpet playing.

Most of the ideas in Mr. Roney's compositions began at the center of
jazz's mainstream language and cut a path outward, often by way of funk,
hip-hop, pop, Brazilian or Afro-Caribbean music.

Mr. Roney made nearly 20 albums as a bandleader, including three for
Warner Bros. at the peak of the Young Lions era, all grounded in his
unshakable linguistic command and his appetite for harmonic adventure.
His recordings for Muse in the late 1980s and early '90s --- especially
\href{https://www.youtube.com/watch?v=DJvapwO30Ms}{his 1987 debut,
``Verses''}~--- featured a mix of A-list jazz musicians from Mr. Roney's
generation and the one before, and they established him as a premier
young bandleader.

In his New York Times review of a 1988 concert by the drummer Tony
Williams's quintet, Jon Pareles
\href{https://www.nytimes.com/1988/03/26/arts/review-jazz-a-drummer-who-goads-the-soloists.html}{singled
out Mr. Roney} as ``the standout soloist, bitingly articulate at fast
tempos and lucidly melodic in gentler passages.''

Profiling Mr. Roney in The Washington Post in 1987, James McBride ---
who later became a prizewinning novelist ---
\href{https://www.washingtonpost.com/archive/lifestyle/1987/12/12/wallace-roney-and-the-quest-to-be-heard/8d34f342-850e-49e6-91ac-641832df07d1/}{declared}:
``His name is Wallace Roney III. He is 27 years old. He is from
Washington, and he is one of the best jazz trumpet players in the
world.''

The two albums that Mr. Roney released in the early 2000s, immediately
after leaving Warner Bros., were among his most memorable, and more
formally ambitious than his early work. They represented a flush of
creativity after years of frustration under contract to a label that
often imposed unwelcome creative demands.

On ``No Room for Argument'' (2000), released on Stretch Records, Mr.
Roney struck a nimble balance between historical reverence and futurist
adventure, pairing a synthesizer with a Fender Rhodes electric piano
and, at one point, mashing up parts of John Coltrane's ``A Love
Supreme'' with Davis's ``Filles de Kilimanjaro.'' Its follow-up,
``Prototype'' (2004), for High Note, featured different sorts of homage:
separate reworkings of the titular Outkast ballad and Al Green's ``Let's
Stay Together.''

\includegraphics{https://static01.nyt.com/images/2020/03/31/obituaries/31Roney1/merlin_13183961_44d7fd11-b0cc-422c-bfd1-a22d9bf3340d-articleLarge.jpg?quality=75\&auto=webp\&disable=upscale}

Mr. Roney won a Grammy in 1994 for his participation in
\href{https://www.youtube.com/watch?v=r_DJJyJ5Ogg\&feature=emb_title}{``A
Tribute to Miles,''} filling the trumpet chair alongside the four
supporting members of Davis's second great quintet: Tony Williams, Wayne
Shorter, Herbie Hancock and Ron Carter. All were younger than Davis ---
and indeed, throughout the latter half of his career, Davis worked
almost exclusively with junior musicians. But before meeting Mr. Roney,
he had never agreed to mentor another trumpet player.

Struck by Mr. Roney's performance at
\href{https://www.nytimes.com/1983/11/09/arts/concert-davis-tribute.html}{a
1983 tribute concert at Radio City Music Hall}, Davis invited the young
trumpeter to join him at his home in Manhattan the next day. A close
friendship blossomed between the 23-year-old upstart and the ailing
elder, one that culminated in a momentous performance at the 1991
Montreux Jazz Festival, just months before Davis's death. It was the
only time Davis publicly revisited material from his back catalog.

With Quincy Jones conducting, the two trumpeters stood shoulder to
shoulder in what would become a
\href{https://www.facebook.com/JazzImprovisers/photos/a.969266783184007/1194385580672125/?type=1\&theater}{timeless
piece of postclassic jazz iconography}. Davis, wizened and
wire-thin,~hunched over a music stand alongside his burly young protégé,
who picked up the slack whenever his idol missed a note.

``A lot of people like to say, `Yeah, well, I hung with Miles, but we
never talked about music,''' Mr. Roney said in
\href{https://www.youtube.com/watch?v=8cxUYcUbdBg}{a 2016 interview}.
``Well, guess what? I did. I loved him because of his music, and he
talked to me about music all the time. You definitely had to earn Miles
Davis's respect, and not everybody could do that.''

Mr. Roney remembered that Davis --- whose birthday was just one day
apart from his --- had once told him, ``You look at me just like how I
used to look at Dizzy,'' referring to his own mentor Dizzy Gillespie.

Wallace Roney III was born on May 25, 1960, in Philadelphia, to Roberta
Sherman, a homemaker, and Wallace Roney Jr., a U.S. Marshal and vice
president of the American Federation of Government Employees. His
parents divorced when he was young, and he lived for a time with his
grandmother, Rosezell Roney.

In his teens, he lived with his father in Washington, enrolling in the
Duke Ellington School of the Arts. His father's friends were not
professional musicians, but they had an abiding devotion to jazz. Mr.
Roney often recalled that they would hold listening parties at which
each person would listen closely to a different instrument as a track
played, and then would compare notes.

The immersion in a music-loving family gave Mr. Roney a head start ---
but he was also loaded with preternatural talent. He had perfect pitch,
and he impressed his father by teaching himself the basics of the
trumpet using the family's horn, which had been lying around unused. At
12, he became the youngest member of the Philadelphia Brass, a
professional classical quintet.

By his midteens, he was already making trips to New York to perform. In
his city debut, in 1976, he played at Ali's Alley, a loft space in SoHo.

``As soon as Mr. Roney commenced to swing, the noise level in the club
immediately dropped off, and those in the middle of conversations or
laughing and joking turned their attention to the bandstand,'' the
critic Stanley Crouch later wrote of that show for
\href{https://www.nytimes.com/2000/09/24/arts/music-don-t-ask-the-critics-ask-wallace-roney-s-peers.html}{a
profile in The New York Times} in 2000. In the youthful trumpeter's
playing, Mr. Crouch wrote, ``the passion for jazz was so thorough that
the atmosphere inside the club was completely rearranged.''

``At the end of the tune, the room took on a crazily jubilant mood, and
the clapping wouldn't stop,'' Mr. Crouch added.

In addition to his fiancée, a vocalist and educator whom he had known
since high school, and his grandmother Rosezell, Mr. Roney is survived
by his sister, Crystal Roney; a brother, the saxophonist Antoine Roney;
two half sisters, April Petus and Marla Majett; a half brother, Michael
Majett; a son, Wallace Vernell Roney, a trumpeter~now on the rise on the
New York scene; and a daughter, Barbara Roney. His marriage to
\href{https://www.nytimes.com/2017/06/27/arts/music/geri-allen-dead-jazz.html}{Geri
Allen}, a noted pianist and frequent musical collaborator during Mr.
Roney's early career, ended in divorce.

In both 1979 and 1980, Mr. Roney won DownBeat magazine's award for best
young jazz musician of the year. A decade later, he pulled off a similar
double victory: He was voted trumpeter to watch in back-to-back DownBeat
critics' polls in 1989 and 1990.

He attended both Howard University and Berklee College of Music before
moving to New York City to pursue a career.

After years of lean times (jazz in particular was in a commercial slump
for much of the 1980s), he received two separate calls within the same
month inviting him to join the bands of Tony Williams and Art Blakey,
both pre-eminent elder drummers. He spent years in both ensembles before
his solo career took off.

Even in later years, Mr. Roney continued to balance his devotion to the
greats of jazz's past with an urge to~make his own way. In 2014, he
starred in the
\href{https://www.npr.org/2014/10/16/356653950/to-miles-from-wayne}{public
debut} of ``Universe,'' a large-ensemble suite that the saxophonist
Wayne Shorter wrote for Davis in the late 1960s, but that had never been
performed.

``I see my music as an extension of `Nefertiti,' `A Love Supreme,' Tony
Williams's Lifetime, Herbie's sextet and Miles' last band,'' Mr. Roney
said in
\href{https://jazztimes.com/features/profiles/wallace-roney-the-man-with-the-golden-horn/}{a
2004 interview with JazzTimes}.

``You could look at it as if Lifetime had a gig one night, and Miles sat
in, and Wayne came and played, and Herbie played and wrote some
arrangements, and Joe Zawinul came and sat in too, and Ron and Me'shell
Ndegeocello played bass, and Prince, Sly Stone, Bennie Maupin and Mos
Def dropped by,'' he said. ``That's part of what I'm doing.''

He added: ``The other part is updating it with stuff that I hear today,
the new synthesizers and the new sounds that appeal to me. I bring all
those elements together and still try to play what I consider
straight-ahead, innovative music.''

\href{https://www.nytimes.com/interactive/2020/obituaries/people-died-coronavirus-obituaries.html?action=click\&pgtype=Article\&state=default\&region=BELOW_MAIN_CONTENT\&context=covid_obits_promo}{}

\hypertarget{those-weve-lost}{%
\section{Those We've Lost}\label{those-weve-lost}}

The coronavirus pandemic has taken an incalculable death toll. This
series is designed to put names and faces to the numbers.

Read more

\includegraphics{https://static01.nyt.com/images/2020/07/30/obituaries/30Pedro/30Pedro-square640.jpg}

\hypertarget{bernaldina-josuxe9-pedro}{%
\section{Bernaldina José Pedro}\label{bernaldina-josuxe9-pedro}}

d. Boa Vista, Brazil

Leader among the Indigenous Macuxi

\includegraphics{https://static01.nyt.com/images/2020/07/31/obituaries/31Swing/merlin_175167783_8913bc90-0d64-43f3-a655-1bb1bf1601c9-square640.jpg}

\hypertarget{john-eric-swing}{%
\section{John Eric Swing}\label{john-eric-swing}}

d. Fountain Valley, Calif.

Champion of Filipino-Americans

\includegraphics{https://static01.nyt.com/images/2020/07/27/obituaries/27Victor/merlin_175001436_38b11f8e-227a-4e2c-9821-7618af9b2524-square640.jpg}

\hypertarget{victor-victor}{%
\section{Victor Victor}\label{victor-victor}}

d. Santo Domingo, Dominican Republic

Beloved musician of the Dominican Republic

\includegraphics{https://static01.nyt.com/images/2020/07/31/obituaries/31Negron/merlin_175160169_516322ae-fd23-4969-b6b2-193ced371105-square640.jpg}

\hypertarget{dr-eddie-negruxf3n}{%
\section{Dr. Eddie Negrón}\label{dr-eddie-negruxf3n}}

d. Fort Walton Beach, Fla.

Internist on Florida's Emerald Coast

\includegraphics{https://static01.nyt.com/images/2020/07/30/obituaries/30Dobson/merlin_175115928_f6b9271c-8f05-4fe1-a38a-5ca4a58f8935-square640.jpg}

\hypertarget{dobby-dobson}{%
\section{Dobby Dobson}\label{dobby-dobson}}

d. Coral Springs, Fla.

Jamaican singer and songwriter

\includegraphics{https://static01.nyt.com/images/2020/08/01/obituaries/28Gonzalez/merlin_175002771_beb57888-3951-409a-ae13-03a94b2e962e-square640.jpg}

\hypertarget{waldemar-gonzalez}{%
\section{Waldemar Gonzalez}\label{waldemar-gonzalez}}

d. White Plains, N.Y.

Teacher and social worker

Advertisement

\protect\hyperlink{after-bottom}{Continue reading the main story}

\hypertarget{site-index}{%
\subsection{Site Index}\label{site-index}}

\hypertarget{site-information-navigation}{%
\subsection{Site Information
Navigation}\label{site-information-navigation}}

\begin{itemize}
\tightlist
\item
  \href{https://help.nytimes.com/hc/en-us/articles/115014792127-Copyright-notice}{©~2020~The
  New York Times Company}
\end{itemize}

\begin{itemize}
\tightlist
\item
  \href{https://www.nytco.com/}{NYTCo}
\item
  \href{https://help.nytimes.com/hc/en-us/articles/115015385887-Contact-Us}{Contact
  Us}
\item
  \href{https://www.nytco.com/careers/}{Work with us}
\item
  \href{https://nytmediakit.com/}{Advertise}
\item
  \href{http://www.tbrandstudio.com/}{T Brand Studio}
\item
  \href{https://www.nytimes.com/privacy/cookie-policy\#how-do-i-manage-trackers}{Your
  Ad Choices}
\item
  \href{https://www.nytimes.com/privacy}{Privacy}
\item
  \href{https://help.nytimes.com/hc/en-us/articles/115014893428-Terms-of-service}{Terms
  of Service}
\item
  \href{https://help.nytimes.com/hc/en-us/articles/115014893968-Terms-of-sale}{Terms
  of Sale}
\item
  \href{https://spiderbites.nytimes.com}{Site Map}
\item
  \href{https://help.nytimes.com/hc/en-us}{Help}
\item
  \href{https://www.nytimes.com/subscription?campaignId=37WXW}{Subscriptions}
\end{itemize}
