Sections

SEARCH

\protect\hyperlink{site-content}{Skip to
content}\protect\hyperlink{site-index}{Skip to site index}

\href{https://www.nytimes.com/section/arts/music}{Music}

\href{https://myaccount.nytimes.com/auth/login?response_type=cookie\&client_id=vi}{}

\href{https://www.nytimes.com/section/todayspaper}{Today's Paper}

\href{/section/arts/music}{Music}\textbar{}Ellis Marsalis, Jazz Pianist
and Music Family Patriarch, Dies at 85

\url{https://nyti.ms/2wUxRF6}

\begin{itemize}
\item
\item
\item
\item
\item
\item
\end{itemize}

\href{https://www.nytimes.com/news-event/coronavirus?action=click\&pgtype=Article\&state=default\&region=TOP_BANNER\&context=storylines_menu}{The
Coronavirus Outbreak}

\begin{itemize}
\tightlist
\item
  live\href{https://www.nytimes.com/2020/08/03/world/coronavirus-covid-19.html?action=click\&pgtype=Article\&state=default\&region=TOP_BANNER\&context=storylines_menu}{Latest
  Updates}
\item
  \href{https://www.nytimes.com/interactive/2020/us/coronavirus-us-cases.html?action=click\&pgtype=Article\&state=default\&region=TOP_BANNER\&context=storylines_menu}{Maps
  and Cases}
\item
  \href{https://www.nytimes.com/interactive/2020/science/coronavirus-vaccine-tracker.html?action=click\&pgtype=Article\&state=default\&region=TOP_BANNER\&context=storylines_menu}{Vaccine
  Tracker}
\item
  \href{https://www.nytimes.com/2020/08/02/us/covid-college-reopening.html?action=click\&pgtype=Article\&state=default\&region=TOP_BANNER\&context=storylines_menu}{College
  Reopening}
\item
  \href{https://www.nytimes.com/live/2020/08/03/business/stock-market-today-coronavirus?action=click\&pgtype=Article\&state=default\&region=TOP_BANNER\&context=storylines_menu}{Economy}
\end{itemize}

Advertisement

\protect\hyperlink{after-top}{Continue reading the main story}

Supported by

\protect\hyperlink{after-sponsor}{Continue reading the main story}

Those we've lost

\hypertarget{ellis-marsalis-jazz-pianist-and-music-family-patriarch-dies-at-85}{%
\section{Ellis Marsalis, Jazz Pianist and Music Family Patriarch, Dies
at
85}\label{ellis-marsalis-jazz-pianist-and-music-family-patriarch-dies-at-85}}

The father of Wynton and Branford Marsalis and a prominent performer and
educator, he succumbed to complications of the coronavirus.

\includegraphics{https://static01.nyt.com/images/2020/04/03/us/politics/01marsalis-obit/01marsalis-obit-articleLarge-v2.jpg?quality=75\&auto=webp\&disable=upscale}

By \href{https://www.nytimes.com/by/giovanni-russonello}{Giovanni
Russonello} and
\href{https://www.nytimes.com/by/michael-levenson}{Michael Levenson}

\begin{itemize}
\item
  Published April 1, 2020Updated April 16, 2020
\item
  \begin{itemize}
  \item
  \item
  \item
  \item
  \item
  \item
  \end{itemize}
\end{itemize}

\emph{This obituary is part of a series about people who have died in
the coronavirus pandemic. Read about others}
\href{https://www.nytimes.com/series/people-who-have-died-of-the-coronavirus}{\emph{here}}\emph{.}

Ellis Marsalis, a pianist and educator who became the guiding force
behind a late-20th-century resurgence in jazz while putting four
musician sons on a path to prominent careers, died on Wednesday in New
Orleans. He was 85.

The cause was complications of Covid-19, the disease caused by the
coronavirus, his son Branford said in a statement.

Mr. Marsalis spent decades as a working musician and teacher in New
Orleans before his eldest sons, Wynton and Branford, gained national
fame in the early 1980s embodying a fresh-faced revival of traditional
jazz.

Mr. Marsalis's star rose along with theirs, and he, too, became a
household name.

``Ellis Marsalis was a legend,'' Mayor LaToya Cantrell of New Orleans
\href{https://twitter.com/mayorcantrell/status/1245523886810238979}{wrote
on Twitter} on Wednesday night. ``He was the prototype of what we mean
when we talk about New Orleans jazz.''

That was not always so. Mr. Marsalis's devotion to midcentury bebop and
its offshoots had long made him something of an outsider in a city with
an abiding loyalty to its early-jazz roots. Still, he secured the
respect of fellow musicians thanks to his unshakable talents as a
pianist and composer, and his supportive but rigorous manner as an
educator.

Once they reached the national stage, the Marsalises' advocacy of
straight-ahead jazz made them renegades of a different sort. Wynton, a
trumpeter, boldly espoused his father's devotion to heroes like Charlie
Parker and Thelonious Monk, and he
\href{https://www.nytimes.com/1988/07/31/arts/music-what-jazz-is-and-isn-t.html}{issued
public broadsides} against the slicker jazz-rock fusion that had largely
displaced acoustic jazz during the late 1960s and '70s.

\includegraphics{https://static01.nyt.com/images/2020/04/03/obituaries/02Marsallis2-02/02Marsallis2-02-articleLarge.jpg?quality=75\&auto=webp\&disable=upscale}

Photogenic, erudite and fabulously talented, Mr. Marsalis's children and
many other young jazz musicians he had taught --- including Terence
Blanchard, Donald Harrison Jr., Harry Connick Jr. and Nicholas Payton
--- became the leaders in a burgeoning traditionalist movement, loosely
referred to as the Young Lions.

``My dad was a giant of a musician and teacher, but an even greater
father,'' Branford Marsalis said in a statement. ``He poured everything
he had into making us the best of what we could be.''

\hypertarget{latest-updates-global-coronavirus-outbreak}{%
\section{\texorpdfstring{\href{https://www.nytimes.com/2020/08/03/world/coronavirus-covid-19.html?action=click\&pgtype=Article\&state=default\&region=MAIN_CONTENT_1\&context=storylines_live_updates}{Latest
Updates: Global Coronavirus
Outbreak}}{Latest Updates: Global Coronavirus Outbreak}}\label{latest-updates-global-coronavirus-outbreak}}

Updated 2020-08-04T05:55:16.339Z

\begin{itemize}
\tightlist
\item
  \href{https://www.nytimes.com/2020/08/03/world/coronavirus-covid-19.html?action=click\&pgtype=Article\&state=default\&region=MAIN_CONTENT_1\&context=storylines_live_updates\#link-4547638f}{Fauci
  defends Birx after she is criticized by Trump.}
\item
  \href{https://www.nytimes.com/2020/08/03/world/coronavirus-covid-19.html?action=click\&pgtype=Article\&state=default\&region=MAIN_CONTENT_1\&context=storylines_live_updates\#link-15e7f995}{Trump
  derides Democrats as lawmakers and administration officials try to
  break stimulus impasse.}
\item
  \href{https://www.nytimes.com/2020/08/03/world/coronavirus-covid-19.html?action=click\&pgtype=Article\&state=default\&region=MAIN_CONTENT_1\&context=storylines_live_updates\#link-e5a2cda}{The
  deadline for 2020 census counting has been moved up by a month.}
\end{itemize}

\href{https://www.nytimes.com/2020/08/03/world/coronavirus-covid-19.html?action=click\&pgtype=Article\&state=default\&region=MAIN_CONTENT_1\&context=storylines_live_updates}{See
more updates}

More live coverage:
\href{https://www.nytimes.com/live/2020/08/03/business/stock-market-today-coronavirus?action=click\&pgtype=Article\&state=default\&region=MAIN_CONTENT_1\&context=storylines_live_updates}{Markets}

In an acknowledgment of the patriarch's influence as well as his own
talents, the National Endowment for the Arts in 2011
\href{https://www.nytimes.com/2011/01/13/arts/music/13nea.html}{named
Mr. Marsalis and his musician sons} Jazz Masters. It is considered the
highest honor for an American jazz musician, and until then it had been
awarded only on an individual basis.

By that point, the Marsalises were widely understood to be jazz's royal
family. Wynton had become the founding artistic director of Jazz at
Lincoln Center in New York, the world's pre-eminent nonprofit
organization devoted to jazz, and he won
\href{https://www.pulitzer.org/winners/wynton-marsalis}{the Pulitzer
Prize for music in 1997}. Branford was a world-renowned saxophonist and
bandleader with three Grammys to his name. Mr. Marsalis's two other
musician sons, Delfeayo, a trombonist, and Jason, a drummer and
vibraphonist, were well established as bandleaders.

\hypertarget{tell-us-about-someone-youve-lost-to-the-coronavirus}{%
\subsection{Tell us about someone you've lost to the
coronavirus}\label{tell-us-about-someone-youve-lost-to-the-coronavirus}}

The Times is telling the stories of those who have died in the pandemic.
Suggest a family member or friend below.

In addition to those sons, Mr. Marsalis is survived by two nonmusician
sons, Mboya and Ellis III; a sister, Yvette; and 15 grandchildren.
Dolores Marsalis, his wife of 58 years, died in 2017.

In an
\href{https://www.nytimes.com/2004/10/03/magazine/the-music-man.html}{interview
with The New York Times Magazine in 2004}, Wynton Marsalis said that his
father had always led by example --- expecting, rather than demanding, a
high level of seriousness from his students.

``My father never put pressure on me.'' he said. ``He's too cool for
that kind of stuff.'' Asked to define his father's brand of cool, he
explained: **``**The house could fall down and everyone would be running
around, and he would still be sitting in his same chair.''

Ellis Louis Marsalis Jr. was born in New Orleans on Nov. 14, 1934. His
mother, Florence (Robertson) Marsalis, was a homemaker.
\href{https://www.nytimes.com/2004/09/24/arts/music/ellis-marsalis-sr-96-jazzmens-patriarch-dies.html}{His
father} owned the Marsalis Motel in suburban New Orleans and was
involved in the civil rights movement. The motel's guests included the
Rev. Dr. Martin Luther King Jr., Representative Adam Clayton Powell Jr.
of New York, the future Supreme Court justice Thurgood Marshall and Ray
Charles.

Mr. Marsalis started out as a saxophonist before switching to the piano
in high school. He earned his bachelor's degree in music education from
Dillard University in New Orleans in 1955 and taught at Xavier
University Preparatory School until enlisting in the Marine Corps in the
late 1950s. There he became a member of the Corps Four, a quartet of
Marines that performed jazz on television and radio to aid in
recruitment.

After leaving the Marines he taught briefly in Breaux Bridge, La., then
returned to New Orleans with Dolores and their four children to work at
his father's motel while playing shows at night.

\href{https://www.nytimes.com/news-event/coronavirus?action=click\&pgtype=Article\&state=default\&region=MAIN_CONTENT_3\&context=storylines_faq}{}

\hypertarget{the-coronavirus-outbreak-}{%
\subsubsection{The Coronavirus Outbreak
›}\label{the-coronavirus-outbreak-}}

\hypertarget{frequently-asked-questions}{%
\paragraph{Frequently Asked
Questions}\label{frequently-asked-questions}}

Updated August 3, 2020

\begin{itemize}
\item ~
  \hypertarget{im-a-small-business-owner-can-i-get-relief}{%
  \paragraph{I'm a small-business owner. Can I get
  relief?}\label{im-a-small-business-owner-can-i-get-relief}}

  \begin{itemize}
  \tightlist
  \item
    The
    \href{https://www.nytimes.com/article/small-business-loans-stimulus-grants-freelancers-coronavirus.html?action=click\&pgtype=Article\&state=default\&region=MAIN_CONTENT_3\&context=storylines_faq}{stimulus
    bills enacted in March} offer help for the millions of American
    small businesses. Those eligible for aid are businesses and
    nonprofit organizations with fewer than 500 workers, including sole
    proprietorships, independent contractors and freelancers. Some
    larger companies in some industries are also eligible. The help
    being offered, which is being managed by the Small Business
    Administration, includes the Paycheck Protection Program and the
    Economic Injury Disaster Loan program. But lots of folks have
    \href{https://www.nytimes.com/interactive/2020/05/07/business/small-business-loans-coronavirus.html?action=click\&pgtype=Article\&state=default\&region=MAIN_CONTENT_3\&context=storylines_faq}{not
    yet seen payouts.} Even those who have received help are confused:
    The rules are draconian, and some are stuck sitting on
    \href{https://www.nytimes.com/2020/05/02/business/economy/loans-coronavirus-small-business.html?action=click\&pgtype=Article\&state=default\&region=MAIN_CONTENT_3\&context=storylines_faq}{money
    they don't know how to use.} Many small-business owners are getting
    less than they expected or
    \href{https://www.nytimes.com/2020/06/10/business/Small-business-loans-ppp.html?action=click\&pgtype=Article\&state=default\&region=MAIN_CONTENT_3\&context=storylines_faq}{not
    hearing anything at all.}
  \end{itemize}
\item ~
  \hypertarget{what-are-my-rights-if-i-am-worried-about-going-back-to-work}{%
  \paragraph{What are my rights if I am worried about going back to
  work?}\label{what-are-my-rights-if-i-am-worried-about-going-back-to-work}}

  \begin{itemize}
  \tightlist
  \item
    Employers have to provide
    \href{https://www.osha.gov/SLTC/covid-19/standards.html}{a safe
    workplace} with policies that protect everyone equally.
    \href{https://www.nytimes.com/article/coronavirus-money-unemployment.html?action=click\&pgtype=Article\&state=default\&region=MAIN_CONTENT_3\&context=storylines_faq}{And
    if one of your co-workers tests positive for the coronavirus, the
    C.D.C.} has said that
    \href{https://www.cdc.gov/coronavirus/2019-ncov/community/guidance-business-response.html}{employers
    should tell their employees} -\/- without giving you the sick
    employee's name -\/- that they may have been exposed to the virus.
  \end{itemize}
\item ~
  \hypertarget{should-i-refinance-my-mortgage}{%
  \paragraph{Should I refinance my
  mortgage?}\label{should-i-refinance-my-mortgage}}

  \begin{itemize}
  \tightlist
  \item
    \href{https://www.nytimes.com/article/coronavirus-money-unemployment.html?action=click\&pgtype=Article\&state=default\&region=MAIN_CONTENT_3\&context=storylines_faq}{It
    could be a good idea,} because mortgage rates have
    \href{https://www.nytimes.com/2020/07/16/business/mortgage-rates-below-3-percent.html?action=click\&pgtype=Article\&state=default\&region=MAIN_CONTENT_3\&context=storylines_faq}{never
    been lower.} Refinancing requests have pushed mortgage applications
    to some of the highest levels since 2008, so be prepared to get in
    line. But defaults are also up, so if you're thinking about buying a
    home, be aware that some lenders have tightened their standards.
  \end{itemize}
\item ~
  \hypertarget{what-is-school-going-to-look-like-in-september}{%
  \paragraph{What is school going to look like in
  September?}\label{what-is-school-going-to-look-like-in-september}}

  \begin{itemize}
  \tightlist
  \item
    It is unlikely that many schools will return to a normal schedule
    this fall, requiring the grind of
    \href{https://www.nytimes.com/2020/06/05/us/coronavirus-education-lost-learning.html?action=click\&pgtype=Article\&state=default\&region=MAIN_CONTENT_3\&context=storylines_faq}{online
    learning},
    \href{https://www.nytimes.com/2020/05/29/us/coronavirus-child-care-centers.html?action=click\&pgtype=Article\&state=default\&region=MAIN_CONTENT_3\&context=storylines_faq}{makeshift
    child care} and
    \href{https://www.nytimes.com/2020/06/03/business/economy/coronavirus-working-women.html?action=click\&pgtype=Article\&state=default\&region=MAIN_CONTENT_3\&context=storylines_faq}{stunted
    workdays} to continue. California's two largest public school
    districts --- Los Angeles and San Diego --- said on July 13, that
    \href{https://www.nytimes.com/2020/07/13/us/lausd-san-diego-school-reopening.html?action=click\&pgtype=Article\&state=default\&region=MAIN_CONTENT_3\&context=storylines_faq}{instruction
    will be remote-only in the fall}, citing concerns that surging
    coronavirus infections in their areas pose too dire a risk for
    students and teachers. Together, the two districts enroll some
    825,000 students. They are the largest in the country so far to
    abandon plans for even a partial physical return to classrooms when
    they reopen in August. For other districts, the solution won't be an
    all-or-nothing approach.
    \href{https://bioethics.jhu.edu/research-and-outreach/projects/eschool-initiative/school-policy-tracker/}{Many
    systems}, including the nation's largest, New York City, are
    devising
    \href{https://www.nytimes.com/2020/06/26/us/coronavirus-schools-reopen-fall.html?action=click\&pgtype=Article\&state=default\&region=MAIN_CONTENT_3\&context=storylines_faq}{hybrid
    plans} that involve spending some days in classrooms and other days
    online. There's no national policy on this yet, so check with your
    municipal school system regularly to see what is happening in your
    community.
  \end{itemize}
\item ~
  \hypertarget{is-the-coronavirus-airborne}{%
  \paragraph{Is the coronavirus
  airborne?}\label{is-the-coronavirus-airborne}}

  \begin{itemize}
  \tightlist
  \item
    The coronavirus
    \href{https://www.nytimes.com/2020/07/04/health/239-experts-with-one-big-claim-the-coronavirus-is-airborne.html?action=click\&pgtype=Article\&state=default\&region=MAIN_CONTENT_3\&context=storylines_faq}{can
    stay aloft for hours in tiny droplets in stagnant air}, infecting
    people as they inhale, mounting scientific evidence suggests. This
    risk is highest in crowded indoor spaces with poor ventilation, and
    may help explain super-spreading events reported in meatpacking
    plants, churches and restaurants.
    \href{https://www.nytimes.com/2020/07/06/health/coronavirus-airborne-aerosols.html?action=click\&pgtype=Article\&state=default\&region=MAIN_CONTENT_3\&context=storylines_faq}{It's
    unclear how often the virus is spread} via these tiny droplets, or
    aerosols, compared with larger droplets that are expelled when a
    sick person coughs or sneezes, or transmitted through contact with
    contaminated surfaces, said Linsey Marr, an aerosol expert at
    Virginia Tech. Aerosols are released even when a person without
    symptoms exhales, talks or sings, according to Dr. Marr and more
    than 200 other experts, who
    \href{https://academic.oup.com/cid/article/doi/10.1093/cid/ciaa939/5867798}{have
    outlined the evidence in an open letter to the World Health
    Organization}.
  \end{itemize}
\end{itemize}

Mr. Marsalis performed and recorded throughout the 1960s and '70s with a
variety of modern and progressive jazz musicians, including the drummer
Ed Blackwell and the eminent horn-playing brothers Cannonball and Nat
Adderley.

He later earned a master's degree in music education from Loyola
University in New Orleans and led the jazz studies program at the New
Orleans Center for Creative Arts for high school students. It was there
that he mentored such future stars as Mr. Blanchard and Mr. Connick as
well as his own children.

He later taught at Virginia Commonwealth University and the University
of New Orleans, where he served for 12 years as the founding director of
its jazz studies department.

Image

Mr. Marsalis in performance at the New Orleans Jazz and Heritage
Festival in 2009. The mayor of New Orleans called him ``the prototype of
what we mean when we talk about New Orleans jazz.''Credit...Associated
Press

\href{https://www.nytimes.com/1979/07/07/archives/jazz-piano-ellis-marsalis-plays-at-carnegie-tavern.html?searchResultPosition=3}{Reviewing
a 1979 performance} by Mr. Marsalis at the Carnegie Tavern in New York
just before his family burst onto the national stage, John S. Wilson of
The New York Times introduced him to his readers. ``Unlike the widely
accepted image of jazz musicians from New Orleans, Mr. Marsalis is not a
traditionalist,'' Mr. Wilson wrote, describing him as ``an eclectic
performer with a light and graceful touch'' and an ``exploratory turn of
mind.''

Four years later, Mr. Marsalis made another New York appearance, at a
next-door locale with a similar name: Carnegie Hall. There he gave a
solo concert, oscillating between original compositions and jazz
standards.

``Mr. Marsalis's interpretations were impressive in their economy and
steadiness,''
\href{https://www.nytimes.com/1983/06/29/arts/ellis-marsalis-s-solo-piano.html}{the
Times critic Stephen Holden wrote}. ``Sticking mainly to the middle
register of the keyboard, the pianist offered richly harmonized
arrangements in which fancy keyboard work was kept to a minimum and
studious melodic invention, rather than pronounced bass patterns,
determined the structures and tempos.''

Before Wynton and then Branford found acclaim, Mr. Marsalis had recorded
only sporadically. But once they all became nationally known, that
changed. In the 1990s, after the Young Lions boom he had helped unleash
led major labels to reinvest in straight-ahead jazz, Mr. Marsalis
released a series of albums for Blue Note and then Columbia.

In 2008, Mr. Marsalis was inducted into the Louisiana Music Hall of
Fame.

He had held a weekly gig for decades at Snug Harbor, one of New
Orleans's premier jazz clubs, before giving it up in December.

Always hungry for knowledge, Mr. Marsalis saw himself as a perpetual
student. In an
\href{http://www.offbeat.com/articles/ellis-marsalis-interview/}{interview
with Offbeat magazine} in 1989, just after joining the faculty at the
University of New Orleans, he said: ``I'd like to get involved in a
course on physics to get a good understanding of the physical aspects of
the universe. There are literature courses I'd like to take. I might one
day. I don't buy the idea that colleges are just for young people.''

Julia Carmel contributed reporting.

Advertisement

\protect\hyperlink{after-bottom}{Continue reading the main story}

\hypertarget{site-index}{%
\subsection{Site Index}\label{site-index}}

\hypertarget{site-information-navigation}{%
\subsection{Site Information
Navigation}\label{site-information-navigation}}

\begin{itemize}
\tightlist
\item
  \href{https://help.nytimes.com/hc/en-us/articles/115014792127-Copyright-notice}{©~2020~The
  New York Times Company}
\end{itemize}

\begin{itemize}
\tightlist
\item
  \href{https://www.nytco.com/}{NYTCo}
\item
  \href{https://help.nytimes.com/hc/en-us/articles/115015385887-Contact-Us}{Contact
  Us}
\item
  \href{https://www.nytco.com/careers/}{Work with us}
\item
  \href{https://nytmediakit.com/}{Advertise}
\item
  \href{http://www.tbrandstudio.com/}{T Brand Studio}
\item
  \href{https://www.nytimes.com/privacy/cookie-policy\#how-do-i-manage-trackers}{Your
  Ad Choices}
\item
  \href{https://www.nytimes.com/privacy}{Privacy}
\item
  \href{https://help.nytimes.com/hc/en-us/articles/115014893428-Terms-of-service}{Terms
  of Service}
\item
  \href{https://help.nytimes.com/hc/en-us/articles/115014893968-Terms-of-sale}{Terms
  of Sale}
\item
  \href{https://spiderbites.nytimes.com}{Site Map}
\item
  \href{https://help.nytimes.com/hc/en-us}{Help}
\item
  \href{https://www.nytimes.com/subscription?campaignId=37WXW}{Subscriptions}
\end{itemize}
