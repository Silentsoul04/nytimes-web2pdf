Sections

SEARCH

\protect\hyperlink{site-content}{Skip to
content}\protect\hyperlink{site-index}{Skip to site index}

\href{https://www.nytimes.com/section/books}{Books}

\href{https://myaccount.nytimes.com/auth/login?response_type=cookie\&client_id=vi}{}

\href{https://www.nytimes.com/section/todayspaper}{Today's Paper}

\href{/section/books}{Books}\textbar{}Patricia Bosworth,
Actress-Turned-Author, Dies at 86

\url{https://nyti.ms/2JEGPZX}

\begin{itemize}
\item
\item
\item
\item
\item
\end{itemize}

\href{https://www.nytimes.com/news-event/coronavirus?action=click\&pgtype=Article\&state=default\&region=TOP_BANNER\&context=storylines_menu}{The
Coronavirus Outbreak}

\begin{itemize}
\tightlist
\item
  live\href{https://www.nytimes.com/2020/08/03/world/coronavirus-covid-19.html?action=click\&pgtype=Article\&state=default\&region=TOP_BANNER\&context=storylines_menu}{Latest
  Updates}
\item
  \href{https://www.nytimes.com/interactive/2020/us/coronavirus-us-cases.html?action=click\&pgtype=Article\&state=default\&region=TOP_BANNER\&context=storylines_menu}{Maps
  and Cases}
\item
  \href{https://www.nytimes.com/interactive/2020/science/coronavirus-vaccine-tracker.html?action=click\&pgtype=Article\&state=default\&region=TOP_BANNER\&context=storylines_menu}{Vaccine
  Tracker}
\item
  \href{https://www.nytimes.com/2020/08/02/us/covid-college-reopening.html?action=click\&pgtype=Article\&state=default\&region=TOP_BANNER\&context=storylines_menu}{College
  Reopening}
\item
  \href{https://www.nytimes.com/live/2020/08/03/business/stock-market-today-coronavirus?action=click\&pgtype=Article\&state=default\&region=TOP_BANNER\&context=storylines_menu}{Economy}
\end{itemize}

Advertisement

\protect\hyperlink{after-top}{Continue reading the main story}

Supported by

\protect\hyperlink{after-sponsor}{Continue reading the main story}

those we've lost

\hypertarget{patricia-bosworth-actress-turned-author-dies-at-86}{%
\section{Patricia Bosworth, Actress-Turned-Author, Dies at
86}\label{patricia-bosworth-actress-turned-author-dies-at-86}}

She gave up the stage for the writing life, publishing biographies of
some famous friends and two powerful memoirs. She died of the
coronavirus.

\includegraphics{https://static01.nyt.com/images/2020/04/04/obituaries/03bosworth/03bosworth-articleLarge.jpg?quality=75\&auto=webp\&disable=upscale}

By Elsa Dixler

\begin{itemize}
\item
  Published April 3, 2020Updated April 16, 2020
\item
  \begin{itemize}
  \item
  \item
  \item
  \item
  \item
  \end{itemize}
\end{itemize}

\emph{This obituary is part of a series about}
\href{https://www.nytimes.com/series/people-who-have-died-of-the-coronavirus}{\emph{people
who have died in the coronavirus pandemic}}\emph{.}

Patricia Bosworth, who gave up acting for the writing life, turning her
knowledge of the theater into a series of biographies and mining her own
extraordinary life for a pair of powerful memoirs, died on Thursday in
Manhattan. She was 86.

Her stepdaughter, Fia Hatsav, said the cause was complications of
pneumonia brought on by the coronavirus.

Ms. Bosworth had some success as an actress. She was admitted to the
Actors Studio in its glory days, learning method acting alongside Marlon
Brando and Marilyn Monroe. She won some important roles onstage and
appeared alongside Audrey Hepburn on film.

But she always wanted to write, and she found material in the many
friendships she had cultivated with luminaries in Hollywood, the theater
world and elsewhere --- Brando, Montgomery Clift and the photographer
Diane Arbus among them.

She became a successful journalist as well, as an editor and writer for
several publications. She was a contributing editor at Vanity Fair for
many years.

Ms. Bosworth's best subject, and the one that underlay most of her work,
was her own eventful life. She explored it in ``Anything Your Little
Heart Desires: An American Family Story'' (1997), which centers on her
charismatic father, a lawyer who defended two of the Hollywood Ten in
the postwar anti-Communist hysteria and saw his career destroyed by the
blacklist; and
\href{https://www.nytimes.com/2017/01/24/books/review-patricia-bosworth-men-in-my-life.html}{``The
Men in My Life: A Memoir of Love and Art in 1950s Manhattan''} (2017),
about her coming-of-age and emergence as a writer.

Suicide haunted her. Her father, who had long abused barbiturates and
alcohol, killed himself, on his second try, in 1959. And her beloved
younger brother shot himself in his dorm room at Reed College in Oregon
in 1953, tormented by depression and conflicted over his homosexuality.

The subjects of Ms. Bosworth's biographies were either suicides (Arbus),
survivors of a relative's suicide (Jane Fonda) or flamboyantly
self-destructive (Clift, Brando). She explained that writing these books
was ``one of the ways I coped with and tried to understand why the two
men I loved most in the world had decided to kill themselves.''

But as challenging as it may have been, Ms. Bosworth's life was hardly
grim.

Patricia Crum was born into privilege on April 24, 1933, in San
Francisco, the daughter of Bartley Cavanaugh Crum and Anna Bosworth
Crum, who was known as Cutsie. Her mother was a former crime reporter
who wrote several novels, among them ``Strumpet Wind'' (1938).

Her father, who was known as Bart, encouraged Patricia's acting
aspirations, and it was he who advised her to take her mother's maiden
name --- depriving future critics of the chance, as she put it, to
castigate a ``crummy performance by Patricia Crum.''

During Ms. Bosworth's childhood, her father practiced law in San
Francisco and served as an adviser to the liberal-leaning
internationalist Wendell Willkie in his Republican presidential campaign
in 1940 and for some years after.

Image

Ms. Bosworth's second memoir, published in 2017, is both the story of a
survivor who struggles with the suicides of her father and brother and
an entertaining account of the author's sexual awakening and her life
among actors in New York.Credit...Patricia Wall/The New York Times

In her first memoir, Ms. Bosworth remembered her parents as glamorous
figures, always leaving for parties or throwing them, their living room
crowded with celebrities. But there were shadows behind the California
sunlight.

Her mother, feeling abandoned by her constantly traveling husband, had
affairs; her father's heartfelt liberalism would run athwart of the
postwar Red Scare. More than one reviewer of ``Anything Your Little
Heart Desires'' compared the Crums' story to an F. Scott Fitzgerald
novel.

Mr. Crum's decline followed his defense of members of the Hollywood Ten,
who had refused to cooperate with the House Un-American Activities
Committee in its attempt to root out suspected Communists in the movie
industry. His corporate clients disappeared. He moved the family to New
York, where he purchased the left-wing newspaper PM and tried to turn it
around as The New York Star. The attempt failed, and he became
despondent, worried about money and harassed by the F.B.I.

Mr. Crum eventually joined a Wall Street law firm and attracted
celebrity clients. He represented Rita Hayworth, for one, in her divorce
from the playboy Prince Aly Kahn. Ms. Bosworth, then a star-struck
teenager, met another client, Montgomery Clift, lounging in the family
living room. She kept one of his cigarette butts for the rest of her
life.

Enrolling at Sarah Lawrence College in Bronxville, N.Y., Ms. Bosworth,
in her first semester, impetuously married a fortune-hunting art student
she had known for six weeks. He quickly became psychologically and
physically abusive, she later wrote.

Ms. Bosworth's parents paid her tuition, but she was left to support her
husband and his grandmother. While continuing her classes, she began to
model, landing a national campaign for Prell shampoo. It was as a model
that she met and formed a bond with Diane Arbus, who at the time was
assisting her husband, the fashion photographer Allan Arbus.

\includegraphics{https://static01.nyt.com/images/2020/04/04/obituaries/03Bosworth2/merlin_116600933_e9f931c3-cfcb-4c7b-ae1d-59108aa588fd-articleLarge.jpg?quality=75\&auto=webp\&disable=upscale}

Her brief marriage over, Ms. Bosworth graduated from Sarah Lawrence in
1955 and auditioned for the acting teacher Lee Strasberg at the Actors
Studio; her acceptance, she later said, was ``one of the high points of
my life.''

The studio, the birthplace of method acting, was in its heyday. In
addition to Monroe and Brando, fellow members included
\href{https://www.nytimes.com/2008/09/28/movies/28newman.html}{Paul
Newman},
\href{https://www.nytimes.com/2014/07/18/theater/elaine-stritch-tart-tongued-broadway-actress-and-singer-is-dead-at-89.html}{Elaine
Stritch},
\href{https://www.nytimes.com/2012/02/04/movies/ben-gazzara-actor-of-stage-and-screen-dies-at-81.html}{Ben
Gazzara} and
\href{https://www.nytimes.com/1980/11/08/archives/steve-mcqueen-50-is-dead-of-a-heart-attack-after-surgery-for-cancer.html}{Steve
McQueen}, who took her for a ride on his motorcycle.

The studio crackled with erotic energy, and in the mid-'50s the casting
couch was an accepted furnishing. Ms. Bosworth was castin ``Blue
Denim,'' directed by Arthur Penn, at the Westport Playhouse in
Connecticut in 1955. Among other roles, she went on to play Laura in
``The Glass Menagerie'' at the Palm Beach Playhouse in Florida in 1956
(``the high point of my acting career'') and a character based on Nora
Ephron in ``Howie,'' a 1959 play by Nora's mother, Phoebe Ephron.

As a film actress Ms. Bosworth played a young nun, the best friend of
Audrey Hepburn's title character, in ``The Nun's Story'' (1959).

Ms. Bosworth ended her adventurous decade by marrying Mel Arrighi, a
playwright and novelist, in 1966. They were together
\href{https://www.nytimes.com/1986/09/17/obituaries/mel-arrighi-52-playwright.html}{until
his death in 1986}. She began writing during those years, contributing
articles about theater to The New York Times and New York magazine. In
time she worked as an editor at McCall's, Harper's Bazaar, Mirabella and
Vanity Fair magazines.

Ms. Bosworth's first book was ``Montgomery Clift: A Biography,''
published in 1978. Rosalyn Drexler, writing in The Times, praised Ms.
Bosworth's ``total immersion in the subject as well as her artistry.''

When ``Diane Arbus: A Biography'' appeared in 1984, the Times critic
\href{https://www.nytimes.com/2018/11/07/obituaries/christopher-lehmann-haupt-dead.html}{Christopher
Lehmann-Haupt} acclaimed it as ``detailed and balanced'' as well as
``highly intelligent.'' (The book was the basis of
\href{https://www.nytimes.com/2006/11/10/movies/10fur.html}{a 2006
film}, ``Fur: An Imaginary Portrait of Diane Arbus,'' starring Nicole
Kidman.)

Ms. Bosworth moved on to ``Marlon Brando,'' a book in the Penguin Lives
series, in 2001, and later to
\href{https://www.nytimes.com/2011/09/25/opinion/sunday/the-dark-tie-between-jane-fonda-and-her-biographer.html}{``Jane
Fonda: The Private Life of a Public Woman''} (2011), a best seller. Ms.
Bosworth had first met Ms. Fonda when they were both students at the
Actors Studio.

Ms. Bosworth told Publishers Weekly that in ``Jane Fonda'' she had tried
to write as much a cultural history as a biography, to give context to
Ms. Fonda's ever-evolving career.

``In the 10 years I took to write her biography, I observed many
Janes,'' she wrote in an essay for The Times in 2011. ``I saw the Jane
with the agenda; the girlish, self-effacing Jane when she's with men;
the armchair shrink Jane who spouts advice about sex and love and
exercise as if by rote whenever she's on TV; the ruthless, hard-as-nails
Jane in business and self-promotion; the generous Jane with friends in
need; the loving grandmother-matriarch Jane; the celebrity Jane who in
May walked down the red carpet at Cannes in a glittery white gown and
left all the young starlets in her dust.''

Image

Ms. Bosworth in 2006 at the New York premiere of the film ``Fur: An
Imaginary Portrait of Diane Arbus,'' based on Ms. Bosworth's Arbus
biography.Credit...Scott Wintrow/Getty Images

Creating a biography, Ms. Bosworth wrote on
\href{http://www.pbosworth.com/}{her website}, was ``like solving a
mystery, always looking for clues.'' Elegantly, and without self-pity or
sentimentality, she eventually turned her attention to the mystery of
her own life.

Her first memoir, ``Anything Your Little Heart Desires,'' was also a
cultural history, backed by prodigious research, as it followed her
father's life from his high-powered law career in San Francisco to his
radicalization by the general strike of 1934 and his involvement in
left-wing causes.

Ms. Bosworth's second memoir, ``The Men in My Life,'' covered her 20s,
from 1953 to 1963. It is both the story of a survivor who struggles with
the suicides of her father and brother and an entertaining account of
her sexual awakening and life among actors in New York.

In addition to her stepdaughter, Ms. Bosworth is survived by her
partner, Douglas Schwalbe; a stepson, Léo Palumbo; and five
step-grandchildren.

She taught literary nonfiction at Columbia University and Barnard
College and for some years ran the Playwright-Directors Unit at the
Actors Studio.

Her final book, ``Protest Song: Paul Robeson, J. Edgar Hoover, and the
Ongoing Fight for Racial Equality,'' is to be published by Farrar,
Straus \& Giroux next year.

Julia Carmel contributed reporting.

\href{https://www.nytimes.com/interactive/2020/obituaries/people-died-coronavirus-obituaries.html?action=click\&pgtype=Article\&state=default\&region=BELOW_MAIN_CONTENT\&context=covid_obits_promo}{}

\hypertarget{those-weve-lost}{%
\section{Those We've Lost}\label{those-weve-lost}}

The coronavirus pandemic has taken an incalculable death toll. This
series is designed to put names and faces to the numbers.

Read more

\includegraphics{https://static01.nyt.com/images/2020/07/30/obituaries/30Pedro/30Pedro-square640.jpg}

\hypertarget{bernaldina-josuxe9-pedro}{%
\section{Bernaldina José Pedro}\label{bernaldina-josuxe9-pedro}}

d. Boa Vista, Brazil

Leader among the Indigenous Macuxi

\includegraphics{https://static01.nyt.com/images/2020/07/31/obituaries/31Swing/merlin_175167783_8913bc90-0d64-43f3-a655-1bb1bf1601c9-square640.jpg}

\hypertarget{john-eric-swing}{%
\section{John Eric Swing}\label{john-eric-swing}}

d. Fountain Valley, Calif.

Champion of Filipino-Americans

\includegraphics{https://static01.nyt.com/images/2020/07/27/obituaries/27Victor/merlin_175001436_38b11f8e-227a-4e2c-9821-7618af9b2524-square640.jpg}

\hypertarget{victor-victor}{%
\section{Victor Victor}\label{victor-victor}}

d. Santo Domingo, Dominican Republic

Beloved musician of the Dominican Republic

\includegraphics{https://static01.nyt.com/images/2020/07/31/obituaries/31Negron/merlin_175160169_516322ae-fd23-4969-b6b2-193ced371105-square640.jpg}

\hypertarget{dr-eddie-negruxf3n}{%
\section{Dr. Eddie Negrón}\label{dr-eddie-negruxf3n}}

d. Fort Walton Beach, Fla.

Internist on Florida's Emerald Coast

\includegraphics{https://static01.nyt.com/images/2020/07/30/obituaries/30Dobson/merlin_175115928_f6b9271c-8f05-4fe1-a38a-5ca4a58f8935-square640.jpg}

\hypertarget{dobby-dobson}{%
\section{Dobby Dobson}\label{dobby-dobson}}

d. Coral Springs, Fla.

Jamaican singer and songwriter

\includegraphics{https://static01.nyt.com/images/2020/08/01/obituaries/28Gonzalez/merlin_175002771_beb57888-3951-409a-ae13-03a94b2e962e-square640.jpg}

\hypertarget{waldemar-gonzalez}{%
\section{Waldemar Gonzalez}\label{waldemar-gonzalez}}

d. White Plains, N.Y.

Teacher and social worker

Advertisement

\protect\hyperlink{after-bottom}{Continue reading the main story}

\hypertarget{site-index}{%
\subsection{Site Index}\label{site-index}}

\hypertarget{site-information-navigation}{%
\subsection{Site Information
Navigation}\label{site-information-navigation}}

\begin{itemize}
\tightlist
\item
  \href{https://help.nytimes.com/hc/en-us/articles/115014792127-Copyright-notice}{©~2020~The
  New York Times Company}
\end{itemize}

\begin{itemize}
\tightlist
\item
  \href{https://www.nytco.com/}{NYTCo}
\item
  \href{https://help.nytimes.com/hc/en-us/articles/115015385887-Contact-Us}{Contact
  Us}
\item
  \href{https://www.nytco.com/careers/}{Work with us}
\item
  \href{https://nytmediakit.com/}{Advertise}
\item
  \href{http://www.tbrandstudio.com/}{T Brand Studio}
\item
  \href{https://www.nytimes.com/privacy/cookie-policy\#how-do-i-manage-trackers}{Your
  Ad Choices}
\item
  \href{https://www.nytimes.com/privacy}{Privacy}
\item
  \href{https://help.nytimes.com/hc/en-us/articles/115014893428-Terms-of-service}{Terms
  of Service}
\item
  \href{https://help.nytimes.com/hc/en-us/articles/115014893968-Terms-of-sale}{Terms
  of Sale}
\item
  \href{https://spiderbites.nytimes.com}{Site Map}
\item
  \href{https://help.nytimes.com/hc/en-us}{Help}
\item
  \href{https://www.nytimes.com/subscription?campaignId=37WXW}{Subscriptions}
\end{itemize}
