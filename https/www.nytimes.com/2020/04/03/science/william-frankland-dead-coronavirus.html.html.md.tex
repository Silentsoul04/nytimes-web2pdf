Sections

SEARCH

\protect\hyperlink{site-content}{Skip to
content}\protect\hyperlink{site-index}{Skip to site index}

\href{https://www.nytimes.com/section/science}{Science}

\href{https://myaccount.nytimes.com/auth/login?response_type=cookie\&client_id=vi}{}

\href{https://www.nytimes.com/section/todayspaper}{Today's Paper}

\href{/section/science}{Science}\textbar{}William Frankland, Pioneering
Allergist, Dies at 108

\url{https://nyti.ms/3e12DNp}

\begin{itemize}
\item
\item
\item
\item
\item
\end{itemize}

\href{https://www.nytimes.com/news-event/coronavirus?action=click\&pgtype=Article\&state=default\&region=TOP_BANNER\&context=storylines_menu}{The
Coronavirus Outbreak}

\begin{itemize}
\tightlist
\item
  live\href{https://www.nytimes.com/2020/08/03/world/coronavirus-covid-19.html?action=click\&pgtype=Article\&state=default\&region=TOP_BANNER\&context=storylines_menu}{Latest
  Updates}
\item
  \href{https://www.nytimes.com/interactive/2020/us/coronavirus-us-cases.html?action=click\&pgtype=Article\&state=default\&region=TOP_BANNER\&context=storylines_menu}{Maps
  and Cases}
\item
  \href{https://www.nytimes.com/interactive/2020/science/coronavirus-vaccine-tracker.html?action=click\&pgtype=Article\&state=default\&region=TOP_BANNER\&context=storylines_menu}{Vaccine
  Tracker}
\item
  \href{https://www.nytimes.com/2020/08/02/us/covid-college-reopening.html?action=click\&pgtype=Article\&state=default\&region=TOP_BANNER\&context=storylines_menu}{College
  Reopening}
\item
  \href{https://www.nytimes.com/live/2020/08/03/business/stock-market-today-coronavirus?action=click\&pgtype=Article\&state=default\&region=TOP_BANNER\&context=storylines_menu}{Economy}
\end{itemize}

Advertisement

\protect\hyperlink{after-top}{Continue reading the main story}

Supported by

\protect\hyperlink{after-sponsor}{Continue reading the main story}

Those We've Lost

\hypertarget{william-frankland-pioneering-allergist-dies-at-108}{%
\section{William Frankland, Pioneering Allergist, Dies at
108}\label{william-frankland-pioneering-allergist-dies-at-108}}

One of the top allergists of the 20th century, Dr. Frankland, who died
of the coronavirus, helped legions of hay fever sneezers by publicizing
pollen counts. Among his patients was Saddam Hussein.

\includegraphics{https://static01.nyt.com/images/2020/04/05/obituaries/05Frankland-obit/03Frankland1-articleLarge.jpg?quality=75\&auto=webp\&disable=upscale}

By \href{https://www.nytimes.com/by/craig-s-smith}{Craig S. Smith}

\begin{itemize}
\item
  Published April 3, 2020Updated April 16, 2020
\item
  \begin{itemize}
  \item
  \item
  \item
  \item
  \item
  \end{itemize}
\end{itemize}

\emph{This obituary is part of a series about}
\href{https://www.nytimes.com/series/people-who-have-died-of-the-coronavirus}{\emph{people
who have died in the coronavirus pandemic}}\emph{.}

Dr. William Frankland, one of the top allergists of the 20th century and
an indomitable researcher who helped legions of hay fever sneezers by
distributing daily pollen counts to the British public, died on Thursday
in London at 108.

His son, Andrew, said the cause was the coronavirus. He lived in a care
home at \href{http://www.thecharterhouse.org/news/}{the historic
Charterhouse complex}, a former monastery in London.

Dr. Frankland, who was among the world's oldest active scientists,
remained remarkably vigorous to the end, despite having come close to
death several times in his long life.

He was born prematurely, weighing just over three pounds, and he
contracted bovine tuberculosis as a child. Later, while serving in the
British Army, he spent years as a malnourished prisoner of war in
Japanese camps. He had another brush with death when he used himself as
an experiment on a biting insect and had an anaphylaxis reaction.

Dr. Frankland was best known in professional circles for a number of
groundbreaking clinical studies. In 1954, he proved that pollen proteins
were the parts of plants most useful in preseason allergy inoculations,
and in 1955 he debunked the efficacy of treating asthma with bacterial
vaccines.

He was an early proponent of using allergen injections to desensitize
patients with severe allergies, and developed immunotherapy serums for
hay fever sufferers with pollen from one of the world's largest pollen
farms, which he operated outside London until the late 1960s.

It was while investigating desensitization to insect bites that Dr.
Frankland allowed the South American insect Rhodnius prolixus to bite
his arm at weekly intervals. The eighth bite sent him into
life-threatening anaphylaxis, from which a nurse revived him with
repeated shots of adrenaline.

Among the tens of thousands of patients that Dr. Frankland treated was
the Iraqi leader Saddam Hussein, who summoned the doctor to Baghdad in
1979. Dr. Frankland found that Mr. Hussein had no allergies but was
suffering from the effects of excessive smoking, consuming as many as 40
cigarettes a day.

``I advised him to stop smoking,'' Dr. Frankland
\href{https://search.proquest.com/openview/6aaa55a46c44a9fb31622a4a71d61f10/1?pq-origsite=gscholar\&cbl=2043523}{told
the medical journal The BMJ}. ``Three and a half months later he was
dramatically better, and because he was so grateful, I was invited back
to Baghdad with my family to have lunch with him.''

Dr. Frankland's research included rare cases. One involved a patient who
\href{https://books.google.com/books?id=dffSAwAAQBAJ\&pg=PA351\&lpg=PA351\&dq=\%22Those+controls+were+not+done+for+your+benefit,+only+mine\%22\&source=bl\&ots=lzpSVI_1uw\&sig=Uhmyajjgts8Xx0rjjVcmMm2jKSc\&hl=en\&sa=X\&ved=0ahUKEwiC58mH4evVAhUs54MKHddLBigQ6AEIJjAA\#v=onepage\&q=\%22Those\%20controls\%20were\%20not\%20done\%20for\%20your\%20benefit\%2C\%20only\%20mine\%22\&f=false}{suspected
that she was allergic} to her partner's semen. She reported, however,
that she had no allergic reaction from sexual encounters with other men,
in effect providing Dr. Frankland with data from a control group, as is
often done in scientific experiments. But she advised him, ``Those
controls were not done for your benefit, only mine.''

Alfred William Frankland was born in Sussex, England, on March 19, 1912,
one of twin boys of Henry and Alice Rose (West) Frankland. His mother
was a musician. His father, a vicar in the Church of England, moved the
family to Britain's Lake District, where the boys grew up surrounded by
farms. It was there that William discovered that he suffered from hay
fever.

He attended St. Bees School in West Cumberland before studying medicine
at Queen's College, Oxford, and St. Mary's Hospital Medical School, now
part of Imperial College London. After finishing his studies, he
enlisted in the Royal Army Medical Corps three days before the outbreak
of World War II, anticipating that doctors would be needed. He was later
sent to Singapore, where he arrived just days before the Japanese attack
on Pearl Harbor.

By chance Dr. Frankland was sent to work in Tanglin Military Hospital in
Singapore rather than the newly opened Alexandra Military Hospital there
--- thus eluding almost certain death. The Alexandra hospital was soon
overrun by Japanese troops, who massacred the doctors, nurses and
patients there. It was one of several times that luck kept him alive.

Dr. Frankland was taken prisoner on Feb. 15, 1942, and spent the
remainder of the war in Japanese prison camps, underfed and overworked,
treating the other men.

On his return to Britain, he took a post at St. Mary's, where he worked
with
\href{https://www.nytimes.com/1955/03/12/archives/sir-alexander-fleming.html}{Alexander
Fleming}, who won the 1945 Nobel Prize in Physiology or Medicine for the
discovery of penicillin.

The mold that had contaminated Dr. Fleming's Petri dishes decades
earlier and led to the development of modern antibiotics had come, in
fact, from the allergy department, which was directly below Dr.
Fleming's laboratory. Dr. Frankland correctly predicted that some
patients would be allergic to the new wonder drug.

Dr. Frankland had a pollen trap installed on the roof of St. Mary's and
began distributing daily pollen counts to the British news media in the
early 1960s. He was one of the first allergists to do so. Pollen counts
are now a staple of weather reports around the world.

Dr. Frankland published more than a hundred articles and academic papers
on allergies, including four that he wrote after turning 100. Among his
many honors, he was named a member of the Order of the British Empire in
2015.

In addition to his son, he is survived by his three daughters, Penelope
Culverhouse, Jenifer Woodhouse and Hilary Crew; 10 grandchildren and six
great-grandchildren.

Before entering Charterhouse, Dr. Frankland had lived alone in a Central
London apartment that he had shared with his wife, Pauline (Jackson)
Frankland, until her death in 2002. He cooked his own meals and, though
he used a walking stick, followed a routine of daily exercises into his
100s.

Given his brushes with death, he was frequently asked what the secret of
his longevity was. He would reply simply, ``Luck.''

\href{https://www.nytimes.com/interactive/2020/obituaries/people-died-coronavirus-obituaries.html?action=click\&pgtype=Article\&state=default\&region=BELOW_MAIN_CONTENT\&context=covid_obits_promo}{}

\hypertarget{those-weve-lost}{%
\section{Those We've Lost}\label{those-weve-lost}}

The coronavirus pandemic has taken an incalculable death toll. This
series is designed to put names and faces to the numbers.

Read more

\includegraphics{https://static01.nyt.com/images/2020/07/30/obituaries/30Pedro/30Pedro-square640.jpg}

\hypertarget{bernaldina-josuxe9-pedro}{%
\section{Bernaldina José Pedro}\label{bernaldina-josuxe9-pedro}}

d. Boa Vista, Brazil

Leader among the Indigenous Macuxi

\includegraphics{https://static01.nyt.com/images/2020/07/31/obituaries/31Swing/merlin_175167783_8913bc90-0d64-43f3-a655-1bb1bf1601c9-square640.jpg}

\hypertarget{john-eric-swing}{%
\section{John Eric Swing}\label{john-eric-swing}}

d. Fountain Valley, Calif.

Champion of Filipino-Americans

\includegraphics{https://static01.nyt.com/images/2020/07/27/obituaries/27Victor/merlin_175001436_38b11f8e-227a-4e2c-9821-7618af9b2524-square640.jpg}

\hypertarget{victor-victor}{%
\section{Victor Victor}\label{victor-victor}}

d. Santo Domingo, Dominican Republic

Beloved musician of the Dominican Republic

\includegraphics{https://static01.nyt.com/images/2020/07/31/obituaries/31Negron/merlin_175160169_516322ae-fd23-4969-b6b2-193ced371105-square640.jpg}

\hypertarget{dr-eddie-negruxf3n}{%
\section{Dr. Eddie Negrón}\label{dr-eddie-negruxf3n}}

d. Fort Walton Beach, Fla.

Internist on Florida's Emerald Coast

\includegraphics{https://static01.nyt.com/images/2020/07/30/obituaries/30Dobson/merlin_175115928_f6b9271c-8f05-4fe1-a38a-5ca4a58f8935-square640.jpg}

\hypertarget{dobby-dobson}{%
\section{Dobby Dobson}\label{dobby-dobson}}

d. Coral Springs, Fla.

Jamaican singer and songwriter

\includegraphics{https://static01.nyt.com/images/2020/08/01/obituaries/28Gonzalez/merlin_175002771_beb57888-3951-409a-ae13-03a94b2e962e-square640.jpg}

\hypertarget{waldemar-gonzalez}{%
\section{Waldemar Gonzalez}\label{waldemar-gonzalez}}

d. White Plains, N.Y.

Teacher and social worker

Advertisement

\protect\hyperlink{after-bottom}{Continue reading the main story}

\hypertarget{site-index}{%
\subsection{Site Index}\label{site-index}}

\hypertarget{site-information-navigation}{%
\subsection{Site Information
Navigation}\label{site-information-navigation}}

\begin{itemize}
\tightlist
\item
  \href{https://help.nytimes.com/hc/en-us/articles/115014792127-Copyright-notice}{©~2020~The
  New York Times Company}
\end{itemize}

\begin{itemize}
\tightlist
\item
  \href{https://www.nytco.com/}{NYTCo}
\item
  \href{https://help.nytimes.com/hc/en-us/articles/115015385887-Contact-Us}{Contact
  Us}
\item
  \href{https://www.nytco.com/careers/}{Work with us}
\item
  \href{https://nytmediakit.com/}{Advertise}
\item
  \href{http://www.tbrandstudio.com/}{T Brand Studio}
\item
  \href{https://www.nytimes.com/privacy/cookie-policy\#how-do-i-manage-trackers}{Your
  Ad Choices}
\item
  \href{https://www.nytimes.com/privacy}{Privacy}
\item
  \href{https://help.nytimes.com/hc/en-us/articles/115014893428-Terms-of-service}{Terms
  of Service}
\item
  \href{https://help.nytimes.com/hc/en-us/articles/115014893968-Terms-of-sale}{Terms
  of Sale}
\item
  \href{https://spiderbites.nytimes.com}{Site Map}
\item
  \href{https://help.nytimes.com/hc/en-us}{Help}
\item
  \href{https://www.nytimes.com/subscription?campaignId=37WXW}{Subscriptions}
\end{itemize}
