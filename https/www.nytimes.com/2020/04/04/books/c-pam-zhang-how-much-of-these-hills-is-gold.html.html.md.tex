Sections

SEARCH

\protect\hyperlink{site-content}{Skip to
content}\protect\hyperlink{site-index}{Skip to site index}

\href{https://www.nytimes.com/section/books}{Books}

\href{https://myaccount.nytimes.com/auth/login?response_type=cookie\&client_id=vi}{}

\href{https://www.nytimes.com/section/todayspaper}{Today's Paper}

\href{/section/books}{Books}\textbar{}How a Chinese-American Novelist
Wrote Herself Into the Wild West

\url{https://nyti.ms/2UKjxs1}

\begin{itemize}
\item
\item
\item
\item
\item
\end{itemize}

Advertisement

\protect\hyperlink{after-top}{Continue reading the main story}

Supported by

\protect\hyperlink{after-sponsor}{Continue reading the main story}

\hypertarget{how-a-chinese-american-novelist-wrote-herself-into-the-wild-west}{%
\section{How a Chinese-American Novelist Wrote Herself Into the Wild
West}\label{how-a-chinese-american-novelist-wrote-herself-into-the-wild-west}}

C Pam Zhang's debut, ``How Much of These Hills Is Gold,'' is one of
several new or forthcoming books by Asian-American writers set in a
period that historically hasn't recognized them.

\includegraphics{https://static01.nyt.com/images/2020/04/03/books/03Zhang1/merlin_170890956_d85eaf29-2a2d-46d8-a392-c861c0dd52d2-articleLarge.jpg?quality=75\&auto=webp\&disable=upscale}

\href{https://www.nytimes.com/by/concepcion-de-leon}{\includegraphics{https://static01.nyt.com/images/2018/07/16/multimedia/author-concepcion-de-leon/author-concepcion-de-leon-thumbLarge.png}}

By \href{https://www.nytimes.com/by/concepcion-de-leon}{Concepción de
León}

\begin{itemize}
\item
  Published April 4, 2020Updated April 7, 2020
\item
  \begin{itemize}
  \item
  \item
  \item
  \item
  \item
  \end{itemize}
\end{itemize}

The day in 2015 that C Pam Zhang was laid off from her first job out of
college, she celebrated with friends at the park, then promptly made
plans to move to Bangkok.

She had been working at a tech start-up in San Francisco for several
years, she said, ``kind of stress-testing this question of whether I
could be happy doing something that was not writing.'' But the layoff
felt liberating, so she decided to live off savings for a while and give
writing a real shot.

``I was like, I'm giving myself a year,'' Zhang said in an interview
last month. ``I've been complaining all this time about not having the
time to write. Will I actually write when I have the time, or am I just
a big fake?''

Over seven months in Thailand, she wrote more than a dozen short
stories, in the process landing on the sort of work she wanted to do:
speculative fiction that dealt with themes like death, migration and
loneliness.

That year, she also drafted a novel. Set during the Gold Rush with
elements of magical realism, it focuses on two orphaned siblings who are
fending for themselves in an American West where tigers roam. The book,
``\href{https://www.nytimes.com/2020/04/07/books/review/how-much-of-these-hills-is-gold-c-pam-zhang.html}{How
Much of These Hills Is Gold},'' is due out from Riverhead on Tuesday.

It is inspired by the emotional currents of Zhang's upbringing,
particularly the loneliness and insecurity of a childhood spent moving
around. It also reckons with grief, something she experienced after
losing her father when she was 22.

``The death of a parent has this really strange gravity, no matter how
many years you get away from it,'' Zhang, now 30, said. ``It warps
everything else around it.''

In ``How Much of These Hills Is Gold,'' the siblings, Lucy and Sam, go
into survival mode after their father dies and they are left penniless
in a hostile town. ``One thing I wanted to reflect on in the book was
how when you mourn in a way that is repressed, it will haunt you,''
Zhang said. ``You can't get away from it.''

Image

C Pam Zhang's ``How Much of These Hills Is Gold'' comes out April 7.

Born in Beijing (the C in her name is short for Chenji), she moved to
the United States, where her parents were already living, when she was
4. She had 10 different addresses by the time she was 18, as her parents
sought out better job opportunities or school systems for her and her
younger sister. But one move stands out. When Zhang was 8, the family
packed up their car and drove from Lexington, Ky., to Salinas, Calif.

``I was so struck by the landscape of America,'' she said, recalling
areas where they were pounded by torrential rain or in the plains of
Oklahoma, where she could see weather patterns from miles away. ``It's
really beautiful but also, in many parts of the country, extremely bleak
and kind of scary.''

Those lasting impressions informed her reading habits. A fan of Laura
Ingalls Wilder and John Steinbeck, she said, ``Eventually I realized
that the people in these books that I loved were always white. I wanted
to write a great American epic in which I saw myself reflected.''

Sarah McGrath, who edited the book, said that reading it reminded her of
Colson Whitehead's
``\href{https://www.nytimes.com/2016/08/03/books/review-the-underground-railroad-colson-whitehead.html}{The
Underground Railroad}'' or Mohsin Hamid's
``\href{https://www.nytimes.com/2017/03/10/books/review/exit-west-mohsin-hamid.html}{Exit
West},'' which ``help me understand our culture and our history in a new
way, not by telling it directly, but by showing it through emotion and
relationships and its art.''

``How Much of These Hills Is Gold'' is one of several new or forthcoming
books by Asian-American writers set in 1800s America. There is ``The
Thousand Crimes of Ming Tsu'' by Tom Lin, forthcoming from Little,
Brown, which is set 150 years ago and follows a Chinese-American
assassin seeking revenge after his wife is abducted.
``\href{https://www.nytimes.com/2020/03/06/books/review/prairie-lotus-linda-sue-park.html}{Prairie
Lotus},'' published last month, is a book for young readers that its
author, the Korean-American writer Linda Sue Park, describes as a
``painful reconciliation'' of her youthful love of Wilder's ``Little
House on the Prairie'' books with the sometimes racist views they
espoused.

Image

``Prairie Lotus,'' by Linda Sue Park, came out in March.

``I would lie in bed and imagine that I was Laura's best friend and make
up adventures that we would have together,'' Park said. But she realized
that, not only were people who looked like her not visible in the story,
``but that even if I had been, Laura's mother, in particular, would
probably have never even let her get close to me.''

This time period --- the Gold Rush and its aftermath --- and
Chinese-Americans' role in it is ripe for re-examination. Until
recently, the roughly 15,000 Chinese-American laborers who worked on the
first Transcontinental Railroad, built in the 1860s, were all but
\href{https://www.nytimes.com/2019/05/14/us/golden-spike-utah-railroad-150th-anniversary.html}{erased
from the historical record} and later barred from obtaining citizenship
by the Chinese Exclusion Act of 1882.

A well-known photograph from the inauguration of the Transcontinental
Railroad inspired a moment in Zhang's book. Lucy, one of the main
characters, ``hears the cheer that goes through the city the day the
last railroad tile is hammered. A golden spike holds track to earth,''
Zhang writes. ``A picture is drawn for the history books, a picture that
shows none of the people who look like her, who built it.''

``How Much of These Hills Is Gold'' is not meant to encapsulate the
Chinese-American experience of that time, but rather to portray ``the
loneliness of being an immigrant and not being allowed to stake a claim
to the place that you live in,'' Zhang said. In light of the
\href{https://www.nytimes.com/2020/03/23/us/chinese-coronavirus-racist-attacks.html}{discrimination
Asian-Americans are facing as the coronavirus has disrupted life},
however, she feels her book is more relevant than ever.

When she began writing it, ``I actually worried that the book's
depictions of naked racism and violence would seem too extreme, too
maudlin,'' she said. ``Now I'm reminded that these
\href{https://www.nytimes.com/2020/03/29/us/politics/coronavirus-asian-americans.html}{ugly
attitudes toward Asian-presenting people} have always lain just under
the veneer of the country, and that they are erupting now.''

After growing up moving from place to place, Zhang has settled in San
Francisco, where she lives with her partner and their cat and dog. She
works part-time as a creative director for a skin-care start-up, and
though she sometimes struggles with the stillness of domesticity, she
values financial security, having grown up in a family in which money
was sometimes tenuous.

``We don't talk enough about how, especially for people who come from
immigrant backgrounds, from poor backgrounds, from impoverished
families, that money is a source of emotional comfort,'' she said.
``It's not just money.''

Thinking about the literature of the West that she has long been drawn
to, and the way it showed ``that ordinary people can lead epic lives
against this epic backdrop,'' Zhang said, she feels similarly about
immigrants' stories. ``How they've crossed entirely new lands and traded
one life for another --- those stories are epic in nature,'' she said.
``They deserve to be told in that way.''

\emph{Follow New York Times Books on}
\href{https://www.facebook.com/nytbooks/}{\emph{Facebook}}\emph{,}
\href{https://twitter.com/nytimesbooks}{\emph{Twitter}} \emph{and}
\href{https://www.instagram.com/nytbooks/}{\emph{Instagram}}\emph{, sign
up for}
\href{https://www.nytimes.com/newsletters/books-review}{\emph{our
newsletter}} \emph{or}
\href{https://www.nytimes.com/interactive/2017/books/books-calendar.html}{\emph{our
literary calendar}}\emph{. And listen to us on the}
\href{https://www.nytimes.com/column/book-review-podcast}{\emph{Book
Review podcast}}\emph{.}

Advertisement

\protect\hyperlink{after-bottom}{Continue reading the main story}

\hypertarget{site-index}{%
\subsection{Site Index}\label{site-index}}

\hypertarget{site-information-navigation}{%
\subsection{Site Information
Navigation}\label{site-information-navigation}}

\begin{itemize}
\tightlist
\item
  \href{https://help.nytimes.com/hc/en-us/articles/115014792127-Copyright-notice}{©~2020~The
  New York Times Company}
\end{itemize}

\begin{itemize}
\tightlist
\item
  \href{https://www.nytco.com/}{NYTCo}
\item
  \href{https://help.nytimes.com/hc/en-us/articles/115015385887-Contact-Us}{Contact
  Us}
\item
  \href{https://www.nytco.com/careers/}{Work with us}
\item
  \href{https://nytmediakit.com/}{Advertise}
\item
  \href{http://www.tbrandstudio.com/}{T Brand Studio}
\item
  \href{https://www.nytimes.com/privacy/cookie-policy\#how-do-i-manage-trackers}{Your
  Ad Choices}
\item
  \href{https://www.nytimes.com/privacy}{Privacy}
\item
  \href{https://help.nytimes.com/hc/en-us/articles/115014893428-Terms-of-service}{Terms
  of Service}
\item
  \href{https://help.nytimes.com/hc/en-us/articles/115014893968-Terms-of-sale}{Terms
  of Sale}
\item
  \href{https://spiderbites.nytimes.com}{Site Map}
\item
  \href{https://help.nytimes.com/hc/en-us}{Help}
\item
  \href{https://www.nytimes.com/subscription?campaignId=37WXW}{Subscriptions}
\end{itemize}
