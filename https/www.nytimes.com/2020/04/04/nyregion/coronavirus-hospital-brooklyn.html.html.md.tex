Sections

SEARCH

\protect\hyperlink{site-content}{Skip to
content}\protect\hyperlink{site-index}{Skip to site index}

\href{https://www.nytimes.com/section/nyregion}{New York}

\href{https://myaccount.nytimes.com/auth/login?response_type=cookie\&client_id=vi}{}

\href{https://www.nytimes.com/section/todayspaper}{Today's Paper}

\href{/section/nyregion}{New York}\textbar{}`Code Blue': A Brooklyn
I.C.U. Fights for Each Life in a Coronavirus Surge

\href{https://nyti.ms/2R64RBd}{https://nyti.ms/2R64RBd}

\begin{itemize}
\item
\item
\item
\item
\item
\item
\end{itemize}

\href{https://www.nytimes.com/news-event/coronavirus?action=click\&pgtype=Article\&state=default\&region=TOP_BANNER\&context=storylines_menu}{The
Coronavirus Outbreak}

\begin{itemize}
\tightlist
\item
  live\href{https://www.nytimes.com/2020/08/08/world/coronavirus-updates.html?action=click\&pgtype=Article\&state=default\&region=TOP_BANNER\&context=storylines_menu}{Latest
  Updates}
\item
  \href{https://www.nytimes.com/interactive/2020/us/coronavirus-us-cases.html?action=click\&pgtype=Article\&state=default\&region=TOP_BANNER\&context=storylines_menu}{Maps
  and Cases}
\item
  \href{https://www.nytimes.com/interactive/2020/science/coronavirus-vaccine-tracker.html?action=click\&pgtype=Article\&state=default\&region=TOP_BANNER\&context=storylines_menu}{Vaccine
  Tracker}
\item
  \href{https://www.nytimes.com/interactive/2020/world/coronavirus-tips-advice.html?action=click\&pgtype=Article\&state=default\&region=TOP_BANNER\&context=storylines_menu}{F.A.Q.}
\item
  \href{https://www.nytimes.com/live/2020/08/07/business/stock-market-today-coronavirus?action=click\&pgtype=Article\&state=default\&region=TOP_BANNER\&context=storylines_menu}{Markets
  \& Economy}
\end{itemize}

Advertisement

\protect\hyperlink{after-top}{Continue reading the main story}

Supported by

\protect\hyperlink{after-sponsor}{Continue reading the main story}

\hypertarget{code-blue-a-brooklyn-icu-fights-for-each-life-in-a-coronavirus-surge}{%
\section{`Code Blue': A Brooklyn I.C.U. Fights for Each Life in a
Coronavirus
Surge}\label{code-blue-a-brooklyn-icu-fights-for-each-life-in-a-coronavirus-surge}}

Nearly every patient was on a ventilator. Some were in their 80s, some
in their 30s. Medical workers were falling fast and had to be
resourceful --- ``the alternative,'' one said, ``is death.''

\includegraphics{https://static01.nyt.com/images/2020/04/05/multimedia/05Virus-Brooklyn-p1-sub/merlin_171193407_c6e0ecf7-b317-420a-9711-bd59d5fc8fad-videoSixteenByNine3000.jpg}

\href{https://www.nytimes.com/by/sheri-fink}{\includegraphics{https://static01.nyt.com/images/2018/08/24/multimedia/author-sheri-fink/author-sheri-fink-thumbLarge.png}}

By \href{https://www.nytimes.com/by/sheri-fink}{Sheri Fink}

\begin{itemize}
\item
  April 4, 2020
\item
  \begin{itemize}
  \item
  \item
  \item
  \item
  \item
  \item
  \end{itemize}
\end{itemize}

The night had been particularly tough. Patient after patient had to be
intubated and put on a ventilator to breathe. At one point, three
``codes'' --- emergency interventions when someone is on the brink of
death --- occurred at once.

Dr. Joshua Rosenberg, a critical care doctor, arrived the next morning
at the Brooklyn Hospital Center. Within hours, he was racing down the
stairwell from the main intensive care unit on the sixth floor to a
temporary one on the third, where he passed one of his favorite medical
students.

``Shouldn't you be home?'' he asked, registering surprise. Clinical
rotations for students had been halted to avoid exposing them to the
coronavirus. ``My mom's here,'' the student replied.

Dr. Rosenberg, 45, let out an expletive and asked which bed she was in.
``I'm rounding there now,'' he said and made sure the student had his
cellphone number.

\includegraphics{https://static01.nyt.com/images/2020/04/05/us/05Virus-Brooklyn2/merlin_171193389_941d0e40-8079-4cb8-86bf-6f2e60cd1e26-articleLarge.jpg?quality=75\&auto=webp\&disable=upscale}

Earlier, residents from the I.C.U. had presented their cases to Dr.
Rosenberg and others, speaking in shorthand and at auctioneer-like
speed. There were so many patients to get through last Monday:

``Admitted for acute hypoxic respiratory failure secondary to likely
Covid-19.''

``Admitted for acute hypoxic respiratory failure secondary to confirmed
Covid-19.''

``Admitted for acute hypoxic respiratory failure, high suspicion of
Covid-19.''

Nearly every person lying in a bed in the new intensive care unit, just
as in the main one, was breathing with the help of a mechanical
ventilator.

There were patients in their 80s and in their 30s. Patients whose asthma
and diabetes helped explain their serious illness. And patients who
seemed to have no risk factors at all. Patients from nursing homes.
Patients who had no homes. Pregnant women, some of whom would not be
conscious when their babies were delivered to increase their odds of
surviving to raise their children.

This was the week that the coronavirus crisis pummeled the Brooklyn
hospital, just as it did others throughout New York City, where the
death toll reached more than 2,000, as the governor warned that vital
equipment and supplies would run short in just a few days, as the mayor
pleaded for more doctors and as hospital officials and political leaders
alike acknowledged that the situation would get even worse.

At the Brooklyn center --- a medium-size independent community hospital
--- that misery was evident. Deaths attributed to the virus more than
quintupled from the previous week. The number of inpatients confirmed to
have Covid-19, the disease caused by the virus, grew from 15 to 105,
with 48 more awaiting results. Hospital leaders estimated that about a
third of doctors and nurses were out sick. The hospital temporarily ran
out of protective plastic gowns, of the main sedative for patients on
ventilators, of a key blood pressure medication. The sense of urgency
and tragedy was heightened by a video, circulating online, showing a
forklift hoisting a body into a refrigerated trailer outside the
hospital.

Amid the unfolding disaster, in a week in which he would see more
deaths, counsel some families to let loved ones go and scramble to save
others, a weary Dr. Rosenberg paused to watch his team tend to their
patients. ``It's making the best of what you can do,'' he said.

Image

Dr. Joshua Rosenberg and the I.C.U. staff during their morning rounds
this past Monday. Nearly every patient that day had been
intubated.Credit...Victor J. Blue for The New York Times

\hypertarget{a-crisis-gathers-strength}{%
\subsection{A Crisis Gathers Strength}\label{a-crisis-gathers-strength}}

Dr. Rosenberg had to stay home the previous week, battling a fever and
intense fatigue from what he assumed was Covid-19 (a test, taken after
he felt better, later came back negative). He could barely climb the
stairs to his bedroom. Returning to work this past Monday, he told a
reporter, was like walking into a storm.

``This is insanity,'' Dr. Rosenberg said to a colleague that day.

Before he left, the intensive care unit had its usual 18 beds. The surge
\href{https://www.nytimes.com/2020/03/26/nyregion/coronavirus-brooklyn-hospital.html}{was
then hitting the emergency department}, leading the hospital to
construct a tent outside and screen scores of people a day. Many, mildly
ill, were reassured and sent home.

But during the time he was gone, the number of people progressing to
severe illness skyrocketed, and the I.C.U. had to expand, then expand
again, effectively doubling. ``In a week's time, we've transitioned from
a crowding outside to a crowding inside,'' said Lenny Singletary, the
hospital's senior vice president for external affairs.

\hypertarget{latest-updates-the-coronavirus-outbreak}{%
\section{\texorpdfstring{\href{https://www.nytimes.com/2020/08/07/world/covid-19-news.html?action=click\&pgtype=Article\&state=default\&region=MAIN_CONTENT_1\&context=storylines_live_updates}{Latest
Updates: The Coronavirus
Outbreak}}{Latest Updates: The Coronavirus Outbreak}}\label{latest-updates-the-coronavirus-outbreak}}

Updated 2020-08-08T12:04:28.992Z

\begin{itemize}
\tightlist
\item
  \href{https://www.nytimes.com/2020/08/07/world/covid-19-news.html?action=click\&pgtype=Article\&state=default\&region=MAIN_CONTENT_1\&context=storylines_live_updates\#link-1f86d03a}{As
  the U.S. relief talks falter again, Trump says he is prepared to act
  on his own.}
\item
  \href{https://www.nytimes.com/2020/08/07/world/covid-19-news.html?action=click\&pgtype=Article\&state=default\&region=MAIN_CONTENT_1\&context=storylines_live_updates\#link-3f64a70a}{Cuomo
  says N.Y. schools can reopen in-person but leaves it up to districts
  to determine if, when and how.}
\item
  \href{https://www.nytimes.com/2020/08/07/world/covid-19-news.html?action=click\&pgtype=Article\&state=default\&region=MAIN_CONTENT_1\&context=storylines_live_updates\#link-14e70066}{Thousands
  of cases went unreported in California when a computer server failed.}
\end{itemize}

\href{https://www.nytimes.com/2020/08/07/world/covid-19-news.html?action=click\&pgtype=Article\&state=default\&region=MAIN_CONTENT_1\&context=storylines_live_updates}{See
more updates}

More live coverage:
\href{https://www.nytimes.com/live/2020/08/07/business/stock-market-today-coronavirus?action=click\&pgtype=Article\&state=default\&region=MAIN_CONTENT_1\&context=storylines_live_updates}{Markets}

Even before the I.C.U.'s morning report had started, Dr. Rosenberg and
other staff members had to rush to an outpatient unit. A middle-aged man
had come to the hospital for dialysis but was sweating profusely. Staff
members were about to help him breathe using a mask with pressurized
air, known as a BiPAP machine.

But Dr. Rosenberg, chair of the hospital's infection control committee,
thought it was a poor idea. There was no way to know right then whether
the man's illness might be caused by the coronavirus, and there were
fears that the device could release virus particles into the air,
potentially spreading the disease. The patient was moved to the
emergency room. ``He has a high chance of getting tubed'' and needing a
ventilator, Dr. Rosenberg told colleagues.

In the new I.C.U., a repurposed chemotherapy infusion unit, blue plastic
gowns fluttered from door hinges, drying after being wiped down for
reuse. A patient bed, tilted up like a slide, held pink plastic bins
overflowing with patient supplies. Dr. Rosenberg's critical care team
assembled in mismatched clothing, masks and protective eyewear, hair and
foot coverings --- wearing much of the scarce equipment all day, not
changing between patients.

With so many staff members out and so many new patients, the array of
doctors, nurses, pharmacists and respiratory therapists who were
accustomed to working in the I.C.U. needed reinforcements. Dr. Rosenberg
welcomed a podiatrist and two of her resident trainees, a neurosurgery
physician assistant, surgery residents and a nurse anesthetist. ``All
people who are good with knives and big needles,'' Dr. Rosenberg
quipped.

Now, some nurses were caring for five critically ill patients at a time,
a ratio he called ``crazy.'' The norm for experienced I.C.U. nurses at
the hospital was just two.

At 10 a.m., Dr. Rosenberg and Dr. James Gasperino, chief of medicine and
critical care, jumped on a call with the hospital leadership about
challenges the center was facing and how it was coping with them.

Image

Dr. James Gasperino is the head of critical care at the hospital. The
I.C.U. under his watch has effectively doubled in size.Credit...Victor
J. Blue for The New York Times

Image

Dr.~Vasantha Kondamudi, the chief medical officer, described a surge in
Covid-19 patients at the hospital.Credit...Victor J. Blue for The New
York Times

The chief medical officer, Dr. Vasantha Kondamudi, later summed it up:
Staff was short, medical residents were falling ill every day, and the
number of patients with suspected or confirmed Covid-19 was ballooning
in nearly every area of the hospital. Yet the crisis had not peaked.

Nurses and others from departments that had cut back on services, like
elective surgery and outpatient clinics, were being trained and
redeployed. ``You're working completely differently,'' said Judy
McLaughlin, senior vice president and chief nursing executive. But even
that wasn't enough: The hospital had requested more than 100 volunteer
doctors and nurses from the city's Medical Reserve Corps and was rapidly
working to vet them.

After the call, Dr. Gasperino conferred in the hallway with the director
of respiratory therapy. The hospital had 98 ventilators, many acquired
in recent days, including small portable devices from the national
stockpile. Employees were running simulations to practice how they might
use each ventilator to treat two patients, a difficult and risky
proposition. ``We're doing this because the alternative is death,'' Dr.
Gasperino said.

An alert sounded on the loudspeaker, interrupting the conversation:
``Code blue, 6B. Code blue, 6B.''

The critical care team was designed to respond to emergencies anywhere
in the hospital. Although he was supposed to be on his way home after an
overnight shift, Dr. Gasperino joined more than than a dozen others
pouring into the patient's room.

``Covid?'' someone asked.

``No, not Covid,'' came the answer.

Young residents stood on either side of the man's bed and took turns
doing chest compressions. Nurses ran out of the room and back in with
supplies. Dr. Gasperino threaded a catheter into a large vein to infuse
medication into the patient's body. The man's pulse returned.

At about the same time, one of the pregnant patients was wheeled from
the intensive care unit and into an operating room for a cesarean
section. She was in her early 30s, and her baby was being delivered
nearly two months early in an effort to save the mother's life. Over the
past day, doctors had ordered two doses of steroid medication to help
the infant's lungs mature.

During rounds earlier that morning, a resident presented the woman's
case. She had been put on a ventilator and sedated the previous evening.
Dr. Rosenberg cursed under his breath: This disease was cruel.

Image

Will Vanderwall, a physician assistant, caring for a patient infected
with the coronavirus. A range of patients of varying ages, backgrounds
and medical histories have fallen critically ill.Credit...Victor J. Blue
for The New York Times

\hypertarget{grasping-for-solutions}{%
\subsection{Grasping for Solutions}\label{grasping-for-solutions}}

As Dr. Rosenberg walked down the corridor, nearly every door he passed
had a neon colored sticker warning that personal protective equipment
must be worn inside. ``COVID'' was handwritten on many of them.

Staff members had separated control boards from some of the ventilators,
so they could adjust their settings and monitor patients without going
inside their rooms unless necessary, reducing exposure to the virus.
Nurses were making a similar adjustment with the pumps that delivered
intravenous medications, adding extension tubing that snaked across
floors into hallways.

Workers rushed in and out of the rooms preparing for procedures. ``Watch
out, don't trip!'' Dr. Rosenberg warned a colleague. Moments later, he
had to repeat the warning. ``Watch out, don't trip!''

Later that day, when a patient became unstable, Dr. Rosenberg passed out
masks with a face shield --- ``they're clean, save them, they're gold''
--- to staff members before they entered the man's room. Dr. Rosenberg
put on a sterile gown and ski goggles, which he said he preferred
because they didn't fog up. He inserted a narrow tube into a patient's
artery to better monitor his vital signs. Procedures performed inside
the room, close to the patient, posed the greatest risk of exposure.

Amid the grimness, Dr. Rosenberg tried to keep the mood positive, his
energy fueled by espresso from an automatic machine in his office. He
called his colleagues ``dude,'' made sports analogies to explain his
points and sometimes asked how their families were dealing with the
stress. Even in the thick of a crisis, he directed questions to trainees
that forced them to think hard about the next step in care for each
patient.

Image

Dr. Rosenberg adjusting his ski goggles before entering the room of an
infected patient.Credit...Victor J. Blue for The New York Times

Image

The hospital was running low on supplies, and doctors wiped down their
gowns for reuse.Credit...Victor J. Blue for The New York Times

Being a teacher came easily to him. He had studied science at Wesleyan
--- earning his degree in three years to save on tuition costs --- and
then taught it to first graders at the Choir Academy of Harlem, a now
shuttered public school that was the home of the famous Boys Choir. He
went on to medical school in Israel, later returning to New York, where
he now lives with his wife and two daughters.

Dr. Rosenberg and his team reviewed the status of one of the many
patients who were receiving a ``Covid cocktail'' of the antimalarial
drug hydroxychloroquine, held up by President Trump as a potential cure,
and the antibiotic azithromycin. Dr. Rosenberg referred to it as a
``maybe-maybe-this-will-work cocktail,'' because only a couple of tiny
studies supported its effectiveness against Covid-19. Still, the doctors
were prescribing it aggressively now, early in the course of
hospitalization, in the hopes that it could prevent the lung damage that
led patients to need ventilators.

The cocktail is generally considered safe, though it may have serious
side effects in certain patients. One man in the I.C.U. developed a
deadly arrhythmia and had to be shocked back to life the night before
Dr. Rosenberg's Monday shift. The doctor told his residents that the
patient should not go back on the drug.

``I don't think the public realizes how often we don't really know''
whether something works, Dr. Rosenberg said. Different coronaviruses can
cause the common cold, which ``affects all of us,'' he said. ``There's
no medicine to get better from it --- it's just time, patience.'' What
scared him with this new coronavirus, though, was the thought that
``time and patience when somebody's on a ventilator is different from
time and patience when someone has the sniffles.''

His team had also begun treating some patients with another medication,
an experimental antiviral drug called remdesivir. But the hospital had
to apply to the manufacturer, Gilead, for emergency permission to use it
on each patient, who had to have a confirmed diagnosis of Covid-19.

\href{https://www.nytimes.com/news-event/coronavirus?action=click\&pgtype=Article\&state=default\&region=MAIN_CONTENT_3\&context=storylines_faq}{}

\hypertarget{the-coronavirus-outbreak-}{%
\subsubsection{The Coronavirus Outbreak
›}\label{the-coronavirus-outbreak-}}

\hypertarget{frequently-asked-questions}{%
\paragraph{Frequently Asked
Questions}\label{frequently-asked-questions}}

Updated August 6, 2020

\begin{itemize}
\item ~
  \hypertarget{why-are-bars-linked-to-outbreaks}{%
  \paragraph{Why are bars linked to
  outbreaks?}\label{why-are-bars-linked-to-outbreaks}}

  \begin{itemize}
  \tightlist
  \item
    Think about a bar. Alcohol is flowing. It can be loud, but it's
    definitely intimate, and you often need to lean in close to hear
    your friend. And strangers have way, way fewer reservations about
    coming up to people in a bar. That's sort of the point of a bar.
    Feeling good and close to strangers. It's no surprise, then, that
    \href{https://www.nytimes.com/2020/07/02/us/coronavirus-bars.html?action=click\&pgtype=Article\&state=default\&region=MAIN_CONTENT_3\&context=storylines_faq}{bars
    have been linked to outbreaks in several states.} Louisiana health
    officials have tied
    \href{https://www.nytimes.com/2020/06/22/us/new-coronavirus-phase.html?action=click\&pgtype=Article\&state=default\&region=MAIN_CONTENT_3\&context=storylines_faq}{at
    least 100 coronavirus cases} to bars in the Tigerland nightlife
    district in Baton Rouge. Minnesota has traced 328 recent cases to
    bars across the state.
    \href{https://www.boisestatepublicradio.org/post/bars-large-venues-close-ada-county-after-surge-coronavirus-prompts-rollback\#stream/0}{In
    Idaho}, health officials shut down bars in Ada County after
    reporting clusters of infections among young adults who had visited
    several bars in downtown Boise. Governors in
    \href{https://www.nytimes.com/2020/07/01/us/california-coronavirus-reopening.html?action=click\&pgtype=Article\&state=default\&region=MAIN_CONTENT_3\&context=storylines_faq}{California},
    \href{https://www.nytimes.com/2020/06/14/us/coronavirus-united-states.html?action=click\&pgtype=Article\&state=default\&region=MAIN_CONTENT_3\&context=storylines_faq}{Texas
    and Arizona}, where coronavirus cases are soaring, have ordered
    hundreds of newly reopened bars to shut down. Less than two weeks
    after Colorado's bars reopened at limited capacity, Gov. Jared Polis
    \href{https://www.denverpost.com/2020/06/30/colorado-bars-closed-coronavirus/}{ordered
    them to close}.
  \end{itemize}
\item ~
  \hypertarget{i-have-antibodies-am-i-now-immune}{%
  \paragraph{I have antibodies. Am I now
  immune?}\label{i-have-antibodies-am-i-now-immune}}

  \begin{itemize}
  \tightlist
  \item
    As of right now,
    \href{https://www.nytimes.com/2020/07/22/health/covid-antibodies-herd-immunity.html?action=click\&pgtype=Article\&state=default\&region=MAIN_CONTENT_3\&context=storylines_faq}{that
    seems likely, for at least several months.} There have been
    frightening accounts of people suffering what seems to be a second
    bout of Covid-19. But experts say these patients may have a
    drawn-out course of infection, with the virus taking a slow toll
    weeks to months after initial exposure. People infected with the
    coronavirus typically
    \href{https://www.nature.com/articles/s41586-020-2456-9}{produce}
    immune molecules called antibodies, which are
    \href{https://www.nytimes.com/2020/05/07/health/coronavirus-antibody-prevalence.html?action=click\&pgtype=Article\&state=default\&region=MAIN_CONTENT_3\&context=storylines_faq}{protective
    proteins made in response to an
    infection}\href{https://www.nytimes.com/2020/05/07/health/coronavirus-antibody-prevalence.html?action=click\&pgtype=Article\&state=default\&region=MAIN_CONTENT_3\&context=storylines_faq}{.
    These antibodies may} last in the body
    \href{https://www.nature.com/articles/s41591-020-0965-6}{only two to
    three months}, which may seem worrisome, but that's perfectly normal
    after an acute infection subsides, said Dr. Michael Mina, an
    immunologist at Harvard University. It may be possible to get the
    coronavirus again, but it's highly unlikely that it would be
    possible in a short window of time from initial infection or make
    people sicker the second time.
  \end{itemize}
\item ~
  \hypertarget{im-a-small-business-owner-can-i-get-relief}{%
  \paragraph{I'm a small-business owner. Can I get
  relief?}\label{im-a-small-business-owner-can-i-get-relief}}

  \begin{itemize}
  \tightlist
  \item
    The
    \href{https://www.nytimes.com/article/small-business-loans-stimulus-grants-freelancers-coronavirus.html?action=click\&pgtype=Article\&state=default\&region=MAIN_CONTENT_3\&context=storylines_faq}{stimulus
    bills enacted in March} offer help for the millions of American
    small businesses. Those eligible for aid are businesses and
    nonprofit organizations with fewer than 500 workers, including sole
    proprietorships, independent contractors and freelancers. Some
    larger companies in some industries are also eligible. The help
    being offered, which is being managed by the Small Business
    Administration, includes the Paycheck Protection Program and the
    Economic Injury Disaster Loan program. But lots of folks have
    \href{https://www.nytimes.com/interactive/2020/05/07/business/small-business-loans-coronavirus.html?action=click\&pgtype=Article\&state=default\&region=MAIN_CONTENT_3\&context=storylines_faq}{not
    yet seen payouts.} Even those who have received help are confused:
    The rules are draconian, and some are stuck sitting on
    \href{https://www.nytimes.com/2020/05/02/business/economy/loans-coronavirus-small-business.html?action=click\&pgtype=Article\&state=default\&region=MAIN_CONTENT_3\&context=storylines_faq}{money
    they don't know how to use.} Many small-business owners are getting
    less than they expected or
    \href{https://www.nytimes.com/2020/06/10/business/Small-business-loans-ppp.html?action=click\&pgtype=Article\&state=default\&region=MAIN_CONTENT_3\&context=storylines_faq}{not
    hearing anything at all.}
  \end{itemize}
\item ~
  \hypertarget{what-are-my-rights-if-i-am-worried-about-going-back-to-work}{%
  \paragraph{What are my rights if I am worried about going back to
  work?}\label{what-are-my-rights-if-i-am-worried-about-going-back-to-work}}

  \begin{itemize}
  \tightlist
  \item
    Employers have to provide
    \href{https://www.osha.gov/SLTC/covid-19/standards.html}{a safe
    workplace} with policies that protect everyone equally.
    \href{https://www.nytimes.com/article/coronavirus-money-unemployment.html?action=click\&pgtype=Article\&state=default\&region=MAIN_CONTENT_3\&context=storylines_faq}{And
    if one of your co-workers tests positive for the coronavirus, the
    C.D.C.} has said that
    \href{https://www.cdc.gov/coronavirus/2019-ncov/community/guidance-business-response.html}{employers
    should tell their employees} -\/- without giving you the sick
    employee's name -\/- that they may have been exposed to the virus.
  \end{itemize}
\item ~
  \hypertarget{what-is-school-going-to-look-like-in-september}{%
  \paragraph{What is school going to look like in
  September?}\label{what-is-school-going-to-look-like-in-september}}

  \begin{itemize}
  \tightlist
  \item
    It is unlikely that many schools will return to a normal schedule
    this fall, requiring the grind of
    \href{https://www.nytimes.com/2020/06/05/us/coronavirus-education-lost-learning.html?action=click\&pgtype=Article\&state=default\&region=MAIN_CONTENT_3\&context=storylines_faq}{online
    learning},
    \href{https://www.nytimes.com/2020/05/29/us/coronavirus-child-care-centers.html?action=click\&pgtype=Article\&state=default\&region=MAIN_CONTENT_3\&context=storylines_faq}{makeshift
    child care} and
    \href{https://www.nytimes.com/2020/06/03/business/economy/coronavirus-working-women.html?action=click\&pgtype=Article\&state=default\&region=MAIN_CONTENT_3\&context=storylines_faq}{stunted
    workdays} to continue. California's two largest public school
    districts --- Los Angeles and San Diego --- said on July 13, that
    \href{https://www.nytimes.com/2020/07/13/us/lausd-san-diego-school-reopening.html?action=click\&pgtype=Article\&state=default\&region=MAIN_CONTENT_3\&context=storylines_faq}{instruction
    will be remote-only in the fall}, citing concerns that surging
    coronavirus infections in their areas pose too dire a risk for
    students and teachers. Together, the two districts enroll some
    825,000 students. They are the largest in the country so far to
    abandon plans for even a partial physical return to classrooms when
    they reopen in August. For other districts, the solution won't be an
    all-or-nothing approach.
    \href{https://bioethics.jhu.edu/research-and-outreach/projects/eschool-initiative/school-policy-tracker/}{Many
    systems}, including the nation's largest, New York City, are
    devising
    \href{https://www.nytimes.com/2020/06/26/us/coronavirus-schools-reopen-fall.html?action=click\&pgtype=Article\&state=default\&region=MAIN_CONTENT_3\&context=storylines_faq}{hybrid
    plans} that involve spending some days in classrooms and other days
    online. There's no national policy on this yet, so check with your
    municipal school system regularly to see what is happening in your
    community.
  \end{itemize}
\end{itemize}

``Do we have a positive test?'' Dr. Rosenberg asked about one patient. A
colleague replied, ``Not yet.'' Test results from a Quest commercial
laboratory in California had been taking about a week, making it harder
to isolate infected patients within the building, provide certain
treatments and even discharge people. Laboratory workers at the Brooklyn
hospital managed to retrofit equipment and start their own testing last
weekend, which doctors considered a game changer.

But with one problem resolved, another arose. This past week, there were
days when the hospital ran short of a drug to treat life-threatening low
blood pressure in many of Dr. Rosenberg's I.C.U. patients, as well as a
sedative that many were receiving to relieve the distress of being on a
ventilator. The doctors ordered substitutes.

The chief pharmacist at the Brooklyn hospital, Robert DiGregorio, worked
until after 2 a.m. on Thursday to try to source more of one drug. Going
forward, Dr. Rosenberg predicted, ``the biggest threat will be
medication shortages.''

Image

Dr. Gasperino at a morning meeting. About a third of the hospital's
doctors and nurses were out sick, and reinforcements came from all areas
of medicine.Credit...Victor J. Blue for The New York Times

\hypertarget{painful-conversations}{%
\subsection{Painful Conversations}\label{painful-conversations}}

Dr. Rosenberg was struck by the range of the patients felled by this
illness --- various ages, ethnicities and medical histories. Some who
had been critically ill, most of them younger, were starting to recover
enough to be taken off a ventilator and breathe on their own.

But as he and his team stopped outside each room, they saw many who were
from nursing homes and had multiple medical problems --- the type of
patients who filled the intensive care unit during flu season. Now some
were extremely sick, with failing organs.

``Very poor prognosis,'' Dr. Rosenberg said about one man, in his 70s,
who had developed kidney damage. ``He's going to pass from this.''

``Has anyone been in contact with the patient's family?'' he asked. He
asked a variation of that in front of other rooms. ``All of these
patients need a palliative care'' consultation, the physician said of
the seriously ill.

The patients were alone. Visitors were no longer allowed into the
hospital, and doctors had to call family members to update them, get
their permission for doing procedures and --- for many --- discuss
end-of-life care.

That day and continuing through the week, Dr. Rosenberg had many
difficult conversations, on the phone and often through translators,
about shifting from trying to extend life to withdrawing life support
and focusing on comfort.

``A lot of family members don't realize how sick the patients are or how
bad the prognosis is with this disease if you develop respiratory
failure,'' he said, particularly in the context of advanced age and
other health conditions. ``The families really want to see their loved
ones.'' The team was using iPads and smartphones to connect them.

Image

Dr. Nerantzakis and Mr. Vanderwall prepare to perform a procedure on a
patient stricken with the coronavirus.Credit...Victor J. Blue for The
New York Times

Image

The two medical professionals, along with Dr. Camille Kim, a podiatry
resident, with the Covid-19 patient.Credit...Victor J. Blue for The New
York Times

He said that the state's laws governing withdrawing patients from
ventilators were complicated. The default, generally, is for doctors to
initiate and continue providing life support unless the patient or proxy
has clear directives otherwise. ``It reflects on the need for these
conversations in primary care well before somebody gets sick and for
that information to be disseminated to family members.''

He added, ``There are an awful lot of really young patients in their 50s
and 60s who I'm sure never thought about this.''

There were fears throughout the week that New York's hospitals would
soon run out of ventilators and be forced to ration them, but doctors at
the Brooklyn center said they had enough for now. Dr. Rosenberg worried
more about having enough staff members and medications.

Still, Dr. Rosenberg said that he and his colleagues were looking at
protocols for how to ration care, developed by intensive care doctors at
other medical centers, in case conditions worsened.

The goal was to expand capacity to avoid the need to limit treatment.
Gary G. Terrinoni, the hospital's president and chief executive, said he
had received donations of food and supplies, but was appealing to the
city and state for physical beds, equipment and funds to ``ensure we can
serve the community'' as his clinical colleagues fought ``the good
fight.''

But even discharging those who no longer needed hospital care to make
space for new patients was sometimes proving difficult. Dr. Rosenberg
worried about getting one of his patients, ready to leave the I.C.U.,
accepted back into a nursing home, where across the city staffing had
fallen short. Government officials were working on sites to accept
released patients, but those had not yet opened.

Even death did not always guarantee an exit. By the end of the week, the
hospital had accepted two refrigerated trailers from the city's medical
examiner. Workers were building shelves in one of them to make space for
more bodies, as overwhelmed funeral homes were failing in some cases to
retrieve them. A tent discouraged onlookers from recording more
cellphone videos.

Meanwhile, patients continued to arrive at the I.C.U. --- some of them
with ties to the 175-year-old institution, near Fort Greene. ``It's like
home for us,'' said Dr. Kondamudi, the chief medical officer.

Dr. Antonio Mendez, the vice chair of the emergency department, was born
at the hospital, and his mother, Josefina, was admitted as an I.C.U.
patient. ``She is a fighter and so are her doctors,'' he said.

On his first day back, Dr. Rosenberg checked her blood gas, a measure of
the effectiveness of her breathing support. It ``looks pretty darn
skippy,'' he said and praised his team for their management of her care.

Late in that long day, Dr. Rosenberg learned that one of the hospital's
own medical residents, whom he knew well, was in the emergency room,
with symptoms of Covid-19 and a worrisome chest X-ray.

``He comes right up,'' he told his team, ``because he's at high risk of
getting intubated.''

To admit the physician to the I.C.U., however, Dr. Rosenberg had to get
more staff. ``We need more nurses,'' he said. Given how overwhelmed they
are, ``they're getting killed.''

Soon after, two nurses who normally worked in the cardiac
catheterization lab walked into the unit to offer their assistance. Dr.
Rosenberg applauded. ``This is the cavalry,'' he said.

Advertisement

\protect\hyperlink{after-bottom}{Continue reading the main story}

\hypertarget{site-index}{%
\subsection{Site Index}\label{site-index}}

\hypertarget{site-information-navigation}{%
\subsection{Site Information
Navigation}\label{site-information-navigation}}

\begin{itemize}
\tightlist
\item
  \href{https://help.nytimes.com/hc/en-us/articles/115014792127-Copyright-notice}{©~2020~The
  New York Times Company}
\end{itemize}

\begin{itemize}
\tightlist
\item
  \href{https://www.nytco.com/}{NYTCo}
\item
  \href{https://help.nytimes.com/hc/en-us/articles/115015385887-Contact-Us}{Contact
  Us}
\item
  \href{https://www.nytco.com/careers/}{Work with us}
\item
  \href{https://nytmediakit.com/}{Advertise}
\item
  \href{http://www.tbrandstudio.com/}{T Brand Studio}
\item
  \href{https://www.nytimes.com/privacy/cookie-policy\#how-do-i-manage-trackers}{Your
  Ad Choices}
\item
  \href{https://www.nytimes.com/privacy}{Privacy}
\item
  \href{https://help.nytimes.com/hc/en-us/articles/115014893428-Terms-of-service}{Terms
  of Service}
\item
  \href{https://help.nytimes.com/hc/en-us/articles/115014893968-Terms-of-sale}{Terms
  of Sale}
\item
  \href{https://spiderbites.nytimes.com}{Site Map}
\item
  \href{https://help.nytimes.com/hc/en-us}{Help}
\item
  \href{https://www.nytimes.com/subscription?campaignId=37WXW}{Subscriptions}
\end{itemize}
