Sections

SEARCH

\protect\hyperlink{site-content}{Skip to
content}\protect\hyperlink{site-index}{Skip to site index}

\href{https://www.nytimes.com/section/obituaries}{Obituaries}

\href{https://myaccount.nytimes.com/auth/login?response_type=cookie\&client_id=vi}{}

\href{https://www.nytimes.com/section/todayspaper}{Today's Paper}

\href{/section/obituaries}{Obituaries}\textbar{}Rafael Gómez Nieto, Last
Member of Unit That Helped Liberate Paris, Dies at 99

\url{https://nyti.ms/2wWv1zA}

\begin{itemize}
\item
\item
\item
\item
\item
\end{itemize}

\href{https://www.nytimes.com/news-event/coronavirus?action=click\&pgtype=Article\&state=default\&region=TOP_BANNER\&context=storylines_menu}{The
Coronavirus Outbreak}

\begin{itemize}
\tightlist
\item
  live\href{https://www.nytimes.com/2020/08/03/world/coronavirus-covid-19.html?action=click\&pgtype=Article\&state=default\&region=TOP_BANNER\&context=storylines_menu}{Latest
  Updates}
\item
  \href{https://www.nytimes.com/interactive/2020/us/coronavirus-us-cases.html?action=click\&pgtype=Article\&state=default\&region=TOP_BANNER\&context=storylines_menu}{Maps
  and Cases}
\item
  \href{https://www.nytimes.com/interactive/2020/science/coronavirus-vaccine-tracker.html?action=click\&pgtype=Article\&state=default\&region=TOP_BANNER\&context=storylines_menu}{Vaccine
  Tracker}
\item
  \href{https://www.nytimes.com/2020/08/02/us/covid-college-reopening.html?action=click\&pgtype=Article\&state=default\&region=TOP_BANNER\&context=storylines_menu}{College
  Reopening}
\item
  \href{https://www.nytimes.com/live/2020/08/03/business/stock-market-today-coronavirus?action=click\&pgtype=Article\&state=default\&region=TOP_BANNER\&context=storylines_menu}{Economy}
\end{itemize}

Advertisement

\protect\hyperlink{after-top}{Continue reading the main story}

Supported by

\protect\hyperlink{after-sponsor}{Continue reading the main story}

Those We've Lost

\hypertarget{rafael-guxf3mez-nieto-last-member-of-unit-that-helped-liberate-paris-dies-at-99}{%
\section{Rafael Gómez Nieto, Last Member of Unit That Helped Liberate
Paris, Dies at
99}\label{rafael-guxf3mez-nieto-last-member-of-unit-that-helped-liberate-paris-dies-at-99}}

Mr. Gómez, a veteran of the Spanish Civil War who served in a French
unit that entered Paris in 1944, was infected by the coronavirus.

\includegraphics{https://static01.nyt.com/images/2020/04/09/obituaries/00gomez/merlin_171179982_873da48c-a127-4e2b-b201-a7e9519010e5-articleLarge.jpg?quality=75\&auto=webp\&disable=upscale}

\href{https://www.nytimes.com/by/raphael-minder}{\includegraphics{https://static01.nyt.com/images/2018/10/15/multimedia/author-raphael-minder/author-raphael-minder-thumbLarge.png}}

By \href{https://www.nytimes.com/by/raphael-minder}{Raphael Minder}

\begin{itemize}
\item
  Published April 5, 2020Updated April 16, 2020
\item
  \begin{itemize}
  \item
  \item
  \item
  \item
  \item
  \end{itemize}
\end{itemize}

\emph{This obituary is part of a series about}
\href{https://www.nytimes.com/series/people-who-have-died-of-the-coronavirus}{\emph{people
who have died in the coronavirus pandemic}}\emph{.}

D-Day was approaching and France's 2nd Armored Division was ready to
take part. Rafael Gómez Nieto, a veteran of the Spanish Civil War, and
the division's mostly Spanish ``Nueve'' unit shipped off from North
Africa to England.

The division, led by Gen. Philippe Leclerc, then crossed the English
Channel and La Nueve made its way through the French countryside. Mr.
Gomez's unit narrowly survived a German encirclement in a French
village.

He and his comrades forged on, reaching the southern edge of Paris on
Aug. 24, 1944. In an armored vehicle named Guernica, he wound his way
through side streets to avoid German soldiers and then raced at full
speed toward City Hall.

His unit turned out to be the first Allied force to take part in the
liberation of Paris.

``I entered Paris with the Guernica, we were Spanish and voilà!'' Mr.
Gómez recalled in a
\href{http://www.canalsur.es/multimedia.html?id=1571558\&jwsource=cl}{documentary}
aired by Canal Sur, a regional Spanish channel. ``People were surprised
to see French troops speaking to them in Spanish.''

Mr. Gómez, the last surviving member of La Nueve, died on Tuesday in
Strasbourg, France. He was 99. He had contracted the coronavirus, his
son, Jean-Paul Gómez, said by phone.

The office of the French president, Emmanuel Macron, said in a
\href{https://www.elysee.fr/emmanuel-macron/2020/04/01/deces-de-rafael-gomez-nieto}{statement,}
that with his death, ``Part of our French and European history has gone.
That of the Spanish Civil War and the Second World War, that of the
odyssey of Spanish Republicans engaged in the fight to suppress the Nazi
yoke.''

Mr. Gómez was born on Jan. 21, 1921, in Adrá, a town in the southern
province of Almería. His father, a career soldier, was stationed in
northeastern Catalonia when Spain's civil war started.

Mr. Gómez was recruited as a teenager into the ``Quinta del Biberón,'' a
celebrated contingent formed in 1938 by the Republican government in a
last-ditch effort to stop the advance of the troops of Gen. Francisco
Franco, who had plunged Spain into a civil war after staging a coup in
July 1936.

Mr. Gómez and the Quinta soon headed for the
\href{https://thespanishcivilwar.com/events-and-battles/battle-of-the-ebro/}{Battle
of the Ebro}, one of the bloodiest episodes of the civil war, during
which the Republicans unsuccessfully sought to turn the tide by
launching a large-scale counterattack across the Ebro river.

When Franco won the war in April 1939, Mr. Gómez and his father escaped
to France and found themselves interned in different camps. After his
release, Mr. Gómez headed for the Algerian city of Oran, where an uncle
lived.

There, he volunteered to join La Nueve, in which 146 of the 160 men were
Spanish. When he told his mother that he was going to fight another war,
``She smashed a glass on the floor out of anger,'' Mr. Gómez recalled in
the documentary. ``I just wanted to fight for the good of humanity,'' he
added.

Supplied by the American military, La Nueve helped drive the Nazis out
of North Africa, under the orders of a French captain, Raymond Dronne.

``We were famous because we had already done a war and knew what a war
was about,'' Mr. Gómez said. ``They put us on the front line and told
us: `La Nueve, go for it.'''

It took six decades for La Nueve to be officially honored by France,
when the authorities in Paris unveiled a plaque in 2004 to commemorate
its special role in the city's liberation.

Robert S. Coale, a professor of Hispanic Studies at the University of
Rouen, France, said in the documentary about Mr. Gómez that La Nueve was
purposefully forgotten after the war both in France and Spain.

``People didn't talk much in France about the Spanish soldiers in
Leclerc's division, because according to the French they were French
soldiers, and people didn't talk in Spain about the anti-Fascist
Spaniards in World War II because we still had Franco.''

After Paris, La Nueve helped liberate Strasbourg and eventually made its
way into Germany. Mr. Gómez reached Berchtesgaden, where Hitler had his
mountain residence, just in time for Germany's surrender.

Only about a dozen members of the unit survived the war because the
French army also put it in the vanguard, or ``always on the front
line,'' Mr. Macron's office said.

Mr. Gómez, who was awarded the Legion of Honor, is survived by his son
and three daughters. He also had 14 grandchildren and several
great-grandchildren. His wife, Florence López, died in 2015.

Like other Republican soldiers, Mr. Gómez hoped that the Allies, after
defeating Germany, would attack Franco's Spain, given the military
support provided to his war effort by Fascist Italy and Nazi Germany,
said Evelyn Mesquida, the author of
``\href{https://www.casadellibro.com/libro-la-nueve-los-espanoles-que-liberaron-paris/9788466659352/3019252}{La
Nueve}'' (2008).

``He had to live with the feeling of being the victim of a treason,''
she said in a phone interview, ``since clearly all the Spanish soldiers
hoped that Franco would be the next target of the Allies, to remove
Fascism.''

\href{https://www.nytimes.com/interactive/2020/obituaries/people-died-coronavirus-obituaries.html?action=click\&pgtype=Article\&state=default\&region=BELOW_MAIN_CONTENT\&context=covid_obits_promo}{}

\hypertarget{those-weve-lost}{%
\section{Those We've Lost}\label{those-weve-lost}}

The coronavirus pandemic has taken an incalculable death toll. This
series is designed to put names and faces to the numbers.

Read more

\includegraphics{https://static01.nyt.com/images/2020/07/30/obituaries/30Pedro/30Pedro-square640.jpg}

\hypertarget{bernaldina-josuxe9-pedro}{%
\section{Bernaldina José Pedro}\label{bernaldina-josuxe9-pedro}}

d. Boa Vista, Brazil

Leader among the Indigenous Macuxi

\includegraphics{https://static01.nyt.com/images/2020/07/31/obituaries/31Swing/merlin_175167783_8913bc90-0d64-43f3-a655-1bb1bf1601c9-square640.jpg}

\hypertarget{john-eric-swing}{%
\section{John Eric Swing}\label{john-eric-swing}}

d. Fountain Valley, Calif.

Champion of Filipino-Americans

\includegraphics{https://static01.nyt.com/images/2020/07/27/obituaries/27Victor/merlin_175001436_38b11f8e-227a-4e2c-9821-7618af9b2524-square640.jpg}

\hypertarget{victor-victor}{%
\section{Victor Victor}\label{victor-victor}}

d. Santo Domingo, Dominican Republic

Beloved musician of the Dominican Republic

\includegraphics{https://static01.nyt.com/images/2020/07/31/obituaries/31Negron/merlin_175160169_516322ae-fd23-4969-b6b2-193ced371105-square640.jpg}

\hypertarget{dr-eddie-negruxf3n}{%
\section{Dr. Eddie Negrón}\label{dr-eddie-negruxf3n}}

d. Fort Walton Beach, Fla.

Internist on Florida's Emerald Coast

\includegraphics{https://static01.nyt.com/images/2020/07/30/obituaries/30Dobson/merlin_175115928_f6b9271c-8f05-4fe1-a38a-5ca4a58f8935-square640.jpg}

\hypertarget{dobby-dobson}{%
\section{Dobby Dobson}\label{dobby-dobson}}

d. Coral Springs, Fla.

Jamaican singer and songwriter

\includegraphics{https://static01.nyt.com/images/2020/08/01/obituaries/28Gonzalez/merlin_175002771_beb57888-3951-409a-ae13-03a94b2e962e-square640.jpg}

\hypertarget{waldemar-gonzalez}{%
\section{Waldemar Gonzalez}\label{waldemar-gonzalez}}

d. White Plains, N.Y.

Teacher and social worker

Advertisement

\protect\hyperlink{after-bottom}{Continue reading the main story}

\hypertarget{site-index}{%
\subsection{Site Index}\label{site-index}}

\hypertarget{site-information-navigation}{%
\subsection{Site Information
Navigation}\label{site-information-navigation}}

\begin{itemize}
\tightlist
\item
  \href{https://help.nytimes.com/hc/en-us/articles/115014792127-Copyright-notice}{©~2020~The
  New York Times Company}
\end{itemize}

\begin{itemize}
\tightlist
\item
  \href{https://www.nytco.com/}{NYTCo}
\item
  \href{https://help.nytimes.com/hc/en-us/articles/115015385887-Contact-Us}{Contact
  Us}
\item
  \href{https://www.nytco.com/careers/}{Work with us}
\item
  \href{https://nytmediakit.com/}{Advertise}
\item
  \href{http://www.tbrandstudio.com/}{T Brand Studio}
\item
  \href{https://www.nytimes.com/privacy/cookie-policy\#how-do-i-manage-trackers}{Your
  Ad Choices}
\item
  \href{https://www.nytimes.com/privacy}{Privacy}
\item
  \href{https://help.nytimes.com/hc/en-us/articles/115014893428-Terms-of-service}{Terms
  of Service}
\item
  \href{https://help.nytimes.com/hc/en-us/articles/115014893968-Terms-of-sale}{Terms
  of Sale}
\item
  \href{https://spiderbites.nytimes.com}{Site Map}
\item
  \href{https://help.nytimes.com/hc/en-us}{Help}
\item
  \href{https://www.nytimes.com/subscription?campaignId=37WXW}{Subscriptions}
\end{itemize}
