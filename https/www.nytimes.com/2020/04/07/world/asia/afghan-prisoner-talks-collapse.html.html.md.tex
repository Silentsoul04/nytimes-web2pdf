Sections

SEARCH

\protect\hyperlink{site-content}{Skip to
content}\protect\hyperlink{site-index}{Skip to site index}

\href{https://www.nytimes.com/section/world/asia}{Asia Pacific}

\href{https://myaccount.nytimes.com/auth/login?response_type=cookie\&client_id=vi}{}

\href{https://www.nytimes.com/section/todayspaper}{Today's Paper}

\href{/section/world/asia}{Asia Pacific}\textbar{}Afghan Prisoner Swap
Hits Wall as Taliban Pull Out of Talks

\url{https://nyti.ms/2Rn08eC}

\begin{itemize}
\item
\item
\item
\item
\item
\end{itemize}

Advertisement

\protect\hyperlink{after-top}{Continue reading the main story}

Supported by

\protect\hyperlink{after-sponsor}{Continue reading the main story}

\hypertarget{afghan-prisoner-swap-hits-wall-as-taliban-pull-out-of-talks}{%
\section{Afghan Prisoner Swap Hits Wall as Taliban Pull Out of
Talks}\label{afghan-prisoner-swap-hits-wall-as-taliban-pull-out-of-talks}}

Discussions with insurgents in Kabul unraveled as Afghan officials
refused to include senior Taliban commanders in the first batch of
prisoners to be released.

\includegraphics{https://static01.nyt.com/images/2020/04/07/world/07afghanistan/merlin_171145764_2c35907b-ddd8-4567-916d-a8d0c123cf54-articleLarge.jpg?quality=75\&auto=webp\&disable=upscale}

\href{https://www.nytimes.com/by/mujib-mashal}{\includegraphics{https://static01.nyt.com/images/2018/10/15/multimedia/author-mujib-mashal/author-mujib-mashal-thumbLarge.png}}

By \href{https://www.nytimes.com/by/mujib-mashal}{Mujib Mashal}

\begin{itemize}
\item
  April 7, 2020
\item
  \begin{itemize}
  \item
  \item
  \item
  \item
  \item
  \end{itemize}
\end{itemize}

KABUL, Afghanistan --- A week of talks between the Afghan government and
the Taliban on a prisoner swap --- seen as crucial to preserving a
\href{https://www.nytimes.com/2020/02/29/world/asia/us-taliban-deal.html}{fragile
peace deal} between the insurgents and the United States --- appeared to
be collapsing on Tuesday, as Taliban leaders ordered their team to pull
out of the discussions.

An agreement
\href{https://www.nytimes.com/2020/02/29/world/asia/us-taliban-deal.html}{signed
between the United States and the Taliban in February} that started the
withdrawal of American troops from Afghanistan calls for the swap of
thousands of prisoners before the two Afghan sides sit together for
talks over a future power-sharing. But the prisoner swap, which was to
be done in batches, has faced opposition and hurdles all along,
threatening the unraveling of a deal that the Trump administration hoped
would signal the end of America's longest war.

After weeks of pressure from American diplomats, the government of
President Ashraf Ghani agreed to a
\href{https://www.nytimes.com/2020/03/23/world/asia/afghanistan-taliban-peace-coronavirus.html}{phased
release of 5,000 Taliban prisoners}. In an unprecedented move, a small
technical team of the insurgents arrived in Kabul for discussions with
Afghan officials over verification of identities before the release. But
those technical discussions now appear to have collapsed after a week as
each side accused the other of insincerity.

``Their release has been delayed under one pretext or another,'' said
Suhail Shaheen, a spokesman for the Taliban's negotiating team.
``Therefore, our technical team will not participate in fruitless
meetings.''

\includegraphics{https://static01.nyt.com/images/2020/04/07/world/07afghanistan2/merlin_171305406_46fcb443-9cd2-416b-9f32-6a8f82c6e404-articleLarge.jpg?quality=75\&auto=webp\&disable=upscale}

Even after the Taliban released a statement saying they were pulling out
of the talks, Afghan officials hoped another meeting between the
technical teams scheduled for late Tuesday would go ahead, but it did
not.

``Discussions on release of Afghan security forces and Taliban prisoners
had entered an important phase ahead of release,'' said Javid Faisal, a
spokesman for Afghanistan's National Security Council. ``Withdrawing
from talks at such a time indicates a lack of seriousness about peace.
We tried our best --- that they need to trust us, and we trust them as
we have to work together. We thought their arrival in Kabul was a big
step.''

The Afghan government has been working under pressure from the United
States, which
\href{https://www.nytimes.com/2020/03/23/world/asia/afghanistan-taliban-peace-coronavirus.html}{cut
\$1 billion in aid} over bickering among political leaders which the
Americans say has undermined what was a tightly choreographed deal.

On Tuesday, NBC News reported that on the same trip last month where
Secretary of State Mike Pompeo announced the aid cut out of frustration,
he delivered a second stern message to Afghan leaders: that if they did
not resolve their dispute to prioritize the peace talks, the United
States could pull out all American troops.

The State Department had no immediate comment on the NBC report.

Image

Secretary of State Mike Pompeo meeting with the Afghan president, Ashraf
Ghani, last month in Kabul, where Mr. Pompeo warned that Afghan
officials must prioritize the peace talks.Credit...EPA, via Shutterstock

The first signs that the talks over the prisoner swap had hit a wall
emerged on Monday, when Matin Bek, Mr. Ghani's head of local governance,
told a news conference the Taliban were demanding that the first batch
of releases include about 15 senior commanders convicted of major
attacks.

Several Afghan officials aware of the discussions said the government
had tried earnestly to use the opportunity of a Taliban delegation in
Kabul to make progress. The International Committee of the Red Cross,
experienced in prisoner swaps, was a third party in the talks to help
overcome mistrust. The Taliban have long said the Afghan government is
creating pretexts for not proceeding, while Afghan officials have said
the insurgents need to understand the releases are time-consuming,
requiring verification and logistical preparation.

The Taliban insisted that the senior commanders be released in the first
batch so they could help verify the rest of the 5,000 prisoners who had
been expected to be released. Afghan officials offered a workaround:
They could not release the senior commanders now, but those commanders
could participate in the verification process and then return to prison.

It was not immediately clear what issue or issues broke down the talks.

Advertisement

\protect\hyperlink{after-bottom}{Continue reading the main story}

\hypertarget{site-index}{%
\subsection{Site Index}\label{site-index}}

\hypertarget{site-information-navigation}{%
\subsection{Site Information
Navigation}\label{site-information-navigation}}

\begin{itemize}
\tightlist
\item
  \href{https://help.nytimes.com/hc/en-us/articles/115014792127-Copyright-notice}{©~2020~The
  New York Times Company}
\end{itemize}

\begin{itemize}
\tightlist
\item
  \href{https://www.nytco.com/}{NYTCo}
\item
  \href{https://help.nytimes.com/hc/en-us/articles/115015385887-Contact-Us}{Contact
  Us}
\item
  \href{https://www.nytco.com/careers/}{Work with us}
\item
  \href{https://nytmediakit.com/}{Advertise}
\item
  \href{http://www.tbrandstudio.com/}{T Brand Studio}
\item
  \href{https://www.nytimes.com/privacy/cookie-policy\#how-do-i-manage-trackers}{Your
  Ad Choices}
\item
  \href{https://www.nytimes.com/privacy}{Privacy}
\item
  \href{https://help.nytimes.com/hc/en-us/articles/115014893428-Terms-of-service}{Terms
  of Service}
\item
  \href{https://help.nytimes.com/hc/en-us/articles/115014893968-Terms-of-sale}{Terms
  of Sale}
\item
  \href{https://spiderbites.nytimes.com}{Site Map}
\item
  \href{https://help.nytimes.com/hc/en-us}{Help}
\item
  \href{https://www.nytimes.com/subscription?campaignId=37WXW}{Subscriptions}
\end{itemize}
