Sections

SEARCH

\protect\hyperlink{site-content}{Skip to
content}\protect\hyperlink{site-index}{Skip to site index}

\href{/section/nyregion}{New York}\textbar{}How Delays and Unheeded
Warnings Hindered New York's Virus Fight

\url{https://nyti.ms/2Xk7ByE}

\begin{itemize}
\item
\item
\item
\item
\item
\item
\end{itemize}

\href{https://www.nytimes.com/news-event/coronavirus?action=click\&pgtype=Article\&state=default\&region=TOP_BANNER\&context=storylines_menu}{The
Coronavirus Outbreak}

\begin{itemize}
\tightlist
\item
  live\href{https://www.nytimes.com/2020/08/04/world/coronavirus-cases.html?action=click\&pgtype=Article\&state=default\&region=TOP_BANNER\&context=storylines_menu}{Latest
  Updates}
\item
  \href{https://www.nytimes.com/interactive/2020/us/coronavirus-us-cases.html?action=click\&pgtype=Article\&state=default\&region=TOP_BANNER\&context=storylines_menu}{Maps
  and Cases}
\item
  \href{https://www.nytimes.com/interactive/2020/science/coronavirus-vaccine-tracker.html?action=click\&pgtype=Article\&state=default\&region=TOP_BANNER\&context=storylines_menu}{Vaccine
  Tracker}
\item
  \href{https://www.nytimes.com/2020/08/02/us/covid-college-reopening.html?action=click\&pgtype=Article\&state=default\&region=TOP_BANNER\&context=storylines_menu}{College
  Reopening}
\item
  \href{https://www.nytimes.com/live/2020/08/04/business/stock-market-today-coronavirus?action=click\&pgtype=Article\&state=default\&region=TOP_BANNER\&context=storylines_menu}{Economy}
\end{itemize}

\includegraphics{https://static01.nyt.com/images/2020/04/03/nyregion/00nyvirus-ticktock1/00nyvirus-ticktock1-articleLarge-v2.jpg?quality=75\&auto=webp\&disable=upscale}

\hypertarget{how-delays-and-unheeded-warnings-hindered-new-yorks-virus-fight}{%
\section{How Delays and Unheeded Warnings Hindered New York's Virus
Fight}\label{how-delays-and-unheeded-warnings-hindered-new-yorks-virus-fight}}

The federal response was chaotic. Even so, the state's and city's own
initial efforts failed to keep pace with the outbreak, The Times found.

Credit...Jonah Markowitz for The New York Times

Supported by

\protect\hyperlink{after-sponsor}{Continue reading the main story}

\href{https://www.nytimes.com/by/j-david-goodman}{\includegraphics{https://static01.nyt.com/images/2018/07/18/nyregion/author-j-david-goodman/author-j-david-goodman-thumbLarge.png}}

By \href{https://www.nytimes.com/by/j-david-goodman}{J. David Goodman}

\begin{itemize}
\item
  Published April 8, 2020Updated July 18, 2020
\item
  \begin{itemize}
  \item
  \item
  \item
  \item
  \item
  \item
  \end{itemize}
\end{itemize}

A 39-year-old woman took Flight 701 from Doha, Qatar, to John F. Kennedy
International Airport in late February, the final leg of her trip home
to New York City from Iran.

A week later, on March 1, she tested positive for the
\href{https://www.nytimes.com/2020/07/13/upshot/coronavirus-response-fax-machines.html}{coronavirus},
the first confirmed case in New York City of an outbreak that had
already devastated China and parts of Europe. The next day,
\href{https://www.nytimes.com/2020/04/09/style/cuomo-ny-tough-video-coronavirus.html}{Gov.
Andrew M. Cuomo}, appearing with Mayor Bill de Blasio at a news
conference, promised that health investigators would track down every
person on the woman's flight. But no one did.

A day later,
\href{https://www.nytimes.com/2020/03/03/nyregion/coronavirus-new-york-state.html}{a
lawyer from New Rochelle}, a New York City suburb, tested positive for
the virus --- an alarming sign because he had not traveled to any
affected country, suggesting community spread was already taking place.

Although city investigators had traced the lawyer's whereabouts and
connections to the most crowded corridors of Manhattan, the state's
efforts focused on the suburb, not the city, and Mr. de Blasio urged the
public not to worry. ``We'll tell you the second we think you should
change your behavior,''
\href{https://www.nytimes.com/2020/03/05/nyregion/coronavirus-new-york-cases.html}{the
mayor said on March 5}.

For many days after the first positive test, as the coronavirus silently
spread throughout the New York region,
\href{https://www.nytimes.com/2020/04/09/style/cuomo-ny-tough-video-coronavirus.html}{Mr.
Cuomo}, Mr. de Blasio and their top aides projected an unswerving
confidence that the outbreak would be readily contained.

There would be cases, they repeatedly said, but New York's hospitals
were some of the best in the world. Plans were in place.
\href{https://www.nytimes.com/2020/07/13/upshot/coronavirus-response-fax-machines.html}{Responses}
had been rehearsed during ``tabletop'' exercises. After all, the city
had been here before --- Ebola, Zika, the H1N1 virus, even Sept. 11.

``Excuse our arrogance as New Yorkers --- I speak for the mayor also on
this one --- we think we have the best health care system on the planet
right here in New York,''
\href{https://www.governor.ny.gov/news/video-audio-photos-rush-transcript-novel-coronavirus-briefing-governor-cuomo-announces-state}{Mr.
Cuomo said on March 2}. ``So, when you're saying, what happened in other
countries versus what happened here, we don't even think it's going to
be as bad as it was in other countries.''

\includegraphics{https://static01.nyt.com/images/2020/04/03/nyregion/00nyvirus-ticktock11/merlin_169898430_0bd836f4-2adb-4cde-bb17-6493371aab4a-articleLarge.jpg?quality=75\&auto=webp\&disable=upscale}

But now, New York City and the surrounding suburbs have become the
epicenter of the pandemic in the United States, with far more cases than
many countries have.
\href{https://www.nytimes.com/2020/04/07/nyregion/coronavirus-new-york-update.html}{More
than 138,000 people in the state have tested positive for the virus},
with nearly all of them in the city and nearby suburbs.

On Tuesday, Mr. Cuomo announced that 731 more people had died of the
virus, the state's highest one-day total yet. The overall death toll in
New York is 5,489 people.

Epidemiologists have pointed to
\href{https://www.nytimes.com/2020/03/23/nyregion/coronavirus-nyc-crowds-density.html}{New
York City's density} and its role as an international hub of commerce
and tourism to explain why the coronavirus has spread so rapidly. And it
seems highly unlikely that any response by the state or city could have
fully stopped the pandemic.

From the earliest days of the crisis, state and city officials were also
hampered by a
\href{https://www.nytimes.com/2020/07/18/us/politics/trump-coronavirus-response-failure-leadership.html}{chaotic
and often dysfunctional federal response}, including
\href{https://www.nytimes.com/2020/03/28/us/testing-coronavirus-pandemic.html}{significant
problems with the expansion of coronavirus testing}, which made it far
harder to gauge the scope of the outbreak.

Normally, New York would get help from Washington in such a time, as it
did after Sept. 11. But
\href{https://www.nytimes.com/2020/04/01/us/politics/coronavirus-trump.html}{President
Trump in February and early March} minimized the coronavirus threat,
clashing with his own medical experts and failing to marshal the might
of the federal government soon after cases emerged in the United States.

As a result, state and city officials often had to make decisions early
on without full assistance from the federal government.

Even so, the initial efforts by New York officials to stem the outbreak
were hampered by their own confused guidance, unheeded warnings, delayed
decisions and political infighting, The New York Times found.

``Flu was coming down, and then you saw this new ominous spike. And it
was Covid. And it was spreading widely in New York City before anyone
knew it,'' said Dr. Thomas R. Frieden, the former head of the Centers
for Disease Control and Prevention and former commissioner of the city's
Health Department. ``You have to move really fast. Hours and days. Not
weeks. Once it gets a head of steam, there is no way to stop it.''

Dr. Frieden said that if the
\href{https://twitter.com/DrTomFrieden/status/1247184873615433729}{state
and city had adopted widespread social-distancing measures a week or two
earlier}, including closing schools, stores and restaurants, then the
estimated death toll from the outbreak might have been reduced by 50 to
80 percent.

But New York mandated those measures after localities in states
including California and Washington had done so.

San Francisco, for example, ordered schools closed on
\href{https://www.kron4.com/health/coronavirus/san-francisco-reports-4-new-coronavirus-cases-bringing-total-to-18/}{March
12 when that city had 18 confirmed cases}; Ohio also ordered its schools
\href{https://www.usnews.com/news/education-news/articles/2020-03-12/ohio-gov-mike-dewine-orders-all-k-12-schools-closed}{closed}
on the same day, with five confirmed cases. Mr. de Blasio ordered
schools in New York to close three days later
\href{https://www.governor.ny.gov/news/governor-cuomo-calls-president-trump-take-comprehensive-federal-action-combat-novel-coronavirus}{when
the city had 329 cases}.

Then seven Bay Area counties
\href{https://www.nytimes.com/2020/03/16/us/california-covid-19.html}{imposed
stay-at-home rules on March 17}. Two days later, the entire
\href{https://www.nytimes.com/2020/03/19/us/California-stay-at-home-order-virus.html}{state
of California ordered the same}. New York State's stay-at-home order
came on the 20th,
\href{https://www.nytimes.com/2020/03/20/us/ny-ca-stay-home-order.html}{and
went into effect on March 22}.

\hypertarget{latest-updates-global-coronavirus-outbreak}{%
\section{\texorpdfstring{\href{https://www.nytimes.com/2020/08/04/world/coronavirus-cases.html?action=click\&pgtype=Article\&state=default\&region=MAIN_CONTENT_1\&context=storylines_live_updates}{Latest
Updates: Global Coronavirus
Outbreak}}{Latest Updates: Global Coronavirus Outbreak}}\label{latest-updates-global-coronavirus-outbreak}}

Updated 2020-08-04T20:42:41.838Z

\begin{itemize}
\tightlist
\item
  \href{https://www.nytimes.com/2020/08/04/world/coronavirus-cases.html?action=click\&pgtype=Article\&state=default\&region=MAIN_CONTENT_1\&context=storylines_live_updates\#link-1228a480}{Novavax
  sees encouraging results from two studies of its experimental
  vaccine.}
\item
  \href{https://www.nytimes.com/2020/08/04/world/coronavirus-cases.html?action=click\&pgtype=Article\&state=default\&region=MAIN_CONTENT_1\&context=storylines_live_updates\#link-4825b93}{Public
  and private schools in Maryland and elsewhere are divided over
  in-person instruction.}
\item
  \href{https://www.nytimes.com/2020/08/04/world/coronavirus-cases.html?action=click\&pgtype=Article\&state=default\&region=MAIN_CONTENT_1\&context=storylines_live_updates\#link-50f7386d}{The
  United Nations calls on policymakers to `plan thoroughly for school
  reopenings.'}
\end{itemize}

\href{https://www.nytimes.com/2020/08/04/world/coronavirus-cases.html?action=click\&pgtype=Article\&state=default\&region=MAIN_CONTENT_1\&context=storylines_live_updates}{See
more updates}

More live coverage:
\href{https://www.nytimes.com/live/2020/08/04/business/stock-market-today-coronavirus?action=click\&pgtype=Article\&state=default\&region=MAIN_CONTENT_1\&context=storylines_live_updates}{Markets}

``New York City as a whole was late in social measures,'' said Isaac B.
Weisfuse, a
\href{https://www.nytimes.com/2008/06/18/nyregion/18flu.html}{former New
York City deputy health commissioner}. ``Any after-action review of the
pandemic in New York City will focus on that issue. It has become the
major issue in the transmission of the virus.''

Interviews with more than 60 people on the front-lines of the unfolding
crisis --- city and state officials, hospital executives, health care
workers, union leaders and emergency medical workers --- revealed how
the virus overwhelmed the city's longstanding preparations, leaving
officials to improvise. Many spoke on the record; others spoke
anonymously to describe private meetings and conversations without fear
of losing their jobs.

``Everything was slow,'' said Councilman Stephen T. Levin, a Brooklyn
Democrat who had called for City Hall to take swifter action as the
outbreak spread. ``You have to adapt really quickly, and nothing we were
doing was adapting quickly.''

Both Mr. Cuomo and Mr. de Blasio have focused intensely in recent days
on vastly expanding the ability of the health care system to treat
coronavirus patients
\href{https://www.nytimes.com/2020/04/06/nyregion/coronavirus-new-york-peak.html}{as
the outbreak nears its peak}. The state and city have set up new
hospital wards, scoured the world for ventilators and protective gear
and aggressively recruited doctors and nurses around the country.

Mr. Cuomo has been praised for his informative daily news conferences,
where he not only focuses on the facts of the pandemic but also seeks to
rally the public's support for efforts to curb the spread. Mr. de Blasio
has also made outreach a priority.

Still, Mr. Cuomo has at times acknowledged the difficulties in fighting
the outbreak.

``I am tired of being behind this virus,''
\href{https://www.governor.ny.gov/news/video-audio-photos-rush-transcript-amid-ongoing-covid-19-pandemic-governor-cuomo-announces-new}{he
said on March 31}. ``We've been playing catch-up. You don't win playing
catch-up.''

The governor's aides said he was referring to the country as a whole,
not New York.

The governor and the mayor emphasized that they had no misgivings about
their initial handling of their response. They said that their efforts
spurred the Trump administration to act more decisively to curb the
outbreak. New York was the first state to obtain federal approval for
its own coronavirus testing.

``Every action I took was criticized at the time as premature,'' Mr.
Cuomo said in an interview. ``The facts have proven my decisions
correct.''

Mr. de Blasio said in a statement, ``We're dealing with a virus that's
only months old and science that changes by the day,'' adding that
``hindsight is a luxury none of us have in the heat of battle.''

\hypertarget{messages-of-confidence-and-confusion}{%
\subsection{Messages of confidence and
confusion}\label{messages-of-confidence-and-confusion}}

Image

The mayor and his aides worried about how a shutdown would affect the
poorest and most vulnerable New Yorkers. Credit...Dave Sanders for The
New York Times

From the start, Mr. de Blasio and Mr. Cuomo projected as much concern
about panic as they did about the virus.

``We can really keep this thing contained,''
\href{https://www.nytimes.com/2020/02/27/nyregion/new-york-coronavirus.html}{Mr.
de Blasio said at a news conference} about virus preparations in late
February.

That tone continued even after the first positive case was announced on
March 1.

``Everybody is doing exactly what we need to do,'' said Mr. Cuomo,
seated with Mr. de Blasio, at a news conference on March 2. ``We have
been ahead of this from Day 1.''

Hospitals also expressed confidence in their plans for responding to a
pandemic, with the Healthcare Association of New York State declaring on
March 2 that its members were
``\href{https://www.hanys.org/communications/pr/2020/2020-03-02_coronavirus.cfm}{prepared
for an influx of patients caused by}Covid-19.''

But few, if any, appeared to have made significant efforts before the
virus hit to greatly increase supplies of ventilators or protective
gear,
\href{https://www.gnyha.org/wp-content/uploads/2020/03/3_GNYHA-Statement-Hospital-Preparedness-for-Coronavirus.pdf}{looking
instead to draw on emergency government stockpiles}.

Officials seemed to speak and act based on the assumption that the virus
had not arrived in the state until that first case --- the woman
traveling from Iran. State and local officials now acknowledge that the
virus was almost certainly in New York much earlier.

Infectious disease specialists had known for weeks before any positive
test had occurred that many of the early cases would be missed because
of significant flaws in federal testing.

Bruce Farber, the chief of infectious diseases for two hospitals within
Northwell Health, the largest hospital system in New York, said that by
late January, it was apparent that cases would soon begin appearing in
the United States. He said he and his colleagues realized, as they
reviewed the strictly limited federal testing criteria during a Feb. 7
meeting, that many infected people would not be identified.

Only those with a fever severe enough to require hospitalization and who
had traveled to China in the previous 14 days could get tested, Dr.
Farber told them, reading from the C.D.C. guidelines.

``It was that moment that I think everybody in the room realized, we're
dead,'' Dr. Farber said.

For both city and state, the initial plan was to trace, isolate and
contain each case. Mr. Cuomo promised that they would go further than
necessary to find every connection to the woman who arrived from Iran.

``Out of an abundance of caution we will be contacting the people who
were on the flight with her from Iran to New York,'' he said.

But no one ever did that work. Local officials could only request an
investigation from the C.D.C., and the agency did not perform one
because they believed at the time she had not been contagious during the
flight, officials said. Neither Mr. Cuomo nor Mr. de Blasio publicly
mentioned finding the plane passengers again.

That's because new cases in the area kept emerging: the lawyer in New
Rochelle who worked in Manhattan but had no connection to the first case
and had not traveled to countries affected by the virus. Then two more
people in New York City tested positive, also unconnected to the
affected countries and, more ominously, to each other.

Image

In mid-March, Mr. Cuomo ordered a ``containment area'' for New Rochelle,
where a cluster had emerged. Credit...Andrew Seng for The New York Times

New York City, at the start of the outbreak, relied on 50 disease
detectives to trace the rapidly rising cases of unconnected infected
people, city officials said.

By comparison, in Wuhan, China, where the pandemic began, more than
9,000 such workers were deployed. New York City added to its original 50
only after the outbreak began to accelerate.

Because of the limits on testing, said the mayor's press secretary,
Freddi Goldstein, ``all the detectives in the world would have been
useless.''

By March 5, Mr. de Blasio seemed to acknowledge the virus had spread
beyond control. ``You have to assume it could be anywhere in the city,''
he said.

Still, not wanting to cause undue alarm, he told New Yorkers to go on
with their normal lives, which left many confused about the danger they
faced.

The city's health commissioner, Dr. Oxiris Barbot, had sought to
reassure commuters, in early February, that ``this is not something that
you're going to contract in the subway or on the bus.'' The mayor
reiterated the point several times in early March.

But there seemed to be little basis for that confidence.

The C.D.C. in early February said it was ``unclear'' if the virus could
be transferred on surfaces and, by March, said that it might
``\href{https://www.cdc.gov/coronavirus/2019-ncov/prevent-getting-sick/how-covid-spreads.html?CDC_AA_refVal=https\%3A\%2F\%2Fwww.cdc.gov\%2Fcoronavirus\%2F2019-ncov\%2Fprepare\%2Ftransmission.html}{be
possible}'' for someone to get infected by touching a contaminated
surface and then touching their face. The virus mainly spreads between
people in close contact, the agency has said, such as occurs on a
crowded subway.

State and city officials blamed the confusing messages on shifting
guidance from the federal government.

But by the second week in March, as the virus continued to spread
widely, Mr. de Blasio also clashed over messaging with his own Health
Department.

The mayor wanted widespread testing, but senior Health Department
officials believed it was a waste of limited resources. They urged
instead a public awareness campaign to tell people with mild symptoms to
stay home and not infect others, or themselves, by going to testing
centers.

City Hall blocked the department from releasing that message to the
public, until the mayor eventually backed down in the third week in
March.

\hypertarget{the-coronavirus-moves-faster-than-the-response}{%
\subsection{The coronavirus moves faster than the
response}\label{the-coronavirus-moves-faster-than-the-response}}

Image

On March 15, Mr. de Blasio finally relented and agreed to close public
schools.~Credit...Sarah Blesener for The New York Times

New York City's system for detecting infectious diseases was flashing
danger.

While only about 100 cases of the coronavirus had been confirmed in the
whole state,
\href{https://a816-health.nyc.gov/hdi/epiquery/disease-reporting}{the
city's surveillance system} was, by the end of the first week in March,
signaling a spike in influenza-like illnesses at emergency rooms. A few
days later, the number of police officers calling out sick jumped
noticeably, as did calls to 911 for fever and cough.

The governor and the mayor began taking limited steps to restrict
people's activities, but even those were met with resistance.

Locals complained when the governor ordered a porous ``containment
area'' for New Rochelle, where a cluster had emerged. It meant closing
schools and gathering places in a one-mile radius of a synagogue at the
center of the outbreak, while allowing movement in and out.

Each day brought some new action.

The governor declared a state of emergency, worked to expand testing
capacity and, later, secured the construction of field hospitals. The
mayor and the governor encouraged work-from-home. They restricted large
gatherings to 500 people, and reduced by half the occupancy for
restaurants and bars. Broadway closed. So did most other big
entertainment venues.

\href{https://www.nytimes.com/news-event/coronavirus?action=click\&pgtype=Article\&state=default\&region=MAIN_CONTENT_3\&context=storylines_faq}{}

\hypertarget{the-coronavirus-outbreak-}{%
\subsubsection{The Coronavirus Outbreak
›}\label{the-coronavirus-outbreak-}}

\hypertarget{frequently-asked-questions}{%
\paragraph{Frequently Asked
Questions}\label{frequently-asked-questions}}

Updated August 4, 2020

\begin{itemize}
\item ~
  \hypertarget{i-have-antibodies-am-i-now-immune}{%
  \paragraph{I have antibodies. Am I now
  immune?}\label{i-have-antibodies-am-i-now-immune}}

  \begin{itemize}
  \tightlist
  \item
    As of right
    now,\href{https://www.nytimes.com/2020/07/22/health/covid-antibodies-herd-immunity.html?action=click\&pgtype=Article\&state=default\&region=MAIN_CONTENT_3\&context=storylines_faq}{that
    seems likely, for at least several months.} There have been
    frightening accounts of people suffering what seems to be a second
    bout of Covid-19. But experts say these patients may have a
    drawn-out course of infection, with the virus taking a slow toll
    weeks to months after initial exposure. People infected with the
    coronavirus typically
    \href{https://www.nature.com/articles/s41586-020-2456-9}{produce}
    immune molecules called antibodies, which are
    \href{https://www.nytimes.com/2020/05/07/health/coronavirus-antibody-prevalence.html?action=click\&pgtype=Article\&state=default\&region=MAIN_CONTENT_3\&context=storylines_faq}{protective
    proteins made in response to an
    infection}\href{https://www.nytimes.com/2020/05/07/health/coronavirus-antibody-prevalence.html?action=click\&pgtype=Article\&state=default\&region=MAIN_CONTENT_3\&context=storylines_faq}{.
    These antibodies may} last in the body
    \href{https://www.nature.com/articles/s41591-020-0965-6}{only two to
    three months}, which may seem worrisome, but that's perfectly normal
    after an acute infection subsides, said Dr. Michael Mina, an
    immunologist at Harvard University. It may be possible to get the
    coronavirus again, but it's highly unlikely that it would be
    possible in a short window of time from initial infection or make
    people sicker the second time.
  \end{itemize}
\item ~
  \hypertarget{im-a-small-business-owner-can-i-get-relief}{%
  \paragraph{I'm a small-business owner. Can I get
  relief?}\label{im-a-small-business-owner-can-i-get-relief}}

  \begin{itemize}
  \tightlist
  \item
    The
    \href{https://www.nytimes.com/article/small-business-loans-stimulus-grants-freelancers-coronavirus.html?action=click\&pgtype=Article\&state=default\&region=MAIN_CONTENT_3\&context=storylines_faq}{stimulus
    bills enacted in March} offer help for the millions of American
    small businesses. Those eligible for aid are businesses and
    nonprofit organizations with fewer than 500 workers, including sole
    proprietorships, independent contractors and freelancers. Some
    larger companies in some industries are also eligible. The help
    being offered, which is being managed by the Small Business
    Administration, includes the Paycheck Protection Program and the
    Economic Injury Disaster Loan program. But lots of folks have
    \href{https://www.nytimes.com/interactive/2020/05/07/business/small-business-loans-coronavirus.html?action=click\&pgtype=Article\&state=default\&region=MAIN_CONTENT_3\&context=storylines_faq}{not
    yet seen payouts.} Even those who have received help are confused:
    The rules are draconian, and some are stuck sitting on
    \href{https://www.nytimes.com/2020/05/02/business/economy/loans-coronavirus-small-business.html?action=click\&pgtype=Article\&state=default\&region=MAIN_CONTENT_3\&context=storylines_faq}{money
    they don't know how to use.} Many small-business owners are getting
    less than they expected or
    \href{https://www.nytimes.com/2020/06/10/business/Small-business-loans-ppp.html?action=click\&pgtype=Article\&state=default\&region=MAIN_CONTENT_3\&context=storylines_faq}{not
    hearing anything at all.}
  \end{itemize}
\item ~
  \hypertarget{what-are-my-rights-if-i-am-worried-about-going-back-to-work}{%
  \paragraph{What are my rights if I am worried about going back to
  work?}\label{what-are-my-rights-if-i-am-worried-about-going-back-to-work}}

  \begin{itemize}
  \tightlist
  \item
    Employers have to provide
    \href{https://www.osha.gov/SLTC/covid-19/standards.html}{a safe
    workplace} with policies that protect everyone equally.
    \href{https://www.nytimes.com/article/coronavirus-money-unemployment.html?action=click\&pgtype=Article\&state=default\&region=MAIN_CONTENT_3\&context=storylines_faq}{And
    if one of your co-workers tests positive for the coronavirus, the
    C.D.C.} has said that
    \href{https://www.cdc.gov/coronavirus/2019-ncov/community/guidance-business-response.html}{employers
    should tell their employees} -\/- without giving you the sick
    employee's name -\/- that they may have been exposed to the virus.
  \end{itemize}
\item ~
  \hypertarget{should-i-refinance-my-mortgage}{%
  \paragraph{Should I refinance my
  mortgage?}\label{should-i-refinance-my-mortgage}}

  \begin{itemize}
  \tightlist
  \item
    \href{https://www.nytimes.com/article/coronavirus-money-unemployment.html?action=click\&pgtype=Article\&state=default\&region=MAIN_CONTENT_3\&context=storylines_faq}{It
    could be a good idea,} because mortgage rates have
    \href{https://www.nytimes.com/2020/07/16/business/mortgage-rates-below-3-percent.html?action=click\&pgtype=Article\&state=default\&region=MAIN_CONTENT_3\&context=storylines_faq}{never
    been lower.} Refinancing requests have pushed mortgage applications
    to some of the highest levels since 2008, so be prepared to get in
    line. But defaults are also up, so if you're thinking about buying a
    home, be aware that some lenders have tightened their standards.
  \end{itemize}
\item ~
  \hypertarget{what-is-school-going-to-look-like-in-september}{%
  \paragraph{What is school going to look like in
  September?}\label{what-is-school-going-to-look-like-in-september}}

  \begin{itemize}
  \tightlist
  \item
    It is unlikely that many schools will return to a normal schedule
    this fall, requiring the grind of
    \href{https://www.nytimes.com/2020/06/05/us/coronavirus-education-lost-learning.html?action=click\&pgtype=Article\&state=default\&region=MAIN_CONTENT_3\&context=storylines_faq}{online
    learning},
    \href{https://www.nytimes.com/2020/05/29/us/coronavirus-child-care-centers.html?action=click\&pgtype=Article\&state=default\&region=MAIN_CONTENT_3\&context=storylines_faq}{makeshift
    child care} and
    \href{https://www.nytimes.com/2020/06/03/business/economy/coronavirus-working-women.html?action=click\&pgtype=Article\&state=default\&region=MAIN_CONTENT_3\&context=storylines_faq}{stunted
    workdays} to continue. California's two largest public school
    districts --- Los Angeles and San Diego --- said on July 13, that
    \href{https://www.nytimes.com/2020/07/13/us/lausd-san-diego-school-reopening.html?action=click\&pgtype=Article\&state=default\&region=MAIN_CONTENT_3\&context=storylines_faq}{instruction
    will be remote-only in the fall}, citing concerns that surging
    coronavirus infections in their areas pose too dire a risk for
    students and teachers. Together, the two districts enroll some
    825,000 students. They are the largest in the country so far to
    abandon plans for even a partial physical return to classrooms when
    they reopen in August. For other districts, the solution won't be an
    all-or-nothing approach.
    \href{https://bioethics.jhu.edu/research-and-outreach/projects/eschool-initiative/school-policy-tracker/}{Many
    systems}, including the nation's largest, New York City, are
    devising
    \href{https://www.nytimes.com/2020/06/26/us/coronavirus-schools-reopen-fall.html?action=click\&pgtype=Article\&state=default\&region=MAIN_CONTENT_3\&context=storylines_faq}{hybrid
    plans} that involve spending some days in classrooms and other days
    online. There's no national policy on this yet, so check with your
    municipal school system regularly to see what is happening in your
    community.
  \end{itemize}
\end{itemize}

Still some people flouted the rules, continuing to gather in public.

But the biggest and most prolonged battle centered on closing the city's
school system, with its 1.1 million students. Doing so would amount to a
virtual shutdown of the city.

Image

Credit...Andrew Seng for The New York Times

Image

Credit...James Estrin/The New York Times

Image

The cluster in New Rochelle eventually led to a drive-through testing
area, and the presence of the National Guard.Credit...Andrew Seng for
The New York Times

The mayor and his aides worried about the effect on the poorest and most
vulnerable New Yorkers. For Mr. de Blasio, whose progressive political
identity has been defined by his attention to the city's have-nots, the
crisis presented a stark and unwelcome choice to harm some New Yorkers
in order to save others.

``If you suddenly in one day close down the schools, how do you make
sure that you are providing for these kids and their parents?'' said
Emma Wolfe, a top aide to Mr. de Blasio. ``We're not in the suburbs. We
can't tell people to stay at home and play around in your yard.''

Behind the scenes, top health officials were growing increasingly
alarmed. Demetre Daskalakis, the city's head of disease control,
threatened to quit if the schools were not closed, a city official said.
City Hall said no such threat reached the mayor's attention.

And Dr. Barbot --- who at the start of the outbreak had insisted that
``New Yorkers remain at low risk'' --- gave a far scarier assessment to
a closed-door meeting of business executives in City Hall on March 12:
Up to 70 percent of city residents could become infected.

The time for containment was over, she added. Mr. de Blasio, seated
beside her at the meeting, stared daggers as she spoke.

``Why don't you shut down restaurants now?'' a chief executive who
attended the meeting recalled someone asking the mayor.

``I'm really concerned about restaurateurs; I'm really concerned about
jobs,'' the mayor responded, the executive recalled. Mr. de Blasio had
urged New Yorkers to start social distancing and work from home where
possible.

The following weekend, even though Mr. de Blasio and Mr. Cuomo had
ordered occupancy limits for restaurants and bars, much of the city's
nightlife appeared to continue apace.

\hypertarget{another-round-of-governor-vs-mayor}{%
\subsection{Another round of governor vs.
mayor}\label{another-round-of-governor-vs-mayor}}

Image

Mr. Cuomo has been praised for his daily news conferences, where he
focuses on the facts of the outbreak.Credit...Cindy Schultz for The New
York Times

State and city officials believed they were doing everything possible to
confront the outbreak, moving from big decision to big decision so
quickly that each day, they said, felt like a year. They blamed the
spread in New York on the federal government, which they say dragged its
feet on testing. For weeks, Mr. Trump
\href{https://www.nytimes.com/video/us/politics/100000007067717/trumps-coronavirus-statements.html}{brushed
aside concerns} that the outbreak would damage the country.

``We have it totally under control,'' Mr. Trump said in late January. A
month later, he advised Americans to ``view this the same as the flu.''

But local officials did have control over closing schools and
businesses. While they waited on making a decision, other major cities
were moving toward shutdowns.

In California, Los Angeles followed San Francisco's lead and ordered its
schools closed on March 13, after 40 cases of the virus had been
confirmed. On that same day, there were nearly four times as many
confirmed cases in New York, but City Hall did not yet support closing
schools.

And even as aides to the mayor and governor, both Democrats, worked
closely together on the response, old rivalries crept in. Though the two
leaders put up a unified front at the outset of the outbreak, it was
clear by the middle of March that a high-stakes version of their
longstanding political battles was playing out. The March 2 news
conference has been their only appearance together.

First, Mr. Cuomo sought to force the mayor's hand on the schools, state
officials said.

In a series of calls during the second weekend in March, the governor
worked with the Greater New York Hospital Association, a powerful
hospital lobby, to address the question of child care for health care
workers in the event that schools closed. That had been a sticking point
for those workers' union, 1199 SEIU, which has deep ties to City Hall.
The hospital association volunteered to raise funds.

The union, which had questioned the need to close schools on that
Friday, was by Sunday calling for closure.

That Sunday morning, March 15, Health Department officials gave Mr. de
Blasio some chilling forecasts of the number of possible dead if more
restrictions were not imposed. Closing schools was necessary, and most
businesses too. By then, the city had a plan to create
\href{https://www.schools.nyc.gov/enrollment/enrollment-help/regional-enrichment-centers}{centers
for the children of health care workers}, as well as emergency medical
workers.

Finally, Mr. de Blasio was persuaded.

As the city prepared an announcement to close the schools, Mr. Cuomo
announced the shutdown during a television appearance. Mr. de Blasio
made it official that evening, and then announced restaurants and bars
would be closed for everything but takeout and delivery.

After the decision on schools, the mayor became more assertive in
suggesting major changes in daily life.

Image

New York didn't shut down fully until March 22.Credit...Bryan Derballa
for The New York Times

Image

Credit...Jordan Gale for The New York Times

Image

Credit...Gabriela Bhaskar for The New York Times

New Yorkers would probably soon have to be kept at home for all but the
most necessary needs, he said on March 17 --- a ``shelter-in-place''
order similar to what had already been implemented in the Bay Area of
California.

This time, Mr. Cuomo was the one who resisted. He favored a more gradual
shutdown.

``I'm as afraid of the fear and the panic as I am of the virus, and I
think that the fear is more contagious than the virus right now,'' the
governor said when asked two days later about the mayor's comments.

He chastised the mayor for a poor communication strategy.

But then California moved first: Gov. Gavin Newsom
\href{https://www.nytimes.com/2020/03/19/us/California-stay-at-home-order-virus.html}{issued
a statewide order} for residents to stay at home. The state had 675
confirmed cases of the virus.

That same day, March 19, New York had more than 4,152.

That night, roughly 20 prominent New York leaders --- including local
members of Congress, two borough presidents, City Council members and
civic and religious figures --- joined a conference call convened by the
state attorney general, Letitia James.

``I was growing very frustrated over the schism between the mayor and
the governor,'' said one person on the call, who captured the sentiment.
After the call, a participant conveyed those feelings to the governor's
office.

Melissa DeRosa, the governor's top aide, said Mr. Cuomo decided on his
plan to ``pause'' New York during an afternoon meeting with his health
commissioner, before the call or Mr. Newsom's order.

The governor had been reviewing disturbing projections about the spread
of the virus since 4:30 a.m., she said.

``OK, let's shut it down,'' she recalled the governor saying. He
announced it the next day.

By that point, March 20, the state had more than 7,000 confirmed cases.

\hypertarget{the-city-reels}{%
\subsection{The city reels}\label{the-city-reels}}

Image

New York City remains virtually shut down as a result of social
distancing and quarantine measures.~Credit...John Taggart for The New
York Times

Leaders of the New York Police Department now start each day with a
review of how many of its 36,000 uniformed officers are sick. By early
April, it was around 19 percent.

``It's been a struggle,'' said the police commissioner, Dermot F. Shea.
Only the fact of a shutdown city has helped.

No parades or protests. No calls to schools. Fewer calls for police in
general. The calls to 911, instead, are mostly for ambulances. First it
was 5,000 a day. Then 6,000. A record almost daily.

That New York City, mammoth and interconnected, would be hit hard by the
pandemic may have been inevitable. But Washington, and then New York,
had options. The state's leaders had power over key decisions.

Their aides argue they moved as fast as possible given the flawed
information they had from the federal government and in the midst of a
fast-moving crisis on a scale none had seen before.

``This is an enemy that we have underestimated from Day 1,'' Mr. Cuomo
said on Monday. ``And we have paid the price dearly.''

Jesse Drucker, Luis Ferré Sadurní, Joseph Goldstein, Jeffery C. Mays,
Jesse McKinley, Brian M. Rosenthal and Michael Rothfeld contributed
reporting.

Advertisement

\protect\hyperlink{after-bottom}{Continue reading the main story}

\hypertarget{site-index}{%
\subsection{Site Index}\label{site-index}}

\hypertarget{site-information-navigation}{%
\subsection{Site Information
Navigation}\label{site-information-navigation}}

\begin{itemize}
\tightlist
\item
  \href{https://help.nytimes.com/hc/en-us/articles/115014792127-Copyright-notice}{©~2020~The
  New York Times Company}
\end{itemize}

\begin{itemize}
\tightlist
\item
  \href{https://www.nytco.com/}{NYTCo}
\item
  \href{https://help.nytimes.com/hc/en-us/articles/115015385887-Contact-Us}{Contact
  Us}
\item
  \href{https://www.nytco.com/careers/}{Work with us}
\item
  \href{https://nytmediakit.com/}{Advertise}
\item
  \href{http://www.tbrandstudio.com/}{T Brand Studio}
\item
  \href{https://www.nytimes.com/privacy/cookie-policy\#how-do-i-manage-trackers}{Your
  Ad Choices}
\item
  \href{https://www.nytimes.com/privacy}{Privacy}
\item
  \href{https://help.nytimes.com/hc/en-us/articles/115014893428-Terms-of-service}{Terms
  of Service}
\item
  \href{https://help.nytimes.com/hc/en-us/articles/115014893968-Terms-of-sale}{Terms
  of Sale}
\item
  \href{https://spiderbites.nytimes.com}{Site Map}
\item
  \href{https://help.nytimes.com/hc/en-us}{Help}
\item
  \href{https://www.nytimes.com/subscription?campaignId=37WXW}{Subscriptions}
\end{itemize}
