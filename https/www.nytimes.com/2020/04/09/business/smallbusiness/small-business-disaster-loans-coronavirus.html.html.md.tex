Sections

SEARCH

\protect\hyperlink{site-content}{Skip to
content}\protect\hyperlink{site-index}{Skip to site index}

\href{https://www.nytimes.com/section/business/smallbusiness}{Entrepreneurship}

\href{https://myaccount.nytimes.com/auth/login?response_type=cookie\&client_id=vi}{}

\href{https://www.nytimes.com/section/todayspaper}{Today's Paper}

\href{/section/business/smallbusiness}{Entrepreneurship}\textbar{}Small
Businesses Wait for Cash as Disaster Loan Program Unravels

\url{https://nyti.ms/2yNyuRu}

\begin{itemize}
\item
\item
\item
\item
\item
\item
\end{itemize}

\href{https://www.nytimes.com/news-event/coronavirus?action=click\&pgtype=Article\&state=default\&region=TOP_BANNER\&context=storylines_menu}{The
Coronavirus Outbreak}

\begin{itemize}
\tightlist
\item
  live\href{https://www.nytimes.com/2020/08/03/world/coronavirus-covid-19.html?action=click\&pgtype=Article\&state=default\&region=TOP_BANNER\&context=storylines_menu}{Latest
  Updates}
\item
  \href{https://www.nytimes.com/interactive/2020/us/coronavirus-us-cases.html?action=click\&pgtype=Article\&state=default\&region=TOP_BANNER\&context=storylines_menu}{Maps
  and Cases}
\item
  \href{https://www.nytimes.com/interactive/2020/science/coronavirus-vaccine-tracker.html?action=click\&pgtype=Article\&state=default\&region=TOP_BANNER\&context=storylines_menu}{Vaccine
  Tracker}
\item
  \href{https://www.nytimes.com/2020/08/02/us/covid-college-reopening.html?action=click\&pgtype=Article\&state=default\&region=TOP_BANNER\&context=storylines_menu}{College
  Reopening}
\item
  \href{https://www.nytimes.com/live/2020/08/03/business/stock-market-today-coronavirus?action=click\&pgtype=Article\&state=default\&region=TOP_BANNER\&context=storylines_menu}{Economy}
\end{itemize}

Advertisement

\protect\hyperlink{after-top}{Continue reading the main story}

Supported by

\protect\hyperlink{after-sponsor}{Continue reading the main story}

\hypertarget{small-businesses-wait-for-cash-as-disaster-loan-program-unravels}{%
\section{Small Businesses Wait for Cash as Disaster Loan Program
Unravels}\label{small-businesses-wait-for-cash-as-disaster-loan-program-unravels}}

Owners were supposed to be able to get up to \$2 million. Now they're
being told the cap is \$15,000 --- if they can get any answers at all.

\includegraphics{https://static01.nyt.com/images/2020/04/10/business/08virus-sbadiaster1-print/merlin_171411543_de2aaacf-7e21-4108-a1ec-260d2d9fc50d-articleLarge.jpg?quality=75\&auto=webp\&disable=upscale}

\href{https://www.nytimes.com/by/stacy-cowley}{\includegraphics{https://static01.nyt.com/images/2018/10/03/multimedia/author-stacy-cowley/author-stacy-cowley-thumbLarge.png}}

By \href{https://www.nytimes.com/by/stacy-cowley}{Stacy Cowley}

\begin{itemize}
\item
  Published April 9, 2020Updated Aug. 3, 2020
\item
  \begin{itemize}
  \item
  \item
  \item
  \item
  \item
  \item
  \end{itemize}
\end{itemize}

Flooded by requests for help like never before, a
\href{https://www.nytimes.com/2020/08/03/business/small-business-loans-coronavirus.html}{federal
disaster loan program} that was supposed to deliver emergency relief to
\href{https://www.nytimes.com/2020/04/20/business/shake-shack-returning-loan-ppp-coronavirus.html}{small
businesses} in just three days has run low on funding and nearly frozen
up entirely. Now, business owners who applied are desperate for cash and
answers about what aid, if any, they are going to receive.

The initiative, known as the
\href{https://www.sba.gov/funding-programs/loans/coronavirus-relief-options/economic-injury-disaster-loan-emergency-advance}{Economic
Injury Disaster Loan} program, is an expansion of an emergency system
run by the Small Business Administration that has for years helped
companies after natural disasters like hurricanes, floods and tornadoes.
To speed billions of dollars in aid along, the government directly funds
the loans, sparing applicants the step of finding a lender willing to
work with them.

But in the face of the pandemic, the
\href{https://www.nytimes.com/2020/04/20/business/shake-shack-returning-loan-ppp-coronavirus.html}{loan
program} is drowning in requests. Many applicants have waited weeks for
approval, with little to no information about where they stand, and
others are being told they'll get a fraction of what they expected.

The program is supposed to offer loans of up to \$2 million, but many
recent applicants said the S.B.A. help line had told them that loans
would be capped at \$15,000 per borrower. That was backed up by
\href{https://int.nyt.com/data/documenthelper/6871-sba-note-about-15000-cap/optimized/full.pdf}{a
message from the agency} that one applicant shared with The New York
Times.

The CARES Act, the \$2 trillion relief bill signed by President Trump
last month, also authorized the S.B.A. to hand out the first \$10,000 as
a grant that didn't have to be paid back. Those funds were supposed to
be available to applicants within three days of their application, even
if they weren't approved for a loan. That hasn't happened, according to
more than 400 applicants who contacted The Times.

S.B.A. officials did not respond to repeated requests for comment.

``I'm afraid I won't see a penny,'' said Virginia Warnken Kelsey, an
opera singer in Branford, Conn., who applied on March 29 and had not
received a response as of Thursday.

Ms. Kelsey had a busy spring season planned, with a tour scheduled to
stop in Belgium and the Netherlands and performances with orchestras in
Oregon and North Carolina. Everything has been canceled. The section of
her website where she posts her engagements simply reads: ``No upcoming
events.'' For her, the loan would be a lifeline of cash to cover her
rent and other bills.

\hypertarget{latest-updates-economy}{%
\section{\texorpdfstring{\href{https://www.nytimes.com/live/2020/08/03/business/stock-market-today-coronavirus?action=click\&pgtype=Article\&state=default\&region=MAIN_CONTENT_1\&context=storylines_live_updates}{Latest
Updates:
Economy}}{Latest Updates: Economy}}\label{latest-updates-economy}}

\href{https://www.nytimes.com/live/2020/08/03/business/stock-market-today-coronavirus?action=click\&pgtype=Article\&state=default\&region=MAIN_CONTENT_1\&context=storylines_live_updates\#the-chicago-fed-president-says-its-up-to-congress-to-save-the-economy}{7h
ago}

\href{https://www.nytimes.com/live/2020/08/03/business/stock-market-today-coronavirus?action=click\&pgtype=Article\&state=default\&region=MAIN_CONTENT_1\&context=storylines_live_updates\#the-chicago-fed-president-says-its-up-to-congress-to-save-the-economy}{The
Chicago Fed president says it's up to Congress to save the economy.}

\href{https://www.nytimes.com/live/2020/08/03/business/stock-market-today-coronavirus?action=click\&pgtype=Article\&state=default\&region=MAIN_CONTENT_1\&context=storylines_live_updates\#faa-says-boeing-has-effectively-mitigated-defects-in-the-737-max}{8h
ago}

\href{https://www.nytimes.com/live/2020/08/03/business/stock-market-today-coronavirus?action=click\&pgtype=Article\&state=default\&region=MAIN_CONTENT_1\&context=storylines_live_updates\#faa-says-boeing-has-effectively-mitigated-defects-in-the-737-max}{F.A.A.
says Boeing has `effectively mitigated' defects in the 737 Max.}

\href{https://www.nytimes.com/live/2020/08/03/business/stock-market-today-coronavirus?action=click\&pgtype=Article\&state=default\&region=MAIN_CONTENT_1\&context=storylines_live_updates\#small-businesses-got-emergency-loans-but-not-what-they-expected}{10h
ago}

\href{https://www.nytimes.com/live/2020/08/03/business/stock-market-today-coronavirus?action=click\&pgtype=Article\&state=default\&region=MAIN_CONTENT_1\&context=storylines_live_updates\#small-businesses-got-emergency-loans-but-not-what-they-expected}{Small
businesses got emergency loans, but not what they expected.}

\href{https://www.nytimes.com/live/2020/08/03/business/stock-market-today-coronavirus?action=click\&pgtype=Article\&state=default\&region=MAIN_CONTENT_1\&context=storylines_live_updates}{See
more updates}

More live coverage:
\href{https://www.nytimes.com/2020/08/03/world/coronavirus-covid-19.html?action=click\&pgtype=Article\&state=default\&region=MAIN_CONTENT_1\&context=storylines_live_updates}{Global}

The disaster loan program's missteps have been overshadowed by
\href{https://www.nytimes.com/2020/04/07/business/coronavirus-ppp-small-business-aid.html}{the
chaotic start} of the federal government's other large small-business
aid effort, the Paycheck Protection Program, which started taking
applications last week. Applicants to that initiative have faced delays
as banks deal with the hasty deployment of a \$349 billion program.

Disaster loan applicants --- many business owners are seeking relief
through both --- have also had to wait, even though the program predates
the crisis. The S.B.A. began taking applications in mid-March, but its
rollout was piecemeal. Each state had to submit its own formal disaster
declaration, and business owners could not apply until their state's
declaration was approved. It took around two weeks for all 50 states to
become eligible.

\includegraphics{https://static01.nyt.com/images/2020/04/10/business/09virus-sbadiaster2-print/merlin_171235518_a0c19bca-262e-44cd-aadb-53356bf09d43-articleLarge.jpg?quality=75\&auto=webp\&disable=upscale}

And even though Congress allocated billions of dollars to fund the
disaster loan program, some applicants said S.B.A. representatives had
told them that funding was running out.

Deb Wood-Schade, who runs a chiropractic wellness business in Aliso
Viejo, Calif., applied in mid-March and was told by phone on Saturday
that she had been approved for a loan of nearly \$25,000 --- enough to
cover six months of her operating expenses. But loan documents she
received on Wednesday suggested that amount had been cut to \$8,300,
covering just two months of her costs.

``Is that all I can get?'' asked Ms. Wood-Schade, who emailed that
question to her S.B.A. loan officer but had not heard back. ``I am
concerned if I take it I won't get the additional funds.''

Senator Ben Cardin, Democrat of Maryland, who pushed for the additional
funding through the CARES Act, said the program simply had to have more
money.

``The fact that S.B.A. is limiting Economic Injury Disaster Loans to an
initial disbursement of \$15,000 shows that there is a clear need for
more resources for this program,'' he said.

The loan program was never designed to handle a disaster of this
magnitude --- one that has
\href{https://www.nytimes.com/2020/04/09/business/economy/unemployment-claim-numbers-coronavirus.html}{sent
unemployment claims soaring} and
\href{https://www.nytimes.com/2020/04/06/business/economy/coronavirus-economy.html}{forced
businesses to close}.

The program's previous peak came in 2006 after Hurricane Katrina. It
disbursed loans of \$1.7 billion that year,
\href{https://fas.org/sgp/crs/misc/R43846.pdf}{according to the
Congressional Research Service}. In early March, Congress allocated
funds to support around \$7 billion in lending in response to the
pandemic. It added another \$10 billion through the CARES Act to fund
the \$10,000 cash grants, saying applicants could get that money even if
their applications were denied.

But the demand has been extraordinary.

If every applicant received the maximum \$10,000 grant, the funding
would cover around one million businesses. But more than three million
applied for disaster loans last week alone, Joseph Amato, the director
of the S.B.A.'s Nevada office, told attendees at a webinar on Monday.
His comments were reported earlier by
\href{https://www.washingtonpost.com/business/2020/04/08/video-sba-official-blasts-big-banks-over-failure-quickly-distribute-loans/}{The
Washington Post}.

In response to the demand, the S.B.A. appears to have also added an
additional restriction on the grants: Dozens of business owners said
they had been told that the grant, if they got it, would be limited to
\$1,000 per employee --- meaning the smallest businesses could not
receive the full amount.

Even early applicants who have been approved for larger loans still have
unanswered questions.

\href{https://www.nytimes.com/2020/04/02/business/small-business-coronavirus-stimulus.html}{Abninder
Mundra}, who owns a franchise of the UPS Store in Portola Valley,
Calif., applied for a loan on March 20 and was approved four days later
for \$210,000. He finally received and signed his closing documents this
week. He was still waiting for the cash to arrive --- and for details
about how the \$10,000 grant would work.

A retail business owner in California, who spoke on the condition of
anonymity because he feared jeopardizing the loan he had been promised,
was relieved to be getting the money needed to support his employees,
but frustrated about the process.

He sought a loan on March 17, right after his state became eligible. In
late March, he received a call from an S.B.A. official who requested
additional documents, then verbally approved a loan of \$500,000. It
took more than a week before he got
\href{https://int.nyt.com/data/documenthelper/6870-sba-eidl-loan-approval-letter/optimized/full.pdf}{a
letter confirming the loan}, along with a pile of closing documents to
sign.

Business owners who applied later are afraid the funding will run out
before their applications are processed.

Image

Kevin Smith, chief executive of the software company Wynexa, has spent
hours on hold trying to find out about his loan. Each time, he has
gotten less information.Credit...Go Nakamura for The New York Times

A loan capped at \$15,000 would be nearly useless to Kevin Smith, the
founder of Wynexa, a software company in Houston. Mr. Smith, who applied
for a loan in late March, is seeking at least \$50,000 to keep his
company and his three employees afloat.

He has called the S.B.A. for updates three times, waiting on hold each
time for up to two hours.

``Each time I've called it's been a different story,'' Mr. Smith said.
He was initially told he would have a response to his application by
April

\begin{enumerate}
\def\labelenumi{\arabic{enumi}.}
\tightlist
\item
  When that date passed and he called again, he was told it would take
  at least two weeks. Now, the S.B.A. is not offering any estimates at
  all, he said.
\end{enumerate}

Several business owners said their frustration was magnified by the
Trump administration's frequent proclamations that small business aid
was flowing freely. ``Any little glitch, we had worked out within
minutes, within hours,'' Mr. Trump said on Tuesday about problems with
the government's still-chaotic paycheck loan program.

Dave Vanslette, who has applied for that program and a disaster loan,
said comments like that were infuriating. He is still waiting for
responses on his applications.

``It would be great if our administration communicated the reality of
the situation instead of saying the process is working perfectly,'' said
Mr. Vanslette, who runs FairwayIQ, a software company in Waltham, Mass.
``This is not my experience.''

Advertisement

\protect\hyperlink{after-bottom}{Continue reading the main story}

\hypertarget{site-index}{%
\subsection{Site Index}\label{site-index}}

\hypertarget{site-information-navigation}{%
\subsection{Site Information
Navigation}\label{site-information-navigation}}

\begin{itemize}
\tightlist
\item
  \href{https://help.nytimes.com/hc/en-us/articles/115014792127-Copyright-notice}{©~2020~The
  New York Times Company}
\end{itemize}

\begin{itemize}
\tightlist
\item
  \href{https://www.nytco.com/}{NYTCo}
\item
  \href{https://help.nytimes.com/hc/en-us/articles/115015385887-Contact-Us}{Contact
  Us}
\item
  \href{https://www.nytco.com/careers/}{Work with us}
\item
  \href{https://nytmediakit.com/}{Advertise}
\item
  \href{http://www.tbrandstudio.com/}{T Brand Studio}
\item
  \href{https://www.nytimes.com/privacy/cookie-policy\#how-do-i-manage-trackers}{Your
  Ad Choices}
\item
  \href{https://www.nytimes.com/privacy}{Privacy}
\item
  \href{https://help.nytimes.com/hc/en-us/articles/115014893428-Terms-of-service}{Terms
  of Service}
\item
  \href{https://help.nytimes.com/hc/en-us/articles/115014893968-Terms-of-sale}{Terms
  of Sale}
\item
  \href{https://spiderbites.nytimes.com}{Site Map}
\item
  \href{https://help.nytimes.com/hc/en-us}{Help}
\item
  \href{https://www.nytimes.com/subscription?campaignId=37WXW}{Subscriptions}
\end{itemize}
