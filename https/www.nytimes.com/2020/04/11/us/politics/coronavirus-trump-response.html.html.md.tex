Sections

SEARCH

\protect\hyperlink{site-content}{Skip to
content}\protect\hyperlink{site-index}{Skip to site index}

\href{https://www.nytimes.com/section/politics}{Politics}

\href{https://myaccount.nytimes.com/auth/login?response_type=cookie\&client_id=vi}{}

\href{https://www.nytimes.com/section/todayspaper}{Today's Paper}

\href{/section/politics}{Politics}\textbar{}He Could Have Seen What Was
Coming: Behind Trump's Failure on the Virus

\url{https://nyti.ms/3ceNp5H}

\begin{itemize}
\item
\item
\item
\item
\item
\item
\end{itemize}

\begin{itemize}
\item
  \href{https://www.nytimes.com/2020/08/03/us/elections/biden-vs-trump.html?action=click\&pgtype=Article\&state=default\&region=TOP_BANNER\&context=storylines_menu}{Election
  Updates}
\item
  \href{https://www.nytimes.com/article/biden-vice-president-2020.html?action=click\&pgtype=Article\&state=default\&region=TOP_BANNER\&context=storylines_menu}{Biden's
  V.P. Search}
\item
  \href{https://www.nytimes.com/interactive/2020/07/24/us/politics/trump-biden-campaign-donors.html?action=click\&pgtype=Article\&state=default\&region=TOP_BANNER\&context=storylines_menu}{Map
  of Donations}
\item
  \href{https://www.nytimes.com/interactive/2020/us/elections/delegate-count-primary-results.html?action=click\&pgtype=Article\&state=default\&region=TOP_BANNER\&context=storylines_menu}{Delegate
  Count}
\item
  \href{https://www.nytimes.com/interactive/2019/us/politics/2020-presidential-candidates.html?action=click\&pgtype=Article\&state=default\&region=TOP_BANNER\&context=storylines_menu}{The
  Candidates}
\item
  \href{https://www.nytimes.com/newsletters/politics?action=click\&pgtype=Article\&state=default\&region=TOP_BANNER\&context=storylines_menu}{Politics
  Newsletter}
\end{itemize}

Advertisement

\protect\hyperlink{after-top}{Continue reading the main story}

Supported by

\protect\hyperlink{after-sponsor}{Continue reading the main story}

\hypertarget{he-could-have-seen-what-was-coming-behind-trumps-failure-on-the-virus}{%
\section{He Could Have Seen What Was Coming: Behind Trump's Failure on
the
Virus}\label{he-could-have-seen-what-was-coming-behind-trumps-failure-on-the-virus}}

An examination reveals the president was warned about the potential for
a pandemic but that internal divisions, lack of planning and his faith
in his own instincts led to a halting response.

\includegraphics{https://static01.nyt.com/images/2020/04/12/us/politics/12dc-virus-reconstruct1/12dc-virus-reconstruct1-articleLarge-v2.jpg?quality=75\&auto=webp\&disable=upscale}

\href{https://www.nytimes.com/by/eric-lipton}{\includegraphics{https://static01.nyt.com/images/2018/12/06/multimedia/author-eric-lipton/author-eric-lipton-thumbLarge.png}}\href{https://www.nytimes.com/by/david-e-sanger}{\includegraphics{https://static01.nyt.com/images/2018/10/03/multimedia/author-david-e-sanger/author-david-e-sanger-thumbLarge.png}}\href{https://www.nytimes.com/by/maggie-haberman}{\includegraphics{https://static01.nyt.com/images/2018/07/12/multimedia/author-maggie-haberman/author-maggie-haberman-thumbLarge.png}}\href{https://www.nytimes.com/by/michael-d-shear}{\includegraphics{https://static01.nyt.com/images/2018/06/13/multimedia/author-michael-d-shear/author-michael-d-shear-thumbLarge-v2.png}}\href{https://www.nytimes.com/by/mark-mazzetti}{\includegraphics{https://static01.nyt.com/images/2018/07/12/multimedia/author-Mark-Mazzetti/author-Mark-Mazzetti-thumbLarge-v4.png}}\href{https://www.nytimes.com/by/julian-e-barnes}{\includegraphics{https://static01.nyt.com/images/2019/12/13/reader-center/author-julian-barnes/author-julian-barnes-thumbLarge.png}}

By \href{https://www.nytimes.com/by/eric-lipton}{Eric Lipton},
\href{https://www.nytimes.com/by/david-e-sanger}{David E. Sanger},
\href{https://www.nytimes.com/by/maggie-haberman}{Maggie Haberman},
\href{https://www.nytimes.com/by/michael-d-shear}{Michael D. Shear},
\href{https://www.nytimes.com/by/mark-mazzetti}{Mark Mazzetti} and
\href{https://www.nytimes.com/by/julian-e-barnes}{Julian E. Barnes}

\begin{itemize}
\item
  Published April 11, 2020Updated May 4, 2020
\item
  \begin{itemize}
  \item
  \item
  \item
  \item
  \item
  \item
  \end{itemize}
\end{itemize}

\href{https://cn.nytimes.com/usa/20200413/coronavirus-trump-response/}{阅读简体中文版}\href{https://cn.nytimes.com/usa/20200413/coronavirus-trump-response/zh-hant/}{閱讀繁體中文版}

\hypertarget{listen-to-this-article}{%
\subsubsection{Listen to This Article}\label{listen-to-this-article}}

Audio Recording by Audm

\emph{To hear more audio stories from publishers, like The New York
Times, download}
\href{https://www.audm.com/?utm_source=nyt\&utm_medium=embed\&utm_campaign=trumps_failure_virus}{\emph{Audm
for iPhone or Android}}\emph{.}

WASHINGTON --- ``Any way you cut it, this is going to be bad,'' a senior
medical adviser at the Department of Veterans Affairs, Dr. Carter
Mecher, wrote on the night of Jan. 28, in an email to a group of public
health experts scattered around the government and universities. ``The
projected size of the outbreak already seems hard to believe.''

A week after the first
\href{https://www.nytimes.com/2020/05/04/us/politics/trump-coronavirus-death-toll.html}{coronavirus}
case had been identified in the United States, and six long weeks before
\href{https://www.nytimes.com/2020/05/04/us/politics/trump-coronavirus-death-toll.html}{President
Trump} finally took aggressive action to confront the danger the nation
was facing --- a pandemic that is now forecast to take tens of thousands
of American lives --- Dr. Mecher was urging the upper ranks of the
nation's public health bureaucracy to wake up and prepare for the
possibility of far more drastic action.

``You guys made fun of me screaming to close the schools,'' he wrote to
\href{https://int.nyt.com/data/documenthelper/6879-2020-covid-19-red-dawn-rising/66f590d5cd41e11bea0f/optimized/full.pdf\#page=1}{the
group, which called itself ``Red Dawn,''} an inside joke based on the
1984 movie about a band of Americans
\href{https://www.youtube.com/watch?v=mRTzUHmx9ZA}{trying to save the
country after a foreign invasion.} ``Now I'm screaming, close the
colleges and universities.''

His was hardly a lone voice. Throughout January, as Mr. Trump repeatedly
played down the seriousness of the virus and focused on other issues, an
array of figures inside his government --- from top White House advisers
to experts deep in the cabinet departments and intelligence agencies ---
identified the threat, sounded alarms and made clear the need for
aggressive action.

The president, though, was slow to absorb the scale of the risk and to
act accordingly, focusing instead on controlling the message, protecting
gains in the economy and batting away warnings from senior officials. It
was a problem, he said, that had come out of nowhere and could not have
been foreseen.

Even after Mr. Trump took his first concrete action at the end of
January ---
\href{https://www.nytimes.com/2020/01/31/business/china-travel-coronavirus.html}{limiting
travel from China} --- public health often had to compete with economic
and political considerations in internal debates, slowing the path
toward belated decisions to seek more money from Congress, obtain
necessary supplies, address shortfalls in testing and ultimately move to
keep much of the nation at home.

Unfolding as it did in the wake of his impeachment by the House and in
the midst of his Senate trial, Mr. Trump's response was colored by his
suspicion of and disdain for what he viewed as the ``Deep State'' ---
the very people in his government whose expertise and long experience
might have guided him more quickly toward steps that would slow the
virus, and likely save lives.

Decision-making was also complicated by a long-running dispute inside
the administration over how to deal with China. The virus at first took
a back seat to a desire not to upset Beijing during trade talks, but
later the impulse to score points against Beijing left the world's two
leading powers further divided as they confronted one of the first truly
global threats of the 21st century.

The shortcomings of Mr. Trump's performance have played out with
remarkable transparency as part of his daily effort to dominate
television screens and the national conversation.

But dozens of interviews with current and former officials and a review
of emails and other records revealed many previously unreported details
and a fuller picture of the roots and extent of his halting response as
the deadly virus spread:

\begin{itemize}
\item
  The National Security Council office responsible for tracking
  pandemics received intelligence reports in early January predicting
  the spread of the virus to the United States, and within weeks was
  raising options like keeping Americans home from work and shutting
  down cities the size of Chicago. Mr. Trump would avoid such steps
  until March.
\item
  Despite Mr. Trump's
  \href{https://www.whitehouse.gov/briefings-statements/remarks-president-trump-vice-president-pence-members-coronavirus-task-force-press-briefing-april-7-2020/}{denial}
  weeks later, he was told at the time about a Jan. 29
  \href{https://www.nytimes.com/2020/04/06/us/politics/navarro-warning-trump-coronavirus.html}{memo}
  produced by his trade adviser, Peter Navarro, laying out in striking
  detail the potential risks of a
  \href{https://www.nytimes.com/2020/04/14/us/politics/trump-authority.html}{coronavirus}
  pandemic: as many as half a million deaths and trillions of dollars in
  economic losses.
\item
  The health and human services secretary, Alex M. Azar II, directly
  warned Mr. Trump of the possibility of a pandemic during a call on
  Jan. 30, the second warning he delivered to the president about the
  virus in two weeks. The president, who was on Air Force One while
  traveling for appearances in the Midwest, responded that Mr. Azar was
  being alarmist.
\item
  Mr. Azar publicly
  \href{https://www.cidrap.umn.edu/news-perspective/2020/02/cdc-flu-surveillance-system-enlisted-hunt-covid-19-cases}{announced}
  in February that the government was establishing a ``surveillance''
  system in five American cities to measure the spread of the virus and
  enable experts to project the next hot spots. It was delayed for
  weeks. The slow start of that plan, on top of the well-documented
  \href{https://www.nytimes.com/2020/03/28/us/testing-coronavirus-pandemic.html}{failures
  to develop the nation's testing capacity}, left administration
  officials with almost no insight into how rapidly the virus was
  spreading. ``We were flying the plane with no instruments,'' one
  official said.
\item
  By the third week in February, the administration's top public health
  experts concluded they should recommend to Mr. Trump a new approach
  that would include warning the American people of the risks and urging
  steps like social distancing and staying home from work. But the White
  House focused instead on messaging and crucial additional weeks went
  by before their views were reluctantly accepted by the president ---
  time when the virus spread largely unimpeded.
\end{itemize}

When Mr. Trump finally
\href{https://www.nytimes.com/2020/03/16/health/coronavirus-social-distancing-crowd-size.html}{agreed
in mid-March} to recommend social distancing across the country,
effectively bringing much of the economy to a halt, he seemed
shellshocked and deflated to some of his closest associates. One
described him as ``subdued'' and ``baffled'' by how the crisis had
played out. An economy that he had wagered his re-election on was
suddenly in shambles.

He only regained his swagger, the associate said, from conducting his
daily White House briefings, at which he often seeks to rewrite the
history of the past several months. He declared at one point that he
\href{https://www.nytimes.com/2020/03/17/us/politics/trump-coronavirus.html}{``felt
it was a pandemic long before it was called a pandemic,''} and insisted
at another that he had to be a
\href{https://www.whitehouse.gov/briefings-statements/remarks-president-trump-vice-president-pence-members-coronavirus-task-force-press-briefing-15/}{``cheerleader
for the country,''} as if that explained why he failed to prepare the
public for what was coming.

Mr. Trump's allies and some administration officials say the criticism
has been unfair. The Chinese government misled other governments, they
say. And they insist that the president was either not getting proper
information, or the people around him weren't conveying the urgency of
the threat. In some cases, they argue, the specific officials he was
hearing from had been discredited in his eyes, but once the right
information got to him through other channels, he made the right calls.

``While the media and Democrats refused to seriously acknowledge this
virus in January and February, President Trump took bold action to
protect Americans and unleash the full power of the federal government
to curb the spread of the virus, expand testing capacities and expedite
vaccine development even when we had no true idea the level of
transmission or asymptomatic spread,'' said Judd Deere, a White House
spokesman.

There were key turning points along the way, opportunities for Mr. Trump
to get ahead of the virus rather than just chase it. There were internal
debates that presented him with stark choices, and moments when he could
have chosen to ask deeper questions and learn more. How he handled them
may shape his re-election campaign. They will certainly shape his
legacy.

\hypertarget{the-containment-illusion}{%
\subsection{The Containment Illusion}\label{the-containment-illusion}}

\emph{By the last week of February, it was clear to the administration's
public health team that schools and businesses in hot spots would have
to close. But in the turbulence of the Trump White House, it took three
more weeks to persuade the president that failure to act quickly to
control the spread of the virus would have dire consequences.}

When Dr. Robert Kadlec, the top disaster response official at the Health
and Human Services Department, convened the White House coronavirus task
force on Feb. 21, his agenda was urgent. There were deep cracks in the
administration's strategy for keeping the virus out of the United
States. They were going to have to lock down the country to prevent it
from spreading. The question was: When?

\includegraphics{https://static01.nyt.com/images/2020/04/12/us/politics/12dc-virus-reconstruct-kadlec/merlin_170064618_871d4dfd-55ad-4873-996c-95a9e3f4a6b6-articleLarge.jpg?quality=75\&auto=webp\&disable=upscale}

There had already been an
\href{https://www.nytimes.com/2020/02/21/world/asia/china-coronavirus-iran.html}{alarming
spike in new cases} around the world and the virus was spreading across
the Middle East. It was becoming apparent that the administration had
botched the rollout of testing to track the virus at home, and a
smaller-scale surveillance program intended to piggyback on a federal
flu tracking system had also been stillborn.

In Washington, the president was not worried,
\href{https://factba.se/transcript/donald-trump-speech-kag-rally-manchester-new-hampshire-february-10-2020}{predicting}
that by April, ``when it gets a little warmer, it miraculously goes
away.'' His White House had yet to ask Congress for additional funding
to prepare for the potential cost of wide-scale infection across the
country, and health care providers were growing increasingly nervous
about the availability of masks, ventilators and other equipment.

What Mr. Trump decided to do next could dramatically shape the course of
the pandemic --- and how many people would get sick and die.

With that in mind, the task force had gathered for a tabletop exercise
--- a real-time version of a full-scale war gaming of a flu pandemic the
administration had run the previous year.
\href{https://www.nytimes.com/2020/03/19/us/politics/trump-coronavirus-outbreak.html}{That
earlier exercise}, also conducted by Mr. Kadlec and called ``Crimson
Contagion,''
\href{https://int.nyt.com/data/documenthelper/6824-2019-10-key-findings-and-after/05bd797500ea55be0724/optimized/full.pdf\#page=18}{predicted
110 million infections}, 7.7 million hospitalizations and 586,000 deaths
following a hypothetical outbreak that started in China.

Facing the likelihood of a real pandemic, the group needed to decide
when to abandon ``containment'' --- the effort to keep the virus outside
the U.S. and to isolate anyone who gets infected --- and embrace
``mitigation'' to thwart the spread of the virus inside the country
until a vaccine becomes available.

Among the questions on the agenda, which was reviewed by The New York
Times, was when the department's secretary, Mr. Azar, should recommend
that Mr. Trump take textbook mitigation measures ``such as school
dismissals and cancellations of mass gatherings,'' which had been
identified as the next appropriate step in
\href{https://www.cdc.gov/flu/pandemic-resources/pdf/community_mitigation-sm.pdf}{a
Bush-era pandemic plan}.

The exercise was sobering. The group --- including Dr. Anthony S. Fauci
of the National Institutes of Health; Dr. Robert R. Redfield of the
Centers for Disease Control and Prevention, and Mr. Azar, who at that
stage was leading the White House Task Force --- concluded they would
soon need to move toward aggressive social distancing, even at the risk
of severe disruption to the nation's economy and the daily lives of
millions of Americans.

Image

The president urged social distancing in mid-March but almost
immediately began talking about reopening the economy.Credit...Andrew
Seng for The New York Times

If Dr. Kadlec had any doubts, they were erased two days later, when he
stumbled upon an email from a researcher at the Georgia
Institute~of~Technology, who was among the group of academics,
government physicians and infectious diseases doctors who had spent
weeks tracking the outbreak in the Red Dawn email chain.

\hypertarget{latest-updates-2020-election}{%
\section{\texorpdfstring{\href{https://www.nytimes.com/2020/08/03/us/elections/biden-vs-trump.html?action=click\&pgtype=Article\&state=default\&region=MAIN_CONTENT_1\&context=storylines_live_updates}{Latest
Updates: 2020
Election}}{Latest Updates: 2020 Election}}\label{latest-updates-2020-election}}

Updated 2020-08-04T01:23:51.312Z

\begin{itemize}
\tightlist
\item
  \href{https://www.nytimes.com/2020/08/03/us/elections/biden-vs-trump.html?action=click\&pgtype=Article\&state=default\&region=MAIN_CONTENT_1\&context=storylines_live_updates\#link-6494b448}{Trump
  assails mail-in voting anew, citing delays in declaring a winner in a
  New York congressional primary.}
\item
  \href{https://www.nytimes.com/2020/08/03/us/elections/biden-vs-trump.html?action=click\&pgtype=Article\&state=default\&region=MAIN_CONTENT_1\&context=storylines_live_updates\#link-3de249e6}{Obama
  issues his first slate of 2020 endorsements.}
\item
  \href{https://www.nytimes.com/2020/08/03/us/elections/biden-vs-trump.html?action=click\&pgtype=Article\&state=default\&region=MAIN_CONTENT_1\&context=storylines_live_updates\#link-54e34d20}{In
  a big shift, Trump is now encouraging mask-wearing in campaign
  emails.}
\end{itemize}

\href{https://www.nytimes.com/2020/08/03/us/elections/biden-vs-trump.html?action=click\&pgtype=Article\&state=default\&region=MAIN_CONTENT_1\&context=storylines_live_updates}{See
more updates}

A 20-year-old Chinese woman had infected five relatives with the virus
even though she never displayed any symptoms herself. The implication
was grave --- apparently healthy people could be unknowingly spreading
the virus --- and supported the need to move quickly to mitigation.

``Is this true?!'' Dr. Kadlec wrote back to the researcher. ``If so we
have a huge whole on our screening and quarantine effort,'' including a
typo where he meant hole. Her response was blunt: ``People are carrying
the virus everywhere.''

Image

The following day, Dr. Kadlec and the others decided to present Mr.
Trump with a plan titled ``Four Steps to Mitigation,'' telling the
president that they needed to begin preparing Americans for a step
rarely taken in United States history.

But over the next several days, a presidential blowup and internal turf
fights would sidetrack such a move. The focus would shift to messaging
and confident predictions of success rather than publicly calling for a
shift to mitigation.

These final days of February, perhaps more than any other moment during
his tenure in the White House, illustrated Mr. Trump's inability or
unwillingness to absorb warnings coming at him. He instead reverted to
his traditional political playbook in the midst of a public health
calamity, squandering vital time as the coronavirus spread silently
across the country.

Dr. Kadlec's group wanted to meet with the president right away, but Mr.
Trump was on a trip to India, so they agreed to make the case to him in
person as soon as he returned two days later. If they could convince him
of the need to shift strategy, they could immediately begin a national
education campaign aimed at preparing the public for the new reality.

A memo dated Feb. 14, prepared in coordination with the National
Security Council and titled ``U.S. Government Response to the 2019 Novel
Coronavirus,'' documented what more drastic measures would look like,
including: ``significantly limiting public gatherings and cancellation
of almost all sporting events, performances, and public and private
meetings that cannot be convened by phone. Consider school closures.
Widespread `stay at home' directives from public and private
organizations with nearly 100\% telework for some.''

The memo did not advocate an immediate national shutdown, but said the
targeted use of ``quarantine and isolation measures'' could be used to
slow the spread in places where ``sustained human-to-human
transmission'' is evident.

Within 24 hours, before they got a chance to make their presentation to
the president, the plan went awry.

Mr. Trump was walking up the steps of Air Force One to head home from
India on Feb. 25 when Dr. Nancy Messonnier, the director of the National
Center for Immunization and Respiratory Diseases,
\href{https://www.cdc.gov/media/releases/2020/t0225-cdc-telebriefing-covid-19.html}{publicly
issued} the blunt warning they had all agreed was necessary.

But Dr. Messonnier had jumped the gun. They had not told the president
yet, much less gotten his consent.

On the 18-hour plane ride home, Mr. Trump fumed as he watched the
\href{https://www.nytimes.com/2020/02/24/business/stock-market-coronavirus.html}{stock
market crash} after Dr. Messonnier's comments. Furious, he called Mr.
Azar when he landed at around 6 a.m. on Feb. 26, raging that Dr.
Messonnier had scared people unnecessarily. Already on thin ice with the
president over a variety of issues and having overseen the failure to
quickly produce an effective and widely available test, Mr. Azar would
soon find his authority reduced.

The meeting that evening with Mr. Trump to advocate social distancing
was canceled, replaced by a news conference in which the president
announced that the White House response would be put under the command
of Vice President Mike Pence.

Image

Vice President Mike Pence visiting a Walmart distribution center in
Gordonsville, Va. this month. He was put in charge of the coronavirus
task force after Mr. Trump clashed with Alex M. Azar II, the health and
human services secretary.Credit...Anna Moneymaker/The New York Times

The push to convince Mr. Trump of the need for more assertive action
stalled. With Mr. Pence and his staff in charge, the focus was clear: no
more alarmist messages. Statements and media appearances by health
officials like Dr. Fauci and Dr. Redfield would be coordinated through
Mr. Pence's office. It would be more than three weeks before Mr. Trump
would announce serious social distancing efforts, a lost period during
which the spread of the virus accelerated rapidly.

Over nearly three weeks from Feb. 26 to March 16, the number of
\href{https://www.nytimes.com/interactive/2020/us/coronavirus-us-cases.html\#map}{confirmed
coronavirus cases} in the United States grew from
\href{https://www.cdc.gov/media/releases/2020/s0226-Covid-19-spread.html}{15}
to 4,226. Since then, nearly half a million Americans have tested
positive for the virus and authorities say hundreds of thousands more
are likely infected.

\hypertarget{the-china-factor}{%
\subsection{The China Factor}\label{the-china-factor}}

\emph{The earliest warnings about coronavirus got caught in the
crosscurrents of the administration's internal disputes over China. It
was the China hawks who pushed earliest for a travel ban. But their
animosity toward China also undercut hopes for a more cooperative
approach by the world's two leading powers to a global crisis.}

It was early January, and the call with a Hong Kong epidemiologist left
Matthew Pottinger rattled.

Mr. Pottinger, the deputy national security adviser and a hawk on China,
took a blunt warning away from the call with the doctor, a longtime
friend: A ferocious, new outbreak that on the surface appeared similar
to the
\href{https://www.nytimes.com/2003/04/27/world/the-sars-epidemic-the-path-from-china-s-provinces-a-crafty-germ-breaks-out.html}{SARS
epidemic of 2003} had emerged in China. It had spread far more quickly
than the government was admitting to, and it wouldn't be long before it
reached other parts of the world.

Image

Matthew Pottinger, left, the deputy national security adviser, was among
those in the administration who pushed for imposing limits on travel
from China.Credit...Andrew Harnik/Associated Press

Mr. Pottinger had worked as a Wall Street Journal correspondent in Hong
Kong during the SARS epidemic, and was still scarred by his experience
documenting the death spread by that highly contagious virus.

Now, seventeen years later, his friend had a blunt message: You need to
be ready. The virus, he warned, which originated in the city of Wuhan,
was being transmitted by people who were showing no symptoms --- an
insight that American health officials had not yet accepted. Mr.
Pottinger declined through a spokesman to comment.

It was one of the earliest warnings to the White House, and it echoed
the intelligence reports making their way to the National Security
Council. While most of the early assessments from the C.I.A. had little
more information than was available publicly, some of the more
specialized corners of the intelligence world were producing
sophisticated and chilling warnings.

In a report to the director of national intelligence, the State
Department's epidemiologist wrote in early January that the virus was
likely to spread across the globe, and warned that the coronavirus could
develop into a pandemic. Working independently, a small outpost of the
Defense Intelligence Agency, the National Center for Medical
Intelligence, came to the same conclusion. Within weeks after getting
initial information about the virus early in the year, biodefense
experts inside the National Security Council, looking at what was
happening in Wuhan, started urging officials to think about what would
be needed to quarantine a city the size of Chicago.

Image

An I.C.U. ward at Papa Giovanni XXIII hospital in Bergamo, Italy last
month where critical Covid-19 patients were hospitalized.Credit...Fabio
Bucciarelli for The New York Times

By mid-January there was growing evidence of the virus spreading outside
China. Mr. Pottinger began convening daily meetings about the
coronavirus. He alerted his boss, Robert C. O'Brien, the national
security adviser.

The early alarms sounded by Mr. Pottinger and other China hawks were
freighted with ideology --- including a push to publicly blame China
that critics in the administration say was a distraction as the
coronavirus spread to Western Europe and eventually the United States.

And they ran into opposition from Mr. Trump's economic advisers, who
worried a tough approach toward China could scuttle a trade deal that
was a pillar of Mr. Trump's re-election campaign.

With his skeptical --- some might even say conspiratorial --- view of
China's ruling Communist Party, Mr. Pottinger initially suspected that
President Xi Jinping's government was keeping a dark secret: that the
virus may have originated in one of the laboratories in Wuhan studying
deadly pathogens. In his view, it might have even been a deadly accident
unleashed on an unsuspecting Chinese population.

During meetings and telephone calls, Mr. Pottinger asked intelligence
agencies --- including officers at the C.I.A. working on Asia and on
weapons of mass destruction --- to search for evidence that might
bolster his theory.

They didn't have any evidence. Intelligence agencies did not detect any
alarm inside the Chinese government that analysts presumed would
accompany the accidental leak of a deadly virus from a government
laboratory. But Mr. Pottinger continued to believe the coronavirus
problem was far worse than the Chinese were acknowledging. Inside the
West Wing, the director of the Domestic Policy Council, Joe Grogan, also
tried to sound alarms that the threat from China was growing.

Mr. Pottinger, backed by Mr. O'Brien, became one of the driving forces
of a campaign in the final weeks of January to convince Mr. Trump to
impose limits on travel from China --- the first substantive step taken
to impede the spread of the virus and one that the president has
repeatedly cited as evidence that he was on top of the problem.

In addition to the opposition from the economic team, Mr. Pottinger and
his allies among the China hawks had to overcome initial skepticism from
the administration's public health experts.

Image

Dr. Anthony Fauci and Dr. Robert Redfield, two leading members of the
administration's public health team, were ready to back a shift in
administration strategy by late February.Credit...Pete Marovich for The
New York Times

Travel restrictions were usually counterproductive to managing
biological outbreaks because they prevented doctors and other
much-needed medical help from easily getting to the affected areas, the
health officials said. And such bans often cause infected people to
flee, spreading the disease further.

But on the morning of Jan. 30, Mr. Azar got a call from Dr. Fauci, Dr.
Redfield and others saying they had changed their minds. The World
Health Organization had
\href{https://www.who.int/news-room/detail/30-01-2020-statement-on-the-second-meeting-of-the-international-health-regulations-(2005)-emergency-committee-regarding-the-outbreak-of-novel-coronavirus-(2019-ncov)}{declared
a global public health emergency} and American officials had discovered
the\href{https://www.cdc.gov/media/releases/2020/p0130-coronavirus-spread.html}{first
confirmed case} of person-to-person transmission inside the United
States.

The economic team, led by Treasury Secretary Steven Mnuchin, continued
to argue that there were big risks in taking a provocative step toward
China and moving to curb global travel. After a debate, Mr. Trump came
down on the side of the hawks and the public health team. The limits on
travel from China were publicly
\href{https://www.whitehouse.gov/presidential-actions/proclamation-suspension-entry-immigrants-nonimmigrants-persons-pose-risk-transmitting-2019-novel-coronavirus/}{announced
on Jan. 31}.

Image

Email sent among federal government physicians and former senior
pandemic advisers by Dr. James Lawler, an infectious diseases specialist
and public health expert at the University of Nebraska Medical Center.

Still, Mr. Trump and other senior officials were wary of further
upsetting Beijing. Besides the concerns about the impact on the trade
deal, they knew that an escalating confrontation was risky because the
United States relies heavily on China for pharmaceuticals and the kinds
of protective equipment most needed to combat the coronavirus.

But the hawks kept pushing in February to take a critical stance toward
China amid the growing crisis. Mr. Pottinger and others --- including
aides to Secretary of State Mike Pompeo --- pressed for government
statements to use the term ``Wuhan Virus.''

Mr. Pompeo tried to hammer the anti-China message at every turn,
eventually even urging leaders of the Group of 7 industrialized
countries to use ``Wuhan virus'' in a joint statement.

Others, including aides to **** Mr. Pence, resisted taking a hard public
line, believing that angering Beijing might lead the Chinese government
to withhold medical supplies, pharmaceuticals and any scientific
research that might ultimately lead to a vaccine.

Image

A temporary hospital for Covid-19 patients in Wuhan, China, where the
virus originated. Crosscurrents in the administration's China policy
complicated its response to the outbreak.Credit...Chinatopix, via
Associated Press

Mr. Trump took a conciliatory approach through the middle of March,
praising the job Mr. Xi was doing.

That changed abruptly, when aides informed Mr. Trump that a Chinese
Foreign Ministry spokesman had publicly spun a new conspiracy about the
origins of Covid-19: that it was brought to China by U.S. Army personnel
who visited the country last October.

Mr. Trump was furious, and he took to his favorite platform to broadcast
a new message. On March 16, he
\href{https://twitter.com/realDonaldTrump/status/1239685852093169664?ref_src=twsrc\%5Etfw\%7Ctwcamp\%5Etweetembed\%7Ctwterm\%5E1239685852093169664\&ref_url=https\%3A\%2F\%2Fwww.bloomberg.com\%2Fnews\%2Farticles\%2F2020-03-17\%2Ftrump-s-chinese-virus-tweet-adds-fuel-to-fire-with-beijing}{wrote
on Twitter} that ``the United States will be powerfully supporting those
industries, like Airlines and others, that are particularly affected by
the Chinese Virus.''

Mr. Trump's decision to escalate the war of words undercut any remaining
possibility of broad cooperation between the governments to address a
global threat. It remains to be seen whether that mutual suspicion will
spill over into efforts to develop treatments or vaccines, both areas
where the two nations are now competing.

One immediate result was a free-for-all across the United States, with
state and local governments and hospitals bidding on the open market for
scarce but essential Chinese-made products. When the state of
Massachusetts managed to procure 1.2 million masks, it fell to the owner
of the New England Patriots, Robert K. Kraft, a Trump ally, to cut
through extensive red tape on both sides of the Pacific to
\href{https://www.nytimes.com/aponline/2020/04/02/sports/football/ap-fbn-patriots-masks-assist.html}{send
his own plane to pick them up.}

\hypertarget{the-consequences-of-chaos}{%
\subsection{The Consequences of Chaos}\label{the-consequences-of-chaos}}

\emph{The chaotic culture of the Trump White House contributed to the
crisis. A lack of planning and a failure to execute, combined with the
president's focus on the news cycle and his preference for following his
gut rather than the data cost time, and perhaps lives.}

Inside the West Wing, Mr. Navarro, Mr. Trump's trade adviser, was widely
seen as quick-tempered, self-important and prone to butting in. He is
among the most outspoken of China hawks and in late January was clashing
with the administration's health experts over limiting travel from
China.

Image

Peter Navarro, Mr. Trump's trade adviser, warned that a pandemic could
cost the United States trillions of dollars and put millions of
Americans at risk of illness or death.Credit...Doug Mills/The New York
Times

So it elicited eye rolls when, after initially being prevented from
joining the coronavirus task force, he circulated a
\href{https://www.nytimes.com/2020/04/06/us/politics/navarro-warning-trump-coronavirus.html}{memo
on Jan. 29} urging Mr. Trump to impose the travel limits, arguing that
failing to confront the outbreak aggressively could be catastrophic,
leading to hundreds of thousands of deaths and trillions of dollars in
economic losses.

The uninvited message could not have conflicted more with the
president's approach at the time of playing down the severity of the
threat. And when aides raised it with Mr. Trump, he responded that he
was unhappy that Mr. Navarro had put his warning in writing.

From the time the virus was first identified as a concern, the
administration's response was plagued by the rivalries and factionalism
that routinely swirl around Mr. Trump and, along with the president's
impulsiveness, undercut decision making and policy development.

Faced with the relentless march of a deadly pathogen, the disagreements
and a lack of long-term planning had significant consequences. They
slowed the president's response and resulted in problems with execution
and planning, including delays in seeking money from Capitol Hill and a
failure to begin broad surveillance testing.

The efforts to shape Mr. Trump's view of the virus began early in
January, when his focus was elsewhere: the fallout from his
\href{https://www.nytimes.com/2020/01/03/world/middleeast/iranian-general-qassem-soleimani-killed.html}{decision
to kill Maj. Gen. Qassim Suleimani}, Iran's security mastermind; his
push for an
\href{https://www.nytimes.com/2019/10/11/business/economy/us-china-trade-deal.html}{initial
trade deal with China}; and his Senate impeachment trial,
\href{https://www.nytimes.com/2020/01/21/us/politics/trump.html}{which
was about to begin}.

Even after Mr. Azar first briefed him about the potential seriousness of
the virus during a phone call on Jan. 18 while the president
\href{https://www.nytimes.com/interactive/2017/04/05/us/politics/tracking-trumps-visits-to-his-branded-properties.html}{was
at his} Mar-a-Lago resort in Florida, Mr. Trump projected confidence
that it would be a passing problem.

``We have it totally under control,''
\href{https://www.cnbc.com/2020/01/22/cnbc-transcript-president-donald-trump-sits-down-with-cnbcs-joe-kernen-at-the-world-economic-forum-in-davos-switzerland.html}{he
told an interviewer} a few days later while attending the World Economic
Forum in Switzerland. ``It's going to be just fine.''

Back in Washington, voices outside of the White House peppered Mr. Trump
with competing assessments about what he should do and how quickly he
should act.

Image

Traders at the New York Stock Exchange on March 9, when stocks suffered
their worst single-day decline in more than a decade. Two days later,
Mr. Trump announced restrictions on travel from Europe.Credit...Ashley
Gilbertson for The New York Times

The efforts to sort out policy behind closed doors were contentious and
sometimes only loosely organized.

That was the case when the National Security Council convened a meeting
on short notice on the afternoon of Jan. 27. The Situation Room was
standing room only, packed with top White House advisers, low-level
staffers, Mr. Trump's social media guru, and several cabinet
secretaries. There was no checklist about the preparations for a
possible pandemic, which would require intensive testing, rapid
acquisition of protective gear, and perhaps serious limitations on
Americans' movements.

Instead, after a 20-minute description by Mr. Azar of his department's
capabilities, the meeting was jolted when Stephen E. Biegun, the newly
installed deputy secretary of state, announced plans to issue a
``\href{https://travel.state.gov/content/travel/en/traveladvisories/ea/travel-advisory-alert-global-level-4-health-advisory-issue.html}{level
four}'' travel warning, strongly discouraging Americans from traveling
to China. The room erupted into bickering.

A few days later, on the evening of Jan. 30, Mick Mulvaney, the acting
White House chief of staff at the time, and Mr. Azar called Air Force
One as the president was making the final decision to go ahead with the
restrictions on China travel. Mr. Azar was blunt, warning that the virus
could develop into a pandemic and arguing that China should be
criticized for failing to be transparent.

Mr. Trump rejected the idea of criticizing China, saying the country had
enough to deal with. And if the president's decision on the travel
restrictions suggested that he fully grasped the seriousness of the
situation, his response to Mr. Azar indicated otherwise.

Stop panicking, Mr. Trump told him.

That sentiment was present throughout February, as the president's top
aides reached for a consistent message but took few concrete steps to
prepare for the possibility of a major public health crisis.

Image

A worker at a Starbucks at an airport in Beijing in January checks a
customer's temperature.Credit...Kevin Frayer/Getty Images

During a briefing on Capitol Hill on Feb. 5, senators urged
administration officials to take the threat more seriously. Several
asked if the administration needed additional money to help local and
state health departments prepare.

Derek Kan, a senior official from the Office of Management and Budget,
replied that the administration had all the money it needed, at least at
that point, to stop the virus, two senators who attended the briefing
said.

``Just left the Administration briefing on Coronavirus,'' Senator
Christopher S. Murphy, Democrat of Connecticut, wrote in a
\href{https://twitter.com/chrismurphyct/status/1225073987639705600?lang=en}{tweet}
shortly after. ``Bottom line: they aren't taking this seriously
enough.''

The administration also struggled to carry out plans it did agree on. In
mid-February, with the effort to roll out widespread testing stalled,
Mr. Azar announced a plan to repurpose a flu-surveillance system in five
major cities to help track the virus among the general population. The
effort all but collapsed even before it got started as Mr. Azar
\href{https://int.nyt.com/data/documenthelper/6873-2020-02-14-cdc-surveillance-fu/51b5187c0fd8b4698a50/optimized/full.pdf\#page=1}{struggled
to win approval}for \$100 million in funding and the
\href{https://www.nytimes.com/2020/03/10/us/coronavirus-testing-delays.html}{C.D.C.
failed to make reliable tests available}.

The number of infections in the United States started to surge through
February and early March, but the Trump administration did not move to
place large-scale orders for masks and other protective equipment, or
critical hospital equipment, such as ventilators. The Pentagon
\href{https://www.nytimes.com/2020/03/17/us/politics/coronavirus-government-army-corps.html}{sat
on standby}, awaiting any orders to help provide temporary hospitals or
other assistance.

Image

Dr. Carter Mecher with the Department of Veterans Affairs argued to
colleagues in late February for so-called targeted layered containment
(TLC) and non-pharmaceutical interventions (NPIs), which are measures
like closing schools and businesses, to limit the spread of the virus.
Mr. Azar and other public health officials came to the same conclusion
around that time.

As February gave way to March, the president continued to be surrounded
by divided factions even as it became clearer that avoiding more
aggressive steps was not tenable.

Mr. Trump had agreed to give an Oval Office address on the evening of
March 11 announcing restrictions on travel from Europe, where the virus
was ravaging Italy. But responding to the views of his business friends
and others, he continued to resist calls for social distancing, school
closures and other steps that would imperil the economy.

Image

Pandemic experts, including Mr. Trump's own former homeland security
adviser, Thomas Bossert, compare notes via the Red Dawn email group,
after Mr. Trump's March 11 announcement that he is limiting travel from
Europe.

But the virus was already multiplying across the country --- and
hospitals were at risk of buckling under the looming wave of severely
ill people, lacking masks and other protective equipment, ventilators
and sufficient intensive care beds. The question loomed over the
president and his aides after weeks of stalling and inaction: What were
they going to do?

The approach that Mr. Azar and others had planned to bring to him weeks
earlier moved to the top of the agenda. Even then, and even by Trump
White House standards, the debate over whether to shut down much of the
country to slow the spread was especially fierce.

Always attuned to anything that could trigger a stock market decline or
an economic slowdown that could hamper his re-election effort, Mr. Trump
also reached out to prominent investors like Stephen A. Schwarzman, the
chief executive of Blackstone Group, a private equity firm.

``Everybody questioned it for a while, not everybody, but a good portion
questioned it,'' Mr. Trump said
\href{https://www.whitehouse.gov/briefings-statements/remarks-president-trump-vice-president-pence-members-coronavirus-task-force-press-briefing-17/}{earlier
this month}. ``They said, let's keep it open. Let's ride it.''

In a tense Oval Office meeting, when Mr. Mnuchin again stressed that the
economy would be ravaged, Mr. O'Brien, the national security adviser,
who had been worried about the virus for weeks, sounded exasperated as
he told Mr. Mnuchin that the economy would be destroyed regardless if
officials did nothing.

Soon after the Oval Office address, Dr. Scott Gottlieb, the former
commissioner of the Food and Drug Administration and a trusted sounding
board inside the White House, visited Mr. Trump, partly at the urging of
Jared Kushner, the president's son-in-law. Dr. Gottlieb's role was to
impress upon the president how serious the crisis could become. Mr.
Pence, by then in charge of the task force, also played a key role at
that point in getting through to the president about the seriousness of
the moment in a way that Mr. Azar had not.

Image

Dr. Deborah Birx eventually helped convince Mr. Trump that stricter
measures needed to be taken.Credit...Anna Moneymaker/The New York Times

But in the end, aides said, it was Dr. Deborah L. Birx, the veteran AIDS
researcher who had joined the task force, who **** helped to persuade
Mr. Trump. Soft-spoken and fond of the kind of charts and graphs Mr.
Trump prefers, Dr. Birx did not have the rough edges that could irritate
the president. He often told people he thought she was elegant.

On Monday, March 16, Mr. Trump
\href{https://www.nytimes.com/2020/03/16/us/politics/trump-coronavirus-guidelines.html}{announced
new social distancing guidelines}, saying they would be in place for two
weeks. The subsequent economic disruptions were so severe that the
president repeatedly suggested that he wanted to lift even those
temporary restrictions. He frequently asked aides why his administration
was still being blamed in news coverage for the widespread failures
involving testing, insisting the responsibility had shifted to the
states.

During the last week in March, Kellyanne Conway, a senior White House
adviser involved in task force meetings, gave voice to concerns other
aides had. She warned Mr. Trump that his wished-for date of Easter to
reopen the country likely couldn't be accomplished. Among other things,
she told him, he would end up being blamed by critics for every
subsequent death caused by the virus.

Within days, he watched images on television of a calamitous situation
at Elmhurst Hospital Center, miles from his childhood home in Queens,
N.Y., where
\href{https://www.nytimes.com/2020/03/25/nyregion/nyc-coronavirus-hospitals.html}{13
people had died} from the coronavirus in 24 hours.

He left the restrictions in place.

Mark Walker contributed reporting from Washington, and Mike Baker from
Seattle. Kitty Bennett contributed research.

\hypertarget{our-2020-election-guide}{%
\section{Our 2020 Election Guide}\label{our-2020-election-guide}}

Updated Aug. 3, 2020

\begin{itemize}
\item
  \begin{center}\rule{0.5\linewidth}{\linethickness}\end{center}

  \hypertarget{the-latest}{%
  \subsection{The Latest}\label{the-latest}}

  \begin{itemize}
  \tightlist
  \item
    President Trump again assails mail-in voting,
    \href{https://www.nytimes.com/2020/08/03/us/politics/trump-mail-in-voting.html?action=click\&pgtype=Article\&state=default\&region=BELOW_MAIN_CONTENT\&context=storylines_guide}{claiming
    without evidence that the process is plagued by fraud}.
  \end{itemize}
\item
  \begin{center}\rule{0.5\linewidth}{\linethickness}\end{center}

  \hypertarget{bidens-vp-search}{%
  \subsection{Biden's V.P. Search}\label{bidens-vp-search}}

  \begin{itemize}
  \tightlist
  \item
    \href{https://www.nytimes.com/article/biden-vice-president-2020.html?action=click\&pgtype=Article\&state=default\&region=BELOW_MAIN_CONTENT\&context=storylines_guide}{Here
    are 13 women} who have been under consideration to be Joe Biden's
    running mate, and why each might be chosen --- and might not be.
  \end{itemize}
\item
  \begin{center}\rule{0.5\linewidth}{\linethickness}\end{center}

  \hypertarget{keep-up-with-our-coverage}{%
  \subsection{Keep Up With Our
  Coverage}\label{keep-up-with-our-coverage}}

  \begin{itemize}
  \tightlist
  \item
    Get an
    \href{https://www.nytimes.com/newsletters/politics?action=click\&pgtype=Article\&state=default\&region=BELOW_MAIN_CONTENT\&context=storylines_guide}{email}
    recapping the day's news
  \end{itemize}

  \begin{itemize}
  \tightlist
  \item
    Download our mobile app on
    \href{https://apps.apple.com/us/app/nytimes/id284862083?ls=1\&mat_click_id=5c79ae7455014fd1bd66b5610c05b8f2-20191112-16948\&referrer=mat_click_id\%3D5c79ae7455014fd1bd66b5610c05b8f2-20191112-16948\%26link_click_id\%3D722930677036718082}{iOS}
    and
    \href{http://a.localytics.com/android?id=com.nytimes.android\&referrer=utm_source\%3Dother_nyt_mobile_web\%26utm_medium\%3DWeb\%2520page\%26utm_term\%3DGeneral\%2520Mobile\%2520Page\%26utm_campaign\%3DNYT\%2520Mobile\%2520General\%2520Page}{Android}
    and turn on Breaking News and Politics alerts
  \end{itemize}
\end{itemize}

Advertisement

\protect\hyperlink{after-bottom}{Continue reading the main story}

\hypertarget{site-index}{%
\subsection{Site Index}\label{site-index}}

\hypertarget{site-information-navigation}{%
\subsection{Site Information
Navigation}\label{site-information-navigation}}

\begin{itemize}
\tightlist
\item
  \href{https://help.nytimes.com/hc/en-us/articles/115014792127-Copyright-notice}{©~2020~The
  New York Times Company}
\end{itemize}

\begin{itemize}
\tightlist
\item
  \href{https://www.nytco.com/}{NYTCo}
\item
  \href{https://help.nytimes.com/hc/en-us/articles/115015385887-Contact-Us}{Contact
  Us}
\item
  \href{https://www.nytco.com/careers/}{Work with us}
\item
  \href{https://nytmediakit.com/}{Advertise}
\item
  \href{http://www.tbrandstudio.com/}{T Brand Studio}
\item
  \href{https://www.nytimes.com/privacy/cookie-policy\#how-do-i-manage-trackers}{Your
  Ad Choices}
\item
  \href{https://www.nytimes.com/privacy}{Privacy}
\item
  \href{https://help.nytimes.com/hc/en-us/articles/115014893428-Terms-of-service}{Terms
  of Service}
\item
  \href{https://help.nytimes.com/hc/en-us/articles/115014893968-Terms-of-sale}{Terms
  of Sale}
\item
  \href{https://spiderbites.nytimes.com}{Site Map}
\item
  \href{https://help.nytimes.com/hc/en-us}{Help}
\item
  \href{https://www.nytimes.com/subscription?campaignId=37WXW}{Subscriptions}
\end{itemize}
