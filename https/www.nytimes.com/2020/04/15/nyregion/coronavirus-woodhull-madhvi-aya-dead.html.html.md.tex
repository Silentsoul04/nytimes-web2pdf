\href{/section/nyregion}{New York}\textbar{}The Heartbreaking Last Texts
of a Hospital Worker on the Front Lines

\url{https://nyti.ms/2VyLCSl}

\begin{itemize}
\item
\item
\item
\item
\item
\item
\end{itemize}

\href{https://www.nytimes.com/news-event/coronavirus?action=click\&pgtype=Article\&state=default\&region=TOP_BANNER\&context=storylines_menu}{The
Coronavirus Outbreak}

\begin{itemize}
\tightlist
\item
  live\href{https://www.nytimes.com/2020/08/04/world/coronavirus-cases.html?action=click\&pgtype=Article\&state=default\&region=TOP_BANNER\&context=storylines_menu}{Latest
  Updates}
\item
  \href{https://www.nytimes.com/interactive/2020/us/coronavirus-us-cases.html?action=click\&pgtype=Article\&state=default\&region=TOP_BANNER\&context=storylines_menu}{Maps
  and Cases}
\item
  \href{https://www.nytimes.com/interactive/2020/science/coronavirus-vaccine-tracker.html?action=click\&pgtype=Article\&state=default\&region=TOP_BANNER\&context=storylines_menu}{Vaccine
  Tracker}
\item
  \href{https://www.nytimes.com/2020/08/02/us/covid-college-reopening.html?action=click\&pgtype=Article\&state=default\&region=TOP_BANNER\&context=storylines_menu}{College
  Reopening}
\item
  \href{https://www.nytimes.com/live/2020/08/04/business/stock-market-today-coronavirus?action=click\&pgtype=Article\&state=default\&region=TOP_BANNER\&context=storylines_menu}{Economy}
\end{itemize}

\includegraphics{https://static01.nyt.com/images/2020/04/13/nyregion/woodhull-header-img/woodhull-header-img-articleLarge.png?quality=75\&auto=webp\&disable=upscale}

Sections

\protect\hyperlink{site-content}{Skip to
content}\protect\hyperlink{site-index}{Skip to site index}

\hypertarget{the-heartbreaking-last-texts-of-a-hospital-worker-on-the-front-lines}{%
\section{The Heartbreaking Last Texts of a Hospital Worker on the Front
Lines}\label{the-heartbreaking-last-texts-of-a-hospital-worker-on-the-front-lines}}

Madhvi Aya worked long hours in the emergency room of a hospital in
Brooklyn that was battered by the coronavirus. Then she caught the virus
herself.

Credit...

Supported by

\protect\hyperlink{after-sponsor}{Continue reading the main story}

By \href{https://www.nytimes.com/by/michael-rothfeld}{Michael Rothfeld},
\href{https://www.nytimes.com/by/jesse-drucker}{Jesse Drucker} and
\href{https://www.nytimes.com/by/william-k-rashbaum}{William K.
Rashbaum}

\begin{itemize}
\item
  Published April 15, 2020Updated April 27, 2020
\item
  \begin{itemize}
  \item
  \item
  \item
  \item
  \item
  \item
  \end{itemize}
\end{itemize}

Lying in a hospital bed last month, Madhvi Aya understood what was
happening to her.

She had been a doctor in India, then trained to become a physician
assistant after she immigrated to the United States. She had worked for
a dozen years at Woodhull Medical Center, a public hospital in Brooklyn,
where she could see the coronavirus tearing
\href{https://www.nytimes.com/2020/04/09/nyregion/coronavirus-queens-corona-jackson-heights-elmhurst.html}{a
merciless path} through the city.

Within days of her last shift as a caregiver, Ms. Aya became a patient.
She had worked in Woodhull's understaffed emergency room, taking medical
histories, ordering tests and asking about symptoms. Now she had become
infected.

Ms. Aya, 61, was alone in a hospital, less than two miles from her
husband and 18-year-old daughter on Long Island, who could not visit
her. She did not have the solace of familiar colleagues; she had been
admitted to a different facility nearer her home. In a text with her
family, she described horrible chest pain from trying to get out of bed.

``I have not improved the way should have been,'' she wrote her husband,
Raj, on March 23.

As she grew sicker, her texts came less frequently and in short,
sporadic bursts.

``I miss you mommy,'' her daughter, Minnoli, wrote on March 25. She
craved the reassurance of her mother's hugs, the comfort of crawling
into her bed. ``Please don't give up hope because I haven't given up. I
need my mommy. I need you to come back to me.''

``Love you,'' Ms. Aya wrote the next day.

``Mom be back.''

Ms. Aya could not keep that promise.

\includegraphics{https://static01.nyt.com/images/2020/04/16/nyregion/16nyvirus-woodhull-1/00nyvirus-woodhull01-articleLarge-v2.jpg?quality=75\&auto=webp\&disable=upscale}

Front-line health care workers
\href{https://www.nytimes.com/2020/03/30/nyregion/ny-coronavirus-doctors-sick.html}{face
a high risk}of contracting the coronavirus, and scores have become sick.
But it is less known how many have died in New York from the virus after
working closely with Covid-19 patients.

Health care systems by and large have not publicly revealed the
identities of those employees, who include
\href{https://www.nytimes.com/2020/03/26/nyregion/nurse-dies-coronavirus-mount-sinai.html}{Kious
Kelly}, a nurse manager at Mount Sinai West in Manhattan, and Dr. Ronald
Verrier, a surgeon at St. Barnabas Hospital in the Bronx.
\href{https://www.nytimes.com/2020/04/27/world/americas/health-workers-attacked.html}{Doctors,
nurses and staffers} who worked in other capacities at hospitals that
have been flooded with virus patients have also died, according to their
families and colleagues.

Ms. Aya's text messages and her family's account of her final days
reveal a woman who spent much of her life devoted to medicine before
succumbing to the
\href{https://www.nytimes.com/2020/04/13/nyregion/coronavirus-nyc-doctors.html}{cruel
and familiar arc} of a patient with Covid-19. Her early mild symptoms
and quarantine at home were followed by a rapidly escalating illness and
long waits for care, until she died alone.

``She was always there for us, whenever we wanted,'' her husband said.
But when she got sick, ``no one was next to her,'' he said.

\hypertarget{latest-updates-global-coronavirus-outbreak}{%
\section{\texorpdfstring{\href{https://www.nytimes.com/2020/08/04/world/coronavirus-cases.html?action=click\&pgtype=Article\&state=default\&region=MAIN_CONTENT_1\&context=storylines_live_updates}{Latest
Updates: Global Coronavirus
Outbreak}}{Latest Updates: Global Coronavirus Outbreak}}\label{latest-updates-global-coronavirus-outbreak}}

Updated 2020-08-04T20:42:41.838Z

\begin{itemize}
\tightlist
\item
  \href{https://www.nytimes.com/2020/08/04/world/coronavirus-cases.html?action=click\&pgtype=Article\&state=default\&region=MAIN_CONTENT_1\&context=storylines_live_updates\#link-1228a480}{Novavax
  sees encouraging results from two studies of its experimental
  vaccine.}
\item
  \href{https://www.nytimes.com/2020/08/04/world/coronavirus-cases.html?action=click\&pgtype=Article\&state=default\&region=MAIN_CONTENT_1\&context=storylines_live_updates\#link-4825b93}{Public
  and private schools in Maryland and elsewhere are divided over
  in-person instruction.}
\item
  \href{https://www.nytimes.com/2020/08/04/world/coronavirus-cases.html?action=click\&pgtype=Article\&state=default\&region=MAIN_CONTENT_1\&context=storylines_live_updates\#link-50f7386d}{The
  United Nations calls on policymakers to `plan thoroughly for school
  reopenings.'}
\end{itemize}

\href{https://www.nytimes.com/2020/08/04/world/coronavirus-cases.html?action=click\&pgtype=Article\&state=default\&region=MAIN_CONTENT_1\&context=storylines_live_updates}{See
more updates}

More live coverage:
\href{https://www.nytimes.com/live/2020/08/04/business/stock-market-today-coronavirus?action=click\&pgtype=Article\&state=default\&region=MAIN_CONTENT_1\&context=storylines_live_updates}{Markets}

Ms. Aya moved to the United States in 1994 to join her husband, who had
immigrated a decade earlier and met her on a return trip to India. She
started working at Woodhull in 2008 and became a senior physician
assistant. Colleagues said she nurtured younger co-workers by drawing on
the experience she had gained as an anesthesiologist and internist in
India, along with her instinct as a caretaker.

``This has been a heavy blow to us all,'' Dr. Robert Chin, Woodhull's
emergency department director, said in an internal email on April 1,
asking for
\href{https://www.gofundme.com/f/in-memory-of-madhvi-aya?utm_source=customer\&utm_medium=email\&utm_campaign=m_pd+share-sheet}{donations}
to help Ms. Aya's family, for whom she had been the primary wage earner.

Like many other hospitals, Woodhull had converted one ward after the
next into makeshift intensive care units when the virus
\href{https://www.nytimes.com/2020/03/20/nyregion/ny-coronavirus-hospitals.html?searchResultPosition=3}{began
its surge in New York}. As the hospital verged on running out of
ventilators, protective gear for medical staff and other equipment, it
appealed to affiliated medical centers for help and transferred patients
elsewhere.

In the week of Ms. Aya's death, Woodhull's emergency department alone
had 20 patients on ventilators, Dr. Chin said.

Another Woodhull employee, a radiology
\href{https://www.facebook.com/permalink.php?story_fbid=2944721018928457\&id=100001717633057}{clerk
named Thomas Soto}, died of the virus at the hospital last week, 12 days
after his first symptoms. Mr. Soto, 59, worked there for decades and was
close to retirement. ``The only reason my dad pushed to work that extra
year was to retire with full pension, and I lost him because of that,''
Jonathan Soto, the older of Mr. Soto's two sons, said through tears.

A former hospital police officer, Herb A. Houchen, 35, returned to
Woodhull as a Covid patient and also died. He had worked at Woodhull for
more than five years and left behind an 11-year-old daughter.

Image

Mr. Houchen was known to family and friends as ``Shaq'' because of his
6-foot-7-inch frame and resemblance to the basketball player Shaquille
O'Neal.Credit...via the family of Herb A. Houchen

Ms. Aya's daughter, Minnoli, said her emotions have ranged from intense
grief to disbelief. She thinks about becoming a doctor herself and is
angry at a health care system that she believes did not protect its
front-line workers. Sometimes she is angry at her mother for not coming
home.

``I just want to be able to hug her and have her tell me everything is
going to be OK,'' Minnoli said.

There is no way to determine how Ms. Aya became infected. While she
worked at Woodhull in early March, front-line employees had not yet been
instructed to wear protective masks for all patients, one staff member
said. Later, as the crisis grew, hospitals realized that people coming
in for apparently unrelated problems were also testing positive for the
virus, potentially exposing unwitting health care workers.

On March 17, Woodhull's administration advised emergency department
workers to wear masks for all patients. A spokesman for New York City's
Health and Hospitals Corporation, which oversees Woodhull, said
protective equipment was available to its health care workers.

Ms. Aya's shifts could be grueling at
\href{https://www.nychealthandhospitals.org/woodhull/}{Woodhull, a
320-bed public hospital} at the intersection of Bedford-Stuyvesant,
Bushwick and Williamsburg. Her husband often drove her to work from
their home in Floral Park as early as 6 a.m. and picked her up 12 hours
later so she could relax in the car.

``We have to take care of our patients first,'' she often said.

Image

Ambulances line up at outside Woodhull Medical Center in Brooklyn, which
has been inundated with patients during the coronavirus
epidemic.~Credit...Victor J. Blue for The New York Times

At the beginning of the outbreak, Ms. Aya worried about bringing the
virus home to her 64-year-old husband, whom she had guided through an
aortic bypass in 2017, and her 86-year-old mother, Malti Masrani, for
whom she had cared after a stroke late last year.

She began coughing around the time of her last shift on March 12, Mr.
Aya said. He drove her to Woodhull the next evening so a doctor could
examine her, picking her up many hours later, after she was tested.

For the next few days, they quarantined on different floors of their
Cape Cod-style home. Ms. Aya had no underlying medical conditions,
family members said.

\href{https://www.nytimes.com/news-event/coronavirus?action=click\&pgtype=Article\&state=default\&region=MAIN_CONTENT_3\&context=storylines_faq}{}

\hypertarget{the-coronavirus-outbreak-}{%
\subsubsection{The Coronavirus Outbreak
›}\label{the-coronavirus-outbreak-}}

\hypertarget{frequently-asked-questions}{%
\paragraph{Frequently Asked
Questions}\label{frequently-asked-questions}}

Updated August 4, 2020

\begin{itemize}
\item ~
  \hypertarget{i-have-antibodies-am-i-now-immune}{%
  \paragraph{I have antibodies. Am I now
  immune?}\label{i-have-antibodies-am-i-now-immune}}

  \begin{itemize}
  \tightlist
  \item
    As of right
    now,\href{https://www.nytimes.com/2020/07/22/health/covid-antibodies-herd-immunity.html?action=click\&pgtype=Article\&state=default\&region=MAIN_CONTENT_3\&context=storylines_faq}{that
    seems likely, for at least several months.} There have been
    frightening accounts of people suffering what seems to be a second
    bout of Covid-19. But experts say these patients may have a
    drawn-out course of infection, with the virus taking a slow toll
    weeks to months after initial exposure. People infected with the
    coronavirus typically
    \href{https://www.nature.com/articles/s41586-020-2456-9}{produce}
    immune molecules called antibodies, which are
    \href{https://www.nytimes.com/2020/05/07/health/coronavirus-antibody-prevalence.html?action=click\&pgtype=Article\&state=default\&region=MAIN_CONTENT_3\&context=storylines_faq}{protective
    proteins made in response to an
    infection}\href{https://www.nytimes.com/2020/05/07/health/coronavirus-antibody-prevalence.html?action=click\&pgtype=Article\&state=default\&region=MAIN_CONTENT_3\&context=storylines_faq}{.
    These antibodies may} last in the body
    \href{https://www.nature.com/articles/s41591-020-0965-6}{only two to
    three months}, which may seem worrisome, but that's perfectly normal
    after an acute infection subsides, said Dr. Michael Mina, an
    immunologist at Harvard University. It may be possible to get the
    coronavirus again, but it's highly unlikely that it would be
    possible in a short window of time from initial infection or make
    people sicker the second time.
  \end{itemize}
\item ~
  \hypertarget{im-a-small-business-owner-can-i-get-relief}{%
  \paragraph{I'm a small-business owner. Can I get
  relief?}\label{im-a-small-business-owner-can-i-get-relief}}

  \begin{itemize}
  \tightlist
  \item
    The
    \href{https://www.nytimes.com/article/small-business-loans-stimulus-grants-freelancers-coronavirus.html?action=click\&pgtype=Article\&state=default\&region=MAIN_CONTENT_3\&context=storylines_faq}{stimulus
    bills enacted in March} offer help for the millions of American
    small businesses. Those eligible for aid are businesses and
    nonprofit organizations with fewer than 500 workers, including sole
    proprietorships, independent contractors and freelancers. Some
    larger companies in some industries are also eligible. The help
    being offered, which is being managed by the Small Business
    Administration, includes the Paycheck Protection Program and the
    Economic Injury Disaster Loan program. But lots of folks have
    \href{https://www.nytimes.com/interactive/2020/05/07/business/small-business-loans-coronavirus.html?action=click\&pgtype=Article\&state=default\&region=MAIN_CONTENT_3\&context=storylines_faq}{not
    yet seen payouts.} Even those who have received help are confused:
    The rules are draconian, and some are stuck sitting on
    \href{https://www.nytimes.com/2020/05/02/business/economy/loans-coronavirus-small-business.html?action=click\&pgtype=Article\&state=default\&region=MAIN_CONTENT_3\&context=storylines_faq}{money
    they don't know how to use.} Many small-business owners are getting
    less than they expected or
    \href{https://www.nytimes.com/2020/06/10/business/Small-business-loans-ppp.html?action=click\&pgtype=Article\&state=default\&region=MAIN_CONTENT_3\&context=storylines_faq}{not
    hearing anything at all.}
  \end{itemize}
\item ~
  \hypertarget{what-are-my-rights-if-i-am-worried-about-going-back-to-work}{%
  \paragraph{What are my rights if I am worried about going back to
  work?}\label{what-are-my-rights-if-i-am-worried-about-going-back-to-work}}

  \begin{itemize}
  \tightlist
  \item
    Employers have to provide
    \href{https://www.osha.gov/SLTC/covid-19/standards.html}{a safe
    workplace} with policies that protect everyone equally.
    \href{https://www.nytimes.com/article/coronavirus-money-unemployment.html?action=click\&pgtype=Article\&state=default\&region=MAIN_CONTENT_3\&context=storylines_faq}{And
    if one of your co-workers tests positive for the coronavirus, the
    C.D.C.} has said that
    \href{https://www.cdc.gov/coronavirus/2019-ncov/community/guidance-business-response.html}{employers
    should tell their employees} -\/- without giving you the sick
    employee's name -\/- that they may have been exposed to the virus.
  \end{itemize}
\item ~
  \hypertarget{should-i-refinance-my-mortgage}{%
  \paragraph{Should I refinance my
  mortgage?}\label{should-i-refinance-my-mortgage}}

  \begin{itemize}
  \tightlist
  \item
    \href{https://www.nytimes.com/article/coronavirus-money-unemployment.html?action=click\&pgtype=Article\&state=default\&region=MAIN_CONTENT_3\&context=storylines_faq}{It
    could be a good idea,} because mortgage rates have
    \href{https://www.nytimes.com/2020/07/16/business/mortgage-rates-below-3-percent.html?action=click\&pgtype=Article\&state=default\&region=MAIN_CONTENT_3\&context=storylines_faq}{never
    been lower.} Refinancing requests have pushed mortgage applications
    to some of the highest levels since 2008, so be prepared to get in
    line. But defaults are also up, so if you're thinking about buying a
    home, be aware that some lenders have tightened their standards.
  \end{itemize}
\item ~
  \hypertarget{what-is-school-going-to-look-like-in-september}{%
  \paragraph{What is school going to look like in
  September?}\label{what-is-school-going-to-look-like-in-september}}

  \begin{itemize}
  \tightlist
  \item
    It is unlikely that many schools will return to a normal schedule
    this fall, requiring the grind of
    \href{https://www.nytimes.com/2020/06/05/us/coronavirus-education-lost-learning.html?action=click\&pgtype=Article\&state=default\&region=MAIN_CONTENT_3\&context=storylines_faq}{online
    learning},
    \href{https://www.nytimes.com/2020/05/29/us/coronavirus-child-care-centers.html?action=click\&pgtype=Article\&state=default\&region=MAIN_CONTENT_3\&context=storylines_faq}{makeshift
    child care} and
    \href{https://www.nytimes.com/2020/06/03/business/economy/coronavirus-working-women.html?action=click\&pgtype=Article\&state=default\&region=MAIN_CONTENT_3\&context=storylines_faq}{stunted
    workdays} to continue. California's two largest public school
    districts --- Los Angeles and San Diego --- said on July 13, that
    \href{https://www.nytimes.com/2020/07/13/us/lausd-san-diego-school-reopening.html?action=click\&pgtype=Article\&state=default\&region=MAIN_CONTENT_3\&context=storylines_faq}{instruction
    will be remote-only in the fall}, citing concerns that surging
    coronavirus infections in their areas pose too dire a risk for
    students and teachers. Together, the two districts enroll some
    825,000 students. They are the largest in the country so far to
    abandon plans for even a partial physical return to classrooms when
    they reopen in August. For other districts, the solution won't be an
    all-or-nothing approach.
    \href{https://bioethics.jhu.edu/research-and-outreach/projects/eschool-initiative/school-policy-tracker/}{Many
    systems}, including the nation's largest, New York City, are
    devising
    \href{https://www.nytimes.com/2020/06/26/us/coronavirus-schools-reopen-fall.html?action=click\&pgtype=Article\&state=default\&region=MAIN_CONTENT_3\&context=storylines_faq}{hybrid
    plans} that involve spending some days in classrooms and other days
    online. There's no national policy on this yet, so check with your
    municipal school system regularly to see what is happening in your
    community.
  \end{itemize}
\end{itemize}

But her cough worsened at home, and she developed a fever. In the early
afternoon of March 18, Mr. Aya dropped his wife off at Long Island
Jewish Medical Center, near their home. He would not see her again.

For an hour and a half, Mr. Aya sat in his car in the hospital parking
lot, texting his wife --- almost always addressing her as ``SH,'' for
``sweetheart''--- to check if she had received a chest X-ray and to say
that he had tried to get in to see her.

``You go home I call you I am waiting,'' she wrote.

At 4:47 a.m. the next day, Ms. Aya texted that she was still waiting for
a bed. When Mr. Aya woke up, he asked if he could bring her coffee. She
said no. She reported her test had come back from Woodhull. Positive.

``I'm so sorry to hear,'' he replied.

They spoke by phone, and she told him to take care of her mother and
bring her daughter home from school.

The next day, Minnoli Aya returned from the University at Buffalo, where
she was a freshman. She believed her mother had pneumonia and hoped to
surprise her. Instead, she learned her mother had contracted the
coronavirus.

``I was just on the floor, and I was broken,'' Minnoli said.

Over the next week, she texted with her mother, who continued to
deteriorate. Doctors called Mr. Aya daily. By the end of the week, his
wife was increasingly having trouble breathing.

By the morning of March 29, doctors got ready to put Ms. Aya on a
ventilator. But there was a life-threatening complication, and they
asked Mr. Aya if he wanted to see his wife for what could be the last
time. He worried that his heart condition would put him at risk if he
caught the virus, and Minnoli could be left without a parent.

The decision not to go, he said, has haunted him. That afternoon, the
hospital called to say that his wife had died.

Minnoli, her father and grandmother could not hug each other, because
they were required to stay six feet apart, even though they lived in the
same house. Nor did they want to plan a funeral service that almost no
one would attend, one where they would not be able to view Ms. Aya's
body. They decided to have her cremated.

Even after her mother died, Minnoli still texted, trying to stay
connected.

``I miss u,'' she wrote before going to bed that night. When she woke
the next morning, Minnoli texted, ``Thank you for coming to me last
night in my dreams.''

Mr. Aya, concerned about Minnoli, arranged for her to speak to a
therapist by video after his wife's death. But he is not sure how long
he can afford the expense because Ms. Aya's health care plan had covered
the family. A representative of her union benefit fund told him by email
that benefits would end 30 days after his wife's death. ``My heart is
broken for you,'' the representative wrote in the email, which was
reviewed by The New York Times.

In the weeks since Ms. Aya has been gone, Minnoli has pored over the
messages still sitting in her phone.

``Hi mommy. College is getting so much more stressful now that it's at
home,'' she had written, three days before her mother's death. ``The
good thing is I'm home but I need you to come back here to me. I hope
you ate dinner and I'm still praying for you and haven't gave up hope.''

``Concentrate,'' Ms. Aya responded.

``I am but I want u home.''

``Home soon.''

``I love you mommy with all my heart.''

``Love you.''

Those were Ms. Aya's final words to her daughter.

Image

Minnoli Aya outside her home in Floral Park, N.Y.Credit...Hilary Swift
for The New York Times

Advertisement

\protect\hyperlink{after-bottom}{Continue reading the main story}

\hypertarget{site-index}{%
\subsection{Site Index}\label{site-index}}

\hypertarget{site-information-navigation}{%
\subsection{Site Information
Navigation}\label{site-information-navigation}}

\begin{itemize}
\tightlist
\item
  \href{https://help.nytimes.com/hc/en-us/articles/115014792127-Copyright-notice}{©~2020~The
  New York Times Company}
\end{itemize}

\begin{itemize}
\tightlist
\item
  \href{https://www.nytco.com/}{NYTCo}
\item
  \href{https://help.nytimes.com/hc/en-us/articles/115015385887-Contact-Us}{Contact
  Us}
\item
  \href{https://www.nytco.com/careers/}{Work with us}
\item
  \href{https://nytmediakit.com/}{Advertise}
\item
  \href{http://www.tbrandstudio.com/}{T Brand Studio}
\item
  \href{https://www.nytimes.com/privacy/cookie-policy\#how-do-i-manage-trackers}{Your
  Ad Choices}
\item
  \href{https://www.nytimes.com/privacy}{Privacy}
\item
  \href{https://help.nytimes.com/hc/en-us/articles/115014893428-Terms-of-service}{Terms
  of Service}
\item
  \href{https://help.nytimes.com/hc/en-us/articles/115014893968-Terms-of-sale}{Terms
  of Sale}
\item
  \href{https://spiderbites.nytimes.com}{Site Map}
\item
  \href{https://help.nytimes.com/hc/en-us}{Help}
\item
  \href{https://www.nytimes.com/subscription?campaignId=37WXW}{Subscriptions}
\end{itemize}
