Sections

SEARCH

\protect\hyperlink{site-content}{Skip to
content}\protect\hyperlink{site-index}{Skip to site index}

\href{/section/us}{U.S.}\textbar{}An Army of Virus Tracers Takes Shape
in Massachusetts

\url{https://nyti.ms/34QGq0p}

\begin{itemize}
\item
\item
\item
\item
\item
\item
\end{itemize}

\href{https://www.nytimes.com/news-event/coronavirus?action=click\&pgtype=Article\&state=default\&region=TOP_BANNER\&context=storylines_menu}{The
Coronavirus Outbreak}

\begin{itemize}
\tightlist
\item
  live\href{https://www.nytimes.com/2020/08/01/world/coronavirus-covid-19.html?action=click\&pgtype=Article\&state=default\&region=TOP_BANNER\&context=storylines_menu}{Latest
  Updates}
\item
  \href{https://www.nytimes.com/interactive/2020/us/coronavirus-us-cases.html?action=click\&pgtype=Article\&state=default\&region=TOP_BANNER\&context=storylines_menu}{Maps
  and Cases}
\item
  \href{https://www.nytimes.com/interactive/2020/science/coronavirus-vaccine-tracker.html?action=click\&pgtype=Article\&state=default\&region=TOP_BANNER\&context=storylines_menu}{Vaccine
  Tracker}
\item
  \href{https://www.nytimes.com/interactive/2020/07/29/us/schools-reopening-coronavirus.html?action=click\&pgtype=Article\&state=default\&region=TOP_BANNER\&context=storylines_menu}{What
  School May Look Like}
\item
  \href{https://www.nytimes.com/live/2020/07/31/business/stock-market-today-coronavirus?action=click\&pgtype=Article\&state=default\&region=TOP_BANNER\&context=storylines_menu}{Economy}
\end{itemize}

\includegraphics{https://static01.nyt.com/images/2020/04/11/autossell/Comp-1-0000500/Comp-1-0000500--videoSixteenByNineJumbo1600.jpg}

Credit...Graphic by Tim Rodenbröker

\hypertarget{an-army-of-virus-tracers-takes-shape-in-massachusetts}{%
\section{An Army of Virus Tracers Takes Shape in
Massachusetts}\label{an-army-of-virus-tracers-takes-shape-in-massachusetts}}

Asian countries have invested heavily in digital contact tracing, which
uses technology to warn people when they have been exposed to the
coronavirus. Massachusetts is using an old-fashioned means: people.

Credit...Graphic by Tim Rodenbröker

Supported by

\protect\hyperlink{after-sponsor}{Continue reading the main story}

\href{https://www.nytimes.com/by/ellen-barry}{\includegraphics{https://static01.nyt.com/images/2018/10/08/multimedia/author-ellen-barry/author-ellen-barry-thumbLarge.png}}

By \href{https://www.nytimes.com/by/ellen-barry}{Ellen Barry}

\begin{itemize}
\item
  Published April 16, 2020Updated June 3, 2020
\item
  \begin{itemize}
  \item
  \item
  \item
  \item
  \item
  \item
  \end{itemize}
\end{itemize}

BOSTON --- Alexandra Cross, a newly minted state public health worker,
dialed a stranger's telephone number on Monday, her heart racing.

It was Ms. Cross's first day as part of Massachusetts's fleet of
\href{https://www.nytimes.com/2020/06/03/health/coronavirus-contact-tracing-apps.html}{contact
tracers}, responsible for tracking down people who have been exposed to
the coronavirus, as soon as possible, and warning them. On her screen
was the name of a woman from Lowell.

``One person who has recently been diagnosed has been in contact with
you,'' the script told her to say. ``Do you have a few minutes to
discuss what that exposure might mean for you?'' Forty-five minutes
later, Ms. Cross hung up the phone. They had giggled and commiserated.
Her file was crammed with information.

She was taking her first steps up a Mount Everest of cases.

Massachusetts is the first state to invest in an ambitious
\href{https://www.nytimes.com/2020/06/21/nyregion/nyc-contact-tracing.html}{contact-tracing}
program, budgeting \$44 million to hire 1,000 people like Ms. Cross. The
program represents a bet on the part of Gov. Charlie Baker that the
state will be able to identify pockets of infection as they emerge, and
prevent infected people from spreading the virus further.

This could help Massachusetts in the coming weeks and months, as it
seeks to relax strict social-distancing measures and reopen its economy.

Contact tracing has helped Asian countries like South Korea and
Singapore contain the spread of the virus, but their systems rely on
\href{https://science.sciencemag.org/content/early/2020/04/09/science.abb6936}{digital
surveillance,} using patients' digital footprints to alert potential
contacts, an intrusion that many Americans would not accept.

Massachusetts is building its response around an old-school,
labor-intensive method: people. Lots of them.

``It's not cheap,'' Governor Baker, a Republican, said. ``But the way I
look at it, the single biggest challenge we're going to have is giving
people confidence and comfort that we know where the virus is.''

The state is
\href{https://www.nbcboston.com/news/local/mass-gov-baker-to-provide-update-on-coronavirus-response-2/2107985/}{currently
experiencing a surge of cases} that is expected to last for the next
week, after which it may start to consider easing social-distancing
rules. Robust contact tracing, combined with ramped-up testing, could
smooth that process, the governor said.

``It's hard to see how we create a sense of safety if we don't have a
program like this in place,'' he said.

The idea of
\href{https://www.theguardian.com/world/2020/apr/04/recruit-volunteer-army-to-trace-coronavirus-contacts-now-urge-top-scientists}{training
a corps of contact tracers} is emerging in many places at the same time,
as leaders think ahead to the point when social-distancing constraints
will be lifted.

\hypertarget{latest-updates-global-coronavirus-outbreak}{%
\section{\texorpdfstring{\href{https://www.nytimes.com/2020/08/01/world/coronavirus-covid-19.html?action=click\&pgtype=Article\&state=default\&region=MAIN_CONTENT_1\&context=storylines_live_updates}{Latest
Updates: Global Coronavirus
Outbreak}}{Latest Updates: Global Coronavirus Outbreak}}\label{latest-updates-global-coronavirus-outbreak}}

Updated 2020-08-02T10:04:29.623Z

\begin{itemize}
\tightlist
\item
  \href{https://www.nytimes.com/2020/08/01/world/coronavirus-covid-19.html?action=click\&pgtype=Article\&state=default\&region=MAIN_CONTENT_1\&context=storylines_live_updates\#link-34047410}{The
  U.S. reels as July cases more than double the total of any other
  month.}
\item
  \href{https://www.nytimes.com/2020/08/01/world/coronavirus-covid-19.html?action=click\&pgtype=Article\&state=default\&region=MAIN_CONTENT_1\&context=storylines_live_updates\#link-780ec966}{Top
  U.S. officials work to break an impasse over the federal jobless
  benefit.}
\item
  \href{https://www.nytimes.com/2020/08/01/world/coronavirus-covid-19.html?action=click\&pgtype=Article\&state=default\&region=MAIN_CONTENT_1\&context=storylines_live_updates\#link-2bc8948}{Its
  outbreak untamed, Melbourne goes into even greater lockdown.}
\end{itemize}

\href{https://www.nytimes.com/2020/08/01/world/coronavirus-covid-19.html?action=click\&pgtype=Article\&state=default\&region=MAIN_CONTENT_1\&context=storylines_live_updates}{See
more updates}

More live coverage:
\href{https://www.nytimes.com/live/2020/07/31/business/stock-market-today-coronavirus?action=click\&pgtype=Article\&state=default\&region=MAIN_CONTENT_1\&context=storylines_live_updates}{Markets}

President Trump was expected to announce that the Centers for Disease
Control and Prevention would hire hundreds of people to perform contact
tracing as part of his push to allow the country to go back to work and
school, a top government official said this week. San Francisco is
\href{https://abc7news.com/san-francisco-contact-tracing-coronavirus-tracnig-bay-area-lockdown-shelter-in-place/6090943/}{assembling
and training 150 volunteers} to augment the contact-tracing efforts of
its own public health department. Ireland is
\href{https://www.irishtimes.com/news/health/large-number-of-public-service-staff-to-be-redeployed-to-contact-tracing-1.4212937}{deploying
1,000 furloughed government workers} to do contact tracing.

This means restoring capacity that has been pared away for decades by
cuts to public health budgets, said Allyson Pollock, the director of the
Institute of Health and Society at Newcastle University in the northeast
of England.

``All our fancy scientists think it's boring, because it's people, you
need people on the ground,'' she said. ``It's the hard bread and butter
of communicable disease control, and we've decimated our services.''

The Massachusetts program is staged by the nonprofit
\href{https://www.pih.org/}{Partners in Health}, whose doctors have led
responses to infectious disease --- Ebola, Zika, drug-resistant
tuberculosis, cholera and typhoid fever, among others --- in the world's
poorest countries.

It is built around one-on-one telephone interviews of newly diagnosed
patients and their contacts, so that subjects must answer the phone when
it rings. Paul Farmer, a physician-anthropologist and one of the group's
founders, said there was no substitute for the bond of trust formed by a
human contact tracer.

``Somebody needs to say to people who are worried and not feeling well,
`We got you,''' he said. ```If this is Covid-19, we got you. And we'll
look out for your contacts, your spouse and your children.' And I think
that's another thing you can do remotely or virtually, is reassure
people that there is no reason to believe everything is lost.''

\includegraphics{https://static01.nyt.com/images/2020/04/15/us/15virus-contacttracing-haiti2/merlin_10961234_7c8a6fec-31d8-4195-8fab-233996edbcf7-articleLarge.jpg?quality=75\&auto=webp\&disable=upscale}

Dr. Farmer, 60, whose
\href{https://www.nytimes.com/2017/10/05/movies/bending-the-arc-review-paul-farmer-partners-in-health.html}{work
on tuberculosis} gave him
\href{https://www.nytimes.com/2003/09/14/books/a-season-in-hell.html}{a
kind of rock-star status} in the public health world, tried to calm the
nerves of around 80 new recruits last week, describing his own
experience investigating a tuberculosis outbreak as a medical student.

His eyeglass frames had broken, something he cheerfully blamed on the
pandemic, and he had to balance them on the bridge of his nose as he
spoke.

``Once you get in someone's space --- this is going to be different, to
the extent that it is virtual --- but sitting with folks in their homes,
the subtext of these tracings was, `Hey, we want to help you and your
family,''' he said.

The downside of human contact tracing is that it is expensive, can
overlook contacts a subject may not recall, and, some argue, is too slow
for a fast-moving virus.

``Using automation to do it, cellphones and triangulations of data, that
is the easiest and fastest way, and probably the most effective way to
do this,'' said Ranu S. Dhillon, a physician at Brigham and Women's
Hospital in Boston, who advised the government of Guinea on the Ebola
outbreak.

``If you're taking one or two days to manually figure out where someone
went, you're adding more time where people can transmit it to others,''
he said.

But human outreach is a standard public health practice, first used in
many countries to
\href{https://academic.oup.com/shm/article-abstract/9/2/195/1643760?redirectedFrom=fulltext}{seek
out sexual partners of men and women} known to be carriers of sexually
transmitted diseases.

It was gritty, solitary work.
\href{https://www.ncbi.nlm.nih.gov/pubmed/18400831}{A contact tracer who
worked in New Zealand in the 1970s} described spending her evenings in
bars and boardinghouses, tracking down subjects based on sketchy
descriptions, like, ``she goes to the hotel at Friday nights and she
drinks Southern Comforts,'' and ``Kathleen --- with a generous
superstructure.''

The work required such abnormal hours and such secrecy, the contract
tracer told the historian Antje Kampf that it was difficult to have a
social life or marriage. ``I sometimes wonder if I was chosen for the
job because I was a loner, or if it was the other way around,'' she
said.

In recent years, jolted by SARS and then by MERS, Asian countries have
poured resources into contact tracing, combining sophisticated digital
tools with large networks of public health outreach workers.

Jim Yong Kim, a co-founder of Partners in Health, who recently stepped
down as president of the World Bank, said he was struck by the contrast
between American and South Korean leaders, who, because of robust
contact tracing,
\href{https://www.nytimes.com/2020/03/23/world/asia/coronavirus-south-korea-flatten-curve.html?searchResultPosition=1}{felt
able to go on the offensive against the virus.}

\href{https://www.nytimes.com/news-event/coronavirus?action=click\&pgtype=Article\&state=default\&region=MAIN_CONTENT_3\&context=storylines_faq}{}

\hypertarget{the-coronavirus-outbreak-}{%
\subsubsection{The Coronavirus Outbreak
›}\label{the-coronavirus-outbreak-}}

\hypertarget{frequently-asked-questions}{%
\paragraph{Frequently Asked
Questions}\label{frequently-asked-questions}}

Updated July 27, 2020

\begin{itemize}
\item ~
  \hypertarget{should-i-refinance-my-mortgage}{%
  \paragraph{Should I refinance my
  mortgage?}\label{should-i-refinance-my-mortgage}}

  \begin{itemize}
  \tightlist
  \item
    \href{https://www.nytimes.com/article/coronavirus-money-unemployment.html?action=click\&pgtype=Article\&state=default\&region=MAIN_CONTENT_3\&context=storylines_faq}{It
    could be a good idea,} because mortgage rates have
    \href{https://www.nytimes.com/2020/07/16/business/mortgage-rates-below-3-percent.html?action=click\&pgtype=Article\&state=default\&region=MAIN_CONTENT_3\&context=storylines_faq}{never
    been lower.} Refinancing requests have pushed mortgage applications
    to some of the highest levels since 2008, so be prepared to get in
    line. But defaults are also up, so if you're thinking about buying a
    home, be aware that some lenders have tightened their standards.
  \end{itemize}
\item ~
  \hypertarget{what-is-school-going-to-look-like-in-september}{%
  \paragraph{What is school going to look like in
  September?}\label{what-is-school-going-to-look-like-in-september}}

  \begin{itemize}
  \tightlist
  \item
    It is unlikely that many schools will return to a normal schedule
    this fall, requiring the grind of
    \href{https://www.nytimes.com/2020/06/05/us/coronavirus-education-lost-learning.html?action=click\&pgtype=Article\&state=default\&region=MAIN_CONTENT_3\&context=storylines_faq}{online
    learning},
    \href{https://www.nytimes.com/2020/05/29/us/coronavirus-child-care-centers.html?action=click\&pgtype=Article\&state=default\&region=MAIN_CONTENT_3\&context=storylines_faq}{makeshift
    child care} and
    \href{https://www.nytimes.com/2020/06/03/business/economy/coronavirus-working-women.html?action=click\&pgtype=Article\&state=default\&region=MAIN_CONTENT_3\&context=storylines_faq}{stunted
    workdays} to continue. California's two largest public school
    districts --- Los Angeles and San Diego --- said on July 13, that
    \href{https://www.nytimes.com/2020/07/13/us/lausd-san-diego-school-reopening.html?action=click\&pgtype=Article\&state=default\&region=MAIN_CONTENT_3\&context=storylines_faq}{instruction
    will be remote-only in the fall}, citing concerns that surging
    coronavirus infections in their areas pose too dire a risk for
    students and teachers. Together, the two districts enroll some
    825,000 students. They are the largest in the country so far to
    abandon plans for even a partial physical return to classrooms when
    they reopen in August. For other districts, the solution won't be an
    all-or-nothing approach.
    \href{https://bioethics.jhu.edu/research-and-outreach/projects/eschool-initiative/school-policy-tracker/}{Many
    systems}, including the nation's largest, New York City, are
    devising
    \href{https://www.nytimes.com/2020/06/26/us/coronavirus-schools-reopen-fall.html?action=click\&pgtype=Article\&state=default\&region=MAIN_CONTENT_3\&context=storylines_faq}{hybrid
    plans} that involve spending some days in classrooms and other days
    online. There's no national policy on this yet, so check with your
    municipal school system regularly to see what is happening in your
    community.
  \end{itemize}
\item ~
  \hypertarget{is-the-coronavirus-airborne}{%
  \paragraph{Is the coronavirus
  airborne?}\label{is-the-coronavirus-airborne}}

  \begin{itemize}
  \tightlist
  \item
    The coronavirus
    \href{https://www.nytimes.com/2020/07/04/health/239-experts-with-one-big-claim-the-coronavirus-is-airborne.html?action=click\&pgtype=Article\&state=default\&region=MAIN_CONTENT_3\&context=storylines_faq}{can
    stay aloft for hours in tiny droplets in stagnant air}, infecting
    people as they inhale, mounting scientific evidence suggests. This
    risk is highest in crowded indoor spaces with poor ventilation, and
    may help explain super-spreading events reported in meatpacking
    plants, churches and restaurants.
    \href{https://www.nytimes.com/2020/07/06/health/coronavirus-airborne-aerosols.html?action=click\&pgtype=Article\&state=default\&region=MAIN_CONTENT_3\&context=storylines_faq}{It's
    unclear how often the virus is spread} via these tiny droplets, or
    aerosols, compared with larger droplets that are expelled when a
    sick person coughs or sneezes, or transmitted through contact with
    contaminated surfaces, said Linsey Marr, an aerosol expert at
    Virginia Tech. Aerosols are released even when a person without
    symptoms exhales, talks or sings, according to Dr. Marr and more
    than 200 other experts, who
    \href{https://academic.oup.com/cid/article/doi/10.1093/cid/ciaa939/5867798}{have
    outlined the evidence in an open letter to the World Health
    Organization}.
  \end{itemize}
\item ~
  \hypertarget{what-are-the-symptoms-of-coronavirus}{%
  \paragraph{What are the symptoms of
  coronavirus?}\label{what-are-the-symptoms-of-coronavirus}}

  \begin{itemize}
  \tightlist
  \item
    Common symptoms
    \href{https://www.nytimes.com/article/symptoms-coronavirus.html?action=click\&pgtype=Article\&state=default\&region=MAIN_CONTENT_3\&context=storylines_faq}{include
    fever, a dry cough, fatigue and difficulty breathing or shortness of
    breath.} Some of these symptoms overlap with those of the flu,
    making detection difficult, but runny noses and stuffy sinuses are
    less common.
    \href{https://www.nytimes.com/2020/04/27/health/coronavirus-symptoms-cdc.html?action=click\&pgtype=Article\&state=default\&region=MAIN_CONTENT_3\&context=storylines_faq}{The
    C.D.C. has also} added chills, muscle pain, sore throat, headache
    and a new loss of the sense of taste or smell as symptoms to look
    out for. Most people fall ill five to seven days after exposure, but
    symptoms may appear in as few as two days or as many as 14 days.
  \end{itemize}
\item ~
  \hypertarget{does-asymptomatic-transmission-of-covid-19-happen}{%
  \paragraph{Does asymptomatic transmission of Covid-19
  happen?}\label{does-asymptomatic-transmission-of-covid-19-happen}}

  \begin{itemize}
  \tightlist
  \item
    So far, the evidence seems to show it does. A widely cited
    \href{https://www.nature.com/articles/s41591-020-0869-5}{paper}
    published in April suggests that people are most infectious about
    two days before the onset of coronavirus symptoms and estimated that
    44 percent of new infections were a result of transmission from
    people who were not yet showing symptoms. Recently, a top expert at
    the World Health Organization stated that transmission of the
    coronavirus by people who did not have symptoms was ``very rare,''
    \href{https://www.nytimes.com/2020/06/09/world/coronavirus-updates.html?action=click\&pgtype=Article\&state=default\&region=MAIN_CONTENT_3\&context=storylines_faq\#link-1f302e21}{but
    she later walked back that statement.}
  \end{itemize}
\end{itemize}

``We're sitting back, hunkering down, waiting to see what the virus is
going to do to us,'' he said. The language of South Korean colleagues,
he said, ``was completely different from the language I was hearing in
the U.S. They were talking about the virus as if it were a person.
Telling me how tricky it was. It was the experience of chasing it
down.''

In a late-night phone call at the end of March, Dr. Kim pitched his idea
to Governor Baker,
\href{https://www.medrxiv.org/content/10.1101/2020.03.03.20030593v1}{pointing
to data from Wuhan, China}, that showed that social distancing alone
could not bring the virus's spread rate low enough to lift the current
restrictions.

``When people say you can't do that, it's too labor-intensive, it makes
no sense to me,'' he said. ``Ask all the people sheltering in place, the
70 percent of people who have lost income --- I would ask those people,
how much is it worth to us to really get on top of it? \$100 billion?
\$500 billion?''

Governor Baker, who listened to the pitch in his car that night, said he
appreciated that Dr. Kim could point to concrete experiences of battling
outbreaks in African countries and in Haiti.

``I have people reaching out to me all day with theoretical things that
we could do, God love 'em,'' he said. He said he had also heard from
``everyone who had a phone-pinging program,'' and that digital tracing
may be integrated into the state's program later, but that human
outreach was critical to reaching people without an easy way to isolate
themselves.

``You've got to be able to connect to people in some way that's
meaningful that's beyond a ping on the phone,'' he said.

Already, Massachusetts' local departments of health had been carrying
out contact tracing, assisted by 1,700 volunteers from the state's
academic public health community.

The 1,000 new jobs, announced at a news conference on April 3, set off a
deluge of applications, now numbering around 15,000. Ms. Cross, 27, who
is training to be a nutritionist, said she was so emotional about taking
part that she wept during her recorded interview.

Image

Dr. Farmer leading a training session, by video, for new contact tracers
on April 9.

``I feel like I have all this energy that I want to funnel into
prevention,'' she said.

Harvey Schwartz, 72, a retired civil rights lawyer from Ipswich, said he
sent in his application within 15 minutes of Governor Baker's
announcement, offering to work without pay.

``I've been spending two and a half hours every morning reading the
news, and getting more and more depressed,'' he said. ``This is the
antidote to that.''

``Cohort One'' of the state's program --- a group of around 80 hires,
among them graduate students, Peace Corps volunteers, medical assistants
and retired nurses --- logged on for the first time last Thursday, to
learn the nuts and bolts of the system.

Each time a citizen tests positive, the results will be immediately
shared with a case investigator through a secure database. Within two
hours, the case investigator will aim to reach the patient by phone and
compile a list of every person he or she had been in close contact with
in the 48 hours before the onset of symptoms.

The names of the contacts --- the expectation is 10 people per new case
--- will then be passed to contact tracers, who will attempt to reach
each one by telephone within 48 hours, calling back three times in
succession to signal the call's importance. For now, tracers are not
leaving messages or callback numbers.

The contact tracers, speaking from their prepared script, will inform
the contact of approximately when they were exposed, and then take an
inventory of symptoms, talk the contact through quarantine requirements,
and help arrange assistance with food or housing if the contact cannot
easily quarantine.

The new contact tracers began placing their first calls over the
weekend.

``It was a real microscopic look into the lives of people who have this
disease,'' said Mr. Schwartz, the former civil rights lawyer, who
reached around 10 people in his first eight-hour shift. To his relief,
they all seemed eager to speak to him.

``Some people were lonely,'' he said. ``Some people were glad the state
was involved. A few people named the governor by name.''

David Novak, 53, a social worker, found that his conversations were
stretching to 30 or 40 minutes apiece, as the people he contacted told
him about their difficulties maintaining distance from their family
members.

One woman told him how strange it is, after a long marriage, to sleep
away from her husband.

``This is where the human element of public health comes in,'' Mr. Novak
said. ``You can use technology to make the humans more efficient, but if
you take the humans out of it, how do you ask questions?''

``You're going to have to talk to them,'' he said.

Advertisement

\protect\hyperlink{after-bottom}{Continue reading the main story}

\hypertarget{site-index}{%
\subsection{Site Index}\label{site-index}}

\hypertarget{site-information-navigation}{%
\subsection{Site Information
Navigation}\label{site-information-navigation}}

\begin{itemize}
\tightlist
\item
  \href{https://help.nytimes.com/hc/en-us/articles/115014792127-Copyright-notice}{©~2020~The
  New York Times Company}
\end{itemize}

\begin{itemize}
\tightlist
\item
  \href{https://www.nytco.com/}{NYTCo}
\item
  \href{https://help.nytimes.com/hc/en-us/articles/115015385887-Contact-Us}{Contact
  Us}
\item
  \href{https://www.nytco.com/careers/}{Work with us}
\item
  \href{https://nytmediakit.com/}{Advertise}
\item
  \href{http://www.tbrandstudio.com/}{T Brand Studio}
\item
  \href{https://www.nytimes.com/privacy/cookie-policy\#how-do-i-manage-trackers}{Your
  Ad Choices}
\item
  \href{https://www.nytimes.com/privacy}{Privacy}
\item
  \href{https://help.nytimes.com/hc/en-us/articles/115014893428-Terms-of-service}{Terms
  of Service}
\item
  \href{https://help.nytimes.com/hc/en-us/articles/115014893968-Terms-of-sale}{Terms
  of Sale}
\item
  \href{https://spiderbites.nytimes.com}{Site Map}
\item
  \href{https://help.nytimes.com/hc/en-us}{Help}
\item
  \href{https://www.nytimes.com/subscription?campaignId=37WXW}{Subscriptions}
\end{itemize}
