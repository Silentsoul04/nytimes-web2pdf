Sections

SEARCH

\protect\hyperlink{site-content}{Skip to
content}\protect\hyperlink{site-index}{Skip to site index}

\href{https://www.nytimes.com/section/nyregion}{New York}

\href{https://myaccount.nytimes.com/auth/login?response_type=cookie\&client_id=vi}{}

\href{https://www.nytimes.com/section/todayspaper}{Today's Paper}

\href{/section/nyregion}{New York}\textbar{}Fernando Miteff, 60,
Graffiti Artist With a Generous Spirit, Dies

\url{https://nyti.ms/2xCk6eB}

\begin{itemize}
\item
\item
\item
\item
\item
\end{itemize}

\href{https://www.nytimes.com/news-event/coronavirus?action=click\&pgtype=Article\&state=default\&region=TOP_BANNER\&context=storylines_menu}{The
Coronavirus Outbreak}

\begin{itemize}
\tightlist
\item
  live\href{https://www.nytimes.com/2020/08/03/world/coronavirus-covid-19.html?action=click\&pgtype=Article\&state=default\&region=TOP_BANNER\&context=storylines_menu}{Latest
  Updates}
\item
  \href{https://www.nytimes.com/interactive/2020/us/coronavirus-us-cases.html?action=click\&pgtype=Article\&state=default\&region=TOP_BANNER\&context=storylines_menu}{Maps
  and Cases}
\item
  \href{https://www.nytimes.com/interactive/2020/science/coronavirus-vaccine-tracker.html?action=click\&pgtype=Article\&state=default\&region=TOP_BANNER\&context=storylines_menu}{Vaccine
  Tracker}
\item
  \href{https://www.nytimes.com/2020/08/02/us/covid-college-reopening.html?action=click\&pgtype=Article\&state=default\&region=TOP_BANNER\&context=storylines_menu}{College
  Reopening}
\item
  \href{https://www.nytimes.com/live/2020/08/03/business/stock-market-today-coronavirus?action=click\&pgtype=Article\&state=default\&region=TOP_BANNER\&context=storylines_menu}{Economy}
\end{itemize}

Advertisement

\protect\hyperlink{after-top}{Continue reading the main story}

Supported by

\protect\hyperlink{after-sponsor}{Continue reading the main story}

Those We've Lost

\hypertarget{fernando-miteff-60-graffiti-artist-with-a-generous-spirit-dies}{%
\section{Fernando Miteff, 60, Graffiti Artist With a Generous Spirit,
Dies}\label{fernando-miteff-60-graffiti-artist-with-a-generous-spirit-dies}}

He brought graffiti back to the subway, only this time his work would
last for only the length of a ride.

By \href{https://www.nytimes.com/by/david-gonzalez}{David Gonzalez}

\begin{itemize}
\item
  Published April 17, 2020Updated May 6, 2020
\item
  \begin{itemize}
  \item
  \item
  \item
  \item
  \item
  \end{itemize}
\end{itemize}

\includegraphics{https://static01.nyt.com/images/2020/05/07/obituaries/15Miteff/merlin_171617127_e663cf58-5fc2-404f-bcd9-033459017fa4-articleLarge.jpg?quality=75\&auto=webp\&disable=upscale}

\emph{This obituary is part of a series about people who have died in
the coronavirus pandemic. Read about others}
\href{https://www.nytimes.com/series/people-who-have-died-of-the-coronavirus}{\emph{here}}\emph{.}

Fernando Miteff liked his art so much, he gave it away.

Using the graffiti tag Nic 707, he was known for giving scraps of paper
adorned with his graceful letter designs and outlines to up-and-coming
artists to guide them, and to fans to thank them. And for the last
decade he did something most straphangers thought had vanished in the
late 1980s: He brought graffiti back to the New York City subway.

But this time, he did it by boarding a train, replacing ads with pieces
by some of the country's best-known and most influential graffiti
artists, like Taki 183, and switching them back at the end of his ride.

``I wanted to bring a new ideology to graffiti,'' he said in a
\href{https://www.nytimes.com/2015/03/02/nyregion/a-graffiti-artist-turns-a-subway-car-into-a-gallery-until-the-end-of-the-line.html}{2015
interview} about his guerrilla subway car exhibits, which he called
\href{http://www.nic707.com/}{InstaFame Phantom Art}. ``I didn't want to
leave a mark that stays. I wanted to leave an impression. As long as you
saw and remembered it, I'm happy with that.''

Mr. Miteff died on April 12 at his home in the Bronx. He was 60. The
cause was complications of Covid-19, said his younger brother Karim, who
managed his archives and was writing a book about Mr. Miteff's graffiti
career.

In a culture known for egos and arguments, Mr. Miteff prided himself on
sharing his love of the art.

``He was always giving, giving and giving,'' Karim Miteff recalled of
his brother's early years. ``He'd sit at McDonald's doodling on napkins
and pass it out. He would be more apt to give away his work than to sell
it.''

Mr. Miteff was born in Buenos Aires to Diana and Alexis Pablo Miteff, a
professional boxer who once fought Muhammad Ali and worked as a
television production manager after retiring. His parents had been
spending a year in Argentina before returning to New York.

Karim said his brother was raised in the Morrisania section of the
Bronx, where he started tagging at 12 after discovering a can of spray
paint in his home's basement. He later founded the Out to Bomb crew, a
loose-knit group of collaborators, and influenced younger artists like
Serve and his protégé, NOC167, who went on to fame.

Though he spent much of his adult life working odd jobs, including
chauffeur and standup comic, he returned to the subways in 2009 after he
and a friend had a brainstorm; it led to his InstaFame project. His easy
personality and sense of humor helped him persuade collaborators ---
from established to up-and-coming ones --- even to paint pieces on the
sidewalk outside art openings.

``He was a funny dude, but he took a lot of people under his wing,''
said
\href{https://www.nytimes.com/2010/02/05/nyregion/05graffiti.html}{Eric
Felisbret, author of ``Graffiti New York,''} a survey of the city's
graffiti history. ``He was completely into graffiti for the love of it.
All those panels he did, he could have only written NIC and that would
put him in the spotlight. Instead, he put his love for the art in the
spotlight.''

\href{https://www.nytimes.com/interactive/2020/obituaries/people-died-coronavirus-obituaries.html?action=click\&pgtype=Article\&state=default\&region=BELOW_MAIN_CONTENT\&context=covid_obits_promo}{}

\hypertarget{those-weve-lost}{%
\section{Those We've Lost}\label{those-weve-lost}}

The coronavirus pandemic has taken an incalculable death toll. This
series is designed to put names and faces to the numbers.

Read more

\includegraphics{https://static01.nyt.com/images/2020/07/30/obituaries/30Pedro/30Pedro-square640.jpg}

\hypertarget{bernaldina-josuxe9-pedro}{%
\section{Bernaldina José Pedro}\label{bernaldina-josuxe9-pedro}}

d. Boa Vista, Brazil

Leader among the Indigenous Macuxi

\includegraphics{https://static01.nyt.com/images/2020/07/31/obituaries/31Swing/merlin_175167783_8913bc90-0d64-43f3-a655-1bb1bf1601c9-square640.jpg}

\hypertarget{john-eric-swing}{%
\section{John Eric Swing}\label{john-eric-swing}}

d. Fountain Valley, Calif.

Champion of Filipino-Americans

\includegraphics{https://static01.nyt.com/images/2020/07/27/obituaries/27Victor/merlin_175001436_38b11f8e-227a-4e2c-9821-7618af9b2524-square640.jpg}

\hypertarget{victor-victor}{%
\section{Victor Victor}\label{victor-victor}}

d. Santo Domingo, Dominican Republic

Beloved musician of the Dominican Republic

\includegraphics{https://static01.nyt.com/images/2020/07/31/obituaries/31Negron/merlin_175160169_516322ae-fd23-4969-b6b2-193ced371105-square640.jpg}

\hypertarget{dr-eddie-negruxf3n}{%
\section{Dr. Eddie Negrón}\label{dr-eddie-negruxf3n}}

d. Fort Walton Beach, Fla.

Internist on Florida's Emerald Coast

\includegraphics{https://static01.nyt.com/images/2020/07/30/obituaries/30Dobson/merlin_175115928_f6b9271c-8f05-4fe1-a38a-5ca4a58f8935-square640.jpg}

\hypertarget{dobby-dobson}{%
\section{Dobby Dobson}\label{dobby-dobson}}

d. Coral Springs, Fla.

Jamaican singer and songwriter

\includegraphics{https://static01.nyt.com/images/2020/08/01/obituaries/28Gonzalez/merlin_175002771_beb57888-3951-409a-ae13-03a94b2e962e-square640.jpg}

\hypertarget{waldemar-gonzalez}{%
\section{Waldemar Gonzalez}\label{waldemar-gonzalez}}

d. White Plains, N.Y.

Teacher and social worker

Advertisement

\protect\hyperlink{after-bottom}{Continue reading the main story}

\hypertarget{site-index}{%
\subsection{Site Index}\label{site-index}}

\hypertarget{site-information-navigation}{%
\subsection{Site Information
Navigation}\label{site-information-navigation}}

\begin{itemize}
\tightlist
\item
  \href{https://help.nytimes.com/hc/en-us/articles/115014792127-Copyright-notice}{©~2020~The
  New York Times Company}
\end{itemize}

\begin{itemize}
\tightlist
\item
  \href{https://www.nytco.com/}{NYTCo}
\item
  \href{https://help.nytimes.com/hc/en-us/articles/115015385887-Contact-Us}{Contact
  Us}
\item
  \href{https://www.nytco.com/careers/}{Work with us}
\item
  \href{https://nytmediakit.com/}{Advertise}
\item
  \href{http://www.tbrandstudio.com/}{T Brand Studio}
\item
  \href{https://www.nytimes.com/privacy/cookie-policy\#how-do-i-manage-trackers}{Your
  Ad Choices}
\item
  \href{https://www.nytimes.com/privacy}{Privacy}
\item
  \href{https://help.nytimes.com/hc/en-us/articles/115014893428-Terms-of-service}{Terms
  of Service}
\item
  \href{https://help.nytimes.com/hc/en-us/articles/115014893968-Terms-of-sale}{Terms
  of Sale}
\item
  \href{https://spiderbites.nytimes.com}{Site Map}
\item
  \href{https://help.nytimes.com/hc/en-us}{Help}
\item
  \href{https://www.nytimes.com/subscription?campaignId=37WXW}{Subscriptions}
\end{itemize}
