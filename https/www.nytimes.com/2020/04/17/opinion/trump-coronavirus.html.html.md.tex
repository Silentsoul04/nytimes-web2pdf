Sections

SEARCH

\protect\hyperlink{site-content}{Skip to
content}\protect\hyperlink{site-index}{Skip to site index}

\href{https://myaccount.nytimes.com/auth/login?response_type=cookie\&client_id=vi}{}

\href{https://www.nytimes.com/section/todayspaper}{Today's Paper}

\href{/section/opinion}{Opinion}\textbar{}Trump's Dow-at-30,000 Dream
Hurt America

\href{https://nyti.ms/3bjgixu}{https://nyti.ms/3bjgixu}

\begin{itemize}
\item
\item
\item
\item
\item
\item
\end{itemize}

Advertisement

\protect\hyperlink{after-top}{Continue reading the main story}

\href{/section/opinion}{Opinion}

Supported by

\protect\hyperlink{after-sponsor}{Continue reading the main story}

\hypertarget{trumps-dow-at-30000-dream-hurt-america}{%
\section{Trump's Dow-at-30,000 Dream Hurt
America}\label{trumps-dow-at-30000-dream-hurt-america}}

Trump is Mark Twain's king, the great American con man.

\href{https://www.nytimes.com/by/roger-cohen}{\includegraphics{https://static01.nyt.com/images/2014/11/01/opinion/cohen-circular/cohen-circular-thumbLarge-v6.png}}

By \href{https://www.nytimes.com/by/roger-cohen}{Roger Cohen}

Opinion Columnist

\begin{itemize}
\item
  April 17, 2020
\item
  \begin{itemize}
  \item
  \item
  \item
  \item
  \item
  \item
  \end{itemize}
\end{itemize}

\includegraphics{https://static01.nyt.com/images/2020/04/17/opinion/17cohen/merlin_171265398_b6587f4d-909b-42a6-aace-841f71d9f429-articleLarge.jpg?quality=75\&auto=webp\&disable=upscale}

It's the money. With
\href{https://www.nytimes.com/2020/05/07/us/politics/trump-coronavirus-fact-check.html}{President
Trump}, that never changes. The Dow at 30,000 was his obsession. Get to
that number and the November election was a lock. Maybe even win with
400 Electoral College votes. A landslide!

The index came close. It was at its high of 29,551 on Feb. 12, more than
three weeks after President Xi Jinping of China, his disastrous delaying
tactics exhausted, warned that the
\href{https://www.nytimes.com/2020/05/07/us/politics/trump-coronavirus-fact-check.html}{coronavirus}
outbreak
``\href{https://apnews.com/14d7dcffa205d9022fa9ea593bb2a8c5}{must be
taken seriously.}'' A Nasdaq record high followed on Feb. 19, almost
three weeks after the World Health Organization declared a ``global
health emergency.''

``We have it totally under control.'' That was
\href{https://www.nytimes.com/2020/03/17/us/politics/trump-coronavirus.html}{Trump's
message at the time}. Jared Kushner, Trump's de facto campaign manager,
liked that. So did Steven Mnuchin, the Treasury secretary. Don't spook
the markets! Champagne on ice! Trump's path to re-election involved
getting enough Americans to say, \emph{I can't stand this guy but, hell,
I'm making money.}

This sordid calculation meant the opportunity to avert the Covid-19
disaster was lost. Warnings were ignored. Chaos prevailed, starting at
the top with a president who can no more think through a process than
feel empathy.

Effective testing was not developed. Medical supplies, masks and
protective suits were not procured. As my colleagues have reported,
Trump was furious in late February when
\href{https://www.nytimes.com/2020/04/11/us/politics/coronavirus-trump-response.html}{a
blunt warning} from a senior health official contributed to a market
dive.

``One day --- it's like a miracle --- it will disappear,'' Trump said of
the virus on Feb. 27. On March 7, guests
\href{https://www.nytimes.com/2020/03/14/us/politics/trump-coronavirus-mar-a-lago.html}{danced
in a conga line at Mar-a-Lago} as Trump hosted family and his
buddy-in-bravado, President Jair Bolsonaro of Brazil, who has called the
virus ``a little flu.''

Seven weeks later, the plague so cavalierly dismissed has taken the
lives of at least 30,000 in the United States. The number of jobless
claims has reached 22 million. The Dow has sunk. Wall Street will not
save Trump. Time for a Plan B. More on that later.

When the Pearl Harbor Commission on this American catastrophe convenes,
even Trump the perennial escape artist will not be able to slither from
history's judgment.

There's nobody left in the presidential entourage who can question his
folly. The toadying of Vice President Mike Pence captures the terror
that reigns in Trump's off-with-his-head court.

Court is the appropriate word. ``When somebody's the president of the
United States, the authority is total,''
\href{https://www.nytimes.com/2020/04/13/us/politics/trump-coronavirus-governors.html}{Trump
said this week}. Prompting Gov. Andrew Cuomo of New York to the timely
reminder, ``We don't have a king in this country.''

The thing is, Trump \emph{is} the king. He's Mark Twain's king, more
precisely. He's the great American swindler, relying on the vastness of
American space to afford him the opportunity to stay just ahead of
disaster by conjuring up one more tall story. Twain's king and duke in
``Huckleberry Finn'' --- claiming to be the dauphin of King Louis XVI of
France and the usurped Duke of Bridgewater --- lie and scam their way
down the Mississippi in the quintessentially American story.

``I felt it was a pandemic long before it was called a pandemic,'' Trump
says on March 17. Twain would have seen material in the grotesqueness of
that. The more fantastic the story, the more it muddies the waters.
Trump needs very muddy waters because he's a petrified coward.

In his whole born-on-third-base life, he has never been held accountable
for anything. And so of course he walks back from his claim of absolute
authority and tells governors to
``\href{https://www.cnn.com/2020/04/16/politics/donald-trump-reopening-guidelines-coronavirus/index.html}{call
your own shots}'' on reopening the economy. His guidelines give him a
veneer of authority without actual responsibility. No buck should stop
in the Oval Office.

In his daily TV ramblings, he tries to blame anyone and anything, the
World Health Organization being the latest. Trump's genius lies in a
sinister capacity to ignore reality and create another by getting
people's blood up in a whirlwind of chaos and distraction. That is how
he got to the Oval Office and how he could remain there.

Plan B is already evident. The disparaged virus that could sink the
market and blow up the path to victory is now the pandemic with
electoral potential. Doesn't corona mean crown, after all? It gives
Trump a daily reality TV show. It attacks cities more than rural areas,
where his vote is concentrated. It permits him to have the Internal
Revenue Service send out stimulus checks for \$1,200
\href{https://www.washingtonpost.com/politics/coming-to-your-1200-relief-check-donald-j-trumps-name/2020/04/14/071016c2-7e82-11ea-8013-1b6da0e4a2b7_story.html}{with}
\href{https://www.washingtonpost.com/politics/coming-to-your-1200-relief-check-donald-j-trumps-name/2020/04/14/071016c2-7e82-11ea-8013-1b6da0e4a2b7_story.html}{\emph{Trump's
name on them}}\emph{.} It creates potential scope for him to claim
emergency powers that allow electoral skulduggery. He knows that the
Mitch McConnell-stacked Supreme Court will either rule for him or make
sure things are slow-rolled enough to protect him. That's his ultimate
source of impunity --- not what the founders had in mind.

The other Plan B element is the great attack on China. Trump,
grotesquely, praised China's ``transparency.'' Forget that. China-U.S.
tensions will ratchet up in the next six months.

\href{https://www.axios.com/biggest-trump-super-pac-test-drives-beijingbiden-campaign-75582a64-6edc-48ec-9b38-f307816b6a32.html}{``Beijing
Biden''} is the new Trump offensive. He will hammer the presumptive
Democratic nominee for supposed weakness on China. He has some
ammunition. Trump is weak on American democracy, which is why Biden
needs to shape up right now and save it. His focus, cutting through the
Trump turbulence, needs to be more unerring than we've seen to date.

\emph{The Times is committed to publishing}
\href{https://www.nytimes.com/2019/01/31/opinion/letters/letters-to-editor-new-york-times-women.html}{\emph{a
diversity of letters}} \emph{to the editor. We'd like to hear what you
think about this or any of our articles. Here are some}
\href{https://help.nytimes.com/hc/en-us/articles/115014925288-How-to-submit-a-letter-to-the-editor}{\emph{tips}}\emph{.
And here's our email:}
\href{mailto:letters@nytimes.com}{\emph{letters@nytimes.com}}\emph{.}

\emph{Follow The New York Times Opinion section on}
\href{https://www.facebook.com/nytopinion}{\emph{Facebook}}\emph{,}
\href{http://twitter.com/NYTOpinion}{\emph{Twitter (@NYTopinion)}}
\emph{and}
\href{https://www.instagram.com/nytopinion/}{\emph{Instagram}}\emph{.}

Advertisement

\protect\hyperlink{after-bottom}{Continue reading the main story}

\hypertarget{site-index}{%
\subsection{Site Index}\label{site-index}}

\hypertarget{site-information-navigation}{%
\subsection{Site Information
Navigation}\label{site-information-navigation}}

\begin{itemize}
\tightlist
\item
  \href{https://help.nytimes.com/hc/en-us/articles/115014792127-Copyright-notice}{©~2020~The
  New York Times Company}
\end{itemize}

\begin{itemize}
\tightlist
\item
  \href{https://www.nytco.com/}{NYTCo}
\item
  \href{https://help.nytimes.com/hc/en-us/articles/115015385887-Contact-Us}{Contact
  Us}
\item
  \href{https://www.nytco.com/careers/}{Work with us}
\item
  \href{https://nytmediakit.com/}{Advertise}
\item
  \href{http://www.tbrandstudio.com/}{T Brand Studio}
\item
  \href{https://www.nytimes.com/privacy/cookie-policy\#how-do-i-manage-trackers}{Your
  Ad Choices}
\item
  \href{https://www.nytimes.com/privacy}{Privacy}
\item
  \href{https://help.nytimes.com/hc/en-us/articles/115014893428-Terms-of-service}{Terms
  of Service}
\item
  \href{https://help.nytimes.com/hc/en-us/articles/115014893968-Terms-of-sale}{Terms
  of Sale}
\item
  \href{https://spiderbites.nytimes.com}{Site Map}
\item
  \href{https://help.nytimes.com/hc/en-us}{Help}
\item
  \href{https://www.nytimes.com/subscription?campaignId=37WXW}{Subscriptions}
\end{itemize}
