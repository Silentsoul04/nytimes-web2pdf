Sections

SEARCH

\protect\hyperlink{site-content}{Skip to
content}\protect\hyperlink{site-index}{Skip to site index}

\href{https://www.nytimes.com/section/world/asia}{Asia Pacific}

\href{https://myaccount.nytimes.com/auth/login?response_type=cookie\&client_id=vi}{}

\href{https://www.nytimes.com/section/todayspaper}{Today's Paper}

\href{/section/world/asia}{Asia Pacific}\textbar{}Powered by Fear,
Indians Embrace Coronavirus Lockdown

\url{https://nyti.ms/3akqjck}

\begin{itemize}
\item
\item
\item
\item
\item
\end{itemize}

\href{https://www.nytimes.com/news-event/coronavirus?action=click\&pgtype=Article\&state=default\&region=TOP_BANNER\&context=storylines_menu}{The
Coronavirus Outbreak}

\begin{itemize}
\tightlist
\item
  live\href{https://www.nytimes.com/2020/08/02/world/coronavirus-updates.html?action=click\&pgtype=Article\&state=default\&region=TOP_BANNER\&context=storylines_menu}{Latest
  Updates}
\item
  \href{https://www.nytimes.com/interactive/2020/us/coronavirus-us-cases.html?action=click\&pgtype=Article\&state=default\&region=TOP_BANNER\&context=storylines_menu}{Maps
  and Cases}
\item
  \href{https://www.nytimes.com/interactive/2020/science/coronavirus-vaccine-tracker.html?action=click\&pgtype=Article\&state=default\&region=TOP_BANNER\&context=storylines_menu}{Vaccine
  Tracker}
\item
  \href{https://www.nytimes.com/interactive/2020/07/29/us/schools-reopening-coronavirus.html?action=click\&pgtype=Article\&state=default\&region=TOP_BANNER\&context=storylines_menu}{What
  School May Look Like}
\item
  \href{https://www.nytimes.com/live/2020/07/31/business/stock-market-today-coronavirus?action=click\&pgtype=Article\&state=default\&region=TOP_BANNER\&context=storylines_menu}{Economy}
\end{itemize}

Advertisement

\protect\hyperlink{after-top}{Continue reading the main story}

Supported by

\protect\hyperlink{after-sponsor}{Continue reading the main story}

\hypertarget{powered-by-fear-indians-embrace-coronavirus-lockdown}{%
\section{Powered by Fear, Indians Embrace Coronavirus
Lockdown}\label{powered-by-fear-indians-embrace-coronavirus-lockdown}}

Volunteers set up their own roadblocks. Neighborhoods set their own
limits. The efforts could help the fractious country battle the outbreak
but risk also stoking divisions and xenophobia.

\includegraphics{https://static01.nyt.com/images/2020/04/17/world/00virus-india-lockdown-1/merlin_171627354_baf5f485-f107-4423-9eb2-728c4af20a77-articleLarge.jpg?quality=75\&auto=webp\&disable=upscale}

\href{https://www.nytimes.com/by/jeffrey-gettleman}{\includegraphics{https://static01.nyt.com/images/2018/10/10/multimedia/author-jeffrey-gettleman/author-jeffrey-gettleman-thumbLarge.png}}\href{https://www.nytimes.com/by/suhasini-raj}{\includegraphics{https://static01.nyt.com/images/2019/11/22/reader-center/author-Suhasini-Raj/author-Suhasini-Raj-thumbLarge.png}}

By \href{https://www.nytimes.com/by/jeffrey-gettleman}{Jeffrey
Gettleman} and \href{https://www.nytimes.com/by/suhasini-raj}{Suhasini
Raj}

\begin{itemize}
\item
  April 19, 2020
\item
  \begin{itemize}
  \item
  \item
  \item
  \item
  \item
  \end{itemize}
\end{itemize}

BHOND, India --- On the long, straight road into the farming village of
Bhond, past fields of tomatoes, eggplant and rust-colored wheat, stood a
police barricade, an incongruous sight for a settlement so small and
remote.

Beyond the sweating police officers lay a second line of defense.
Villagers armed with sticks, their faces covered with fraying bandannas,
blocked the road. Fearful of the spread of the coronavirus, they were
determined to enforce the government's stay-at-home orders and keep
outsiders from entering their hamlet.

No one is paying these men. They are out here all day, every day, under
the withering sun, even as the farms behind them collapse under debt.

``Police or no police, we will continue,'' said Mubarik Khan, a tomato
farmer who has been guarding the gates to Bhond, about 50 miles from New
Delhi, for the past three weeks. ``I'm worried, we're all worried, but I
feel a sense of duty to be out here.''

A fractious country of 1.3 billion people where it has long been
difficult to get individuals and communities to follow the rules, India
has pursued its coronavirus lockdown --- the world's largest --- with
remarkable zeal.

People aren't just dutifully following the law. Many are going above and
beyond it. Volunteer virus patrol squads are popping up everywhere,
casting an extra net of vigilance over the entire country. Neighborhoods
are imposing extra rules and sealing themselves off.

\includegraphics{https://static01.nyt.com/images/2020/04/17/world/00virus-india-lockdown-2/merlin_171627318_4ef32d97-8083-44cd-a105-eacdb2b041dc-articleLarge.jpg?quality=75\&auto=webp\&disable=upscale}

The volunteer efforts could help India protect its people from the
pandemic, given the state of most Indian hospitals, the enormous
population and living conditions like packed slums that leave its people
particularly susceptible to outbreaks.

India has reported about 16,000 confirmed infections and 500 deaths, far
less per capita than many richer countries. But its testing rates are
also lower, and some health experts believe the virus may be lurking
here and there, undetected.

Many Indians are falling in line because they fear falling ill in a
country with a weak health care system offering treatments they cannot
afford. But the popularity of India's prime minister, Narendra Modi,
explains part of the obedience. For many people here, this is Mr. Modi's
lockdown, and what he says goes. His government is India's most powerful
in decades, so many Indians are scared to break his rules.

\hypertarget{latest-updates-global-coronavirus-outbreak}{%
\section{\texorpdfstring{\href{https://www.nytimes.com/2020/08/01/world/coronavirus-covid-19.html?action=click\&pgtype=Article\&state=default\&region=MAIN_CONTENT_1\&context=storylines_live_updates}{Latest
Updates: Global Coronavirus
Outbreak}}{Latest Updates: Global Coronavirus Outbreak}}\label{latest-updates-global-coronavirus-outbreak}}

Updated 2020-08-02T17:52:35.962Z

\begin{itemize}
\tightlist
\item
  \href{https://www.nytimes.com/2020/08/01/world/coronavirus-covid-19.html?action=click\&pgtype=Article\&state=default\&region=MAIN_CONTENT_1\&context=storylines_live_updates\#link-34047410}{The
  U.S. reels as July cases more than double the total of any other
  month.}
\item
  \href{https://www.nytimes.com/2020/08/01/world/coronavirus-covid-19.html?action=click\&pgtype=Article\&state=default\&region=MAIN_CONTENT_1\&context=storylines_live_updates\#link-780ec966}{Top
  U.S. officials work to break an impasse over the federal jobless
  benefit.}
\item
  \href{https://www.nytimes.com/2020/08/01/world/coronavirus-covid-19.html?action=click\&pgtype=Article\&state=default\&region=MAIN_CONTENT_1\&context=storylines_live_updates\#link-2bc8948}{Its
  outbreak untamed, Melbourne goes into even greater lockdown.}
\end{itemize}

\href{https://www.nytimes.com/2020/08/01/world/coronavirus-covid-19.html?action=click\&pgtype=Article\&state=default\&region=MAIN_CONTENT_1\&context=storylines_live_updates}{See
more updates}

More live coverage:
\href{https://www.nytimes.com/live/2020/07/31/business/stock-market-today-coronavirus?action=click\&pgtype=Article\&state=default\&region=MAIN_CONTENT_1\&context=storylines_live_updates}{Markets}

Praising his countrymen for behaving like a
\href{https://www.narendramodi.in/text-of-pm-s-address-to-the-nation-549264}{``disciplined
soldier,''} Mr. Modi has tried to cultivate a sense of fraternity under
the lockdown. Recently he asked all Indians to
\href{https://www.theweek.in/news/india/2020/03/23/covid-19-india-observes-janta-curfew-pm-modi-says-long-battle-ahead.html}{stand
in their doorways at a certain time and clap and make noise}. Likewise
for a
\href{https://economictimes.indiatimes.com/news/politics-and-nation/millions-of-indians-respond-to-pms-appeal-light-candles-diyas-turn-on-mobile-phone-torches/articleshow/74997232.cms}{nationwide
candle-lighting ceremony}. In both instances, millions obeyed.

India's lockdown is nearly a month old, and Mr. Modi recently extended
it to May 3. As it grinds on, it has won praise but also elicited
concerns about overzealous enforcement, especially targeting the poor
and minorities.

Image

A police checkpoint in Mumbai last month. People across India have
followed the instructions to stay indoors.Credit...Atul Loke for The New
York Times

Lower castes are being shunned more than usual. The term ``social
distancing'' plays straight into centuries of ostracism of certain
groups who until recent times were called ``untouchable.''

Muslims, a large minority in a Hindu-dominated land, are also facing a
burst of bigotry and attacks. The Indian government keeps pointing out
that
\href{https://www.nytimes.com/2020/04/12/world/asia/india-coronavirus-muslims-bigotry.html}{an
Islamic seminary in New Delhi was responsible for spreading thousands of
infections}. Now many Indians believe that all Muslims carry a higher
risk of spreading the coronavirus.

``This is one of the problems of overzealousness,'' said Adarsh Shastri,
a politician in the Indian National Congress, the leading opposition
party. ``People get a chance to enforce the laws per their own personal
prejudice.''

As in the United States and other countries, the lockdown has snarled
the supply chain. Farmers have been cut off from their markets, and
hungry people from food.

Some of these problems have been made worse by the way lockdown rules
are interpreted. For example, produce trucks are supposed to be allowed
to pass through checkpoints. But many Indians now fear truck drivers as
virus vectors. Trucks packed with vegetables have been turned back by
police officers and volunteer guards.

In perhaps an acknowledgment that the lockdown has been especially
tight, the government plans to encourage officially on Monday the
unshackling of industries such as agriculture, rubber and tea
plantations, cargo freight and water conservation projects --- some of
which were supposed to be open anyway. The new guidance will cover only
areas without many infections.

Image

Workers fumigating a vegetable market in Mumbai last month.Credit...Atul
Loke for The New York Times

Indians across the country have followed the instructions to retreat
indoors, no matter how cramped their living spaces. One member at a time
emerges to get food, which is usually not every day, and always with a
mask on.

Still, fear keeps growing. More communities are imposing their own
measures to tighten the lockdown further and all but stop the flow of
people.

In one case, in Delhi,
\href{https://www.deccanherald.com/national/father-always-out-roaming-during-lockdown-son-calls-police-820942.html}{a
son turned in his own father} for stepping outside. In another, in West
Bengal State, some families who wanted to maintain social distancing
asked their loved ones to
\href{https://www.theweek.in/news/india/2020/03/29/returning-labourers-in-bengal-village-made-to-stay-in-tree-houses.html}{sleep
in trees}.

Neighborhood associations, especially in wealthy areas, have become
extra careful. One association in Ghaziabad, near Delhi, tried to
require all residents to download the government's official coronavirus
app on their phones, until several residents complained it was an
invasion of privacy.

In Jor Bagh, a New Delhi enclave of parks, bird song and
multimillion-dollar apartments, the leaders of the residents'
association have curtailed the movement of residents, guests and staff,
barring workers like private security guards or pizza deliverymen who
are supposed to be exempt from the restrictions.

\href{https://www.nytimes.com/news-event/coronavirus?action=click\&pgtype=Article\&state=default\&region=MAIN_CONTENT_3\&context=storylines_faq}{}

\hypertarget{the-coronavirus-outbreak-}{%
\subsubsection{The Coronavirus Outbreak
›}\label{the-coronavirus-outbreak-}}

\hypertarget{frequently-asked-questions}{%
\paragraph{Frequently Asked
Questions}\label{frequently-asked-questions}}

Updated July 27, 2020

\begin{itemize}
\item ~
  \hypertarget{should-i-refinance-my-mortgage}{%
  \paragraph{Should I refinance my
  mortgage?}\label{should-i-refinance-my-mortgage}}

  \begin{itemize}
  \tightlist
  \item
    \href{https://www.nytimes.com/article/coronavirus-money-unemployment.html?action=click\&pgtype=Article\&state=default\&region=MAIN_CONTENT_3\&context=storylines_faq}{It
    could be a good idea,} because mortgage rates have
    \href{https://www.nytimes.com/2020/07/16/business/mortgage-rates-below-3-percent.html?action=click\&pgtype=Article\&state=default\&region=MAIN_CONTENT_3\&context=storylines_faq}{never
    been lower.} Refinancing requests have pushed mortgage applications
    to some of the highest levels since 2008, so be prepared to get in
    line. But defaults are also up, so if you're thinking about buying a
    home, be aware that some lenders have tightened their standards.
  \end{itemize}
\item ~
  \hypertarget{what-is-school-going-to-look-like-in-september}{%
  \paragraph{What is school going to look like in
  September?}\label{what-is-school-going-to-look-like-in-september}}

  \begin{itemize}
  \tightlist
  \item
    It is unlikely that many schools will return to a normal schedule
    this fall, requiring the grind of
    \href{https://www.nytimes.com/2020/06/05/us/coronavirus-education-lost-learning.html?action=click\&pgtype=Article\&state=default\&region=MAIN_CONTENT_3\&context=storylines_faq}{online
    learning},
    \href{https://www.nytimes.com/2020/05/29/us/coronavirus-child-care-centers.html?action=click\&pgtype=Article\&state=default\&region=MAIN_CONTENT_3\&context=storylines_faq}{makeshift
    child care} and
    \href{https://www.nytimes.com/2020/06/03/business/economy/coronavirus-working-women.html?action=click\&pgtype=Article\&state=default\&region=MAIN_CONTENT_3\&context=storylines_faq}{stunted
    workdays} to continue. California's two largest public school
    districts --- Los Angeles and San Diego --- said on July 13, that
    \href{https://www.nytimes.com/2020/07/13/us/lausd-san-diego-school-reopening.html?action=click\&pgtype=Article\&state=default\&region=MAIN_CONTENT_3\&context=storylines_faq}{instruction
    will be remote-only in the fall}, citing concerns that surging
    coronavirus infections in their areas pose too dire a risk for
    students and teachers. Together, the two districts enroll some
    825,000 students. They are the largest in the country so far to
    abandon plans for even a partial physical return to classrooms when
    they reopen in August. For other districts, the solution won't be an
    all-or-nothing approach.
    \href{https://bioethics.jhu.edu/research-and-outreach/projects/eschool-initiative/school-policy-tracker/}{Many
    systems}, including the nation's largest, New York City, are
    devising
    \href{https://www.nytimes.com/2020/06/26/us/coronavirus-schools-reopen-fall.html?action=click\&pgtype=Article\&state=default\&region=MAIN_CONTENT_3\&context=storylines_faq}{hybrid
    plans} that involve spending some days in classrooms and other days
    online. There's no national policy on this yet, so check with your
    municipal school system regularly to see what is happening in your
    community.
  \end{itemize}
\item ~
  \hypertarget{is-the-coronavirus-airborne}{%
  \paragraph{Is the coronavirus
  airborne?}\label{is-the-coronavirus-airborne}}

  \begin{itemize}
  \tightlist
  \item
    The coronavirus
    \href{https://www.nytimes.com/2020/07/04/health/239-experts-with-one-big-claim-the-coronavirus-is-airborne.html?action=click\&pgtype=Article\&state=default\&region=MAIN_CONTENT_3\&context=storylines_faq}{can
    stay aloft for hours in tiny droplets in stagnant air}, infecting
    people as they inhale, mounting scientific evidence suggests. This
    risk is highest in crowded indoor spaces with poor ventilation, and
    may help explain super-spreading events reported in meatpacking
    plants, churches and restaurants.
    \href{https://www.nytimes.com/2020/07/06/health/coronavirus-airborne-aerosols.html?action=click\&pgtype=Article\&state=default\&region=MAIN_CONTENT_3\&context=storylines_faq}{It's
    unclear how often the virus is spread} via these tiny droplets, or
    aerosols, compared with larger droplets that are expelled when a
    sick person coughs or sneezes, or transmitted through contact with
    contaminated surfaces, said Linsey Marr, an aerosol expert at
    Virginia Tech. Aerosols are released even when a person without
    symptoms exhales, talks or sings, according to Dr. Marr and more
    than 200 other experts, who
    \href{https://academic.oup.com/cid/article/doi/10.1093/cid/ciaa939/5867798}{have
    outlined the evidence in an open letter to the World Health
    Organization}.
  \end{itemize}
\item ~
  \hypertarget{what-are-the-symptoms-of-coronavirus}{%
  \paragraph{What are the symptoms of
  coronavirus?}\label{what-are-the-symptoms-of-coronavirus}}

  \begin{itemize}
  \tightlist
  \item
    Common symptoms
    \href{https://www.nytimes.com/article/symptoms-coronavirus.html?action=click\&pgtype=Article\&state=default\&region=MAIN_CONTENT_3\&context=storylines_faq}{include
    fever, a dry cough, fatigue and difficulty breathing or shortness of
    breath.} Some of these symptoms overlap with those of the flu,
    making detection difficult, but runny noses and stuffy sinuses are
    less common.
    \href{https://www.nytimes.com/2020/04/27/health/coronavirus-symptoms-cdc.html?action=click\&pgtype=Article\&state=default\&region=MAIN_CONTENT_3\&context=storylines_faq}{The
    C.D.C. has also} added chills, muscle pain, sore throat, headache
    and a new loss of the sense of taste or smell as symptoms to look
    out for. Most people fall ill five to seven days after exposure, but
    symptoms may appear in as few as two days or as many as 14 days.
  \end{itemize}
\item ~
  \hypertarget{does-asymptomatic-transmission-of-covid-19-happen}{%
  \paragraph{Does asymptomatic transmission of Covid-19
  happen?}\label{does-asymptomatic-transmission-of-covid-19-happen}}

  \begin{itemize}
  \tightlist
  \item
    So far, the evidence seems to show it does. A widely cited
    \href{https://www.nature.com/articles/s41591-020-0869-5}{paper}
    published in April suggests that people are most infectious about
    two days before the onset of coronavirus symptoms and estimated that
    44 percent of new infections were a result of transmission from
    people who were not yet showing symptoms. Recently, a top expert at
    the World Health Organization stated that transmission of the
    coronavirus by people who did not have symptoms was ``very rare,''
    \href{https://www.nytimes.com/2020/06/09/world/coronavirus-updates.html?action=click\&pgtype=Article\&state=default\&region=MAIN_CONTENT_3\&context=storylines_faq\#link-1f302e21}{but
    she later walked back that statement.}
  \end{itemize}
\end{itemize}

``I decided to go beyond the government's mandate and imposed severe
measures to protect my community,'' said Sonny Sarna, president of the
Jor Bagh association.

Image

The distribution of food in Delhi on April 10. Strict interpretations of
the government's lockdown measures have disrupted supply
chains.Credit...Rebecca Conway for The New York Times

Some residents grumbled about the inconvenience of not having food
delivered to their doors. But those grumbles stopped after
\href{https://indianexpress.com/article/cities/delhi/delhi-pizza-coronavirus-covid-19-6364606/}{a
pizza deliveryman in another Delhi neighborhood got sick}and the
occupants of 72 homes he had recently served were put under quarantine.
Most Jor Bagh residents seem appreciative of the extra rules.

In rural areas, volunteer virus squads patrol the roads day and night.
Some carry sticks, sickles and pockets full of nails for puncturing the
tires of cars they deem suspicious.

``You can call us civil defense,'' said Monu Manesar, the head protector
of his village in Haryana State.

As a van approached, he shouted: ``Hey, stop! What are you carrying in
there?''

He peered into the van and saw boxes of bees from a honey farm.

``It's a honeycomb,'' he told the three men helping him. ``Let them
go.''

Mr. Manesar is a district coordinator for
\href{https://www.news18.com/news/india/bulandshahr-violence-bajrang-dal-leader-who-complained-of-cow-slaughter-arrested-for-cops-murder-bjp-and-vhp-men-other-accused-1959599.html}{Bajrang
Dal}, a Hindu nationalist group that over the years has been blamed for
attacks on non-Hindus. The fact that some of these virus patrol squads
include the same people who have targeted minorities in the past may
explain recent hate crimes connected to the coronavirus.

Image

Sorting tomatoes in the village of Bhond in the state of
Haryana.Credit...Rebecca Conway for The New York Times

Nearly two weeks ago, Sahimuddin, a reserve police officer and a Muslim
who, like many in India, goes by one name, was riding his motorcycle on
a rural road about 40 miles south of Delhi. A group of farmers manning a
barricade at one village questioned him and then called ahead to the
next village to be on the lookout.

As Mr. Sahimuddin approached the next village, several Hindu farmers at
a barricade threw a noose around his neck and yanked him off his bike.
They beat him viciously, nearly crushing his windpipe, Mr. Sahimuddin's
family said. Police officers corroborated their account.

He is now in a hospital, voiceless and struggling to breathe.

``You don't know how much this angers me,'' said his wife, Sameena.
``Those men had no business stopping my husband.''

In another community, in southern India, upper caste residents recently
dug a five-foot-deep trench around the homes of several Dalits, the
lowest on the caste ladder, to keep the communities separate.

On Sunday evening, possibly as a response to the growing reports of hate
crimes, Mr. Modi said on Twitter that the
\href{https://twitter.com/PMOIndia/status/1251839308085915649}{coronavirus
does not discriminate based on race, religion, caste or creed}.

``We are in this together,'' he said.

Hari Kumar, Shalini Venugopal and Sameer Yasir contributed reporting
from New Delhi.

Advertisement

\protect\hyperlink{after-bottom}{Continue reading the main story}

\hypertarget{site-index}{%
\subsection{Site Index}\label{site-index}}

\hypertarget{site-information-navigation}{%
\subsection{Site Information
Navigation}\label{site-information-navigation}}

\begin{itemize}
\tightlist
\item
  \href{https://help.nytimes.com/hc/en-us/articles/115014792127-Copyright-notice}{©~2020~The
  New York Times Company}
\end{itemize}

\begin{itemize}
\tightlist
\item
  \href{https://www.nytco.com/}{NYTCo}
\item
  \href{https://help.nytimes.com/hc/en-us/articles/115015385887-Contact-Us}{Contact
  Us}
\item
  \href{https://www.nytco.com/careers/}{Work with us}
\item
  \href{https://nytmediakit.com/}{Advertise}
\item
  \href{http://www.tbrandstudio.com/}{T Brand Studio}
\item
  \href{https://www.nytimes.com/privacy/cookie-policy\#how-do-i-manage-trackers}{Your
  Ad Choices}
\item
  \href{https://www.nytimes.com/privacy}{Privacy}
\item
  \href{https://help.nytimes.com/hc/en-us/articles/115014893428-Terms-of-service}{Terms
  of Service}
\item
  \href{https://help.nytimes.com/hc/en-us/articles/115014893968-Terms-of-sale}{Terms
  of Sale}
\item
  \href{https://spiderbites.nytimes.com}{Site Map}
\item
  \href{https://help.nytimes.com/hc/en-us}{Help}
\item
  \href{https://www.nytimes.com/subscription?campaignId=37WXW}{Subscriptions}
\end{itemize}
