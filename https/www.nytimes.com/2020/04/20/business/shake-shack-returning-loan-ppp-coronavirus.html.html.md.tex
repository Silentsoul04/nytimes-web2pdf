Sections

SEARCH

\protect\hyperlink{site-content}{Skip to
content}\protect\hyperlink{site-index}{Skip to site index}

\href{https://www.nytimes.com/section/business}{Business}

\href{https://myaccount.nytimes.com/auth/login?response_type=cookie\&client_id=vi}{}

\href{https://www.nytimes.com/section/todayspaper}{Today's Paper}

\href{/section/business}{Business}\textbar{}`The Big Guys Get Bailed
Out': Restaurants Vie for Relief Funds

\url{https://nyti.ms/2wTU8Tz}

\begin{itemize}
\item
\item
\item
\item
\item
\end{itemize}

\href{https://www.nytimes.com/news-event/coronavirus?action=click\&pgtype=Article\&state=default\&region=TOP_BANNER\&context=storylines_menu}{The
Coronavirus Outbreak}

\begin{itemize}
\tightlist
\item
  live\href{https://www.nytimes.com/2020/08/04/world/coronavirus-covid-19.html?action=click\&pgtype=Article\&state=default\&region=TOP_BANNER\&context=storylines_menu}{Latest
  Updates}
\item
  \href{https://www.nytimes.com/interactive/2020/us/coronavirus-us-cases.html?action=click\&pgtype=Article\&state=default\&region=TOP_BANNER\&context=storylines_menu}{Maps
  and Cases}
\item
  \href{https://www.nytimes.com/interactive/2020/science/coronavirus-vaccine-tracker.html?action=click\&pgtype=Article\&state=default\&region=TOP_BANNER\&context=storylines_menu}{Vaccine
  Tracker}
\item
  \href{https://www.nytimes.com/2020/08/02/us/covid-college-reopening.html?action=click\&pgtype=Article\&state=default\&region=TOP_BANNER\&context=storylines_menu}{College
  Reopening}
\item
  \href{https://www.nytimes.com/live/2020/08/03/business/stock-market-today-coronavirus?action=click\&pgtype=Article\&state=default\&region=TOP_BANNER\&context=storylines_menu}{Economy}
\end{itemize}

Advertisement

\protect\hyperlink{after-top}{Continue reading the main story}

Supported by

\protect\hyperlink{after-sponsor}{Continue reading the main story}

\hypertarget{the-big-guys-get-bailed-out-restaurants-vie-for-relief-funds}{%
\section{`The Big Guys Get Bailed Out': Restaurants Vie for Relief
Funds}\label{the-big-guys-get-bailed-out-restaurants-vie-for-relief-funds}}

Shake Shack was among the larger companies criticized for seeking
small-business emergency loans from the federal government.

\includegraphics{https://static01.nyt.com/images/2020/04/20/business/20virus-shakeshack/merlin_171363681_0e7ea4b5-8416-44a0-abab-5c102c691f4b-articleLarge.jpg?quality=75\&auto=webp\&disable=upscale}

By \href{http://www.nytimes.com/by/david-yaffe-bellany}{David
Yaffe-Bellany}

\begin{itemize}
\item
  April 20, 2020
\item
  \begin{itemize}
  \item
  \item
  \item
  \item
  \item
  \end{itemize}
\end{itemize}

Buried deep in the
\href{https://assets.documentcloud.org/documents/20059055/final-final-cares-act.pdf}{900-page
stimulus package} that Congress passed in March, a single paragraph has
sparked an outcry from small restaurants as major chains and mom-and-pop
places alike scramble to survive a devastating financial crisis.

The provision, in a section outlining which small businesses qualify for
loans from the federal government, allowed big chains like Shake Shack,
Potbelly and Ruth's Chris Steak House to get tens of millions of dollars
while many smaller restaurants walked away with nothing when the \$349
billion fund was
\href{https://www.nytimes.com/2020/04/16/business/coronavirus-sba-loans-out-of-money.html}{exhausted
last week}. On Monday, Congress and the White House were nearing a deal
to replenish that fund with \$300 billion in additional relief.

The inequity caused widespread outrage. Independent owners said it would
create
\href{https://www.esquire.com/food-drink/restaurants/a32190986/cares-act-chains-independent-restaurants/}{a
post-pandemic landscape} in which chains dominated and small, vibrant
restaurants collapsed. Some lawmakers said the outcome had
\href{https://www.wsj.com/articles/how-ruths-chris-got-an-extra-helping-of-small-business-aid-money-11587312001}{violated
the spirit of the legislation}.

``The big guys get bailed out, and the little guys don't,'' said Danny
Abrams, who has laid off all 310 of his employees across six restaurants
he owns in New York.

On Sunday, Shake Shack, with 189 outlets and nearly 8,000 employees in
the United States, acknowledged that the Small Business Administration's
Paycheck Protection Program had been carried out unevenly, and said it
would return the \$10 million it had received.

But the potential new funding and Shake Shack's return of its money may
come too late for the thousands of independent restaurateurs across the
United States who are grasping for a lifeline.

From the beginning, the federal loan program has been plagued by
glitches, as overwhelming demand and confusion about how it would work
slowed the approval process in many industries. Banks turned away some
would-be borrowers because there were too many applicants. Other
companies lost valuable time because their bankers didn't know all the
details about how the program would function.

In the restaurant industry, small-business owners who have been forced
to close their dining rooms are concerned that the program's fine print
ended up helping large chains.

Under the terms, businesses that employ fewer than 500 people are
eligible for loans, which will be forgiven if the borrower does not lay
off workers or rehires them by June 30.

But a subsection of the legislation, under the heading ``business
concerns with more than 1 physical location,'' states that certain types
of businesses, including restaurant and hotel chains, with no more than
500 employees ``per physical location'' are also eligible.

\hypertarget{latest-updates-economy}{%
\section{\texorpdfstring{\href{https://www.nytimes.com/live/2020/08/03/business/stock-market-today-coronavirus?action=click\&pgtype=Article\&state=default\&region=MAIN_CONTENT_1\&context=storylines_live_updates}{Latest
Updates:
Economy}}{Latest Updates: Economy}}\label{latest-updates-economy}}

\href{https://www.nytimes.com/live/2020/08/03/business/stock-market-today-coronavirus?action=click\&pgtype=Article\&state=default\&region=MAIN_CONTENT_1\&context=storylines_live_updates\#the-chicago-fed-president-says-its-up-to-congress-to-save-the-economy}{13h
ago}

\href{https://www.nytimes.com/live/2020/08/03/business/stock-market-today-coronavirus?action=click\&pgtype=Article\&state=default\&region=MAIN_CONTENT_1\&context=storylines_live_updates\#the-chicago-fed-president-says-its-up-to-congress-to-save-the-economy}{The
Chicago Fed president says it's up to Congress to save the economy.}

\href{https://www.nytimes.com/live/2020/08/03/business/stock-market-today-coronavirus?action=click\&pgtype=Article\&state=default\&region=MAIN_CONTENT_1\&context=storylines_live_updates\#faa-says-boeing-has-effectively-mitigated-defects-in-the-737-max}{14h
ago}

\href{https://www.nytimes.com/live/2020/08/03/business/stock-market-today-coronavirus?action=click\&pgtype=Article\&state=default\&region=MAIN_CONTENT_1\&context=storylines_live_updates\#faa-says-boeing-has-effectively-mitigated-defects-in-the-737-max}{F.A.A.
says Boeing has `effectively mitigated' defects in the 737 Max.}

\href{https://www.nytimes.com/live/2020/08/03/business/stock-market-today-coronavirus?action=click\&pgtype=Article\&state=default\&region=MAIN_CONTENT_1\&context=storylines_live_updates\#small-businesses-got-emergency-loans-but-not-what-they-expected}{16h
ago}

\href{https://www.nytimes.com/live/2020/08/03/business/stock-market-today-coronavirus?action=click\&pgtype=Article\&state=default\&region=MAIN_CONTENT_1\&context=storylines_live_updates\#small-businesses-got-emergency-loans-but-not-what-they-expected}{Small
businesses got emergency loans, but not what they expected.}

\href{https://www.nytimes.com/live/2020/08/03/business/stock-market-today-coronavirus?action=click\&pgtype=Article\&state=default\&region=MAIN_CONTENT_1\&context=storylines_live_updates}{See
more updates}

More live coverage:
\href{https://www.nytimes.com/2020/08/04/world/coronavirus-covid-19.html?action=click\&pgtype=Article\&state=default\&region=MAIN_CONTENT_1\&context=storylines_live_updates}{Global}

Restaurant and hotel trade associations lobbied for that provision. On
March 18, the National Restaurant Association sent a letter to Congress
requesting a recovery fund specifically for its industry. But as drafts
of the legislation started circulating, it became clear that Congress
was not going to grant a restaurant-specific bailout.

So the group called members of Congress to advocate for a carve-out in
the small-business program that would make all restaurants eligible for
loans, regardless of size.

``There was no one in the industry that was calling against this at that
point --- everybody was in support of this carve-out,'' said Sean
Kennedy, the executive vice president for public affairs at the National
Restaurant Association. ``This pandemic is a tidal wave that is crashing
against every restaurant, no matter how big, small or well funded it may
be.''

Meanwhile, at lunches last month, Senate Republicans were asking one of
the architects of the federal loan program, Senator Marco Rubio of
Florida, who heads the small-business committee, why the funding did not
extend to larger companies. He responded that it was meant for small
businesses, and that he was negotiating its parameters with the
Democrats.

During those negotiations, Mr. Rubio and several colleagues from both
parties, including Senator Chuck Schumer of New York, the Democratic
leader, agreed to expand the program to cover some hospitality
establishments so more employers would have access to the loans.

\includegraphics{https://static01.nyt.com/images/2020/04/20/business/20VIRUS-SBALOANS-1/merlin_171754443_953fdb07-91ea-4c44-92ad-766309dfb9b2-articleLarge.jpg?quality=75\&auto=webp\&disable=upscale}

But the program was not large enough to fund everyone, pitting companies
of different sizes from various industries against one another in a race
for cash. In an open letter on Sunday, Shake Shack's chairman, Danny
Meyer, and its chief executive, Randy Garutti, said the chain had
decided to return the government loan after securing additional capital
through an equity transaction.

``We're thankful for that, and we've decided to immediately return the
entire \$10 million P.P.P. loan we received last week to the S.B.A. so
that those restaurants who need it most can get it,'' the letter said.

Potbelly, a chain of 400 restaurants, also received \$10 million. And
the parent company of Ruth's Chris Steak House, Ruth's Hospitality,
which has more than 5,000 employees, landed \$20 million by seeking
loans for two separate subsidiaries. Representatives for Potbelly and
Ruth's Hospitality did not respond to requests for comment.

After those loans became public last week in securities filings, chefs,
food critics and restaurant owners across the country
\href{https://www.esquire.com/food-drink/restaurants/a32190986/cares-act-chains-independent-restaurants/}{expressed
outrage} that a federal program intended to help small businesses was
channeling funds to corporate chains with other sources of funding.

``There was a lot of anger and frustration,'' said Andrew Rigie,
executive director of the New York City Hospitality Alliance. ``We need
to get the money into the hands of independent restaurant owners.''

Mr. Abrams, the restaurant owner in New York, said the episode recalled
the 2008 financial crisis, when big banks were bailed out by the federal
government while small businesses suffered. ``It felt a little bit like
déjà vu,'' he said.

As soon as the loans became available, Mr. Abrams said, he sent the
relevant documents to his bank, Capital One, to start applying. But
Capital One was unable to process his application before the funds were
exhausted. ``It just doesn't seem equitable,'' Mr. Abrams said.

Practically overnight, the pandemic has upended restaurants across the
United States. Many large chains and well-funded restaurant groups have
the resources to ride out a protracted shutdown. But independent
restaurants, which make up about two-thirds of the American dining
landscape, may not survive.

Image

Many restaurants, forced to close their dining rooms, are trying to get
by with delivery or takeout.Credit...Brittainy Newman/The New York Times

Some restaurants have tried to continue as delivery or grocery
operations, but the economics of those models are difficult to master.
Since March, eight million restaurant employees, or two-thirds of the
work force, have been laid off or furloughed, according to the National
Restaurant Association. And the industry has lost \$30 billion, with an
additional \$50 billion expected to disappear by the end of April.

While industry leaders have urged unity, a division between the haves
and the have-nots has quickly emerged. In contrast to struggling
mom-and-pop restaurants, fast-food companies like McDonald's and Burger
King have continued to record decent sales at their drive-throughs.

The corporate muscle of those fast-food companies also put their
franchisees in an enviable position compared with most small
restaurants. Most U.S. franchisees at Restaurant Brands
International,~which owns Burger King and Popeyes,~applied for the
small-business loans, with help from ``franchisee liquidity teams'' that
walked the owners through the process, according to Jose Cil, the
company's chief executive.

``The CARES Act is several hundred pages. These are complicated,
technical regulations,'' Mr. Cil said, referring to the federal stimulus
legislation. ``So our teams are quickly becoming expert at that.''

But industry officials maintain that the stimulus package is not enough
to keep restaurants in business.

In a letter on Monday, the National Restaurant Association asked
Democratic and Republican leaders in Congress to dedicate a recovery
fund for the restaurant industry. The restaurant group is also calling
for changes to the loan program that would allow borrowers to spend more
of the funds on nonpayroll expenses and push back the date by which
employees must be rehired for loans to be forgiven.

``The restaurant industry has been the hardest hit by the coronavirus
mandates --- suffering more sales and job losses than any other industry
in the country,'' wrote Mr. Kennedy, the group's public affairs
official. ``For an industry with sales that exceed the agriculture,
airline, railroad, ground transportation and spectator sports industries
combined, a restaurant relief and recovery program is desperately
needed.''

Austin Ramzy and Jim Tankersley contributed reporting.

Advertisement

\protect\hyperlink{after-bottom}{Continue reading the main story}

\hypertarget{site-index}{%
\subsection{Site Index}\label{site-index}}

\hypertarget{site-information-navigation}{%
\subsection{Site Information
Navigation}\label{site-information-navigation}}

\begin{itemize}
\tightlist
\item
  \href{https://help.nytimes.com/hc/en-us/articles/115014792127-Copyright-notice}{©~2020~The
  New York Times Company}
\end{itemize}

\begin{itemize}
\tightlist
\item
  \href{https://www.nytco.com/}{NYTCo}
\item
  \href{https://help.nytimes.com/hc/en-us/articles/115015385887-Contact-Us}{Contact
  Us}
\item
  \href{https://www.nytco.com/careers/}{Work with us}
\item
  \href{https://nytmediakit.com/}{Advertise}
\item
  \href{http://www.tbrandstudio.com/}{T Brand Studio}
\item
  \href{https://www.nytimes.com/privacy/cookie-policy\#how-do-i-manage-trackers}{Your
  Ad Choices}
\item
  \href{https://www.nytimes.com/privacy}{Privacy}
\item
  \href{https://help.nytimes.com/hc/en-us/articles/115014893428-Terms-of-service}{Terms
  of Service}
\item
  \href{https://help.nytimes.com/hc/en-us/articles/115014893968-Terms-of-sale}{Terms
  of Sale}
\item
  \href{https://spiderbites.nytimes.com}{Site Map}
\item
  \href{https://help.nytimes.com/hc/en-us}{Help}
\item
  \href{https://www.nytimes.com/subscription?campaignId=37WXW}{Subscriptions}
\end{itemize}
