Sections

SEARCH

\protect\hyperlink{site-content}{Skip to
content}\protect\hyperlink{site-index}{Skip to site index}

\href{https://www.nytimes.com/section/health}{Health}

\href{https://myaccount.nytimes.com/auth/login?response_type=cookie\&client_id=vi}{}

\href{https://www.nytimes.com/section/todayspaper}{Today's Paper}

\href{/section/health}{Health}\textbar{}How Coronavirus Infected Some,
but Not All, in a Restaurant

\url{https://nyti.ms/2Kn7JGa}

\begin{itemize}
\item
\item
\item
\item
\item
\item
\end{itemize}

\href{https://www.nytimes.com/news-event/coronavirus?action=click\&pgtype=Article\&state=default\&region=TOP_BANNER\&context=storylines_menu}{The
Coronavirus Outbreak}

\begin{itemize}
\tightlist
\item
  live\href{https://www.nytimes.com/2020/08/01/world/coronavirus-covid-19.html?action=click\&pgtype=Article\&state=default\&region=TOP_BANNER\&context=storylines_menu}{Latest
  Updates}
\item
  \href{https://www.nytimes.com/interactive/2020/us/coronavirus-us-cases.html?action=click\&pgtype=Article\&state=default\&region=TOP_BANNER\&context=storylines_menu}{Maps
  and Cases}
\item
  \href{https://www.nytimes.com/interactive/2020/science/coronavirus-vaccine-tracker.html?action=click\&pgtype=Article\&state=default\&region=TOP_BANNER\&context=storylines_menu}{Vaccine
  Tracker}
\item
  \href{https://www.nytimes.com/interactive/2020/07/29/us/schools-reopening-coronavirus.html?action=click\&pgtype=Article\&state=default\&region=TOP_BANNER\&context=storylines_menu}{What
  School May Look Like}
\item
  \href{https://www.nytimes.com/live/2020/07/31/business/stock-market-today-coronavirus?action=click\&pgtype=Article\&state=default\&region=TOP_BANNER\&context=storylines_menu}{Economy}
\end{itemize}

Advertisement

\protect\hyperlink{after-top}{Continue reading the main story}

Supported by

\protect\hyperlink{after-sponsor}{Continue reading the main story}

\hypertarget{how-coronavirus-infected-some-but-not-all-in-a-restaurant}{%
\section{How Coronavirus Infected Some, but Not All, in a
Restaurant}\label{how-coronavirus-infected-some-but-not-all-in-a-restaurant}}

A limited study by Chinese researchers suggests the role played by air
currents in spreading the illness in enclosed spaces.

\includegraphics{https://static01.nyt.com/images/2020/04/20/science/20VIRUS-RESTAURANT/merlin_170571381_4af0869a-7d8b-4f27-87f3-2d35eca11eed-articleLarge.jpg?quality=75\&auto=webp\&disable=upscale}

\href{https://www.nytimes.com/by/kenneth-chang}{\includegraphics{https://static01.nyt.com/images/2018/02/16/multimedia/author-kenneth-chang/author-kenneth-chang-thumbLarge.jpg}}

By \href{https://www.nytimes.com/by/kenneth-chang}{Kenneth Chang}

\begin{itemize}
\item
  April 20, 2020
\item
  \begin{itemize}
  \item
  \item
  \item
  \item
  \item
  \item
  \end{itemize}
\end{itemize}

\href{https://cn.nytimes.com/health/20200421/airflow-coronavirus-restaurants/}{阅读简体中文版}\href{https://cn.nytimes.com/health/20200421/airflow-coronavirus-restaurants/zh-hant/}{閱讀繁體中文版}

In January, at a restaurant in Guangzhou, China, one diner infected with
the novel coronavirus but not yet feeling sick appeared to have spread
the disease to nine other people. One of the restaurant's
air-conditioners apparently blew the virus particles around the dining
room.

There were 73 other diners who ate that day on the same floor of the
five-story restaurant, and the good news is they did not become sick.
Neither did the eight employees who were working on the floor at the
time.

Chinese researchers described the incident in
\href{https://wwwnc.cdc.gov/eid/article/26/7/20-0764_article\#comment}{a
paper that is to be published in the July issue of the Emerging
Infectious Diseases}, a journal published by the Centers for Disease
Control and Prevention. The field study has limitations. The
researchers, for example, did not perform experiments to simulate the
airborne transmission.

That outbreak illustrates some of the challenges that restaurants will
face when they try to reopen. Ventilation systems can create complex
patterns of airflow and keep viruses aloft, so simply spacing tables six
feet apart --- the minimum distance that the C.D.C. advises you keep
from other people --- may not be sufficient to safeguard restaurant
patrons.

The social nature of dining out could increase the risk. The longer
people linger in a contaminated area, the more virus particles they
would likely inhale. Eating is also one activity that cannot be
accomplished while wearing a mask. Virus-laden droplets can be expelled
into the air
\href{https://www.nytimes.com/interactive/2020/04/14/science/coronavirus-transmission-cough-6-feet-ar-ul.html}{through
breathing and talking, not just through coughs and sneezes}.

As the C.D.C. now advises, ``Avoid large and small gatherings in private
places and public spaces, such a friend's house, parks, restaurants,
shops, or any other place.''

On the other hand, all of the people who became sick at the restaurant
in China were either at the same table as the infected person or at one
of two neighboring tables. The fact that people farther away remained
healthy is a hopeful hint that the coronavirus is primarily transmitted
through larger respiratory droplets, which fall out of the air more
quickly than smaller droplets known as aerosols, which can float for
hours.

\hypertarget{latest-updates-global-coronavirus-outbreak}{%
\section{\texorpdfstring{\href{https://www.nytimes.com/2020/08/01/world/coronavirus-covid-19.html?action=click\&pgtype=Article\&state=default\&region=MAIN_CONTENT_1\&context=storylines_live_updates}{Latest
Updates: Global Coronavirus
Outbreak}}{Latest Updates: Global Coronavirus Outbreak}}\label{latest-updates-global-coronavirus-outbreak}}

Updated 2020-08-02T06:31:57.992Z

\begin{itemize}
\tightlist
\item
  \href{https://www.nytimes.com/2020/08/01/world/coronavirus-covid-19.html?action=click\&pgtype=Article\&state=default\&region=MAIN_CONTENT_1\&context=storylines_live_updates\#link-34047410}{The
  U.S. reels as July cases more than double the total of any other
  month.}
\item
  \href{https://www.nytimes.com/2020/08/01/world/coronavirus-covid-19.html?action=click\&pgtype=Article\&state=default\&region=MAIN_CONTENT_1\&context=storylines_live_updates\#link-780ec966}{Top
  U.S. officials work to break an impasse over the federal jobless
  benefit.}
\item
  \href{https://www.nytimes.com/2020/08/01/world/coronavirus-covid-19.html?action=click\&pgtype=Article\&state=default\&region=MAIN_CONTENT_1\&context=storylines_live_updates\#link-2bc8948}{Its
  outbreak untamed, Melbourne goes into even greater lockdown.}
\end{itemize}

\href{https://www.nytimes.com/2020/08/01/world/coronavirus-covid-19.html?action=click\&pgtype=Article\&state=default\&region=MAIN_CONTENT_1\&context=storylines_live_updates}{See
more updates}

More live coverage:
\href{https://www.nytimes.com/live/2020/07/31/business/stock-market-today-coronavirus?action=click\&pgtype=Article\&state=default\&region=MAIN_CONTENT_1\&context=storylines_live_updates}{Markets}

``I think it's a well-done study with the limitations of being a field
study,'' said Werner E. Bischoff, the medical director of infection
prevention and health system epidemiology at the Wake Forest School of
Medicine in North Carolina. Dr. Bischoff was not involved with the
research.

On Jan. 24, a family went to lunch at the restaurant in Guangzhou, a
sprawling metropolis in southern China located 80 miles from Hong Kong.

The family had left Wuhan, 520 miles to the north and the hot spot of
the initial coronavirus outbreak, one day before Chinese officials
imposed a lockdown on the city and the surrounding province of Hubei to
slow the spread of the disease.

At lunch, the five members of the family --- designated Family A in the
paper --- appeared healthy. But later in the day, one of them, a
63-year-old woman, experienced a fever and a cough and went to the
hospital where she tested positive for the coronavirus.

Within two weeks, nine others who ate lunch on that floor of the
Guangzhou restaurant that day also tested positive. Four were relatives
of the first infected woman. They could have been infected outside of
the restaurant.

But for the other five, the restaurant appears to have been the source
of the virus.

\includegraphics{https://static01.nyt.com/images/2020/04/20/science/20VIRUS-RESTAURANT2/20VIRUS-RESTAURANT2-articleLarge.jpg?quality=75\&auto=webp\&disable=upscale}

Family A's table was on the west side of the 1,500-square foot dining
room, between tables where two unrelated families, B and C, were also
having lunch. Family B and Family A overlapped for a period of 53
minutes, and three of its members --- a couple and their daughter ---
became sick. Family C sat next to Family A at the other neighboring
table along the same side the room, overlapping for 73 minutes, and two
of its members --- a mother and her daughter --- became ill.

\href{https://www.nytimes.com/news-event/coronavirus?action=click\&pgtype=Article\&state=default\&region=MAIN_CONTENT_3\&context=storylines_faq}{}

\hypertarget{the-coronavirus-outbreak-}{%
\subsubsection{The Coronavirus Outbreak
›}\label{the-coronavirus-outbreak-}}

\hypertarget{frequently-asked-questions}{%
\paragraph{Frequently Asked
Questions}\label{frequently-asked-questions}}

Updated July 27, 2020

\begin{itemize}
\item ~
  \hypertarget{should-i-refinance-my-mortgage}{%
  \paragraph{Should I refinance my
  mortgage?}\label{should-i-refinance-my-mortgage}}

  \begin{itemize}
  \tightlist
  \item
    \href{https://www.nytimes.com/article/coronavirus-money-unemployment.html?action=click\&pgtype=Article\&state=default\&region=MAIN_CONTENT_3\&context=storylines_faq}{It
    could be a good idea,} because mortgage rates have
    \href{https://www.nytimes.com/2020/07/16/business/mortgage-rates-below-3-percent.html?action=click\&pgtype=Article\&state=default\&region=MAIN_CONTENT_3\&context=storylines_faq}{never
    been lower.} Refinancing requests have pushed mortgage applications
    to some of the highest levels since 2008, so be prepared to get in
    line. But defaults are also up, so if you're thinking about buying a
    home, be aware that some lenders have tightened their standards.
  \end{itemize}
\item ~
  \hypertarget{what-is-school-going-to-look-like-in-september}{%
  \paragraph{What is school going to look like in
  September?}\label{what-is-school-going-to-look-like-in-september}}

  \begin{itemize}
  \tightlist
  \item
    It is unlikely that many schools will return to a normal schedule
    this fall, requiring the grind of
    \href{https://www.nytimes.com/2020/06/05/us/coronavirus-education-lost-learning.html?action=click\&pgtype=Article\&state=default\&region=MAIN_CONTENT_3\&context=storylines_faq}{online
    learning},
    \href{https://www.nytimes.com/2020/05/29/us/coronavirus-child-care-centers.html?action=click\&pgtype=Article\&state=default\&region=MAIN_CONTENT_3\&context=storylines_faq}{makeshift
    child care} and
    \href{https://www.nytimes.com/2020/06/03/business/economy/coronavirus-working-women.html?action=click\&pgtype=Article\&state=default\&region=MAIN_CONTENT_3\&context=storylines_faq}{stunted
    workdays} to continue. California's two largest public school
    districts --- Los Angeles and San Diego --- said on July 13, that
    \href{https://www.nytimes.com/2020/07/13/us/lausd-san-diego-school-reopening.html?action=click\&pgtype=Article\&state=default\&region=MAIN_CONTENT_3\&context=storylines_faq}{instruction
    will be remote-only in the fall}, citing concerns that surging
    coronavirus infections in their areas pose too dire a risk for
    students and teachers. Together, the two districts enroll some
    825,000 students. They are the largest in the country so far to
    abandon plans for even a partial physical return to classrooms when
    they reopen in August. For other districts, the solution won't be an
    all-or-nothing approach.
    \href{https://bioethics.jhu.edu/research-and-outreach/projects/eschool-initiative/school-policy-tracker/}{Many
    systems}, including the nation's largest, New York City, are
    devising
    \href{https://www.nytimes.com/2020/06/26/us/coronavirus-schools-reopen-fall.html?action=click\&pgtype=Article\&state=default\&region=MAIN_CONTENT_3\&context=storylines_faq}{hybrid
    plans} that involve spending some days in classrooms and other days
    online. There's no national policy on this yet, so check with your
    municipal school system regularly to see what is happening in your
    community.
  \end{itemize}
\item ~
  \hypertarget{is-the-coronavirus-airborne}{%
  \paragraph{Is the coronavirus
  airborne?}\label{is-the-coronavirus-airborne}}

  \begin{itemize}
  \tightlist
  \item
    The coronavirus
    \href{https://www.nytimes.com/2020/07/04/health/239-experts-with-one-big-claim-the-coronavirus-is-airborne.html?action=click\&pgtype=Article\&state=default\&region=MAIN_CONTENT_3\&context=storylines_faq}{can
    stay aloft for hours in tiny droplets in stagnant air}, infecting
    people as they inhale, mounting scientific evidence suggests. This
    risk is highest in crowded indoor spaces with poor ventilation, and
    may help explain super-spreading events reported in meatpacking
    plants, churches and restaurants.
    \href{https://www.nytimes.com/2020/07/06/health/coronavirus-airborne-aerosols.html?action=click\&pgtype=Article\&state=default\&region=MAIN_CONTENT_3\&context=storylines_faq}{It's
    unclear how often the virus is spread} via these tiny droplets, or
    aerosols, compared with larger droplets that are expelled when a
    sick person coughs or sneezes, or transmitted through contact with
    contaminated surfaces, said Linsey Marr, an aerosol expert at
    Virginia Tech. Aerosols are released even when a person without
    symptoms exhales, talks or sings, according to Dr. Marr and more
    than 200 other experts, who
    \href{https://academic.oup.com/cid/article/doi/10.1093/cid/ciaa939/5867798}{have
    outlined the evidence in an open letter to the World Health
    Organization}.
  \end{itemize}
\item ~
  \hypertarget{what-are-the-symptoms-of-coronavirus}{%
  \paragraph{What are the symptoms of
  coronavirus?}\label{what-are-the-symptoms-of-coronavirus}}

  \begin{itemize}
  \tightlist
  \item
    Common symptoms
    \href{https://www.nytimes.com/article/symptoms-coronavirus.html?action=click\&pgtype=Article\&state=default\&region=MAIN_CONTENT_3\&context=storylines_faq}{include
    fever, a dry cough, fatigue and difficulty breathing or shortness of
    breath.} Some of these symptoms overlap with those of the flu,
    making detection difficult, but runny noses and stuffy sinuses are
    less common.
    \href{https://www.nytimes.com/2020/04/27/health/coronavirus-symptoms-cdc.html?action=click\&pgtype=Article\&state=default\&region=MAIN_CONTENT_3\&context=storylines_faq}{The
    C.D.C. has also} added chills, muscle pain, sore throat, headache
    and a new loss of the sense of taste or smell as symptoms to look
    out for. Most people fall ill five to seven days after exposure, but
    symptoms may appear in as few as two days or as many as 14 days.
  \end{itemize}
\item ~
  \hypertarget{does-asymptomatic-transmission-of-covid-19-happen}{%
  \paragraph{Does asymptomatic transmission of Covid-19
  happen?}\label{does-asymptomatic-transmission-of-covid-19-happen}}

  \begin{itemize}
  \tightlist
  \item
    So far, the evidence seems to show it does. A widely cited
    \href{https://www.nature.com/articles/s41591-020-0869-5}{paper}
    published in April suggests that people are most infectious about
    two days before the onset of coronavirus symptoms and estimated that
    44 percent of new infections were a result of transmission from
    people who were not yet showing symptoms. Recently, a top expert at
    the World Health Organization stated that transmission of the
    coronavirus by people who did not have symptoms was ``very rare,''
    \href{https://www.nytimes.com/2020/06/09/world/coronavirus-updates.html?action=click\&pgtype=Article\&state=default\&region=MAIN_CONTENT_3\&context=storylines_faq\#link-1f302e21}{but
    she later walked back that statement.}
  \end{itemize}
\end{itemize}

An air-conditioning unit next to Family C blew air in the southward
direction across all three tables; some of the air likely bounced off
the wall, back in the direction of Family C.

Because the coronavirus had not yet spread widely beyond Wuhan, public
health officials were able to trace the recent contacts of Families B
and C and determine that the restaurant was the only likely place where
they would have crossed paths with the virus.

The researchers did not state in the paper whether any of the other
diners who did not contract the coronavirus were members of the three
affected families or if they were all customers at 12 other tables. The
73 people were quarantined for 14 days and did not develop symptoms.

``We conclude that in this outbreak, droplet transmission was prompted
by air-conditioned ventilation,'' the authors wrote. ``The key factor
for infection was the direction of the airflow.''

Harvey V. Fineberg, who leads the Standing Committee on Emerging
Infectious Diseases and 21st Century Health Threats at the National
Academies of Sciences, Engineering and Medicine, described the report as
``provocative and eye-opening.''

He said restaurants should be mindful of the direction of airflow in
arranging tables. Germicidal ultraviolet lights could also be installed
to destroy floating virus particles, Dr. Fineberg said. And the paper's
findings could have implications beyond restaurants.

``It's illuminating for the kind of thing we need to keep learning about
as we try to configure safe work spaces,'' Dr. Fineberg said. ``Not just
safe restaurant and entertainment venues but where you go to work.''

\textbf{\emph{{[}}\href{http://on.fb.me/1paTQ1h}{\emph{Like the Science
Times page on Facebook.}}} ****** \emph{\textbar{} Sign up for the}
\textbf{\href{http://nyti.ms/1MbHaRU}{\emph{Science Times
newsletter.}}\emph{{]}}}

Advertisement

\protect\hyperlink{after-bottom}{Continue reading the main story}

\hypertarget{site-index}{%
\subsection{Site Index}\label{site-index}}

\hypertarget{site-information-navigation}{%
\subsection{Site Information
Navigation}\label{site-information-navigation}}

\begin{itemize}
\tightlist
\item
  \href{https://help.nytimes.com/hc/en-us/articles/115014792127-Copyright-notice}{©~2020~The
  New York Times Company}
\end{itemize}

\begin{itemize}
\tightlist
\item
  \href{https://www.nytco.com/}{NYTCo}
\item
  \href{https://help.nytimes.com/hc/en-us/articles/115015385887-Contact-Us}{Contact
  Us}
\item
  \href{https://www.nytco.com/careers/}{Work with us}
\item
  \href{https://nytmediakit.com/}{Advertise}
\item
  \href{http://www.tbrandstudio.com/}{T Brand Studio}
\item
  \href{https://www.nytimes.com/privacy/cookie-policy\#how-do-i-manage-trackers}{Your
  Ad Choices}
\item
  \href{https://www.nytimes.com/privacy}{Privacy}
\item
  \href{https://help.nytimes.com/hc/en-us/articles/115014893428-Terms-of-service}{Terms
  of Service}
\item
  \href{https://help.nytimes.com/hc/en-us/articles/115014893968-Terms-of-sale}{Terms
  of Sale}
\item
  \href{https://spiderbites.nytimes.com}{Site Map}
\item
  \href{https://help.nytimes.com/hc/en-us}{Help}
\item
  \href{https://www.nytimes.com/subscription?campaignId=37WXW}{Subscriptions}
\end{itemize}
