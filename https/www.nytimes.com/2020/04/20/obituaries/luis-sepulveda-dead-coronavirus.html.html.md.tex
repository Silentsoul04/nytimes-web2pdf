Sections

SEARCH

\protect\hyperlink{site-content}{Skip to
content}\protect\hyperlink{site-index}{Skip to site index}

\href{https://www.nytimes.com/section/obituaries}{Obituaries}

\href{https://myaccount.nytimes.com/auth/login?response_type=cookie\&client_id=vi}{}

\href{https://www.nytimes.com/section/todayspaper}{Today's Paper}

\href{/section/obituaries}{Obituaries}\textbar{}Luis Sepúlveda, Chilean
Writer Exiled by Pinochet, Dies at 70

\url{https://nyti.ms/3ez4yJ7}

\begin{itemize}
\item
\item
\item
\item
\item
\end{itemize}

\href{https://www.nytimes.com/news-event/coronavirus?action=click\&pgtype=Article\&state=default\&region=TOP_BANNER\&context=storylines_menu}{The
Coronavirus Outbreak}

\begin{itemize}
\tightlist
\item
  live\href{https://www.nytimes.com/2020/08/03/world/coronavirus-covid-19.html?action=click\&pgtype=Article\&state=default\&region=TOP_BANNER\&context=storylines_menu}{Latest
  Updates}
\item
  \href{https://www.nytimes.com/interactive/2020/us/coronavirus-us-cases.html?action=click\&pgtype=Article\&state=default\&region=TOP_BANNER\&context=storylines_menu}{Maps
  and Cases}
\item
  \href{https://www.nytimes.com/interactive/2020/science/coronavirus-vaccine-tracker.html?action=click\&pgtype=Article\&state=default\&region=TOP_BANNER\&context=storylines_menu}{Vaccine
  Tracker}
\item
  \href{https://www.nytimes.com/2020/08/02/us/covid-college-reopening.html?action=click\&pgtype=Article\&state=default\&region=TOP_BANNER\&context=storylines_menu}{College
  Reopening}
\item
  \href{https://www.nytimes.com/live/2020/08/03/business/stock-market-today-coronavirus?action=click\&pgtype=Article\&state=default\&region=TOP_BANNER\&context=storylines_menu}{Economy}
\end{itemize}

Advertisement

\protect\hyperlink{after-top}{Continue reading the main story}

Supported by

\protect\hyperlink{after-sponsor}{Continue reading the main story}

Those We've Lost

\hypertarget{luis-sepuxfalveda-chilean-writer-exiled-by-pinochet-dies-at-70}{%
\section{Luis Sepúlveda, Chilean Writer Exiled by Pinochet, Dies at
70}\label{luis-sepuxfalveda-chilean-writer-exiled-by-pinochet-dies-at-70}}

His best-known work was ``The Old Man Who Read Love Stories.'' Mr.
Sepúlveda was among the first people in Spain hospitalized with the new
coronavirus.

\includegraphics{https://static01.nyt.com/images/2020/04/22/obituaries/17sepulveda-virus-lost/merlin_171642249_e874f506-5e67-4436-9158-dca8b8892eb1-articleLarge.jpg?quality=75\&auto=webp\&disable=upscale}

\href{https://www.nytimes.com/by/raphael-minder}{\includegraphics{https://static01.nyt.com/images/2018/10/15/multimedia/author-raphael-minder/author-raphael-minder-thumbLarge.png}}

By \href{https://www.nytimes.com/by/raphael-minder}{Raphael Minder}

\begin{itemize}
\item
  April 20, 2020
\item
  \begin{itemize}
  \item
  \item
  \item
  \item
  \item
  \end{itemize}
\end{itemize}

\emph{This obituary is part of a series about people who have died in
the coronavirus pandemic. Read about others}
\href{https://www.nytimes.com/series/people-who-have-died-of-the-coronavirus}{\emph{here}}\emph{.}

MADRID --- Luis Sepúlveda, a Chilean writer whose stay among Indigenous
people in the Amazon led to his most celebrated novel and who was jailed
during the dictatorship of Augusto Pinochet, died April 16 in Oviedo,
Spain. He was 70.

The cause was the novel coronavirus, according to Tusquets, his
publishing house in Barcelona. Mr. Sepúlveda, who was hospitalized in
February, was among the first wave of people in Spain to be diagnosed
with the coronavirus.

Mr. Sepúlveda published several novels, children's stories and travel
books, and he also wrote and directed films. He acquired fame with his
novel ``The Old Man Who Read Love Stories'' (1988), which tells the
story of a man who, together with his wife, leaves his mountain village
to take part in the colonization of the Amazon.

The book was inspired by Mr. Sepúlveda's stay in the 1970s with the
region's Shuar Indigenous people. A
\href{https://www.nytimes.com/1994/05/01/books/flight-to-amazonia.html?searchResultPosition=5}{review}
in The New York Times by David Unger compared it to one of the early
works of Gabriel García Márquez.

``In its simple language and philosophical underpinnings, it is magical,
thanks to the author's skill at describing jungle life,'' Mr. Unger
wrote. Mr. Sepúlveda wrote the screenplay for a 2001 movie version
starring Richard Dreyfuss.

Mr. Sepúlveda was born on Oct. 4, 1949, in Ovalle, a small city in
central Chile. His father owned a restaurant and was a Communist
militant. His mother, who was of Mapuche Indigenous descent, worked as a
nurse. As a teenager, he joined the Communist Youth and then studied
theater at the University of Chile.

After Gen. Pinochet staged a coup and took charge of Chile in 1973, Mr.
Sepúlveda was among a large number of left-wing intellectuals and
political activists jailed by the regime.

His prison sentence was eventually turned into house arrest. He fled and
went underground, but was recaptured and sentenced again, this time to
28 years in prison. With support from Amnesty International, his
sentence was eventually changed to exile and he spent some time in the
Amazon.

In the late 1970s, Mr. Sepúlveda moved around Latin America including
Nicaragua, where he joined a leftist militant group. He lived for a time
in Germany, where he worked for Greenpeace, serving on one of its ships.

In 1997, he settled in Spain's northern region of Asturias, where he
renewed his relationship with Carmen Yáñez, a poet whom he had married
in Chile before the coup.

Ms. Yáñez had been detained and tortured by Pinochet's police, but she
eventually was granted political asylum in Sweden, where she lived until
she rejoined Mr. Sepúlveda in Spain. She and five children, Carlos,
Paulina, Max, Leon and Sebastián, survive him.

\href{https://www.nytimes.com/interactive/2020/obituaries/people-died-coronavirus-obituaries.html?action=click\&pgtype=Article\&state=default\&region=BELOW_MAIN_CONTENT\&context=covid_obits_promo}{}

\hypertarget{those-weve-lost}{%
\section{Those We've Lost}\label{those-weve-lost}}

The coronavirus pandemic has taken an incalculable death toll. This
series is designed to put names and faces to the numbers.

Read more

\includegraphics{https://static01.nyt.com/images/2020/07/30/obituaries/30Pedro/30Pedro-square640.jpg}

\hypertarget{bernaldina-josuxe9-pedro}{%
\section{Bernaldina José Pedro}\label{bernaldina-josuxe9-pedro}}

d. Boa Vista, Brazil

Leader among the Indigenous Macuxi

\includegraphics{https://static01.nyt.com/images/2020/07/31/obituaries/31Swing/merlin_175167783_8913bc90-0d64-43f3-a655-1bb1bf1601c9-square640.jpg}

\hypertarget{john-eric-swing}{%
\section{John Eric Swing}\label{john-eric-swing}}

d. Fountain Valley, Calif.

Champion of Filipino-Americans

\includegraphics{https://static01.nyt.com/images/2020/07/27/obituaries/27Victor/merlin_175001436_38b11f8e-227a-4e2c-9821-7618af9b2524-square640.jpg}

\hypertarget{victor-victor}{%
\section{Victor Victor}\label{victor-victor}}

d. Santo Domingo, Dominican Republic

Beloved musician of the Dominican Republic

\includegraphics{https://static01.nyt.com/images/2020/07/31/obituaries/31Negron/merlin_175160169_516322ae-fd23-4969-b6b2-193ced371105-square640.jpg}

\hypertarget{dr-eddie-negruxf3n}{%
\section{Dr. Eddie Negrón}\label{dr-eddie-negruxf3n}}

d. Fort Walton Beach, Fla.

Internist on Florida's Emerald Coast

\includegraphics{https://static01.nyt.com/images/2020/07/30/obituaries/30Dobson/merlin_175115928_f6b9271c-8f05-4fe1-a38a-5ca4a58f8935-square640.jpg}

\hypertarget{dobby-dobson}{%
\section{Dobby Dobson}\label{dobby-dobson}}

d. Coral Springs, Fla.

Jamaican singer and songwriter

\includegraphics{https://static01.nyt.com/images/2020/08/01/obituaries/28Gonzalez/merlin_175002771_beb57888-3951-409a-ae13-03a94b2e962e-square640.jpg}

\hypertarget{waldemar-gonzalez}{%
\section{Waldemar Gonzalez}\label{waldemar-gonzalez}}

d. White Plains, N.Y.

Teacher and social worker

Advertisement

\protect\hyperlink{after-bottom}{Continue reading the main story}

\hypertarget{site-index}{%
\subsection{Site Index}\label{site-index}}

\hypertarget{site-information-navigation}{%
\subsection{Site Information
Navigation}\label{site-information-navigation}}

\begin{itemize}
\tightlist
\item
  \href{https://help.nytimes.com/hc/en-us/articles/115014792127-Copyright-notice}{©~2020~The
  New York Times Company}
\end{itemize}

\begin{itemize}
\tightlist
\item
  \href{https://www.nytco.com/}{NYTCo}
\item
  \href{https://help.nytimes.com/hc/en-us/articles/115015385887-Contact-Us}{Contact
  Us}
\item
  \href{https://www.nytco.com/careers/}{Work with us}
\item
  \href{https://nytmediakit.com/}{Advertise}
\item
  \href{http://www.tbrandstudio.com/}{T Brand Studio}
\item
  \href{https://www.nytimes.com/privacy/cookie-policy\#how-do-i-manage-trackers}{Your
  Ad Choices}
\item
  \href{https://www.nytimes.com/privacy}{Privacy}
\item
  \href{https://help.nytimes.com/hc/en-us/articles/115014893428-Terms-of-service}{Terms
  of Service}
\item
  \href{https://help.nytimes.com/hc/en-us/articles/115014893968-Terms-of-sale}{Terms
  of Sale}
\item
  \href{https://spiderbites.nytimes.com}{Site Map}
\item
  \href{https://help.nytimes.com/hc/en-us}{Help}
\item
  \href{https://www.nytimes.com/subscription?campaignId=37WXW}{Subscriptions}
\end{itemize}
