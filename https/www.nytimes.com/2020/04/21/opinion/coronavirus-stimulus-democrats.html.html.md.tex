Sections

SEARCH

\protect\hyperlink{site-content}{Skip to
content}\protect\hyperlink{site-index}{Skip to site index}

\href{https://myaccount.nytimes.com/auth/login?response_type=cookie\&client_id=vi}{}

\href{https://www.nytimes.com/section/todayspaper}{Today's Paper}

\href{/section/opinion}{Opinion}\textbar{}Will Democrats Fold Again?

\href{https://nyti.ms/2XPW4HL}{https://nyti.ms/2XPW4HL}

\begin{itemize}
\item
\item
\item
\item
\item
\end{itemize}

Advertisement

\protect\hyperlink{after-top}{Continue reading the main story}

\href{/section/opinion}{Opinion}

Supported by

\protect\hyperlink{after-sponsor}{Continue reading the main story}

\hypertarget{will-democrats-fold-again}{%
\section{Will Democrats Fold Again?}\label{will-democrats-fold-again}}

They keep accepting flawed bills.

\href{https://www.nytimes.com/by/david-leonhardt}{\includegraphics{https://static01.nyt.com/images/2018/04/02/opinion/david-leonhardt/david-leonhardt-thumbLarge.png}}

By \href{https://www.nytimes.com/by/david-leonhardt}{David Leonhardt}

Opinion Columnist

\begin{itemize}
\item
  April 21, 2020
\item
  \begin{itemize}
  \item
  \item
  \item
  \item
  \item
  \end{itemize}
\end{itemize}

\includegraphics{https://static01.nyt.com/images/2020/04/21/opinion/21leonhardt-newsletter/merlin_171192057_147e0a4e-9dd3-4edd-8305-be94ecc94987-articleLarge.jpg?quality=75\&auto=webp\&disable=upscale}

\emph{This article is part of David Leonhardt's newsletter. You can}
\href{https://www.nytimes.com/newsletters/opiniontoday?action=click\&module=Intentional\&pgtype=Article}{\emph{sign
up here}} \emph{to receive it each weekday.}

Congressional Democrats believe that fiscal aid to states is one of the
most effective forms of economic stimulus, and many economists
\href{https://www.economy.com/mark-zandi/documents/Stimulus-Impact-2008.pdf}{agree}.
But Democrats have not insisted on a major aid program for states.
Instead, they may be on the verge of passing a fourth coronavirus bill
without such a program.

Congressional Democrats also believe that the 2020 elections could be
\href{https://www.nytimes.com/2020/04/03/opinion/wisconsin-primary-coronavirus.html}{a
chaotic mess}, in which many people are forced to choose between voting
and protecting their health. Election experts
\href{https://electionlawblog.org/?p=110034}{agree}. Yet Democrats have
not insisted on the money and new rules necessary to hold elections
safely during a pandemic.

And many Democrats believe that the United States badly needs an
aggressive national program of
\href{https://www.nytimes.com/2020/04/06/health/coronavirus-testing-us.html}{virus
testing}. The first three coronavirus bills passed by Congress and
signed by President Trump didn't include a national testing program.

Yesterday, congressional Democrats began to show more backbone and said
they
\href{https://www.nytimes.com/2020/04/20/us/politics/congress-coronavirus-bill.html}{would
not pass} a new coronavirus bill --- organized around the expansion of a
small-business loan program --- unless it included a national testing
program.

We'll see if they stick to it.

Democrats don't control the Senate or the White House, so obviously they
can't dictate every aspect of the coronavirus response. Given this
reality, they have won some important concessions in recent weeks,
including
\href{https://www.nytimes.com/2020/03/25/us/politics/coronavirus-senate-deal.html}{much
more help} for unemployed workers. But Democrats have more political
leverage than they've been
\href{https://www.nytimes.com/2020/04/06/opinion/coronavirus-stimulus-democrats.html}{willing
to use} so far.

When Barack Obama was president, congressional Republicans recognized
that the president's party would take much of the blame for problems in
the country. As a result, they often adopted a tough (and sometimes
\href{https://www.simonandschuster.com/books/The-Cynic/Alec-MacGillis/9781501112034}{cynical})
negotiating stance.

During the Trump presidency, Democrats have not been willing to be so
tough, even in the service of policies many nonpartisan observers
believe would help the country. Democrats insist that they will have
more chances --- that the scale of the virus crisis means that Trump and
congressional Republicans will be desperate to pass yet another bill in
coming weeks and that state aid and election protection can be added to
those bills.

Perhaps. At some point, though, Democrats will have to decide when
they're going to stop accepting bills that they know fall short of what
the country needs.

\textbf{For more:}

\begin{itemize}
\item
  \href{https://nymag.com/intelligencer/2020/04/trump-coronavirus-open-state-governors-protests.html}{Jonathan
  Chait}, New York magazine: Trump ``is hurling all responsibility to
  state governments, leaving it to them to devise effective tests and to
  decide when to relax social distancing. At the same time, he is
  starving them of the resources to handle the job \ldots{} Trump's
  seemingly paradoxical stance is an attempt to hoard credit and shirk
  risk, straddling the demands of his business allies with the pleas of
  his public-health advisers.''
\item
  \href{https://www.washingtonpost.com/opinions/2020/04/20/war-against-states/}{Paul
  Waldman}, The Washington Post: ``State and local budgets are suddenly
  facing all kinds of new costs related to the pandemic, while at the
  same time tax revenues have fallen off a cliff. If they don't get
  help, they'll have to start laying people off and slashing state
  services, which will only make the recession deeper and longer.''
\item
  \href{https://www.businessinsider.com/trump-treatment-state-governors-governments-coronavirus-response-getting-worse-2020-4}{Linette
  Lopez}, Business Insider: ``While the federal government may have some
  guidelines for how states open up, it is not doing anything to ease
  that transition.''
\end{itemize}

\emph{If you are not a subscriber to this newsletter, you can}
\href{https://www.nytimes.com/newsletters/david-leonhardt}{\emph{subscribe
here}}\emph{. You can also join me on}
\href{https://twitter.com/DLeonhardt}{\emph{Twitter (@DLeonhardt)}}
\emph{and}
\href{https://www.facebook.com/DavidRLeonhardt/}{\emph{Facebook}}\emph{.}

\emph{Follow The New York Times Opinion section on}
\href{https://www.facebook.com/nytopinion}{\emph{Facebook}}\emph{,}
\href{http://twitter.com/NYTOpinion}{\emph{Twitter (@NYTopinion)}}
\emph{and}
\href{https://www.instagram.com/nytopinion/}{\emph{Instagram}}\emph{.}

Advertisement

\protect\hyperlink{after-bottom}{Continue reading the main story}

\hypertarget{site-index}{%
\subsection{Site Index}\label{site-index}}

\hypertarget{site-information-navigation}{%
\subsection{Site Information
Navigation}\label{site-information-navigation}}

\begin{itemize}
\tightlist
\item
  \href{https://help.nytimes.com/hc/en-us/articles/115014792127-Copyright-notice}{©~2020~The
  New York Times Company}
\end{itemize}

\begin{itemize}
\tightlist
\item
  \href{https://www.nytco.com/}{NYTCo}
\item
  \href{https://help.nytimes.com/hc/en-us/articles/115015385887-Contact-Us}{Contact
  Us}
\item
  \href{https://www.nytco.com/careers/}{Work with us}
\item
  \href{https://nytmediakit.com/}{Advertise}
\item
  \href{http://www.tbrandstudio.com/}{T Brand Studio}
\item
  \href{https://www.nytimes.com/privacy/cookie-policy\#how-do-i-manage-trackers}{Your
  Ad Choices}
\item
  \href{https://www.nytimes.com/privacy}{Privacy}
\item
  \href{https://help.nytimes.com/hc/en-us/articles/115014893428-Terms-of-service}{Terms
  of Service}
\item
  \href{https://help.nytimes.com/hc/en-us/articles/115014893968-Terms-of-sale}{Terms
  of Sale}
\item
  \href{https://spiderbites.nytimes.com}{Site Map}
\item
  \href{https://help.nytimes.com/hc/en-us}{Help}
\item
  \href{https://www.nytimes.com/subscription?campaignId=37WXW}{Subscriptions}
\end{itemize}
