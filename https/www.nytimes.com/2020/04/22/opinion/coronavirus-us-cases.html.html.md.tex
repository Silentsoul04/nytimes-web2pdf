Sections

SEARCH

\protect\hyperlink{site-content}{Skip to
content}\protect\hyperlink{site-index}{Skip to site index}

\href{https://myaccount.nytimes.com/auth/login?response_type=cookie\&client_id=vi}{}

\href{https://www.nytimes.com/section/todayspaper}{Today's Paper}

\href{/section/opinion}{Opinion}\textbar{}Another Problem With the U.S.
Virus Response, in a Chart

\href{https://nyti.ms/2VsbJex}{https://nyti.ms/2VsbJex}

\begin{itemize}
\item
\item
\item
\item
\item
\end{itemize}

Advertisement

\protect\hyperlink{after-top}{Continue reading the main story}

\href{/section/opinion}{Opinion}

Supported by

\protect\hyperlink{after-sponsor}{Continue reading the main story}

\hypertarget{another-problem-with-the-us-virus-response-in-a-chart}{%
\section{Another Problem With the U.S. Virus Response, in a
Chart}\label{another-problem-with-the-us-virus-response-in-a-chart}}

A very slow decline from the peak.

\href{https://www.nytimes.com/by/david-leonhardt}{\includegraphics{https://static01.nyt.com/images/2018/04/02/opinion/david-leonhardt/david-leonhardt-thumbLarge.png}}

By \href{https://www.nytimes.com/by/david-leonhardt}{David Leonhardt}

Opinion Columnist

\begin{itemize}
\item
  April 22, 2020
\item
  \begin{itemize}
  \item
  \item
  \item
  \item
  \item
  \end{itemize}
\end{itemize}

\includegraphics{https://static01.nyt.com/images/2020/04/22/opinion/22leonhardt-chart/22leonhardt-chart-articleLarge-v2.png?quality=75\&auto=webp\&disable=upscale}

\emph{This article is part of David Leonhardt's newsletter. You can}
\href{https://www.nytimes.com/newsletters/opiniontoday?action=click\&module=Intentional\&pgtype=Article}{\emph{sign
up here}} \emph{to receive it each weekday.}

If you've been following the charts
\href{https://www.nytimes.com/interactive/2020/us/coronavirus-us-cases.html}{showing}
the number of new coronavirus cases in the United States each day, you
may have noticed a worrisome pattern in the last few days.

The number of new cases appears to have peaked about a week and a half
ago. But the decline since then has been very modest. There are still
about 30,000 Americans being diagnosed each day. The seven-day moving
average of new cases --- a measure that smooths out daily fluctuations
--- has declined only 2 percent since it peaked 11 days ago.

As you can see in the chart above, that's not typical. In other
countries, the number of new cases has usually declined much more
sharply after peaking.

Why? It's impossible to know for certain with a virus as complex and
unknown as this one. But there is an obvious potential cause: Many
political leaders in the United States, including President Trump, are
not following the advice of public health experts.

Those experts have urged a range of measures: continued social
distancing until the number of cases falls further; a rapid expansion of
virus testing; and planning an extensive program of ``contact tracing''
and quarantining, to allow for gradual reopening. The United States is
taking some of these steps, but only some.

In the meantime, Trump is also encouraging protests that defy social
distancing, which will only spread the virus. And he is calling for an
across-the-board reduction in immigration, which will do little if
anything to reduce the virus's spread.

Some states are also taking counterproductive steps. Wisconsin's
insistence on holding an election in the midst of a pandemic has
evidently led to some additional cases, as
\href{https://www.jsonline.com/story/news/local/milwaukee/2020/04/20/coronavirus-milwaukee-7-new-cases-may-tied-april-7-election/5168669002/}{Alison
Dirr of the Milwaukee Journal Sentinel} reported.

So maybe we shouldn't be surprised that the United States isn't reducing
the spread of the virus as well as many other countries. We don't seem
to be trying as hard.

\emph{If you are not a subscriber to this newsletter, you can}
\href{https://www.nytimes.com/newsletters/david-leonhardt}{\emph{subscribe
here}}\emph{. You can also join me on}
\href{https://twitter.com/DLeonhardt}{\emph{Twitter (@DLeonhardt)}}
\emph{and}
\href{https://www.facebook.com/DavidRLeonhardt/}{\emph{Facebook}}\emph{.}

\emph{Follow The New York Times Opinion section on}
\href{https://www.facebook.com/nytopinion}{\emph{Facebook}}\emph{,}
\href{http://twitter.com/NYTOpinion}{\emph{Twitter (@NYTopinion)}}
\emph{and}
\href{https://www.instagram.com/nytopinion/}{\emph{Instagram}}\emph{.}

Advertisement

\protect\hyperlink{after-bottom}{Continue reading the main story}

\hypertarget{site-index}{%
\subsection{Site Index}\label{site-index}}

\hypertarget{site-information-navigation}{%
\subsection{Site Information
Navigation}\label{site-information-navigation}}

\begin{itemize}
\tightlist
\item
  \href{https://help.nytimes.com/hc/en-us/articles/115014792127-Copyright-notice}{©~2020~The
  New York Times Company}
\end{itemize}

\begin{itemize}
\tightlist
\item
  \href{https://www.nytco.com/}{NYTCo}
\item
  \href{https://help.nytimes.com/hc/en-us/articles/115015385887-Contact-Us}{Contact
  Us}
\item
  \href{https://www.nytco.com/careers/}{Work with us}
\item
  \href{https://nytmediakit.com/}{Advertise}
\item
  \href{http://www.tbrandstudio.com/}{T Brand Studio}
\item
  \href{https://www.nytimes.com/privacy/cookie-policy\#how-do-i-manage-trackers}{Your
  Ad Choices}
\item
  \href{https://www.nytimes.com/privacy}{Privacy}
\item
  \href{https://help.nytimes.com/hc/en-us/articles/115014893428-Terms-of-service}{Terms
  of Service}
\item
  \href{https://help.nytimes.com/hc/en-us/articles/115014893968-Terms-of-sale}{Terms
  of Sale}
\item
  \href{https://spiderbites.nytimes.com}{Site Map}
\item
  \href{https://help.nytimes.com/hc/en-us}{Help}
\item
  \href{https://www.nytimes.com/subscription?campaignId=37WXW}{Subscriptions}
\end{itemize}
