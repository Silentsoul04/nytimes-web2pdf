Sections

SEARCH

\protect\hyperlink{site-content}{Skip to
content}\protect\hyperlink{site-index}{Skip to site index}

\href{https://www.nytimes.com/section/world/africa}{Africa}

\href{https://myaccount.nytimes.com/auth/login?response_type=cookie\&client_id=vi}{}

\href{https://www.nytimes.com/section/todayspaper}{Today's Paper}

\href{/section/world/africa}{Africa}\textbar{}`Instead of Coronavirus,
the Hunger Will Kill Us.' A Global Food Crisis Looms.

\url{https://nyti.ms/3eHELic}

\begin{itemize}
\item
\item
\item
\item
\item
\item
\end{itemize}

\href{https://www.nytimes.com/news-event/coronavirus?action=click\&pgtype=Article\&state=default\&region=TOP_BANNER\&context=storylines_menu}{The
Coronavirus Outbreak}

\begin{itemize}
\tightlist
\item
  live\href{https://www.nytimes.com/2020/08/04/world/coronavirus-covid-19.html?action=click\&pgtype=Article\&state=default\&region=TOP_BANNER\&context=storylines_menu}{Latest
  Updates}
\item
  \href{https://www.nytimes.com/interactive/2020/us/coronavirus-us-cases.html?action=click\&pgtype=Article\&state=default\&region=TOP_BANNER\&context=storylines_menu}{Maps
  and Cases}
\item
  \href{https://www.nytimes.com/interactive/2020/science/coronavirus-vaccine-tracker.html?action=click\&pgtype=Article\&state=default\&region=TOP_BANNER\&context=storylines_menu}{Vaccine
  Tracker}
\item
  \href{https://www.nytimes.com/2020/08/02/us/covid-college-reopening.html?action=click\&pgtype=Article\&state=default\&region=TOP_BANNER\&context=storylines_menu}{College
  Reopening}
\item
  \href{https://www.nytimes.com/live/2020/08/03/business/stock-market-today-coronavirus?action=click\&pgtype=Article\&state=default\&region=TOP_BANNER\&context=storylines_menu}{Economy}
\end{itemize}

Advertisement

\protect\hyperlink{after-top}{Continue reading the main story}

Supported by

\protect\hyperlink{after-sponsor}{Continue reading the main story}

\hypertarget{instead-of-coronavirus-the-hunger-will-kill-us-a-global-food-crisis-looms}{%
\section{`Instead of Coronavirus, the Hunger Will Kill Us.' A Global
Food Crisis
Looms.}\label{instead-of-coronavirus-the-hunger-will-kill-us-a-global-food-crisis-looms}}

The world has never faced a hunger emergency like this, experts say. It
could double the number of people facing acute hunger to 265 million by
the end of this year.

\includegraphics{https://static01.nyt.com/images/2020/04/21/world/00virus-hunger/merlin_171373365_5c38a120-dfa2-4e21-a6cf-aefcefd2e49c-articleLarge.jpg?quality=75\&auto=webp\&disable=upscale}

By \href{https://www.nytimes.com/by/abdi-latif-dahir}{Abdi Latif Dahir}

\begin{itemize}
\item
  Published April 22, 2020Updated May 13, 2020
\item
  \begin{itemize}
  \item
  \item
  \item
  \item
  \item
  \item
  \end{itemize}
\end{itemize}

\href{https://cn.nytimes.com/world/20200423/coronavirus-hunger-crisis/}{阅读简体中文版}\href{https://cn.nytimes.com/world/20200423/coronavirus-hunger-crisis/zh-hant/}{閱讀繁體中文版}

NAIROBI, Kenya --- In the largest slum in Kenya's capital, people
desperate to eat set off a stampede during a recent giveaway of flour
and cooking oil, leaving scores injured and two people dead.

In India, thousands of workers are lining up twice a day for bread and
fried vegetables to keep
\href{https://www.nytimes.com/2020/05/06/us/politics/coronavirus-hunger-food-stamps.html}{hunger}
at bay.

And across Colombia, poor households are hanging red clothing and flags
from their windows and balconies as a sign that they are hungry.

``We don't have any money, and now we need to survive,'' said Pauline
Karushi, who lost her job at a jewelry business in Nairobi, and lives in
two rooms with her child and four other relatives. ``That means not
eating much.''

The
\href{https://www.nytimes.com/2020/05/13/us/politics/coronavirus-hunger-food-banks.html}{coronavirus
pandemic} has brought
\href{https://www.nytimes.com/2020/05/13/us/politics/coronavirus-hunger-food-banks.html}{hunger}
to millions of people around the world. National lockdowns and social
distancing measures are drying up work and incomes, and are likely to
disrupt agricultural production and supply routes --- leaving millions
to worry how they will get enough to eat.

\includegraphics{https://static01.nyt.com/images/2020/04/21/world/00virus-hunger2/00virus-hunger2-articleLarge-v2.jpg?quality=75\&auto=webp\&disable=upscale}

The coronavirus has sometimes been called an equalizer because it has
sickened both rich and poor, but when it comes to food, the commonality
ends. It is poor people, including large segments of poorer nations, who
are now going hungry and facing the prospect of starving.

``The coronavirus has been anything but a great equalizer,'' said Asha
Jaffar, a volunteer who brought food to families in the Nairobi slum of
Kibera after the fatal stampede. ``It's been the great revealer, pulling
the curtain back on the class divide and exposing how deeply unequal
this country is.''

Already, 135 million people
\href{https://www.fsinplatform.org/sites/default/files/resources/files/GRFC_2020_ONLINE_200420.pdf}{had
been facing acute food shortages}, but now with the pandemic, 130
million more could go hungry in 2020, said Arif Husain, chief economist
at the World Food Program, a United Nations agency. Altogether, an
estimated 265 million people could be pushed to the brink of starvation
by year's end.

``We've never seen anything like this before,'' Mr. Husain said. ``It
wasn't a pretty picture to begin with, but this makes it truly
unprecedented and uncharted territory.''

\hypertarget{latest-updates-global-coronavirus-outbreak}{%
\section{\texorpdfstring{\href{https://www.nytimes.com/2020/08/04/world/coronavirus-covid-19.html?action=click\&pgtype=Article\&state=default\&region=MAIN_CONTENT_1\&context=storylines_live_updates}{Latest
Updates: Global Coronavirus
Outbreak}}{Latest Updates: Global Coronavirus Outbreak}}\label{latest-updates-global-coronavirus-outbreak}}

Updated 2020-08-04T09:15:14.275Z

\begin{itemize}
\tightlist
\item
  \href{https://www.nytimes.com/2020/08/04/world/coronavirus-covid-19.html?action=click\&pgtype=Article\&state=default\&region=MAIN_CONTENT_1\&context=storylines_live_updates\#link-6b644638}{`Long
  days, long nights': Washington prepares for a prolonged fight over
  virus relief.}
\item
  \href{https://www.nytimes.com/2020/08/04/world/coronavirus-covid-19.html?action=click\&pgtype=Article\&state=default\&region=MAIN_CONTENT_1\&context=storylines_live_updates\#link-7af9fca0}{Israel's
  rocky reopening of its schools may be a lesson for the U.S.}
\item
  \href{https://www.nytimes.com/2020/08/04/world/coronavirus-covid-19.html?action=click\&pgtype=Article\&state=default\&region=MAIN_CONTENT_1\&context=storylines_live_updates\#link-33bf9168}{Hurricane
  Isaias arrives in North Carolina as officials along the East Coast
  scramble.}
\end{itemize}

\href{https://www.nytimes.com/2020/08/04/world/coronavirus-covid-19.html?action=click\&pgtype=Article\&state=default\&region=MAIN_CONTENT_1\&context=storylines_live_updates}{See
more updates}

More live coverage:
\href{https://www.nytimes.com/live/2020/08/03/business/stock-market-today-coronavirus?action=click\&pgtype=Article\&state=default\&region=MAIN_CONTENT_1\&context=storylines_live_updates}{Markets}

The world has experienced severe hunger crises before, but those were
regional and caused by one factor or another --- extreme weather,
economic downturns, wars or political instability.

Image

The Mandawi wholesale market in Central Kabul, Afghanistan, in
March.Credit...Jim Huylebroek for The New York Times

This hunger crisis, experts say, is global and caused by a multitude of
factors linked to the coronavirus pandemic and the ensuing interruption
of the economic order: the sudden loss in income for countless millions
who were already living hand-to-mouth; the collapse in oil prices;
widespread shortages of hard currency from tourism drying up; overseas
workers not having earnings to send home; and ongoing problems like
climate change, violence, population dislocations and humanitarian
disasters.

Already, from Honduras to South Africa to India, protests and
\href{https://twitter.com/IOL/status/1252255761951895552?s=20}{looting}
have broken out amid frustrations from lockdowns and worries about
hunger. With classes shut down, over 368 million children
\href{https://cdn.wfp.org/2020/school-feeding-map/index.html}{have lost}
the nutritious meals and snacks they normally receive in school.

There is no shortage of food globally, or mass starvation from the
pandemic --- yet. But logistical problems in planting, harvesting and
transporting food will leave poor countries exposed in the coming
months, especially those reliant on imports, said Johan Swinnen,
director general of the International Food Policy Research Institute in
Washington.

While the system of food distribution and retailing in rich nations is
organized and automated, he said, systems in developing countries are
``labor intensive,'' making ``these supply chains much more vulnerable
to Covid-19 and social distancing regulations.''

Yet even if there is no major surge in food prices, the food security
situation for poor people is likely to deteriorate significantly
worldwide. This is especially true for economies like Sudan and Zimbabwe
that were struggling before the outbreak, or those like Iran that have
increasingly used oil revenues to finance critical goods like food and
medicine.

In Venezuela, the pandemic could deal a devastating blow to millions
already living in
\href{https://www.nytimes.com/2019/05/17/world/americas/venezuela-economy.html}{the
world's largest economic collapse outside wartime}.

Image

A virtually deserted Candelaria Square in Caracas, Venezuela, during the
nationwide lockdown in March.Credit...Adriana Loureiro Fernandez for The
New York Times

In the sprawling Petare slum on the outskirts of the capital, Caracas, a
nationwide lockdown has left Freddy Bastardo and five others in his
household without jobs. Their government-supplied rations, which had
arrived only once every two months before the crisis, have long run out.

``We are already thinking of selling things that we don't use in the
house to be able to eat,'' said Mr. Bastardo, 25, a security guard. ``I
have neighbors who don't have food, and I'm worried that if protests
start, we wouldn't be able to get out of here.''

Uncertainty over food is also building in India, where daily-wage
workers with little or no social safety net face a future where
\href{https://www.nytimes.com/2020/03/30/world/asia/coronavirus-india-lockdown.html}{hunger
is a more immediate threat than the virus}.

As wages have dried up, half a million people are estimated to have left
cities to walk home, setting off the nation's ``largest mass migration
since independence,'' said Amitabh Behar, the chief executive of Oxfam
India.

On a recent evening, hundreds of migrant workers, who have been stuck in
New Delhi after a lockdown was imposed in March with little warning, sat
under the shade of a bridge waiting for food to arrive. The Delhi
government has set up soup kitchens, yet workers like Nihal Singh go
hungry as the throngs at these centers have increased in recent days.

``Instead of coronavirus, the hunger will kill us,'' said Mr. Singh, who
was hoping to eat his first meal in a day. Migrants waiting in food
lines have fought each other over a plate of rice and lentils. Mr. Singh
said he was ashamed to beg for food but had no other option.

\href{https://www.nytimes.com/news-event/coronavirus?action=click\&pgtype=Article\&state=default\&region=MAIN_CONTENT_3\&context=storylines_faq}{}

\hypertarget{the-coronavirus-outbreak-}{%
\subsubsection{The Coronavirus Outbreak
›}\label{the-coronavirus-outbreak-}}

\hypertarget{frequently-asked-questions}{%
\paragraph{Frequently Asked
Questions}\label{frequently-asked-questions}}

Updated August 3, 2020

\begin{itemize}
\item ~
  \hypertarget{im-a-small-business-owner-can-i-get-relief}{%
  \paragraph{I'm a small-business owner. Can I get
  relief?}\label{im-a-small-business-owner-can-i-get-relief}}

  \begin{itemize}
  \tightlist
  \item
    The
    \href{https://www.nytimes.com/article/small-business-loans-stimulus-grants-freelancers-coronavirus.html?action=click\&pgtype=Article\&state=default\&region=MAIN_CONTENT_3\&context=storylines_faq}{stimulus
    bills enacted in March} offer help for the millions of American
    small businesses. Those eligible for aid are businesses and
    nonprofit organizations with fewer than 500 workers, including sole
    proprietorships, independent contractors and freelancers. Some
    larger companies in some industries are also eligible. The help
    being offered, which is being managed by the Small Business
    Administration, includes the Paycheck Protection Program and the
    Economic Injury Disaster Loan program. But lots of folks have
    \href{https://www.nytimes.com/interactive/2020/05/07/business/small-business-loans-coronavirus.html?action=click\&pgtype=Article\&state=default\&region=MAIN_CONTENT_3\&context=storylines_faq}{not
    yet seen payouts.} Even those who have received help are confused:
    The rules are draconian, and some are stuck sitting on
    \href{https://www.nytimes.com/2020/05/02/business/economy/loans-coronavirus-small-business.html?action=click\&pgtype=Article\&state=default\&region=MAIN_CONTENT_3\&context=storylines_faq}{money
    they don't know how to use.} Many small-business owners are getting
    less than they expected or
    \href{https://www.nytimes.com/2020/06/10/business/Small-business-loans-ppp.html?action=click\&pgtype=Article\&state=default\&region=MAIN_CONTENT_3\&context=storylines_faq}{not
    hearing anything at all.}
  \end{itemize}
\item ~
  \hypertarget{what-are-my-rights-if-i-am-worried-about-going-back-to-work}{%
  \paragraph{What are my rights if I am worried about going back to
  work?}\label{what-are-my-rights-if-i-am-worried-about-going-back-to-work}}

  \begin{itemize}
  \tightlist
  \item
    Employers have to provide
    \href{https://www.osha.gov/SLTC/covid-19/standards.html}{a safe
    workplace} with policies that protect everyone equally.
    \href{https://www.nytimes.com/article/coronavirus-money-unemployment.html?action=click\&pgtype=Article\&state=default\&region=MAIN_CONTENT_3\&context=storylines_faq}{And
    if one of your co-workers tests positive for the coronavirus, the
    C.D.C.} has said that
    \href{https://www.cdc.gov/coronavirus/2019-ncov/community/guidance-business-response.html}{employers
    should tell their employees} -\/- without giving you the sick
    employee's name -\/- that they may have been exposed to the virus.
  \end{itemize}
\item ~
  \hypertarget{should-i-refinance-my-mortgage}{%
  \paragraph{Should I refinance my
  mortgage?}\label{should-i-refinance-my-mortgage}}

  \begin{itemize}
  \tightlist
  \item
    \href{https://www.nytimes.com/article/coronavirus-money-unemployment.html?action=click\&pgtype=Article\&state=default\&region=MAIN_CONTENT_3\&context=storylines_faq}{It
    could be a good idea,} because mortgage rates have
    \href{https://www.nytimes.com/2020/07/16/business/mortgage-rates-below-3-percent.html?action=click\&pgtype=Article\&state=default\&region=MAIN_CONTENT_3\&context=storylines_faq}{never
    been lower.} Refinancing requests have pushed mortgage applications
    to some of the highest levels since 2008, so be prepared to get in
    line. But defaults are also up, so if you're thinking about buying a
    home, be aware that some lenders have tightened their standards.
  \end{itemize}
\item ~
  \hypertarget{what-is-school-going-to-look-like-in-september}{%
  \paragraph{What is school going to look like in
  September?}\label{what-is-school-going-to-look-like-in-september}}

  \begin{itemize}
  \tightlist
  \item
    It is unlikely that many schools will return to a normal schedule
    this fall, requiring the grind of
    \href{https://www.nytimes.com/2020/06/05/us/coronavirus-education-lost-learning.html?action=click\&pgtype=Article\&state=default\&region=MAIN_CONTENT_3\&context=storylines_faq}{online
    learning},
    \href{https://www.nytimes.com/2020/05/29/us/coronavirus-child-care-centers.html?action=click\&pgtype=Article\&state=default\&region=MAIN_CONTENT_3\&context=storylines_faq}{makeshift
    child care} and
    \href{https://www.nytimes.com/2020/06/03/business/economy/coronavirus-working-women.html?action=click\&pgtype=Article\&state=default\&region=MAIN_CONTENT_3\&context=storylines_faq}{stunted
    workdays} to continue. California's two largest public school
    districts --- Los Angeles and San Diego --- said on July 13, that
    \href{https://www.nytimes.com/2020/07/13/us/lausd-san-diego-school-reopening.html?action=click\&pgtype=Article\&state=default\&region=MAIN_CONTENT_3\&context=storylines_faq}{instruction
    will be remote-only in the fall}, citing concerns that surging
    coronavirus infections in their areas pose too dire a risk for
    students and teachers. Together, the two districts enroll some
    825,000 students. They are the largest in the country so far to
    abandon plans for even a partial physical return to classrooms when
    they reopen in August. For other districts, the solution won't be an
    all-or-nothing approach.
    \href{https://bioethics.jhu.edu/research-and-outreach/projects/eschool-initiative/school-policy-tracker/}{Many
    systems}, including the nation's largest, New York City, are
    devising
    \href{https://www.nytimes.com/2020/06/26/us/coronavirus-schools-reopen-fall.html?action=click\&pgtype=Article\&state=default\&region=MAIN_CONTENT_3\&context=storylines_faq}{hybrid
    plans} that involve spending some days in classrooms and other days
    online. There's no national policy on this yet, so check with your
    municipal school system regularly to see what is happening in your
    community.
  \end{itemize}
\item ~
  \hypertarget{is-the-coronavirus-airborne}{%
  \paragraph{Is the coronavirus
  airborne?}\label{is-the-coronavirus-airborne}}

  \begin{itemize}
  \tightlist
  \item
    The coronavirus
    \href{https://www.nytimes.com/2020/07/04/health/239-experts-with-one-big-claim-the-coronavirus-is-airborne.html?action=click\&pgtype=Article\&state=default\&region=MAIN_CONTENT_3\&context=storylines_faq}{can
    stay aloft for hours in tiny droplets in stagnant air}, infecting
    people as they inhale, mounting scientific evidence suggests. This
    risk is highest in crowded indoor spaces with poor ventilation, and
    may help explain super-spreading events reported in meatpacking
    plants, churches and restaurants.
    \href{https://www.nytimes.com/2020/07/06/health/coronavirus-airborne-aerosols.html?action=click\&pgtype=Article\&state=default\&region=MAIN_CONTENT_3\&context=storylines_faq}{It's
    unclear how often the virus is spread} via these tiny droplets, or
    aerosols, compared with larger droplets that are expelled when a
    sick person coughs or sneezes, or transmitted through contact with
    contaminated surfaces, said Linsey Marr, an aerosol expert at
    Virginia Tech. Aerosols are released even when a person without
    symptoms exhales, talks or sings, according to Dr. Marr and more
    than 200 other experts, who
    \href{https://academic.oup.com/cid/article/doi/10.1093/cid/ciaa939/5867798}{have
    outlined the evidence in an open letter to the World Health
    Organization}.
  \end{itemize}
\end{itemize}

``The lockdown has trampled on our dignity,'' he said.

Image

Waiting in line for meals in New Delhi, where daily-wage workers with
little or no social safety net say hunger is a more immediate threat
than the virus.Credit...Rebecca Conway for The New York Times

Refugees and people living in conflict zones are likely to be hit the
hardest.

The curfews and restrictions on movement are already devastating the
meager incomes of displaced people in Uganda and Ethiopia, the delivery
of seeds and farming tools in South Sudan and the distribution of food
aid in the Central African Republic. Containment measures in Niger,
which hosts almost 60,000 refugees fleeing conflict in Mali, have led to
surges in the pricing of food, according to the International Rescue
Committee.

The effects of the restrictions ``may cause more suffering than the
disease itself,'' said Kurt Tjossem, regional vice president for East
Africa at the International Rescue Committee.

Ahmad Bayoush, a construction worker who had been displaced to Idlib
Province in northern Syria, said he and many others had signed up to
receive food from aid groups, but that it had yet to arrive.

``I am expecting real hunger if it continues like this in the north,''
he said.

The pandemic is also slowing efforts to deal with
\href{https://www.nytimes.com/2020/02/21/world/africa/locusts-kenya-east-africa.html}{the
historic locust plague} that has been ravaging the East and Horn of
Africa. The outbreak is the worst the region has seen in decades and
comes on the heels of a year marked by extreme droughts and floods. But
the arrival of billions of
\href{http://www.fao.org/ag/locusts/en/info/info/index.html}{new swarms}
could further deepen food insecurity, said Cyril Ferrand, head of the
Food and Agriculture Organization's resilience team in eastern Africa.

Travel bans and airport closures, Mr. Ferrand said, are interrupting the
supply of pesticides that could help limit the locust population and
save pastureland and crops.

Image

Seeking shelter under a tree as locusts take flight in Laisamis, a town
in Marsabit County, Kenya, in February.Credit...Khadija Farah for The
New York Times

As many go hungry, there is concern in a number of countries that food
shortages will lead to social discord. In Colombia, residents of the
coastal state of La Guajira have begun blocking roads to call attention
to their need for food. In South Africa, rioters have broken into
neighborhood food kiosks and
\href{https://www.amnesty.org/en/latest/news/2020/04/southern-africa-government-intervention-required-as-millions-face-hunger-under-covid19-lockdown-regimes/}{faced
off with the police}.

And even charitable food giveaways can expose people to the virus when
throngs appear, as happened in Nairobi's shantytown of Kibera earlier
this month.

``People called each other and came rushing,'' said Valentine Akinyi,
who works at the district government office where the food was
distributed. ``People have lost jobs. It showed you how hungry they
are.''

To assuage the impact of this crisis, some governments are
\href{https://twitter.com/RwandaTrade/status/1241672944461451264}{fixing
prices on food items}, delivering free food and putting in place
\href{https://www.president.go.ke/2020/03/25/presidential-address-on-the-state-interventions-to-cushion-kenyans-against-economic-effects-of-covid-19-pandemic-on-25th-march-2020/}{plans
to send money transfers} to the poorest households.

Yet communities across the world are also taking matters into their own
hands. Some are
\href{https://secure.changa.co.ke/myweb/share/39216}{raising money}
through crowdfunding platforms, while others have begun programs
to\href{https://www.rappi.com.co/tienda/menu-solidario}{buy meals} for
needy families.

On a recent afternoon, Ms. Jaffar and a group of volunteers made their
way through Kibera, bringing items like sugar, flour, rice and sanitary
pads to dozens of families. A native of the area herself, Ms. Jaffar
said she started the food drive after hearing so many stories from
families who said they and their children were going to sleep hungry.

The food drive has so far reached 500 families. But with all the calls
for assistance she's getting, she said, ``that's a drop in the ocean.''

Reporting was contributed by Anatoly Kurmanaev and Isayen Herrera from
Caracas, Venezuela; Paulina Villegas from Mexico City; Julie Turkewitz
from Bogotá, Colombia; Ben Hubbard and Hwaida Saad from Beirut, Lebanon;
Sameer Yasir from New Delhi; and Hannah Beech from Bangkok.

Advertisement

\protect\hyperlink{after-bottom}{Continue reading the main story}

\hypertarget{site-index}{%
\subsection{Site Index}\label{site-index}}

\hypertarget{site-information-navigation}{%
\subsection{Site Information
Navigation}\label{site-information-navigation}}

\begin{itemize}
\tightlist
\item
  \href{https://help.nytimes.com/hc/en-us/articles/115014792127-Copyright-notice}{©~2020~The
  New York Times Company}
\end{itemize}

\begin{itemize}
\tightlist
\item
  \href{https://www.nytco.com/}{NYTCo}
\item
  \href{https://help.nytimes.com/hc/en-us/articles/115015385887-Contact-Us}{Contact
  Us}
\item
  \href{https://www.nytco.com/careers/}{Work with us}
\item
  \href{https://nytmediakit.com/}{Advertise}
\item
  \href{http://www.tbrandstudio.com/}{T Brand Studio}
\item
  \href{https://www.nytimes.com/privacy/cookie-policy\#how-do-i-manage-trackers}{Your
  Ad Choices}
\item
  \href{https://www.nytimes.com/privacy}{Privacy}
\item
  \href{https://help.nytimes.com/hc/en-us/articles/115014893428-Terms-of-service}{Terms
  of Service}
\item
  \href{https://help.nytimes.com/hc/en-us/articles/115014893968-Terms-of-sale}{Terms
  of Sale}
\item
  \href{https://spiderbites.nytimes.com}{Site Map}
\item
  \href{https://help.nytimes.com/hc/en-us}{Help}
\item
  \href{https://www.nytimes.com/subscription?campaignId=37WXW}{Subscriptions}
\end{itemize}
