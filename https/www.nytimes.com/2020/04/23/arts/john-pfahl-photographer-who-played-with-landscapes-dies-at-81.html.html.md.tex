Sections

SEARCH

\protect\hyperlink{site-content}{Skip to
content}\protect\hyperlink{site-index}{Skip to site index}

\href{https://www.nytimes.com/section/arts}{Arts}

\href{https://myaccount.nytimes.com/auth/login?response_type=cookie\&client_id=vi}{}

\href{https://www.nytimes.com/section/todayspaper}{Today's Paper}

\href{/section/arts}{Arts}\textbar{}John Pfahl, Photographer Who Played
With Landscapes, Dies at 81

\url{https://nyti.ms/2Kwi3LQ}

\begin{itemize}
\item
\item
\item
\item
\item
\end{itemize}

\href{https://www.nytimes.com/news-event/coronavirus?action=click\&pgtype=Article\&state=default\&region=TOP_BANNER\&context=storylines_menu}{The
Coronavirus Outbreak}

\begin{itemize}
\tightlist
\item
  live\href{https://www.nytimes.com/2020/08/03/world/coronavirus-covid-19.html?action=click\&pgtype=Article\&state=default\&region=TOP_BANNER\&context=storylines_menu}{Latest
  Updates}
\item
  \href{https://www.nytimes.com/interactive/2020/us/coronavirus-us-cases.html?action=click\&pgtype=Article\&state=default\&region=TOP_BANNER\&context=storylines_menu}{Maps
  and Cases}
\item
  \href{https://www.nytimes.com/interactive/2020/science/coronavirus-vaccine-tracker.html?action=click\&pgtype=Article\&state=default\&region=TOP_BANNER\&context=storylines_menu}{Vaccine
  Tracker}
\item
  \href{https://www.nytimes.com/2020/08/02/us/covid-college-reopening.html?action=click\&pgtype=Article\&state=default\&region=TOP_BANNER\&context=storylines_menu}{College
  Reopening}
\item
  \href{https://www.nytimes.com/live/2020/08/03/business/stock-market-today-coronavirus?action=click\&pgtype=Article\&state=default\&region=TOP_BANNER\&context=storylines_menu}{Economy}
\end{itemize}

Advertisement

\protect\hyperlink{after-top}{Continue reading the main story}

Supported by

\protect\hyperlink{after-sponsor}{Continue reading the main story}

THOSE WE'VE LOST

\hypertarget{john-pfahl-photographer-who-played-with-landscapes-dies-at-81}{%
\section{John Pfahl, Photographer Who Played With Landscapes, Dies at
81}\label{john-pfahl-photographer-who-played-with-landscapes-dies-at-81}}

Long before Photoshop, he manipulated landscapes with everyday objects,
and found beauty in compost piles and other peculiar places.

\includegraphics{https://static01.nyt.com/images/2020/04/24/obituaries/22pfahl1/merlin_171785127_fd05c480-c8df-4d97-ace4-d52f34612199-articleLarge.jpg?quality=75\&auto=webp\&disable=upscale}

\href{https://www.nytimes.com/by/richard-sandomir}{\includegraphics{https://static01.nyt.com/images/2018/12/10/multimedia/author-richard-sandomir/author-richard-sandomir-thumbLarge.png}}

By \href{https://www.nytimes.com/by/richard-sandomir}{Richard Sandomir}

\begin{itemize}
\item
  Published April 23, 2020Updated April 28, 2020
\item
  \begin{itemize}
  \item
  \item
  \item
  \item
  \item
  \end{itemize}
\end{itemize}

\emph{This obituary is part of a series about people who have died in
the coronavirus pandemic. Read about others}
\href{https://www.nytimes.com/series/people-who-have-died-of-the-coronavirus}{\emph{here}}\emph{.}

\href{https://johnpfahl.com/}{John Pfahl,} an inventive landscape
photographer renowned for manipulating the natural world by inserting
into it objects like rope, foil, lace, tape and, once, a pie pan, died
on April 15 in Buffalo, N.Y. He was 81.

His sister-in-law, Cathy Pfahl, said that the cause was the new
coronavirus, but that he had also had heart problems, mild dementia and
Parkinson's disease.

Mr. Pfahl developed a reputation as a masterly if quirky landscape
photographer over more than 40 years. In addition to his manipulations,
he found beauty in peculiar vistas like the
\href{https://johnpfahl.com/pages/smoke/S1.html}{belching smoke of a
coke plant in Lackawanna, N.Y.,}the rotting fruit and vegetables of his
compost pile, and a stately hill of road salt --- often as statements
about the environmental impact of industrialization.

``I want to make pictures that work on a more mysterious level, that
approach the truth by a more circuitous route,'' he
\href{https://johnpfahl.com/pages/extras/ArtStatement.html}{wrote}in an
artist's statement on his website.

Image

Mr. Pfahl at an exhibition of his photography in 2007 in Buffalo, N.Y.
``I want to make pictures that work on a more mysterious level, that
approach the truth by a more circuitous route,'' he said.Credit...Sharon
Cantillon/Buffalo News

In ``Altered Landscapes,'' a series of predigital manipulated photos
that Mr. Pfahl shot in the 1970s, he brought playfulness and visual
pun-making to his work. In ``Triangle, Bermuda,'' he laid a triangle of
black string that led from a beach in Bermuda to the waters of the
Atlantic Ocean.

In another landscape, he parodied
\href{https://www.anseladams.com/}{Ansel Adams's} famous photograph
\href{https://www.moma.org/collection/works/53904}{``Moonrise,
Hernandez, New Mexico,''} by placing a pie pan among boulders at Capitol
Reef National Park in Utah. The pan appears to be the same size and of
similar brightness as the distant moon directly above it. He called the
picture ``Moonrise Over Pie Pan.''

\includegraphics{https://static01.nyt.com/images/2020/04/24/obituaries/22pfahl3/merlin_171781338_d4a6af62-0c84-4d3f-9883-296657dc8f9e-articleLarge.jpg?quality=75\&auto=webp\&disable=upscale}

Some of his photographs may seem whimsical, but there was serious
conceptual rationale and intricate mathematical calculations behind his
image tinkering.

``It's a witty reflection on how we tend to conflate pictures of the
natural world with the natural world itself,'' said Lisa Hostetler, the
curator in charge of the department of photography at
the\href{https://www.eastman.org/}{George Eastman Museum} in Rochester,
N.Y., which has a large collection of Mr. Pfahl's work. (He was a
longtime trustee there.)

By inserting mundane objects into landscapes, Ms. Hostetler added, Mr.
Pfahl was making a point about how cameras distort three-dimensional
space. ``He was showing us that while we believe in a picture, it looks
real, it looks normal, but it's actually false,'' she said.

Decades later, he embraced digital technology to alter pictures of
pastoral scenes in the British Isles.

John Alfred Pfahl was born on Feb. 17, 1939, in Manhattan and raised in
Wanaque, N.J., where he grew up hiking local wilderness trails, an early
introduction to the natural world. His father, Hans, was a floor manager
for a series of factories, and his mother, Anna (Gerhardt) Pfahl, was a
homemaker. Both were German immigrants.

Image

``Moonrise over Pie Pan,'' (1977), a parody of Ansel Adams's ``Moonrise,
Hernandez, New Mexico.''Credit...John Pfahl/Janet Borden Gallery

Mr. Pfahl graduated with a bachelor of fine arts degree in 1961 from the
Syracuse University School of Art, where he had majored in advertising
but found his career path when he took elective courses in photography.

After two years in the Army in an engineering battalion at Fort Belvoir,
Va., he worked as an assistant to Paul Elfenbein, an advertising
photographer in Manhattan, then moved to Los Angeles to assist Herbert
Bruce Cross, an architectural photographer.

Returning to Syracuse University, he earned a master's degree in color
photography in 1968. Soon after, he was hired to teach photography at
the Rochester Institute Technology, where he remained until 1985, giving
him time to pursue photography full-time.

Mr. Pfahl did not ignore magnificent landscapes, like waterfalls, but he
often photographed them within the context of the industrialization that
sometimes shrouded their beauty.

In the 1980s, not long after the Three Mile Island nuclear accident in
Pennsylvania, he photographed the picturesque settings where power
companies often built nuclear plants, dams and generators. He invariably
set the power plants in the far reaches of the pictures, letting their
lush surroundings dominate.

Image

``Trojan Nuclear Plant, Columbia River, Oregon'' (1982)Credit...John
Pfahl/Janet Borden Gallery

``For me, power plants in the natural landscape represent only the most
extreme example of man's willful domination over the wilderness,'' Mr.
Pfahl wrote. ``It is the arena where the needs and ambitions of an
ever-expanding population collide most forcefully with the finite
resources of nature.''

He also found inspiration closer to home: the decaying watermelon rinds,
pumpkin shreds, pears, fennel and oranges in his compost pile, which he
called a ``daybook of both memorable and mundane meals that grace my
table.''

When the pictures,
\href{https://www.albrightknox.org/artworks/p19954a-10392-very-rich-hours-compost-pile-pineapple}{``From
the Very Rich Hours of a Compost Pile,''} were exhibited at the Nina
Freudenheim Gallery in Buffalo in 1995, Richard Huntington, the art
critic of The Buffalo News, praised them for displaying unerring
composition and ``vivid depiction of rot.''

``Is nothing ever amiss in a John Pfahl photograph?'' Mr. Huntington
wrote. ``His vision is so controlled, so precise, that I suspect that
when the world sees him coming it quickly rearranges itself, getting all
the various parts in order, just like someone does with the house when
they see an honored guest coming up the walkway.''

He found inspiration, too, in piles of leaves, a sand pit, a hill filled
with demolition material and a tire farm near his home. Adams may have
had the Sierra, but, as Mr. Pfahl wrote, he had these modest mountains
to call his own.

Image

``Bethlehem \#16, Lackawanna, N.Y.'' (1968)Credit...John Pfahl/Janet
Borden Gallery

``I try to imbue these piles of raw and recycled materials, through
judicious use of light, atmosphere and scale, with the majesty of
mountains I recall from summers in the Rockies and the Alps,'' he wrote.

Mr. Pfahl, who died in a Buffalo hospital, is survived by his wife,
Bonnie Gordon, an artist and a former professor of design at Buffalo
State College whom he married in a roadside chapel in Albuquerque, N.M.,
and a brother, Walter.

In the 1970s, Mr. Pfahl traveled around the United States asking
strangers to let him shoot landscapes from inside their homes. Many
accommodated him, intrigued by his idiosyncratic vision. He would then
take down their curtains, wash their windows and move their furniture
for a series that he called ``Picture Windows.''

``I liked the idea that my photographic vantage points were not solely
determined by myself,'' he wrote. ``They were predetermined by others,
sometimes years earlier, and patiently waited for me to discover them.''

\href{https://www.nytimes.com/interactive/2020/obituaries/people-died-coronavirus-obituaries.html?action=click\&pgtype=Article\&state=default\&region=BELOW_MAIN_CONTENT\&context=covid_obits_promo}{}

\hypertarget{those-weve-lost}{%
\section{Those We've Lost}\label{those-weve-lost}}

The coronavirus pandemic has taken an incalculable death toll. This
series is designed to put names and faces to the numbers.

Read more

\includegraphics{https://static01.nyt.com/images/2020/07/30/obituaries/30Pedro/30Pedro-square640.jpg}

\hypertarget{bernaldina-josuxe9-pedro}{%
\section{Bernaldina José Pedro}\label{bernaldina-josuxe9-pedro}}

d. Boa Vista, Brazil

Leader among the Indigenous Macuxi

\includegraphics{https://static01.nyt.com/images/2020/07/31/obituaries/31Swing/merlin_175167783_8913bc90-0d64-43f3-a655-1bb1bf1601c9-square640.jpg}

\hypertarget{john-eric-swing}{%
\section{John Eric Swing}\label{john-eric-swing}}

d. Fountain Valley, Calif.

Champion of Filipino-Americans

\includegraphics{https://static01.nyt.com/images/2020/07/27/obituaries/27Victor/merlin_175001436_38b11f8e-227a-4e2c-9821-7618af9b2524-square640.jpg}

\hypertarget{victor-victor}{%
\section{Victor Victor}\label{victor-victor}}

d. Santo Domingo, Dominican Republic

Beloved musician of the Dominican Republic

\includegraphics{https://static01.nyt.com/images/2020/07/31/obituaries/31Negron/merlin_175160169_516322ae-fd23-4969-b6b2-193ced371105-square640.jpg}

\hypertarget{dr-eddie-negruxf3n}{%
\section{Dr. Eddie Negrón}\label{dr-eddie-negruxf3n}}

d. Fort Walton Beach, Fla.

Internist on Florida's Emerald Coast

\includegraphics{https://static01.nyt.com/images/2020/07/30/obituaries/30Dobson/merlin_175115928_f6b9271c-8f05-4fe1-a38a-5ca4a58f8935-square640.jpg}

\hypertarget{dobby-dobson}{%
\section{Dobby Dobson}\label{dobby-dobson}}

d. Coral Springs, Fla.

Jamaican singer and songwriter

\includegraphics{https://static01.nyt.com/images/2020/08/01/obituaries/28Gonzalez/merlin_175002771_beb57888-3951-409a-ae13-03a94b2e962e-square640.jpg}

\hypertarget{waldemar-gonzalez}{%
\section{Waldemar Gonzalez}\label{waldemar-gonzalez}}

d. White Plains, N.Y.

Teacher and social worker

Advertisement

\protect\hyperlink{after-bottom}{Continue reading the main story}

\hypertarget{site-index}{%
\subsection{Site Index}\label{site-index}}

\hypertarget{site-information-navigation}{%
\subsection{Site Information
Navigation}\label{site-information-navigation}}

\begin{itemize}
\tightlist
\item
  \href{https://help.nytimes.com/hc/en-us/articles/115014792127-Copyright-notice}{©~2020~The
  New York Times Company}
\end{itemize}

\begin{itemize}
\tightlist
\item
  \href{https://www.nytco.com/}{NYTCo}
\item
  \href{https://help.nytimes.com/hc/en-us/articles/115015385887-Contact-Us}{Contact
  Us}
\item
  \href{https://www.nytco.com/careers/}{Work with us}
\item
  \href{https://nytmediakit.com/}{Advertise}
\item
  \href{http://www.tbrandstudio.com/}{T Brand Studio}
\item
  \href{https://www.nytimes.com/privacy/cookie-policy\#how-do-i-manage-trackers}{Your
  Ad Choices}
\item
  \href{https://www.nytimes.com/privacy}{Privacy}
\item
  \href{https://help.nytimes.com/hc/en-us/articles/115014893428-Terms-of-service}{Terms
  of Service}
\item
  \href{https://help.nytimes.com/hc/en-us/articles/115014893968-Terms-of-sale}{Terms
  of Sale}
\item
  \href{https://spiderbites.nytimes.com}{Site Map}
\item
  \href{https://help.nytimes.com/hc/en-us}{Help}
\item
  \href{https://www.nytimes.com/subscription?campaignId=37WXW}{Subscriptions}
\end{itemize}
