Sections

SEARCH

\protect\hyperlink{site-content}{Skip to
content}\protect\hyperlink{site-index}{Skip to site index}

\href{https://www.nytimes.com/section/us}{U.S.}

\href{https://myaccount.nytimes.com/auth/login?response_type=cookie\&client_id=vi}{}

\href{https://www.nytimes.com/section/todayspaper}{Today's Paper}

\href{/section/us}{U.S.}\textbar{}Risky Strategy Has Produced Wins for
Democrats in Fights Over Pandemic Aid

\href{https://nyti.ms/3auE7Bk}{https://nyti.ms/3auE7Bk}

\begin{itemize}
\item
\item
\item
\item
\item
\item
\end{itemize}

Advertisement

\protect\hyperlink{after-top}{Continue reading the main story}

Supported by

\protect\hyperlink{after-sponsor}{Continue reading the main story}

\hypertarget{risky-strategy-has-produced-wins-for-democrats-in-fights-over-pandemic-aid}{%
\section{Risky Strategy Has Produced Wins for Democrats in Fights Over
Pandemic
Aid}\label{risky-strategy-has-produced-wins-for-democrats-in-fights-over-pandemic-aid}}

Now another struggle looms as Senator Mitch McConnell threatens a
go-slow approach on the next round of economic stimulus, including aid
to states.

\includegraphics{https://static01.nyt.com/images/2020/04/22/us/politics/22dc-virus-assess1/merlin_171800877_ca45ddc1-8847-46af-8fae-74a7db66f9f9-articleLarge.jpg?quality=75\&auto=webp\&disable=upscale}

\href{https://www.nytimes.com/by/carl-hulse}{\includegraphics{https://static01.nyt.com/images/2018/06/14/multimedia/author-carl-hulse/author-carl-hulse-thumbLarge.png}}

By \href{https://www.nytimes.com/by/carl-hulse}{Carl Hulse}

\begin{itemize}
\item
  April 23, 2020
\item
  \begin{itemize}
  \item
  \item
  \item
  \item
  \item
  \item
  \end{itemize}
\end{itemize}

WASHINGTON --- In January 2018, Senate Democrats took a politically
risky stand, shutting down the government to insist on protections for
hundreds of thousands of undocumented immigrants. Gleeful Republicans
saw the obstruction strategy as a huge blunder and pounded the
Democrats, who
\href{https://www.nytimes.com/2018/01/22/us/politics/congress-votes-to-end-government-shutdown.html}{caved
after only a few days} of sharp attacks and cut a deal to reopen.

Times --- and circumstances --- have changed.

Democrats have now blocked two consecutive coronavirus rescue packages
pushed by Republicans and withstood withering criticism to win
concessions --- and hundreds of billions of dollars --- they said were
vital. At nearly \$500 billion, the latest measure to move through
Congress this week ended up being almost twice the size and much broader
in scope than the original bill Senator Mitch McConnell, Republican of
Kentucky and the majority leader, tried to ram through two weeks earlier
without negotiations.

It was a potentially dangerous strategy for Democrats, particularly in
an election year, that left them open to accusations from President
Trump and congressional Republicans that they were denying desperately
needed money at a crucial moment for businesses trying to survive in the
face of the pandemic. It may have also reduced their leverage in the
next fight over a much larger stimulus measure that is likely to top \$1
trillion.

But their willingness to take on those risks reflects Democrats'
confidence that the terrain of the current debate --- a public health
crisis and economic disaster that will require the broadest government
relief effort since the post-World War II era --- plays to their core
strengths as a party. It is also based in part on their belief that Mr.
Trump, whose re-election hopes are likely to rise or fall based on the
public perception of his administration's response to the pandemic, has
a strong incentive to compromise with them.

``We think we are right,'' Senator Chuck Schumer of New York, the
Democratic leader, said in an interview explaining his party's stance.
``People were just united that this was a serious crisis, and it was
sort of obvious what McConnell was trying to do.''

The same was true last month, when Democrats
\href{https://www.nytimes.com/2020/03/26/us/coronavirus-senate-stimulus-package.html}{twice
voted to block a sweeping economic stimulus package} that contained
jobless aid, direct payments to Americans and business bailouts while
they held out for their priorities, including stricter oversight
requirements over how the Trump administration would spend the vast
amounts of money. That \$2.2 trillion measure passed unanimously.

``My overall observation is it's pretty hard to win a spending contest
with a Democrat,'' Mr. McConnell said in a brief interview on Tuesday.
``They always want to spend more on everything.''

Mr. McConnell has evidently had enough of a dynamic that seems to be
empowering Democrats. After the latest aid bill passed the Senate on
Tuesday, he cited mounting deficit spending --- Congress has now
appropriated more than \$2.7 trillion in only seven weeks to confront
the pandemic --- and declared that Republicans would entertain no more
coronavirus rescue packages until all lawmakers were back in Washington
for a normal Senate session. That scenario could delay additional aid
while allowing a fuller debate on emerging proposals, rather than
negotiation and approval by a handful of top lawmakers in a nearly empty
Capitol.

The Republican leader also sought to play down Democratic gains in the
bill, emphasizing that they had failed to secure additional aid to state
and local governments that they had aggressively sought.

``It's unfortunate that it took our Democratic colleagues 12 days to
agree to a deal that contains essentially nothing that Republicans ever
opposed,'' Mr. McConnell said.

But the measure did contain multiple things that Mr. McConnell initially
rejected as he sought approval of a bare-bones infusion of \$250 billion
into a small-business loan program that had quickly run dry of funds.
Instead, that program got \$320 billion in new funds, including \$60
billion secured by Democrats to be funneled through smaller community
lenders to reach businesses that can struggle to get loans from big
banks.

Also included were \$60 billion to replenish exhausted Small Business
Administration disaster relief accounts, \$75 billion for hospitals and
\$25 billion for Covid-19 testing, plus a mandate that the Trump
administration establish a strategy to help states vastly step up the
deployment of tests throughout the country --- a move Republicans had
opposed.

``Of the four major things we pushed for, we got three over Republican
resistance,'' Mr. Schumer said. ``But they knew they needed us.''

\includegraphics{https://static01.nyt.com/images/2020/04/22/us/politics/22dc-virus-assess2/merlin_171800904_88074946-3bb7-472f-9b38-a85d00a58c56-articleLarge.jpg?quality=75\&auto=webp\&disable=upscale}

Speaker Nancy Pelosi, who early on warned Mr. McConnell that his
proposal would not clear the Democratically controlled House, called the
outcome a clear win for her party's priorities. She said the legislation
would not have been delayed at all if Republicans had accepted a
Democratic counteroffer two weeks ago.

``They like to say, `Oh, we held up,''' Ms. Pelosi told reporters. ``No,
we didn't hold it up. They held up. And now we have prevailed.''

Democrats had substantial help in pushing back against Republicans,
including a willing negotiator in Treasury Secretary Steven Mnuchin, who
has shown a
\href{https://www.nytimes.com/2020/04/17/us/politics/coronavirus-mnuchin-republicans.html}{tendency
to side with Democrats} that has unnerved Republicans. Their efforts to
place conditions on the small-business loan funds also benefited from a
public outcry over how the program was being administered, after it
emerged that the
\href{https://www.nytimes.com/2020/04/20/business/shake-shack-returning-loan-ppp-coronavirus.html}{money
had been flowing more easily to large chains and publicly traded
companies} than to the smaller mom-and-pop operations it was intended to
help.

``We needed to make sure they were working for everybody, not just the
most-connected businesses,'' said Senator Chris Van Hollen, Democrat of
Maryland, who said the business owners he consulted wanted Democrats to
institute changes in the loan effort, not rush to inject more funding
into a flawed program.

Even after the final agreement was struck, Republicans continued to slam
Democrats for the delay. Senator John Barrasso of Wyoming, the chamber's
No. 3 Republican, called it ``disgraceful.''

``For Chuck Schumer and Nancy Pelosi to hold up that money for these
people and hold them all hostage to create leverage is unconscionable to
me,'' he said. ``They seemed to have no sense of urgency or sense of the
crisis that is hitting this country.''

The Trump campaign released a scathing ad on Monday attacking Ms. Pelosi
for blocking the funding, juxtaposing footage of poor Americans
struggling in the pandemic with clips of an appearance the speaker made
on late-night TV in which she showed off her favorite ice creams.
``Nancy Antoinette,'' it called her.

Even as congressional Democrats were celebrating their gains in the
legislation, they were under fire from progressive lawmakers and
advocacy groups who saw the package as insufficient. Some on the left
faulted Democrats for not extracting more from the Republicans,
particularly for states and cities that are being pushed to the
financial brink by the pandemic, or winning more direct aid for
beleaguered Americans wondering how to pay rent when they cannot expect
another check from the government.

``This is a win for McConnell and Trump,'' said Ezra Levin, an executive
director of the group Indivisible. ``This Covid 3.5 package is nothing
close to what families and workers need right now.''

Mr. Schumer and Ms. Pelosi disputed that notion and promised a
``robust'' Phase 4 of the pandemic relief effort. They noted that the
Trump administration was also already mapping plans for another bill
that would include infrastructure investments and aid for states.
Democrats said that action would become inevitable as states and cities
face decisions on laying off emergency workers and cutting other
services and as the public clamors for help.

Mr. McConnell appears to be digging in, telling the conservative radio
host Hugh Hewitt on Wednesday that he wanted to ``push the pause
button'' on coronavirus relief legislation, and that ``this whole
business of additional assistance for state and local governments needs
to be thoroughly evaluated.''

But with the pandemic continuing to roil the economy and facing intense
pressure to respond from Democrats, the White House, governors of both
parties and some of his own lawmakers, Mr. McConnell may once again find
himself in the unusual position of struggling to hold the line.

Emily Cochrane contributed reporting.

Advertisement

\protect\hyperlink{after-bottom}{Continue reading the main story}

\hypertarget{site-index}{%
\subsection{Site Index}\label{site-index}}

\hypertarget{site-information-navigation}{%
\subsection{Site Information
Navigation}\label{site-information-navigation}}

\begin{itemize}
\tightlist
\item
  \href{https://help.nytimes.com/hc/en-us/articles/115014792127-Copyright-notice}{©~2020~The
  New York Times Company}
\end{itemize}

\begin{itemize}
\tightlist
\item
  \href{https://www.nytco.com/}{NYTCo}
\item
  \href{https://help.nytimes.com/hc/en-us/articles/115015385887-Contact-Us}{Contact
  Us}
\item
  \href{https://www.nytco.com/careers/}{Work with us}
\item
  \href{https://nytmediakit.com/}{Advertise}
\item
  \href{http://www.tbrandstudio.com/}{T Brand Studio}
\item
  \href{https://www.nytimes.com/privacy/cookie-policy\#how-do-i-manage-trackers}{Your
  Ad Choices}
\item
  \href{https://www.nytimes.com/privacy}{Privacy}
\item
  \href{https://help.nytimes.com/hc/en-us/articles/115014893428-Terms-of-service}{Terms
  of Service}
\item
  \href{https://help.nytimes.com/hc/en-us/articles/115014893968-Terms-of-sale}{Terms
  of Sale}
\item
  \href{https://spiderbites.nytimes.com}{Site Map}
\item
  \href{https://help.nytimes.com/hc/en-us}{Help}
\item
  \href{https://www.nytimes.com/subscription?campaignId=37WXW}{Subscriptions}
\end{itemize}
