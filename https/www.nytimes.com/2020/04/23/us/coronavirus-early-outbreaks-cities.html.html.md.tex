Sections

SEARCH

\protect\hyperlink{site-content}{Skip to
content}\protect\hyperlink{site-index}{Skip to site index}

\href{https://www.nytimes.com/section/us}{U.S.}

\href{https://myaccount.nytimes.com/auth/login?response_type=cookie\&client_id=vi}{}

\href{https://www.nytimes.com/section/todayspaper}{Today's Paper}

\href{/section/us}{U.S.}\textbar{}Hidden Outbreaks Spread Through U.S.
Cities Far Earlier Than Americans Knew, Estimates Say

\url{https://nyti.ms/2VysPYx}

\begin{itemize}
\item
\item
\item
\item
\item
\item
\end{itemize}

\href{https://www.nytimes.com/news-event/coronavirus?action=click\&pgtype=Article\&state=default\&region=TOP_BANNER\&context=storylines_menu}{The
Coronavirus Outbreak}

\begin{itemize}
\tightlist
\item
  live\href{https://www.nytimes.com/2020/08/01/world/coronavirus-covid-19.html?action=click\&pgtype=Article\&state=default\&region=TOP_BANNER\&context=storylines_menu}{Latest
  Updates}
\item
  \href{https://www.nytimes.com/interactive/2020/us/coronavirus-us-cases.html?action=click\&pgtype=Article\&state=default\&region=TOP_BANNER\&context=storylines_menu}{Maps
  and Cases}
\item
  \href{https://www.nytimes.com/interactive/2020/science/coronavirus-vaccine-tracker.html?action=click\&pgtype=Article\&state=default\&region=TOP_BANNER\&context=storylines_menu}{Vaccine
  Tracker}
\item
  \href{https://www.nytimes.com/interactive/2020/07/29/us/schools-reopening-coronavirus.html?action=click\&pgtype=Article\&state=default\&region=TOP_BANNER\&context=storylines_menu}{What
  School May Look Like}
\item
  \href{https://www.nytimes.com/live/2020/07/31/business/stock-market-today-coronavirus?action=click\&pgtype=Article\&state=default\&region=TOP_BANNER\&context=storylines_menu}{Economy}
\end{itemize}

Advertisement

\protect\hyperlink{after-top}{Continue reading the main story}

Supported by

\protect\hyperlink{after-sponsor}{Continue reading the main story}

\hypertarget{hidden-outbreaks-spread-through-us-cities-far-earlier-than-americans-knew-estimates-say}{%
\section{Hidden Outbreaks Spread Through U.S. Cities Far Earlier Than
Americans Knew, Estimates
Say}\label{hidden-outbreaks-spread-through-us-cities-far-earlier-than-americans-knew-estimates-say}}

By \href{https://www.nytimes.com/by/benedict-carey}{Benedict Carey} and
\href{https://www.nytimes.com/by/james-glanz}{James Glanz}

\begin{itemize}
\item
  Published April 23, 2020Updated July 6, 2020
\item
  \begin{itemize}
  \item
  \item
  \item
  \item
  \item
  \item
  \end{itemize}
\end{itemize}

By the time New York City confirmed its first case of the
\href{https://www.nytimes.com/2020/07/04/health/239-experts-with-one-big-claim-the-coronavirus-is-airborne.html}{coronavirus}
on March 1, thousands of
\href{https://www.nytimes.com/2020/05/14/health/talking-coronavirus-infect.html}{infections}
were already silently spreading through the city, a hidden explosion of
a disease that many still viewed as a remote threat as the city awaited
the first signs of spring.

Hidden outbreaks were also spreading almost completely undetected in
Boston, San Francisco, Chicago and Seattle, long before testing showed
that each city had a major problem, according to a model of the
\href{https://www.nytimes.com/2020/07/04/health/239-experts-with-one-big-claim-the-coronavirus-is-airborne.html}{spread}
of the disease by researchers at Northeastern University who shared
their results with The New York Times.

Even in early February --- while the world focused on China --- the
virus was not only likely to be spreading in multiple American cities,
but also seeding blooms of infection elsewhere in the United States, the
researchers found.

In five major U.S. cities, as of March 1

there were only 23 confirmed cases of coronavirus.

But according to the Northeastern model, there could have actually been

about 28,000 infections in those cities by then.

Boston

2,300

Seattle

2,300

Chicago

3,300

San Francisco

9,300

New York

10,700

In five major U.S. cities, as of March 1 there were only 23 confirmed
cases of coronavirus.

But according to the Northeastern model, there could have actually been
about 28,000 infections in those cities by then.

Boston

2,300

Seattle

2,300

Chicago

3,300

San Francisco

9,300

New York

10,700

Note: Numbers are median estimates that the Northeastern model
calculated for each city. The true number of infections could have been
substantially higher or lower than shown here.

By Derek Watkins

As political leaders grappled in February with the question of whether
the outbreak would become serious enough to order measures like school
closures and remote work, little or no systematic testing for the virus
was taking place.

``Meanwhile, in the background, you have this silent chain of
transmission of thousands of people,'' said Alessandro Vespignani,
director of the Network Science Institute at Northeastern University in
Boston, who led the research team.

Modeling the spread of a disease is inherently inexact, involving
estimates of how often people come in contact and transmit the virus as
they travel, work and socialize. The model estimates all infections,
including those in people who may experience mild or no
\href{https://www.nytimes.com/2020/04/30/well/live/coronavirus-days-5-through-10.html}{symptoms}
and those that are never detected in testing.

Other disease researchers said the findings of Dr. Vespignani's team
were broadly in line with their own analyses. The research offers the
first clear accounting of how far behind the United States was in
detecting the virus. And the findings provide a warning of what can
recur, the researchers say, if social distancing restrictions are lifted
too quickly.

Dr. Robert R. Redfield, the director of the Centers for Disease Control
and Prevention, said last week that American health officials had been
successful in tracking the first known cases and their contacts in the
United States before the outbreak got out of control.

``Through Feb. 27, this country only had 14 cases,'' he said during a
briefing. ``We did that isolation and that contact tracing, and it was
very successful. But then, when the virus more exploded, it got beyond
the public health capacity.''

\includegraphics{https://static01.nyt.com/images/2020/04/22/us/00HIDDEN-OUTBREAKS1/merlin_171417732_7e553e66-5f39-4ad2-941b-19a283a5cb5b-articleLarge.jpg?quality=75\&auto=webp\&disable=upscale}

But the new estimates of coronavirus infections are vastly higher than
those official counts.

By late February, as the world's attention shifted to a dire outbreak in
Italy, those 14 known American cases were a tiny fraction of the
thousands of undetected infections that the researchers estimated were
spreading from person to person across this country.

And more cases may have been arriving in the United States by the day.

``Knowing the number of flights coming into New York from Italy, it was
like watching a horrible train wreck in slow motion,'' said Adriana
Heguy, director of the Genome Technology Center at New York University's
Grossman School of Medicine.

\hypertarget{latest-updates-global-coronavirus-outbreak}{%
\section{\texorpdfstring{\href{https://www.nytimes.com/2020/08/01/world/coronavirus-covid-19.html?action=click\&pgtype=Article\&state=default\&region=MAIN_CONTENT_1\&context=storylines_live_updates}{Latest
Updates: Global Coronavirus
Outbreak}}{Latest Updates: Global Coronavirus Outbreak}}\label{latest-updates-global-coronavirus-outbreak}}

Updated 2020-08-02T07:42:09.613Z

\begin{itemize}
\tightlist
\item
  \href{https://www.nytimes.com/2020/08/01/world/coronavirus-covid-19.html?action=click\&pgtype=Article\&state=default\&region=MAIN_CONTENT_1\&context=storylines_live_updates\#link-34047410}{The
  U.S. reels as July cases more than double the total of any other
  month.}
\item
  \href{https://www.nytimes.com/2020/08/01/world/coronavirus-covid-19.html?action=click\&pgtype=Article\&state=default\&region=MAIN_CONTENT_1\&context=storylines_live_updates\#link-780ec966}{Top
  U.S. officials work to break an impasse over the federal jobless
  benefit.}
\item
  \href{https://www.nytimes.com/2020/08/01/world/coronavirus-covid-19.html?action=click\&pgtype=Article\&state=default\&region=MAIN_CONTENT_1\&context=storylines_live_updates\#link-2bc8948}{Its
  outbreak untamed, Melbourne goes into even greater lockdown.}
\end{itemize}

\href{https://www.nytimes.com/2020/08/01/world/coronavirus-covid-19.html?action=click\&pgtype=Article\&state=default\&region=MAIN_CONTENT_1\&context=storylines_live_updates}{See
more updates}

More live coverage:
\href{https://www.nytimes.com/live/2020/07/31/business/stock-market-today-coronavirus?action=click\&pgtype=Article\&state=default\&region=MAIN_CONTENT_1\&context=storylines_live_updates}{Markets}

Dr. Heguy's team and another at the Icahn School of Medicine at Mount
Sinai have found through genetic analysis that the seeds of most
infections in New York came from multiple locations in Europe, rather
than directly from China.

``We weren't testing, and if you're not testing you don't know,'' Dr.
Heguy said. The new estimates suggesting that thousands of infections
were spreading silently in the first months of the year ``don't seem
surprising at all,'' she said.

There are other signs that the outbreak was worse at an earlier point
than previously known. This week, health officials in Santa Clara
County, Calif., announced a
\href{https://www.nytimes.com/2020/04/22/us/santa-clara-county-coronavirus-death.html}{newly
discovered coronavirus-linked death} on Feb. 6, weeks earlier than what
had been previously thought to be the first death caused by the virus in
the United States.

Some scientists cautioned that the new report's estimates of an
enormous, unseen wave of infections could be too high ---~even though
testing surveillance lagged at the time.

``Even with these corrections, it's still on the high side --- this is
higher than I would have expected,'' said Dr. Donald Burke, a professor
of epidemiology at the University of Pittsburgh Graduate School of
Public Health.

Others said that the findings were in line with the fragmentary evidence
that had been available until now. Lauren Ancel Meyers, a professor of
biology and statistics at the University of Texas at Austin, said that
her own risk estimates and
\href{https://covid-19.tacc.utexas.edu/projections/}{most recent
projections} reveal a grim stealthiness of early coronavirus spread.

``By the time you see a few cases, it's pretty certain that you already
have an outbreak underway,'' Dr. Meyers said.

Dr. Vespignani's approach models the outbreak over time based on what is
known about the virus and where it has been detected. It estimates the
spread of the disease by simulating the movements of individual people
based on where people fly, how they move around, when they go to school
and other data. By running the model under various conditions --- when
schools are closed, say --- his team estimates where the virus may have
spread undetected.

Unseen carriers of the disease, many of them with mild symptoms or none
at all, can still spread the virus. For that reason, by the time leaders
in many cities and states took action, it was already too late to slow
the initial spread.

\href{https://www.nytimes.com/news-event/coronavirus?action=click\&pgtype=Article\&state=default\&region=MAIN_CONTENT_3\&context=storylines_faq}{}

\hypertarget{the-coronavirus-outbreak-}{%
\subsubsection{The Coronavirus Outbreak
›}\label{the-coronavirus-outbreak-}}

\hypertarget{frequently-asked-questions}{%
\paragraph{Frequently Asked
Questions}\label{frequently-asked-questions}}

Updated July 27, 2020

\begin{itemize}
\item ~
  \hypertarget{should-i-refinance-my-mortgage}{%
  \paragraph{Should I refinance my
  mortgage?}\label{should-i-refinance-my-mortgage}}

  \begin{itemize}
  \tightlist
  \item
    \href{https://www.nytimes.com/article/coronavirus-money-unemployment.html?action=click\&pgtype=Article\&state=default\&region=MAIN_CONTENT_3\&context=storylines_faq}{It
    could be a good idea,} because mortgage rates have
    \href{https://www.nytimes.com/2020/07/16/business/mortgage-rates-below-3-percent.html?action=click\&pgtype=Article\&state=default\&region=MAIN_CONTENT_3\&context=storylines_faq}{never
    been lower.} Refinancing requests have pushed mortgage applications
    to some of the highest levels since 2008, so be prepared to get in
    line. But defaults are also up, so if you're thinking about buying a
    home, be aware that some lenders have tightened their standards.
  \end{itemize}
\item ~
  \hypertarget{what-is-school-going-to-look-like-in-september}{%
  \paragraph{What is school going to look like in
  September?}\label{what-is-school-going-to-look-like-in-september}}

  \begin{itemize}
  \tightlist
  \item
    It is unlikely that many schools will return to a normal schedule
    this fall, requiring the grind of
    \href{https://www.nytimes.com/2020/06/05/us/coronavirus-education-lost-learning.html?action=click\&pgtype=Article\&state=default\&region=MAIN_CONTENT_3\&context=storylines_faq}{online
    learning},
    \href{https://www.nytimes.com/2020/05/29/us/coronavirus-child-care-centers.html?action=click\&pgtype=Article\&state=default\&region=MAIN_CONTENT_3\&context=storylines_faq}{makeshift
    child care} and
    \href{https://www.nytimes.com/2020/06/03/business/economy/coronavirus-working-women.html?action=click\&pgtype=Article\&state=default\&region=MAIN_CONTENT_3\&context=storylines_faq}{stunted
    workdays} to continue. California's two largest public school
    districts --- Los Angeles and San Diego --- said on July 13, that
    \href{https://www.nytimes.com/2020/07/13/us/lausd-san-diego-school-reopening.html?action=click\&pgtype=Article\&state=default\&region=MAIN_CONTENT_3\&context=storylines_faq}{instruction
    will be remote-only in the fall}, citing concerns that surging
    coronavirus infections in their areas pose too dire a risk for
    students and teachers. Together, the two districts enroll some
    825,000 students. They are the largest in the country so far to
    abandon plans for even a partial physical return to classrooms when
    they reopen in August. For other districts, the solution won't be an
    all-or-nothing approach.
    \href{https://bioethics.jhu.edu/research-and-outreach/projects/eschool-initiative/school-policy-tracker/}{Many
    systems}, including the nation's largest, New York City, are
    devising
    \href{https://www.nytimes.com/2020/06/26/us/coronavirus-schools-reopen-fall.html?action=click\&pgtype=Article\&state=default\&region=MAIN_CONTENT_3\&context=storylines_faq}{hybrid
    plans} that involve spending some days in classrooms and other days
    online. There's no national policy on this yet, so check with your
    municipal school system regularly to see what is happening in your
    community.
  \end{itemize}
\item ~
  \hypertarget{is-the-coronavirus-airborne}{%
  \paragraph{Is the coronavirus
  airborne?}\label{is-the-coronavirus-airborne}}

  \begin{itemize}
  \tightlist
  \item
    The coronavirus
    \href{https://www.nytimes.com/2020/07/04/health/239-experts-with-one-big-claim-the-coronavirus-is-airborne.html?action=click\&pgtype=Article\&state=default\&region=MAIN_CONTENT_3\&context=storylines_faq}{can
    stay aloft for hours in tiny droplets in stagnant air}, infecting
    people as they inhale, mounting scientific evidence suggests. This
    risk is highest in crowded indoor spaces with poor ventilation, and
    may help explain super-spreading events reported in meatpacking
    plants, churches and restaurants.
    \href{https://www.nytimes.com/2020/07/06/health/coronavirus-airborne-aerosols.html?action=click\&pgtype=Article\&state=default\&region=MAIN_CONTENT_3\&context=storylines_faq}{It's
    unclear how often the virus is spread} via these tiny droplets, or
    aerosols, compared with larger droplets that are expelled when a
    sick person coughs or sneezes, or transmitted through contact with
    contaminated surfaces, said Linsey Marr, an aerosol expert at
    Virginia Tech. Aerosols are released even when a person without
    symptoms exhales, talks or sings, according to Dr. Marr and more
    than 200 other experts, who
    \href{https://academic.oup.com/cid/article/doi/10.1093/cid/ciaa939/5867798}{have
    outlined the evidence in an open letter to the World Health
    Organization}.
  \end{itemize}
\item ~
  \hypertarget{what-are-the-symptoms-of-coronavirus}{%
  \paragraph{What are the symptoms of
  coronavirus?}\label{what-are-the-symptoms-of-coronavirus}}

  \begin{itemize}
  \tightlist
  \item
    Common symptoms
    \href{https://www.nytimes.com/article/symptoms-coronavirus.html?action=click\&pgtype=Article\&state=default\&region=MAIN_CONTENT_3\&context=storylines_faq}{include
    fever, a dry cough, fatigue and difficulty breathing or shortness of
    breath.} Some of these symptoms overlap with those of the flu,
    making detection difficult, but runny noses and stuffy sinuses are
    less common.
    \href{https://www.nytimes.com/2020/04/27/health/coronavirus-symptoms-cdc.html?action=click\&pgtype=Article\&state=default\&region=MAIN_CONTENT_3\&context=storylines_faq}{The
    C.D.C. has also} added chills, muscle pain, sore throat, headache
    and a new loss of the sense of taste or smell as symptoms to look
    out for. Most people fall ill five to seven days after exposure, but
    symptoms may appear in as few as two days or as many as 14 days.
  \end{itemize}
\item ~
  \hypertarget{does-asymptomatic-transmission-of-covid-19-happen}{%
  \paragraph{Does asymptomatic transmission of Covid-19
  happen?}\label{does-asymptomatic-transmission-of-covid-19-happen}}

  \begin{itemize}
  \tightlist
  \item
    So far, the evidence seems to show it does. A widely cited
    \href{https://www.nature.com/articles/s41591-020-0869-5}{paper}
    published in April suggests that people are most infectious about
    two days before the onset of coronavirus symptoms and estimated that
    44 percent of new infections were a result of transmission from
    people who were not yet showing symptoms. Recently, a top expert at
    the World Health Organization stated that transmission of the
    coronavirus by people who did not have symptoms was ``very rare,''
    \href{https://www.nytimes.com/2020/06/09/world/coronavirus-updates.html?action=click\&pgtype=Article\&state=default\&region=MAIN_CONTENT_3\&context=storylines_faq\#link-1f302e21}{but
    she later walked back that statement.}
  \end{itemize}
\end{itemize}

A few cities with early outbreaks, notably Seattle, are believed to have
avoided enormous growth later by heeding the models available at the
time and taking action well ahead of the rest of the country.

``We knew the numbers we saw were just the tip of the iceberg, and that
there were much greater numbers below the surface,'' Jenny A. Durkan,
the mayor of Seattle, said in an interview. ``We had to act.''

Image

A highway sign in Seattle on April 15.Credit...Ruth Fremson/The New York
Times

Image

As early as March 14, much of Seattle was quiet as people avoided
gathering in normally busy public places, like Pioneer
Square.Credit...Grant Hindsley for The New York Times

\href{https://www.nytimes.com/2020/04/08/nyregion/new-york-coronavirus-response-delays.html}{City
and state officials in New York acted more slowly}, waiting until known
cases were at a higher level to shut down schools and issue a
stay-at-home order.
\href{https://www.nytimes.com/2020/03/16/nyregion/coronavirus-bill-de-blasio.html}{Mayor
Bill de Blasio was reluctant to embrace shutdowns} until mid-March,
citing the impact they would have on vulnerable New Yorkers.

``Even while we learn new things about this virus almost daily, one
thing remains consistent: New Yorkers were put at risk by the federal
government's total failure to provide us with adequate testing
capability,'' said the mayor's press secretary, Freddi Goldstein.

In mid-February, a month before New York City schools were closed, New
York City and San Francisco already had more than 600 people with
unidentified infections, and Seattle, Chicago and Boston already had
more than 100 people, the findings estimate. By March 1, as New York
confirmed its first case, the numbers there may already have surpassed
10,000.

From these primary travel hubs and a few other cities, the model shows,
the disease was then spread to other locations in the United States.

Dr. Vespignani said he and his research team warned officials of the
silent spread, posting some of their early projections in mid-February.
``We were talking to officials here, and it was the same reaction we got
in Italy, in the U.K., in Spain,'' Dr. Vespignani said. ``They told me,
`OK, that's happening on your computer, not in reality.' Look,'' he
added, ``No one's going to shut down a country based on a model.''

The virus moved under the radar swiftly in February and March, doctors
and researchers said, because few cities or states had adequate
surveillance systems in place. And testing, if it was being done at all,
was haphazard. Emergency rooms were busy preparing for the predicted
onslaught and likely missed some early virus-related deaths, and did not
have the time or tools to verify infections on the fly, experts said.

It was mid-March before teams at N.Y.U. and Mount Sinai began taking
samples for testing in New York. On Thursday, Gov. Andrew M. Cuomo
announced results from antibody testing of grocery store
shoppers---~which researchers warned were preliminary and could change
--- that suggests
\href{https://www.nytimes.com/2020/04/23/nyregion/coronavirus-antibodies-test-ny.html}{one
of every five} New Yorkers may have been infected.

The new findings from the model produce a range of possible outcomes for
when the virus may have infected 10 people in each city. In New York,
for example, the model shows that the first 10 infected people could
have been walking the streets of the city as early as the last week in
January, or as late as the middle of February. From there, the
infections in the centers of the outbreak grew exponentially.

Trevor Bedford, an associate professor at the Fred Hutchinson Cancer
Research Center and the University of Washington in Seattle, said it
became clear in late February that ``community transmission'' --- an
infectious outbreak --- was probably silently underway in Washington
after a single test result came back positive for someone who had no
symptoms.

Whatever the precise scale of the initial outbreak, that same dynamic
will accelerate once measures to mitigate the spread are relaxed without
other public health measures in place, Dr. Burke said. ``When you take
away social distancing, everything will go right through the roof,'' he
said.

Advertisement

\protect\hyperlink{after-bottom}{Continue reading the main story}

\hypertarget{site-index}{%
\subsection{Site Index}\label{site-index}}

\hypertarget{site-information-navigation}{%
\subsection{Site Information
Navigation}\label{site-information-navigation}}

\begin{itemize}
\tightlist
\item
  \href{https://help.nytimes.com/hc/en-us/articles/115014792127-Copyright-notice}{©~2020~The
  New York Times Company}
\end{itemize}

\begin{itemize}
\tightlist
\item
  \href{https://www.nytco.com/}{NYTCo}
\item
  \href{https://help.nytimes.com/hc/en-us/articles/115015385887-Contact-Us}{Contact
  Us}
\item
  \href{https://www.nytco.com/careers/}{Work with us}
\item
  \href{https://nytmediakit.com/}{Advertise}
\item
  \href{http://www.tbrandstudio.com/}{T Brand Studio}
\item
  \href{https://www.nytimes.com/privacy/cookie-policy\#how-do-i-manage-trackers}{Your
  Ad Choices}
\item
  \href{https://www.nytimes.com/privacy}{Privacy}
\item
  \href{https://help.nytimes.com/hc/en-us/articles/115014893428-Terms-of-service}{Terms
  of Service}
\item
  \href{https://help.nytimes.com/hc/en-us/articles/115014893968-Terms-of-sale}{Terms
  of Sale}
\item
  \href{https://spiderbites.nytimes.com}{Site Map}
\item
  \href{https://help.nytimes.com/hc/en-us}{Help}
\item
  \href{https://www.nytimes.com/subscription?campaignId=37WXW}{Subscriptions}
\end{itemize}
