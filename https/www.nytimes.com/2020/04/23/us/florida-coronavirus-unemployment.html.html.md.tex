Sections

SEARCH

\protect\hyperlink{site-content}{Skip to
content}\protect\hyperlink{site-index}{Skip to site index}

\href{https://www.nytimes.com/section/us}{U.S.}

\href{https://myaccount.nytimes.com/auth/login?response_type=cookie\&client_id=vi}{}

\href{https://www.nytimes.com/section/todayspaper}{Today's Paper}

\href{/section/us}{U.S.}\textbar{}`Florida Is a Terrible State to Be an
Unemployed Person'

\url{https://nyti.ms/2VTHKLE}

\begin{itemize}
\item
\item
\item
\item
\item
\item
\end{itemize}

Advertisement

\protect\hyperlink{after-top}{Continue reading the main story}

Supported by

\protect\hyperlink{after-sponsor}{Continue reading the main story}

\hypertarget{florida-is-a-terrible-state-to-be-an-unemployed-person}{%
\section{`Florida Is a Terrible State to Be an Unemployed
Person'}\label{florida-is-a-terrible-state-to-be-an-unemployed-person}}

Florida has emerged as one of the slowest states in the nation to
process an avalanche of unemployment claims since the coronavirus hit.

\includegraphics{https://static01.nyt.com/images/2020/04/23/us/23VIRUS-FLORIDA-virgile/merlin_171835740_8bd3c107-fb1a-4ac7-b24d-b7bf039cf90f-articleLarge.jpg?quality=75\&auto=webp\&disable=upscale}

\href{https://www.nytimes.com/by/patricia-mazzei}{\includegraphics{https://static01.nyt.com/images/2018/11/28/multimedia/author-patricia-mazzei/author-patricia-mazzei-thumbLarge.png}}\href{https://www.nytimes.com/by/sabrina-tavernise}{\includegraphics{https://static01.nyt.com/images/2018/06/13/multimedia/author-sabrina-tavernise/author-sabrina-tavernise-thumbLarge.jpg}}

By \href{https://www.nytimes.com/by/patricia-mazzei}{Patricia Mazzei}
and \href{https://www.nytimes.com/by/sabrina-tavernise}{Sabrina
Tavernise}

\begin{itemize}
\item
  Published April 23, 2020Updated April 24, 2020
\item
  \begin{itemize}
  \item
  \item
  \item
  \item
  \item
  \item
  \end{itemize}
\end{itemize}

MIAMI --- After Ernst Virgile lost his job at the Fort Lauderdale
airport, he sat up late at the computer in his living room night after
night, trying to apply for unemployment, refreshing his browser again
and again while his wife was sleeping.

He worried about their three children, and about making payments on the
car and the house he and his wife, who also lost her job at the airport
on the same day, worked so hard to buy in 2018. He sometimes got in the
car and drove around by himself to be able to worry without having to
hide it.

On April 3, after more than a week of trying, Mr. Virgile, 37, was
finally able to get his claim lodged in Florida's overwhelmed system.
But he has yet to see a penny of unemployment compensation.

``My job is to not let them see,'' he said of his children. ``I have to
stay strong. But I need to pay the water and electric. What am I going
to do? We don't have anything saved.''

New unemployment figures released on Thursday showed that another 4.4
million people
\href{https://www.nytimes.com/2020/04/23/business/economy/unemployment-claims-coronavirus.html?action=click\&module=Top\%20Stories\&pgtype=Homepage}{filed
unemployment claims last week} amid the coronavirus crisis, bringing the
five-week total to more than 26 million.

Many states are scrambling to process
\href{https://www.nytimes.com/2020/04/16/business/economy/unemployment-numbers-coronavirus.html}{an
avalanche of jobless claims}, struggling with overloaded websites and
unanswered phones. But Florida has emerged as one of the slowest in the
nation.

Hundreds of thousands of workers --- many from Florida's once-booming
service industry --- have been waiting for weeks for a check. It has
taken some as long as that to file. As the website became unusable under
the weight of the traffic, the state agreed this month to accept paper
applications, a tacit acknowledgment that the system was all but broken.
Florida's breakdown became a national symbol of distress, when footage
of
\href{https://www.local10.com/news/local/2020/04/07/here-is-where-to-download-application-for-unemployment-benefits/}{a
snaking line} for those applications outside the public library in
Hialeah, a blue-collar city outside Miami, went viral.

The debacle has become an embarrassment for
\href{https://www.nytimes.com/2020/04/01/us/coronavirus-florida-de-santis-trump.html}{Gov.
Ron DeSantis}, a Republican, who has had to repeatedly address the
shortcomings. He called the system ``cumbersome'' last week and
acknowledged that only 34,000 of 850,000 pending claims had been paid, a
number that rose over the past week to 117,000.

The governor appointed an unemployment czar and signed executive orders
waiving some requirements to ease the traffic on the website.

But the fixes follow what experts say has been an intentional weakening
of the unemployment system over a decade, aimed at reducing taxes on
employers, that has left Florida particularly ill-equipped to handle the
crisis. The state pays one of the lowest levels of benefits in the
nation: The maximum is just \$275 a week.

``Florida is a terrible state to be an unemployed person,'' said Michele
Evermore, an unemployment insurance expert at the National Employment
Law Project in Washington. ``It's hard to get in. Once you do, it's easy
to get disqualified. The benefit level is way below average. And that
was before the crisis.''

Mr. DeSantis said that easing the benefits gridlock is his top priority,
and has blamed the record number of claims for the breakdown. ``Not
nearly enough'' applications have been processed, he said.

The governor inherited a website and benefit restrictions from his
predecessor, former Gov. Rick Scott, that saved the state and employers
money --- and kept the benefit rate low --- when the economy was
chugging along. But the system has proved unworkable in a crisis.

Data for precise national comparisons will not be out for some time. But
as a measure of the state's sluggishness, Ms. Evermore pointed out that
its unemployment trust fund, a kind of bank account for unemployment
funds, contains more money now than it did on March 1, while many other
states are in danger of running out in a few weeks.

``Florida's fund has gone up,'' she said. ``This means benefits aren't
getting to people in their time of need.''

The situation is urgent. State Representative Anna V. Eskamani, an
Orlando Democrat, said her office has been trying to assist some 800
people from across the state who have reached out, desperate for help.
She and her staff members track each request on an Excel spreadsheet.
Some contacts come in via Instagram message.

``Somebody called me this morning from Ormond Beach,'' Ms. Eskamani
said. ``I got an email from someone in Panama City last night. We're
just trying to show up for everyone as best we can.''

Some people have received two weeks of benefits, Ms. Eskamani said. But
those successful cases are few and far between.

A labor union representing South Florida hotel and casino workers, Unite
Here Local 355, said 98 percent of its 7,000 members were out of work.
``Virtually none of our members have received unemployment benefits,''
said Wendi Walsh, the union's secretary-treasurer.

Many have been locked out when they do not remember a PIN they created
years ago. Calling operators to reset it results in long waits and
disconnected phone calls.

José Garrido, 41, a former doorman at the Fontainebleau hotel in Miami
Beach, filed his application on March 22. But he has yet to receive any
benefits.

``I'm short on rent,'' said Mr. Garrido, a father of two. ``We are not
going to have any health insurance.''

``I'm terrified,'' he said.

\includegraphics{https://static01.nyt.com/images/2020/04/23/us/23VIRUS-FLORIDA-desantis/merlin_171687399_be38e5dd-98e4-4e36-8687-aaecff650006-articleLarge.jpg?quality=75\&auto=webp\&disable=upscale}

The current unemployment system in Florida dates back to 2011, when the
state legislature and Mr. Scott, a Republican and the governor at the
time,
\href{https://www.nytimes.com/2011/05/08/us/08florida.html}{enacted a
series of major changes}. It was not long after the Great Recession, and
the federal government had increased unemployment taxes on businesses.
In response, the Republican-controlled legislature set out to reduce
that benefit to be able to bring taxes back down.

Floridians, who once could file by phone, now had to file online, and
faced a set of new electronic filing requirements that made the process
of establishing eligibility one of the most onerous in the nation, Ms.
Evermore said. The online system was hard to use, offered very little
customer service and limited access for Spanish speakers. A cumbersome
skills test had appeared. People had to prove, on a complicated online
form, that they had applied to at least five jobs a week. Benefits went
to as few as 12 weeks from 26.

The result
\href{https://www.nytimes.com/2014/01/11/us/floridas-site-said-to-delay-millions-in-aid-to-jobless.html}{was
disastrous} for unemployed Floridians. By 2015,
\href{https://s27147.pcdn.co/wp-content/uploads/Aint-No-Sunshine-Florida-Unemployment-Insurance.pdf}{just
39 percent of workers who applied for benefits ever received a first
payment}, compared with 68 percent nationally. That number has not
changed much, Ms. Evermore said.

When the virus hit, Florida was at the bottom of the pack. Just 11
percent of unemployed Floridians were receiving unemployment insurance
in 2019, compared with about 52 percent of unemployed people in
Massachusetts and 57 percent in New Jersey, according to data from the
Department of Labor. Florida was the second-worst in the country last
year by a hair, just after North Carolina.

Mr. DeSantis said last week that the state was slow to process claims
before the virus.

``If you applied in January, I mean, it was a cumbersome process --- it
would take several weeks,'' he said. ``But when the unemployment rate is
3 percent, it's a little bit different than what we have now.''

This week, Mr. DeSantis sounded more exasperated: ``Look, this system,
the fact that the state paid \$77 million for this thing --- it's a
jalopy.''

State Senator Joe Gruters of Sarasota, chairman of the Republican Party
of Florida, went further in a Twitter post earlier this month: ``\$77
million? Someone should go to jail over that.''

Mr. Scott, now the state's junior senator, has taken the brunt of the
criticism for the troubled system. Asked about Mr. DeSantis's blunt
assessment, a spokesman for Mr. Scott said in a statement on Wednesday
that the former governor did not ``have time for dumb political
squabbles.'' Mr. Scott has noted that Deloitte Consulting, the
well-connected contractor hired to build the website, was chosen by his
predecessor, Charlie Crist.

Mr. Scott ran on jobs after the recession, a period during which Florida
paid so many claims that it wiped out the unemployment trust fund. The
reforms he enacted were intended to adapt to economic conditions,
raising overall payments when unemployment was high and jobs scarce but
bringing them down in boom times.

Then there was the glitchy website. State audits in
\href{https://flauditor.gov/pages/pdf_files/2015-107.pdf}{2015},
\href{https://flauditor.gov/pages/pdf_files/2017-039.pdf}{2016} and
\href{https://flauditor.gov/pages/pdf_files/2019-183.pdf}{2019} found a
slew of problems, but neither Mr. Scott nor Mr. DeSantis did much to fix
them, and the troubles persisted.

To deal with the crush of applications that began in mid-March, the
Florida Department of Economic Opportunity installed 100 new servers and
promised to reassign 2,000 employees from other state agencies to enter
data from paper applications. The department built a makeshift portal to
accommodate more applicants, but those applications must still be
migrated to the original system.

Another portal, for gig workers and independent contractors who now
qualify for federal benefits, is still pending. Secretary Jonathan R.
Satter of the Department of Management Services, whom Mr. DeSantis
tapped last week to oversee the unemployment morass, said on April 16
that the portal would debut in a week to 10 days.

Among the requirements waived by Mr. DeSantis is that applicants show
they have been job hunting, because there are so few jobs to be had. But
the governor says he lacks the executive authority to raise the maximum
benefit of \$275 a week. (A new federal relief law will augment that; it
pays
\href{https://slack-redir.net/link?url=https\%3A\%2F\%2Fwww.nytimes.com\%2Finteractive\%2F2020\%2F04\%2F23\%2Fbusiness\%2Feconomy\%2Funemployment-benefits-stimulus-coronavirus.html}{\$600
per week} on top of state benefits.)

``We're the most Scrooge-like state in the country on benefits,'' said
State Senator José Javier Rodríguez, a Miami Democrat.

The Department of Economic Opportunity has not said if claims would be
retroactive to the date of every applicant's layoff. Claims filed early
on in the crisis were excluded from a waiver that scrapped the ``wait
week'' typically required before people can get their first benefit
checks.

``We want to maximize the benefits for them,'' said Tiffany Vause, a
spokeswoman. ``A lot of that is going to be done on a case-by-case
basis.''

Image

Esther Ortega, a laid-off stadium bartender, filed a paper application
for unemployment earlier this month after trying frantically for days to
file online.Credit...Angel Valentin for The New York Times

But for people going weeks without income, the delay is becoming dire.
Esther Ortega, a laid-off stadium bartender, filed a paper application
earlier this month after trying frantically for days to file online. She
could have gotten it for free at the Hialeah public library but did not
want to wait in line and possibly bring the virus home to her
11-year-old daughter and her aunt and stepfather, who are both in their
70s, so she paid to print it at a nearby store.

She has not seen any benefits yet.

``Every day at 6 a.m., believe me, I open my app on my phone to check my
bank account,'' said Ms. Ortega, who is 41 and became a widow in
February. ``It's very frustrating. It's just energy draining. Like, I
don't know what the governor is waiting for. ''

Patricia Mazzei reported from Miami, and Sabrina Tavernise from
Washington. Emily Badger and Alicia Parlapiano contributed reporting
from Washington.

Advertisement

\protect\hyperlink{after-bottom}{Continue reading the main story}

\hypertarget{site-index}{%
\subsection{Site Index}\label{site-index}}

\hypertarget{site-information-navigation}{%
\subsection{Site Information
Navigation}\label{site-information-navigation}}

\begin{itemize}
\tightlist
\item
  \href{https://help.nytimes.com/hc/en-us/articles/115014792127-Copyright-notice}{©~2020~The
  New York Times Company}
\end{itemize}

\begin{itemize}
\tightlist
\item
  \href{https://www.nytco.com/}{NYTCo}
\item
  \href{https://help.nytimes.com/hc/en-us/articles/115015385887-Contact-Us}{Contact
  Us}
\item
  \href{https://www.nytco.com/careers/}{Work with us}
\item
  \href{https://nytmediakit.com/}{Advertise}
\item
  \href{http://www.tbrandstudio.com/}{T Brand Studio}
\item
  \href{https://www.nytimes.com/privacy/cookie-policy\#how-do-i-manage-trackers}{Your
  Ad Choices}
\item
  \href{https://www.nytimes.com/privacy}{Privacy}
\item
  \href{https://help.nytimes.com/hc/en-us/articles/115014893428-Terms-of-service}{Terms
  of Service}
\item
  \href{https://help.nytimes.com/hc/en-us/articles/115014893968-Terms-of-sale}{Terms
  of Sale}
\item
  \href{https://spiderbites.nytimes.com}{Site Map}
\item
  \href{https://help.nytimes.com/hc/en-us}{Help}
\item
  \href{https://www.nytimes.com/subscription?campaignId=37WXW}{Subscriptions}
\end{itemize}
