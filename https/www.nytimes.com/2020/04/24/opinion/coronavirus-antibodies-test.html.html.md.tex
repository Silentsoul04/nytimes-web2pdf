Sections

SEARCH

\protect\hyperlink{site-content}{Skip to
content}\protect\hyperlink{site-index}{Skip to site index}

\href{https://myaccount.nytimes.com/auth/login?response_type=cookie\&client_id=vi}{}

\href{https://www.nytimes.com/section/todayspaper}{Today's Paper}

\href{/section/opinion}{Opinion}\textbar{}There Are More Cases Than We
Thought. Is That Good News?

\href{https://nyti.ms/3cI1xEx}{https://nyti.ms/3cI1xEx}

\begin{itemize}
\item
\item
\item
\item
\item
\end{itemize}

Advertisement

\protect\hyperlink{after-top}{Continue reading the main story}

\href{/section/opinion}{Opinion}

Supported by

\protect\hyperlink{after-sponsor}{Continue reading the main story}

\hypertarget{there-are-more-cases-than-we-thought-is-that-good-news}{%
\section{There Are More Cases Than We Thought. Is That Good
News?}\label{there-are-more-cases-than-we-thought-is-that-good-news}}

Not really. Here's why.

\href{https://www.nytimes.com/by/david-leonhardt}{\includegraphics{https://static01.nyt.com/images/2018/04/02/opinion/david-leonhardt/david-leonhardt-thumbLarge.png}}

By \href{https://www.nytimes.com/by/david-leonhardt}{David Leonhardt}

Opinion Columnist

\begin{itemize}
\item
  April 24, 2020
\item
  \begin{itemize}
  \item
  \item
  \item
  \item
  \item
  \end{itemize}
\end{itemize}

\includegraphics{https://static01.nyt.com/images/2020/04/24/opinion/24leonhardt-newsletter/merlin_171835386_a7980b50-87b0-4e96-bedc-34cd0b227369-articleLarge.jpg?quality=75\&auto=webp\&disable=upscale}

\emph{This article is part of David Leonhardt's newsletter. You can}
\href{https://www.nytimes.com/newsletters/opiniontoday?action=click\&module=Intentional\&pgtype=Article}{\emph{sign
up here}} \emph{to receive it each weekday.}

My first reaction to the news that the coronavirus seems to have spread
more widely than initially understood --- potentially to
\href{https://www.nytimes.com/2020/04/23/nyregion/coronavirus-antibodies-test-ny.html}{20
percent of New York City residents}, for example --- was optimism.

If more people have had the virus, it means that its death rate is
lower. That's just math. We have a decent idea of how many people
\href{https://coronavirus.jhu.edu/map.html}{have died} from the virus.
If the total pool of people who have had it is larger than the early
estimates suggested, the chances that any individual patient will die
from it are, by definition, smaller --- closer to about 0.5 percent on
average, instead of 3 or 4 percent, as initially seemed possible.

But as I spent some time talking to public health experts yesterday, my
optimism faded. The new statistics still suggest that the overall death
toll could be catastrophic, and on the high end of the range of the
\href{https://www.washingtonpost.com/outlook/2020/04/14/coronavirus-models-ihme-ic/}{various
statistical models}.

How could that be? There are two main reasons.

One, the fact that more people may have already had the virus also
suggests that it's more contagious than the initial numbers suggested
--- that any one person with the virus tends to pass it to a greater
number of others. And if it's more contagious, it may be harder to
contain in coming months. As society begins to reopen, the virus could
spread more quickly. The number of Americans who get it before a vaccine
is developed would then be larger.

Two, even if the death rate is lower than feared, it's still very high.
``It is still, with these new findings, many times more deadly than
influenza,''
\href{https://twitter.com/cmyeaton?ref_src=twsrc\%5Egoogle\%7Ctwcamp\%5Eserp\%7Ctwgr\%5Eauthor}{Caitlin
Rivers}, an epidemic researcher at Johns Hopkins University, told me.
The best current guess is that the death rate for coronavirus is about
five times higher than that of seasonal influenza.

A few basic calculations show how scary a 0.5 percent death rate is. If
about one in three Americans ultimately get the virus --- or 110 million
people --- more than 500,000 would die. If 200 million people get it, 1
million would die.

\href{https://www.nytimes.com/topic/person/ezekiel-j-emanuel}{Ezekiel
Emanuel} of the University of Pennsylvania pointed out to me that about
20 percent of virus fatalities so far in the United States have been
among people aged between 35 and 64. If the total number of deaths ends
up in the ranges I've mentioned here, the virus could end up being the
No. 1 killer of people in that age group, surpassing both cancer and
heart disease.

The latest news, Emanuel said, ``doesn't make any of the goals you want
to reach easier.''

As I've
\href{https://www.nytimes.com/2020/04/10/opinion/coronavirus-social-distancing.html}{written
before}, it's likely that we have a long and very difficult fight ahead
of us.

\textbf{For more} \ldots{}

\begin{itemize}
\tightlist
\item
  \href{https://globalhealth.washington.edu/faculty/jared-baeten}{Jared
  Baeten}, a vice dean of the School of Public Health at the University
  of Washington, offered this perspective on the virus's spread, in an
  email:
\end{itemize}

\begin{quote}
It isn't a surprise --- the evidence has been growing for weeks that the
number of infections, especially asymptomatic infections, is far greater
than are being found through testing. Thus, the death rate is lower.
That gives me a bit of optimism, at least for individuals who get a
positive test. But, it doesn't make me optimistic yet --- there's still
a large fraction of the population out there who hasn't been infected
yet.
\end{quote}

\begin{itemize}
\tightlist
\item
  \href{https://twitter.com/ScottGottliebMD/status/1251207647166160901}{Scott
  Gottlieb} of the American Enterprise Institute, writing about two
  studies that suggested a wider than expected spread of infections
  \href{https://www.nytimes.com/2020/04/21/health/coronavirus-antibodies-california.html}{in
  California}: ``This probably aligns with what overall national
  exposure may be, on order of about 5 percent once we do wide serology
  \ldots{} The data so far suggest that nationally, total exposure is
  still low.''
\end{itemize}

\emph{If you are not a subscriber to this newsletter, you can}
\href{https://www.nytimes.com/newsletters/david-leonhardt}{\emph{subscribe
here}}\emph{. You can also join me on}
\href{https://twitter.com/DLeonhardt}{\emph{Twitter (@DLeonhardt)}}
\emph{and}
\href{https://www.facebook.com/DavidRLeonhardt/}{\emph{Facebook}}\emph{.}

\emph{Follow The New York Times Opinion section on}
\href{https://www.facebook.com/nytopinion}{\emph{Facebook}}\emph{,}
\href{http://twitter.com/NYTOpinion}{\emph{Twitter (@NYTopinion)}}
\emph{and}
\href{https://www.instagram.com/nytopinion/}{\emph{Instagram}}\emph{.}

Advertisement

\protect\hyperlink{after-bottom}{Continue reading the main story}

\hypertarget{site-index}{%
\subsection{Site Index}\label{site-index}}

\hypertarget{site-information-navigation}{%
\subsection{Site Information
Navigation}\label{site-information-navigation}}

\begin{itemize}
\tightlist
\item
  \href{https://help.nytimes.com/hc/en-us/articles/115014792127-Copyright-notice}{©~2020~The
  New York Times Company}
\end{itemize}

\begin{itemize}
\tightlist
\item
  \href{https://www.nytco.com/}{NYTCo}
\item
  \href{https://help.nytimes.com/hc/en-us/articles/115015385887-Contact-Us}{Contact
  Us}
\item
  \href{https://www.nytco.com/careers/}{Work with us}
\item
  \href{https://nytmediakit.com/}{Advertise}
\item
  \href{http://www.tbrandstudio.com/}{T Brand Studio}
\item
  \href{https://www.nytimes.com/privacy/cookie-policy\#how-do-i-manage-trackers}{Your
  Ad Choices}
\item
  \href{https://www.nytimes.com/privacy}{Privacy}
\item
  \href{https://help.nytimes.com/hc/en-us/articles/115014893428-Terms-of-service}{Terms
  of Service}
\item
  \href{https://help.nytimes.com/hc/en-us/articles/115014893968-Terms-of-sale}{Terms
  of Sale}
\item
  \href{https://spiderbites.nytimes.com}{Site Map}
\item
  \href{https://help.nytimes.com/hc/en-us}{Help}
\item
  \href{https://www.nytimes.com/subscription?campaignId=37WXW}{Subscriptions}
\end{itemize}
