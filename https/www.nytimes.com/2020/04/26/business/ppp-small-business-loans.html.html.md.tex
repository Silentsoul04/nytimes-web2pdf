Sections

SEARCH

\protect\hyperlink{site-content}{Skip to
content}\protect\hyperlink{site-index}{Skip to site index}

\href{https://www.nytimes.com/section/business}{Business}

\href{https://myaccount.nytimes.com/auth/login?response_type=cookie\&client_id=vi}{}

\href{https://www.nytimes.com/section/todayspaper}{Today's Paper}

\href{/section/business}{Business}\textbar{}Small-Business Loan Program,
Chaotic From Start, Gets 2nd Round

\url{https://nyti.ms/355QrGS}

\begin{itemize}
\item
\item
\item
\item
\item
\end{itemize}

\href{https://www.nytimes.com/news-event/coronavirus?action=click\&pgtype=Article\&state=default\&region=TOP_BANNER\&context=storylines_menu}{The
Coronavirus Outbreak}

\begin{itemize}
\tightlist
\item
  live\href{https://www.nytimes.com/2020/08/03/world/coronavirus-covid-19.html?action=click\&pgtype=Article\&state=default\&region=TOP_BANNER\&context=storylines_menu}{Latest
  Updates}
\item
  \href{https://www.nytimes.com/interactive/2020/us/coronavirus-us-cases.html?action=click\&pgtype=Article\&state=default\&region=TOP_BANNER\&context=storylines_menu}{Maps
  and Cases}
\item
  \href{https://www.nytimes.com/interactive/2020/science/coronavirus-vaccine-tracker.html?action=click\&pgtype=Article\&state=default\&region=TOP_BANNER\&context=storylines_menu}{Vaccine
  Tracker}
\item
  \href{https://www.nytimes.com/2020/08/02/us/covid-college-reopening.html?action=click\&pgtype=Article\&state=default\&region=TOP_BANNER\&context=storylines_menu}{College
  Reopening}
\item
  \href{https://www.nytimes.com/live/2020/08/03/business/stock-market-today-coronavirus?action=click\&pgtype=Article\&state=default\&region=TOP_BANNER\&context=storylines_menu}{Economy}
\end{itemize}

Advertisement

\protect\hyperlink{after-top}{Continue reading the main story}

Supported by

\protect\hyperlink{after-sponsor}{Continue reading the main story}

\hypertarget{small-business-loan-program-chaotic-from-start-gets-2nd-round}{%
\section{Small-Business Loan Program, Chaotic From Start, Gets 2nd
Round}\label{small-business-loan-program-chaotic-from-start-gets-2nd-round}}

The Paycheck Protection Program distributed \$349 billion in less than
two weeks, but lenders and borrowers confronted confusion at every step.

\includegraphics{https://static01.nyt.com/images/2020/04/26/business/00sba5/merlin_171925779_16652b3b-8e6f-42c5-b812-990bacde136f-articleLarge.jpg?quality=75\&auto=webp\&disable=upscale}

\href{https://www.nytimes.com/by/stacy-cowley}{\includegraphics{https://static01.nyt.com/images/2018/10/03/multimedia/author-stacy-cowley/author-stacy-cowley-thumbLarge.png}}\href{https://www.nytimes.com/by/alan-rappeport}{\includegraphics{https://static01.nyt.com/images/2018/06/12/multimedia/author-alan-rappeport/author-alan-rappeport-thumbLarge-v2.png}}\href{https://www.nytimes.com/by/emily-flitter}{\includegraphics{https://static01.nyt.com/images/2019/06/19/reader-center/author-emily-flitter/author-emily-flitter-thumbLarge.png}}

By \href{https://www.nytimes.com/by/stacy-cowley}{Stacy Cowley},
\href{https://www.nytimes.com/by/alan-rappeport}{Alan Rappeport} and
\href{https://www.nytimes.com/by/emily-flitter}{Emily Flitter}

\begin{itemize}
\item
  Published April 26, 2020Updated May 13, 2020
\item
  \begin{itemize}
  \item
  \item
  \item
  \item
  \item
  \end{itemize}
\end{itemize}

One day before the federal government's \$349 billion aid program for
\href{https://www.nytimes.com/2020/05/13/business/paycheck-protection-program-small-business.html}{small
businesses} was set to go live, the chief executive of a Minnesota bank
was frantically dialing officials in Washington. Grand Rapids State Bank
needed more time to understand the program, said the executive, Noah W.
Wilcox, even as it faced a crush of borrowers. His pleas went unheeded.

``Somebody put a stake in the ground and it just wasn't moving,'' said
Mr. Wilcox, who is also the chairman of the Independent Community
Bankers of America, which represents about 5,000 institutions.
``Secretary Mnuchin was not budging one inch from the date he initially
set.''

Late on April 2, just hours before the opening, the Treasury Department
released details on the Paycheck Protection Program. Treasury Secretary
Steven Mnuchin told would-be borrowers that they would receive funds
within a day, but the program, which promised speed, also brought chaos.

The rules confused lenders, including small community banks as well as
Wall Street firms less familiar with the Small Business Administration,
which was establishing the program. They were unsure about who would
qualify for
\href{https://www.nytimes.com/2020/05/13/business/paycheck-protection-program-small-business.html}{loans},
how the loans would be distributed and how they would eventually be
forgiven.

Some of the program's rules were lax, allowing more than 200 publicly
traded companies to obtain loans totaling more than \$750 million. Yet,
some small businesses that did get loans called the rules too
restrictive, saying that it would be more helpful if they could use the
funds to retool their operations for after the pandemic rather than
paying employees.

The government has since published new guidance strongly discouraging
public companies from using the program and urged those that did take
the money to return it. Some have; others haven't.

And the cash ran out in 13 days, leaving many small business owners
waiting in line and increasingly desperate. As the federal government
prepares to replenish the fund with \$310 billion more, lenders expect
the second round, which will open for applications at 10:30 a.m. on
Monday, to be depleted even faster.

``As soon as they turn the switch on, that money will be gone,'' said
Tony Wilkinson, the chief executive of the National Association of
Government Guaranteed Lenders, a trade group.

\hypertarget{demand-through-the-roof}{%
\subsection{Demand `Through the Roof'}\label{demand-through-the-roof}}

\includegraphics{https://static01.nyt.com/images/2020/04/26/business/00sba4a/merlin_171925731_a8e50e5f-d88f-4b10-84c4-c50bd3b3aab8-articleLarge.jpg?quality=75\&auto=webp\&disable=upscale}

Small companies --- those with under 500 workers --- employ nearly half
of America's private sector work force. Most run on thin margins and
have scant savings. For restaurants, gyms and other small businesses
that depend entirely on people walking in the door, sales fell to zero
after stay-at-home orders went into effect.

\hypertarget{latest-updates-economy}{%
\section{\texorpdfstring{\href{https://www.nytimes.com/live/2020/08/03/business/stock-market-today-coronavirus?action=click\&pgtype=Article\&state=default\&region=MAIN_CONTENT_1\&context=storylines_live_updates}{Latest
Updates:
Economy}}{Latest Updates: Economy}}\label{latest-updates-economy}}

\href{https://www.nytimes.com/live/2020/08/03/business/stock-market-today-coronavirus?action=click\&pgtype=Article\&state=default\&region=MAIN_CONTENT_1\&context=storylines_live_updates\#the-chicago-fed-president-says-its-up-to-congress-to-save-the-economy}{9h
ago}

\href{https://www.nytimes.com/live/2020/08/03/business/stock-market-today-coronavirus?action=click\&pgtype=Article\&state=default\&region=MAIN_CONTENT_1\&context=storylines_live_updates\#the-chicago-fed-president-says-its-up-to-congress-to-save-the-economy}{The
Chicago Fed president says it's up to Congress to save the economy.}

\href{https://www.nytimes.com/live/2020/08/03/business/stock-market-today-coronavirus?action=click\&pgtype=Article\&state=default\&region=MAIN_CONTENT_1\&context=storylines_live_updates\#faa-says-boeing-has-effectively-mitigated-defects-in-the-737-max}{10h
ago}

\href{https://www.nytimes.com/live/2020/08/03/business/stock-market-today-coronavirus?action=click\&pgtype=Article\&state=default\&region=MAIN_CONTENT_1\&context=storylines_live_updates\#faa-says-boeing-has-effectively-mitigated-defects-in-the-737-max}{F.A.A.
says Boeing has `effectively mitigated' defects in the 737 Max.}

\href{https://www.nytimes.com/live/2020/08/03/business/stock-market-today-coronavirus?action=click\&pgtype=Article\&state=default\&region=MAIN_CONTENT_1\&context=storylines_live_updates\#small-businesses-got-emergency-loans-but-not-what-they-expected}{12h
ago}

\href{https://www.nytimes.com/live/2020/08/03/business/stock-market-today-coronavirus?action=click\&pgtype=Article\&state=default\&region=MAIN_CONTENT_1\&context=storylines_live_updates\#small-businesses-got-emergency-loans-but-not-what-they-expected}{Small
businesses got emergency loans, but not what they expected.}

\href{https://www.nytimes.com/live/2020/08/03/business/stock-market-today-coronavirus?action=click\&pgtype=Article\&state=default\&region=MAIN_CONTENT_1\&context=storylines_live_updates}{See
more updates}

More live coverage:
\href{https://www.nytimes.com/2020/08/03/world/coronavirus-covid-19.html?action=click\&pgtype=Article\&state=default\&region=MAIN_CONTENT_1\&context=storylines_live_updates}{Global}

The Paycheck Protection Program was intended as a backstop to dissuade
layoffs. Under the program, small businesses could borrow up to two and
a half times their average monthly payroll cost. If they used the money
to retain workers and keep paying them for at least eight weeks, the
loan would be forgiven in full, and they could use a portion of the cash
for certain other expenses, like rent and utilities.

``We knew demand would be through the roof,'' said Jim Donnelly, the
chief commercial officer at Bangor Savings Bank in Maine.

The program was hastily designed after a difficult debate in Congress.
Democrats had wanted the government to provide cash infusions to small
businesses through tax rebates or by having the Treasury work directly
with payroll processors to administer payments. They had also wanted it
to cover 16 weeks of payroll, rather than eight. Republicans wanted to
steer the program through private sector financial institutions. They
won.

The government adapted a decades-old program run by the S.B.A. that
guaranteed small business loans issued by a network of banks nationwide.
That allowed the government to essentially outsource most of the
program's legwork to banks, making its adoption speedier. But the S.B.A.
and the banks had never before operated a program of this scale. Last
year, the S.B.A. backed around \$30 billion in loans. Now, it was
expected to process more than 10 times that volume, in just a few weeks.

And the administration wanted the money out the door immediately
beginning on Friday, April 3. ``This will be up and running tomorrow,''
Mr. Mnuchin said on Thursday night. ``You get the money. You'll get it
the same day.''

To bankers, that was an absurd promise. It typically took them days to
prepare even the simplest loan paperwork and go over the fine print with
borrowers. As Mr. Mnuchin spoke, they were still waiting for critical
information and materials. For example, the S.B.A. usually required
lenders to use a specific promissory note for loans it guaranteed. The
agency had not yet provided that note for the new program's loans.

Image

The aid program began on April 3 and quickly ran out of
cash.Credit...Nick Oxford for The New York Times

Image

Energy drinks fueled BancFirst's war room.Credit...Nick Oxford for The
New York Times

Many community banks decided to simply start lending and accept the risk
that some details would need to be hashed out later. BancFirst,
Oklahoma's largest S.B.A. lender, set up a war room, enlisted employees
from across the bank, and ran six-hour work shifts around the clock
starting Friday night.

``We went 24/7 for four consecutive days,'' said Kent Faison, the
president of BancFirst's commercial capital group. The volume was a
``fire hose'' like he had never seen before. Last year, his group made
around 130 S.B.A. loans. This month, it issued more than 5,500.

Larger banks struggled to handle the staggering customer demand,
especially as bank divisions that don't typically deal with the S.B.A.
were pulled in. Banks also imposed their own rules on who to lend to.

Bank of America started accepting applications right away, but its rules
blocked many of its customers. JPMorgan Chase's customers found
themselves trapped in an enormous backlog. Wells Fargo, constrained by
lending restrictions imposed by the Federal Reserve for its history of
bad behavior, told most of its applicants that it would not be able to
help them. And Citibank waited days to even begin taking applications
from most of its small business customers.

\hypertarget{who-gets-a-loan-who-doesnt}{%
\subsection{Who Gets a Loan? Who
Doesn't?}\label{who-gets-a-loan-who-doesnt}}

Image

``I feel a strong moral obligation to keep this money circulating in the
economy,'' said Thomas Fennell, owner of a translation business in
Omaha.Credit...Calla Kessler/The New York Times

As borrowers' applications flooded in, gaps became apparent. Under
longstanding S.B.A. rules, landlords, money lenders, political lobbyists
and some other businesses were excluded. So were felons convicted within
the last five years --- an exclusion that cut off some people trying to
rebuild their lives, and the employees who worked for them.

Other companies that seemed to test the program's boundaries slipped
through. In addition to several big restaurant chains, mining companies,
drugmakers, software developers and manufacturers with access to equity
markets and commercial loans were able to get loans using the
taxpayer-funded program.

For some small businesses, the program worked. In late March, sales for
Thomas Fennell's translation business in Omaha, C3 Translators, dried
up. He feared he would soon run out of cash to keep paying salaries for
himself and his husband, the company's two employees.

Mr. Fennell applied for a loan through the First National Bank of Omaha
two days after the program opened. A week later, his application was
approved. His check, for just under \$20,000, arrived on April 17.

Because the loan will cover payroll, he plans to spend some of his
savings to hire a local company to build a website for his business.

``I feel a strong moral obligation to keep this money circulating in the
economy,'' he said.

But even a successful loan application can pose challenges.

Naomi Pomeroy has two restaurants in Portland, Ore.: Beast, a 26-seat
establishment serving a six-course tasting menu at two communal tables,
and, across the street, Expatriate, a high-end cocktail bar. On
Thursday, she got an email saying that her application for a loan for
one of the two restaurants had been approved.

The news would have thrilled thousands of other business owners. Ms.
Pomeroy greeted it with grim reserve.

``I hesitate to take the money,'' she said.

To have the loan fully forgiven, she would have to rehire her employees
immediately and pay them for two months, even though she cannot yet
reopen her restaurant. The last of the 30 people she laid off five weeks
earlier just started receiving unemployment benefits --- and, with the
\$600 a week in additional federal benefits included in the stimulus
bill, all of them are being paid more on unemployment than they earned
working at her restaurants.

If Ms. Pomeroy takes the loan, she would prefer to spend the money
retooling her spaces to fit social distancing guidelines. For Beast,
that would mean getting rid of the two enormous tables and finding a way
to turn a profit while serving fewer customers at a time. But
renovations are not a permissible use of the emergency funds.

``If we could only open back up at 50 percent capacity and I was forced
to hire back 100 percent of my staff, then that doesn't make sense,''
she said. Ms. Pomeroy, who helped found an advocacy group, the
Independent Restaurant Coalition, said she was not sure the aid program
would really benefit anyone in the restaurant industry.

Lawmakers and Treasury officials played down the program's hitches. On
Sunday, Mr. Mnuchin told Fox News that the first round ``impacted about
30 million workers,'' and that he expected the second round to support
roughly the same number.

Lenders fear that many business owners will be left wanting, again, if
the new funding runs out fast. Economists have estimated that more than
\$1 trillion might be needed to fully meet the program's demand.

Bankers are preparing for a frenzy on Monday morning. Mr. Faison, at
BancFirst, already has more than 850 applications ready to go.

Many lenders are telling applicants to brace for disappointment.

Robert A. Klein, who runs a management consulting business in Great
Neck, N.Y., was one of thousands of Chase customers who got an email
this week warning him that the bank had ``an extremely large volume of
applications'' ahead of his.

``We expect that funds could run out again quickly,'' Chase wrote. ``We
wanted to give you this information, so that you can decide if you would
like to try applying with another lender.''

Jeanna Smialek contributed reporting.

Advertisement

\protect\hyperlink{after-bottom}{Continue reading the main story}

\hypertarget{site-index}{%
\subsection{Site Index}\label{site-index}}

\hypertarget{site-information-navigation}{%
\subsection{Site Information
Navigation}\label{site-information-navigation}}

\begin{itemize}
\tightlist
\item
  \href{https://help.nytimes.com/hc/en-us/articles/115014792127-Copyright-notice}{©~2020~The
  New York Times Company}
\end{itemize}

\begin{itemize}
\tightlist
\item
  \href{https://www.nytco.com/}{NYTCo}
\item
  \href{https://help.nytimes.com/hc/en-us/articles/115015385887-Contact-Us}{Contact
  Us}
\item
  \href{https://www.nytco.com/careers/}{Work with us}
\item
  \href{https://nytmediakit.com/}{Advertise}
\item
  \href{http://www.tbrandstudio.com/}{T Brand Studio}
\item
  \href{https://www.nytimes.com/privacy/cookie-policy\#how-do-i-manage-trackers}{Your
  Ad Choices}
\item
  \href{https://www.nytimes.com/privacy}{Privacy}
\item
  \href{https://help.nytimes.com/hc/en-us/articles/115014893428-Terms-of-service}{Terms
  of Service}
\item
  \href{https://help.nytimes.com/hc/en-us/articles/115014893968-Terms-of-sale}{Terms
  of Sale}
\item
  \href{https://spiderbites.nytimes.com}{Site Map}
\item
  \href{https://help.nytimes.com/hc/en-us}{Help}
\item
  \href{https://www.nytimes.com/subscription?campaignId=37WXW}{Subscriptions}
\end{itemize}
