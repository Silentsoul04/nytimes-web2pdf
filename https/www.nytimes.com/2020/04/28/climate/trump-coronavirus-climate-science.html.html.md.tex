Sections

SEARCH

\protect\hyperlink{site-content}{Skip to
content}\protect\hyperlink{site-index}{Skip to site index}

\href{https://www.nytimes.com/section/climate}{Climate}

\href{https://myaccount.nytimes.com/auth/login?response_type=cookie\&client_id=vi}{}

\href{https://www.nytimes.com/section/todayspaper}{Today's Paper}

\href{/section/climate}{Climate}\textbar{}Trump's Response to Virus
Reflects a Long Disregard for Science

\url{https://nyti.ms/2VHM5CI}

\begin{itemize}
\item
\item
\item
\item
\item
\item
\end{itemize}

\href{https://www.nytimes.com/section/climate?action=click\&pgtype=Article\&state=default\&region=TOP_BANNER\&context=storylines_menu}{Climate
and Environment}

\begin{itemize}
\tightlist
\item
  \href{https://www.nytimes.com/2020/07/30/climate/sea-level-inland-floods.html?action=click\&pgtype=Article\&state=default\&region=TOP_BANNER\&context=storylines_menu}{Rising
  Seas}
\item
  \href{https://www.nytimes.com/interactive/2020/climate/trump-environment-rollbacks.html?action=click\&pgtype=Article\&state=default\&region=TOP_BANNER\&context=storylines_menu}{Trump's
  Changes}
\item
  \href{https://www.nytimes.com/interactive/2020/04/19/climate/climate-crash-course-1.html?action=click\&pgtype=Article\&state=default\&region=TOP_BANNER\&context=storylines_menu}{Climate
  101}
\item
  \href{https://www.nytimes.com/interactive/2018/08/30/climate/how-much-hotter-is-your-hometown.html?action=click\&pgtype=Article\&state=default\&region=TOP_BANNER\&context=storylines_menu}{Is
  Your Hometown Hotter?}
\item
  \href{https://www.nytimes.com/newsletters/climate-change?action=click\&pgtype=Article\&state=default\&region=TOP_BANNER\&context=storylines_menu}{Newsletter}
\end{itemize}

Advertisement

\protect\hyperlink{after-top}{Continue reading the main story}

Supported by

\protect\hyperlink{after-sponsor}{Continue reading the main story}

News Analysis

\hypertarget{trumps-response-to-virus-reflects-a-long-disregard-for-science}{%
\section{Trump's Response to Virus Reflects a Long Disregard for
Science}\label{trumps-response-to-virus-reflects-a-long-disregard-for-science}}

The president's Covid-19 response has extended the administration's
longstanding practice of undermining scientific expertise for political
purposes.

\includegraphics{https://static01.nyt.com/images/2020/04/24/climate/24CLI-VIRUSSCIENCE1/24CLI-VIRUSSCIENCE1-articleLarge-v2.jpg?quality=75\&auto=webp\&disable=upscale}

\href{https://www.nytimes.com/by/lisa-friedman}{\includegraphics{https://static01.nyt.com/images/2018/07/18/multimedia/author-lisa-friedman/author-lisa-friedman-thumbLarge.png}}\href{https://www.nytimes.com/by/brad-plumer}{\includegraphics{https://static01.nyt.com/images/2018/02/20/multimedia/author-brad-plumer/author-brad-plumer-thumbLarge.jpg}}

By \href{https://www.nytimes.com/by/lisa-friedman}{Lisa Friedman} and
\href{https://www.nytimes.com/by/brad-plumer}{Brad Plumer}

\begin{itemize}
\item
  April 28, 2020
\item
  \begin{itemize}
  \item
  \item
  \item
  \item
  \item
  \item
  \end{itemize}
\end{itemize}

WASHINGTON --- At a March visit with doctors and researchers at the
Centers for Disease Control and Prevention, the public health agency at
the heart of the fight against the coronavirus, President Trump
\href{https://www.whitehouse.gov/briefings-statements/remarks-president-trump-tour-centers-disease-control-prevention-atlanta-ga/}{spoke
words of praise} for the scientific acumen in the building ---
particularly his own.

``Every one of these doctors said, `How do you know so much about this?'
Maybe I have a natural ability,'' Mr. Trump said.

It was a striking boast, even amid a grave health crisis in which Mr.
Trump has repeatedly contradicted medical experts in favor of his own
judgment. But a disregard for scientific advice has been a
\href{https://www.nytimes.com/2019/12/28/climate/trump-administration-war-on-science.html}{defining
characteristic} of Mr. Trump's administration.

As the nation confronts one of its worst public health disasters in
generations, a moment that demands a leader willing to marshal the full
might of the American scientific establishment, the White House is
occupied by a president whose administration, critics say, has
diminished the conclusions of scientists in formulating policy, who
personally harbors a suspicion of expert knowledge, and who often puts
his political instincts ahead of the facts.

``Donald Trump is the most anti-science and anti-environment president
we've ever had,'' said Douglas Brinkley, a presidential historian at
Rice University. The president's actions, he said, have eroded one of
the United States's most enviable assets: the government's deep
scientific expertise, built over decades. ``It's extraordinarily crazy
and reckless,'' he said.

Judd Deere, a White House spokesman, said in a statement that Mr.
Trump's handling of the coronavirus outbreak ``has put the full power of
the federal government to work to slow the spread, save lives, and place
this great country on a data-driven path to opening up again.''

Well before winning the presidency, Mr. Trump had publicly questioned
science by
\href{https://www.nytimes.com/2020/03/09/health/trump-vaccines.html}{expressing
skepticism about vaccines} and
\href{https://www.politifact.com/factchecks/2016/jun/03/hillary-clinton/yes-donald-trump-did-call-climate-change-chinese-h/}{suggesting
climate change was a hoax} fabricated by China.

Once in office, Mr. Trump's administration quickly began work on one of
its most far-reaching policies --- the systematic downplaying or
ignoring of science in order to weaken environmental health and global
warming regulations. Automakers, farmers and others had sought
regulatory relief, saying that more flexible rules would still ensure
progress on environmental protection while avoiding bureaucratic
mandates. However, in implementing the rollbacks, the administration has
\href{https://www.nytimes.com/2018/06/09/climate/trump-administration-science.html}{marginalized
key scientists}, disbanded expert advisory boards and
\href{https://www.nytimes.com/2019/06/08/climate/rod-schoonover-testimony.html}{suppressed}
or
\href{https://www.nytimes.com/2020/03/02/climate/goks-uncertainty-language-interior.html}{altered}
findings that make clear the dangers of pollution and global warming.

\includegraphics{https://static01.nyt.com/images/2020/04/27/climate/27CLI-VIRUSSCIENCE5/27CLI-VIRUSSCIENCE5-articleLarge.jpg?quality=75\&auto=webp\&disable=upscale}

More recently, as the coronavirus outbreak engulfed the nation, Mr.
Trump has repeatedly clashed with his own public health experts.

He was
\href{https://www.nytimes.com/2020/04/11/us/politics/coronavirus-trump-response.html?action=click\&module=RelatedLinks\&pgtype=Article}{slow
to react} to early internal warnings to take the outbreak more seriously
and has promoted the use of various drugs to fight the virus even as
\href{https://www.nytimes.com/2020/04/21/health/nih-covid-19-treatment.html}{scientists
said} there was no proof they would be effective. On Thursday, he
suggested that
\href{https://www.nytimes.com/2020/04/24/health/sunlight-coronavirus-trump.html}{injecting
disinfectants} might help defeat Covid-19, drawing global condemnation
and ridicule.

\href{https://www.nytimes.com/section/climate?action=click\&pgtype=Article\&state=default\&region=MAIN_CONTENT_1\&context=storylines_keepup}{}

\hypertarget{climate-and-environment-}{%
\subsubsection{Climate and Environment
›}\label{climate-and-environment-}}

\hypertarget{keep-up-on-the-latest-climate-news}{%
\paragraph{Keep Up on the Latest Climate
News}\label{keep-up-on-the-latest-climate-news}}

Updated July 30, 2020

Here's what you need to know about the latest climate change news this
week:

\begin{itemize}
\item
  \begin{itemize}
  \tightlist
  \item
    \href{https://www.nytimes.com/2020/07/30/climate/bangladesh-floods.html?action=click\&pgtype=Article\&state=default\&region=MAIN_CONTENT_1\&context=storylines_keepup}{Floods
    in}\href{https://www.nytimes.com/2020/07/30/climate/bangladesh-floods.html?action=click\&pgtype=Article\&state=default\&region=MAIN_CONTENT_1\&context=storylines_keepup}{Bangladesh}
    are punishing the people least responsible for climate change.
  \item
    As climate change raises sea levels,
    \href{https://www.nytimes.com/2020/07/30/climate/sea-level-inland-floods.html?action=click\&pgtype=Article\&state=default\&region=MAIN_CONTENT_1\&context=storylines_keepup}{storm
    surges and high tides} are likely to push farther inland.
  \item
    The E.P.A. inspector general plans to investigate whether a rollback
    of fuel efficiency standards
    \href{https://www.nytimes.com/2020/07/27/climate/trump-fuel-efficiency-rule.html?action=click\&pgtype=Article\&state=default\&region=MAIN_CONTENT_1\&context=storylines_keepup}{violated
    government rules}.
  \end{itemize}
\end{itemize}

And last week Mr. Trump publicly
\href{https://www.cnn.com/2020/04/23/politics/fauci-testing-capacity-not-overly-confident/index.html}{downplayed
a warning} by Dr. Anthony Fauci, the administration's most visible
medical expert, that the United States still lacked adequate capacity to
test for the coronavirus. ``I don't agree with him on that, no,'' Mr.
Trump said. ``I think we're doing a great job on testing.''

The president also suggested that the virus might be gone by the fall, a
line that
\href{https://www.nytimes.com/2020/04/22/us/politics/trump-coronavirus-fall.html}{was
immediately countered} by Dr. Fauci, who said: ``We will have
coronavirus in the fall. I am convinced of that.''

Historians and foreign policy experts said the administration's
disregard for scientific expertise --- combined with the nation's
broader retreat from international trade agreements and cross-border
defense alliances like NATO --- is diminishing the nation's status on
the world stage. ``America's friends feel like they don't even recognize
us,'' said Kori Schake, director of foreign and defense policy studies
at the American Enterprise Institute, a conservative research
organization.

Other critics noted that Mr. Trump's decision to withdraw the United
States from the Paris Agreement, a 2015 pact among nations to combat
climate change, has left the world adrift on one of the biggest
challenges to face humanity. And now, amid a sweeping global pandemic,
Mr. Trump has said he will halt funding for the World Health
Organization.

Part of what elevated America after World War II, Dr. Schake said, was
that ``we represented modernity in all its advantages,'' whether by
creating a polio vaccine or landing a man on the moon. ``It will be a
real struggle to restore the admiration for the United States that is
such an important part of our power in the world,'' she said.

The administration faces immense challenges in navigating the
coronavirus outbreak. Shutdowns nationwide have already pushed 26
million people into unemployment. But health experts
\href{https://www.nytimes.com/2020/04/06/upshot/coronavirus-four-benchmarks-reopening.html}{have
converged on a broad agreement} that sending people back to work too
soon, before measures like a robust testing system are in place, risks
causing a surge of new infections, deepening the crisis.

In many cases, the administration's guidance broadly follows that
scientific understanding. But experts have also warned that Mr. Trump's
\href{https://www.nytimes.com/2020/04/12/us/when-lockdown-ending-coronavirus.html}{frequent
exhortations to quickly reopen the economy} threaten to muddle a vital
public health message at a precarious time.

Image

``We will have coronavirus in the fall,'' Dr. Anthony Fauci said last
week. ``I am convinced of that.'' Credit...Doug Mills/The New York Times

``It's precisely because we're in this uncertain and perilous moment
that it's all the more important to rely on the best scientific
advice,'' said Lawrence Gostin, a professor of public health law at
Georgetown University.

Mr. Deere, the White House spokesman, said any suggestion that Mr. Trump
hasn't consulted and relied upon health experts and scientific advisers
``is just false.'' On Friday Mr. Trump announced a
\href{https://twitter.com/whitehouse/status/1254904154977337345}{phased
approach} to reopening the economy that the White House said is ``based
on the advice of public health experts.''

Past administrations have, to varying degrees, disregarded scientific
findings that conflicted with political or policy priorities. For
example, the Reagan administration was criticized by health experts for
being slow to respond to the AIDS crisis in the 1980s. And in 2011,
President Barack Obama's top health official overruled Food and Drug
Administration scientists who had found that over-the-counter emergency
contraceptives were safe for minors.

But within the Trump administration, the
\href{https://www.nytimes.com/2019/12/28/climate/trump-administration-war-on-science.html}{attacks
on science} and expertise have been far more broad.

``Scientists tell them inconvenient things,'' said Jerry Taylor,
president of the Niskanen Center, a centrist research organization, and
former climate change denialist who now advocates for the acceptance of
climate science. ``Whether we're talking about the E.P.A. or we're
talking about climate change broadly speaking, or we're talking about
the coronavirus, his administration is constantly engaged in magical
thinking.''

Critics of the administration's actions both on environmental matters
and the virus say that federal policy has been shaped to favor
short-term economic gain at the expense of public health.

With much of the nation sheltering at home from the coronavirus ---
bringing commerce to a halt, sending unemployment skyrocketing and
causing turmoil in the financial markets --- the motivations to restart
the economy are powerful. But Mr. Taylor of the Niskanen Center said
that some conservatives were incorrectly diagnosing the stay-at-home
orders as the main driver of the nation's woes rather than the virus
itself.

Mr. Taylor likened it to the argument that government action to fight
climate change would be too costly in various ways --- an argument that
overlooks the
\href{https://www.nytimes.com/2018/11/19/climate/climate-disasters.html}{significant
costs} of inaction. ``If we leave the underlying problem unattended,''
he said, ``the economic cost will be far greater.''

Image

Nearly 100 environmental regulations have been rolled back since
President Trump took office.Credit...Victor J. Blue for The New York
Times

Meanwhile, the pandemic hasn't slowed the administration's environmental
rollbacks.

Over the past month the Environmental Protection Agency has issued
several deregulatory policies, including on
\href{https://www.nytimes.com/2020/04/16/climate/epa-mercury-coal.html}{mercury
pollution} and
\href{https://www.nytimes.com/2020/03/30/climate/trump-fuel-economy.html}{automobile
emissions}, overruling advice from the agency's own independent advisory
board that such findings
\href{https://yosemite.epa.gov/sab/sabproduct.nsf/MeetingCalBOARD/D87AC6491A9811C1852584CD006F3CC6?OpenDocument}{lacked
scientific rigor}. The E.P.A. also
\href{https://www.nytimes.com/2020/04/14/climate/coronavirus-soot-clean-air-regulations.html}{refused
to tighten air quality standards}, despite
\href{https://www.nytimes.com/2020/04/07/climate/air-pollution-coronavirus-covid.html}{preliminary
research} suggesting that long-term exposure to dirty air could
exacerbate the risk of death from the coronavirus.

The administration has maintained that it can safeguard health and the
environment while loosening restrictions on industry. Andrea Woods, a
spokeswoman for the E.P.A., said, ``We have never ignored the science in
making the very tough policy decisions required of the agency.''

The parallels between the administration's environmental rollbacks and
its coronavirus response are not exact. When it comes to the coronavirus
outbreak, there is still an important counterweight to many of Mr.
Trump's impulses, most notably Dr. Fauci. Asked last week if he felt
that experts at the National Institutes of Health were unable to speak
their minds or oppose Mr. Trump, Dr. Fauci was unequivocal. ``Absolutely
no,'' he said.

That stands in contrast to the administration's approach on issues like
climate change, where officials who have spoken out have found
themselves sidelined.

In July, Rod Schoonover, a State Department intelligence analyst,
\href{https://www.nytimes.com/2019/07/10/climate/rod-schoonover-resigns.html}{resigned
in protest} after the White House blocked his discussion of climate
science in Congressional testimony. In other instances, the
administration has
\href{https://www.nytimes.com/2019/02/20/climate/climate-national-security-threat.html}{promoted
climate denialists' work} and
\href{https://www.nytimes.com/2020/03/02/climate/goks-uncertainty-language-interior.html}{allowed
them to insert misrepresentations of scientific facts} into federal
documents.

Image

Dr. Nancy Messonnier at a coronavirus briefing in January.Credit...Shawn
Thew/EPA, via Shutterstock

Still, there have been some prominent staff shake-ups at health
agencies.

Before the pandemic began, the C.D.C.
\href{https://www.reuters.com/article/us-health-coronavirus-china-cdc-exclusiv/exclusive-u-s-slashed-cdc-staff-inside-china-prior-to-coronavirus-outbreak-idUSKBN21C3N5}{had
reduced its staff in Beijing} from approximately 47 to 14 under the
Trump administration, a move that critics have said may have complicated
its ability to confront the outbreak earlier. An agency spokesman said
it had been done to focus more on ``technical collaboration'' with
China, which requires fewer people.

In February, Nancy Messonnier, a top C.D.C. official, was removed from
overseeing the agency's coronavirus response. Dr. Messonnier
\href{https://www.nytimes.com/2020/03/07/us/politics/trump-coronavirus.html}{had
warned} that Americans need to prepare for a ``significant disruption''
at a time when Mr. Trump was insisting that the virus was ``very well
under control in our country.''

Last week, Rick Bright
\href{https://www.nytimes.com/2020/04/22/us/politics/rick-bright-trump-hydroxychloroquine-coronavirus.html}{was
dismissed} as the director of the Biomedical Advanced Research and
Development Authority, the agency involved in work on coronavirus
treatments. Mr. Bright said he had been removed after urging caution in
expanding access to hydroxychloroquine, the controversial treatment
embraced by Mr. Trump. He also said the administration had put
``politics and cronyism ahead of science.''

Mr. Trump has said he ``never heard'' of Dr. Bright. Mr. Deere, the
White House spokesman, accused critics of waging a campaign ``to
criticize this president for discussing anything that might provide hope
to the American people.''

Sheila Kaplan contributed reporting

Advertisement

\protect\hyperlink{after-bottom}{Continue reading the main story}

\hypertarget{site-index}{%
\subsection{Site Index}\label{site-index}}

\hypertarget{site-information-navigation}{%
\subsection{Site Information
Navigation}\label{site-information-navigation}}

\begin{itemize}
\tightlist
\item
  \href{https://help.nytimes.com/hc/en-us/articles/115014792127-Copyright-notice}{©~2020~The
  New York Times Company}
\end{itemize}

\begin{itemize}
\tightlist
\item
  \href{https://www.nytco.com/}{NYTCo}
\item
  \href{https://help.nytimes.com/hc/en-us/articles/115015385887-Contact-Us}{Contact
  Us}
\item
  \href{https://www.nytco.com/careers/}{Work with us}
\item
  \href{https://nytmediakit.com/}{Advertise}
\item
  \href{http://www.tbrandstudio.com/}{T Brand Studio}
\item
  \href{https://www.nytimes.com/privacy/cookie-policy\#how-do-i-manage-trackers}{Your
  Ad Choices}
\item
  \href{https://www.nytimes.com/privacy}{Privacy}
\item
  \href{https://help.nytimes.com/hc/en-us/articles/115014893428-Terms-of-service}{Terms
  of Service}
\item
  \href{https://help.nytimes.com/hc/en-us/articles/115014893968-Terms-of-sale}{Terms
  of Sale}
\item
  \href{https://spiderbites.nytimes.com}{Site Map}
\item
  \href{https://help.nytimes.com/hc/en-us}{Help}
\item
  \href{https://www.nytimes.com/subscription?campaignId=37WXW}{Subscriptions}
\end{itemize}
