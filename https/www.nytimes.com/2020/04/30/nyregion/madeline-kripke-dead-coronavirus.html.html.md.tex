Sections

SEARCH

\protect\hyperlink{site-content}{Skip to
content}\protect\hyperlink{site-index}{Skip to site index}

\href{https://www.nytimes.com/section/nyregion}{New York}

\href{https://myaccount.nytimes.com/auth/login?response_type=cookie\&client_id=vi}{}

\href{https://www.nytimes.com/section/todayspaper}{Today's Paper}

\href{/section/nyregion}{New York}\textbar{}Madeline Kripke, Doyenne of
Dictionaries, Is Dead at 76

\url{https://nyti.ms/3d3clNK}

\begin{itemize}
\item
\item
\item
\item
\item
\end{itemize}

\href{https://www.nytimes.com/news-event/coronavirus?action=click\&pgtype=Article\&state=default\&region=TOP_BANNER\&context=storylines_menu}{The
Coronavirus Outbreak}

\begin{itemize}
\tightlist
\item
  live\href{https://www.nytimes.com/2020/08/03/world/coronavirus-covid-19.html?action=click\&pgtype=Article\&state=default\&region=TOP_BANNER\&context=storylines_menu}{Latest
  Updates}
\item
  \href{https://www.nytimes.com/interactive/2020/us/coronavirus-us-cases.html?action=click\&pgtype=Article\&state=default\&region=TOP_BANNER\&context=storylines_menu}{Maps
  and Cases}
\item
  \href{https://www.nytimes.com/interactive/2020/science/coronavirus-vaccine-tracker.html?action=click\&pgtype=Article\&state=default\&region=TOP_BANNER\&context=storylines_menu}{Vaccine
  Tracker}
\item
  \href{https://www.nytimes.com/2020/08/02/us/covid-college-reopening.html?action=click\&pgtype=Article\&state=default\&region=TOP_BANNER\&context=storylines_menu}{College
  Reopening}
\item
  \href{https://www.nytimes.com/live/2020/08/03/business/stock-market-today-coronavirus?action=click\&pgtype=Article\&state=default\&region=TOP_BANNER\&context=storylines_menu}{Economy}
\end{itemize}

Advertisement

\protect\hyperlink{after-top}{Continue reading the main story}

Supported by

\protect\hyperlink{after-sponsor}{Continue reading the main story}

Those We've Lost

\hypertarget{madeline-kripke-doyenne-of-dictionaries-is-dead-at-76}{%
\section{Madeline Kripke, Doyenne of Dictionaries, Is Dead at
76}\label{madeline-kripke-doyenne-of-dictionaries-is-dead-at-76}}

A woman of many words, mostly unspoken, she amassed a lexicographic
trove of some 20,000 books, much of it crammed into her Greenwich
Village apartment.

\includegraphics{https://static01.nyt.com/images/2020/05/01/obituaries/25kripke-virus-lost01/merlin_172032759_ac4657dd-a433-4784-97bf-d5a6fb4903a4-articleLarge.jpg?quality=75\&auto=webp\&disable=upscale}

\href{https://www.nytimes.com/by/sam-roberts}{\includegraphics{https://static01.nyt.com/images/2018/02/20/multimedia/author-sam-roberts/author-sam-roberts-thumbLarge.jpg}}

By \href{https://www.nytimes.com/by/sam-roberts}{Sam Roberts}

\begin{itemize}
\item
  April 30, 2020
\item
  \begin{itemize}
  \item
  \item
  \item
  \item
  \item
  \end{itemize}
\end{itemize}

\emph{This obituary is part of a series about people who have died in
the coronavirus pandemic. Read about others}
\href{https://www.nytimes.com/series/people-who-have-died-of-the-coronavirus}{\emph{here}}\emph{.}

Madeline Kripke, who kept one of the world's largest private collection
of dictionaries, much of it crammed into her Greenwich Village
apartment, could be defined this way: liberal {[}adj., as in giving
unstintingly{]}, compleat {[}adj., meaning having all the requisite
skills{]} and sui generis {[}adj., in a class by itself{]}.

Beginning with the Webster's Collegiate that her parents gave her in the
fifth grade, she accumulated an estimated 20,000 volumes as diverse as a
Latin dictionary printed in 1502, Jonathan Swift's 1722 booklet titled
``The Benefits of Farting Explained,'' and the New York Metropolitan
Transportation Authority's 1980 guide to pickpocket slang.

Ms. Kripke (pronounced KRIP-key) died on April 25 in Manhattan at 76.
Her brother,
\href{https://www.nytimes.com/2006/01/28/books/philosopher-65-lectures-not-about-what-am-i-but-what-is-i.html}{Saul
Kripke}, a noted philosopher and
\href{https://www.britannica.com/biography/Saul-Kripke}{professor} at
the City University of New York Graduate Center, said the cause was the
coronavirus and complications of pneumonia.

One question that none of Ms. Kripke's reference books answers is what
will happen to her collection. After avoiding eviction in the mid-1990s
by agreeing to remove the volumes stacked in the hallway, she had hoped
to transfer the whole enchilada {[}slang for the entirety{]} from her
apartment and three warehouses to a university or, if she had her
druthers {[}n., preference{]}, to install it in her own dictionary
library, which she never got to build.

``Unfortunately, it appears that no clear plan existed for her
collection,'' her brother, her only immediate survivor, said in a phone
interview. ``We are now in touch with some of her expert friends for
advice.''

Those friends are legion {[}adj., multitudinous{]}, thanks to Ms.
Kripke's generosity and virtuosity as a resource on etymology {[}n., the
derivation of words{]}, pronunciation and usage and especially every
variety of vulgarity and slang, from the indigenous argot of Argentina
to the patois of vaudeville, the London underworld, cowboys, hipsters
and generations of teenagers.

But Ms. Kripke was not an indiscriminate amasser, said Ammon Shea, the
author of
``\href{https://www.nytimes.com/2008/08/03/books/review/Baker-t.html}{Reading
the OED}: One Man, One Year, 21,730 Pages'' (2008). ``Madeline,'' he
said, ``built a cathedral of the English lexicographic tradition, tens
of thousands of carefully chosen items.''

Madeline Faith Kripke was born on Sept. 9, 1943, in New London, Conn.,
where her father, Rabbi
\href{https://www.nytimes.com/2014/05/04/us/rabbi-myer-kripke-100-early-buffett-friend-and-investor-dies.html}{Myer
S. Kripke}, headed a Conservative Jewish congregation. Her mother,
Dorothy (Karp) Kripke, was an author of children's religious books.

Madeline grew up in Omaha, where her father was the rabbi of Beth El
Synagogue and where her parents were friends of the investor Warren
Buffett (and beneficiaries of his financial advice).

\includegraphics{https://static01.nyt.com/images/2020/05/01/obituaries/25kripke-virus-lost02/merlin_172032762_68f409bb-214b-4da0-83f2-9f317d6fdd65-articleLarge.jpg?quality=75\&auto=webp\&disable=upscale}

The Webster's Collegiate she received from her parents, she told Daniel
Krieger for a profile about her on the website
\href{https://narratively.com/the-dame-of-dictionaries/}{Narratively,}
``unlocked the world for me because I could read at any vocabulary level
I wanted.'' Which she did, conscientiously documenting the words she
didn't understand.

``I realized that dictionaries were each infinitely explorable,'' she
told Mr. Krieger, ``so they opened me to new possibilities in a mix of
serendipity, discovery and revelation.''

After earning a bachelor's degree in English from Barnard College, she
remained in New York in the 1960s, living as a cross between a beatnik
and a hippie, she said, then working as a welfare case worker, a
teacher, and a copy editor and proofreader --- skills she would apply to
her collecting.

She was self-taught as a lexicographer. ``She approached her collection
and study with the same scholarship and discipline with which her father
approached religion,'' said Tom Dalzell, a slang expert, ``and with
which her brother approaches modal logic, philosophy of language,
metaphysics, epistemology and recursion theory.''

Jesse Sheidlower, a former editor at the ``Oxford English Dictionary,''
said of Ms. Kripke, ``She didn't just accumulate material; she read it
all, and could tell you the editor's personality based on the changes
made across varying editions of a work."

While she later revived her childhood practice of recording unfamiliar
words in a notebook, Ms. Kripke never exploited her command of language
in poetry or prose, except for the occasional verse, like her ode to
Icarus, which began, ``He must have been high when he first tried to
fly.''

The comprehensiveness of her collection amazed many in the lexicographic
world.

Simon Winchester, the author of
``\href{https://www.nytimes.com/2003/10/12/books/you-could-look-it-up.html}{The
Meaning of Everything}: The Story of the Oxford English Dictionary''
(2003), said in an email: ``I would challenge her to find this volume of
Czech loanwords or that collection of Greenland slang or Common Terms in
Astrophysics --- and she'd always say, `Yes, I'm sure I have it
somewhere,' and would dive in like a truffle hound and come up for air
holding the volume in triumph, and I would retire, always defeated.''

Ms. Kripke's linguistic-related ephemera included an instruction manual
for dictionary salesmen and a pivotal letter from George Merriam to his
brother Charles. The letter captured ``the moment when the brothers
hatch a plan for getting the rights to Noah Webster's dictionary --- the
Big Bang moment that leads directly to the creation of Merriam-Webster
dictionaries,'' said John Morse, a former president and publisher of
Merriam-Webster.

About one-fifth of Ms. Kripke's collection represents what Mr.
Winchester described in The New York Review of Books in 2012 as ``the
very living and breathing edge of the English language: the ragged and
ill-defined omnium gatherum of informal, witty, clever, newborn, and
usually impermanent words that constitute what for the past two
centuries has been known as slang.''

Armed with a flashlight, she would hunt down ``A Classical Dictionary of
the Vulgar Tongue'' from 1785; or ``The Pocket Dictionary of Prison
Slanguage'' (1941), by Clinton T. Duffy, a former warden of San Quentin;
or the pornographic comic books known as Tijuana bibles.

Ms. Kripke sold books, but she acquired even more, with surpassing
dedication. As a young collector, she once coveted a 1694 edition of
``The Ladies Dictionary,'' which she had found in a London shop at a
time when she had only enough money for a planned train trip to France
to meet a friend in Nice.

She bought the book and hitchhiked to Nice instead.

\href{https://www.nytimes.com/interactive/2020/obituaries/people-died-coronavirus-obituaries.html?action=click\&pgtype=Article\&state=default\&region=BELOW_MAIN_CONTENT\&context=covid_obits_promo}{}

\hypertarget{those-weve-lost}{%
\section{Those We've Lost}\label{those-weve-lost}}

The coronavirus pandemic has taken an incalculable death toll. This
series is designed to put names and faces to the numbers.

Read more

\includegraphics{https://static01.nyt.com/images/2020/07/30/obituaries/30Pedro/30Pedro-square640.jpg}

\hypertarget{bernaldina-josuxe9-pedro}{%
\section{Bernaldina José Pedro}\label{bernaldina-josuxe9-pedro}}

d. Boa Vista, Brazil

Leader among the Indigenous Macuxi

\includegraphics{https://static01.nyt.com/images/2020/07/31/obituaries/31Swing/merlin_175167783_8913bc90-0d64-43f3-a655-1bb1bf1601c9-square640.jpg}

\hypertarget{john-eric-swing}{%
\section{John Eric Swing}\label{john-eric-swing}}

d. Fountain Valley, Calif.

Champion of Filipino-Americans

\includegraphics{https://static01.nyt.com/images/2020/07/27/obituaries/27Victor/merlin_175001436_38b11f8e-227a-4e2c-9821-7618af9b2524-square640.jpg}

\hypertarget{victor-victor}{%
\section{Victor Victor}\label{victor-victor}}

d. Santo Domingo, Dominican Republic

Beloved musician of the Dominican Republic

\includegraphics{https://static01.nyt.com/images/2020/07/31/obituaries/31Negron/merlin_175160169_516322ae-fd23-4969-b6b2-193ced371105-square640.jpg}

\hypertarget{dr-eddie-negruxf3n}{%
\section{Dr. Eddie Negrón}\label{dr-eddie-negruxf3n}}

d. Fort Walton Beach, Fla.

Internist on Florida's Emerald Coast

\includegraphics{https://static01.nyt.com/images/2020/07/30/obituaries/30Dobson/merlin_175115928_f6b9271c-8f05-4fe1-a38a-5ca4a58f8935-square640.jpg}

\hypertarget{dobby-dobson}{%
\section{Dobby Dobson}\label{dobby-dobson}}

d. Coral Springs, Fla.

Jamaican singer and songwriter

\includegraphics{https://static01.nyt.com/images/2020/08/01/obituaries/28Gonzalez/merlin_175002771_beb57888-3951-409a-ae13-03a94b2e962e-square640.jpg}

\hypertarget{waldemar-gonzalez}{%
\section{Waldemar Gonzalez}\label{waldemar-gonzalez}}

d. White Plains, N.Y.

Teacher and social worker

Advertisement

\protect\hyperlink{after-bottom}{Continue reading the main story}

\hypertarget{site-index}{%
\subsection{Site Index}\label{site-index}}

\hypertarget{site-information-navigation}{%
\subsection{Site Information
Navigation}\label{site-information-navigation}}

\begin{itemize}
\tightlist
\item
  \href{https://help.nytimes.com/hc/en-us/articles/115014792127-Copyright-notice}{©~2020~The
  New York Times Company}
\end{itemize}

\begin{itemize}
\tightlist
\item
  \href{https://www.nytco.com/}{NYTCo}
\item
  \href{https://help.nytimes.com/hc/en-us/articles/115015385887-Contact-Us}{Contact
  Us}
\item
  \href{https://www.nytco.com/careers/}{Work with us}
\item
  \href{https://nytmediakit.com/}{Advertise}
\item
  \href{http://www.tbrandstudio.com/}{T Brand Studio}
\item
  \href{https://www.nytimes.com/privacy/cookie-policy\#how-do-i-manage-trackers}{Your
  Ad Choices}
\item
  \href{https://www.nytimes.com/privacy}{Privacy}
\item
  \href{https://help.nytimes.com/hc/en-us/articles/115014893428-Terms-of-service}{Terms
  of Service}
\item
  \href{https://help.nytimes.com/hc/en-us/articles/115014893968-Terms-of-sale}{Terms
  of Sale}
\item
  \href{https://spiderbites.nytimes.com}{Site Map}
\item
  \href{https://help.nytimes.com/hc/en-us}{Help}
\item
  \href{https://www.nytimes.com/subscription?campaignId=37WXW}{Subscriptions}
\end{itemize}
