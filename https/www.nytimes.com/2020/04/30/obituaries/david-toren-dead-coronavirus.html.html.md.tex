Sections

SEARCH

\protect\hyperlink{site-content}{Skip to
content}\protect\hyperlink{site-index}{Skip to site index}

\href{https://www.nytimes.com/section/obituaries}{Obituaries}

\href{https://myaccount.nytimes.com/auth/login?response_type=cookie\&client_id=vi}{}

\href{https://www.nytimes.com/section/todayspaper}{Today's Paper}

\href{/section/obituaries}{Obituaries}\textbar{}David Toren, Who Fought
to Recover Nazi-Looted Art, Dies at 94

\url{https://nyti.ms/2zDexx0}

\begin{itemize}
\item
\item
\item
\item
\item
\end{itemize}

\href{https://www.nytimes.com/news-event/coronavirus?action=click\&pgtype=Article\&state=default\&region=TOP_BANNER\&context=storylines_menu}{The
Coronavirus Outbreak}

\begin{itemize}
\tightlist
\item
  live\href{https://www.nytimes.com/2020/08/03/world/coronavirus-covid-19.html?action=click\&pgtype=Article\&state=default\&region=TOP_BANNER\&context=storylines_menu}{Latest
  Updates}
\item
  \href{https://www.nytimes.com/interactive/2020/us/coronavirus-us-cases.html?action=click\&pgtype=Article\&state=default\&region=TOP_BANNER\&context=storylines_menu}{Maps
  and Cases}
\item
  \href{https://www.nytimes.com/interactive/2020/science/coronavirus-vaccine-tracker.html?action=click\&pgtype=Article\&state=default\&region=TOP_BANNER\&context=storylines_menu}{Vaccine
  Tracker}
\item
  \href{https://www.nytimes.com/2020/08/02/us/covid-college-reopening.html?action=click\&pgtype=Article\&state=default\&region=TOP_BANNER\&context=storylines_menu}{College
  Reopening}
\item
  \href{https://www.nytimes.com/live/2020/08/03/business/stock-market-today-coronavirus?action=click\&pgtype=Article\&state=default\&region=TOP_BANNER\&context=storylines_menu}{Economy}
\end{itemize}

Advertisement

\protect\hyperlink{after-top}{Continue reading the main story}

Supported by

\protect\hyperlink{after-sponsor}{Continue reading the main story}

Those We've Lost

\hypertarget{david-toren-who-fought-to-recover-nazi-looted-art-dies-at-94}{%
\section{David Toren, Who Fought to Recover Nazi-Looted Art, Dies at
94}\label{david-toren-who-fought-to-recover-nazi-looted-art-dies-at-94}}

Mr. Toren, who died of the coronavirus, was a patent lawyer who
recovered a relative's stolen painting amid a large cache of works
discovered in Germany.

\includegraphics{https://static01.nyt.com/images/2020/05/01/obituaries/01toren-print1/29toren-virus-lost01-articleLarge.jpg?quality=75\&auto=webp\&disable=upscale}

By Catherine Hickley

\begin{itemize}
\item
  April 30, 2020
\item
  \begin{itemize}
  \item
  \item
  \item
  \item
  \item
  \end{itemize}
\end{itemize}

\emph{This obituary is part of a series about people who have died in
the coronavirus pandemic. Read about others}
\href{https://www.nytimes.com/series/people-who-have-died-of-the-coronavirus}{\emph{here}}\emph{.}

David Toren, a Holocaust survivor and patent lawyer who waged a
single-minded quest to recover art looted from his family by the Nazis,
died on April 19 at his home in Manhattan. He was 94.

The cause was the novel coronavirus, his son, Peter Toren, said.

Mr. Toren's campaign to recover the stolen works drew headlines when a
painting by Max Liebermann, ``Two Riders on the
Beach,''\href{https://www.nytimes.com/2014/08/18/arts/design/experts-say-a-second-work-in-munich-was-looted.html}{surfaced
in the collection}of Cornelius Gurlitt, an elderly recluse. Mr. Gurlitt
had hoarded the art he inherited from his father, a dealer for the
Nazis, in his homes in Munich and Salzburg. Mr. Toren had been searching
for the painting for years.

Images of the work with other rediscovered paintings were displayed at a
news conference given by the state prosecutor in Augsburg, Germany, in
November 2013. By then Mr. Toren was blind, a consequence of a severe
case of shingles, but he could remember last seeing the painting hanging
in his great-uncle's villa in Germany 75 years earlier --- on Nov. 10,
1938, the day after Kristallnacht. The Gestapo eventually seized his
great-uncle's art collection, and ``Two Riders'' wound up in the hands
of an unscrupulous museum director, who sold it to Mr. Gurlitt's father.

``I always liked that painting because I liked horses,'' Mr. Toren said
in 2014. ``I will get it back.''

And he did --- finally --- the next year, after a host of bureaucratic
delays. Asked then if he felt a sense of closure, he responded with
characteristic emphasis. ``No!'' he said. ``We are looking for more
important paintings'' from his great-uncle's collection.

The discovery of Mr. Gurlitt's trove, which encompassed about 1,500
works by Claude Monet, Henri Matisse, Otto Dix and other artists,
revived interest in the issue of Nazi-looted art. Mr. Toren was one of
several heirs to claim that the collection held stolen works. Since its
discovery, a total of 13 have been identified as looted and have been
returned.

Klaus-Günther Tarnowski was born on April 30, 1925, and grew up with an
elder brother in a well-to-do family in an elegant neighborhood of
Breslau, now Wrocław, in Poland, but then part of Germany. (He changed
his name after fleeing.) His father, Dr. Georg Martin Tarnowski, was a
successful lawyer and the proud leader of the Breslau chapter of the
Association of Jewish War Veterans. His mother was Maria (Friedmann)
Tarnowski.

Georg Tarnowski's first wife, who died during World War I, was a niece
of the painter Lesser Ury, and the walls of the family apartment were
adorned with the paintings she had received for every birthday.

\includegraphics{https://static01.nyt.com/images/2020/05/01/obituaries/01toren-print2/merlin_172034925_5769ba9f-bd1f-492f-8b39-6ef6e1cb04f6-articleLarge.jpg?quality=75\&auto=webp\&disable=upscale}

At 10, after the Nazis had seized power, Klaus-Günther began attending
the prestigious Zwinger Gymnasium, where he encountered anti-Semitism
and bullying by teachers and pupils alike. His non-Jewish nanny was
compelled to leave the household. In a 2014 interview, he recalled being
heartbroken at having to give away his two pet parakeets, Habakkuk and
Zephaniah. Breslau's cinemas and ice cream parlors posted signs on their
doors warning, ``Jews Not Welcome.''

Kristallnacht remained etched in Mr. Toren's memory 75 years later. He
remembered watching unseen from the balcony with his parents as the
Jewish-run liquor store below was plundered and bottles were smashed
against the walls. The next morning, the Gestapo came for his father.
After three weeks at Buchenwald, he was allowed to return home gaunt,
his head shaved.

He managed to squeeze his son, now 14, onto what would prove to be the
last Kindertransport ** evacuation to Sweden before World War II broke
out. Mr. Toren left on Aug. 23, 1939, only days before Germany invaded
Poland, and never saw his parents again. Both perished at Auschwitz.

In Sweden, Mr. Toren managed to track down a distant relative, who
agreed to finance his tuition. He passed his exams and went on to study
chemistry in Stockholm. In 1948, he was recruited by Haganah, a Zionist
paramilitary organization later absorbed into the Israel Defense Forces,
and he left for Israel.

Mr. Toren was born with a degenerative eye condition, retinitis
pigmentosa. Discharged from the Israeli army because his eyesight was
inadequate, he began working as a chemist, but found the work dull. He
applied for a position in Tel Aviv as an assistant to a patent lawyer.
In Israel he met Sarah Brown, an American social worker who later became
a psychoanalyst, and they married in 1953. She died in 2019. In addition
to his son, Mr. Toren is survived by two grandchildren.

After a brief spell in London, the couple moved to New York in 1955, and
Mr. Toren took up night studies at New York Law School. After graduating
in 1960, he, too, became a patent lawyer. As one of America's top
German-speakers in the field, he often attracted German companies as
clients, many of whom had had strong Nazi connections.

``My father didn't have any misgivings about representing clients like
this because he felt that if he didn't do the work, someone else would,
and he deserved to profit financially,'' Peter Toren said by email. ``He
also charged what he called the `Nazi premium,' which was about 20
percent to 25 percent more than other clients. I don't think this was
listed in the billing statements.''

Mr. Toren contracted shingles in 2007; a devastating side effect was the
complete loss of his eyesight within three days. After that, well into
his 80s, he retired. In his last years the quest for his family's stolen
art became his preoccupation.

He sued Germany and the state of Bavaria in 2014 in federal court in
Washington. The case, which is pending, aims to recover 54 of his
great-uncle's paintings whose location is unknown.

``We will continue to search for and hopefully locate additional works
of art looted by the Nazis,'' Peter Toren, also a lawyer, said. ``My
father would have expected no less.''

\href{https://www.nytimes.com/interactive/2020/obituaries/people-died-coronavirus-obituaries.html?action=click\&pgtype=Article\&state=default\&region=BELOW_MAIN_CONTENT\&context=covid_obits_promo}{}

\hypertarget{those-weve-lost}{%
\section{Those We've Lost}\label{those-weve-lost}}

The coronavirus pandemic has taken an incalculable death toll. This
series is designed to put names and faces to the numbers.

Read more

\includegraphics{https://static01.nyt.com/images/2020/07/30/obituaries/30Pedro/30Pedro-square640.jpg}

\hypertarget{bernaldina-josuxe9-pedro}{%
\section{Bernaldina José Pedro}\label{bernaldina-josuxe9-pedro}}

d. Boa Vista, Brazil

Leader among the Indigenous Macuxi

\includegraphics{https://static01.nyt.com/images/2020/07/31/obituaries/31Swing/merlin_175167783_8913bc90-0d64-43f3-a655-1bb1bf1601c9-square640.jpg}

\hypertarget{john-eric-swing}{%
\section{John Eric Swing}\label{john-eric-swing}}

d. Fountain Valley, Calif.

Champion of Filipino-Americans

\includegraphics{https://static01.nyt.com/images/2020/07/27/obituaries/27Victor/merlin_175001436_38b11f8e-227a-4e2c-9821-7618af9b2524-square640.jpg}

\hypertarget{victor-victor}{%
\section{Victor Victor}\label{victor-victor}}

d. Santo Domingo, Dominican Republic

Beloved musician of the Dominican Republic

\includegraphics{https://static01.nyt.com/images/2020/07/31/obituaries/31Negron/merlin_175160169_516322ae-fd23-4969-b6b2-193ced371105-square640.jpg}

\hypertarget{dr-eddie-negruxf3n}{%
\section{Dr. Eddie Negrón}\label{dr-eddie-negruxf3n}}

d. Fort Walton Beach, Fla.

Internist on Florida's Emerald Coast

\includegraphics{https://static01.nyt.com/images/2020/07/30/obituaries/30Dobson/merlin_175115928_f6b9271c-8f05-4fe1-a38a-5ca4a58f8935-square640.jpg}

\hypertarget{dobby-dobson}{%
\section{Dobby Dobson}\label{dobby-dobson}}

d. Coral Springs, Fla.

Jamaican singer and songwriter

\includegraphics{https://static01.nyt.com/images/2020/08/01/obituaries/28Gonzalez/merlin_175002771_beb57888-3951-409a-ae13-03a94b2e962e-square640.jpg}

\hypertarget{waldemar-gonzalez}{%
\section{Waldemar Gonzalez}\label{waldemar-gonzalez}}

d. White Plains, N.Y.

Teacher and social worker

Advertisement

\protect\hyperlink{after-bottom}{Continue reading the main story}

\hypertarget{site-index}{%
\subsection{Site Index}\label{site-index}}

\hypertarget{site-information-navigation}{%
\subsection{Site Information
Navigation}\label{site-information-navigation}}

\begin{itemize}
\tightlist
\item
  \href{https://help.nytimes.com/hc/en-us/articles/115014792127-Copyright-notice}{©~2020~The
  New York Times Company}
\end{itemize}

\begin{itemize}
\tightlist
\item
  \href{https://www.nytco.com/}{NYTCo}
\item
  \href{https://help.nytimes.com/hc/en-us/articles/115015385887-Contact-Us}{Contact
  Us}
\item
  \href{https://www.nytco.com/careers/}{Work with us}
\item
  \href{https://nytmediakit.com/}{Advertise}
\item
  \href{http://www.tbrandstudio.com/}{T Brand Studio}
\item
  \href{https://www.nytimes.com/privacy/cookie-policy\#how-do-i-manage-trackers}{Your
  Ad Choices}
\item
  \href{https://www.nytimes.com/privacy}{Privacy}
\item
  \href{https://help.nytimes.com/hc/en-us/articles/115014893428-Terms-of-service}{Terms
  of Service}
\item
  \href{https://help.nytimes.com/hc/en-us/articles/115014893968-Terms-of-sale}{Terms
  of Sale}
\item
  \href{https://spiderbites.nytimes.com}{Site Map}
\item
  \href{https://help.nytimes.com/hc/en-us}{Help}
\item
  \href{https://www.nytimes.com/subscription?campaignId=37WXW}{Subscriptions}
\end{itemize}
