Sections

SEARCH

\protect\hyperlink{site-content}{Skip to
content}\protect\hyperlink{site-index}{Skip to site index}

\href{https://www.nytimes.com/spotlight/podcasts}{Podcasts}

\href{https://myaccount.nytimes.com/auth/login?response_type=cookie\&client_id=vi}{}

\href{https://www.nytimes.com/section/todayspaper}{Today's Paper}

\href{/spotlight/podcasts}{Podcasts}\textbar{}Fiona Ex Machina

\url{https://nyti.ms/3d0KNJ1}

\begin{itemize}
\item
\item
\item
\item
\item
\end{itemize}

Advertisement

\protect\hyperlink{after-top}{Continue reading the main story}

transcript

Back to Still Processing

bars

0:00/0:00

-0:00

transcript

\hypertarget{fiona-ex-machina}{%
\subsection{Fiona Ex Machina}\label{fiona-ex-machina}}

\hypertarget{hosted-by-wesley-morris-and-jenna-wortham-produced-by-hans-buetow-and-sydney-harper}{%
\subsubsection{Hosted by Wesley Morris and Jenna Wortham. Produced by
Hans Buetow and Sydney
Harper.}\label{hosted-by-wesley-morris-and-jenna-wortham-produced-by-hans-buetow-and-sydney-harper}}

\hypertarget{fiona-apple-blows-our-minds-again}{%
\paragraph{Fiona Apple blows our minds,
again.}\label{fiona-apple-blows-our-minds-again}}

Thursday, April 30th, 2020

\begin{itemize}
\item
  {[}music{]}
\item
  jenna wortham\\
  This is ``Still Processing.'' I'm Jenna Wortham.
\item
  wesley morris\\
  I'm Wesley Morris. And Jenna, I have a confession.
\item
  jenna wortham\\
  Oh.
\item
  wesley morris\\
  I have not always loved Fiona Apple.
\item
  jenna wortham\\
  What? I'm sorry. What?
\item
  wesley morris\\
  No, it's true.
\item
  jenna wortham\\
  You know what this week's episode is about.
\item
  wesley morris\\
  {[}LAUGHING{]} No, it's true. I did not enjoy her. It started for me
  like it did for everybody else, my relationship with this woman, in
  1996 when she was a pretty popular recording artist out of nowhere.
  She was a teenager, and she shows up, and she's singing these bluesy
  torch songs. The first single off of her first album title is called
  ``Shadow Boxer.'' And I'm like, I had just started listening to Nina
  Simone at this point, and I'm like, Oh ma'am, please, this is not ---
\item
  jenna wortham\\
  {[}LAUGHING{]}
\item
  wesley morris\\
  I don't want this from you. You don't have what it takes. And then
  ``Criminal'' happens. Now where were you when you first saw that
  video?
\item
  jenna wortham\\
  Listen, I'm not afraid to say it. I was a chunky teenager. So my
  little chunky butt would sit down in front of the TV, with a bag of
  cheetos, and I would watch these music videos. And I definitely
  remember when ``Criminal'' just kept playing.
\item
  {[}music - fiona apple, "criminal"{]}
\item
  fiona apple (singing)\\
  I've been a bad, bad girl.
\end{itemize}

wesley morris

Basically, in the video, Fiona Apple is writhing around on the floor.
She's taking pictures of her friends. She is very thin and taking off
her clothes at some point, and she's in lingerie. And do you remember
that part?

jenna wortham

I mean, god yes, who can forget it? That image of Fiona Apple with the
two braids, crouched and hidden in the closet was just emblazoned across
my young mind. And the sad thing is I wanted to be in that druggie den
with her. And it was hard. I struggled with the popularity of that video
because it also meant the popularity of the aesthetic of that video,
which was rail thin, really strung out-looking, waify white girls.

wesley morris

It is immediately deemed part of what we --- if anybody who was around
in the late 90s remembers as being ``heroin chic.'' And the fear of
heroin chic, and the controversy of so-called heroin chic, Kate Moss
being the epitome of that. This is basically skinny white people rolling
around on the floor and being the subject of many a fashion campaign.
The video goes on to win Best Female Video at the 1997 Video Music
Awards. And Fiona Apple goes up to accept her Moonman trophy.

\begin{itemize}
\item
  archived recording (fiona apple)\\
  I didn't prepare a speech, and I'm sorry, but I'm glad that I didn't
  because I'm not going to do this like everybody else does it. `Cause
  everybody that I should be thanking, I'm really sorry, but I have to
  use this time. See, Maya Angelou said that we as human beings at our
  best can only create opportunities. And I'm going to use this
  opportunity the way that I want to use it.

  {[}CHEERS AND APPLAUSE{]}

  So what I want to say is, everybody out there that's watching,
  everybody that's watching this world, this world is bull ---
  {[}MUTED{]}

  {[}APPLAUSE{]}

  And you shouldn't model your life --- wait a second --- you shouldn't
  model your life about what you think that we think is cool, and what
  we're wearing, and what we're saying and everything. Go with yourself.
  Go with yourself.

  {[}CHEERING{]}
\end{itemize}

jenna wortham

People were really confused by her speech. They didn't know what to make
of it. She was called rude. She was called combative.

wesley morris

They called her crazy.

jenna wortham

Yes, they called her crazy, and she was seen as entitled and bratty. And
she essentially was banished, and she retreated. And it really created
the impression that she was reclusive and totally a mystery from then on
out.

wesley morris

Inscrutable, stubborn, difficult to work with.

jenna wortham

As if, like, those aren't things that men always are. It's just always
hilarious to me when it's a woman or a person of color. It's like, Oh,
they're difficult. But it made me really respect her because I was like
yes, I will go with myself. I will do what I want to do.

wesley morris

Well, what it did for me was make me really curious about what this
person was going to do musically, right? I was curious about how the
person who has this enormous backlash against her for speaking what is
honestly the truth --- I'm really curious about what that person does
next. Two years later, my entire world gets blown up when she puts out
this album called ``When the Pawn.'' This album to me was the moment
where I was officially in love with this person.

jenna wortham

Mmm.

wesley morris

Because she took that moment in 1997, in front of the whole world, and
she found a way to not retreat from it, but to amplify the fury she felt
about how she was treated during that period, where she is made famous
almost kind of against her will, and held up as being an icon of
something that she didn't want to be iconic for being. And the reason
that you and I are talking about Fiona Apple is because she's still
speaking. And she's still speaking in the way she wants to speak, and
she's finding new ways of saying what she wants to say. And a couple of
weeks ago, she releases a new album, her first album in eight years, and
it's called ``Fetch the Bolt Cutters.'' It's the perfect capstone on
20-something years of living with and learning to love this woman.

jenna wortham

It's also a perfect accompaniment to living in this moment, and being on
lockdown, and being in crisis, and feeling lots of big emotions that we
don't know what to do with. And lo and behold, here we have a piece of
art that is exactly rooted in what you do with feelings of frustration,
pent up rage, anger, and ultimately funnels them into action. It's a
really perfect companion for Covid, even though it's not meant to be.
It's been in the works for a long time. But it actually turns out to be
the perfect companion for right now.

wesley morris

So we are going to take a break. And when we come back, we're going to
talk about Fiona Apple's latest masterpiece, ``Fetch the Bolt Cutters.''

{[}music - kindness, "world restart"{]}

{[}music - fiona apple, "fetch the bolt cutters"{]}

\begin{itemize}
\item
  fiona apple (singing)\\
  Fetch the bolt cutters. I've been in here too long.

  Fetch the bolt cutters.
\end{itemize}

jenna wortham

Wesley, you know what gives me the shivers in this song?

wesley morris

What?

jenna wortham

It's the way she says ``fetch the bolt cutters,'' right? It's not like a
scream. It's not like (SHOUTING) fetch the bolt cutters! It's just this
quiet ---

wesley morris

Ring the alarm!

jenna wortham

Right, right, right. It's not this alarmist battle cry. It's just this
very resolved, astute observation that it's time to fetch the bolt
cutters. And it gives me the chills because there's no --- it's not
like, do you want to get them? Let me help you if you've been thinking
about getting the bolt cutters and liberating yourself. No, no, no, it's
time. We're leaving. Go get the bolt cutters. And since the song came
out, I've noticed it across just so many different places in social
media, but in particular, it's being used to score videos and Instagram
stories. And it's everything from somebody breaking down a chicken or a
piece of meat for dinner, which is really grim and gruesome in all the
best ways. You know, I saw recently a friend was playing the song while
she was preparing a really luxurious bath for herself. Somebody else was
using it to dance. I mean, I was playing it while I was cooking the
other night and then posted a story with it. And I think it really is a
soundtrack for the coronavirus pandemic. It feels so appropriate because
even though, physically, we really can't fetch the bolt cutters. We've
been actually told many times to not fetch the bolt cutters, actually to
put them away, put the lock back on the door, stay inside. But
metaphorically, this idea that it's time to step outside of ourselves.
It's time to step outside of our comfort zones. It's time to do things
differently and to think about other people, or to think about, I guess,
ourselves in relationship to other people and make different decisions,
and unlock the parts of yourself that have been yearning to come out.
Because literally it's now or never, like we've faced what has felt like
end times. So if not now, then when? And I think that's why people --- I
think that's why women in particular are really responding to it because
that reminder is coming along at a time when we most need it.

wesley morris

The calm with which she says fetch the bolt cutters strikes me as, ``My
water just broke.'' She says that, and what I hear is a person
responding to an almost involuntary force that she knew was coming this
whole time. And it is finally here, and she is ready to release it into
the world.

jenna wortham

It's just this declaration that it's time for whatever was holding me
back in the past, or whatever --- I just imagine a pair of pliers. It's
just like these things are being extracted. It's like whatever vestige
of the past, whatever thing you don't need anymore, whatever thorn in
your foot, whatever is in your way, it's like we're plucking it out one
by one and setting it to the side. So for me, the visual is not like a
fence coming down, or a gate lock being broken. It's actually like these
tiny, methodical, little extractions and excisions being made, like a
process is happening by which through we will then exit.

wesley morris

Mmm. Now look, I don't think that Fiona Apple is trying to sound like
anybody at any particular time. I don't know what she's listening to
now. But there's this part of this song where it sounds to me like a
little bit Rickie Lee Jones in one spot, this move into Beyoncé, and
then back into Rickie Lee Jones, for just like 12 seconds. And it's just
really beautiful.

{[}music - fiona apple, "fetch the bolt cutters"{]}

\begin{itemize}
\tightlist
\item
  fiona apple (singing)\\
  While I'd not yet found my bearings those it girls hit the ground
  comparing the way I was to the way she was, saying I'm not stylish
  enough, and I cry too much. And I listened because I hadn't found my
  own voice yet. So all I could hear was the noise that people make when
  they don't know shit. But I didn't know that yet.
\end{itemize}

wesley morris

The way she sort of sing-talks that, and then she lowers her voice, and
she kind of runs the words together. I heard that, and I thought, I hear
Beyoncé doing the verses in ``Hold Up,'' and she's doing them as a sort
of lullaby. And there's just something about this part of the song that
is so fragrant to me, the allusion to these two women, these two very
different singer-songwriters, just really got me.

jenna wortham

Mmm. It's really interesting to invoke our queen and patron saint,
Beyoncé, in relationship to this album because there's actually a lot in
``Fetch the Bolt Cutters'' that makes me think a lot about ``Lemonade,''
in particular. And not really sonically yet, even though I do see the
point that you just made. But I think just tonally, it's really, really,
really satisfying to listen to women at a very particular point in time
--- their 40s. Beyoncé is almost there, Fiona is a little bit over that.
These are both women who've lived most of their teen years in the public
eye. These are women who both have been agents of their sexuality and
have had their sexuality be used to kind of sell music, and to sell
them. And to sort of watch them take control over what that looks like
and what that means to them is really gratifying. And I'm listening to
the song itself, ``Fetch the Bolt Cutters,'' I'm hearing Fiona Apple do
her thing. But in my mind's eye, I'm seeing Beyoncé in that incredible
marigold, lemony dress with all the ruffles, and her hair is just
flowing, walking down the street, twirling the bat and bashing out the
windows.

wesley morris

{[}LAUGHING{]}

jenna wortham

And it's just remarkable. They're both so resigned to destruction, but
no longer self-destruction. And I'm a little bit younger than both of
them, but I'm also coming into that self-awareness of not just kind of
who I've been, but sort of taking all of that knowledge and sort of
paying it forward and thinking about who I want to be, especially in
relationship to myself and other women in my world. And they do kind of
offer these blueprints for how to take in all that emotion and kind of
how to push it back out in a way that's cathartic and constructive, at
least to them.

wesley morris

The thing that comes to mind when you say that is ``Heavy Balloon.'' The
opening line is like, People like us play with a heavy balloon. It's
clearly about depression. But it also, even through the depression, it's
an attempt to reach out. It's the most rhythmic and grooving song on the
album. And it's the song on the album that sounds most like a song on
her first album, ``Tidal,'' called ``The First Taste.''

jenna wortham

Mm-hm.

wesley morris

It is this evocation of an earlier self, a lighter self.

{[}music - fiona apple, "heavy balloon"{]}

\begin{itemize}
\tightlist
\item
  fiona apple (singing)\\
  People like us, we play with a heavy balloon. We keep it up to keep
  the devil at bay but it always falls way too soon.
\end{itemize}

jenna wortham

But I think you feel that interplay between the heaviness and the
darkness because she's talking about all this extra weight, and how that
density makes it hard to rise up. But then, you get to the chorus.

{[}music - fiona apple, "heavy balloon"{]}

\begin{itemize}
\tightlist
\item
  fiona apple (singing)\\
  But you know what? I spread like strawberries. I climb like peas and
  beans. I've been sucking it in so long that I'm busting at the seams.
\end{itemize}

jenna wortham

Whenever I hear it, I'm like throwing my head back and I'm like staring
up at the sky, and I'm just like gyrating my hips to myself. I'm like,
yeah, spread like strawberries. And she said in an interview that she
was really moved by the idea about the ways that strawberries grow out
and they cover an entire garden. And so there is this levity in that
imagery of just like, yeah, I might be heavy, I might be depressed. But
I'm actually going to keep doing my thing. I'm going to keep growing.
And that song is one my favorites on the album for sure.

wesley morris

Oh, it's great.

jenna wortham

It's the one I play the loudest.

wesley morris

I am in awe of this woman. The way she is able to capture feelings that
are so mundane, to be able to take those kitchen staples and turn them
into objects of determination and pain. Geez, Louise.

{[}music - fiona apple, "heavy balloon"{]}

\begin{itemize}
\tightlist
\item
  fiona apple (singing)\\
  I spread like strawberries. I climb like peas and beans.
\end{itemize}

jenna wortham

I really want to talk about the song ``Ladies'' off the album. Fiona
Apple is talking to her ex's new girlfriend --- or you can pluralize it,
new girlfriends. I mean, she's really just talking to the lineage ---
the line of women that are coming after her. And the song is both a
warning and both a little bit of a preemptive apology for kind of what
they're in for. I like to think about this song as kind of its own
whisper network, like Fiona's trying to tell these women what it's going
to be like to be in a relationship with him. But I also don't feel like
she's blaming them for the choices that they're going to make. She just
wants them to make informed choices. If this is your choice and you make
it, OK, but I'll be here for you on the other side if you want me to be.
Because I've been there too, and I know what that's like. And damn, what
a gesture. That takes a lot of self-reflection (LAUGHING) and
introspection.

wesley morris

Yeah. Even the idea that she's just sort of saying, you know, I might've
left some stuff at his place, take it. When you go, you can take
whatever I left.

jenna wortham

She even says though, There's a dress that's not mine, but don't throw
it out. It was there when I got there. And let's be at peace with the
way time works, and the way life works.

{[}music - fiona apple, "ladies"{]}

\begin{itemize}
\tightlist
\item
  fiona apple (singing)\\
  Ladies, ladies, ladies, ladies. Ruminations on the looming effect and
  the parallax view and the figure and the form ---
\end{itemize}

wesley morris

This song also has this concept of parallax, which is a thing that Fiona
Apple has sung about in other songs on other albums. And it's a useful
relationship concept, in a lot of ways, which is that, imagine looking
at an object through your viewfinder on a camera and the lens on your
camera. You're looking at the same object, but in two different ways
through two different lenses. And seems pretty useful when it comes to
looking at people, too. Sometimes you're looking at a person through one
set of circumstances, and another set of circumstances reveals the exact
same person in a completely different way. It's taken her a long time,
perhaps, to get to the point where she is able to practice her own
parallaxing in order to look at, not only old relationships, but to
understand that there is just an inherent parallax in any relationship.
And once you get some distance on it, you're able to see yourself as
somebody's future ex or current relationship.

jenna wortham

Mmm. I really like the generosity of that song. Because it's a very
natural thing to think about your current partner's exes, or your last
partner, who they date after you. It's very easy to get caught up in
these spirals and to compare and despair, and to try to weigh your own
enough-ness against these shadowy images of other people and other
women. And I really feel like she's trying to open a conversation and be
like, we are not each other's enemies. I'm no longer playing this game.
And come over for tea. Even more, like, let's go on a hike together. I
just want you to know, I'm not gunning for you. And you start to realize
the ways in which you've been taught and conditioned to only see other
women as rivals or people you should compare yourself against, rather
than see them as what they can be, which are collaborators, true loves,
like deep, deep friends, sisters.

wesley morris

I love how loungey this song sounds. It's got this really good bass
line, and the groove is familiar to me as a kind of a little bit of an
R\&B vamp, where the story being told is one thing, but there is also a
way that you have to perform the story. And I hear it, and I'm just
like, Nina Simone could do this. Aretha could do this. Gladys Knight
could do this. These three very, very different singers could all
respond to this song in their characteristic ways without necessarily
even having to do much of a rearrangement ---

jenna wortham

Oh, yeah.

wesley morris

--- of the song itself. But this kind of world-weary, been through it
all way of (LAUGHING) communicating.

jenna wortham

Oh, yeah. I love, love, love the kind of fatigue in it and the humor.
And I can totally just see Fiona Apple with a black fedora on, and a
white tank top and suspenders, and her sinewy arms are out, and she's in
the back of some sort of smoky cabaret. {[}LAUGHTER{]} And she's just
like cocked to the side and clapping in this very cynical way.

{[}music - fiona apple, "ladies"{]}

\begin{itemize}
\tightlist
\item
  fiona apple (singing)\\
  Yet another woman to whom I won't get through.
\end{itemize}

wesley morris

The thing about Fiona Apple is she's always been about warning somebody
about something.

jenna wortham

{[}LAUGHING{]}

wesley morris

Whether it's a potential relationship, or herself about herself. And
this album is the first time all of that experience, and all of that
knowledge, and all of that rejection, and avoidance, and judgment, and
perception has been harnessed for the powers of what can only be
described as wisdom.

jenna wortham

Mmm. Well Wesley, I am just now reigniting my love affair with Fiona
Apple. But you've been deep in it with her for the last 20 years or so.
So help me understand the work that she's done from the last time I was
paying attention, with ``Tidal,'' to now, with ``Fetch the Bolt
Cutters.''

wesley morris

OK, so if we want to understand a little bit where Fiona Apple is in
2020, relative to where she was say, even in 2012, I think it might be
useful to actually just go back to 2012 and listen to one of the best
songs on her previous great album, which is called ``The Idler Wheel,''
the I-D-L-E-R wheel. And the song that came to mind that is useful, both
in terms of how we've been talking about her today, and also in terms of
her adventurousness as a singer, is this song called ``Regret.'' It's
just structured really beautifully, and it sounds like she's built this
one kind of seemingly straightforward, beautiful song, and then
something happens.

{[}music - fiona apple, "regret"{]}

\begin{itemize}
\tightlist
\item
  fiona apple (singing)\\
  Oh, I ran out of white doves' feathers to soak up the hot piss that
  comes from your mouth every time you address me.
\end{itemize}

wesley morris

There's a kind of singing that I really love and is very hard to do. I'm
talking specifically about ugly singing, or unusual singing, right?
There aren't a lot of people who have the skill, and I would say even
more important, the bravery to do it, to try it. The recorded music
history is a lot of such people. I mean, you can go back to someone like
Louis Armstrong, or Screamin' Jay Hawkins, and Nina Simone, and Yoko
Ono, and Tom Waits, the patron saint of back of the throat, bottom of
the earth singing. People who are much more interested in the truth of
the sound of the voice than of the alleged beauty of the presentation.
It's just such a beautiful, beautiful song about real ugly feelings. And
I feel like that's the thing that I love her so much for is she isn't
afraid to make ugly feelings sound ugly. She doesn't run away from where
she is emotionally. I mean, this song is stripped down to almost
nothing. In that beautiful moment at the end, where she --- you hear the
piano top close.

jenna wortham

Yeah. (LAUGHING)

wesley morris

She's like, I'm done. I've said it all.

jenna wortham

Yeah. And you know Wesley, maybe some of the distance that has been
closed, or crossed, or worked out in the last eight years, I guess,
between this song coming out, ``Regret,'' on ``The Idler Wheel'' and
then ``Fetch the Bolt Cutters'' is that Fiona Apple has figured out at
least what she's going to do with all those raw nerve endings. She's
learned a little bit how to process them, how to sit with them, and how
to be OK. (LAUGHING) Right? And she's teaching us also right now how to
be OK with them. Because one of the differences between a song like
``Regret'' and then something like ``Heavy Balloon,'' that pain and
grief of a song like ``Regret'' has been transmuted. And it's not that
she's avoiding it, or she's not dealing with it, it's just that she has
metabolized it and she has channeled it into something else.

And that's why I think, even though ``Fetch the Bolt Cutters'' is still
so, so, so emotionally dense, and the frequency is so high, it can feel
peaceful, it can feel calm. Because she's worked out how she feels about
these things. And now she's telling us so that we can do the same.

{[}music{]}

\begin{itemize}
\item
  speaker 1\\
  Hey, Jenna and Wesley.
\item
  speaker 2\\
  Hi, Jenna, hi, Wesley.
\item
  speaker 4\\
  Hey, y'all.
\end{itemize}

{[}music{]}

jenna wortham

A few weeks ago, we asked you to tell us how you're taking care during
these times, and how people are taking care of you. And the responses
were overwhelming.

wesley morris

We heard from people all over the planet.

\begin{itemize}
\item
  mike\\
  This is Mike coming to you from Chicago.
\item
  speaker 5\\
  Cambridge, U.K.
\item
  speaker 6\\
  I live in Greenpoint in Brooklyn, New York.
\item
  speaker 7\\
  In Lisbon, Portugal's capital in mainland Europe.
\end{itemize}

wesley morris

And you guys are doing amazing work just taking care of yourselves.

\begin{itemize}
\item
  speaker 8\\
  Hi, Wesley and Jenna. I'm calling in with a story of how my husband
  has taken care of me during this difficult time. I had a birthday in
  the end of March, and had plans to celebrate, but obviously wasn't
  able to do that. I work for one of the major museums in the city,
  which is shut down. And so my husband surprised me by asking a bunch
  of our friends to create something. They emailed it to him, he printed
  it out, and he hung it up. He got actually 70 submissions of works of
  art, and he had so many submissions, there are two rooms in the house
  filled with art created by our loved ones. And he surprised me with my
  own museum with art that I could come and look at and take care of.
  And it not only took care of me, but it was a bright spot for a lot of
  people as well, where they stopped for a second, and sat down, and
  created something, or they had their kid make something. And so my
  husband was able to do something that took care of me, but also took
  care of a lot of our friends, even for a brief second. It was a very
  unusual 36th birthday.
\item
  claire grindinger\\
  OK. Hey, Jenna. Hey, Wesley. My name's Claire Grindinger. I'm a senior
  at Washington University in St. Louis, but right now I'm home in
  Dallas, Texas, cause of the coronavirus. I am right now running, as
  you might be able to tell. On March 29, I was supposed to run the St.
  Louis Go marathon, but I decided to run it virtually. And so my
  parents are coming around. They're following me on my route, bringing
  me a jacket, food, water and whatnot because there are no water
  stations, and just really helping me out. So I feel really supported
  and lucky to be here with them. So thank you.
\item
  inna kim\\
  Hi, this is Inna. I am in New Jersey. I moved from South Korea a
  couple of months ago, and this is not easy situation for all. But I
  have a routine. I exercise. I bake bunch. And I also have a baby, who
  is 10 months old, which makes me keep going and super busy. And for my
  family in South Korea, I try to contact them every day, sending a
  bunch of photos of baby, especially to my mom. And actually, I just
  sent her gift box, which has a bunch of lemons and gingers for her
  immune system. And I hope she's fine and well. And yeah, thank you.
\item
  speaker 9\\
  Hi, Jenna. Hi, Wesley. I'm currently studying in Cambridge, the U.K.,
  where I am in self-isolation, of course. My little sister does not
  live with me. So she's in London right now. But we have been having
  FaceTime calls where we have dance parties. And we have karaoke
  sessions where we do what we would normally do if we were living
  together, which is sing, scream a Disney song at the top of our lungs.
  But we're doing it virtually over the internet. It's really worked to
  distract me and just really make me feel silly and forget about the
  absurdity of what we're currently going through. So I hope you two are
  taking care, and I will see you on the internet.
\item
  mike pesoli\\
  Hey, Jenna. Hey, Wesley. This is Mike coming to you from Chicago. I am
  currently taking care of myself during this crisis of Covid-19 by
  throwing myself into literature. I know it might sound basic, but I
  find books to be a great way into someone else's psyche or into
  someone else's experience. And for the few lucky people in my life, I
  will even go as far as calling my local bookstore --- my favorite is
  Unabridged in Chicago --- and I will send them a book that I think
  they might really love. And then in that way, I'm taking care of a
  small business that really means something to me, I'm taking care of
  someone that I love, and I'm taking care of myself because it makes me
  feel good to give something to someone in this moment when so many of
  us can feel helpless. Cheers.
\item
  speaker 10\\
  Hi, Jenna and Wesley. Something that I've been doing with my
  community, and my loved ones, my family and friends, is just trying to
  be vulnerable from afar, really getting over that fear of admitting
  I'm not doing well right now. It's a huge relief. It's kind of like
  its own way of me taking care right now is just by that, letting go.
  And knowing that that doesn't mean that you can't handle things, but
  your ability to reach out to a friend and say, look, I can't handle
  everything I'm feeling right now. Are you available for me to share?
  And be well.
\end{itemize}

{[}music{]}

wesley morris

Thank you to every single person who submitted something. We really,
really appreciate it.

jenna wortham

And please keep taking care of yourself and your loved ones.

{[}music{]}

That's our show. Next week, we're going to be talking about all this
newfound closeness that even while living in a broken, fragmented world,
we're finding new ways to be together, intimate, close, online and
through our devices.

``Still Processing'' is a product of The New York Times. This week, like
all the past couple of weeks, it was recorded in our living rooms.

wesley morris

It's produced by Hans Beutow and Sydney Harper.

jenna wortham

Our editors are Sara Sarasohn, Sasha Weiss, Wendy Door, and Lisa Tobin.

wesley morris

And our engineer is Jake Gorski.

jenna wortham

Our theme music is by Kindness. It's called ``World Restart,'' from the
album ``Otherness.'' You can find all our old episodes and the show
notes for this one and those at nytimes.com/stillprocessing.

{[}music{]}

\href{https://www.nytimes.com/column/still-processing-podcast}{\includegraphics{https://static01.nyt.com/images/2019/09/15/podcasts/still-processing-album-art-2/still-processing-album-art-2-square320.jpg}Still
Processing}Subscribe:

\begin{itemize}
\tightlist
\item
  \href{https://itunes.apple.com/us/podcast/id1151436460}{Apple
  Podcasts}
\item
  \href{https://www.google.com/podcasts?feed=aHR0cHM6Ly9yc3MuYXJ0MTkuY29tL255dC1zdGlsbC1wcm9jZXNzaW5n}{Google
  Podcasts}
\end{itemize}

\hypertarget{fiona-ex-machina-1}{%
\section{Fiona Ex Machina}\label{fiona-ex-machina-1}}

\hypertarget{fiona-apple-blows-our-minds-again-1}{%
\subsection{Fiona Apple blows our minds,
again.}\label{fiona-apple-blows-our-minds-again-1}}

Hosted by Wesley Morris and Jenna Wortham. Produced by Hans Buetow and
Sydney Harper.

Transcript

transcript

Back to Still Processing

bars

0:00/0:00

-0:00

transcript

\hypertarget{fiona-ex-machina-2}{%
\subsection{Fiona Ex Machina}\label{fiona-ex-machina-2}}

\hypertarget{hosted-by-wesley-morris-and-jenna-wortham-produced-by-hans-buetow-and-sydney-harper-1}{%
\subsubsection{Hosted by Wesley Morris and Jenna Wortham. Produced by
Hans Buetow and Sydney
Harper.}\label{hosted-by-wesley-morris-and-jenna-wortham-produced-by-hans-buetow-and-sydney-harper-1}}

\hypertarget{fiona-apple-blows-our-minds-again-2}{%
\paragraph{Fiona Apple blows our minds,
again.}\label{fiona-apple-blows-our-minds-again-2}}

Thursday, April 30th, 2020

\begin{itemize}
\item
  {[}music{]}
\item
  jenna wortham\\
  This is ``Still Processing.'' I'm Jenna Wortham.
\item
  wesley morris\\
  I'm Wesley Morris. And Jenna, I have a confession.
\item
  jenna wortham\\
  Oh.
\item
  wesley morris\\
  I have not always loved Fiona Apple.
\item
  jenna wortham\\
  What? I'm sorry. What?
\item
  wesley morris\\
  No, it's true.
\item
  jenna wortham\\
  You know what this week's episode is about.
\item
  wesley morris\\
  {[}LAUGHING{]} No, it's true. I did not enjoy her. It started for me
  like it did for everybody else, my relationship with this woman, in
  1996 when she was a pretty popular recording artist out of nowhere.
  She was a teenager, and she shows up, and she's singing these bluesy
  torch songs. The first single off of her first album title is called
  ``Shadow Boxer.'' And I'm like, I had just started listening to Nina
  Simone at this point, and I'm like, Oh ma'am, please, this is not ---
\item
  jenna wortham\\
  {[}LAUGHING{]}
\item
  wesley morris\\
  I don't want this from you. You don't have what it takes. And then
  ``Criminal'' happens. Now where were you when you first saw that
  video?
\item
  jenna wortham\\
  Listen, I'm not afraid to say it. I was a chunky teenager. So my
  little chunky butt would sit down in front of the TV, with a bag of
  cheetos, and I would watch these music videos. And I definitely
  remember when ``Criminal'' just kept playing.
\item
  {[}music - fiona apple, "criminal"{]}
\item
  fiona apple (singing)\\
  I've been a bad, bad girl.
\end{itemize}

wesley morris

Basically, in the video, Fiona Apple is writhing around on the floor.
She's taking pictures of her friends. She is very thin and taking off
her clothes at some point, and she's in lingerie. And do you remember
that part?

jenna wortham

I mean, god yes, who can forget it? That image of Fiona Apple with the
two braids, crouched and hidden in the closet was just emblazoned across
my young mind. And the sad thing is I wanted to be in that druggie den
with her. And it was hard. I struggled with the popularity of that video
because it also meant the popularity of the aesthetic of that video,
which was rail thin, really strung out-looking, waify white girls.

wesley morris

It is immediately deemed part of what we --- if anybody who was around
in the late 90s remembers as being ``heroin chic.'' And the fear of
heroin chic, and the controversy of so-called heroin chic, Kate Moss
being the epitome of that. This is basically skinny white people rolling
around on the floor and being the subject of many a fashion campaign.
The video goes on to win Best Female Video at the 1997 Video Music
Awards. And Fiona Apple goes up to accept her Moonman trophy.

\begin{itemize}
\item
  archived recording (fiona apple)\\
  I didn't prepare a speech, and I'm sorry, but I'm glad that I didn't
  because I'm not going to do this like everybody else does it. `Cause
  everybody that I should be thanking, I'm really sorry, but I have to
  use this time. See, Maya Angelou said that we as human beings at our
  best can only create opportunities. And I'm going to use this
  opportunity the way that I want to use it.

  {[}CHEERS AND APPLAUSE{]}

  So what I want to say is, everybody out there that's watching,
  everybody that's watching this world, this world is bull ---
  {[}MUTED{]}

  {[}APPLAUSE{]}

  And you shouldn't model your life --- wait a second --- you shouldn't
  model your life about what you think that we think is cool, and what
  we're wearing, and what we're saying and everything. Go with yourself.
  Go with yourself.

  {[}CHEERING{]}
\end{itemize}

jenna wortham

People were really confused by her speech. They didn't know what to make
of it. She was called rude. She was called combative.

wesley morris

They called her crazy.

jenna wortham

Yes, they called her crazy, and she was seen as entitled and bratty. And
she essentially was banished, and she retreated. And it really created
the impression that she was reclusive and totally a mystery from then on
out.

wesley morris

Inscrutable, stubborn, difficult to work with.

jenna wortham

As if, like, those aren't things that men always are. It's just always
hilarious to me when it's a woman or a person of color. It's like, Oh,
they're difficult. But it made me really respect her because I was like
yes, I will go with myself. I will do what I want to do.

wesley morris

Well, what it did for me was make me really curious about what this
person was going to do musically, right? I was curious about how the
person who has this enormous backlash against her for speaking what is
honestly the truth --- I'm really curious about what that person does
next. Two years later, my entire world gets blown up when she puts out
this album called ``When the Pawn.'' This album to me was the moment
where I was officially in love with this person.

jenna wortham

Mmm.

wesley morris

Because she took that moment in 1997, in front of the whole world, and
she found a way to not retreat from it, but to amplify the fury she felt
about how she was treated during that period, where she is made famous
almost kind of against her will, and held up as being an icon of
something that she didn't want to be iconic for being. And the reason
that you and I are talking about Fiona Apple is because she's still
speaking. And she's still speaking in the way she wants to speak, and
she's finding new ways of saying what she wants to say. And a couple of
weeks ago, she releases a new album, her first album in eight years, and
it's called ``Fetch the Bolt Cutters.'' It's the perfect capstone on
20-something years of living with and learning to love this woman.

jenna wortham

It's also a perfect accompaniment to living in this moment, and being on
lockdown, and being in crisis, and feeling lots of big emotions that we
don't know what to do with. And lo and behold, here we have a piece of
art that is exactly rooted in what you do with feelings of frustration,
pent up rage, anger, and ultimately funnels them into action. It's a
really perfect companion for Covid, even though it's not meant to be.
It's been in the works for a long time. But it actually turns out to be
the perfect companion for right now.

wesley morris

So we are going to take a break. And when we come back, we're going to
talk about Fiona Apple's latest masterpiece, ``Fetch the Bolt Cutters.''

{[}music - kindness, "world restart"{]}

{[}music - fiona apple, "fetch the bolt cutters"{]}

\begin{itemize}
\item
  fiona apple (singing)\\
  Fetch the bolt cutters. I've been in here too long.

  Fetch the bolt cutters.
\end{itemize}

jenna wortham

Wesley, you know what gives me the shivers in this song?

wesley morris

What?

jenna wortham

It's the way she says ``fetch the bolt cutters,'' right? It's not like a
scream. It's not like (SHOUTING) fetch the bolt cutters! It's just this
quiet ---

wesley morris

Ring the alarm!

jenna wortham

Right, right, right. It's not this alarmist battle cry. It's just this
very resolved, astute observation that it's time to fetch the bolt
cutters. And it gives me the chills because there's no --- it's not
like, do you want to get them? Let me help you if you've been thinking
about getting the bolt cutters and liberating yourself. No, no, no, it's
time. We're leaving. Go get the bolt cutters. And since the song came
out, I've noticed it across just so many different places in social
media, but in particular, it's being used to score videos and Instagram
stories. And it's everything from somebody breaking down a chicken or a
piece of meat for dinner, which is really grim and gruesome in all the
best ways. You know, I saw recently a friend was playing the song while
she was preparing a really luxurious bath for herself. Somebody else was
using it to dance. I mean, I was playing it while I was cooking the
other night and then posted a story with it. And I think it really is a
soundtrack for the coronavirus pandemic. It feels so appropriate because
even though, physically, we really can't fetch the bolt cutters. We've
been actually told many times to not fetch the bolt cutters, actually to
put them away, put the lock back on the door, stay inside. But
metaphorically, this idea that it's time to step outside of ourselves.
It's time to step outside of our comfort zones. It's time to do things
differently and to think about other people, or to think about, I guess,
ourselves in relationship to other people and make different decisions,
and unlock the parts of yourself that have been yearning to come out.
Because literally it's now or never, like we've faced what has felt like
end times. So if not now, then when? And I think that's why people --- I
think that's why women in particular are really responding to it because
that reminder is coming along at a time when we most need it.

wesley morris

The calm with which she says fetch the bolt cutters strikes me as, ``My
water just broke.'' She says that, and what I hear is a person
responding to an almost involuntary force that she knew was coming this
whole time. And it is finally here, and she is ready to release it into
the world.

jenna wortham

It's just this declaration that it's time for whatever was holding me
back in the past, or whatever --- I just imagine a pair of pliers. It's
just like these things are being extracted. It's like whatever vestige
of the past, whatever thing you don't need anymore, whatever thorn in
your foot, whatever is in your way, it's like we're plucking it out one
by one and setting it to the side. So for me, the visual is not like a
fence coming down, or a gate lock being broken. It's actually like these
tiny, methodical, little extractions and excisions being made, like a
process is happening by which through we will then exit.

wesley morris

Mmm. Now look, I don't think that Fiona Apple is trying to sound like
anybody at any particular time. I don't know what she's listening to
now. But there's this part of this song where it sounds to me like a
little bit Rickie Lee Jones in one spot, this move into Beyoncé, and
then back into Rickie Lee Jones, for just like 12 seconds. And it's just
really beautiful.

{[}music - fiona apple, "fetch the bolt cutters"{]}

\begin{itemize}
\tightlist
\item
  fiona apple (singing)\\
  While I'd not yet found my bearings those it girls hit the ground
  comparing the way I was to the way she was, saying I'm not stylish
  enough, and I cry too much. And I listened because I hadn't found my
  own voice yet. So all I could hear was the noise that people make when
  they don't know shit. But I didn't know that yet.
\end{itemize}

wesley morris

The way she sort of sing-talks that, and then she lowers her voice, and
she kind of runs the words together. I heard that, and I thought, I hear
Beyoncé doing the verses in ``Hold Up,'' and she's doing them as a sort
of lullaby. And there's just something about this part of the song that
is so fragrant to me, the allusion to these two women, these two very
different singer-songwriters, just really got me.

jenna wortham

Mmm. It's really interesting to invoke our queen and patron saint,
Beyoncé, in relationship to this album because there's actually a lot in
``Fetch the Bolt Cutters'' that makes me think a lot about ``Lemonade,''
in particular. And not really sonically yet, even though I do see the
point that you just made. But I think just tonally, it's really, really,
really satisfying to listen to women at a very particular point in time
--- their 40s. Beyoncé is almost there, Fiona is a little bit over that.
These are both women who've lived most of their teen years in the public
eye. These are women who both have been agents of their sexuality and
have had their sexuality be used to kind of sell music, and to sell
them. And to sort of watch them take control over what that looks like
and what that means to them is really gratifying. And I'm listening to
the song itself, ``Fetch the Bolt Cutters,'' I'm hearing Fiona Apple do
her thing. But in my mind's eye, I'm seeing Beyoncé in that incredible
marigold, lemony dress with all the ruffles, and her hair is just
flowing, walking down the street, twirling the bat and bashing out the
windows.

wesley morris

{[}LAUGHING{]}

jenna wortham

And it's just remarkable. They're both so resigned to destruction, but
no longer self-destruction. And I'm a little bit younger than both of
them, but I'm also coming into that self-awareness of not just kind of
who I've been, but sort of taking all of that knowledge and sort of
paying it forward and thinking about who I want to be, especially in
relationship to myself and other women in my world. And they do kind of
offer these blueprints for how to take in all that emotion and kind of
how to push it back out in a way that's cathartic and constructive, at
least to them.

wesley morris

The thing that comes to mind when you say that is ``Heavy Balloon.'' The
opening line is like, People like us play with a heavy balloon. It's
clearly about depression. But it also, even through the depression, it's
an attempt to reach out. It's the most rhythmic and grooving song on the
album. And it's the song on the album that sounds most like a song on
her first album, ``Tidal,'' called ``The First Taste.''

jenna wortham

Mm-hm.

wesley morris

It is this evocation of an earlier self, a lighter self.

{[}music - fiona apple, "heavy balloon"{]}

\begin{itemize}
\tightlist
\item
  fiona apple (singing)\\
  People like us, we play with a heavy balloon. We keep it up to keep
  the devil at bay but it always falls way too soon.
\end{itemize}

jenna wortham

But I think you feel that interplay between the heaviness and the
darkness because she's talking about all this extra weight, and how that
density makes it hard to rise up. But then, you get to the chorus.

{[}music - fiona apple, "heavy balloon"{]}

\begin{itemize}
\tightlist
\item
  fiona apple (singing)\\
  But you know what? I spread like strawberries. I climb like peas and
  beans. I've been sucking it in so long that I'm busting at the seams.
\end{itemize}

jenna wortham

Whenever I hear it, I'm like throwing my head back and I'm like staring
up at the sky, and I'm just like gyrating my hips to myself. I'm like,
yeah, spread like strawberries. And she said in an interview that she
was really moved by the idea about the ways that strawberries grow out
and they cover an entire garden. And so there is this levity in that
imagery of just like, yeah, I might be heavy, I might be depressed. But
I'm actually going to keep doing my thing. I'm going to keep growing.
And that song is one my favorites on the album for sure.

wesley morris

Oh, it's great.

jenna wortham

It's the one I play the loudest.

wesley morris

I am in awe of this woman. The way she is able to capture feelings that
are so mundane, to be able to take those kitchen staples and turn them
into objects of determination and pain. Geez, Louise.

{[}music - fiona apple, "heavy balloon"{]}

\begin{itemize}
\tightlist
\item
  fiona apple (singing)\\
  I spread like strawberries. I climb like peas and beans.
\end{itemize}

jenna wortham

I really want to talk about the song ``Ladies'' off the album. Fiona
Apple is talking to her ex's new girlfriend --- or you can pluralize it,
new girlfriends. I mean, she's really just talking to the lineage ---
the line of women that are coming after her. And the song is both a
warning and both a little bit of a preemptive apology for kind of what
they're in for. I like to think about this song as kind of its own
whisper network, like Fiona's trying to tell these women what it's going
to be like to be in a relationship with him. But I also don't feel like
she's blaming them for the choices that they're going to make. She just
wants them to make informed choices. If this is your choice and you make
it, OK, but I'll be here for you on the other side if you want me to be.
Because I've been there too, and I know what that's like. And damn, what
a gesture. That takes a lot of self-reflection (LAUGHING) and
introspection.

wesley morris

Yeah. Even the idea that she's just sort of saying, you know, I might've
left some stuff at his place, take it. When you go, you can take
whatever I left.

jenna wortham

She even says though, There's a dress that's not mine, but don't throw
it out. It was there when I got there. And let's be at peace with the
way time works, and the way life works.

{[}music - fiona apple, "ladies"{]}

\begin{itemize}
\tightlist
\item
  fiona apple (singing)\\
  Ladies, ladies, ladies, ladies. Ruminations on the looming effect and
  the parallax view and the figure and the form ---
\end{itemize}

wesley morris

This song also has this concept of parallax, which is a thing that Fiona
Apple has sung about in other songs on other albums. And it's a useful
relationship concept, in a lot of ways, which is that, imagine looking
at an object through your viewfinder on a camera and the lens on your
camera. You're looking at the same object, but in two different ways
through two different lenses. And seems pretty useful when it comes to
looking at people, too. Sometimes you're looking at a person through one
set of circumstances, and another set of circumstances reveals the exact
same person in a completely different way. It's taken her a long time,
perhaps, to get to the point where she is able to practice her own
parallaxing in order to look at, not only old relationships, but to
understand that there is just an inherent parallax in any relationship.
And once you get some distance on it, you're able to see yourself as
somebody's future ex or current relationship.

jenna wortham

Mmm. I really like the generosity of that song. Because it's a very
natural thing to think about your current partner's exes, or your last
partner, who they date after you. It's very easy to get caught up in
these spirals and to compare and despair, and to try to weigh your own
enough-ness against these shadowy images of other people and other
women. And I really feel like she's trying to open a conversation and be
like, we are not each other's enemies. I'm no longer playing this game.
And come over for tea. Even more, like, let's go on a hike together. I
just want you to know, I'm not gunning for you. And you start to realize
the ways in which you've been taught and conditioned to only see other
women as rivals or people you should compare yourself against, rather
than see them as what they can be, which are collaborators, true loves,
like deep, deep friends, sisters.

wesley morris

I love how loungey this song sounds. It's got this really good bass
line, and the groove is familiar to me as a kind of a little bit of an
R\&B vamp, where the story being told is one thing, but there is also a
way that you have to perform the story. And I hear it, and I'm just
like, Nina Simone could do this. Aretha could do this. Gladys Knight
could do this. These three very, very different singers could all
respond to this song in their characteristic ways without necessarily
even having to do much of a rearrangement ---

jenna wortham

Oh, yeah.

wesley morris

--- of the song itself. But this kind of world-weary, been through it
all way of (LAUGHING) communicating.

jenna wortham

Oh, yeah. I love, love, love the kind of fatigue in it and the humor.
And I can totally just see Fiona Apple with a black fedora on, and a
white tank top and suspenders, and her sinewy arms are out, and she's in
the back of some sort of smoky cabaret. {[}LAUGHTER{]} And she's just
like cocked to the side and clapping in this very cynical way.

{[}music - fiona apple, "ladies"{]}

\begin{itemize}
\tightlist
\item
  fiona apple (singing)\\
  Yet another woman to whom I won't get through.
\end{itemize}

wesley morris

The thing about Fiona Apple is she's always been about warning somebody
about something.

jenna wortham

{[}LAUGHING{]}

wesley morris

Whether it's a potential relationship, or herself about herself. And
this album is the first time all of that experience, and all of that
knowledge, and all of that rejection, and avoidance, and judgment, and
perception has been harnessed for the powers of what can only be
described as wisdom.

jenna wortham

Mmm. Well Wesley, I am just now reigniting my love affair with Fiona
Apple. But you've been deep in it with her for the last 20 years or so.
So help me understand the work that she's done from the last time I was
paying attention, with ``Tidal,'' to now, with ``Fetch the Bolt
Cutters.''

wesley morris

OK, so if we want to understand a little bit where Fiona Apple is in
2020, relative to where she was say, even in 2012, I think it might be
useful to actually just go back to 2012 and listen to one of the best
songs on her previous great album, which is called ``The Idler Wheel,''
the I-D-L-E-R wheel. And the song that came to mind that is useful, both
in terms of how we've been talking about her today, and also in terms of
her adventurousness as a singer, is this song called ``Regret.'' It's
just structured really beautifully, and it sounds like she's built this
one kind of seemingly straightforward, beautiful song, and then
something happens.

{[}music - fiona apple, "regret"{]}

\begin{itemize}
\tightlist
\item
  fiona apple (singing)\\
  Oh, I ran out of white doves' feathers to soak up the hot piss that
  comes from your mouth every time you address me.
\end{itemize}

wesley morris

There's a kind of singing that I really love and is very hard to do. I'm
talking specifically about ugly singing, or unusual singing, right?
There aren't a lot of people who have the skill, and I would say even
more important, the bravery to do it, to try it. The recorded music
history is a lot of such people. I mean, you can go back to someone like
Louis Armstrong, or Screamin' Jay Hawkins, and Nina Simone, and Yoko
Ono, and Tom Waits, the patron saint of back of the throat, bottom of
the earth singing. People who are much more interested in the truth of
the sound of the voice than of the alleged beauty of the presentation.
It's just such a beautiful, beautiful song about real ugly feelings. And
I feel like that's the thing that I love her so much for is she isn't
afraid to make ugly feelings sound ugly. She doesn't run away from where
she is emotionally. I mean, this song is stripped down to almost
nothing. In that beautiful moment at the end, where she --- you hear the
piano top close.

jenna wortham

Yeah. (LAUGHING)

wesley morris

She's like, I'm done. I've said it all.

jenna wortham

Yeah. And you know Wesley, maybe some of the distance that has been
closed, or crossed, or worked out in the last eight years, I guess,
between this song coming out, ``Regret,'' on ``The Idler Wheel'' and
then ``Fetch the Bolt Cutters'' is that Fiona Apple has figured out at
least what she's going to do with all those raw nerve endings. She's
learned a little bit how to process them, how to sit with them, and how
to be OK. (LAUGHING) Right? And she's teaching us also right now how to
be OK with them. Because one of the differences between a song like
``Regret'' and then something like ``Heavy Balloon,'' that pain and
grief of a song like ``Regret'' has been transmuted. And it's not that
she's avoiding it, or she's not dealing with it, it's just that she has
metabolized it and she has channeled it into something else.

And that's why I think, even though ``Fetch the Bolt Cutters'' is still
so, so, so emotionally dense, and the frequency is so high, it can feel
peaceful, it can feel calm. Because she's worked out how she feels about
these things. And now she's telling us so that we can do the same.

{[}music{]}

\begin{itemize}
\item
  speaker 1\\
  Hey, Jenna and Wesley.
\item
  speaker 2\\
  Hi, Jenna, hi, Wesley.
\item
  speaker 4\\
  Hey, y'all.
\end{itemize}

{[}music{]}

jenna wortham

A few weeks ago, we asked you to tell us how you're taking care during
these times, and how people are taking care of you. And the responses
were overwhelming.

wesley morris

We heard from people all over the planet.

\begin{itemize}
\item
  mike\\
  This is Mike coming to you from Chicago.
\item
  speaker 5\\
  Cambridge, U.K.
\item
  speaker 6\\
  I live in Greenpoint in Brooklyn, New York.
\item
  speaker 7\\
  In Lisbon, Portugal's capital in mainland Europe.
\end{itemize}

wesley morris

And you guys are doing amazing work just taking care of yourselves.

\begin{itemize}
\item
  speaker 8\\
  Hi, Wesley and Jenna. I'm calling in with a story of how my husband
  has taken care of me during this difficult time. I had a birthday in
  the end of March, and had plans to celebrate, but obviously wasn't
  able to do that. I work for one of the major museums in the city,
  which is shut down. And so my husband surprised me by asking a bunch
  of our friends to create something. They emailed it to him, he printed
  it out, and he hung it up. He got actually 70 submissions of works of
  art, and he had so many submissions, there are two rooms in the house
  filled with art created by our loved ones. And he surprised me with my
  own museum with art that I could come and look at and take care of.
  And it not only took care of me, but it was a bright spot for a lot of
  people as well, where they stopped for a second, and sat down, and
  created something, or they had their kid make something. And so my
  husband was able to do something that took care of me, but also took
  care of a lot of our friends, even for a brief second. It was a very
  unusual 36th birthday.
\item
  claire grindinger\\
  OK. Hey, Jenna. Hey, Wesley. My name's Claire Grindinger. I'm a senior
  at Washington University in St. Louis, but right now I'm home in
  Dallas, Texas, cause of the coronavirus. I am right now running, as
  you might be able to tell. On March 29, I was supposed to run the St.
  Louis Go marathon, but I decided to run it virtually. And so my
  parents are coming around. They're following me on my route, bringing
  me a jacket, food, water and whatnot because there are no water
  stations, and just really helping me out. So I feel really supported
  and lucky to be here with them. So thank you.
\item
  inna kim\\
  Hi, this is Inna. I am in New Jersey. I moved from South Korea a
  couple of months ago, and this is not easy situation for all. But I
  have a routine. I exercise. I bake bunch. And I also have a baby, who
  is 10 months old, which makes me keep going and super busy. And for my
  family in South Korea, I try to contact them every day, sending a
  bunch of photos of baby, especially to my mom. And actually, I just
  sent her gift box, which has a bunch of lemons and gingers for her
  immune system. And I hope she's fine and well. And yeah, thank you.
\item
  speaker 9\\
  Hi, Jenna. Hi, Wesley. I'm currently studying in Cambridge, the U.K.,
  where I am in self-isolation, of course. My little sister does not
  live with me. So she's in London right now. But we have been having
  FaceTime calls where we have dance parties. And we have karaoke
  sessions where we do what we would normally do if we were living
  together, which is sing, scream a Disney song at the top of our lungs.
  But we're doing it virtually over the internet. It's really worked to
  distract me and just really make me feel silly and forget about the
  absurdity of what we're currently going through. So I hope you two are
  taking care, and I will see you on the internet.
\item
  mike pesoli\\
  Hey, Jenna. Hey, Wesley. This is Mike coming to you from Chicago. I am
  currently taking care of myself during this crisis of Covid-19 by
  throwing myself into literature. I know it might sound basic, but I
  find books to be a great way into someone else's psyche or into
  someone else's experience. And for the few lucky people in my life, I
  will even go as far as calling my local bookstore --- my favorite is
  Unabridged in Chicago --- and I will send them a book that I think
  they might really love. And then in that way, I'm taking care of a
  small business that really means something to me, I'm taking care of
  someone that I love, and I'm taking care of myself because it makes me
  feel good to give something to someone in this moment when so many of
  us can feel helpless. Cheers.
\item
  speaker 10\\
  Hi, Jenna and Wesley. Something that I've been doing with my
  community, and my loved ones, my family and friends, is just trying to
  be vulnerable from afar, really getting over that fear of admitting
  I'm not doing well right now. It's a huge relief. It's kind of like
  its own way of me taking care right now is just by that, letting go.
  And knowing that that doesn't mean that you can't handle things, but
  your ability to reach out to a friend and say, look, I can't handle
  everything I'm feeling right now. Are you available for me to share?
  And be well.
\end{itemize}

{[}music{]}

wesley morris

Thank you to every single person who submitted something. We really,
really appreciate it.

jenna wortham

And please keep taking care of yourself and your loved ones.

{[}music{]}

That's our show. Next week, we're going to be talking about all this
newfound closeness that even while living in a broken, fragmented world,
we're finding new ways to be together, intimate, close, online and
through our devices.

``Still Processing'' is a product of The New York Times. This week, like
all the past couple of weeks, it was recorded in our living rooms.

wesley morris

It's produced by Hans Beutow and Sydney Harper.

jenna wortham

Our editors are Sara Sarasohn, Sasha Weiss, Wendy Door, and Lisa Tobin.

wesley morris

And our engineer is Jake Gorski.

jenna wortham

Our theme music is by Kindness. It's called ``World Restart,'' from the
album ``Otherness.'' You can find all our old episodes and the show
notes for this one and those at nytimes.com/stillprocessing.

{[}music{]}

Previous

More episodes ofStill Processing

\href{https://www.nytimes.com/2020/07/23/podcasts/hamilton-ziwe-discomfort.html?action=click\&module=audio-series-bar\&region=header\&pgtype=Article}{\includegraphics{https://static01.nyt.com/images/2020/07/23/multimedia/23stillprocessing-pix/23stillprocessing-pix-thumbLarge.jpg}}

July 23, 2020~~•~ 38:10Ziwe May Destroy Hamilton

\href{https://www.nytimes.com/2020/07/16/podcasts/reparations-for-aunt-jemima.html?action=click\&module=audio-series-bar\&region=header\&pgtype=Article}{\includegraphics{https://static01.nyt.com/images/2020/07/18/multimedia/16stillprocessing-pix/16stillprocessing-pix-thumbLarge.jpg}}

July 16, 2020~~•~ 35:35Reparations for Aunt Jemima!

\href{https://www.nytimes.com/2020/07/09/podcasts/still-processing-black-lives-matter.html?action=click\&module=audio-series-bar\&region=header\&pgtype=Article}{\includegraphics{https://static01.nyt.com/images/2020/07/12/podcasts/09stillprocessing-image/xx-stillprocessing-thumbLarge.jpg}}

July 9, 2020~~•~ 26:29So Y'all Finally Get It

\href{https://www.nytimes.com/2020/05/14/podcasts/still-processing-westworld-hollywood-utopia-dystopia.html?action=click\&module=audio-series-bar\&region=header\&pgtype=Article}{\includegraphics{https://static01.nyt.com/images/2020/05/16/podcasts/14stillprocessing-image/14stillprocessing-image-thumbLarge-v2.jpg}}

May 14, 2020New Loop, America

\href{https://www.nytimes.com/2020/05/07/podcasts/still-processing-internet-vulnerability-sondheim-parks-recreation.html?action=click\&module=audio-series-bar\&region=header\&pgtype=Article}{\includegraphics{https://static01.nyt.com/images/2020/04/28/pageoneplus/28sondheimjp-sp/28sondheimjp-sp-thumbLarge-v4.jpg}}

May 7, 2020Does This Phone Make Me Look Human?

\href{https://www.nytimes.com/2020/04/30/podcasts/still-processing-fiona-apple-fetch-bolt-cutters.html?action=click\&module=audio-series-bar\&region=header\&pgtype=Article}{\includegraphics{https://static01.nyt.com/images/2020/05/03/multimedia/30stillpro-image/30stillpro-image-thumbLarge.jpg}}

May 1, 2020Fiona Ex Machina

\href{https://www.nytimes.com/2020/04/23/podcasts/still-processing-halle-berry-sharon-stone-catwoman-quarantine.html?action=click\&module=audio-series-bar\&region=header\&pgtype=Article}{\includegraphics{https://static01.nyt.com/images/2020/04/25/arts/23stillprocessing/23stillprocessing-thumbLarge-v3.jpg}}

April 23, 2020Halle Berry? Hallelujah.

\href{https://www.nytimes.com/2020/04/16/podcasts/still-processing-AIDS-survive-coronavirus.html?action=click\&module=audio-series-bar\&region=header\&pgtype=Article}{\includegraphics{https://static01.nyt.com/images/2020/04/20/us/16stillprocessing/16stillprocessing-thumbLarge-v3.jpg}}

April 16, 2020How to Learn From a Plague

\href{https://www.nytimes.com/2020/04/09/podcasts/still-processing-tiger-king.html?action=click\&module=audio-series-bar\&region=header\&pgtype=Article}{\includegraphics{https://static01.nyt.com/images/2020/04/11/podcasts/09stillprocessing-image2/09stillprocessing-image2-thumbLarge-v2.jpg}}

April 9, 2020~~•~ 39:49Frosted Flakes

\href{https://www.nytimes.com/2020/04/02/podcasts/high-fidelity-zoe-kravitz.html?action=click\&module=audio-series-bar\&region=header\&pgtype=Article}{\includegraphics{https://static01.nyt.com/images/2020/04/05/arts/02still-processing-highfidelity/13highfidelity-thumbLarge.jpg}}

April 2, 2020~~•~ 40:55Delicious Vinyl

\href{https://www.nytimes.com/2020/03/26/podcasts/still-processing-quarantine.html?action=click\&module=audio-series-bar\&region=header\&pgtype=Article}{\includegraphics{https://static01.nyt.com/images/2020/03/29/podcasts/26stillprocessing1/26stillprocessing1-thumbLarge.jpg}}

March 26, 2020~~•~ 30:47A Pod From Both Our Houses

\href{https://www.nytimes.com/2019/11/07/podcasts/still-processing-parasite-watchmen-bong-joon-ho.html?action=click\&module=audio-series-bar\&region=header\&pgtype=Article}{\includegraphics{https://static01.nyt.com/images/2019/11/08/arts/07stilpr-parasite/00parasite-1-thumbLarge.jpg}}

November 7, 2019Wake

\href{https://www.nytimes.com/column/still-processing-podcast}{See All
Episodes ofStill Processing}

Next

Published April 30, 2020Updated May 12, 2020

\begin{itemize}
\item
\item
\item
\item
\item
\end{itemize}

By \href{https://www.nytimes.com/by/wesley-morris}{Wesley Morris} and
\href{https://www.nytimes.com/by/jenna-wortham}{Jenna Wortham}

``Fetch the Bolt Cutters'' is Fiona Apple's master class in channeling
frustration and anger into what can only be called wisdom. Also, we hear
from listeners all over the planet, sharing how they are taking care of
the people in their lives.

Image

Fiona Apple performing in 2019 in Inglewood, Calif.Credit...Kevin
Mazur/Getty Images for The Chris Cornell Estate

Discussed this week:

\begin{itemize}
\item
  ``\href{https://www.youtube.com/watch?v=N541HLPeG6Y\&list=OLAK5uy_nvvTVXc476e1vHBGAN7Y-DH9_sZjOhgx8}{Fetch
  the Bolt Cutters}'' (Fiona Apple, 2020)
\item
  \href{https://www.youtube.com/watch?v=42gNkySFycA}{Fiona Apple on the
  VMAs in 1997}
\item
  ``\href{https://www.youtube.com/watch?v=9W9Un3rIzns}{Regret}'' (Fiona
  Apple, ``The Idler Wheel Is Wiser than the Driver of the Screw and
  Whipping Cords Will Serve You More than Ropes Will Ever Do,'' 2012)
\item
  ``\href{https://www.youtube.com/watch?v=FFOzayDpWoI}{Criminal}''
  (Fiona Apple, ``Tidal,'' 1996)
\item
  ``\href{https://www.youtube.com/watch?v=PeonBmeFR8o}{Hold Up}''
  (Beyoncé Knowles, ``Lemonade,'' 2016)
\end{itemize}

``Still Processing'' is produced by Hans Buetow and Sydney Harper and
edited by Sara Sarasohn and Sasha Weiss, with editorial oversight from
Wendy Dorr and Lisa Tobin. Our engineer is Jake Gorski. Our theme music
is by Kindness. It's called ``World Restart,'' from the album
``Otherness.''

Advertisement

\protect\hyperlink{after-bottom}{Continue reading the main story}

\hypertarget{site-index}{%
\subsection{Site Index}\label{site-index}}

\hypertarget{site-information-navigation}{%
\subsection{Site Information
Navigation}\label{site-information-navigation}}

\begin{itemize}
\tightlist
\item
  \href{https://help.nytimes.com/hc/en-us/articles/115014792127-Copyright-notice}{©~2020~The
  New York Times Company}
\end{itemize}

\begin{itemize}
\tightlist
\item
  \href{https://www.nytco.com/}{NYTCo}
\item
  \href{https://help.nytimes.com/hc/en-us/articles/115015385887-Contact-Us}{Contact
  Us}
\item
  \href{https://www.nytco.com/careers/}{Work with us}
\item
  \href{https://nytmediakit.com/}{Advertise}
\item
  \href{http://www.tbrandstudio.com/}{T Brand Studio}
\item
  \href{https://www.nytimes.com/privacy/cookie-policy\#how-do-i-manage-trackers}{Your
  Ad Choices}
\item
  \href{https://www.nytimes.com/privacy}{Privacy}
\item
  \href{https://help.nytimes.com/hc/en-us/articles/115014893428-Terms-of-service}{Terms
  of Service}
\item
  \href{https://help.nytimes.com/hc/en-us/articles/115014893968-Terms-of-sale}{Terms
  of Sale}
\item
  \href{https://spiderbites.nytimes.com}{Site Map}
\item
  \href{https://help.nytimes.com/hc/en-us}{Help}
\item
  \href{https://www.nytimes.com/subscription?campaignId=37WXW}{Subscriptions}
\end{itemize}
