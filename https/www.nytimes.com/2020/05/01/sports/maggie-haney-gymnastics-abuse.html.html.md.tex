Sections

SEARCH

\protect\hyperlink{site-content}{Skip to
content}\protect\hyperlink{site-index}{Skip to site index}

\href{https://www.nytimes.com/section/sports}{Sports}

\href{https://myaccount.nytimes.com/auth/login?response_type=cookie\&client_id=vi}{}

\href{https://www.nytimes.com/section/todayspaper}{Today's Paper}

\href{/section/sports}{Sports}\textbar{}Olympic Gymnast Recalls
Emotional Abuse `So Twisted That I Thought It Couldn't Be Real'

\url{https://nyti.ms/3d5tsi4}

\begin{itemize}
\item
\item
\item
\item
\item
\end{itemize}

Advertisement

\protect\hyperlink{after-top}{Continue reading the main story}

Supported by

\protect\hyperlink{after-sponsor}{Continue reading the main story}

\hypertarget{olympic-gymnast-recalls-emotional-abuse-so-twisted-that-i-thought-it-couldnt-be-real}{%
\section{Olympic Gymnast Recalls Emotional Abuse `So Twisted That I
Thought It Couldn't Be
Real'}\label{olympic-gymnast-recalls-emotional-abuse-so-twisted-that-i-thought-it-couldnt-be-real}}

Laurie Hernandez won medals at the 2016 Games. But she said her love of
the sport was tainted under training with Maggie Haney, who is now
suspended by the sport's American federation.

\includegraphics{https://static01.nyt.com/images/2020/05/02/sports/01gymnastics-haney1-print/merlin_107061637_65e759f3-3aa0-423b-86ff-ad9509d4d37f-articleLarge.jpg?quality=75\&auto=webp\&disable=upscale}

By \href{https://www.nytimes.com/by/juliet-macur}{Juliet Macur}

\begin{itemize}
\item
  May 1, 2020
\item
  \begin{itemize}
  \item
  \item
  \item
  \item
  \item
  \end{itemize}
\end{itemize}

For the longest time, Laurie Hernandez, an Olympic champion gymnast,
thought she was crazy for thinking that her coach had emotionally abused
her.

When that coach screamed at her for the tiniest of mistakes or lapses in
focus, calling her weak, lazy or messed up in the head, with an
obscenity for emphasis, Hernandez rationalized that all top coaches
pushed their young gymnasts that way. When she stood frozen and bawling
at the gym, her heart racing as she struggled to breathe, scared to do
anything that would elicit her coach's wrath, Hernandez --- an
adolescent at the time --- assumed she just couldn't handle the
pressure.

She considered it possible that many coaches publicly berated their
gymnasts for gaining weight and developing the curves that come with
puberty, and that they pushed athletes to train on dislocated knees and
broken wrists. She had become convinced that was how Olympians were
made.

When she complained about mistreatment, she recalled, her coach would
tell her to stop taking things so personally. And then Hernandez would
end up apologizing for causing trouble.

``I thought I deserved all of it,'' Hernandez, 19, said in an interview
with The New York Times on Thursday, the first time she has spoken
publicly about the abuse she says ignited eating disorders and
depression that she continues to battle.

``The toughest part about it was that there were no bruises or marks to
show that it was real,'' she said. ``It was all just so twisted that I
thought it couldn't be real.''

This week, Hernandez received a measure of validation that the mental
anguish she experienced under her former coach, Maggie Haney, was, in
fact, real. U.S.A. Gymnastics, after a weekslong hearing on accusations
that Haney had verbally and emotionally abused athletes,
\href{https://www.nytimes.com/2020/04/29/sports/gymnastics-coach-banned-maggie-haney.html}{suspended
her from coaching for eight years}.

A panel organized by the gymnastics federation evaluated evidence from
at least five of Haney's athletes, including Hernandez and
\href{https://www.ocregister.com/2020/02/14/riley-mccusker-now-training-in-arizona-after-maggie-haney-suspension/}{Riley
McCusker}, a strong candidate for the Tokyo Olympics next summer, and
found that Haney had engaged in ``severe aggressive behavior'' toward
gymnasts that included trying to control a minor by ridiculing her.

Haney, 42, did not respond to two requests for comment. Her lawyer,
Russell Prince, said that Haney planned to appeal the ruling and that
the hearing process was ``fundamentally flawed in a multitude of ways.''

The suspension was a significant step by U.S.A. Gymnastics to change the
sport's
\href{https://www.nytimes.com/2018/08/18/sports/gymnasts-coach-abuse.html}{overall
culture}, which many gymnasts criticized in the aftermath of the sexual
abuse scandal involving the former national team doctor, Lawrence G.
Nassar. He is now serving what amounts to a lifetime prison sentence for
molesting more than 200 girls and women.

\href{https://www.nytimes.com/2018/11/10/sports/-usa-gymnastics-scandal-.html}{Emotional
and verbal abuse is rampant in the sport}, many gymnasts said then, with
athletes so afraid of their coaches that they rarely speak out about any
mistreatment.

Hernandez did not tell her mother details about Haney's conduct until
2016, weeks after she competed in the Rio Olympics, and she did so then
only because her mother had overheard her in a FaceTime conversation
with a former teammate about Haney's pulling that gymnast's hair.

The two teenagers were laughing about it, and Wanda Hernandez, a social
worker, stopped them to say: ``I'm sorry, but what are you saying? No
one should be touching you or grabbing you.''

After her daughter, then 16, told her everything, Wanda Hernandez
immediately called Haney to terminate her services and sent a complaint
to U.S.A. Gymnastics. She said she eventually had to demand that Haney
stay away from Laurie because Haney kept making contact.

From there, Wanda Hernandez kept pressing for more oversight in the
sport. In 2018, she spoke with Morinari Watanabe, the president of the
international gymnastics federation, at a meet in Chicago and discussed
the need to teach gymnasts that they should feel safe speaking up about
abuse.

Still, it took U.S.A. Gymnastics four years to penalize Haney, and the
national federation acknowledged in a statement on Friday that the
``investigation and resolution process must be faster in the future.''

The Hernandez family considered the outcome a move forward for the
sport.

``I thought they were just going to try to sweep it under the rug,''
Laurie Hernandez said of U.S.A. Gymnastics. ``But, wow, they did the
right thing. I can't believe they actually did the right thing.''

Hernandez, who is from Old Bridge Township, N.J., was 5 when she began
training with MG Elite, Haney's team in central New Jersey. ****
Hernandez remembers being bubbly and outgoing back then, but by the time
she was a teenager she found herself crying some mornings before getting
out of bed as she anticipated seeing Haney at the gym.

She said Haney sometimes screamed at her so loudly that people could
hear it from the parking lot.

The gym was Haney's dominion. The team handbook says in underlined red
ink: ``Parents are NOT allowed to stay and watch practice,'' and ``There
are NO EXCEPTIONS!''

Hernandez recalls happy times with Haney, like team banquets and
sleepovers at her house or trips to get ice cream, but at the gym
Hernandez invariably felt stuck. She wanted to leave, she said, but
couldn't speak up because she was just a child, unable to walk out or
drive home. Feeling helpless, she would often cry, angering Haney even
more.

Early on, Hernandez sought help from her mother, who contacted the
coach. The next day, Haney showed up at the gym in a rage, telling
Laurie Hernandez, ``So, your mother called me.'' Then, Hernandez said,
Haney punished the entire team by assigning extra conditioning work.

Eventually, Hernandez stopped telling her family about anything that
happened at the gym, fending for herself instead. Puberty was an
especially rough time.

``Maggie was saying to me, `You already have a boxy body type, so we're
going to have to keep an eye on your weight,' or shout, `You're so
busty!' for everyone to hear,'' said Hernandez, who started wearing two
sports bras to flatten out her chest.

Struggling with her weight and pubescent body changes, Hernandez ended
up binge-eating and purging. When she lost weight, Haney would praise
her and Hernandez felt encouraged that the vomiting had worked.

``Any compliment was like holy water,'' Hernandez said. ``It went from
one day walking on eggshells with her to her saying the next day that
`we're in this together.' She really knew how to mess with your head.''

\includegraphics{https://static01.nyt.com/images/2020/05/02/sports/01gymnastics-haney2-print/01gymnastics-haney-2-articleLarge.jpg?quality=75\&auto=webp\&disable=upscale}

Other teammates, she said, were similarly trying to cope. One, she said,
was so scared of Haney that she would throw up before every practice.

The coach, she said, would also talk badly about Wanda Hernandez at the
gym, saying things like ``you should trust nobody because your parents
are just going to kick you to the curb.''

``Maggie would really trash-talk her, and it hurt a lot at first,''
Hernandez said. ``But then I'd say, `No, Mom, please don't complain to
Maggie because you're only going to make her mad. She's doing the best
she can.'''

Hernandez said she learned how to block out the abuse.

``As soon as Maggie raised her voice past a certain point, that space in
my brain would pull the parachute and I couldn't hear her,'' she said.
``As soon as practice was over, it was like my brain was a computer and
I wiped out everything.''

That's how Hernandez remembers making it through a practice after she
fractured a growth plate in her wrist when she fell from the balance
beam. She recalls Haney accusing her of being dramatic when she
complained about the pain, much as the coach had reacted when Hernandez
dislocated her knee at a national team training camp. The knee injury
required surgery.

Hernandez competed at the Rio Olympics with a torn abdominal muscle,
helping the United States claim
\href{https://www.nytimes.com/interactive/2016/08/09/sports/olympics-womens-gymnastics-team-all-around.html}{the
team gold medal} and winning a silver medal on the balance beam.

After the Olympics, Hernandez parlayed her success into stardom outside
the sport. She competed on
\href{https://www.nj.com/entertainment/celebrities/2016/11/laurie_hernandez_dancing_with_the_stars_ranked.html}{``Dancing
With the Stars''} and won, in just one of her many television
appearances. But in spite of the medals, the sponsorships and the two
books she wrote, Hernandez regrets training with Haney.

``I'm grateful that I got to the Olympics, but at what cost?'' she said.

Hernandez said that she had struggled with ``full-on major depression''
and that she now receives treatment for it. She coped with her sadness
by overeating and still has ``a ginormous fear of doing something wrong,
perfectionism to the extreme.''

In 2018, Hernandez moved to California to start anew in her sport after
taking time away to enjoy her Olympic success and heal emotionally.

Now at Gym-Max Gymnastics in Costa Mesa, Calif., Hernandez trains under
Jenny Zhang and Howie Liang. It feels like heaven, she said.

It took a while to adjust, but when she makes a mistake now, she doesn't
brace for a coach to yell at her. With the change, she knew that she
really did love gymnastics.

``Whether I make it to Tokyo or not, I'm doing something I love on my
own terms,'' Hernandez said, ``and people treat me the way I want to be
treated, and that makes me happy.'' She relaxes by singing or playing a
guitar or ukulele, and she leans on her toy golden doodle, Honey, a
registered emotional support dog. Hernandez bought him as soon as she
found out she would be testifying against Haney.

After testifying for 80 minutes at the U.S.A. Gymnastics hearing in
February, Hernandez left feeling uneasy, wondering if she had ``just
thrown Maggie under the bus.'' She thought about apologizing and asked
herself, yet again, if she had deserved Haney's brutal treatment.

But several other gymnasts had testified about the abuse. She wasn't
alone, and that comforted her.

At least for one moment, Hernandez felt that a weight had been lifted.

``It was really nice to finally tell people what happened,'' she said.
``I did it for little Laurie and all the little kids out there.''

Danielle Allentuck contributed reporting.

Advertisement

\protect\hyperlink{after-bottom}{Continue reading the main story}

\hypertarget{site-index}{%
\subsection{Site Index}\label{site-index}}

\hypertarget{site-information-navigation}{%
\subsection{Site Information
Navigation}\label{site-information-navigation}}

\begin{itemize}
\tightlist
\item
  \href{https://help.nytimes.com/hc/en-us/articles/115014792127-Copyright-notice}{©~2020~The
  New York Times Company}
\end{itemize}

\begin{itemize}
\tightlist
\item
  \href{https://www.nytco.com/}{NYTCo}
\item
  \href{https://help.nytimes.com/hc/en-us/articles/115015385887-Contact-Us}{Contact
  Us}
\item
  \href{https://www.nytco.com/careers/}{Work with us}
\item
  \href{https://nytmediakit.com/}{Advertise}
\item
  \href{http://www.tbrandstudio.com/}{T Brand Studio}
\item
  \href{https://www.nytimes.com/privacy/cookie-policy\#how-do-i-manage-trackers}{Your
  Ad Choices}
\item
  \href{https://www.nytimes.com/privacy}{Privacy}
\item
  \href{https://help.nytimes.com/hc/en-us/articles/115014893428-Terms-of-service}{Terms
  of Service}
\item
  \href{https://help.nytimes.com/hc/en-us/articles/115014893968-Terms-of-sale}{Terms
  of Sale}
\item
  \href{https://spiderbites.nytimes.com}{Site Map}
\item
  \href{https://help.nytimes.com/hc/en-us}{Help}
\item
  \href{https://www.nytimes.com/subscription?campaignId=37WXW}{Subscriptions}
\end{itemize}
