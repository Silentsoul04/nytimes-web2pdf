Sections

SEARCH

\protect\hyperlink{site-content}{Skip to
content}\protect\hyperlink{site-index}{Skip to site index}

\href{https://www.nytimes.com/section/travel}{Travel}

\href{https://myaccount.nytimes.com/auth/login?response_type=cookie\&client_id=vi}{}

\href{https://www.nytimes.com/section/todayspaper}{Today's Paper}

\href{/section/travel}{Travel}\textbar{}Help! My Flight Was Canceled and
I Still Can't Get a Refund

\url{https://nyti.ms/2VU3A2I}

\begin{itemize}
\item
\item
\item
\item
\item
\item
\end{itemize}

\href{https://www.nytimes.com/news-event/coronavirus?action=click\&pgtype=Article\&state=default\&region=TOP_BANNER\&context=storylines_menu}{The
Coronavirus Outbreak}

\begin{itemize}
\tightlist
\item
  live\href{https://www.nytimes.com/2020/08/04/world/coronavirus-cases.html?action=click\&pgtype=Article\&state=default\&region=TOP_BANNER\&context=storylines_menu}{Latest
  Updates}
\item
  \href{https://www.nytimes.com/interactive/2020/us/coronavirus-us-cases.html?action=click\&pgtype=Article\&state=default\&region=TOP_BANNER\&context=storylines_menu}{Maps
  and Cases}
\item
  \href{https://www.nytimes.com/interactive/2020/science/coronavirus-vaccine-tracker.html?action=click\&pgtype=Article\&state=default\&region=TOP_BANNER\&context=storylines_menu}{Vaccine
  Tracker}
\item
  \href{https://www.nytimes.com/2020/08/02/us/covid-college-reopening.html?action=click\&pgtype=Article\&state=default\&region=TOP_BANNER\&context=storylines_menu}{College
  Reopening}
\item
  \href{https://www.nytimes.com/live/2020/08/04/business/stock-market-today-coronavirus?action=click\&pgtype=Article\&state=default\&region=TOP_BANNER\&context=storylines_menu}{Economy}
\end{itemize}

Advertisement

\protect\hyperlink{after-top}{Continue reading the main story}

Supported by

\protect\hyperlink{after-sponsor}{Continue reading the main story}

Tripped Up

\hypertarget{help-my-flight-was-canceled-and-i-still-cant-get-a-refund}{%
\section{Help! My Flight Was Canceled and I Still Can't Get a
Refund}\label{help-my-flight-was-canceled-and-i-still-cant-get-a-refund}}

There's a reason we are deliberately pounding the topic of travel
refunds into the ground: It's a big deal and there's lots of confusion
and contradictory information out there.

\includegraphics{https://static01.nyt.com/images/2020/05/03/travel/03TrippedUp-airline-refunds/03TrippedUp-airline-refunds-articleLarge.jpg?quality=75\&auto=webp\&disable=upscale}

By Sarah Firshein

\begin{itemize}
\item
  May 1, 2020
\item
  \begin{itemize}
  \item
  \item
  \item
  \item
  \item
  \item
  \end{itemize}
\end{itemize}

\emph{Travel and travel planning are being disrupted by the worldwide
spread of the coronavirus. For the latest updates,
read}\href{https://www.nytimes.com/news-event/coronavirus?action=click\&module=RelatedLinks\&pgtype=Article}{\emph{The
New York Times's Covid-19 coverage here}}\emph{.}

\hypertarget{dear-tripped-up}{%
\subsubsection{\texorpdfstring{\textbf{Dear Tripped
Up,}}{Dear Tripped Up,}}\label{dear-tripped-up}}

In September 2019 B.C. --- ``before coronavirus'' --- my family and I
booked a round-trip flight from Tampa, Fla., to London with British
Airways, scheduled to depart in March. Then the coronavirus hit and our
flight was canceled by the airline. We contacted British Airways to
request a refund, but they have refused, stating they will only provide
a voucher for future use. Philip

\hypertarget{dear-philip}{%
\subsubsection{Dear Philip,}\label{dear-philip}}

By now, you've probably read a number of coronavirus-related stories
about refunds for canceled travel plans --- I've written a handful
myself. But there's a reason we're deliberately pounding the topic into
the ground: It's a big deal and there's lots of confusion and
contradictory information out there.

To begin, airlines are required to issue cash refunds when they cancel
(or significantly delay) flights because of the
\href{https://www.nytimes.com/2020/04/22/us/coronavirus-updates.html}{coronavirus
pandemic.} That applies to domestic airlines as well as international
airlines for canceled flights to, within or from the United States.

As you know,
\href{https://www.nytimes.com/article/coronavirus-travel-restrictions.html}{travel
restrictions and closed borders} have asphyxiated commercial aviation;
airlines have slashed
\href{https://www.nytimes.com/2020/03/16/travel/international-airlines-service-cuts-coronavirus.html}{routes
and schedules.} As of last month, airlines
\href{https://www.iata.org/en/pressroom/ceoblog/passenger-tickets-refunds/}{still
owed passengers \$35 billion} for flights that could not or cannot take
place, according to the International Air Transport Association, an
industry group.

In an emailed statement, a British Airways spokeswoman said that you
are, in fact, entitled to a refund; the airline's customer service team
has reached out to you to resolve the issue. ``If a customer's flight
has been canceled, they should call us to discuss their options. They
can rebook, refund or choose to take a voucher to fly at a later date,''
she said.

The fact that you called and were initially told the exact opposite
tracks with scores of other readers who feel that refunds have been
turned into an all-out game of Frogger.

\hypertarget{latest-updates-global-coronavirus-outbreak}{%
\section{\texorpdfstring{\href{https://www.nytimes.com/2020/08/04/world/coronavirus-cases.html?action=click\&pgtype=Article\&state=default\&region=MAIN_CONTENT_1\&context=storylines_live_updates}{Latest
Updates: Global Coronavirus
Outbreak}}{Latest Updates: Global Coronavirus Outbreak}}\label{latest-updates-global-coronavirus-outbreak}}

Updated 2020-08-04T20:08:28.255Z

\begin{itemize}
\tightlist
\item
  \href{https://www.nytimes.com/2020/08/04/world/coronavirus-cases.html?action=click\&pgtype=Article\&state=default\&region=MAIN_CONTENT_1\&context=storylines_live_updates\#link-1228a480}{Novavax
  sees encouraging results from two studies of its experimental
  vaccine.}
\item
  \href{https://www.nytimes.com/2020/08/04/world/coronavirus-cases.html?action=click\&pgtype=Article\&state=default\&region=MAIN_CONTENT_1\&context=storylines_live_updates\#link-4825b93}{Public
  and private schools in Maryland and elsewhere are divided over
  in-person instruction.}
\item
  \href{https://www.nytimes.com/2020/08/04/world/coronavirus-cases.html?action=click\&pgtype=Article\&state=default\&region=MAIN_CONTENT_1\&context=storylines_live_updates\#link-4d1eafa8}{N.Y.C.'s
  health commissioner resigns after clashing with the mayor over the
  virus.}
\end{itemize}

\href{https://www.nytimes.com/2020/08/04/world/coronavirus-cases.html?action=click\&pgtype=Article\&state=default\&region=MAIN_CONTENT_1\&context=storylines_live_updates}{See
more updates}

More live coverage:
\href{https://www.nytimes.com/live/2020/08/04/business/stock-market-today-coronavirus?action=click\&pgtype=Article\&state=default\&region=MAIN_CONTENT_1\&context=storylines_live_updates}{Markets}

``At best, airlines are just hiding the fact that customers are owed
refunds if they want one,'' said Scott Keyes, an aviation industry
expert and the founder of the website
\href{https://scottscheapflights.com/}{Scott's Cheap Flights}. ``At
worst, they're actively not giving those refunds even when they're
asked. The thinking is: Right now we are in such a cash crunch that we
want to maintain as much cash as possible --- we can deal with the
reputation ramifications and legal ramifications down the road.''

Micky, another reader, got an email from Delta Air Lines with the
subject line: ``Your Trip Has Been Canceled.'' The body of the message
stated that a canceled plane ticket had been turned into an eCredit;
there were zero mentions of refunds.

I reached out to Delta (which just posted its first quarterly loss in
five years) and was able to get Micky a refund. I also got the direct
link to the refund form
(\href{https://www.delta.com/refunds/refundsHomeAction.action}{here it
is}), which is otherwise buried in the FAQs on Delta's
\href{https://www.delta.com/us/en/travel-update-center/overview}{coronavirus
landing page.}

``To save customers time, we proactively issued eCredits so they could
easily reschedule without having to wait further. That said, customers
whose flights have been canceled or significantly delayed by Delta are
absolutely eligible for refunds upon request, in keeping with our
longstanding policy. We'll evaluate opportunities to clarify this in
future correspondence,'' said a company spokeswoman.

Last month, fueled by a surge in passenger complaints, the United States
Department of Transportation issued a statement reminding airlines of
their duty to issue refunds for canceled flights. But enforcement has
been less than straightforward.

Class-action lawsuits against several major airlines, including Delta,
Southwest Airlines and United Airlines, are certain to crank up the
pressure. In the meantime, there are three concrete things one can try
that don't involve hiring a lawyer.

First, send the airline a direct message on social media. It may sound
sophomoric, but several industry players have told me (anecdotally) that
this can be a consistently reliable tactic. One reader also shared her
luck with this approach.

\href{https://www.nytimes.com/news-event/coronavirus?action=click\&pgtype=Article\&state=default\&region=MAIN_CONTENT_3\&context=storylines_faq}{}

\hypertarget{the-coronavirus-outbreak-}{%
\subsubsection{The Coronavirus Outbreak
›}\label{the-coronavirus-outbreak-}}

\hypertarget{frequently-asked-questions}{%
\paragraph{Frequently Asked
Questions}\label{frequently-asked-questions}}

Updated August 4, 2020

\begin{itemize}
\item ~
  \hypertarget{i-have-antibodies-am-i-now-immune}{%
  \paragraph{I have antibodies. Am I now
  immune?}\label{i-have-antibodies-am-i-now-immune}}

  \begin{itemize}
  \tightlist
  \item
    As of right
    now,\href{https://www.nytimes.com/2020/07/22/health/covid-antibodies-herd-immunity.html?action=click\&pgtype=Article\&state=default\&region=MAIN_CONTENT_3\&context=storylines_faq}{that
    seems likely, for at least several months.} There have been
    frightening accounts of people suffering what seems to be a second
    bout of Covid-19. But experts say these patients may have a
    drawn-out course of infection, with the virus taking a slow toll
    weeks to months after initial exposure. People infected with the
    coronavirus typically
    \href{https://www.nature.com/articles/s41586-020-2456-9}{produce}
    immune molecules called antibodies, which are
    \href{https://www.nytimes.com/2020/05/07/health/coronavirus-antibody-prevalence.html?action=click\&pgtype=Article\&state=default\&region=MAIN_CONTENT_3\&context=storylines_faq}{protective
    proteins made in response to an
    infection}\href{https://www.nytimes.com/2020/05/07/health/coronavirus-antibody-prevalence.html?action=click\&pgtype=Article\&state=default\&region=MAIN_CONTENT_3\&context=storylines_faq}{.
    These antibodies may} last in the body
    \href{https://www.nature.com/articles/s41591-020-0965-6}{only two to
    three months}, which may seem worrisome, but that's perfectly normal
    after an acute infection subsides, said Dr. Michael Mina, an
    immunologist at Harvard University. It may be possible to get the
    coronavirus again, but it's highly unlikely that it would be
    possible in a short window of time from initial infection or make
    people sicker the second time.
  \end{itemize}
\item ~
  \hypertarget{im-a-small-business-owner-can-i-get-relief}{%
  \paragraph{I'm a small-business owner. Can I get
  relief?}\label{im-a-small-business-owner-can-i-get-relief}}

  \begin{itemize}
  \tightlist
  \item
    The
    \href{https://www.nytimes.com/article/small-business-loans-stimulus-grants-freelancers-coronavirus.html?action=click\&pgtype=Article\&state=default\&region=MAIN_CONTENT_3\&context=storylines_faq}{stimulus
    bills enacted in March} offer help for the millions of American
    small businesses. Those eligible for aid are businesses and
    nonprofit organizations with fewer than 500 workers, including sole
    proprietorships, independent contractors and freelancers. Some
    larger companies in some industries are also eligible. The help
    being offered, which is being managed by the Small Business
    Administration, includes the Paycheck Protection Program and the
    Economic Injury Disaster Loan program. But lots of folks have
    \href{https://www.nytimes.com/interactive/2020/05/07/business/small-business-loans-coronavirus.html?action=click\&pgtype=Article\&state=default\&region=MAIN_CONTENT_3\&context=storylines_faq}{not
    yet seen payouts.} Even those who have received help are confused:
    The rules are draconian, and some are stuck sitting on
    \href{https://www.nytimes.com/2020/05/02/business/economy/loans-coronavirus-small-business.html?action=click\&pgtype=Article\&state=default\&region=MAIN_CONTENT_3\&context=storylines_faq}{money
    they don't know how to use.} Many small-business owners are getting
    less than they expected or
    \href{https://www.nytimes.com/2020/06/10/business/Small-business-loans-ppp.html?action=click\&pgtype=Article\&state=default\&region=MAIN_CONTENT_3\&context=storylines_faq}{not
    hearing anything at all.}
  \end{itemize}
\item ~
  \hypertarget{what-are-my-rights-if-i-am-worried-about-going-back-to-work}{%
  \paragraph{What are my rights if I am worried about going back to
  work?}\label{what-are-my-rights-if-i-am-worried-about-going-back-to-work}}

  \begin{itemize}
  \tightlist
  \item
    Employers have to provide
    \href{https://www.osha.gov/SLTC/covid-19/standards.html}{a safe
    workplace} with policies that protect everyone equally.
    \href{https://www.nytimes.com/article/coronavirus-money-unemployment.html?action=click\&pgtype=Article\&state=default\&region=MAIN_CONTENT_3\&context=storylines_faq}{And
    if one of your co-workers tests positive for the coronavirus, the
    C.D.C.} has said that
    \href{https://www.cdc.gov/coronavirus/2019-ncov/community/guidance-business-response.html}{employers
    should tell their employees} -\/- without giving you the sick
    employee's name -\/- that they may have been exposed to the virus.
  \end{itemize}
\item ~
  \hypertarget{should-i-refinance-my-mortgage}{%
  \paragraph{Should I refinance my
  mortgage?}\label{should-i-refinance-my-mortgage}}

  \begin{itemize}
  \tightlist
  \item
    \href{https://www.nytimes.com/article/coronavirus-money-unemployment.html?action=click\&pgtype=Article\&state=default\&region=MAIN_CONTENT_3\&context=storylines_faq}{It
    could be a good idea,} because mortgage rates have
    \href{https://www.nytimes.com/2020/07/16/business/mortgage-rates-below-3-percent.html?action=click\&pgtype=Article\&state=default\&region=MAIN_CONTENT_3\&context=storylines_faq}{never
    been lower.} Refinancing requests have pushed mortgage applications
    to some of the highest levels since 2008, so be prepared to get in
    line. But defaults are also up, so if you're thinking about buying a
    home, be aware that some lenders have tightened their standards.
  \end{itemize}
\item ~
  \hypertarget{what-is-school-going-to-look-like-in-september}{%
  \paragraph{What is school going to look like in
  September?}\label{what-is-school-going-to-look-like-in-september}}

  \begin{itemize}
  \tightlist
  \item
    It is unlikely that many schools will return to a normal schedule
    this fall, requiring the grind of
    \href{https://www.nytimes.com/2020/06/05/us/coronavirus-education-lost-learning.html?action=click\&pgtype=Article\&state=default\&region=MAIN_CONTENT_3\&context=storylines_faq}{online
    learning},
    \href{https://www.nytimes.com/2020/05/29/us/coronavirus-child-care-centers.html?action=click\&pgtype=Article\&state=default\&region=MAIN_CONTENT_3\&context=storylines_faq}{makeshift
    child care} and
    \href{https://www.nytimes.com/2020/06/03/business/economy/coronavirus-working-women.html?action=click\&pgtype=Article\&state=default\&region=MAIN_CONTENT_3\&context=storylines_faq}{stunted
    workdays} to continue. California's two largest public school
    districts --- Los Angeles and San Diego --- said on July 13, that
    \href{https://www.nytimes.com/2020/07/13/us/lausd-san-diego-school-reopening.html?action=click\&pgtype=Article\&state=default\&region=MAIN_CONTENT_3\&context=storylines_faq}{instruction
    will be remote-only in the fall}, citing concerns that surging
    coronavirus infections in their areas pose too dire a risk for
    students and teachers. Together, the two districts enroll some
    825,000 students. They are the largest in the country so far to
    abandon plans for even a partial physical return to classrooms when
    they reopen in August. For other districts, the solution won't be an
    all-or-nothing approach.
    \href{https://bioethics.jhu.edu/research-and-outreach/projects/eschool-initiative/school-policy-tracker/}{Many
    systems}, including the nation's largest, New York City, are
    devising
    \href{https://www.nytimes.com/2020/06/26/us/coronavirus-schools-reopen-fall.html?action=click\&pgtype=Article\&state=default\&region=MAIN_CONTENT_3\&context=storylines_faq}{hybrid
    plans} that involve spending some days in classrooms and other days
    online. There's no national policy on this yet, so check with your
    municipal school system regularly to see what is happening in your
    community.
  \end{itemize}
\end{itemize}

``Delta did not notify us, but I saw online that the flights were
canceled,'' Linda wrote. ``No emails. No phone calls. Nothing got a
refund confirmation. Well, until I sent a private instant message
through Facebook. Almost immediate reply. Asked for our names and
confirmation number. Bingo. Refund confirmation received.''

The second course of action is to
\href{https://airconsumer.dot.gov/escomplaint/ConsumerForm.cfm}{file a
complaint} with the transportation department. The federal agency tracks
complaints by airline and releases them monthly; March complaints will
be released in May and April complaints will be released in June. That
data will be used by the department's Aviation Enforcement Office to
monitor airlines' compliance with the refund policy, according to an
agency spokeswoman.

Last tip: If you do manage to get an airline representative on the
phone, do everything you can to get yourself passed around to another
agent if the initial point of contact is unwilling to budge. That may
mean asking to speak with a manager or asking to be transferred to a
different department (then asking to be transferred back). Be patient;
you'll need a good chunk of time in order to let your options play out.

Or, as Mr. Keyes recently
\href{https://twitter.com/scottsflights/status/1253023249744588800}{put
it on Twitter:} ``When a parent says `no' what's the \#1 thing every kid
knows to do? Go ask the other parent.''

\begin{center}\rule{0.5\linewidth}{\linethickness}\end{center}

\href{https://twitter.com/sfirshein?lang=en}{Sarah Firshein} is a
Brooklyn-based writer. If you need advice about a best-laid travel plan
that went awry, \textbf{\href{mailto:travel@nytimes.com}{send an email
to travel@nytimes.com}.}

\begin{center}\rule{0.5\linewidth}{\linethickness}\end{center}

\textbf{Follow New York Times Travel} ******
\emph{on}\href{https://www.instagram.com/nytimestravel/}{\emph{Instagram}}\emph{,}\href{https://twitter.com/nytimestravel}{\emph{Twitter}}
\emph{and}\href{https://www.facebook.com/nytimestravel/}{\emph{Facebook}}\emph{.
And}\href{https://www.nytimes.com/newsletters/traveldispatch}{\emph{sign
up for our weekly Travel Dispatch newsletter}} \emph{to receive expert
tips on traveling smarter and inspiration for your next vacation.
Dreaming up a future getaway or just armchair traveling? Check out
our}\href{https://www.nytimes.com/interactive/2020/travel/places-to-visit.html}{\emph{52
Places list}}\emph{.}

Advertisement

\protect\hyperlink{after-bottom}{Continue reading the main story}

\hypertarget{site-index}{%
\subsection{Site Index}\label{site-index}}

\hypertarget{site-information-navigation}{%
\subsection{Site Information
Navigation}\label{site-information-navigation}}

\begin{itemize}
\tightlist
\item
  \href{https://help.nytimes.com/hc/en-us/articles/115014792127-Copyright-notice}{©~2020~The
  New York Times Company}
\end{itemize}

\begin{itemize}
\tightlist
\item
  \href{https://www.nytco.com/}{NYTCo}
\item
  \href{https://help.nytimes.com/hc/en-us/articles/115015385887-Contact-Us}{Contact
  Us}
\item
  \href{https://www.nytco.com/careers/}{Work with us}
\item
  \href{https://nytmediakit.com/}{Advertise}
\item
  \href{http://www.tbrandstudio.com/}{T Brand Studio}
\item
  \href{https://www.nytimes.com/privacy/cookie-policy\#how-do-i-manage-trackers}{Your
  Ad Choices}
\item
  \href{https://www.nytimes.com/privacy}{Privacy}
\item
  \href{https://help.nytimes.com/hc/en-us/articles/115014893428-Terms-of-service}{Terms
  of Service}
\item
  \href{https://help.nytimes.com/hc/en-us/articles/115014893968-Terms-of-sale}{Terms
  of Sale}
\item
  \href{https://spiderbites.nytimes.com}{Site Map}
\item
  \href{https://help.nytimes.com/hc/en-us}{Help}
\item
  \href{https://www.nytimes.com/subscription?campaignId=37WXW}{Subscriptions}
\end{itemize}
