Sections

SEARCH

\protect\hyperlink{site-content}{Skip to
content}\protect\hyperlink{site-index}{Skip to site index}

\href{/section/world/asia}{Asia Pacific}\textbar{}In a City Frozen in
Fear, Time Freezes, Too

\url{https://nyti.ms/2yYTlkP}

\begin{itemize}
\item
\item
\item
\item
\item
\end{itemize}

\href{https://www.nytimes.com/news-event/coronavirus?action=click\&pgtype=Article\&state=default\&region=TOP_BANNER\&context=storylines_menu}{The
Coronavirus Outbreak}

\begin{itemize}
\tightlist
\item
  live\href{https://www.nytimes.com/2020/08/02/world/coronavirus-updates.html?action=click\&pgtype=Article\&state=default\&region=TOP_BANNER\&context=storylines_menu}{Latest
  Updates}
\item
  \href{https://www.nytimes.com/interactive/2020/us/coronavirus-us-cases.html?action=click\&pgtype=Article\&state=default\&region=TOP_BANNER\&context=storylines_menu}{Maps
  and Cases}
\item
  \href{https://www.nytimes.com/interactive/2020/science/coronavirus-vaccine-tracker.html?action=click\&pgtype=Article\&state=default\&region=TOP_BANNER\&context=storylines_menu}{Vaccine
  Tracker}
\item
  \href{https://www.nytimes.com/interactive/2020/07/29/us/schools-reopening-coronavirus.html?action=click\&pgtype=Article\&state=default\&region=TOP_BANNER\&context=storylines_menu}{What
  School May Look Like}
\item
  \href{https://www.nytimes.com/live/2020/07/31/business/stock-market-today-coronavirus?action=click\&pgtype=Article\&state=default\&region=TOP_BANNER\&context=storylines_menu}{Economy}
\end{itemize}

\includegraphics{https://static01.nyt.com/images/2020/05/01/world/01virus-india-dispatch-1/01virus-india-dispatch-1-articleLarge-v3.jpg?quality=75\&auto=webp\&disable=upscale}

new DELHI DISPATCH

\hypertarget{in-a-city-frozen-in-fear-time-freezes-too}{%
\section{In a City Frozen in Fear, Time Freezes,
Too}\label{in-a-city-frozen-in-fear-time-freezes-too}}

Unusually clear skies, sunny days and empty parks beckon, but in a city
locked down by the coronavirus, almost no one is heeding their call.

People talked to market stall owners from their home near Jama Masjid in
New Delhi, last week.Credit...

Supported by

\protect\hyperlink{after-sponsor}{Continue reading the main story}

By \href{https://www.nytimes.com/by/jeffrey-gettleman}{Jeffrey
Gettleman}

Photographs by Rebecca Conway

\begin{itemize}
\item
  Published May 1, 2020Updated May 5, 2020
\item
  \begin{itemize}
  \item
  \item
  \item
  \item
  \item
  \end{itemize}
\end{itemize}

NEW DELHI --- The first thing that disappeared was the annoying sound of
a power drill up the street, from a house under construction.

Then the newspapers.

Then the fruit sellers, the taxis, the rickshaws, and chicken.

Day by day,
\href{https://www.nytimes.com/2020/03/25/world/asia/india-lockdown-coronavirus.html}{life
under coronavirus lockdown in India} took away something else, usually
something good. And nearly six weeks into it, much of this country is
still frozen.

In many cities like New Delhi, practically nothing is moving on the
roads. People stay indoors, as instructed, emerging only to collect the
basic necessities. One friend who gets his food delivered told me he
hasn't left his house for a month.

All the airlines are grounded. Schools and offices are closed. The only
businesses that I've seen operating are food shops, pharmacies and
banks. The banks have lines running out the door and down the sidewalk
where red circles have been spray-painted for people to stand in, six
feet apart, like little islands.

The other day, I drove to Delhi's outskirts. India is a place rightly
known for teeming crowds and riotous traffic. There seems to be a
national aversion to sticking to your lane, so I felt almost guilty
blazing down an empty highway, past miles of shuttered shops, with no
one to cut me off.

\includegraphics{https://static01.nyt.com/images/2020/05/01/world/01virus-delhi-dispatch-3/merlin_172069254_84c99e32-c5cc-4187-ab0a-3006f42e3620-articleLarge.jpg?quality=75\&auto=webp\&disable=upscale}

Image

Sunil Kumar Jain served a customer at his pharmacy in the containment
zone market on Thursday.

Image

Lining up outside a bank in New Delhi on Thursday.

Whenever we turned off the highway, every village, no matter how small,
was barricaded --- some with oil drums, others with rope. Behind the
barricades stood villagers carrying sticks to keep strangers out and
wearing frayed bandannas over their faces, the virus vigilantes.

Even the sky above us is different these days. New Delhi is usually one
of the world's most polluted cities; its ceiling is invariably smudge
gray. But now with so few cars and factories running, the air here is
\href{https://www.nytimes.com/2020/04/08/world/asia/india-pollution-coronavirus.html}{cleaner
than it has been in decades}.

The weather that first weekend under lockdown, in late March, was
especially lovely: mid-80s, breezy, clear skies. So on the following
Monday when I saw The Times's driver, Jag Singh, one of the few Indians
I now see on a regular basis because of our isolation, I asked if he had
managed to get outside.

\hypertarget{latest-updates-global-coronavirus-outbreak}{%
\section{\texorpdfstring{\href{https://www.nytimes.com/2020/08/01/world/coronavirus-covid-19.html?action=click\&pgtype=Article\&state=default\&region=MAIN_CONTENT_1\&context=storylines_live_updates}{Latest
Updates: Global Coronavirus
Outbreak}}{Latest Updates: Global Coronavirus Outbreak}}\label{latest-updates-global-coronavirus-outbreak}}

Updated 2020-08-02T17:52:35.962Z

\begin{itemize}
\tightlist
\item
  \href{https://www.nytimes.com/2020/08/01/world/coronavirus-covid-19.html?action=click\&pgtype=Article\&state=default\&region=MAIN_CONTENT_1\&context=storylines_live_updates\#link-34047410}{The
  U.S. reels as July cases more than double the total of any other
  month.}
\item
  \href{https://www.nytimes.com/2020/08/01/world/coronavirus-covid-19.html?action=click\&pgtype=Article\&state=default\&region=MAIN_CONTENT_1\&context=storylines_live_updates\#link-780ec966}{Top
  U.S. officials work to break an impasse over the federal jobless
  benefit.}
\item
  \href{https://www.nytimes.com/2020/08/01/world/coronavirus-covid-19.html?action=click\&pgtype=Article\&state=default\&region=MAIN_CONTENT_1\&context=storylines_live_updates\#link-2bc8948}{Its
  outbreak untamed, Melbourne goes into even greater lockdown.}
\end{itemize}

\href{https://www.nytimes.com/2020/08/01/world/coronavirus-covid-19.html?action=click\&pgtype=Article\&state=default\&region=MAIN_CONTENT_1\&context=storylines_live_updates}{See
more updates}

More live coverage:
\href{https://www.nytimes.com/live/2020/07/31/business/stock-market-today-coronavirus?action=click\&pgtype=Article\&state=default\&region=MAIN_CONTENT_1\&context=storylines_live_updates}{Markets}

``No.'' Did his neighbors? Again, ``no.''

Having seen the photos of some Americans rushing to beaches as soon as
they were allowed, I asked why he thought Indians felt so constrained.

His answer came quickly: ``Everyone's scared. People are saying if they
get sick, where will they go?''

That explained a lot. It explained why I wasn't seeing anyone in my
neighborhood venturing into the parks or strolling under the banyan
trees. It explained why few people in India were testing the lockdown
limits.

It's not that they are automatically more willing to follow the rules
than say, California's sun lovers who were not nearly social distancing.
It's that in this case, Indians are more scared.

Image

Few shoppers were in a market last week, on the evening of the start of
Ramadan, near Jama Masjid, a major mosque.~

Image

Friday prayer at the start of Ramadan.

Image

A Muslim family and their Imam offer prayers before breaking their fast
on the first evening of Ramadan in New Delhi.

They're scared of catching a highly contagious disease, and they don't
trust that a beleaguered health care system will save them. Or they're
scared about how they'll pay for it, even if they get the care they
need.

India has a lot of great doctors but the ratio of doctors or hospital
beds per person is much lower than in the West. And many people here
survive on a few dollars a day.

India seems to be doing better than many richer countries in containing
the virus, at least so far. With reported infections relatively low,
around 35,000,
\href{https://www.nytimes.com/2020/04/28/world/asia/india-coronavirus-lockdown.html}{the
government is trying to loosen the national tourniquet} and reopen some
industries, such as agriculture and select manufacturing.

But many Indians don't want to take the tourniquet off, even if it is
stanching the economy's flow.

A good chunk of the population has decided that the best way to protect
themselves is not only to stick to the lockdown rules, but to go above
and beyond them, like the volunteer virus squads who sealed off entire
villages. Even people in my Delhi neighborhood have become lockdown
enthusiasts.

\href{https://www.nytimes.com/news-event/coronavirus?action=click\&pgtype=Article\&state=default\&region=MAIN_CONTENT_3\&context=storylines_faq}{}

\hypertarget{the-coronavirus-outbreak-}{%
\subsubsection{The Coronavirus Outbreak
›}\label{the-coronavirus-outbreak-}}

\hypertarget{frequently-asked-questions}{%
\paragraph{Frequently Asked
Questions}\label{frequently-asked-questions}}

Updated July 27, 2020

\begin{itemize}
\item ~
  \hypertarget{should-i-refinance-my-mortgage}{%
  \paragraph{Should I refinance my
  mortgage?}\label{should-i-refinance-my-mortgage}}

  \begin{itemize}
  \tightlist
  \item
    \href{https://www.nytimes.com/article/coronavirus-money-unemployment.html?action=click\&pgtype=Article\&state=default\&region=MAIN_CONTENT_3\&context=storylines_faq}{It
    could be a good idea,} because mortgage rates have
    \href{https://www.nytimes.com/2020/07/16/business/mortgage-rates-below-3-percent.html?action=click\&pgtype=Article\&state=default\&region=MAIN_CONTENT_3\&context=storylines_faq}{never
    been lower.} Refinancing requests have pushed mortgage applications
    to some of the highest levels since 2008, so be prepared to get in
    line. But defaults are also up, so if you're thinking about buying a
    home, be aware that some lenders have tightened their standards.
  \end{itemize}
\item ~
  \hypertarget{what-is-school-going-to-look-like-in-september}{%
  \paragraph{What is school going to look like in
  September?}\label{what-is-school-going-to-look-like-in-september}}

  \begin{itemize}
  \tightlist
  \item
    It is unlikely that many schools will return to a normal schedule
    this fall, requiring the grind of
    \href{https://www.nytimes.com/2020/06/05/us/coronavirus-education-lost-learning.html?action=click\&pgtype=Article\&state=default\&region=MAIN_CONTENT_3\&context=storylines_faq}{online
    learning},
    \href{https://www.nytimes.com/2020/05/29/us/coronavirus-child-care-centers.html?action=click\&pgtype=Article\&state=default\&region=MAIN_CONTENT_3\&context=storylines_faq}{makeshift
    child care} and
    \href{https://www.nytimes.com/2020/06/03/business/economy/coronavirus-working-women.html?action=click\&pgtype=Article\&state=default\&region=MAIN_CONTENT_3\&context=storylines_faq}{stunted
    workdays} to continue. California's two largest public school
    districts --- Los Angeles and San Diego --- said on July 13, that
    \href{https://www.nytimes.com/2020/07/13/us/lausd-san-diego-school-reopening.html?action=click\&pgtype=Article\&state=default\&region=MAIN_CONTENT_3\&context=storylines_faq}{instruction
    will be remote-only in the fall}, citing concerns that surging
    coronavirus infections in their areas pose too dire a risk for
    students and teachers. Together, the two districts enroll some
    825,000 students. They are the largest in the country so far to
    abandon plans for even a partial physical return to classrooms when
    they reopen in August. For other districts, the solution won't be an
    all-or-nothing approach.
    \href{https://bioethics.jhu.edu/research-and-outreach/projects/eschool-initiative/school-policy-tracker/}{Many
    systems}, including the nation's largest, New York City, are
    devising
    \href{https://www.nytimes.com/2020/06/26/us/coronavirus-schools-reopen-fall.html?action=click\&pgtype=Article\&state=default\&region=MAIN_CONTENT_3\&context=storylines_faq}{hybrid
    plans} that involve spending some days in classrooms and other days
    online. There's no national policy on this yet, so check with your
    municipal school system regularly to see what is happening in your
    community.
  \end{itemize}
\item ~
  \hypertarget{is-the-coronavirus-airborne}{%
  \paragraph{Is the coronavirus
  airborne?}\label{is-the-coronavirus-airborne}}

  \begin{itemize}
  \tightlist
  \item
    The coronavirus
    \href{https://www.nytimes.com/2020/07/04/health/239-experts-with-one-big-claim-the-coronavirus-is-airborne.html?action=click\&pgtype=Article\&state=default\&region=MAIN_CONTENT_3\&context=storylines_faq}{can
    stay aloft for hours in tiny droplets in stagnant air}, infecting
    people as they inhale, mounting scientific evidence suggests. This
    risk is highest in crowded indoor spaces with poor ventilation, and
    may help explain super-spreading events reported in meatpacking
    plants, churches and restaurants.
    \href{https://www.nytimes.com/2020/07/06/health/coronavirus-airborne-aerosols.html?action=click\&pgtype=Article\&state=default\&region=MAIN_CONTENT_3\&context=storylines_faq}{It's
    unclear how often the virus is spread} via these tiny droplets, or
    aerosols, compared with larger droplets that are expelled when a
    sick person coughs or sneezes, or transmitted through contact with
    contaminated surfaces, said Linsey Marr, an aerosol expert at
    Virginia Tech. Aerosols are released even when a person without
    symptoms exhales, talks or sings, according to Dr. Marr and more
    than 200 other experts, who
    \href{https://academic.oup.com/cid/article/doi/10.1093/cid/ciaa939/5867798}{have
    outlined the evidence in an open letter to the World Health
    Organization}.
  \end{itemize}
\item ~
  \hypertarget{what-are-the-symptoms-of-coronavirus}{%
  \paragraph{What are the symptoms of
  coronavirus?}\label{what-are-the-symptoms-of-coronavirus}}

  \begin{itemize}
  \tightlist
  \item
    Common symptoms
    \href{https://www.nytimes.com/article/symptoms-coronavirus.html?action=click\&pgtype=Article\&state=default\&region=MAIN_CONTENT_3\&context=storylines_faq}{include
    fever, a dry cough, fatigue and difficulty breathing or shortness of
    breath.} Some of these symptoms overlap with those of the flu,
    making detection difficult, but runny noses and stuffy sinuses are
    less common.
    \href{https://www.nytimes.com/2020/04/27/health/coronavirus-symptoms-cdc.html?action=click\&pgtype=Article\&state=default\&region=MAIN_CONTENT_3\&context=storylines_faq}{The
    C.D.C. has also} added chills, muscle pain, sore throat, headache
    and a new loss of the sense of taste or smell as symptoms to look
    out for. Most people fall ill five to seven days after exposure, but
    symptoms may appear in as few as two days or as many as 14 days.
  \end{itemize}
\item ~
  \hypertarget{does-asymptomatic-transmission-of-covid-19-happen}{%
  \paragraph{Does asymptomatic transmission of Covid-19
  happen?}\label{does-asymptomatic-transmission-of-covid-19-happen}}

  \begin{itemize}
  \tightlist
  \item
    So far, the evidence seems to show it does. A widely cited
    \href{https://www.nature.com/articles/s41591-020-0869-5}{paper}
    published in April suggests that people are most infectious about
    two days before the onset of coronavirus symptoms and estimated that
    44 percent of new infections were a result of transmission from
    people who were not yet showing symptoms. Recently, a top expert at
    the World Health Organization stated that transmission of the
    coronavirus by people who did not have symptoms was ``very rare,''
    \href{https://www.nytimes.com/2020/06/09/world/coronavirus-updates.html?action=click\&pgtype=Article\&state=default\&region=MAIN_CONTENT_3\&context=storylines_faq\#link-1f302e21}{but
    she later walked back that statement.}
  \end{itemize}
\end{itemize}

A few weeks ago when I biked with my two sons to the neighborhood milk
depot --- perfectly permissible under lockdown rules --- someone leaned
out a window and boomed: ``Go home!''

My kids' eyes widened. I pedaled away shaken. Did he mean back to our
apartment? Or to America? Many people here have blamed foreigners for
bringing India the coronavirus.

So my family now does what everyone does: we stay inside, looking out
the windows at one beautiful day passing after another. We used to sit
around the table and share upcoming plans. Now we don't have any.

Image

A financial district in central New Delhi on Friday.

Image

Migrant workers waiting to return to their homes in other states
exercising at a makeshift shelter in New Delhi, last month.

Image

Healthcare workers carry out house-to-house calls on residents in New
Delhi, last month.

We are marooned in the present tense. We couldn't leave New Delhi even
if we wanted to. I crave sitting in the open night air in a Rajasthani
village, listening to tabla music. Or simply standing in a freshly
turned field that smells of earth and shaking a farmer's hand.

Lockdown is a blow to what I do and why I've dragged my family into
this. Foreign correspondents relocate themselves and their families to
soak up new experiences and transmit as much of that as possible to
readers back home --- not just the news but also the feel of a place,
the humanity. We wither doing Zoom calls from our couches.

But there is so much suffering around me, I'm not focusing on that right
now.

I get out a couple of times a week with my journalist pass, and just a
few days ago I met a mother and her 9-year-old daughter moving from
spray-painted circle to spray-painted circle down a sidewalk in a food
line. The mother was a maid who had lost her job because of the
lockdown. The daughter told me coronavirus came from stones that fell
from the sky.

They had zero money and I could tell from how listlessly they accepted
their two pieces of fried bread and two lumps of potato curry that they
hated taking handouts.

Behind them, in the bright sun, were hundreds of people just like them,
marching slowly forward.

Image

New Delhi railway station on Thursday.

Advertisement

\protect\hyperlink{after-bottom}{Continue reading the main story}

\hypertarget{site-index}{%
\subsection{Site Index}\label{site-index}}

\hypertarget{site-information-navigation}{%
\subsection{Site Information
Navigation}\label{site-information-navigation}}

\begin{itemize}
\tightlist
\item
  \href{https://help.nytimes.com/hc/en-us/articles/115014792127-Copyright-notice}{©~2020~The
  New York Times Company}
\end{itemize}

\begin{itemize}
\tightlist
\item
  \href{https://www.nytco.com/}{NYTCo}
\item
  \href{https://help.nytimes.com/hc/en-us/articles/115015385887-Contact-Us}{Contact
  Us}
\item
  \href{https://www.nytco.com/careers/}{Work with us}
\item
  \href{https://nytmediakit.com/}{Advertise}
\item
  \href{http://www.tbrandstudio.com/}{T Brand Studio}
\item
  \href{https://www.nytimes.com/privacy/cookie-policy\#how-do-i-manage-trackers}{Your
  Ad Choices}
\item
  \href{https://www.nytimes.com/privacy}{Privacy}
\item
  \href{https://help.nytimes.com/hc/en-us/articles/115014893428-Terms-of-service}{Terms
  of Service}
\item
  \href{https://help.nytimes.com/hc/en-us/articles/115014893968-Terms-of-sale}{Terms
  of Sale}
\item
  \href{https://spiderbites.nytimes.com}{Site Map}
\item
  \href{https://help.nytimes.com/hc/en-us}{Help}
\item
  \href{https://www.nytimes.com/subscription?campaignId=37WXW}{Subscriptions}
\end{itemize}
