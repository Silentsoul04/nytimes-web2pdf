Sections

SEARCH

\protect\hyperlink{site-content}{Skip to
content}\protect\hyperlink{site-index}{Skip to site index}

\href{https://www.nytimes.com/section/nyregion}{New York}

\href{https://myaccount.nytimes.com/auth/login?response_type=cookie\&client_id=vi}{}

\href{https://www.nytimes.com/section/todayspaper}{Today's Paper}

\href{/section/nyregion}{New York}\textbar{}15 Funerals a Day: The Pace
of Death Stuns Brooklyn Muslims

\url{https://nyti.ms/2YwnRgw}

\begin{itemize}
\item
\item
\item
\item
\item
\end{itemize}

\href{https://www.nytimes.com/news-event/coronavirus?action=click\&pgtype=Article\&state=default\&region=TOP_BANNER\&context=storylines_menu}{The
Coronavirus Outbreak}

\begin{itemize}
\tightlist
\item
  live\href{https://www.nytimes.com/2020/08/04/world/coronavirus-covid-19.html?action=click\&pgtype=Article\&state=default\&region=TOP_BANNER\&context=storylines_menu}{Latest
  Updates}
\item
  \href{https://www.nytimes.com/interactive/2020/us/coronavirus-us-cases.html?action=click\&pgtype=Article\&state=default\&region=TOP_BANNER\&context=storylines_menu}{Maps
  and Cases}
\item
  \href{https://www.nytimes.com/interactive/2020/science/coronavirus-vaccine-tracker.html?action=click\&pgtype=Article\&state=default\&region=TOP_BANNER\&context=storylines_menu}{Vaccine
  Tracker}
\item
  \href{https://www.nytimes.com/2020/08/02/us/covid-college-reopening.html?action=click\&pgtype=Article\&state=default\&region=TOP_BANNER\&context=storylines_menu}{College
  Reopening}
\item
  \href{https://www.nytimes.com/live/2020/08/03/business/stock-market-today-coronavirus?action=click\&pgtype=Article\&state=default\&region=TOP_BANNER\&context=storylines_menu}{Economy}
\end{itemize}

Advertisement

\protect\hyperlink{after-top}{Continue reading the main story}

Supported by

\protect\hyperlink{after-sponsor}{Continue reading the main story}

NEW YORK SHUTTERED

\hypertarget{15-funerals-a-day-the-pace-of-death-stuns-brooklyn-muslims}{%
\section{15 Funerals a Day: The Pace of Death Stuns Brooklyn
Muslims}\label{15-funerals-a-day-the-pace-of-death-stuns-brooklyn-muslims}}

Al-Rayaan Muslim Funeral Services has turned to family members and
fellow mourners to offer prayers and move bodies.

By \href{https://www.nytimes.com/by/todd-heisler}{Todd Heisler}

May 4, 2020

\includegraphics{https://static01.nyt.com/images/2020/05/05/nyregion/05nyvirus-photo-muslim1/00nyvirus-photo-muslim1-articleLarge.jpg?quality=75\&auto=webp\&disable=upscale}

All day long, wood coffins are carried in and out of Al-Rayaan Muslim
Funeral Services in Brooklyn. What is meant to be a peaceful, reflective
moment for grieving families has given way to a chaotic rhythm. Workers
climb into a refrigerated truck and carefully carry the dead into the
funeral home for a prayer, then back out to be transported to their
final resting place.

They do this an average of 15 times a day in recent weeks. Before
coronavirus hit, the home was holding only 20 to 30 funerals a month.

\hypertarget{new-york-shuttered}{%
\subsubsection{\texorpdfstring{\href{https://www.nytimes.com/spotlight/new-york-shuttered}{New
York Shuttered}}{New York Shuttered}}\label{new-york-shuttered}}

We're documenting life in the city as the threat from the coronavirus
spreads.

Image

Inside Al-Rayaan, Imam Ahmed Ali Uzir, left, led a prayer for five
people who died from Covid-19.

Outside the home on this stretch of Coney Island Avenue, families gather
in small clusters waiting for their turn for a viewing. Many of the dead
are immigrants from Pakistan and Bangladesh, and some don't have family
in the United States. Mourners nearby are asked to join in the
recitation of the Janazah --- a Muslim funeral prayer.

Inside, empty caskets, delivered twice a week now, lean against the
walls. Each simple wooden coffin has a name written in marker.

Image

Imtiaz Ahmed, center, owns Al-Rayaan with his brother-in-law. He has
been working long hours coordinating hospital pickups, viewings and
burials.

Since the coronavirus took hold of New York in March, burial rituals in
the city have become more complicated.
\href{https://www.nytimes.com/2020/04/30/nyregion/coronavirus-nyc-funeral-home-morgue-bodies.html}{Funeral
homes everywhere are backed up} and overwhelmed. Many, like Al-Rayaan,
have had to rely on refrigerated trucks to store the dead since social
distancing restrictions, along with
\href{https://www.nytimes.com/interactive/2020/04/27/upshot/coronavirus-deaths-new-york-city.html}{a
significant spike in the number of deaths} in the city, slowed down the
pace of burials. It's particularly disruptive for Muslims, who rarely
practice embalming and whose religion dictates that the dead must be
\href{https://www.nytimes.com/2020/07/10/world/africa/coronavirus-capetown-south-africa.html}{buried}
quickly.

``The earth is waiting. Allah is asking for that person to be buried as
soon as possible. You never want that grave to wait for you,'' said
Zafar Iqbal, who owns Al-Rayaan with his brother in law, Imtiaz Ahmed.

Image

Imam Ahmed Ali Uzir, right, led a prayer in front of a minivan that
would take the deceased to a cemetery on Long Island.

Mr. Ahmed, an immigrant from Pakistan who used to drive a taxi, was
initially reluctant to enter the funeral business. But his
brother-in-law persuaded him, and three years ago they started
Al-Rayaan.

They chose the narrow storefront on Coney Island Avenue partly because
of its proximity to the mosque next door.

``I'm the guy who didn't want to do this. Now I'm the guy who manages
everything here,'' Mr. Ahmed said.

Image

Caskets are loaded into a van outside Al-Rayaan. Each casket has the
name of the deceased written in marker, along with the burial location.

Despite the grueling pace, Mr. Ahmed knows he is serving his community
in a time of need. When the pandemic took hold in March, many of his
workers, afraid of catching the virus, stopped coming to work. Family
and friends have since stepped in to help handle the arduous task of
keeping up with the pace of death.

\href{https://www.nytimes.com/news-event/coronavirus?action=click\&pgtype=Article\&state=default\&region=MAIN_CONTENT_3\&context=storylines_faq}{}

\hypertarget{the-coronavirus-outbreak-}{%
\subsubsection{The Coronavirus Outbreak
›}\label{the-coronavirus-outbreak-}}

\hypertarget{frequently-asked-questions}{%
\paragraph{Frequently Asked
Questions}\label{frequently-asked-questions}}

Updated August 3, 2020

\begin{itemize}
\item ~
  \hypertarget{im-a-small-business-owner-can-i-get-relief}{%
  \paragraph{I'm a small-business owner. Can I get
  relief?}\label{im-a-small-business-owner-can-i-get-relief}}

  \begin{itemize}
  \tightlist
  \item
    The
    \href{https://www.nytimes.com/article/small-business-loans-stimulus-grants-freelancers-coronavirus.html?action=click\&pgtype=Article\&state=default\&region=MAIN_CONTENT_3\&context=storylines_faq}{stimulus
    bills enacted in March} offer help for the millions of American
    small businesses. Those eligible for aid are businesses and
    nonprofit organizations with fewer than 500 workers, including sole
    proprietorships, independent contractors and freelancers. Some
    larger companies in some industries are also eligible. The help
    being offered, which is being managed by the Small Business
    Administration, includes the Paycheck Protection Program and the
    Economic Injury Disaster Loan program. But lots of folks have
    \href{https://www.nytimes.com/interactive/2020/05/07/business/small-business-loans-coronavirus.html?action=click\&pgtype=Article\&state=default\&region=MAIN_CONTENT_3\&context=storylines_faq}{not
    yet seen payouts.} Even those who have received help are confused:
    The rules are draconian, and some are stuck sitting on
    \href{https://www.nytimes.com/2020/05/02/business/economy/loans-coronavirus-small-business.html?action=click\&pgtype=Article\&state=default\&region=MAIN_CONTENT_3\&context=storylines_faq}{money
    they don't know how to use.} Many small-business owners are getting
    less than they expected or
    \href{https://www.nytimes.com/2020/06/10/business/Small-business-loans-ppp.html?action=click\&pgtype=Article\&state=default\&region=MAIN_CONTENT_3\&context=storylines_faq}{not
    hearing anything at all.}
  \end{itemize}
\item ~
  \hypertarget{what-are-my-rights-if-i-am-worried-about-going-back-to-work}{%
  \paragraph{What are my rights if I am worried about going back to
  work?}\label{what-are-my-rights-if-i-am-worried-about-going-back-to-work}}

  \begin{itemize}
  \tightlist
  \item
    Employers have to provide
    \href{https://www.osha.gov/SLTC/covid-19/standards.html}{a safe
    workplace} with policies that protect everyone equally.
    \href{https://www.nytimes.com/article/coronavirus-money-unemployment.html?action=click\&pgtype=Article\&state=default\&region=MAIN_CONTENT_3\&context=storylines_faq}{And
    if one of your co-workers tests positive for the coronavirus, the
    C.D.C.} has said that
    \href{https://www.cdc.gov/coronavirus/2019-ncov/community/guidance-business-response.html}{employers
    should tell their employees} -\/- without giving you the sick
    employee's name -\/- that they may have been exposed to the virus.
  \end{itemize}
\item ~
  \hypertarget{should-i-refinance-my-mortgage}{%
  \paragraph{Should I refinance my
  mortgage?}\label{should-i-refinance-my-mortgage}}

  \begin{itemize}
  \tightlist
  \item
    \href{https://www.nytimes.com/article/coronavirus-money-unemployment.html?action=click\&pgtype=Article\&state=default\&region=MAIN_CONTENT_3\&context=storylines_faq}{It
    could be a good idea,} because mortgage rates have
    \href{https://www.nytimes.com/2020/07/16/business/mortgage-rates-below-3-percent.html?action=click\&pgtype=Article\&state=default\&region=MAIN_CONTENT_3\&context=storylines_faq}{never
    been lower.} Refinancing requests have pushed mortgage applications
    to some of the highest levels since 2008, so be prepared to get in
    line. But defaults are also up, so if you're thinking about buying a
    home, be aware that some lenders have tightened their standards.
  \end{itemize}
\item ~
  \hypertarget{what-is-school-going-to-look-like-in-september}{%
  \paragraph{What is school going to look like in
  September?}\label{what-is-school-going-to-look-like-in-september}}

  \begin{itemize}
  \tightlist
  \item
    It is unlikely that many schools will return to a normal schedule
    this fall, requiring the grind of
    \href{https://www.nytimes.com/2020/06/05/us/coronavirus-education-lost-learning.html?action=click\&pgtype=Article\&state=default\&region=MAIN_CONTENT_3\&context=storylines_faq}{online
    learning},
    \href{https://www.nytimes.com/2020/05/29/us/coronavirus-child-care-centers.html?action=click\&pgtype=Article\&state=default\&region=MAIN_CONTENT_3\&context=storylines_faq}{makeshift
    child care} and
    \href{https://www.nytimes.com/2020/06/03/business/economy/coronavirus-working-women.html?action=click\&pgtype=Article\&state=default\&region=MAIN_CONTENT_3\&context=storylines_faq}{stunted
    workdays} to continue. California's two largest public school
    districts --- Los Angeles and San Diego --- said on July 13, that
    \href{https://www.nytimes.com/2020/07/13/us/lausd-san-diego-school-reopening.html?action=click\&pgtype=Article\&state=default\&region=MAIN_CONTENT_3\&context=storylines_faq}{instruction
    will be remote-only in the fall}, citing concerns that surging
    coronavirus infections in their areas pose too dire a risk for
    students and teachers. Together, the two districts enroll some
    825,000 students. They are the largest in the country so far to
    abandon plans for even a partial physical return to classrooms when
    they reopen in August. For other districts, the solution won't be an
    all-or-nothing approach.
    \href{https://bioethics.jhu.edu/research-and-outreach/projects/eschool-initiative/school-policy-tracker/}{Many
    systems}, including the nation's largest, New York City, are
    devising
    \href{https://www.nytimes.com/2020/06/26/us/coronavirus-schools-reopen-fall.html?action=click\&pgtype=Article\&state=default\&region=MAIN_CONTENT_3\&context=storylines_faq}{hybrid
    plans} that involve spending some days in classrooms and other days
    online. There's no national policy on this yet, so check with your
    municipal school system regularly to see what is happening in your
    community.
  \end{itemize}
\item ~
  \hypertarget{is-the-coronavirus-airborne}{%
  \paragraph{Is the coronavirus
  airborne?}\label{is-the-coronavirus-airborne}}

  \begin{itemize}
  \tightlist
  \item
    The coronavirus
    \href{https://www.nytimes.com/2020/07/04/health/239-experts-with-one-big-claim-the-coronavirus-is-airborne.html?action=click\&pgtype=Article\&state=default\&region=MAIN_CONTENT_3\&context=storylines_faq}{can
    stay aloft for hours in tiny droplets in stagnant air}, infecting
    people as they inhale, mounting scientific evidence suggests. This
    risk is highest in crowded indoor spaces with poor ventilation, and
    may help explain super-spreading events reported in meatpacking
    plants, churches and restaurants.
    \href{https://www.nytimes.com/2020/07/06/health/coronavirus-airborne-aerosols.html?action=click\&pgtype=Article\&state=default\&region=MAIN_CONTENT_3\&context=storylines_faq}{It's
    unclear how often the virus is spread} via these tiny droplets, or
    aerosols, compared with larger droplets that are expelled when a
    sick person coughs or sneezes, or transmitted through contact with
    contaminated surfaces, said Linsey Marr, an aerosol expert at
    Virginia Tech. Aerosols are released even when a person without
    symptoms exhales, talks or sings, according to Dr. Marr and more
    than 200 other experts, who
    \href{https://academic.oup.com/cid/article/doi/10.1093/cid/ciaa939/5867798}{have
    outlined the evidence in an open letter to the World Health
    Organization}.
  \end{itemize}
\end{itemize}

``It's not easy for any funeral home to take 15 funerals a day,'' Mr.
Ahmed said.

Image

Mr. Ahmed's family has been working at Al-Rayaan to help keep up with
the number of funerals, which has risen to about 15 each day, compared
to 20 to 30 a month before the coronavirus arrived.

In the last two months they have buried 200 people, including many who
died from Covid-19. There was Ferzana Ahsan, a pharmacist from Pakistan;
Aurangzabe Iqbal, a green taxi driver and father of four whose body was
shipped to his native Pakistan a day before his 40th birthday; and
Ferdous Hasan, whose family in Bangladesh arranged for him to be buried
in New Jersey.

Image

Iqbal Chaudhry, center, mourned his brother, Aurangzabe Iqbal, who spent
a week in the hospital on a ventilator before he died.

Last Tuesday, as the last minivan of the day drove away from Al-Rayaan,
Mr. Ahmed stood alone on the sidewalk. The afternoon sun warmed the air
as he enjoyed a rare moment of quiet. After moving 14 bodies that day,
including seven that were shipped back to Pakistan and the others
destined for cemeteries in New Jersey and Long Island, he was hoping to
call it an early day for once.

``Death is certain. It doesn't matter if there's an epidemic or not,''
he said.

Then a visitor arrived and his phone rang. He went back inside.

Image

Imam Ahmed Ali Uzir said a final prayer for two people who were
transported from Al-Rayaan to Marlboro Muslim Memorial Cemetery in New
Jersey.

Advertisement

\protect\hyperlink{after-bottom}{Continue reading the main story}

\hypertarget{site-index}{%
\subsection{Site Index}\label{site-index}}

\hypertarget{site-information-navigation}{%
\subsection{Site Information
Navigation}\label{site-information-navigation}}

\begin{itemize}
\tightlist
\item
  \href{https://help.nytimes.com/hc/en-us/articles/115014792127-Copyright-notice}{©~2020~The
  New York Times Company}
\end{itemize}

\begin{itemize}
\tightlist
\item
  \href{https://www.nytco.com/}{NYTCo}
\item
  \href{https://help.nytimes.com/hc/en-us/articles/115015385887-Contact-Us}{Contact
  Us}
\item
  \href{https://www.nytco.com/careers/}{Work with us}
\item
  \href{https://nytmediakit.com/}{Advertise}
\item
  \href{http://www.tbrandstudio.com/}{T Brand Studio}
\item
  \href{https://www.nytimes.com/privacy/cookie-policy\#how-do-i-manage-trackers}{Your
  Ad Choices}
\item
  \href{https://www.nytimes.com/privacy}{Privacy}
\item
  \href{https://help.nytimes.com/hc/en-us/articles/115014893428-Terms-of-service}{Terms
  of Service}
\item
  \href{https://help.nytimes.com/hc/en-us/articles/115014893968-Terms-of-sale}{Terms
  of Sale}
\item
  \href{https://spiderbites.nytimes.com}{Site Map}
\item
  \href{https://help.nytimes.com/hc/en-us}{Help}
\item
  \href{https://www.nytimes.com/subscription?campaignId=37WXW}{Subscriptions}
\end{itemize}
