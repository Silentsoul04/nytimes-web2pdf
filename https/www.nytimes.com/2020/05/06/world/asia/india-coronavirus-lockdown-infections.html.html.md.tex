Sections

SEARCH

\protect\hyperlink{site-content}{Skip to
content}\protect\hyperlink{site-index}{Skip to site index}

\href{https://www.nytimes.com/section/world/asia}{Asia Pacific}

\href{https://myaccount.nytimes.com/auth/login?response_type=cookie\&client_id=vi}{}

\href{https://www.nytimes.com/section/todayspaper}{Today's Paper}

\href{/section/world/asia}{Asia Pacific}\textbar{}As India Loosens Its
Strict Lockdown, Coronavirus Deaths Jump Sharply

\url{https://nyti.ms/2L5FhZE}

\begin{itemize}
\item
\item
\item
\item
\item
\end{itemize}

\href{https://www.nytimes.com/news-event/coronavirus?action=click\&pgtype=Article\&state=default\&region=TOP_BANNER\&context=storylines_menu}{The
Coronavirus Outbreak}

\begin{itemize}
\tightlist
\item
  live\href{https://www.nytimes.com/2020/08/02/world/coronavirus-updates.html?action=click\&pgtype=Article\&state=default\&region=TOP_BANNER\&context=storylines_menu}{Latest
  Updates}
\item
  \href{https://www.nytimes.com/interactive/2020/us/coronavirus-us-cases.html?action=click\&pgtype=Article\&state=default\&region=TOP_BANNER\&context=storylines_menu}{Maps
  and Cases}
\item
  \href{https://www.nytimes.com/interactive/2020/science/coronavirus-vaccine-tracker.html?action=click\&pgtype=Article\&state=default\&region=TOP_BANNER\&context=storylines_menu}{Vaccine
  Tracker}
\item
  \href{https://www.nytimes.com/interactive/2020/07/29/us/schools-reopening-coronavirus.html?action=click\&pgtype=Article\&state=default\&region=TOP_BANNER\&context=storylines_menu}{What
  School May Look Like}
\item
  \href{https://www.nytimes.com/live/2020/07/31/business/stock-market-today-coronavirus?action=click\&pgtype=Article\&state=default\&region=TOP_BANNER\&context=storylines_menu}{Economy}
\end{itemize}

Advertisement

\protect\hyperlink{after-top}{Continue reading the main story}

Supported by

\protect\hyperlink{after-sponsor}{Continue reading the main story}

\hypertarget{as-india-loosens-its-strict-lockdown-coronavirus-deaths-jump-sharply}{%
\section{As India Loosens Its Strict Lockdown, Coronavirus Deaths Jump
Sharply}\label{as-india-loosens-its-strict-lockdown-coronavirus-deaths-jump-sharply}}

The streets have suddenly come alive, especially at night, in many areas
where social distancing is impossible.

\includegraphics{https://static01.nyt.com/images/2020/05/06/world/06virus-india-1/06virus-india-1-articleLarge-v3.jpg?quality=75\&auto=webp\&disable=upscale}

\href{https://www.nytimes.com/by/jeffrey-gettleman}{\includegraphics{https://static01.nyt.com/images/2018/10/10/multimedia/author-jeffrey-gettleman/author-jeffrey-gettleman-thumbLarge.png}}\href{https://www.nytimes.com/by/kai-schultz}{\includegraphics{https://static01.nyt.com/images/2019/11/22/reader-center/author-kai-schultz/author-kai-schultz-thumbLarge.png}}

By \href{https://www.nytimes.com/by/jeffrey-gettleman}{Jeffrey
Gettleman} and \href{https://www.nytimes.com/by/kai-schultz}{Kai
Schultz}

\begin{itemize}
\item
  Published May 6, 2020Updated May 7, 2020
\item
  \begin{itemize}
  \item
  \item
  \item
  \item
  \item
  \end{itemize}
\end{itemize}

NEW DELHI --- Part of India's success in blunting the spread of the
coronavirus had been
\href{https://www.nytimes.com/2020/05/01/world/asia/india-coronavirus-delhi.html}{a
fierce lockdown that was zealously obeyed}.

But in the last few days, the government has loosened up the rules,
drawing people into the streets. And now the dangerous contagion appears
to be spreading more aggressively.

The doubling rate of infections --- the amount of time it takes for the
number of coronavirus cases to double --- has dropped from around 12
days to 9.5. The daily death rate has shot up from a few dozen in
mid-April to more than 100 now.

This is still a long way from the devastating toll that the United
States and several European countries have been enduring. The total
number of reported infections in India is around 50,000, giving it a
much lower per capita case rate than many other countries.

But a tour of Delhi shows how much has changed just this week.

The streets of working-class neighborhoods that last week were deserted
are thronged with people. Bicycle rickshaws dart in and out of traffic.
Pedestrians flow down the sides of the road. Most wear masks, as
required, but many wear them off their chins with their noses and
sometimes even their mouths exposed.

\includegraphics{https://static01.nyt.com/images/2020/05/06/world/06virus-india-2/merlin_172080561_16c76cbe-a89c-48fe-9017-d67ad9028dcc-articleLarge.jpg?quality=75\&auto=webp\&disable=upscale}

At liquor stores, which reopened Monday for the first time in six weeks
--- the government was desperate for liquor tax revenue --- there is
nothing close to social distancing. Instead, there is utter chaos.

Police officers try to beat back the crowds with long sticks, but it's
no use. The crowds just keep growing, with lines at stores sometimes
stretching nearly a mile. At one shop in Delhi on Tuesday, hundreds of
men converged to buy whiskey, pushing and shoving to get closer to the
front, crumpled bills in their fingers, wild looks in their eyes. Many
packed so close to each other that they rested their hands on the sweaty
backs of people in front of them.

As the heat rises --- it hit 104 degrees in New Delhi a few days ago ---
people who live in cramped quarters, sometimes eight to a room, are
finding it unbearable to stay indoors as the government has ordered. So
they spill outside. They mingle in the streets. They gather.

\hypertarget{latest-updates-global-coronavirus-outbreak}{%
\section{\texorpdfstring{\href{https://www.nytimes.com/2020/08/01/world/coronavirus-covid-19.html?action=click\&pgtype=Article\&state=default\&region=MAIN_CONTENT_1\&context=storylines_live_updates}{Latest
Updates: Global Coronavirus
Outbreak}}{Latest Updates: Global Coronavirus Outbreak}}\label{latest-updates-global-coronavirus-outbreak}}

Updated 2020-08-02T17:52:35.962Z

\begin{itemize}
\tightlist
\item
  \href{https://www.nytimes.com/2020/08/01/world/coronavirus-covid-19.html?action=click\&pgtype=Article\&state=default\&region=MAIN_CONTENT_1\&context=storylines_live_updates\#link-34047410}{The
  U.S. reels as July cases more than double the total of any other
  month.}
\item
  \href{https://www.nytimes.com/2020/08/01/world/coronavirus-covid-19.html?action=click\&pgtype=Article\&state=default\&region=MAIN_CONTENT_1\&context=storylines_live_updates\#link-780ec966}{Top
  U.S. officials work to break an impasse over the federal jobless
  benefit.}
\item
  \href{https://www.nytimes.com/2020/08/01/world/coronavirus-covid-19.html?action=click\&pgtype=Article\&state=default\&region=MAIN_CONTENT_1\&context=storylines_live_updates\#link-2bc8948}{Its
  outbreak untamed, Melbourne goes into even greater lockdown.}
\end{itemize}

\href{https://www.nytimes.com/2020/08/01/world/coronavirus-covid-19.html?action=click\&pgtype=Article\&state=default\&region=MAIN_CONTENT_1\&context=storylines_live_updates}{See
more updates}

More live coverage:
\href{https://www.nytimes.com/live/2020/07/31/business/stock-market-today-coronavirus?action=click\&pgtype=Article\&state=default\&region=MAIN_CONTENT_1\&context=storylines_live_updates}{Markets}

``There's no police around, nobody is enforcing the lockdown, people are
out everywhere,'' said an exasperated Delhi shopkeeper who goes by one
name, Mehtab.

The virus's hot spots are India's crowded urban areas, especially New
Delhi, the political capital, and Mumbai, the business capital. About a
third of all reported infections are from these two cities, each home to
around 20 million people.

In Mumbai, the police officers enforcing the lockdown seem exhausted,
and withdraw from neighborhoods at night.

Image

People lining up to receive free food in New Delhi last
week.Credit...Rebecca Conway for The New York Times

``Every night when the police leave, the people get out as if there is a
party,'' said Rakhi Jadhav, a local representative in eastern Mumbai.

Dharavi, one of Mumbai's biggest slums, where a million or so people are
squeezed into less than one square mile of shacks and narrow alleyways,
is emerging as a serious concern. At least 600 residents there have been
infected and it's nearly impossible to socially distance. People live
face to face and share communal toilets, often dozens of people using
just one.

Across the city, people are watching Dharavi closely. Officials are
beginning to worry they might not have the resources they need to
prevent a much wider outbreak.

``Testing labs, beds, facilities --- they are all being overburdened
with asymptomatic and mildly infected patients,'' said Pradip Awate, an
epidemiologist and chief surveillance officer in Maharashtra state, home
to Mumbai.

Six weeks ago, India's prime minister,
\href{https://www.nytimes.com/2020/03/24/world/asia/india-coronavirus-lockdown.html}{Narendra
Modi, imposed a strict nationwide lockdown} for India's 1.3 billion
people. Mr. Modi ordered Indians to stay in their homes, and his
government shut down just about everything from schools and offices to
railroads and the country's airspace. Even the borders separating
India's states were sealed.

Image

Health workers last week tracing infected individuals' contacts in the
Dharavi neighborhood.Credit...Atul Loke for The New York Times

Many Indians obeyed the rules, wary of catching the virus and not
trusting India's beleaguered health care system to save them.

But as the weeks have dragged on, the economic costs have piled up and
people are becoming more desperate. Many Indians live hand-to-mouth on
the equivalent of a few dollars a day, and countless millions have been
thrown out of work.
\href{https://www.nytimes.com/2020/03/29/world/asia/coronavirus-india-migrants.html}{Endless
streams of migrant workers have trudged hundreds of miles back to their
villages}, hoping to rely on their families and farms to survive.

\href{https://www.nytimes.com/news-event/coronavirus?action=click\&pgtype=Article\&state=default\&region=MAIN_CONTENT_3\&context=storylines_faq}{}

\hypertarget{the-coronavirus-outbreak-}{%
\subsubsection{The Coronavirus Outbreak
›}\label{the-coronavirus-outbreak-}}

\hypertarget{frequently-asked-questions}{%
\paragraph{Frequently Asked
Questions}\label{frequently-asked-questions}}

Updated July 27, 2020

\begin{itemize}
\item ~
  \hypertarget{should-i-refinance-my-mortgage}{%
  \paragraph{Should I refinance my
  mortgage?}\label{should-i-refinance-my-mortgage}}

  \begin{itemize}
  \tightlist
  \item
    \href{https://www.nytimes.com/article/coronavirus-money-unemployment.html?action=click\&pgtype=Article\&state=default\&region=MAIN_CONTENT_3\&context=storylines_faq}{It
    could be a good idea,} because mortgage rates have
    \href{https://www.nytimes.com/2020/07/16/business/mortgage-rates-below-3-percent.html?action=click\&pgtype=Article\&state=default\&region=MAIN_CONTENT_3\&context=storylines_faq}{never
    been lower.} Refinancing requests have pushed mortgage applications
    to some of the highest levels since 2008, so be prepared to get in
    line. But defaults are also up, so if you're thinking about buying a
    home, be aware that some lenders have tightened their standards.
  \end{itemize}
\item ~
  \hypertarget{what-is-school-going-to-look-like-in-september}{%
  \paragraph{What is school going to look like in
  September?}\label{what-is-school-going-to-look-like-in-september}}

  \begin{itemize}
  \tightlist
  \item
    It is unlikely that many schools will return to a normal schedule
    this fall, requiring the grind of
    \href{https://www.nytimes.com/2020/06/05/us/coronavirus-education-lost-learning.html?action=click\&pgtype=Article\&state=default\&region=MAIN_CONTENT_3\&context=storylines_faq}{online
    learning},
    \href{https://www.nytimes.com/2020/05/29/us/coronavirus-child-care-centers.html?action=click\&pgtype=Article\&state=default\&region=MAIN_CONTENT_3\&context=storylines_faq}{makeshift
    child care} and
    \href{https://www.nytimes.com/2020/06/03/business/economy/coronavirus-working-women.html?action=click\&pgtype=Article\&state=default\&region=MAIN_CONTENT_3\&context=storylines_faq}{stunted
    workdays} to continue. California's two largest public school
    districts --- Los Angeles and San Diego --- said on July 13, that
    \href{https://www.nytimes.com/2020/07/13/us/lausd-san-diego-school-reopening.html?action=click\&pgtype=Article\&state=default\&region=MAIN_CONTENT_3\&context=storylines_faq}{instruction
    will be remote-only in the fall}, citing concerns that surging
    coronavirus infections in their areas pose too dire a risk for
    students and teachers. Together, the two districts enroll some
    825,000 students. They are the largest in the country so far to
    abandon plans for even a partial physical return to classrooms when
    they reopen in August. For other districts, the solution won't be an
    all-or-nothing approach.
    \href{https://bioethics.jhu.edu/research-and-outreach/projects/eschool-initiative/school-policy-tracker/}{Many
    systems}, including the nation's largest, New York City, are
    devising
    \href{https://www.nytimes.com/2020/06/26/us/coronavirus-schools-reopen-fall.html?action=click\&pgtype=Article\&state=default\&region=MAIN_CONTENT_3\&context=storylines_faq}{hybrid
    plans} that involve spending some days in classrooms and other days
    online. There's no national policy on this yet, so check with your
    municipal school system regularly to see what is happening in your
    community.
  \end{itemize}
\item ~
  \hypertarget{is-the-coronavirus-airborne}{%
  \paragraph{Is the coronavirus
  airborne?}\label{is-the-coronavirus-airborne}}

  \begin{itemize}
  \tightlist
  \item
    The coronavirus
    \href{https://www.nytimes.com/2020/07/04/health/239-experts-with-one-big-claim-the-coronavirus-is-airborne.html?action=click\&pgtype=Article\&state=default\&region=MAIN_CONTENT_3\&context=storylines_faq}{can
    stay aloft for hours in tiny droplets in stagnant air}, infecting
    people as they inhale, mounting scientific evidence suggests. This
    risk is highest in crowded indoor spaces with poor ventilation, and
    may help explain super-spreading events reported in meatpacking
    plants, churches and restaurants.
    \href{https://www.nytimes.com/2020/07/06/health/coronavirus-airborne-aerosols.html?action=click\&pgtype=Article\&state=default\&region=MAIN_CONTENT_3\&context=storylines_faq}{It's
    unclear how often the virus is spread} via these tiny droplets, or
    aerosols, compared with larger droplets that are expelled when a
    sick person coughs or sneezes, or transmitted through contact with
    contaminated surfaces, said Linsey Marr, an aerosol expert at
    Virginia Tech. Aerosols are released even when a person without
    symptoms exhales, talks or sings, according to Dr. Marr and more
    than 200 other experts, who
    \href{https://academic.oup.com/cid/article/doi/10.1093/cid/ciaa939/5867798}{have
    outlined the evidence in an open letter to the World Health
    Organization}.
  \end{itemize}
\item ~
  \hypertarget{what-are-the-symptoms-of-coronavirus}{%
  \paragraph{What are the symptoms of
  coronavirus?}\label{what-are-the-symptoms-of-coronavirus}}

  \begin{itemize}
  \tightlist
  \item
    Common symptoms
    \href{https://www.nytimes.com/article/symptoms-coronavirus.html?action=click\&pgtype=Article\&state=default\&region=MAIN_CONTENT_3\&context=storylines_faq}{include
    fever, a dry cough, fatigue and difficulty breathing or shortness of
    breath.} Some of these symptoms overlap with those of the flu,
    making detection difficult, but runny noses and stuffy sinuses are
    less common.
    \href{https://www.nytimes.com/2020/04/27/health/coronavirus-symptoms-cdc.html?action=click\&pgtype=Article\&state=default\&region=MAIN_CONTENT_3\&context=storylines_faq}{The
    C.D.C. has also} added chills, muscle pain, sore throat, headache
    and a new loss of the sense of taste or smell as symptoms to look
    out for. Most people fall ill five to seven days after exposure, but
    symptoms may appear in as few as two days or as many as 14 days.
  \end{itemize}
\item ~
  \hypertarget{does-asymptomatic-transmission-of-covid-19-happen}{%
  \paragraph{Does asymptomatic transmission of Covid-19
  happen?}\label{does-asymptomatic-transmission-of-covid-19-happen}}

  \begin{itemize}
  \tightlist
  \item
    So far, the evidence seems to show it does. A widely cited
    \href{https://www.nature.com/articles/s41591-020-0869-5}{paper}
    published in April suggests that people are most infectious about
    two days before the onset of coronavirus symptoms and estimated that
    44 percent of new infections were a result of transmission from
    people who were not yet showing symptoms. Recently, a top expert at
    the World Health Organization stated that transmission of the
    coronavirus by people who did not have symptoms was ``very rare,''
    \href{https://www.nytimes.com/2020/06/09/world/coronavirus-updates.html?action=click\&pgtype=Article\&state=default\&region=MAIN_CONTENT_3\&context=storylines_faq\#link-1f302e21}{but
    she later walked back that statement.}
  \end{itemize}
\end{itemize}

Many out-of-work laborers are still marooned in India's cities with no
source of income and no way to feed their families except for meals
provided by charities or the government.

With the pressure building, the central government and many states
loosened up the lockdown rules this week, allowing small wedding
ceremonies to resume, buses to operate and many businesses to open,
including salons, pet shops and electrical stores.

So now Indians are wondering if the easing of the lockdown has led to
the surge in reported infections or if this is the right time, with
cases already rising, to allow people to interact more easily. The
increase in reported infections could also be from an increase in the
number of tests, from about 20,000 in late March to more than a million
today. India has reported just one positive case for every 25 tests,
compared with one infection for every six tests in the United States.

Image

Migrant workers on a special train awaiting departure from Bhiwandi
station in Mharashtra State on Sunday.Credit...Atul Loke for The New
York Times

Still, Dr. S.D. Gupta, a public health expert and a member of the
government's Covid-19 task force, said testing would need to ramp up
significantly to capture the full scope of India's coronavirus outbreak.

States like Tamil Nadu and Gujarat are witnessing sharp jumps in the
number of cases. Officials in West Bengal have been accused of
\href{https://www.thehindu.com/news/cities/kolkata/extremely-high-mortality-rate-in-west-bengal-says-central-team/article31501093.ece}{sitting
on higher-than-usual mortality figures}. In Punjab, the rate at which
new cases were doubling dropped from 18 days to four this past week,
largely because of
\href{https://timesofindia.indiatimes.com/city/chandigarh/punjabs-double-whammy-no-let-up-in-spiralling-cases/articleshow/75545299.cms}{an
aggressive outbreak among religious pilgrims} returning from
Maharashtra, India's worst afflicted state.

Dr. Gupta said it was difficult to say whether India was in the midst of
experiencing a dangerous upward trajectory of new cases. But he warned
that the country's growth curve of new infections had still not peaked.
With so many asymptomatic patients unaware of their status, he said,
officials needed to remain especially vigilant as more restrictions were
eased in the coming weeks.

``What we are seeing is the tip of the iceberg,'' he said. ``Unless we
do testing, we will never know how many cases we have. That's a
problem.''

Suhasini Raj and Hari Kumar contributed reporting.

Advertisement

\protect\hyperlink{after-bottom}{Continue reading the main story}

\hypertarget{site-index}{%
\subsection{Site Index}\label{site-index}}

\hypertarget{site-information-navigation}{%
\subsection{Site Information
Navigation}\label{site-information-navigation}}

\begin{itemize}
\tightlist
\item
  \href{https://help.nytimes.com/hc/en-us/articles/115014792127-Copyright-notice}{©~2020~The
  New York Times Company}
\end{itemize}

\begin{itemize}
\tightlist
\item
  \href{https://www.nytco.com/}{NYTCo}
\item
  \href{https://help.nytimes.com/hc/en-us/articles/115015385887-Contact-Us}{Contact
  Us}
\item
  \href{https://www.nytco.com/careers/}{Work with us}
\item
  \href{https://nytmediakit.com/}{Advertise}
\item
  \href{http://www.tbrandstudio.com/}{T Brand Studio}
\item
  \href{https://www.nytimes.com/privacy/cookie-policy\#how-do-i-manage-trackers}{Your
  Ad Choices}
\item
  \href{https://www.nytimes.com/privacy}{Privacy}
\item
  \href{https://help.nytimes.com/hc/en-us/articles/115014893428-Terms-of-service}{Terms
  of Service}
\item
  \href{https://help.nytimes.com/hc/en-us/articles/115014893968-Terms-of-sale}{Terms
  of Sale}
\item
  \href{https://spiderbites.nytimes.com}{Site Map}
\item
  \href{https://help.nytimes.com/hc/en-us}{Help}
\item
  \href{https://www.nytimes.com/subscription?campaignId=37WXW}{Subscriptions}
\end{itemize}
