Sections

SEARCH

\protect\hyperlink{site-content}{Skip to
content}\protect\hyperlink{site-index}{Skip to site index}

\href{https://www.nytimes.com/section/nyregion}{New York}

\href{https://myaccount.nytimes.com/auth/login?response_type=cookie\&client_id=vi}{}

\href{https://www.nytimes.com/section/todayspaper}{Today's Paper}

\href{/section/nyregion}{New York}\textbar{}De Blasio Strips Control of
Virus Tracing From Health Department

\url{https://nyti.ms/2LbjEqM}

\begin{itemize}
\item
\item
\item
\item
\item
\end{itemize}

\href{https://www.nytimes.com/news-event/coronavirus?action=click\&pgtype=Article\&state=default\&region=TOP_BANNER\&context=storylines_menu}{The
Coronavirus Outbreak}

\begin{itemize}
\tightlist
\item
  live\href{https://www.nytimes.com/2020/08/04/world/coronavirus-covid-19.html?action=click\&pgtype=Article\&state=default\&region=TOP_BANNER\&context=storylines_menu}{Latest
  Updates}
\item
  \href{https://www.nytimes.com/interactive/2020/us/coronavirus-us-cases.html?action=click\&pgtype=Article\&state=default\&region=TOP_BANNER\&context=storylines_menu}{Maps
  and Cases}
\item
  \href{https://www.nytimes.com/interactive/2020/science/coronavirus-vaccine-tracker.html?action=click\&pgtype=Article\&state=default\&region=TOP_BANNER\&context=storylines_menu}{Vaccine
  Tracker}
\item
  \href{https://www.nytimes.com/2020/08/02/us/covid-college-reopening.html?action=click\&pgtype=Article\&state=default\&region=TOP_BANNER\&context=storylines_menu}{College
  Reopening}
\item
  \href{https://www.nytimes.com/live/2020/08/04/business/stock-market-today-coronavirus?action=click\&pgtype=Article\&state=default\&region=TOP_BANNER\&context=storylines_menu}{Economy}
\end{itemize}

Advertisement

\protect\hyperlink{after-top}{Continue reading the main story}

Supported by

\protect\hyperlink{after-sponsor}{Continue reading the main story}

\hypertarget{de-blasio-strips-control-of-virus-tracing-from-health-department}{%
\section{De Blasio Strips Control of Virus Tracing From Health
Department}\label{de-blasio-strips-control-of-virus-tracing-from-health-department}}

Leading health officials expressed serious concerns over the transfer of
contact tracing to the agency that runs public hospitals, a departure
from current and past practice.

\includegraphics{https://static01.nyt.com/images/2020/05/07/nyregion/07nyvirus-contacttracing1/merlin_172283187_0cce69be-e71d-47b2-a06b-2c5e69a8a898-articleLarge.jpg?quality=75\&auto=webp\&disable=upscale}

\href{https://www.nytimes.com/by/j-david-goodman}{\includegraphics{https://static01.nyt.com/images/2018/07/18/nyregion/author-j-david-goodman/author-j-david-goodman-thumbLarge.png}}\href{https://www.nytimes.com/by/william-k-rashbaum}{\includegraphics{https://static01.nyt.com/images/2018/06/13/multimedia/author-william-k-rashbaum/author-william-k-rashbaum-thumbLarge.jpg}}\href{https://www.nytimes.com/by/jeffery-c-mays}{\includegraphics{https://static01.nyt.com/images/2018/07/18/multimedia/author-jeffery-c-mays/author-jeffery-c-mays-thumbLarge.png}}

By \href{https://www.nytimes.com/by/j-david-goodman}{J. David Goodman},
\href{https://www.nytimes.com/by/william-k-rashbaum}{William K.
Rashbaum} and \href{https://www.nytimes.com/by/jeffery-c-mays}{Jeffery
C. Mays}

\begin{itemize}
\item
  Published May 7, 2020Updated June 16, 2020
\item
  \begin{itemize}
  \item
  \item
  \item
  \item
  \item
  \end{itemize}
\end{itemize}

New York City will soon assemble an army of more than 1,000 disease
detectives to trace the contacts of every person who tests positive for
the coronavirus, an approach seen as crucial to quelling the outbreak
and paving the way to reopen the hobbled city.

But that effort will not be led by the city's renowned Health
Department, which for decades has conducted
\href{https://www.nytimes.com/2020/06/21/nyregion/nyc-contact-tracing.html}{contact
tracing} for diseases such as tuberculosis, H.I.V. and Ebola.

Instead, in a sharp departure from current and past practice, the city
is going to put the vast new public health apparatus in the hands of its
public hospital system, Health and Hospitals, city officials
acknowledged on Thursday night after being approached by The New York
Times about the changes.

The decision, which Mayor Bill de Blasio announced at his daily briefing
on Friday, puzzled current and former health officials, who questioned
the wisdom of changing what has worked before, especially during a
pandemic.

The department conducted tracing of coronavirus cases at the start of
the outbreak, and had been doing so again recently, in preparation for
the city's expansion.

Dr. Mary T. Bassett, a
\href{https://www.nytimes.com/2018/08/02/nyregion/nyc-health-mary-bassett-resignation.html}{former
city health commissioner} under Mr. de Blasio and now the director of
the FXB Center for Health and Human Rights at Harvard University, said
that the three key elements of handling the coronavirus --- testing,
tracing and quarantine --- have long been performed by the Health
Department.

``These are core functions of public health agencies around the world,
including New York City, which has decades of experience,'' Dr. Bassett
said in an email. ``To confront Covid-19, it makes sense to build on
this expertise.''

\hypertarget{latest-updates-global-coronavirus-outbreak}{%
\section{\texorpdfstring{\href{https://www.nytimes.com/2020/08/04/world/coronavirus-covid-19.html?action=click\&pgtype=Article\&state=default\&region=MAIN_CONTENT_1\&context=storylines_live_updates}{Latest
Updates: Global Coronavirus
Outbreak}}{Latest Updates: Global Coronavirus Outbreak}}\label{latest-updates-global-coronavirus-outbreak}}

Updated 2020-08-04T15:28:30.260Z

\begin{itemize}
\tightlist
\item
  \href{https://www.nytimes.com/2020/08/04/world/coronavirus-covid-19.html?action=click\&pgtype=Article\&state=default\&region=MAIN_CONTENT_1\&context=storylines_live_updates\#link-6b644638}{`Long
  days, long nights': Washington prepares for a prolonged fight over
  virus relief.}
\item
  \href{https://www.nytimes.com/2020/08/04/world/coronavirus-covid-19.html?action=click\&pgtype=Article\&state=default\&region=MAIN_CONTENT_1\&context=storylines_live_updates\#link-7af9fca0}{Israel's
  rocky reopening of its schools may be a lesson for the U.S.}
\item
  \href{https://www.nytimes.com/2020/08/04/world/coronavirus-covid-19.html?action=click\&pgtype=Article\&state=default\&region=MAIN_CONTENT_1\&context=storylines_live_updates\#link-5c0d6427}{As
  Isaias makes landfall, the virus makes it trickier to shelter from the
  storm.}
\end{itemize}

\href{https://www.nytimes.com/2020/08/04/world/coronavirus-covid-19.html?action=click\&pgtype=Article\&state=default\&region=MAIN_CONTENT_1\&context=storylines_live_updates}{See
more updates}

More live coverage:
\href{https://www.nytimes.com/live/2020/08/04/business/stock-market-today-coronavirus?action=click\&pgtype=Article\&state=default\&region=MAIN_CONTENT_1\&context=storylines_live_updates}{Markets}

The move also angered some within the ranks of the Health Department,
which has been at odds with Mr. de Blasio since the early days of the
outbreak, when top officials in the department clashed with City Hall
over the timing of school closures and public health messaging.

In announcing the move Friday morning, the mayor said, ``Everything at
Health and Hospitals has been based on speed and intensity and
precision, and they've done an amazing job.''

Dr. Mitchell Katz, the head of Health and Hospitals, said that the move
was made because his agency, a public benefit corporation rather than a
city department, could more quickly hire contact tracers and enter into
contracts for testing and other needed services.

His agency, he said, would also oversee the hotels for housing people
who could not quarantine at home, as well as some coronavirus testing.

But he said the tracing itself would be supervised by a team of roughly
50 Health Department experts who will be detailed to Health and
Hospitals to run the operation.

``So we'll be able to have the best of both worlds,'' he said. ``We'll
be able to have the people who are used to supervising this kind of work
still doing the supervision of it. But we'll be able to hire as quickly
as possible and get as many tests done'' as possible.

\includegraphics{https://static01.nyt.com/images/2020/05/07/nyregion/07nyvirus-contacttracing2/07nyvirus-contacttracing2-articleLarge.jpg?quality=75\&auto=webp\&disable=upscale}

The city's public hospital system struggled with budgetary shortfalls
even before the coronavirus outbreak sent thousands of sick and dying
patients into its 11 hospitals. Their emergency rooms have been the
front lines of the crisis, as doctors and nurses faced a wave of
patients at facilities including
\href{https://www.nytimes.com/2020/03/25/nyregion/nyc-coronavirus-hospitals.html}{Elmhurst
Hospital in Queens} and
\href{https://www.nytimes.com/2020/04/15/nyregion/coronavirus-woodhull-madhvi-aya-dead.html}{Woodhull
Hospital in Brooklyn}. Staff members died, even some who
\href{https://www.nytimes.com/2020/05/04/nyregion/coronavirus-ny-hospital-workers.html}{did
not work directly with patients}.

A spokeswoman for Mr. de Blasio declined to comment about any
dissatisfaction expressed by Health Department officials and declined to
make the city health commissioner, Dr. Oxiris Barbot, available to
discuss the matter.

Contact tracing is one part of an accepted strategy for responding to
disease outbreaks, and it is now at the center of a national discussion
of how and when to begin allowing economic activity to restart. State
officials included the number of contact tracers working in a region as
one key metric for determining when businesses in that region would able
to reopen.

Mr. de Blasio, in recent days, has talked repeatedly about the large new
public health system that will be needed to contain the virus as the
number of new cases recedes. Officials have said the efforts are
essential to reviving the city's economic life.

\href{https://www.nytimes.com/news-event/coronavirus?action=click\&pgtype=Article\&state=default\&region=MAIN_CONTENT_3\&context=storylines_faq}{}

\hypertarget{the-coronavirus-outbreak-}{%
\subsubsection{The Coronavirus Outbreak
›}\label{the-coronavirus-outbreak-}}

\hypertarget{frequently-asked-questions}{%
\paragraph{Frequently Asked
Questions}\label{frequently-asked-questions}}

Updated August 3, 2020

\begin{itemize}
\item ~
  \hypertarget{im-a-small-business-owner-can-i-get-relief}{%
  \paragraph{I'm a small-business owner. Can I get
  relief?}\label{im-a-small-business-owner-can-i-get-relief}}

  \begin{itemize}
  \tightlist
  \item
    The
    \href{https://www.nytimes.com/article/small-business-loans-stimulus-grants-freelancers-coronavirus.html?action=click\&pgtype=Article\&state=default\&region=MAIN_CONTENT_3\&context=storylines_faq}{stimulus
    bills enacted in March} offer help for the millions of American
    small businesses. Those eligible for aid are businesses and
    nonprofit organizations with fewer than 500 workers, including sole
    proprietorships, independent contractors and freelancers. Some
    larger companies in some industries are also eligible. The help
    being offered, which is being managed by the Small Business
    Administration, includes the Paycheck Protection Program and the
    Economic Injury Disaster Loan program. But lots of folks have
    \href{https://www.nytimes.com/interactive/2020/05/07/business/small-business-loans-coronavirus.html?action=click\&pgtype=Article\&state=default\&region=MAIN_CONTENT_3\&context=storylines_faq}{not
    yet seen payouts.} Even those who have received help are confused:
    The rules are draconian, and some are stuck sitting on
    \href{https://www.nytimes.com/2020/05/02/business/economy/loans-coronavirus-small-business.html?action=click\&pgtype=Article\&state=default\&region=MAIN_CONTENT_3\&context=storylines_faq}{money
    they don't know how to use.} Many small-business owners are getting
    less than they expected or
    \href{https://www.nytimes.com/2020/06/10/business/Small-business-loans-ppp.html?action=click\&pgtype=Article\&state=default\&region=MAIN_CONTENT_3\&context=storylines_faq}{not
    hearing anything at all.}
  \end{itemize}
\item ~
  \hypertarget{what-are-my-rights-if-i-am-worried-about-going-back-to-work}{%
  \paragraph{What are my rights if I am worried about going back to
  work?}\label{what-are-my-rights-if-i-am-worried-about-going-back-to-work}}

  \begin{itemize}
  \tightlist
  \item
    Employers have to provide
    \href{https://www.osha.gov/SLTC/covid-19/standards.html}{a safe
    workplace} with policies that protect everyone equally.
    \href{https://www.nytimes.com/article/coronavirus-money-unemployment.html?action=click\&pgtype=Article\&state=default\&region=MAIN_CONTENT_3\&context=storylines_faq}{And
    if one of your co-workers tests positive for the coronavirus, the
    C.D.C.} has said that
    \href{https://www.cdc.gov/coronavirus/2019-ncov/community/guidance-business-response.html}{employers
    should tell their employees} -\/- without giving you the sick
    employee's name -\/- that they may have been exposed to the virus.
  \end{itemize}
\item ~
  \hypertarget{should-i-refinance-my-mortgage}{%
  \paragraph{Should I refinance my
  mortgage?}\label{should-i-refinance-my-mortgage}}

  \begin{itemize}
  \tightlist
  \item
    \href{https://www.nytimes.com/article/coronavirus-money-unemployment.html?action=click\&pgtype=Article\&state=default\&region=MAIN_CONTENT_3\&context=storylines_faq}{It
    could be a good idea,} because mortgage rates have
    \href{https://www.nytimes.com/2020/07/16/business/mortgage-rates-below-3-percent.html?action=click\&pgtype=Article\&state=default\&region=MAIN_CONTENT_3\&context=storylines_faq}{never
    been lower.} Refinancing requests have pushed mortgage applications
    to some of the highest levels since 2008, so be prepared to get in
    line. But defaults are also up, so if you're thinking about buying a
    home, be aware that some lenders have tightened their standards.
  \end{itemize}
\item ~
  \hypertarget{what-is-school-going-to-look-like-in-september}{%
  \paragraph{What is school going to look like in
  September?}\label{what-is-school-going-to-look-like-in-september}}

  \begin{itemize}
  \tightlist
  \item
    It is unlikely that many schools will return to a normal schedule
    this fall, requiring the grind of
    \href{https://www.nytimes.com/2020/06/05/us/coronavirus-education-lost-learning.html?action=click\&pgtype=Article\&state=default\&region=MAIN_CONTENT_3\&context=storylines_faq}{online
    learning},
    \href{https://www.nytimes.com/2020/05/29/us/coronavirus-child-care-centers.html?action=click\&pgtype=Article\&state=default\&region=MAIN_CONTENT_3\&context=storylines_faq}{makeshift
    child care} and
    \href{https://www.nytimes.com/2020/06/03/business/economy/coronavirus-working-women.html?action=click\&pgtype=Article\&state=default\&region=MAIN_CONTENT_3\&context=storylines_faq}{stunted
    workdays} to continue. California's two largest public school
    districts --- Los Angeles and San Diego --- said on July 13, that
    \href{https://www.nytimes.com/2020/07/13/us/lausd-san-diego-school-reopening.html?action=click\&pgtype=Article\&state=default\&region=MAIN_CONTENT_3\&context=storylines_faq}{instruction
    will be remote-only in the fall}, citing concerns that surging
    coronavirus infections in their areas pose too dire a risk for
    students and teachers. Together, the two districts enroll some
    825,000 students. They are the largest in the country so far to
    abandon plans for even a partial physical return to classrooms when
    they reopen in August. For other districts, the solution won't be an
    all-or-nothing approach.
    \href{https://bioethics.jhu.edu/research-and-outreach/projects/eschool-initiative/school-policy-tracker/}{Many
    systems}, including the nation's largest, New York City, are
    devising
    \href{https://www.nytimes.com/2020/06/26/us/coronavirus-schools-reopen-fall.html?action=click\&pgtype=Article\&state=default\&region=MAIN_CONTENT_3\&context=storylines_faq}{hybrid
    plans} that involve spending some days in classrooms and other days
    online. There's no national policy on this yet, so check with your
    municipal school system regularly to see what is happening in your
    community.
  \end{itemize}
\item ~
  \hypertarget{is-the-coronavirus-airborne}{%
  \paragraph{Is the coronavirus
  airborne?}\label{is-the-coronavirus-airborne}}

  \begin{itemize}
  \tightlist
  \item
    The coronavirus
    \href{https://www.nytimes.com/2020/07/04/health/239-experts-with-one-big-claim-the-coronavirus-is-airborne.html?action=click\&pgtype=Article\&state=default\&region=MAIN_CONTENT_3\&context=storylines_faq}{can
    stay aloft for hours in tiny droplets in stagnant air}, infecting
    people as they inhale, mounting scientific evidence suggests. This
    risk is highest in crowded indoor spaces with poor ventilation, and
    may help explain super-spreading events reported in meatpacking
    plants, churches and restaurants.
    \href{https://www.nytimes.com/2020/07/06/health/coronavirus-airborne-aerosols.html?action=click\&pgtype=Article\&state=default\&region=MAIN_CONTENT_3\&context=storylines_faq}{It's
    unclear how often the virus is spread} via these tiny droplets, or
    aerosols, compared with larger droplets that are expelled when a
    sick person coughs or sneezes, or transmitted through contact with
    contaminated surfaces, said Linsey Marr, an aerosol expert at
    Virginia Tech. Aerosols are released even when a person without
    symptoms exhales, talks or sings, according to Dr. Marr and more
    than 200 other experts, who
    \href{https://academic.oup.com/cid/article/doi/10.1093/cid/ciaa939/5867798}{have
    outlined the evidence in an open letter to the World Health
    Organization}.
  \end{itemize}
\end{itemize}

``You're talking about tens of thousands of people who will be tested
daily,'' the mayor said in an interview on MSNBC on Sunday. ``You're
talking about a tracing apparatus that, anyone who is tested positive,
we ask them, we interview them: who they've been close with in recent
days, who they've come in contact with.''

Before the shift, hiring was being conducted by the nonprofit Fund for
Public Health in New York City in partnership with the Health
Department; salaries for contact tracers
\href{https://web.archive.org/web/20200428080921/https://1w20ju1nsz1k2xqrjx3ccsd1-wpengine.netdna-ssl.com/wp-content/uploads/sites/76/2020/04/CT-I.pdf}{started
at \$57,000 a year}. Thousands had already applied.

Dr. Thomas R. Frieden, the former director of the Centers for Disease
Control and Prevention and former New York City health commissioner,
said that the city's Health Department is the ``greatest in the world''
and that ``if any health department can excel at contact tracing, New
York City can.''

David Harvey, executive director of the National Coalition of STD
Directors, a membership organization of health officials who deal with
contact tracing to limit the spread of sexually transmitted diseases,
said that while there was a ``role for the health care delivery system
to coordinate with the health department,'' the group was concerned that
the mayor's decision might slow the implementation of contract tracing
in the city.

``New York is leading the nation as the epicenter of this epidemic and
it~is a leader of the nation,'' Mr. Harvey added. ``We are worried about
anything that delays the rollout of an expanded contact tracing
program.''

The Health Department has the authority to issue quarantine orders in
New York City. Such orders are likely to be an important part of the
program, which would not only involve getting in touch with people who
have had close contact with an infected person, but also getting those
contacts to quarantine themselves, either at home or, possibly, in a
city-funded hotel.

\href{https://www.nytimes.com/2014/10/24/nyregion/tracing-patients-possible-contacts-creates-host-of-challenges-for-the-city.html}{During
the Ebola outbreak early in Mr. de Blasio's first term}, the Health
Department created a telephone center to track potential contacts.

As the rapid spread of infections overwhelmed the system in late March
and April, the city largely stopped doing contact tracing on confirmed
Covid-19 cases. In recent weeks, health workers have been doing a few
case investigations to prepare to ramp up the system, and they plan to
restart tracing on a broad scale once hundreds of new people have been
hired.

Yet the city has never tried to trace a disease as commonplace in the
population as Covid-19, Dr. Shama Ahuja, who oversees tuberculosis
contact tracing for the Health Department, said in an interview last
week.

``It is like tracing the flu,'' she said. ``The level of this effort is
unprecedented.''

Sharon Otterman contributed reporting.

Advertisement

\protect\hyperlink{after-bottom}{Continue reading the main story}

\hypertarget{site-index}{%
\subsection{Site Index}\label{site-index}}

\hypertarget{site-information-navigation}{%
\subsection{Site Information
Navigation}\label{site-information-navigation}}

\begin{itemize}
\tightlist
\item
  \href{https://help.nytimes.com/hc/en-us/articles/115014792127-Copyright-notice}{©~2020~The
  New York Times Company}
\end{itemize}

\begin{itemize}
\tightlist
\item
  \href{https://www.nytco.com/}{NYTCo}
\item
  \href{https://help.nytimes.com/hc/en-us/articles/115015385887-Contact-Us}{Contact
  Us}
\item
  \href{https://www.nytco.com/careers/}{Work with us}
\item
  \href{https://nytmediakit.com/}{Advertise}
\item
  \href{http://www.tbrandstudio.com/}{T Brand Studio}
\item
  \href{https://www.nytimes.com/privacy/cookie-policy\#how-do-i-manage-trackers}{Your
  Ad Choices}
\item
  \href{https://www.nytimes.com/privacy}{Privacy}
\item
  \href{https://help.nytimes.com/hc/en-us/articles/115014893428-Terms-of-service}{Terms
  of Service}
\item
  \href{https://help.nytimes.com/hc/en-us/articles/115014893968-Terms-of-sale}{Terms
  of Sale}
\item
  \href{https://spiderbites.nytimes.com}{Site Map}
\item
  \href{https://help.nytimes.com/hc/en-us}{Help}
\item
  \href{https://www.nytimes.com/subscription?campaignId=37WXW}{Subscriptions}
\end{itemize}
