Sections

SEARCH

\protect\hyperlink{site-content}{Skip to
content}\protect\hyperlink{site-index}{Skip to site index}

\href{https://www.nytimes.com/section/politics}{Politics}

\href{https://myaccount.nytimes.com/auth/login?response_type=cookie\&client_id=vi}{}

\href{https://www.nytimes.com/section/todayspaper}{Today's Paper}

\href{/section/politics}{Politics}\textbar{}U.S. Drops Michael Flynn
Case, in Move Backed by Trump

\begin{itemize}
\item
\item
\item
\item
\item
\item
\end{itemize}

Advertisement

\protect\hyperlink{after-top}{Continue reading the main story}

Supported by

\protect\hyperlink{after-sponsor}{Continue reading the main story}

\hypertarget{us-drops-michael-flynn-case-in-move-backed-by-trump}{%
\section{U.S. Drops Michael Flynn Case, in Move Backed by
Trump}\label{us-drops-michael-flynn-case-in-move-backed-by-trump}}

The extraordinary move came after Mr. Flynn, the former national
security adviser, fought the case in court for months, a reversal after
pleading guilty twice and cooperating with investigators.

\includegraphics{https://static01.nyt.com/images/2020/05/07/us/politics/07dc-flynn-sub/merlin_141043815_6e17b22c-bef5-44e7-b52c-1155f04eed76-articleLarge.jpg?quality=75\&auto=webp\&disable=upscale}

\href{https://www.nytimes.com/by/adam-goldman}{\includegraphics{https://static01.nyt.com/images/2018/07/12/multimedia/author-adam-goldman/author-adam-goldman-thumbLarge.png}}\href{https://www.nytimes.com/by/katie-benner}{\includegraphics{https://static01.nyt.com/images/2018/02/16/multimedia/author-katie-benner/author-katie-benner-thumbLarge-v2.png}}

By \href{https://www.nytimes.com/by/adam-goldman}{Adam Goldman} and
\href{https://www.nytimes.com/by/katie-benner}{Katie Benner}

\begin{itemize}
\item
  Published May 7, 2020Updated June 10, 2020
\item
  \begin{itemize}
  \item
  \item
  \item
  \item
  \item
  \item
  \end{itemize}
\end{itemize}

WASHINGTON --- After an extraordinary public campaign by President Trump
and his allies, the Justice Department dropped its criminal case on
Thursday against
\href{https://www.nytimes.com/2020/05/07/us/politics/michael-flynn-case.html}{Michael
T. Flynn}, Mr. Trump's first national security adviser.

\href{https://www.nytimes.com/2020/05/07/us/politics/michael-flynn-case.html}{Mr.
Flynn} had previously pleaded guilty twice to lying to F.B.I. agents
about his conversations with a Russian diplomat during the presidential
transition in late 2016.

The move was the latest example of Attorney General William P. Barr's
efforts
\href{https://www.nytimes.com/2020/03/06/us/barr-mueller-investigation.html}{to
chisel away at the results of the Russia investigation}. Documents that
Mr. Flynn's lawyers cited as evidence of prosecutorial misconduct were
turned over as part of a review by an outside prosecutor whom Mr. Barr
assigned
\href{https://www.nytimes.com/2020/02/14/us/politics/michael-flynn-prosecutors-barr.html}{to
re-examine the case}. Mr. Barr has cast doubt not only on some of the
prosecutions in the investigation but also on its premise, assigning
another independent prosecutor to scrutinize its origins.

The decision for the government to throw out a case after a defendant
had already pleaded guilty was also highly unusual. Former prosecutors
\href{https://www.nytimes.com/2020/05/07/us/politics/michael-flynn-case.html}{struggled
to point to any precedent} and portrayed the Justice Department's
justification as dubious.

By abandoning the case, the department undid what had been one of the
first significant acts of the special counsel investigation into
possible ties between the Trump campaign and Russia's 2016 election
interference --- the prosecution of a retired top Army general turned
national security adviser who pleaded guilty to lying to investigators.

Though
\href{https://www.nytimes.com/2017/02/13/us/politics/donald-trump-national-security-adviser-michael-flynn.html}{Mr.
Trump fired Mr. Flynn} weeks into his presidency for lying to Vice
President Mike Pence about the conversations with the diplomat, he has
long complained that a corrupt few at the F.B.I. mistreated Mr. Flynn
and suggested he might pardon him. Law enforcement officials dropping
the charges took issue with the F.B.I.'s interview of Mr. Flynn in early
2017 as part of the Russia investigation that Robert S. Mueller III
later took over.

Agents' questioning ``was untethered to, and unjustified by, the
F.B.I.'s counterintelligence investigation into Mr. Flynn,'' the acting
United States attorney in Washington, Timothy Shea, wrote
\href{https://int.nyt.com/data/documenthelper/6936-michael-flynn-motion-to-dismiss/fa06f5e13a0ec71843b6/optimized/full.pdf\#page=1}{in
a motion to dismiss the charges}. Prosecutors said that the case fell
short of the legal standard that Mr. Flynn's lies be ``materially''
relevant to the matter under investigation.

``The government is not persuaded that the Jan. 24, 2017, interview was
conducted with a legitimate investigative basis and therefore does not
believe Mr. Flynn's statements were material even if untrue,'' Mr. Shea
wrote.

Democrats condemned the move. ``A politicized and thoroughly corrupt
Department of Justice is going to let the president's crony simply walk
away,'' said Representative Jerrold Nadler, Democrat of New York and the
chairman of the House Judiciary Committee. ``Americans are right to be
furious and worried about the continued erosion of our rule of law.''

He said he would ask the Justice Department inspector general to
investigate and work to secure Mr. Barr's testimony before his committee
as soon as possible.

In dropping the charges, law enforcement officials abandoned the stance
of the career prosecutors who had been on the case, who had argued that
Mr. Flynn's conversations with the Russian ambassador to the United
States at the time, Sergey I. Kislyak, ``went to the heart'' of the
F.B.I.'s Trump-Russia investigation.

Mr. Trump told reporters on Thursday that Mr. Flynn was ``an innocent
man'' and accused Obama administration officials of targeting him to try
to ``take down a president.'' He angrily tore into his unnamed
persecutors. ``I hope a lot of people are going to pay a big price
because they're dishonest, crooked people,'' Mr. Trump said. ``They're
scum --- and I say it a lot, they're scum, they're human scum. This
should never have happened in this country.''

Mr. Barr explained the decision as an effort to ``restore confidence in
the system'' and that law enforcement officials had a duty to dismiss
the charges. He said he was ``doing the law's bidding,'' not Mr.
Trump's, and the Justice Department said that it did not brief the White
House before it dropped the charges.

``Partisan feelings are so strong that people have lost any sense of
justice,'' Mr. Barr said
\href{https://www.cbsnews.com/news/attorney-general-william-barr-on-michael-flynn-obamacare-and-coronavirus-restrictions-transcript/}{in
an interview with CBS News}.

Asked whether Mr. Flynn lied, Mr. Barr said that ``people sometimes
plead to things that turn out not to be crimes.''

Sidney Powell, Mr. Flynn's lawyer, said on Fox Business Network that she
and her client were ``relieved and gratified'' that Mr. Barr withdrew
the case. She called the decision ``a restoration of the rule of law.''

Mr. Flynn pleaded guilty in December 2017 to lying to the F.B.I. about
his conversations with Mr. Kislyak during the presidential transition.
He had also entered a guilty plea a second time in 2018 at an aborted
sentencing hearing. At the time, Mr. Flynn said he knew that lying to
the F.B.I. was a crime and accepted responsibility for what he had done.

The climactic move by the Justice Department appeared to be brewing in
recent days after the outside prosecutors reviewing the case said they
had found new documents that were possibly exculpatory, raising the
hopes of Mr. Flynn's lawyers and vocal supporters that he would be
exonerated.

One of the documents showed that the F.B.I. was on the verge of closing
a counterintelligence investigation into Mr. Flynn's ties to Russia in
early January 2017. Mr. Shea, a longtime trusted adviser of Mr. Barr's,
said in the court filing that F.B.I. agents had no reason to interview
Mr. Flynn at the White House weeks later.

In a possible sign of disagreement, Brandon L. Van Grack, the Justice
Department lawyer who led the prosecution of Mr. Flynn, abruptly
withdrew from the case on Thursday. Mr. Flynn's lawyers have repeatedly
attacked Mr. Van Grack by name in court filings, citing his ``incredible
malfeasance.'' Other prosecutors in Mr. Shea's office were stunned by
the decision to drop the case, according to a person who spoke with
several lawyers in the office.

Neither Mr. Van Grack nor other career prosecutors who have worked on
the Flynn case over the past two and a half years signed their names to
Mr. Shea's court filing.

It is now up to the federal judge in Washington overseeing the case,
Emmet G. Sullivan, to decide whether to dismiss it and close off the
possibility that Mr. Flynn could be tried again for the same crime. If
the judge wants, he could ask for written submissions and hold a hearing
on that topic.

Judge Sullivan, who accepted Mr. Flynn's original guilty plea, could
also weigh in on whether he believes any of the new materials that the
government has produced to Mr. Flynn's lawyers represent a violation on
the part of the Justice Department or its lawyers who worked on the
case.

The White House was prepared for the possibility of Mr. Trump pardoning
Mr. Flynn last week, according to two people familiar with the
discussions. But some advisers urged the president to let the case play
out.

Mr. Flynn, a decorated lieutenant general and former head of the Defense
Intelligence Agency, was an early supporter of Mr. Trump's 2016
presidential campaign,
\href{https://www.youtube.com/watch?v=UFBAjhxjQ90}{joining the crowd} in
a ``lock her up'' chant about Hillary Clinton, then the presumptive
Democratic presidential nominee, at the Republican National Convention
that year.

After winning the election, Mr. Trump named Mr. Flynn his national
security adviser. In late December 2016, Mr. Flynn and Mr. Kislyak spoke
on the phone shortly after the Obama administration placed sanctions on
Russia for interfering in the election, and Mr. Flynn advised that the
Russians hold off on escalating in response, undermining that
administration's foreign policy.

The F.B.I. had been close to closing its inquiry into Mr. Flynn. But as
investigators discovered the conversations in early January through
routine government wiretaps of Mr. Kislyak, and as they learned in
subsequent days that he had lied to other White House officials about
them, they began to conclude that they had reason to suspect that Mr.
Flynn constituted a national security threat.

Law enforcement officials warned the White House that Russia could have
blackmailed Mr. Flynn. But seeing no move by Mr. Trump to address the
issue, F.B.I. agents decided to question Mr. Flynn, where he repeatedly
made false statements about his talks with Mr. Kislyak.

Yet the agents
\href{https://intelligence.house.gov/uploadedfiles/am33.pdf}{``felt like
it was not clear to them that he was, you know, lying or dissembling},''
Andrew G. McCabe, the F.B.I. deputy director at the time, told
congressional investigators in an interview released on Thursday as part
of \href{https://intelligence.house.gov/russiainvestigation/}{thousands
of pages} of witness testimony from the House Intelligence Committee's
own Russia inquiry.

Investigators knew Mr. Flynn's statements were ``inconsistent'' with
what he had told Mr. Kislyak, Mr. McCabe said. The transcripts seemed
unambiguous: Mr. Flynn brought up the sanctions, according to Mary
McCord, a former top national security lawyer at the Justice Department
who reviewed the documents. It was hard to imagine that Mr. Flynn
``would forget talking about something he raised,'' she told special
counsel investigators, according to court papers.

But Mr. Shea argued that the F.B.I. should never have interviewed Mr.
Flynn. Agents had no need to speak with him because they were on the
verge of closing the inquiry and already knew what was said on the call
because of the transcripts.

Mr. Shea also dismissed F.B.I. concerns about Mr. Flynn lying to Mr.
Pence and to a White House spokesman at the time, Sean Spicer, about the
call. Both provided misinformation to the public, but Mr. Shea said
their statements did not ``provide a separate or distinct basis for an
investigation.''

He added that if F.B.I. officials were so worried, they should have
talked with Mr. Pence and Mr. Spicer.

``The frail and shifting justifications for its ongoing probe of Mr.
Flynn, as well as the irregular procedure that preceded his interview,
suggests that the F.B.I. was eager to interview Mr. Flynn irrespective
of any underlying investigation,'' Mr. Shea wrote.

Less than a month after the interview, Mr. Flynn was
\href{https://www.nytimes.com/2017/02/13/us/politics/donald-trump-national-security-adviser-michael-flynn.html}{forced
to resign}. According to the White House at the time, he was forced out
because he was not forthcoming with Mr. Pence about his
\href{https://www.nytimes.com/interactive/2017/02/14/us/politics/flynn-call-russia-timeline.html}{conversations
with} Mr. Kislyak.

Mr. Shea also said the government could not prove at any trial that Mr.
Flynn lied. The admission was striking because agents had given Mr.
Flynn repeated opportunities during questioning to correct his
misstatements. Prosecutors had previously said that Mr. Flynn lied
because he was locked into the story he had told Mr. Pence.

Mr. Shea also said he was uncertain that prosecutors could prove that
Mr. Flynn's false statements were material to the Russia investigation.

Mr. Van Grack had said in earlier court filings that the ``topics of
sanctions went to the heart of the F.B.I.'s counterintelligence
investigation.'' He said that ``any effort to undermine those sanctions
could have been evidence of links or coordination between the Trump
campaign and Russia.''

Judge Sullivan had also said that Mr. Flynn's statements were material
to the Russia inquiry.

Mr. Flynn
\href{https://www.nytimes.com/2018/12/04/us/politics/michael-flynn-special-counsel-sentencing-memo.html}{cooperated
extensively} with the Justice Department before hiring new lawyers last
year and later moving
\href{https://www.nytimes.com/2020/01/14/us/politics/michael-flynn-withdraws-guilty-plea.html}{to
withdraw his plea}. Ms. Powell has accused prosecutors in a blizzard of
court filings of ``bad faith,'' pressuring her client to cooperate and
withholding exculpatory evidence. Judge Sullivan
\href{https://www.nytimes.com/2019/12/16/us/politics/michael-flynn-sentencing.html}{forcefully
rejected} most of those claims.

Last summer, Ms. Powell wrote to Mr. Barr to suggest that new
prosecutors review the case, expressing her ``fervent hope that you and
the Department of Justice will use this case to restore integrity and
trust in the department.''

She then confidently predicted three months later on Fox Business's
``Lou Dobbs Tonight'' that the case would be dropped. At the time,
prosecutors had not yet begun the review of it.

Reporting was contributed by Michael Crowley, Charlie Savage and Julian
E. Barnes from Washington, and Maggie Haberman from New York.

Advertisement

\protect\hyperlink{after-bottom}{Continue reading the main story}

\hypertarget{site-index}{%
\subsection{Site Index}\label{site-index}}

\hypertarget{site-information-navigation}{%
\subsection{Site Information
Navigation}\label{site-information-navigation}}

\begin{itemize}
\tightlist
\item
  \href{https://help.nytimes.com/hc/en-us/articles/115014792127-Copyright-notice}{©~2020~The
  New York Times Company}
\end{itemize}

\begin{itemize}
\tightlist
\item
  \href{https://www.nytco.com/}{NYTCo}
\item
  \href{https://help.nytimes.com/hc/en-us/articles/115015385887-Contact-Us}{Contact
  Us}
\item
  \href{https://www.nytco.com/careers/}{Work with us}
\item
  \href{https://nytmediakit.com/}{Advertise}
\item
  \href{http://www.tbrandstudio.com/}{T Brand Studio}
\item
  \href{https://www.nytimes.com/privacy/cookie-policy\#how-do-i-manage-trackers}{Your
  Ad Choices}
\item
  \href{https://www.nytimes.com/privacy}{Privacy}
\item
  \href{https://help.nytimes.com/hc/en-us/articles/115014893428-Terms-of-service}{Terms
  of Service}
\item
  \href{https://help.nytimes.com/hc/en-us/articles/115014893968-Terms-of-sale}{Terms
  of Sale}
\item
  \href{https://spiderbites.nytimes.com}{Site Map}
\item
  \href{https://help.nytimes.com/hc/en-us}{Help}
\item
  \href{https://www.nytimes.com/subscription?campaignId=37WXW}{Subscriptions}
\end{itemize}
