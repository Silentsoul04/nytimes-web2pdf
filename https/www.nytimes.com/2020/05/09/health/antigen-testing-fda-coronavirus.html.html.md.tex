Sections

SEARCH

\protect\hyperlink{site-content}{Skip to
content}\protect\hyperlink{site-index}{Skip to site index}

\href{https://www.nytimes.com/section/health}{Health}

\href{https://myaccount.nytimes.com/auth/login?response_type=cookie\&client_id=vi}{}

\href{https://www.nytimes.com/section/todayspaper}{Today's Paper}

\href{/section/health}{Health}\textbar{}F.D.A. Approves First Antigen
Test for Detecting the Coronavirus

\url{https://nyti.ms/3fBFJN4}

\begin{itemize}
\item
\item
\item
\item
\item
\end{itemize}

\href{https://www.nytimes.com/news-event/coronavirus?action=click\&pgtype=Article\&state=default\&region=TOP_BANNER\&context=storylines_menu}{The
Coronavirus Outbreak}

\begin{itemize}
\tightlist
\item
  live\href{https://www.nytimes.com/2020/08/04/world/coronavirus-cases.html?action=click\&pgtype=Article\&state=default\&region=TOP_BANNER\&context=storylines_menu}{Latest
  Updates}
\item
  \href{https://www.nytimes.com/interactive/2020/us/coronavirus-us-cases.html?action=click\&pgtype=Article\&state=default\&region=TOP_BANNER\&context=storylines_menu}{Maps
  and Cases}
\item
  \href{https://www.nytimes.com/interactive/2020/science/coronavirus-vaccine-tracker.html?action=click\&pgtype=Article\&state=default\&region=TOP_BANNER\&context=storylines_menu}{Vaccine
  Tracker}
\item
  \href{https://www.nytimes.com/2020/08/02/us/covid-college-reopening.html?action=click\&pgtype=Article\&state=default\&region=TOP_BANNER\&context=storylines_menu}{College
  Reopening}
\item
  \href{https://www.nytimes.com/live/2020/08/04/business/stock-market-today-coronavirus?action=click\&pgtype=Article\&state=default\&region=TOP_BANNER\&context=storylines_menu}{Economy}
\end{itemize}

Advertisement

\protect\hyperlink{after-top}{Continue reading the main story}

Supported by

\protect\hyperlink{after-sponsor}{Continue reading the main story}

\hypertarget{fda-approves-first-antigen-test-for-detecting-the-coronavirus}{%
\section{F.D.A. Approves First Antigen Test for Detecting the
Coronavirus}\label{fda-approves-first-antigen-test-for-detecting-the-coronavirus}}

The test offers medical workers and health authorities an inexpensive
tool for fast, mass screening.

\includegraphics{https://static01.nyt.com/images/2020/05/10/us/politics/10virus-test-fda/merlin_172349061_8b491ab0-d461-49df-8320-19c330caee86-articleLarge.jpg?quality=75\&auto=webp\&disable=upscale}

By \href{https://www.nytimes.com/by/andrew-jacobs}{Andrew Jacobs}

\begin{itemize}
\item
  Published May 9, 2020Updated May 19, 2020
\item
  \begin{itemize}
  \item
  \item
  \item
  \item
  \item
  \end{itemize}
\end{itemize}

The Food and Drug Administration has approved the first antigen test
that can rapidly detect whether a person has been infected by the
coronavirus, a significant advancement that promises to greatly expand
the nation's testing capacity.

The test, by the
\href{https://ir.quidel.com/news/news-release-details/2020/Quidel-Receives-Emergency-Authorization-for-Rapid-Antigen-COVID-19-Diagnostic-Assay/default.aspx}{Quidel
Corporation} of San Diego, was given emergency use authorization late
Friday by the F.D.A.,
\href{https://www.fda.gov/media/137886/download}{according to a notice}
on the agency's website.

Unlike commonly available coronavirus tests that use polymerase chain
reaction, or PCR, antigen diagnostics work by quickly detecting
fragments of virus in a sample. The newly approved Quidel test will rely
on specimens collected from nasal swabs, according to the F.D.A., and
they can only be processed by the company's lab instruments.

``Diagnostic testing is one of the pillars of our nation's response to
Covid-19, and the F.D.A. continues to take actions to help make these
critical products available,'' the agency said in a statement on
Saturday. ``One of the main advantages of an antigen test is the speed
of the test, which can provide results in minutes.'' The F.D.A. said it
expects to grant emergency clearance for other antigen tests in the near
future.

A shortage of coronavirus tests in the United States has hampered
efforts to contain the pandemic, and has limited the capabilities of
states seeking to ease the lockdowns and social distancing measures that
have throttled the nation's economy.

Although antigen tests are extremely accurate in detecting positive
infections, they cannot detect all active infections and they have a
higher chance of false negatives than PCR tests. Positive test results
may also need to be confirmed with the slower but more accurate PCR
test, which relies on detecting the presence of genetic material.

Experts said the approval of an antigen test for Covid-19 would boost
testing efforts by giving medical workers and health authorities an
inexpensive tool for mass rapid testing. Further developed, antigen
tests also hold potential for use at home, in the manner of a home
pregnancy kit.

``I am very enthusiastic about antigen testing because of its ability to
be scaled up to millions of tests a day, and because it has a much more
rapid turnaround,'' said Dr. Ashish Jha, director of the Harvard Global
Health Institute. ``A lot of us have been looking forward to this
moment.''

Dr. Jha said he still had questions about the sensitivity and
specificity of the test. He noted that even an antigen test that
produces false positives is preferable to one that results in false
negatives, because infected people given a negative result could
unknowingly spread the virus to others.

``It's ideal to have low false positives and false negatives, but
usually you're trading off between the two,'' Dr. Jha said. ``The reason
you're wiling to live with more false positives is you want to be able
to screen people, and if they are negative you want to feel confident
that they are truly negative.''

Advertisement

\protect\hyperlink{after-bottom}{Continue reading the main story}

\hypertarget{site-index}{%
\subsection{Site Index}\label{site-index}}

\hypertarget{site-information-navigation}{%
\subsection{Site Information
Navigation}\label{site-information-navigation}}

\begin{itemize}
\tightlist
\item
  \href{https://help.nytimes.com/hc/en-us/articles/115014792127-Copyright-notice}{©~2020~The
  New York Times Company}
\end{itemize}

\begin{itemize}
\tightlist
\item
  \href{https://www.nytco.com/}{NYTCo}
\item
  \href{https://help.nytimes.com/hc/en-us/articles/115015385887-Contact-Us}{Contact
  Us}
\item
  \href{https://www.nytco.com/careers/}{Work with us}
\item
  \href{https://nytmediakit.com/}{Advertise}
\item
  \href{http://www.tbrandstudio.com/}{T Brand Studio}
\item
  \href{https://www.nytimes.com/privacy/cookie-policy\#how-do-i-manage-trackers}{Your
  Ad Choices}
\item
  \href{https://www.nytimes.com/privacy}{Privacy}
\item
  \href{https://help.nytimes.com/hc/en-us/articles/115014893428-Terms-of-service}{Terms
  of Service}
\item
  \href{https://help.nytimes.com/hc/en-us/articles/115014893968-Terms-of-sale}{Terms
  of Sale}
\item
  \href{https://spiderbites.nytimes.com}{Site Map}
\item
  \href{https://help.nytimes.com/hc/en-us}{Help}
\item
  \href{https://www.nytimes.com/subscription?campaignId=37WXW}{Subscriptions}
\end{itemize}
