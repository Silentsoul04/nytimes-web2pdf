Sections

SEARCH

\protect\hyperlink{site-content}{Skip to
content}\protect\hyperlink{site-index}{Skip to site index}

\href{https://www.nytimes.com/section/nyregion}{New York}

\href{https://myaccount.nytimes.com/auth/login?response_type=cookie\&client_id=vi}{}

\href{https://www.nytimes.com/section/todayspaper}{Today's Paper}

\href{/section/nyregion}{New York}\textbar{}Jimmy Glenn, Ex-Boxer Whose
Times Square Bar Endures, Dies at 89

\url{https://nyti.ms/3fDt5gW}

\begin{itemize}
\item
\item
\item
\item
\item
\item
\end{itemize}

\href{https://www.nytimes.com/news-event/coronavirus?action=click\&pgtype=Article\&state=default\&region=TOP_BANNER\&context=storylines_menu}{The
Coronavirus Outbreak}

\begin{itemize}
\tightlist
\item
  live\href{https://www.nytimes.com/2020/08/03/world/coronavirus-covid-19.html?action=click\&pgtype=Article\&state=default\&region=TOP_BANNER\&context=storylines_menu}{Latest
  Updates}
\item
  \href{https://www.nytimes.com/interactive/2020/us/coronavirus-us-cases.html?action=click\&pgtype=Article\&state=default\&region=TOP_BANNER\&context=storylines_menu}{Maps
  and Cases}
\item
  \href{https://www.nytimes.com/interactive/2020/science/coronavirus-vaccine-tracker.html?action=click\&pgtype=Article\&state=default\&region=TOP_BANNER\&context=storylines_menu}{Vaccine
  Tracker}
\item
  \href{https://www.nytimes.com/2020/08/02/us/covid-college-reopening.html?action=click\&pgtype=Article\&state=default\&region=TOP_BANNER\&context=storylines_menu}{College
  Reopening}
\item
  \href{https://www.nytimes.com/live/2020/08/03/business/stock-market-today-coronavirus?action=click\&pgtype=Article\&state=default\&region=TOP_BANNER\&context=storylines_menu}{Economy}
\end{itemize}

Advertisement

\protect\hyperlink{after-top}{Continue reading the main story}

Supported by

\protect\hyperlink{after-sponsor}{Continue reading the main story}

Those We've Lost

\hypertarget{jimmy-glenn-ex-boxer-whose-times-square-bar-endures-dies-at-89}{%
\section{Jimmy Glenn, Ex-Boxer Whose Times Square Bar Endures, Dies at
89}\label{jimmy-glenn-ex-boxer-whose-times-square-bar-endures-dies-at-89}}

A fighter, trainer, cutman and manager, he once owned both a bar and a
gym. Only the bar survives. He died of coronavirus complications.

\includegraphics{https://static01.nyt.com/images/2020/05/13/obituaries/08Glenn/merlin_172292619_2a1abb1b-a347-49d5-aef7-b900256a4a12-articleLarge.jpg?quality=75\&auto=webp\&disable=upscale}

\href{https://www.nytimes.com/by/richard-sandomir}{\includegraphics{https://static01.nyt.com/images/2018/12/10/multimedia/author-richard-sandomir/author-richard-sandomir-thumbLarge.png}}

By \href{https://www.nytimes.com/by/richard-sandomir}{Richard Sandomir}

\begin{itemize}
\item
  Published May 9, 2020Updated May 11, 2020
\item
  \begin{itemize}
  \item
  \item
  \item
  \item
  \item
  \item
  \end{itemize}
\end{itemize}

\emph{This obituary is part of a series about people who have died in
the coronavirus pandemic. Read about others}
\href{https://www.nytimes.com/series/people-who-have-died-of-the-coronavirus}{\emph{here}}\emph{.}

Jimmy Glenn, a former amateur boxer and trainer whose gym on 42nd Street
succumbed to Times Square's redevelopment, but whose nearby bar has
resisted the area's drastic changes for nearly a half-century, died on
Thursday in Manhattan. He was 89.

The cause was complications of the coronavirus, his son Adam Glenn said.

Mr. Glenn opened the bar Jimmy's Corner, on West 44th Street between
Sixth and Seventh Avenues, in 1971, the year Joe Frazier defeated
Muhammad Ali at Madison Square Garden in the so-called Fight of the
Century. He made it a shrine to boxing, filling it with photos, posters
and other memorabilia.

If Mr. Glenn was not off training a fighter or working in a boxer's
corner as a cutman, he was at the bar six nights a week. It was a small
and simple joint, serving drinks and snacks, and it reduced costs by not
serving meals.

Seven years after opening Jimmy's Corner, he opened the Times Square
Boxing Club, over a bar in a small building on 42nd Street near
Broadway. It was the newest gym in the city, but it was never a
moneymaker; Mr. Glenn subsidized it with earnings from Jimmy's Corner.

But for Mr. Glenn, a friendly man with a gentle voice, owning a gym was
a natural business. He loved training young fighters. One night he sat
at his bar and talked with The Daily News about the ambitious young
boxers he was mentoring. ``Once you get boxing in your blood,'' he
observed, ``it don't leave.''

He closed the gym 15 years later, but Jimmy's Corner endured.

\includegraphics{https://static01.nyt.com/images/2020/05/08/obituaries/08Glenn3/merlin_172324458_6e152ef7-f1b3-4e46-aa8b-825c63018ac6-articleLarge.jpg?quality=75\&auto=webp\&disable=upscale}

James Lee Glenn was born on Aug. 18, 1930, in Monticello, S.C. His
mother, Susie (Glenn) Robertson, was a domestic worker. When James was
about 7 his mother left him with his grandfather, a sharecropper, and
moved to Washington. He reunited with her a few years later in Harlem,
when she married Jonah Robertson, a custodian and pastor. Jimmy spent
most of World War II with his grandfather but returned to Harlem in
1944.

Mr. Glenn began boxing in New York at the Police Athletic League and
studied the splendid
\href{https://www.nytimes.com/1989/04/13/obituaries/sugar-ray-robinson-boxing-s-best-is-dead.html}{Sugar
Ray Robinson} when he worked out at a Harlem gym. In late 1946, when he
was 16, he went to the Garden to see Robinson beat Tommy Bell to win his
first championship belt.

Mr. Glenn's boxing career was modest. As an amateur fighter, from the
mid-1940s until around 1950, he had 14 wins and two defeats as a
middleweight and welterweight. His most famous bout was a loss to Floyd
Patterson, the future heavyweight champion.

``He beat me,''
\href{https://www.secondsout.com/columns/thomas-hauser/happy-birthday-jimmy-glenn}{Mr.
Glenn told Seconds Out}, a boxing website, in 2005. ``Knocked me down a
few times, broke my tooth. But I went the distance.''

Mr. Glenn volunteered to train amateurs at a church community center in
Harlem and stayed there for about 15 years while working odd jobs,
including as a truck driver and a house painter.

He developed a reputation as a skilled trainer and cutman for boxers
like \href{https://boxrec.com/media/index.php/Jimmy_Glenn}{Patterson,
Michael Spinks, Jameel McCline, Aaron Davis, Bobby Cassidy and Terrence
Alli.} Most recently he managed Travis Peterkin, a light heavyweight. In
the late 1980s he trained Mark Gastineau, a former defensive end for the
New York Jets, in his brief career as a boxer.

Image

The survival of Jimmy's Corner, one writer observed, ``has the feel of
an underdog story.''Credit...Piotr Redlinski for The New York Times

Adam Glenn said that his father's perseverance --- and a friendly
landlord --- kept Jimmy's Corner in business, attracting boxers,
promoters and fans, who would drink there and watch bouts on television
surrounded by pictures of Mr. Glenn with Muhammad Ali (a friend who
occasionally visited Mr. Glenn's gym) and posters that promoted Joe
Louis-Max Schmeling and Ali-George Foreman bouts.

``He and my mom built a solid business that required 24-hour
attention,'' Adam Glenn said. ``He had to make rules to keep the pimps
and mob out, and that meant standing up every day and telling people,
`This is my place and nobody is going to mess with us.'''

The landlord, the Durst Organization, did not try to push the bar out
with exorbitant rent.

``They loved my dad,'' Adam Glenn said.

In addition to him, Mr. Glenn is survived by another son, James Jr.;
five daughters, Denise Mercado, Cheryl Glenn-Mitchell, Delana Glenn,
Anita Costa and Tanya Glenn; and six siblings. His wife, Swietlana
Garbarska**,** known as Swannie Glenn, died in 2015. His previous
marriage, to Wynola Ann Flemming, ended in divorce.

In 2018,
\href{https://www.newyorker.com/magazine/2018/05/21/the-underdog-story-of-times-squares-still-gritty-jimmys-corner}{The
New Yorker visited Jimmy's Corner} and took note of Mr. Glenn's silvery
horseshoe mustache and marble-handled walking cane; his cheap prices,
like \$3 draft beers and \$3.50 drinks from the rail; and his permanence
in a changing neighborhood.

``From a distance of half a century,'' The New Yorker wrote, ``the bar's
survival, in the heart of Times Square, has the feel of an underdog
story. The decades-long makeover of the neighborhood, from a convivial
Gomorrah to an outpost of Disneyland, couldn't dislodge the place.''

Jimmy's Corner has been closed during the coronavirus pandemic. But Adam
Glenn vowed that it would reopen.

''

\href{https://www.nytimes.com/interactive/2020/obituaries/people-died-coronavirus-obituaries.html?action=click\&pgtype=Article\&state=default\&region=BELOW_MAIN_CONTENT\&context=covid_obits_promo}{}

\hypertarget{those-weve-lost}{%
\section{Those We've Lost}\label{those-weve-lost}}

The coronavirus pandemic has taken an incalculable death toll. This
series is designed to put names and faces to the numbers.

Read more

\includegraphics{https://static01.nyt.com/images/2020/07/30/obituaries/30Pedro/30Pedro-square640.jpg}

\hypertarget{bernaldina-josuxe9-pedro}{%
\section{Bernaldina José Pedro}\label{bernaldina-josuxe9-pedro}}

d. Boa Vista, Brazil

Leader among the Indigenous Macuxi

\includegraphics{https://static01.nyt.com/images/2020/07/31/obituaries/31Swing/merlin_175167783_8913bc90-0d64-43f3-a655-1bb1bf1601c9-square640.jpg}

\hypertarget{john-eric-swing}{%
\section{John Eric Swing}\label{john-eric-swing}}

d. Fountain Valley, Calif.

Champion of Filipino-Americans

\includegraphics{https://static01.nyt.com/images/2020/07/27/obituaries/27Victor/merlin_175001436_38b11f8e-227a-4e2c-9821-7618af9b2524-square640.jpg}

\hypertarget{victor-victor}{%
\section{Victor Victor}\label{victor-victor}}

d. Santo Domingo, Dominican Republic

Beloved musician of the Dominican Republic

\includegraphics{https://static01.nyt.com/images/2020/07/31/obituaries/31Negron/merlin_175160169_516322ae-fd23-4969-b6b2-193ced371105-square640.jpg}

\hypertarget{dr-eddie-negruxf3n}{%
\section{Dr. Eddie Negrón}\label{dr-eddie-negruxf3n}}

d. Fort Walton Beach, Fla.

Internist on Florida's Emerald Coast

\includegraphics{https://static01.nyt.com/images/2020/07/30/obituaries/30Dobson/merlin_175115928_f6b9271c-8f05-4fe1-a38a-5ca4a58f8935-square640.jpg}

\hypertarget{dobby-dobson}{%
\section{Dobby Dobson}\label{dobby-dobson}}

d. Coral Springs, Fla.

Jamaican singer and songwriter

\includegraphics{https://static01.nyt.com/images/2020/08/01/obituaries/28Gonzalez/merlin_175002771_beb57888-3951-409a-ae13-03a94b2e962e-square640.jpg}

\hypertarget{waldemar-gonzalez}{%
\section{Waldemar Gonzalez}\label{waldemar-gonzalez}}

d. White Plains, N.Y.

Teacher and social worker

Advertisement

\protect\hyperlink{after-bottom}{Continue reading the main story}

\hypertarget{site-index}{%
\subsection{Site Index}\label{site-index}}

\hypertarget{site-information-navigation}{%
\subsection{Site Information
Navigation}\label{site-information-navigation}}

\begin{itemize}
\tightlist
\item
  \href{https://help.nytimes.com/hc/en-us/articles/115014792127-Copyright-notice}{©~2020~The
  New York Times Company}
\end{itemize}

\begin{itemize}
\tightlist
\item
  \href{https://www.nytco.com/}{NYTCo}
\item
  \href{https://help.nytimes.com/hc/en-us/articles/115015385887-Contact-Us}{Contact
  Us}
\item
  \href{https://www.nytco.com/careers/}{Work with us}
\item
  \href{https://nytmediakit.com/}{Advertise}
\item
  \href{http://www.tbrandstudio.com/}{T Brand Studio}
\item
  \href{https://www.nytimes.com/privacy/cookie-policy\#how-do-i-manage-trackers}{Your
  Ad Choices}
\item
  \href{https://www.nytimes.com/privacy}{Privacy}
\item
  \href{https://help.nytimes.com/hc/en-us/articles/115014893428-Terms-of-service}{Terms
  of Service}
\item
  \href{https://help.nytimes.com/hc/en-us/articles/115014893968-Terms-of-sale}{Terms
  of Sale}
\item
  \href{https://spiderbites.nytimes.com}{Site Map}
\item
  \href{https://help.nytimes.com/hc/en-us}{Help}
\item
  \href{https://www.nytimes.com/subscription?campaignId=37WXW}{Subscriptions}
\end{itemize}
