Sections

SEARCH

\protect\hyperlink{site-content}{Skip to
content}\protect\hyperlink{site-index}{Skip to site index}

\href{https://www.nytimes.com/section/us}{U.S.}

\href{https://myaccount.nytimes.com/auth/login?response_type=cookie\&client_id=vi}{}

\href{https://www.nytimes.com/section/todayspaper}{Today's Paper}

\href{/section/us}{U.S.}\textbar{}The Class Divide: Remote Learning at 2
Schools, Private and Public

\url{https://nyti.ms/2LhkTEM}

\begin{itemize}
\item
\item
\item
\item
\item
\item
\end{itemize}

\href{https://www.nytimes.com/news-event/coronavirus?action=click\&pgtype=Article\&state=default\&region=TOP_BANNER\&context=storylines_menu}{The
Coronavirus Outbreak}

\begin{itemize}
\tightlist
\item
  live\href{https://www.nytimes.com/2020/08/04/world/coronavirus-cases.html?action=click\&pgtype=Article\&state=default\&region=TOP_BANNER\&context=storylines_menu}{Latest
  Updates}
\item
  \href{https://www.nytimes.com/interactive/2020/us/coronavirus-us-cases.html?action=click\&pgtype=Article\&state=default\&region=TOP_BANNER\&context=storylines_menu}{Maps
  and Cases}
\item
  \href{https://www.nytimes.com/interactive/2020/science/coronavirus-vaccine-tracker.html?action=click\&pgtype=Article\&state=default\&region=TOP_BANNER\&context=storylines_menu}{Vaccine
  Tracker}
\item
  \href{https://www.nytimes.com/2020/08/02/us/covid-college-reopening.html?action=click\&pgtype=Article\&state=default\&region=TOP_BANNER\&context=storylines_menu}{College
  Reopening}
\item
  \href{https://www.nytimes.com/live/2020/08/04/business/stock-market-today-coronavirus?action=click\&pgtype=Article\&state=default\&region=TOP_BANNER\&context=storylines_menu}{Economy}
\end{itemize}

Advertisement

\protect\hyperlink{after-top}{Continue reading the main story}

Supported by

\protect\hyperlink{after-sponsor}{Continue reading the main story}

\hypertarget{the-class-divide-remote-learning-at-2-schools-private-and-public}{%
\section{The Class Divide: Remote Learning at 2 Schools, Private and
Public}\label{the-class-divide-remote-learning-at-2-schools-private-and-public}}

Some private schools provide online luxury learning during the pandemic.
As many public schools struggle to adjust, the nation's educational gaps
widen.

\includegraphics{https://static01.nyt.com/images/2020/05/07/us/00VIRUS-FIRSTGRADE-warach/merlin_172284309_3b5173f2-87db-4d0b-b376-489a828f793e-articleLarge.jpg?quality=75\&auto=webp\&disable=upscale}

\href{https://www.nytimes.com/by/dana-goldstein}{\includegraphics{https://static01.nyt.com/images/2018/06/12/multimedia/author-dana-goldstein/author-dana-goldstein-thumbLarge.png}}

By \href{https://www.nytimes.com/by/dana-goldstein}{Dana Goldstein}

\begin{itemize}
\item
  Published May 9, 2020Updated June 5, 2020
\item
  \begin{itemize}
  \item
  \item
  \item
  \item
  \item
  \item
  \end{itemize}
\end{itemize}

For Rachel Warach's class, the 133rd morning of first grade, numbered on
a poster board behind her, was similar to all of the previous mornings.

Her students from across Chicago spent 15 minutes working quietly on
math problems and writing in their journals. They split into small
reading groups, with Ms. Warach bouncing between them to offer feedback.
Later, there was an Earth Day discussion of ``The Lorax'' and a math
lesson on sorting everyday objects --- rolls of tape, coins, pens ---
according to shape.

There was a break for lunch and recess, followed by Hebrew class. All as
Oisabel sprawled on the floor, Shira snuggled against her mom, and a
father whispered to his son, ``Can you take that blanket off your head,
please?''

This is first grade at a private school determined to make remote
education during the coronavirus as similar as possible to what it
looked like before the pandemic. Chicago Jewish Day School provides four
hours and 15 minutes of daily live instruction, including yoga, art and
music. Students even do messy baking projects over Zoom, with
\href{https://www.nytimes.com/2020/05/10/us/coronavirus-giving-birth-new-mothers.html}{parents}
as sous chefs.

It bears little resemblance to the more typical experience that Jacob
Rios is having in Philadelphia, where he attends first grade at a public
school, Spruance Elementary.

Jacob did not see his teacher via video screen until late April; the
district spent the first several weeks of the shutdown focused on
training staff members to use remote teaching tools, distributing
laptops to students and getting meals to low-income families, which make
up a majority of the district's population.

Now Jacob's teacher, Dolores Morris, meets with her students each
morning for an hour --- Jacob's only live video instruction, according
to his mother. About 11 of the 26 students in the class attend daily,
Ms. Morris said.

A close look at these two very different first-grade classes in two of
America's largest cities shows how
\href{https://www.nytimes.com/news-event/coronavirus}{the coronavirus
pandemic} has done nothing to level the playing field of American
education, and instead has widened the gaps that have always existed.

About 10 percent of American children attend private schools, not all of
which have been leaders in online education. And there are disparities
in the public system, too, where some schools have done much more than
others to get online instruction up and running effectively. But what
the pandemic has made clear is that remote education, especially of the
youngest students, requires a rare mix of enthusiastic school
leadership, teacher expertise and homes equipped with everything
children need to learn effectively.

\hypertarget{latest-updates-global-coronavirus-outbreak}{%
\section{\texorpdfstring{\href{https://www.nytimes.com/2020/08/04/world/coronavirus-cases.html?action=click\&pgtype=Article\&state=default\&region=MAIN_CONTENT_1\&context=storylines_live_updates}{Latest
Updates: Global Coronavirus
Outbreak}}{Latest Updates: Global Coronavirus Outbreak}}\label{latest-updates-global-coronavirus-outbreak}}

Updated 2020-08-04T20:42:41.838Z

\begin{itemize}
\tightlist
\item
  \href{https://www.nytimes.com/2020/08/04/world/coronavirus-cases.html?action=click\&pgtype=Article\&state=default\&region=MAIN_CONTENT_1\&context=storylines_live_updates\#link-1228a480}{Novavax
  sees encouraging results from two studies of its experimental
  vaccine.}
\item
  \href{https://www.nytimes.com/2020/08/04/world/coronavirus-cases.html?action=click\&pgtype=Article\&state=default\&region=MAIN_CONTENT_1\&context=storylines_live_updates\#link-4825b93}{Public
  and private schools in Maryland and elsewhere are divided over
  in-person instruction.}
\item
  \href{https://www.nytimes.com/2020/08/04/world/coronavirus-cases.html?action=click\&pgtype=Article\&state=default\&region=MAIN_CONTENT_1\&context=storylines_live_updates\#link-50f7386d}{The
  United Nations calls on policymakers to `plan thoroughly for school
  reopenings.'}
\end{itemize}

\href{https://www.nytimes.com/2020/08/04/world/coronavirus-cases.html?action=click\&pgtype=Article\&state=default\&region=MAIN_CONTENT_1\&context=storylines_live_updates}{See
more updates}

More live coverage:
\href{https://www.nytimes.com/live/2020/08/04/business/stock-market-today-coronavirus?action=click\&pgtype=Article\&state=default\&region=MAIN_CONTENT_1\&context=storylines_live_updates}{Markets}

At Chicago Jewish Day School, students who need extra help are being
tutored in phonics via Zoom, or meeting remotely with a social worker.
The school has sent home books, dry-erase boards, markers and other
needed supplies. Parents have provided the rest: internet access, iPads,
and quiet study nooks in well-appointed homes filled with pianos, books
and tasteful wooden play kitchens.

The system has been up and running since mid-March.

Remote learning at the Chicago school is not perfect. There are spotty
Wi-Fi connections, stray emojis in the chat panel and children who
wander away from the screen. But there is little doubt that in a nation
of over 100,000 shuttered schools, these children continue to receive a
luxury good --- one whose list price is \$28,000 per year.

In Ms. Morris's class in Philadelphia, Jacob is one of the more
fortunate students. His mother, Brenda Rios, sits by his side to help
him with assignments. She is off work from her usual part-time job
preparing meals at a preschool.

Because so many parents of the other students are essential workers ---
prison guards, cleaners, nursing assistants --- Ms. Morris knows they
may not be available to offer hands-on support. Still, she is trying to
look on the bright side.

``I'm thanking God that I can at least see their faces,'' she said.

\includegraphics{https://static01.nyt.com/images/2020/05/07/us/00VIRUS-FIRSTGRADE-morris/merlin_172255164_ab36055c-6068-4fe6-94ac-bfd8e995bd56-articleLarge.jpg?quality=75\&auto=webp\&disable=upscale}

That is rare in the world of coronavirus-altered learning. The Center on
Reinventing Public Education, a think tank, examined the remote learning
policies of 100 public school districts and charter networks nationwide.
It found that just 22 of them are requiring real-time teaching --- and
just 10 of those systems are teaching live in all grades, including
early elementary school.

The country's three largest districts, in New York City, Los Angeles and
Chicago, are not requiring teachers to do any live video instruction,
though some individual schools are choosing to do so.

It is a different story in many private schools, both independent and
parochial. Although associations said they did not have any hard data on
the average number of hours that students in their networks were
receiving live instruction, examples from around the country typically
show a gap with public schools.

The reasons are clear: Private school students are more likely to live
in homes with good internet access, computers and physical space for
children to focus on academics. Parents are less likely to be working
outside the home and are more available to guide young children through
getting online and staying logged in --- entering user names and
passwords, navigating between windows and programs.

And unlike their public-school counterparts, private schoolteachers are
generally not unionized, giving their employers more leverage in laying
out demands for remote work. Some public school unions
\href{https://www.nytimes.com/2020/04/21/us/coronavirus-teachers-unions-school-home.html}{have
won} strict limits on video-teaching requirements, arguing that it can
be difficult for educators to teach live from home when many are also
taking care of their own children, whose schools and day cares are also
closed.

In Philadelphia, Ms. Morris, a 42-year veteran, is in her last semester
before retirement, and it looks nothing like the farewell she expected.
Nevertheless, she has thrown herself into learning the technology to
teach remotely. Often, she is texting and emailing with parents while
simultaneously interacting with her students via Google Classroom.

A recent Monday morning was devoted to a phonics lesson on the sound
``oy.'' Ms. Morris used Google Classroom to display vocabulary words on
slides --- ``enjoy,'' ``soil,'' ``annoy'' --- and Jacob's mother, Ms.
Rios, helped him complete an online activity identifying the various
spellings of the sound.

Ms. Rios, home alone with three sons, said she appreciated Ms. Morris's
dedication to her students at a difficult time. Still, the transition
online had been rocky. At first, Ms. Rios was not sure how to operate
the district-provided Chromebook. Since then, much of the day's activity
has revolved around worksheets and compliance checks, which can be
maddening to submit online.

For one art lesson, Jacob watched a video about Vincent van Gogh, then
had to fill out an ``exit ticket,'' writing what he had learned about
the painter. Like any first grader, Jacob needed help to craft complete
sentences on the computer. Then, after submitting his answer, Ms. Rios
was required to click to another screen to report that he had finished
the activity.

\href{https://www.nytimes.com/news-event/coronavirus?action=click\&pgtype=Article\&state=default\&region=MAIN_CONTENT_3\&context=storylines_faq}{}

\hypertarget{the-coronavirus-outbreak-}{%
\subsubsection{The Coronavirus Outbreak
›}\label{the-coronavirus-outbreak-}}

\hypertarget{frequently-asked-questions}{%
\paragraph{Frequently Asked
Questions}\label{frequently-asked-questions}}

Updated August 4, 2020

\begin{itemize}
\item ~
  \hypertarget{i-have-antibodies-am-i-now-immune}{%
  \paragraph{I have antibodies. Am I now
  immune?}\label{i-have-antibodies-am-i-now-immune}}

  \begin{itemize}
  \tightlist
  \item
    As of right
    now,\href{https://www.nytimes.com/2020/07/22/health/covid-antibodies-herd-immunity.html?action=click\&pgtype=Article\&state=default\&region=MAIN_CONTENT_3\&context=storylines_faq}{that
    seems likely, for at least several months.} There have been
    frightening accounts of people suffering what seems to be a second
    bout of Covid-19. But experts say these patients may have a
    drawn-out course of infection, with the virus taking a slow toll
    weeks to months after initial exposure. People infected with the
    coronavirus typically
    \href{https://www.nature.com/articles/s41586-020-2456-9}{produce}
    immune molecules called antibodies, which are
    \href{https://www.nytimes.com/2020/05/07/health/coronavirus-antibody-prevalence.html?action=click\&pgtype=Article\&state=default\&region=MAIN_CONTENT_3\&context=storylines_faq}{protective
    proteins made in response to an
    infection}\href{https://www.nytimes.com/2020/05/07/health/coronavirus-antibody-prevalence.html?action=click\&pgtype=Article\&state=default\&region=MAIN_CONTENT_3\&context=storylines_faq}{.
    These antibodies may} last in the body
    \href{https://www.nature.com/articles/s41591-020-0965-6}{only two to
    three months}, which may seem worrisome, but that's perfectly normal
    after an acute infection subsides, said Dr. Michael Mina, an
    immunologist at Harvard University. It may be possible to get the
    coronavirus again, but it's highly unlikely that it would be
    possible in a short window of time from initial infection or make
    people sicker the second time.
  \end{itemize}
\item ~
  \hypertarget{im-a-small-business-owner-can-i-get-relief}{%
  \paragraph{I'm a small-business owner. Can I get
  relief?}\label{im-a-small-business-owner-can-i-get-relief}}

  \begin{itemize}
  \tightlist
  \item
    The
    \href{https://www.nytimes.com/article/small-business-loans-stimulus-grants-freelancers-coronavirus.html?action=click\&pgtype=Article\&state=default\&region=MAIN_CONTENT_3\&context=storylines_faq}{stimulus
    bills enacted in March} offer help for the millions of American
    small businesses. Those eligible for aid are businesses and
    nonprofit organizations with fewer than 500 workers, including sole
    proprietorships, independent contractors and freelancers. Some
    larger companies in some industries are also eligible. The help
    being offered, which is being managed by the Small Business
    Administration, includes the Paycheck Protection Program and the
    Economic Injury Disaster Loan program. But lots of folks have
    \href{https://www.nytimes.com/interactive/2020/05/07/business/small-business-loans-coronavirus.html?action=click\&pgtype=Article\&state=default\&region=MAIN_CONTENT_3\&context=storylines_faq}{not
    yet seen payouts.} Even those who have received help are confused:
    The rules are draconian, and some are stuck sitting on
    \href{https://www.nytimes.com/2020/05/02/business/economy/loans-coronavirus-small-business.html?action=click\&pgtype=Article\&state=default\&region=MAIN_CONTENT_3\&context=storylines_faq}{money
    they don't know how to use.} Many small-business owners are getting
    less than they expected or
    \href{https://www.nytimes.com/2020/06/10/business/Small-business-loans-ppp.html?action=click\&pgtype=Article\&state=default\&region=MAIN_CONTENT_3\&context=storylines_faq}{not
    hearing anything at all.}
  \end{itemize}
\item ~
  \hypertarget{what-are-my-rights-if-i-am-worried-about-going-back-to-work}{%
  \paragraph{What are my rights if I am worried about going back to
  work?}\label{what-are-my-rights-if-i-am-worried-about-going-back-to-work}}

  \begin{itemize}
  \tightlist
  \item
    Employers have to provide
    \href{https://www.osha.gov/SLTC/covid-19/standards.html}{a safe
    workplace} with policies that protect everyone equally.
    \href{https://www.nytimes.com/article/coronavirus-money-unemployment.html?action=click\&pgtype=Article\&state=default\&region=MAIN_CONTENT_3\&context=storylines_faq}{And
    if one of your co-workers tests positive for the coronavirus, the
    C.D.C.} has said that
    \href{https://www.cdc.gov/coronavirus/2019-ncov/community/guidance-business-response.html}{employers
    should tell their employees} -\/- without giving you the sick
    employee's name -\/- that they may have been exposed to the virus.
  \end{itemize}
\item ~
  \hypertarget{should-i-refinance-my-mortgage}{%
  \paragraph{Should I refinance my
  mortgage?}\label{should-i-refinance-my-mortgage}}

  \begin{itemize}
  \tightlist
  \item
    \href{https://www.nytimes.com/article/coronavirus-money-unemployment.html?action=click\&pgtype=Article\&state=default\&region=MAIN_CONTENT_3\&context=storylines_faq}{It
    could be a good idea,} because mortgage rates have
    \href{https://www.nytimes.com/2020/07/16/business/mortgage-rates-below-3-percent.html?action=click\&pgtype=Article\&state=default\&region=MAIN_CONTENT_3\&context=storylines_faq}{never
    been lower.} Refinancing requests have pushed mortgage applications
    to some of the highest levels since 2008, so be prepared to get in
    line. But defaults are also up, so if you're thinking about buying a
    home, be aware that some lenders have tightened their standards.
  \end{itemize}
\item ~
  \hypertarget{what-is-school-going-to-look-like-in-september}{%
  \paragraph{What is school going to look like in
  September?}\label{what-is-school-going-to-look-like-in-september}}

  \begin{itemize}
  \tightlist
  \item
    It is unlikely that many schools will return to a normal schedule
    this fall, requiring the grind of
    \href{https://www.nytimes.com/2020/06/05/us/coronavirus-education-lost-learning.html?action=click\&pgtype=Article\&state=default\&region=MAIN_CONTENT_3\&context=storylines_faq}{online
    learning},
    \href{https://www.nytimes.com/2020/05/29/us/coronavirus-child-care-centers.html?action=click\&pgtype=Article\&state=default\&region=MAIN_CONTENT_3\&context=storylines_faq}{makeshift
    child care} and
    \href{https://www.nytimes.com/2020/06/03/business/economy/coronavirus-working-women.html?action=click\&pgtype=Article\&state=default\&region=MAIN_CONTENT_3\&context=storylines_faq}{stunted
    workdays} to continue. California's two largest public school
    districts --- Los Angeles and San Diego --- said on July 13, that
    \href{https://www.nytimes.com/2020/07/13/us/lausd-san-diego-school-reopening.html?action=click\&pgtype=Article\&state=default\&region=MAIN_CONTENT_3\&context=storylines_faq}{instruction
    will be remote-only in the fall}, citing concerns that surging
    coronavirus infections in their areas pose too dire a risk for
    students and teachers. Together, the two districts enroll some
    825,000 students. They are the largest in the country so far to
    abandon plans for even a partial physical return to classrooms when
    they reopen in August. For other districts, the solution won't be an
    all-or-nothing approach.
    \href{https://bioethics.jhu.edu/research-and-outreach/projects/eschool-initiative/school-policy-tracker/}{Many
    systems}, including the nation's largest, New York City, are
    devising
    \href{https://www.nytimes.com/2020/06/26/us/coronavirus-schools-reopen-fall.html?action=click\&pgtype=Article\&state=default\&region=MAIN_CONTENT_3\&context=storylines_faq}{hybrid
    plans} that involve spending some days in classrooms and other days
    online. There's no national policy on this yet, so check with your
    municipal school system regularly to see what is happening in your
    community.
  \end{itemize}
\end{itemize}

Sometimes during live lessons, Ms. Rios can see via the video feed that
another child is confused --- they have not opened the right window or
clicked on the right link --- and does not have an adult nearby to help
them follow along.

``I was almost in tears today,'' she said. ``It's excruciating to watch
that --- a child who wants to learn and isn't able to.''

Ms. Morris, too, is frustrated by the limitations of online learning,
especially by the fact that she cannot always see students' reactions
while she is presenting material to them, to check that they understand.
She can tell that using the system is difficult for first graders,
because even some strong students are submitting blank assignments,
meaning they most likely did the work, but their answers did not get
recorded.

In Chicago, there are many reasons the Jewish Day School was able to
handle the transition to remote learning so well. The school closed for
students ahead of most others in Illinois. That allowed administrators
to spend several days, before the building shut down, training staff
members on how to use online tools.

The school's curriculum is based around hands-on activities and
discussion, which means young children learning from home do not need to
be as adept at typing as in schools that assign more structured, written
worksheets.

Image

Rachel Warach uses Zoom to discuss a reading assignment with her student
Roey.

And crucially, families in the school are generally stable economically
and available to closely supervise their children's education.

\href{https://www.nytimes.com/2020/04/28/us/coronavirus-schools-reopen.html}{Given
the possibility} that schools will remain at least partly closed in the
fall, Chicago Jewish Day School is now marketing itself as a leader in
remote learning, with a
\href{https://vimeo.com/409618723?fbclid=IwAR3GArK9URuzo5b3YMRZtVaP6hRp0eektfrxCQV-_INSp0g8oQpStZHn7YU}{slick
video aimed at parents}. School leaders hope to increase enrollment at a
time when requests for financial aid may go up as donations decrease
because of the economic downturn. Already, 57 percent of families at the
school receive some assistance with tuition.

Parents in Ms. Warach's class said they had been pleasantly surprised by
the effectiveness of online first grade.

Among them is Caroline Musin Berkowitz, a nonprofit manager, and her
husband, a legal analyst. They are both working from their apartment
while taking care of their two young children. Having 6-year-old Shira
engaged with school for most of the day, sitting across from her parents
at the dining room table with headphones on, provides some respite.

The family has no qualms about re-enrolling Shira in the fall, even
though they are not getting the exact experience they thought they were
paying for.

``We made a choice to go with private school over public school for so
many reasons,'' Ms. Musin Berkowitz said, ``and the idea of a global
pandemic and school moving to online was not one of them.''

Now, she added, ``I can't even describe how beneficial it's been.''

Advertisement

\protect\hyperlink{after-bottom}{Continue reading the main story}

\hypertarget{site-index}{%
\subsection{Site Index}\label{site-index}}

\hypertarget{site-information-navigation}{%
\subsection{Site Information
Navigation}\label{site-information-navigation}}

\begin{itemize}
\tightlist
\item
  \href{https://help.nytimes.com/hc/en-us/articles/115014792127-Copyright-notice}{©~2020~The
  New York Times Company}
\end{itemize}

\begin{itemize}
\tightlist
\item
  \href{https://www.nytco.com/}{NYTCo}
\item
  \href{https://help.nytimes.com/hc/en-us/articles/115015385887-Contact-Us}{Contact
  Us}
\item
  \href{https://www.nytco.com/careers/}{Work with us}
\item
  \href{https://nytmediakit.com/}{Advertise}
\item
  \href{http://www.tbrandstudio.com/}{T Brand Studio}
\item
  \href{https://www.nytimes.com/privacy/cookie-policy\#how-do-i-manage-trackers}{Your
  Ad Choices}
\item
  \href{https://www.nytimes.com/privacy}{Privacy}
\item
  \href{https://help.nytimes.com/hc/en-us/articles/115014893428-Terms-of-service}{Terms
  of Service}
\item
  \href{https://help.nytimes.com/hc/en-us/articles/115014893968-Terms-of-sale}{Terms
  of Sale}
\item
  \href{https://spiderbites.nytimes.com}{Site Map}
\item
  \href{https://help.nytimes.com/hc/en-us}{Help}
\item
  \href{https://www.nytimes.com/subscription?campaignId=37WXW}{Subscriptions}
\end{itemize}
