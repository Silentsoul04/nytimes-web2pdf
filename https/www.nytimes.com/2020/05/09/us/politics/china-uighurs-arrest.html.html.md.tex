Sections

SEARCH

\protect\hyperlink{site-content}{Skip to
content}\protect\hyperlink{site-index}{Skip to site index}

\href{https://www.nytimes.com/section/politics}{Politics}

\href{https://myaccount.nytimes.com/auth/login?response_type=cookie\&client_id=vi}{}

\href{https://www.nytimes.com/section/todayspaper}{Today's Paper}

\href{/section/politics}{Politics}\textbar{}Sister Fights to Free Uighur
Businessman Held in China After U.S. Trip

\url{https://nyti.ms/3bil4ur}

\begin{itemize}
\item
\item
\item
\item
\item
\end{itemize}

Advertisement

\protect\hyperlink{after-top}{Continue reading the main story}

Supported by

\protect\hyperlink{after-sponsor}{Continue reading the main story}

\hypertarget{sister-fights-to-free-uighur-businessman-held-in-china-after-us-trip}{%
\section{Sister Fights to Free Uighur Businessman Held in China After
U.S.
Trip}\label{sister-fights-to-free-uighur-businessman-held-in-china-after-us-trip}}

Ekpar Asat came to the United States for the State Department's most
prestigious program for foreign citizens. Chinese security officers
detained him weeks after he returned home.

\includegraphics{https://static01.nyt.com/images/2020/05/09/us/politics/09dc-uighurs1/merlin_172066014_1855d4ee-ae26-42b1-82eb-decb787a88b8-articleLarge.jpg?quality=75\&auto=webp\&disable=upscale}

\href{https://www.nytimes.com/by/edward-wong}{\includegraphics{https://static01.nyt.com/images/2018/09/24/multimedia/author-edward-wong/author-edward-wong-thumbLarge-v5.png}}

By \href{https://www.nytimes.com/by/edward-wong}{Edward Wong}

\begin{itemize}
\item
  Published May 9, 2020Updated July 6, 2020
\item
  \begin{itemize}
  \item
  \item
  \item
  \item
  \item
  \end{itemize}
\end{itemize}

\href{https://cn.nytimes.com/china/20200511/china-uighurs-arrest/}{阅读简体中文版}\href{https://cn.nytimes.com/china/20200511/china-uighurs-arrest/zh-hant/}{閱讀繁體中文版}

WASHINGTON --- When Ekpar Asat saw his older sister for the last time
one winter night in Manhattan, he promised her he would return to the
United States in a few months with their parents to watch her graduate
with a master's degree from Harvard Law School --- the first ethnic
\href{https://www.nytimes.com/2020/07/06/world/asia/china-xinjiang-uighur-court.html}{Uighur}
to do so.

But three weeks after returning to China from that trip in 2016, when he
was attending a prestigious \href{https://eca.state.gov/ivlp}{State
Department leadership training program}, he disappeared into the shadows
of a vast
\href{https://www.nytimes.com/2018/09/08/world/asia/china-uighur-muslim-detention-camp.html}{detention
system} in the country's northwest.

This winter, his sister, Rayhan Asat, heard that he had been sentenced
to 15 years in prison on suspicion of inciting ethnic hatred.

``It's so upsetting,'' she said. ``After he came to the U.S., he had an
elevated profile, and then he was labeled an enemy of the state.''

Ms. Asat, a lawyer and permanent American resident in Washington,
described her 34-year-old brother as a model citizen --- an entrepreneur
who founded a social media app for
\href{https://www.nytimes.com/2020/07/01/technology/china-uighurs-hackers-malware-hackers-smartphones.html}{Uighurs}
and took part in state-organized events. Their parents are members of
the Communist Party. The mother retired as a chemistry professor and the
father as a civil servant in water resources administration.

But Ms. Asat said the State Department project, the
\href{https://eca.state.gov/facesofexchange/}{International Visitor
Leadership Program}, which for 80 years has trained many prominent
foreign citizens, appears to have tainted her brother in the eyes of
Chinese security officers.

Though department officials have raised Mr. Asat's case with Chinese
counterparts, Ms. Asat is now calling for the agency to escalate its
efforts --- especially because he appears to have been detained and
convicted of crimes based on his association with the program, she said.
She stressed, though, that the weeks he had spent in the United States
had been a positive experience.

``I told the State Department this was the very reason that he was in
danger,'' Ms. Asat, 36, said. ``When the United States organizes these
programs and brings over these people, we have to protect them after the
fact so these programs can be successful.''

Morgan Ortagus, a State Department spokeswoman, said diplomats would
continue to raise Mr. Asat's case with the Chinese government.

``We call on Beijing to immediately release all those arbitrarily
detained, and to end its draconian policies which have forcibly
indoctrinated and intimidated its own citizens in Xinjiang,'' she added,
referring to China's northwest region, where most Uighurs live.

John Ullyot, a spokesman for the White House National Security Council,
said the Trump administration ``has called out the Chinese Communist
Party for its treatment of millions of Uighurs like Ekpar Asat whom it
arbitrarily detains, indoctrinates and forces into labor to erase their
ethnic identity and religious beliefs.''

\includegraphics{https://static01.nyt.com/images/2020/05/09/us/politics/09dc-uighurs2/merlin_172066023_f7fc2dd0-18b2-4733-b74d-35dfd792009d-articleLarge.jpg?quality=75\&auto=webp\&disable=upscale}

The State Department mentioned Mr. Asat in its
\href{https://www.state.gov/wp-content/uploads/2020/03/CHINA-INCLUSIVE-2019-HUMAN-RIGHTS-REPORT.pdf}{2019
human rights report} on China, which said he apparently had been
detained in 2016 ``after participating in a program in the United
States'' and then was sentenced to 15 years in prison. The report, which
identifies Mr. Asat by his official Chinese name, Aikebaier Aisaiti,
also described the ``mass arbitrary detention'' of
\href{https://www.nytimes.com/2019/02/21/business/china-xinjiang-uighur-dna-thermo-fisher.html}{Muslims}
in Xinjiang.

The Uighurs are a Turkic-speaking Muslim group that
\href{https://www.nytimes.com/2009/07/12/weekinreview/12wong.html}{calls
Xinjiang their homeland}. The Chinese Communist Party says it is
\href{https://www.nytimes.com/2019/12/30/world/asia/china-xinjiang-muslims-labor.html}{clamping
down} on separatism and religious extremism in the region, but most
Uighurs suffering from the
\href{https://www.nytimes.com/2019/05/22/world/asia/china-surveillance-xinjiang.html}{hard-line
anti-Muslim policies} are
\href{https://www.nytimes.com/2019/12/28/world/asia/china-xinjiang-children-boarding-schools.html}{innocent
citizens}.

\href{https://duihua.org/team_member/john-kamm/}{John Kamm}, a human
rights advocate in California who has advised Ms. Asat, said the State
Department should make public cases of persecution that might be linked
to its programs.

``I think it's simply prudent,'' he said. ``I'm not calling for
canceling the program.''

The Chinese Foreign Ministry in Beijing said it had no information on
Mr. Asat, and the Xinjiang regional government did not answer submitted
questions.

Mr. Asat was detained around the time that officials in Xinjiang were
\href{https://www.nytimes.com/interactive/2019/11/16/world/asia/china-xinjiang-documents.html}{building
up an internment camp system} that has held at least one million Muslims
over four years. Since 2018, Secretary of State Mike Pompeo
\href{https://www.nytimes.com/2019/05/05/us/politics/pompeo-china-sanctions.html}{has
denounced the camps}.

The fact that Mr. Asat ran a social media app with about 100,000 users
might also have made Chinese officials suspicious, since they are wary
of
\href{https://www.nytimes.com/2019/01/10/business/china-twitter-censorship-online.html}{online
mass communications}. But Ms. Asat said her brother was careful to abide
by censorship rules.

Ms. Asat said she learned of her brother's fate in January. The Chinese
Embassy in Washington sent an email to the office of
\href{https://www.coons.senate.gov/}{Senator Chris Coons}, Democrat of
Delaware, telling them about Mr. Asat's 15-year sentence. That was in
response to
\href{https://int.nyt.com/data/documenthelper/6938-senators-ekpar-asat/b23f1159370b5d1ad5a8/optimized/full.pdf\#page=1}{a
bipartisan letter} that the senator and six other lawmakers, including
\href{https://www.young.senate.gov/}{Senator Todd Young}, Republican of
Indiana, sent
\href{https://www.nytimes.com/2020/04/05/opinion/coronavirus-china-us.html}{Ambassador
Cui Tiankai} in December urging China to release Mr. Asat.

Mr. Coons said in a statement that he was ``deeply concerned'' about the
repression in Xinjiang. ``China must be held accountable for this gross
abuse of human rights against Aikebaier and other innocent people,'' he
added.

Senator Mitch McConnell, the majority leader, said this week that the
Senate would
\href{https://www.reuters.com/article/us-usa-china-xinjiang/senate-leader-expects-vote-soon-on-seeking-china-sanctions-over-uighurs-idUSKBN22J3E4?il=0}{move
soon} to pass a bipartisan bill, the
\href{https://www.nytimes.com/2019/12/27/us/politics/trumps-human-rights-congress.html}{Uighur
Human Rights Policy Act}, that would compel President Trump to take
tough measures against China on Xinjiang, including imposing sanctions
on some officials. Tensions between Washington and Beijing have risen
sharply during the pandemic, which began in central China.

Ms. Asat said the accusation of ethnic hatred was absurd. She pointed to
state-run news reports that featured her brother's business or his
participation in government-sponsored events. ``They always said he was
a bridge builder between the ethnic minorities and the majority Han,''
she said.

Image

A sculpture of Mao Zedong and a Uighur man in the Xinjiang region of
China. The area is under a heavy security presence.Credit...Giulia
Marchi for The New York Times

In 2014, Mr. Asat was among a small group of businesspeople selected to
meet with Max Baucus, then the American ambassador to China, who had
traveled to Xinjiang.

The United States Embassy in Beijing then encouraged Mr. Asat to apply
for the State Department's leadership program, Ms. Asat said. After
being accepted, he flew to the United States in February 2016, where he
completed a three-week program for journalists that introduced him to
institutions and figures across Washington and five states. There were
eight other participants.

``He was by far the most curious of the group; he was constantly asking
questions,'' said Sheridan Bell, a former diplomat and senior program
manager at \href{https://www.meridian.org/}{Meridian International
Center}, a nonprofit group that runs the program for the State
Department.

``He focused his questions on the line between doing what your bosses
want you to do and telling the truth --- the narrow line between doing
your job as a journalist and propaganda,'' he added. ``He was very
cautious. I can't imagine he said things that would have gotten him into
trouble.''

The State Department's website says the program has
\href{https://eca.state.gov/highlight/launching-80th-anniversary-international-visitor-leadership-program}{trained
more than 225,000 people} since its inception in 1940.
\href{https://eca.state.gov/facesofexchange/}{Prominent alumni} include
Margaret Thatcher, the former prime minister of Britain, and
\href{https://www.nytimes.com/2019/03/22/world/australia/jacinda-ardern-new-zealand-leader.html}{Jacinda
Ardern}, the current prime minister of New Zealand. The courses focus on
different industries and professions.

Mr. Bell said Chinese participants are often affiliated with the
Communist Party or the government. One alumnus of a U.S. foreign policy
program is
\href{https://www.nytimes.com/2020/02/28/opinion/coronavirus-china-government.html}{Xie
Feng}, now the Foreign Ministry's commissioner in Hong Kong. There have
been occasional ethnic minority participants, including Tibetans.

Mr. Asat was effusive about the program. While in Washington, he wrote
on WeChat, the Chinese social media app: ``It's an incredible honor to
participate in the I.V.L.P. I hope to gain profound insight into
American culture and media!''

He posted photographs of his travels. He posed with Americans to whom he
had given traditional Uighur hats and scarves. He sat in a Formula One
racecar in Florida, attended an N.B.A. game in Indiana and posed next to
the CNN logo at the network's headquarters in Georgia.

Image

Secretary of State Mike Pompeo has condemned China's actions towards
Uighurs, but Ms. Asat says more needs to be done on her brother's
case.Credit...Pool photo by Andrew Harnik

Ms. Asat, who was studying for her master of laws at Harvard after
having lived outside China since 2009, flew to Washington for a job
interview while her brother was there. The two had dinner at Clyde's of
Georgetown, then went for a stroll by the shops along M Street.

She saw him once more afterward, for a few hours in New York. He told
her that since he now had a visa to enter the United States, he would
return in May with their parents for her graduation.

``He was inspired,'' she said. ``He wanted to spend the summer here and
learn more English. He liked the tech landscape and wanted to learn more
about how to innovate his company.''

Ms. Asat said she felt guilty for not having spent more time with her
brother. ``There is so much regret,'' she said.

After Mr. Asat returned to their home city of Urumqi, nothing seemed
unusual at first. He helped sponsor an international boxing tournament
that was covered by China Central Television, the state-run network. He
stood in the ring next to a famous Uighur boxer holding up a trophy.

Then Ms. Asat stopped hearing from him. No new messages appeared on his
WeChat account after April 7.

Her parents told her they would not attend her graduation.

During what should have been a high point in her life --- graduating
from Harvard --- Ms. Asat was distraught and in tears.

She soon began working in the Washington office of a Wall Street law
firm and talking to American officials. She noticed her brother's
employees were still posting news items to his app, Bagdax, as were
public users of the app, until April 2017, which suggests its existence
was not the reason for his detention.

As
\href{https://www.nytimes.com/2019/12/07/world/europe/uighur-whistleblower.html}{public
reports} have emerged of the mass detentions in Xinjiang, Ms. Asat has
tried quietly to get the State Department, the National Security Council
and congressional offices to press harder for her brother's freedom.

She sometimes speaks briefly with family members in Xinjiang, but it is
all small talk. They avoid discussing her brother.

In early March, two months after finding out about his sentencing, she
decided to speak publicly at Harvard Law School about his case and the
crisis in Xinjiang.

``I know from trying to help Rayhan Asat over the past four years that
she is a person of real courage and integrity,'' said
\href{https://hls.harvard.edu/faculty/directory/10010/Alford}{William P.
Alford}, the vice dean and law professor who hosted the talk. ``The case
of her brother, arrested right after returning from the U.S. and largely
shrouded by Chinese authorities, is tragic.''

Ms. Asat said she was aware her family might suffer reprisals as a
result of her speaking out. That
\href{https://www.nytimes.com/2018/10/18/world/asia/uighur-muslims-china-detainment.html}{has
happened to other Uighurs} abroad. But she said her brother's ordeal had
made her realize that no matter what she and her family do to conform as
model citizens, the Chinese government sees Uighurs ``forever as
outsiders.''

``When you grow up in China, you're not supposed to engage in
politics,'' she said. ``But China is pushing me to become political.''

Advertisement

\protect\hyperlink{after-bottom}{Continue reading the main story}

\hypertarget{site-index}{%
\subsection{Site Index}\label{site-index}}

\hypertarget{site-information-navigation}{%
\subsection{Site Information
Navigation}\label{site-information-navigation}}

\begin{itemize}
\tightlist
\item
  \href{https://help.nytimes.com/hc/en-us/articles/115014792127-Copyright-notice}{©~2020~The
  New York Times Company}
\end{itemize}

\begin{itemize}
\tightlist
\item
  \href{https://www.nytco.com/}{NYTCo}
\item
  \href{https://help.nytimes.com/hc/en-us/articles/115015385887-Contact-Us}{Contact
  Us}
\item
  \href{https://www.nytco.com/careers/}{Work with us}
\item
  \href{https://nytmediakit.com/}{Advertise}
\item
  \href{http://www.tbrandstudio.com/}{T Brand Studio}
\item
  \href{https://www.nytimes.com/privacy/cookie-policy\#how-do-i-manage-trackers}{Your
  Ad Choices}
\item
  \href{https://www.nytimes.com/privacy}{Privacy}
\item
  \href{https://help.nytimes.com/hc/en-us/articles/115014893428-Terms-of-service}{Terms
  of Service}
\item
  \href{https://help.nytimes.com/hc/en-us/articles/115014893968-Terms-of-sale}{Terms
  of Sale}
\item
  \href{https://spiderbites.nytimes.com}{Site Map}
\item
  \href{https://help.nytimes.com/hc/en-us}{Help}
\item
  \href{https://www.nytimes.com/subscription?campaignId=37WXW}{Subscriptions}
\end{itemize}
