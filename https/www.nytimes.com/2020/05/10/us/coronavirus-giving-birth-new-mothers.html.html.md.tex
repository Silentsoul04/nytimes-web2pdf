Sections

SEARCH

\protect\hyperlink{site-content}{Skip to
content}\protect\hyperlink{site-index}{Skip to site index}

\href{https://www.nytimes.com/section/us}{U.S.}

\href{https://myaccount.nytimes.com/auth/login?response_type=cookie\&client_id=vi}{}

\href{https://www.nytimes.com/section/todayspaper}{Today's Paper}

\href{/section/us}{U.S.}\textbar{}Pregnant in a Pandemic? These 5
Mothers of Newborns Have Advice

\url{https://nyti.ms/2xTJ2yq}

\begin{itemize}
\item
\item
\item
\item
\item
\item
\end{itemize}

\href{https://www.nytimes.com/spotlight/at-home?action=click\&pgtype=Article\&state=default\&region=TOP_BANNER\&context=at_home_menu}{At
Home}

\begin{itemize}
\tightlist
\item
  \href{https://www.nytimes.com/2020/07/28/books/time-for-a-literary-road-trip.html?action=click\&pgtype=Article\&state=default\&region=TOP_BANNER\&context=at_home_menu}{Take:
  A Literary Road Trip}
\item
  \href{https://www.nytimes.com/2020/07/29/magazine/bored-with-your-home-cooking-some-smoky-eggplant-will-fix-that.html?action=click\&pgtype=Article\&state=default\&region=TOP_BANNER\&context=at_home_menu}{Cook:
  Smoky Eggplant}
\item
  \href{https://www.nytimes.com/2020/07/27/travel/moose-michigan-isle-royale.html?action=click\&pgtype=Article\&state=default\&region=TOP_BANNER\&context=at_home_menu}{Look
  Out: For Moose}
\item
  \href{https://www.nytimes.com/interactive/2020/at-home/even-more-reporters-editors-diaries-lists-recommendations.html?action=click\&pgtype=Article\&state=default\&region=TOP_BANNER\&context=at_home_menu}{Explore:
  Reporters' Obsessions}
\end{itemize}

Advertisement

\protect\hyperlink{after-top}{Continue reading the main story}

Supported by

\protect\hyperlink{after-sponsor}{Continue reading the main story}

\hypertarget{pregnant-in-a-pandemic-these-5-mothers-of-newborns-have-advice}{%
\section{Pregnant in a Pandemic? These 5 Mothers of Newborns Have
Advice}\label{pregnant-in-a-pandemic-these-5-mothers-of-newborns-have-advice}}

Try to laugh, trust your own judgment and concentrate on your baby.

\includegraphics{https://static01.nyt.com/images/2020/05/07/nyregion/nyregionspecial/poonam-c-section/poonam-c-section-videoSixteenByNineJumbo1600.jpg}

\href{https://www.nytimes.com/by/shreeya-sinha}{\includegraphics{https://static01.nyt.com/images/2018/07/12/multimedia/author-shreeya-sinha/author-shreeya-sinha-thumbLarge.png}}

By \href{https://www.nytimes.com/by/shreeya-sinha}{Shreeya Sinha}

\begin{itemize}
\item
  Published May 10, 2020Updated May 13, 2020
\item
  \begin{itemize}
  \item
  \item
  \item
  \item
  \item
  \item
  \end{itemize}
\end{itemize}

Before the coronavirus, expectant mothers relied on support from family
and friends to help them recover from painful deliveries and adapt to
unpredictable life with a newborn. Now many parents have to go it alone
and make gut-wrenching decisions:

Is a
\href{https://www.nytimes.com/2020/03/30/parenting/home-birth-coronavirus-hospital.html}{home
birth less risky}? Who will take care of the kids if your partner is
with you at the hospital? Should you allow visitors to help with the
laundry, or struggle to do it yourself to limit your exposure to others?

The New York Times asked parents to tell us how they are coping. We
heard from more than 800 readers in at least two dozen states and five
countries. Here are some of their stories and advice for new mothers,
edited for length and clarity.

\hypertarget{there-are-some-silver-linings-for-somebody-who-is-a-single-mom}{%
\subsection{`There are some silver linings for somebody who is a single
mom.'}\label{there-are-some-silver-linings-for-somebody-who-is-a-single-mom}}

\emph{Emily, 37, a manager at a Maryland state office, delivered her
first child, Romero, at home. She asked that her full name not be used.}

I had planned a home birth since I was five minutes pregnant. When I
first started telling people I was planning this, before the pandemic, I
was getting comments like, ``Are you sure you are safe?'' Now people are
asking, ``Can I get the contact information?''

I set up an aerial yoga hammock in what was once my dining room. I set
up my space that I've been calling the ``birth gym.'' I was in labor
with him for two days and 17 hours. There was a point in labor when I
ran out of contractions and we had the conversation: Should we go to the
hospital? At that point if they saw me and the baby's position, they
would most likely give me a C-section. The midwives gave me homeopathic
stuff, and then they coached me with my mom through two hours of squats.
I had a very deep level of trust in the midwives because I went through
hours of preparation and training.

I felt the risk of exposure from going to hospital would've been higher.
Because of the way that the hospital takes precautions, I could only
have one person with me, excluding the midwives. I was worried about
being separated from my baby.

My partner, he's sort of a friendly sperm donor. He's in a high-risk
profession, so he's not safe to see the baby. I have to weigh the risk
of someone coming in and being exposed versus hemorrhaging because I
carried too much laundry downstairs versus the risk of the baby getting
an infection because the laundry didn't get done because I couldn't
carry it.

There are some silver linings for somebody who is a single mom because
all of a sudden everyone needs accommodations. The possibility that I
could telework for longer, and then be able to breastfeed and not have
to put him in day care too young, has increased.

\includegraphics{https://static01.nyt.com/images/2020/05/08/us/00VIRUS-BIRTHING/00VIRUS-BIRTHING-articleLarge.jpg?quality=75\&auto=webp\&disable=upscale}

\hypertarget{advice-to-mothers}{%
\subsubsection{\texorpdfstring{\textbf{Advice to
mothers:}}{Advice to mothers:}}\label{advice-to-mothers}}

You got to trust your own judgment as a mom. As a parent, you are the
most in tune to what the best decision for your kid is.

\hypertarget{im-grieving-the-postpartum-that-i-could-have-had}{%
\subsection{`I'm grieving the postpartum that I could have
had.'}\label{im-grieving-the-postpartum-that-i-could-have-had}}

\emph{Carly Buxton, 35, a market researcher in Virginia, delivered
Callum, her second child, at a hospital.}

I became a postpartum doula after the birth of my first child, and I had
such high hopes for how I wanted to call on my village of support this
time around. But instead, it's all FaceTime kisses and waves through
glass porch doors.

I'm grieving the postpartum that I could have had. I'm grieving the fact
that I could've had time with my husband while my daughter was at
preschool. He's barely able to hold the baby because he's running around
with a toddler. **** I'm grieving my parents who have only met the new
baby three times. On top of that, I have guilt because our situation is
so much better than others. We are able to order and pay for groceries.
There's all this stuff I should be thankful for, but not being able to
be thankful for it makes it worse. There's this persistent fear that
we're not in the clear. We mothers who are delivering babies right now
--- this is history-making.

Image

Carly Buxton delivered Callum, her second child.

\hypertarget{advice-to-mothers-1}{%
\subsubsection{\texorpdfstring{\textbf{Advice to
mothers:}}{Advice to mothers:}}\label{advice-to-mothers-1}}

Connect to people you can trust who are knowledgeable and skilled in
this field who can walk you through your decision-making process. People
are scrambling right now to build virtual support networks. Be open to
that kind of stuff. It doesn't feel the same. Try to laugh.

\hypertarget{i-didnt-really-get-to-feel-my-babys-skin-or-kiss-her-until-we-were-discharged}{%
\subsection{`I didn't really get to feel my baby's skin or kiss her
until we were
discharged.'}\label{i-didnt-really-get-to-feel-my-babys-skin-or-kiss-her-until-we-were-discharged}}

\emph{Danielle Galiano, 33, an I.T. consultant in New Jersey, delivered
Céleste}, \emph{her second child, at a hospital.}

There was anxiety, and we had to make hard decisions. Each step of the
way our birth plan started to change. We couldn't bring our daughter
with us, so we had to have our parents come help. Our doctor told us
given Covid-19, a C-section would be more optimal, so we could plan for
all doctors to be there. It was a bit surreal giving birth with a mask
and surgical gear. I didn't really get to feel my baby's skin or kiss
her until we were discharged.

I went once a week to doctor's appointments with masks --- people were
being tested in their cars outside.

We've tried to stock up as much as we can, but it's always a lingering
concern. For example, I'm breastfeeding and I needed a nipple shield.
Thank God we found it on Target. We buy certain things in advance, but
we are also
\href{https://www.nytimes.com/2020/03/30/parenting/coronavirus-baby-formula-shortages-wipes-diapers.html}{trying
not to stockpile} because it adds to the issues everyone is facing.

Image

Danielle Galiano delivered her second child, Céleste.

\hypertarget{advice-to-mothers-2}{%
\subsubsection{\texorpdfstring{\textbf{Advice to
mothers:}}{Advice to mothers:}}\label{advice-to-mothers-2}}

I'm lucky I have a supportive loving partner doing this with me. If you
really sit and think, it can get overwhelming. Try to concentrate on
your baby and doing the best you can.

\hypertarget{it-is-akin-i-imagine-to-having-a-baby-in-post-apocalyptic-times}{%
\subsection{`It is akin, I imagine, to having a baby in post-apocalyptic
times.'}\label{it-is-akin-i-imagine-to-having-a-baby-in-post-apocalyptic-times}}

\emph{Lindsey Gordon, 33, a former bar manager in New Orleans, delivered
her first child, Atlas, at a hospital.}

My doctor insisted I be induced on my due date because she was worried
about waiting any longer with the current pandemic scare. I was only
allowed one person with me throughout the entire birthing process. I
conceded to be induced entirely against my wishes but in times like
these, you do what you have to do. I was calm until the baby arrived ---
seemingly healthy and perfect. Now I am a pile of nerves and fear. No
one is allowed to see our baby. It is heartbreaking that his
grandparents won't get to hold him for the foreseeable future. It is
akin, I imagine, to having a baby in post-apocalyptic times. Everything
is surreal, scary and weird --- searching for a sense of normalcy is a
futile endeavor.

My post-op appointment at two weeks was a video call. I'm showing my
C-section scar in my phone camera. I cannot imagine having medical
issues post-op right now. I had to buy an infant scale to avoid having
to go back in, because he's got weight issues.

There's only so much you can do remotely. It was a terrible problem with
the whole breastfeeding thing. I found it exponentially more difficult
than the C-section. I've tried several times to get in touch with my
lactation specialist to no avail.

Image

Lindsey Gordon delivered Atlas, her first child.

\hypertarget{advice-to-mothers-3}{%
\subsubsection{\texorpdfstring{\textbf{Advice to
mothers:}}{Advice to mothers:}}\label{advice-to-mothers-3}}

Do the best you can. Be as strong as you can. It's hard. Try to focus on
health and the baby and yourself --- that's really all that matters.

\hypertarget{no-one-has-met-her-no-ones-going-to-meet-her}{%
\subsection{`No one has met her. No one's going to meet
her.'}\label{no-one-has-met-her-no-ones-going-to-meet-her}}

\emph{Emily Fazio, 33, a teacher in Illinois, delivered Josie, her
second child, at a hospital.}

I had no idea what my nurses actually looked like. They were all in
P.P.E., We joked, ``Oh, I'll run into you in the grocery store, and I
won't know that was you.'' There were no visitors. My husband was able
to come in with me. We showed up to the E.R. in the middle of night.
They screened us outside the building to make sure we were asymptomatic.
It was a lot of precautions. I felt safe. We were discharged after 24
hours, which they normally don't do.

No one has met her. No one's going to meet her. My grandparents looked
at the baby in the driveway and left, so that we didn't potentially
expose them after we'd been in the hospital.

These last few weeks have been very difficult, but life with a newborn
(and a toddler) are always difficult. Is it more difficult because we
haven't had anyone to come hold the baby and play with the toddler so we
can shower or nap? Of course. But with some hindsight and some memories
of what it was like after our first was born, this period is difficult
no matter what, and I don't know that any changed circumstances would
make it much easier.

Postpartum recovery is not pretty. For the first couple of days, Maggie,
the toddler, was sad I couldn't lift her up --- it physically hurt.
Between that and the baby, she was feeling rejected. It's hard to say,
``No, Mama can't pick you up. Ask Daddy right now.'' We're just doing
that right now and without help.

You go through a lot of toilet paper postpartum, too because you go to
the bathroom, at least if you have a vaginal delivery. I'm supposed to
be giving vitamin D drops, but don't have them because of delays in
procuring.

Image

Emily Fazio delivered Josie, her second child.~

\hypertarget{advice-to-mothers-4}{%
\subsubsection{\texorpdfstring{\textbf{Advice to
mothers:}}{Advice to mothers:}}\label{advice-to-mothers-4}}

The most important thing is that your babies feel love. So, even on days
that are tough, when I lose my patience, when I'm so tired I'm just
running on steam, when nothing on my to-do list has been done, I've been
trying to ground myself in that: My babies know that I love them. That's
what matters most. I'm just doing the best I can.

Advertisement

\protect\hyperlink{after-bottom}{Continue reading the main story}

\hypertarget{site-index}{%
\subsection{Site Index}\label{site-index}}

\hypertarget{site-information-navigation}{%
\subsection{Site Information
Navigation}\label{site-information-navigation}}

\begin{itemize}
\tightlist
\item
  \href{https://help.nytimes.com/hc/en-us/articles/115014792127-Copyright-notice}{©~2020~The
  New York Times Company}
\end{itemize}

\begin{itemize}
\tightlist
\item
  \href{https://www.nytco.com/}{NYTCo}
\item
  \href{https://help.nytimes.com/hc/en-us/articles/115015385887-Contact-Us}{Contact
  Us}
\item
  \href{https://www.nytco.com/careers/}{Work with us}
\item
  \href{https://nytmediakit.com/}{Advertise}
\item
  \href{http://www.tbrandstudio.com/}{T Brand Studio}
\item
  \href{https://www.nytimes.com/privacy/cookie-policy\#how-do-i-manage-trackers}{Your
  Ad Choices}
\item
  \href{https://www.nytimes.com/privacy}{Privacy}
\item
  \href{https://help.nytimes.com/hc/en-us/articles/115014893428-Terms-of-service}{Terms
  of Service}
\item
  \href{https://help.nytimes.com/hc/en-us/articles/115014893968-Terms-of-sale}{Terms
  of Sale}
\item
  \href{https://spiderbites.nytimes.com}{Site Map}
\item
  \href{https://help.nytimes.com/hc/en-us}{Help}
\item
  \href{https://www.nytimes.com/subscription?campaignId=37WXW}{Subscriptions}
\end{itemize}
