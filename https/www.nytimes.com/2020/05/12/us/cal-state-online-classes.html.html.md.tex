Sections

SEARCH

\protect\hyperlink{site-content}{Skip to
content}\protect\hyperlink{site-index}{Skip to site index}

\href{https://www.nytimes.com/section/us}{U.S.}

\href{https://myaccount.nytimes.com/auth/login?response_type=cookie\&client_id=vi}{}

\href{https://www.nytimes.com/section/todayspaper}{Today's Paper}

\href{/section/us}{U.S.}\textbar{}Fearing a Second Wave, Cal State Will
Keep Classes Online in the Fall

\href{https://nyti.ms/2LkKj4A}{https://nyti.ms/2LkKj4A}

\begin{itemize}
\item
\item
\item
\item
\item
\item
\end{itemize}

\href{https://www.nytimes.com/news-event/coronavirus?action=click\&pgtype=Article\&state=default\&region=TOP_BANNER\&context=storylines_menu}{The
Coronavirus Outbreak}

\begin{itemize}
\tightlist
\item
  live\href{https://www.nytimes.com/2020/08/08/world/coronavirus-updates.html?action=click\&pgtype=Article\&state=default\&region=TOP_BANNER\&context=storylines_menu}{Latest
  Updates}
\item
  \href{https://www.nytimes.com/interactive/2020/us/coronavirus-us-cases.html?action=click\&pgtype=Article\&state=default\&region=TOP_BANNER\&context=storylines_menu}{Maps
  and Cases}
\item
  \href{https://www.nytimes.com/interactive/2020/science/coronavirus-vaccine-tracker.html?action=click\&pgtype=Article\&state=default\&region=TOP_BANNER\&context=storylines_menu}{Vaccine
  Tracker}
\item
  \href{https://www.nytimes.com/interactive/2020/world/coronavirus-tips-advice.html?action=click\&pgtype=Article\&state=default\&region=TOP_BANNER\&context=storylines_menu}{F.A.Q.}
\item
  \href{https://www.nytimes.com/live/2020/08/07/business/stock-market-today-coronavirus?action=click\&pgtype=Article\&state=default\&region=TOP_BANNER\&context=storylines_menu}{Markets
  \& Economy}
\end{itemize}

Advertisement

\protect\hyperlink{after-top}{Continue reading the main story}

Supported by

\protect\hyperlink{after-sponsor}{Continue reading the main story}

\hypertarget{fearing-a-second-wave-cal-state-will-keep-classes-online-in-the-fall}{%
\section{Fearing a Second Wave, Cal State Will Keep Classes Online in
the
Fall}\label{fearing-a-second-wave-cal-state-will-keep-classes-online-in-the-fall}}

The move by the nation's largest four-year public university system
comes as many other schools insist they will find a way to bring
students back to campus despite the coronavirus.

\includegraphics{https://static01.nyt.com/images/2020/05/12/us/12VIRUS-CALSTATE/12VIRUS-CALSTATE-articleLarge.jpg?quality=75\&auto=webp\&disable=upscale}

By Shawn Hubler

\begin{itemize}
\item
  May 12, 2020
\item
  \begin{itemize}
  \item
  \item
  \item
  \item
  \item
  \item
  \end{itemize}
\end{itemize}

SACRAMENTO --- In the most sweeping sign yet of the long-term impact of
the coronavirus on American higher education, California State
University, the nation's largest four-year public university system,
said on Tuesday that classes at its 23 campuses would be canceled for
the fall semester, with instruction taking place almost exclusively
online.

The system is the first large American university to tell students they
will not be returning to campus in the fall. Most of the nation's
colleges and universities have gone out of their way to say they intend
to reopen, but they are also making backup plans for online classes.

The pandemic has had a
\href{https://www.nytimes.com/2020/04/15/us/coronavirus-colleges-universities-admissions.html}{devastating
impact on the finances} of colleges and universities, a large number of
which were already struggling before virus-related closures. Many are
concerned about growing signs that a large number of students
\href{https://www.nytimes.com/2020/05/01/us/coronavirus-college-enrollment.html}{will
choose to sit out} the fall semester if classes remain virtual, or
demand hefty cuts in tuition.

A
\href{https://www.nytimes.com/2020/03/25/us/politics/colleges-universities-coronavirus.html}{\$14
billion federal bailout} passed by Congress this spring will not be
enough to save some universities if enrollment drops significantly,
experts said, and for many students, the in-person experience is a
significant part of higher education's draw.

But the chancellor of the California State University system, Timothy P.
White,
\href{https://www2.calstate.edu/csu-system/board-of-trustees/livestream/Pages/livestream.aspx}{told
the board of trustees} on Tuesday that the risks were too great for the
more than 480,000 undergraduates enrolled at the Cal States, as they are
known, to return to campus in the fall. Classes will continue virtually,
as they have since March.

``Our university, when open without restrictions and fully in person, as
is the traditional norm of the past, is a place where over 500,000
people come together in close and vibrant proximity with each other on a
daily basis,'' he said. ``That approach, sadly, just isn't in the cards
now.''

McGill University in Montreal, one of Canada's most prestigious
universities,
\href{https://www.nytimes.com/2020/05/11/world/coronavirus-news.html\#link-1cc52594}{made
a similar announcement on Monday}, saying it will offer most of its
courses online in September.

Mr. White allowed for the possibility of exceptions. If health and
safety precautions permit, clinical classes in the nursing program could
be held in person, he said, as could certain science labs and other
essential instruction.

\hypertarget{the-coronavirus-outbreak}{%
\subsubsection{The Coronavirus
Outbreak}\label{the-coronavirus-outbreak}}

\hypertarget{back-to-school}{%
\paragraph{Back to School}\label{back-to-school}}

Updated Aug. 8, 2020

The latest highlights as the first students return to U.S. schools.

\begin{itemize}
\item
  \begin{itemize}
  \tightlist
  \item
    Health experts say New York State schools are
    \href{https://www.nytimes.com/2020/08/07/health/coronavirus-ny-schools-reopen.html?action=click\&pgtype=Article\&state=default\&region=MAIN_CONTENT_2\&context=storylines_keepup}{in
    a good position to reopen}, and Gov. Andrew M. Cuomo has
    \href{https://www.nytimes.com/2020/08/07/nyregion/cuomo-schools-reopening.html?action=click\&pgtype=Article\&state=default\&region=MAIN_CONTENT_2\&context=storylines_keepup}{cleared
    the way}.
  \item
    Many schools spent the summer focused on reopening classrooms. What
    if they had
    \href{https://www.nytimes.com/2020/08/07/us/remote-learning-fall-2020.html?action=click\&pgtype=Article\&state=default\&region=MAIN_CONTENT_2\&context=storylines_keepup}{focused
    on improving remote learning} instead?
  \item
    A mother in Germany describes how her family
    \href{https://www.nytimes.com/2020/08/07/parenting/germany-schools-reopening-children.html?action=click\&pgtype=Article\&state=default\&region=MAIN_CONTENT_2\&context=storylines_keepup}{coped
    with the anxiety and uncertainty} of going back to school there.
  \item
    A high school freshman tested positive after two days in class. A
    yearbook editor worries about access to sporting events. We spoke to
    students about
    \href{https://www.nytimes.com/2020/08/06/us/coronavirus-students.html?action=click\&pgtype=Article\&state=default\&region=MAIN_CONTENT_2\&context=storylines_keepup}{what
    school is like in the age of Covid-19.}
  \end{itemize}
\end{itemize}

Experts said Cal State's decision could have a significant impact.

``Cal State is an extraordinarily large and important university system
and an awful lot of other institutions will watch this development
carefully,'' said Terry W. Hartle, senior vice president of the American
Council on Education, a trade association of college presidents.

The Chronicle of Higher Education has been
\href{https://www.chronicle.com/article/Here-s-a-List-of-Colleges-/248626}{keeping
a running tally} of what American colleges are planning to do for the
fall. Only a handful of schools, mostly small ones, have said they are
leaning toward online-only classes, including Wayne State University in
Detroit, a virus hot spot, and Sierra College outside Sacramento. A few
say they are planning a hybrid model. But the vast majority say they are
planning for in-person classes.

Brown University's president, Christina Paxson,
\href{https://www.nytimes.com/2020/04/26/opinion/coronavirus-colleges-universities.html}{said
in a New York Times Op-Ed} late last month that reopening campuses this
fall ``should be a national priority.''

Size, location and population density could play a big role in what
universities decide, Mr. Hartle said.

On Monday, Bradley University in Peoria, Ill., population 111,000,
announced its ``commitment to resume on-campus classes'' for about 5,000
students in the fall, saying, ``The midsize of Bradley and the
small-city setting of Peoria make it easier for students to maintain
safe distances and avoid unnecessary exposure to potentially dangerous
germs.''

California's other four-year university system, the University of
California, with nearly 300,000 students on 10 campuses, has not
announced whether its fall classes will be held online, in-person or a
mix. But faculty members there say plans are being drawn up for all
three contingencies. The Board of Regents is expected to discuss
systemwide plans at a meeting next week.

Cal State's announcement came as Dr. Anthony S. Fauci, the nation's top
infectious disease expert,
\href{https://www.nytimes.com/2020/05/12/us/politics/fauci-cdc-coronavirus-senate-testimony.html?action=click\&module=Spotlight\&pgtype=Homepage}{told
a Senate panel} on Tuesday that needless ``suffering and death'' could
result if states move too quickly to reopen schools and businesses.

Mr. White, the Cal State chancellor, noted that academic researchers and
public health experts were predicting a ``second, smaller wave'' of the
coronavirus this summer, ``followed by a very significant wave'' in the
fall and another wave in the first quarter of next year.

With no vaccine on the immediate horizon, Mr. White told the system's
trustees that ``it would be irresponsible'' to postpone a decision on
in-person classes until summer, only to be forced to retreat hastily to
remote learning in the fall.

Better, he said, to plan for the worst and hope for the best in
September.

``This,'' he said, ``is our new and expensive reality.''

Anemona Hartocollis contributed reporting from New York.

Advertisement

\protect\hyperlink{after-bottom}{Continue reading the main story}

\hypertarget{site-index}{%
\subsection{Site Index}\label{site-index}}

\hypertarget{site-information-navigation}{%
\subsection{Site Information
Navigation}\label{site-information-navigation}}

\begin{itemize}
\tightlist
\item
  \href{https://help.nytimes.com/hc/en-us/articles/115014792127-Copyright-notice}{©~2020~The
  New York Times Company}
\end{itemize}

\begin{itemize}
\tightlist
\item
  \href{https://www.nytco.com/}{NYTCo}
\item
  \href{https://help.nytimes.com/hc/en-us/articles/115015385887-Contact-Us}{Contact
  Us}
\item
  \href{https://www.nytco.com/careers/}{Work with us}
\item
  \href{https://nytmediakit.com/}{Advertise}
\item
  \href{http://www.tbrandstudio.com/}{T Brand Studio}
\item
  \href{https://www.nytimes.com/privacy/cookie-policy\#how-do-i-manage-trackers}{Your
  Ad Choices}
\item
  \href{https://www.nytimes.com/privacy}{Privacy}
\item
  \href{https://help.nytimes.com/hc/en-us/articles/115014893428-Terms-of-service}{Terms
  of Service}
\item
  \href{https://help.nytimes.com/hc/en-us/articles/115014893968-Terms-of-sale}{Terms
  of Sale}
\item
  \href{https://spiderbites.nytimes.com}{Site Map}
\item
  \href{https://help.nytimes.com/hc/en-us}{Help}
\item
  \href{https://www.nytimes.com/subscription?campaignId=37WXW}{Subscriptions}
\end{itemize}
