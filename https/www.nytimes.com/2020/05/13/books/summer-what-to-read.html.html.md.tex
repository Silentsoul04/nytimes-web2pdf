\href{/section/books}{Books}\textbar{}Let Books Create Your Summer

\url{https://nyti.ms/2WrmEWl}

\begin{itemize}
\item
\item
\item
\item
\item
\end{itemize}

\href{https://www.nytimes.com/spotlight/at-home?action=click\&pgtype=Article\&state=default\&region=TOP_BANNER\&context=at_home_menu}{At
Home}

\begin{itemize}
\tightlist
\item
  \href{https://www.nytimes.com/2020/07/28/books/time-for-a-literary-road-trip.html?action=click\&pgtype=Article\&state=default\&region=TOP_BANNER\&context=at_home_menu}{Take:
  A Literary Road Trip}
\item
  \href{https://www.nytimes.com/2020/07/29/magazine/bored-with-your-home-cooking-some-smoky-eggplant-will-fix-that.html?action=click\&pgtype=Article\&state=default\&region=TOP_BANNER\&context=at_home_menu}{Cook:
  Smoky Eggplant}
\item
  \href{https://www.nytimes.com/2020/07/27/travel/moose-michigan-isle-royale.html?action=click\&pgtype=Article\&state=default\&region=TOP_BANNER\&context=at_home_menu}{Look
  Out: For Moose}
\item
  \href{https://www.nytimes.com/interactive/2020/at-home/even-more-reporters-editors-diaries-lists-recommendations.html?action=click\&pgtype=Article\&state=default\&region=TOP_BANNER\&context=at_home_menu}{Explore:
  Reporters' Obsessions}
\end{itemize}

\includegraphics{https://static01.nyt.com/images/2020/05/22/arts/22Summer-Books/22Summer-Books-articleLarge.jpg?quality=75\&auto=webp\&disable=upscale}

Sections

\protect\hyperlink{site-content}{Skip to
content}\protect\hyperlink{site-index}{Skip to site index}

\hypertarget{let-books-create-your-summer}{%
\section{Let Books Create Your
Summer}\label{let-books-create-your-summer}}

This season's reading will have to do more than usual to help us get
away from it all while we can't get away.

Credit...Eric Petersen

Supported by

\protect\hyperlink{after-sponsor}{Continue reading the main story}

\href{https://www.nytimes.com/by/sarah-lyall}{\includegraphics{https://static01.nyt.com/images/2018/02/20/multimedia/author-sarah-lyall/author-sarah-lyall-thumbLarge.jpg}}

By \href{https://www.nytimes.com/by/sarah-lyall}{Sarah Lyall}

\begin{itemize}
\item
  May 13, 2020
\item
  \begin{itemize}
  \item
  \item
  \item
  \item
  \item
  \end{itemize}
\end{itemize}

``The Daughter of Time'' (1951), by the Scottish mystery writer
Josephine Tey, begins with its detective laid up in the hospital with a
broken leg, isolated and dispirited. But he pulls himself together. From
his bed, he conducts a thrilling literary investigation into one of the
great historical mysteries of the ages: whether Richard III really
killed his nephews, the poor doomed princes in the tower. It's a
particularly satisfying response to an unexpected confinement.

Here in 2020, as the spring of our anxiety gives way to the summer of
our discontent, we face a similar predicament. We can't go far. Our
worlds feel muffled, sad, small, lonely, scary, boring. We're caring for
older relatives, looking after young children, stranded in the city,
yearning for places we can't visit and people we can't see.

Our relationship with books feels different, too. The long and lazy days
of summer are usually a time to relax and soften into the pleasures of
reading, ideally someplace near the water, a cool drink in your hand,
the sun radiating overhead, nothing to do but enjoy yourself. But when
pleasure is scarce, when the outdoors is a luxury, when we feel stuck in
an endless frightening present --- when summer isn't really summer ---
what does it mean for summer reading?

It may be tempting to binge-watch our way through these next months. But
TV washes over you. Reading draws you in. Books that absorb us, books
that calm us down, books that comfort us, books that remind us we are
not alone but part of the grand sweep of history, books that surprise
and enchant us --- this is what we're looking for. Maybe this literary
summer will mean reading a succession of fiendish thrillers; or maybe it
will mean finally tackling Trollope. Whatever works. We are making this
up as we go along.

Many years ago, when I was little, my family spent the summer in a
rented house on a tiny Greek island. Left alone to a near-criminal
degree, my brother and I ran wild. We got lost; we were wounded by sea
urchin spines. That sort of freedom feels distant now, irrelevant to our
current circumstance. But we were also so bored. (No internet, no TV,
dearth of other kids.) We loved to read, and when we finished the books
we'd brought with us, we foraged.

\includegraphics{https://static01.nyt.com/images/2020/05/13/books/13lyall-summer-reading-combo2/13lyall-summer-reading-combo-articleLarge.jpg?quality=75\&auto=webp\&disable=upscale}

People leave weird reading material in their rental houses. So if you're
lucky enough to be in someone else's bucolic cottage this summer, be
sure to investigate the shelves. In Greece, a lot of the books on those
shelves were in Greek. But, in English, we found an old stack of
mysteries by Ngaio Marsh, the excitingly named New Zealand author. I
also read Robert K. Massie's ``Nicholas and Alexandra'' (1967), which
stood out not for the rigor of its scholarship or the sweep of its story
but for the thrilling (and mostly unintelligible; I was only 7)
descriptions of blood disorders, murderous intent and people who take a
long time to die.

Even if we can't go anywhere right now, we can still find serendipitous
books at home, left behind by someone else or stashed, forgotten, in a
cupboard. Their moment has come.

This summer is a great time to fall into fictional worlds within other
fictional worlds, embodiments of the novel's ability to tantalize and
liberate our imaginations. The youngish-adult novel ``Born to Trot''
(1950), by Marguerite Henry, is also written from the perspective of a
person in bed; it's a further reminder that physical confinement is no
barrier to imaginative freedom. A young man falls ill with an
unspecified something and is sent to a sanitarium for a long rest. He is
rescued by a book --- the tale of the great American harness horse
Hambletonian --- and the second book is cunningly there within the
first, embedded inside the story.

Image

For other pleasingly nested books, try Arthur Phillips's ``The Tragedy
of Arthur'' (2011), which weaves into the main story a newly discovered
play that might be a lost Shakespeare work; or Anthony Horowitz's
whodunit-within-a-whodunit ``Magpie Murders'' (2016), the tale of an
editor who is reading a manuscript, the entirety of which is there for
us to read, too; or Italo Calvino's classic second-person love letter to
fiction, ``If on a Winter's Night a Traveler'' (1979).

Magic and fantasy are not for everyone, but oh, they are for me right
now, when I long to step through the window into another world. There's
nothing like a magical book from your childhood to transport you, and I
adore the stories of Edward Eager from the 1950s and 1960s, with their
bookish protagonists and their cross-referential magical adventures,
homages to the British writer E. Nesbit. But grown-ups have their own
versions in the books of Neil Gaiman, China Miéville, Susanna Clarke
(read
\href{https://www.nytimes.com/2004/09/05/books/review/jonathan-strange-mr-norrell39-hogwarts-for-grownups.html}{``Jonathan
Strange \& Mr. Norrell,''} for starters), Lev Grossman's ``Magicians''
series or the latest by Leigh Bardugo,
\href{https://www.nytimes.com/2019/10/03/books/leigh-bardugo-ninth-house.html}{``Ninth
House,''} set at a version of Yale teeming with magic. For science
fiction that pierces your heart, read Ted Chiang.

Image

Sometimes lately it feels that a good day is one in which you
successfully change out of your pajamas. But even now there is the
tantalizing prospect of incremental self-improvement.

This summer's project might be the Russian novels you never got to
before, or the first four volumes of Robert Caro's magisterial biography
of Lyndon Johnson, or ``How Proust Can Change Your Life'' (1997), in
which Alain de Botton reads Proust so you don't have to.

Maybe you want to learn how not to kill the person you love. For that,
there's Esther Perel's ``Mating in Captivity'' (2006), the thesis of
which --- that it is fiendishly hard to reconcile erotic excitement with
too \emph{much} togetherness --- is being played out by millions of
couples across the world right now.

No matter who we're living with, we'd love to get away. One of my
favorite moments in C.S. Lewis's Narnia series comes in ``The Voyage of
the Dawn Treader'' (1952), when the characters see a painting of a great
ship, begin to smell and hear and feel the sea, and find themselves
suddenly transported to that ship and that sea. Some books are perfect
for reminding us what that feels like, the sights and scents and freedom
of the outdoors. John Vaillant's nonfiction work ``The Golden Spruce''
(2005) is a missing-person mystery, a history of logging and its effect
on the environment, and a stunning evocation of the beauty and wildness
of the great forests and bodies of water in British Columbia. ``Swimming
Home'' (2012), by Deborah Levy, and ``Call Me by Your Name'' (2007), by
Andre Aciman, make you feel that you, too, are on a sexy European
vacation by a pool. (With remarkably different results.)

Image

Credit...

Finally, if you find yourself worrying, as many of us are, about the
capacity of ordinary people to withstand emotional hardship, there's
nothing like the quiet beauty of Marilynne Robinson's great series ---
start with ``Gilead'' (2004) and keep going --- to restore your faith in
the complicated goodness of the human heart. A fourth book in the series
is coming this fall, giving us something to look forward to and assuring
us that even when times are hard, good books will always welcome us in.

Advertisement

\protect\hyperlink{after-bottom}{Continue reading the main story}

\hypertarget{site-index}{%
\subsection{Site Index}\label{site-index}}

\hypertarget{site-information-navigation}{%
\subsection{Site Information
Navigation}\label{site-information-navigation}}

\begin{itemize}
\tightlist
\item
  \href{https://help.nytimes.com/hc/en-us/articles/115014792127-Copyright-notice}{©~2020~The
  New York Times Company}
\end{itemize}

\begin{itemize}
\tightlist
\item
  \href{https://www.nytco.com/}{NYTCo}
\item
  \href{https://help.nytimes.com/hc/en-us/articles/115015385887-Contact-Us}{Contact
  Us}
\item
  \href{https://www.nytco.com/careers/}{Work with us}
\item
  \href{https://nytmediakit.com/}{Advertise}
\item
  \href{http://www.tbrandstudio.com/}{T Brand Studio}
\item
  \href{https://www.nytimes.com/privacy/cookie-policy\#how-do-i-manage-trackers}{Your
  Ad Choices}
\item
  \href{https://www.nytimes.com/privacy}{Privacy}
\item
  \href{https://help.nytimes.com/hc/en-us/articles/115014893428-Terms-of-service}{Terms
  of Service}
\item
  \href{https://help.nytimes.com/hc/en-us/articles/115014893968-Terms-of-sale}{Terms
  of Sale}
\item
  \href{https://spiderbites.nytimes.com}{Site Map}
\item
  \href{https://help.nytimes.com/hc/en-us}{Help}
\item
  \href{https://www.nytimes.com/subscription?campaignId=37WXW}{Subscriptions}
\end{itemize}
