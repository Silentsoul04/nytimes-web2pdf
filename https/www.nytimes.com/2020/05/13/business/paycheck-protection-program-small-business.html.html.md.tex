Sections

SEARCH

\protect\hyperlink{site-content}{Skip to
content}\protect\hyperlink{site-index}{Skip to site index}

\href{https://www.nytimes.com/section/business}{Business}

\href{https://myaccount.nytimes.com/auth/login?response_type=cookie\&client_id=vi}{}

\href{https://www.nytimes.com/section/todayspaper}{Today's Paper}

\href{/section/business}{Business}\textbar{}Senators Question Lower
Relief Loan Amounts for Small Businesses

\url{https://nyti.ms/2WsB3S9}

\begin{itemize}
\item
\item
\item
\item
\item
\end{itemize}

Advertisement

\protect\hyperlink{after-top}{Continue reading the main story}

Supported by

\protect\hyperlink{after-sponsor}{Continue reading the main story}

\hypertarget{senators-question-lower-relief-loan-amounts-for-small-businesses}{%
\section{Senators Question Lower Relief Loan Amounts for Small
Businesses}\label{senators-question-lower-relief-loan-amounts-for-small-businesses}}

After learning that some received less than they asked for under the
Paycheck Protection Program, three Democrats want more scrutiny of the
banks.

\href{https://www.nytimes.com/by/emily-flitter}{\includegraphics{https://static01.nyt.com/images/2019/06/19/reader-center/author-emily-flitter/author-emily-flitter-thumbLarge.png}}

By \href{https://www.nytimes.com/by/emily-flitter}{Emily Flitter}

\begin{itemize}
\item
  Published May 13, 2020Updated June 10, 2020
\item
  \begin{itemize}
  \item
  \item
  \item
  \item
  \item
  \end{itemize}
\end{itemize}

Three Senate Democrats want Trump administration officials to keep a
closer eye on the banks handing out aid to small businesses after some
companies said they had received less than they expected, without
explanation.

In a letter to Treasury Secretary Steven Mnuchin and Small Business
Administrator Jovita Carranza on Wednesday, three members of the Senate
Banking Committee asked for more oversight of banks making loans under
the
\href{https://www.nytimes.com/2020/06/10/business/Small-business-loans-ppp.html}{Paycheck
Protection Program}. The program is designed to provide forgivable loans
that help small businesses pay their employees and some of their
overhead during the coronavirus lockdowns.

But some business owners told The New York Times that they
\href{https://www.nytimes.com/2020/05/11/business/coronavirus-aid-banks.html}{had
not received as much money} as they had asked for, and said they had
been told that the decision was made by the banks --- not the Small
Business Administration, which is funding the program.

``Whether inadvertent or intentional, this troubling report warrants a
response from your agencies,'' the senators, Sherrod Brown of Ohio,
Robert Menendez of New Jersey and Tina Smith of Minnesota, wrote in the
letter.

The issue poses a significant problem for borrowers. Treasury rules say
that for Paycheck Protection Program loans to be forgiven, businesses
must spend the money in precise ways, including rehiring a certain
portion of their employees. If the amount they receive isn't enough to
rehire enough employees, their loans cannot be forgiven.

The senators asked Mr. Mnuchin and Ms. Carranza to offer those borrowers
options for getting additional funds to cover the gap. They also asked
whether the officials had a plan for identifying which borrowers had not
gotten as much money as they needed and determining the reason for the
discrepancy.

It was not clear how many small businesses received less money than they
asked for under the program, but borrowers seeking loans from Citizens
Financial Group and Bank of America reported experiencing that problem.

At least seven borrowers posted about the issue on a page on Bank of
America's website that its customers can use as a discussion forum. The
Times spoke with four additional borrowers experiencing the problem. A
Bank of America representative advised users on the website to contact
the bank for help, and a bank spokesman told The Times that the bank was
working to correct any problems that borrowers reported.

The senators said small businesses needed all the help they could get
while competing against larger, wealthier clients for banks' attention.
``Answers to these questions will help ensure that eligible small
businesses are not further disadvantaged by lenders shorting them on
P.P.P. loans,'' they wrote.

Advertisement

\protect\hyperlink{after-bottom}{Continue reading the main story}

\hypertarget{site-index}{%
\subsection{Site Index}\label{site-index}}

\hypertarget{site-information-navigation}{%
\subsection{Site Information
Navigation}\label{site-information-navigation}}

\begin{itemize}
\tightlist
\item
  \href{https://help.nytimes.com/hc/en-us/articles/115014792127-Copyright-notice}{©~2020~The
  New York Times Company}
\end{itemize}

\begin{itemize}
\tightlist
\item
  \href{https://www.nytco.com/}{NYTCo}
\item
  \href{https://help.nytimes.com/hc/en-us/articles/115015385887-Contact-Us}{Contact
  Us}
\item
  \href{https://www.nytco.com/careers/}{Work with us}
\item
  \href{https://nytmediakit.com/}{Advertise}
\item
  \href{http://www.tbrandstudio.com/}{T Brand Studio}
\item
  \href{https://www.nytimes.com/privacy/cookie-policy\#how-do-i-manage-trackers}{Your
  Ad Choices}
\item
  \href{https://www.nytimes.com/privacy}{Privacy}
\item
  \href{https://help.nytimes.com/hc/en-us/articles/115014893428-Terms-of-service}{Terms
  of Service}
\item
  \href{https://help.nytimes.com/hc/en-us/articles/115014893968-Terms-of-sale}{Terms
  of Sale}
\item
  \href{https://spiderbites.nytimes.com}{Site Map}
\item
  \href{https://help.nytimes.com/hc/en-us}{Help}
\item
  \href{https://www.nytimes.com/subscription?campaignId=37WXW}{Subscriptions}
\end{itemize}
