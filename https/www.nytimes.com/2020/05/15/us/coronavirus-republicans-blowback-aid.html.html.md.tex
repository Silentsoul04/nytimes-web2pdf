Sections

SEARCH

\protect\hyperlink{site-content}{Skip to
content}\protect\hyperlink{site-index}{Skip to site index}

\href{https://www.nytimes.com/section/us}{U.S.}

\href{https://myaccount.nytimes.com/auth/login?response_type=cookie\&client_id=vi}{}

\href{https://www.nytimes.com/section/todayspaper}{Today's Paper}

\href{/section/us}{U.S.}\textbar{}With Go-Slow Approach, Republicans
Risk Political Blowback on Pandemic Aid

\href{https://nyti.ms/3bFCYaI}{https://nyti.ms/3bFCYaI}

\begin{itemize}
\item
\item
\item
\item
\item
\item
\end{itemize}

Advertisement

\protect\hyperlink{after-top}{Continue reading the main story}

Supported by

\protect\hyperlink{after-sponsor}{Continue reading the main story}

News Analysis

\hypertarget{with-go-slow-approach-republicans-risk-political-blowback-on-pandemic-aid}{%
\section{With Go-Slow Approach, Republicans Risk Political Blowback on
Pandemic
Aid}\label{with-go-slow-approach-republicans-risk-political-blowback-on-pandemic-aid}}

With jobless numbers rising, Republicans' reluctance to negotiate on
another round of pandemic relief is proving difficult to sustain.

\includegraphics{https://static01.nyt.com/images/2020/05/14/us/politics/14dc-assess/14dc-assess-articleLarge.jpg?quality=75\&auto=webp\&disable=upscale}

\href{https://www.nytimes.com/by/carl-hulse}{\includegraphics{https://static01.nyt.com/images/2018/06/14/multimedia/author-carl-hulse/author-carl-hulse-thumbLarge.png}}

By \href{https://www.nytimes.com/by/carl-hulse}{Carl Hulse}

\begin{itemize}
\item
  May 15, 2020
\item
  \begin{itemize}
  \item
  \item
  \item
  \item
  \item
  \item
  \end{itemize}
\end{itemize}

WASHINGTON --- More than eight weeks and almost \$2.8 trillion federal
dollars into an urgent response to the coronavirus pandemic,
congressional Republicans and the Trump administration have made it
clear that they are in no rush to engage with Democrats on another round
of costly relief measures.

But their resistance --- born of spending fatigue and policy divisions
--- is proving increasingly unsustainable, given tens of millions of
anxious Americans out of work, businesses and schools shuttered and an
election looming.

Even as the House moved forward on Friday with a Democratic recovery
measure that Republicans abhor, President Trump and party leaders
offered new assurances that they would draft their own legislation at
some point, reflecting their growing unease at being portrayed as
hostile to providing additional federal help.

``Phase 4 is going to happen,'' Mr. Trump told reporters at the White
House, using the shorthand for the next round of coronavirus aid, just
minutes after saying it ``could'' happen. ``It's going to happen in a
much better way for the American people.'' Only last week, he said he
was in ``no rush'' to take up such a bill.

His pivot followed an unusual one by Senator Mitch McConnell, Republican
of Kentucky and the majority leader, who after saying on Monday that he
felt no urgency to provide more immediate help, told Fox News on
Thursday that another round of recovery legislation was likely.

Republicans' reluctance to commit to more relief measures has provided
an opening for Democrats, who have been pounding Senate Republicans ---
particularly endangered incumbents facing the voters in November --- for
their stance.

They have seized in particular on Mr. McConnell's brushoff --- ``I don't
think we have yet felt the urgency of acting immediately,'' he said ---
as a serious blunder by the usually disciplined Mr. McConnell.

``I'd urge the constituents of senators in every state to call them and
ask them that question: Do you agree with Senator McConnell?'' Senator
Chuck Schumer of New York, the Democratic leader, said Thursday on the
Senate floor.

Members of both parties concede that the \$3 trillion measure that
Democrats were speeding through the House is several bridges too far,
considering its giant cost and the underpinning of progressive policies
on immigration and other issues that could never clear the
Republican-controlled Senate. Republicans branded it an outlandish
liberal wish list. Most have rejected it outright, and very few were
expected to back it in the vote set for Friday --- conveniently
scheduled by Democrats to coincide with more gloomy unemployment news.

But this week demonstrated the difficulty of maintaining that stance.
Jobless claims soared to 36.5 million over two months, and Jerome H.
Powell, the chair of the Federal Reserve,
\href{https://www.nytimes.com/2020/05/13/business/economy/fed-chair-powell-economy-virus-support.html}{warned
on Wednesday} that Congress must be prepared to enact more fiscal
stimulus to avoid long-term economic damage.

Representative Peter T. King of New York, a Republican who said he would
support Democrats' \$3 trillion aid bill despite his opposition to many
of its provisions, said the dire economic situation of so many states
and cities meant Congress had to start somewhere.

``States are going to go under,'' Mr. King said. ``You've got to get
negotiations going. You can't stand in each other's corners yelling back
and forth.''

Yet that seems to be exactly what is happening, as the pandemic
continues to exact its toll. Almost since the
\href{https://www.nytimes.com/2020/04/21/us/politics/congress-business-relief-ppp.html}{last
rescue package was approved} in late April, Mr. McConnell has been
insistent that it was time for Congress to ``pause'' and evaluate how
the trillions in spending already pushed out the door were working
before allocating more.

Mr. McConnell was not happy with how the previous negotiations unfolded,
with Democrats successfully holding out for concessions and the White
House, represented by Treasury Secretary Steven Mnuchin, readily
acquiescing. He made it clear he would not go down that road again.

He began the week sticking to that position, telling reporters on
Tuesday that he would not so much as ``interact'' with Democrats until
Senate Republicans and the White House had reached a joint decision on
whether to move ahead, and what should be included in any package.

But by Thursday, he opened the door slightly to another round of
legislation, even as he left the timing uncertain.

``We do anticipate having to act again at some point,'' he said in the
interview with Fox News, saying that he agreed with Mr. Powell. ``I am
certainly not ruling out another fiscal package.''

Ms. Pelosi was not waiting, pushing forward her House bill put together
solely by Democrats, while noting the call by the Federal Reserve chair
for aggressive intervention by Congress to shore up the economy. In the
absence of a Republican negotiating partner, she said, ``our
conversation is with the American people.''

She also warned that opposing the legislation could have political
consequences.

``If you vote against this and all this funding for your state, then you
have to go home and defend it,'' she said Thursday evening during a call
with the caucus, according to multiple people familiar with the remarks.
``And if you can defend that `no' vote, then you're a better politician
than me.''

Earlier Thursday, Mr. McConnell shredded the House legislation on the
Senate floor as an ``1,800-page seasonal catalog of left-wing oddities''
and a ``totally unserious effort.'' He also said that Democrats ``cannot
stop salivating over the possibilities for partisan gain.''

Whether they are salivating or not, it is true that Democrats see the
opportunity to score political points if Republicans stick to their
standpat stance on pandemic relief. Democrats are trying to position
themselves as the alternatives to go-slow Republicans, portraying theirs
as the party racing to the rescue of suffering and cash-short Americans,
as evidenced by the support former Vice President Joseph R. Biden Jr.
gave this week for rent and mortgage forgiveness for those struggling to
pay.

At the same time, Democratic state parties in Arizona, Georgia and North
Carolina, among others, are hammering Republican incumbents, pressing
them on whether they side with Mr. McConnell or unemployed constituents
being thrown off their health insurance. One broadside aimed at Senator
Thom Tillis in North Carolina was typical.

``Does Tillis Agree With McConnell That There's No `Urgency' to Act as
New Reports Show Millions Without Health Insurance and a Looming `Long,
Painful' Downturn?'' asked a statement distributed by the North Carolina
Democratic Party.

Republicans say they believe they can counter the Democratic drumbeat by
focusing on how over-the-top the House proposal is, with its plethora of
liberal policies --- and by emphasizing that the party's
\href{https://www.nytimes.com/2020/05/14/us/politics/coronavirus-workers-income-lawmakers.html}{most
progressive wing still was not satisfied}.

Republicans are also putting forward rescue packages of their own,
though none have been embraced by either the party leadership or the
White House. It was notable that Senator Cory Gardner of Colorado,
perhaps his party's most endangered incumbent, on Thursday got behind a
proposal by Senator Josh Hawley, Republican of Missouri, that would have
the federal government underwrite minimum payroll for workers and
provide new grants to struggling businesses.

Some Republicans say the absence of a negotiation is in itself a
negotiating tactic as the two sides maneuver for position. Many see more
legislation as inevitable given the immense needs and the intensifying
political pressure and say the foundation for an agreement exists, with
business aid, payroll guarantees, help for state and local governments
and some form of liability protections as building blocks.

``There is a deal to be had here,'' said Representative Tom Cole of
Oklahoma, a senior Republican on the Appropriations Committee and a
veteran legislative negotiator. ``It is just not going to be this deal,
and it is not going to be rushed as quickly as the Democrats would like
to do it.''

In the meantime, the parties remain in their corners shouting, but
Democrats believe the clamor will ultimately force Republicans to come
to the table.

Advertisement

\protect\hyperlink{after-bottom}{Continue reading the main story}

\hypertarget{site-index}{%
\subsection{Site Index}\label{site-index}}

\hypertarget{site-information-navigation}{%
\subsection{Site Information
Navigation}\label{site-information-navigation}}

\begin{itemize}
\tightlist
\item
  \href{https://help.nytimes.com/hc/en-us/articles/115014792127-Copyright-notice}{©~2020~The
  New York Times Company}
\end{itemize}

\begin{itemize}
\tightlist
\item
  \href{https://www.nytco.com/}{NYTCo}
\item
  \href{https://help.nytimes.com/hc/en-us/articles/115015385887-Contact-Us}{Contact
  Us}
\item
  \href{https://www.nytco.com/careers/}{Work with us}
\item
  \href{https://nytmediakit.com/}{Advertise}
\item
  \href{http://www.tbrandstudio.com/}{T Brand Studio}
\item
  \href{https://www.nytimes.com/privacy/cookie-policy\#how-do-i-manage-trackers}{Your
  Ad Choices}
\item
  \href{https://www.nytimes.com/privacy}{Privacy}
\item
  \href{https://help.nytimes.com/hc/en-us/articles/115014893428-Terms-of-service}{Terms
  of Service}
\item
  \href{https://help.nytimes.com/hc/en-us/articles/115014893968-Terms-of-sale}{Terms
  of Sale}
\item
  \href{https://spiderbites.nytimes.com}{Site Map}
\item
  \href{https://help.nytimes.com/hc/en-us}{Help}
\item
  \href{https://www.nytimes.com/subscription?campaignId=37WXW}{Subscriptions}
\end{itemize}
