Sections

SEARCH

\protect\hyperlink{site-content}{Skip to
content}\protect\hyperlink{site-index}{Skip to site index}

\href{https://www.nytimes.com/section/obituaries}{Obituaries}

\href{https://myaccount.nytimes.com/auth/login?response_type=cookie\&client_id=vi}{}

\href{https://www.nytimes.com/section/todayspaper}{Today's Paper}

\href{/section/obituaries}{Obituaries}\textbar{}Peter Brancazio, Who
Explored the Physics of Sports, Dies at 81

\url{https://nyti.ms/2z6wXX0}

\begin{itemize}
\item
\item
\item
\item
\item
\end{itemize}

\href{https://www.nytimes.com/news-event/coronavirus?action=click\&pgtype=Article\&state=default\&region=TOP_BANNER\&context=storylines_menu}{The
Coronavirus Outbreak}

\begin{itemize}
\tightlist
\item
  live\href{https://www.nytimes.com/2020/08/03/world/coronavirus-covid-19.html?action=click\&pgtype=Article\&state=default\&region=TOP_BANNER\&context=storylines_menu}{Latest
  Updates}
\item
  \href{https://www.nytimes.com/interactive/2020/us/coronavirus-us-cases.html?action=click\&pgtype=Article\&state=default\&region=TOP_BANNER\&context=storylines_menu}{Maps
  and Cases}
\item
  \href{https://www.nytimes.com/interactive/2020/science/coronavirus-vaccine-tracker.html?action=click\&pgtype=Article\&state=default\&region=TOP_BANNER\&context=storylines_menu}{Vaccine
  Tracker}
\item
  \href{https://www.nytimes.com/2020/08/02/us/covid-college-reopening.html?action=click\&pgtype=Article\&state=default\&region=TOP_BANNER\&context=storylines_menu}{College
  Reopening}
\item
  \href{https://www.nytimes.com/live/2020/08/03/business/stock-market-today-coronavirus?action=click\&pgtype=Article\&state=default\&region=TOP_BANNER\&context=storylines_menu}{Economy}
\end{itemize}

Advertisement

\protect\hyperlink{after-top}{Continue reading the main story}

Supported by

\protect\hyperlink{after-sponsor}{Continue reading the main story}

Those We've Lost

\hypertarget{peter-brancazio-who-explored-the-physics-of-sports-dies-at-81}{%
\section{Peter Brancazio, Who Explored the Physics of Sports, Dies at
81}\label{peter-brancazio-who-explored-the-physics-of-sports-dies-at-81}}

He used science to demystify the myths of rising fastballs and Michael
Jordan's seemingly long hang time. He died of complications of the novel
coronavirus.

\includegraphics{https://static01.nyt.com/images/2020/05/20/obituaries/14BRANCAZIO/14BRANCAZIO-articleLarge.jpg?quality=75\&auto=webp\&disable=upscale}

\href{https://www.nytimes.com/by/richard-sandomir}{\includegraphics{https://static01.nyt.com/images/2018/12/10/multimedia/author-richard-sandomir/author-richard-sandomir-thumbLarge.png}}

By \href{https://www.nytimes.com/by/richard-sandomir}{Richard Sandomir}

\begin{itemize}
\item
  Published May 16, 2020Updated May 20, 2020
\item
  \begin{itemize}
  \item
  \item
  \item
  \item
  \item
  \end{itemize}
\end{itemize}

\emph{This obituary is part of a series about people who have died in
the coronavirus pandemic. Read about others}
\href{https://www.nytimes.com/series/people-who-have-died-of-the-coronavirus}{\emph{here}}\emph{.}

Peter J. Brancazio, a physics professor who debunked concepts like the
rising fastball (physically impossible) and Michael Jordan's apparently
endless hang time (much shorter than fans believed), died on April 25 in
Manhasset, N.Y. He was 81.

The cause was complications of the novel coronavirus, his son Larry
said.

Professor Brancazio, who taught at Brooklyn College for more than 30
years, was one of a small number of sports-minded physicists whose
research anticipated the use of the advanced statistics that are now
accessible through computerized tracking technology. His work, which he
began in the 1980s, was filled with terms like launching angle (how high
a ball is hit, in degrees) and spin rate (the measurement of a pitch in
revolutions per minute) that are now part of baseball's lingua franca.
(Launch angle, not launching angle, is the term now widely used.)

Although he was obsessed with basketball, Professor Brancazio was best
known for what he had to say about baseball, notably his explanation
that a so-called rising fastball could not rise --- even if pitches
thrown by fireballers like Nolan Ryan had seemingly been doing that for
decades.

``The rising fastball is an illusion,'' Professor Brancazio told The
Kansas City Star in 1987.

Gravity, he said, makes everything fall, even baseballs, and no one can
throw one fast enough and with enough spin to overcome gravity's natural
force. ``The rising fastball just looks as if it's rising,'' he said.
``It's really just not dropping as far'' as a typical fastball.

A fastball thrown at 90 miles per hour and 1,800 revolutions per minute
would drop three feet when it reached home plate, he said. But a
fastball that is thrown with still more backspin will fall only two and
a half feet, a six-inch difference that creates the illusion of rising.

Professor Brancazio, whose tools included a calculator and a TRS-80
computer, wrote about his research in professional journals; in
magazines like Popular Mechanics; and in the 1984 book ``Sport Science:
Physical Laws and Optimum Performance.''

In a June 1991 segment of the ABC News program ``PrimeTime Live,'' he
presented a scientific solution to a sports question:
\href{https://www.scienceabc.com/pure-sciences/secret-michael-jordan-slam-dunks-basketball-math-physics-hang-time-jump.html}{Why
does Michael Jordan seem to be able to fly}during a spectacular slam
dunk?

Several fans were asked during the segment to guess how long Jordan
seemed to hang in the air. Their guesses ranged from six to 10 seconds.

No, Professor Brancazio, said. Even Jordan was subject to gravity. His
hang time was only 0.9 seconds.

Later that year, Professor Brancazio elaborated on the physics of hang
time for Popular Mechanics. In an article about the science of slam
dunks, he devised a formula that determined that a 36-inch vertical leap
would equal hang time of 0.87 seconds and that a four-foot vertical leap
would equal one second.

``No small part of Jordan's greatness is the fact that he seems to cover
enormous horizontal distances in the air,'' Professor Brancazio wrote.
``He accentuates this illusion by releasing his shots on the way down,
rather than at the peak of his trajectory.''

Peter John Brancazio was born on March 22, 1939, in the Astoria section
of Queens. His mother, Ann (Salomone) Brancazio, was an actuarial worker
for The Hartford, an insurance company. His father, also named Peter,
sorted mail for the Post Office.

When Professor Brancazio and his future wife, Ronnie Kramer, were dating
as teenagers, she gave him a gift that would help guide him in his
professional life: a telescope. ``It made him want to study astronomy,''
she said.

After graduating with a bachelor's degree in engineering science from
New York University in 1959, he earned a master's in nuclear engineering
from Columbia University a year later. He began teaching physics at
Brooklyn College in 1963 while working toward a Ph.D. in astrophysics
from N.Y.U.

During his 34 years at Brooklyn College, he was also a director of the
college's observatory.

Professor Brancazio wrote his first sports article, about basketball,
for The American Journal of Physics in 1981. In it, he calculated the
optimum launching angles for shots from various distances on the floor.

Having distilled the lessons of shooting on the schoolyards of Astoria,
he found that a ball was best launched at an angle of 45 degrees, plus
half the angle of the incline from the shooter's hand to the front of
the rim of the basket, or about 50 to 55 degrees.

He had, he admitted, a personal reason for writing the paper.

``In truth,'' he wrote, ``the major purpose of this research was to find
some means to compensate for the author's stature (5' 10'' in sneakers),
inability to leap more than eight inches off the floor, and advancing
age.''

Image

Professor Brancazio wrote his first article about sports for The
American Journal of Physics in 1981. Three years later, he wrote the
book ``Sport Science,'' which a colleague said was ``a favorite among
physicists for its clear, accurate treatment.''

His intellectual detour into baseball, basketball and other sports
enlivened his classes and made him part of a small group of physicists
who brought science to sports, among them the Yale professor Robert
Adair, who wrote the 1990 book ``The Physics of Baseball.''

Michael Lisa, a professor of physics at the Ohio State University, said
that when he did the research for his 2016 book ``The Physics of
Sports,'' Professor Brancazio's book had been an inspiration. ``His book
is a favorite among physicists for its clear, accurate treatment,''
Professor Lisa wrote in an email. ``Meanwhile, it is popular among a
broader readership for its compelling approach, obviously driven by a
passion for sports coupled with a scientific mind.''

Professor Brancazio had no doubt that the people he most wanted to
impress --- athletes --- would disdain his research. And he knew why, or
at least why they did in the era before advanced training techniques
transformed athletic achievement.

``Larry Bird does not need to be told to release his shots at the
optimum launching angle,''
\href{https://aapt.scitation.org/doi/abs/10.1119/1.15601}{he wrote in
The American Journal of Physics} in 1988, ``nor does Dwight Gooden have
to understand the Magnus effect in order to throw a devastating
curveball.''

Professor Brancazio retired from Brooklyn College in 1997 and then
briefly taught adult education courses there and at Queens College. He
lectured on science, religion and astronomy at Hutton House, part of
Long Island University, from 1999 until last year.

In addition to his wife and his son Larry, Professor Brancazio is
survived by another son, David, and five grandchildren.

Professor Brancazio became a sought-after physicist in the news media
when sports met science. During Game 1 of the 1991 World Series, for
instance, CBS introduced SuperVision, a computerized animation of the
path and speed of pitches. One pitch, by Jack Morris of the Minnesota
Twins, clocked in at 94 miles per hour when it left his right hand and
was the same speed when it landed in the catcher's mitt.

CBS's analysts were impressed. But when asked a day later, Professor
Brancazio said that a ball could not maintain the same speed on its path
of 60 feet 6 inches.

``The ball has to slow down by air resistance,''
\href{https://www.nytimes.com/1991/10/22/sports/tv-sports-the-no-braking-ball-cbs-versus-physics.html}{he
told The New York Times} in 1991. ``No way it can maintain speed or pick
up speed. It should lose 9 percent of its speed along the way.''

The inventor of SuperVision acknowledged the error, saying that the
speeds had probably been rounded off --- the ball might have left
Morris's hand at 94.4 m.p.h. but had landed at 93.6.

A pitch that maintained its speed, it turned out, was as magical as a
rising fastball.

\href{https://www.nytimes.com/interactive/2020/obituaries/people-died-coronavirus-obituaries.html?action=click\&pgtype=Article\&state=default\&region=BELOW_MAIN_CONTENT\&context=covid_obits_promo}{}

\hypertarget{those-weve-lost}{%
\section{Those We've Lost}\label{those-weve-lost}}

The coronavirus pandemic has taken an incalculable death toll. This
series is designed to put names and faces to the numbers.

Read more

\includegraphics{https://static01.nyt.com/images/2020/07/30/obituaries/30Pedro/30Pedro-square640.jpg}

\hypertarget{bernaldina-josuxe9-pedro}{%
\section{Bernaldina José Pedro}\label{bernaldina-josuxe9-pedro}}

d. Boa Vista, Brazil

Leader among the Indigenous Macuxi

\includegraphics{https://static01.nyt.com/images/2020/07/31/obituaries/31Swing/merlin_175167783_8913bc90-0d64-43f3-a655-1bb1bf1601c9-square640.jpg}

\hypertarget{john-eric-swing}{%
\section{John Eric Swing}\label{john-eric-swing}}

d. Fountain Valley, Calif.

Champion of Filipino-Americans

\includegraphics{https://static01.nyt.com/images/2020/07/27/obituaries/27Victor/merlin_175001436_38b11f8e-227a-4e2c-9821-7618af9b2524-square640.jpg}

\hypertarget{victor-victor}{%
\section{Victor Victor}\label{victor-victor}}

d. Santo Domingo, Dominican Republic

Beloved musician of the Dominican Republic

\includegraphics{https://static01.nyt.com/images/2020/07/31/obituaries/31Negron/merlin_175160169_516322ae-fd23-4969-b6b2-193ced371105-square640.jpg}

\hypertarget{dr-eddie-negruxf3n}{%
\section{Dr. Eddie Negrón}\label{dr-eddie-negruxf3n}}

d. Fort Walton Beach, Fla.

Internist on Florida's Emerald Coast

\includegraphics{https://static01.nyt.com/images/2020/07/30/obituaries/30Dobson/merlin_175115928_f6b9271c-8f05-4fe1-a38a-5ca4a58f8935-square640.jpg}

\hypertarget{dobby-dobson}{%
\section{Dobby Dobson}\label{dobby-dobson}}

d. Coral Springs, Fla.

Jamaican singer and songwriter

\includegraphics{https://static01.nyt.com/images/2020/08/01/obituaries/28Gonzalez/merlin_175002771_beb57888-3951-409a-ae13-03a94b2e962e-square640.jpg}

\hypertarget{waldemar-gonzalez}{%
\section{Waldemar Gonzalez}\label{waldemar-gonzalez}}

d. White Plains, N.Y.

Teacher and social worker

Advertisement

\protect\hyperlink{after-bottom}{Continue reading the main story}

\hypertarget{site-index}{%
\subsection{Site Index}\label{site-index}}

\hypertarget{site-information-navigation}{%
\subsection{Site Information
Navigation}\label{site-information-navigation}}

\begin{itemize}
\tightlist
\item
  \href{https://help.nytimes.com/hc/en-us/articles/115014792127-Copyright-notice}{©~2020~The
  New York Times Company}
\end{itemize}

\begin{itemize}
\tightlist
\item
  \href{https://www.nytco.com/}{NYTCo}
\item
  \href{https://help.nytimes.com/hc/en-us/articles/115015385887-Contact-Us}{Contact
  Us}
\item
  \href{https://www.nytco.com/careers/}{Work with us}
\item
  \href{https://nytmediakit.com/}{Advertise}
\item
  \href{http://www.tbrandstudio.com/}{T Brand Studio}
\item
  \href{https://www.nytimes.com/privacy/cookie-policy\#how-do-i-manage-trackers}{Your
  Ad Choices}
\item
  \href{https://www.nytimes.com/privacy}{Privacy}
\item
  \href{https://help.nytimes.com/hc/en-us/articles/115014893428-Terms-of-service}{Terms
  of Service}
\item
  \href{https://help.nytimes.com/hc/en-us/articles/115014893968-Terms-of-sale}{Terms
  of Sale}
\item
  \href{https://spiderbites.nytimes.com}{Site Map}
\item
  \href{https://help.nytimes.com/hc/en-us}{Help}
\item
  \href{https://www.nytimes.com/subscription?campaignId=37WXW}{Subscriptions}
\end{itemize}
