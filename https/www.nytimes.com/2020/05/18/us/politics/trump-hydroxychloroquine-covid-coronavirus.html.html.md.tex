Sections

SEARCH

\protect\hyperlink{site-content}{Skip to
content}\protect\hyperlink{site-index}{Skip to site index}

\href{https://www.nytimes.com/section/politics}{Politics}

\href{https://myaccount.nytimes.com/auth/login?response_type=cookie\&client_id=vi}{}

\href{https://www.nytimes.com/section/todayspaper}{Today's Paper}

\href{/section/politics}{Politics}\textbar{}Trump Says He's Taking
Hydroxychloroquine, Prompting Warning From Health Experts

\url{https://nyti.ms/3e0Jp9w}

\begin{itemize}
\item
\item
\item
\item
\item
\item
\end{itemize}

\href{https://www.nytimes.com/news-event/coronavirus?action=click\&pgtype=Article\&state=default\&region=TOP_BANNER\&context=storylines_menu}{The
Coronavirus Outbreak}

\begin{itemize}
\tightlist
\item
  live\href{https://www.nytimes.com/2020/08/01/world/coronavirus-covid-19.html?action=click\&pgtype=Article\&state=default\&region=TOP_BANNER\&context=storylines_menu}{Latest
  Updates}
\item
  \href{https://www.nytimes.com/interactive/2020/us/coronavirus-us-cases.html?action=click\&pgtype=Article\&state=default\&region=TOP_BANNER\&context=storylines_menu}{Maps
  and Cases}
\item
  \href{https://www.nytimes.com/interactive/2020/science/coronavirus-vaccine-tracker.html?action=click\&pgtype=Article\&state=default\&region=TOP_BANNER\&context=storylines_menu}{Vaccine
  Tracker}
\item
  \href{https://www.nytimes.com/interactive/2020/07/29/us/schools-reopening-coronavirus.html?action=click\&pgtype=Article\&state=default\&region=TOP_BANNER\&context=storylines_menu}{What
  School May Look Like}
\item
  \href{https://www.nytimes.com/live/2020/07/31/business/stock-market-today-coronavirus?action=click\&pgtype=Article\&state=default\&region=TOP_BANNER\&context=storylines_menu}{Economy}
\end{itemize}

Advertisement

\protect\hyperlink{after-top}{Continue reading the main story}

Supported by

\protect\hyperlink{after-sponsor}{Continue reading the main story}

\hypertarget{trump-says-hes-taking-hydroxychloroquine-prompting-warning-from-health-experts}{%
\section{Trump Says He's Taking Hydroxychloroquine, Prompting Warning
From Health
Experts}\label{trump-says-hes-taking-hydroxychloroquine-prompting-warning-from-health-experts}}

His announcement drew immediate criticism from a range of medical
experts, who warned not just of the dangers it posed for the president's
health but also of the example it set.

\includegraphics{https://static01.nyt.com/images/2020/05/18/us/politics/18dc-virus-trump1/18dc-virus-trump1-articleLarge-v2.jpg?quality=75\&auto=webp\&disable=upscale}

By \href{https://www.nytimes.com/by/annie-karni}{Annie Karni} and
\href{https://www.nytimes.com/by/katie-thomas}{Katie Thomas}

\begin{itemize}
\item
  May 18, 2020
\item
  \begin{itemize}
  \item
  \item
  \item
  \item
  \item
  \item
  \end{itemize}
\end{itemize}

WASHINGTON --- President Trump said on Monday that he had been taking
\href{https://www.nytimes.com/article/hydroxychloroquine-coronavirus.html}{hydroxychloroquine},
an antimalarial drug the Food and Drug Administration warned could cause
serious heart problems for coronavirus patients. He said he was taking
the drug as a preventive measure and continued to test negative for the
coronavirus.

``All I can tell you is so far I seem to be OK,'' Mr. Trump said, adding
that he had been taking the drug for about a week and a half, with the
approval of the White House physician. ``I get a lot of tremendously
positive news on the hydroxy,'' Mr. Trump continued, explaining that his
decision to try the drug was based on one of his favorite refrains:
``What do you have to lose?''

But Mr. Trump's announcement surprised many of his aides and drew
immediate criticism from a range of medical experts, who warned not just
of the dangers it posed for the president's health but also of the
example it set.

``My concern would be that the public not hear comments about the use of
hydroxychloroquine and believe that taking this drug to prevent Covid-19
infection is without hazards. In fact, there are serious hazards,'' said
Dr. Steven E. Nissen, the chief academic officer of the Miller Family
Heart, Vascular \& Thoracic Institute at the Cleveland Clinic.

Dr. Scott Solomon, a professor of medicine at Harvard Medical School,
said Mr. Trump's decision to try the drug was up to him and his
physician. ``But what is irresponsible is the example he is setting,''
Dr. Solomon said.

Mr. Trump publicly embraced hydroxychloroquine as a ``game changer'' in
the fight against the virus in March, and his endorsement, amplified by
Fox News hosts like Laura Ingraham and Sean Hannity, caused a run on the
drug, making it scarce for those who took it for lupus and rheumatoid
arthritis, for which it is regularly prescribed.

But on Monday night, Dr. Manny Alvarez, the senior managing editor for
Fox News's health news,
\href{https://twitter.com/justinbaragona/status/1262508060410613765}{said
on air} that the president's statement was ``highly irresponsible'' and
asked what had changed since studies showed the drug had no benefits.

\hypertarget{latest-updates-global-coronavirus-outbreak}{%
\section{\texorpdfstring{\href{https://www.nytimes.com/2020/08/01/world/coronavirus-covid-19.html?action=click\&pgtype=Article\&state=default\&region=MAIN_CONTENT_1\&context=storylines_live_updates}{Latest
Updates: Global Coronavirus
Outbreak}}{Latest Updates: Global Coronavirus Outbreak}}\label{latest-updates-global-coronavirus-outbreak}}

Updated 2020-08-02T07:42:09.613Z

\begin{itemize}
\tightlist
\item
  \href{https://www.nytimes.com/2020/08/01/world/coronavirus-covid-19.html?action=click\&pgtype=Article\&state=default\&region=MAIN_CONTENT_1\&context=storylines_live_updates\#link-34047410}{The
  U.S. reels as July cases more than double the total of any other
  month.}
\item
  \href{https://www.nytimes.com/2020/08/01/world/coronavirus-covid-19.html?action=click\&pgtype=Article\&state=default\&region=MAIN_CONTENT_1\&context=storylines_live_updates\#link-780ec966}{Top
  U.S. officials work to break an impasse over the federal jobless
  benefit.}
\item
  \href{https://www.nytimes.com/2020/08/01/world/coronavirus-covid-19.html?action=click\&pgtype=Article\&state=default\&region=MAIN_CONTENT_1\&context=storylines_live_updates\#link-2bc8948}{Its
  outbreak untamed, Melbourne goes into even greater lockdown.}
\end{itemize}

\href{https://www.nytimes.com/2020/08/01/world/coronavirus-covid-19.html?action=click\&pgtype=Article\&state=default\&region=MAIN_CONTENT_1\&context=storylines_live_updates}{See
more updates}

More live coverage:
\href{https://www.nytimes.com/live/2020/07/31/business/stock-market-today-coronavirus?action=click\&pgtype=Article\&state=default\&region=MAIN_CONTENT_1\&context=storylines_live_updates}{Markets}

Mr. Trump first said he was considering taking the drug himself in
April. But in recent weeks he had notably stopped promoting it, as did
the Fox News hosts. But he then
\href{https://www.nytimes.com/2020/04/26/us/politics/trump-disinfectant-coronavirus.html}{suggested
at a news conference} that injecting disinfectants into the human body
could help combat the virus, causing confused callers
\href{https://slack-redir.net/link?url=https\%3A\%2F\%2Fwww.nytimes.com\%2F2020\%2F04\%2F24\%2Fus\%2Fpolitics\%2Ftrump-inject-disinfectant-bleach-coronavirus.html}{to
flood state health hotlines} and the makers of Clorox and Lysol to plead
with Americans not to inject or ingest their products.

His announcement on Monday came less than a month after the F.D.A.
\href{https://www.nytimes.com/2020/04/24/health/fda-hydroxychloroquine-coronavirus.html?smid=nytcore-ios-share}{issued
a safety warning about the drug}, noting that it could cause dangerous
abnormalities in heart rhythm in coronavirus patients and should not be
used outside clinical trials or in hospitals where patients were closely
monitored for heart problems.

But by that time hydroxychloroquine had become a divisive issue within
the Trump administration. Dr. Rick Bright, who led the federal agency
involved in developing a
\href{https://www.nytimes.com/2020/04/30/health/coronavirus-antiviral-drugs.html}{coronavirus}
vaccine, said he was removed from his post after he pressed for rigorous
vetting of the drug.

\includegraphics{https://static01.nyt.com/images/2020/05/18/us/politics/18dc-virus-trump2/merlin_171899859_fce27f06-652a-4d9e-b9ab-329b41b29031-articleLarge.jpg?quality=75\&auto=webp\&disable=upscale}

Dr. Bright said he was pressured to direct money toward
hydroxychloroquine, one of several ``potentially dangerous drugs
promoted by those with political connections.''

On Monday, the president not only promoted the drug but also said he was
taking it. And he made it clear that his decision was based on trusting
anecdotal evidence, and his own gut, over the warnings of the
government, or any data.

In that sense his position was consistent with his view of other expert
medical advice --- he has also refused to follow the guidelines of the
Centers for Disease Control and Prevention and wear a face mask. And
before becoming president he had
\href{https://www.statnews.com/2019/04/26/trump-vaccinations-measles/}{alleged
that there was a link} between the number of vaccines children got in
early infancy and the development of autism.

``I take it because I think I hear very good things,'' Mr. Trump said,
citing a letter he received from an unnamed doctor in Westchester, N.Y.,
promoting the use of hydroxychloroquine.

``I want the people of this nation to feel good. I don't want them being
sick,'' Mr. Trump said at the end of a round table with restaurant
executives at the White House. ``And there is a very good chance that
this has an impact, especially early on.''

Mr. Trump said he started taking the drug about 10 days ago, around the
same time
\href{https://www.nytimes.com/2020/05/10/us/politics/white-house-coronavirus-trump.html}{two
White House aides tested positive for the coronavirus}, prompting the
fears of the president and other top officials that the virus would
spread rapidly through the West Wing.

\href{https://www.nytimes.com/news-event/coronavirus?action=click\&pgtype=Article\&state=default\&region=MAIN_CONTENT_3\&context=storylines_faq}{}

\hypertarget{the-coronavirus-outbreak-}{%
\subsubsection{The Coronavirus Outbreak
›}\label{the-coronavirus-outbreak-}}

\hypertarget{frequently-asked-questions}{%
\paragraph{Frequently Asked
Questions}\label{frequently-asked-questions}}

Updated July 27, 2020

\begin{itemize}
\item ~
  \hypertarget{should-i-refinance-my-mortgage}{%
  \paragraph{Should I refinance my
  mortgage?}\label{should-i-refinance-my-mortgage}}

  \begin{itemize}
  \tightlist
  \item
    \href{https://www.nytimes.com/article/coronavirus-money-unemployment.html?action=click\&pgtype=Article\&state=default\&region=MAIN_CONTENT_3\&context=storylines_faq}{It
    could be a good idea,} because mortgage rates have
    \href{https://www.nytimes.com/2020/07/16/business/mortgage-rates-below-3-percent.html?action=click\&pgtype=Article\&state=default\&region=MAIN_CONTENT_3\&context=storylines_faq}{never
    been lower.} Refinancing requests have pushed mortgage applications
    to some of the highest levels since 2008, so be prepared to get in
    line. But defaults are also up, so if you're thinking about buying a
    home, be aware that some lenders have tightened their standards.
  \end{itemize}
\item ~
  \hypertarget{what-is-school-going-to-look-like-in-september}{%
  \paragraph{What is school going to look like in
  September?}\label{what-is-school-going-to-look-like-in-september}}

  \begin{itemize}
  \tightlist
  \item
    It is unlikely that many schools will return to a normal schedule
    this fall, requiring the grind of
    \href{https://www.nytimes.com/2020/06/05/us/coronavirus-education-lost-learning.html?action=click\&pgtype=Article\&state=default\&region=MAIN_CONTENT_3\&context=storylines_faq}{online
    learning},
    \href{https://www.nytimes.com/2020/05/29/us/coronavirus-child-care-centers.html?action=click\&pgtype=Article\&state=default\&region=MAIN_CONTENT_3\&context=storylines_faq}{makeshift
    child care} and
    \href{https://www.nytimes.com/2020/06/03/business/economy/coronavirus-working-women.html?action=click\&pgtype=Article\&state=default\&region=MAIN_CONTENT_3\&context=storylines_faq}{stunted
    workdays} to continue. California's two largest public school
    districts --- Los Angeles and San Diego --- said on July 13, that
    \href{https://www.nytimes.com/2020/07/13/us/lausd-san-diego-school-reopening.html?action=click\&pgtype=Article\&state=default\&region=MAIN_CONTENT_3\&context=storylines_faq}{instruction
    will be remote-only in the fall}, citing concerns that surging
    coronavirus infections in their areas pose too dire a risk for
    students and teachers. Together, the two districts enroll some
    825,000 students. They are the largest in the country so far to
    abandon plans for even a partial physical return to classrooms when
    they reopen in August. For other districts, the solution won't be an
    all-or-nothing approach.
    \href{https://bioethics.jhu.edu/research-and-outreach/projects/eschool-initiative/school-policy-tracker/}{Many
    systems}, including the nation's largest, New York City, are
    devising
    \href{https://www.nytimes.com/2020/06/26/us/coronavirus-schools-reopen-fall.html?action=click\&pgtype=Article\&state=default\&region=MAIN_CONTENT_3\&context=storylines_faq}{hybrid
    plans} that involve spending some days in classrooms and other days
    online. There's no national policy on this yet, so check with your
    municipal school system regularly to see what is happening in your
    community.
  \end{itemize}
\item ~
  \hypertarget{is-the-coronavirus-airborne}{%
  \paragraph{Is the coronavirus
  airborne?}\label{is-the-coronavirus-airborne}}

  \begin{itemize}
  \tightlist
  \item
    The coronavirus
    \href{https://www.nytimes.com/2020/07/04/health/239-experts-with-one-big-claim-the-coronavirus-is-airborne.html?action=click\&pgtype=Article\&state=default\&region=MAIN_CONTENT_3\&context=storylines_faq}{can
    stay aloft for hours in tiny droplets in stagnant air}, infecting
    people as they inhale, mounting scientific evidence suggests. This
    risk is highest in crowded indoor spaces with poor ventilation, and
    may help explain super-spreading events reported in meatpacking
    plants, churches and restaurants.
    \href{https://www.nytimes.com/2020/07/06/health/coronavirus-airborne-aerosols.html?action=click\&pgtype=Article\&state=default\&region=MAIN_CONTENT_3\&context=storylines_faq}{It's
    unclear how often the virus is spread} via these tiny droplets, or
    aerosols, compared with larger droplets that are expelled when a
    sick person coughs or sneezes, or transmitted through contact with
    contaminated surfaces, said Linsey Marr, an aerosol expert at
    Virginia Tech. Aerosols are released even when a person without
    symptoms exhales, talks or sings, according to Dr. Marr and more
    than 200 other experts, who
    \href{https://academic.oup.com/cid/article/doi/10.1093/cid/ciaa939/5867798}{have
    outlined the evidence in an open letter to the World Health
    Organization}.
  \end{itemize}
\item ~
  \hypertarget{what-are-the-symptoms-of-coronavirus}{%
  \paragraph{What are the symptoms of
  coronavirus?}\label{what-are-the-symptoms-of-coronavirus}}

  \begin{itemize}
  \tightlist
  \item
    Common symptoms
    \href{https://www.nytimes.com/article/symptoms-coronavirus.html?action=click\&pgtype=Article\&state=default\&region=MAIN_CONTENT_3\&context=storylines_faq}{include
    fever, a dry cough, fatigue and difficulty breathing or shortness of
    breath.} Some of these symptoms overlap with those of the flu,
    making detection difficult, but runny noses and stuffy sinuses are
    less common.
    \href{https://www.nytimes.com/2020/04/27/health/coronavirus-symptoms-cdc.html?action=click\&pgtype=Article\&state=default\&region=MAIN_CONTENT_3\&context=storylines_faq}{The
    C.D.C. has also} added chills, muscle pain, sore throat, headache
    and a new loss of the sense of taste or smell as symptoms to look
    out for. Most people fall ill five to seven days after exposure, but
    symptoms may appear in as few as two days or as many as 14 days.
  \end{itemize}
\item ~
  \hypertarget{does-asymptomatic-transmission-of-covid-19-happen}{%
  \paragraph{Does asymptomatic transmission of Covid-19
  happen?}\label{does-asymptomatic-transmission-of-covid-19-happen}}

  \begin{itemize}
  \tightlist
  \item
    So far, the evidence seems to show it does. A widely cited
    \href{https://www.nature.com/articles/s41591-020-0869-5}{paper}
    published in April suggests that people are most infectious about
    two days before the onset of coronavirus symptoms and estimated that
    44 percent of new infections were a result of transmission from
    people who were not yet showing symptoms. Recently, a top expert at
    the World Health Organization stated that transmission of the
    coronavirus by people who did not have symptoms was ``very rare,''
    \href{https://www.nytimes.com/2020/06/09/world/coronavirus-updates.html?action=click\&pgtype=Article\&state=default\&region=MAIN_CONTENT_3\&context=storylines_faq\#link-1f302e21}{but
    she later walked back that statement.}
  \end{itemize}
\end{itemize}

As for taking hydroxychloroquine, ``I'm not going to get hurt by it,''
Mr. Trump said, adding that he was sharing the news to be transparent
with Americans and appearing to enjoy the shock value of his
announcement. ``It has been around for 40 years for malaria, for lupus,
for other things.''

Later on Monday night, the White House physician, Dr. Sean P. Conley,
\href{https://int.nyt.com/data/documenthelper/6959-letter-from-white-house-physic/e3e29d81b7d6339b9f56/optimized/full.pdf\#page=1}{released
a statement} that linked Mr. Trump's decision to take the drug to the
``support staff'' who tested positive for the virus, an apparent
reference to the president's personal valet. ``After numerous
discussions he and I had regarding the evidence for and against the use
of hydroxychloroquine, we concluded the potential benefit from treatment
outweighed the relative risks,'' Dr. Conley said. He also said the
president ``is in very good health and has remained symptom free.''

Early studies of hydroxychloroquine in the laboratory suggesting that
the drug could block the coronavirus from attacking cells prompted
initial enthusiasm. But the studies of the drug in humans so far have
pointed to serious side effects.

``I think it's a very bad idea to be taking hydroxychloroquine as a
preventive medication,'' said Dr. Eric Topol, a cardiologist and the
director of the Scripps Research Translational Institute in La Jolla,
Calif. ``There are no data to support that, there's no evidence and in
fact there is no compelling evidence to support its use at all at this
point.''

Dr. Topol said the risk of developing a potentially fatal arrhythmia
because of hydroxychloroquine could come without warning and did not
happen only in people with heart conditions. ``We can't predict that. In
fact, it can happen in people who are healthy,'' he said. ``It could
happen in anyone.''

Mr. Trump has never provided the public with a full picture of his
health. In 2018, the White House physician reported that Mr. Trump had
an LDL cholesterol level of 143, well above the desired level of 100 or
less. Some cardiologists who are not associated with the White House
said
\href{https://www.nytimes.com/2018/01/17/us/politics/trump-physical-heart-health-cholesterol.html}{his
cholesterol levels raised heart concerns}.

Mr. Trump made a trip in November to Walter Reed National Military
Medical Center that was not listed on his public schedule. He stayed for
about two hours for what White House officials said were routine tests,
but since the visit had not been revealed in advance and came only nine
months after
\href{https://slack-redir.net/link?url=https\%3A\%2F\%2Fwww.nytimes.com\%2F2019\%2F02\%2F14\%2Fus\%2Fpolitics\%2Ftrump-obese.html\%3Faction\%3Dclick\%26module\%3DRelatedLinks\%26pgtype\%3DArticle}{his
last annual physical}, it touched off much discussion about whether the
president had an undisclosed health issue.

Mr. Trump, 73, is the oldest man ever sworn in for a first term as
president, and he is known for his love of fast food and takes pride in
not exercising. At his
\href{https://www.nytimes.com/2019/02/14/us/politics/trump-obese.html}{checkup
last year} he weighed 243 pounds, which is considered obese for a man of
his reported height of 6 feet 3 inches. He has been reported in the past
to be taking rosuvastatin, a lipid-lowering drug, to control his
cholesterol.

Neil Cavuto, a Fox News host, reacted to the president's announcement
with a grim warning that once might have shocked his network's viewers.
To anyone with pre-existing conditions, he said: ``It will kill you. I
cannot stress enough. This will kill you.''

Annie Karni reported from Washington, and Katie Thomas from New York.

Advertisement

\protect\hyperlink{after-bottom}{Continue reading the main story}

\hypertarget{site-index}{%
\subsection{Site Index}\label{site-index}}

\hypertarget{site-information-navigation}{%
\subsection{Site Information
Navigation}\label{site-information-navigation}}

\begin{itemize}
\tightlist
\item
  \href{https://help.nytimes.com/hc/en-us/articles/115014792127-Copyright-notice}{©~2020~The
  New York Times Company}
\end{itemize}

\begin{itemize}
\tightlist
\item
  \href{https://www.nytco.com/}{NYTCo}
\item
  \href{https://help.nytimes.com/hc/en-us/articles/115015385887-Contact-Us}{Contact
  Us}
\item
  \href{https://www.nytco.com/careers/}{Work with us}
\item
  \href{https://nytmediakit.com/}{Advertise}
\item
  \href{http://www.tbrandstudio.com/}{T Brand Studio}
\item
  \href{https://www.nytimes.com/privacy/cookie-policy\#how-do-i-manage-trackers}{Your
  Ad Choices}
\item
  \href{https://www.nytimes.com/privacy}{Privacy}
\item
  \href{https://help.nytimes.com/hc/en-us/articles/115014893428-Terms-of-service}{Terms
  of Service}
\item
  \href{https://help.nytimes.com/hc/en-us/articles/115014893968-Terms-of-sale}{Terms
  of Sale}
\item
  \href{https://spiderbites.nytimes.com}{Site Map}
\item
  \href{https://help.nytimes.com/hc/en-us}{Help}
\item
  \href{https://www.nytimes.com/subscription?campaignId=37WXW}{Subscriptions}
\end{itemize}
