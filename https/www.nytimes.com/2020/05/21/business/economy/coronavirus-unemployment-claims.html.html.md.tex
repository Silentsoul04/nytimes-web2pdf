Sections

SEARCH

\protect\hyperlink{site-content}{Skip to
content}\protect\hyperlink{site-index}{Skip to site index}

\href{https://www.nytimes.com/section/business/economy}{Economy}

\href{https://myaccount.nytimes.com/auth/login?response_type=cookie\&client_id=vi}{}

\href{https://www.nytimes.com/section/todayspaper}{Today's Paper}

\href{/section/business/economy}{Economy}\textbar{}Many Jobs May Vanish
Forever as Layoffs Mount

\href{https://nyti.ms/3bRjL5Q}{https://nyti.ms/3bRjL5Q}

\begin{itemize}
\item
\item
\item
\item
\item
\item
\end{itemize}

\href{https://www.nytimes.com/news-event/coronavirus?action=click\&pgtype=Article\&state=default\&region=TOP_BANNER\&context=storylines_menu}{The
Coronavirus Outbreak}

\begin{itemize}
\tightlist
\item
  live\href{https://www.nytimes.com/2020/08/08/world/coronavirus-updates.html?action=click\&pgtype=Article\&state=default\&region=TOP_BANNER\&context=storylines_menu}{Latest
  Updates}
\item
  \href{https://www.nytimes.com/interactive/2020/us/coronavirus-us-cases.html?action=click\&pgtype=Article\&state=default\&region=TOP_BANNER\&context=storylines_menu}{Maps
  and Cases}
\item
  \href{https://www.nytimes.com/interactive/2020/science/coronavirus-vaccine-tracker.html?action=click\&pgtype=Article\&state=default\&region=TOP_BANNER\&context=storylines_menu}{Vaccine
  Tracker}
\item
  \href{https://www.nytimes.com/interactive/2020/world/coronavirus-tips-advice.html?action=click\&pgtype=Article\&state=default\&region=TOP_BANNER\&context=storylines_menu}{F.A.Q.}
\item
  \href{https://www.nytimes.com/live/2020/08/07/business/stock-market-today-coronavirus?action=click\&pgtype=Article\&state=default\&region=TOP_BANNER\&context=storylines_menu}{Markets
  \& Economy}
\end{itemize}

Advertisement

\protect\hyperlink{after-top}{Continue reading the main story}

Supported by

\protect\hyperlink{after-sponsor}{Continue reading the main story}

\hypertarget{many-jobs-may-vanish-forever-as-layoffs-mount}{%
\section{Many Jobs May Vanish Forever as Layoffs
Mount}\label{many-jobs-may-vanish-forever-as-layoffs-mount}}

With over 38 million U.S. unemployment claims in nine weeks, one
economist says the situation is ``grimmer than we thought.''

\includegraphics{https://static01.nyt.com/images/2020/05/21/business/21virus-jobless1/21virus-jobless1-articleLarge.jpg?quality=75\&auto=webp\&disable=upscale}

\href{https://www.nytimes.com/by/patricia-cohen}{\includegraphics{https://static01.nyt.com/images/2018/02/16/multimedia/author-patricia-cohen/author-patricia-cohen-thumbLarge.jpg}}

By \href{https://www.nytimes.com/by/patricia-cohen}{Patricia Cohen}

\begin{itemize}
\item
  Published May 21, 2020Updated June 11, 2020
\item
  \begin{itemize}
  \item
  \item
  \item
  \item
  \item
  \item
  \end{itemize}
\end{itemize}

Even as states begin to reopen for business, a further 2.4 million
workers joined the nation's
\href{https://www.nytimes.com/2020/06/11/business/economy/unemployment-claims-coronavirus.html}{unemployment}
rolls last week, and there is growing concern among economists that many
of the lost jobs are gone for good.

The Labor Department's report of
\href{https://www.nytimes.com/2020/06/04/business/economy/coronavirus-unemployment-claims.html}{new
jobless claims}, released Thursday, brought the total to 38.6 million
since mid-March, when the coronavirus outbreak forced widespread
shutdowns.

While workers and their employers have expressed optimism that most of
the joblessness will be temporary, many who are studying the pandemic's
impact are increasingly worried about the employment situation.

``I hate to say it, but this is going to take longer and look grimmer
than we thought,'' Nicholas Bloom, an economist at Stanford University,
said of the path to recovery.

Mr. Bloom is a co-author of an
\href{https://bfi.uchicago.edu/wp-content/uploads/BFI_WP_202059.pdf}{analysis}
that estimates 42 percent of recent layoffs will result in permanent job
loss.

``Firms intend to hire these people back,'' he said, referring to a
\href{https://www.frbatlanta.org/research/surveys/business-uncertainty}{recent
survey of businesses by the Federal Reserve Bank of Atlanta}. ``But we
know from the past that these aspirations often don't turn out to be
true.''

The precariousness of the path ahead
\href{https://www.nytimes.com/2020/05/21/business/economy/fed-chair-warns-this-is-a-downturn-without-modern-precedent.html}{was
underscored Thursday} by the Federal Reserve chair,
\href{https://www.nytimes.com/2020/05/19/business/coronavirus-stocks-economy.html}{Jerome
H. Powell}. ``We are now experiencing a whole new level of uncertainty,
as questions only the virus can answer complicate the outlook,'' he said
in remarks for delivery at an online forum.

The economy that does come back is likely to look quite different from
the one that closed. If social distancing rules become the new normal,
causing thinner crowds in restaurants, theaters and stores, at sports
arenas, and on airplanes, then fewer workers will be required.

Large companies already expect more of their workers to continue to work
remotely and say they plan to
\href{https://www.prnewswire.com/news-releases/deloitte-cfo-signals-survey-executive-teams-are-largely-focused-on-adapting-operations-for-near-term-performance-and-evolving-their-businesses-for-a-post-crisis-future-301061671.html}{reduce
their real estate footprint,} which will reduce the foot traffic that
feeds nearby restaurants, shops, nail salons and other businesses.

Concerns about working in close quarters and too much social interaction
could also accelerate the trend toward automation, some economists say.

\hypertarget{latest-updates-the-coronavirus-outbreak-and-the-economy}{%
\section{\texorpdfstring{\href{https://www.nytimes.com/live/2020/08/07/business/stock-market-today-coronavirus?action=click\&pgtype=Article\&state=default\&region=MAIN_CONTENT_1\&context=storylines_live_updates}{Latest
Updates: The Coronavirus Outbreak and the
Economy}}{Latest Updates: The Coronavirus Outbreak and the Economy}}\label{latest-updates-the-coronavirus-outbreak-and-the-economy}}

\href{https://www.nytimes.com/live/2020/08/07/business/stock-market-today-coronavirus?action=click\&pgtype=Article\&state=default\&region=MAIN_CONTENT_1\&context=storylines_live_updates\#wealthy-families-are-throwing-a-lifeline-to-distressed-businesses}{15h
ago}

\href{https://www.nytimes.com/live/2020/08/07/business/stock-market-today-coronavirus?action=click\&pgtype=Article\&state=default\&region=MAIN_CONTENT_1\&context=storylines_live_updates\#wealthy-families-are-throwing-a-lifeline-to-distressed-businesses}{Wealthy
families are throwing a lifeline to distressed businesses.}

\href{https://www.nytimes.com/live/2020/08/07/business/stock-market-today-coronavirus?action=click\&pgtype=Article\&state=default\&region=MAIN_CONTENT_1\&context=storylines_live_updates\#the-publisher-of-the-onion-jezebel-and-other-websites-lays-off-15-employees}{16h
ago}

\href{https://www.nytimes.com/live/2020/08/07/business/stock-market-today-coronavirus?action=click\&pgtype=Article\&state=default\&region=MAIN_CONTENT_1\&context=storylines_live_updates\#the-publisher-of-the-onion-jezebel-and-other-websites-lays-off-15-employees}{The
publisher of The Onion, Jezebel and other websites lays off 15
employees.}

\href{https://www.nytimes.com/live/2020/08/07/business/stock-market-today-coronavirus?action=click\&pgtype=Article\&state=default\&region=MAIN_CONTENT_1\&context=storylines_live_updates\#canada-outlines-its-response-to-the-new-us-aluminum-tariff}{21h
ago}

\href{https://www.nytimes.com/live/2020/08/07/business/stock-market-today-coronavirus?action=click\&pgtype=Article\&state=default\&region=MAIN_CONTENT_1\&context=storylines_live_updates\#canada-outlines-its-response-to-the-new-us-aluminum-tariff}{Canada
outlines its response to the new U.S. aluminum tariff.}

\href{https://www.nytimes.com/live/2020/08/07/business/stock-market-today-coronavirus?action=click\&pgtype=Article\&state=default\&region=MAIN_CONTENT_1\&context=storylines_live_updates}{See
more updates}

More live coverage:
\href{https://www.nytimes.com/2020/08/07/world/covid-19-news.html?action=click\&pgtype=Article\&state=default\&region=MAIN_CONTENT_1\&context=storylines_live_updates}{Global}

New jobs are being created, mostly at low wages --- for delivery
drivers, warehouse workers and cleaners. But many more jobs will vanish.

``I think we're in for a very long haul,'' Mr. Bloom at Stanford said.

6

million

38.6 million

5

Claims were filed in

the last nine weeks

4

Initial jobless claims, per week

Seasonally adjusted

3

2

RECESSION

1

'06

'08

'09

'12

'16

'20

6

million

38.6 million

5

Claims were filed in

the last nine weeks

4

3

RECESSION

2

Initial jobless claims, per week

Seasonally adjusted

1

'06

'08

'09

'12

'16

'20

Source: Department of Labor

By The New York Times

Torsten Slok, chief economist for Deutsche Bank Securities, agreed that
the government's latest report pointed to lasting job losses. Even with
states reopening, ``the hemorrhaging has continued,'' he said.

``I fear that maybe there is something more fundamental going on,''
particularly in occupations most affected by social distancing rules,
Mr. Slok added. He expects the official jobless rate for May to approach
20 percent, up from the
\href{https://www.nytimes.com/interactive/2020/05/08/business/economy/april-jobs-report.html}{14.7
percent} reported by the Labor Department for April.

A household
\href{https://www.census.gov/library/stories/2020/05/new-household-pulse-survey-shows-concern-over-food-security-loss-of-income.html}{survey
from the Census Bureau} released Wednesday offered further evidence of
the widespread pain: 47 percent of adults said they or a member of their
household had lost employment income since mid-March. Nearly 40 percent
expected the loss to continue over the next four weeks.

Nonetheless, Larry Kudlow, director of the National Economic Council,
knocked down the idea of extending unemployment benefits. ``I do not
believe that more government spending is going to give us a strong and
durable recovery,'' he said Thursday at an event sponsored by The
Washington Post.

Emergency relief and expanded unemployment benefits that Congress
approved in late March have helped tide households over. Roughly
three-quarters of people who are eligible for a \$1,200 stimulus payment
from the federal government have received it, according to the Treasury
Department.

Workers who have successfully applied for unemployment benefits are
getting the extra \$600 weekly supplement from the federal government,
and most states have finally begun to carry out the Pandemic
Unemployment Assistance program, which extends benefits to freelancers,
self-employed workers and others who don't routinely qualify. The total
number of new pandemic insurance claims reported, though, was inflated
by nearly a million because of a data entry mistake from Massachusetts,
according to the state's Executive Office of Labor and Workforce
Development.

Mistakes, lags in reporting and processing, and the weeding out of
duplicate claims and reports have clouded the unemployment picture in
some places.

What is clear, though, is that many states are still struggling to keep
up with the overwhelming demand, drawing desperate complaints from
jobless workers who have been waiting two months or more to receive
their first benefit check. Indiana, Wyoming, Hawaii and Missouri are
among the states with large backlogs of incompletely processed claims.
Another is Kentucky, where nearly one in three workers are unemployed.

The \$600 supplement has become
\href{https://www.washingtonpost.com/politics/trump-expresses-opposition-to-extending-unemployment-benefits-enacted-in-response-to-pandemic/2020/05/19/19ae0e50-9a12-11ea-a282-386f56d579e6_story.html}{a
point of contention}, drawing criticism from the White House and
Republican congressional leaders who object to the notion that some
workers --- particularly low-wage ones --- are
\href{https://bfi.uchicago.edu/working-paper/2020-62/}{getting more
money in unemployment benefits} than they would on the job. But many
have also
\href{https://www.epi.org/blog/16-2-million-workers-have-likely-lost-employer-provided-health-insurance-since-the-coronavirus-shock-began/}{lost
their employer-provided health insurance} and other benefits.

Sami Adamson, a freelance scenic artist for theater, events and
television shows, received the letter with her login credentials to
collect benefits from New Jersey only on Monday, more than two months
after she first applied.

She said her partner, who is in the same line of work, had filed for
jobless benefits in New York and quickly received his payments.

By the time Ms. Adamson heard from New Jersey, a design studio had
called her for a temporary assignment. She plans to eventually reclaim
the lost weeks of benefits, but for now she is helping to make face
shields in a large warehouse where assembly-line workers are spaced
apart, handling plastic, foam and elastic.

``I don't think I'll need aid for the next two or three weeks,'' Ms.
Adamson said, ``but I'm not sure too far ahead of that.''

Nearly half of the states have yet to provide the additional 13 weeks of
unemployment insurance that the federal government has promised to those
who exhausted their state benefits. Workers in Florida --- which
provides just 12 weeks of benefits, the fewest anywhere --- are
particularly feeling this pinch. And while several states, including
those that pay the average of 26 weeks, have offered additional weeks of
coverage during the pandemic, Florida has not.

Small-business owners who were hoping the Paycheck Protection Program
would enable them to keep their workers on the payroll contend the
program is not operating as intended.

Roy Surdej, who owns Peaches Boutique in Chicago, applied for a loan
after he was forced to close and the pandemic eliminated the season's
wave of proms,
\href{https://www.nytimes.com/2019/11/12/style/quinceaera-genz-millennial.html}{quinceañeras}
and graduation celebrations.

Image

Peaches Boutique in Chicago was closed at what should have been the peak
of a season of dressy events.Credit...Lucy Hewett for The New York Times

Image

``It's devastating for us,'' said the boutique's owner, with no sign of
when he can reopen and rehire.Credit...Lucy Hewett for The New York
Times

Under the program, the loan turns into a grant if he rehires the
100-person staff he had built up in February in anticipation of selling
thousands of ruffled, sequined and strappy dresses during the spring
rush. But he said that would be impossible, given the income he had lost
and the restrictions that continue to pre-empt social gatherings.

``No way can I qualify for full forgiveness,'' said Mr. Surdej, who said
revenue had dried up. ``It's devastating for us,'' he added, saying he
had no clue when he would be able to reopen and begin rehiring. ``If the
government can't adjust the dates to allow us to use it properly so we
can survive, then I won't use it.''

At the same time, the
\href{https://www.nytimes.com/2020/05/19/business/stock-market-today-coronavirus.html}{Congressional
Budget Office warned} that businesses able to use the Paycheck
Protection Program might end up laying off workers when the program
expires at the end of June.

Several states have warned workers that they risk losing their benefits
if they refuse an offer to work. Federal rules enacted during the
pandemic say that workers are not compelled to return to unsafe working
conditions, but just what constitutes such conditions is not necessarily
clear.

On Tuesday, Democratic senators sent a letter to Labor Secretary Eugene
Scalia to ``clarify the circumstances'' so that workers are not ``forced
to choose between going back to work in unsafe conditions, or continuing
to social distance and losing their only source of income.''

Workers with child care responsibilities can stay on unemployment if
public schools are closed, but once the term ends, a lack of day care or
summer programs is not considered a legitimate reason. Nor are
self-imposed quarantines.

Officials can lift stay-at-home and business restrictions, but then what
happens? ``There are lingering concerns about health, family situations,
kids not in school, relatives who are sick and needing care,'' said Carl
Tannenbaum, chief economist at Northern Trust. ``There's going to be a
very slow and gradual process of reopening and restoring employment
beyond just a declaration from the statehouse or the county seat.''

Tiffany Hsu, Jeanna Smialek and Alan Rappeport contributed reporting.

Advertisement

\protect\hyperlink{after-bottom}{Continue reading the main story}

\hypertarget{site-index}{%
\subsection{Site Index}\label{site-index}}

\hypertarget{site-information-navigation}{%
\subsection{Site Information
Navigation}\label{site-information-navigation}}

\begin{itemize}
\tightlist
\item
  \href{https://help.nytimes.com/hc/en-us/articles/115014792127-Copyright-notice}{©~2020~The
  New York Times Company}
\end{itemize}

\begin{itemize}
\tightlist
\item
  \href{https://www.nytco.com/}{NYTCo}
\item
  \href{https://help.nytimes.com/hc/en-us/articles/115015385887-Contact-Us}{Contact
  Us}
\item
  \href{https://www.nytco.com/careers/}{Work with us}
\item
  \href{https://nytmediakit.com/}{Advertise}
\item
  \href{http://www.tbrandstudio.com/}{T Brand Studio}
\item
  \href{https://www.nytimes.com/privacy/cookie-policy\#how-do-i-manage-trackers}{Your
  Ad Choices}
\item
  \href{https://www.nytimes.com/privacy}{Privacy}
\item
  \href{https://help.nytimes.com/hc/en-us/articles/115014893428-Terms-of-service}{Terms
  of Service}
\item
  \href{https://help.nytimes.com/hc/en-us/articles/115014893968-Terms-of-sale}{Terms
  of Sale}
\item
  \href{https://spiderbites.nytimes.com}{Site Map}
\item
  \href{https://help.nytimes.com/hc/en-us}{Help}
\item
  \href{https://www.nytimes.com/subscription?campaignId=37WXW}{Subscriptions}
\end{itemize}
