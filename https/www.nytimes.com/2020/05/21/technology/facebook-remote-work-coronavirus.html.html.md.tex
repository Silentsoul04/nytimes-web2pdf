Sections

SEARCH

\protect\hyperlink{site-content}{Skip to
content}\protect\hyperlink{site-index}{Skip to site index}

\href{https://www.nytimes.com/section/technology}{Technology}

\href{https://myaccount.nytimes.com/auth/login?response_type=cookie\&client_id=vi}{}

\href{https://www.nytimes.com/section/todayspaper}{Today's Paper}

\href{/section/technology}{Technology}\textbar{}Facebook Starts Planning
for Permanent Remote Workers

\url{https://nyti.ms/3e5PGku}

\begin{itemize}
\item
\item
\item
\item
\item
\end{itemize}

\href{https://www.nytimes.com/news-event/coronavirus?action=click\&pgtype=Article\&state=default\&region=TOP_BANNER\&context=storylines_menu}{The
Coronavirus Outbreak}

\begin{itemize}
\tightlist
\item
  live\href{https://www.nytimes.com/2020/08/03/world/coronavirus-covid-19.html?action=click\&pgtype=Article\&state=default\&region=TOP_BANNER\&context=storylines_menu}{Latest
  Updates}
\item
  \href{https://www.nytimes.com/interactive/2020/us/coronavirus-us-cases.html?action=click\&pgtype=Article\&state=default\&region=TOP_BANNER\&context=storylines_menu}{Maps
  and Cases}
\item
  \href{https://www.nytimes.com/interactive/2020/science/coronavirus-vaccine-tracker.html?action=click\&pgtype=Article\&state=default\&region=TOP_BANNER\&context=storylines_menu}{Vaccine
  Tracker}
\item
  \href{https://www.nytimes.com/2020/08/02/us/covid-college-reopening.html?action=click\&pgtype=Article\&state=default\&region=TOP_BANNER\&context=storylines_menu}{College
  Reopening}
\item
  \href{https://www.nytimes.com/live/2020/08/03/business/stock-market-today-coronavirus?action=click\&pgtype=Article\&state=default\&region=TOP_BANNER\&context=storylines_menu}{Economy}
\end{itemize}

Advertisement

\protect\hyperlink{after-top}{Continue reading the main story}

Supported by

\protect\hyperlink{after-sponsor}{Continue reading the main story}

\hypertarget{facebook-starts-planning-for-permanent-remote-workers}{%
\section{Facebook Starts Planning for Permanent Remote
Workers}\label{facebook-starts-planning-for-permanent-remote-workers}}

The move is a stark change from an office-centric culture. But there's a
catch: Salaries are likely to change to match local costs of living.

\includegraphics{https://static01.nyt.com/images/2020/05/21/business/21facebook/merlin_165501837_9b3f8675-055a-4bfa-a174-9aa433c175f0-articleLarge.jpg?quality=75\&auto=webp\&disable=upscale}

By \href{https://www.nytimes.com/by/kate-conger}{Kate Conger}

\begin{itemize}
\item
  May 21, 2020
\item
  \begin{itemize}
  \item
  \item
  \item
  \item
  \item
  \end{itemize}
\end{itemize}

OAKLAND, Calif. --- Facebook said on Thursday that it would allow many
employees to work from home permanently. But there's a catch: They may
not be able to keep their big Silicon Valley salaries in more affordable
parts of the country.

Mark Zuckerberg, Facebook's chief executive, told workers during a staff
meeting that was livestreamed on his Facebook page that within a decade
as many as half of the company's more than 48,000 employees would work
from home.

``It's clear that Covid has changed a lot about our lives, and that
certainly includes the way that most of us work,'' Mr. Zuckerberg said.
``Coming out of this period, I expect that remote work is going to be a
growing trend as well.''

Facebook's decision, the first among tech's biggest companies, is a
stark change for a business culture built around getting workers into
giant offices and keeping them there. Using free shuttle buses, free
cafeterias and personal services like dry cleaning, tech companies have
done as much as possible over the years to give employees little reason
to go home, let alone avoid the office.

If other giant companies follow suit, tech employment could start to
shift away from expensive hubs like Silicon Valley, Seattle and New
York. The option to work from home could also provide more reason for
tech workers who complain that their enviable salaries still aren't
enough to buy a home in San Francisco or San Jose
to\href{https://www.nytimes.com/2019/05/09/technology/uber-lyft-low-tax-millennials.html}{consider
settling in other parts of the country}.

Mr. Zuckerberg's announcement followed similar
\href{https://www.nytimes.com/2020/05/08/technology/coronavirus-work-from-home.html}{decisions
at Twitter and the payments company Square}, both led by Jack Dorsey.
Mr. Dorsey said last week that employees at his companies would be
allowed to work from home indefinitely. At Google, employees have been
told they can work from home through the end of the year, but the
company has not made any indications about permanent plans.

There are signs that remote work is popular among technologists. After
Mr. Dorsey's announcement, Google searches for ``Twitter jobs'' spiked,
according to Google Trends.

Aaron Levie, the chief executive of the business technology company Box,
wrote on Twitter that ``the push happening around remote work is as
game-changing for the future of tech as the launch of the iPhone'' more
than a decade ago.

Tech executives have long believed that person-to-person communication
was a big part of the creativity that went into generating popular
products. They built giant campuses that reflected that belief, from the
ornate offices of Apple, Google and Facebook in Silicon Valley to the
new Amazon headquarters in Seattle.

Still, the biggest tech companies were trying to expand beyond their
main offices before the pandemic, as an older generation of companies
like Intel had done. Amazon, for example, intends to open
a\href{https://www.nytimes.com/2019/02/14/nyregion/amazon-hq2-queens.html}{second
headquarters} in Virginia. The coronavirus pandemic could accelerate
those plans.

\hypertarget{latest-updates-economy}{%
\section{\texorpdfstring{\href{https://www.nytimes.com/live/2020/08/03/business/stock-market-today-coronavirus?action=click\&pgtype=Article\&state=default\&region=MAIN_CONTENT_1\&context=storylines_live_updates}{Latest
Updates:
Economy}}{Latest Updates: Economy}}\label{latest-updates-economy}}

\href{https://www.nytimes.com/live/2020/08/03/business/stock-market-today-coronavirus?action=click\&pgtype=Article\&state=default\&region=MAIN_CONTENT_1\&context=storylines_live_updates\#the-chicago-fed-president-says-its-up-to-congress-to-save-the-economy}{10h
ago}

\href{https://www.nytimes.com/live/2020/08/03/business/stock-market-today-coronavirus?action=click\&pgtype=Article\&state=default\&region=MAIN_CONTENT_1\&context=storylines_live_updates\#the-chicago-fed-president-says-its-up-to-congress-to-save-the-economy}{The
Chicago Fed president says it's up to Congress to save the economy.}

\href{https://www.nytimes.com/live/2020/08/03/business/stock-market-today-coronavirus?action=click\&pgtype=Article\&state=default\&region=MAIN_CONTENT_1\&context=storylines_live_updates\#faa-says-boeing-has-effectively-mitigated-defects-in-the-737-max}{10h
ago}

\href{https://www.nytimes.com/live/2020/08/03/business/stock-market-today-coronavirus?action=click\&pgtype=Article\&state=default\&region=MAIN_CONTENT_1\&context=storylines_live_updates\#faa-says-boeing-has-effectively-mitigated-defects-in-the-737-max}{F.A.A.
says Boeing has `effectively mitigated' defects in the 737 Max.}

\href{https://www.nytimes.com/live/2020/08/03/business/stock-market-today-coronavirus?action=click\&pgtype=Article\&state=default\&region=MAIN_CONTENT_1\&context=storylines_live_updates\#small-businesses-got-emergency-loans-but-not-what-they-expected}{13h
ago}

\href{https://www.nytimes.com/live/2020/08/03/business/stock-market-today-coronavirus?action=click\&pgtype=Article\&state=default\&region=MAIN_CONTENT_1\&context=storylines_live_updates\#small-businesses-got-emergency-loans-but-not-what-they-expected}{Small
businesses got emergency loans, but not what they expected.}

\href{https://www.nytimes.com/live/2020/08/03/business/stock-market-today-coronavirus?action=click\&pgtype=Article\&state=default\&region=MAIN_CONTENT_1\&context=storylines_live_updates}{See
more updates}

More live coverage:
\href{https://www.nytimes.com/2020/08/03/world/coronavirus-covid-19.html?action=click\&pgtype=Article\&state=default\&region=MAIN_CONTENT_1\&context=storylines_live_updates}{Global}

``Before the virus happened, a lot of the discussion about the tech
sector was about how to bring people to work sites and create affordable
housing,'' said Robert Silverman, a professor of urban and regional
planning at the State University of New York at Buffalo. ``This is kind
of a natural progression.''

An employee exodus from the biggest urban tech hubs, combined with
layoffs, could have dramatic local impacts. Housing costs in the Bay
Area, for example, have fallen since the pandemic began, according to
the rental firm Zumper. Rents in San Francisco fell 7 percent in April,
and were down 15 percent in Menlo Park, Facebook's home.

Mr. Zuckerberg long worried that employees who worked remotely would
lose productivity. Facebook once
\href{https://www.reuters.com/article/us-facebook-benefits-idUSKBN0U02P620151217}{provided
cash bonuses} to employees who lived within 10 miles of its
headquarters. In 2018, Facebook expanded its main campus with elaborate
new
offices\href{https://www.bloomberg.com/news/photo-essays/2018-09-04/here-s-a-first-look-inside-facebook-s-new-frank-gehry-designed-office}{designed
by the star architect Frank Gehry}, including a 3.6-acre roof garden
with more than 200 trees.

Just last year, Facebook started moving into a 43-story office tower
that it had leased in San Francisco, and the company is still reportedly
in talks for a
\href{https://commercialobserver.com/2020/05/facebook-closing-on-740000-square-feet-at-farley-post-office/\#.XsbsRx55EbV.twitter}{significant
office expansion in New York}, as well.

In March, the coronavirus lockdown forced companies to send employees
home. Many tech companies, including Facebook, emptied their offices
before local shelter-in-place orders.

Now, more than two months later, executives are discovering that their
remote workers performed better than expected. Mr. Zuckerberg said
employees remained focused even though they were working from home.

Facebook will begin by allowing new hires who are senior engineers to
work remotely, and then allow current employees to apply for permission
to work from home if they have positive performance reviews.

Starting in January, Facebook's employee compensation will be adjusted
based on the cost of living in the locations where workers choose to
live. Facebook will make sure employees are honest about their location
by checking where they log in to internal systems from, he said.

Mr. Zuckerberg said the shift could offer more benefits than
inconveniences for the company. Allowing remote work will allow Facebook
to broaden its recruitment, retain valuable employees, reduce the
climate impact caused by commutes and expand the diversity of its work
force, Mr. Zuckerberg said.

So far, Facebook, Square and Twitter are being far more aggressive than
their counterparts in the industry. Their work is mostly done in
software code, which can be handled remotely.

At Apple, on the other hand, many employees are hardware engineers who
need to be in the company's lab, particularly because of the company's
secrecy around its products. Tim Cook, Apple's chief executive, said in
April that the company's main office in Silicon Valley would be closed
until at least June and has not updated that timeline.

Start-ups could also find it difficult to manage a remote work force.
Allowing workers to live in the Midwest could keep costs down, but
Silicon Valley has a giant talent pool from which start-ups draw their
workers. Also, many venture capitalists, mostly based in Silicon Valley
and San Francisco, expect the companies they invest in to be based
nearby.

At Los Angeles-based Snap, the maker of Snapchat, employees are allowed
to work at home through September. Evan Spiegel, Snap's chief executive,
said in an
\href{https://www.nytimes.com/2020/05/21/technology/robot-deliveries.html}{interview}
that he was reassessing the situation regularly and considering guidance
from health authorities about when to reopen.

``People want certainty, and there's a huge amount of pressure as a
leader to make definitive statements,'' Mr. Spiegel said on Wednesday.
``I think it's important that we remain flexible in a situation that is
changing rapidly.''

Advertisement

\protect\hyperlink{after-bottom}{Continue reading the main story}

\hypertarget{site-index}{%
\subsection{Site Index}\label{site-index}}

\hypertarget{site-information-navigation}{%
\subsection{Site Information
Navigation}\label{site-information-navigation}}

\begin{itemize}
\tightlist
\item
  \href{https://help.nytimes.com/hc/en-us/articles/115014792127-Copyright-notice}{©~2020~The
  New York Times Company}
\end{itemize}

\begin{itemize}
\tightlist
\item
  \href{https://www.nytco.com/}{NYTCo}
\item
  \href{https://help.nytimes.com/hc/en-us/articles/115015385887-Contact-Us}{Contact
  Us}
\item
  \href{https://www.nytco.com/careers/}{Work with us}
\item
  \href{https://nytmediakit.com/}{Advertise}
\item
  \href{http://www.tbrandstudio.com/}{T Brand Studio}
\item
  \href{https://www.nytimes.com/privacy/cookie-policy\#how-do-i-manage-trackers}{Your
  Ad Choices}
\item
  \href{https://www.nytimes.com/privacy}{Privacy}
\item
  \href{https://help.nytimes.com/hc/en-us/articles/115014893428-Terms-of-service}{Terms
  of Service}
\item
  \href{https://help.nytimes.com/hc/en-us/articles/115014893968-Terms-of-sale}{Terms
  of Sale}
\item
  \href{https://spiderbites.nytimes.com}{Site Map}
\item
  \href{https://help.nytimes.com/hc/en-us}{Help}
\item
  \href{https://www.nytimes.com/subscription?campaignId=37WXW}{Subscriptions}
\end{itemize}
