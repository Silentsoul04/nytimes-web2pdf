Sections

SEARCH

\protect\hyperlink{site-content}{Skip to
content}\protect\hyperlink{site-index}{Skip to site index}

\href{https://www.nytimes.com/section/world/asia}{Asia Pacific}

\href{https://myaccount.nytimes.com/auth/login?response_type=cookie\&client_id=vi}{}

\href{https://www.nytimes.com/section/todayspaper}{Today's Paper}

\href{/section/world/asia}{Asia Pacific}\textbar{}`Lionhearted' Girl
Bikes Dad Across India, Inspiring a Nation

\url{https://nyti.ms/3ebHUpm}

\begin{itemize}
\item
\item
\item
\item
\item
\end{itemize}

\href{https://www.nytimes.com/news-event/coronavirus?action=click\&pgtype=Article\&state=default\&region=TOP_BANNER\&context=storylines_menu}{The
Coronavirus Outbreak}

\begin{itemize}
\tightlist
\item
  live\href{https://www.nytimes.com/2020/08/02/world/coronavirus-updates.html?action=click\&pgtype=Article\&state=default\&region=TOP_BANNER\&context=storylines_menu}{Latest
  Updates}
\item
  \href{https://www.nytimes.com/interactive/2020/us/coronavirus-us-cases.html?action=click\&pgtype=Article\&state=default\&region=TOP_BANNER\&context=storylines_menu}{Maps
  and Cases}
\item
  \href{https://www.nytimes.com/interactive/2020/science/coronavirus-vaccine-tracker.html?action=click\&pgtype=Article\&state=default\&region=TOP_BANNER\&context=storylines_menu}{Vaccine
  Tracker}
\item
  \href{https://www.nytimes.com/interactive/2020/07/29/us/schools-reopening-coronavirus.html?action=click\&pgtype=Article\&state=default\&region=TOP_BANNER\&context=storylines_menu}{What
  School May Look Like}
\item
  \href{https://www.nytimes.com/live/2020/07/31/business/stock-market-today-coronavirus?action=click\&pgtype=Article\&state=default\&region=TOP_BANNER\&context=storylines_menu}{Economy}
\end{itemize}

Advertisement

\protect\hyperlink{after-top}{Continue reading the main story}

Supported by

\protect\hyperlink{after-sponsor}{Continue reading the main story}

\hypertarget{lionhearted-girl-bikes-dad-across-india-inspiring-a-nation}{%
\section{`Lionhearted' Girl Bikes Dad Across India, Inspiring a
Nation}\label{lionhearted-girl-bikes-dad-across-india-inspiring-a-nation}}

A 15-year-old migrant girl pedaled hundreds of miles to bring her
injured father back to their home village. India's cycling federation
has taken notice.

\includegraphics{https://static01.nyt.com/images/2020/06/02/world/22india-bike-promo/merlin_172735011_9aa26fba-ff92-4896-827c-9acb6bb9db2c-articleLarge.jpg?quality=75\&auto=webp\&disable=upscale}

\href{https://www.nytimes.com/by/jeffrey-gettleman}{\includegraphics{https://static01.nyt.com/images/2018/10/10/multimedia/author-jeffrey-gettleman/author-jeffrey-gettleman-thumbLarge.png}}\href{https://www.nytimes.com/by/suhasini-raj}{\includegraphics{https://static01.nyt.com/images/2019/11/22/reader-center/author-Suhasini-Raj/author-Suhasini-Raj-thumbLarge.png}}

By \href{https://www.nytimes.com/by/jeffrey-gettleman}{Jeffrey
Gettleman} and \href{https://www.nytimes.com/by/suhasini-raj}{Suhasini
Raj}

\begin{itemize}
\item
  May 22, 2020
\item
  \begin{itemize}
  \item
  \item
  \item
  \item
  \item
  \end{itemize}
\end{itemize}

NEW DELHI --- She was a 15-year-old with a simple mission: bring papa
home.

Jyoti Kumari and her dad had nearly no money, no transport, and their
village was halfway across India.

And her dad, an out-of-work migrant laborer, was injured and could
barely walk.

So Jyoti told her dad: Let \emph{me} take you home. He thought the idea
was crazy but went along with it. She then jumped on a \$20 purple bike
bought with the last of their savings. With her dad perched on the rear,
she pedaled from the outskirts of New Delhi to their home village, 700
miles away.

"Don't worry, mummy,'' she reassured her mother along the way, using
borrowed cellphones. ``I will get Papa home good.''

During the past two months under India's coronavirus lockdown, millions
of migrant laborers and their families have poured out of India's
cities, desperate and penniless, as they try to get back to their native
villages where they can rely on family networks to survive.

Many haven't made it. Some have been crushed by trains; others run over
by trucks. A few have simply collapsed while trudging down a long, hot
highway, dead from exhaustion.

But amid all this pain and sadness now emerges a tale of devotion and
straight-up grit. The Indian press has seized upon this feel-good story,
gushing about Jyoti
\href{https://www.dnaindia.com/lifestyle/report-she-is-indeed-my-shravan-kumar-15-year-old-girl-brings-father-from-gurugram-to-bihar-on-bicycle-2825465}{the
``lionhearted}.''

And a few days ago, the story got even better.

While resting up in her village, Jyoti received a call from the
\href{http://www.cyclingfederationofindia.org/about.html}{Cycling
Federation of India}. Convinced she had the right stuff, Onkar Singh,
the federation's chairman, invited her to New Delhi for a tryout with
the national team.

``She has great talent,'' Mr. Singh said.

Reached by phone on Friday in her village of Sirhulli, in Bihar, one of
India's poorest states, Jyoti said in a scratchy voice barely above a
whisper, because she still sounded exhausted: ``I'm elated, I really
want to go.''

CHINA

New

Delhi

PAKISTAN

Gurugram

Sirhulli

BIHAR

INDIA

Bay of

Bengal

400 MILES

By The New York Times

As India struggles with the coronavirus and the severe measures to
contain it, the plight of the nation's migrant workers has become a
crisis within a crisis.

Within hours of a national lockdown imposed on March 25,
\href{https://www.nytimes.com/2020/03/29/world/asia/coronavirus-india-migrants.html}{thousands
upon thousands of migrant laborers began to bolt from the cities}. Many
had gravitated to urban areas for work and lived hand-to-mouth, as
rickshaw pullers, tea sellers, brick haulers on construction sites.

\includegraphics{https://static01.nyt.com/images/2020/05/22/world/22india-bike-2sub/merlin_172294200_f627260f-67a3-414f-85c3-0a82ec783531-articleLarge.jpg?quality=75\&auto=webp\&disable=upscale}

But once the lockdown eviscerated their chances of getting any work,
they feared running out of money and food and began long, treacherous
journeys back to their home villages.

Scholars estimate that tens of millions are on the move, the biggest
migration of human beings across the subcontinent since the partition of
India and Pakistan in 1947.

\hypertarget{latest-updates-global-coronavirus-outbreak}{%
\section{\texorpdfstring{\href{https://www.nytimes.com/2020/08/01/world/coronavirus-covid-19.html?action=click\&pgtype=Article\&state=default\&region=MAIN_CONTENT_1\&context=storylines_live_updates}{Latest
Updates: Global Coronavirus
Outbreak}}{Latest Updates: Global Coronavirus Outbreak}}\label{latest-updates-global-coronavirus-outbreak}}

Updated 2020-08-02T17:52:35.962Z

\begin{itemize}
\tightlist
\item
  \href{https://www.nytimes.com/2020/08/01/world/coronavirus-covid-19.html?action=click\&pgtype=Article\&state=default\&region=MAIN_CONTENT_1\&context=storylines_live_updates\#link-34047410}{The
  U.S. reels as July cases more than double the total of any other
  month.}
\item
  \href{https://www.nytimes.com/2020/08/01/world/coronavirus-covid-19.html?action=click\&pgtype=Article\&state=default\&region=MAIN_CONTENT_1\&context=storylines_live_updates\#link-780ec966}{Top
  U.S. officials work to break an impasse over the federal jobless
  benefit.}
\item
  \href{https://www.nytimes.com/2020/08/01/world/coronavirus-covid-19.html?action=click\&pgtype=Article\&state=default\&region=MAIN_CONTENT_1\&context=storylines_live_updates\#link-2bc8948}{Its
  outbreak untamed, Melbourne goes into even greater lockdown.}
\end{itemize}

\href{https://www.nytimes.com/2020/08/01/world/coronavirus-covid-19.html?action=click\&pgtype=Article\&state=default\&region=MAIN_CONTENT_1\&context=storylines_live_updates}{See
more updates}

More live coverage:
\href{https://www.nytimes.com/live/2020/07/31/business/stock-market-today-coronavirus?action=click\&pgtype=Article\&state=default\&region=MAIN_CONTENT_1\&context=storylines_live_updates}{Markets}

When it becomes a matter of survival, said Priya Deshingkar, a professor
of migration and development at the University of Sussex, migrant
laborers ``will try to go home, because that is where their real social
safety net lies.''

That's exactly why Jyoti hit the road.

Her father, Mohan Paswan, a rickshaw driver from a lower rung of India's
caste system, was injured in a traffic accident in January and was
running out of money even before the lockdown. He was among the legions
of migrant workers performing menial jobs in the shadows of Gurugram, a
satellite city of New Delhi and home to corridors of shimmering glass
towers and many millionaires.

Jyoti came out from their village in Bihar to care for Mr. Paswan. She
had dropped out of school a year ago because the family didn't have
enough money. Things got even worse after the lockdown, with their
landlord threatening to kick them out and then cutting off their
electricity.

When Jyoti came up with the escape plan, her father shook his head.

``I said, `Look, daughter, it's not four or five kilometers that you
will drag me from here. It's 12-, 13-hundred kilometers. How will we
go?'' he said in a video broadcast by the BBC's Hindi service.

The two bought a simple girl's bike for the equivalent of about \$20. On
May 8, they set off, Jyoti at the handlebars, dad sitting pillion on
back. Jyoti was pretty confident on a bike, having ridden a lot in her
village.

Many days they had little food. They slept at gas stations. They lived
off the generosity of strangers. Jyoti said that except for one short
lift on a truck, she pedaled nearly 100 miles a day. It wasn't easy. Her
father is big, and he was carrying a bag.

\href{https://www.nytimes.com/news-event/coronavirus?action=click\&pgtype=Article\&state=default\&region=MAIN_CONTENT_3\&context=storylines_faq}{}

\hypertarget{the-coronavirus-outbreak-}{%
\subsubsection{The Coronavirus Outbreak
›}\label{the-coronavirus-outbreak-}}

\hypertarget{frequently-asked-questions}{%
\paragraph{Frequently Asked
Questions}\label{frequently-asked-questions}}

Updated July 27, 2020

\begin{itemize}
\item ~
  \hypertarget{should-i-refinance-my-mortgage}{%
  \paragraph{Should I refinance my
  mortgage?}\label{should-i-refinance-my-mortgage}}

  \begin{itemize}
  \tightlist
  \item
    \href{https://www.nytimes.com/article/coronavirus-money-unemployment.html?action=click\&pgtype=Article\&state=default\&region=MAIN_CONTENT_3\&context=storylines_faq}{It
    could be a good idea,} because mortgage rates have
    \href{https://www.nytimes.com/2020/07/16/business/mortgage-rates-below-3-percent.html?action=click\&pgtype=Article\&state=default\&region=MAIN_CONTENT_3\&context=storylines_faq}{never
    been lower.} Refinancing requests have pushed mortgage applications
    to some of the highest levels since 2008, so be prepared to get in
    line. But defaults are also up, so if you're thinking about buying a
    home, be aware that some lenders have tightened their standards.
  \end{itemize}
\item ~
  \hypertarget{what-is-school-going-to-look-like-in-september}{%
  \paragraph{What is school going to look like in
  September?}\label{what-is-school-going-to-look-like-in-september}}

  \begin{itemize}
  \tightlist
  \item
    It is unlikely that many schools will return to a normal schedule
    this fall, requiring the grind of
    \href{https://www.nytimes.com/2020/06/05/us/coronavirus-education-lost-learning.html?action=click\&pgtype=Article\&state=default\&region=MAIN_CONTENT_3\&context=storylines_faq}{online
    learning},
    \href{https://www.nytimes.com/2020/05/29/us/coronavirus-child-care-centers.html?action=click\&pgtype=Article\&state=default\&region=MAIN_CONTENT_3\&context=storylines_faq}{makeshift
    child care} and
    \href{https://www.nytimes.com/2020/06/03/business/economy/coronavirus-working-women.html?action=click\&pgtype=Article\&state=default\&region=MAIN_CONTENT_3\&context=storylines_faq}{stunted
    workdays} to continue. California's two largest public school
    districts --- Los Angeles and San Diego --- said on July 13, that
    \href{https://www.nytimes.com/2020/07/13/us/lausd-san-diego-school-reopening.html?action=click\&pgtype=Article\&state=default\&region=MAIN_CONTENT_3\&context=storylines_faq}{instruction
    will be remote-only in the fall}, citing concerns that surging
    coronavirus infections in their areas pose too dire a risk for
    students and teachers. Together, the two districts enroll some
    825,000 students. They are the largest in the country so far to
    abandon plans for even a partial physical return to classrooms when
    they reopen in August. For other districts, the solution won't be an
    all-or-nothing approach.
    \href{https://bioethics.jhu.edu/research-and-outreach/projects/eschool-initiative/school-policy-tracker/}{Many
    systems}, including the nation's largest, New York City, are
    devising
    \href{https://www.nytimes.com/2020/06/26/us/coronavirus-schools-reopen-fall.html?action=click\&pgtype=Article\&state=default\&region=MAIN_CONTENT_3\&context=storylines_faq}{hybrid
    plans} that involve spending some days in classrooms and other days
    online. There's no national policy on this yet, so check with your
    municipal school system regularly to see what is happening in your
    community.
  \end{itemize}
\item ~
  \hypertarget{is-the-coronavirus-airborne}{%
  \paragraph{Is the coronavirus
  airborne?}\label{is-the-coronavirus-airborne}}

  \begin{itemize}
  \tightlist
  \item
    The coronavirus
    \href{https://www.nytimes.com/2020/07/04/health/239-experts-with-one-big-claim-the-coronavirus-is-airborne.html?action=click\&pgtype=Article\&state=default\&region=MAIN_CONTENT_3\&context=storylines_faq}{can
    stay aloft for hours in tiny droplets in stagnant air}, infecting
    people as they inhale, mounting scientific evidence suggests. This
    risk is highest in crowded indoor spaces with poor ventilation, and
    may help explain super-spreading events reported in meatpacking
    plants, churches and restaurants.
    \href{https://www.nytimes.com/2020/07/06/health/coronavirus-airborne-aerosols.html?action=click\&pgtype=Article\&state=default\&region=MAIN_CONTENT_3\&context=storylines_faq}{It's
    unclear how often the virus is spread} via these tiny droplets, or
    aerosols, compared with larger droplets that are expelled when a
    sick person coughs or sneezes, or transmitted through contact with
    contaminated surfaces, said Linsey Marr, an aerosol expert at
    Virginia Tech. Aerosols are released even when a person without
    symptoms exhales, talks or sings, according to Dr. Marr and more
    than 200 other experts, who
    \href{https://academic.oup.com/cid/article/doi/10.1093/cid/ciaa939/5867798}{have
    outlined the evidence in an open letter to the World Health
    Organization}.
  \end{itemize}
\item ~
  \hypertarget{what-are-the-symptoms-of-coronavirus}{%
  \paragraph{What are the symptoms of
  coronavirus?}\label{what-are-the-symptoms-of-coronavirus}}

  \begin{itemize}
  \tightlist
  \item
    Common symptoms
    \href{https://www.nytimes.com/article/symptoms-coronavirus.html?action=click\&pgtype=Article\&state=default\&region=MAIN_CONTENT_3\&context=storylines_faq}{include
    fever, a dry cough, fatigue and difficulty breathing or shortness of
    breath.} Some of these symptoms overlap with those of the flu,
    making detection difficult, but runny noses and stuffy sinuses are
    less common.
    \href{https://www.nytimes.com/2020/04/27/health/coronavirus-symptoms-cdc.html?action=click\&pgtype=Article\&state=default\&region=MAIN_CONTENT_3\&context=storylines_faq}{The
    C.D.C. has also} added chills, muscle pain, sore throat, headache
    and a new loss of the sense of taste or smell as symptoms to look
    out for. Most people fall ill five to seven days after exposure, but
    symptoms may appear in as few as two days or as many as 14 days.
  \end{itemize}
\item ~
  \hypertarget{does-asymptomatic-transmission-of-covid-19-happen}{%
  \paragraph{Does asymptomatic transmission of Covid-19
  happen?}\label{does-asymptomatic-transmission-of-covid-19-happen}}

  \begin{itemize}
  \tightlist
  \item
    So far, the evidence seems to show it does. A widely cited
    \href{https://www.nature.com/articles/s41591-020-0869-5}{paper}
    published in April suggests that people are most infectious about
    two days before the onset of coronavirus symptoms and estimated that
    44 percent of new infections were a result of transmission from
    people who were not yet showing symptoms. Recently, a top expert at
    the World Health Organization stated that transmission of the
    coronavirus by people who did not have symptoms was ``very rare,''
    \href{https://www.nytimes.com/2020/06/09/world/coronavirus-updates.html?action=click\&pgtype=Article\&state=default\&region=MAIN_CONTENT_3\&context=storylines_faq\#link-1f302e21}{but
    she later walked back that statement.}
  \end{itemize}
\end{itemize}

As they moved down the long roads under the withering sun, many people
teased them, saying it was ridiculous for a girl to pedal while her
father sat on the back.

``Father would become upset when he heard such things, but I told him
not to worry as people did not know that he was wounded,''
Jyoti\href{https://thewire.in/rights/jyoti-kumari-bihar-gurgaon-cycle-covid-19-lockdown}{told
an interviewer at The Wire}, an Indian publication. (Over the past few
days, she has done
\href{https://www.tribuneindia.com/news/nation/15-year-old-girl-brings-father-from-gurugram-to-bihar-on-bicycle-87464}{a
lot of interviews}).

After they arrived in their village last weekend, her father went into a
quarantine center --- many states in India have tried to sequester
migrant laborers returning from the cities, to stop the coronavirus from
reaching deep into the countryside. How well that is working is unclear.

Image

Migrants travelling home by cycle rickshaw from Punjab to Bihar, travel
from Andhel village in the state of Uttar Pradesh, this
month.Credit...Rebecca Conway for The New York Times

As the lockdown has relaxed, infections are rising quickly. India has
reported around 120,000,
\href{https://timesofindia.indiatimes.com/india/indias-virus-cases-cross-100000-jump-at-fastest-pace-in-asia/articleshow/75829868.cms}{its
curve one of the}steepest among large countries. Many health experts
believe the real numbers are even higher but hidden because of India's
relatively low rates of testing.

Jyoti's mother convinced village elders to let her quarantine at home.
She was exhausted and soon besieged by reporters.

Then, a few days later, on Thursday morning, she got The Call.

The Cycling Federation of India, which scouts young talent and sends the
best to international competitions, including the Olympics, tracked her
down through a journalist. Mr. Singh, the chairman, said he had been
moved by how far she pedaled with a heavy person on the back.\\
``And luggage,'' Mr. Singh was quick to add.

The federation is planning to bring her to New Delhi via ``something
comfortable, like an AC train,'' he said, referring to the
air-conditioned service. She will then do a series of cycling tests.

As for how much she actually rode versus the help she received from
trucks, Mr. Singh acknowledged that maybe the story had become a bit
stretched, totally understandable in times like these.

But one thing was not in doubt, Mr. Singh said.

``She has guts.''

Jeffrey Gettleman reported from New Delhi and Suhasini Raj from Lucknow,
India. Hari Kumar contributed reporting from New Delhi.

Advertisement

\protect\hyperlink{after-bottom}{Continue reading the main story}

\hypertarget{site-index}{%
\subsection{Site Index}\label{site-index}}

\hypertarget{site-information-navigation}{%
\subsection{Site Information
Navigation}\label{site-information-navigation}}

\begin{itemize}
\tightlist
\item
  \href{https://help.nytimes.com/hc/en-us/articles/115014792127-Copyright-notice}{©~2020~The
  New York Times Company}
\end{itemize}

\begin{itemize}
\tightlist
\item
  \href{https://www.nytco.com/}{NYTCo}
\item
  \href{https://help.nytimes.com/hc/en-us/articles/115015385887-Contact-Us}{Contact
  Us}
\item
  \href{https://www.nytco.com/careers/}{Work with us}
\item
  \href{https://nytmediakit.com/}{Advertise}
\item
  \href{http://www.tbrandstudio.com/}{T Brand Studio}
\item
  \href{https://www.nytimes.com/privacy/cookie-policy\#how-do-i-manage-trackers}{Your
  Ad Choices}
\item
  \href{https://www.nytimes.com/privacy}{Privacy}
\item
  \href{https://help.nytimes.com/hc/en-us/articles/115014893428-Terms-of-service}{Terms
  of Service}
\item
  \href{https://help.nytimes.com/hc/en-us/articles/115014893968-Terms-of-sale}{Terms
  of Sale}
\item
  \href{https://spiderbites.nytimes.com}{Site Map}
\item
  \href{https://help.nytimes.com/hc/en-us}{Help}
\item
  \href{https://www.nytimes.com/subscription?campaignId=37WXW}{Subscriptions}
\end{itemize}
