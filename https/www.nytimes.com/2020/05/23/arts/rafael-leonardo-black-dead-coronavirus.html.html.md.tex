Sections

SEARCH

\protect\hyperlink{site-content}{Skip to
content}\protect\hyperlink{site-index}{Skip to site index}

\href{https://www.nytimes.com/section/arts}{Arts}

\href{https://myaccount.nytimes.com/auth/login?response_type=cookie\&client_id=vi}{}

\href{https://www.nytimes.com/section/todayspaper}{Today's Paper}

\href{/section/arts}{Arts}\textbar{}Rafael Leonardo Black, Solitary and
Self-Trained Artist, Dies at 71

\url{https://nyti.ms/2LXonN2}

\begin{itemize}
\item
\item
\item
\item
\item
\end{itemize}

\href{https://www.nytimes.com/news-event/coronavirus?action=click\&pgtype=Article\&state=default\&region=TOP_BANNER\&context=storylines_menu}{The
Coronavirus Outbreak}

\begin{itemize}
\tightlist
\item
  live\href{https://www.nytimes.com/2020/08/03/world/coronavirus-covid-19.html?action=click\&pgtype=Article\&state=default\&region=TOP_BANNER\&context=storylines_menu}{Latest
  Updates}
\item
  \href{https://www.nytimes.com/interactive/2020/us/coronavirus-us-cases.html?action=click\&pgtype=Article\&state=default\&region=TOP_BANNER\&context=storylines_menu}{Maps
  and Cases}
\item
  \href{https://www.nytimes.com/interactive/2020/science/coronavirus-vaccine-tracker.html?action=click\&pgtype=Article\&state=default\&region=TOP_BANNER\&context=storylines_menu}{Vaccine
  Tracker}
\item
  \href{https://www.nytimes.com/2020/08/02/us/covid-college-reopening.html?action=click\&pgtype=Article\&state=default\&region=TOP_BANNER\&context=storylines_menu}{College
  Reopening}
\item
  \href{https://www.nytimes.com/live/2020/08/03/business/stock-market-today-coronavirus?action=click\&pgtype=Article\&state=default\&region=TOP_BANNER\&context=storylines_menu}{Economy}
\end{itemize}

Advertisement

\protect\hyperlink{after-top}{Continue reading the main story}

Supported by

\protect\hyperlink{after-sponsor}{Continue reading the main story}

Those We've Lost

\hypertarget{rafael-leonardo-black-solitary-and-self-trained-artist-dies-at-71}{%
\section{Rafael Leonardo Black, Solitary and Self-Trained Artist, Dies
at
71}\label{rafael-leonardo-black-solitary-and-self-trained-artist-dies-at-71}}

After laboring for years in his small Brooklyn apartment, he had his
first New York gallery show when he was 64. He died from complications
of Covid-19.

\includegraphics{https://static01.nyt.com/images/2020/05/26/obituaries/26Black1/22Black1-articleLarge-v2.jpg?quality=75\&auto=webp\&disable=upscale}

\href{https://www.nytimes.com/by/holland-cotter}{\includegraphics{https://static01.nyt.com/images/2018/02/16/multimedia/author-holland-cotter/author-holland-cotter-thumbLarge.jpg}}

By \href{https://www.nytimes.com/by/holland-cotter}{Holland Cotter}

\begin{itemize}
\item
  Published May 23, 2020Updated May 25, 2020
\item
  \begin{itemize}
  \item
  \item
  \item
  \item
  \item
  \end{itemize}
\end{itemize}

\emph{This obituary is part of a series about people who have died in
the coronavirus pandemic. Read about others}
\href{https://www.nytimes.com/series/people-who-have-died-of-the-coronavirus}{\emph{here}}\emph{.}

Rafael Leonardo Black, a self-trained artist who spent more than 40
years creating elaborate pictorial mythologies steeped in art history
and popular culture, and who had his first New York gallery show at 64,
died on May 15 in Brooklyn. He was 71.

The cause was complications of Covid-19, said
\href{http://www.francisnaumann.com/BLACK/Obit.html}{Francis M.
Naumann}, the art dealer who represents him.

Mr. Black's debut, in 2013 at Francis Naumann Fine Art in Manhattan,
consisted of collagelike pencil drawings of historically diverse figures
and scenes brought together under umbrella themes. The work was so
minutely detailed that the gallery provided magnifying glasses to view
it. The exhibition was accompanied by a multipage guide, with numbered
charts of the compositions and annotations by the artist identifying the
figures depicted.

\includegraphics{https://static01.nyt.com/images/2020/05/25/obituaries/25Black-new/25Black-new-articleLarge.jpg?quality=75\&auto=webp\&disable=upscale}

A 1982 drawing called ``Oneirology,'' Mr. Black explained, ``presents
the towering beauty as well as the horrors of the 20th century.'' The
tutelary spirit was Picasso, represented by a centrally placed
mini-version of one of his 1930s ``Weeping Woman'' paintings, around
which circulated figures of Coco Chanel, André Breton and the
19th-century Queen Ranavalona of Madagascar, known for her love of
French fashion, who was mentioned by Marcel Proust.

If these images occupy positions on the beauty side of Mr. Black's
20th-century equation, the horror component had at least equal weight.
It included images of three Latin American dictators, a Shell Oil
refinery and a portrait, excerpted from a photograph, of a Roman
Catholic cardinal in deep conversation with Joseph Goebbels.

Mr. Black himself refrained from any obvious passing of judgment. The
players in ``Oneirology,'' including Orpheus, Andy Warhol and three
giraffes, commingle as if at a party. They are all overseen by a guiding
star in the form of a glowing image of the disco diva Grace Jones's
smiling lips.

Rafael Leonardo Black was born on Jan. 6, 1949, on Aruba, a Caribbean
island that remains Dutch territory. He started drawing as a child,
encouraged by his parents.

When asked, in an interview published for his 2013 show, whether he had
been named for Italian Renaissance artists, he said that ``Rafael'' was
in honor of an uncle, Rafael Hodge, and that ``Leonardo'' came from
``Rafael Leónidas Trujillo Molina, the first person of color to became
president of the Dominican Republic,'' adding, ``My mother admired
him.'' (People who knew Mr. Black called him Ray.)

Image

Mr. Black's 2016 work ``Eudaimonicon.'' His complex, labor-intensive
drawings were often years in the making.Credit...via Francis M. Naumann
Fine Art, New York

A prodigious early reader --- he was proficient in Dutch, French and
Spanish, as well as English --- he came to New York City in 1965 to
attend high school. He spent a lot of time in museums and almost
immediately began exploring the city's rock counterculture, going to
small nightclubs, where he heard Jimi Hendrix.

After hanging out, sketchbook in hand, at the offices of the music
magazine Crawdaddy in 1967, he was invited to illustrate its reviews of
two major albums: the Beatles' ``Sgt. Pepper's Lonely Hearts Club Band''
and the Jimi Hendrix Experience's ``Are You Experienced.'' Densely
composed but done with fine-grained precision in black and white, the
results set a model for his later work.

Image

''Flamenco Pharmacy,'' 2016.Credit...via Francis M. Naumann Fine Art,
New York

That same year he began studying at Columbia University, where he
majored in art history. ``At Columbia, a wider world of the arts opened
up to me,'' he said in the gallery interview; the historical styles that
particularly interested him were Symbolism and Surrealism.

But he left Columbia in his senior year before graduating. ``My grades
were not good, so they asked me to take some time and come back later,''
he said. ``I left, but I never came back.''

Supporting himself with various jobs --- he was a typist in a law firm,
a salesman at Gimbels and then at Macy's, and a hospital receptionist
--- he continued to read voraciously. Books on comparative mythology by
Joseph Campbell especially interested him, as did the work of the
African-American poet, novelist and essayist Ishmael Reed.

Always his life revolved around his art. He lived alone in a small
apartment that doubled as his studio in the Clinton Hill neighborhood of
Brooklyn. There he devoted himself to his complex, labor-intensive
drawings, which were often years in the making.

Image

``Pique-nique,'' 1991.Credit...Francis M. Naumann Fine Art, New York

\href{https://www.nytimes.com/2013/06/05/nyregion/discovered-at-64-a-brooklyn-artist-takes-his-place.html}{Jim
Dwyer of The New York Times}visited Mr. Black at the time of his Naumann
show. ``For more than three decades, Mr. Black, 64, has made a portal to
the world in dense, miniature renderings of ancient myth and modern
figures,'' Mr. Dwyer wrote. ``Until recently, few people ever saw his
work because he had almost no visitors.''

``Day after day, year after year,'' he continued, ``he labored like a
monk.''

Speaking of his art to Mr. Dwyer, Mr. Black said he had ``just never
made the effort to sell it,'' although, he added, ``I've always done it
since --- well, I guess, since I've known myself.''

He is survived by a nephew, Jean Murphy.

Mr. Black's first New York solo show, titled ``Insider Art'' --- a
reference to his profound knowledge of art history --- proved to be his
last. Most of the pictures sold, earning him more money than he had ever
had. He contemplated traveling back to Aruba, but never did. He
preferred his daily studio routine, a painstaking mode of production
that meant that he left little new work behind.

``What I do is read and make my pictures,'' he told Mr. Dwyer. ``People
who become what are called artists don't stop. There's a saying:
`Everybody writes poems at 15; real poets write them at 50.'''

\href{https://www.nytimes.com/interactive/2020/obituaries/people-died-coronavirus-obituaries.html?action=click\&pgtype=Article\&state=default\&region=BELOW_MAIN_CONTENT\&context=covid_obits_promo}{}

\hypertarget{those-weve-lost}{%
\section{Those We've Lost}\label{those-weve-lost}}

The coronavirus pandemic has taken an incalculable death toll. This
series is designed to put names and faces to the numbers.

Read more

\includegraphics{https://static01.nyt.com/images/2020/07/30/obituaries/30Pedro/30Pedro-square640.jpg}

\hypertarget{bernaldina-josuxe9-pedro}{%
\section{Bernaldina José Pedro}\label{bernaldina-josuxe9-pedro}}

d. Boa Vista, Brazil

Leader among the Indigenous Macuxi

\includegraphics{https://static01.nyt.com/images/2020/07/31/obituaries/31Swing/merlin_175167783_8913bc90-0d64-43f3-a655-1bb1bf1601c9-square640.jpg}

\hypertarget{john-eric-swing}{%
\section{John Eric Swing}\label{john-eric-swing}}

d. Fountain Valley, Calif.

Champion of Filipino-Americans

\includegraphics{https://static01.nyt.com/images/2020/07/27/obituaries/27Victor/merlin_175001436_38b11f8e-227a-4e2c-9821-7618af9b2524-square640.jpg}

\hypertarget{victor-victor}{%
\section{Victor Victor}\label{victor-victor}}

d. Santo Domingo, Dominican Republic

Beloved musician of the Dominican Republic

\includegraphics{https://static01.nyt.com/images/2020/07/31/obituaries/31Negron/merlin_175160169_516322ae-fd23-4969-b6b2-193ced371105-square640.jpg}

\hypertarget{dr-eddie-negruxf3n}{%
\section{Dr. Eddie Negrón}\label{dr-eddie-negruxf3n}}

d. Fort Walton Beach, Fla.

Internist on Florida's Emerald Coast

\includegraphics{https://static01.nyt.com/images/2020/07/30/obituaries/30Dobson/merlin_175115928_f6b9271c-8f05-4fe1-a38a-5ca4a58f8935-square640.jpg}

\hypertarget{dobby-dobson}{%
\section{Dobby Dobson}\label{dobby-dobson}}

d. Coral Springs, Fla.

Jamaican singer and songwriter

\includegraphics{https://static01.nyt.com/images/2020/08/01/obituaries/28Gonzalez/merlin_175002771_beb57888-3951-409a-ae13-03a94b2e962e-square640.jpg}

\hypertarget{waldemar-gonzalez}{%
\section{Waldemar Gonzalez}\label{waldemar-gonzalez}}

d. White Plains, N.Y.

Teacher and social worker

Advertisement

\protect\hyperlink{after-bottom}{Continue reading the main story}

\hypertarget{site-index}{%
\subsection{Site Index}\label{site-index}}

\hypertarget{site-information-navigation}{%
\subsection{Site Information
Navigation}\label{site-information-navigation}}

\begin{itemize}
\tightlist
\item
  \href{https://help.nytimes.com/hc/en-us/articles/115014792127-Copyright-notice}{©~2020~The
  New York Times Company}
\end{itemize}

\begin{itemize}
\tightlist
\item
  \href{https://www.nytco.com/}{NYTCo}
\item
  \href{https://help.nytimes.com/hc/en-us/articles/115015385887-Contact-Us}{Contact
  Us}
\item
  \href{https://www.nytco.com/careers/}{Work with us}
\item
  \href{https://nytmediakit.com/}{Advertise}
\item
  \href{http://www.tbrandstudio.com/}{T Brand Studio}
\item
  \href{https://www.nytimes.com/privacy/cookie-policy\#how-do-i-manage-trackers}{Your
  Ad Choices}
\item
  \href{https://www.nytimes.com/privacy}{Privacy}
\item
  \href{https://help.nytimes.com/hc/en-us/articles/115014893428-Terms-of-service}{Terms
  of Service}
\item
  \href{https://help.nytimes.com/hc/en-us/articles/115014893968-Terms-of-sale}{Terms
  of Sale}
\item
  \href{https://spiderbites.nytimes.com}{Site Map}
\item
  \href{https://help.nytimes.com/hc/en-us}{Help}
\item
  \href{https://www.nytimes.com/subscription?campaignId=37WXW}{Subscriptions}
\end{itemize}
