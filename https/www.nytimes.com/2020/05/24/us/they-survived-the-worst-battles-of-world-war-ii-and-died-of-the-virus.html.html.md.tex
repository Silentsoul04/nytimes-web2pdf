Sections

SEARCH

\protect\hyperlink{site-content}{Skip to
content}\protect\hyperlink{site-index}{Skip to site index}

\href{https://www.nytimes.com/section/us}{U.S.}

\href{https://myaccount.nytimes.com/auth/login?response_type=cookie\&client_id=vi}{}

\href{https://www.nytimes.com/section/todayspaper}{Today's Paper}

\href{/section/us}{U.S.}\textbar{}They Survived the Worst Battles of
World War II. And Died of the Virus.

\url{https://nyti.ms/3d14PDp}

\begin{itemize}
\item
\item
\item
\item
\item
\end{itemize}

\href{https://www.nytimes.com/news-event/coronavirus?action=click\&pgtype=Article\&state=default\&region=TOP_BANNER\&context=storylines_menu}{The
Coronavirus Outbreak}

\begin{itemize}
\tightlist
\item
  live\href{https://www.nytimes.com/2020/08/01/world/coronavirus-covid-19.html?action=click\&pgtype=Article\&state=default\&region=TOP_BANNER\&context=storylines_menu}{Latest
  Updates}
\item
  \href{https://www.nytimes.com/interactive/2020/us/coronavirus-us-cases.html?action=click\&pgtype=Article\&state=default\&region=TOP_BANNER\&context=storylines_menu}{Maps
  and Cases}
\item
  \href{https://www.nytimes.com/interactive/2020/science/coronavirus-vaccine-tracker.html?action=click\&pgtype=Article\&state=default\&region=TOP_BANNER\&context=storylines_menu}{Vaccine
  Tracker}
\item
  \href{https://www.nytimes.com/interactive/2020/07/29/us/schools-reopening-coronavirus.html?action=click\&pgtype=Article\&state=default\&region=TOP_BANNER\&context=storylines_menu}{What
  School May Look Like}
\item
  \href{https://www.nytimes.com/live/2020/07/31/business/stock-market-today-coronavirus?action=click\&pgtype=Article\&state=default\&region=TOP_BANNER\&context=storylines_menu}{Economy}
\end{itemize}

Advertisement

\protect\hyperlink{after-top}{Continue reading the main story}

Supported by

\protect\hyperlink{after-sponsor}{Continue reading the main story}

\hypertarget{they-survived-the-worst-battles-of-world-war-ii-and-died-of-the-virus}{%
\section{They Survived the Worst Battles of World War II. And Died of
the
Virus.}\label{they-survived-the-worst-battles-of-world-war-ii-and-died-of-the-virus}}

Inside the Holyoke Soldiers' Home was a man who had served as a jailer
to Hitler's top aide. A man who had rescued Japanese kamikaze pilots
from the sea. A man who carried memories of a concentration camp.

\includegraphics{https://static01.nyt.com/images/2020/05/22/us/00virus-veterans01/merlin_172465614_55e405de-aba8-4122-9568-9d3be2ee000d-articleLarge.jpg?quality=75\&auto=webp\&disable=upscale}

\href{https://www.nytimes.com/by/ellen-barry}{\includegraphics{https://static01.nyt.com/images/2018/10/08/multimedia/author-ellen-barry/author-ellen-barry-thumbLarge.png}}

By \href{https://www.nytimes.com/by/ellen-barry}{Ellen Barry}

\begin{itemize}
\item
  Published May 24, 2020Updated May 25, 2020
\item
  \begin{itemize}
  \item
  \item
  \item
  \item
  \item
  \end{itemize}
\end{itemize}

HOLYOKE, Mass. --- In 1945, James Leach Miller returned from the war and
said nothing.

He said nothing about it to his wife, not for 64 years of marriage. He
folded up his Army uniform, with the medals still pinned to it, and put
it in the basement, where his older boy would sometimes take it out to
play soldiers.

He joined the fire department. He went to church on Sundays. He never
complained.

``That generation, they didn't air their problems,'' said his younger
son, Michael. ``He would say, `It was not a good time. I've had better
times.' He would not embellish.''

Mr. Miller was already in his 70s when he began to tell Michael, an Air
Force flight engineer, little bits about landing on Omaha Beach on
D-Day. ``Fragments would come out,'' his son said. The deafening roar as
they waited for the beach to clear, crowded into a landing ship with
other 21-year-olds. A blur that lasted 24 hours. The buzz-drone of
Messerschmitts. Dust clouds. Mud.

Michael once offered to take him back to Normandy --- World War II
veterans were making the journey --- but his father shook his head and
said, ``I've been there once.''

This story comes up for a reason. Mr. Miller, 96, who survived what was
for Americans the bloodiest battle of World War II,
\href{https://obits.masslive.com/obituaries/masslive/obituary.aspx?n=james-miller\&pid=195864379\&fhid=25292}{died
of complications from the coronavirus on March 30} inside the
\href{https://www.mass.gov/orgs/soldiers-home-in-holyoke}{Holyoke
Soldiers' Home}. The virus has spread in more than 40 veterans' homes in
more than 20 states, leading to the deaths of at least 300 people.

The
\href{https://www.nytimes.com/2020/03/31/us/coronavirus-holyoke-veterans-home.html}{conditions
inside} the 247-bed, state-run home,
\href{https://www.nytimes.com/aponline/2020/05/22/us/ap-us-virus-outbreak-veterans-home.html}{where
Mr. Miller had lived for five years}, were
\href{https://www.gazettenet.com/Employees-allege-lax-COVID-19-safety-protocols-at-Holyoke-Soldiers-Home-33631412}{so
chaotic} that his children cannot recount them without breaking down.

When Mr. Miller lay weak and gasping that weekend, his two daughters, in
a car in the parking lot, pleaded with a nurse on duty over an iPhone to
give him morphine or atropine to relieve his suffering. ``She said, `we
can't do it,' and she started to cry,'' said his daughter Linda McKee.
``There was no one there giving orders.''

Michael Miller, at his father's bedside, did the only thing he could do
--- moistened his lips with a sponge on a wooden stick.

``At that point, he was choking,'' Ms. McKee said. ``He died with no
care whatsoever.''

Image

James Leach Miller.

The question of what went wrong at the Holyoke Soldiers' Home will be
with Massachusetts for a long time.

With
\href{https://www.wbur.org/news/2020/05/01/holyoke-soldiers-home-coronavirus-cases-deaths-outage-families}{scarce
protective gear} and a shortage of staff, the facility's administrators
\href{https://www.masslive.com/coronavirus/2020/05/staffing-concerns-predated-coronavirus-outbreak-at-holyoke-soldiers-home.html}{combined
wards of infected and uninfected men}, and the
\href{https://www.propublica.org/article/superintendent-bragged-about-va-review-of-short-staffed-soldiers-home-two-months-later-73-veterans-are-dead}{virus
spread quickly} through a fragile population.

Of the 210 veterans who were living in the facility in late March, 89
are now dead, 74 having tested positive for the virus. Almost
three-quarters of the veterans inside were infected. It is one of the
highest death tolls of any end-of-life facility in the country.

Multiple investigations have been opened, several of which seek to
determine whether state officials should be charged with negligence
under civil or criminal law. The facility's superintendent, Bennett
Walsh,
\href{https://www.bostonglobe.com/2020/05/02/metro/holyoke-soldiers-home-has-been-overlooked-understaffed-led-by-inexperience/}{a
retired Marine Corps lieutenant colonel}with no nursing home experience,
was placed on administrative leave on March 30.

\hypertarget{latest-updates-global-coronavirus-outbreak}{%
\section{\texorpdfstring{\href{https://www.nytimes.com/2020/08/01/world/coronavirus-covid-19.html?action=click\&pgtype=Article\&state=default\&region=MAIN_CONTENT_1\&context=storylines_live_updates}{Latest
Updates: Global Coronavirus
Outbreak}}{Latest Updates: Global Coronavirus Outbreak}}\label{latest-updates-global-coronavirus-outbreak}}

Updated 2020-08-02T07:42:09.613Z

\begin{itemize}
\tightlist
\item
  \href{https://www.nytimes.com/2020/08/01/world/coronavirus-covid-19.html?action=click\&pgtype=Article\&state=default\&region=MAIN_CONTENT_1\&context=storylines_live_updates\#link-34047410}{The
  U.S. reels as July cases more than double the total of any other
  month.}
\item
  \href{https://www.nytimes.com/2020/08/01/world/coronavirus-covid-19.html?action=click\&pgtype=Article\&state=default\&region=MAIN_CONTENT_1\&context=storylines_live_updates\#link-780ec966}{Top
  U.S. officials work to break an impasse over the federal jobless
  benefit.}
\item
  \href{https://www.nytimes.com/2020/08/01/world/coronavirus-covid-19.html?action=click\&pgtype=Article\&state=default\&region=MAIN_CONTENT_1\&context=storylines_live_updates\#link-2bc8948}{Its
  outbreak untamed, Melbourne goes into even greater lockdown.}
\end{itemize}

\href{https://www.nytimes.com/2020/08/01/world/coronavirus-covid-19.html?action=click\&pgtype=Article\&state=default\&region=MAIN_CONTENT_1\&context=storylines_live_updates}{See
more updates}

More live coverage:
\href{https://www.nytimes.com/live/2020/07/31/business/stock-market-today-coronavirus?action=click\&pgtype=Article\&state=default\&region=MAIN_CONTENT_1\&context=storylines_live_updates}{Markets}

But many in the state are
\href{https://www.bostonglobe.com/2020/05/12/opinion/did-baker-administration-pay-lip-service-holyoke-soldiers-home/}{revisiting
decisions} made since 2015, when a moderate, technocratic Republican
governor, Charlie Baker, was elected on
\href{https://www.bostonglobe.com/metro/2016/12/06/baker-cut-million-from-state-budget/mV9k1k6H4Ce9xSkqGC0sHM/story.html}{a
promise to rein in spending.}

The facility's budget increased by 14 percent over the last five years,
according to a spokesman for the state's health department. Even so,
there were
\href{https://pioneerinstitute.org/covid/holyoke-soldiers-home-study-targets-inadequate-nursing-home-staffing-standards/}{persistent
shortfalls in staffing}, and the local unions complained that workers
were frequently pressured to stay for unplanned double shifts. The
facility's previous superintendent
\href{https://www.masslive.com/news/2015/12/holyoke_soldiers_home_meeting_1.html}{stepped
down in 2015,} declaring that the home could not safely care for the
population on the existing budget.

All this was well known before the coronavirus arrived in the state this
spring, said Erin O'Brien, an associate professor of political science
at the University of Massachusetts, Boston.

``All these regular Massachusetts folks that are now outraged, I don't
disagree, but veterans' programs require funding,'' she said. ``When you
vote to shrink government, it has ramifications.''

\hypertarget{they-each-had-stories}{%
\subsection{They each had stories}\label{they-each-had-stories}}

In 1952, young men were returning to the industrial towns of western
Massachusetts
\href{https://www.bostonglobe.com/2020/04/01/metro/soldiers-homes-sprung-national-effort-care-wounded-aged-veterans/}{after
serving in World War II}. They were kids from poor families. And
\href{https://archives.lib.state.ma.us/bitstream/handle/2452/237911/ocm39986874-1949-SB-0575.pdf?sequence=1\&isAllowed=y}{they
were damaged}: shellshocked, learning to live without limbs, unable to
communicate what they had seen.

It was
\href{http://holyokemass.com/2010/04/27/dedication-of-soldiers-home-in-holyoke-attracts-15000/}{to
these men} that Gov. Paul Dever, who had fought in the war himself,
dedicated the Holyoke Soldiers' Home, promising to protect injured
veterans from what he called ``the scissors of false economy.''

Fifteen thousand people lined the streets for that day's parade, and the
facility --- built on a hill and illuminated with floodlights --- became
a source of great pride in this part of the state.

The men in its wards had some stories.

There was Emilio DiPalma, a retired crane operator, who died of the
coronavirus on April 8.

At 19, an Army staff sergeant, Mr. DiPalma had
\href{https://www.youtube.com/watch?v=CzlBPFDCNog}{guarded Hermann
Goering}, the driving force behind the Nazi concentration camps, during
the Nuremberg trials. Mr. DiPalma called him ``Hermann the German.''
They didn't get along.

\includegraphics{https://static01.nyt.com/images/2020/05/22/us/00virus-veterans02/merlin_172729797_88c4ed87-d8f0-4ba7-a980-5c5618d1fdb4-articleLarge.jpg?quality=75\&auto=webp\&disable=upscale}

In
\href{https://www.amazon.com/Just-Kid-Guard-Nuremberg-Trials-ebook/dp/B004HFRLJC}{his
memoir, ``Just a Kid, A Guard at the Nuremberg Trials,''}Mr. DiPalma
\href{https://www.witf.org/2020/05/22/remembering-her-father-a-wwii-veteran-who-stood-guard-at-the-nuremberg-trials/}{recalled
Goering as arrogant and uncooperative}, often berating him in rapid-fire
German. Goering used to ask his young guard to bring him cups of water,
which Mr. DiPalma poured out of a chlorinated pouch.

Goering hated the taste of it, and would grimace and hand it back,
remarking ``Bah, Amerikanisch.'' After a few rounds of this, ``I had had
it with Hermann's antics,'' Mr. DiPalma said.

So the next time, Mr. DiPalma brought him a cup of water from the
toilet. Mr. Goering drank it down, and said, ``Ah, gute wasser!''

``He smiled, and so did I,'' Mr. DiPalma wrote. ``I guess I felt it was
my little contribution to the war effort.''

There was
\href{https://www.legacy.com/obituaries/gazettenet/obituary.aspx?n=samuel-joseph-lococo\&pid=196051871\&fhid=15489}{Sam
Lococo}, a retired postal worker, who contracted coronavirus and died
April 16.

At 20, Mr. Lococo had joined the Navy
and\href{https://www.bostonherald.com/2020/04/20/final-salute-for-soldiers-home-vet-who-died-of-coronavirus/}{shipped
out to the South Pacific}. He lived in fear of attacks by Japanese
kamikaze pilots. And at the same time, he was part of a team that sent
out whaleboats to
\href{https://www.spokesman.com/stories/1995/jul/31/kamikaze-pilot-is-reunited-with-his-us-rescuers/}{rescue
these pilots after they had crashed into the Pacific}.

In
\href{https://forbeslibrary.org/store/product/touched-with-fire-an-american-community-in-world-war-ii/}{an
interview with a local historian}, he recalled looking into the face of
one of those battered and half-drowned men, and seeing terror.

``The Japanese had been taught that the Americans were savages, so
probably he was afraid of us,'' he said. ``He kept saying in English,
`You are going to kill me. You are going to kill me.''' They pulled him
from the sea, dressed his wounds in the sick bay, and transferred him to
the U.S.S. Lexington.

And this was the point of the story. ``We treated that pilot like a
king,'' Mr. Lococo said.

Image

At 20, Sam Lococo had joined the Navy and shipped out to the South
Pacific.

Then there were those like Mr. Miller, who didn't talk about the war.

``As far as his service, what he encountered in Europe, I really am at a
loss, ma'am,'' said his oldest son, James P. Miller. ``Dad probably just
didn't want to talk about it. It was past.''

But from time to time, he startled the people around him with his swift,
instinctive response to crisis, James Miller said.

There was a time when a lawn-mower blade flew off an engine in the shop
where he worked, and sliced into a man's leg so deeply that the other
workers started screaming and ran out, but Jim went to the hurt man and
bound him up, in his quiet way.

Michael recalls sitting with his father and a V.A. psychologist
screening him for signs of post-traumatic stress disorder. ``She said,
so what do you do if you're reading the newspaper and you see something
that upsets you? He said, `I turn the page and I read the funny
papers.'''

Only once, in fact, did he see his father overcome with emotion about
the war.

It was in the 1990s, and Jim Miller learned for the first time that
there were people who denied that the Holocaust had occurred. And Mr.
Miller --- a man who never got upset about anything --- was as angry as
his son had ever seen him. ``It's like he had a hot-point button,''
Michael said.

Mr. Miller dug out a box of old photos, and drove them to a small
Holocaust museum in Springfield, which
\href{https://collections.ushmm.org/search/catalog/irn43802}{eventually
sent them to the U.S. Holocaust Memorial Museum in Washington, D.C}.

They showed corpses lined up in front of the Nordhausen concentration
camp. It showed other things. Boxcars. Ovens. Bones.

\href{https://www.nytimes.com/news-event/coronavirus?action=click\&pgtype=Article\&state=default\&region=MAIN_CONTENT_3\&context=storylines_faq}{}

\hypertarget{the-coronavirus-outbreak-}{%
\subsubsection{The Coronavirus Outbreak
›}\label{the-coronavirus-outbreak-}}

\hypertarget{frequently-asked-questions}{%
\paragraph{Frequently Asked
Questions}\label{frequently-asked-questions}}

Updated July 27, 2020

\begin{itemize}
\item ~
  \hypertarget{should-i-refinance-my-mortgage}{%
  \paragraph{Should I refinance my
  mortgage?}\label{should-i-refinance-my-mortgage}}

  \begin{itemize}
  \tightlist
  \item
    \href{https://www.nytimes.com/article/coronavirus-money-unemployment.html?action=click\&pgtype=Article\&state=default\&region=MAIN_CONTENT_3\&context=storylines_faq}{It
    could be a good idea,} because mortgage rates have
    \href{https://www.nytimes.com/2020/07/16/business/mortgage-rates-below-3-percent.html?action=click\&pgtype=Article\&state=default\&region=MAIN_CONTENT_3\&context=storylines_faq}{never
    been lower.} Refinancing requests have pushed mortgage applications
    to some of the highest levels since 2008, so be prepared to get in
    line. But defaults are also up, so if you're thinking about buying a
    home, be aware that some lenders have tightened their standards.
  \end{itemize}
\item ~
  \hypertarget{what-is-school-going-to-look-like-in-september}{%
  \paragraph{What is school going to look like in
  September?}\label{what-is-school-going-to-look-like-in-september}}

  \begin{itemize}
  \tightlist
  \item
    It is unlikely that many schools will return to a normal schedule
    this fall, requiring the grind of
    \href{https://www.nytimes.com/2020/06/05/us/coronavirus-education-lost-learning.html?action=click\&pgtype=Article\&state=default\&region=MAIN_CONTENT_3\&context=storylines_faq}{online
    learning},
    \href{https://www.nytimes.com/2020/05/29/us/coronavirus-child-care-centers.html?action=click\&pgtype=Article\&state=default\&region=MAIN_CONTENT_3\&context=storylines_faq}{makeshift
    child care} and
    \href{https://www.nytimes.com/2020/06/03/business/economy/coronavirus-working-women.html?action=click\&pgtype=Article\&state=default\&region=MAIN_CONTENT_3\&context=storylines_faq}{stunted
    workdays} to continue. California's two largest public school
    districts --- Los Angeles and San Diego --- said on July 13, that
    \href{https://www.nytimes.com/2020/07/13/us/lausd-san-diego-school-reopening.html?action=click\&pgtype=Article\&state=default\&region=MAIN_CONTENT_3\&context=storylines_faq}{instruction
    will be remote-only in the fall}, citing concerns that surging
    coronavirus infections in their areas pose too dire a risk for
    students and teachers. Together, the two districts enroll some
    825,000 students. They are the largest in the country so far to
    abandon plans for even a partial physical return to classrooms when
    they reopen in August. For other districts, the solution won't be an
    all-or-nothing approach.
    \href{https://bioethics.jhu.edu/research-and-outreach/projects/eschool-initiative/school-policy-tracker/}{Many
    systems}, including the nation's largest, New York City, are
    devising
    \href{https://www.nytimes.com/2020/06/26/us/coronavirus-schools-reopen-fall.html?action=click\&pgtype=Article\&state=default\&region=MAIN_CONTENT_3\&context=storylines_faq}{hybrid
    plans} that involve spending some days in classrooms and other days
    online. There's no national policy on this yet, so check with your
    municipal school system regularly to see what is happening in your
    community.
  \end{itemize}
\item ~
  \hypertarget{is-the-coronavirus-airborne}{%
  \paragraph{Is the coronavirus
  airborne?}\label{is-the-coronavirus-airborne}}

  \begin{itemize}
  \tightlist
  \item
    The coronavirus
    \href{https://www.nytimes.com/2020/07/04/health/239-experts-with-one-big-claim-the-coronavirus-is-airborne.html?action=click\&pgtype=Article\&state=default\&region=MAIN_CONTENT_3\&context=storylines_faq}{can
    stay aloft for hours in tiny droplets in stagnant air}, infecting
    people as they inhale, mounting scientific evidence suggests. This
    risk is highest in crowded indoor spaces with poor ventilation, and
    may help explain super-spreading events reported in meatpacking
    plants, churches and restaurants.
    \href{https://www.nytimes.com/2020/07/06/health/coronavirus-airborne-aerosols.html?action=click\&pgtype=Article\&state=default\&region=MAIN_CONTENT_3\&context=storylines_faq}{It's
    unclear how often the virus is spread} via these tiny droplets, or
    aerosols, compared with larger droplets that are expelled when a
    sick person coughs or sneezes, or transmitted through contact with
    contaminated surfaces, said Linsey Marr, an aerosol expert at
    Virginia Tech. Aerosols are released even when a person without
    symptoms exhales, talks or sings, according to Dr. Marr and more
    than 200 other experts, who
    \href{https://academic.oup.com/cid/article/doi/10.1093/cid/ciaa939/5867798}{have
    outlined the evidence in an open letter to the World Health
    Organization}.
  \end{itemize}
\item ~
  \hypertarget{what-are-the-symptoms-of-coronavirus}{%
  \paragraph{What are the symptoms of
  coronavirus?}\label{what-are-the-symptoms-of-coronavirus}}

  \begin{itemize}
  \tightlist
  \item
    Common symptoms
    \href{https://www.nytimes.com/article/symptoms-coronavirus.html?action=click\&pgtype=Article\&state=default\&region=MAIN_CONTENT_3\&context=storylines_faq}{include
    fever, a dry cough, fatigue and difficulty breathing or shortness of
    breath.} Some of these symptoms overlap with those of the flu,
    making detection difficult, but runny noses and stuffy sinuses are
    less common.
    \href{https://www.nytimes.com/2020/04/27/health/coronavirus-symptoms-cdc.html?action=click\&pgtype=Article\&state=default\&region=MAIN_CONTENT_3\&context=storylines_faq}{The
    C.D.C. has also} added chills, muscle pain, sore throat, headache
    and a new loss of the sense of taste or smell as symptoms to look
    out for. Most people fall ill five to seven days after exposure, but
    symptoms may appear in as few as two days or as many as 14 days.
  \end{itemize}
\item ~
  \hypertarget{does-asymptomatic-transmission-of-covid-19-happen}{%
  \paragraph{Does asymptomatic transmission of Covid-19
  happen?}\label{does-asymptomatic-transmission-of-covid-19-happen}}

  \begin{itemize}
  \tightlist
  \item
    So far, the evidence seems to show it does. A widely cited
    \href{https://www.nature.com/articles/s41591-020-0869-5}{paper}
    published in April suggests that people are most infectious about
    two days before the onset of coronavirus symptoms and estimated that
    44 percent of new infections were a result of transmission from
    people who were not yet showing symptoms. Recently, a top expert at
    the World Health Organization stated that transmission of the
    coronavirus by people who did not have symptoms was ``very rare,''
    \href{https://www.nytimes.com/2020/06/09/world/coronavirus-updates.html?action=click\&pgtype=Article\&state=default\&region=MAIN_CONTENT_3\&context=storylines_faq\#link-1f302e21}{but
    she later walked back that statement.}
  \end{itemize}
\end{itemize}

``He wanted people to remember,'' Michael said. ``I think, having lived
through all the physical issues, the psychological issues, if someone
says it never really happened, he was like, `Oh, my gosh, you folks, you
have no idea.'''

\hypertarget{they-were-trying-to-do-their-jobs}{%
\subsection{`They were trying to do their
jobs'}\label{they-were-trying-to-do-their-jobs}}

Mr. Miller's children had worried about the Soldiers' Home, enough to
request repeated private meetings with its superintendent, Mr. Walsh.
The trouble, they said, was staffing.

``When you live through those cuts, and have someone physically there,
you feel it every day,'' Ms. McKee said.

``They were trying to do their jobs,'' she said of the staff, ``they
just didn't have the means.''

Image

A cleaning crew outside of the soldiers' home in March. Three-quarters
of the people inside, staff and residents, were
infected.Credit...Jessica Rinaldi/The Boston Globe, via Associated Press

The home had
\href{https://www.propublica.org/article/superintendent-bragged-about-va-review-of-short-staffed-soldiers-home-two-months-later-73-veterans-are-dead}{passed
three successive yearly inspections}, meeting or provisionally meeting
the standards set by the U.S. Department of Veterans Affairs. But the
union representing most of the staff, Chapter 888 of
S.E.I.U.,\href{https://www.cnn.com/2020/04/06/us/holyoke-soldiers-home-coronavirus/index.html}{warned
persistently} that the facility was operating at 80 percent staffing
levels..

By March 14, the home was closed to most visitors, like most nursing
facilities in the state. A man in a dementia unit began showing
symptoms, declining so fast that it alarmed Joseph Ramirez, the vice
chair of the union chapter.

``We're used to seeing death, we know what it looks like when it comes,
but I was in shock, I was just like, `Oh, my god,''' he said. The man
was not fully isolated, and staff who treated him were rotated to other
units. ``What they had us doing, we were spreading it around,'' he said.

Image

A view of the Soldiers' Home atop the hill in Holyoke, Mass.Credit...Cj
Gunther/EPA, via Shutterstock

By the third week of March, a quarter of the staff was not reporting to
work,
\href{https://www.gazettenet.com/Soldiers-Home-superintendent-responds-33796782}{Mr.
Walsh has said through his lawyer}. To accommodate the low staffing,
medical staff decided to consolidate two units, crowding together
infected and uninfected veterans.

Mr. Walsh has said his superiors approved that decision, and were
routinely updated on the distress the facility was in. He said he had
called for help from the National Guard, but been refused.

``No one was kept in the dark,'' he said in a statement.

Gov. Baker has said little about these assertions, citing an ongoing
investigation.

Brooke Karanovich, a spokeswoman for the state Executive Office of
Health and Human services, called the deaths at the Soldiers' Home ``a
reminder of the insidious nature of Covid-19.''

She added: ``We are deeply saddened by the extent of the outbreak and
the loss of life."

As for Mr. Miller's children, they have trouble describing that last
weekend without crying.

``We're very bitter because of the way he died,'' said Ms. McKee.

She and her sister, Susan, sat in the parking lot, peering into her
father's room through her brother's iPhone. She heard spasms of coughing
from her father's roommates; two of the three would die that weekend.
She saw a large refrigerated truck pull up to a loading dock in the back
of the facility, for the bodies.

``It was complete panic,'' Ms. McKee said. ``It was pandemonium. Nobody
knew where to turn. ''

Inside, Michael sat with his father, holding his hand and praying,
reassuring him that he wasn't alone. Watching him breathe, stop
breathing, and start breathing again.

``I wouldn't wish that upon anybody,'' he said. ``It's something I will
remember for the rest of my life.''

Mr. Miller died on March 30, on the day when a cascade of scrutiny began
to fall on the facility. From his father's bedside, Michael could see a
group of public health officials making their way through the units.

But his attention was with his father, who was breathing but no longer
responding, and the strangeness of surviving Omaha Beach to die that
way.

``That's the irony, he landed on Normandy beach and your chances of
survival weren't great,'' he said. ``And he made it.''

Advertisement

\protect\hyperlink{after-bottom}{Continue reading the main story}

\hypertarget{site-index}{%
\subsection{Site Index}\label{site-index}}

\hypertarget{site-information-navigation}{%
\subsection{Site Information
Navigation}\label{site-information-navigation}}

\begin{itemize}
\tightlist
\item
  \href{https://help.nytimes.com/hc/en-us/articles/115014792127-Copyright-notice}{©~2020~The
  New York Times Company}
\end{itemize}

\begin{itemize}
\tightlist
\item
  \href{https://www.nytco.com/}{NYTCo}
\item
  \href{https://help.nytimes.com/hc/en-us/articles/115015385887-Contact-Us}{Contact
  Us}
\item
  \href{https://www.nytco.com/careers/}{Work with us}
\item
  \href{https://nytmediakit.com/}{Advertise}
\item
  \href{http://www.tbrandstudio.com/}{T Brand Studio}
\item
  \href{https://www.nytimes.com/privacy/cookie-policy\#how-do-i-manage-trackers}{Your
  Ad Choices}
\item
  \href{https://www.nytimes.com/privacy}{Privacy}
\item
  \href{https://help.nytimes.com/hc/en-us/articles/115014893428-Terms-of-service}{Terms
  of Service}
\item
  \href{https://help.nytimes.com/hc/en-us/articles/115014893968-Terms-of-sale}{Terms
  of Sale}
\item
  \href{https://spiderbites.nytimes.com}{Site Map}
\item
  \href{https://help.nytimes.com/hc/en-us}{Help}
\item
  \href{https://www.nytimes.com/subscription?campaignId=37WXW}{Subscriptions}
\end{itemize}
