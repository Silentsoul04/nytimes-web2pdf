Sections

SEARCH

\protect\hyperlink{site-content}{Skip to
content}\protect\hyperlink{site-index}{Skip to site index}

\href{https://www.nytimes.com/section/travel}{Travel}

\href{https://myaccount.nytimes.com/auth/login?response_type=cookie\&client_id=vi}{}

\href{https://www.nytimes.com/section/todayspaper}{Today's Paper}

\href{/section/travel}{Travel}\textbar{}Help! One Company Refused to
Refund Travelers More Than \$100,000

\url{https://nyti.ms/3ekH9dp}

\begin{itemize}
\item
\item
\item
\item
\item
\item
\end{itemize}

\href{https://www.nytimes.com/news-event/coronavirus?action=click\&pgtype=Article\&state=default\&region=TOP_BANNER\&context=storylines_menu}{The
Coronavirus Outbreak}

\begin{itemize}
\tightlist
\item
  live\href{https://www.nytimes.com/2020/08/04/world/coronavirus-cases.html?action=click\&pgtype=Article\&state=default\&region=TOP_BANNER\&context=storylines_menu}{Latest
  Updates}
\item
  \href{https://www.nytimes.com/interactive/2020/us/coronavirus-us-cases.html?action=click\&pgtype=Article\&state=default\&region=TOP_BANNER\&context=storylines_menu}{Maps
  and Cases}
\item
  \href{https://www.nytimes.com/interactive/2020/science/coronavirus-vaccine-tracker.html?action=click\&pgtype=Article\&state=default\&region=TOP_BANNER\&context=storylines_menu}{Vaccine
  Tracker}
\item
  \href{https://www.nytimes.com/2020/08/02/us/covid-college-reopening.html?action=click\&pgtype=Article\&state=default\&region=TOP_BANNER\&context=storylines_menu}{College
  Reopening}
\item
  \href{https://www.nytimes.com/live/2020/08/04/business/stock-market-today-coronavirus?action=click\&pgtype=Article\&state=default\&region=TOP_BANNER\&context=storylines_menu}{Economy}
\end{itemize}

Advertisement

\protect\hyperlink{after-top}{Continue reading the main story}

Supported by

\protect\hyperlink{after-sponsor}{Continue reading the main story}

Tripped Up

\hypertarget{help-one-company-refused-to-refund-travelers-more-than-100000}{%
\section{Help! One Company Refused to Refund Travelers More Than
\$100,000}\label{help-one-company-refused-to-refund-travelers-more-than-100000}}

Then our columnist intervened with the Boston-based tour operator
Overseas Adventure Travel.

\includegraphics{https://static01.nyt.com/images/2020/05/25/travel/26trippedup-OAT/25trippedup-OAT-articleLarge.jpg?quality=75\&auto=webp\&disable=upscale}

By Sarah Firshein

\begin{itemize}
\item
  Published May 25, 2020Updated May 27, 2020
\item
  \begin{itemize}
  \item
  \item
  \item
  \item
  \item
  \item
  \end{itemize}
\end{itemize}

\hypertarget{dear-tripped-up}{%
\subsubsection{\texorpdfstring{\textbf{Dear Tripped
Up,}}{Dear Tripped Up,}}\label{dear-tripped-up}}

My trip to Egypt with \href{https://www.oattravel.com/}{Overseas
Adventure Travel} was scheduled to depart in late March. O.A.T. canceled
the tour because of Covid-19, which was a relief. The company notified
me of the cancellation and offered either a credit for rebooking or a
full refund that would have included airfare. The next week, I learned
via email that I had been rebooked on the same trip next year. Unbeknown
to me, the stated policy had changed: O.A.T. was no longer offering
refunds. I told them that I had a medical condition and did not know
when --- or if --- I would be able to travel, and asked whether they
were just going to keep my \$17,500 if I couldn't travel by the end of
2021. Answer: Yes.

I feel I am being held hostage by O.A.T. How are they allowed to keep my
money? Roz

\hypertarget{hi-roz}{%
\subsubsection{Hi Roz,}\label{hi-roz}}

They're not. But, if the extraordinary number of reader complaints I
have received --- more than a dozen and counting --- are any indication,
they have been doing so anyway.

Overseas Adventure Travel is part of Grand Circle Corporation, a family
of travel companies based in Boston. The small-group and cruising
company has been
\href{https://www.oattravel.com/oat-difference/awards}{recognized
nationally.}

But Massachusetts happens to be one of only a handful of states with
specific laws that guarantee consumers protection against
\href{https://www.mass.gov/doc/940-cmr-15-the-sale-of-travel-services/download}{travel
sellers}, including requiring that tour operators offer the option of
cash refunds (in addition to vouchers or credits for rebooking) when
they fail to provide agreed-upon, paid-for services. According to
\href{https://www.mass.gov/doc/940-cmr-15-the-sale-of-travel-services/download}{the
law,} the cash refund must be ``an amount equal to the fair market
retail value of any undelivered, purchased travel service.''

Translation: When they cancel your trip, they are legally required to
offer you the money back.

But even when individual consumers know their rights, they have few
options at their immediate disposal when a customer-service
representative --- usually the only public-facing proxy for a company's
official or unofficial policies --- refuses to relent on refunds.

\hypertarget{latest-updates-global-coronavirus-outbreak}{%
\section{\texorpdfstring{\href{https://www.nytimes.com/2020/08/04/world/coronavirus-cases.html?action=click\&pgtype=Article\&state=default\&region=MAIN_CONTENT_1\&context=storylines_live_updates}{Latest
Updates: Global Coronavirus
Outbreak}}{Latest Updates: Global Coronavirus Outbreak}}\label{latest-updates-global-coronavirus-outbreak}}

Updated 2020-08-04T20:08:28.255Z

\begin{itemize}
\tightlist
\item
  \href{https://www.nytimes.com/2020/08/04/world/coronavirus-cases.html?action=click\&pgtype=Article\&state=default\&region=MAIN_CONTENT_1\&context=storylines_live_updates\#link-1228a480}{Novavax
  sees encouraging results from two studies of its experimental
  vaccine.}
\item
  \href{https://www.nytimes.com/2020/08/04/world/coronavirus-cases.html?action=click\&pgtype=Article\&state=default\&region=MAIN_CONTENT_1\&context=storylines_live_updates\#link-4825b93}{Public
  and private schools in Maryland and elsewhere are divided over
  in-person instruction.}
\item
  \href{https://www.nytimes.com/2020/08/04/world/coronavirus-cases.html?action=click\&pgtype=Article\&state=default\&region=MAIN_CONTENT_1\&context=storylines_live_updates\#link-4d1eafa8}{N.Y.C.'s
  health commissioner resigns after clashing with the mayor over the
  virus.}
\end{itemize}

\href{https://www.nytimes.com/2020/08/04/world/coronavirus-cases.html?action=click\&pgtype=Article\&state=default\&region=MAIN_CONTENT_1\&context=storylines_live_updates}{See
more updates}

More live coverage:
\href{https://www.nytimes.com/live/2020/08/04/business/stock-market-today-coronavirus?action=click\&pgtype=Article\&state=default\&region=MAIN_CONTENT_1\&context=storylines_live_updates}{Markets}

As \href{https://travellaw.com/}{Adam Anolik,} a San Francisco- based
travel-industry lawyer, explained over email, that's why oversight ---
forcing a company to comply with state laws --- can feel like an uphill
battle. ``The outcome can often turn on who cancels, which is why a lot
of suppliers and travelers are playing chicken right now. In reality,
many of these statutes are seldom enforced. This pandemic could cause
some of them to be dusted off,'' said Mr. Anolik.

This is the third
\href{https://www.nytimes.com/column/tripped-up}{Tripped Up} column in a
row that addresses the issue of refunds. Although travel has stopped and
is only starting up again --- slowly and in only a few destinations ---
the aftershocks of that screeching halt, brought on by the coronavirus,
continue to reverberate.

As travel companies now suffer a cash crunch, they are facing off with
travelers over
\href{https://www.nytimes.com/2020/05/12/travel/refunds-or-credits-travelers-and-businesses-face-off.html}{credits
and refunds}. Airlines are
\href{https://www.nytimes.com/2020/05/01/travel/trip-refund-airlines.html}{sidestepping}
refund regulations established by the United States Transportation
Department and the European Union, betting that negative press (and even
class-action lawsuits) are still preferable to negative-balance bank
accounts.

It's not hard to surmise, just by reading your email, what happened at
O.A.T.: the realization that issuing refunds en masse would bleed the
company dry. Tweaking the immortal words of Biggie: no money, mo'
problems.

To determine if my hunch was correct, I reached out to O.A.T. While they
didn't answer my question directly, I was able to recoup more than
\$100,000, collectively, for you and 10 other readers. Some got total
refunds, while others (including you) received partial refunds or
continue to wait for certain fees and sums to clear.

In an emailed statement, an O.A.T. spokeswoman said the company is
``working to improve our processes and to better address the needs of
each traveler whose trip was canceled or postponed due to the pandemic.
We are either rebooking travelers on another trip or providing a
refund.'' Since mid-March, she said, O.A.T. has refunded more than 5,000
travelers --- amounting to more than \$12 million.

TripAdvisor, ConsumerAffairs and other review sites show lingering
frustrations about the company's coronavirus policies, but several Times
readers, according to emails I've received, have made headway by filing
parallel complaints with the \href{https://www.bbb.org/}{Better Business
Bureau} and the
\href{https://www.mass.gov/how-to/file-a-consumer-complaint}{Massachusetts
Attorney General's office}. That agency said that it has received about
275 consumer complaints so far about canceled O.A.T. trips and has been
pressuring the company to comply with the state's travel-seller laws.

Other travelers looking for refunds may have a better chance if they're
dealing with companies headquartered within Hawaii, Washington state,
Illinois or California. In Hawaii, consumers have the right to a cash
refund (minus previously disclosed cancellation fees) within two weeks
of requesting one. Washington requires travel companies to issue cash
refunds within 14 days (or 30 days when the funds are already paid to a
vendor, as might be the case for a tour operator) when they cancel a
service. In Illinois, the Travel Promotion Consumer Protection Act was
created in the 1980s to safeguard against travel scams that promise
too-good-to-be-true deals. California, with some of the strongest
consumer protections in the country, requires cash refunds for
undelivered services within 30 days from whichever date is earliest: the
scheduled departure date, the date the refund is requested or the date
the service was canceled by the travel company. (Mr. Anolik has a
detailed breakdown of other state-by-state travel laws on
\href{https://www.travellaw.com/page/travel-law-faq}{his firm's
website.})

\href{https://www.nytimes.com/news-event/coronavirus?action=click\&pgtype=Article\&state=default\&region=MAIN_CONTENT_3\&context=storylines_faq}{}

\hypertarget{the-coronavirus-outbreak-}{%
\subsubsection{The Coronavirus Outbreak
›}\label{the-coronavirus-outbreak-}}

\hypertarget{frequently-asked-questions}{%
\paragraph{Frequently Asked
Questions}\label{frequently-asked-questions}}

Updated August 4, 2020

\begin{itemize}
\item ~
  \hypertarget{i-have-antibodies-am-i-now-immune}{%
  \paragraph{I have antibodies. Am I now
  immune?}\label{i-have-antibodies-am-i-now-immune}}

  \begin{itemize}
  \tightlist
  \item
    As of right
    now,\href{https://www.nytimes.com/2020/07/22/health/covid-antibodies-herd-immunity.html?action=click\&pgtype=Article\&state=default\&region=MAIN_CONTENT_3\&context=storylines_faq}{that
    seems likely, for at least several months.} There have been
    frightening accounts of people suffering what seems to be a second
    bout of Covid-19. But experts say these patients may have a
    drawn-out course of infection, with the virus taking a slow toll
    weeks to months after initial exposure. People infected with the
    coronavirus typically
    \href{https://www.nature.com/articles/s41586-020-2456-9}{produce}
    immune molecules called antibodies, which are
    \href{https://www.nytimes.com/2020/05/07/health/coronavirus-antibody-prevalence.html?action=click\&pgtype=Article\&state=default\&region=MAIN_CONTENT_3\&context=storylines_faq}{protective
    proteins made in response to an
    infection}\href{https://www.nytimes.com/2020/05/07/health/coronavirus-antibody-prevalence.html?action=click\&pgtype=Article\&state=default\&region=MAIN_CONTENT_3\&context=storylines_faq}{.
    These antibodies may} last in the body
    \href{https://www.nature.com/articles/s41591-020-0965-6}{only two to
    three months}, which may seem worrisome, but that's perfectly normal
    after an acute infection subsides, said Dr. Michael Mina, an
    immunologist at Harvard University. It may be possible to get the
    coronavirus again, but it's highly unlikely that it would be
    possible in a short window of time from initial infection or make
    people sicker the second time.
  \end{itemize}
\item ~
  \hypertarget{im-a-small-business-owner-can-i-get-relief}{%
  \paragraph{I'm a small-business owner. Can I get
  relief?}\label{im-a-small-business-owner-can-i-get-relief}}

  \begin{itemize}
  \tightlist
  \item
    The
    \href{https://www.nytimes.com/article/small-business-loans-stimulus-grants-freelancers-coronavirus.html?action=click\&pgtype=Article\&state=default\&region=MAIN_CONTENT_3\&context=storylines_faq}{stimulus
    bills enacted in March} offer help for the millions of American
    small businesses. Those eligible for aid are businesses and
    nonprofit organizations with fewer than 500 workers, including sole
    proprietorships, independent contractors and freelancers. Some
    larger companies in some industries are also eligible. The help
    being offered, which is being managed by the Small Business
    Administration, includes the Paycheck Protection Program and the
    Economic Injury Disaster Loan program. But lots of folks have
    \href{https://www.nytimes.com/interactive/2020/05/07/business/small-business-loans-coronavirus.html?action=click\&pgtype=Article\&state=default\&region=MAIN_CONTENT_3\&context=storylines_faq}{not
    yet seen payouts.} Even those who have received help are confused:
    The rules are draconian, and some are stuck sitting on
    \href{https://www.nytimes.com/2020/05/02/business/economy/loans-coronavirus-small-business.html?action=click\&pgtype=Article\&state=default\&region=MAIN_CONTENT_3\&context=storylines_faq}{money
    they don't know how to use.} Many small-business owners are getting
    less than they expected or
    \href{https://www.nytimes.com/2020/06/10/business/Small-business-loans-ppp.html?action=click\&pgtype=Article\&state=default\&region=MAIN_CONTENT_3\&context=storylines_faq}{not
    hearing anything at all.}
  \end{itemize}
\item ~
  \hypertarget{what-are-my-rights-if-i-am-worried-about-going-back-to-work}{%
  \paragraph{What are my rights if I am worried about going back to
  work?}\label{what-are-my-rights-if-i-am-worried-about-going-back-to-work}}

  \begin{itemize}
  \tightlist
  \item
    Employers have to provide
    \href{https://www.osha.gov/SLTC/covid-19/standards.html}{a safe
    workplace} with policies that protect everyone equally.
    \href{https://www.nytimes.com/article/coronavirus-money-unemployment.html?action=click\&pgtype=Article\&state=default\&region=MAIN_CONTENT_3\&context=storylines_faq}{And
    if one of your co-workers tests positive for the coronavirus, the
    C.D.C.} has said that
    \href{https://www.cdc.gov/coronavirus/2019-ncov/community/guidance-business-response.html}{employers
    should tell their employees} -\/- without giving you the sick
    employee's name -\/- that they may have been exposed to the virus.
  \end{itemize}
\item ~
  \hypertarget{should-i-refinance-my-mortgage}{%
  \paragraph{Should I refinance my
  mortgage?}\label{should-i-refinance-my-mortgage}}

  \begin{itemize}
  \tightlist
  \item
    \href{https://www.nytimes.com/article/coronavirus-money-unemployment.html?action=click\&pgtype=Article\&state=default\&region=MAIN_CONTENT_3\&context=storylines_faq}{It
    could be a good idea,} because mortgage rates have
    \href{https://www.nytimes.com/2020/07/16/business/mortgage-rates-below-3-percent.html?action=click\&pgtype=Article\&state=default\&region=MAIN_CONTENT_3\&context=storylines_faq}{never
    been lower.} Refinancing requests have pushed mortgage applications
    to some of the highest levels since 2008, so be prepared to get in
    line. But defaults are also up, so if you're thinking about buying a
    home, be aware that some lenders have tightened their standards.
  \end{itemize}
\item ~
  \hypertarget{what-is-school-going-to-look-like-in-september}{%
  \paragraph{What is school going to look like in
  September?}\label{what-is-school-going-to-look-like-in-september}}

  \begin{itemize}
  \tightlist
  \item
    It is unlikely that many schools will return to a normal schedule
    this fall, requiring the grind of
    \href{https://www.nytimes.com/2020/06/05/us/coronavirus-education-lost-learning.html?action=click\&pgtype=Article\&state=default\&region=MAIN_CONTENT_3\&context=storylines_faq}{online
    learning},
    \href{https://www.nytimes.com/2020/05/29/us/coronavirus-child-care-centers.html?action=click\&pgtype=Article\&state=default\&region=MAIN_CONTENT_3\&context=storylines_faq}{makeshift
    child care} and
    \href{https://www.nytimes.com/2020/06/03/business/economy/coronavirus-working-women.html?action=click\&pgtype=Article\&state=default\&region=MAIN_CONTENT_3\&context=storylines_faq}{stunted
    workdays} to continue. California's two largest public school
    districts --- Los Angeles and San Diego --- said on July 13, that
    \href{https://www.nytimes.com/2020/07/13/us/lausd-san-diego-school-reopening.html?action=click\&pgtype=Article\&state=default\&region=MAIN_CONTENT_3\&context=storylines_faq}{instruction
    will be remote-only in the fall}, citing concerns that surging
    coronavirus infections in their areas pose too dire a risk for
    students and teachers. Together, the two districts enroll some
    825,000 students. They are the largest in the country so far to
    abandon plans for even a partial physical return to classrooms when
    they reopen in August. For other districts, the solution won't be an
    all-or-nothing approach.
    \href{https://bioethics.jhu.edu/research-and-outreach/projects/eschool-initiative/school-policy-tracker/}{Many
    systems}, including the nation's largest, New York City, are
    devising
    \href{https://www.nytimes.com/2020/06/26/us/coronavirus-schools-reopen-fall.html?action=click\&pgtype=Article\&state=default\&region=MAIN_CONTENT_3\&context=storylines_faq}{hybrid
    plans} that involve spending some days in classrooms and other days
    online. There's no national policy on this yet, so check with your
    municipal school system regularly to see what is happening in your
    community.
  \end{itemize}
\end{itemize}

Back to O. A. T.: In a follow-up note a few weeks ago, you said that you
feel the company provides a ``good travel service.'' But, you wrote,
``pushing all the risk of uncertainty onto the client by refusing a
refund makes me fearful to ever do business with O.A.T. again.''

You raise an important point. Even though people can't travel right now,
many of us are continuing to dream about our next trip --- and that
means making conscious and subconscious decisions about which companies
to spend money with once the pandemic has passed.

\begin{center}\rule{0.5\linewidth}{\linethickness}\end{center}

\href{https://twitter.com/sfirshein?lang=en}{Sarah Firshein} is a
Brooklyn-based travel writer. If you need advice about a best-laid
travel plan that went awry,
\textbf{\href{mailto:travel@nytimes.com}{send an email to
travel@nytimes.com}.}

\begin{center}\rule{0.5\linewidth}{\linethickness}\end{center}

\textbf{WE CAN DREAM ABOUT TRAVEL} \emph{Follow New York Times Travel
on}
\href{https://www.instagram.com/nytimestravel/}{\emph{Instagram}}\emph{,}
\href{https://twitter.com/nytimestravel}{\emph{Twitter}} \emph{and}
\href{https://www.facebook.com/nytimestravel/}{\emph{Facebook}}\emph{.
And}
\href{https://www.nytimes.com/newsletters/traveldispatch?action=click\&module=inline\&pgtype=Article}{\emph{sign
up for our}} **
\href{https://www.nytimes.com/newsletters/traveldispatch}{\emph{Travel
Dispatch newsletter}}\emph{: Each week you'll receive tips on traveling
smarter, stories on hot destinations and access to photos from all over
the world.}

Advertisement

\protect\hyperlink{after-bottom}{Continue reading the main story}

\hypertarget{site-index}{%
\subsection{Site Index}\label{site-index}}

\hypertarget{site-information-navigation}{%
\subsection{Site Information
Navigation}\label{site-information-navigation}}

\begin{itemize}
\tightlist
\item
  \href{https://help.nytimes.com/hc/en-us/articles/115014792127-Copyright-notice}{©~2020~The
  New York Times Company}
\end{itemize}

\begin{itemize}
\tightlist
\item
  \href{https://www.nytco.com/}{NYTCo}
\item
  \href{https://help.nytimes.com/hc/en-us/articles/115015385887-Contact-Us}{Contact
  Us}
\item
  \href{https://www.nytco.com/careers/}{Work with us}
\item
  \href{https://nytmediakit.com/}{Advertise}
\item
  \href{http://www.tbrandstudio.com/}{T Brand Studio}
\item
  \href{https://www.nytimes.com/privacy/cookie-policy\#how-do-i-manage-trackers}{Your
  Ad Choices}
\item
  \href{https://www.nytimes.com/privacy}{Privacy}
\item
  \href{https://help.nytimes.com/hc/en-us/articles/115014893428-Terms-of-service}{Terms
  of Service}
\item
  \href{https://help.nytimes.com/hc/en-us/articles/115014893968-Terms-of-sale}{Terms
  of Sale}
\item
  \href{https://spiderbites.nytimes.com}{Site Map}
\item
  \href{https://help.nytimes.com/hc/en-us}{Help}
\item
  \href{https://www.nytimes.com/subscription?campaignId=37WXW}{Subscriptions}
\end{itemize}
