Sections

SEARCH

\protect\hyperlink{site-content}{Skip to
content}\protect\hyperlink{site-index}{Skip to site index}

\href{https://myaccount.nytimes.com/auth/login?response_type=cookie\&client_id=vi}{}

\href{https://www.nytimes.com/section/todayspaper}{Today's Paper}

How to Use Zoom Like a Theater or Film Professional

\href{https://nyti.ms/3gtYaE5}{https://nyti.ms/3gtYaE5}

\begin{itemize}
\item
\item
\item
\item
\item
\item
\end{itemize}

\href{https://www.nytimes.com/spotlight/at-home?action=click\&pgtype=Article\&state=default\&region=TOP_BANNER\&context=at_home_menu}{At
Home}

\begin{itemize}
\tightlist
\item
  \href{https://www.nytimes.com/2020/08/04/arts/television/sam-jay-netflix-special.html?action=click\&pgtype=Article\&state=default\&region=TOP_BANNER\&context=at_home_menu}{Watch:
  Sam Jay}
\item
  \href{https://www.nytimes.com/interactive/2020/at-home/even-more-reporters-editors-diaries-lists-recommendations.html?action=click\&pgtype=Article\&state=default\&region=TOP_BANNER\&context=at_home_menu}{Peruse:
  Reporters' Google Docs}
\item
  \href{https://www.nytimes.com/2020/08/04/dining/colombian-empanadas-carlos-gaviria.html?action=click\&pgtype=Article\&state=default\&region=TOP_BANNER\&context=at_home_menu}{Make:
  Empanadas}
\item
  \href{https://www.nytimes.com/2020/08/06/arts/design/street-art-nyc-george-floyd.html?action=click\&pgtype=Article\&state=default\&region=TOP_BANNER\&context=at_home_menu}{Explore:
  N.Y.C. Street Art}
\end{itemize}

Advertisement

\protect\hyperlink{after-top}{Continue reading the main story}

Supported by

\protect\hyperlink{after-sponsor}{Continue reading the main story}

\hypertarget{how-to-use-zoom-like-a-theater-or-film-professional}{%
\section{How to Use Zoom Like a Theater or Film
Professional}\label{how-to-use-zoom-like-a-theater-or-film-professional}}

Tips for putting your best face forward, if only for an office staff
meeting.

\includegraphics{https://static01.nyt.com/images/2020/05/27/t-magazine/27tmag-zoom/27tmag-zoom-articleLarge.jpg?quality=75\&auto=webp\&disable=upscale}

By \href{https://www.nytimes.com/by/alex-hawgood}{Alex Hawgood}

\begin{itemize}
\item
  Published May 27, 2020Updated June 18, 2020
\item
  \begin{itemize}
  \item
  \item
  \item
  \item
  \item
  \item
  \end{itemize}
\end{itemize}

Since the pandemic forced theaters to close their doors in mid-March,
there has been no shortage of noble attempts to recreate the magic of
the stage on streaming platforms including Zoom, Instagram Live and
YouTube (where the Broadway interview show
``\href{https://www.starsinthehouse.com/}{Stars in the House}'' streams
two theater productions daily). Virtual theater productions, of course,
are mediated through technology and thus not experienced as they would
be from the front mezzanine, and yet they're not fully polished or
always prerecorded, either. Part of the thrill of theater is its
immediacy and, along with it, the potential for something to fall to
pieces. That aspect remains perfectly intact, and recent performances
have been felled by late starts, spotty internet and A/V connections,
missed cues and poor lighting. ``It can all feel like the local portions
of the Jerry Lewis telethon,'' jokes the producer and writer Jonathan
Tolins. ``As though these actors might be in their laundry rooms.''

Surely, stumbles can be endearing --- who wouldn't delight in watching
Jake Gyllenhaal, bathed in yellowish light, belt ``Move On'' from
``\href{https://www.thehudsonbroadway.com/whatson/sunday-in-the-park-with-george/}{Sunday
in the Park With George}'' (1983) in front of an average-looking window
blind? --- and all the more so because they are relatable. Increasingly,
us non-thespians, too, are expected to present ourselves via digital
platforms, and to issue compelling performances, whether for a
conference meeting, a job interview, a friendly catch-up or an Instagram
Live segment about cutting a spouse's hair. Really, all the digital
world's a stage. So, for help navigating it, why not look to Broadway?

Luckily, in addition to trials, there have been triumphs. Sometimes this
has **** been on account of the choice of work. Intimate narratives with
homey settings, such as ``Buyer \& Cellar,'' Tolins's 2013 one-man play
about an actor hired to work as a faux shopkeeper in Barbra Streisand's
Malibu basement, a benefit revival of which Broadway.com livestreamed
last month, make for easier transfers, requiring less coordination and
perhaps less suspension of disbelief from their audiences. But that
production's team was also thoughtful about technical aspects --- in his
midtown apartment, Michael Urie, its star, had three ring lights
(circular fixtures that distribute light evenly), two cameras and one
cameraman (his partner, Ryan Spahn). A two-camera setup might be
overkill for your standard Zoom call, but chances are a few of the below
tips will feel doable and help you stand out from the crowd amid the
stranger-than-fiction reality we're all living.

\includegraphics{https://static01.nyt.com/images/2020/05/27/t-magazine/27tmag-zoom-02/27tmag-zoom-02-articleLarge.jpg?quality=75\&auto=webp\&disable=upscale}

\hypertarget{lights}{%
\subsubsection{\texorpdfstring{\textbf{Lights}}{Lights}}\label{lights}}

Clothes and makeup can be important --- dress the part but stay away
from houndstooth, which might register onscreen as a dizzying blur ---
but lighting, which can be the difference between looking like yourself
and a washed-out or shadowy version thereof, is a trickier art. On its
own, the fluorescent light emitted from digital devices is about as
flattering as that from a bug zapper. First, set your computer between
an external light source and your desk chair, so that your face acquires
a glow. Your computer can also give you a boost: While delivering the
introductory remarks for the streaming run of ``Buyer \& Cellar,''
Tolins had his computer's desktop background set to solid yellow to
create a warmer tone and add a bit of dimension to his complexion, which
is on the paler side. For those with darker skin tones, the comedic
performer \href{https://www.instagram.com/ziwef/?hl=en}{Ziwe Fumudoh},
who writes for the late-night comedy series ``Desus \& Mero'' and films
weekly Instagram Live shows from her Bushwick apartment, suggests ``a
whitish, bluish'' light, which can be achieved with Apple's night mode.
As of late, Fumudoh has also been using a
\href{https://fotodioxpro.com/products/vlog-led-ringlight}{Fotodiox ring
light} to remove pesky shadows and illuminate the eyes.

Not that a natural look is always the right one.
\href{http://www.alexbickel.com/}{Alex Bickel}, a film colorist
responsible for the iconic bluish tint in ``Moonlight'' (2016), recently
bought a
\href{https://www.usa.lighting.philips.com/consumer/smart-wifi-led}{smart
LED light from Philips} that, when installed in a floor lamp or a
flashlight, offers a seemingly infinite mix of colors and special
effects and transforms the whole room. (Bickel chose a flickering
lavender option for a kaleidoscopic thank-you video he messaged to a
friend, an emergency-room doctor in New York.) And
\href{http://samlevydp.com/}{Sam Levy}, a cinematographer who's worked
with Greta Gerwig and Frank Ocean, among others, and his wife, the
filmmaker Karen Cinorre, have been experimenting with taping colored gel
swatches by
\href{https://www.leefilters.com/lighting/colour-list.html}{Lee Filters}
over their camera lenses. These sorts of overlays are also popular with
Broadway technicians tasked with setting the mood onstage. Purple might
be used for a cheerful pantomime, the brand's website suggests, whereas
a Victorian melodrama might call for a pale green. Levy also recommends
using Photo Booth or your camera app to get a preview of your light and
look before an important call.

Image

Credit...Sofía Probert

\hypertarget{camera-and-set-and-sound}{%
\subsubsection{\texorpdfstring{\textbf{Camera (and Set and
Sound)}}{Camera (and Set and Sound)}}\label{camera-and-set-and-sound}}

You don't have to worry much about the camera device itself, which
however unkind, is essentially fixed when it comes to standard streaming
platforms. But there are other staging elements that benefit from
attention. ``Watching all of these new productions on Zoom or YouTube,
you can tell who knows how to film themselves and who doesn't,'' says
\href{https://www.saraisaacsoncasting.com/}{Sara Isaacson}, a Los
Angeles-based casting director who consults actors on the dos and don'ts
of self-taped auditions. ``I always tell my actors to make sure there
are no distractions behind them, like a refrigerator door or a dying
plant,'' she says. She recommends painting a small square of any
available wall space a neutral color so that the background won't take
away from the action at hand.

For sound purposes, you might want to cover your \emph{other} walls with
blankets. Soon after Alison Koch was brought on as the digital content
producer of ``Soundstage,'' Playwrights Horizons's audio-only play and
musical project, she says she realized that ``recording episodes while
huddled in a makeshift studio in a dressing room wasn't going to cut
it.'' And so the team decamped to professional studios, which naturally
offer top-of-the-line compressors, microphone preamplifiers and more.
But \href{https://www.instagram.com/gare_saw_bear/?hl=en}{Gary Atturio},
a sound engineer and mixer who has worked on both musical episodes for
the series, one being Kirsten Childs's ``Edge of Night'' (out next
month), says that some aspects of a studio can be imitated at home.
``You want a room that's very isolated --- no street noise, no air
conditioning, no cat meowing --- and then you want to eliminate that
boxy sound you'd get in a new, empty apartment.'' In addition to
blankets, furniture also helps dampen reverberation. ``It's actually
more about the room than the microphone,'' Atturio says, though decent
dedicated mics, which provide a cleaner, more focused sound than
internal laptop models, are
\href{https://www.bluedesigns.com/products/yeti/}{available} for just
over \$100.

But what if you need a set with drama, or simply want to enjoy the nice
weather and take a call on your porch or fire escape? Earlier this
month, the actor and producer
\href{https://www.instagram.com/erichbergen/?hl=en}{Erich Bergen} did a
location scout (via FaceTime) of Bette Midler's Dutchess County, N.Y.,
home for a virtual appearance the actress was set to make at a gala for
the \href{https://www.nyrp.org/}{New York Restoration Project}, a
nonprofit that Midler founded in 1995 to support city gardens and parks,
and landed, fittingly, on a corner of her garden. ``When filming
outside,'' he warns, ``be wary that an iPhone microphone can only carry
sound from so far away.'' In other words, speak into the mic. And,
should a great gust of wind or honking truck pass by, mute accordingly.

Image

Credit...Sofía Probert

\hypertarget{action}{%
\subsubsection{\texorpdfstring{\textbf{Action}}{Action}}\label{action}}

``Eye line is super important. I don't want to feel like I'm being
stared at directly in an audition, so it's better to look slightly to
the side of the lens,'' says Isaacson. Though she adds that she
occasionally encounters someone ``who refuses to look at me and only
shows their profile, which is really weird.'' On most devices, there's a
green light right next to the lens, which is a good place to gaze (make
sure that your camera is sitting just above eye-level, which will show
your face at a favorable angle and encourage good posture). And, it goes
without saying, try not to stare at the image of your visage projected
back at you. (If you can't help yourself, you may be meant for the stage
after all --- nevertheless, opt to hide self-view.) Another place not to
look would be your notes --- Tolins, for example, taped a printout of
his speech to his monitor --- or the room around you. The camera doesn't
lie, and you never know when someone might pin your image and get a
detailed view of you barely listening, or nervously ad-libbing.

The last step might be to not think about anything you just read. ``Give
yourself time to get set up and make sure everything works. That peace
of mind will come through and allow you to forget about the devices and
just focus on connecting with people,'' says Levy. Finally, if something
does go wrong, just do your best to recover and get on with the show.
``I don't like that we're here,'' Levy adds, ``but I like the honesty
and the human persistence built into this moment.''

Advertisement

\protect\hyperlink{after-bottom}{Continue reading the main story}

\hypertarget{site-index}{%
\subsection{Site Index}\label{site-index}}

\hypertarget{site-information-navigation}{%
\subsection{Site Information
Navigation}\label{site-information-navigation}}

\begin{itemize}
\tightlist
\item
  \href{https://help.nytimes.com/hc/en-us/articles/115014792127-Copyright-notice}{©~2020~The
  New York Times Company}
\end{itemize}

\begin{itemize}
\tightlist
\item
  \href{https://www.nytco.com/}{NYTCo}
\item
  \href{https://help.nytimes.com/hc/en-us/articles/115015385887-Contact-Us}{Contact
  Us}
\item
  \href{https://www.nytco.com/careers/}{Work with us}
\item
  \href{https://nytmediakit.com/}{Advertise}
\item
  \href{http://www.tbrandstudio.com/}{T Brand Studio}
\item
  \href{https://www.nytimes.com/privacy/cookie-policy\#how-do-i-manage-trackers}{Your
  Ad Choices}
\item
  \href{https://www.nytimes.com/privacy}{Privacy}
\item
  \href{https://help.nytimes.com/hc/en-us/articles/115014893428-Terms-of-service}{Terms
  of Service}
\item
  \href{https://help.nytimes.com/hc/en-us/articles/115014893968-Terms-of-sale}{Terms
  of Sale}
\item
  \href{https://spiderbites.nytimes.com}{Site Map}
\item
  \href{https://help.nytimes.com/hc/en-us}{Help}
\item
  \href{https://www.nytimes.com/subscription?campaignId=37WXW}{Subscriptions}
\end{itemize}
