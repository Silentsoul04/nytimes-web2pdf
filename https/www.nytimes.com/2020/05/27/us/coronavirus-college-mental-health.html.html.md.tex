\href{/section/us}{U.S.}\textbar{}Scattered to the Winds, College
Students Mourn Lost Semester

\url{https://nyti.ms/2M4dMQO}

\begin{itemize}
\item
\item
\item
\item
\item
\item
\end{itemize}

\href{https://www.nytimes.com/news-event/coronavirus?action=click\&pgtype=Article\&state=default\&region=TOP_BANNER\&context=storylines_menu}{The
Coronavirus Outbreak}

\begin{itemize}
\tightlist
\item
  live\href{https://www.nytimes.com/2020/08/02/world/coronavirus-updates.html?action=click\&pgtype=Article\&state=default\&region=TOP_BANNER\&context=storylines_menu}{Latest
  Updates}
\item
  \href{https://www.nytimes.com/interactive/2020/us/coronavirus-us-cases.html?action=click\&pgtype=Article\&state=default\&region=TOP_BANNER\&context=storylines_menu}{Maps
  and Cases}
\item
  \href{https://www.nytimes.com/interactive/2020/science/coronavirus-vaccine-tracker.html?action=click\&pgtype=Article\&state=default\&region=TOP_BANNER\&context=storylines_menu}{Vaccine
  Tracker}
\item
  \href{https://www.nytimes.com/interactive/2020/07/29/us/schools-reopening-coronavirus.html?action=click\&pgtype=Article\&state=default\&region=TOP_BANNER\&context=storylines_menu}{What
  School May Look Like}
\item
  \href{https://www.nytimes.com/live/2020/07/31/business/stock-market-today-coronavirus?action=click\&pgtype=Article\&state=default\&region=TOP_BANNER\&context=storylines_menu}{Economy}
\end{itemize}

\includegraphics{https://static01.nyt.com/images/2020/05/13/us/00virus-collegestress1/merlin_172453281_6653ffa6-5684-4a91-b93c-68dc3d270761-articleLarge.jpg?quality=75\&auto=webp\&disable=upscale}

Sections

\protect\hyperlink{site-content}{Skip to
content}\protect\hyperlink{site-index}{Skip to site index}

\hypertarget{scattered-to-the-winds-college-students-mourn-lost-semester}{%
\section{Scattered to the Winds, College Students Mourn Lost
Semester}\label{scattered-to-the-winds-college-students-mourn-lost-semester}}

The unpredictability of life during the pandemic has been hard on
everyone, but it has caused particular stress for many college students.

Joshua McCartney, 24, on the grounds of his family's camp in Maine,
where he is sheltering from the pandemic.~Credit...Tristan Spinski for
The New York Times

Supported by

\protect\hyperlink{after-sponsor}{Continue reading the main story}

\href{https://www.nytimes.com/by/anemona-hartocollis}{\includegraphics{https://static01.nyt.com/images/2018/06/13/multimedia/author-anemona-hartocollis/author-anemona-hartocollis-thumbLarge-v3.jpg}}

By \href{https://www.nytimes.com/by/anemona-hartocollis}{Anemona
Hartocollis}

\begin{itemize}
\item
  Published May 27, 2020Updated July 3, 2020
\item
  \begin{itemize}
  \item
  \item
  \item
  \item
  \item
  \item
  \end{itemize}
\end{itemize}

From his little room in the empty summer camp where he was sheltering
from \href{https://www.nytimes.com/news-event/coronavirus}{the
pandemic}, like a character in an apocalyptic movie, Joshua McCartney
wrote a letter to the provost of Denison University, where he was a
senior.

His classmates had been scattered to the four winds, breaking up the
community they depended on for support. They were anxious and stressed,
he said, and the
\href{https://www.nytimes.com/2020/07/03/your-money/students-unemployment-insurance-coronavirus.html}{university}
did not seem to get it.

``The sheer volume of panic attacks, nightmares and tears that have been
related to me in the past two weeks is staggering,'' he wrote in a
letter co-signed by 23 other students.

Stress and
\href{https://www.nytimes.com/2020/06/25/sports/ncaafootball/college-football-coronavirus-cases.html}{college}
seem to go hand in hand, but the sudden emptying out of campuses across
the United States has increased the anxiety for many students, who find
themselves isolated from their peers,
\href{https://www.nytimes.com/2020/04/08/us/coronavirus-college-students.html}{packed
together with their parents} and full of worry over what the future
holds.

One in five students say their mental health has significantly worsened
during the
\href{https://www.nytimes.com/2020/07/03/your-money/students-unemployment-insurance-coronavirus.html}{coronavirus}
pandemic, according to a survey conducted in April by Active Minds, a
mental health advocacy group focused on college students.

Almost all of the 2,000 students surveyed said the virus had caused them
stress or anxiety, and 80 percent said they had experienced loneliness
or isolation because of it. Nearly half said that a major source of
stress was the financial impact of the virus on them or their families.

As the virus spread, young people shifted from talking about getting a
degree, finding a job and falling in love to ``pandemic anxiety,'' said
Boaz Gaon, founder of the mental-health-oriented social networking app
Wisdo, which has 20,000 active college-age users. ``They've developed
sleeplessness,'' Mr. Gaon said. ``They're looking for new friends,
purpose in life.''

Of course, most people are resilient, and sadness and anxiety can be an
understandable reaction to this wrenching moment. ``We don't need to
tell people they're suffering from a mental health problem when they're
having an appropriate response to very challenging circumstances,'' said
Victor Schwartz, a psychiatrist with the Jed Foundation, who advises
colleges on mental health issues.

Still, Mr. McCartney can testify that the feelings he and others are
experiencing are intense. To prove it, he has collected a series of
testimonials from his friends:

\begin{quote}
``The things that have gotten me through the roughest times in college
have been the community, my friends and roommates. I have lost that
support network.''

``I just spent an hour crying, and that was my self-care of the month.''

``Josh, I'm serious --- I don't think I can do this much more. I have no
motivation, and I can't work this hard for much longer.''
\end{quote}

\includegraphics{https://static01.nyt.com/images/2020/05/13/us/00virus-collegestress2/merlin_172434852_0fb741e9-32b5-4d80-85a9-a7c799b85f3a-articleLarge.jpg?quality=75\&auto=webp\&disable=upscale}

\hypertarget{i-felt-really-bad-because-i-could-relate-to-it}{%
\subsection{`I felt really bad because I could relate to
it.'}\label{i-felt-really-bad-because-i-could-relate-to-it}}

While Maite Rodriguez was away at college, her mother moved into her
childhood room. A junior at New York University, Ms. Rodriguez returned
home to Newark in March to find her mother's Bible studies laid out on
her old desk, and her mother's jewelry scattered around her old room.

``I literally don't have a space in my house that's my own,'' she said.

She could hardly blame her parents. They were down, too. They live over
the family restaurant and bar, which was closed because of the virus.
Her mother went downstairs nonetheless to clean it every day. Her father
slept most of the day and watched TV all night, the schedule he kept
when the bar was open. Ms. Rodriguez found her sleep habits changing to
match.

\hypertarget{latest-updates-global-coronavirus-outbreak}{%
\section{\texorpdfstring{\href{https://www.nytimes.com/2020/08/01/world/coronavirus-covid-19.html?action=click\&pgtype=Article\&state=default\&region=MAIN_CONTENT_1\&context=storylines_live_updates}{Latest
Updates: Global Coronavirus
Outbreak}}{Latest Updates: Global Coronavirus Outbreak}}\label{latest-updates-global-coronavirus-outbreak}}

Updated 2020-08-02T17:52:35.962Z

\begin{itemize}
\tightlist
\item
  \href{https://www.nytimes.com/2020/08/01/world/coronavirus-covid-19.html?action=click\&pgtype=Article\&state=default\&region=MAIN_CONTENT_1\&context=storylines_live_updates\#link-34047410}{The
  U.S. reels as July cases more than double the total of any other
  month.}
\item
  \href{https://www.nytimes.com/2020/08/01/world/coronavirus-covid-19.html?action=click\&pgtype=Article\&state=default\&region=MAIN_CONTENT_1\&context=storylines_live_updates\#link-780ec966}{Top
  U.S. officials work to break an impasse over the federal jobless
  benefit.}
\item
  \href{https://www.nytimes.com/2020/08/01/world/coronavirus-covid-19.html?action=click\&pgtype=Article\&state=default\&region=MAIN_CONTENT_1\&context=storylines_live_updates\#link-2bc8948}{Its
  outbreak untamed, Melbourne goes into even greater lockdown.}
\end{itemize}

\href{https://www.nytimes.com/2020/08/01/world/coronavirus-covid-19.html?action=click\&pgtype=Article\&state=default\&region=MAIN_CONTENT_1\&context=storylines_live_updates}{See
more updates}

More live coverage:
\href{https://www.nytimes.com/live/2020/07/31/business/stock-market-today-coronavirus?action=click\&pgtype=Article\&state=default\&region=MAIN_CONTENT_1\&context=storylines_live_updates}{Markets}

Her family joked that she was the breadwinner, because Ms. Rodriguez was
still being paid \$150 every two weeks for the student job she could no
longer do.

She faithfully took her classes online, sitting on the bed she shared
with her mother. Instead of clearing her head by walking across campus
between classes, she walked around the kitchen.

She noticed that some of her peers rarely checked into class. But she
could no longer knock on their dorm room doors to ask how they were
doing.

Instead, she caught glimpses of their home lives on Zoom --- upscale
residences with floor-to-ceiling windows or a student sitting stoically
through a screaming match between his parents.

Out of sympathy for him, Ms. Rodriguez turned off the sound on her
computer. ``I felt really bad because I could relate to it,'' she said.

\hypertarget{i-had-some-kind-of-purpose}{%
\subsection{`I had some kind of
purpose.'}\label{i-had-some-kind-of-purpose}}

When they sent students home, colleges began offering mental health
counseling remotely. But requests for counseling showed a decrease
rather than an increase, according to Kim Coplin, the Denison provost to
whom Mr. McCartney wrote his letter. She said other colleges and
universities showed a similar pattern. ``We don't have any data
available at this time to give us definitive answers,'' Dr. Coplin said.

Students say the answer is obvious --- lack of privacy at home.

Joshua Osvaldo Arrayales had a therapist and a nutritionist at N.Y.U.,
where he went through a gender transition. But now that he has returned
home to San Diego, he no longer consults them. ``They did tell us we
were allowed to videoconference,'' he said. ``I thought it was best to
maybe not openly talk about how my parents have made a lot of my life
difficult.''

He has been trying to wean himself from his anti-anxiety medication,
because it has been hard to get a steady supply.

He dreaded going home to face his family.

``I feel like being thrown into my home life, which over the years has
exacerbated both my anxiety and my depression,'' he said, ``I'm going to
have a setback.''

His parents are still at work, his mother at a grocery store, his father
in the food service industry. Mr. Arrayales takes care of his little
sister and walks the dog. He has mixed feelings about being home --- he
wants to take care of his family and be taken care of, yet he misses his
independence.

But he is anxious to regain his life and identity in college. ``I think
it's just like, I feel really lonely,'' he said. ``I think that in New
York, going to school, I had some kind of purpose, in a way, because I
had to be at work. I had to be at class. I made plans with people, so I
had to be there. Now it's like I don't have to be anywhere, and even if
I do, I'm already there. I'm already at home.''

\href{https://www.nytimes.com/news-event/coronavirus?action=click\&pgtype=Article\&state=default\&region=MAIN_CONTENT_3\&context=storylines_faq}{}

\hypertarget{the-coronavirus-outbreak-}{%
\subsubsection{The Coronavirus Outbreak
›}\label{the-coronavirus-outbreak-}}

\hypertarget{frequently-asked-questions}{%
\paragraph{Frequently Asked
Questions}\label{frequently-asked-questions}}

Updated July 27, 2020

\begin{itemize}
\item ~
  \hypertarget{should-i-refinance-my-mortgage}{%
  \paragraph{Should I refinance my
  mortgage?}\label{should-i-refinance-my-mortgage}}

  \begin{itemize}
  \tightlist
  \item
    \href{https://www.nytimes.com/article/coronavirus-money-unemployment.html?action=click\&pgtype=Article\&state=default\&region=MAIN_CONTENT_3\&context=storylines_faq}{It
    could be a good idea,} because mortgage rates have
    \href{https://www.nytimes.com/2020/07/16/business/mortgage-rates-below-3-percent.html?action=click\&pgtype=Article\&state=default\&region=MAIN_CONTENT_3\&context=storylines_faq}{never
    been lower.} Refinancing requests have pushed mortgage applications
    to some of the highest levels since 2008, so be prepared to get in
    line. But defaults are also up, so if you're thinking about buying a
    home, be aware that some lenders have tightened their standards.
  \end{itemize}
\item ~
  \hypertarget{what-is-school-going-to-look-like-in-september}{%
  \paragraph{What is school going to look like in
  September?}\label{what-is-school-going-to-look-like-in-september}}

  \begin{itemize}
  \tightlist
  \item
    It is unlikely that many schools will return to a normal schedule
    this fall, requiring the grind of
    \href{https://www.nytimes.com/2020/06/05/us/coronavirus-education-lost-learning.html?action=click\&pgtype=Article\&state=default\&region=MAIN_CONTENT_3\&context=storylines_faq}{online
    learning},
    \href{https://www.nytimes.com/2020/05/29/us/coronavirus-child-care-centers.html?action=click\&pgtype=Article\&state=default\&region=MAIN_CONTENT_3\&context=storylines_faq}{makeshift
    child care} and
    \href{https://www.nytimes.com/2020/06/03/business/economy/coronavirus-working-women.html?action=click\&pgtype=Article\&state=default\&region=MAIN_CONTENT_3\&context=storylines_faq}{stunted
    workdays} to continue. California's two largest public school
    districts --- Los Angeles and San Diego --- said on July 13, that
    \href{https://www.nytimes.com/2020/07/13/us/lausd-san-diego-school-reopening.html?action=click\&pgtype=Article\&state=default\&region=MAIN_CONTENT_3\&context=storylines_faq}{instruction
    will be remote-only in the fall}, citing concerns that surging
    coronavirus infections in their areas pose too dire a risk for
    students and teachers. Together, the two districts enroll some
    825,000 students. They are the largest in the country so far to
    abandon plans for even a partial physical return to classrooms when
    they reopen in August. For other districts, the solution won't be an
    all-or-nothing approach.
    \href{https://bioethics.jhu.edu/research-and-outreach/projects/eschool-initiative/school-policy-tracker/}{Many
    systems}, including the nation's largest, New York City, are
    devising
    \href{https://www.nytimes.com/2020/06/26/us/coronavirus-schools-reopen-fall.html?action=click\&pgtype=Article\&state=default\&region=MAIN_CONTENT_3\&context=storylines_faq}{hybrid
    plans} that involve spending some days in classrooms and other days
    online. There's no national policy on this yet, so check with your
    municipal school system regularly to see what is happening in your
    community.
  \end{itemize}
\item ~
  \hypertarget{is-the-coronavirus-airborne}{%
  \paragraph{Is the coronavirus
  airborne?}\label{is-the-coronavirus-airborne}}

  \begin{itemize}
  \tightlist
  \item
    The coronavirus
    \href{https://www.nytimes.com/2020/07/04/health/239-experts-with-one-big-claim-the-coronavirus-is-airborne.html?action=click\&pgtype=Article\&state=default\&region=MAIN_CONTENT_3\&context=storylines_faq}{can
    stay aloft for hours in tiny droplets in stagnant air}, infecting
    people as they inhale, mounting scientific evidence suggests. This
    risk is highest in crowded indoor spaces with poor ventilation, and
    may help explain super-spreading events reported in meatpacking
    plants, churches and restaurants.
    \href{https://www.nytimes.com/2020/07/06/health/coronavirus-airborne-aerosols.html?action=click\&pgtype=Article\&state=default\&region=MAIN_CONTENT_3\&context=storylines_faq}{It's
    unclear how often the virus is spread} via these tiny droplets, or
    aerosols, compared with larger droplets that are expelled when a
    sick person coughs or sneezes, or transmitted through contact with
    contaminated surfaces, said Linsey Marr, an aerosol expert at
    Virginia Tech. Aerosols are released even when a person without
    symptoms exhales, talks or sings, according to Dr. Marr and more
    than 200 other experts, who
    \href{https://academic.oup.com/cid/article/doi/10.1093/cid/ciaa939/5867798}{have
    outlined the evidence in an open letter to the World Health
    Organization}.
  \end{itemize}
\item ~
  \hypertarget{what-are-the-symptoms-of-coronavirus}{%
  \paragraph{What are the symptoms of
  coronavirus?}\label{what-are-the-symptoms-of-coronavirus}}

  \begin{itemize}
  \tightlist
  \item
    Common symptoms
    \href{https://www.nytimes.com/article/symptoms-coronavirus.html?action=click\&pgtype=Article\&state=default\&region=MAIN_CONTENT_3\&context=storylines_faq}{include
    fever, a dry cough, fatigue and difficulty breathing or shortness of
    breath.} Some of these symptoms overlap with those of the flu,
    making detection difficult, but runny noses and stuffy sinuses are
    less common.
    \href{https://www.nytimes.com/2020/04/27/health/coronavirus-symptoms-cdc.html?action=click\&pgtype=Article\&state=default\&region=MAIN_CONTENT_3\&context=storylines_faq}{The
    C.D.C. has also} added chills, muscle pain, sore throat, headache
    and a new loss of the sense of taste or smell as symptoms to look
    out for. Most people fall ill five to seven days after exposure, but
    symptoms may appear in as few as two days or as many as 14 days.
  \end{itemize}
\item ~
  \hypertarget{does-asymptomatic-transmission-of-covid-19-happen}{%
  \paragraph{Does asymptomatic transmission of Covid-19
  happen?}\label{does-asymptomatic-transmission-of-covid-19-happen}}

  \begin{itemize}
  \tightlist
  \item
    So far, the evidence seems to show it does. A widely cited
    \href{https://www.nature.com/articles/s41591-020-0869-5}{paper}
    published in April suggests that people are most infectious about
    two days before the onset of coronavirus symptoms and estimated that
    44 percent of new infections were a result of transmission from
    people who were not yet showing symptoms. Recently, a top expert at
    the World Health Organization stated that transmission of the
    coronavirus by people who did not have symptoms was ``very rare,''
    \href{https://www.nytimes.com/2020/06/09/world/coronavirus-updates.html?action=click\&pgtype=Article\&state=default\&region=MAIN_CONTENT_3\&context=storylines_faq\#link-1f302e21}{but
    she later walked back that statement.}
  \end{itemize}
\end{itemize}

\hypertarget{i-just-kind-of-cry-and-i-dont-know-why-im-crying}{%
\subsection{`I just kind of cry, and I don't know why I'm
crying.'}\label{i-just-kind-of-cry-and-i-dont-know-why-im-crying}}

The mental health and academic performance of students are intertwined,
so universities have tried to provide academic as well as psychological
help remotely. ``We continued to provide free tutoring, academic
advising, virtual study tables and library support and resources,'' Dr.
Coplin, the Denison provost, said.

But Grace Horn, Mr. McCartney's classmate at Denison, found that
counseling online had its limits. ``Now there's just a disconnect that
you can't breach virtually,'' she said.

Ms. Horn, home in Atlanta, worries about the future. She left school in
the middle of applying for jobs after graduation, and her divorced
mother is unemployed.

At home, she has the ideal quarantine setup: her own bedroom, her own
bathroom, a little balcony outside of her room. She makes Chex Mix as if
she were still in her dorm. When it is time to socialize, she joins her
mother and, from a distance, the backyard neighbors. She said her mood
was volatile.

``It really oscillates from great hope, and I'm graduating and I have
great friends, and we're talking on Zoom all the time to these great
deep sadnesses where I just kind of cry, and I don't know why I'm
crying,'' she said.

Image

The grounds of the New England Music Camp at the Snow Pond Center for
the Arts, along Messalonskee Lake, in Maine.Credit...Tristan Spinski for
The New York Times

\hypertarget{everything-has-been-ended}{%
\subsection{`Everything has been
ended.'}\label{everything-has-been-ended}}

In March, Mr. McCartney drove to the summer camp his family has operated
for generations in Sidney, Maine, from Denison's campus in Granville,
Ohio.

His father picked up his sister in Chicago, stopped at their apartment
in New York City to pick up the cat and joined him. His mother went
south to Florida, to tend to her parents.

There is a charter school on the camp grounds, also closed for the
pandemic. He set himself up in a classroom. The high-speed Wi-Fi was
excellent.

In some ways, it was a cozy setup. His father and sister live in
separate staff apartments. His aunt and uncle live in the main house.
His cousin and his cousin's husband have moved in as well. Another
cousin, a carpenter, lives in town and comes to visit, but, at least at
first, was wary of getting out of his car for fear of infection.

But Mr. McCartney feels trapped.

``I spend all day in this room,'' Mr. McCartney said, just before final
exams this month. ``I eat breakfast alone. I come up here, do my
homework, attend class, do more work. I talk with my friends, usually
about work. We try not to talk about the virus.''

He was heartened when the provost got back to him within hours of his
email, telling him that she, too, was having adjustment problems.

About a week ago, he celebrated his commencement online, with family
members dropping in remotely. It was a happy event, but the anxiety rose
up in him as soon as it was over. Medical school beckons in the future.
But he is not exactly sure what to do right now.

``Everything has been ended --- closed or canceled or disallowed,'' he
said.

Advertisement

\protect\hyperlink{after-bottom}{Continue reading the main story}

\hypertarget{site-index}{%
\subsection{Site Index}\label{site-index}}

\hypertarget{site-information-navigation}{%
\subsection{Site Information
Navigation}\label{site-information-navigation}}

\begin{itemize}
\tightlist
\item
  \href{https://help.nytimes.com/hc/en-us/articles/115014792127-Copyright-notice}{©~2020~The
  New York Times Company}
\end{itemize}

\begin{itemize}
\tightlist
\item
  \href{https://www.nytco.com/}{NYTCo}
\item
  \href{https://help.nytimes.com/hc/en-us/articles/115015385887-Contact-Us}{Contact
  Us}
\item
  \href{https://www.nytco.com/careers/}{Work with us}
\item
  \href{https://nytmediakit.com/}{Advertise}
\item
  \href{http://www.tbrandstudio.com/}{T Brand Studio}
\item
  \href{https://www.nytimes.com/privacy/cookie-policy\#how-do-i-manage-trackers}{Your
  Ad Choices}
\item
  \href{https://www.nytimes.com/privacy}{Privacy}
\item
  \href{https://help.nytimes.com/hc/en-us/articles/115014893428-Terms-of-service}{Terms
  of Service}
\item
  \href{https://help.nytimes.com/hc/en-us/articles/115014893968-Terms-of-sale}{Terms
  of Sale}
\item
  \href{https://spiderbites.nytimes.com}{Site Map}
\item
  \href{https://help.nytimes.com/hc/en-us}{Help}
\item
  \href{https://www.nytimes.com/subscription?campaignId=37WXW}{Subscriptions}
\end{itemize}
