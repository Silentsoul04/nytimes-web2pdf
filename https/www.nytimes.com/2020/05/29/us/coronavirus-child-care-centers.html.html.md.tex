Sections

SEARCH

\protect\hyperlink{site-content}{Skip to
content}\protect\hyperlink{site-index}{Skip to site index}

\href{https://www.nytimes.com/section/us}{U.S.}

\href{https://myaccount.nytimes.com/auth/login?response_type=cookie\&client_id=vi}{}

\href{https://www.nytimes.com/section/todayspaper}{Today's Paper}

\href{/section/us}{U.S.}\textbar{}As Day Care Centers Reopen, Will
Parents Send Their Children?

\url{https://nyti.ms/2TNMOBd}

\begin{itemize}
\item
\item
\item
\item
\item
\item
\end{itemize}

\href{https://www.nytimes.com/news-event/coronavirus?action=click\&pgtype=Article\&state=default\&region=TOP_BANNER\&context=storylines_menu}{The
Coronavirus Outbreak}

\begin{itemize}
\tightlist
\item
  live\href{https://www.nytimes.com/2020/08/04/world/coronavirus-cases.html?action=click\&pgtype=Article\&state=default\&region=TOP_BANNER\&context=storylines_menu}{Latest
  Updates}
\item
  \href{https://www.nytimes.com/interactive/2020/us/coronavirus-us-cases.html?action=click\&pgtype=Article\&state=default\&region=TOP_BANNER\&context=storylines_menu}{Maps
  and Cases}
\item
  \href{https://www.nytimes.com/interactive/2020/science/coronavirus-vaccine-tracker.html?action=click\&pgtype=Article\&state=default\&region=TOP_BANNER\&context=storylines_menu}{Vaccine
  Tracker}
\item
  \href{https://www.nytimes.com/2020/08/02/us/covid-college-reopening.html?action=click\&pgtype=Article\&state=default\&region=TOP_BANNER\&context=storylines_menu}{College
  Reopening}
\item
  \href{https://www.nytimes.com/live/2020/08/04/business/stock-market-today-coronavirus?action=click\&pgtype=Article\&state=default\&region=TOP_BANNER\&context=storylines_menu}{Economy}
\end{itemize}

Advertisement

\protect\hyperlink{after-top}{Continue reading the main story}

Supported by

\protect\hyperlink{after-sponsor}{Continue reading the main story}

\hypertarget{as-day-care-centers-reopen-will-parents-send-their-children}{%
\section{As Day Care Centers Reopen, Will Parents Send Their
Children?}\label{as-day-care-centers-reopen-will-parents-send-their-children}}

Reopening requires cash, safety renovations and parental confidence ---
all of which are in short supply.

\includegraphics{https://static01.nyt.com/images/2020/05/22/us/00virus-childcare01/merlin_172524804_8f48d27d-8b78-4c1b-a191-38994d16398c-articleLarge.jpg?quality=75\&auto=webp\&disable=upscale}

\href{https://www.nytimes.com/by/dana-goldstein}{\includegraphics{https://static01.nyt.com/images/2018/06/12/multimedia/author-dana-goldstein/author-dana-goldstein-thumbLarge.png}}\href{https://www.nytimes.com/by/julie-bosman}{\includegraphics{https://static01.nyt.com/images/2018/11/09/multimedia/author-julie-bosman/author-julie-bosman-thumbLarge.png}}

By \href{https://www.nytimes.com/by/dana-goldstein}{Dana Goldstein} and
\href{https://www.nytimes.com/by/julie-bosman}{Julie Bosman}

\begin{itemize}
\item
  Published May 29, 2020Updated May 30, 2020
\item
  \begin{itemize}
  \item
  \item
  \item
  \item
  \item
  \item
  \end{itemize}
\end{itemize}

Venice Ray was eager to return to work when Texas announced last week
that child care centers, like the one she was laid off from in March,
could immediately reopen. But re-enrolling her 4-year-old son? That gave
her pause.

As many restaurants, hair salons and shopping malls across the country
\href{https://www.nytimes.com/interactive/2020/us/states-reopen-map-coronavirus.html}{welcome
back} customers,
\href{http://www.hunt-institute.org/covid-19-resources/}{some states are
allowing} day care centers and preschools to reopen, acknowledging that
child care plays a foundational role in the American economy.

But for millions of working parents like Ms. Ray, the choice to send
their children back to a place known for spreading germs, even in more
normal times, is not an easy one. And in an industry operating on
razor-thin margins, the survival of many child care centers is in doubt.

\href{https://www.nytimes.com/news-event/coronavirus}{The coronavirus}
cost the industry
\href{https://www.bls.gov/news.release/empsit.t17.htm}{more than 355,000
jobs} in March and April, about a third of the pre-pandemic total. And a
survey by an industry group showed that many providers were so short of
cash that they
\href{https://www.naeyc.org/sites/default/files/globally-shared/downloads/PDFs/our-work/public-policy-advocacy/effects_of_coronavirus_on_child_care.final.pdf}{could
go out of business permanently}, unable to pay rent, mortgage payments
or other fixed costs.

Democrats in Congress are
\href{https://www.nytimes.com/2020/05/27/upshot/virus-childcare-bailout-democrats.html}{introducing
bills this week and next} that would spend \$50 billion to keep centers
afloat and provide tuition relief to families. The bills would also help
put in place new safety measures, which could be vital to ensuring that
parents feel safe re-enrolling their children.

In a majority of American families, both parents hold jobs, making child
care essential for a functioning economy. But the United States is rare
among industrialized nations in
\href{https://www.nytimes.com/2019/08/15/upshot/why-americans-resist-child-care.html}{not
providing a universal option}, leaving families with the burden of
figuring it out themselves. Many struggle even in the best of times.

\hypertarget{latest-updates-global-coronavirus-outbreak}{%
\section{\texorpdfstring{\href{https://www.nytimes.com/2020/08/04/world/coronavirus-cases.html?action=click\&pgtype=Article\&state=default\&region=MAIN_CONTENT_1\&context=storylines_live_updates}{Latest
Updates: Global Coronavirus
Outbreak}}{Latest Updates: Global Coronavirus Outbreak}}\label{latest-updates-global-coronavirus-outbreak}}

Updated 2020-08-04T20:57:54.346Z

\begin{itemize}
\tightlist
\item
  \href{https://www.nytimes.com/2020/08/04/world/coronavirus-cases.html?action=click\&pgtype=Article\&state=default\&region=MAIN_CONTENT_1\&context=storylines_live_updates\#link-1228a480}{Novavax
  sees encouraging results from two studies of its experimental
  vaccine.}
\item
  \href{https://www.nytimes.com/2020/08/04/world/coronavirus-cases.html?action=click\&pgtype=Article\&state=default\&region=MAIN_CONTENT_1\&context=storylines_live_updates\#link-4825b93}{Public
  and private schools in Maryland and elsewhere are divided over
  in-person instruction.}
\item
  \href{https://www.nytimes.com/2020/08/04/world/coronavirus-cases.html?action=click\&pgtype=Article\&state=default\&region=MAIN_CONTENT_1\&context=storylines_live_updates\#link-50f7386d}{The
  United Nations calls on policymakers to `plan thoroughly for school
  reopenings.'}
\end{itemize}

\href{https://www.nytimes.com/2020/08/04/world/coronavirus-cases.html?action=click\&pgtype=Article\&state=default\&region=MAIN_CONTENT_1\&context=storylines_live_updates}{See
more updates}

More live coverage:
\href{https://www.nytimes.com/live/2020/08/04/business/stock-market-today-coronavirus?action=click\&pgtype=Article\&state=default\&region=MAIN_CONTENT_1\&context=storylines_live_updates}{Markets}

Experts now worry that if licensed centers disappear during the
pandemic, more families will resort to ad hoc arrangements, such as
relying on relatives, friends or neighbors who lack experience, let
alone formal training in safety or education.

But for families balancing professional and economic pressures with
health concerns, the idea of re-enrolling in center-based care can be
profoundly anxiety-provoking, even with new sanitation and social
distancing practices required by federal and state guidelines.

That is the situation the Ray family found itself in. Ms. Ray is
confident that the new procedures will make it safe for her to return to
work when her employer, a day care and preschool associated with a
Longview, Texas, church, reopens on June 2.

But as a mother, she felt unsure about whether her son should go back,
too. In particular, she was concerned about her in-laws, who have been
babysitting her older children. What if the youngest, their 4-year-old
son, brought the virus home and passed it on to them?

``It's a really tough one,'' she said. After days of debate, she and her
husband decided that Ms. Ray would go back to work --- the family needs
her \$475-per-week salary, and she loves her job --- but their son would
go to his grandparents for care.

A similar conversation took place in Alison Larkin's home in Chicago.
Before the coronavirus outbreak, Ms. Larkin, a 35-year-old social work
consultant, had a dream situation for her toddler, Clementine: a coveted
spot in a respected day care just a few minutes away.

But now she and her husband, who have been trading child care shifts
while working from home, are preparing for their current arrangement to
last until there is a vaccine, even though Illinois is allowing
\href{https://www.chicagotribune.com/coronavirus/ct-coronavirus-illinois-child-care-facilities-reopen-daycare-20200522-tbsreoatqfeblldhkx7lmfe3km-story.html}{day
care centers to reopen this week}.

``My sense of things is that the virus is not going anywhere for quite
some time,'' Ms. Larkin said. ``This could be an indefinite state of
being.''

Mandy Zaransky-Hurst, a mother and corporate executive in Chicago, has
been sorely missing her 4-month-old's day care since it was shut down in
March. She has a full-time job as the chief operating officer of a
training and development company. Her husband, who is also working from
home, is a lawyer.

\includegraphics{https://static01.nyt.com/images/2020/05/22/us/00virus-childcare02/merlin_172524879_5a63b2d9-e5c3-4268-81f0-66d8d842dbb1-articleLarge.jpg?quality=75\&auto=webp\&disable=upscale}

``It's not sustainable,'' Ms. Zaransky-Hurst said of their current
arrangement, which requires her to frequently rise at 4 a.m. to begin a
10-hour workday, while also caring for their 6-year-old.

But at the same time, she worries that day care is not safe, given the
constant flow of teachers and children in and out.

\href{https://www.nytimes.com/news-event/coronavirus?action=click\&pgtype=Article\&state=default\&region=MAIN_CONTENT_3\&context=storylines_faq}{}

\hypertarget{the-coronavirus-outbreak-}{%
\subsubsection{The Coronavirus Outbreak
›}\label{the-coronavirus-outbreak-}}

\hypertarget{frequently-asked-questions}{%
\paragraph{Frequently Asked
Questions}\label{frequently-asked-questions}}

Updated August 4, 2020

\begin{itemize}
\item ~
  \hypertarget{i-have-antibodies-am-i-now-immune}{%
  \paragraph{I have antibodies. Am I now
  immune?}\label{i-have-antibodies-am-i-now-immune}}

  \begin{itemize}
  \tightlist
  \item
    As of right
    now,\href{https://www.nytimes.com/2020/07/22/health/covid-antibodies-herd-immunity.html?action=click\&pgtype=Article\&state=default\&region=MAIN_CONTENT_3\&context=storylines_faq}{that
    seems likely, for at least several months.} There have been
    frightening accounts of people suffering what seems to be a second
    bout of Covid-19. But experts say these patients may have a
    drawn-out course of infection, with the virus taking a slow toll
    weeks to months after initial exposure. People infected with the
    coronavirus typically
    \href{https://www.nature.com/articles/s41586-020-2456-9}{produce}
    immune molecules called antibodies, which are
    \href{https://www.nytimes.com/2020/05/07/health/coronavirus-antibody-prevalence.html?action=click\&pgtype=Article\&state=default\&region=MAIN_CONTENT_3\&context=storylines_faq}{protective
    proteins made in response to an
    infection}\href{https://www.nytimes.com/2020/05/07/health/coronavirus-antibody-prevalence.html?action=click\&pgtype=Article\&state=default\&region=MAIN_CONTENT_3\&context=storylines_faq}{.
    These antibodies may} last in the body
    \href{https://www.nature.com/articles/s41591-020-0965-6}{only two to
    three months}, which may seem worrisome, but that's perfectly normal
    after an acute infection subsides, said Dr. Michael Mina, an
    immunologist at Harvard University. It may be possible to get the
    coronavirus again, but it's highly unlikely that it would be
    possible in a short window of time from initial infection or make
    people sicker the second time.
  \end{itemize}
\item ~
  \hypertarget{im-a-small-business-owner-can-i-get-relief}{%
  \paragraph{I'm a small-business owner. Can I get
  relief?}\label{im-a-small-business-owner-can-i-get-relief}}

  \begin{itemize}
  \tightlist
  \item
    The
    \href{https://www.nytimes.com/article/small-business-loans-stimulus-grants-freelancers-coronavirus.html?action=click\&pgtype=Article\&state=default\&region=MAIN_CONTENT_3\&context=storylines_faq}{stimulus
    bills enacted in March} offer help for the millions of American
    small businesses. Those eligible for aid are businesses and
    nonprofit organizations with fewer than 500 workers, including sole
    proprietorships, independent contractors and freelancers. Some
    larger companies in some industries are also eligible. The help
    being offered, which is being managed by the Small Business
    Administration, includes the Paycheck Protection Program and the
    Economic Injury Disaster Loan program. But lots of folks have
    \href{https://www.nytimes.com/interactive/2020/05/07/business/small-business-loans-coronavirus.html?action=click\&pgtype=Article\&state=default\&region=MAIN_CONTENT_3\&context=storylines_faq}{not
    yet seen payouts.} Even those who have received help are confused:
    The rules are draconian, and some are stuck sitting on
    \href{https://www.nytimes.com/2020/05/02/business/economy/loans-coronavirus-small-business.html?action=click\&pgtype=Article\&state=default\&region=MAIN_CONTENT_3\&context=storylines_faq}{money
    they don't know how to use.} Many small-business owners are getting
    less than they expected or
    \href{https://www.nytimes.com/2020/06/10/business/Small-business-loans-ppp.html?action=click\&pgtype=Article\&state=default\&region=MAIN_CONTENT_3\&context=storylines_faq}{not
    hearing anything at all.}
  \end{itemize}
\item ~
  \hypertarget{what-are-my-rights-if-i-am-worried-about-going-back-to-work}{%
  \paragraph{What are my rights if I am worried about going back to
  work?}\label{what-are-my-rights-if-i-am-worried-about-going-back-to-work}}

  \begin{itemize}
  \tightlist
  \item
    Employers have to provide
    \href{https://www.osha.gov/SLTC/covid-19/standards.html}{a safe
    workplace} with policies that protect everyone equally.
    \href{https://www.nytimes.com/article/coronavirus-money-unemployment.html?action=click\&pgtype=Article\&state=default\&region=MAIN_CONTENT_3\&context=storylines_faq}{And
    if one of your co-workers tests positive for the coronavirus, the
    C.D.C.} has said that
    \href{https://www.cdc.gov/coronavirus/2019-ncov/community/guidance-business-response.html}{employers
    should tell their employees} -\/- without giving you the sick
    employee's name -\/- that they may have been exposed to the virus.
  \end{itemize}
\item ~
  \hypertarget{should-i-refinance-my-mortgage}{%
  \paragraph{Should I refinance my
  mortgage?}\label{should-i-refinance-my-mortgage}}

  \begin{itemize}
  \tightlist
  \item
    \href{https://www.nytimes.com/article/coronavirus-money-unemployment.html?action=click\&pgtype=Article\&state=default\&region=MAIN_CONTENT_3\&context=storylines_faq}{It
    could be a good idea,} because mortgage rates have
    \href{https://www.nytimes.com/2020/07/16/business/mortgage-rates-below-3-percent.html?action=click\&pgtype=Article\&state=default\&region=MAIN_CONTENT_3\&context=storylines_faq}{never
    been lower.} Refinancing requests have pushed mortgage applications
    to some of the highest levels since 2008, so be prepared to get in
    line. But defaults are also up, so if you're thinking about buying a
    home, be aware that some lenders have tightened their standards.
  \end{itemize}
\item ~
  \hypertarget{what-is-school-going-to-look-like-in-september}{%
  \paragraph{What is school going to look like in
  September?}\label{what-is-school-going-to-look-like-in-september}}

  \begin{itemize}
  \tightlist
  \item
    It is unlikely that many schools will return to a normal schedule
    this fall, requiring the grind of
    \href{https://www.nytimes.com/2020/06/05/us/coronavirus-education-lost-learning.html?action=click\&pgtype=Article\&state=default\&region=MAIN_CONTENT_3\&context=storylines_faq}{online
    learning},
    \href{https://www.nytimes.com/2020/05/29/us/coronavirus-child-care-centers.html?action=click\&pgtype=Article\&state=default\&region=MAIN_CONTENT_3\&context=storylines_faq}{makeshift
    child care} and
    \href{https://www.nytimes.com/2020/06/03/business/economy/coronavirus-working-women.html?action=click\&pgtype=Article\&state=default\&region=MAIN_CONTENT_3\&context=storylines_faq}{stunted
    workdays} to continue. California's two largest public school
    districts --- Los Angeles and San Diego --- said on July 13, that
    \href{https://www.nytimes.com/2020/07/13/us/lausd-san-diego-school-reopening.html?action=click\&pgtype=Article\&state=default\&region=MAIN_CONTENT_3\&context=storylines_faq}{instruction
    will be remote-only in the fall}, citing concerns that surging
    coronavirus infections in their areas pose too dire a risk for
    students and teachers. Together, the two districts enroll some
    825,000 students. They are the largest in the country so far to
    abandon plans for even a partial physical return to classrooms when
    they reopen in August. For other districts, the solution won't be an
    all-or-nothing approach.
    \href{https://bioethics.jhu.edu/research-and-outreach/projects/eschool-initiative/school-policy-tracker/}{Many
    systems}, including the nation's largest, New York City, are
    devising
    \href{https://www.nytimes.com/2020/06/26/us/coronavirus-schools-reopen-fall.html?action=click\&pgtype=Article\&state=default\&region=MAIN_CONTENT_3\&context=storylines_faq}{hybrid
    plans} that involve spending some days in classrooms and other days
    online. There's no national policy on this yet, so check with your
    municipal school system regularly to see what is happening in your
    community.
  \end{itemize}
\end{itemize}

``How can companies really expect their employees to return to work in a
normal fashion?'' she asked. ``What true flexibility and understanding
will companies give to employees who can't send their kids back to day
care?''

So far, there have not been major coronavirus outbreaks reported in
child care centers in the United States, although one in Canada was the
site of an alarming
\href{https://montreal.ctvnews.ca/16-infected-north-of-montreal-in-quebec-s-first-covid-19-outbreak-in-a-daycare-1.4927853}{cluster
of cases}.

The industry typically serves more than
\href{https://cdn2.hubspot.net/hubfs/3957809/State\%20Fact\%20Sheets\%202019/2019StateFactSheets-Overview.pdf}{12
million children in the United States} under 6, the
\href{https://www.childtrends.org/most-child-care-providers-in-the-united-states-are-based-in-homes-not-centers}{majority
of whom} attend day cares in private homes. Providers are licensed by
state and local governments, and must follow regulations for education,
health and safety. Generally,
\href{https://www.childcare.gov/index.php/consumer-education/ratios-and-group-sizes}{experts
recommend} that one trained caretaker be responsible for no more than
four infants, six toddlers or 10 preschoolers.

Under normal circumstances, the physical space required for a center to
maintain social distancing depends heavily on the developmental stage of
the children being cared for. Infants are more interested in adults than
other babies, and are easier to space out than toddlers and
preschoolers, who are social and frenetic. The space required for
effective social distancing will vary greatly depending on the ages and
temperament of children in a center.

\href{https://www.cdc.gov/coronavirus/2019-ncov/downloads/php/CDC-Activities-Initiatives-for-COVID-19-Response.pdf}{Coronavirus
guidelines} from the Centers for Disease Control and Prevention call for
child care centers to disinfect surfaces and shared objects several
times per day; avoid toys, like stuffed animals, that cannot be easily
washed; stagger children's arrival time to limit contact between
parents; and to seat and nap children at least six feet apart from one
another. States are issuing their own guidelines, which can differ
significantly.

Tennille Smalls, who runs a
\href{https://www.instagram.com/gentlehandsacademydaycare/}{child care
center} out of her home in New Haven, Conn., is among the providers
trying to make the necessary changes. Her business is surviving during
the pandemic, but only precariously. Enrollment has fallen to four
children from nine; their parents are essential health care workers.

Ms. Smalls received about \$11,000 in federal relief loans, not nearly
enough to cover her lost revenue and the costs of moving the center to a
larger space over the summer, which is necessary for children and
employees to maintain social distancing guidelines.

She has already signed a lease and begun planning renovations, such as
installing toddler sinks and mounted dispensers for sanitizer, masks and
gloves. To make ends meet, she and her business partner have reduced
their own salaries.

``I'm nervous, but I'm not fearful,'' Ms. Smalls said. She hopes her
spacious new location will reassure nervous parents.

``Families will be able to tour it and say, `This is somewhere that I'm
going to feel comfortable putting my most prized possession.' ''

Ben Casselman contributed reporting.

Advertisement

\protect\hyperlink{after-bottom}{Continue reading the main story}

\hypertarget{site-index}{%
\subsection{Site Index}\label{site-index}}

\hypertarget{site-information-navigation}{%
\subsection{Site Information
Navigation}\label{site-information-navigation}}

\begin{itemize}
\tightlist
\item
  \href{https://help.nytimes.com/hc/en-us/articles/115014792127-Copyright-notice}{©~2020~The
  New York Times Company}
\end{itemize}

\begin{itemize}
\tightlist
\item
  \href{https://www.nytco.com/}{NYTCo}
\item
  \href{https://help.nytimes.com/hc/en-us/articles/115015385887-Contact-Us}{Contact
  Us}
\item
  \href{https://www.nytco.com/careers/}{Work with us}
\item
  \href{https://nytmediakit.com/}{Advertise}
\item
  \href{http://www.tbrandstudio.com/}{T Brand Studio}
\item
  \href{https://www.nytimes.com/privacy/cookie-policy\#how-do-i-manage-trackers}{Your
  Ad Choices}
\item
  \href{https://www.nytimes.com/privacy}{Privacy}
\item
  \href{https://help.nytimes.com/hc/en-us/articles/115014893428-Terms-of-service}{Terms
  of Service}
\item
  \href{https://help.nytimes.com/hc/en-us/articles/115014893968-Terms-of-sale}{Terms
  of Sale}
\item
  \href{https://spiderbites.nytimes.com}{Site Map}
\item
  \href{https://help.nytimes.com/hc/en-us}{Help}
\item
  \href{https://www.nytimes.com/subscription?campaignId=37WXW}{Subscriptions}
\end{itemize}
