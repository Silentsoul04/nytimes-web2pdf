Sections

SEARCH

\protect\hyperlink{site-content}{Skip to
content}\protect\hyperlink{site-index}{Skip to site index}

\href{https://www.nytimes.com/section/world/asia}{Asia Pacific}

\href{https://myaccount.nytimes.com/auth/login?response_type=cookie\&client_id=vi}{}

\href{https://www.nytimes.com/section/todayspaper}{Today's Paper}

\href{/section/world/asia}{Asia Pacific}\textbar{}China and India Brawl
at 14,000 Feet Along the Border

\url{https://nyti.ms/2ySscR2}

\begin{itemize}
\item
\item
\item
\item
\item
\end{itemize}

Advertisement

\protect\hyperlink{after-top}{Continue reading the main story}

Supported by

\protect\hyperlink{after-sponsor}{Continue reading the main story}

\hypertarget{china-and-india-brawl-at-14000-feet-along-the-border}{%
\section{China and India Brawl at 14,000 Feet Along the
Border}\label{china-and-india-brawl-at-14000-feet-along-the-border}}

As China projects its power across Asia, and along the disputed
India-China border in the Himalayas, India is feeling surrounded. Both
sides insist they don't want a war, but thousands of troops have been
sent.

\includegraphics{https://static01.nyt.com/images/2020/05/31/world/30india-china-border-top/merlin_160161276_6aa98d55-c96e-454d-aebe-41cd311cc0a1-articleLarge.jpg?quality=75\&auto=webp\&disable=upscale}

\href{https://www.nytimes.com/by/jeffrey-gettleman}{\includegraphics{https://static01.nyt.com/images/2018/10/10/multimedia/author-jeffrey-gettleman/author-jeffrey-gettleman-thumbLarge.png}}\href{https://www.nytimes.com/by/steven-lee-myers}{\includegraphics{https://static01.nyt.com/images/2018/10/15/multimedia/author-steven-lee-myers/author-steven-lee-myers-thumbLarge.png}}

By \href{https://www.nytimes.com/by/jeffrey-gettleman}{Jeffrey
Gettleman} and \href{https://www.nytimes.com/by/steven-lee-myers}{Steven
Lee Myers}

\begin{itemize}
\item
  Published May 30, 2020Updated July 11, 2020
\item
  \begin{itemize}
  \item
  \item
  \item
  \item
  \item
  \end{itemize}
\end{itemize}

\href{https://cn.nytimes.com/world/20200603/india-china-border/}{阅读简体中文版}\href{https://cn.nytimes.com/world/20200603/india-china-border/zh-hant}{閱讀繁體中文版}

NEW DELHI --- High in the Himalayas, an enormous fistfight erupted in
early May between the soldiers of
\href{https://www.nytimes.com/2020/06/18/world/asia/india-china-border.html}{China
and India}. Brawls at 14,000 feet along their inhospitable and disputed
frontier are not terribly unusual, but what happened next was.

A few days later, Chinese troops confronted Indian soldiers again, this
time at several other remote border points in the Himalayas, some more
than 1,000 miles apart. Since then both armies have rushed in thousands
of reinforcements. Indian analysts say that China has beefed up its
forces with dump trucks, excavators, troop carriers, artillery and
armored vehicles and that China is now occupying Indian territory.

No shots have been fired, as the de facto border code dictates, but the
soldiers have fought fiercely with rocks, wooden clubs and their hands
in a handful of clashes. In one melee at the glacial lake Pangong Tso,
several Indian troops were hurt badly enough that they had to be
evacuated by helicopter, and Indian analysts said Chinese troops were
injured as well.

Nobody thinks China and India are about to go to war. But the escalating
buildup has turned into their most serious confrontation since 2017 and
may be a sign of more trouble to come as the world's two most populous
countries increasingly bump up against each other in one of the bleakest
and most remote borderlands on earth.

Line of Actual

Control

CHINA

PAKISTAN

Diamer Bhasha dam

CHINA

NEPAL

GILGIT-

BALTISTAN

Controlled by Pakistan

Area of detail

Disputed borders

or cease-fire lines

CHINA

INDIA

Galwan Valley

Controlled by India

Bay of

Bengal

Pangong Tso

Arabian

Sea

JAMMU

AND KASHMIR

LADAKH

CHINA

CHINA

PAKISTAN

Lipulekh

Pass

INDIA

Naku La

NEPAL

BHUTAN

100 miles

BANGLADESH

Line of Actual

Control

Diamer Bhasha dam

CHINA

CHINA

PAKISTAN

GILGIT-

BALTISTAN

Controlled by Pakistan

NEPAL

Disputed borders

or cease-fire lines

CHINA

Area of detail

Galwan Valley

Controlled by India

INDIA

Pangong Tso

JAMMU

AND KASHMIR

LADAKH

CHINA

Bay of

Bengal

CHINA

Arabian

Sea

PAKISTAN

Lipulekh

Pass

INDIA

Naku La

NEPAL

BHUTAN

100 miles

BANGLADESH

Line of Actual

Control

Diamer

Bhasha dam

CHINA

CHINA

GILGIT-

BALTISTAN

Controlled by Pakistan

PAKISTAN

Disputed borders

or cease-fire lines

NEPAL

CHINA

Area of detail

Galwan Valley

Controlled by India

INDIA

LADAKH

JAMMU

AND KASHMIR

Pangong Tso

Bay of

Bengal

CHINA

CHINA

Arabian

Sea

PAKISTAN

Lipulekh

Pass

INDIA

Naku La

NEPAL

100 miles

Line of Actual

Control

CHINA

CHINA

GILGIT-

BALTISTAN

Controlled by Pakistan

1

Disputed borders

or cease-fire lines

Area of detail

CHINA

2

INDIA

Controlled by India

LADAKH

3

JAMMU

AND KASHMIR

Bay of

Bengal

Arabian

Sea

CHINA

CHINA

PAKISTAN

4

INDIA

5

NEPAL

100 miles

1

2

3

4

5

Diamer Bhasha dam

Galwan Valley

Pangong Tso

Lipulekh Pass

Naku La

CHINA

Disputed borders

or cease-fire lines

1

Ctrl. by Pakistan

CHINA

2

3

Ctrl. by India

CHINA

CHINA

PAKISTAN

4

INDIA

NEPAL

5

100 miles

1

2

3

4

5

Diamer Bhasha dam

Galwan Valley

Pangong Tso

Lipulekh Pass

Naku La

CHINA

PAKISTAN

NEPAL

Area of detail

INDIA

Bay of

Bengal

Arabian

Sea

Source: Satellite image via Microsoft Corporation Earthstar Geographics

By Jugal K. Patel

President Trump, unsolicited, stepped in on Wednesday,
\href{https://twitter.com/realDonaldTrump/status/1265604027678670848}{offering
on Twitter to mediate} what he called ``a raging border dispute.''

For India, the Chinese incursions and maneuvers at multiple points along
the more than 2,100-mile border have raised suspicions of a concerted
campaign to exert pressure on the government of Prime Minister
\href{https://www.nytimes.com/2020/06/17/world/asia/india-china-border-clashes.html}{Narendra
Modi}.

With the world consumed by the coronavirus pandemic, China has
\href{https://www.nytimes.com/2020/05/24/world/asia/china-hong-kong-taiwan.html}{acted
forcefully} to defend its territorial claims, including in the
Himalayas. In recent weeks, the Chinese have sunk a Vietnamese fishing
boat in the South China Sea; swarmed a Malaysian offshore oil rig;
menaced Taiwan; and severely tightened their grip on the semiautonomous
region of Hong Kong.

The confrontation with India ``fits a broader pattern of Chinese
assertiveness,'' said Tanvi Madan, director of the India Project at the
Brookings Institution in Washington, noting that it was the fourth
flare-up since China's authoritarian leader, Xi Jinping, rose to power
at the end of 2012.

\includegraphics{https://static01.nyt.com/images/2020/05/30/world/30india-china-border-2/merlin_172878099_25a35567-6ada-4ab9-9233-8e50ddda3241-articleLarge.jpg?quality=75\&auto=webp\&disable=upscale}

India's government has disclosed few details about what has actually
happened, saying in a statement only that it was the ``Chinese side that
has recently undertaken activity hindering India's normal patrolling
patterns.''

Mr. Modi, who is usually outspoken in defense of his country's
interests, appears intent on avoiding an escalation, analysts said.

``The military skirmishes and standoffs with India seem to reflect
Beijing's calculation that India's still increasing Covid-19 infections,
coupled with its economic downturn, place it in no position to wage a
border conflict,'' said Brahma Chellaney, professor of strategic studies
at the New Delhi-based Center for Policy Research.

Image

An Indian Army base in 2017 in Haa, Bhutan, close to a disputed border
with China.Credit...Gilles Sabrié for The New York Times

All this, he added, could also be ``Beijing's way of sending a political
message'' to India not to get too close to the United States and to back
off its criticism of the way China has handled the coronavirus.

Even before the scuffling, India was feeling increasingly hemmed in by
China's expanding economic and geopolitical influence in South Asia.

To the west, the Chinese are working with Pakistan, India's archenemy,
and recently agreed to help construct an enormous
\href{https://asia.nikkei.com/Spotlight/Belt-and-Road/China-ignores-India-over-dam-project-in-Pakistani-Kashmir}{dam
on the border of Pakistan-administered Kashmir}, an area India claims.

To the east, China's new friend, Nepal, just produced a map that
challenges where the Indian border lies;
\href{https://thewire.in/external-affairs/army-chief-general-m-m-naravane-nepal-lipulekh-china}{India
has blamed China for stirring up the trouble}. Nepal was once a close
ally, but
\href{https://www.cfr.org/blog/bilateral-mishap-view-nepal}{after India
encouraged a punishing trade blockade} in 2015, Nepal drifted closer to
China.

To the south, deep in the tropics,
\href{https://www.defencenews.in/article/India-Worries-China-may-be-Constructing-South-China-Sea-Like-Artificial-Island-in-Maldives-830589}{the
Chinese have taken over an island in the Maldives}, a few hundred miles
off India's coast. Indian military experts say China has brought in
millions of pounds of sand, expanding the island for possible use as an
airstrip or submarine base.

Image

Prime Minister Narendra Modi of India and China's leader, Xi Jinping,
during a summit in Mamallapuram, India, in October.Credit...Indian Press
Information Bureau/EPA, via Shutterstock

``Obviously, the Chinese aim is to pressurize India,'' said D.S. Hooda,
a retired general in India's army.

China's foreign ministry has blamed India for the recent tensions but
tried to play down the confrontation. That is in stark contrast to
similar border skirmishes in 2017,
\href{https://www.nytimes.com/2017/07/26/world/asia/dolam-plateau-china-india-bhutan.html}{when
the two countries squared off for 73 days} over another contested
Himalayan border region near Bhutan, leading to a dangerous spike in
nationalistic sentiment on both sides.

``The Chinese border troops are committed to upholding China's
territorial and sovereignty security, responding resolutely to India's
trespassing and infringing activities and maintaining peace and
tranquillity,'' a spokesman, Zhao Lijian, said after the first public
reports of clashes emerged in mid-May.

He urged India to ``refrain from taking any unilateral actions that may
complicate the situation.''

Both countries run patrols along the disputed border, known as the Line
of Actual Control, the precise location of which can be blurry. The
packs of soldiers marching up and down the mountains are under strict
orders not to shoot at each other, security analysts said, but that
doesn't stop them from throwing rocks. Or the occasional punch.

Sometimes, big passing patrols collide. A few years ago,
\href{https://theprint.in/report/visuals-show-india-china-clash-at-ladakh-was-serious-troops-injured/7015/}{another
Indo-China brawl broke out} --- and was captured on video --- at the
same mountain lake where some of the clashes happened this month.

China has not officially acknowledged any recent deployment of forces to
the Himalayas. But Global Times, a tabloid controlled by the Communist
Party, cited a source close to the People's Liberation Army in
\href{http://www.globaltimes.cn/content/1188681.shtml}{a May 18 article}
who said China's military bolstered its forces in response to what it
considered illegal construction by India in or near Chinese territory.

China has a superior military, which analysts believe could force India
to back down.
\href{https://twitter.com/ajaishukla/status/1265267130825465856}{Ajai
Shukla, a former Indian Army colonel,} estimated that China had brought
in three brigades of the People's Liberation Army --- amounting to
thousands of troops --- and India had deployed around 3,000
reinforcements.

``If they want to evict the Chinese, the Indian Army would have to start
a shooting war,'' Mr. Shukla said. He doesn't think that will happen and
added that India's options are ``limited by not wanting this to
escalate.''

Just a few months ago, Mr. Modi and Mr. Xi were sipping fresh coconuts
together during
\href{https://www.nytimes.com/2019/10/11/world/asia/narendra-modi-xi-jinping-india-china.html}{a
quick summit meeting in southern India}. A good relationship would help
both countries in their aspirations for world power.

Image

Border Roads Organization workers rest near Pangong Lake in Ladakh,
India, in 2018.Credit...Manish Swarup/Associated Press

Still, they have become increasingly watchful of each other, especially
in the high Himalayas, where few ever go.

India has recently stepped up efforts to improve the roads its military
uses to crisscross the mountain passes in the
\href{https://www.nytimes.com/2020/07/11/world/asia/india-china-border-ladakh.html}{Ladakh
region}, on the border of Tibet. These roads are not easy to build. They
snake across a gravelly landscape of high altitude rivers, glaciers and
passes at 17,000 feet above sea level.

Analysts said that China did not intend to start a war but that it
wanted to frustrate India's road-building efforts. The race to make
these high mountain roads is becoming increasingly fraught. The 2017
standoff between India and China began when Indian troops physically
blocked a Chinese road crew in a disputed region claimed by Bhutan, a
close ally of India's.

China is also sensitive about the Indian border because it abuts two
regions within China that Beijing is especially concerned about: Tibet
and Xinjiang.

The spark of the recent tensions seems to have been one particular new
road that the Indians have been building to reach a military airstrip at
India's northernmost border outpost, which was the site of another
border standoff in 2013.

The two countries have established mechanisms for resolving border
conflicts since 1962, when they went to war in the Himalayas, with India
losing badly.

``There hasn't been a shot fired in years,'' Ms. Madan said, adding that
the last death from a border skirmish happened in 1975.

Still, tensions could easily flare.

``All of this is happening in the area where they fought in the '62
war,'' she said. ``There is a lot of baggage associated with this on
both sides.''

Jeffrey Gettleman reported from New Delhi, and Steven Lee Myers from
Seoul, South Korea. Hari Kumar contributed reporting from New Delhi, and
Salman Masood from Islamabad, Pakistan. Claire Fu contributed research
from Beijing.

Advertisement

\protect\hyperlink{after-bottom}{Continue reading the main story}

\hypertarget{site-index}{%
\subsection{Site Index}\label{site-index}}

\hypertarget{site-information-navigation}{%
\subsection{Site Information
Navigation}\label{site-information-navigation}}

\begin{itemize}
\tightlist
\item
  \href{https://help.nytimes.com/hc/en-us/articles/115014792127-Copyright-notice}{©~2020~The
  New York Times Company}
\end{itemize}

\begin{itemize}
\tightlist
\item
  \href{https://www.nytco.com/}{NYTCo}
\item
  \href{https://help.nytimes.com/hc/en-us/articles/115015385887-Contact-Us}{Contact
  Us}
\item
  \href{https://www.nytco.com/careers/}{Work with us}
\item
  \href{https://nytmediakit.com/}{Advertise}
\item
  \href{http://www.tbrandstudio.com/}{T Brand Studio}
\item
  \href{https://www.nytimes.com/privacy/cookie-policy\#how-do-i-manage-trackers}{Your
  Ad Choices}
\item
  \href{https://www.nytimes.com/privacy}{Privacy}
\item
  \href{https://help.nytimes.com/hc/en-us/articles/115014893428-Terms-of-service}{Terms
  of Service}
\item
  \href{https://help.nytimes.com/hc/en-us/articles/115014893968-Terms-of-sale}{Terms
  of Sale}
\item
  \href{https://spiderbites.nytimes.com}{Site Map}
\item
  \href{https://help.nytimes.com/hc/en-us}{Help}
\item
  \href{https://www.nytimes.com/subscription?campaignId=37WXW}{Subscriptions}
\end{itemize}
