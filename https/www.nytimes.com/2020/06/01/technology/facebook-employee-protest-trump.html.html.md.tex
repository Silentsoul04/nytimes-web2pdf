Sections

SEARCH

\protect\hyperlink{site-content}{Skip to
content}\protect\hyperlink{site-index}{Skip to site index}

\href{https://www.nytimes.com/section/technology}{Technology}

\href{https://myaccount.nytimes.com/auth/login?response_type=cookie\&client_id=vi}{}

\href{https://www.nytimes.com/section/todayspaper}{Today's Paper}

\href{/section/technology}{Technology}\textbar{}Facebook Employees Stage
Virtual Walkout to Protest Trump Posts

\url{https://nyti.ms/3eCkV71}

\begin{itemize}
\item
\item
\item
\item
\item
\item
\end{itemize}

\href{https://www.nytimes.com/news-event/george-floyd-protests-minneapolis-new-york-los-angeles?action=click\&pgtype=Article\&state=default\&region=TOP_BANNER\&context=storylines_menu}{Race
and America}

\begin{itemize}
\tightlist
\item
  \href{https://www.nytimes.com/2020/07/26/us/protests-portland-seattle-trump.html?action=click\&pgtype=Article\&state=default\&region=TOP_BANNER\&context=storylines_menu}{Protesters
  Return to Other Cities}
\item
  \href{https://www.nytimes.com/2020/07/24/us/portland-oregon-protests-white-race.html?action=click\&pgtype=Article\&state=default\&region=TOP_BANNER\&context=storylines_menu}{Portland
  at the Center}
\item
  \href{https://www.nytimes.com/2020/07/23/podcasts/the-daily/portland-protests.html?action=click\&pgtype=Article\&state=default\&region=TOP_BANNER\&context=storylines_menu}{Podcast:
  Showdown in Portland}
\item
  \href{https://www.nytimes.com/interactive/2020/07/16/us/black-lives-matter-protests-louisville-breonna-taylor.html?action=click\&pgtype=Article\&state=default\&region=TOP_BANNER\&context=storylines_menu}{45
  Days in Louisville}
\end{itemize}

Advertisement

\protect\hyperlink{after-top}{Continue reading the main story}

Supported by

\protect\hyperlink{after-sponsor}{Continue reading the main story}

\hypertarget{facebook-employees-stage-virtual-walkout-to-protest-trump-posts}{%
\section{Facebook Employees Stage Virtual Walkout to Protest Trump
Posts}\label{facebook-employees-stage-virtual-walkout-to-protest-trump-posts}}

While Twitter started labeling some of the president's inflammatory
messages, Facebook's chief executive, Mark Zuckerberg, has said his
company should leave them alone.

\includegraphics{https://static01.nyt.com/images/2020/06/01/business/01fbrevolt/merlin_165501543_70f18712-040e-474f-a69d-dd78065610ab-articleLarge.jpg?quality=75\&auto=webp\&disable=upscale}

\href{https://www.nytimes.com/by/sheera-frenkel}{\includegraphics{https://static01.nyt.com/images/2018/06/14/multimedia/author-sheera-frenkel/author-sheera-frenkel-thumbLarge.png}}\href{https://www.nytimes.com/by/mike-isaac}{\includegraphics{https://static01.nyt.com/images/2018/02/16/multimedia/author-mike-isaac/author-mike-isaac-thumbLarge.jpg}}\href{https://www.nytimes.com/by/cecilia-kang}{\includegraphics{https://static01.nyt.com/images/2019/01/29/multimedia/author-cecilia-kang/author-cecilia-kang-thumbLarge.png}}\href{https://www.nytimes.com/by/gabriel-dance}{\includegraphics{https://static01.nyt.com/images/2018/02/16/multimedia/author-gabriel-dance/author-gabriel-dance-thumbLarge.jpg}}

By \href{https://www.nytimes.com/by/sheera-frenkel}{Sheera Frenkel},
\href{https://www.nytimes.com/by/mike-isaac}{Mike Isaac},
\href{https://www.nytimes.com/by/cecilia-kang}{Cecilia Kang} and
\href{https://www.nytimes.com/by/gabriel-dance}{Gabriel J.X. Dance}

\begin{itemize}
\item
  June 1, 2020
\item
  \begin{itemize}
  \item
  \item
  \item
  \item
  \item
  \item
  \end{itemize}
\end{itemize}

OAKLAND, Calif. --- Hundreds of Facebook employees, in rare public
criticism on Monday of their own company, protested executives' decision
not to do anything about inflammatory posts that President Trump had
placed on the giant social media platform over the past week.

Many of the employees, who said they refused to work in order to show
their support for demonstrators across the country, added an automated
message to their digital profiles and email responses saying that they
were out of the office in a show of protest.

The protest group --- conducting a virtual ``walkout'' of sorts since
most Facebook employees are working from home because of the coronavirus
pandemic --- was one of a number of clusters of employees pressing
Facebook executives to take a tougher stand on Mr. Trump's posts.

Inside the company, staff members have circulated petitions and
threatened to resign, and a number of employees wrote publicly about
their unhappiness on Twitter and elsewhere. More than a dozen current
and former employees have described the unrest as the most serious
challenge to the leadership of Mark Zuckerberg, the chief executive,
since the company was founded 15 years ago.

``The hateful rhetoric advocating violence against black demonstrators
by the US President does not warrant defense under the guise of freedom
of expression,'' one Facebook employee wrote in an internal message
board, according to a copy of the text viewed by The New York Times.

The employee added: ``Along with Black employees in the company, and all
persons with a moral conscience, I am calling for Mark to immediately
take down the President's post advocating violence, murder and imminent
threat against Black people.'' The Times agreed to withhold the
employee's name.

Mr. Zuckerberg has argued on a number of occasions that Facebook should
take a hands-off approach to what people post, including lies from
elected officials and others in power. He has repeatedly said the public
should be allowed to decide what to believe.

That stand was tested last week when Twitter added fact-check and
warning labels to two tweets from the president that broke Twitter's
rules around voter suppression and glorification of violence. But as
Twitter acted on Mr. Trump's tweets, Facebook
\href{https://www.nytimes.com/2020/05/29/technology/twitter-facebook-zuckerberg-trump.html}{left
his posts on its platform alone}. Mr. Zuckerberg said Mr. Trump's posts
did not violate the social network's rules.

\includegraphics{https://static01.nyt.com/images/2020/06/01/business/01fbrevolt2/merlin_172958523_806fc2de-cde0-4c6f-a166-2796c32eef34-articleLarge.jpg?quality=75\&auto=webp\&disable=upscale}

Image

Mr. Trump's post on Twitter, which the platform modified.

``Personally, I have a visceral negative reaction to this kind of
divisive and inflammatory rhetoric,'' Mr. Zuckerberg said in a post to
his Facebook page on Friday. ``But I'm responsible for reacting not just
in my personal capacity but as the leader of an institution committed to
free expression.''

Mr. Zuckerberg spoke briefly with Mr. Trump in a telephone call on
Friday, according to two people familiar with the matter. The call,
which was previously reported by
\href{https://www.axios.com/trump-facebook-zuckerberg-phone-call-d8d1016e-4e17-4906-b4f4-dc3e5c00bca7.html}{Axios},
was described as ``productive,'' though it was not clear what was said.
Mr. Zuckerberg explained his position to employees in a live-streamed
question and answer session later that day.

In a video of the session that was reviewed by The Times, hundreds of
employees voiced opposition by posting comments alongside the session,
and some questioned whether any black people had been involved in making
the decision.

``The lack of backbone, and this weak leadership, will be judged by
history. Hate speech should never be compared to free speech,'' one
employee wrote. ``The president (sic) is literally threatening for the
National Guard to shoot citizens. Maybe when we're in the middle of a
race war the policy will change.''

Mr. Zuckerberg said the posts were different from those that threaten
violence because they were about the use of ``state force,'' which is
currently allowed.

While there was some support for the chief executive during the
livestream, the results of an internal poll taken during the session and
posted by a staff member showed that more than 1,000 Facebook employees
voted against Mr. Zuckerberg's choice. Nineteen of the respondents said
they agreed with the decision.

In response to the walkout on Monday, Mr. Zuckerberg has moved his
weekly meeting with employees to Tuesday from Thursday. The meeting will
be a chance for employees to question Mr. Zuckerberg directly.

A Facebook spokeswoman said Monday morning that executives welcomed
feedback from employees. ``We recognize the pain many of our people are
feeling right now, especially our Black community,'' said Liz Bourgeois,
the spokeswoman. ``We encourage employees to speak openly when they
disagree with leadership.''

Mr. Zuckerberg's post last week explaining his decision on Mr. Trump's
posts frustrated many inside the company. More than a dozen Facebook
employees tweeted that they disagreed with Mr. Zuckerberg's decision,
including the head of design of Facebook's portal product, Andrew Crow.

An engineer for the platform, Lauren Tan, posted about the situation on
Friday. ``Facebook's inaction in taking down Trump's post inciting
violence makes me ashamed to work here,'' Ms. Tan wrote
\href{https://twitter.com/sugarpirate_/status/1266470996162146304}{in a
tweet}. ``Silence is complicity.''

Two senior Facebook employees told The New York Times that they had
informed their managers that they would resign if Mr. Zuckerberg did not
reverse his decision. Another person, who was supposed to start work at
the company next month, told Facebook they were no longer willing to
accept a position at the company because of Mr. Zuckerberg's decision.

Over the weekend, several petitions circulated among Facebook employees
calling for the company to make personnel changes and for more diversity
of voices among Mr. Zuckerberg's top lieutenants.

Image

Mr. Zuckerberg, left, with Joel Kaplan, Facebook's vice president of
global public policy, in the Senate in 2018. Mr. Kaplan is considered a
strong conservative voice in the company.Credit...Tom Brenner/The New
York Times

In private online chats, employees have called for the resignation of
Joel Kaplan, Facebook's vice president of global policy. Mr. Kaplan is
seen as being a strong conservative voice within the company. In 2018,
he upset some employees when he sat in the front row of the confirmation
hearings of Supreme Court Justice Brett Kavanaugh, who was a close
friend.

Roger McNamee, a venture capitalist who was an early investor in
Facebook but in recent years has turned into an aggressive critic of the
company, said Facebook's decision to leave Mr. Trump's posts alone was
typical of a longtime pattern of behavior among big social media
companies.

``Internet platforms that are pervasive --- as Facebook and Google are
globally --- must always align with power, including authoritarians. It
is a matter of self-preservation,'' Mr. McNamee said. ``Facebook has
been a key tool for authoritarians in Brazil, the Philippines, Cambodia
and Myanmar. In the U.S., Facebook has consistently ignored or altered
its terms of service to the benefit of Trump. Until last week, Twitter
did the same thing.''

Mr. Zuckerberg and Sheryl Sandberg, the company's chief operating
officer, planned to host a call on Monday evening with civil rights
leaders who have lashed out publicly against Facebook's protection of
Mr. Trump's posts. The call was expected to include Vanita Gupta of the
National Leadership Conference, Rashad Robinson of Color of Change and
Sherrilyn Ifill of the NAACP Legal Defense and Educational Fund.

The civil rights leaders said they would push back on Mr. Zuckerberg's
position on Mr. Trump's posts, which they see as violations of
Facebook's community standards that do not permit voter suppression or
the incitement of violence, even by political figures.

``It's really important for Mark Zuckerberg to contend with the fact
that he is prioritizing free expression while our democracy is literally
burning,'' said Ms. Gupta, who organized the call with the executives.

On Sunday, Mr. Zuckerberg wrote that he would be donating \$10 million
to groups working on racial justice. The move, coupled with his earlier
post expressing solidarity with the demonstrators, did little to quell
the internal protest.

Mr. Robinson, the civil rights group leader, said Mr. Zuckerberg's
financial pledge was ``one of the most insulting things I've ever
seen.'' The donation of money, he said, doesn't change Facebook's policy
of protecting Mr. Trump's comments that contain falsehoods and appear to
violate the company's policies.

Facebook executives have long acknowledged that the company has failed
to attract a diverse work force.

``There's a long history of Facebook, as a company, not seeing or being
responsive to black employees,'' said Mark Luckie, who quit the company
in 2018 and
published\href{https://www.facebook.com/notes/mark-s-luckie/facebook-is-failing-its-black-employees-and-its-black-users/1931075116975013/}{a
memo} titled ``Facebook is failing its black employees and its black
users.''

Like many Silicon Valley companies, Facebook had a severe lack of
diversity, especially among executives, Mr. Luckie said in an interview.
``When you don't have a diverse group of people at the top of the
company, you don't understand the issues involved or why your employees
are upset.''

In 2014, 2 percent of Facebook's employees were black. In 2019, that
number had increased to 3.8 percent, according to the company's
diversity report.

In the post to the internal message board, the dissenting Facebook
employee ended his comment with a quote from the Rev. Dr. Martin Luther
King Jr., the slain civil rights leader.

``Our lives begin to end the day we become silent about things that
matter,'' the quote read.

Sheera Frenkel reported from Oakland, Mike Isaac from San Francisco,
Cecilia Kang from Washington and Gabriel J.X. Dance from Staunton, Va.

Advertisement

\protect\hyperlink{after-bottom}{Continue reading the main story}

\hypertarget{site-index}{%
\subsection{Site Index}\label{site-index}}

\hypertarget{site-information-navigation}{%
\subsection{Site Information
Navigation}\label{site-information-navigation}}

\begin{itemize}
\tightlist
\item
  \href{https://help.nytimes.com/hc/en-us/articles/115014792127-Copyright-notice}{©~2020~The
  New York Times Company}
\end{itemize}

\begin{itemize}
\tightlist
\item
  \href{https://www.nytco.com/}{NYTCo}
\item
  \href{https://help.nytimes.com/hc/en-us/articles/115015385887-Contact-Us}{Contact
  Us}
\item
  \href{https://www.nytco.com/careers/}{Work with us}
\item
  \href{https://nytmediakit.com/}{Advertise}
\item
  \href{http://www.tbrandstudio.com/}{T Brand Studio}
\item
  \href{https://www.nytimes.com/privacy/cookie-policy\#how-do-i-manage-trackers}{Your
  Ad Choices}
\item
  \href{https://www.nytimes.com/privacy}{Privacy}
\item
  \href{https://help.nytimes.com/hc/en-us/articles/115014893428-Terms-of-service}{Terms
  of Service}
\item
  \href{https://help.nytimes.com/hc/en-us/articles/115014893968-Terms-of-sale}{Terms
  of Sale}
\item
  \href{https://spiderbites.nytimes.com}{Site Map}
\item
  \href{https://help.nytimes.com/hc/en-us}{Help}
\item
  \href{https://www.nytimes.com/subscription?campaignId=37WXW}{Subscriptions}
\end{itemize}
