Sections

SEARCH

\protect\hyperlink{site-content}{Skip to
content}\protect\hyperlink{site-index}{Skip to site index}

\href{https://www.nytimes.com/section/technology}{Technology}

\href{https://myaccount.nytimes.com/auth/login?response_type=cookie\&client_id=vi}{}

\href{https://www.nytimes.com/section/todayspaper}{Today's Paper}

\href{/section/technology}{Technology}\textbar{}Misinformation About
George Floyd Protests Surges on Social Media

\url{https://nyti.ms/2Xmm3FV}

\begin{itemize}
\item
\item
\item
\item
\item
\end{itemize}

\href{https://www.nytimes.com/news-event/george-floyd-protests-minneapolis-new-york-los-angeles?action=click\&pgtype=Article\&state=default\&region=TOP_BANNER\&context=storylines_menu}{Race
and America}

\begin{itemize}
\tightlist
\item
  \href{https://www.nytimes.com/2020/07/26/us/protests-portland-seattle-trump.html?action=click\&pgtype=Article\&state=default\&region=TOP_BANNER\&context=storylines_menu}{Protesters
  Return to Other Cities}
\item
  \href{https://www.nytimes.com/2020/07/24/us/portland-oregon-protests-white-race.html?action=click\&pgtype=Article\&state=default\&region=TOP_BANNER\&context=storylines_menu}{Portland
  at the Center}
\item
  \href{https://www.nytimes.com/2020/07/23/podcasts/the-daily/portland-protests.html?action=click\&pgtype=Article\&state=default\&region=TOP_BANNER\&context=storylines_menu}{Podcast:
  Showdown in Portland}
\item
  \href{https://www.nytimes.com/interactive/2020/07/16/us/black-lives-matter-protests-louisville-breonna-taylor.html?action=click\&pgtype=Article\&state=default\&region=TOP_BANNER\&context=storylines_menu}{45
  Days in Louisville}
\end{itemize}

Advertisement

\protect\hyperlink{after-top}{Continue reading the main story}

Supported by

\protect\hyperlink{after-sponsor}{Continue reading the main story}

\hypertarget{misinformation-about-george-floyd-protests-surges-on-social-media}{%
\section{Misinformation About George Floyd Protests Surges on Social
Media}\label{misinformation-about-george-floyd-protests-surges-on-social-media}}

In the universe of false online information, Mr. Floyd remains alive and
George Soros is to blame for the protests.

\includegraphics{https://static01.nyt.com/images/2020/06/01/business/01unrest-disinfo/merlin_173069919_3a77e392-c604-4c90-b4fd-c76e3463c034-articleLarge.jpg?quality=75\&auto=webp\&disable=upscale}

By \href{https://www.nytimes.com/by/davey-alba}{Davey Alba}

\begin{itemize}
\item
  Published June 1, 2020Updated June 22, 2020
\item
  \begin{itemize}
  \item
  \item
  \item
  \item
  \item
  \end{itemize}
\end{itemize}

\href{https://www.nytimes.com/es/2020/06/03/espanol/ciencia-y-tecnologia/george-floyd-desinformacion-fake-news.html}{Leer
en español}

\emph{{[}Follow the live updates on}
\href{https://www.nytimes.com/2020/06/22/us/seattle-shooting-roosevelt-statue-nascar-noose.html}{\emph{Seattle,
Bubba Wallace, statues and the confederate flag}}\emph{.{]}}

On Twitter and Facebook, hundreds of posts are circulating saying that
George Floyd is not actually dead.

Conspiracy theorists are baselessly arguing that George Soros, the
billionaire investor and Democratic donor, is funding the
\href{https://www.nytimes.com/live/2020/george-floyd-protests-today-06-01}{spreading
protests} against police brutality.

And conservative commentators are asserting with little evidence that
\href{https://www.nytimes.com/article/what-antifa-trump.html}{antifa,
the far-left antifascism activist movement}, coordinated the riots and
looting that sprang from the protests.

Untruths, conspiracy theories and other false information are running
rampant online as the furor over Mr. Floyd, an African-American man who
\href{https://www.nytimes.com/2020/05/31/us/george-floyd-investigation.html}{was
killed last week in police custody} in Minneapolis, has built. The
misinformation has surged as the protests have dominated conversation,
far outpacing the volume of online posts and media mentions about last
year's
\href{https://www.nytimes.com/news-event/hong-kong-protests}{protests in
Hong Kong} and
\href{https://www.nytimes.com/2019/04/15/business/yellow-vests-movement-inequality.html}{Yellow
Vest movement} in France, according to the media insights company Zignal
Labs.

At its peak on Friday, Mr. Floyd and the protests around his death were
mentioned 8.8 million times, said Zignal Labs, which analyzed global
television broadcasts and social media. In contrast, news of the Hong
Kong protests reached 1.5 million mentions a day and the Yellow Vest
movement 941,000.

``The combination of evolving events, sustained attention and, most of
all, deep existing divisions make this moment a perfect storm for
disinformation,'' said Graham Brookie, director of the Atlantic
Council's Digital Forensic Research Lab. ``All of it is toxic, and make
our very real challenges and divisions harder to address.''

The collision of racial tensions and political polarization during
\href{https://www.nytimes.com/news-event/coronavirus?action=click\&pgtype=Article\&state=default\&module=styln-coronavirus\&variant=show\&region=TOP_BANNER\&context=storylines_menu}{the
coronavirus pandemic} has supersized the misinformation, researchers
said. Much of it is being shared by the conspiracy group QAnon and
far-right commentators as well as by those on the left, Mr. Brookie
said.

President Trump himself has stoked the divisive information. Over the
past few days, he posted on Twitter that
\href{https://www.nytimes.com/reuters/2020/05/31/us/31reuters-minneapolis-police-trump-antifa.html}{antifa
was a ``Terrorist Organization''} and urged the public to
\href{https://www.nytimes.com/2020/05/30/us/politics/trump-threatens-protesters-dogs-weapons.html}{show
up for a ``MAGA Night''} counterprotest at the White House.

Along with that, people are experiencing high levels of fear,
uncertainty and anger, said Claire Wardle, executive director of First
Draft, an organization that fights online disinformation. That creates
``the worst possible context for a healthy information environment,''
she said.

Twitter and Facebook did not immediately have a comment.

Here are three significant categories of falsehoods that have surfaced
on social media platforms about Mr. Floyd's death and the protests.

\hypertarget{george-floyds-fake-death}{%
\subsection{George Floyd's `Fake'
Death}\label{george-floyds-fake-death}}

The unfounded rumor that Mr. Floyd is alive is emblematic of the
misinformation narrative that a newsworthy event was staged. This has
become an increasingly common refrain over the years, with conspiracy
theorists saying, among other examples, that the 1969 moon landing and
the 2012 massacre at Sandy Hook Elementary School were hoaxes.

On Friday, the YouTube conspiracy channel JonXArmy shared a 22-minute
video that falsely asserted Mr. Floyd's death had been faked. The video
was shared nearly 100 times on Facebook, mostly in groups
\href{https://www.nytimes.com/2020/02/09/us/politics/qanon-trump-conspiracy-theory.html}{run
by QAnon}, reaching 1.3 million people, according to data from
CrowdTangle, a Facebook-owned tool that analyzes interactions across
social media.

Jon Miller, who runs the JonXArmy channel, did not immediately respond
to requests for comment. YouTube said on its site that it had removed
the video, citing its policy on hate speech.

On Twitter, posts stating that ``George Floyd is not dead'' were also
tweeted hundreds of times over the past week, with the phrase peaking at
15 mentions in a 10-minute span on Monday morning, according to
Dataminr, a social media monitoring service.

In thousands of other posts on Facebook and Twitter, people falsely
stated that Derek Chauvin, the Minnesota police officer who was
\href{https://www.nytimes.com/2020/05/29/us/minneapolis-police-george-floyd.html}{charged
with third-degree murder and second-degree manslaughter} in Mr. Floyd's
death, was an actor and that the entire incident had been faked by the
deep state.

\hypertarget{the-george-soros-conspiracy}{%
\subsection{The George Soros
Conspiracy}\label{the-george-soros-conspiracy}}

The false idea that Mr. Soros funded the protests spiked on social media
over the past week, showing how new events can resurrect old conspiracy
theories. Mr. Soros has for years been cast as an
\href{https://www.nytimes.com/2018/05/29/us/roseanne-george-soros-twitter.html}{anticonservative
villain by a loose network of activists and political figures on the
right} and has become a convenient boogeyman for all manner of ills.

On Twitter, Mr. Soros was mentioned in 34,000 tweets in connection with
Mr. Floyd's death over the past week, according to Dataminr. Over 90
videos in five languages mentioning Soros conspiracies were also posted
to YouTube over the past seven days, according to an analysis by The New
York Times.

On Facebook, 72,000 posts mentioned Mr. Soros in the past week, up from
12,600 the week before, according to The Times's analysis. Of the 10
most engaged posts about Mr. Soros on the social network, nine featured
false conspiracies linking him to the unrest. They were collectively
shared over 110,000 times.

Two of the top Facebook posts sharing Soros conspiracies were from
Texas' agriculture commissioner, ****
\href{https://www.facebook.com/MillerForTexas/}{Sid Miller}, an
outspoken supporter of Mr. Trump.

``I have no doubt in my mind that George Soros is funding these
so-called `spontaneous' protests,'' Mr. Miller wrote in one of the
posts. ``Soros is pure evil and is hell-bent on destroying our
country!''

Mr. Miller did not immediately respond to a request for comment.

Farshad Shadloo, a YouTube spokesman, said that the Soros conspiracy
videos did not violate the company's guidelines but that the site wasn't
recommending them.

A spokeswoman for Mr. Soros said, ``We deplore the false notion that the
people taking to the streets to express their anguish are paid, by
George Soros or anyone else.''

\hypertarget{antifa-misinformation}{%
\subsection{Antifa Misinformation}\label{antifa-misinformation}}

The unsubstantiated theory that antifa activists are responsible for the
riots and looting was the biggest piece of protest misinformation
tracked by Zignal Labs, which looked at certain categories of
falsehoods. Of 873,000 pieces of misinformation linked to the protests,
575,800 were mentions of antifa, Zignal Labs said.

The antifa narrative gained traction because ``long-established networks
of hyperpartisan social media influencers now work together like a
well-oiled machine,'' said Erin Gallagher, a social media researcher.

That began when Mr. Trump tweeted on Sunday that ``ANTIFA led
anarchists'' and ``Radical Left Anarchists'' were to blame for the
unrest, without providing specifics. Then he called antifa ``a Terrorist
Organization.''

Dan Bongino, a conservative political commentator who has unsuccessfully
run for a House seat several times, then took up the call. On the ``Fox
and Friends'' television show on Monday, Mr. Bongino said antifa
activists were responsible for a ``sophisticated'' attack on the White
House and called it an ``insurrection.''

He did not immediately respond to a request for comment.

Those assertions soon spread around social media. More than 6,000
Facebook posts linking the antifa movement to the protests appeared in
the last seven days, collecting over 1.3 million likes and shares,
according to The Times's analysis.

And on Twitter, a fake ``manual'' specifying ``riot orders'' that was
supposedly issued by Democrats directing antifa activists to stir up
trouble circulated prominently. But the so-called manual was a
resurrection of an old hoax linked to the April 2015 riots in Baltimore
over the death of Freddie Gray in police custody, the fact-checking
website
\href{https://www.snopes.com/fact-check/floyd-instruction-manual-protesters/}{Snopes
reported}.

Sheera Frenkel contributed reporting. Ben Decker contributed research.

Advertisement

\protect\hyperlink{after-bottom}{Continue reading the main story}

\hypertarget{site-index}{%
\subsection{Site Index}\label{site-index}}

\hypertarget{site-information-navigation}{%
\subsection{Site Information
Navigation}\label{site-information-navigation}}

\begin{itemize}
\tightlist
\item
  \href{https://help.nytimes.com/hc/en-us/articles/115014792127-Copyright-notice}{©~2020~The
  New York Times Company}
\end{itemize}

\begin{itemize}
\tightlist
\item
  \href{https://www.nytco.com/}{NYTCo}
\item
  \href{https://help.nytimes.com/hc/en-us/articles/115015385887-Contact-Us}{Contact
  Us}
\item
  \href{https://www.nytco.com/careers/}{Work with us}
\item
  \href{https://nytmediakit.com/}{Advertise}
\item
  \href{http://www.tbrandstudio.com/}{T Brand Studio}
\item
  \href{https://www.nytimes.com/privacy/cookie-policy\#how-do-i-manage-trackers}{Your
  Ad Choices}
\item
  \href{https://www.nytimes.com/privacy}{Privacy}
\item
  \href{https://help.nytimes.com/hc/en-us/articles/115014893428-Terms-of-service}{Terms
  of Service}
\item
  \href{https://help.nytimes.com/hc/en-us/articles/115014893968-Terms-of-sale}{Terms
  of Sale}
\item
  \href{https://spiderbites.nytimes.com}{Site Map}
\item
  \href{https://help.nytimes.com/hc/en-us}{Help}
\item
  \href{https://www.nytimes.com/subscription?campaignId=37WXW}{Subscriptions}
\end{itemize}
