\href{/section/health}{Health}\textbar{}Monster or Machine? A Profile of
the Coronavirus at 6 Months

\url{https://nyti.ms/2AynuYR}

\begin{itemize}
\item
\item
\item
\item
\item
\item
\end{itemize}

\href{https://www.nytimes.com/news-event/coronavirus?action=click\&pgtype=Article\&state=default\&region=TOP_BANNER\&context=storylines_menu}{The
Coronavirus Outbreak}

\begin{itemize}
\tightlist
\item
  live\href{https://www.nytimes.com/2020/07/31/world/coronavirus-covid-19.html?action=click\&pgtype=Article\&state=default\&region=TOP_BANNER\&context=storylines_menu}{Latest
  Updates}
\item
  \href{https://www.nytimes.com/interactive/2020/us/coronavirus-us-cases.html?action=click\&pgtype=Article\&state=default\&region=TOP_BANNER\&context=storylines_menu}{Maps
  and Cases}
\item
  \href{https://www.nytimes.com/interactive/2020/science/coronavirus-vaccine-tracker.html?action=click\&pgtype=Article\&state=default\&region=TOP_BANNER\&context=storylines_menu}{Vaccine
  Tracker}
\item
  \href{https://www.nytimes.com/interactive/2020/07/29/us/schools-reopening-coronavirus.html?action=click\&pgtype=Article\&state=default\&region=TOP_BANNER\&context=storylines_menu}{What
  School May Look Like}
\item
  \href{https://www.nytimes.com/live/2020/07/31/business/stock-market-today-coronavirus?action=click\&pgtype=Article\&state=default\&region=TOP_BANNER\&context=storylines_menu}{Economy}
\end{itemize}

\includegraphics{https://static01.nyt.com/images/2020/06/02/science/02CORONAVIRUSPROFILE-02/02CORONAVIRUSPROFILE-02-articleLarge.jpg?quality=75\&auto=webp\&disable=upscale}

Sections

\protect\hyperlink{site-content}{Skip to
content}\protect\hyperlink{site-index}{Skip to site index}

\hypertarget{monster-or-machine-a-profile-of-the-coronavirus-at-6-months}{%
\section{Monster or Machine? A Profile of the Coronavirus at 6
Months}\label{monster-or-machine-a-profile-of-the-coronavirus-at-6-months}}

Our ``hidden enemy,'' in plain sight.

Credit...Richard McGuire

Supported by

\protect\hyperlink{after-sponsor}{Continue reading the main story}

By Alan Burdick

\begin{itemize}
\item
  Published June 2, 2020Updated June 15, 2020
\item
  \begin{itemize}
  \item
  \item
  \item
  \item
  \item
  \item
  \end{itemize}
\end{itemize}

\href{https://www.nytimes.com/es/2020/06/02/espanol/ciencia-y-tecnologia/perfil-coronavirus-covid.html}{Leer
en español}

\hypertarget{listen-to-this-audio}{%
\subsubsection{Listen to This Audio}\label{listen-to-this-audio}}

Audio Recording by Audm

\emph{To hear more audio stories from publishers like The New York
Times,
download}\href{https://www.audm.com/?utm_source=nytmag\&utm_medium=embed\&utm_campaign=left_behind_draper}{**}\href{https://www.audm.com/?utm_source=nyt\&utm_medium=embed\&utm_campaign=monster_or_machine}{\emph{Audm
for iPhone or Android.}}

A virus, at heart, is information, a packet of data that benefits from
being shared.

The information at stake is genetic: instructions to make more
\href{https://www.nytimes.com/2020/06/15/health/coronavirus-underlying-conditions.html}{virus}.
Unlike a truly living organism, a
\href{https://www.nytimes.com/2020/06/15/health/coronavirus-underlying-conditions.html}{virus}
cannot replicate on its own; it cannot move, grow, persist or
perpetuate. It needs a host. The viral code breaks into a living cell,
hijacks the genetic machinery and instructs it to produce new code ---
new virus.

President Trump has characterized the response to the pandemic as a
``medical war,'' and described the virus behind it as, by turns,
``genius,'' a ``hidden enemy'' and ``a monster.'' It would be more
accurate to say that we find ourselves at odds with a microscopic
photocopy machine. Not even that: an assembly manual for a photocopier,
model SARS-CoV-2.

For at least six months now, the virus has replicated among us. The toll
has been devastating. Officially, more than six million people worldwide
have been infected so far, and 370,000 have died. (The actual numbers
are certainly higher.) The United States, which has seen the largest
share of cases and casualties, recently surpassed 100,000 deaths,
one-fourth the number of all Americans who died in World War II.
Businesses are shuttered --- in 10 weeks, some 40 million Americans have
lost their jobs --- and
\href{https://www.nytimes.com/2020/04/08/business/economy/coronavirus-food-banks.html}{food
banks are overrun}. The virus has fueled widespread frustration and
exposed our deepest faults: of color, class and privilege, between the
deliverers and the delivered to.

Still, summer --- summer! --- has all but arrived. We step out to look,
breathe, vent. The pause is illusory. Cases are falling in New York, the
epicenter in the United States, but
\href{https://www.nytimes.com/interactive/2020/us/coronavirus-us-cases.html?action=click\&pgtype=Article\&state=default\&module=styln-coronavirus-markets\&variant=show\&region=TOP_BANNER\&context=storylines_menu\#states}{firmly
rising} in Wisconsin, Virginia, Alabama, Arkansas, North and South
Carolina, and other states. China, where the pandemic originated, and
South Korea saw recent resurgences. Health officials fear another major
wave of infections in the fall, and
\href{https://www.nytimes.com/2020/05/08/health/coronavirus-pandemic-curve-scenarios.html}{a
possible wave train beyond}.

``We are really early in this disease,'' Dr. Ashish Jha, the director of
the Harvard Global Health Institute,
\href{https://www.nytimes.com/2020/05/03/world/asia/coronavirus-spread-where-why.html?campaign_id=9\&emc=edit_nn_20200504\&instance_id=18202\&nl=the-morning\&regi_id=102543212\&segment_id=26556\&te=1\&user_id=11229ce0c34ff5caaf09af6410292613}{told
The Times recently}. ``If this were a baseball game, it would be the
second inning.''

There may be
\href{https://www.nytimes.com/2020/03/24/science/viruses-coranavirus-biology.html}{trillions
of species of virus} in the world. They infect bacteria, mostly, but
also abalone,
\href{https://www.nytimes.com/2020/01/28/science/bats-coronavirus-Wuhan.html}{bats},
beans, beetles, blackberries, cassavas, cats, dogs, hermit crabs,
mosquitoes, potatoes, pangolins, ticks and the Tasmanian devil. They
give birds cancer and turn bananas black. Of the trillions, a few
hundred thousand kinds of viruses are known, and fewer than 7,000 have
names. Only about 250, including SARS-CoV-2, have the mechanics to
infect us.

In our information age, we have grown familiar with computer viruses and
with memes going viral; now here is the real thing to remind us what the
metaphor means. A mere wisp of data has grounded more than half of the
world's commercial airplanes, sharply reduced global carbon emissions
and doubled the stock price of Zoom. It has infiltrated our language ---
``social distancing,'' ``immunocompromised shoppers'' --- and our
\href{https://www.nytimes.com/2020/04/13/style/why-weird-dreams-coronavirus.html}{dreams}.
It has postponed sports, political conventions, and the premieres of the
next Spider-Man, Black Widow, Wonder Woman and James Bond films. Because
of the virus, the U.S. Supreme Court renders rulings by telephone, and
\href{https://www.nytimes.com/2020/04/17/arts/coronavirus-nature-genre.html}{wild
boars} roam the empty streets of Barcelona.

It also has prompted a collaborative response unlike any our species has
seen. Teams of scientists,
\href{https://www.nytimes.com/2020/04/01/world/europe/coronavirus-science-research-cooperation.html}{working
across national boundaries}, are racing to understand the virus's
weaknesses, develop treatments and vaccine candidates, and to accurately
forecast its next moves. Medical workers are risking their lives to tend
to the sick. Those of us at home do what we can: share instructions for
how to make a surgical mask from a pillowcase; sing and cheer from
windows and doorsteps; send condolences; offer hope.

``We're mounting a reaction against the virus that is truly
unprecedented,'' said Dr. Melanie Ott, director of the Gladstone
Institute of Virology in San Francisco.

So far the match is deadlocked. We gather, analyze, disseminate, probe:
What is this thing? What must be done? When can life return to normal?
And we hide, while the latest iteration of an ancient biochemical cipher
ticks on, advancing itself at our expense.

\hypertarget{a-fearsome-envelope}{%
\subsection{A fearsome envelope}\label{a-fearsome-envelope}}

Who knows when viruses first came about. Perhaps, as one theory holds,
they began as free-living microbes that, through natural selection, were
stripped down and became parasites. Maybe they began as genetic cogs
within microbes, then gained the ability to venture out and invade other
cells. Or maybe viruses came first, shuttling and replicating in the
primordial protein soup, gaining shades of complexity --- enzymes, outer
membranes --- that gave rise to cells and, eventually, us. They are
sacks of code --- double- or single-stranded, DNA or RNA --- and
sometimes called capsid-encoding organisms, or C.E.O.s

As viruses go, SARS-CoV-2 is big --- its genome is more than twice the
size of that of the average flu virus and about one-half larger than
Ebola's. But it is still tiny: 10,000 times smaller than a millimeter,
barely one-thousandth the width of a human hair, smaller even than the
wavelength of light from a germicidal lamp. If a person were the size of
Earth, the virus would be the size of a person. Picture a human lung
cell as a cramped office just big enough for a desk, a chair and a copy
machine. SARS-CoV-2 is an oily envelope stuck to the door.

It was
\href{https://www.nytimes.com/2020/01/08/health/china-pneumonia-outbreak-virus.html}{formally
identified} on Jan. 7 by scientists in China. For weeks beforehand, a
mysterious respiratory ailment had been circulating in the city of
Wuhan. Health officials were worried that it might be a reappearance of
severe acute respiratory syndrome, or SARS, an alarming viral illness
that emerged abruptly in 2002, infected more than 8,000 people and
killed nearly 800 in the next several months, then was quarantined into
oblivion.

The scientists had gathered fluid samples from three patients and, with
nucleic-acid extractors and other tools, compared the genome of the
pathogen with that of known ones. A transmission electron microscope
\href{https://www.nejm.org/doi/full/10.1056/NEJMoa2001017}{revealed the
culprit}: spherical, with ``quite distinctive spikes'' reminiscent of a
crown or the corona of the sun. It was a coronavirus, and a novel one.

\includegraphics{https://static01.nyt.com/images/2020/06/02/science/02CORONAVIRUSPROFILE/02CORONAVIRUSPROFILE-articleLarge.jpg?quality=75\&auto=webp\&disable=upscale}

In later colorized images, the virus resembles
\href{https://www.youtube.com/watch?v=oBQvvCY2Mj0}{small garish orbs of
lint} or the papery eggs of certain spiders, adhering by the dozens to
much larger cells. Recently a visual team, working closely with
researchers, created ``the most accurate
\href{https://vimeo.com/417208044/758c67edaf}{model of the SARS-CoV-2
viral particle} currently available'': a barbed, multicolored globe with
the texture of fine moss, like something out of Dr. Seuss, or a sunken
naval mine draped in algae and sponges.

Once upon a time, our pathogens were crudely named: Spanish flu, Asian
flu, yellow fever, Black Death. Now we have H1N1, MERS (Middle East
Respiratory Syndrome), H.I.V. --- strings of letters as streamlined as
the viruses themselves, codes for codes. The new coronavirus was
temporarily named
\href{https://www.who.int/docs/default-source/coronaviruse/situation-reports/20200130-sitrep-10-ncov.pdf?sfvrsn=d0b2e480_2}{2019-nCoV}.
On Feb. 11, the International Committee on Taxonomy of Viruses
officially renamed it SARS-CoV-2, to indicate that it was very closely
related to the SARS virus, another coronavirus.

\hypertarget{latest-updates-global-coronavirus-outbreak}{%
\section{\texorpdfstring{\href{https://www.nytimes.com/2020/07/31/world/coronavirus-covid-19.html?action=click\&pgtype=Article\&state=default\&region=MAIN_CONTENT_1\&context=storylines_live_updates}{Latest
Updates: Global Coronavirus
Outbreak}}{Latest Updates: Global Coronavirus Outbreak}}\label{latest-updates-global-coronavirus-outbreak}}

Updated 2020-07-31T12:09:36.060Z

\begin{itemize}
\tightlist
\item
  \href{https://www.nytimes.com/2020/07/31/world/coronavirus-covid-19.html?action=click\&pgtype=Article\&state=default\&region=MAIN_CONTENT_1\&context=storylines_live_updates\#link-626650}{The
  deal with Sanofi and GlaxoSmithKline is the biggest so far with the
  U.S. government.}
\item
  \href{https://www.nytimes.com/2020/07/31/world/coronavirus-covid-19.html?action=click\&pgtype=Article\&state=default\&region=MAIN_CONTENT_1\&context=storylines_live_updates\#link-3d16bd29}{As
  states scramble to put out fires, Fauci and other top U.S. health
  officials will go back before Congress.}
\item
  \href{https://www.nytimes.com/2020/07/31/world/coronavirus-covid-19.html?action=click\&pgtype=Article\&state=default\&region=MAIN_CONTENT_1\&context=storylines_live_updates\#link-622686c1}{Five
  key developments you may have missed on Thursday.}
\end{itemize}

\href{https://www.nytimes.com/2020/07/31/world/coronavirus-covid-19.html?action=click\&pgtype=Article\&state=default\&region=MAIN_CONTENT_1\&context=storylines_live_updates}{See
more updates}

More live coverage:
\href{https://www.nytimes.com/live/2020/07/31/business/stock-market-today-coronavirus?action=click\&pgtype=Article\&state=default\&region=MAIN_CONTENT_1\&context=storylines_live_updates}{Markets}

Before the emergence of the original SARS, the study of coronaviruses
was a professional backwater. ``There has been such a deluge of
attention on we coronavirologists,'' said Susan R. Weiss, a virologist
at the University of Pennsylvania. ``It is quite in contrast to
previously being mostly ignored.''

There are hundreds of kinds of coronaviruses. Two of them, SARS-CoV and
MERS-CoV, can be deadly; four cause one-third of common colds. Many
infect animals with which humans associate, including camels, cats,
chickens, and bats. All are RNA viruses. Our coronavirus, like the
others, is a string of roughly 30,000 biochemical building blocks called
nucleotides enclosed in a membrane of both protein and lipid.

``I've always been impressed by coronaviruses,'' said Anthony Fehr, a
virologist at Kansas University. ``They are extremely complex in the way
that they get around and start to take over a cell. They make more genes
and more proteins than most other RNA viruses, which gives them more
options to shut down the host cell.''

The core code of SARS-CoV-2 contains genes for
\href{https://www.nytimes.com/interactive/2020/04/03/science/coronavirus-genome-bad-news-wrapped-in-protein.html?searchResultPosition=1}{as
many as 29 proteins}: the instructions to replicate the code. One
protein, S, provides the spikes on the surface of the virus and unlocks
the door to the target cell. The others, on entry, separate and attend
to their tasks: turning off the cell's alarm system; commandeering the
copier to make new viral proteins; folding viral envelopes, and helping
new viruses bubble out of the cell by the thousands.

``I usually picture it as an entity that comes into the cell and then it
falls apart,'' Dr. Ott said. ``It has to fall apart to build some
mini-factories in the cell to reproduce itself, and has to come together
as an entity at the end to infect other cells.''

For medical researchers, these proteins are key to understanding why the
virus is so successful, and how it might be neutralized. For instance,
to break into a cell, the S protein binds to a receptor called
angiotensin converting enzyme 2, or ACE2, like a hand on a doorknob. The
S protein on this coronavirus is nearly identical in structure to the
one in the first SARS --- ``SARS Classic'' --- but some data suggests
that it binds to the target enzyme far more strongly. Some researchers
think this may partly explain why the new virus infects humans so
efficiently.

Every pathogen evolves along a path between impact and stealth. Too mild
and the illness does not spread from person to person; too visible and
the carrier, unwell and aware, stays home or is avoided --- and the
illness does not spread. ``SARS infected 8,000 people, and was contained
quickly, in part because it didn't spread before symptoms appeared,''
Dr. Weiss noted.

By comparison, SARS-CoV-2 seems to have achieved an admirable balance.
``No aspect of the virus is extraordinary,'' said Dr. Pardis Sabeti, a
computational geneticist at the Broad Institute who helped sequence the
Ebola virus in 2014. ``It's the combination of things that makes it
extraordinary.''

SARS Classic settled quickly into human lung cells, causing a person to
cough but also announcing its presence. In contrast, its successor tends
to colonize first the nose and throat, sometimes causing few initial
symptoms. Some cells there are thought to be rich in the surface enzyme
ACE2 --- the doorknob that SARS-CoV-2 turns so readily. The virus
replicates quietly, and quietly spreads: One study found that a person
carrying SARS-CoV-2 is most contagious two to three days before they are
aware that they might be ill.

From there, the virus can move into the lungs. The delicate alveoli,
which gather oxygen essential to the body, become inflamed and struggle
to do their job. The texture of the lungs turns from airy froth to gummy
marshmallow. The patient may develop pneumonia; some, drowning
internally and desperate for oxygen, go into acute respiratory distress
and require a ventilator.

The virus can settle in still further: damaging the muscular walls of
the heart; attacking the lining of the blood vessels and generating
clots; inducing strokes, seizures and inflammation of the brain; and
damaging the kidneys. Often the greatest damage is inflicted not by the
virus but by the body's attempt to fight it off with a dangerous
``cytokine
storm''\href{https://www.nytimes.com/2020/04/01/health/coronavirus-cytokine-storm-immune-system.html}{of
immune system molecules}.

The result is an illness with a perplexing array of faces. A dry cough
and a low fever at the outset, sometimes. Shortness of breath or
difficulty breathing, sometimes. Maybe you lose your sense of smell or
taste. Maybe your toes become
\href{https://www.nytimes.com/2020/05/01/health/coronavirus-covid-toe.html}{red
and inflamed}, as if you had frostbite. For some patients it feels like
\href{https://www.sfgate.com/bayarea/article/Coronavirus-updates-COVID-19-Bay-Area-deaths-cases-15225947.phphttps://www.sfgate.com/bayarea/article/Coronavirus-updates-COVID-19-Bay-Area-deaths-cases-15225947.php}{a
heart attack}, or it causes delusion or disorientation.

Often it feels like nothing at all; according to the Centers for Disease
Control and Prevention,
\href{https://www.nytimes.com/2020/03/31/health/coronavirus-asymptomatic-transmission.htmlhttps://www.nytimes.com/2020/03/31/health/coronavirus-asymptomatic-transmission.html}{35
percent} of people who contract the virus experience few to no symptoms,
although they can continue to spread it. ``The virus acts like no
pathogen humanity has ever seen,'' the journal Science
\href{https://www.sciencemag.org/news/2020/04/how-does-coronavirus-kill-clinicians-trace-ferocious-rampage-through-body-brain-toes}{recently
noted}.

More to the point, the pathogen has gone largely unseen. ``It has these
perfect properties to spread throughout the entire human population,''
Dr. Fehr said. ``If we didn't know what a virus was'' --- and didn't
take proper precautions --- ``this virus would infect virtually every
human on the planet. It still might do that.''

\hypertarget{data-vs-data}{%
\subsection{Data vs. data}\label{data-vs-data}}

On Jan. 10, the Wuhan health commission in China reported that in the
previous weeks, 41 people had contracted the illness caused by the
coronavirus, and that one had died --- the first known casualty at the
time.

That same day, Chinese scientists publicly released the complete genome
of the virus. The blueprint, which could be simulated and synthesized in
the lab, was almost as good as a physical sample, and easier for
researchers worldwide to obtain. Analyses appeared in journals and on
preprint servers like bioRxiv, on sites like
\href{https://nextstrain.org/}{nextstrain.org} and
\href{http://virological.org/}{virological.org}: clues to the virus's
origin, its errors and its weaknesses. From then on, the new coronavirus
began to replicate not only physically in human cells but also
figuratively, and likely to its own detriment, in the human mind.

Dr. Ott entered medicine in the 1980s, when AIDS was still new and
terrifyingly unknown. ``Compare that time to today, there are a lot of
similarities,'' she said. ``A new virus, a rush to understand, a rush to
a cure or a vaccine. What's fundamentally different now is that we have
generated this community of collaboration and data-sharing. It's really
mind-blowing.''

Three hours after the virus's code was published, Inovio
Pharmaceuticals, based in San Diego,
\href{https://www.voanews.com/science-health/coronavirus-outbreak/new-tech-could-make-corona}{began
work} on a vaccine against it --- one of more than 100 such efforts now
underway around the world. Dr. Sabeti's lab quickly got to work
developing diagnostic tests. Dr. Ott and Dr. Weiss soon managed to
obtain samples of live virus, which allowed them to ``actually look at
what's going on'' when it infects cells in the lab, Dr. Ott said.

``The cell is mounting a profound battle to prevent the virus from
entering or, on entering, to alarm everyone around it so it can't
spread,'' she said. ``The virus's intent is to overcome this initial
surge of defense, to set up shop long enough to reproduce itself and to
spread.''

Image

Credit...Richard McGuire

With so many proteins in its tool kit, the virus has many ways to
counter our immune system; these also offer targets for
\href{https://www.nytimes.com/2020/04/30/health/coronavirus-antiviral-drugs.html}{potential
vaccines and drugs}. Researchers are working every angle. Most vaccine
efforts are focused on disrupting the spike proteins, which allow entry
into the cell. The drug remdesivir targets the virus's replication
machinery. Dr. Fehr studies how the virus disables our immune system.

``I use the analogy of Star Wars,'' he said. ``The virus is the Dark
Side. We have a cellular defense system of hundreds of antiviral
proteins'' --- Jedi knights --- ``to defend ourselves. Our lab is
studying one specific Jedi that uses one particular weapon, and how the
virus fights back.''

These battles, fought on the field of biochemistry, strain the alphabet
to describe. The Jedi in this analogy are particular enzymes
(poly-ADP-ribose polymerases, or PARPS, if you must know) that are
produced in infected cells and wield a molecule that attaches to certain
invading proteins --- ``we don't know what these are yet,'' Dr. Fehr
said --- and disrupts them. In response, the virus has an enzyme of its
own that sweeps away our Jedi like dust from a sandcrawler.

Carolyn Machamer, a cell biologist at the Johns Hopkins School of
Medicine, is studying the later stages of the process, to learn how the
virus manages to navigate and assemble itself within a host cell and
depart it. Among the
\href{https://cellbio.jhmi.edu/people/faculty/carolyn-machamer-phd}{research
topics} listed on her university webpage are coronaviruses but also
``intracellular protein trafficking'' and ``exocytosis of large cargo.''

\href{https://www.nytimes.com/news-event/coronavirus?action=click\&pgtype=Article\&state=default\&region=MAIN_CONTENT_3\&context=storylines_faq}{}

\hypertarget{the-coronavirus-outbreak-}{%
\subsubsection{The Coronavirus Outbreak
›}\label{the-coronavirus-outbreak-}}

\hypertarget{frequently-asked-questions}{%
\paragraph{Frequently Asked
Questions}\label{frequently-asked-questions}}

Updated July 27, 2020

\begin{itemize}
\item ~
  \hypertarget{should-i-refinance-my-mortgage}{%
  \paragraph{Should I refinance my
  mortgage?}\label{should-i-refinance-my-mortgage}}

  \begin{itemize}
  \tightlist
  \item
    \href{https://www.nytimes.com/article/coronavirus-money-unemployment.html?action=click\&pgtype=Article\&state=default\&region=MAIN_CONTENT_3\&context=storylines_faq}{It
    could be a good idea,} because mortgage rates have
    \href{https://www.nytimes.com/2020/07/16/business/mortgage-rates-below-3-percent.html?action=click\&pgtype=Article\&state=default\&region=MAIN_CONTENT_3\&context=storylines_faq}{never
    been lower.} Refinancing requests have pushed mortgage applications
    to some of the highest levels since 2008, so be prepared to get in
    line. But defaults are also up, so if you're thinking about buying a
    home, be aware that some lenders have tightened their standards.
  \end{itemize}
\item ~
  \hypertarget{what-is-school-going-to-look-like-in-september}{%
  \paragraph{What is school going to look like in
  September?}\label{what-is-school-going-to-look-like-in-september}}

  \begin{itemize}
  \tightlist
  \item
    It is unlikely that many schools will return to a normal schedule
    this fall, requiring the grind of
    \href{https://www.nytimes.com/2020/06/05/us/coronavirus-education-lost-learning.html?action=click\&pgtype=Article\&state=default\&region=MAIN_CONTENT_3\&context=storylines_faq}{online
    learning},
    \href{https://www.nytimes.com/2020/05/29/us/coronavirus-child-care-centers.html?action=click\&pgtype=Article\&state=default\&region=MAIN_CONTENT_3\&context=storylines_faq}{makeshift
    child care} and
    \href{https://www.nytimes.com/2020/06/03/business/economy/coronavirus-working-women.html?action=click\&pgtype=Article\&state=default\&region=MAIN_CONTENT_3\&context=storylines_faq}{stunted
    workdays} to continue. California's two largest public school
    districts --- Los Angeles and San Diego --- said on July 13, that
    \href{https://www.nytimes.com/2020/07/13/us/lausd-san-diego-school-reopening.html?action=click\&pgtype=Article\&state=default\&region=MAIN_CONTENT_3\&context=storylines_faq}{instruction
    will be remote-only in the fall}, citing concerns that surging
    coronavirus infections in their areas pose too dire a risk for
    students and teachers. Together, the two districts enroll some
    825,000 students. They are the largest in the country so far to
    abandon plans for even a partial physical return to classrooms when
    they reopen in August. For other districts, the solution won't be an
    all-or-nothing approach.
    \href{https://bioethics.jhu.edu/research-and-outreach/projects/eschool-initiative/school-policy-tracker/}{Many
    systems}, including the nation's largest, New York City, are
    devising
    \href{https://www.nytimes.com/2020/06/26/us/coronavirus-schools-reopen-fall.html?action=click\&pgtype=Article\&state=default\&region=MAIN_CONTENT_3\&context=storylines_faq}{hybrid
    plans} that involve spending some days in classrooms and other days
    online. There's no national policy on this yet, so check with your
    municipal school system regularly to see what is happening in your
    community.
  \end{itemize}
\item ~
  \hypertarget{is-the-coronavirus-airborne}{%
  \paragraph{Is the coronavirus
  airborne?}\label{is-the-coronavirus-airborne}}

  \begin{itemize}
  \tightlist
  \item
    The coronavirus
    \href{https://www.nytimes.com/2020/07/04/health/239-experts-with-one-big-claim-the-coronavirus-is-airborne.html?action=click\&pgtype=Article\&state=default\&region=MAIN_CONTENT_3\&context=storylines_faq}{can
    stay aloft for hours in tiny droplets in stagnant air}, infecting
    people as they inhale, mounting scientific evidence suggests. This
    risk is highest in crowded indoor spaces with poor ventilation, and
    may help explain super-spreading events reported in meatpacking
    plants, churches and restaurants.
    \href{https://www.nytimes.com/2020/07/06/health/coronavirus-airborne-aerosols.html?action=click\&pgtype=Article\&state=default\&region=MAIN_CONTENT_3\&context=storylines_faq}{It's
    unclear how often the virus is spread} via these tiny droplets, or
    aerosols, compared with larger droplets that are expelled when a
    sick person coughs or sneezes, or transmitted through contact with
    contaminated surfaces, said Linsey Marr, an aerosol expert at
    Virginia Tech. Aerosols are released even when a person without
    symptoms exhales, talks or sings, according to Dr. Marr and more
    than 200 other experts, who
    \href{https://academic.oup.com/cid/article/doi/10.1093/cid/ciaa939/5867798}{have
    outlined the evidence in an open letter to the World Health
    Organization}.
  \end{itemize}
\item ~
  \hypertarget{what-are-the-symptoms-of-coronavirus}{%
  \paragraph{What are the symptoms of
  coronavirus?}\label{what-are-the-symptoms-of-coronavirus}}

  \begin{itemize}
  \tightlist
  \item
    Common symptoms
    \href{https://www.nytimes.com/article/symptoms-coronavirus.html?action=click\&pgtype=Article\&state=default\&region=MAIN_CONTENT_3\&context=storylines_faq}{include
    fever, a dry cough, fatigue and difficulty breathing or shortness of
    breath.} Some of these symptoms overlap with those of the flu,
    making detection difficult, but runny noses and stuffy sinuses are
    less common.
    \href{https://www.nytimes.com/2020/04/27/health/coronavirus-symptoms-cdc.html?action=click\&pgtype=Article\&state=default\&region=MAIN_CONTENT_3\&context=storylines_faq}{The
    C.D.C. has also} added chills, muscle pain, sore throat, headache
    and a new loss of the sense of taste or smell as symptoms to look
    out for. Most people fall ill five to seven days after exposure, but
    symptoms may appear in as few as two days or as many as 14 days.
  \end{itemize}
\item ~
  \hypertarget{does-asymptomatic-transmission-of-covid-19-happen}{%
  \paragraph{Does asymptomatic transmission of Covid-19
  happen?}\label{does-asymptomatic-transmission-of-covid-19-happen}}

  \begin{itemize}
  \tightlist
  \item
    So far, the evidence seems to show it does. A widely cited
    \href{https://www.nature.com/articles/s41591-020-0869-5}{paper}
    published in April suggests that people are most infectious about
    two days before the onset of coronavirus symptoms and estimated that
    44 percent of new infections were a result of transmission from
    people who were not yet showing symptoms. Recently, a top expert at
    the World Health Organization stated that transmission of the
    coronavirus by people who did not have symptoms was ``very rare,''
    \href{https://www.nytimes.com/2020/06/09/world/coronavirus-updates.html?action=click\&pgtype=Article\&state=default\&region=MAIN_CONTENT_3\&context=storylines_faq\#link-1f302e21}{but
    she later walked back that statement.}
  \end{itemize}
\end{itemize}

On entering the cell, components of the virus set up shop in a
subregion, or organelle, called the Golgi complex, which resembles a
stack of pancakes and serves as the cell's mail-sorting center. Dr.
Machamer has been working to understand how the virus commandeers the
unit to route all the newly replicated viral bits, scattered throughout
the cell, for final assembly.

The subject was ``poorly studied,'' she conceded. Most drug research has
focused on the early stages, like blocking infection at the very outset
or disrupting replication inside the cell. ``Like I said, it hasn't
gotten a whole lot of attention,'' she said. ``But I think it will now,
because I think we have some really interesting targets that could
possibly yield new types of drugs.''

The line of inquiry dates back to her postdoctoral days. She was
studying the Golgi complex --- ``the organelle is really bizarre'' ---
even then. ``It's following what you're interested in, that's what basic
science is about. It's, like, you don't actually set out to cure the
world or anything, but you follow your nose.''

For all the attention the virus has received, it is still new to science
and rich in unknowns. ``I'm still very focused on the question, How does
the virus get into the body?'' Dr. Ott said. ``Which cells does it
infect in the upper airway? How does it get into the lower airway, and
from there to other organs? It's absolutely not clear what the path is,
or what the vulnerable path types are.

And most pressing: Why are so many of us asymptomatic? ``How does the
virus manage to do this without leaving traces in some people, but in
others there's a giant reaction?'' she said. ``That's the biggest
question currently, and the most urgent.''

\hypertarget{mistakes-are-made}{%
\subsection{Mistakes are made}\label{mistakes-are-made}}

Even a photocopier is imperfect, and SARS-CoV-2 is no exception. When
the virus commandeers a host cell to copy itself, invariably mistakes
are made, an incorrect nucleotide swapped for the right one, for
instance. In theory, such mutations, or an accumulation of them, could
make a virus more infectious or deadly, or less so, but in the vast
majority of cases, they do not affect a virus's performance.

What's important to note is that the process is random and incessant.
Humans describe the contest between host and virus as a war, but the
virus is not at war. Our enemy has no agency; it does not develop
``strategies'' for escaping our medicines or the activity of our immune
systems.

Unlike some viruses, SARS-CoV-2 has a proofreading protein --- NSP14 ---
that clips out mistakes. Even still, errors slip through. The virus
acquires \href{https://bedford.io/blog/ncov-cryptic-transmission/}{two
mutations a month}, on average, which is
\href{https://jvi.asm.org/content/84/19/9733}{less than half the error
rate of the flu} --- and increases the possibility that a vaccine or
drug treatment, once developed, will not be quickly outdated. ``So far
it's been relatively faithful,'' Dr. Ott said. ``That's good for us.''

By March, \href{http://www.graphen.ai/covid.html}{at least 1,388}
variants of the coronavirus had been detected around the world, all
functionally identical as far as scientists could tell. Arrayed as an
ancestral tree, these lineages reveal where and when the virus spread.
For instance, the first confirmed case of Covid-19 in New York was
announced on March 1, but an analysis of samples
\href{https://www.nytimes.com/2020/04/08/science/new-york-coronavirus-cases-europe-genomes.htmlhttps://www.nytimes.com/2020/04/08/science/new-york-coronavirus-cases-europe-genomes.html}{revealed
that} the virus had begun to circulate in the region weeks earlier.
Unlike early cases on the West Coast, which were seeded by people
arriving from China, these cases were seeded
\href{https://nextstrain.org/narratives/ncov/sit-rep/2020-04-17?n=5}{from
Europe}, and in turn
\href{https://www.nytimes.com/2020/05/07/us/new-york-city-coronavirus-outbreak.html}{seeded
cases throughout much of the country}.

The roots can be traced back still further. The first known patient was
hospitalized in Wuhan on Dec. 16, 2019, and first felt ill on Dec. 1;
the first infection would have occurred still earlier. Sometime before
that the virus, or its progenitor, was in a bat --- the genome is 96
percent similar to a bat virus. How long ago it made that jump, and
acquired the mutations necessary to do so, is unclear. In any case, and
contrary to
\href{https://www.nytimes.com/2020/03/13/world/asia/coronavirus-china-conspiracy-theory.html}{certain
conspiracy theories}, SARS-CoV-2 was not engineered in a laboratory.

Image

Credit...Richard McGuire

``Those scenarios are so unlikely as to be impossible,'' said Dr. Robert
Garry, a microbiologist at Tulane University and an expert on emerging
diseases. In March, a team of researchers including Dr. Garry published
\href{https://www.nature.com/articles/S41591-020-0820-9}{a paper} in
Nature Medicine comparing the genome and protein structures of the novel
virus to those of other coronaviruses. The novel distinctions were
``most likely the result of natural selection,'' they concluded. ``Our
analyses clearly show that SARS-CoV-2 is not a laboratory construct or a
purposefully manipulated virus.''

In our species, the virus has found prime habitat. It seems to do most
of its replicating in the upper respiratory tract, Dr. Garry noted:
``That makes it easier to spread with your voice, so there may be more
opportunities for it to spread casually, and perhaps earlier in the
course of the disease.''

And there we have it: an organism, or whatever the right word is,
ideally adapted to human conversation, the louder the better. Our
communication is its transmission. Consider where so many outbreaks have
begun: funerals, parties, call centers, sports arenas, meatpacking
plants, dorm rooms, cruise ships, prisons. In February, a
\href{https://www.nytimes.com/2020/04/12/us/coronavirus-biogen-boston-superspreader.html}{medical
conference in Boston} led to more than 70 cases in two weeks. In
Arkansas, several cases were linked to ``a high school swim party that
I'm sure everybody thought was harmless,'' Gov. Asa Hutchinson said.
After a choir rehearsal in Mount Vernon, Wash., 28 members of the choir
fell ill. Not even song is safe anymore.

The virus has no trouble finding us. But we are still struggling to find
it; a recent model by epidemiologists at Columbia University estimated
that for every documented infection in the United States, 12 more go
undetected. Who has it, or had it, and who does not? A firm grasp of the
virus's whereabouts --- using diagnostic tests, antibody tests and
contact tracing --- is essential to our bid to return normal life. But
humanity's immune response has been uneven.

In late May, in an open letter, a group of former White House science
advisers warned that, to prepare for an anticipated resurgence of the
pandemic later this year, the federal government needed to begin
preparing immediately to avoid the ``extraordinary shortage of
supplies'' that occurred this spring.

``The virus is here, it's everywhere,'' Dr. Rick Bright, the former
director of the Biomedical Advanced Research and Development Authority,
\href{https://www.rev.com/blog/transcripts/dr-rick-bright-testimony-transcript-vaccine-expert-whistleblower-ousted-by-trump-testifies}{told
the U.S. Senate} in mid-May. ``We need to unleash the voices of the
scientists in our public health system in the United States, so they can
be heard.'' Right now, he added, ``There is no master coordinated plan
on how to respond to this outbreak.''

SARS-CoV-2 virus has no plan. It doesn't need one; absent a vaccine, the
virus is here to stay. ``This is a pretty efficient pathogen,'' Dr.
Garry said. ``It's very good at what it does.''

\hypertarget{the-next-wave}{%
\subsection{The next wave}\label{the-next-wave}}

``The virus spreads because of an intrinsic, latent quality in the
culture,'' the media theorist Douglas Rushkoff, who two decades ago
coined the phrase ``going viral,''
\href{https://rushkoff.com/digital-trends-trump-media-virus/}{wrote
recently}. ``Both biological and media viruses say less about themselves
than they do about their hosts.''

To know SARS-CoV-2 is to know ourselves in reflection. It is mechanical,
unreflecting, consistently on-message --- the purest near-living
expression of data management to be found on Earth. It is, and does, and
is more. There is no ``I'' in a virus.

We are exactly its opposite: human, and everything that implies. Masters
of information, suckers for misinformation; slaves to emotion, ego and
wishful thinking. But also: inquiring, willful, optimistic. In our best
moments, we strive to learn, and to advance more than our individual
selves.

``The best thing to come out of this pandemic is that everyone has
become a virologist in some way,'' Dr. Ott said. She has a regular
trivia night with her family in Germany, over Zoom. Lately, the topic
has centered on viruses, and she has been impressed by how much they
know. ``There's so much more knowledge around,'' she said. ``A lot of
wrong info around, also. But people have become so literate, because we
all want it to go away.''

Dr. Sabeti agreed, up to a point. She expressed a deep curiosity about
viruses --- they are ``formidable opponents to understand'' --- but said
that, this time around, she found herself less interested in the purely
intellectual pursuit.

``For me right now, the place that I'm in, I really just most want to
stop this virus,'' she said. ``It's so frustrating and disappointing, to
say the least, to be in this position in which we have stopped the
world, in which we've created social distancing, in which we have
created mass amounts of human devastation and collateral damage because
we just weren't prepared.

``I don't care to understand it,'' she said. ``For me, it's. \ldots{} I
get up in the morning and my motivation is just: Stop this thing, and
figure out how to never have this happen again.''

\textbf{\emph{{[}}\href{http://on.fb.me/1paTQ1h}{\emph{Like the Science
Times page on Facebook.}}} ****** \emph{\textbar{} Sign up for the}
\textbf{\href{http://nyti.ms/1MbHaRU}{\emph{Science Times
newsletter.}}\emph{{]}}}

Advertisement

\protect\hyperlink{after-bottom}{Continue reading the main story}

\hypertarget{site-index}{%
\subsection{Site Index}\label{site-index}}

\hypertarget{site-information-navigation}{%
\subsection{Site Information
Navigation}\label{site-information-navigation}}

\begin{itemize}
\tightlist
\item
  \href{https://help.nytimes.com/hc/en-us/articles/115014792127-Copyright-notice}{©~2020~The
  New York Times Company}
\end{itemize}

\begin{itemize}
\tightlist
\item
  \href{https://www.nytco.com/}{NYTCo}
\item
  \href{https://help.nytimes.com/hc/en-us/articles/115015385887-Contact-Us}{Contact
  Us}
\item
  \href{https://www.nytco.com/careers/}{Work with us}
\item
  \href{https://nytmediakit.com/}{Advertise}
\item
  \href{http://www.tbrandstudio.com/}{T Brand Studio}
\item
  \href{https://www.nytimes.com/privacy/cookie-policy\#how-do-i-manage-trackers}{Your
  Ad Choices}
\item
  \href{https://www.nytimes.com/privacy}{Privacy}
\item
  \href{https://help.nytimes.com/hc/en-us/articles/115014893428-Terms-of-service}{Terms
  of Service}
\item
  \href{https://help.nytimes.com/hc/en-us/articles/115014893968-Terms-of-sale}{Terms
  of Sale}
\item
  \href{https://spiderbites.nytimes.com}{Site Map}
\item
  \href{https://help.nytimes.com/hc/en-us}{Help}
\item
  \href{https://www.nytimes.com/subscription?campaignId=37WXW}{Subscriptions}
\end{itemize}
