Sections

SEARCH

\protect\hyperlink{site-content}{Skip to
content}\protect\hyperlink{site-index}{Skip to site index}

\href{https://myaccount.nytimes.com/auth/login?response_type=cookie\&client_id=vi}{}

\href{https://www.nytimes.com/section/todayspaper}{Today's Paper}

\href{/section/opinion}{Opinion}\textbar{}America, We Break It, It's
Gone

\href{https://nyti.ms/2U41vjA}{https://nyti.ms/2U41vjA}

\begin{itemize}
\item
\item
\item
\item
\item
\item
\end{itemize}

Advertisement

\protect\hyperlink{after-top}{Continue reading the main story}

\href{/section/opinion}{Opinion}

Supported by

\protect\hyperlink{after-sponsor}{Continue reading the main story}

\hypertarget{america-we-break-it-its-gone}{%
\section{America, We Break It, It's
Gone}\label{america-we-break-it-its-gone}}

Where can we find the leadership to save the U.S.?

\href{https://www.nytimes.com/by/thomas-l-friedman}{\includegraphics{https://static01.nyt.com/images/2018/04/02/opinion/thomas-l-friedman/thomas-l-friedman-thumbLarge.png}}

By \href{https://www.nytimes.com/by/thomas-l-friedman}{Thomas L.
Friedman}

Opinion Columnist

\begin{itemize}
\item
  June 2, 2020
\item
  \begin{itemize}
  \item
  \item
  \item
  \item
  \item
  \item
  \end{itemize}
\end{itemize}

\includegraphics{https://static01.nyt.com/images/2020/06/02/opinion/02friedman1/merlin_173118384_fb527a6b-ce3d-4c04-ab11-ec674d09e8a7-articleLarge.jpg?quality=75\&auto=webp\&disable=upscale}

On Nov. 9, 2016, the morning after our last presidential election,
\href{https://www.nytimes.com/2016/11/09/opinion/were-near-the-breaking-point.html}{my
column began} by recalling words from an immigrant, my friend Lesley
Goldwasser, who came to America from Zimbabwe in the 1980s. Surveying
our political scene a few years earlier, Lesley had remarked to me:
``You Americans kick around your country like it's a football. But it's
not a football. It's a Fabergé egg. You can break it.'' I then added:
``With
\href{https://www.nytimes.com/2020/06/04/world/americas/trump-george-floyd.html}{Donald
Trump} now elected president, I have more fear than I've ever had in my
63 years that we could do just that --- break our country, that we could
become so irreparably divided that our national government will not
function.''

Well, I am now 66, and my fears have all come true --- and worse. I am
not at all certain we will be able to conduct a free and fair election
in November or have a peaceful transition of presidential power in
January. We are edging toward a cultural civil war, only this time we
are not lucky: Abraham Lincoln is not the president.

Lincoln, in our darkest, most divisive hour, was able to dig deep into
his soul and find the words ``with malice toward none, with charity for
all \ldots{} let us strive on to finish the work we are in'' and
establish ``a just and lasting peace among ourselves and with all
nations.''

Instead, we have Donald Trump, a man whose first instinct, when the
country is being ripped apart, was to have peaceful protesters
tear-gassed and shoved aside so that he could walk to a nearby church
just for a photo op outside holding a Bible. He did not open that Bible
to read a healing passage. He did not enter the church to host a healing
dialogue. He posed for a photo op to drive up his support among white
evangelicals. Trump was holding the Bible upside down.

What to do? Where can we find the leadership needed to calm this
situation, deal with its underlying causes and at least get us through
the 2020 election?

Three years ago, I might have hoped that Senate Republicans would step
in and restrain Trump. But now we all know better. The Senate Republican
caucus today is nothing but a political brothel. Mitch McConnell is the
madame. And McConnell and his caucus rent themselves out by the night to
whoever will energize the Republican base to keep them in power and
secure the economic benefits for their wealthiest donors.

Those energizers have been Sarah Palin, the Tea Party, coal companies,
industrial polluters and now Trump's most rabid supporters. It doesn't
matter who. The red light is always on above the door of the Senate
G.O.P. caucus room.

How about the social media barons? Will they save us from the toxic
waste they now circulate? Certainly not Facebook's Mark Zuckerberg, who
is clearly the Rupert Murdoch of his generation. He's always justifying
his cowardly choices with vacuous bromides about ``free speech,'' but
he's obviously just in it for the money --- no matter how much his
platform is used to destroy our democracy.

It is interesting to note that scientists tell us that people with the
coronavirus who are loud and obnoxious in a closed room are the biggest
super-spreaders of that pathogen. And internet experts tell us that
people who are loud and obnoxious online are the biggest super-spreaders
of political pathogens. That's because Facebook's whole business model
is to encourage and reward enragement because it drives more engagement.
Sorry, help is not on the way from Zuck.

So where to look? It is not hopeless. I hope America's principled
business leaders, and there are many, can find a way to come together to
lead a healing discussion, maybe through the Business Roundtable, in the
absence of a president willing and able to do so.

AT\&T Chairman Randall Stephenson eloquently called for exactly this
\href{https://www.cnbc.com/2020/06/02/atts-randall-stephenson-calls-on-fellow-ceos-to-speak-up-for-justice.html}{on
CNBC's ``Squawk Box''} on Tuesday morning. (AT\&T is a donor to Planet
Word, a museum to promote reading and literacy that my wife is building
in Washington, D.C.)

``All of us C.E.O.s have large African-American employee bodies,'' he
said. ``We owe it to them to make sure we're speaking to this and that
we're asking our policymakers to step up \ldots{} and just say it: `We
got a problem. We have a big problem. And it needs to be dealt with.'''

Stephenson added: ``This is about doing justice and making sure that
we're putting in place procedures \ldots{} to address what seem to be
constant and recurring injustices \ldots{} as it relates to interaction
of law enforcement'' with the black community.

How can business make an immediate difference? Obviously by empowering
politicians who want to address police reforms, but, just as important,
by amplifying local social entrepreneurs working in disadvantaged
neighborhoods to help their residents realize their full potential.

I am from Minneapolis. I was born in the Northside, a few miles from the
street where George Floyd was killed. No one there is doing more today
to make sure that disadvantaged families in that neighborhood have the
tools to succeed than my friend Sondra Samuels, the C.E.O. and president
**** of the \href{https://northsideachievement.org/}{Northside
Achievement Zone}. NAZ is working with parents, students and local
partners to drive a culture shift in predominantly black North
Minneapolis to end multigenerational poverty through education and
building family stability.

Sondra told me the right response to the killing of Floyd has to be
``both/and'' not ``either/or.'' We need both an immediate end to the
looting, burning and infiltration of white supremacist groups that is
destroying the homes and businesses of good people in cities all over
the country \emph{and} we need deeper civil rights, voting rights,
education, environmental and policing reforms for this generation.

NAZ and its 30 nonprofit partners and schools will tell you that it is a
struggle, often two steps forward and one step back, but they have been
making a quantifiable difference in getting kids the tools to succeed
and get to college, and at the same time providing parents the support
they need to stabilize their homes, increase their parenting skills and
ensure upward mobility.

NAZ has also significantly increased access to quality early-learning
opportunities for the 1,000 families it works with and helped provide
measurable improvements in reading proficiency and other learning
metrics critical for life success.

If you are depressed and want to do something that will have a lasting
impact for the common good --- not just denounce looters and scream at
Trump on your television screen --- check out the NAZ website \emph{and
hit the donate button!}

Finally, I think remarkable leadership is coming from some local
politicians --- so many great mayors of all colors and political
stripes. Every time I hear Atlanta Mayor Keisha Lance Bottoms speak ---
whether about dealing with the coronavirus, injustice or the rioting in
her town --- I want to ask Joe Biden: ``Are you interviewing her for
vice president?''

\includegraphics{https://static01.nyt.com/images/2020/06/02/opinion/02friedman2/merlin_164800188_38a6536e-2486-41b3-b696-dcdab453a9c4-articleLarge.jpg?quality=75\&auto=webp\&disable=upscale}

And I was really impressed how, to help quell the violence in Atlanta,
she enlisted the local rapper Killer Mike at her press conference, who
told the city:

``It is your duty not to burn your own house down for anger with an
enemy. It is your duty to fortify your own house so that you may be a
house of refuge in times of organization. Now is the time to plot, plan,
strategize, organize and mobilize. It is time to beat up prosecutors you
don't like at the voting booth. It is time to hold mayoral offices
accountable, chiefs and deputy chiefs. I'd like to appreciate our mayor
for talking to us like a black mama and telling us to take our ass home,
and I'd like to thank my friends for convincing me to come here.''

Help is not on the way from this White House or this G.O.P., but the
country is full of problem-solvers. We need to ignore Trump as much as
possible; he's made himself part of the problem. But we can connect,
elevate, amplify and empower the business leaders, social entrepreneurs
and local leaders who are rising and ready to be the solution.

\emph{The Times is committed to publishing}
\href{https://www.nytimes.com/2019/01/31/opinion/letters/letters-to-editor-new-york-times-women.html}{\emph{a
diversity of letters}} \emph{to the editor. We'd like to hear what you
think about this or any of our articles. Here are some}
\href{https://help.nytimes.com/hc/en-us/articles/115014925288-How-to-submit-a-letter-to-the-editor}{\emph{tips}}\emph{.
And here's our email:}
\href{mailto:letters@nytimes.com}{\emph{letters@nytimes.com}}\emph{.}

\emph{Follow The New York Times Opinion section on}
\href{https://www.facebook.com/nytopinion}{\emph{Facebook}}\emph{,}
\href{http://twitter.com/NYTOpinion}{\emph{Twitter (@NYTopinion)}}
\emph{and}
\href{https://www.instagram.com/nytopinion/}{\emph{Instagram}}\emph{.}

Advertisement

\protect\hyperlink{after-bottom}{Continue reading the main story}

\hypertarget{site-index}{%
\subsection{Site Index}\label{site-index}}

\hypertarget{site-information-navigation}{%
\subsection{Site Information
Navigation}\label{site-information-navigation}}

\begin{itemize}
\tightlist
\item
  \href{https://help.nytimes.com/hc/en-us/articles/115014792127-Copyright-notice}{©~2020~The
  New York Times Company}
\end{itemize}

\begin{itemize}
\tightlist
\item
  \href{https://www.nytco.com/}{NYTCo}
\item
  \href{https://help.nytimes.com/hc/en-us/articles/115015385887-Contact-Us}{Contact
  Us}
\item
  \href{https://www.nytco.com/careers/}{Work with us}
\item
  \href{https://nytmediakit.com/}{Advertise}
\item
  \href{http://www.tbrandstudio.com/}{T Brand Studio}
\item
  \href{https://www.nytimes.com/privacy/cookie-policy\#how-do-i-manage-trackers}{Your
  Ad Choices}
\item
  \href{https://www.nytimes.com/privacy}{Privacy}
\item
  \href{https://help.nytimes.com/hc/en-us/articles/115014893428-Terms-of-service}{Terms
  of Service}
\item
  \href{https://help.nytimes.com/hc/en-us/articles/115014893968-Terms-of-sale}{Terms
  of Sale}
\item
  \href{https://spiderbites.nytimes.com}{Site Map}
\item
  \href{https://help.nytimes.com/hc/en-us}{Help}
\item
  \href{https://www.nytimes.com/subscription?campaignId=37WXW}{Subscriptions}
\end{itemize}
