Sections

SEARCH

\protect\hyperlink{site-content}{Skip to
content}\protect\hyperlink{site-index}{Skip to site index}

\href{https://www.nytimes.com/section/us}{U.S.}

\href{https://myaccount.nytimes.com/auth/login?response_type=cookie\&client_id=vi}{}

\href{https://www.nytimes.com/section/todayspaper}{Today's Paper}

\href{/section/us}{U.S.}\textbar{}Protests Draw Shoulder-to-Shoulder
Crowds After Months of Virus Isolation

\url{https://nyti.ms/2U6VHWF}

\begin{itemize}
\item
\item
\item
\item
\item
\item
\end{itemize}

\href{https://www.nytimes.com/news-event/george-floyd-protests-minneapolis-new-york-los-angeles?action=click\&pgtype=Article\&state=default\&region=TOP_BANNER\&context=storylines_menu}{Race
and America}

\begin{itemize}
\tightlist
\item
  \href{https://www.nytimes.com/2020/07/26/us/protests-portland-seattle-trump.html?action=click\&pgtype=Article\&state=default\&region=TOP_BANNER\&context=storylines_menu}{Protesters
  Return to Other Cities}
\item
  \href{https://www.nytimes.com/2020/07/24/us/portland-oregon-protests-white-race.html?action=click\&pgtype=Article\&state=default\&region=TOP_BANNER\&context=storylines_menu}{Portland
  at the Center}
\item
  \href{https://www.nytimes.com/2020/07/23/podcasts/the-daily/portland-protests.html?action=click\&pgtype=Article\&state=default\&region=TOP_BANNER\&context=storylines_menu}{Podcast:
  Showdown in Portland}
\item
  \href{https://www.nytimes.com/interactive/2020/07/16/us/black-lives-matter-protests-louisville-breonna-taylor.html?action=click\&pgtype=Article\&state=default\&region=TOP_BANNER\&context=storylines_menu}{45
  Days in Louisville}
\end{itemize}

Advertisement

\protect\hyperlink{after-top}{Continue reading the main story}

Supported by

\protect\hyperlink{after-sponsor}{Continue reading the main story}

\hypertarget{protests-draw-shoulder-to-shoulder-crowds-after-months-of-virus-isolation}{%
\section{Protests Draw Shoulder-to-Shoulder Crowds After Months of Virus
Isolation}\label{protests-draw-shoulder-to-shoulder-crowds-after-months-of-virus-isolation}}

Much of the country stayed inside, separated, as a way to slow the
spread of the coronavirus. Now protests are creating crowds, threatening
a resurgence.

\includegraphics{https://static01.nyt.com/images/2020/06/02/us/02UNREST-VIRUS-mlps/merlin_173084217_dd54b766-ca55-40df-b1b9-000c78fc59a0-articleLarge.jpg?quality=75\&auto=webp\&disable=upscale}

\href{https://www.nytimes.com/by/julie-bosman}{\includegraphics{https://static01.nyt.com/images/2018/11/09/multimedia/author-julie-bosman/author-julie-bosman-thumbLarge.png}}\href{https://www.nytimes.com/by/amy-harmon}{\includegraphics{https://static01.nyt.com/images/2020/04/29/reader-center/author-amy-harmon/author-amy-harmon-thumbLarge-v2.png}}

By \href{https://www.nytimes.com/by/julie-bosman}{Julie Bosman} and
\href{https://www.nytimes.com/by/amy-harmon}{Amy Harmon}

\begin{itemize}
\item
  Published June 2, 2020Updated July 1, 2020
\item
  \begin{itemize}
  \item
  \item
  \item
  \item
  \item
  \item
  \end{itemize}
\end{itemize}

CHICAGO --- For days, Kate Dixon has been watching the videos of
demonstrations from her home in a Denver suburb: the images of young
people packed shoulder to shoulder, the crowds shouting in unison on
downtown streets, the occasional détente between protester and police
officer that ends in a hug.

``You want that to be a wonderful moment,'' said Ms. Dixon, a
stay-at-home mother who has been sewing face masks in her spare time.
``But your heart just hurts at all the illness this could be causing.''

In the last week, the United States has abruptly shifted from one
crippling crisis to the next. Most Americans had been under stay-at-home
orders for months to slow the coronavirus pandemic, restrictions that
were gradually eased throughout May, freeing people in many states to
begin venturing back into shops and restaurants.

Then came Memorial Day in Minneapolis, when George Floyd, a black man,
died after pleading that he could not breathe as a white police officer
pressed his knee into Mr. Floyd's neck. His death has prompted cascading
\href{https://www.nytimes.com/2020/06/03/us/confederate-statues-george-floyd.html}{protests}
in hundreds of cities, where demonstrators have called for an end to
police brutality and racist institutions.

Suddenly America no longer looks like a nation cooped up at home.

The
\href{https://www.nytimes.com/2020/07/01/nyregion/nyc-coronavirus-protests.html}{demonstrations
have spurred fears that they could cause a deadly resurgence of the
coronavirus}. And for those sympathetic to a growing movement, deciding
whether to attend protests has been complicated: Some people have
avoided them entirely, reasoning that the chance of contracting the
coronavirus in a crowd is too high. Others have joined despite the
risks.

``The police violence against black people --- that's a pandemic, too,''
said Kelli Ann Thomas, a community organizer who joined protests in
Miami. ``People are willing to risk their lives, to risk their health,
to show solidarity with black people.''

No one has studied the precise dynamics of how the virus may be
transmitted under the mix of conditions that prevail at mass protests.
And because of delays between exposure to the virus and the start of
symptoms, and then hospitalizations and deaths, the impact of the
protests on virus spread will not be known for several weeks.

Health experts know that the virus is far less likely to be spread
outdoors than indoors. And masks reduce the chance of infected people
transmitting the respiratory droplets that contain the virus.

But many uncertainties remain. Yelling, shouting and singing can
increase how far those droplets are projected. Crowds and the length of
time an uninfected person is near someone who is infected also increase
the risk of transmission.

Protests have revealed a range of precautions, from people wearing
tightly secured masks to others with no face covering. On Monday, a
nurse's assistant who works at a nursing home was among the protesters
at the site where Mr. Floyd died in Minneapolis, showing up in a mask
and scrubs.

Contagion was weighing on the mind of Jamie Schwesnedl in the first days
of protests. Mr. Schwesnedl owns Moon Palace Books, down the street from
a police station that was set on fire last weekend after demonstrators
clashed with the police.

Black people were already suffering from a disproportionate number of
coronavirus infections; now many members of their communities are
protesting the death of Mr. Floyd.

``The idea of getting a whole lot of people together to yell, which we
know is one of the most effective ways to transmit the virus, and having
that happen around a lot of people of color and neighborhoods and
communities of color, is very stressful,'' said Mr. Schwesnedl, who is
white. ``It just adds this whole stress of how this is going to impact
the infection rates.''

In Denver, Tay Anderson, a protest leader and school board member, has
been worried about the disparate impact of the coronavirus on black
Colorado residents like himself. Thousands of people marched through
Denver and laid down, shoulder to shoulder, on the lawn of the Capitol
in silent demonstrations.

He put out a call on social media for all protesters to join him in
getting tested for the coronavirus on Saturday morning at the Pepsi
Center, a sports and concert arena where Denver has been running free,
large-scale testing.

``WE ARE STILL IN A PANDEMIC,'' he wrote on Twitter.

In Los Angeles, city leaders have expressed alarm that the protests
could fuel the spread of the coronavirus, while at the same time voicing
support for the rights of demonstrators to gather and saying that they
shared in the outrage over the death of Mr. Floyd.

Mayor Eric Garcetti warned that the gatherings could become
``superspreader events'' --- not unlike during the 1918 flu pandemic
when, after the first wave of infections, some cities began holding
parades and large gatherings that led to a second, more deadly wave.

Some of the city's virus testing sites have had to be shut down briefly,
and while jail populations had been reduced over concerns about the
spread of the virus in tightly controlled institutions, they now are
filling again after mass arrests.

Though the pandemic has
\href{https://www.nytimes.com/2020/06/01/us/coronavirus-united-states.html}{slowed
somewhat in recent weeks}, the virus continues to infect thousands and
kill hundreds every day. Many of the country's big cities have been
adding several hundred new cases every day. On Monday, nearly 1,000 new
cases were identified in Los Angeles County, Calif., more than 500 in
Cook County, Ill., and more than 400 in New York City. In the county
that includes Minneapolis, more than 1,000 new cases have been
identified over the past week and officials have urged protesters to
seek coronavirus testing.

Epidemiologists said the protests would almost certainly lead to more
cases of the virus. Large gatherings of people are known to have led to
chains of transmission in other settings. And police tactics such as
spraying tear gas, which causes people to cough; herding protesters into
smaller areas for crowd control; and placing arrested individuals in
buses, vans and holding cells also increase the risk of infection.

\href{https://reichlab.io/covid19-forecast-hub/}{An aggregate model}
assembled by researchers at the University of Massachusetts projects
that the nation will see between 5,000 to 7,000 deaths from Covid-19
each week over the next month.

But public health experts emphasized that police violence against black
people in America also represents a public health crisis. The anger over
economic, social and health disparities fueling the protests, health
experts said, are reflected in sharply higher rates of
coronavirus-related death and illness among black Americans. Several
counseled a ``harm reduction'' approach that would allow people to join
the demonstrations as safely as possible.

``Last week, all the news was about Covid; this week, all the news is
about the protests,'' said Eleanor Murray, an assistant professor of
epidemiology at Boston University School of Public Health. ``But really,
these are two pieces of the same conversation.''

Yolanda Williams, the host of the podcast ``Parenting Decolonized,"
which focuses on parenting black children, said she has not attended
protests in Little Rock, Ark., where she lives. She is a single mother
of a toddler, and her area has seen a spike in Covid-19 cases.

But she had a message on her Facebook page for those among her followers
who are white: You should be showing up.

``I know it's scary, but if you are committed to dismantling white
supremacy, it's you that needs to be out there, en masse, protesting as
loud for civil rights as you did for the Women's March,'' she said in
the post.

\includegraphics{https://static01.nyt.com/images/2020/06/02/us/02UNREST-VIRUS-sf/merlin_173097513_714197d1-e806-4ecd-966f-cd155468eb17-articleLarge.jpg?quality=75\&auto=webp\&disable=upscale}

In an interview on Tuesday, Ms. Williams said she was trying to drive
home the point that it is far riskier for black people to attend
protests than it is for white people.

``The problem is, we are having to choose from either dying from Covid
or dying from cops,'' she said. ``Put yourself on the line like we are
doing every day. Put your body on the line.''

For many public health experts who have spent weeks advising
policymakers and the public on how to reduce their risk of getting or
inadvertently spreading the coronavirus, the mass demonstrations have
forced a shift in perspective.

Tiffany Rodriguez, an epidemiologist who has rarely left her home since
mid-March, said it was often impossible to maintain the recommended six
feet of distance at the protest she attended in Boston on Sunday. But
``understanding that police brutality is a public health epidemic,'' she
said she felt compelled to go.

Julie Bosman reported from Chicago, and Amy Harmon from New York.
Reporting was contributed by Patricia Mazzei from Miami, Jack Healy from
Denver, Tim Arango from Los Angeles, Dionne Searcey from Minneapolis and
Mitch Smith from Overland Park, Kan.

Advertisement

\protect\hyperlink{after-bottom}{Continue reading the main story}

\hypertarget{site-index}{%
\subsection{Site Index}\label{site-index}}

\hypertarget{site-information-navigation}{%
\subsection{Site Information
Navigation}\label{site-information-navigation}}

\begin{itemize}
\tightlist
\item
  \href{https://help.nytimes.com/hc/en-us/articles/115014792127-Copyright-notice}{©~2020~The
  New York Times Company}
\end{itemize}

\begin{itemize}
\tightlist
\item
  \href{https://www.nytco.com/}{NYTCo}
\item
  \href{https://help.nytimes.com/hc/en-us/articles/115015385887-Contact-Us}{Contact
  Us}
\item
  \href{https://www.nytco.com/careers/}{Work with us}
\item
  \href{https://nytmediakit.com/}{Advertise}
\item
  \href{http://www.tbrandstudio.com/}{T Brand Studio}
\item
  \href{https://www.nytimes.com/privacy/cookie-policy\#how-do-i-manage-trackers}{Your
  Ad Choices}
\item
  \href{https://www.nytimes.com/privacy}{Privacy}
\item
  \href{https://help.nytimes.com/hc/en-us/articles/115014893428-Terms-of-service}{Terms
  of Service}
\item
  \href{https://help.nytimes.com/hc/en-us/articles/115014893968-Terms-of-sale}{Terms
  of Sale}
\item
  \href{https://spiderbites.nytimes.com}{Site Map}
\item
  \href{https://help.nytimes.com/hc/en-us}{Help}
\item
  \href{https://www.nytimes.com/subscription?campaignId=37WXW}{Subscriptions}
\end{itemize}
