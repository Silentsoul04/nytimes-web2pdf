Sections

SEARCH

\protect\hyperlink{site-content}{Skip to
content}\protect\hyperlink{site-index}{Skip to site index}

\href{/section/politics}{Politics}\textbar{}How Trump's Idea for a Photo
Op Led to Havoc in a Park

\url{https://nyti.ms/2U5gcTp}

\begin{itemize}
\item
\item
\item
\item
\item
\item
\end{itemize}

\href{https://www.nytimes.com/news-event/george-floyd-protests-minneapolis-new-york-los-angeles?action=click\&pgtype=Article\&state=default\&region=TOP_BANNER\&context=storylines_menu}{Race
and America}

\begin{itemize}
\tightlist
\item
  \href{https://www.nytimes.com/2020/07/26/us/protests-portland-seattle-trump.html?action=click\&pgtype=Article\&state=default\&region=TOP_BANNER\&context=storylines_menu}{Protesters
  Return to Other Cities}
\item
  \href{https://www.nytimes.com/2020/07/24/us/portland-oregon-protests-white-race.html?action=click\&pgtype=Article\&state=default\&region=TOP_BANNER\&context=storylines_menu}{Portland
  at the Center}
\item
  \href{https://www.nytimes.com/2020/07/23/podcasts/the-daily/portland-protests.html?action=click\&pgtype=Article\&state=default\&region=TOP_BANNER\&context=storylines_menu}{Podcast:
  Showdown in Portland}
\item
  \href{https://www.nytimes.com/interactive/2020/07/16/us/black-lives-matter-protests-louisville-breonna-taylor.html?action=click\&pgtype=Article\&state=default\&region=TOP_BANNER\&context=storylines_menu}{45
  Days in Louisville}
\end{itemize}

\includegraphics{https://static01.nyt.com/images/2020/06/02/us/politics/02dc-unrest-tictoc-1/merlin_173090628_9c9a1070-0e82-48f4-9bde-d3cdc874e370-videoSixteenByNine3000.jpg}

\hypertarget{how-trumps-idea-for-a-photo-op-led-to-havoc-in-a-park}{%
\section{How Trump's Idea for a Photo Op Led to Havoc in a
Park}\label{how-trumps-idea-for-a-photo-op-led-to-havoc-in-a-park}}

When the history of the Trump presidency is written, the clash with
protesters that preceded President Trump's walk across Lafayette Square
may be remembered as one of its defining moments.

Credit...

Supported by

\protect\hyperlink{after-sponsor}{Continue reading the main story}

By \href{https://www.nytimes.com/by/peter-baker}{Peter Baker},
\href{https://www.nytimes.com/by/maggie-haberman}{Maggie Haberman},
\href{https://www.nytimes.com/by/katie-rogers}{Katie Rogers},
\href{https://www.nytimes.com/by/zolan-kanno-youngs}{Zolan Kanno-Youngs}
and \href{https://www.nytimes.com/by/katie-benner}{Katie Benner}

Videos by \href{https://www.nytimes.com/by/haley-willis}{Haley Willis},
\href{http://nytimes.com/by/christiaan-triebert}{Christiaan Triebert}
and David Botti

\begin{itemize}
\item
  Published June 2, 2020Updated July 28, 2020
\item
  \begin{itemize}
  \item
  \item
  \item
  \item
  \item
  \item
  \end{itemize}
\end{itemize}

\href{https://cn.nytimes.com/usa/20200603/trump-walk-lafayette-square/}{阅读简体中文版}\href{https://cn.nytimes.com/usa/20200603/trump-walk-lafayette-square/zh-hant/}{閱讀繁體中文版}\href{https://www.nytimes.com/es/2020/06/03/espanol/mundo/trump-foto-iglesia-protestas.html}{Leer
en español}

WASHINGTON --- After a weekend of protests that led all the way to his
own front yard and forced him to
\href{https://www.nytimes.com/2020/06/03/us/politics/trump-protests.html}{briefly
retreat to a bunker beneath the White House}, President Trump arrived in
the Oval Office on Monday agitated over the television images, annoyed
that anyone would think he was hiding and eager for action.

He wanted to send the military into American cities, an idea that
provoked a heated, voices-raised fight among his advisers. But by the
end of the day, urged on by his daughter Ivanka Trump, he came up with a
more personal way of demonstrating toughness --- he would march across
\href{https://www.nytimes.com/2020/07/28/us/politics/lafayette-square-park-police-protests.html}{Lafayette
Square} to a church damaged by fire the night before.

The only problem: A plan developed earlier in the day to expand the
security perimeter around the White House had not been carried out. When
Attorney General William P. Barr strode out of the White House gates for
a personal inspection early Monday evening, he discovered that
protesters were still on the northern edge of the square. For the
president to make it to St. John's Church, they would have to be cleared
out. Mr. Barr gave the order to disperse them.

What ensued was a burst of violence unlike any seen in the shadow of the
White House in generations. As he prepared for his surprise march to the
church, Mr. Trump first went before cameras in the Rose Garden to
declare himself ``your president of law and order'' but also ``an ally
of all peaceful protesters,'' even as peaceful protesters just a block
away and clergy members on the church patio were routed by smoke and
flash grenades and some form of chemical spray deployed by
shield-bearing riot officers and mounted police.

\includegraphics{https://static01.nyt.com/images/2020/02/06/us/02-vid-dcclip6-image/02-vid-dcclip6-image-videoSixteenByNineJumbo1600.jpg}

After a day in which he
\href{https://www.nytimes.com/2020/06/01/us/politics/trump-governors.html}{berated
``weak'' governors} and lectured them to ``dominate'' the demonstrators,
the president emerged from the White House, followed by a phalanx of
aides and Secret Service agents as he made his way to the church, where
he posed stern-faced, holding up a Bible that his daughter pulled out of
her \$1,540 MaxMara bag.

The resulting photographs of Mr. Trump striding purposefully across the
square satisfied his long-held desire to project strength, images that
members of his re-election campaign team quickly began recirculating and
pinning to their Twitter home pages once he was safely back in the
fortified White House.

The scene of mayhem that preceded the walk --- barely 1,000 feet from
the symbol of American democracy --- evoked images more commonly
associated with authoritarian countries, but that did not bother the
president, who has long flirted with overseas strongmen and has
expressed envy of their ability to dominate.

Throughout his time in office, Mr. Trump has generated concern over what
critics see as his autocratic instincts, including his claims to
untrammeled power to ``do whatever I want,'' his attacks on
quasi-autonomous institutions of government like the F.B.I. or
inspectors general and his efforts to discredit independent sources of
information that anger him, like the news media he denounces as the
``enemy of the people.''

And when the history of the Trump presidency is written, the clash at
Lafayette Square may be remembered as one of its defining moments.

Mr. Trump and his inner circle considered it a triumph that would
resonate with many middle Americans turned off by scenes of urban riots
and looting that have accompanied nonviolent protests of the
\href{https://www.nytimes.com/2020/05/31/us/george-floyd-investigation.html}{police
killing of a subdued black man in Minneapolis}.

But critics, including some fellow Republicans, were aghast at the use
of force against Americans who posed no visible threat at the time, all
to facilitate what they deemed a ham-handed photo opportunity featuring
all white faces. Some Democratic senators used words like
\href{https://twitter.com/RonWyden/status/1267605801549664256}{``fascist''}
and
\href{https://twitter.com/KamalaHarris/status/1267603100656898049}{``dictator''}
to describe the president's words and actions.

Bishop Mariann Edgar Budde of the Episcopal Diocese of Washington, who
was not consulted beforehand,
\href{https://www.nytimes.com/2020/06/02/us/politics/trump-church.html}{said
she was ``outraged''} over the use of one of her churches as a political
backdrop to boast of squelching protests against racism. Even some White
House officials privately expressed dismay that the president's
entourage had not thought to include a single person of color.

Mayor Muriel E. Bowser of Washington sharply objected on Tuesday and
said the federal government had even privately broached the idea of
taking over the city's police force, which she pledged to resist. ``I
don't think the military should be used in the streets of American
cities against Americans,'' she said, ``and I definitely don't think it
should be done for a show.''

Arlington County in suburban Virginia withdrew its police from those
assembled to guard the White House and other federal sites after the
Lafayette Square clash. Even beforehand, Democratic governors in
Virginia, New York and Delaware refused to send National Guard troops
requested by the Trump administration.

\includegraphics{https://static01.nyt.com/images/2020/02/06/us/02-vid-dcclip3a-image/02-vid-dcclip3a-image-videoSixteenByNineJumbo1600.jpg}

The spectacle staged by the White House also left military leaders
struggling to explain themselves in response to criticism from retired
officers that they had allowed themselves to be used as political props.
Defense Secretary Mark T. Esper and Gen. Mark A. Milley, the chairman of
the Joint Chiefs of Staff, put out word through military officials that
they did not know in advance about the dispersal of the protesters or
about the president's planned photo op, insisting that they thought they
were accompanying him to review the troops.

The police action cleared the way for the photo op, but it hardly
quelled the anger in the streets. By Tuesday afternoon, demonstrators
had returned to the edge of Lafayette Square --- where new tall fences
had been erected overnight --- and shouted their discontent at the line
of black-clad officers.

``Take off the riot gear, I don't see no riot here,'' they chanted.

Aides on Tuesday defended Mr. Trump's walk to the church, given that a
small fire had been set in its basement during demonstrations over the
weekend. ``The president very much felt when he saw those images on
Sunday night --- that crossed a terrible line, that goes way beyond
peaceful protesting,'' Kellyanne Conway, his counselor, told reporters.

But she distanced him from the decisions on how to disperse the crowd.
``Clearly, the president doesn't know how law enforcement is handling
his movement,'' she said.

This account of the clash is based on descriptions by reporters at the
scene, interviews with dozens of protesters, White House aides, law
enforcement officials, city leaders and others involved in the tense day
as well as an analysis of video footage from The New York Times's visual
investigations team.

\hypertarget{morning-at-the-white-house}{%
\subsection{Morning at the White
House}\label{morning-at-the-white-house}}

Mr. Trump was stirred up on Monday morning as he met with national
security and law enforcement advisers to discuss what could be done
about the street unrest. The advisers told him that he could not let the
nation's capital be overrun, that the symbolism was too important and
that he had to get it under control that night.

Among the ideas put on the table was invoking the Insurrection Act, a
two-century-old law that would enable the president to send in
active-duty military to quell disturbances over the objections of
governors. The act has long been controversial. President George Bush
invoked it in 1992 to respond to the Rodney King riots only at the
request of California. But in the civil rights era, presidents sent in
troops to enforce desegregation over the resistance of racist governors.

Its use is so charged that President George W. Bush hesitated to invoke
it to respond to Hurricane Katrina for fear of looking like he was
overriding local and state leaders.

\includegraphics{https://static01.nyt.com/images/2020/06/02/us/politics/02dc-unrest-tictoc-protest/merlin_173083536_3794a6fb-0349-48ee-91bf-bd228b847a36-articleLarge.jpg?quality=75\&auto=webp\&disable=upscale}

Vice President Mike Pence favored the idea, reasoning that it would
allow quicker action than calling up National Guard units, and he was
backed by Mr. Esper. But Mr. Barr and General Milley warned against it.
The attorney general cited concerns about states' rights, while General
Milley assured the president that he had enough force already in the
nation's capital to secure the city and expressed worry about putting
active-duty soldiers in such a role.

Several officials came away with different impressions of where Mark
Meadows, the White House chief of staff, stood on the issue, but the
discussion grew increasingly heated as voices were raised and tensions
escalated.

Mr. Trump and Mr. Pence then conducted a conference call with the
nation's governors in which the president berated them for being
``weak'' and ``fools,'' advising them to ``dominate'' the demonstrators.
Mr. Esper talked about controlling ``the battlespace.''

The president rhapsodized about the crackdown in Minneapolis once the
National Guard moved in. ``It's a beautiful thing to watch,'' he said.
``It just can't be any better. There's no experiment needed. You don't
have to do tests.''

\hypertarget{midday-planning}{%
\subsection{Midday Planning}\label{midday-planning}}

In Washington, Mr. Barr was in charge of the federal response and an
alphabet soup of agencies had contributed officers, agents and troops to
defend the White House and other federal installations, including the
Secret Service, the United States Park Police, National Guard, Capitol
Police, the Bureau of Alcohol, Tobacco, Firearms and Explosives, the
Marshal's Service, the Bureau of Prisons, Customs and Border Protection
and Immigration and Customs Enforcement.

\includegraphics{https://static01.nyt.com/images/2020/02/06/us/02-vid-dcclip2-image/02-vid-dcclip2-image-videoSixteenByNineJumbo1600.jpg}

Mr. Barr was concerned about demonstrations near the White House over
the weekend that had resulted in a small basement fire at St. John's and
graffiti on the Treasury Department headquarters, so he resolved to push
the security perimeter farther from the mansion.

Reinforcements were summoned. Just before noon, an alert went out to
every Washington-area agent with Homeland Security Investigations, a
division of ICE, telling them to prepare to assist with any
demonstration, according to an email labeled with a ``high'' severity.
The F.B.I. deployed its elite hostage rescue team, highly armed and
trained agents more accustomed to arresting dangerous suspects than
dealing with riots. And ICE deployed its ``special response teams'' to
protect agency facilities and be on call for more.

But others were reluctant to help. Mr. Trump was so aggressive on the
call with governors that when Gov. Ralph Northam of Virginia received a
request to send up to 5,000 of his state's National Guard troops, he
grew concerned. His staff contacted Ms. Bowser's office and discovered
that the mayor had not even been notified of the request. At that point,
Mr. Northam turned the White House down. Similarly, Gov. Andrew Cuomo of
New York called off buses of National Guard troops that were to head to
Washington.

By midafternoon on Monday, protesters had gathered again on H Street at
the north side of Lafayette Square, this time peacefully. The Rev. Gini
Gerbasi, the rector of St. John's Church in Georgetown and a former
assistant rector at St. John's, arrived around 4 p.m. with cases of
water for the demonstrators. Joining her on the church patio were about
20 clergy members who passed out snacks.

\includegraphics{https://static01.nyt.com/images/2020/02/06/us/02-vid-dcclip1-image/02-vid-dcclip1-image-videoSixteenByNineJumbo1600.jpg}

Next to them on the patio, a group affiliated with Black Lives Matter
mixed water and soap in squeeze bottles as emergency eye wash if
protesters were tear-gassed by the police.

While there were occasionally some aggressive encounters with the
police, Ms. Gerbasi said, it was largely calm. ``There were a few tense
moments,'' she said. ``But it was peaceful.''

Inside the White House nearby, Mr. Trump was coming up with his plan to
walk to the church. Several administration officials said it was his own
idea; two officials said that during a senior staff meeting on Tuesday,
Mr. Meadows credited the president's daughter. It was crafted during an
Oval Office meeting the day before that included Ms. Trump; Mr. Meadows;
Jared Kushner, the president's son-in-law and senior adviser; and Hope
Hicks, another top adviser.

At some point, Anthony Ornato, a Secret Service veteran who serves as
deputy chief of staff for operations, was brought in to coordinate the
logistics of the visit. Ms. Hicks came up with the visuals for how it
would look. But officials privately conceded that little thought was
given to what Mr. Trump would do once he actually got to the church.
There was some discussion of going inside, but it was boarded up.

The president and his team decided he would first make a statement in
the Rose Garden in which he would express sympathy for the family of
George Floyd, the black man who died in Minneapolis when a police
officer kneeled on his neck for nearly nine minutes, but then he would
take a strong stance in favor of reclaiming the streets. He would
threaten to invoke the Insurrection Act if governors and mayors did not
do a better job of security. Reporters were told a statement would be
coming, but the march to the church was kept a secret.

Mr. Barr made a trip out of the White House and into Lafayette Square
only to find that the plan to expand the security perimeter had not been
carried out. He ordered the law enforcement officers on the ground to
complete the expansion, which would mean dispersing protesters, but
there was not enough time to do so before the president's planned
statement.

Image

Attorney General William P. Barr on Monday outside the White House. He
gave the order to disperse the demonstrators.Credit...Alex
Brandon/Associated Press

\hypertarget{before-the-clash}{%
\subsection{Before the Clash}\label{before-the-clash}}

At 5:07 p.m., National Guard trucks loaded with troops headed north on
West Executive Avenue, a lane on the White House compound between the
West Wing and the Eisenhower Executive Office Building, and drove past
the visitors' entrance, went out the gates and turned right onto
Pennsylvania Avenue.

Shortly after, two members of the Secret Service counterassault team
appeared on the roof of the West Wing with guns and binoculars, peering
north toward Lafayette Square. While snipers are stationed on the main
roof of the White House from time to time, they are not usually deployed
on top of the West Wing, and the sight was jarring for regulars at the
building.

The White House press corps was summoned to the Rose Garden at 6:03 p.m.
Outside the gates and across Lafayette Square, some of the officers in
riot gear kneeled down and some protesters initially thought they were
expressing solidarity as the police have done in other cities, but in
fact they were putting on their gas masks.

At 6:17 p.m., a large phalanx of officers wearing Secret Service
uniforms began advancing on protesters, climbing or jumping over
barriers at the edge of the square at H Street and Madison Place.
Officials said later that the police warned protesters to disperse three
times, but if they did, reporters on the scene as well as many
demonstrators did not hear it.

\includegraphics{https://static01.nyt.com/images/2020/06/02/autossell/02-vid-dcclip3b-cover/02-vid-dcclip3b-cover-videoSixteenByNineJumbo1600.png}

Some form of chemical agent was fired at protesters, flash bang grenades
went off and mounted police moved toward the crowds. ``People were
dropping to the ground'' at the sound of bangs and pops that sounded
like gunfire, Ms. Gerbasi said. ``We started seeing and smelling tear
gas, and people were running at us.''

By 6:30 p.m., she said, ``Suddenly the police were on the patio of St.
John's Church in a line, literally pushing and shoving people off of the
patio.''

Julia Dominick, a seminarian with the Virginia Theological Seminary in
Alexandria, Va., and a former emergency room nurse, was tending to a
hurt protester when a police line advanced.

``There was not a warning,'' she said. ``I've never been in a war. I've
never been shot at. I've never been afraid in that way. Those sounds and
the gas, it will be with me.'' (No police agency acknowledged using tear
gas, but reporters and protesters on the scene said there was clearly a
chemical irritant of some kind.)

At 6:43 p.m., Mr. Trump made his statement in the Rose Garden, finishing
seven minutes later, and then headed back through the White House to
emerge on the north side and walk out the gates and into the park. Mr.
Barr, Mr. Esper, General Milley, Mr. Meadows, Ms. Trump, Mr. Kushner and
others followed him, but Mr. Pence and his staff hung back as the
building emptied and watched on television instead.

The president's movement surprised nearly everyone, as he intended,
including law enforcement. The Washington police chief said he was
notified only moments beforehand. Park Police commanders on the scene
were as surprised as everyone else to see the president in the park.

\includegraphics{https://static01.nyt.com/images/2020/06/02/video/02-vid-dcclip4-COVER/02-vid-dcclip4-COVER-videoSixteenByNineJumbo1600.png}

When he reached St. John's, Mr. Trump made no pretense of any intent
other than posing for photographs --- he held up the Bible carried by
his daughter, then gathered a few top advisers next to him in a line. He
made no formal remarks and then, having accomplished his purpose, headed
back to the White House, passing in front of a wall with new graffiti
saying, ``Fuck Trump.''

The police and other forces pursued demonstrators around the capital the
rest of the evening, with military helicopters even swooping low
overhead in what were called shows of force. Mr. Barr and General Milley
at different points roamed the streets.

\includegraphics{https://static01.nyt.com/images/2020/06/02/video/02-vid-dcclip5-COVER/02-vid-dcclip5-COVER-videoSixteenByNineJumbo1600.png}

By Tuesday morning, Mr. Trump boasted of success. ``D.C. had no problems
last night,''
\href{https://twitter.com/realDonaldTrump/status/1267808120136511489}{he
wrote on Twitter}. ``Many arrests. Great job done by all. Overwhelming
force. Domination. Likewise, Minneapolis was great (thank you President
Trump!).''

By Tuesday afternoon, the crowds were back and even bigger.

Peter Baker, Katie Rogers, Zolan Kanno-Youngs and Katie Benner reported
from Washington, and Maggie Haberman from New York. Reporting was
contributed by Helene Cooper, Thomas Gibbons-Neff, Annie Daniel, Annie
Karni, Jonathan Martin, Douglas Mills, Eric Schmitt, Erin Schaff and
Jennifer Steinhauer from Washington.

Video sources: Aaron Fenster, via Storyful; Ben Warren; Agencia EFE, via
Associated Press; U.S. Network Pool, via Reuters; Scott Thuman; U.S.
Pool via Reuters; Google Earth; and ADS-B Exchange.

Advertisement

\protect\hyperlink{after-bottom}{Continue reading the main story}

\hypertarget{site-index}{%
\subsection{Site Index}\label{site-index}}

\hypertarget{site-information-navigation}{%
\subsection{Site Information
Navigation}\label{site-information-navigation}}

\begin{itemize}
\tightlist
\item
  \href{https://help.nytimes.com/hc/en-us/articles/115014792127-Copyright-notice}{©~2020~The
  New York Times Company}
\end{itemize}

\begin{itemize}
\tightlist
\item
  \href{https://www.nytco.com/}{NYTCo}
\item
  \href{https://help.nytimes.com/hc/en-us/articles/115015385887-Contact-Us}{Contact
  Us}
\item
  \href{https://www.nytco.com/careers/}{Work with us}
\item
  \href{https://nytmediakit.com/}{Advertise}
\item
  \href{http://www.tbrandstudio.com/}{T Brand Studio}
\item
  \href{https://www.nytimes.com/privacy/cookie-policy\#how-do-i-manage-trackers}{Your
  Ad Choices}
\item
  \href{https://www.nytimes.com/privacy}{Privacy}
\item
  \href{https://help.nytimes.com/hc/en-us/articles/115014893428-Terms-of-service}{Terms
  of Service}
\item
  \href{https://help.nytimes.com/hc/en-us/articles/115014893968-Terms-of-sale}{Terms
  of Sale}
\item
  \href{https://spiderbites.nytimes.com}{Site Map}
\item
  \href{https://help.nytimes.com/hc/en-us}{Help}
\item
  \href{https://www.nytimes.com/subscription?campaignId=37WXW}{Subscriptions}
\end{itemize}
