Sections

SEARCH

\protect\hyperlink{site-content}{Skip to
content}\protect\hyperlink{site-index}{Skip to site index}

\href{https://www.nytimes.com/section/technology}{Technology}

\href{https://myaccount.nytimes.com/auth/login?response_type=cookie\&client_id=vi}{}

\href{https://www.nytimes.com/section/todayspaper}{Today's Paper}

\href{/section/technology}{Technology}\textbar{}TikTok (Yes, TikTok) Is
the Future

\url{https://nyti.ms/2z55XYc}

\begin{itemize}
\item
\item
\item
\item
\item
\end{itemize}

Advertisement

\protect\hyperlink{after-top}{Continue reading the main story}

Supported by

\protect\hyperlink{after-sponsor}{Continue reading the main story}

on tech

\hypertarget{tiktok-yes-tiktok-is-the-future}{%
\section{TikTok (Yes, TikTok) Is the
Future}\label{tiktok-yes-tiktok-is-the-future}}

Whether serious or silly, TikTok is an outlet for expression unlike
anything that came before.

\includegraphics{https://static01.nyt.com/images/2020/06/03/business/03ontech-videoSTILL/03ontech-videoSTILL-threeByTwoMediumAt2X.png}

\href{https://www.nytimes.com/by/shira-ovide}{\includegraphics{https://static01.nyt.com/images/2020/03/18/reader-center/author-shira-ovide/author-shira-ovide-thumbLarge-v2.png}}

By \href{https://www.nytimes.com/by/shira-ovide}{Shira Ovide}

\begin{itemize}
\item
  June 3, 2020
\item
  \begin{itemize}
  \item
  \item
  \item
  \item
  \item
  \end{itemize}
\end{itemize}

\emph{This article is part of the On Tech newsletter. You can}
\href{https://www.nytimes.com/newsletters/signup/OT}{\emph{sign up
here}} \emph{to receive it weekdays.}

I had mostly avoided
\href{https://www.nytimes.com/2020/07/26/technology/tiktok-china-ban-model.html}{TikTok};
it made me feel old. But for me and
\href{https://www.cnn.com/2020/05/05/tech/tiktok-bytedance-coronavirus-intl-hnk/index.html}{many
of you}, TikTok has become a needed dose of silliness during the
pandemic --- and more recently, a unique home for grieving and activism.

Alongside short videos of a
hamster\href{https://www.tiktok.com/@hamstarz/video/6829822490131434757}{jamming
on the piano} and an
\href{https://www.tiktok.com/@cakelover363/video/6824418668445846789?lang=en}{incredible
watermelon carving}, there are
\href{https://www.tiktok.com/@kareemrahma/video/6831762210218446085?lang=US}{scenes
of the protests} against the killing of George Floyd in Minneapolis, and
a
\href{https://www.tiktok.com/@theleftistdude/video/6826796142299254021}{history
lesson} on the 1921 massacre of residents of a black neighborhood in
Tulsa.

What's unusual about TikTok is that it's not another place to see what's
happening. It's a distilled expression of how people are feeling. At its
best, a TikTok video gives me a sense of someone's essence --- and taken
together, of our collective essence.

TikTok feels familiar, but its soul is unlike that of other social media
that came before it. It can be mindless fun, but it's also a force to
pay attention to. TikTok is the first entertainment powerhouse born in
and built for the smartphone age --- and it might change everything.

It's also the first time that Americans have had to consider that U.S.
companies might not always rule the internet. There's a lot of
importance wrapped in
\href{https://www.nytimes.com/2019/03/10/style/what-is-tik-tok.html}{a
(mostly) goofy app}.

Last month, a reader named Richard wrote us asking, ``Can you explain
why TikTok is all the rage?''

Well, the magic is TikTok makes it easy
\href{https://www.nytimes.com/interactive/2019/10/10/arts/TIK-TOK.html}{to
be creative} and to watch others' best work. A 60-second limit on videos
means users don't need to create much filler, and there's often a common
thread with many videos set to the same song or riffing on a
``challenge'' like
\href{https://www.vulture.com/2020/06/tiktok-wipe-it-down-challenge-bmw-kenny-explainer.html}{cleaning
mirrors}.

TikTok makes it easy to watch by pushing you videos that its computers
predict you will like. You don't need to search or know whom to watch.
(But that is also why TikTok can operate like a bubble. I might see
Black Lives Matter videos, while you might see only celebrities
dancing.)

TikTok doesn't necessarily show you the reality of the world. It's about
expression, but it's not like anything we're used to.

Netflix, YouTube and most other internet video services grafted existing
business behaviors onto new distribution models. TikTok blew up all of
that. It wasn't made for cord cutters. It's for people who never watched
TV at all.

If you're on TikTok to talk politics, you'll find irreverent
\href{https://melmagazine.com/en-us/story/camp-pence-tiktok-memes-lgbtq-conversion-therapy}{political
in-jokes} and
\href{https://www.nytimes.com/2020/02/27/style/tiktok-politics-bernie-trump.html}{none
of the usual TV-like conventions}. Hollywood productions are absent.
Whether fun or solemn, everything is tailored to TikTok's id.

TikTok does have many of the familiar internet problems like overreach
of
\href{https://www.nytimes.com/2020/05/14/technology/tiktok-kids-privacy.html}{data
collection},
\href{https://www.latimes.com/local/lanow/la-me-ln-tik-tok-lewd-acts-arrest-20190214-story.html}{stalking}
and harmful
\href{https://www.mediamatters.org/fake-news/tiktok-hosting-videos-spreading-misinformation-about-coronavirus-despite-platforms-new}{misinformation}.

The biggest questions stem from
\href{https://www.nytimes.com/2019/11/18/technology/tiktok-alex-zhu-interview.html}{TikTok's
ownership} by the Chinese internet conglomerate ByteDance. Some American
\href{https://www.politico.com/newsletters/morning-tech/2020/05/28/house-democrats-join-gop-in-going-after-tiktok-787945}{politicians
worry} that TikTok is a conduit for China to siphon Americans' data.
(TikTok says it doesn't do this.)

TikTok faced questions last year on whether it was
\href{https://www.bloomberg.com/news/newsletters/2019-09-17/hong-kong-protests-raise-censorship-concerns-for-hot-app-tiktok?sref=7ooTCNG1}{hiding
videos from Hong Kong's protests} to appease the Chinese government. The
company said it didn't.

I don't know whether those fears are valid. But TikTok is definitely
a\href{https://www.nytimes.com/2019/11/05/business/tiktok-china-bytedance.html}{mind
bender}. It's one of the first Chinese internet services that is
globally popular. That's a challenge for Americans who are used to U.S.
internet companies dominating much of the world.

TikTok might be rewiring entertainment, giving the next generation of
activists new ways to tell stories and challenging the global internet
order.

\emph{Hey, you are someone who appreciates smart conversations about
technology. Join my DealBook colleague}
\href{https://www.nytimes.com/by/andrew-ross-sorkin}{\emph{Andrew Ross
Sorkin}} \emph{and the veteran technology journalist}
\href{https://www.nytimes.com/column/kara-swisher}{\emph{Kara Swisher}}
\emph{for a discussion about how the tech giants are dealing with free
speech, the risks and opportunities created by the pandemic and more.}
\href{https://timesevents.nytimes.com/dealbookdebrief0604}{\emph{R.S.V.P.
here}} \emph{for the call, which will be on Thursday at 11 a.m.
Eastern.}

\begin{center}\rule{0.5\linewidth}{\linethickness}\end{center}

\hypertarget{tip-of-the-week}{%
\subsubsection{Tip of the Week}\label{tip-of-the-week}}

\hypertarget{how-to-make-your-own-tiktok-videos--for-cheap}{%
\subsection{How to make your own TikTok videos --- for
cheap}\label{how-to-make-your-own-tiktok-videos--for-cheap}}

\href{https://www.nytimes.com/by/brian-x-chen}{\emph{Brian X.
Chen}}\emph{, a consumer technology writer at the The New York Times,
suggests some apps and products to help you create your own online
videos and photos.}

It's hard to become famous on social media. (I have firsthand experience
\href{https://www.nytimes.com/2018/12/05/technology/personaltech/instagram-influencers-dogs-food.html}{failing
to make my dog, Max, an Instagram celebrity}.) But if you want to give
it a shot, you don't have to splurge on fancy cameras and lights to make
videos and photos look better. You can just use your smartphone camera
and a few tools.

Here are some low-cost hacks I've used over the years:

\begin{itemize}
\tightlist
\item
  \textbf{A phone tripod.} My wife occasionally posts cooking videos to
  demonstrate her recipes, and
  \href{https://www.amazon.com/UBeesize-Portable-Adjustable-Universal-Compatible/dp/B06Y2VP3C7}{this
  tiny \$20 phone tripod} fits nicely on the kitchen counter while
  holding the smartphone stable at different angles. That beats spending
  \$300 to \$400 on a GoPro camera.
\end{itemize}

\begin{itemize}
\item
  \textbf{A work light.} Professional photographers spend hundreds of
  dollars on light kits. You know what else works great? A
  \href{https://petapixel.com/2018/07/26/these-portraits-were-shot-with-a-20-work-light-from-home-depot/}{\$20
  work light from the hardware store}. These powerful lights were
  designed for outdoor construction, but they do a miraculous job at
  lighting for indoor photography.

  The light is very harsh, though. To diffuse it, I tape a piece of
  parchment paper over the light's metal grill.
\end{itemize}

\begin{itemize}
\tightlist
\item
  \textbf{A good photo-editing app.} There are plenty of cheap photo and
  video editing apps to do touch-ups before posting your selfies.
  \href{https://apps.apple.com/us/app/vsco-photo-video-editor/id588013838}{VSCO}
  charges for special filters and editing tools, but the free basic
  features will get you one small step closer to internet stardom.
\end{itemize}

\begin{center}\rule{0.5\linewidth}{\linethickness}\end{center}

\hypertarget{before-we-go-}{%
\subsection{Before we go \ldots{}}\label{before-we-go-}}

\begin{itemize}
\item
  \textbf{Tough questions for the Facebook boss:} Mark Zuckerberg told
  Facebook employees on Tuesday that
  \href{https://www.nytimes.com/2020/06/02/technology/zuckerberg-defends-facebook-trump-posts.html}{he
  stood by the company's hands-off approach} to recent inflammatory
  posts by President Trump, despite dissent from some employees and
  outsiders, my colleagues reported. Facing fury at times during a
  virtual meeting with employees, Zuckerberg said it was ``a tough
  decision,'' but that he made a thoroughly considered call based on the
  company's policies.
\item
  \textbf{There are no magic bullets for our}
  \textbf{\href{https://www.nytimes.com/2020/05/27/technology/public-transportation-cities-pandemic.html}{city
  transportation hellscape}} \textbf{but\ldots{}} Brian, our consumer
  tech writer, tried and loved electric bikes, and he said they're an
  \href{https://www.nytimes.com/2020/06/03/technology/personaltech/e-bikes-are-having-their-moment-they-deserve-it.html}{effective
  and fun transportation option} for commuters looking to reduce the
  risk of the coronavirus and avoid nightmare traffic. (I was converted
  long ago to the joys of biking for transportation, so yea!) Check out
  Brian's recommendations on what to consider if you're e-bike curious.
\item
  \textbf{If you were confused about the black squares on Instagram:} My
  colleagues
  \href{https://www.nytimes.com/2020/06/02/style/instagram-blackout.html}{debate}
  whether people sharing images on Instagram of black boxes on Tuesday
  was an effective symbol of solidarity for people abused by police, or
  a way for people to avoid doing something meaningful about racism.
\end{itemize}

\hypertarget{hugs-to-this}{%
\subsubsection{Hugs to this}\label{hugs-to-this}}

Sticking with today's TikTok theme: Here is a
\href{https://www.tiktok.com/@miaamabile/video/6807566845948873990}{mewing
kitten} in the couch cushions.

\begin{center}\rule{0.5\linewidth}{\linethickness}\end{center}

\emph{We want to hear from you. Tell us what you think of this
newsletter and what else you'd like us to explore. You can reach us at}
\href{mailto:ontech@nytimes.com?subject=On\%20Tech\%20Feedback}{\emph{ontech@nytimes.com.}}

\emph{Get this newsletter in your inbox every
weekday;}\href{https://www.nytimes.com/newsletters/signup/OT}{\emph{please
sign up here}}\emph{.}

Advertisement

\protect\hyperlink{after-bottom}{Continue reading the main story}

\hypertarget{site-index}{%
\subsection{Site Index}\label{site-index}}

\hypertarget{site-information-navigation}{%
\subsection{Site Information
Navigation}\label{site-information-navigation}}

\begin{itemize}
\tightlist
\item
  \href{https://help.nytimes.com/hc/en-us/articles/115014792127-Copyright-notice}{©~2020~The
  New York Times Company}
\end{itemize}

\begin{itemize}
\tightlist
\item
  \href{https://www.nytco.com/}{NYTCo}
\item
  \href{https://help.nytimes.com/hc/en-us/articles/115015385887-Contact-Us}{Contact
  Us}
\item
  \href{https://www.nytco.com/careers/}{Work with us}
\item
  \href{https://nytmediakit.com/}{Advertise}
\item
  \href{http://www.tbrandstudio.com/}{T Brand Studio}
\item
  \href{https://www.nytimes.com/privacy/cookie-policy\#how-do-i-manage-trackers}{Your
  Ad Choices}
\item
  \href{https://www.nytimes.com/privacy}{Privacy}
\item
  \href{https://help.nytimes.com/hc/en-us/articles/115014893428-Terms-of-service}{Terms
  of Service}
\item
  \href{https://help.nytimes.com/hc/en-us/articles/115014893968-Terms-of-sale}{Terms
  of Sale}
\item
  \href{https://spiderbites.nytimes.com}{Site Map}
\item
  \href{https://help.nytimes.com/hc/en-us}{Help}
\item
  \href{https://www.nytimes.com/subscription?campaignId=37WXW}{Subscriptions}
\end{itemize}
