Sections

SEARCH

\protect\hyperlink{site-content}{Skip to
content}\protect\hyperlink{site-index}{Skip to site index}

\href{https://www.nytimes.com/section/politics}{Politics}

\href{https://myaccount.nytimes.com/auth/login?response_type=cookie\&client_id=vi}{}

\href{https://www.nytimes.com/section/todayspaper}{Today's Paper}

\href{/section/politics}{Politics}\textbar{}Esper Breaks With Trump on
Using Troops Against Protesters

\url{https://nyti.ms/2BwJLqH}

\begin{itemize}
\item
\item
\item
\item
\item
\end{itemize}

\href{https://www.nytimes.com/news-event/george-floyd-protests-minneapolis-new-york-los-angeles?action=click\&pgtype=Article\&state=default\&region=TOP_BANNER\&context=storylines_menu}{Race
and America}

\begin{itemize}
\tightlist
\item
  \href{https://www.nytimes.com/2020/07/26/us/protests-portland-seattle-trump.html?action=click\&pgtype=Article\&state=default\&region=TOP_BANNER\&context=storylines_menu}{Protesters
  Return to Other Cities}
\item
  \href{https://www.nytimes.com/2020/07/24/us/portland-oregon-protests-white-race.html?action=click\&pgtype=Article\&state=default\&region=TOP_BANNER\&context=storylines_menu}{Portland
  at the Center}
\item
  \href{https://www.nytimes.com/2020/07/23/podcasts/the-daily/portland-protests.html?action=click\&pgtype=Article\&state=default\&region=TOP_BANNER\&context=storylines_menu}{Podcast:
  Showdown in Portland}
\item
  \href{https://www.nytimes.com/interactive/2020/07/16/us/black-lives-matter-protests-louisville-breonna-taylor.html?action=click\&pgtype=Article\&state=default\&region=TOP_BANNER\&context=storylines_menu}{45
  Days in Louisville}
\end{itemize}

Advertisement

\protect\hyperlink{after-top}{Continue reading the main story}

Supported by

\protect\hyperlink{after-sponsor}{Continue reading the main story}

\hypertarget{esper-breaks-with-trump-on-using-troops-against-protesters}{%
\section{Esper Breaks With Trump on Using Troops Against
Protesters}\label{esper-breaks-with-trump-on-using-troops-against-protesters}}

Mark Esper's comments reflected the turmoil within the military over
President Trump, who has said he could put active-duty troops on the
streets to perform law enforcement functions.

\includegraphics{https://static01.nyt.com/images/2020/06/03/us/politics/03dc-unrest-military1/merlin_173089044_90a4efff-f245-4b8a-9ed1-7d0550719e05-articleLarge.jpg?quality=75\&auto=webp\&disable=upscale}

\href{https://www.nytimes.com/by/eric-schmitt}{\includegraphics{https://static01.nyt.com/images/2018/06/12/multimedia/author-eric-schmitt/author-eric-schmitt-thumbLarge-v2.png}}\href{https://www.nytimes.com/by/helene-cooper}{\includegraphics{https://static01.nyt.com/images/2018/08/24/multimedia/author-helene-cooper/author-helene-cooper-thumbLarge.png}}\href{https://www.nytimes.com/by/thomas-gibbons-neff}{\includegraphics{https://static01.nyt.com/images/2018/07/12/multimedia/author-thomas-gibbons-neff/author-thomas-gibbons-neff-thumbLarge.png}}\href{https://www.nytimes.com/by/maggie-haberman}{\includegraphics{https://static01.nyt.com/images/2018/07/12/multimedia/author-maggie-haberman/author-maggie-haberman-thumbLarge.png}}

By \href{https://www.nytimes.com/by/eric-schmitt}{Eric Schmitt},
\href{https://www.nytimes.com/by/helene-cooper}{Helene Cooper},
\href{https://www.nytimes.com/by/thomas-gibbons-neff}{Thomas
Gibbons-Neff} and
\href{https://www.nytimes.com/by/maggie-haberman}{Maggie Haberman}

\begin{itemize}
\item
  Published June 3, 2020Updated June 11, 2020
\item
  \begin{itemize}
  \item
  \item
  \item
  \item
  \item
  \end{itemize}
\end{itemize}

WASHINGTON --- Defense Secretary Mark T. Esper broke with President
Trump on Wednesday and said that active-duty military troops should not
be sent to control the wave of protests in American cities, at least for
now. His words were at odds with his commander in chief, who on Monday
threatened to do exactly that.

Mr. Esper's comments reflected the turmoil within the
\href{https://www.nytimes.com/2020/06/04/us/politics/trump-military-protests.html}{military
over Mr. Trump}, who in seeking to put American troops on the streets
alarmed top Pentagon officials fearful that the military would be seen
as participating in a move toward martial law.

Speaking at a news conference at the Pentagon, the defense secretary
said that the deployment of active-duty troops in a domestic law
enforcement role ``should only be used as a matter of last resort and
only in the most urgent and dire of situations.''

The president was angered by Mr. Esper's remarks, and excoriated him
later at the White House, an administration official said. Asked on
Wednesday whether Mr. Trump still had confidence in Mr. Esper, the White
House press secretary, Kayleigh McEnany, said that ``as of right now,
Secretary Esper is still Secretary Esper,'' but that ``should the
president lose faith, we will all learn about that in the future.''

Senior Pentagon leaders are now so concerned about losing public support
--- and that of their active-duty and reserve personnel, 40 percent of
whom are people of color --- that Gen.
\href{https://www.nytimes.com/2020/07/09/us/politics/milley-trump-confederate-base-names.html}{Mark
A. Milley}, the chairman of the Joint Chiefs of Staff,
\href{https://int.nyt.com/data/documenthelper/6990-milley-memo/fc4fb1c4459fbdbc87a7/optimized/full.pdf\#page=1}{released
a message} to top military commanders on Wednesday affirming that every
member of the armed forces swears an oath to defend the Constitution,
which he said ``gives Americans the right to freedom of speech and
peaceful assembly.''

Mr. Esper and General Milley acted after they came under sharp
criticism, including from retired military officers, for walking with
Mr. Trump to a church near the White House after peaceful protesters had
been forcibly cleared.

As anger mounted over the president's photo op at the church, former
Defense Secretary Jim Mattis offered a withering denunciation of the
president's leadership.

``Donald Trump is the first president in my lifetime who does not try to
unite the American people --- does not even pretend to try,''
\href{https://int.nyt.com/data/documenthelper/6991-mattis-statement/5ce1ea06ba0f4fc8eeb3/optimized/full.pdf\#page=1}{Mr.
Mattis said in a statement}. ``Instead he tries to divide us. We are
witnessing the consequences of three years of this deliberate effort. We
are witnessing the consequences of three years without mature
leadership.''

Mr. Trump
\href{https://twitter.com/realDonaldTrump/status/1268347256748507136?s=20}{responded
late Wednesday on Twitter} to Mr. Mattis's rebuke, saying that he had
had ``the honor of firing'' Mr. Mattis. In reality, Mr. Mattis
\href{https://www.nytimes.com/2018/12/20/us/politics/jim-mattis-defense-secretary-trump.html}{resigned
in protest in December 2018} over Mr. Trump's decision to withdraw
American troops from eastern Syria.

``His primary strength was not military, but rather personal public
relations. I gave him a new life, things to do, and battles to win, but
he seldom `brought home the bacon,''' the president tweeted. ``I didn't
like his `leadership' style or much else about him, and many others
agree. Glad he is gone!''

Other former military figures were less focused on Mr. Trump than on the
specter of the military being used to police protesters.

``We are at the most dangerous time for civil-military relations I've
seen in my lifetime,'' Adm. Sandy Winnefeld, a retired vice chairman of
the Joint Chiefs of Staff, said in an email. ``It is especially
important to reserve the use of federal forces for only the most dire
circumstances that actually threaten the survival of the nation. Our
senior-most military leaders need to ensure their political chain of
command understands these things.''

Pentagon officials note that the military is trained in using lethal
power against foreign adversaries, not in law enforcement, and what is
appropriate in Falluja is not in Farragut Square.

On Monday, after major protests over the weekend across the United
States, as well as late-night looting, Mr. Trump had discussed invoking
the little-used 1807 Insurrection Act to deploy active-duty troops in
American cities. He was dissuaded by General Milley and William P. Barr,
the attorney general, officials said. Officials said Mr. Esper initially
seemed to back the president's position. Still, on Monday in the Rose
Garden, Mr. Trump declared himself ``your president of law and order.''

\includegraphics{https://static01.nyt.com/images/2020/06/03/us/politics/03dc-unrest-military2/merlin_173160816_c125456c-b144-4544-bd4e-f2b1ff4d06b0-articleLarge.jpg?quality=75\&auto=webp\&disable=upscale}

Whether or not he had ever intended to make good on his threat, about
1,600 troops had been ordered to hold at bases just outside Washington,
with soldiers drawn from a rapid-reaction unit of the 82nd Airborne
Division at Fort Bragg, N.C., and a military police unit at Fort Drum,
N.Y. More than 2,000 National Guard forces are already inside the city,
a number that is expected to double in the next few days.

The Army had made a decision to send a unit of the 82nd Airborne's rapid
deployment force, about 200 troops, home from the capital region. But
Mr. Trump ordered Mr. Esper during the angry meeting at the White House
to reverse it, the administration official said. The reversal was
\href{https://www.nytimes.com/aponline/2020/06/03/us/politics/ap-us-america-protests-troops-depart.html?searchResultPosition=1}{first
reported by The Associated Press}.

Despite calls for calm from senior Pentagon leaders, the troops on the
ground in Washington on Wednesday night appeared to be ramping up for a
more militarized show of force. National Guard units pushed solidly
ahead of the police near the White House, almost becoming the public
face of the security presence. They also blocked the streets with Army
transport trucks and extended the perimeter against protesters.

Although Mr. Esper's comments at the Pentagon made clear that a rise in
violence in cities nationwide could prompt a change in his stance, his
statement was clear. Saying that the Insurrection Act should be invoked
only in the ``most urgent and dire of situations,'' he added that ``we
are not in one of those situations now.''

Mr. Esper, a West Point graduate who once served in the 101st Airborne
Division, said, ``I do not support invoking the Insurrection Act.''

At the White House, Ms. McEnany said that, for now, Mr. Trump was
``relying on surging the streets with National Guard.'' But, she noted:
``The Insurrection Act is a tool available. The president has the sole
authority and, if needed, he will use it.''

General Milley has been able to influence Mr. Trump in ways that Mr.
Esper, who the president views with skepticism, has not, White House
officials said.

Mr. Esper's explicit opposition to invoking the act came only days after
he described the country as a ``battle space'' to be cleared, a comment
that drew harsh condemnation from a number of former senior military
officials --- the kind who usually do not criticize the successors
across the Pentagon leadership. The use of the term, bandied about in
battlefield command centers, implies a piece of terrain, disassembled in
grid squares, characterized by threats and awaiting one solution:
military force through violence.

Mr. Esper also backtracked about what he knew beforehand about Mr.
Trump's
\href{https://www.nytimes.com/2020/06/02/us/politics/trump-walk-lafayette-square.html}{visit
to a church across from the White House}.

Mr. Esper said this week that he was unaware of his destination when he
set out with the president on Monday night for what he thought was a
visit to view troops near Lafayette Square. ``I didn't know where I was
going,'' Mr. Esper told NBC News in an interview on Tuesday. ``I wanted
to see how much damage actually happened.''

White House officials were furious, and Mr. Esper tried to walk back his
comments on Wednesday. He acknowledged that he did know that he was
accompanying Mr. Trump to St. John's Church for what turned out to be a
photo op after the authorities used some form of chemical spray against
protesters to clear the way.

Mr. Esper also said it took nearly 24 hours for the authorities to
determine that a flight of helicopters that descended to rooftop level
--- kicking up debris and sending peaceful protesters running for cover
--- belonged to the District of Columbia National Guard. He said that
episode was under investigation.

Mr. Esper's remarks about the delay in finding information on the
helicopter mission stand in stark contrast to the level of military
planning that occurred beforehand. An email obtained by The New York
Times indicated that Ryan McCarthy, the Army secretary, and Gen. James
C. McConville, the Army chief of staff, made clear their intent for the
evening, including the clearance of airspace. The two men, officials
said, were on hand in a command center in Washington belonging to the
F.B.I., where they pored over maps, looking at streets.

Image

National Guard troops deployed on Tuesday night outside the White
House.Credit...Erin Schaff/The New York Times

Compounding the problematic use of military helicopters to intimidate
protesters was the fact that one of the aircraft, a Lakota helicopter,
was adorned with a red cross, denoting its medical and, therefore, not
hostile affiliation.

Perhaps the most tortured of the Pentagon top leadership so far has been
General Milley, who is seen clearly in a video of the movement across
Lafayette Square walking behind Mr. Trump and wearing combat fatigues.
General Milley, who has since been criticized from a host of voices,
both military and civilian, spent the hours after the photo op walking
the streets of Washington talking to National Guards troops there.

He spoke of the need to protect the peaceful protests, in remarks that
appeared jarring to some because they came in the hours after the
president's photo op.

The comments from Mr. Esper and the letter from General Milley followed
a memo on Monday night from the Air Force chief of staff, Gen. David L.
Goldfein, deploring as a ``national tragedy'' the killing of George
Floyd, who died after he was in police custody in Minneapolis. General
Goldfein said that every American ``should be outraged.''

Since then, other messages to the armed forces have been released by
several service chiefs and secretaries --- all carefully drafted and in
no way criticizing Mr. Trump or his policies, but expressing solidarity
with American values and the military's history of staying out of
politics.

General McConville and Mr. McCarthy sent a letter to troops and their
families underscoring the ``right of the people peaceably to assemble
and to petition the government for a redress of grievances.''

The Navy's top officer, Adm. Michael M. Gilday, said in a message on
Wednesday to all sailors: ``I think we need to listen. We have black
Americans in our Navy and in our communities that are in deep pain right
now. They are hurting.''

And Chief Master Sgt. Kaleth O. Wright of the Air Force, who is black,
wrote an
\href{https://twitter.com/cmsaf18/status/1267572332907954177}{extraordinary
Twitter thread declaring}, ``I am George Floyd.''

Advertisement

\protect\hyperlink{after-bottom}{Continue reading the main story}

\hypertarget{site-index}{%
\subsection{Site Index}\label{site-index}}

\hypertarget{site-information-navigation}{%
\subsection{Site Information
Navigation}\label{site-information-navigation}}

\begin{itemize}
\tightlist
\item
  \href{https://help.nytimes.com/hc/en-us/articles/115014792127-Copyright-notice}{©~2020~The
  New York Times Company}
\end{itemize}

\begin{itemize}
\tightlist
\item
  \href{https://www.nytco.com/}{NYTCo}
\item
  \href{https://help.nytimes.com/hc/en-us/articles/115015385887-Contact-Us}{Contact
  Us}
\item
  \href{https://www.nytco.com/careers/}{Work with us}
\item
  \href{https://nytmediakit.com/}{Advertise}
\item
  \href{http://www.tbrandstudio.com/}{T Brand Studio}
\item
  \href{https://www.nytimes.com/privacy/cookie-policy\#how-do-i-manage-trackers}{Your
  Ad Choices}
\item
  \href{https://www.nytimes.com/privacy}{Privacy}
\item
  \href{https://help.nytimes.com/hc/en-us/articles/115014893428-Terms-of-service}{Terms
  of Service}
\item
  \href{https://help.nytimes.com/hc/en-us/articles/115014893968-Terms-of-sale}{Terms
  of Sale}
\item
  \href{https://spiderbites.nytimes.com}{Site Map}
\item
  \href{https://help.nytimes.com/hc/en-us}{Help}
\item
  \href{https://www.nytimes.com/subscription?campaignId=37WXW}{Subscriptions}
\end{itemize}
