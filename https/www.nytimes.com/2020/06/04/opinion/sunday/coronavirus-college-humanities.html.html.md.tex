Sections

SEARCH

\protect\hyperlink{site-content}{Skip to
content}\protect\hyperlink{site-index}{Skip to site index}

\href{https://www.nytimes.com/section/opinion/sunday}{Sunday Review}

\href{https://myaccount.nytimes.com/auth/login?response_type=cookie\&client_id=vi}{}

\href{https://www.nytimes.com/section/todayspaper}{Today's Paper}

\href{/section/opinion/sunday}{Sunday Review}\textbar{}The End of
College as We Knew It?

\href{https://nyti.ms/37bt7c2}{https://nyti.ms/37bt7c2}

\begin{itemize}
\item
\item
\item
\item
\item
\item
\end{itemize}

\href{https://www.nytimes.com/news-event/coronavirus?action=click\&pgtype=Article\&state=default\&region=TOP_BANNER\&context=storylines_menu}{The
Coronavirus Outbreak}

\begin{itemize}
\tightlist
\item
  live\href{https://www.nytimes.com/2020/08/08/world/coronavirus-updates.html?action=click\&pgtype=Article\&state=default\&region=TOP_BANNER\&context=storylines_menu}{Latest
  Updates}
\item
  \href{https://www.nytimes.com/interactive/2020/us/coronavirus-us-cases.html?action=click\&pgtype=Article\&state=default\&region=TOP_BANNER\&context=storylines_menu}{Maps
  and Cases}
\item
  \href{https://www.nytimes.com/interactive/2020/science/coronavirus-vaccine-tracker.html?action=click\&pgtype=Article\&state=default\&region=TOP_BANNER\&context=storylines_menu}{Vaccine
  Tracker}
\item
  \href{https://www.nytimes.com/interactive/2020/world/coronavirus-tips-advice.html?action=click\&pgtype=Article\&state=default\&region=TOP_BANNER\&context=storylines_menu}{F.A.Q.}
\item
  \href{https://www.nytimes.com/live/2020/08/07/business/stock-market-today-coronavirus?action=click\&pgtype=Article\&state=default\&region=TOP_BANNER\&context=storylines_menu}{Markets
  \& Economy}
\end{itemize}

Advertisement

\protect\hyperlink{after-top}{Continue reading the main story}

\href{/section/opinion}{Opinion}

Supported by

\protect\hyperlink{after-sponsor}{Continue reading the main story}

\hypertarget{the-end-of-college-as-we-knew-it}{%
\section{The End of College as We Knew
It?}\label{the-end-of-college-as-we-knew-it}}

Restaurants get eulogies. Airlines get bailouts. Shakespeare gets kicked
when he's down.

\href{https://www.nytimes.com/by/frank-bruni}{\includegraphics{https://static01.nyt.com/images/2018/04/03/opinion/frank-bruni/frank-bruni-thumbLarge.png}}

By \href{https://www.nytimes.com/by/frank-bruni}{Frank Bruni}

Opinion Columnist

\begin{itemize}
\item
  June 4, 2020
\item
  \begin{itemize}
  \item
  \item
  \item
  \item
  \item
  \item
  \end{itemize}
\end{itemize}

\includegraphics{https://static01.nyt.com/images/2020/06/08/opinion/sunday/08BruniSub2/04BruniSub2-articleLarge.jpg?quality=75\&auto=webp\&disable=upscale}

We need doctors right now. My God, we need doctors: to evaluate the
coronavirus's assault, assess the body's response and figure out where,
in that potentially deadly tumble of events, there's a chance to
intervene.

We need research scientists. It falls to them to map every last wrinkle
of this invader and find its Achilles' heel.

But we also need Achilles. We need Homer. We need writers, philosophers,
historians. They'll be the ones to chart the social, cultural and
political challenges of this pandemic --- and of all the other dynamics
that have pushed the United States so harrowingly close to the edge. In
terms of restoring faith in the American project and reseeding common
ground, they're beyond essential.

And I'm not sure we get that.

Colleges and universities are in trouble --- serious trouble. They're
agonizing over whether they can safely welcome students back to campus
in the fall or must try to replicate the educational experience
imperfectly online. They're confronting
\href{https://abcnews.go.com/Business/coronavirus-pandemic-brings-staggering-losses-colleges-universities/story?id=70359686}{sharply
reduced revenue},
\href{https://www.wsj.com/articles/public-universities-see-state-funding-disappear-effective-immediately-11587653753?mod=article_inline}{severe
budget cuts}, warfare between administrators and faculty, and even
\href{https://www.nbcnews.com/news/us-news/students-25-universities-sue-refunds-after-campuses-close-due-coronavirus-n1200746}{lawsuits
from students} who want refunds for a derailed spring semester. And a
devastated economy leaves their very missions and identities in limbo,
all but guaranteeing that more students will approach higher education
in a brutally practical fashion, as an on-ramp to employment and nothing
more.

``If one were to invent a crisis uniquely and diabolically designed to
undermine the foundations of traditional colleges and universities, it
might look very much like the current global pandemic,'' Brian
Rosenberg, who just finished a nearly 17-year stretch as president of
Macalester College,
\href{https://www.chronicle.com/article/How-Should-Colleges-Prepare/248507}{wrote
in The Chronicle of Higher Education recently}. That wasn't a renegade
take. It was a representative, even restrained, one.

When I later exchanged emails with him, he expanded on it. He observed
that the physically close-knit nature of the classroom and the campus
puts colleges ``not far behind cruise ships and assisted-living
facilities'' as ideal theaters of contagion. He noted that this
contagion came along when higher education was already on the defensive
--- maligned by conservative politicians for
\href{https://www.washingtonpost.com/sf/national/2017/11/25/elitists-crybabies-and-junky-degrees/?utm_term=.032f27b188bc}{its
supposed elitism} and resented by students and their families for its
hefty price tag.

Now, he said, he can detect people taking ``a ghoulish pleasure'' in its
travails. Restaurants get eulogies. Airlines get bailouts. Universities
get kicked when they're down. ``That says a lot about our societal
priorities,'' Rosenberg said.

But not all aspects of university life will be equally undermined. Homer
could be in particular peril, dismissed along with the rest of the
humanities as a fusty luxury, a disposable lark. And that chills
Rosenberg.

``Here is the problem,'' he told me. ``A society without a grounding in
ethics, self-reflection, empathy and beauty is one that has lost its
way.''

``We are seeing that play out,'' he added --- and this was \emph{before}
George Floyd's
\href{https://www.nytimes.com/2020/05/31/us/george-floyd-investigation.html}{anguished
pleas} and the fury and the fires. He pointed to the empathy deficit in
Americans openly hostile to social-distancing directives, which was
followed by the empathy void that put a knee to Floyd's neck. ``I can
only imagine how George Eliot or Shakespeare would write about such
people,'' he said.

We don't have to imagine, because Shakespeare, Eliot and scores of the
other writers and thinkers at the core of a liberal arts education
lavished attention on the conflict between individual desires and
communal obligations, on the toxic fruits of fear and on the dangerous
lure of ignorance. That's why we read them. That's why we should
continue to, especially now.

``This is not only a public health crisis and an economic crisis, though
Lord knows it's both of those,'' said
\href{https://english.columbia.edu/content/andrew-delbanco}{Andrew
Delbanco}, a professor of American studies at Columbia University and
the president of \href{http://www.teaglefoundation.org/Home}{the Teagle
Foundation}, a philanthropy that promotes the liberal arts. ``It's also
a values crisis. It raises all kinds of deep human questions: What are
our responsibilities to other people? Does representative democracy
work? How do we get to a place where something like bipartisanship could
emerge again?''

The answers will sooner come from history, philosophy and literature
than from drug companies, social media and outer space. Put another way,
whom do you trust: Pfizer, Mark Zuckerberg and Elon Musk, or the Rev.
Dr. Martin Luther King Jr., Plato and Jane Austen? It's not a close
call.

What a mess we're in. What disruption we're in for. It will probably
look like this in higher education: Dozens and potentially hundreds of
small four-year colleges go under, some of them within
\href{https://www.wsj.com/articles/coronavirus-pushes-colleges-to-the-breaking-point-forcing-hard-choices-about-education-11588256157}{the
next year} and others over the next five. Online instruction
proliferates, because the pandemic has forced more schools to experiment
with it, because it could be a way for them to expand enrollment and
thus revenues, and because it's more accessible to financially strapped
students who are wedging classes between shifts at work.

The already pronounced divide between richly endowed, largely
residential schools and more socioeconomically diverse ones that depend
on public funding grows wider as state and local governments face
unprecedented financial distress. A shrinking minority of students get a
boutique college experience. Then there's everybody else.

\includegraphics{https://static01.nyt.com/images/2020/06/08/opinion/08bruni2/merlin_87917062_7a45720e-d8e0-4721-b3a4-cd8e04b5dc62-articleLarge.jpg?quality=75\&auto=webp\&disable=upscale}

``We always knew that America was moving more and more toward very
different groups of people,''
\href{https://www.suny.edu/sunycon/2016/speakers/gail-mellow/}{Gail
Mellow}, the former president of LaGuardia Community College in Queens,
told me. Now that movement is accelerating.

\hypertarget{the-coronavirus-outbreak}{%
\subsubsection{The Coronavirus
Outbreak}\label{the-coronavirus-outbreak}}

\hypertarget{back-to-school}{%
\paragraph{Back to School}\label{back-to-school}}

Updated Aug. 8, 2020

The latest highlights as the first students return to U.S. schools.

\begin{itemize}
\item
  \begin{itemize}
  \tightlist
  \item
    Health experts say New York State schools are
    \href{https://www.nytimes.com/2020/08/07/health/coronavirus-ny-schools-reopen.html?action=click\&pgtype=Article\&state=default\&region=MAIN_CONTENT_2\&context=storylines_keepup}{in
    a good position to reopen}, and Gov. Andrew M. Cuomo has
    \href{https://www.nytimes.com/2020/08/07/nyregion/cuomo-schools-reopening.html?action=click\&pgtype=Article\&state=default\&region=MAIN_CONTENT_2\&context=storylines_keepup}{cleared
    the way}.
  \item
    Many schools spent the summer focused on reopening classrooms. What
    if they had
    \href{https://www.nytimes.com/2020/08/07/us/remote-learning-fall-2020.html?action=click\&pgtype=Article\&state=default\&region=MAIN_CONTENT_2\&context=storylines_keepup}{focused
    on improving remote learning} instead?
  \item
    A mother in Germany describes how her family
    \href{https://www.nytimes.com/2020/08/07/parenting/germany-schools-reopening-children.html?action=click\&pgtype=Article\&state=default\&region=MAIN_CONTENT_2\&context=storylines_keepup}{coped
    with the anxiety and uncertainty} of going back to school there.
  \item
    A high school freshman tested positive after two days in class. A
    yearbook editor worries about access to sporting events. We spoke to
    students about
    \href{https://www.nytimes.com/2020/08/06/us/coronavirus-students.html?action=click\&pgtype=Article\&state=default\&region=MAIN_CONTENT_2\&context=storylines_keepup}{what
    school is like in the age of Covid-19.}
  \end{itemize}
\end{itemize}

And if the economy doesn't do some spectacular turnaround, more students
will demand a financial payoff from college that's as immediate and
certain as possible. For computer science and chemistry departments,
that's a boon. For English, comparative literature, classics and
anthropology? A bust.

They're already hurting: The percentage of college students getting
degrees in the humanities has
\href{https://www.theatlantic.com/ideas/archive/2018/08/the-humanities-face-a-crisisof-confidence/567565/}{declined
sharply over the past decade} while the popularity of more obviously
job-related majors connected to, say,
\href{https://www.forbes.com/sites/michaeltnietzel/2019/01/07/whither-the-humanities-the-ten-year-trend-in-college-majors/\#4009403a64ad}{health
care and technology surged}. And the pandemic provides extra incentive
for schools to redirect money from the humanities to the sciences,
because that's where big grants for biomedical research are.

To solve our short-term problems, that emphasis makes sense. But to
solve our long-term ones? To apply the lessons of the Spanish flu of
1918 and the urban riots of 1968 to the misery and rage of 2020? I want
as many broadly educated, deeply reflective citizens and leaders as
possible.

Like Andrea Romero, 19, a computer science major at Purdue University
who, as part of its
\href{https://cla.purdue.edu/academic/cornerstone/index.html}{Cornerstone
program}, which encourages all undergraduates to dip into the
humanities, took a class in ``transformative texts.'' In an essay about
being forced by the pandemic to leave campus, return home and linger
there, she invoked Homer's ``Odyssey'' --- specifically, Odysseus'
\href{https://www.greekmyths-greekmythology.com/calypso-odysseus-greek-myth/}{consignment
to the nymph Calypso's island}. The hero's life there is pleasant, even
good. But the ease of a given moment can't --- and shouldn't --- erase
the commitments and aspirations beyond it.

``I look forward to my return to `Ithaca','' Romero wrote, likening the
Purdue campus to Odysseus's destination. ``Until this day arrives, I
have learned that it is valid to feel disappointed and fortunate at the
same time.''

Mrinali Dhembla, 21, told me that her double major in political science
and Chinese language at Hunter College, which is part of the City
University of New York, isn't perfectly tailored to a given profession.
But it has allowed her to see and evaluate America's predicament through
the lens of other struggles, taught her to watch for the way some people
try to profit from others' pain, taken her outside of her narrowest self
and given her ``more sensitivity and warmheartedness,'' she said.

Lexi Robinson, 21, just graduated from Central Michigan University.
Although her major was public and nonprofit administration, she also
delved into the humanities, for example taking a religion and social
issues course that she found especially meaningful. It sounded an alarm
about moral absolutism. ``Whatever side you're on, you think the other
is telling blatant lies,'' she told me, adding that such a viewpoint is
a dead end for democracy. ``How do we ever come to a middle ground?''

At Ursinus College in Pennsylvania this spring,
\href{https://www.ursinus.edu/live/profiles/83-stephanie-mackler}{Stephanie
Mackler}, an associate professor of education, asked the students in one
of her seminars to write about the merits of the liberal arts. Matt
Schmitz, 20, who is majoring in psychology and educational studies,
reflected on the story of Galileo. It's about so much more than
astronomy, he wrote; it's a window into humans' investment in
established fictions over discomfiting truths. To study the humanities,
Schmitz observed, is to connect to something grander: ``Without it,
humanity would be left to aimlessly wander from day to day and problem
to problem.''

Rodrigo Vazquez, 28, is pursuing a master's degree in applied
mathematics at the University of Nevada, Las Vegas, where he got a
bachelor's degree in economics. But he also majored in English, which
opened vistas to him that he still savors. Confined like so many
Americans to his house over recent months, he told me that he staved off
loneliness with reading: not just Camus's ``The Plague,''
\href{https://www.theguardian.com/books/2020/mar/05/publishers-report-sales-boom-in-novels-about-fictional-epidemics-camus-the-plague-dean-koontz}{an
obvious choice}, but also Proust's ``Swann's Way'' and Melville's
``Moby-Dick.'' They made him feel connected to human struggle across
time.

``Moby-Dick.'' Now there's a transformative text about our investments
in --- and responsibilities to --- one another.

Consider the celebrated passage in which Ishmael describes being roped
to Queequeg, who dangles over their ship's side to attend to a whale
carcass. If one man gets sucked into the heaving water, both men go
under. And Ishmael reflects ``that my own individuality was now merged
in a joint stock company of two; that my free will had received a mortal
wound; and that another's mistake or misfortune might plunge innocent me
into unmerited disaster and death.''

``This situation of mine,'' he adds, ``was the precise situation of
every mortal that breathes.''

Or these days, that struggles to breathe.

Image

Columbia University in Manhattan.Credit...Hiroko Masuike/The New York
Times

A vaccine for the coronavirus won't inoculate anyone against the
ideological arrogance, conspiracy theories and other internet-abetted
passions and prejudices that drive Americans apart. But the perspective,
discernment and skepticism that a liberal arts education can nurture
just might.

Science may produce better versions of tear gas and lighter versions of
riot gear, God help us. But it can't compete with the humanities for
telling us how and why certain societies unravel and others thrive.

Maybe that's so obviously self-evident that amid all the raging need in
our country, governments will dig deeper to expand the opportunity of
college. Maybe college students will demand enlightenment on top of, or
even before, job training.

``I think we're going to have a lot of surprises,'' said
\href{https://www.english.ucsb.edu/people/newfield-christopher}{Christopher
Newfield}, a professor of literature and American studies at the
University of California, Santa Barbara, who has written extensively
about the degradation of higher education over recent decades. ``People
are not linear.'' They could well flock \emph{to} Melville. ``I wouldn't
bet my house on it,'' he said, but added, ``I'd bet \emph{a room} of my
house on it.''

We need doctors, all right, but not all doctors are the same, as
\href{https://www.egonzehnder.com/office/new-york/consultant/benito-cachinero}{Benito
Cachinero-Sánchez}, the vice chair of the Library of America's board of
directors, reminded me. If he were choosing between two physicians, he
said, he would
\href{https://www.theatlantic.com/health/archive/2018/07/medicine-doctors-fiction/566342/}{go
with one who has read Chekhov}, ``because he's a fuller human being and
he's going to treat \emph{me} like a fuller human being.''

Current events show that when it comes to treating one another like
fuller human beings, we need all the help we can get.

\emph{I invite you to sign up for my free}
\href{https://www.nytimes.com/newsletters/frank-bruni}{\emph{weekly
email newsletter}}\emph{. You can follow me on Twitter
(}\href{https://twitter.com/FrankBruni}{\emph{@FrankBruni}}\emph{).}

\emph{Listen to}
\href{https://www.nytimes.com/column/the-argument}{\emph{``The
Argument'' podcast}} \emph{every Thursday morning, with Ross Douthat,
Michelle Goldberg and me.}

Advertisement

\protect\hyperlink{after-bottom}{Continue reading the main story}

\hypertarget{site-index}{%
\subsection{Site Index}\label{site-index}}

\hypertarget{site-information-navigation}{%
\subsection{Site Information
Navigation}\label{site-information-navigation}}

\begin{itemize}
\tightlist
\item
  \href{https://help.nytimes.com/hc/en-us/articles/115014792127-Copyright-notice}{©~2020~The
  New York Times Company}
\end{itemize}

\begin{itemize}
\tightlist
\item
  \href{https://www.nytco.com/}{NYTCo}
\item
  \href{https://help.nytimes.com/hc/en-us/articles/115015385887-Contact-Us}{Contact
  Us}
\item
  \href{https://www.nytco.com/careers/}{Work with us}
\item
  \href{https://nytmediakit.com/}{Advertise}
\item
  \href{http://www.tbrandstudio.com/}{T Brand Studio}
\item
  \href{https://www.nytimes.com/privacy/cookie-policy\#how-do-i-manage-trackers}{Your
  Ad Choices}
\item
  \href{https://www.nytimes.com/privacy}{Privacy}
\item
  \href{https://help.nytimes.com/hc/en-us/articles/115014893428-Terms-of-service}{Terms
  of Service}
\item
  \href{https://help.nytimes.com/hc/en-us/articles/115014893968-Terms-of-sale}{Terms
  of Sale}
\item
  \href{https://spiderbites.nytimes.com}{Site Map}
\item
  \href{https://help.nytimes.com/hc/en-us}{Help}
\item
  \href{https://www.nytimes.com/subscription?campaignId=37WXW}{Subscriptions}
\end{itemize}
