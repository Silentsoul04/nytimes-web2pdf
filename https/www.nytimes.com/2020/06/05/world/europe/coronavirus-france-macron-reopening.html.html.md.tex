Sections

SEARCH

\protect\hyperlink{site-content}{Skip to
content}\protect\hyperlink{site-index}{Skip to site index}

\href{https://www.nytimes.com/section/world/europe}{Europe}

\href{https://myaccount.nytimes.com/auth/login?response_type=cookie\&client_id=vi}{}

\href{https://www.nytimes.com/section/todayspaper}{Today's Paper}

\href{/section/world/europe}{Europe}\textbar{}Macron Beat Back the
Coronavirus. France Is Not Impressed.

\href{https://nyti.ms/3cG51Hb}{https://nyti.ms/3cG51Hb}

\begin{itemize}
\item
\item
\item
\item
\item
\end{itemize}

\href{https://www.nytimes.com/news-event/coronavirus?action=click\&pgtype=Article\&state=default\&region=TOP_BANNER\&context=storylines_menu}{The
Coronavirus Outbreak}

\begin{itemize}
\tightlist
\item
  live\href{https://www.nytimes.com/2020/08/08/world/coronavirus-updates.html?action=click\&pgtype=Article\&state=default\&region=TOP_BANNER\&context=storylines_menu}{Latest
  Updates}
\item
  \href{https://www.nytimes.com/interactive/2020/us/coronavirus-us-cases.html?action=click\&pgtype=Article\&state=default\&region=TOP_BANNER\&context=storylines_menu}{Maps
  and Cases}
\item
  \href{https://www.nytimes.com/interactive/2020/science/coronavirus-vaccine-tracker.html?action=click\&pgtype=Article\&state=default\&region=TOP_BANNER\&context=storylines_menu}{Vaccine
  Tracker}
\item
  \href{https://www.nytimes.com/interactive/2020/world/coronavirus-tips-advice.html?action=click\&pgtype=Article\&state=default\&region=TOP_BANNER\&context=storylines_menu}{F.A.Q.}
\item
  \href{https://www.nytimes.com/live/2020/08/07/business/stock-market-today-coronavirus?action=click\&pgtype=Article\&state=default\&region=TOP_BANNER\&context=storylines_menu}{Markets
  \& Economy}
\end{itemize}

Advertisement

\protect\hyperlink{after-top}{Continue reading the main story}

Supported by

\protect\hyperlink{after-sponsor}{Continue reading the main story}

\hypertarget{macron-beat-back-the-coronavirus-france-is-not-impressed}{%
\section{Macron Beat Back the Coronavirus. France Is Not
Impressed.}\label{macron-beat-back-the-coronavirus-france-is-not-impressed}}

President Emmanuel Macron has gotten little credit for his country's
relative success in battling the contagion. Instead, he remains
unpopular and subject to the usual dose of resentments.

\includegraphics{https://static01.nyt.com/images/2020/06/04/world/VIRUS-FRANCE-MACRON01sub/merlin_171567093_7a7cffd5-a978-4573-ac78-9ed03e8671e1-articleLarge.jpg?quality=75\&auto=webp\&disable=upscale}

\href{https://www.nytimes.com/by/adam-nossiter}{\includegraphics{https://static01.nyt.com/images/2018/10/15/multimedia/author-adam-nossiter/author-adam-nossiter-thumbLarge.png}}

By \href{https://www.nytimes.com/by/adam-nossiter}{Adam Nossiter}

\begin{itemize}
\item
  June 5, 2020
\item
  \begin{itemize}
  \item
  \item
  \item
  \item
  \item
  \end{itemize}
\end{itemize}

PARIS --- President Emmanuel Macron's government has beaten back the
coronavirus, prevented mass layoffs, propped up the salaries of the
unemployed, staved off long food lines, and achieved a lower death rate
than its neighbors, Germany excepted.

Mr. Macron ordered a strict lockdown that lasted nearly two months, and
when it was over the virus was barely circulating. But while the early
response could be faulted for some sluggishness and a shortage of masks,
and more than 29,000 people died, France has fared better than many in
the pandemic, especially when compared with the United States, Italy,
Spain and especially Britain.

Just don't tell that to the French, who resent Mr. Macron for it more
than ever.

The French expect much of their leaders, and almost always find them
wanting.
\href{https://www.nytimes.com/2020/02/25/world/europe/macron-france-pensions.html?searchResultPosition=13}{Mr.
Macron is no exception.} In fact, the better the results, the less
willing, it seems, the French are to applaud their president. That
pattern has held virtually since Mr. Macron took office in 2017, casting
a shadow over a term expiring in two years.

Mr. Macron reduced unemployment and created more jobs, but the French
resented him for loosening labor protections. He evened out the
country's helter-skelter pension system, and there were
\href{https://www.nytimes.com/2019/12/05/world/europe/france-strike-macron.html?searchResultPosition=33}{months
of strikes} by aggrieved unions and citizens distrustful of his
intentions.

Even as the French celebrated their provisional release from lockdown
this week with the much-anticipated partial
\href{https://www.nytimes.com/2020/06/02/world/europe/coronavirus-paris-cafe-reopen-france.html}{reopening
of cafes and restaurants}, the coronavirus has only reinforced the
paradox of the president's uneasy relationship with his own citizens.

\includegraphics{https://static01.nyt.com/images/2020/06/04/world/VIRUS-FRANCE-MACRON02/merlin_172913808_e7169b0a-a9ac-4b61-8834-3f246ae4d9c2-articleLarge.jpg?quality=75\&auto=webp\&disable=upscale}

``Deconfinement is going pretty well,'' said Olivier Galland, a
sociologist at the National Center for Scientific Research. ``But the
French don't seem satisfied. But I don't think they can ever be
satisfied.''

On Friday the head of the government's scientific council, the
immunologist Jean-François Delfraissy, declared the epidemic ``under
control'' in France, in an interview on French radio. Still, the French,
far more than their European neighbors, have judged the government's
performance in response to the health crisis harshly.

``Distrust is a structural element of French society, stable and
well-established,'' Mr. Galland wrote in a recent paper on ``The Great
Depression of the French'' for Telos*,* a widely followed political
science website.

On average, over half of Europe's citizens, outside of France, view
their government's performance in response to the virus favorably, even
in countries with far worse records. In France, 66 percent have an
unfavorable view, according to a recent Figaro poll.

\hypertarget{latest-updates-the-coronavirus-outbreak}{%
\section{\texorpdfstring{\href{https://www.nytimes.com/2020/08/07/world/covid-19-news.html?action=click\&pgtype=Article\&state=default\&region=MAIN_CONTENT_1\&context=storylines_live_updates}{Latest
Updates: The Coronavirus
Outbreak}}{Latest Updates: The Coronavirus Outbreak}}\label{latest-updates-the-coronavirus-outbreak}}

Updated 2020-08-08T12:04:28.992Z

\begin{itemize}
\tightlist
\item
  \href{https://www.nytimes.com/2020/08/07/world/covid-19-news.html?action=click\&pgtype=Article\&state=default\&region=MAIN_CONTENT_1\&context=storylines_live_updates\#link-1f86d03a}{As
  the U.S. relief talks falter again, Trump says he is prepared to act
  on his own.}
\item
  \href{https://www.nytimes.com/2020/08/07/world/covid-19-news.html?action=click\&pgtype=Article\&state=default\&region=MAIN_CONTENT_1\&context=storylines_live_updates\#link-3f64a70a}{Cuomo
  says N.Y. schools can reopen in-person but leaves it up to districts
  to determine if, when and how.}
\item
  \href{https://www.nytimes.com/2020/08/07/world/covid-19-news.html?action=click\&pgtype=Article\&state=default\&region=MAIN_CONTENT_1\&context=storylines_live_updates\#link-14e70066}{Thousands
  of cases went unreported in California when a computer server failed.}
\end{itemize}

\href{https://www.nytimes.com/2020/08/07/world/covid-19-news.html?action=click\&pgtype=Article\&state=default\&region=MAIN_CONTENT_1\&context=storylines_live_updates}{See
more updates}

More live coverage:
\href{https://www.nytimes.com/live/2020/08/07/business/stock-market-today-coronavirus?action=click\&pgtype=Article\&state=default\&region=MAIN_CONTENT_1\&context=storylines_live_updates}{Markets}

Mr. Macron stiffened and looked impatient when he was asked recently on
French television about his unpopularity.

``Look, I don't sit around feeling sorry for myself,'' he said. ``I'm
looking ahead.''

``For decades this country has known doubt and division,'' Mr. Macron
added. ``I don't believe in miracles. This distrustful France exists. It
hasn't changed.''

Whatever credit France's government gets from its success in dealing
with the virus has gone instead to Mr. Macron's understated prime
minister, Édouard Philippe.

``The news is pretty good,'' Mr. Philippe said simply last week, after
looking at the post-lockdown results. Over 60 percent found him
convincing in a poll by the independent Odoxa polling firm for Le Figaro
and France-Info.

In a sign of his political ascendancy, Mr. Philippe was on the cover of
this week's L'Obs*,* a popular weekly newsmagazine, with the headline,
``The Tough Guy.''

``Can Macron do without Édouard Philippe?'' the magazine asked,
alighting on speculation that Mr. Macron would jettison a prime minister
who has stolen the spotlight once the crisis ended.

Yet the government's measures --- a tightly enforced lockdown,
mobilization of French technology like high-speed trains to save
patients, and closely followed counsel from scientists --- were Mr.
Macron's. That is the French way: the president decides, and the nation
follows.

Image

Mr. Macron ordered a strict lockdown that lasted nearly two
months.Credit...Dmitry Kostyukov for The New York Times

But that means Mr. Macron takes the blame, too, for the early shortage
of masks, which the government did not initially admit to and a
spokeswoman minimized. The affair riled the French media for several
weeks, but has since largely dropped from view. On the streets some wear
masks but many do not.

``What's most problematic is that we're actually being lied to,'' said
Marie Balaril, 27, a social-sciences instructor at a Paris university,
as she recalled the government's refusal to acknowledge that the country
had faced a mask shortage.

The president has vigorously defended his record. ``Let's be honest,''
Mr. Macron said in the recent television interview. ``At the beginning
of March nobody was talking about masks.''

``When I look around, nobody was ready,'' he said. ``Nobody. Nobody.''

The per capita death rate in France is higher than the United States,
with more than 100,000 deaths. But France has a population density --- a
key variable in the epidemic --- more than three times greater. France's
\href{https://www.nytimes.com/2020/05/20/world/europe/virus-paris-reopening.html?action=click\&module=RelatedLinks\&pgtype=Article}{hospitalization
and death curves have been in sharp decline} since about the second week
of April.

Image

Masks distribution at the Gare du Nord in Paris. The country had faced a
shortage of masks.Credit...Andrea Mantovani for The New York Times

In contrast to those on the street, many experts and others interviewed
gave the government good marks.

Guillaume Chiche, a parliamentarian who recently deserted Mr. Macron's
party --- another sign of the French president's waning popularity ---
said the government's actions ``were very strong.''

\href{https://www.nytimes.com/news-event/coronavirus?action=click\&pgtype=Article\&state=default\&region=MAIN_CONTENT_3\&context=storylines_faq}{}

\hypertarget{the-coronavirus-outbreak-}{%
\subsubsection{The Coronavirus Outbreak
›}\label{the-coronavirus-outbreak-}}

\hypertarget{frequently-asked-questions}{%
\paragraph{Frequently Asked
Questions}\label{frequently-asked-questions}}

Updated August 6, 2020

\begin{itemize}
\item ~
  \hypertarget{why-are-bars-linked-to-outbreaks}{%
  \paragraph{Why are bars linked to
  outbreaks?}\label{why-are-bars-linked-to-outbreaks}}

  \begin{itemize}
  \tightlist
  \item
    Think about a bar. Alcohol is flowing. It can be loud, but it's
    definitely intimate, and you often need to lean in close to hear
    your friend. And strangers have way, way fewer reservations about
    coming up to people in a bar. That's sort of the point of a bar.
    Feeling good and close to strangers. It's no surprise, then, that
    \href{https://www.nytimes.com/2020/07/02/us/coronavirus-bars.html?action=click\&pgtype=Article\&state=default\&region=MAIN_CONTENT_3\&context=storylines_faq}{bars
    have been linked to outbreaks in several states.} Louisiana health
    officials have tied
    \href{https://www.nytimes.com/2020/06/22/us/new-coronavirus-phase.html?action=click\&pgtype=Article\&state=default\&region=MAIN_CONTENT_3\&context=storylines_faq}{at
    least 100 coronavirus cases} to bars in the Tigerland nightlife
    district in Baton Rouge. Minnesota has traced 328 recent cases to
    bars across the state.
    \href{https://www.boisestatepublicradio.org/post/bars-large-venues-close-ada-county-after-surge-coronavirus-prompts-rollback\#stream/0}{In
    Idaho}, health officials shut down bars in Ada County after
    reporting clusters of infections among young adults who had visited
    several bars in downtown Boise. Governors in
    \href{https://www.nytimes.com/2020/07/01/us/california-coronavirus-reopening.html?action=click\&pgtype=Article\&state=default\&region=MAIN_CONTENT_3\&context=storylines_faq}{California},
    \href{https://www.nytimes.com/2020/06/14/us/coronavirus-united-states.html?action=click\&pgtype=Article\&state=default\&region=MAIN_CONTENT_3\&context=storylines_faq}{Texas
    and Arizona}, where coronavirus cases are soaring, have ordered
    hundreds of newly reopened bars to shut down. Less than two weeks
    after Colorado's bars reopened at limited capacity, Gov. Jared Polis
    \href{https://www.denverpost.com/2020/06/30/colorado-bars-closed-coronavirus/}{ordered
    them to close}.
  \end{itemize}
\item ~
  \hypertarget{i-have-antibodies-am-i-now-immune}{%
  \paragraph{I have antibodies. Am I now
  immune?}\label{i-have-antibodies-am-i-now-immune}}

  \begin{itemize}
  \tightlist
  \item
    As of right now,
    \href{https://www.nytimes.com/2020/07/22/health/covid-antibodies-herd-immunity.html?action=click\&pgtype=Article\&state=default\&region=MAIN_CONTENT_3\&context=storylines_faq}{that
    seems likely, for at least several months.} There have been
    frightening accounts of people suffering what seems to be a second
    bout of Covid-19. But experts say these patients may have a
    drawn-out course of infection, with the virus taking a slow toll
    weeks to months after initial exposure. People infected with the
    coronavirus typically
    \href{https://www.nature.com/articles/s41586-020-2456-9}{produce}
    immune molecules called antibodies, which are
    \href{https://www.nytimes.com/2020/05/07/health/coronavirus-antibody-prevalence.html?action=click\&pgtype=Article\&state=default\&region=MAIN_CONTENT_3\&context=storylines_faq}{protective
    proteins made in response to an
    infection}\href{https://www.nytimes.com/2020/05/07/health/coronavirus-antibody-prevalence.html?action=click\&pgtype=Article\&state=default\&region=MAIN_CONTENT_3\&context=storylines_faq}{.
    These antibodies may} last in the body
    \href{https://www.nature.com/articles/s41591-020-0965-6}{only two to
    three months}, which may seem worrisome, but that's perfectly normal
    after an acute infection subsides, said Dr. Michael Mina, an
    immunologist at Harvard University. It may be possible to get the
    coronavirus again, but it's highly unlikely that it would be
    possible in a short window of time from initial infection or make
    people sicker the second time.
  \end{itemize}
\item ~
  \hypertarget{im-a-small-business-owner-can-i-get-relief}{%
  \paragraph{I'm a small-business owner. Can I get
  relief?}\label{im-a-small-business-owner-can-i-get-relief}}

  \begin{itemize}
  \tightlist
  \item
    The
    \href{https://www.nytimes.com/article/small-business-loans-stimulus-grants-freelancers-coronavirus.html?action=click\&pgtype=Article\&state=default\&region=MAIN_CONTENT_3\&context=storylines_faq}{stimulus
    bills enacted in March} offer help for the millions of American
    small businesses. Those eligible for aid are businesses and
    nonprofit organizations with fewer than 500 workers, including sole
    proprietorships, independent contractors and freelancers. Some
    larger companies in some industries are also eligible. The help
    being offered, which is being managed by the Small Business
    Administration, includes the Paycheck Protection Program and the
    Economic Injury Disaster Loan program. But lots of folks have
    \href{https://www.nytimes.com/interactive/2020/05/07/business/small-business-loans-coronavirus.html?action=click\&pgtype=Article\&state=default\&region=MAIN_CONTENT_3\&context=storylines_faq}{not
    yet seen payouts.} Even those who have received help are confused:
    The rules are draconian, and some are stuck sitting on
    \href{https://www.nytimes.com/2020/05/02/business/economy/loans-coronavirus-small-business.html?action=click\&pgtype=Article\&state=default\&region=MAIN_CONTENT_3\&context=storylines_faq}{money
    they don't know how to use.} Many small-business owners are getting
    less than they expected or
    \href{https://www.nytimes.com/2020/06/10/business/Small-business-loans-ppp.html?action=click\&pgtype=Article\&state=default\&region=MAIN_CONTENT_3\&context=storylines_faq}{not
    hearing anything at all.}
  \end{itemize}
\item ~
  \hypertarget{what-are-my-rights-if-i-am-worried-about-going-back-to-work}{%
  \paragraph{What are my rights if I am worried about going back to
  work?}\label{what-are-my-rights-if-i-am-worried-about-going-back-to-work}}

  \begin{itemize}
  \tightlist
  \item
    Employers have to provide
    \href{https://www.osha.gov/SLTC/covid-19/standards.html}{a safe
    workplace} with policies that protect everyone equally.
    \href{https://www.nytimes.com/article/coronavirus-money-unemployment.html?action=click\&pgtype=Article\&state=default\&region=MAIN_CONTENT_3\&context=storylines_faq}{And
    if one of your co-workers tests positive for the coronavirus, the
    C.D.C.} has said that
    \href{https://www.cdc.gov/coronavirus/2019-ncov/community/guidance-business-response.html}{employers
    should tell their employees} -\/- without giving you the sick
    employee's name -\/- that they may have been exposed to the virus.
  \end{itemize}
\item ~
  \hypertarget{what-is-school-going-to-look-like-in-september}{%
  \paragraph{What is school going to look like in
  September?}\label{what-is-school-going-to-look-like-in-september}}

  \begin{itemize}
  \tightlist
  \item
    It is unlikely that many schools will return to a normal schedule
    this fall, requiring the grind of
    \href{https://www.nytimes.com/2020/06/05/us/coronavirus-education-lost-learning.html?action=click\&pgtype=Article\&state=default\&region=MAIN_CONTENT_3\&context=storylines_faq}{online
    learning},
    \href{https://www.nytimes.com/2020/05/29/us/coronavirus-child-care-centers.html?action=click\&pgtype=Article\&state=default\&region=MAIN_CONTENT_3\&context=storylines_faq}{makeshift
    child care} and
    \href{https://www.nytimes.com/2020/06/03/business/economy/coronavirus-working-women.html?action=click\&pgtype=Article\&state=default\&region=MAIN_CONTENT_3\&context=storylines_faq}{stunted
    workdays} to continue. California's two largest public school
    districts --- Los Angeles and San Diego --- said on July 13, that
    \href{https://www.nytimes.com/2020/07/13/us/lausd-san-diego-school-reopening.html?action=click\&pgtype=Article\&state=default\&region=MAIN_CONTENT_3\&context=storylines_faq}{instruction
    will be remote-only in the fall}, citing concerns that surging
    coronavirus infections in their areas pose too dire a risk for
    students and teachers. Together, the two districts enroll some
    825,000 students. They are the largest in the country so far to
    abandon plans for even a partial physical return to classrooms when
    they reopen in August. For other districts, the solution won't be an
    all-or-nothing approach.
    \href{https://bioethics.jhu.edu/research-and-outreach/projects/eschool-initiative/school-policy-tracker/}{Many
    systems}, including the nation's largest, New York City, are
    devising
    \href{https://www.nytimes.com/2020/06/26/us/coronavirus-schools-reopen-fall.html?action=click\&pgtype=Article\&state=default\&region=MAIN_CONTENT_3\&context=storylines_faq}{hybrid
    plans} that involve spending some days in classrooms and other days
    online. There's no national policy on this yet, so check with your
    municipal school system regularly to see what is happening in your
    community.
  \end{itemize}
\end{itemize}

``Now, they seem logical. But at the time they were anything but
neutral,'' Mr. Chiche said, pointing to the moves to prop up salaries,
ban religious ceremonies, and impose the lockdown. ``I think they made
choices that were optimal.''

Still, he joined 13 other members of parliament who deserted the French
president's party in May, depriving it of its majority --- a symbolic
blow widely interpreted in the French media as an ominous sign for Mr.
Macron's future.

Mr. Chiche, an ex-Socialist, has been one of the president's critics on
the left, a group judging him too favorable to business and urging him
to ``define a new horizon,'' as Mr. Chiche put it.

Frederic Keck, an anthropologist and biosecurity expert, also at the
National Center for Scientific Research, called Mr. Macron's handling of
the pandemic ``pretty good.''

``Very centralized management around the president. Very French, but
also relatively efficient,'' he said.

``This dissatisfaction is the reflection of an excessive demand for
security,'' Mr. Keck added. But he, too, noted that Mr. Macron was not
getting much credit.

Over half the French approve of the government's reopening plan. But
they don't approve of Mr. Macron: Just 30 to 40 percent judged him up to
dealing with the epidemic. In another Figaro poll, 62 percent of
respondents found Mr. Macron's manner ``arrogant'' and
``authoritarian.''

In some ways Mr. Macron is his own worst enemy, with a style that can
come off as imperious. His speeches during the crisis were lengthy and
literary, both trademarks. He first reproached the French for lacking
``a sense of responsibilities,'' then later praised them for their
discipline.

Image

Mr. Macron visiting a supermarket in Brittany in April.Credit...Pool
photo by Stephane Mahe

``He likes these lyrical effusions, and people just aren't keen on
that,'' Mr. Galland said.

In the recent television appearance, Mr. Macron was shown meeting a
group of unhappy top chefs by videoconference from the Élysée Palace.

The chefs --- some of the most famous names in French cuisine, including
Alain Ducasse --- didn't conceal their frustration at being forced to
stay closed during the lockdown.

``We're not optimistic about the survival of about half of our
restaurants,'' Mr. Ducasse said.

Mr. Macron was not impressed. He smiled slightly at the grumbling, then
administered a lesson to the complaining chefs.

``Look, I like liberty as much as you,'' Mr. Macron said. ``But what
you've got to remember is that it's good to exercise this liberty in a
country like France. It's good to live in a country where the state is
strong.''

He added, pointedly, ``There are other countries where the state is
letting people fail.''

Constant Meheut contributed reporting.

Advertisement

\protect\hyperlink{after-bottom}{Continue reading the main story}

\hypertarget{site-index}{%
\subsection{Site Index}\label{site-index}}

\hypertarget{site-information-navigation}{%
\subsection{Site Information
Navigation}\label{site-information-navigation}}

\begin{itemize}
\tightlist
\item
  \href{https://help.nytimes.com/hc/en-us/articles/115014792127-Copyright-notice}{©~2020~The
  New York Times Company}
\end{itemize}

\begin{itemize}
\tightlist
\item
  \href{https://www.nytco.com/}{NYTCo}
\item
  \href{https://help.nytimes.com/hc/en-us/articles/115015385887-Contact-Us}{Contact
  Us}
\item
  \href{https://www.nytco.com/careers/}{Work with us}
\item
  \href{https://nytmediakit.com/}{Advertise}
\item
  \href{http://www.tbrandstudio.com/}{T Brand Studio}
\item
  \href{https://www.nytimes.com/privacy/cookie-policy\#how-do-i-manage-trackers}{Your
  Ad Choices}
\item
  \href{https://www.nytimes.com/privacy}{Privacy}
\item
  \href{https://help.nytimes.com/hc/en-us/articles/115014893428-Terms-of-service}{Terms
  of Service}
\item
  \href{https://help.nytimes.com/hc/en-us/articles/115014893968-Terms-of-sale}{Terms
  of Sale}
\item
  \href{https://spiderbites.nytimes.com}{Site Map}
\item
  \href{https://help.nytimes.com/hc/en-us}{Help}
\item
  \href{https://www.nytimes.com/subscription?campaignId=37WXW}{Subscriptions}
\end{itemize}
