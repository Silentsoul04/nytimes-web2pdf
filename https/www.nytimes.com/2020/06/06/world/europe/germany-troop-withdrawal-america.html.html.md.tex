Sections

SEARCH

\protect\hyperlink{site-content}{Skip to
content}\protect\hyperlink{site-index}{Skip to site index}

\href{https://www.nytimes.com/section/world/europe}{Europe}

\href{https://myaccount.nytimes.com/auth/login?response_type=cookie\&client_id=vi}{}

\href{https://www.nytimes.com/section/todayspaper}{Today's Paper}

\href{/section/world/europe}{Europe}\textbar{}Has `America First' Become
`Trump First'? Germans Wonder

\url{https://nyti.ms/3gZH4y4}

\begin{itemize}
\item
\item
\item
\item
\item
\end{itemize}

Advertisement

\protect\hyperlink{after-top}{Continue reading the main story}

Supported by

\protect\hyperlink{after-sponsor}{Continue reading the main story}

\hypertarget{has-america-first-become-trump-first-germans-wonder}{%
\section{Has `America First' Become `Trump First'? Germans
Wonder}\label{has-america-first-become-trump-first-germans-wonder}}

One week after Chancellor Angela Merkel told President Trump she would
not attend the Group of 7 meeting he wanted to host, he decided to
withdraw 9,500 troops from her country.

\includegraphics{https://static01.nyt.com/images/2020/06/06/world/06germany-troops/merlin_165411357_6b0b7062-6d9d-40ab-95ba-6e630f0c6f8b-articleLarge.jpg?quality=75\&auto=webp\&disable=upscale}

\href{https://www.nytimes.com/by/katrin-bennhold}{\includegraphics{https://static01.nyt.com/images/2018/07/13/multimedia/author-katrin-bennhold/author-katrin-bennhold-thumbLarge.png}}

By \href{https://www.nytimes.com/by/katrin-bennhold}{Katrin Bennhold}

\begin{itemize}
\item
  June 6, 2020
\item
  \begin{itemize}
  \item
  \item
  \item
  \item
  \item
  \end{itemize}
\end{itemize}

BERLIN --- When Chancellor Angela Merkel told President Trump last week
that she would not attend the Group of 7 meeting he wanted to host in
Washington this month, the call between the two leaders, normally
respectful in tone, turned testy.

Ms. Merkel cited the ongoing pandemic. Mr. Trump responded with a
wide-ranging monologue about his frustrations with the Group of 7 and
NATO and the World Health Organization. America was doing great, he
said, even as citizens rioted in cities across the country. The pandemic
was China's fault.

They hung up after only 20 minutes.

``It was not a nice call,'' said one official who was listening and
recounted the exchange.

One week later, Germans learned that the United States planned to
\href{https://www.nytimes.com/2020/06/05/world/europe/trump-troops-europe-nato-germany.html?searchResultPosition=2}{cut
its troop presence} in their country by more than a quarter. Some 9,500
soldiers who have helped keep peace on the continent are to leave within
the next three months. There had been no warning, and even today there
is not yet an official notification.

It is not clear whether the two episodes are related. But together they
signal a breakdown in relations between the United States and Europe's
most influential country, not seen since World War II as communication
collapses and interests diverge over nearly every important issue,
including Russia, Iran, China, and trade and security.

Trust between Ms. Merkel and Mr. Trump was lost long ago. Now, officials
and analysts say, something much more fundamental was slipping away ---
trust in the strategic foundation of the trans-Atlantic alliance itself.

The lack of consultation on the decision, and the uncertainty and
unpredictability in dealing with Mr. Trump --- his decision to leave the
W.H.O. similarly surprised allies --- have become hallmarks of his years
in office.

In the view of European officials, the United States has gone from being
the indispensable ally to the undependable one. It is a frustrating turn
of events that they have neither sought nor desired.

``It's yet another wake-up call for us Europeans to take our fate into
our own hands,'' said Johann David Wadephul, a senior German lawmaker
from Ms. Merkel's Christian Democrats.

\includegraphics{https://static01.nyt.com/images/2020/06/06/world/06germany-troops2/06germany-troops2-articleLarge.jpg?quality=75\&auto=webp\&disable=upscale}

By unilaterally withdrawing troops from the United States' most
important European ally, Mr. Trump is hurting NATO, or the North
Atlantic Treaty Organization, and directly playing into the hands of
President Vladimir V. Putin of Russia, who has long resented America's
military footprint on the continent, said Thomas Kleine-Brockhoff, the
Berlin-based vice president of the research group, the German Marshall
Fund.

Mr. Trump's strategic rival was neither Mr. Putin nor President Xi
Jinping of China, Mr. Kleine-Brockhoff concluded. ``His systemic rival
is Angela Merkel,'' he said.

The lack of chemistry between Ms. Merkel, a quantum physicist, and Mr.
Trump, a celebrity millionaire, is not new. What is new is that Mr.
Trump appears to have abandoned any pretense of being on the same side.

``Merkel represents everything Trump loathes: Globalism,
multilateralism, international law,'' Mr. Kleine-Brockhoff said. ``Trump
aligns more with the well-known authoritarian leaders in the world.''

Mr. Trump, Germans worry, is in the process of redefining American
national interest and a strong trans-Atlantic alliance is not part of
it.

``He thinks he's exerting power and leverage and the might of the United
States,'' Mr. Kleine-Brockhoff said. ``But should the troops really be
brought home **** within the next three months, he would deprive the
United States of 25 percent of its deterrence capability in Europe.''

It is a radical departure from American postwar foreign policy.

Seasoned diplomats on both sides of the Atlantic say American-German
relations should be considered critically important, even more so after
Britain's decision to leave the European Union.

Germany is the richest, most populous country in Europe, the continent's
economic powerhouse and an important American economic partner. German
companies employ roughly 700,000 people in the United States.

At the same time, some 35,000 American troops are stationed in Germany,
one of the most important military hubs for the United States. And some
12,000 German civilians are employed on these bases. Tens of thousands
of other jobs also depend on the American presence. The troop withdrawal
will hurt Germany economically.

Image

The United States military hospital in Landstuhl treats many of the
soldiers who are hurt in combat in Iraq or Afghanistan.Credit...Ralph
Orlowski/Reuters

But it will hurt the United States strategically, officials say.

In addition to withdrawing permanent troops, the president plans to
limit the maximum number of troops in Germany to 25,000, less than half
the current maximum. That's probably more important than his plan to
withdraw 9,500 troops, said Ivo Daalder, the head of the Chicago Council
on Global Affairs, a think tank. ``The issue is less about the troops
stationed permanently in Germany than how many troops you can rotate in
at any onetime,'' Mr. Daalder said.

Almost all American military flights to Iraq or Afghanistan pass through
Ramstein, in southwestern Germany, the biggest American base outside the
United States.

The United States military hospital in Landstuhl treats many of the
soldiers who are hurt in combat in Iraq or Afghanistan. U.S. military
missions in Africa are coordinated from southwest Germany, too.

Above all, American troops in Germany have served as a deterrent to an
increasingly aggressive Russia, analysts said.

Nicholas Burns, a former official in the administration of George W.
Bush and now professor at the Harvard Kennedy School, said the troop
withdrawal served Mr. Putin's long-term objective of dividing the West.

``This is a significant political and symbolic blow to our immediate
priority in Europe: strengthening U.S. strategic connections to Germany,
the most important European power, especially following Britain's exit
from the E.U.,'' Mr. Burns said.

Mr. Trump's decision to withdraw troops is in line with his ``America
First'' vision of limiting American deployments overseas, and his
insistence that allies must shoulder more of the burden for their own
defense.

But before the presidential elections in November, some say ``America
First'' seems to have morphed into ``Trump First.''

``It's all about him, it's not about a vision of the world, not about
politics, it's about him, about his need for validation --- and
sometimes his need for revenge,'' said Norbert Röttgen, chairman of
Germany's foreign affairs committee and one of several candidates hoping
to succeed Ms. Merkel as chancellor next year.

Image

The lack of chemistry between Ms. Merkel, a quantum physicist, and Mr.
Trump, a celebrity millionaire, is not news.Credit...Erin Schaff/The New
York Times

German officials are already bracing for more disruptive announcements
from Washington in the months before the American election --- and
possibly after.

Many worry that Mr. Trump will unilaterally speed up the time table for
troop withdrawals from Afghanistan, giving the Taliban the upper hand in
peace talks. Some even expect him to bring troops back from South Korea.

``He is nervous and under pressure and the tighter it gets for him, the
more critical the situation is for him, the more he will lash out,'' Mr.
Mr. Röttgen said.

Some fear that if Mr. Trump is re-elected, his first announcement will
be that the United States is leaving NATO. Ultimately, Mr.
Kleine-Brockhoff said, the question is: ``How much can Trump destroy?''

Mr. Trump has long complained about the expense of protecting the United
States' allies in NATO. Since taking office, he has singled Germany out
as a wealthy nation that spends proportionately little on its defense.

Some of the complaints are legitimate, analysts say.

``There's plenty to criticize about how Germany spends its defense
euros,'' said Ivo Daalder, a former U.S. ambassador to NATO. But, he
said, the way to get the country to spend more was to ``come up with
common strategies, which is what NATO does.''

Mr. Trump's idea that Germans were freeloading on the presence of
American troops in Germany was simply wrong, he said.

Germany, he said, pays a lot to host the American forces and makes a
significant amount of land available to U.S. and NATO for training. The
only place in Europe anyone can do live fire exercises, for example, is
in Bavaria.

``We are in NATO not as a favor to our allies but to ensure our own
security,'' said Mr. Daalder. ``We deploy troops in Germany and
elsewhere to prevent wars so we don't have to fight them.''

Steven Erlanger contributed reporting from Brussels and Michael Crowley
from Washington.

Advertisement

\protect\hyperlink{after-bottom}{Continue reading the main story}

\hypertarget{site-index}{%
\subsection{Site Index}\label{site-index}}

\hypertarget{site-information-navigation}{%
\subsection{Site Information
Navigation}\label{site-information-navigation}}

\begin{itemize}
\tightlist
\item
  \href{https://help.nytimes.com/hc/en-us/articles/115014792127-Copyright-notice}{©~2020~The
  New York Times Company}
\end{itemize}

\begin{itemize}
\tightlist
\item
  \href{https://www.nytco.com/}{NYTCo}
\item
  \href{https://help.nytimes.com/hc/en-us/articles/115015385887-Contact-Us}{Contact
  Us}
\item
  \href{https://www.nytco.com/careers/}{Work with us}
\item
  \href{https://nytmediakit.com/}{Advertise}
\item
  \href{http://www.tbrandstudio.com/}{T Brand Studio}
\item
  \href{https://www.nytimes.com/privacy/cookie-policy\#how-do-i-manage-trackers}{Your
  Ad Choices}
\item
  \href{https://www.nytimes.com/privacy}{Privacy}
\item
  \href{https://help.nytimes.com/hc/en-us/articles/115014893428-Terms-of-service}{Terms
  of Service}
\item
  \href{https://help.nytimes.com/hc/en-us/articles/115014893968-Terms-of-sale}{Terms
  of Sale}
\item
  \href{https://spiderbites.nytimes.com}{Site Map}
\item
  \href{https://help.nytimes.com/hc/en-us}{Help}
\item
  \href{https://www.nytimes.com/subscription?campaignId=37WXW}{Subscriptions}
\end{itemize}
