Sections

SEARCH

\protect\hyperlink{site-content}{Skip to
content}\protect\hyperlink{site-index}{Skip to site index}

\href{/section/world/europe}{Europe}\textbar{}Giving Your Number to
Strangers? It's Not Flirting; It's a Rule

\url{https://nyti.ms/2MDS9a1}

\begin{itemize}
\item
\item
\item
\item
\item
\end{itemize}

\href{https://www.nytimes.com/news-event/coronavirus?action=click\&pgtype=Article\&state=default\&region=TOP_BANNER\&context=storylines_menu}{The
Coronavirus Outbreak}

\begin{itemize}
\tightlist
\item
  live\href{https://www.nytimes.com/2020/08/01/world/coronavirus-covid-19.html?action=click\&pgtype=Article\&state=default\&region=TOP_BANNER\&context=storylines_menu}{Latest
  Updates}
\item
  \href{https://www.nytimes.com/interactive/2020/us/coronavirus-us-cases.html?action=click\&pgtype=Article\&state=default\&region=TOP_BANNER\&context=storylines_menu}{Maps
  and Cases}
\item
  \href{https://www.nytimes.com/interactive/2020/science/coronavirus-vaccine-tracker.html?action=click\&pgtype=Article\&state=default\&region=TOP_BANNER\&context=storylines_menu}{Vaccine
  Tracker}
\item
  \href{https://www.nytimes.com/interactive/2020/07/29/us/schools-reopening-coronavirus.html?action=click\&pgtype=Article\&state=default\&region=TOP_BANNER\&context=storylines_menu}{What
  School May Look Like}
\item
  \href{https://www.nytimes.com/live/2020/07/31/business/stock-market-today-coronavirus?action=click\&pgtype=Article\&state=default\&region=TOP_BANNER\&context=storylines_menu}{Economy}
\end{itemize}

\includegraphics{https://static01.nyt.com/images/2020/06/02/world/00berlin-dispatch1/merlin_172972389_a0a15c76-acf2-4388-9f7e-415de2fda3b0-articleLarge.jpg?quality=75\&auto=webp\&disable=upscale}

Berlin Dispatch

\hypertarget{giving-your-number-to-strangers-its-not-flirting-its-a-rule}{%
\section{Giving Your Number to Strangers? It's Not Flirting; It's a
Rule}\label{giving-your-number-to-strangers-its-not-flirting-its-a-rule}}

As Berlin emerges from its lockdown, residents are getting used to
social distancing in bars and swimming pools --- and to handing out
their phone number several times a day to help in contact tracing.

Tomi Paasonen, right, and Colin Comfort, second from right, celebrated
their marriage with a small group of friends in a Berlin park last
month.~Credit...

Supported by

\protect\hyperlink{after-sponsor}{Continue reading the main story}

By \href{https://www.nytimes.com/by/katrin-bennhold}{Katrin Bennhold}

Photographs by Emile Ducke

\begin{itemize}
\item
  Published June 7, 2020Updated June 25, 2020
\item
  \begin{itemize}
  \item
  \item
  \item
  \item
  \item
  \end{itemize}
\end{itemize}

BERLIN --- I've given my phone number to a lot of strangers over the
past week.

I scribbled it down for the charming barista who made my latte the other
morning. I handed it to the waiter who took my first restaurant booking
in more than two months. I gave it to a hairdresser, to an ice cream
vendor, even to the guy behind plexiglass at the open-air swimming pool
that just reopened.

``That mask looks great on you,'' he said and winked at me over his own
mask in one of those I-can't-believe-I'm-having-this-conversation
moments we have every day now.

I swear I wasn't flirting. I was just trying to have a swim.

Berlin has been emerging from its coronavirus lockdown in full force,
and the cost of freedom, the price of having a snippet of our pre-corona
lives --- date night with your husband, a hair cut (a hair cut!) --- is
handing over contact details.

Call them corona logs: A registry of who was where when and for how long
has become mandatory so the detectives of the German health authority
can trace the contacts of a newly infected person.

It is one way in which
\href{https://www.nytimes.com/2020/06/25/world/europe/germany-coronavirus-reopening.html}{Germany}'s
new normal looks anything but normal. In a country where privacy is
something of a national religion, Germans now casually hand over their
private address at every turn of daily life.

At the \href{http://refinerycoffee.de/en/}{Refinery}, a trendy coffee
shop near the river Spree in central Berlin, Sabine Baum, a graphic
designer, was adding her details to the dog-eared handwritten list on
the counter as she waited for her porridge one recent morning.

``Somehow it feels OK because it's just on paper and not online,'' she
said.

\includegraphics{https://static01.nyt.com/images/2020/06/02/world/00berlin-dispatch2/merlin_172972362_9d4fa590-447c-4328-999f-43a3f8546d76-articleLarge.jpg?quality=75\&auto=webp\&disable=upscale}

Image

In the beer garden at the Tiergarten, groups of guests were separated by
plexiglass.

The longing for normality is a powerful incentive to put up with things
that in early March --- a lifetime ago --- would have seemed either
unacceptable or totally absurd to many Germans. Like wearing something
in a sauna (face masks might become mandatory when saunas reopen next
month).

The lockdown in Berlin was always softer than in neighboring capitals.
Streets and parks emptied, but never completely.

\hypertarget{latest-updates-global-coronavirus-outbreak}{%
\section{\texorpdfstring{\href{https://www.nytimes.com/2020/08/01/world/coronavirus-covid-19.html?action=click\&pgtype=Article\&state=default\&region=MAIN_CONTENT_1\&context=storylines_live_updates}{Latest
Updates: Global Coronavirus
Outbreak}}{Latest Updates: Global Coronavirus Outbreak}}\label{latest-updates-global-coronavirus-outbreak}}

Updated 2020-08-02T07:42:09.613Z

\begin{itemize}
\tightlist
\item
  \href{https://www.nytimes.com/2020/08/01/world/coronavirus-covid-19.html?action=click\&pgtype=Article\&state=default\&region=MAIN_CONTENT_1\&context=storylines_live_updates\#link-34047410}{The
  U.S. reels as July cases more than double the total of any other
  month.}
\item
  \href{https://www.nytimes.com/2020/08/01/world/coronavirus-covid-19.html?action=click\&pgtype=Article\&state=default\&region=MAIN_CONTENT_1\&context=storylines_live_updates\#link-780ec966}{Top
  U.S. officials work to break an impasse over the federal jobless
  benefit.}
\item
  \href{https://www.nytimes.com/2020/08/01/world/coronavirus-covid-19.html?action=click\&pgtype=Article\&state=default\&region=MAIN_CONTENT_1\&context=storylines_live_updates\#link-2bc8948}{Its
  outbreak untamed, Melbourne goes into even greater lockdown.}
\end{itemize}

\href{https://www.nytimes.com/2020/08/01/world/coronavirus-covid-19.html?action=click\&pgtype=Article\&state=default\&region=MAIN_CONTENT_1\&context=storylines_live_updates}{See
more updates}

More live coverage:
\href{https://www.nytimes.com/live/2020/07/31/business/stock-market-today-coronavirus?action=click\&pgtype=Article\&state=default\&region=MAIN_CONTENT_1\&context=storylines_live_updates}{Markets}

But then streets and parks were never that busy to start with. Less
densely populated than Manhattan, London and many Asian cities, Berlin
is a mellow metropolis. Long before the virus hit, social distancing was
the norm at prominent sights like the Brandenburg Gate (even if we
didn't call it that).

There are moments when Germany feels --- almost --- normal again.
Children walking to school in the mornings. Restaurants buzzing at
night. A wedding in a public park. There is even a hint of rush hour as
more people trade their home office for their real one.

But one swimming pool now has a traffic light outside its shower area to
regulate the number of people inside. Vending machines that used to sell
candy and condoms now sell disinfectant and masks. A theater has removed
500 of its 700 seats to allow for distancing during performances.

There are the drive-in church services and drive-in coronavirus test
centers. There are the masks --- and the people waving doctor's notes
exempting them from wearing masks.

Image

At the swimming pool in the Olympic Stadium, there are now regulations
on how many swimmers can be in the pool at once.

Image

 At soccer practice, children train individually with their own balls at
clearly marked locations two meters apart.

In the Tiergarten, Berlin's equivalent of Central Park, the beer garden
has reopened. Walking one recent afternoon toward the cluster of trestle
tables on the banks of the Neuer See, a small boating lake, the scene
ahead of me was so normal it was surreal. It was busy! Children were
taking turns on the wooden swing. A couple of row boats drifted lazily
across the lake.

It was only when I got closer that I saw the plexiglass dividers on the
tables.

``Knock knock!'' said the older woman sitting on the other end of our
table as she waved to my 4-year-old son through the plastic.

The man at the boat stand was disinfecting a pile of life vests. We,
too, rented a boat that afternoon (after writing down our names, address
and phone numbers on another dog-eared paper list).

``Can I touch the oar?'' my 8-year-old daughter asked uncertainly. She
had noticed the couple getting into the rowing boat next to us wearing
rubber gloves.

Her question echoed something I heard on the soccer field, when my older
daughter's team resumed practicing last month.

``The kids ask me: `Can I touch the ball? Can I pass the ball?''' said
Emilia Rahaus, a 19-year-old coach.

Coronavirus soccer, at least in the amateur sphere, looks nothing like
real soccer. There are no games. The children --- seven on each half of
the field --- train individually with their own balls at clearly marked
locations two meters apart. The goals are disinfected after use.

``It's so familiar and yet so different,'' Ms. Rahaus said. Behind her,
on the far end of the field, a banner in the blue-white color of West
Berlin's traditional soccer club fluttered in the afternoon breeze:
``Youth is our future.''

Image

Theater seats were removed in the auditorium of the Berliner Ensemble~
to comply with social distancing rules.

Image

Dinner at `noto,' a reopened restaurant in Berlin's Mitte district.

That motto made me think of the chorus of tired mothers (and it is
mainly mothers) at the school gate in the morning. Schools have
reopened. But because of distancing rules classes remain maddeningly
sporadic.

\href{https://www.nytimes.com/news-event/coronavirus?action=click\&pgtype=Article\&state=default\&region=MAIN_CONTENT_3\&context=storylines_faq}{}

\hypertarget{the-coronavirus-outbreak-}{%
\subsubsection{The Coronavirus Outbreak
›}\label{the-coronavirus-outbreak-}}

\hypertarget{frequently-asked-questions}{%
\paragraph{Frequently Asked
Questions}\label{frequently-asked-questions}}

Updated July 27, 2020

\begin{itemize}
\item ~
  \hypertarget{should-i-refinance-my-mortgage}{%
  \paragraph{Should I refinance my
  mortgage?}\label{should-i-refinance-my-mortgage}}

  \begin{itemize}
  \tightlist
  \item
    \href{https://www.nytimes.com/article/coronavirus-money-unemployment.html?action=click\&pgtype=Article\&state=default\&region=MAIN_CONTENT_3\&context=storylines_faq}{It
    could be a good idea,} because mortgage rates have
    \href{https://www.nytimes.com/2020/07/16/business/mortgage-rates-below-3-percent.html?action=click\&pgtype=Article\&state=default\&region=MAIN_CONTENT_3\&context=storylines_faq}{never
    been lower.} Refinancing requests have pushed mortgage applications
    to some of the highest levels since 2008, so be prepared to get in
    line. But defaults are also up, so if you're thinking about buying a
    home, be aware that some lenders have tightened their standards.
  \end{itemize}
\item ~
  \hypertarget{what-is-school-going-to-look-like-in-september}{%
  \paragraph{What is school going to look like in
  September?}\label{what-is-school-going-to-look-like-in-september}}

  \begin{itemize}
  \tightlist
  \item
    It is unlikely that many schools will return to a normal schedule
    this fall, requiring the grind of
    \href{https://www.nytimes.com/2020/06/05/us/coronavirus-education-lost-learning.html?action=click\&pgtype=Article\&state=default\&region=MAIN_CONTENT_3\&context=storylines_faq}{online
    learning},
    \href{https://www.nytimes.com/2020/05/29/us/coronavirus-child-care-centers.html?action=click\&pgtype=Article\&state=default\&region=MAIN_CONTENT_3\&context=storylines_faq}{makeshift
    child care} and
    \href{https://www.nytimes.com/2020/06/03/business/economy/coronavirus-working-women.html?action=click\&pgtype=Article\&state=default\&region=MAIN_CONTENT_3\&context=storylines_faq}{stunted
    workdays} to continue. California's two largest public school
    districts --- Los Angeles and San Diego --- said on July 13, that
    \href{https://www.nytimes.com/2020/07/13/us/lausd-san-diego-school-reopening.html?action=click\&pgtype=Article\&state=default\&region=MAIN_CONTENT_3\&context=storylines_faq}{instruction
    will be remote-only in the fall}, citing concerns that surging
    coronavirus infections in their areas pose too dire a risk for
    students and teachers. Together, the two districts enroll some
    825,000 students. They are the largest in the country so far to
    abandon plans for even a partial physical return to classrooms when
    they reopen in August. For other districts, the solution won't be an
    all-or-nothing approach.
    \href{https://bioethics.jhu.edu/research-and-outreach/projects/eschool-initiative/school-policy-tracker/}{Many
    systems}, including the nation's largest, New York City, are
    devising
    \href{https://www.nytimes.com/2020/06/26/us/coronavirus-schools-reopen-fall.html?action=click\&pgtype=Article\&state=default\&region=MAIN_CONTENT_3\&context=storylines_faq}{hybrid
    plans} that involve spending some days in classrooms and other days
    online. There's no national policy on this yet, so check with your
    municipal school system regularly to see what is happening in your
    community.
  \end{itemize}
\item ~
  \hypertarget{is-the-coronavirus-airborne}{%
  \paragraph{Is the coronavirus
  airborne?}\label{is-the-coronavirus-airborne}}

  \begin{itemize}
  \tightlist
  \item
    The coronavirus
    \href{https://www.nytimes.com/2020/07/04/health/239-experts-with-one-big-claim-the-coronavirus-is-airborne.html?action=click\&pgtype=Article\&state=default\&region=MAIN_CONTENT_3\&context=storylines_faq}{can
    stay aloft for hours in tiny droplets in stagnant air}, infecting
    people as they inhale, mounting scientific evidence suggests. This
    risk is highest in crowded indoor spaces with poor ventilation, and
    may help explain super-spreading events reported in meatpacking
    plants, churches and restaurants.
    \href{https://www.nytimes.com/2020/07/06/health/coronavirus-airborne-aerosols.html?action=click\&pgtype=Article\&state=default\&region=MAIN_CONTENT_3\&context=storylines_faq}{It's
    unclear how often the virus is spread} via these tiny droplets, or
    aerosols, compared with larger droplets that are expelled when a
    sick person coughs or sneezes, or transmitted through contact with
    contaminated surfaces, said Linsey Marr, an aerosol expert at
    Virginia Tech. Aerosols are released even when a person without
    symptoms exhales, talks or sings, according to Dr. Marr and more
    than 200 other experts, who
    \href{https://academic.oup.com/cid/article/doi/10.1093/cid/ciaa939/5867798}{have
    outlined the evidence in an open letter to the World Health
    Organization}.
  \end{itemize}
\item ~
  \hypertarget{what-are-the-symptoms-of-coronavirus}{%
  \paragraph{What are the symptoms of
  coronavirus?}\label{what-are-the-symptoms-of-coronavirus}}

  \begin{itemize}
  \tightlist
  \item
    Common symptoms
    \href{https://www.nytimes.com/article/symptoms-coronavirus.html?action=click\&pgtype=Article\&state=default\&region=MAIN_CONTENT_3\&context=storylines_faq}{include
    fever, a dry cough, fatigue and difficulty breathing or shortness of
    breath.} Some of these symptoms overlap with those of the flu,
    making detection difficult, but runny noses and stuffy sinuses are
    less common.
    \href{https://www.nytimes.com/2020/04/27/health/coronavirus-symptoms-cdc.html?action=click\&pgtype=Article\&state=default\&region=MAIN_CONTENT_3\&context=storylines_faq}{The
    C.D.C. has also} added chills, muscle pain, sore throat, headache
    and a new loss of the sense of taste or smell as symptoms to look
    out for. Most people fall ill five to seven days after exposure, but
    symptoms may appear in as few as two days or as many as 14 days.
  \end{itemize}
\item ~
  \hypertarget{does-asymptomatic-transmission-of-covid-19-happen}{%
  \paragraph{Does asymptomatic transmission of Covid-19
  happen?}\label{does-asymptomatic-transmission-of-covid-19-happen}}

  \begin{itemize}
  \tightlist
  \item
    So far, the evidence seems to show it does. A widely cited
    \href{https://www.nature.com/articles/s41591-020-0869-5}{paper}
    published in April suggests that people are most infectious about
    two days before the onset of coronavirus symptoms and estimated that
    44 percent of new infections were a result of transmission from
    people who were not yet showing symptoms. Recently, a top expert at
    the World Health Organization stated that transmission of the
    coronavirus by people who did not have symptoms was ``very rare,''
    \href{https://www.nytimes.com/2020/06/09/world/coronavirus-updates.html?action=click\&pgtype=Article\&state=default\&region=MAIN_CONTENT_3\&context=storylines_faq\#link-1f302e21}{but
    she later walked back that statement.}
  \end{itemize}
\end{itemize}

My son is allowed only half days in nursery. One daughter goes to school
two days a week, the other 90 minutes a day.

``Home schooling was a breeze compared to this,'' one fellow mother
sighed.

She recently called in sick at work because the logistics of multiple
children with wildly different schedules was just too much. Her husband
earns more, so he can't afford to miss his Zoom meetings, which leaves
the burden of child-care squarely on her.

``So much for school openings helping parents get back to work,'' she
said.

And something else has crept into conversations: Is it all necessary?

Germany has been so successful at containing the pandemic that few
people here know anyone who got infected.

``Trust me, it's just the flu,'' a taxi driver told me recently. His
mask was casually suspended from the rearview mirror.

Even our pediatrician called all the precautions ``quite exaggerated.''

Every Saturday,
\href{https://www.nytimes.com/2020/05/18/world/europe/coronavirus-germany-far-right.html}{an
eclectic mix of coronavirus protesters} take to the streets to air their
frustrations with restrictions. The sight of them --- masked and
standing six feet apart --- was telling of yet another new normal:
``When we used to protest we weren't allowed to cover our faces,'' my
mother chuckled. ``Nowadays you \emph{have} to cover your face.''

If social distancing during a protest is tricky, try social distancing
in the pool. Outside the entrance of my local pool, there was a graphic
of what it looks like in theory: four swimmers in a five-square-meter
area of water.

In practice, it looked a little different.

Image

A boat excursion on a lake in the Tiergarten.

That weird dance we all do on the sidewalk to avoid getting too close to
someone? Picture that in a swimming lane.

``Watch your distance!'' a man shouted, when I overtook him.

I told my husband about it later. A Brit who has been reprimanded more
than once for crossing an empty road against a red light, he is
convinced most Germans love all the new rules they can now enforce on
other citizens.

We were having a date. Our first in corona times.

Before the pandemic, date night had been a fixture of our lives, a
longstanding tradition that had carried us through the disruptions of
three babies and two intense jobs. Every Friday, an evening of
uninterrupted conversation. Call it marital maintenance.

So when our \href{http://jungbluth-restaurant.de/}{favorite local
restaurant} reopened after two months of lockdown, it was a Big Deal, up
there with a visit to the hair dresser.

I dressed up in my new face mask, a snazzy number with a flower pattern
I had bought specially for the occasion (and that had so impressed the
guy at the pool).

It all went well until I went down the one-way system to the bathroom
the wrong way and got told off by another patron.

``See?'' my husband said.

Advertisement

\protect\hyperlink{after-bottom}{Continue reading the main story}

\hypertarget{site-index}{%
\subsection{Site Index}\label{site-index}}

\hypertarget{site-information-navigation}{%
\subsection{Site Information
Navigation}\label{site-information-navigation}}

\begin{itemize}
\tightlist
\item
  \href{https://help.nytimes.com/hc/en-us/articles/115014792127-Copyright-notice}{©~2020~The
  New York Times Company}
\end{itemize}

\begin{itemize}
\tightlist
\item
  \href{https://www.nytco.com/}{NYTCo}
\item
  \href{https://help.nytimes.com/hc/en-us/articles/115015385887-Contact-Us}{Contact
  Us}
\item
  \href{https://www.nytco.com/careers/}{Work with us}
\item
  \href{https://nytmediakit.com/}{Advertise}
\item
  \href{http://www.tbrandstudio.com/}{T Brand Studio}
\item
  \href{https://www.nytimes.com/privacy/cookie-policy\#how-do-i-manage-trackers}{Your
  Ad Choices}
\item
  \href{https://www.nytimes.com/privacy}{Privacy}
\item
  \href{https://help.nytimes.com/hc/en-us/articles/115014893428-Terms-of-service}{Terms
  of Service}
\item
  \href{https://help.nytimes.com/hc/en-us/articles/115014893968-Terms-of-sale}{Terms
  of Sale}
\item
  \href{https://spiderbites.nytimes.com}{Site Map}
\item
  \href{https://help.nytimes.com/hc/en-us}{Help}
\item
  \href{https://www.nytimes.com/subscription?campaignId=37WXW}{Subscriptions}
\end{itemize}
