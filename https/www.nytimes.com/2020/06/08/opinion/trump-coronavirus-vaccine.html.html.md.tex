Sections

SEARCH

\protect\hyperlink{site-content}{Skip to
content}\protect\hyperlink{site-index}{Skip to site index}

\href{https://myaccount.nytimes.com/auth/login?response_type=cookie\&client_id=vi}{}

\href{https://www.nytimes.com/section/todayspaper}{Today's Paper}

\href{/section/opinion}{Opinion}\textbar{}Could Trump Turn a Vaccine
Into a Campaign Stunt?

\href{https://nyti.ms/3h1k5CO}{https://nyti.ms/3h1k5CO}

\begin{itemize}
\item
\item
\item
\item
\item
\item
\end{itemize}

Advertisement

\protect\hyperlink{after-top}{Continue reading the main story}

\href{/section/opinion}{Opinion}

Supported by

\protect\hyperlink{after-sponsor}{Continue reading the main story}

\hypertarget{could-trump-turn-a-vaccine-into-a-campaign-stunt}{%
\section{Could Trump Turn a Vaccine Into a Campaign
Stunt?}\label{could-trump-turn-a-vaccine-into-a-campaign-stunt}}

In a desperate search for a boost, he could release a coronavirus
vaccine that has not been shown to be safe and effective as an October
surprise.

By \href{https://hcmg.wharton.upenn.edu/profile/zemanuel/}{Ezekiel J.
Emanuel} and Paul A. Offit

Dr. Emanuel and Dr. Offit are professors at the University of
Pennsylvania.

\begin{itemize}
\item
  June 8, 2020
\item
  \begin{itemize}
  \item
  \item
  \item
  \item
  \item
  \item
  \end{itemize}
\end{itemize}

\includegraphics{https://static01.nyt.com/images/2020/06/08/opinion/08Emanuel-Offit/merlin_172389771_00c37a56-baa0-4544-817c-ebf281076c0b-articleLarge.jpg?quality=75\&auto=webp\&disable=upscale}

Oct. 23, 2020, 9 a.m., with 10 days before the election, Fox New
releases a poll showing President Trump trailing Joe Biden by eight
percentage points.

Oct. 23, 2020, 3 p.m., at a hastily convened news conference, President
Trump announces that the Food and Drug Administration has just issued an
Emergency Use Authorization for a
\href{https://www.nytimes.com/interactive/2020/06/09/magazine/covid-vaccine.html}{coronavirus
vaccine}. Mr. Trump declares victory over Covid-19, demands that all
businesses reopen immediately and predicts a rapid economic recovery.

Given how this president has behaved, this incredibly dangerous scenario
is not far-fetched. In a desperate search for a political boost, he
could release a coronavirus vaccine before it had been thoroughly tested
and shown to be safe and effective.

There are
\href{https://www.who.int/who-documents-detail/draft-landscape-of-covid-19-candidate-vaccines}{123
candidate Covid-19 vaccines in development, and 10 are in human trials}.
Many have not even been tested, or only perfunctorily tested, in
animals. In July, the National Institutes of Health is
\href{https://www.nbcnews.com/health/health-news/nih-director-large-scale-vaccine-testing-should-be-ready-july-n1207751}{planning}
to begin randomized phase III trials to test whether some of the 10
vaccines prevent infection with coronavirus. Some pharmaceutical
companies are planning to start their own trials at about the same time.
Astra Zeneca has already
\href{https://www.astrazeneca.com/media-centre/press-releases/2020/astrazeneca-advances-response-to-global-covid-19-challenge-as-it-receives-first-commitments-for-oxfords-potential-new-vaccine.html}{mentioned}
it plans to begin delivery of its vaccine in October.

\emph{{[}}\href{https://www.nytimes.com/interactive/2020/science/coronavirus-vaccine-tracker.html}{\emph{Follow
our Live Coronavirus Vaccine Tracker}}\emph{.{]}}

Pfizer is
\href{https://www.cnbc.com/2020/05/05/pfizer-biontech-are-set-to-begin-us-coronavirus-vaccine-trial.html}{planning}
to give its vaccine to approximately
\href{https://www.cnbc.com/2020/05/05/pfizer-biontech-are-set-to-begin-us-coronavirus-vaccine-trial.html}{8,000
patients}. The N.I.H. is planning to enroll 30,000 participants ---
20,000 getting a candidate vaccine and 10,000 research controls.

By comparison, the Phase III effectiveness trial for one rotavirus
vaccine,
\href{https://www.cdc.gov/vaccines/pubs/surv-manual/chpt13-rotavirus.html}{RotaTeq},
to prevent diarrhea involved about 70,000 infants from 2001 to 2004 and
another rotavirus vaccine trial,
\href{https://www.nejm.org/doi/10.1056/NEJMoa052664?url_ver=Z39.88-2003\&rfr_id=ori\%3Arid\%3Acrossref.org\&rfr_dat=cr_pub++0www.ncbi.nlm.nih.gov}{Rotarix},
involved 63,000 infants, from 2003 to 2006.

Researchers are expecting that it will be likely to take at least
another eight to 12 months to determine whether these coronavirus
vaccines are effective. Scientists have to wait until a sufficient
number of patients are exposed to coronavirus to see if the vaccine
really reduces the infection rate, as well as how many people develop
uncommon side effects. For
\href{https://www.sciencedirect.com/science/article/pii/S0140673617318214}{comparison},
the effectiveness trial for the rotavirus vaccines took about four years
and the human papillomavirus vaccine studies to prevent cervical cancer
took seven years.

So could Mr. Trump really pull an ``October Surprise'' with a vaccine
less than five months from today?

One highly unlikely possibility is that recruitment of volunteers in a
coronavirus ``hot spot'' would be so rapid that it would allow for an
adequate assessment of the vaccine's safety and effectiveness very
quickly.

There is another scenario that is far more ominous: Three months after
the N.I.H. trials begin in July --- so, mid October --- studies reveal
many patients are developing high levels of antibodies to the
coronavirus without severe side effects. As the White House did with its
relentless promotion of hydroxychloroquine as a cure, it would badger
the F.D.A. to permit use of the vaccine. More pressure would come from
drug companies, some of whom may spend up to \$1 billion on research and
are intensely competing for prestige and glory. They are planning to
begin manufacturing their vaccine candidates at-risk --- that is, before
completed studies showing their vaccine is actually effective.

Cognizant of the fate of
\href{https://www.washingtonpost.com/health/2020/05/05/rick-bright-hydroxychloroquine-whistleblower-complaint/}{Rick
Bright} --- the head of the Biomedical Advanced Research and Development
Authority, who was summarily demoted when he resisted the president's
wishes to ramp up purchase of hydroxychloroquine --- the F.D.A. could
issue an Emergency Use Authorization for one or more vaccines. These
authorizations only require that the F.D.A. finds it ``reasonable to
believe'' that a vaccine ``may be effective'' in preventing a
life-threatening disease for it to be put on the market, without being
formally licensed.

An emergency authorization would allow Mr. Trump to hold his news
conference and declare victory. But like President George W. Bush's
``Mission Accomplished'' proclamation, it has the potential to be a
travesty. Millions of vaccines could be distributed without proof that
the vaccine can prevent disease or transmission.

No vaccine since the 1950s has been approved and licensed without
completing large, prospective, placebo-controlled studies of safety and
effectiveness.

Even if a vaccine generates antibodies, it does not prove that the
vaccine is effective at preventing infection; it only makes it more
likely that the vaccine would be effective. Indeed, about half of the
vaccines for other diseases that work and are on the market actually
\href{https://www.ncbi.nlm.nih.gov/pmc/articles/PMC2897268/\#:~:text=Although\%20the\%20immune\%20system\%20is,well\%20as\%20quantity\%2C\%20are\%20important.}{lack
clear immunological correlates} for protection, meaning they are
effective but patients' antibodies, immune cells or other markers do not
identify whether a patient is protected. Even with the initial trials,
we are likely to have scant data on whether older people will mount an
immune reaction and be protected.

Giving people a false sense of being protected will most likely lead to
serious outbreaks of the disease as people reduce their compliance with
physical distancing and other public health measures.

If only 20,000 participants receive the vaccine, serious but rare side
effects might be missed. If such harms eventually arise, it could
further erode a fragile vaccine confidence and threaten the ability to
get enough people vaccinated to establish herd immunity. That would be a
disaster.

We were once in a situation very similar to the current one. Like
Covid-19 today, polio in the 1950s was a horrific disease feared by
every parent. Each summer 1,500 children died and as many as 30,000
became paralyzed for the rest of their lives. Jonas Salk produced his
vaccine and tested it on 700 children in the Pittsburgh area. It was
safe and produced antibodies. But proof that it was effective at
preventing polio was demanded. A randomized, controlled trial was
required before the vaccine would be licensed and distributed. More than
400,000 children got the vaccine and 200,000 got placebo. Only after
this effectiveness trial was completed was the Salk vaccine licensed and
all children finally protected from the dreaded disease.

The F.D.A. must require more than the production of antibodies to
approve a vaccine, even for an emergency authorization, much less
licensing. Only when the independent data safety and monitoring board
composed of physicians, researchers and biostatisticians reviews the
accumulated trial data to assess the safety and effectiveness of the
vaccines, should the F.D.A. be allowed to decide on approval.

Thousands of Americans have already died as Donald Trump has perpetually
postponed effective public health interventions and made poor
therapeutic recommendations. We must be on alert to prevent him from
corrupting the rigorous assessment of safety and effectiveness of
Covid-19 vaccines in order to pull an October vaccine surprise to try to
win re-election.

Ezekiel Emanuel,
\href{https://www.google.com/search?q=\%40ZekeEmanuel\&rlz=1C5CHFA_enUS745US745\&oq=\%40ZekeEmanuel\&aqs=chrome..69i57.1069j0j4\&sourceid=chrome\&ie=UTF-8}{@ZekeEmanuel},
is professor of medical ethics and health policy at the University of
Pennsylvania, a member of Joe Biden's coronavirus task force, and author
of the forthcoming book ``Which Country has the World's Best Health
Care?'' ** Paul Offit is professor of pediatrics at the University of
Pennsylvania, co-inventor of the rotavirus vaccine and author of
``Overkill: When Modern Medicine Goes Too Far.''

\emph{The Times is committed to publishing}
\href{https://www.nytimes.com/2019/01/31/opinion/letters/letters-to-editor-new-york-times-women.html}{\emph{a
diversity of letters}} \emph{to the editor. We'd like to hear what you
think about this or any of our articles. Here are some}
\href{https://help.nytimes.com/hc/en-us/articles/115014925288-How-to-submit-a-letter-to-the-editor}{\emph{tips}}\emph{.
And here's our email:}
\href{mailto:letters@nytimes.com}{\emph{letters@nytimes.com}}\emph{.}

\emph{Follow The New York Times Opinion section on}
\href{https://www.facebook.com/nytopinion}{\emph{Facebook}}\emph{,}
\href{http://twitter.com/NYTOpinion}{\emph{Twitter (@NYTopinion)}}
\emph{and}
\href{https://www.instagram.com/nytopinion/}{\emph{Instagram}}\emph{.}

Advertisement

\protect\hyperlink{after-bottom}{Continue reading the main story}

\hypertarget{site-index}{%
\subsection{Site Index}\label{site-index}}

\hypertarget{site-information-navigation}{%
\subsection{Site Information
Navigation}\label{site-information-navigation}}

\begin{itemize}
\tightlist
\item
  \href{https://help.nytimes.com/hc/en-us/articles/115014792127-Copyright-notice}{©~2020~The
  New York Times Company}
\end{itemize}

\begin{itemize}
\tightlist
\item
  \href{https://www.nytco.com/}{NYTCo}
\item
  \href{https://help.nytimes.com/hc/en-us/articles/115015385887-Contact-Us}{Contact
  Us}
\item
  \href{https://www.nytco.com/careers/}{Work with us}
\item
  \href{https://nytmediakit.com/}{Advertise}
\item
  \href{http://www.tbrandstudio.com/}{T Brand Studio}
\item
  \href{https://www.nytimes.com/privacy/cookie-policy\#how-do-i-manage-trackers}{Your
  Ad Choices}
\item
  \href{https://www.nytimes.com/privacy}{Privacy}
\item
  \href{https://help.nytimes.com/hc/en-us/articles/115014893428-Terms-of-service}{Terms
  of Service}
\item
  \href{https://help.nytimes.com/hc/en-us/articles/115014893968-Terms-of-sale}{Terms
  of Sale}
\item
  \href{https://spiderbites.nytimes.com}{Site Map}
\item
  \href{https://help.nytimes.com/hc/en-us}{Help}
\item
  \href{https://www.nytimes.com/subscription?campaignId=37WXW}{Subscriptions}
\end{itemize}
