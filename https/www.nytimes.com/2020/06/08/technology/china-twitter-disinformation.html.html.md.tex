Sections

SEARCH

\protect\hyperlink{site-content}{Skip to
content}\protect\hyperlink{site-index}{Skip to site index}

\href{https://www.nytimes.com/section/technology}{Technology}

\href{https://myaccount.nytimes.com/auth/login?response_type=cookie\&client_id=vi}{}

\href{https://www.nytimes.com/section/todayspaper}{Today's Paper}

\href{/section/technology}{Technology}\textbar{}Behind China's Twitter
Campaign, a Murky Supporting Chorus

\url{https://nyti.ms/2Ydw0oC}

\begin{itemize}
\item
\item
\item
\item
\item
\item
\end{itemize}

\href{https://www.nytimes.com/news-event/coronavirus?action=click\&pgtype=Article\&state=default\&region=TOP_BANNER\&context=storylines_menu}{The
Coronavirus Outbreak}

\begin{itemize}
\tightlist
\item
  live\href{https://www.nytimes.com/2020/08/01/world/coronavirus-covid-19.html?action=click\&pgtype=Article\&state=default\&region=TOP_BANNER\&context=storylines_menu}{Latest
  Updates}
\item
  \href{https://www.nytimes.com/interactive/2020/us/coronavirus-us-cases.html?action=click\&pgtype=Article\&state=default\&region=TOP_BANNER\&context=storylines_menu}{Maps
  and Cases}
\item
  \href{https://www.nytimes.com/interactive/2020/science/coronavirus-vaccine-tracker.html?action=click\&pgtype=Article\&state=default\&region=TOP_BANNER\&context=storylines_menu}{Vaccine
  Tracker}
\item
  \href{https://www.nytimes.com/interactive/2020/07/29/us/schools-reopening-coronavirus.html?action=click\&pgtype=Article\&state=default\&region=TOP_BANNER\&context=storylines_menu}{What
  School May Look Like}
\item
  \href{https://www.nytimes.com/live/2020/07/31/business/stock-market-today-coronavirus?action=click\&pgtype=Article\&state=default\&region=TOP_BANNER\&context=storylines_menu}{Economy}
\end{itemize}

Advertisement

\protect\hyperlink{after-top}{Continue reading the main story}

Supported by

\protect\hyperlink{after-sponsor}{Continue reading the main story}

\hypertarget{behind-chinas-twitter-campaign-a-murky-supporting-chorus}{%
\section{Behind China's Twitter Campaign, a Murky Supporting
Chorus}\label{behind-chinas-twitter-campaign-a-murky-supporting-chorus}}

Swarms of accounts are amplifying Beijing's brash new messaging as the
country tries to shape the global narrative about the coronavirus and
much else.

\includegraphics{https://static01.nyt.com/images/2020/06/04/business/00china-tweet/00china-tweet-articleLarge.jpg?quality=75\&auto=webp\&disable=upscale}

By \href{https://www.nytimes.com/by/raymond-zhong}{Raymond Zhong}, Aaron
Krolik, \href{https://www.nytimes.com/by/paul-mozur}{Paul Mozur},
\href{https://www.nytimes.com/by/ronen-bergman}{Ronen Bergman} and
\href{https://www.nytimes.com/by/edward-wong}{Edward Wong}

\begin{itemize}
\item
  Published June 8, 2020Updated June 10, 2020
\item
  \begin{itemize}
  \item
  \item
  \item
  \item
  \item
  \item
  \end{itemize}
\end{itemize}

\href{https://cn.nytimes.com/technology/20200609/china-twitter-disinformation/}{阅读简体中文版}\href{https://cn.nytimes.com/technology/20200609/china-twitter-disinformation/zh}{閱讀繁體中文版}

As the Trump administration lashes out at China over a range of
grievances, Beijing's top diplomats and representatives are using the
president's favorite online megaphone --- Twitter --- to slap back with
a pugnaciousness that is best described as Trumpian.

Behind China's combative new messengers, a murky hallelujah chorus of
sympathetic accounts has emerged to repost them and cheer them on. Many
are new to the platform. Some do little else but amplify the Beijing
line.

No doubt some of these accounts are run by patriotic, tech-savvy Chinese
people who get around their government's ban on Twitter and other
Western platforms. But an analysis by The New York Times found that many
of the accounts behaved with a single-mindedness that could suggest a
coordinated campaign of the type that nation states have carried out on
Twitter in the past.

Of the roughly 4,600 accounts that reposted China's leading envoys and
state-run news outlets during a recent week, many acted suspiciously,
The Times found. One in six tweeted with extremely high frequency
despite having few followers, as if they were being used as
loudspeakers, not as sharing platforms.

Nearly one in seven tweeted almost nothing of their own, instead filling
their feeds with reposts of the official Chinese accounts and others.

In all, one third of the accounts had been created in the last three
months, as the war of words with the Trump administration heated up. One
in seven had zero followers.

The United States and China are battling to dominate the global
narrative. China was criticized for its
\href{https://www.nytimes.com/2020/02/01/world/asia/china-coronavirus.html}{early
mishandling} of the coronavirus outbreak, but it has regained confidence
as other countries have made their own stumbles. With
\href{https://www.nytimes.com/2020/06/02/world/asia/china-george-floyd.html}{the
United States in turmoil}, upended first by the epidemic and now by
protests, Beijing sees a chance to define itself as a global leader, one
\href{https://www.nytimes.com/2020/05/24/world/asia/china-hong-kong-taiwan.html}{unafraid
to press its interests} in Hong Kong and the region.

\includegraphics{https://static01.nyt.com/images/2020/06/05/business/00chinatweet-3/merlin_172516875_610f8014-7613-4b74-a762-0a750f29698e-articleLarge.jpg?quality=75\&auto=webp\&disable=upscale}

It is far from clear that the Chinese government is behind the swarms of
accounts helping to spread its gospel on Twitter. Online information
campaigns are becoming increasingly sophisticated as malicious actors
get better at disguising their digital activity, security experts say.
They now rarely make telltale mistakes such as using social media
accounts that were all created on the same day, follow one another and
post the same material.

Campaigns are often uncovered one small piece at a time. Twitter has
declared operations to be state-backed after identifying
\href{https://blog.twitter.com/en_us/topics/company/2019/info-ops-disclosure-data-september-2019.html}{as
few as six accounts}.

Much is unknown about China's covert influence activities in particular.
\href{https://www.nytimes.com/interactive/2019/09/18/world/asia/hk-twitter.html}{Twitter
last year suspended more than 200,000 accounts} that it called a
\href{https://www.nytimes.com/2019/08/19/technology/hong-kong-protests-china-disinformation-facebook-twitter.html}{Chinese
state-backed operation} aimed at discrediting Hong Kong's protesters,
though it said little about how it came to that conclusion.

\hypertarget{latest-updates-economy}{%
\section{\texorpdfstring{\href{https://www.nytimes.com/live/2020/07/31/business/stock-market-today-coronavirus?action=click\&pgtype=Article\&state=default\&region=MAIN_CONTENT_1\&context=storylines_live_updates}{Latest
Updates:
Economy}}{Latest Updates: Economy}}\label{latest-updates-economy}}

\href{https://www.nytimes.com/live/2020/07/31/business/stock-market-today-coronavirus?action=click\&pgtype=Article\&state=default\&region=MAIN_CONTENT_1\&context=storylines_live_updates\#kodaks-chief-executive-was-given-stock-options-then-the-share-price-spiked-1000-percent}{34h
ago}

\href{https://www.nytimes.com/live/2020/07/31/business/stock-market-today-coronavirus?action=click\&pgtype=Article\&state=default\&region=MAIN_CONTENT_1\&context=storylines_live_updates\#kodaks-chief-executive-was-given-stock-options-then-the-share-price-spiked-1000-percent}{Kodak's
chief executive was given stock options. Then the share price spiked
1,000 percent.}

\href{https://www.nytimes.com/live/2020/07/31/business/stock-market-today-coronavirus?action=click\&pgtype=Article\&state=default\&region=MAIN_CONTENT_1\&context=storylines_live_updates\#fitch-ratings-downgrades-its-outlook-on-us-debt}{37h
ago}

\href{https://www.nytimes.com/live/2020/07/31/business/stock-market-today-coronavirus?action=click\&pgtype=Article\&state=default\&region=MAIN_CONTENT_1\&context=storylines_live_updates\#fitch-ratings-downgrades-its-outlook-on-us-debt}{Fitch
Ratings downgrades its outlook on U.S. debt.}

\href{https://www.nytimes.com/live/2020/07/31/business/stock-market-today-coronavirus?action=click\&pgtype=Article\&state=default\&region=MAIN_CONTENT_1\&context=storylines_live_updates\#us-sanctions-more-chinese-officials-over-human-rights-violations-as-tensions-flare}{44h
ago}

\href{https://www.nytimes.com/live/2020/07/31/business/stock-market-today-coronavirus?action=click\&pgtype=Article\&state=default\&region=MAIN_CONTENT_1\&context=storylines_live_updates\#us-sanctions-more-chinese-officials-over-human-rights-violations-as-tensions-flare}{U.S.
sanctions more Chinese officials over human rights violations as
tensions flare}

\href{https://www.nytimes.com/live/2020/07/31/business/stock-market-today-coronavirus?action=click\&pgtype=Article\&state=default\&region=MAIN_CONTENT_1\&context=storylines_live_updates}{See
more updates}

More live coverage:
\href{https://www.nytimes.com/2020/08/01/world/coronavirus-covid-19.html?action=click\&pgtype=Article\&state=default\&region=MAIN_CONTENT_1\&context=storylines_live_updates}{Global}

Still, The Times's findings add to other recent evidence suggesting that
Twitter is being manipulated to amplify pro-Beijing messages. Next Dim,
a data firm in Israel, discovered two mundane-looking tweets praising
China's coronavirus response that were liked and reposted hundreds of
thousands of times in March, possibly with the help of strategically
placed influencer accounts.

The U.S. State Department found inauthentic-seeming accounts that in
April cited
\href{https://www.cam.ac.uk/research/news/covid-19-genetic-network-analysis-provides-snapshot-of-pandemic-origins}{a
Cambridge University study} to raise doubts that the coronavirus
originated in China. The most active of these accounts referred to the
study in scores of tweets, even though the study's lead author dismissed
that interpretation of its findings.

Neither Next Dim's findings nor the State Department's have been
previously reported.

``Improving the health of the public conversation is a priority for our
company,'' Twitter said in a statement. ``Platform manipulation,
including spam and other attempts to undermine the public conversation,
is a violation of the Twitter Rules.''

The State Department has
\href{https://www.nytimes.com/2020/03/28/us/politics/china-russia-coronavirus-disinformation.html}{denounced
China's efforts} to burnish its image and drown out criticism during the
pandemic, comparing them to Russia's disinformation campaigns. Both
countries are using a range of tools to ``shape and tilt any given
information environment in their favor,'' said Lea Gabrielle,
coordinator of the department's Global Engagement Center.

``I think the Chinese Communist Party is still trying to define its
relationship with Twitter,'' said
\href{https://www.cnas.org/publications/reports/dangerous-synergies}{Kristine
Lee}, a fellow at the Center for a New American Security. ``But the
Covid-19 pandemic has served as an important period of
experimentation.''

The U.S.-China tongue-lashing adds to the questions vexing Twitter about
\href{https://www.nytimes.com/2020/05/30/technology/twitter-trump-dorsey.html}{how
it treats inflammatory or misleading remarks} from world leaders. Mr.
Trump has
\href{https://twitter.com/realdonaldtrump/status/1266326065833824257}{accused
the company} of censoring him and other Republicans while ignoring
questionable posts by Democrats and the Chinese government.

Beijing's Twitter brigade includes Hua Chunying, the head of the foreign
ministry's information department. Since joining the platform in
October, Ms. Hua has attracted more than half a million followers with
her
\href{https://twitter.com/SpokespersonCHN/status/1266741986096107520}{feisty
put-downs} of the United States.

In \href{https://www.thepaper.cn/newsDetail_forward_3900567}{a Communist
Party journal} last year, Ms. Hua wrote that China had to find a voice
in international affairs that was commensurate with its economic
strength. ``We have walked closer to the center of the world stage than
ever before, but we still do not grasp the microphone completely in our
hands,'' she wrote.

One reason, she wrote: a lack of ``fighting spirit.''

Another foreign ministry spokesman, Zhao Lijian,
\href{https://twitter.com/zlj517/status/1238111898828066823}{became
notorious after tweeting} that the U.S. military might have brought the
coronavirus to China. Twitter later added a
\href{https://www.nytimes.com/2020/05/28/technology/trump-twitter-fact-check.html}{fact-checking
label} to Mr. Zhao's post.

The Times analyzed all of the tweets that Ms. Hua, Mr. Zhao and 12 other
Twitter users linked to the Chinese government posted between May 18 and
May 25.

The other users included the foreign ministry's main account, as well as
the accounts of China's ambassadors to the United States and Britain.
They also included nine accounts run by state news outlets.

That week, Beijing moved to tighten its
\href{https://www.nytimes.com/2020/05/21/world/asia/hong-kong-china.html}{control
over Hong Kong}. Mr. Trump threatened to cut off
\href{https://www.nytimes.com/2020/05/18/health/coronavirus-who-china-trump.html}{funding
to the World Health Organization}. American officials congratulated
Taiwan's president on the start of her second term. China, which claims
Taiwan as its territory,
\href{http://www.xinhuanet.com/english/2020-05/20/c_139072817.htm}{was
furious}.

Ms. Hua mused about
\href{https://twitter.com/SpokespersonCHN/status/1263053023137263616}{whether
the coronavirus actually originated in the United States}: ``Scientists
at the US NIH began developing a \#COVID19 vaccine on January 11. There
were reports of cases as early as November last year. Any explanation or
investigation?'' Her post, which refers to the National Institutes of
Health, was liked 4,600 times.

The Times's analysis found that hundreds of the 4,600 accounts that
reposted the Chinese government mouthpieces that week behaved
suspiciously. Many were incessant tweeters despite having limited
followings. After excluding accounts that had zero followers and had
tweeted five times or fewer, over a sixth of the accounts had posted 100
or more times for every follower.

A few accounts repeatedly retweeted at set lengths of time after the
original post --- 9 hours and 49 minutes after, 19 hours and 34 minutes
after --- suggesting that software had been used to schedule their
tweets. Twitter has since suspended some of the accounts for violating
its policies.

When contacted by The Times, several pro-China Twitter users denied
being part of a government campaign but acknowledged that they joined
the platform specifically to follow the foreign ministry
representatives. The ministry did not respond to a request for comment.

Image

Zhao Lijian, a Chinese foreign ministry spokesman, has gained notoriety
for his pugnacious tweets.Credit...Wu Hong/EPA, via Shutterstock

Others said they were either curious about Mr. Trump's tweets about
China or felt demonized by them.

``He has done so many shameless things for re-election,'' one user,
@beautifullady76, said in a Twitter message. ``Countless Chinese people
are angry and everyone has the right to the truth. We just want to say a
fair word for China!''

Public records show that Beijing is trying to expand its influence on
the Western internet. China's internet regulator has sought out
contractors to help it ``make use of overseas social media platforms to
develop online propaganda on major themes,'' procurement documents show.

Much of this kind of activity may not appear in official documents,
however. The regulator did not respond to a request for comment.

``There's no reason to think that the parts of the Chinese government
that are formally in charge of conducting information operations are not
able to conduct operations that are as sophisticated as others','' said
Camille François of the network analysis company Graphika. ``They just
haven't been publicly exposed and dissected yet.''

Researchers remain on the lookout. ProPublica tracked
\href{https://www.propublica.org/article/how-china-built-a-twitter-propaganda-machine-then-let-it-loose-on-coronavirus}{10,000
fishy accounts} that posted about the coronavirus and the Hong Kong
protests. Alkemy, an Italian digital marketing firm,
\href{https://formiche.net/2020/03/china-unleashed-twitter-bots-covid19-propaganda-italy/}{found
that inauthentic-looking users} were behind many posts celebrating
Chinese medical aid to Italy.

In March, two tweets lauding China's handling of the outbreak were liked
and reposted hundreds of thousands of times. The posts were not
shocking, funny or newsworthy, and originated from users with modest
followings.

That caught the attention of Next Dim, an Israeli company that uses
network analytics to identify and prevent financial crime.

``While scanning Twitter, our systems automatically discovered a huge
irregularity,'' said Next Dim's chief executive, Netta Marrom. Too huge,
he believes, to be the result of chance.

On March 12, the first user,
\href{https://twitter.com/manisha_kataki/status/1238007207700180992}{@manisha\_kataki},
posted a video showing workers disinfecting streets in China. ``At this
rate, China will be back in action very soon, may be much faster than
the world expects,'' the user wrote.

The next day, another user, @Ejiketion, retweeted the post, marveling at
how China had locked down cities and built coronavirus hospitals. In the
West, by contrast, ``We washing our hands LOL,'' @Ejiketion wrote. The
account has since been deleted.

The two posts together received more than 382,000 retweets and 1.1
million likes, many of them within the first two days. That made them
roughly as popular as
\href{https://twitter.com/elonmusk/status/1236029449042198528}{Elon
Musk's tweet}, also from March, in which the head of Tesla called the
coronavirus panic ``dumb.''

Two other posts that also retweeted @manisha\_kataki but translated
@Ejiketion's comment into
\href{https://twitter.com/andruklepe/status/1238573314534264838}{Spanish}
and
\href{https://twitter.com/buenvcosv/status/1238927468041588737}{French}
received a combined 67,000 retweets and 181,000 likes.

Next Dim identified around 20 Twitter users whose followers accounted
for thousands of the retweets of @manisha\_kataki's and @Ejiketion's
posts. Some of these users had immense followings but rarely tweeted
about China.

Image

Tiananmen Square in Beijing during the annual meeting of China's
legislature last month.Credit...Roman Pilipey/EPA, via Shutterstock

Next Dim's analysis uncovered other signs that the two tweets'
popularity may not have been organic. Few of the first users to retweet
@manisha\_kataki's post were followers of the account, which means they
were unlikely to have seen the tweet in their timelines. Thousands of
accounts reposted both tweets, even though @Ejiketion's tweet was itself
a repost of @manisha\_kataki's.

Neither @manisha\_kataki nor @Ejiketion responded to requests for
comment.

Wang Yiwei and Lin Qiqing contributed research.

Advertisement

\protect\hyperlink{after-bottom}{Continue reading the main story}

\hypertarget{site-index}{%
\subsection{Site Index}\label{site-index}}

\hypertarget{site-information-navigation}{%
\subsection{Site Information
Navigation}\label{site-information-navigation}}

\begin{itemize}
\tightlist
\item
  \href{https://help.nytimes.com/hc/en-us/articles/115014792127-Copyright-notice}{©~2020~The
  New York Times Company}
\end{itemize}

\begin{itemize}
\tightlist
\item
  \href{https://www.nytco.com/}{NYTCo}
\item
  \href{https://help.nytimes.com/hc/en-us/articles/115015385887-Contact-Us}{Contact
  Us}
\item
  \href{https://www.nytco.com/careers/}{Work with us}
\item
  \href{https://nytmediakit.com/}{Advertise}
\item
  \href{http://www.tbrandstudio.com/}{T Brand Studio}
\item
  \href{https://www.nytimes.com/privacy/cookie-policy\#how-do-i-manage-trackers}{Your
  Ad Choices}
\item
  \href{https://www.nytimes.com/privacy}{Privacy}
\item
  \href{https://help.nytimes.com/hc/en-us/articles/115014893428-Terms-of-service}{Terms
  of Service}
\item
  \href{https://help.nytimes.com/hc/en-us/articles/115014893968-Terms-of-sale}{Terms
  of Sale}
\item
  \href{https://spiderbites.nytimes.com}{Site Map}
\item
  \href{https://help.nytimes.com/hc/en-us}{Help}
\item
  \href{https://www.nytimes.com/subscription?campaignId=37WXW}{Subscriptions}
\end{itemize}
