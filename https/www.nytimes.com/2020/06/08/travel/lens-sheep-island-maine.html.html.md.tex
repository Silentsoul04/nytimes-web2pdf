\href{/section/travel}{Travel}\textbar{}Shearing Sheep, and Hewing to
Tradition, on an Island in Maine

\url{https://nyti.ms/30lIAVw}

\begin{itemize}
\item
\item
\item
\item
\item
\item
\end{itemize}

\includegraphics{https://static01.nyt.com/images/2020/06/08/travel/08travel-sheep-22/merlin_173130474_f1fc3a8a-5bb7-4387-8770-c590148c063b-articleLarge.jpg?quality=75\&auto=webp\&disable=upscale}

Sections

\protect\hyperlink{site-content}{Skip to
content}\protect\hyperlink{site-index}{Skip to site index}

The World Through a Lens

\hypertarget{shearing-sheep-and-hewing-to-tradition-on-an-island-in-maine}{%
\section{Shearing Sheep, and Hewing to Tradition, on an Island in
Maine}\label{shearing-sheep-and-hewing-to-tradition-on-an-island-in-maine}}

In a remote area of Maine, the Wakeman family maintains the traditions
of island shepherding, the cycles of which have been largely unchanged
for centuries.

Grant Estell carries an armful of wool away from a sorting
table.Credit...

Supported by

\protect\hyperlink{after-sponsor}{Continue reading the main story}

Photographs by Greta Rybus

Text by Galen Koch and Greta Rybus

\begin{itemize}
\item
  June 8, 2020
\item
  \begin{itemize}
  \item
  \item
  \item
  \item
  \item
  \item
  \end{itemize}
\end{itemize}

\emph{With travel restrictions in place worldwide, we've launched a new
series,}
\href{https://www.nytimes.com/column/the-world-through-a-lens}{\emph{The
World Through a Lens}}\emph{, in which photojournalists help transport
you, virtually, to some of our planet's most beautiful and intriguing
places. This week, Greta Rybus shares a collection of photographs from a
set of islands in Maine.}

\begin{center}\rule{0.5\linewidth}{\linethickness}\end{center}

Three miles off the coast of Maine, in a remote area northeast of Acadia
National Park, lies a cluster of islands --- including Little Nash
Island, Big Nash Island and Flat Island --- populated only by sheep.

The Wakeman family, who live on the nearby mainland, are the year-round
caretakers. Alfie Wakeman works full-time as a pediatric provider in the
local clinic. His wife, Eleni, works full-time as a speech-language
pathologist and the assistant fire chief for the local volunteer fire
department. Their three daughters --- Wren, Lilly and Evie --- are all
college-age or newly graduated.

\includegraphics{https://static01.nyt.com/images/2020/06/08/travel/08travel-sheep-16/merlin_173131266_edce7bf3-1b42-46a2-8ace-98201d14a3ac-articleLarge.jpg?quality=75\&auto=webp\&disable=upscale}

Image

Evie Wakeman cares for a lamb named Pinney. Lambs may be brought in for
additional care if their mother rejects them or isn't able to feed them.
A lamb that needs additional care will be bottle fed, warmed and
reintroduced to its mother. If reintroduction isn't successful, the lamb
will become a ``dooryard sheep'' at the Wakeman household.

Each spring, Alfie leaves his medical practice for three weeks to live
on Big Nash Island for the lambing season. (In his text messages, Alfie
includes smiley faces when he talks about going to the island, or about
new lambs; sad faces punctuate his texts when he discusses leaving the
island.) The sheep, wild and self-sufficient, are able to thrive off the
providence of the island. But every so often a sick lamb needs special
care.

Image

Alfie Wakeman. ``It feels like I do bit of everything,'' he said.
``Today I'm a farmer, tomorrow I'll be a fishermen, then Monday I'll be
in the O.R. assisting surgery.''

Image

Eleni Wakeman. ``It's about realizing that everything has value and
everything has beauty,'' she said. ``You're rooting for this baby lamb,
and if it dies, it's heartbreaking --- but there's something really
beautiful in what you did and how you connected.''

About a century ago, a 10-year-old girl named Jenny Cirone --- the
daughter of the lighthouse keeper on
\href{https://www.lighthousefriends.com/light.asp?ID=765}{Little Nash
Island} --- began raising sheep. She would go on to tend her flock for
more than 80 years.

Alfie, Eleni and their daughters knew Jenny well. They lived next door
to her and helped her care for the island and its sheep. They still
understand this part of the world largely through Jenny and her stories.

Image

Lambs scattered on the island.

Image

Historical evidence suggests that sheep have been raised on Big Nash
Island for more than 300 years. The current flock has roots going back
about a century.

Image

Evie Wakeman holds a lamb after it's been checked and tended to. The
sheep are wild and typically only have close human contact twice a year,
during the spring and fall roundup.

Jenny knew everything about the island and the ocean around it. She
hauled lobster traps with Alfie almost until the day she died, a month
shy of 92. She remembered each sheep, its lineage, how much wool it
made. She gave names to every lamb, and to each spot on the ocean floor
that was good for a lobster trap.

Image

During the fall roundup, the flock is culled, and selected sheep are
pulled from the herd to be brought to the mainland and slaughtered.

Image

``The island wool is so magnificent,'' said Geri Valentine, a shearer.
``It's this clean, lustrous, shiny beautiful wool. For hand spinning,
it's like a dream.''

At the end of lambing season, a community gathers on Big Nash to help
round up and shear the sheep. (The other islands' sheep will be sheared,
too, but those require smaller crews.) The volunteers --- around 40
people --- include a handful of knitters and spinners; they often wear
sweaters made of \href{https://www.starcroftfiber.com/yarn/}{Nash Island
wool}. Some show up because they live down the road and are accustomed
to pitching in. Others are lured by an adoration of good wool. Still
others come because of the island itself --- for the tradition, for the
memory of Jenny.

Image

Alfie Wakeman spends about a month on the island each year for lambing,
although he is often joined by others. This year he is accompanied by
his daughter, Evie, a freshman in college. (Her courses are all online
due to the pandemic.)

Image

The Wakeman family doesn't use dogs to round up sheep. Instead they rely
on a community. As people work to gather the sheep, they must remain
vigilant: If one sheep bolts in the wrong direction, all might follow.

Image

A lamb nestles into one of the island's hillocks.

Before they're sheared, the sheep must be rounded up --- a process that
requires considerable patience. Around 20 people sweep the island
methodically; no animals can be left behind on the little hills or rocky
beaches, and the sheep shouldn't be spooked. (Sheep are notoriously
skittish.) Everyone joins in --- their arms outstretched, their hands
sometimes clasped together --- as they funnel the sheep toward a corral
made of salvaged driftwood.

Image

Sheep being funneled toward a corral.

Image

The group discusses logistics before the shearing.

When the corral is full, the crew works to pull lambs from underneath
the sheep, moving them to a separate pen; there, the rams are castrated
and the ewes' tails are docked. Each lamb and sheep is carefully checked
and given any necessary care. Meanwhile, the shearers skim whirring
blades along the bodies of the sheep, their hands and the clippers
hidden under the thick wool. (Much of shearing is done blindly, by
feel.)

Image

Alfie Wakeman pets a sheep as it awaits shearing.

Image

Like barn raising, community shearing is a tradition whose success
depends on the coordinated effort of a group of people.

The work is physically demanding, but the shearers move quickly, often
without pausing for food or water. After hours of labor, and once the
last sheep's wool has been removed, the shearers return their tools to
their cases, the blades slick with lanolin, and the group migrates to a
cabin --- Big Nash's lone building --- for a potluck meal: baked beans,
a salad, turkey, rhubarb cakes. A simple rule is announced: ``The
shearers eat first.''

Image

Wren Wakeman puts away her clippers. The addition of electric shears has
been one of the only changes to the shearing process in more than a
century.~

Image

After shearing, the wool is sorted, assessed and cleaned by hand. Every
burr and bit of grass is picked out.

Image

On community work days, food is brought and shared by everyone. Many
volunteers are part of the local farming community and bring food made
from locally grown ingredients.

Another roundup will happen again in the fall: The sheep will be
gathered, checked and tended to. Some will stay on the island, growing
thick with wool, while most of the males and a handful of ewes will be
brought to the mainland to be processed as meat.

Image

Sheep aboard the Wakemans' boat.

Image

During lambing season, Alfie Wakeman tends to the lambs on several
islands.

Image

Pinney the lamb goes for a ride on a boat. Some lambs need additional
care, and are kept close for frequent bottle feedings. To transport them
on and off the boat, they are often snuggled into a bucket.

The sheep chosen for slaughter will be scooped up, their soft woolen
bodies carried from the driftwood pen, down the rocky beach, to a
dinghy. Then, from the dinghy to the family's lobster boat, until sheep
are packed from bulkhead to transom, calm and blinking in the sun.
Volunteers will sit on the sides of the boat or climb onto its top as it
motors back to the mainland. A waiting truck will bring the sheep to the
local butcher.

Image

Aboard the Wakemans' lobster boat.

Maine was once a land of shepherds. Its islands and coastal communities
were dotted with the fleeced bodies of sheep, its shrubs and trees
grazed into oblivion. Historical photos show wide expanses of pasture
that have now become thick with forests and houses.

Back then, there were more families like the Wakemans, who raised their
own animals and grew their own food, who gathered people together to
share both their work and a meal, who used dark humor and whispered
their thanks on the days when animals gave up their wool or became food.

Image

Many sheep will complete their life cycle on the island, their bones and
bodies becoming part of the island itself.

Some of the sheep spend their entire lives on these islands, from birth
to death. They \emph{become} the islands. Their sun-bleached bones are
entrenched in the earth, embedded in the grassy knolls and wetlands
where they once grazed, their bodies decomposing to nourish a new
generation.

Image

The island in the wake of the Wakemans' lobster boat.

Image

In the distance: the Nash Island lighthouse.

Jenny Cirone is also here; her gravestone sits at the far end of Big
Nash, her ashes buried in the place with the best view of the
lighthouse. She, too, is a part of the island --- the grass, the sea,
the sheep, the story.

\begin{center}\rule{0.5\linewidth}{\linethickness}\end{center}

\href{http://www.gretarybus.com/}{\emph{Greta Rybus}} \emph{is a
photojournalist based in Portland, Maine. You can follow her work on}
\href{https://www.instagram.com/gretarybus/}{\emph{Instagram}}\emph{.}

\href{http://www.galenkoch.com/}{\emph{Galen Koch}} \emph{is a
documentarian and radio journalist based in Scarborough, Maine. You can
follow her project,} \href{http://www.thefirstcoast.org/}{\emph{The
First Coast}}\emph{, on}
\href{https://www.instagram.com/thefirstcoast/?hl=en}{\emph{Instagram}}\emph{.}

\emph{\textbf{Follow New York Times Travel}} \emph{on}
\href{https://www.instagram.com/nytimestravel/}{\emph{Instagram}}\emph{,}
\href{https://twitter.com/nytimestravel}{\emph{Twitter}} \emph{and}
\href{https://www.facebook.com/nytimestravel/}{\emph{Facebook}}\emph{.
And}
\href{https://www.nytimes.com/newsletters/traveldispatch}{\emph{sign up
for our weekly Travel Dispatch newsletter}} \emph{to receive expert tips
on traveling smarter and inspiration for your next vacation.}

Advertisement

\protect\hyperlink{after-bottom}{Continue reading the main story}

\hypertarget{site-index}{%
\subsection{Site Index}\label{site-index}}

\hypertarget{site-information-navigation}{%
\subsection{Site Information
Navigation}\label{site-information-navigation}}

\begin{itemize}
\tightlist
\item
  \href{https://help.nytimes.com/hc/en-us/articles/115014792127-Copyright-notice}{©~2020~The
  New York Times Company}
\end{itemize}

\begin{itemize}
\tightlist
\item
  \href{https://www.nytco.com/}{NYTCo}
\item
  \href{https://help.nytimes.com/hc/en-us/articles/115015385887-Contact-Us}{Contact
  Us}
\item
  \href{https://www.nytco.com/careers/}{Work with us}
\item
  \href{https://nytmediakit.com/}{Advertise}
\item
  \href{http://www.tbrandstudio.com/}{T Brand Studio}
\item
  \href{https://www.nytimes.com/privacy/cookie-policy\#how-do-i-manage-trackers}{Your
  Ad Choices}
\item
  \href{https://www.nytimes.com/privacy}{Privacy}
\item
  \href{https://help.nytimes.com/hc/en-us/articles/115014893428-Terms-of-service}{Terms
  of Service}
\item
  \href{https://help.nytimes.com/hc/en-us/articles/115014893968-Terms-of-sale}{Terms
  of Sale}
\item
  \href{https://spiderbites.nytimes.com}{Site Map}
\item
  \href{https://help.nytimes.com/hc/en-us}{Help}
\item
  \href{https://www.nytimes.com/subscription?campaignId=37WXW}{Subscriptions}
\end{itemize}
