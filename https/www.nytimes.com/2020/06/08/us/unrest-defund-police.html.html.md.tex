Sections

SEARCH

\protect\hyperlink{site-content}{Skip to
content}\protect\hyperlink{site-index}{Skip to site index}

\href{https://www.nytimes.com/section/us}{U.S.}

\href{https://myaccount.nytimes.com/auth/login?response_type=cookie\&client_id=vi}{}

\href{https://www.nytimes.com/section/todayspaper}{Today's Paper}

\href{/section/us}{U.S.}\textbar{}After Protests, Politicians Reconsider
Police Budgets and Discipline

\url{https://nyti.ms/30o0CGI}

\begin{itemize}
\item
\item
\item
\item
\item
\item
\end{itemize}

\href{https://www.nytimes.com/news-event/george-floyd-protests-minneapolis-new-york-los-angeles?action=click\&pgtype=Article\&state=default\&region=TOP_BANNER\&context=storylines_menu}{Race
and America}

\begin{itemize}
\tightlist
\item
  \href{https://www.nytimes.com/interactive/2020/07/03/us/george-floyd-protests-crowd-size.html?action=click\&pgtype=Article\&state=default\&region=TOP_BANNER\&context=storylines_menu}{Black
  Lives Matter Movement}
\item
  \href{https://www.nytimes.com/interactive/2020/06/28/us/i-cant-breathe-police-arrest.html?action=click\&pgtype=Article\&state=default\&region=TOP_BANNER\&context=storylines_menu}{History
  of `I Can't Breathe'}
\item
  \href{https://www.nytimes.com/interactive/2020/06/10/upshot/black-lives-matter-attitudes.html?action=click\&pgtype=Article\&state=default\&region=TOP_BANNER\&context=storylines_menu}{How
  Public Opinion Shifted}
\item
  \href{https://www.nytimes.com/interactive/2020/07/16/us/black-lives-matter-protests-louisville-breonna-taylor.html?action=click\&pgtype=Article\&state=default\&region=TOP_BANNER\&context=storylines_menu}{45
  Days in Louisville}
\end{itemize}

Advertisement

\protect\hyperlink{after-top}{Continue reading the main story}

Supported by

\protect\hyperlink{after-sponsor}{Continue reading the main story}

\hypertarget{after-protests-politicians-reconsider-police-budgets-and-discipline}{%
\section{After Protests, Politicians Reconsider Police Budgets and
Discipline}\label{after-protests-politicians-reconsider-police-budgets-and-discipline}}

Elected officials are exploring changes ranging from defunding police
departments to requiring more accountability.

\includegraphics{https://static01.nyt.com/images/2020/06/08/us/08UNREST-DEFUND-wdc/08UNREST-DEFUND-wdc-articleLarge.jpg?quality=75\&auto=webp\&disable=upscale}

By \href{https://www.nytimes.com/by/dionne-searcey}{Dionne Searcey},
\href{https://www.nytimes.com/by/john-eligon}{John Eligon} and
\href{https://www.nytimes.com/by/farah-stockman}{Farah Stockman}

\begin{itemize}
\item
  June 8, 2020
\item
  \begin{itemize}
  \item
  \item
  \item
  \item
  \item
  \item
  \end{itemize}
\end{itemize}

MINNEAPOLIS --- In an abrupt change of course, the mayor of New York
vowed to cut the budget of the nation's largest police force. In Los
Angeles, the mayor called for redirecting millions of dollars from
policing after protesters gathered outside his home. And in Minneapolis,
City Council members pledged to dismantle their
\href{https://www.nytimes.com/2020/06/09/us/ca-defund-police.html}{police}
force and completely reinvent how public safety is handled. As tens of
thousands of people have demonstrated against police violence over the
past two weeks, calls have emerged in cities across the country for
fundamental changes to American policing.

The pleas for change have taken a variety of forms --- including
measures to restrict police use of military-style equipment and efforts
to require officers to face strict discipline in cases of misconduct.
Parks, universities and schools have distanced themselves from local
police departments, severing contracts. In some places, the calls for
change have gone still further, aiming to abolish police departments,
shift police funds into social services or
\href{https://www.nytimes.com/2020/06/09/us/ca-defund-police.html}{defund
police} departments partly or entirely.

``It is a critical time that we can see concrete change,'' said the Rev.
Al Sharpton, who last week addressed the crowd gathered for
\href{https://www.nytimes.com/2020/06/08/us/george-floyd-viewing-funeral-houston-unrest.html}{a
memorial service for George Floyd}, the black man who died after a white
police officer pressed his knee into his neck for nearly nine minutes in
Minneapolis last month. ``The legislation and the policy changes will be
the ones that determine the victory of this movement.''

\includegraphics{https://static01.nyt.com/images/2017/01/29/podcasts/the-daily-album-art/the-daily-album-art-articleInline-v2.jpg?quality=75\&auto=webp\&disable=upscale}

\hypertarget{listen-to-the-daily-the-case-for-defunding-the-police}{%
\subsubsection{Listen to `The Daily': The Case for Defunding the
Police}\label{listen-to-the-daily-the-case-for-defunding-the-police}}

Protesters across the country are calling for the abolition of policing.
But what would that actually look like?

transcript

Back to The Daily

bars

0:00/25:49

-25:49

transcript

\hypertarget{listen-to-the-daily-the-case-for-defunding-the-police-1}{%
\subsection{Listen to `The Daily': The Case for Defunding the
Police}\label{listen-to-the-daily-the-case-for-defunding-the-police-1}}

\hypertarget{hosted-by-michael-barbaro-produced-by-rachel-quester-luke-vander-ploeg-and-asthaa-chaturvedi-with-help-from-annie-brown-and-edited-by-mj-davis-lin-and-lisa-chow}{%
\subsubsection{Hosted by Michael Barbaro; produced by Rachel Quester,
Luke Vander Ploeg and Asthaa Chaturvedi; with help from Annie Brown; and
edited by M.J. Davis Lin and Lisa
Chow}\label{hosted-by-michael-barbaro-produced-by-rachel-quester-luke-vander-ploeg-and-asthaa-chaturvedi-with-help-from-annie-brown-and-edited-by-mj-davis-lin-and-lisa-chow}}

\hypertarget{protesters-across-the-country-are-calling-for-the-abolition-of-policing-but-what-would-that-actually-look-like}{%
\paragraph{Protesters across the country are calling for the abolition
of policing. But what would that actually look
like?}\label{protesters-across-the-country-are-calling-for-the-abolition-of-policing-but-what-would-that-actually-look-like}}

\begin{itemize}
\item
  michael barbaro\\
  From The New York Times, I'm Michael Barbaro. This is ``The Daily.''
\item
  {[}music{]}\\
  Today, several major U.S. cities are now proposing ways to defund and
  even dismantle their police departments. John Eligon on the thinking
  behind those plans and what they might look like in practice.

  It's Tuesday, June 9.
\item
  archived recording (protestors)\\
  (CHANTING) I can't breathe! I can't breathe!
\end{itemize}

john eligon

In the early days of the protests after George Floyd was killed ---

\begin{itemize}
\tightlist
\item
  archived recording (protestors)\\
  (CHANTING) No justice, no peace! No justice, no peace!
\end{itemize}

john eligon

--- it was just pure emotions and raw rage.

\begin{itemize}
\tightlist
\item
  archived recording (protestors)\\
  {[}EXPLETIVE{]} these racist {[}EXPLETIVE{]} police!
\end{itemize}

john eligon

But pretty soon, once the more fiery protests and fiery unrest died
down, then we started seeing the organizers come in and talking about
what they want. And one thing we quickly saw were these face masks that
people were wearing. They were black, and they had yellow writing on
them. And they said, ``Defund police.''

michael barbaro

Hmm.

john eligon

And from there, you start hearing these calls at protest, at rallies.

\begin{itemize}
\tightlist
\item
  archived recording (protestors)\\
  (CHANTING) Defund the police! Defund the police!
\end{itemize}

john eligon

You start hearing, ``Defund the police.''

You start hearing calls to abolish the police. You start seeing people
waving signs. And it became clear that this was an opening that a lot of
activists saw to take this moment of a very brutal police killing and
turn it into something much larger.

\begin{itemize}
\item
  archived recording (protestor 1)\\
  Do the right thing!
\item
  archived recording (protestor 2)\\
  Defund the police!
\end{itemize}

michael barbaro

So John, what do these concepts --- defund, dismantle, abolish the
police --- what exactly do they mean?

john eligon

To defund, when activists say that, what they mean is taking money away
from the police department's budget and redirect it toward other things
--- whether that be social services, agencies, maybe mental health
agencies --- that can do functions that police are often called on to
do.

michael barbaro

Mm-hmm.

john eligon

But if you fully defund it, you can get to a space where the police
department is abolished. And so essentially, what that means is that
there is no more police department as we know it. You don't call these
men and women in blue shirts to come racing to your door with their guns
in hand. It means that they have to figure out some other form of
providing that public safety, and the police department would not be
that form.

michael barbaro

And where did these concepts come from?

john eligon

Well, at their core, they come from the problems and issues that
especially communities of color, especially black communities, have had
with policing. They see police coming into their communities to
brutalize them, not to protect and serve them. And that has really
influenced this desire to keep the police away, to do something else.
And we've seen, basically, that governments and police forces, they
respond with certain reforms. We've seen efforts for body cameras. We've
seen diversifying the police departments. We've seen changes to the
rules on use of force.

michael barbaro

Mm-hmm.

john eligon

But what became clear to a lot of today's activists, and what they say
explicitly, is that these reforms are not working. If you look at since
Michael Brown was killed in Ferguson, Missouri six years ago, the police
have continued to kill people at high rates, and especially black people
at disproportionately high rates. And so for them, the only solution is
to tear it down and build something new.

michael barbaro

So John, what might it actually look like in practice to defund or
abolish a police department?

john eligon

So for instance, if someone is homeless and they're struggling on the
streets, a person can call 9-1-1, and instead of an armed police officer
being sent out, perhaps there can be an outreach worker from a homeless
services agency. Or if you have someone having a mental health episode,
then again, you can call 9-1-1, and instead of a police officer, maybe a
health care worker, a mental health worker will come out. And the idea
behind it is to really cut down the interactions between armed police
officers and civilians. And by doing that, the hope is that it will
reduce their conflict and the potential for people getting hurt or
killed by police officers.

michael barbaro

Right. I mean, that makes a certain sense, especially for a community
where there's not a lot of violent crime. But every community is
different, right? And some towns, some cities --- I think about New York
City, for example --- have a significantly higher rate of violent crime
that would seem to require having armed police. So how do activists
think about that?

john eligon

For a lot of the activists that I spoke to, the issue was about
centering public safety on communities. And one activist that I spoke
to, Arianna Nason, she said essentially it's going to be up to each
community to decide what public safety looks like for itself.

\begin{itemize}
\tightlist
\item
  arianna nason\\
  It's going to be up to every community to decide what they need. We
  can't decide that.
\end{itemize}

john eligon

So maybe that's armed patrols. Maybe that's mental health workers. Maybe
that's some sort of mobile units with social workers sitting in it, and
people are trained in using force and different things. One of the big
ideas is this idea of community policing, community watch. And it's
interesting. I had said isn't an issue, though, with community policing
or community patrols, neighborhood watch, that if I walk into that
community, as a black man with dreadlocks, if I walk into one of these
communities, we see what happens with neighborhood watch. We see Trayvon
Martin. We see Ahmaud Arbery.

\begin{itemize}
\tightlist
\item
  john eligon\\
  Should that be a concern, then? I guess with this community-type based
  model that certain people who look a certain way might go into the
  neighborhood, and that community might decide to take it into their
  own hands and then take it overboard, I guess.
\end{itemize}

john eligon

And she took off her sunglasses. She looked at me, and she said ---

\begin{itemize}
\tightlist
\item
  arianna nason\\
  No, I get that. And I'll be really real with you. For me, personally,
  I don't have all the answers for that. I don't. And I wish I did. A
  lot of it is ---
\end{itemize}

john eligon

Honestly, I don't really know the answer to that right now.

michael barbaro

Huh.

john eligon

She was not sure exactly what the answer was. And see, this is all to
say, it's still very tricky and very much a work in progress. But what
she did say is that the current system also is not working for me
either. So it's a matter of what are they going to do differently? And
they believe that something drastically different needs to be done.

michael barbaro

Mm-hmm. As best you can tell, would any of the familiar elements of an
existing police department --- I'm thinking, for the sake of argument,
homicide detectives, special victims units that investigate sexual
assault or rape --- do those remain? Do they take a different form? Do
they adopt a different name? Has that been fleshed out?

john eligon

I would say, no, it's not been fleshed out. Because again, we go back to
the fact that this is not going to be some federal commission, or even
state commission or a city commission for anywhere that's going to come
up with, like, these are the rules for public safety now. And these are
all things that need to be worked out. And I think what people say with
things like homicide investigations, with sex crimes investigations and
things like that, they say a couple of things. One, the police are not
doing a good job at those anyways. So you have lots of cities where the
clearance rate on homicides and other investigations is miserable. You
had, even here in Minneapolis, there was a big scandal with all the rape
kits that they had untested. So they had a lot of issues with crimes
that were not being investigated properly. And then, the second thing
that people say is that those jobs can be taken up by specialized,
trained people. You can build new institutions to do those things that
aren't necessary policing. I did talk to one council member who said,
maybe there's still police, but for very, very limited role, and many of
their responsibilities are farmed out. You know, anything short of some
sort of active violence, you don't need police for. So in some people's
eyes, that would still be a police force. But one thing that the people
who are most ardent about abolishing the police or defunding the police,
even, they make it clear that they don't just want a system in which
it's police in another name, police with another uniform on.

{[}music{]}

And these demands to defund the police, they've actually been brewing in
Minneapolis for several years now. Ever since a police killing back in
2015, there's been several local activist groups working on it. And
those activist groups came together this past weekend in what was
probably the biggest and most clearest demand for defunding the police.

michael barbaro

We'll be right back.

\begin{itemize}
\tightlist
\item
  archived recording (protestors)\\
  (CHANTING) Abolish the M.P.D.!
\end{itemize}

john eligon

So there were hundreds of activists who went and gathered in front of
the mayor Jacob Frey's house.

\begin{itemize}
\item
  archived recording (protestors)\\
  Abolish the M.P.D.! And they had a megaphone. They were chanting. They
  were screaming. And sure enough, the mayor came out to talk to the
  protesters. He kind of made his way through the crowd, walked up to
  the front, and you had one of the lead organizers for a group called
  the Black Visions Collective. Kandace Montgomery, she was standing up
  on a riser there, talking down in a megaphone to the mayor.
\item
  archived recording (kandace montgomery)\\
  Jacob Frey, we have a yes-or-no question for you. Yes or no, will you
  commit to defunding Minneapolis Police Department?
\item
  archived recording (jacob frey)\\
  {[}INAUDIBLE{]}
\end{itemize}

john eligon

And you could tell, like, there is this hesitation because he knows this
is not going to go well, right? You have all these very vocal, very
ardent activists around you who want you to defund the police.

\begin{itemize}
\item
  archived recording (kandace montgomery)\\
  Will you defund the Minneapolis Police Department?
\item
  archived recording\\
  {[}CROWD MURMURING{]}
\item
  archived recording (kandace montgomery)\\
  All right, be quiet y'all. Be quiet, because it's important that we
  actually hear this. It's important that we hear this, because if y'all
  don't know, he's up for re-election next year.
\item
  archived recording\\
  {[}CROWD CHEERING{]}
\end{itemize}

john eligon

And then ---

\begin{itemize}
\item
  archived recording (jacob frey)\\
  I do not support the full abolition of the police department.
\item
  archived recording (kandace montgomery)\\
  All right!
\item
  archived recording (speaker)\\
  You're wasting our time! Get the {[}EXPLETIVE{]} out of here!
\end{itemize}

john eligon

And he gives his answer --- he does not support full abolition of the
police.

\begin{itemize}
\tightlist
\item
  archived recording (crowd)\\
  (CHANTING) Go home, Jacob, go home! Go home, Jacob, go home!
\end{itemize}

john eligon

And he turns around, and he just kind of walked off into the sea of
people.

\begin{itemize}
\tightlist
\item
  archived recording (crowd)\\
  (CHANTING) Shame! Shame! Shame! Shame! Shame! Shame! Shame! Shame!
  Shame!
\end{itemize}

john eligon

So after this event, there was already an event planned for the
following day by some of these same activists organizations, in which
they were going to bring council members who were supportive of their
cause onto a stage in a park in the Southern part of the city. And they
were going to try to get them to make a commitment to defunding the
police.

michael barbaro

So a lot like what they had done to the mayor?

john eligon

Exactly.

\begin{itemize}
\tightlist
\item
  archived recording (lisa bender)\\
  Hi, Minneapolis. You look so beautiful today. I'm Lisa Bender. I'm the
  president of the Minneapolis City Council.
\end{itemize}

john eligon

And so we had this gathering where there were hundreds of residents.

\begin{itemize}
\tightlist
\item
  archived recording (lisa bender)\\
  Our efforts at incremental reform have failed. Period.
\end{itemize}

john eligon

And you had council members.

\begin{itemize}
\tightlist
\item
  archived recording (lisa bender)\\
  Our commitment is to do what's necessary to keep every single member
  of our community safe, and to tell the truth that the Minneapolis
  Police are not doing that. {[}CROWD CHEERS{]}
\end{itemize}

john eligon

And you had nine of them who went up on the stage.

\begin{itemize}
\tightlist
\item
  archived recording (council member 1)\\
  We are here today to begin the process of ending the Minneapolis
  Police Department. {[}CROWD CHEERS{]}
\end{itemize}

john eligon

And then all nine of them each read a part pledging to defend the
police.

\begin{itemize}
\tightlist
\item
  archived recording (council member 2)\\
  All of us on this stage support this statement, and we stand with the
  people of Minneapolis in fighting for a safer community. {[}CROWD
  CHEERS{]}
\end{itemize}

john eligon

The last council member, he basically said, and we are all committing to
this pledge. And at that moment, it was like this emotional eruption.

\begin{itemize}
\tightlist
\item
  archived recording (speaker)\\
  {[}INAUDIBLE{]}, get up, y'all. We are transforming our city right
  now. Get up! get
\end{itemize}

john eligon

You had white people, black people, Asian people, all putting their
fists in the air, shouting, defund the police, defund police.

\begin{itemize}
\tightlist
\item
  archived recording (crowd)\\
  (CHANTING) Defund M.P.D.! Defund M.P.D.! Defund M.P.D.!
\end{itemize}

michael barbaro

So just to be clear, this is not a vote, and not necessarily even a
pledge to vote, but this is a public commitment to defund the police ---
to do the very thing that the mayor, when asked, declined to agree to 24
hours before.

john eligon

Yes, exactly. This is a pledge that they are going to defund the police.
It is not a vote. It is not anything set in stone or written. But these
are putting them all on record in front of many community members,
saying that we are going to do this. And I even asked the activists
about that. I said, we've heard politicians say things before and not
keep those pledges. But this is something that they saw they've been
working on with them together in tandem. So I think there's a level of
trust there that this pledge has really meant something. And you could
see it in the reaction of the people who were there. They were really
describing it as their Civil Rights Movement, their Voting Rights Act
moment.

michael barbaro

Wow. And John, can the members of the City Council who were in that
park, making this pledge, do they have the actual authority to take away
funding from the police department?

john eligon

Yes, they absolutely have voting authority to do that. The council
actually controls the police department's budget. And what's more
significant about this moment is that because there were nine of them,
those nine seats represents a veto-proof majority. So even if the mayor,
Jacob Frey, does not want this to happen, if that coalition sticks
together, they can do this on their own. And I think what we're seeing
is this sentiment is growing in traction in certain places. Like we
already have in New York and Los Angeles, the mayors in both of those
cities have already said that they are going to be redirecting funds
that were intended for the police toward other parts of the city, toward
other agencies in the city.

michael barbaro

I'm curious what the appetite for this kind of change to policing is,
beyond the cities where there are largely Democratic city councils and
mayors, and where this is now under discussion.

john eligon

That's a very key question, right? We're already seeing conservatives
coming out against this and talking about this is as very radical
leftist step to be taking. We see Donald Trump already tweeting about
it. So certainly, this is something that, for conservative communities,
something like this would be a tougher sell. And so again, policing is a
very local thing. So what you experience and what the police force does
or does not look like in Minneapolis is going to be very different than
what it does or does not look like in Edina, which is just outside of
Minneapolis, or any other suburb. So it's going to be, in some ways, a
patchwork of public safety, I think, if these things start happening
around the country.

michael barbaro

And I guess an open question is whether or not this has entered the
mainstream, even of the Democratic Party. Just a few hours before you
and I began to talk, Joe Biden came out and said he does not support
defunding the police.

john eligon

Yeah, this is certainly not something that is part of the mainstream or
moderate Democratic platform. That said, you do get some people who
might be in these more moderate spaces, you do get their attention and
you do get their ear, is this sense that policing is not working ---
which is just the basis of what these defund or abolish the police
efforts are about, is that the system is not working. And so you will
get even the more moderate folks to say that, to buy into that. And that
may not result in them supporting a defund or abolishment, but will it
support more stringent reforms, more significant reforms to police? So
we'll see what happens.

michael barbaro

I wonder how the activists that you're talking to see the challenge of
explaining what these concepts are going to mean. Because in this
moment, I think many Americans are really hearing these calls ---
defund, dismantle, abolish --- for the first time. And they may be very
wary of them, and they may see them as quite radical.

john eligon

What the activists will tell you is that while it might sound radical
for many Americans, this actually is not all that radical for a large
section of this country.

\begin{itemize}
\item
  john eligon\\
  What was your name, sir?
\item
  yahzerah brazelton\\
  Yaazirah
\item
  john eligon\\
  How do you spell that?
\item
  yahzerah brazelton\\
  Y-A-H- ---
\end{itemize}

john eligon

If you go to black and brown communities --- like I went up to the North
Side of Minneapolis --- and you talk to people about their experiences
with the police there, it is not the experience of expecting an officer
to come and help you. It's exactly the opposite. And I was speaking with
a couple there, Amanda and Yaazirah Brazelton.

\begin{itemize}
\item
  yahzerah brazelton\\
  It's about time for a change.
\item
  amanda brazelton\\
  A change, yeah.
\item
  yahzerah brazelton\\
  Yeah. About time for a change.
\end{itemize}

john eligon

And they were telling me that from a young age, essentially, they
already had horrific experiences with the police.

\begin{itemize}
\tightlist
\item
  yahzerah brazelton\\
  {[}INAUDIBLE{]} I have police put guns in my face, you know, at seven
  years old, coming to my house with my mother and my father arguing,
  just regular argument that happens with a husband and wife.
\end{itemize}

john eligon

Yaazirah, he was seven years old when the police came to his house when
his parents were having an argument.

\begin{itemize}
\tightlist
\item
  yahzerah brazelton\\
  And they put guns in my face and put us all on the ground.
\end{itemize}

john eligon

And then they stuck a gun in his face.

\begin{itemize}
\tightlist
\item
  yahzerah brazelton\\
  They traumatized me in childhood, so I was really against white police
  officers since.
\end{itemize}

john eligon

And his wife Amanda, she was 14 when she was in a car with white people,
and she's black.

\begin{itemize}
\tightlist
\item
  amanda brazelton\\
  We were driving in the car. All my white friends got out. And as soon
  as I got out, they pulled guns, yelled at ---
\end{itemize}

john eligon

And when they were pulled over, the cops let the white people out, but
then they pulled guns on her.

\begin{itemize}
\tightlist
\item
  amanda brazelton\\
  I'm 14 years old, in the backseat of the car, not doing nothing wrong.
  And that was my first real incident with the police in that
  interaction.
\end{itemize}

john eligon

And so the way they see the police department is not a force where you
call and then an Officer Friendly shows up.

\begin{itemize}
\tightlist
\item
  amanda brazelton\\
  Man, I got a houseful of kids that are scared of the police because of
  what they've seen.
\end{itemize}

john eligon

It's one where Amanda recalled her children have already had run-ins
with the police that when someone was breaking into their house once,
she didn't call the police, but she called family and friends.

\begin{itemize}
\tightlist
\item
  amanda brazelton\\
  And I called him. I called my brother. I called my uncle. I called my
  dad, my mom, before called the police.
\end{itemize}

john eligon

Why didn't you call the police?

\begin{itemize}
\tightlist
\item
  amanda brazelton\\
  Because they kill black people. They'll call me, I'll get killed in my
  own home.
\end{itemize}

john eligon

So it's already a lived experience, a lived reality for people in many
black communities, that the police are essentially a force that only
exists, in their eyes, to harass them, to brutalize them and not to
protect them. And what the activists hope is that people who are scared
that abolishing police will suddenly lead to a breakdown in their
communities and just rampant violence, they're saying, no, this will
create outcomes that will make the community safer and better, not just
for affluent white communities, but for all communities across the
country. And so what activists are asking is that people who see this as
a radical idea, who can't envision a world without police, they're
asking them to just walk in these people's shoes, understand what
they're going through.

{[}music{]}

michael barbaro

John, thank you very much.

john eligon

Thank you.

\begin{itemize}
\tightlist
\item
  archived recording (lisa bender)\\
  The nine members of the city council that came from every corner of
  our city to stand together to make this commitment, we don't have all
  the answers.
\end{itemize}

michael barbaro

In an interview on Monday, the president of the Minneapolis City
Council, one of the nine members who has pledged to defund the city's
police department, acknowledged that implementing the plan would likely
take years.

\begin{itemize}
\tightlist
\item
  archived recording (lisa bender)\\
  And if you look back at the last 150 years of our police department,
  it is becoming increasingly clear that that model of policing isn't
  working. I hope it won't take 150 years to get to that looking
  forward, that next solution. But we have a lot of wisdom in our
  community. We have invested in {[}INAUDIBLE{]} ---
\end{itemize}

{[}music{]}

michael barbaro

We'll be right back.

Here's what else you need to know today.

\begin{itemize}
\tightlist
\item
  archived recording (karen bass)\\
  Good morning, everyone. The Justice in Policing Act establishes a
  bold, transformative vision of policing in America.
\end{itemize}

michael barbaro

On Monday, Congressional Democrats introduced the most sweeping federal
plan to reform the police in modern memory.

\begin{itemize}
\tightlist
\item
  archived recording (karen bass)\\
  Never again should the world be subjected to witnessing what we saw on
  the streets in Minneapolis --- the slow murder of an individual by a
  uniformed police officer.
\end{itemize}

michael barbaro

The legislation would reduce the legal protections that now shield
police officers accused of misconduct from being prosecuted, and would
impose new restrictions to prevent police officers from using deadly
force. The measure is expected to quickly pass in the House, where
there's a Democratic majority, but faces an uncertain future in the
Republican-controlled Senate. Meanwhile, state-level reforms continue.
On Monday, New York's legislature banned the use of chokeholds by police
and repealed a statute that effectively hid the disciplinary records of
police officers.

And ---

\begin{itemize}
\tightlist
\item
  archived recording (dr. tedros adhonom)\\
  Almost 7 million cases of Covid-19 have now been reported to W.H.O.,
  and almost 400,000 deaths. Although the situation in Europe is
  improving, globally, it's worsening.
\end{itemize}

michael barbaro

The World Health Organization said that the number of new daily
infections from the coronavirus hit a record high --- more than 136,000
on Sunday --- and warned that mass protests in places like the U.S.
could further spread the virus.

\begin{itemize}
\tightlist
\item
  archived recording (dr. tedros adhonom)\\
  We encourage all those protesting around the world to do so safely.
  Clean your hands, cover your cough and wear a mask if you attend a
  protest.
\end{itemize}

michael barbaro

Finally, the National Bureau of Economic Research said that because of
the pandemic, the United States economy officially entered a recession
in February, ending the longest economic expansion on record. It began
in 2009 and lasted 128 months.

{[}music{]}

That's it for ``The Daily.'' I'm Michael Barbaro. My colleague Caitlin
Dickerson will host the show tomorrow.

Democrats in Congress on Monday unveiled legislation aimed at ending
excessive use of force by the police and making it easier to identify,
track and prosecute police misconduct. The measures were seen as the
most expansive intervention into policing that federal lawmakers have
proposed in recent memory.

The legislation would curtail protections that shield police officers
accused of misconduct from being prosecuted and would set restrictions
aimed at barring officers from using deadly force except as a last
resort. The fate of the measures was far from certain; they were
expected to pass swiftly in the Democratic-led House, but President
Trump and Republican lawmakers have yet to signal what measures, if any,
they would accept. The legislation under consideration does not
contemplate defunding police departments and falls short of what many
protesters have demanded.

For his part, Mr. Trump on Monday
\href{https://www.nytimes.com/2020/06/08/us/politics/defund-police-trump.html}{discarded
proposals to remove funds} from police departments. ``We won't be
defunding our police,'' he said. ``We won't be dismantling our police.''
His attorney general, William P. Barr, said that it would be wrong to
reduce police budgets in part because he felt the country needed more
policing to preserve public safety, and warned that the nation would see
``chaos'' and ``more killings'' should any major city disband its
department.

Former Vice President Joseph R. Biden Jr., the presumptive Democratic
presidential nominee, ``does not believe that police should be
defunded,''
\href{https://www.nytimes.com/2020/06/08/us/politics/biden-defund-the-police.html}{a
campaign spokesman said on Monday}, adding that Mr. Biden ``supports the
urgent need for reform'' as well as financial support for community
policing programs.

Around the country, city and state leaders were weighing overhauls of
their policing policies, aware of the delicate balance of voters'
concerns about crime versus their repulsion at police brutality.

In Albany, New York State lawmakers on Monday began passing a
wide-ranging package of bills targeting police misconduct, overcoming
deep-seated opposition from law enforcement unions. The measures, many
of which have languished for years, include
\href{https://www.nysenate.gov/legislation/bills/2019/s6670/amendment/b}{a
ban on the use of chokeholds} as well as the repeal of a decades-old
statute that has effectively
\href{https://www.nytimes.com/2020/06/05/nyregion/police-misconduct-records-are-secret-protests-may-finally-change-that.html}{hidden
the disciplinary records} of police officers from public view.

Last week, a City Council budget meeting in Nashville stretched on for
more than eight hours, coming to a close well after midnight as
residents organized by a coalition of community groups lined up to
demand that the police budget be cut.

The idea of removing money from police forces, once largely put forth
for years by academics and advocacy groups, appeared to be shifting into
the spotlight, as activists and elected officials in cities like
Nashville, Portland, Ore., and Denver weighed the possibility.

``This is totally new,'' said Stacie Gilmore, City Council member for a
largely Latino and African-American district in Denver who had received
2,500 emails in the past three days demanding the city defund the
police. ``We're always scrambling to get enough resources. Our Police
Department by default serves as social worker, therapist, family
counselor, career counselor. We don't need the police to do that job
anymore. It's not working for communities of color.''

Late last week, after several days of protests, Mayor Ted Wheeler of
Portland announced an end to school resource officers, freeing up \$1
million to be used elsewhere with community input, according to Tim
Becker, a spokesman for the mayor.

Around the country, the calls from activists and other leaders for
defunding police departments have taken on different meanings in
different places. Most pleas for defunding the police do not signal a
wish to end efforts at public safety. Rather, officials say they want to
stop spending millions of dollars on certain items for the police, such
as military-style equipment. Some proposals seek to trim the number of
officers, a prospect that could force a debate over union contracts.

The end goal, advocates say, is to put an end to horrific scenes like
the death of Mr. Floyd in Minneapolis.

In that city, council members took a first major step toward dismantling
its police force on Sunday when nine of them, a veto-proof majority,
pledged to revamp policing. Specifics were uncertain but council members
promised to listen to concerns from community groups and cautioned
changes would take time.

``We're reclaiming the conversation of public safety and we're saying,
`It doesn't have to be fear-based, it doesn't have to be
punishment-based,''' said Alondra Cano, a council member.

\includegraphics{https://static01.nyt.com/images/2020/06/08/us/08UNREST-DEFUND-nypd/merlin_173322660_2879d025-0946-4aa1-88b0-69419ebf62f2-articleLarge.jpg?quality=75\&auto=webp\&disable=upscale}

Other lawmakers and leaders say defunding police departments could have
unintended consequences. Some people worry about safety if fewer armed
officers are on patrol, especially in summer months when crime rates
tend to spike.

Jim Cooper, a Democratic state legislator in California, urged cities to
proceed with caution when they consider cutting police budgets.

``You still have bad people out there who do bad things,'' said Mr.
Cooper, who spent 30 years in law enforcement. ``And most of the crime
is in underserved neighborhoods, not in SoHo or Beverly Hills.''

After 10 nights of mass protests and several videos documenting police
violence in New York, Mayor Bill de Blasio on Sunday vowed to cut an
unspecified amount from the
\href{https://www.nytimes.com/2020/06/15/nyregion/nypd-plainclothes-cops.html}{New
York Police Department}'s \$6 billion budget and redirect it toward
youth and other social programs.

Earlier, Mr. de Blasio had expressed substantial skepticism about the
wisdom of cutting police funding, even as he acknowledged that all
agencies might face cuts should the federal government fail to come
through with more coronavirus relief.

In Los Angeles, Mayor Eric Garcetti last week agreed to redirect \$150
million from the Police Department's nearly \$2 billion budget and other
city programs to health and education programs among others. The move
came after calls from members of Black Lives Matter Los Angeles and the
City Council.

Officials from police unions have pushed back against the idea with
sharp rebukes in some cases. In Los Angeles, the union issued a
statement saying that a crisis response team should be sent to the mayor
``because Eric has apparently lost his damn mind.'' Union members
\href{https://abc7.com/mayor-eric-garcetti-lapd-george-floyd-police-misconduct/6234091/}{warned
that spending cuts would lead to more crime.}

Image

A rally in support of defunding the police was held at Powderhorn Park
in Minneapolis on Sunday.Credit...Laylah Amatullah Barrayn for The New
York Times

In Minneapolis,
\href{https://www.nytimes.com/interactive/2020/06/03/us/minneapolis-police-use-of-force.html}{police
have used force against black people} at a rate at least seven times as
often as they have against white people over the past five years,
according to the city's data.

That statistic helps explain why the idea of abolishing the police force
makes sense to some African-Americans. Some black people say police
departments have not served to protect their communities, but rather to
harass and brutalize them. Amanda Brazelton, a resident of Minneapolis's
predominantly black North Side, said she supported using money that now
goes to the police to instead create community-led safety efforts.

Ms. Brazelton said negative interactions with the police started when
she was 14 and riding in a car that was pulled over. The officers did
nothing when her white friends got out, she said. But when she stepped
out of the car, the officers pulled weapons on her and yelled.

Now, if there is an issue at her home or she feels in danger, Ms.
Brazelton, a 30-year-old caterer, said she would call friends or family
before the police.

``As crazy as it seems, it could be something for the better,'' Ms.
Brazelton said of abolishing the police. ``They kill black people.''

There is a difference between defunding the police and abolishing the
police, said Arianna Nason, a member of the MPD150 Collective, a
coalition of community activists in Minneapolis.

She envisions a city where community watch groups or app-based safety
groups could respond to crimes.

The prospect that neighborhood watch groups could stereotype and
endanger people of color is also a concern among some people. Ms. Nason
said she understood that, but that danger already existed in the current
system.

``A lot of it is a leap of faith,'' she said. ``I want to choose to
believe in humanity. I want to choose to believe that this moment feels
different because it is different.''

Dionne Searcey and John Eligon reported from Minneapolis, and Farah
Stockman from Boston. Reporting was contributed by Maggie Haberman,
Thomas Kaplan and Catie Edmondson from Washington; Astead W. Herndon
from Houston; Dana Rubinstein and Richard A. Oppel Jr. from New York;
Shawn Hubler from Sacramento; Eric Killelea from Minneapolis; and Luis
Ferré-Sadurní from Albany, N.Y.

Advertisement

\protect\hyperlink{after-bottom}{Continue reading the main story}

\hypertarget{site-index}{%
\subsection{Site Index}\label{site-index}}

\hypertarget{site-information-navigation}{%
\subsection{Site Information
Navigation}\label{site-information-navigation}}

\begin{itemize}
\tightlist
\item
  \href{https://help.nytimes.com/hc/en-us/articles/115014792127-Copyright-notice}{©~2020~The
  New York Times Company}
\end{itemize}

\begin{itemize}
\tightlist
\item
  \href{https://www.nytco.com/}{NYTCo}
\item
  \href{https://help.nytimes.com/hc/en-us/articles/115015385887-Contact-Us}{Contact
  Us}
\item
  \href{https://www.nytco.com/careers/}{Work with us}
\item
  \href{https://nytmediakit.com/}{Advertise}
\item
  \href{http://www.tbrandstudio.com/}{T Brand Studio}
\item
  \href{https://www.nytimes.com/privacy/cookie-policy\#how-do-i-manage-trackers}{Your
  Ad Choices}
\item
  \href{https://www.nytimes.com/privacy}{Privacy}
\item
  \href{https://help.nytimes.com/hc/en-us/articles/115014893428-Terms-of-service}{Terms
  of Service}
\item
  \href{https://help.nytimes.com/hc/en-us/articles/115014893968-Terms-of-sale}{Terms
  of Sale}
\item
  \href{https://spiderbites.nytimes.com}{Site Map}
\item
  \href{https://help.nytimes.com/hc/en-us}{Help}
\item
  \href{https://www.nytimes.com/subscription?campaignId=37WXW}{Subscriptions}
\end{itemize}
