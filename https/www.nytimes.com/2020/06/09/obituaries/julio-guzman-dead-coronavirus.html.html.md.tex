Sections

SEARCH

\protect\hyperlink{site-content}{Skip to
content}\protect\hyperlink{site-index}{Skip to site index}

\href{https://www.nytimes.com/section/obituaries}{Obituaries}

\href{https://myaccount.nytimes.com/auth/login?response_type=cookie\&client_id=vi}{}

\href{https://www.nytimes.com/section/todayspaper}{Today's Paper}

\href{/section/obituaries}{Obituaries}\textbar{}Julio Guzman, Salvadoran
Refugee Who Started a Church, Dies at 64

\url{https://nyti.ms/3hei3PF}

\begin{itemize}
\item
\item
\item
\item
\item
\end{itemize}

\href{https://www.nytimes.com/news-event/coronavirus?action=click\&pgtype=Article\&state=default\&region=TOP_BANNER\&context=storylines_menu}{The
Coronavirus Outbreak}

\begin{itemize}
\tightlist
\item
  live\href{https://www.nytimes.com/2020/08/03/world/coronavirus-covid-19.html?action=click\&pgtype=Article\&state=default\&region=TOP_BANNER\&context=storylines_menu}{Latest
  Updates}
\item
  \href{https://www.nytimes.com/interactive/2020/us/coronavirus-us-cases.html?action=click\&pgtype=Article\&state=default\&region=TOP_BANNER\&context=storylines_menu}{Maps
  and Cases}
\item
  \href{https://www.nytimes.com/interactive/2020/science/coronavirus-vaccine-tracker.html?action=click\&pgtype=Article\&state=default\&region=TOP_BANNER\&context=storylines_menu}{Vaccine
  Tracker}
\item
  \href{https://www.nytimes.com/2020/08/02/us/covid-college-reopening.html?action=click\&pgtype=Article\&state=default\&region=TOP_BANNER\&context=storylines_menu}{College
  Reopening}
\item
  \href{https://www.nytimes.com/live/2020/08/03/business/stock-market-today-coronavirus?action=click\&pgtype=Article\&state=default\&region=TOP_BANNER\&context=storylines_menu}{Economy}
\end{itemize}

Advertisement

\protect\hyperlink{after-top}{Continue reading the main story}

Supported by

\protect\hyperlink{after-sponsor}{Continue reading the main story}

Those We've Lost

\hypertarget{julio-guzman-salvadoran-refugee-who-started-a-church-dies-at-64}{%
\section{Julio Guzman, Salvadoran Refugee Who Started a Church, Dies at
64}\label{julio-guzman-salvadoran-refugee-who-started-a-church-dies-at-64}}

Mr. Guzman proclaimed his faith in God even as he lay dying of Covid-19
in the hospital. He started his church in New Jersey 30 years ago.

\includegraphics{https://static01.nyt.com/images/2020/06/13/obituaries/06Guzman/merlin_173252502_6e9ebd44-3c5e-4f90-8481-b3ce0a847834-articleLarge.jpg?quality=75\&auto=webp\&disable=upscale}

By \href{https://www.nytimes.com/by/rod-nordland}{Rod Nordland}

\begin{itemize}
\item
  Published June 9, 2020Updated June 15, 2020
\item
  \begin{itemize}
  \item
  \item
  \item
  \item
  \item
  \end{itemize}
\end{itemize}

\emph{This obituary is part of a series about people who have died in
the coronavirus pandemic. Read about others}
\href{https://www.nytimes.com/interactive/2020/obituaries/people-died-coronavirus-obituaries.html}{\emph{here}}\emph{.}

Thirty years ago, Julio Guzman, a refugee from El Salvador, founded an
evangelical church ---
\href{https://www.facebook.com/pages/Iglesia-Cristiana-Buenas-Nuevas/113296232037352}{Iglesia
Cristiana Buenas Nuevas} --- in North Bergen, N.J. His wife, Ana Guzman,
was co-pastor.

Their congregation grew to include nearly 200 mostly Spanish-speaking
worshipers. The couple raised four children to adulthood, but not before
they had suffered a grievous trial: the loss of their son Daniel, who
died, at 5, of a brain tumor in 1995.

Another grievous trial was in store for the Guzman family. On April 4,
Mr. Guzman died of Covid-19 at Hackensack University Medical Center in
New Jersey, his family said. He was 64.

Like so many families whose members have been hospitalized during the
pandemic, the Guzmans were forbidden to visit. But Mr. Guzman was able
to text a final message to his eldest son, William: ``I love you. No
matter what happens, God always has a plan.''

Mrs. Guzman said, ``He never lost his faith.''

Julio Cesar Guzman was born in San Salvador on Sept. 24, 1955. Like so
many Salvadorans of his generation Mr. Guzman, at 29, fled the country's
rampant
\href{https://www.nytimes.com/2018/12/10/us/el-salvador-ms-13.html}{gang
violence} and headed north, slipping across the U.S. border. He
eventually gained legal residency, found work as a welder and became a
citizen.

He studied scripture and Hebrew at a school run by a synagogue in Jersey
City, and founded Iglesia Buenas Nuevas in 1990, when he married Ana.

Those who knew Mr. Guzman agreed that his most striking quality was
humility. He was a short man, powerfully built, with a deep reservoir of
warmth and an outgoing nature.

``His first instinct was to help people,'' said Mayra Amaya, a
goddaughter of Mr. Guzman and close family friend who grew up with his
children in North Bergen. ``He had a loving heart, and he never judged
you,'' she continued, ``so a lot of people went to him for advice.''

He did not, Ms. Amaya said, offer theology as a solution for thorny life
problems. His advice was as practical as it was humane.

According to the family, Mr. Guzman's first visit to Jerusalem, in 2012,
was a high point in his life. He began guiding tours of the Holy Land
for groups from his congregation, some of whom he baptized or married
there.

The couple had recently come home from Israel when the pandemic hit, and
Mr. Guzman was among the first religious leaders in his community to
recognize the urgency of social distancing. He suspended in-person
worship in early March, and moved services online.

But he continued counseling people in person. Within a week or so, he
began to show symptoms of the virus. They quickly became severe.

``Julio was always there for everyone,'' Ms. Amaya said, ``but at the
end, none of us could be there for him.'' Pastor Guzman took a different
view: ``When God calls you, God calls you,'' he texted her.

\href{https://www.nytimes.com/interactive/2020/obituaries/people-died-coronavirus-obituaries.html?action=click\&pgtype=Article\&state=default\&region=BELOW_MAIN_CONTENT\&context=covid_obits_promo}{}

\hypertarget{those-weve-lost}{%
\section{Those We've Lost}\label{those-weve-lost}}

The coronavirus pandemic has taken an incalculable death toll. This
series is designed to put names and faces to the numbers.

Read more

\includegraphics{https://static01.nyt.com/images/2020/07/30/obituaries/30Pedro/30Pedro-square640.jpg}

\hypertarget{bernaldina-josuxe9-pedro}{%
\section{Bernaldina José Pedro}\label{bernaldina-josuxe9-pedro}}

d. Boa Vista, Brazil

Leader among the Indigenous Macuxi

\includegraphics{https://static01.nyt.com/images/2020/07/31/obituaries/31Swing/merlin_175167783_8913bc90-0d64-43f3-a655-1bb1bf1601c9-square640.jpg}

\hypertarget{john-eric-swing}{%
\section{John Eric Swing}\label{john-eric-swing}}

d. Fountain Valley, Calif.

Champion of Filipino-Americans

\includegraphics{https://static01.nyt.com/images/2020/07/27/obituaries/27Victor/merlin_175001436_38b11f8e-227a-4e2c-9821-7618af9b2524-square640.jpg}

\hypertarget{victor-victor}{%
\section{Victor Victor}\label{victor-victor}}

d. Santo Domingo, Dominican Republic

Beloved musician of the Dominican Republic

\includegraphics{https://static01.nyt.com/images/2020/07/31/obituaries/31Negron/merlin_175160169_516322ae-fd23-4969-b6b2-193ced371105-square640.jpg}

\hypertarget{dr-eddie-negruxf3n}{%
\section{Dr. Eddie Negrón}\label{dr-eddie-negruxf3n}}

d. Fort Walton Beach, Fla.

Internist on Florida's Emerald Coast

\includegraphics{https://static01.nyt.com/images/2020/07/30/obituaries/30Dobson/merlin_175115928_f6b9271c-8f05-4fe1-a38a-5ca4a58f8935-square640.jpg}

\hypertarget{dobby-dobson}{%
\section{Dobby Dobson}\label{dobby-dobson}}

d. Coral Springs, Fla.

Jamaican singer and songwriter

\includegraphics{https://static01.nyt.com/images/2020/08/01/obituaries/28Gonzalez/merlin_175002771_beb57888-3951-409a-ae13-03a94b2e962e-square640.jpg}

\hypertarget{waldemar-gonzalez}{%
\section{Waldemar Gonzalez}\label{waldemar-gonzalez}}

d. White Plains, N.Y.

Teacher and social worker

Advertisement

\protect\hyperlink{after-bottom}{Continue reading the main story}

\hypertarget{site-index}{%
\subsection{Site Index}\label{site-index}}

\hypertarget{site-information-navigation}{%
\subsection{Site Information
Navigation}\label{site-information-navigation}}

\begin{itemize}
\tightlist
\item
  \href{https://help.nytimes.com/hc/en-us/articles/115014792127-Copyright-notice}{©~2020~The
  New York Times Company}
\end{itemize}

\begin{itemize}
\tightlist
\item
  \href{https://www.nytco.com/}{NYTCo}
\item
  \href{https://help.nytimes.com/hc/en-us/articles/115015385887-Contact-Us}{Contact
  Us}
\item
  \href{https://www.nytco.com/careers/}{Work with us}
\item
  \href{https://nytmediakit.com/}{Advertise}
\item
  \href{http://www.tbrandstudio.com/}{T Brand Studio}
\item
  \href{https://www.nytimes.com/privacy/cookie-policy\#how-do-i-manage-trackers}{Your
  Ad Choices}
\item
  \href{https://www.nytimes.com/privacy}{Privacy}
\item
  \href{https://help.nytimes.com/hc/en-us/articles/115014893428-Terms-of-service}{Terms
  of Service}
\item
  \href{https://help.nytimes.com/hc/en-us/articles/115014893968-Terms-of-sale}{Terms
  of Sale}
\item
  \href{https://spiderbites.nytimes.com}{Site Map}
\item
  \href{https://help.nytimes.com/hc/en-us}{Help}
\item
  \href{https://www.nytimes.com/subscription?campaignId=37WXW}{Subscriptions}
\end{itemize}
