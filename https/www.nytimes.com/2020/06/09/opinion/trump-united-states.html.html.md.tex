Sections

SEARCH

\protect\hyperlink{site-content}{Skip to
content}\protect\hyperlink{site-index}{Skip to site index}

\href{https://myaccount.nytimes.com/auth/login?response_type=cookie\&client_id=vi}{}

\href{https://www.nytimes.com/section/todayspaper}{Today's Paper}

\href{/section/opinion}{Opinion}\textbar{}Let's Change Our Motto to `Out
of Many, We'

\href{https://nyti.ms/30qI5td}{https://nyti.ms/30qI5td}

\begin{itemize}
\item
\item
\item
\item
\item
\item
\end{itemize}

Advertisement

\protect\hyperlink{after-top}{Continue reading the main story}

\href{/section/opinion}{Opinion}

Supported by

\protect\hyperlink{after-sponsor}{Continue reading the main story}

\hypertarget{lets-change-our-motto-to-out-of-many-we}{%
\section{Let's Change Our Motto to `Out of Many,
We'}\label{lets-change-our-motto-to-out-of-many-we}}

It is heading toward ``Out of Many, None,'' and I fear it becoming ``Out
of Many, Me.''

\href{https://www.nytimes.com/by/thomas-l-friedman}{\includegraphics{https://static01.nyt.com/images/2018/04/02/opinion/thomas-l-friedman/thomas-l-friedman-thumbLarge.png}}

By \href{https://www.nytimes.com/by/thomas-l-friedman}{Thomas L.
Friedman}

Opinion Columnist

\begin{itemize}
\item
  June 9, 2020
\item
  \begin{itemize}
  \item
  \item
  \item
  \item
  \item
  \item
  \end{itemize}
\end{itemize}

\includegraphics{https://static01.nyt.com/images/2020/06/09/opinion/09friedman/09friedman-articleLarge.jpg?quality=75\&auto=webp\&disable=upscale}

There are so many prisms through which to view the tectonic events
taking place on America's streets since the police killing of George
Floyd in Minneapolis, but to my mind the most important is that our
country is in the process of renegotiating its founding motto, carried
on the seal of the United States: ``E pluribus unum,'' or ``Out of many,
one.''

I'd say that our motto used to be ``Out of many, one,'' but it's now
heading for ``Out of many, none.'' I fear it could become ``Out of many,
me.'' But I am certain that if we're to thrive in the 21st century it
needs to be ``Out of many, we.''

Why do I say this? Two reasons. First, I was born and raised in
Minneapolis and I have come to realize how much its good sides and ugly
sides --- both of which have been on national display lately --- are a
microcosm of the broad national struggle over what exactly our motto
should be today.

Out of many, one? Out of many, none? Out of many, me? Out of many, we?

I was born in 1953 on the Northside, in the same part of the city as my
parents were born**,** after their parents immigrated from Eastern
Europe. The Northside was basically a ghetto of mostly Jews and blacks,
who were not integrated there but isolated together by walls of racism
and anti-Semitism.

After World War II, much of the Northside Jewish community made an
exodus, en masse, to one suburb --- St. Louis Park, because it did not
have restrictions on home sales to Jews and had enough housing stock to
take them all. Practically overnight a suburb that had been almost 100
percent white, Christian and Scandinavian became 80 percent white
Christian Scandinavian and 20 percent Jewish. If Sweden and Israel had a
baby it would have been St. Louis Park**.**

Meanwhile, the African-Americans, weighed down by structural racism ---
with its bad schools, zoning restrictions, polluting highway and
factories, all reinforcing multigenerational poverty --- mostly could
not escape the Northside, which exploded in riots in 1967. When I
graduated from high school in St. Louis Park in 1971, we had two
African-Americans in our school of about 2,500.

I also had an aunt and uncle who had moved to the small town of Willmar,
in West Central Minnesota, to start a steel company in 1949, and I spent
summers visiting them. For many years they, and two other Jewish
families there, constituted ``diversity'' in the virtually all-white,
largely Protestant/Catholic Willmar.

After high school, I left Minnesota to discover the world. I returned
some 40 years later to write a book (partly about Minnesota) in 2015,
\href{https://www.thomaslfriedman.com/thank-you-for-being-late/}{``Thank
You for Being Late,''} **** and I found that the world had discovered
\href{https://discoverstlouispark.com/about-st-louis-park/}{St. Louis
Park} and Willmar.

By then, St. Louis Park High School had become 58 percent white, 27
percent black, 9 percent Latinx, 5 percent Asian and 1 percent Native
American. The black student body was roughly split between
African-Americans and recent immigrants from Somalia, and my high
school, which had essentially no Muslims in my day, now had more Muslims
than Jews.

In May 2019 I visited Willmar High School to research
\href{https://www.nytimes.com/2019/05/14/opinion/trump-willmar-minnesota.html}{a
column} about its transformation since my boyhood and found that its
student body comprised young people from some 30 countries across Latin
America, the Middle East and Asia --- and nearly half the town of 21,000
was made up of Latinx, Somali and other East African and Asian
immigrants.

Have no illusions; necessity was the mother of inclusion. Willmar, like
so many Minnesota towns, needed workers at all skill levels. But that's
often how walls first get broken down. Towns in Minnesota today that
cannot manage diversity know that they will most likely wither. And they
are seeing places like Willmar and St. Louis Park, which still have
plenty of racial issues to manage, thrive by becoming more diverse.

As I noted in my book, ``Minnesota nice'' --- the state's informal motto
--- covered for a lot of structural racism and police brutality over the
years, and still does. George Floyd's death was not a freak event. But
it's also true that the state is full of people who want to get caught
trying to reverse that. (Check out the Itasca
\href{https://www.theitascaproject.com/}{Project} and the
\href{https://northsideachievement.org/}{Northside Achievement Zone} as
just two among many examples.) Floyd's killing has shown them that the
effort needs to get into a whole new gear, though.

And that brings me back to our national motto. It was easy to say ``Out
of many, one'' when most of the ``many'' were white and from Europe and
when the black and brown minority was small and formally and then
informally not treated as equal members of the ``one.''

But as St. Louis Park and Willmar testify, even a state like Minnesota
is now just so much more diverse. And like the country, its major cities
will become minority majorities over the next two decades.
Unfortunately, this new level of diversity, rather than being a source
of our strength, has lately become a source of paralysis.

That is how we got into ``Out of many, none.''

Our founders created a system of divided powers, but they assumed that
politicians would in the end compromise to get stuff done. Lately,
however, polarization has become so tribal that compromise is impossible
and the system has frozen into a veto machine, the political scientist
Frank Fukuyama observed. So, we can't do anything big or hard --- or
together --- anymore.

``As many people point out, it wasn't symmetric polarization,''
\href{https://www.the-american-interest.com/2020/05/13/political-decay-in-the-time-of-coronavirus/}{Fukuyama
said in a Zoom discussion for The American Interest}. ``There's been a
shift clearly to the left by the Democratic Party, represented by Bernie
Sanders, but the real thing that changed was a shift by the Republican
Party to a position that was very unfamiliar to Reagan Republicans, in
which the state itself became the enemy for a lot of the Tea Party wing
of the party. And then it's captured by the Trump wing that was kind of
an identitarian right-wing nationalist group. And that has led, I think,
to the current crisis that we're in, where fundamental decisions are
really deadlocked.''

This paralysis has led some on the right to long for a third motto,
``Out of many, me'' --- or as Donald Trump once proclaimed, ``I alone
can fix it.'' Trump believes that he can simply cut through the
paralysis by seizing more executive power, the Constitution be damned,
but he is not alone in this view. The leaders of Russia, China, Hungary,
Turkey and Brazil all share this authoritarian impulse.

This is a fantasy. The only way we are going to remain America is if our
motto becomes ``Out of many, we.''

``Out of many, we'' acknowledges that ``we the people'' are now more
diverse than ever --- that diversity, when it can be made to work, is a
tremendous source of resilience, innovation, creativity and renewal. But
for that diversity to be a strength again for America, it cannot be
based any longer on a white majority learning ``tolerance'' for
nonwhites --- the descendants of slaves and immigrants.

Tolerance is important to be sure. But ``Out of many, we'' summons us
all --- people of every color --- to a deeper commitment to pluralism: a
robust appreciation of the distinctive contribution of every community
and a commitment beyond rhetoric to make sure that each one has the
schools, governance and policing that enables that contribution.

I like how Kay Coles James, the first African-American and the first
woman to head the Heritage Foundation, a conservative think tank, put it
in a
\href{https://www.foxnews.com/opinion/george-floyds-senseless-killing-end-racism-americas-cancer-kay-coles-james}{recent
essay} on Foxnews.com: ``It's time America takes responsibility and
expands human flourishing to all of its citizens --- not just the
majority of them.'' (Hat tip to
\href{https://www.washingtonpost.com/opinions/this-is-what-happens-when-bigotry-dominates-the-main-conservative-media-platform/2020/06/08/c1deaf50-a9ba-11ea-a9d9-a81c1a491c52_story.html}{Michael
Gerson for quoting this}.)

It is clearly the fear of living in such a diverse America that has
brought a hard core of whites to stick with Trump no matter what he does
and to encourage the Republican Party to try to hold onto power any way
possible --- through gerrymandering, voter suppression, control of the
Senate through sparsely populated non-diverse states, and the courts ---
in order to keep winning the Electoral College while losing the popular
vote.

That is not a sustainable strategy for sustaining America. We need a
healthy conservative party in America --- one that embraces diversity
but offers conservative principles for how to get the most out of it.
The G.O.P. can't just keep trying to hold the presidency through
maneuvers while losing the national vote by bigger and bigger margins.

If that continues, America, this great experiment, will eventually just
blow apart. And then our tombstone will read: ``Out of many --- just
bits, pieces and fragments.'' We can't let that happen.

\emph{The Times is committed to publishing}
\href{https://www.nytimes.com/2019/01/31/opinion/letters/letters-to-editor-new-york-times-women.html}{\emph{a
diversity of letters}} \emph{to the editor. We'd like to hear what you
think about this or any of our articles. Here are some}
\href{https://help.nytimes.com/hc/en-us/articles/115014925288-How-to-submit-a-letter-to-the-editor}{\emph{tips}}\emph{.
And here's our email:}
\href{mailto:letters@nytimes.com}{\emph{letters@nytimes.com}}\emph{.}

\emph{Follow The New York Times Opinion section on}
\href{https://www.facebook.com/nytopinion}{\emph{Facebook}}\emph{,}
\href{http://twitter.com/NYTOpinion}{\emph{Twitter (@NYTopinion)}}
\emph{and}
\href{https://www.instagram.com/nytopinion/}{\emph{Instagram}}\emph{.}

Advertisement

\protect\hyperlink{after-bottom}{Continue reading the main story}

\hypertarget{site-index}{%
\subsection{Site Index}\label{site-index}}

\hypertarget{site-information-navigation}{%
\subsection{Site Information
Navigation}\label{site-information-navigation}}

\begin{itemize}
\tightlist
\item
  \href{https://help.nytimes.com/hc/en-us/articles/115014792127-Copyright-notice}{©~2020~The
  New York Times Company}
\end{itemize}

\begin{itemize}
\tightlist
\item
  \href{https://www.nytco.com/}{NYTCo}
\item
  \href{https://help.nytimes.com/hc/en-us/articles/115015385887-Contact-Us}{Contact
  Us}
\item
  \href{https://www.nytco.com/careers/}{Work with us}
\item
  \href{https://nytmediakit.com/}{Advertise}
\item
  \href{http://www.tbrandstudio.com/}{T Brand Studio}
\item
  \href{https://www.nytimes.com/privacy/cookie-policy\#how-do-i-manage-trackers}{Your
  Ad Choices}
\item
  \href{https://www.nytimes.com/privacy}{Privacy}
\item
  \href{https://help.nytimes.com/hc/en-us/articles/115014893428-Terms-of-service}{Terms
  of Service}
\item
  \href{https://help.nytimes.com/hc/en-us/articles/115014893968-Terms-of-sale}{Terms
  of Sale}
\item
  \href{https://spiderbites.nytimes.com}{Site Map}
\item
  \href{https://help.nytimes.com/hc/en-us}{Help}
\item
  \href{https://www.nytimes.com/subscription?campaignId=37WXW}{Subscriptions}
\end{itemize}
