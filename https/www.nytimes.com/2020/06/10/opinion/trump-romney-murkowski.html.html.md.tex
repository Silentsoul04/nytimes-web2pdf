Sections

SEARCH

\protect\hyperlink{site-content}{Skip to
content}\protect\hyperlink{site-index}{Skip to site index}

\href{https://myaccount.nytimes.com/auth/login?response_type=cookie\&client_id=vi}{}

\href{https://www.nytimes.com/section/todayspaper}{Today's Paper}

\href{/section/opinion}{Opinion}\textbar{}Trump Trembles: Mitt's on the
Move

\href{https://nyti.ms/3dMSsvc}{https://nyti.ms/3dMSsvc}

\begin{itemize}
\item
\item
\item
\item
\item
\item
\end{itemize}

Advertisement

\protect\hyperlink{after-top}{Continue reading the main story}

\href{/section/opinion}{Opinion}

Supported by

\protect\hyperlink{after-sponsor}{Continue reading the main story}

\hypertarget{trump-trembles-mitts-on-the-move}{%
\section{Trump Trembles: Mitt's on the
Move}\label{trump-trembles-mitts-on-the-move}}

Just remember that nothing counts if you don't vote.

\href{https://www.nytimes.com/by/gail-collins}{\includegraphics{https://static01.nyt.com/images/2018/04/03/opinion/gail-collins/gail-collins-thumbLarge.png}}

By \href{https://www.nytimes.com/by/gail-collins}{Gail Collins}

Opinion Columnist

\begin{itemize}
\item
  June 10, 2020
\item
  \begin{itemize}
  \item
  \item
  \item
  \item
  \item
  \item
  \end{itemize}
\end{itemize}

\includegraphics{https://static01.nyt.com/images/2020/06/10/opinion/10collins2/merlin_173402001_bd838db9-6c18-4763-872b-84b429bd5745-articleLarge.jpg?quality=75\&auto=webp\&disable=upscale}

Go Mitt!

Back in 2016 when Donald Trump was elected, the world was so upside down
you might have believed almost any prediction of our strange future. But
I'll bet you'd still have been skeptical if somebody told you Mitt
Romney would turn into an inspirational political figure.

Yes! The same guy who ran one of the most boring presidential races in
modern history. (``There's no question it's not good being poor.'')
Early this year he turned into
\href{https://www.nytimes.com/2020/02/05/us/politics/romney-trump-impeachment.html}{an
impeachment hero}. Now he's calling for ``a voice against racism'' and
\href{https://www.nytimes.com/2020/06/07/us/politics/mitt-romney-george-floyd-protests.html}{marching
for Black Lives Matter.}

And driving Donald Trump nuts. No holier a grail than that. Yet there's
still one more little thing. Mitt can't bring himself to say that people
should support Joe Biden. He's keeping his own voting plans secret.
There's a lot of speculation he's going to write in his wife's name,
like he did in 2016.

That's a pretty pale rebellion. It's like saying you're going to sit
home on Election Day and sulk.

Same story for Alaska's Lisa Murkowski, who
\href{https://www.nytimes.com/2020/06/04/us/politics/murkowski-mattis-trump.html}{says
she's ``struggling''} to figure out what to do. Last week Murkowski
responded fervently in support of former Defense Secretary Jim Mattis's
\href{https://www.theatlantic.com/politics/archive/2020/06/james-mattis-denounces-trump-protests-militarization/612640/}{anti-Trump
essay} in which he urged Americans to ``unite without him'' and create a
better world. There are, everybody knows, a great many things the
country needs to do to make a brighter tomorrow. But the ``without him''
part does seem to be the critical priority.

Cheers to both Murkowski and Romney. It's true that neither of them is
up for re-election this year, but if you cast your eyes across the
Republican Senate majority, you will see many, many folks in safe seats
who are nevertheless afraid to cross the president, even when his
behavior causes them to go home and weep, or drink, or sit up all night
playing solitaire on the computer.

You may remember that Romney was a Republican candidate for president in
2008 and 2012. But it's OK if you don't. Neither campaign was very
memorable, although I personally will never forget when he placated
social conservatives by announcing that the only reason he had supported
abortion rights as governor of Massachusetts was that he really didn't
understand what an embryo was.

There are two ways to look at his current profile in courage. One is
that Romney is a very rich 73-year-old in a safe seat who can do pretty
much anything he wants. Another is that it's still \ldots{} courageous.
Take your pick.

\includegraphics{https://static01.nyt.com/images/2020/06/10/opinion/10collins1/merlin_173376612_3c6cdc70-d4d4-471a-a231-0af4383507f3-articleLarge.jpg?quality=75\&auto=webp\&disable=upscale}

Murkowski was once elected to the Senate on a write-in vote after a Tea
Party candidate won the Republican nomination. Since then she's bucked
her president on repealing Obamacare, but supported him on everything
from cabinet nominees to impeachment. All the while busily delivering
tons and tons of stuff to Alaska's powerful energy interests. She's got
a certain amount of independence, but you still have to
\href{https://www.washingtonpost.com/news/powerpost/paloma/the-energy-202/2020/06/08/the-energy-202-lisa-murkowski-s-wins-for-alaska-help-protect-her-against-trump-attacks/5edd2548602ff12947e86583/}{get
those drilling rights}.

There must be a ton of Republican officials who are at least a little
tempted to reject the dreaded concept of Four More Years. But if they
want credit for showing some spine, they have to follow through and vote
for the only other real candidate in the race. Otherwise, it's just a
wasted vote at best, and maybe a real boost for Trump. Perhaps you
remember in 2016, when people who would never in a billion years have
supported Donald Trump showed their lack of enthusiasm for Hillary
Clinton by voting for the Green Party. Their defection was enough to
turn the tide in several critical swing states.
\href{https://www.nytimes.com/2018/11/09/opinion/third-party-midterms.html}{In
Michigan, for instance,} Trump won all 16 electoral votes with a margin
of 10,704 voters. Green Party candidate Jill Stein got 51,463.

The White House is already sniping at Romney and Murkowski for their
rebellion. Trump vowed to campaign against Murkowski no matter who was
running against her --- ``good or bad. \ldots{} If you have a pulse, I'm
with you.'' White House press secretary Kayleigh McEnany suggested
nobody cared if Romney said ``three words outside on Pennsylvania
Avenue.'' The words presumably were ``Black Lives Matter,'' but we have
already learned that McEnany is a spokesperson who has a certain amount
of trouble with speaking.

``But,'' she added, ``I would note this: that President Trump won 8
percent of the black vote. Mitt Romney won 2 percent of the black
vote.'' Actually, the two candidates got pretty much the same
African-American support, which would be minimal.

The difference between them, of course, is that Romney's been growing on
racial issues while Trump has been shrinking. If it's possible to get
tinier.

A lot of Republicans who are horrified by the president don't have the
gumption to criticize him at all. After Trump tweeted that the
75-year-old demonstrator who was seriously injured by the police in
Buffalo might have been ``an ANTIFA provocateur,'' reporters cornered
Senator Marco Rubio on what he thought.
\href{https://www.cnn.com/2020/06/09/politics/trump-buffalo-protester-set-up-tweet/index.html}{Rubio
pleaded ignorance}: ``I don't read Twitter, I only write on it.''

So give M and R credit for taking a stand. But this thing about not
voting, or going for the Libertarian, or writing in Brad Pitt, is crazy.
They've got to back Biden. Then after he takes office, they can attack
every single thing he tries to do for the next four years. Everybody
wins.

\emph{The Times is committed to publishing}
\href{https://www.nytimes.com/2019/01/31/opinion/letters/letters-to-editor-new-york-times-women.html}{\emph{a
diversity of letters}} \emph{to the editor. We'd like to hear what you
think about this or any of our articles. Here are some}
\href{https://help.nytimes.com/hc/en-us/articles/115014925288-How-to-submit-a-letter-to-the-editor}{\emph{tips}}\emph{.
And here's our email:}
\href{mailto:letters@nytimes.com}{\emph{letters@nytimes.com}}\emph{.}

\emph{Follow The New York Times Opinion section on}
\href{https://www.facebook.com/nytopinion}{\emph{Facebook}}\emph{,}
\href{http://twitter.com/NYTOpinion}{\emph{Twitter (@NYTopinion)}}
\emph{and}
\href{https://www.instagram.com/nytopinion/}{\emph{Instagram}}\emph{.}

Advertisement

\protect\hyperlink{after-bottom}{Continue reading the main story}

\hypertarget{site-index}{%
\subsection{Site Index}\label{site-index}}

\hypertarget{site-information-navigation}{%
\subsection{Site Information
Navigation}\label{site-information-navigation}}

\begin{itemize}
\tightlist
\item
  \href{https://help.nytimes.com/hc/en-us/articles/115014792127-Copyright-notice}{©~2020~The
  New York Times Company}
\end{itemize}

\begin{itemize}
\tightlist
\item
  \href{https://www.nytco.com/}{NYTCo}
\item
  \href{https://help.nytimes.com/hc/en-us/articles/115015385887-Contact-Us}{Contact
  Us}
\item
  \href{https://www.nytco.com/careers/}{Work with us}
\item
  \href{https://nytmediakit.com/}{Advertise}
\item
  \href{http://www.tbrandstudio.com/}{T Brand Studio}
\item
  \href{https://www.nytimes.com/privacy/cookie-policy\#how-do-i-manage-trackers}{Your
  Ad Choices}
\item
  \href{https://www.nytimes.com/privacy}{Privacy}
\item
  \href{https://help.nytimes.com/hc/en-us/articles/115014893428-Terms-of-service}{Terms
  of Service}
\item
  \href{https://help.nytimes.com/hc/en-us/articles/115014893968-Terms-of-sale}{Terms
  of Sale}
\item
  \href{https://spiderbites.nytimes.com}{Site Map}
\item
  \href{https://help.nytimes.com/hc/en-us}{Help}
\item
  \href{https://www.nytimes.com/subscription?campaignId=37WXW}{Subscriptions}
\end{itemize}
