Sections

SEARCH

\protect\hyperlink{site-content}{Skip to
content}\protect\hyperlink{site-index}{Skip to site index}

\href{https://www.nytimes.com/section/technology}{Technology}

\href{https://myaccount.nytimes.com/auth/login?response_type=cookie\&client_id=vi}{}

\href{https://www.nytimes.com/section/todayspaper}{Today's Paper}

\href{/section/technology}{Technology}\textbar{}Amazon Pauses Police Use
of Its Facial Recognition Software

\url{https://nyti.ms/30wnILc}

\begin{itemize}
\item
\item
\item
\item
\item
\end{itemize}

Advertisement

\protect\hyperlink{after-top}{Continue reading the main story}

Supported by

\protect\hyperlink{after-sponsor}{Continue reading the main story}

\hypertarget{amazon-pauses-police-use-of-its-facial-recognition-software}{%
\section{Amazon Pauses Police Use of Its Facial Recognition
Software}\label{amazon-pauses-police-use-of-its-facial-recognition-software}}

The company said it hoped the moratorium ``might give Congress enough
time to put in place appropriate rules'' for the technology.

\includegraphics{https://static01.nyt.com/images/2020/06/10/business/10UNREST-AMAZON-sub/merlin_146147871_e6de7acc-ca13-4a75-85c9-bcaa6c869a18-articleLarge.jpg?quality=75\&auto=webp\&disable=upscale}

By \href{https://www.nytimes.com/by/karen-weise}{Karen Weise} and
\href{https://www.nytimes.com/by/natasha-singer}{Natasha Singer}

\begin{itemize}
\item
  June 10, 2020
\item
  \begin{itemize}
  \item
  \item
  \item
  \item
  \item
  \end{itemize}
\end{itemize}

SEATTLE --- Amazon said on Wednesday that it was putting a one-year
pause on letting the police use its
\href{https://www.nytimes.com/2019/05/20/technology/amazon-facial-recognition.html}{facial
recognition} tool, in a major sign of the growing concerns that the
technology may lead to
\href{https://www.nytimes.com/2018/02/09/technology/facial-recognition-race-artificial-intelligence.html}{unfair
treatment} of African-Americans.

The technology giant did not explain its reasoning in its brief blog
post about the change, but the move came amid the nationwide protests
over racism and biased policing. Amazon's technology had been criticized
in the past for misidentifying people of color.

In its
\href{https://blog.aboutamazon.com/policy/we-are-implementing-a-one-year-moratorium-on-police-use-of-rekognition}{blog
post}, the company
\href{https://blog.aboutamazon.com/policy/we-are-implementing-a-one-year-moratorium-on-police-use-of-rekognition}{said}
it hoped the moratorium on its service, Rekognition, ``might give
Congress enough time to put in place appropriate rules'' for the ethical
use of facial recognition.

The announcement was a striking change for Amazon, a prominent supplier
of facial recognition software to law enforcement. More than other big
technology companies, Amazon has resisted calls to slow its deployment.
In the past, Amazon had said its tools were accurate but were improperly
used by researchers.

On Monday, IBM said it would stop selling facial recognition products,
and last year, the leading maker of police body cameras
\href{https://www.nytimes.com/2019/06/27/opinion/police-cam-facial-recognition.html}{banned}
the use of facial recognition on its products at the recommendation of
its independent ethics board, which said the technology ``is not
currently reliable enough to ethically justify its use.''
\href{https://www.nytimes.com/2020/06/09/technology/facial-recognition-software.html}{Google
has advocated} a temporary ban on the technology.

The American Civil Liberties Union applauded Amazon in a statement for
``finally recognizing the dangers face recognition poses to Black and
Brown communities and civil rights more broadly.'' But it said that the
company should extend the moratorium on law enforcement use of its
system until Congress passed a law regulating the technology.

``Face recognition technology gives governments the unprecedented power
to spy on us wherever we go,'' Nicole Ozer, technology and civil
liberties director for the A.C.L.U. of Northern California, said in the
statement. ``It fuels police abuse. This surveillance technology must be
stopped.''

Law enforcement agencies use facial recognition technology to
\href{https://www.nytimes.com/2019/06/09/opinion/facial-recognition-police-new-york-city.html}{identify}
suspects and missing children. The systems work by trying to match
facial pattern data extracted from photos or video with those in
databases like driver's license records. The authorities used the
technology to help identify the suspect
\href{https://www.nytimes.com/2019/04/29/us/capital-gazette-shooting-suspect.html}{in
the mass shooting} at a newspaper last year in Annapolis, Md.

But civil liberties groups have warned that the technology can be used
at a distance to secretly identify individuals --- such as protesters
attending demonstrations --- potentially chilling Americans' right to
free speech or simply limiting their ability to go about their business
anonymously in public. Some cities, including San Francisco, and
Cambridge, Mass., have passed bans on the technology.

This week, Democrats in the House introduced a police reform law that
would ban the use of facial recognition technology with police recording
equipment. Some lawmakers have long worried about the technology,
questioning manufacturers and the public agencies that use their
products on how it affects civil rights and privacy.

Civil liberties advocates began a campaign to ban the use of facial
recognition by law enforcement in 2018, after a report by academic
researchers found racial bias in the systems. The report found that
facial technologies made by IBM and Microsoft were able to correctly
identify the gender of white men in photographs about 100 percent of the
time. But the systems were much less accurate in their ability to
identify the gender of darker-skinned women.

IBM and Microsoft quickly improved their systems. Amazon found itself
under heightened scrutiny.

For the past two years, the A.C.L.U. has led a campaign to push Amazon
to stop selling the technology to law enforcement agencies. The group
obtained documents, using open information laws, from police departments
that showed how Amazon was aggressively marketing its technology to law
enforcement.

The A.C.L.U. also tested Amazon's technology using the head shots of
members of Congress and comparing them against a database of publicly
available mug shots. The group reported that the Amazon technology
incorrectly matched 28 members of Congress with people who had been
arrested, amounting to a 5 percent error rate among legislators. At the
time, Amazon disputed the findings, saying that the group had used its
system differently than law enforcement customers did.

Rep. Jimmy Gomez, a California Democrat and one of the lawmakers
misidentified in the A.C.L.U. test, said he met with Amazon about the
issue almost a dozen times. He said Amazon was less open to criticism
than its tech peers.

``They were avoiding taking any responsibility for their technology in
my opinion,'' Mr. Gomez said on Wednesday after the company's
announcement. ``They always had some excuse.''

Mr. Gomez, who is vice chairman of the House Committee on Oversight and
Reform, said he was glad to see Amazon halt police sales.

``Amazon can sense that the American people don't want platitudes when
it comes to dealing with disparities right now,'' he said. ``They want
concrete action.''

Amazon introduced Rekognition in 2016 as a low-cost, ``highly scalable''
way to identify images, including people, in vast databases. Soon after,
it began
\href{https://www.nytimes.com/2018/05/22/technology/amazon-facial-recognition.html}{pitching
the police} on the tool to help investigations, and law enforcement
agencies began adopting the technology.

In an interview on the PBS show ``Frontline'' earlier this year, Andy
Jassy, the chief executive of Amazon Web Services, said he did not think
the company knew how many police departments were deploying the
technology.

Last fall, Jeff Bezos, Amazon's chief executive, said the company was
drafting privacy legislation for facial recognition. But he indicated
that Amazon would continue selling the tools in the meantime.

``It's a perfect example of something that has really positive uses, so
you don't want to put the brakes on it,'' Mr. Bezos said. ``At the same
time, there is lots of potential for abuses with that kind of
technology, so you want regulations.''

He said he would welcome ``good regulations'' on the issue. ``That kind
of stability I think would be healthy for the whole industry,'' he said.

Mr. Bezos did not provide details for what the company's proposed
legislation would entail.

Mr. Gomez said he had not seen any model legislation proposed by Amazon,
adding, ``That would have been news to me.''

Karen Weise reported from Seattle, and Natasha Singer from New York.
David McCabe contributed reporting from Washington.

Advertisement

\protect\hyperlink{after-bottom}{Continue reading the main story}

\hypertarget{site-index}{%
\subsection{Site Index}\label{site-index}}

\hypertarget{site-information-navigation}{%
\subsection{Site Information
Navigation}\label{site-information-navigation}}

\begin{itemize}
\tightlist
\item
  \href{https://help.nytimes.com/hc/en-us/articles/115014792127-Copyright-notice}{©~2020~The
  New York Times Company}
\end{itemize}

\begin{itemize}
\tightlist
\item
  \href{https://www.nytco.com/}{NYTCo}
\item
  \href{https://help.nytimes.com/hc/en-us/articles/115015385887-Contact-Us}{Contact
  Us}
\item
  \href{https://www.nytco.com/careers/}{Work with us}
\item
  \href{https://nytmediakit.com/}{Advertise}
\item
  \href{http://www.tbrandstudio.com/}{T Brand Studio}
\item
  \href{https://www.nytimes.com/privacy/cookie-policy\#how-do-i-manage-trackers}{Your
  Ad Choices}
\item
  \href{https://www.nytimes.com/privacy}{Privacy}
\item
  \href{https://help.nytimes.com/hc/en-us/articles/115014893428-Terms-of-service}{Terms
  of Service}
\item
  \href{https://help.nytimes.com/hc/en-us/articles/115014893968-Terms-of-sale}{Terms
  of Sale}
\item
  \href{https://spiderbites.nytimes.com}{Site Map}
\item
  \href{https://help.nytimes.com/hc/en-us}{Help}
\item
  \href{https://www.nytimes.com/subscription?campaignId=37WXW}{Subscriptions}
\end{itemize}
