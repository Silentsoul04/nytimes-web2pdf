Sections

SEARCH

\protect\hyperlink{site-content}{Skip to
content}\protect\hyperlink{site-index}{Skip to site index}

\href{https://www.nytimes.com/section/health}{Health}

\href{https://myaccount.nytimes.com/auth/login?response_type=cookie\&client_id=vi}{}

\href{https://www.nytimes.com/section/todayspaper}{Today's Paper}

\href{/section/health}{Health}\textbar{}What We're Learning About Online
Learning

\url{https://nyti.ms/2UFFjNb}

\begin{itemize}
\item
\item
\item
\item
\item
\end{itemize}

Advertisement

\protect\hyperlink{after-top}{Continue reading the main story}

Supported by

\protect\hyperlink{after-sponsor}{Continue reading the main story}

mind

\hypertarget{what-were-learning-about-online-learning}{%
\section{What We're Learning About Online
Learning}\label{what-were-learning-about-online-learning}}

As virtual classrooms and online learning proliferate, researchers are
working to quantify what works and what doesn't.

\includegraphics{https://static01.nyt.com/images/2020/06/16/science/00ONLINE-SCHOOL1/merlin_170752293_fc28b2da-b3e6-46cf-bf19-1776c8813397-articleLarge.jpg?quality=75\&auto=webp\&disable=upscale}

\href{https://www.nytimes.com/by/benedict-carey}{\includegraphics{https://static01.nyt.com/images/2018/02/16/multimedia/author-benedict-carey/author-benedict-carey-thumbLarge.jpg}}

By \href{https://www.nytimes.com/by/benedict-carey}{Benedict Carey}

\begin{itemize}
\item
  June 13, 2020
\item
  \begin{itemize}
  \item
  \item
  \item
  \item
  \item
  \end{itemize}
\end{itemize}

Over four days in mid-March, Cindy Hansen, an 11th grade English teacher
at Timpanogos High School in Orem, Utah, had to go fully virtual, and
took her class of some 30 students reading ``The Great Gatsby'' online.

Ms. Hansen had no experience with virtual courses and, like teachers
around the country, had to experiment. She decided to upload video
lessons --- presenting the text of ``Gatsby'' along with a small window
in the corner of the screen, in which she read aloud key passages and
assigned essays.

The transition seemed to be proceeding smoothly until, after several
lessons, she received a note from a student who rarely spoke up in
class.

``He's one of my sweetest students, and he wrote, `Ms. Hansen, those
videos are glitchy --- I can't really see the text,''' she said in a
phone interview. ``I had just assumed they were fine. Well, they were
horrible, and the poor kid felt frustrated. I'm glad he said
something.'' She quickly fixed the problem, she said, by reshooting the
videos directly on the teaching site instead of uploading them.

After this spring's on-the-fly experiment in online classes, teachers
and school districts across the country are preparing for what will be
anything but a normal fall semester. Some districts stumbled in the
transition, \href{https://www.washingtonpost.com}{with classes
zoom-bombed and interrupted}; many strained to address
\href{https://www.nytimes.com/2020/04/06/us/coronavirus-schools-attendance-absent.html}{serious
inequities in access to computers}. Recent research finds that
\href{https://www.nytimes.com/2020/06/05/us/coronavirus-education-lost-learning.html}{most
students fell months behind} during the last term of the year, with the
heaviest impact on low-income students.

Other schools, like Timpanogos, transitioned with less disruption, in
part by mobilizing facilitators, coaches and other staff members to
support both teachers and students who were in danger of logging off and
checking out, according to a report by researchers.

Now, most districts are facing a future in which online courses will
likely be part of the curriculum, whether that entails students
returning in shifts or classrooms remaining closed because of local
outbreaks. And underlying that adjustment is a more fundamental
question: How efficiently do students learn using virtual lessons?

``What we're finding in the research thus far is it's generally harder
to keep students engaged with virtual lessons,'' no matter the content,
said Jered Borup, an associate professor in learning technologies at
George Mason University. ``Over all, though, that is not the
distinguishing feature here. Rather, it's what supports the student has
when learning virtually. That makes all the difference.''

Research comparing in-person to online learning comes from many
disciplines and does not benefit from the kinds of controls that
scientists prefer; courses, teachers, students and class composition
vary too much to make comparisons easily.

Physical presence matters, in ways that are not captured by the
scientific method. ``Look, I did fine in Ms. Hansen's class --- I just
bought the audiobooks and read `Gatsby' on my own,'' one student, Ethan
Avery, said in a phone interview. ``But in some other classes. \ldots{}
I'm personally a terrible procrastinator, and not having that physical
reminder, sitting in class and the teachers grilling me, `Ethan, this is
due Friday,' I fell behind. That was the rough part.''

The two most authoritative reviews of the research to date, examining
the results of nearly 300 studies, come to a similar conclusion.
Students tend to learn less efficiently than usual in online courses, as
a rule, and depending on the course. But if they have a facilitator or
mentor on hand, someone to help with the technology and focus their
attention --- an approach sometimes called blended learning --- they
perform about as well in many virtual classes, and sometimes better.

One state that has applied this approach broadly, for nearly two
decades, is Michigan. A state-supported nonprofit institute called
Michigan Virtual offers scores of online courses, in languages, the
sciences, history and professional development. It also offers 23
virtual advanced placement (A.P.) courses, for college credit.

``We find that if students have support and a schedule --- they do the
lesson every weekday at 9 a.m., for instance --- they tend to do better
than just tuning in here and there,'' said Joe Freidhoff, vice president
of Michigan Virtual. ``The mantra of online learning is, `Your own time,
your own pace, your own path.' In fact, each of these factors matter
greatly, and some structure seems to help.''

In 2012, the institute added a research arm, to track the progress of
its students. In the 2018-19 school year, more than 120,000 students
took at least one of its virtual courses; the vast majority of students
were in high school. The pass rate was 50 percent for those living below
the state's poverty line, and 70 percent for those living above it,
averages roughly in line with the public high schools.

The story was different for Michigan Virtual's A.P. students. In the
2018-19 academic year, 807 students took least one of its virtual A.P.
classes. The final exams are graded on a scale from 1 to 5, with scores
of 3 or above having a chance to earn college credit. The virtual
learners' overall average score was 3.21, compared to 3.04 among
Michigan peers who took the course in a classroom. The national average
on those same tests was 2.89.

``On these exams, our students consistently exceed state and national
averages,'' Dr. Freidhoff said. ``Of course, being A.P. students, they
tend to be very self-directed, motivated students.''

In its scramble to shift courses online in mid-March, the Timpanogos
district put facilitators in place, both for teachers who needed them
and to check in on some students. It lent Chromebooks to every student
that did not have a computer at home. And it implemented a policy that,
by all accounts, took pressure off the sudden transition: Students could
opt for a ``P'' for pass, if struggling with a virtual class, without
taking a hit to their G.P.A.

``It was a little overwhelming at first,'' said Briley Andersen, another
of Ms. Hansen's students. ``My physics and computer science classes were
taking almost all my time, so I ended up taking a P in those.'' She
added, ``As long as there's good communication with a teacher, you get
the hang of it. If not, it takes too long to figure out what you're
supposed to do.''

Michelle Jensen, who is employed by the district as a learning coach,
provided guidance to teachers --- including Ms. Hansen --- and to
students when possible. ``The rationale was, do no harm,'' she said.
``These students are going to have 13 years of education, at least, and
our approach to this one term was, help them learn how to make this
adjustment.''

In a review of Timpanogos's transition, a research team led by Dr. Borup
and Ms. Jensen found that it was largely the nondigital measures that
mattered most. Teachers offered virtual office hours to students, and
contacted them when activity fell off. When those interventions weren't
effective, counselors worked with the family.

The last term of the 2020 school year was, in effect, a hard lesson for
much of the educational system in what virtual classes could and could
not provide. The content is there, and accessible, in any well-prepared
course.

But if the evidence thus far is any guide, virtual education will depend
for its success on old-school principles: creative, attentive teaching
and patient support from parents. As ``The Great Gatsby'' concludes:
``So we beat on, boats against the current, borne back ceaselessly into
the past.''

\textbf{\emph{{[}}\href{http://on.fb.me/1paTQ1h}{\emph{Like the Science
Times page on Facebook.}}} ****** \emph{\textbar{} Sign up for the}
\textbf{\href{http://nyti.ms/1MbHaRU}{\emph{Science Times
newsletter.}}\emph{{]}}}

Advertisement

\protect\hyperlink{after-bottom}{Continue reading the main story}

\hypertarget{site-index}{%
\subsection{Site Index}\label{site-index}}

\hypertarget{site-information-navigation}{%
\subsection{Site Information
Navigation}\label{site-information-navigation}}

\begin{itemize}
\tightlist
\item
  \href{https://help.nytimes.com/hc/en-us/articles/115014792127-Copyright-notice}{©~2020~The
  New York Times Company}
\end{itemize}

\begin{itemize}
\tightlist
\item
  \href{https://www.nytco.com/}{NYTCo}
\item
  \href{https://help.nytimes.com/hc/en-us/articles/115015385887-Contact-Us}{Contact
  Us}
\item
  \href{https://www.nytco.com/careers/}{Work with us}
\item
  \href{https://nytmediakit.com/}{Advertise}
\item
  \href{http://www.tbrandstudio.com/}{T Brand Studio}
\item
  \href{https://www.nytimes.com/privacy/cookie-policy\#how-do-i-manage-trackers}{Your
  Ad Choices}
\item
  \href{https://www.nytimes.com/privacy}{Privacy}
\item
  \href{https://help.nytimes.com/hc/en-us/articles/115014893428-Terms-of-service}{Terms
  of Service}
\item
  \href{https://help.nytimes.com/hc/en-us/articles/115014893968-Terms-of-sale}{Terms
  of Sale}
\item
  \href{https://spiderbites.nytimes.com}{Site Map}
\item
  \href{https://help.nytimes.com/hc/en-us}{Help}
\item
  \href{https://www.nytimes.com/subscription?campaignId=37WXW}{Subscriptions}
\end{itemize}
