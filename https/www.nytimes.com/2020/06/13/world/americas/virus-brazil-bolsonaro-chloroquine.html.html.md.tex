Sections

SEARCH

\protect\hyperlink{site-content}{Skip to
content}\protect\hyperlink{site-index}{Skip to site index}

\href{https://www.nytimes.com/section/world/americas}{Americas}

\href{https://myaccount.nytimes.com/auth/login?response_type=cookie\&client_id=vi}{}

\href{https://www.nytimes.com/section/todayspaper}{Today's Paper}

\href{/section/world/americas}{Americas}\textbar{}Brazil President
Embraces Unproven `Cure' as Pandemic Surges

\url{https://nyti.ms/2Ax0c6d}

\begin{itemize}
\item
\item
\item
\item
\item
\end{itemize}

\href{https://www.nytimes.com/news-event/coronavirus?action=click\&pgtype=Article\&state=default\&region=TOP_BANNER\&context=storylines_menu}{The
Coronavirus Outbreak}

\begin{itemize}
\tightlist
\item
  live\href{https://www.nytimes.com/2020/08/04/world/coronavirus-cases.html?action=click\&pgtype=Article\&state=default\&region=TOP_BANNER\&context=storylines_menu}{Latest
  Updates}
\item
  \href{https://www.nytimes.com/interactive/2020/us/coronavirus-us-cases.html?action=click\&pgtype=Article\&state=default\&region=TOP_BANNER\&context=storylines_menu}{Maps
  and Cases}
\item
  \href{https://www.nytimes.com/interactive/2020/science/coronavirus-vaccine-tracker.html?action=click\&pgtype=Article\&state=default\&region=TOP_BANNER\&context=storylines_menu}{Vaccine
  Tracker}
\item
  \href{https://www.nytimes.com/2020/08/02/us/covid-college-reopening.html?action=click\&pgtype=Article\&state=default\&region=TOP_BANNER\&context=storylines_menu}{College
  Reopening}
\item
  \href{https://www.nytimes.com/live/2020/08/04/business/stock-market-today-coronavirus?action=click\&pgtype=Article\&state=default\&region=TOP_BANNER\&context=storylines_menu}{Economy}
\end{itemize}

Advertisement

\protect\hyperlink{after-top}{Continue reading the main story}

Supported by

\protect\hyperlink{after-sponsor}{Continue reading the main story}

\hypertarget{brazil-president-embraces-unproven-cure-as-pandemic-surges}{%
\section{Brazil President Embraces Unproven `Cure' as Pandemic
Surges}\label{brazil-president-embraces-unproven-cure-as-pandemic-surges}}

President Jair Bolsonaro hailed hydroxychloroquine as a godsend while he
railed against quarantine measures and other best practices, undermining
the country's coronavirus response.

\includegraphics{https://static01.nyt.com/images/2020/06/13/world/13brazil-hydroxychloroquine/13brazil-hydroxychloroquine-articleLarge.jpg?quality=75\&auto=webp\&disable=upscale}

By \href{https://www.nytimes.com/by/ernesto-londono}{Ernesto Londoño}
and Mariana Simões

\begin{itemize}
\item
  Published June 13, 2020Updated June 18, 2020
\item
  \begin{itemize}
  \item
  \item
  \item
  \item
  \item
  \end{itemize}
\end{itemize}

\emph{{[}Read more on}
\href{https://www.nytimes.com/article/brazil-coronavirus-cases.html}{\emph{Brazil's
Coronavirus cases and deaths}}\emph{.{]}}

RIO DE JANEIRO --- The coronavirus was taking root in Latin America when
President
\href{https://www.nytimes.com/2020/07/07/world/americas/brazil-bolsonaro-coronavirus.html}{Jair
Bolsonaro} of Brazil startled the medical community with a claim: A
miracle drug was on hand.

``God is Brazilian, the cure is right here!''
\href{https://oglobo.globo.com/sociedade/coronavirus/deus-brasileiro-a-cura-ta-ai-diz-bolsonaro-sobre-remedio-ainda-em-teste-contra-covid-19-1-24337060}{the
president exclaimed} in late March to a throng of supporters.
``\href{https://www.nytimes.com/2020/06/15/health/fda-hydroxychloroquine-malaria.html}{Chloroquine}
is working everywhere.''

Since then, the virus has ripped through Brazil. More than 41,000 people
have died --- Brazil has now passed Britain and has recorded more
fatalities than any country other than the United States --- and the
daily death toll is now the highest in the world, bucking the downward
trend that is allowing other major economies to reopen.

Experts point to Mr. Bolsonaro's rejection of the emerging scientific
consensus on how to fight the pandemic --- including his promotion of
unproven remedies such as chloroquine and
\href{https://www.nytimes.com/2020/06/20/health/hydroxychloroquine-coronavirus-trial.html}{hydroxychloroquine}
--- as one of the factors that helped tilt the country into its
\href{https://www.nytimes.com/2020/05/12/world/americas/latin-america-virus-death.html}{current
health crisis}.

\includegraphics{https://static01.nyt.com/images/2020/06/12/world/12Brazil-Hydroxychloroquine2/merlin_173248497_62cf155e-e9f1-43f0-a971-b52978539f08-articleLarge.jpg?quality=75\&auto=webp\&disable=upscale}

Mr. Bolsonaro ordered the armed forces to mass produce it in the
military's pharmaceutical laboratory and ordered a large supply of the
drug's ingredients from India.

``Decisions are being made not based on evidence and empirical data but
rather on anecdotal reports,'' said
\href{https://www.sabin.org/updates/pressreleases/dr-denise-garrett-joins-sabin-director-coalition-against-typhoid}{Denise
Garrett}, a Brazilian-American epidemiologist who worked at the Centers
for Disease Control and Prevention for more than 20 years. ``Bolsonaro
invested a huge amount of money into an action that has not been proven
to be effective at the expense of increasing testing and contact
tracing.''

Between February, when Brazil identified its
\href{https://www.nytimes.com/2020/02/26/world/americas/brazil-italy-coronavirus.html}{first
coronavirus case}, and June, when Brazil's coronavirus caseload topped
828,000, lagging only behind the United States, the country had months
to learn from other nations that had been ravaged by the virus and
prepare for the pandemic.

Instead, Mr. Bolsonaro has led the country down what health experts call
a perilous path: He sabotaged quarantine measures adopted by governors,
encouraged mass rallies and repeatedly dismissed the danger of the
virus, asserting that it was a ``measly cold'' and that people with
``athletic backgrounds,'' like himself, were impervious to serious
complications.

Earlier this week, Mr. Bolsonaro's administration
\href{https://www.nytimes.com/2020/06/08/world/americas/brazil-coronavirus-statistics.html}{stopped
disclosing comprehensive coronavirus statistics,} leaving Brazilians
without an official tally showing the trajectory and scope of the
outbreak. The data was restored after the Supreme Court ordered the
ministry to resume publishing it.

\hypertarget{latest-updates-global-coronavirus-outbreak}{%
\section{\texorpdfstring{\href{https://www.nytimes.com/2020/08/04/world/coronavirus-cases.html?action=click\&pgtype=Article\&state=default\&region=MAIN_CONTENT_1\&context=storylines_live_updates}{Latest
Updates: Global Coronavirus
Outbreak}}{Latest Updates: Global Coronavirus Outbreak}}\label{latest-updates-global-coronavirus-outbreak}}

Updated 2020-08-04T19:28:21.450Z

\begin{itemize}
\tightlist
\item
  \href{https://www.nytimes.com/2020/08/04/world/coronavirus-cases.html?action=click\&pgtype=Article\&state=default\&region=MAIN_CONTENT_1\&context=storylines_live_updates\#link-4825b93}{Public
  and private schools in Maryland and elsewhere are divided over
  in-person instruction.}
\item
  \href{https://www.nytimes.com/2020/08/04/world/coronavirus-cases.html?action=click\&pgtype=Article\&state=default\&region=MAIN_CONTENT_1\&context=storylines_live_updates\#link-4d1eafa8}{N.Y.C.'s
  health commissioner resigns after clashing with the mayor over the
  virus.}
\item
  \href{https://www.nytimes.com/2020/08/04/world/coronavirus-cases.html?action=click\&pgtype=Article\&state=default\&region=MAIN_CONTENT_1\&context=storylines_live_updates\#link-6b644638}{`Long
  days, long nights': Washington prepares for a prolonged fight over
  virus relief.}
\end{itemize}

\href{https://www.nytimes.com/2020/08/04/world/coronavirus-cases.html?action=click\&pgtype=Article\&state=default\&region=MAIN_CONTENT_1\&context=storylines_live_updates}{See
more updates}

More live coverage:
\href{https://www.nytimes.com/live/2020/08/04/business/stock-market-today-coronavirus?action=click\&pgtype=Article\&state=default\&region=MAIN_CONTENT_1\&context=storylines_live_updates}{Markets}

Under his tenure, decisions about medical and scientific protocols
became measures of political loyalty. As the coronavirus crisis
worsened, Mr. Bolsonaro leaned on the health ministry to embrace
widespread use of chloroquine and hydroxychloroquine, straining his
relationship with the two physicians who have served as health
ministers. One was fired in April and the other one
\href{https://www.nytimes.com/2020/05/15/world/americas/brazil-health-minister-bolsonaro.html}{lasted
less than a month}on the job.

Image

A protest against Mr. Bolsonaro last month in São Paulo.Credit...Victor
Moriyama for The New York Times

Their successor, an active duty general with no medical experience,
agreed to
\href{https://www.saude.gov.br/noticias/agencia-saude/46919-ministerio-da-saude-divulga-diretrizes-para-tratamento-medicamentoso-de-pacientes}{issue
guidance}encouraging doctors to prescribe the drug widely for Covid-19
patients.

Chloroquine and Hydroxychloroquine are both anti-malaria drugs but they
have distinct secondary uses;
\href{https://www.nytimes.com/article/hydroxychloroquine-coronavirus.html}{hydroxychloroquine
also treats lupus and rheumatoid arthritis}. Both drugs are among the
pharmaceuticals being studied as potential remedies for Covid-19, but
neither drug has been approved as a reliable treatment for Covid-19
patients.

The United States Food and Drug Administration
\href{https://www.fda.gov/drugs/drug-safety-and-availability/fda-cautions-against-use-hydroxychloroquine-or-chloroquine-covid-19-outside-hospital-setting-or}{warned
against} use of the two drugs in Covid-19 patients outside of hospital
settings because they can cause heart problems.

Margareth Dalcolmo, a prominent pulmonologist and researcher at Fiocruz,
a government agency that does health care research in Rio de Janeiro,
said Brazil's embrace of the drug set a dangerous precedent --- and is
hampering the necessary research.

``Today chloroquine became a political panacea, which is harmful for
science,'' she said in an interview. ``What we have, as I see it, is an
unfortunate politicization of pharmaceuticals.''

The controversy over hydroxychloroquine has also reverberated outside
Brazil.

In mid-May, President Trump
\href{https://www.nytimes.com/2020/05/18/us/politics/trump-hydroxychloroquine-covid-coronavirus.html}{said
he had begun taking the drug}as a preventive measure, which generated
consternation among doctors.

Later that month, the White House announced it was donating two million
doses of the drug to Brazil so
\href{https://www.whitehouse.gov/briefings-statements/joint-statement-united-states-america-federative-republic-brazil-regarding-health-cooperation/}{it
could be used} ``to treat Brazilians who become infected.''

Image

President Trump said last month that he has been taking the drug as a
preventive measure.Credit...Doug Mills/The New York Times

Representative Eliot Engel, the New York Democrat who chairs the House
Foreign Affairs Committee, called that decision appalling. ``It's
irresponsible that President Trump and Jair Bolsonaro have put politics
over science,''
\href{https://twitter.com/HouseForeign/status/1267826331930128386?s=20}{he
said in a statement on Twitter.}

Studies into the drug's uses continue. Earlier this month, the
\href{https://www.nytimes.com/2020/06/03/health/hydroxychloroquine-coronavirus-trump.html}{first
carefully controlled trial} of hydroxychloroquine found the drug does
not prevent the disease in people who have been exposed to a sick
patient.

Also, a prominent medical journal, The Lancet, took the rare step
of\href{https://www.nytimes.com/reuters/2020/06/05/world/europe/05reuters-health-coronavirus-hydroxychloroquine-lancet.html}{retracting
a widely-read study} earlier this month that found the drug could be
dangerous in Covid-19 patients. The study, which was withdrawn after its
underlying data was called into question, led doctors to halt some
clinical trials.

In Brazil, the battle over hydroxychloroquine began in March as doctors
were preparing for a crush of patients and testing a variety of drugs
based on treatment protocols that had shown promise in other countries.

Marcelo Kalichsztein, a prominent pulmonologist in Rio de Janeiro, began
prescribing hydroxychloroquine to coronavirus patients soon after they
developed symptoms, along with the antibiotic azithromycin and a zinc
supplement. He did so having found the research
of\href{https://www.nytimes.com/2020/05/12/magazine/didier-raoult-hydroxychloroquine.html}{the
French microbiologist, Dr. Didier Raoult}, persuasive. But Dr. Raoult's
\href{https://retractionwatch.com/2020/04/06/hydroxychlorine-covid-19-study-did-not-meet-publishing-societys-expected-standard/}{research
was discredited}, and the scientific group that published it said later
that the paper had not met its standards.

Image

The French microbiologist Didier Raoult's research into
hydroxychloroquine as a potential treatment for Covid-19 has been
subject to criticisms over methodology.Credit...Christophe Simon/Agence
France-Presse --- Getty Images

``This is a brand-new disease and we don't have a silver bullet,'' Dr.
Kalichsztein said. ``We were all searching for a medication that would
stop the virus in the very first stage.''

\href{https://www.nytimes.com/news-event/coronavirus?action=click\&pgtype=Article\&state=default\&region=MAIN_CONTENT_3\&context=storylines_faq}{}

\hypertarget{the-coronavirus-outbreak-}{%
\subsubsection{The Coronavirus Outbreak
›}\label{the-coronavirus-outbreak-}}

\hypertarget{frequently-asked-questions}{%
\paragraph{Frequently Asked
Questions}\label{frequently-asked-questions}}

Updated August 4, 2020

\begin{itemize}
\item ~
  \hypertarget{i-have-antibodies-am-i-now-immune}{%
  \paragraph{I have antibodies. Am I now
  immune?}\label{i-have-antibodies-am-i-now-immune}}

  \begin{itemize}
  \tightlist
  \item
    As of right
    now,\href{https://www.nytimes.com/2020/07/22/health/covid-antibodies-herd-immunity.html?action=click\&pgtype=Article\&state=default\&region=MAIN_CONTENT_3\&context=storylines_faq}{that
    seems likely, for at least several months.} There have been
    frightening accounts of people suffering what seems to be a second
    bout of Covid-19. But experts say these patients may have a
    drawn-out course of infection, with the virus taking a slow toll
    weeks to months after initial exposure. People infected with the
    coronavirus typically
    \href{https://www.nature.com/articles/s41586-020-2456-9}{produce}
    immune molecules called antibodies, which are
    \href{https://www.nytimes.com/2020/05/07/health/coronavirus-antibody-prevalence.html?action=click\&pgtype=Article\&state=default\&region=MAIN_CONTENT_3\&context=storylines_faq}{protective
    proteins made in response to an
    infection}\href{https://www.nytimes.com/2020/05/07/health/coronavirus-antibody-prevalence.html?action=click\&pgtype=Article\&state=default\&region=MAIN_CONTENT_3\&context=storylines_faq}{.
    These antibodies may} last in the body
    \href{https://www.nature.com/articles/s41591-020-0965-6}{only two to
    three months}, which may seem worrisome, but that's perfectly normal
    after an acute infection subsides, said Dr. Michael Mina, an
    immunologist at Harvard University. It may be possible to get the
    coronavirus again, but it's highly unlikely that it would be
    possible in a short window of time from initial infection or make
    people sicker the second time.
  \end{itemize}
\item ~
  \hypertarget{im-a-small-business-owner-can-i-get-relief}{%
  \paragraph{I'm a small-business owner. Can I get
  relief?}\label{im-a-small-business-owner-can-i-get-relief}}

  \begin{itemize}
  \tightlist
  \item
    The
    \href{https://www.nytimes.com/article/small-business-loans-stimulus-grants-freelancers-coronavirus.html?action=click\&pgtype=Article\&state=default\&region=MAIN_CONTENT_3\&context=storylines_faq}{stimulus
    bills enacted in March} offer help for the millions of American
    small businesses. Those eligible for aid are businesses and
    nonprofit organizations with fewer than 500 workers, including sole
    proprietorships, independent contractors and freelancers. Some
    larger companies in some industries are also eligible. The help
    being offered, which is being managed by the Small Business
    Administration, includes the Paycheck Protection Program and the
    Economic Injury Disaster Loan program. But lots of folks have
    \href{https://www.nytimes.com/interactive/2020/05/07/business/small-business-loans-coronavirus.html?action=click\&pgtype=Article\&state=default\&region=MAIN_CONTENT_3\&context=storylines_faq}{not
    yet seen payouts.} Even those who have received help are confused:
    The rules are draconian, and some are stuck sitting on
    \href{https://www.nytimes.com/2020/05/02/business/economy/loans-coronavirus-small-business.html?action=click\&pgtype=Article\&state=default\&region=MAIN_CONTENT_3\&context=storylines_faq}{money
    they don't know how to use.} Many small-business owners are getting
    less than they expected or
    \href{https://www.nytimes.com/2020/06/10/business/Small-business-loans-ppp.html?action=click\&pgtype=Article\&state=default\&region=MAIN_CONTENT_3\&context=storylines_faq}{not
    hearing anything at all.}
  \end{itemize}
\item ~
  \hypertarget{what-are-my-rights-if-i-am-worried-about-going-back-to-work}{%
  \paragraph{What are my rights if I am worried about going back to
  work?}\label{what-are-my-rights-if-i-am-worried-about-going-back-to-work}}

  \begin{itemize}
  \tightlist
  \item
    Employers have to provide
    \href{https://www.osha.gov/SLTC/covid-19/standards.html}{a safe
    workplace} with policies that protect everyone equally.
    \href{https://www.nytimes.com/article/coronavirus-money-unemployment.html?action=click\&pgtype=Article\&state=default\&region=MAIN_CONTENT_3\&context=storylines_faq}{And
    if one of your co-workers tests positive for the coronavirus, the
    C.D.C.} has said that
    \href{https://www.cdc.gov/coronavirus/2019-ncov/community/guidance-business-response.html}{employers
    should tell their employees} -\/- without giving you the sick
    employee's name -\/- that they may have been exposed to the virus.
  \end{itemize}
\item ~
  \hypertarget{should-i-refinance-my-mortgage}{%
  \paragraph{Should I refinance my
  mortgage?}\label{should-i-refinance-my-mortgage}}

  \begin{itemize}
  \tightlist
  \item
    \href{https://www.nytimes.com/article/coronavirus-money-unemployment.html?action=click\&pgtype=Article\&state=default\&region=MAIN_CONTENT_3\&context=storylines_faq}{It
    could be a good idea,} because mortgage rates have
    \href{https://www.nytimes.com/2020/07/16/business/mortgage-rates-below-3-percent.html?action=click\&pgtype=Article\&state=default\&region=MAIN_CONTENT_3\&context=storylines_faq}{never
    been lower.} Refinancing requests have pushed mortgage applications
    to some of the highest levels since 2008, so be prepared to get in
    line. But defaults are also up, so if you're thinking about buying a
    home, be aware that some lenders have tightened their standards.
  \end{itemize}
\item ~
  \hypertarget{what-is-school-going-to-look-like-in-september}{%
  \paragraph{What is school going to look like in
  September?}\label{what-is-school-going-to-look-like-in-september}}

  \begin{itemize}
  \tightlist
  \item
    It is unlikely that many schools will return to a normal schedule
    this fall, requiring the grind of
    \href{https://www.nytimes.com/2020/06/05/us/coronavirus-education-lost-learning.html?action=click\&pgtype=Article\&state=default\&region=MAIN_CONTENT_3\&context=storylines_faq}{online
    learning},
    \href{https://www.nytimes.com/2020/05/29/us/coronavirus-child-care-centers.html?action=click\&pgtype=Article\&state=default\&region=MAIN_CONTENT_3\&context=storylines_faq}{makeshift
    child care} and
    \href{https://www.nytimes.com/2020/06/03/business/economy/coronavirus-working-women.html?action=click\&pgtype=Article\&state=default\&region=MAIN_CONTENT_3\&context=storylines_faq}{stunted
    workdays} to continue. California's two largest public school
    districts --- Los Angeles and San Diego --- said on July 13, that
    \href{https://www.nytimes.com/2020/07/13/us/lausd-san-diego-school-reopening.html?action=click\&pgtype=Article\&state=default\&region=MAIN_CONTENT_3\&context=storylines_faq}{instruction
    will be remote-only in the fall}, citing concerns that surging
    coronavirus infections in their areas pose too dire a risk for
    students and teachers. Together, the two districts enroll some
    825,000 students. They are the largest in the country so far to
    abandon plans for even a partial physical return to classrooms when
    they reopen in August. For other districts, the solution won't be an
    all-or-nothing approach.
    \href{https://bioethics.jhu.edu/research-and-outreach/projects/eschool-initiative/school-policy-tracker/}{Many
    systems}, including the nation's largest, New York City, are
    devising
    \href{https://www.nytimes.com/2020/06/26/us/coronavirus-schools-reopen-fall.html?action=click\&pgtype=Article\&state=default\&region=MAIN_CONTENT_3\&context=storylines_faq}{hybrid
    plans} that involve spending some days in classrooms and other days
    online. There's no national policy on this yet, so check with your
    municipal school system regularly to see what is happening in your
    community.
  \end{itemize}
\end{itemize}

Dr. Kalichsztein, who contracted the virus in early April and took
hydroxychloroquine, said the treatment had been effective in preventing
the disease from reaching an inflammatory stage among more than 100
patients whose care he oversaw.

Doctors began sharing their experiences with the drug and tips on how to
mitigate the risk of heart complications in Zoom meetings and group
chats on WhatsApp.

While these discussions were happening out of sight, Nise Yamaguchi, a
São Paulo immunologist and oncologist, emerged as a high profile
champion for the drug, arguing in television interviews that it had the
potential to prevent patients from becoming sick enough to require
hospitalization.

Dr. Yamaguchi, who caught Mr. Bolsonaro's attention and was summoned to
meet with him, said she never intended to become embroiled in the heated
political debate that has added to Brazil's polarization.

``The doctors and scientists that act based on academic research can't
allow themselves to be guided by political matters, since the health of
the patient is paramount,'' she said in an email.

But by mid April, hydroxychloroquine became something of a litmus test
among Brazilians who revere and loath the far-right president, who has
invested a lot of political capital --- and public funds --- in the
drug.

Image

Consumers waited in line to enter a store on Thursday, the first day of
trade reopening on the streets of São Paulo.Credit...Victor Moriyama for
The New York Times

Staunch supporters of the president clamored for more widespread use of
the drug in YouTube videos, memes and tweets that claimed that a
lifesaving drug was being maligned by leftists. They promoted their
posts with hashtags that included \#BolsonaroWasRight,
\#HydroxychloroquineSavesLives and \#HydroxychloroquineNow.

Eduardo Bolsonaro, one of the president's sons, posted a video on his
YouTube channel on April 17 that attacked a study that warned about the
side effects of the drug.

``The left is coming together to create fake studies about
chloroquine,'' \href{https://www.youtube.com/watch?v=q5--FDXaOg4}{he
wrote in the video caption}. ``The objective is to demonize the
medicine, even though they know that it's effective to save lives.''

Mr. Bolsonaro's minister for human rights, Damares Alves, an Evangelical
pastor, \href{https://www.instagram.com/p/CALEk5FAIhx/}{called the
medication a ``miracle.}''

The charged political debate surrounding the use of the drug could
interfere with ongoing trials, said Dr. Garrett, the former C.D.C.
expert.

``Either volunteers won't want to be part of it because they are
contaminated by the political debate or the ones who will be part of it
may be doing it driven by political ideology,'' she said.

And that, she said, would be ``very unfortunate for public health.''

Advertisement

\protect\hyperlink{after-bottom}{Continue reading the main story}

\hypertarget{site-index}{%
\subsection{Site Index}\label{site-index}}

\hypertarget{site-information-navigation}{%
\subsection{Site Information
Navigation}\label{site-information-navigation}}

\begin{itemize}
\tightlist
\item
  \href{https://help.nytimes.com/hc/en-us/articles/115014792127-Copyright-notice}{©~2020~The
  New York Times Company}
\end{itemize}

\begin{itemize}
\tightlist
\item
  \href{https://www.nytco.com/}{NYTCo}
\item
  \href{https://help.nytimes.com/hc/en-us/articles/115015385887-Contact-Us}{Contact
  Us}
\item
  \href{https://www.nytco.com/careers/}{Work with us}
\item
  \href{https://nytmediakit.com/}{Advertise}
\item
  \href{http://www.tbrandstudio.com/}{T Brand Studio}
\item
  \href{https://www.nytimes.com/privacy/cookie-policy\#how-do-i-manage-trackers}{Your
  Ad Choices}
\item
  \href{https://www.nytimes.com/privacy}{Privacy}
\item
  \href{https://help.nytimes.com/hc/en-us/articles/115014893428-Terms-of-service}{Terms
  of Service}
\item
  \href{https://help.nytimes.com/hc/en-us/articles/115014893968-Terms-of-sale}{Terms
  of Sale}
\item
  \href{https://spiderbites.nytimes.com}{Site Map}
\item
  \href{https://help.nytimes.com/hc/en-us}{Help}
\item
  \href{https://www.nytimes.com/subscription?campaignId=37WXW}{Subscriptions}
\end{itemize}
