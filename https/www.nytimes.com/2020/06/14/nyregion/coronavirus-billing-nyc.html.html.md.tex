Sections

SEARCH

\protect\hyperlink{site-content}{Skip to
content}\protect\hyperlink{site-index}{Skip to site index}

\href{https://www.nytimes.com/section/nyregion}{New York}

\href{https://myaccount.nytimes.com/auth/login?response_type=cookie\&client_id=vi}{}

\href{https://www.nytimes.com/section/todayspaper}{Today's Paper}

\href{/section/nyregion}{New York}\textbar{}She Survived the
Coronavirus. Then She Got a \$400,000 Medical Bill.

\url{https://nyti.ms/3fuikN3}

\begin{itemize}
\item
\item
\item
\item
\item
\item
\end{itemize}

\href{https://www.nytimes.com/news-event/coronavirus?action=click\&pgtype=Article\&state=default\&region=TOP_BANNER\&context=storylines_menu}{The
Coronavirus Outbreak}

\begin{itemize}
\tightlist
\item
  live\href{https://www.nytimes.com/2020/08/04/world/coronavirus-cases.html?action=click\&pgtype=Article\&state=default\&region=TOP_BANNER\&context=storylines_menu}{Latest
  Updates}
\item
  \href{https://www.nytimes.com/interactive/2020/us/coronavirus-us-cases.html?action=click\&pgtype=Article\&state=default\&region=TOP_BANNER\&context=storylines_menu}{Maps
  and Cases}
\item
  \href{https://www.nytimes.com/interactive/2020/science/coronavirus-vaccine-tracker.html?action=click\&pgtype=Article\&state=default\&region=TOP_BANNER\&context=storylines_menu}{Vaccine
  Tracker}
\item
  \href{https://www.nytimes.com/2020/08/02/us/covid-college-reopening.html?action=click\&pgtype=Article\&state=default\&region=TOP_BANNER\&context=storylines_menu}{College
  Reopening}
\item
  \href{https://www.nytimes.com/live/2020/08/04/business/stock-market-today-coronavirus?action=click\&pgtype=Article\&state=default\&region=TOP_BANNER\&context=storylines_menu}{Economy}
\end{itemize}

Advertisement

\protect\hyperlink{after-top}{Continue reading the main story}

Supported by

\protect\hyperlink{after-sponsor}{Continue reading the main story}

\hypertarget{she-survived-the-coronavirus-then-she-got-a-400000-medical-bill}{%
\section{She Survived the Coronavirus. Then She Got a \$400,000 Medical
Bill.}\label{she-survived-the-coronavirus-then-she-got-a-400000-medical-bill}}

Patients who were treated for the virus are largely supposed to be
exempt from receiving large bills. One hospital erroneously~sent one.

\includegraphics{https://static01.nyt.com/images/2020/06/11/nyregion/00nyvirus-billing/00nyvirus-billing-articleLarge.jpg?quality=75\&auto=webp\&disable=upscale}

\href{https://www.nytimes.com/by/joseph-goldstein}{\includegraphics{https://static01.nyt.com/images/2018/07/16/multimedia/author-joseph-goldstein/author-joseph-goldstein-thumbLarge.png}}

By \href{https://www.nytimes.com/by/joseph-goldstein}{Joseph Goldstein}

\begin{itemize}
\item
  Published June 14, 2020Updated June 15, 2020
\item
  \begin{itemize}
  \item
  \item
  \item
  \item
  \item
  \item
  \end{itemize}
\end{itemize}

Janet Mendez started receiving bills soon after returning in April to
her mother's home from Mount Sinai Morningside hospital, where she
nearly died of Covid-19. First, there was one for \$31,165. Unable to
work and finding it difficult to walk, Ms. Mendez decided to put the
bill out of her mind and focus on her recovery.

The next one was impossible to ignore: an invoice for \$401,885.57,
although it noted that the hospital would reduce the bill by
\$326,851.63 as a ``financial assistance benefit.'' But that still left
a tab of more than \$75,000.

``Oh my God, how am I going to pay all this money?'' Ms. Mendez, 33,
recalled thinking. The answer came to her in about a second: ``I'm not
going to be able to pay all this.''

Ms. Mendez is optimistic that her insurance company will cover a large
part of the costs, but only after receiving a series of harassing phone
calls from the hospital about payment.

A spokesman for the hospital told The Times that Ms. Mendez erroneously
received a bill that should have gone directly to her insurance company
or the government. Coronavirus patients, through a series of federal aid
packages, are supposed to be largely exempt from paying for the bulk of
their care.

But mistakes are likely to occur, particularly given the number of
people who have recently lost their health insurance amid an economic
downturn and widespread job loss. And when they do happen, patients like
Ms. Mendez will be the ones to have to sort out the complicated billing
process at a time when they are still recovering from Covid-19.

``We're looking at a tsunami,'' said Elisabeth Benjamin, a vice
president at the Community Service Society of New York, which is trying
to help Ms. Mendez get her bill reduced. ``The earthquake has struck,
and now we're waiting for the bills to roll on in.''

\hypertarget{latest-updates-global-coronavirus-outbreak}{%
\section{\texorpdfstring{\href{https://www.nytimes.com/2020/08/04/world/coronavirus-cases.html?action=click\&pgtype=Article\&state=default\&region=MAIN_CONTENT_1\&context=storylines_live_updates}{Latest
Updates: Global Coronavirus
Outbreak}}{Latest Updates: Global Coronavirus Outbreak}}\label{latest-updates-global-coronavirus-outbreak}}

Updated 2020-08-05T07:58:24.076Z

\begin{itemize}
\tightlist
\item
  \href{https://www.nytimes.com/2020/08/04/world/coronavirus-cases.html?action=click\&pgtype=Article\&state=default\&region=MAIN_CONTENT_1\&context=storylines_live_updates\#link-762df92}{As
  talks drag on, McConnell signals openness to jobless aid extension,
  and negotiators agree on a deadline.}
\item
  \href{https://www.nytimes.com/2020/08/04/world/coronavirus-cases.html?action=click\&pgtype=Article\&state=default\&region=MAIN_CONTENT_1\&context=storylines_live_updates\#link-1228a480}{Novavax
  sees encouraging results from two studies of its experimental
  vaccine.}
\item
  \href{https://www.nytimes.com/2020/08/04/world/coronavirus-cases.html?action=click\&pgtype=Article\&state=default\&region=MAIN_CONTENT_1\&context=storylines_live_updates\#link-794484ed}{Mississippians
  must now wear masks in public, governor says.}
\end{itemize}

\href{https://www.nytimes.com/2020/08/04/world/coronavirus-cases.html?action=click\&pgtype=Article\&state=default\&region=MAIN_CONTENT_1\&context=storylines_live_updates}{See
more updates}

More live coverage:
\href{https://www.nytimes.com/live/2020/08/04/business/stock-market-today-coronavirus?action=click\&pgtype=Article\&state=default\&region=MAIN_CONTENT_1\&context=storylines_live_updates}{Markets}

When Ms. Mendez got over the initial shock and examined her bill more
closely, she was struck by how vague and arbitrary the charges seemed.
She was billed \$3,550 for ``inpatient charges'' and another \$42,714.52
for ``pharmacy,'' but without any breakdown of what medicines she
received or how much each cost.

Ms. Mendez said the bill should have at least been itemized, listing
each drug she was being charged for --- and the price. She was, after
all, unconscious for much of her hospitalization.

``I don't know what medicines they put in me,'' she said. ``I can't say
they did this, or they didn't do this.''

Most of the line items on her hospital bill are vague. Some of the most
expensive are four entries that simply read ``Medical --- Cardiac
Care.'' Each one ranges from \$41,000 to \$82,000.

Part of the confusion was that Ms. Mendez had recently changed health
insurers, and she had arrived at the hospital struggling to breathe and
without her new insurance information. The hospital billing department
concluded she was uninsured and sent her a bill directly.

``To be clear, neither this patient nor any Mount Sinai patient should
receive a bill or be expected to directly pay for their Covid-19 care,''
a spokesman for Mount Sinai Health System, Jason Kaplan, wrote in an
email, describing it as an isolated error.

Image

Ms. Mendez's insurance company told her she is likely to owe less than
\$10,000 for her treatment.~Credit...Calla Kessler/The New York Times

While eye-popping medical bills are nothing new, Covid-19 patients are
supposed to be largely exempt. During Ms. Mendez's hospitalization, a
huge bailout of hospitals was taking shape.

In New York City, hospitals received more than \$3 billion in federal
payments last month from an early round of bailout payments. The
hospital where Ms. Mendez was treated, Mount Sinai Morningside (formerly
Mount Sinai St. Luke's),
\href{https://data.cdc.gov/Administrative/Provider-Relief-Fund-COVID-19-High-Impact-Payments/b58h-s9zx/data}{received}
at least \$63.7 million.

The federal dollars are intended to help compensate hospitals and health
care providers for the expense of treating Covid-19 patients like Ms.
Mendez. The money is also meant to help make up for
\href{https://www.hhs.gov/coronavirus/cares-act-provider-relief-fund/index.html\#collapseTwo}{the
revenue hospitals} lost as elective procedures were canceled and
non-Covid patients dwindled.

The money comes with some conditions that are intended to protect
patients from medical debt. For instance, health care providers are not
permitted to
\href{https://www.hhs.gov/sites/default/files/provider-relief-fund-general-distribution-faqs.pdf}{seek
extra payment} from patients with health insurance who received care at
an out-of-network hospital. Nor can they ``balance-bill'' --- that is,
bill the patient for the difference between what the insurer will pay
and the hospital's charges.

But the protections do not fully insulate patients. Even if a hospital
takes federal money, some of the doctors who treat patients there can
send their own bills to patients directly.

\href{https://www.nytimes.com/news-event/coronavirus?action=click\&pgtype=Article\&state=default\&region=MAIN_CONTENT_3\&context=storylines_faq}{}

\hypertarget{the-coronavirus-outbreak-}{%
\subsubsection{The Coronavirus Outbreak
›}\label{the-coronavirus-outbreak-}}

\hypertarget{frequently-asked-questions}{%
\paragraph{Frequently Asked
Questions}\label{frequently-asked-questions}}

Updated August 4, 2020

\begin{itemize}
\item ~
  \hypertarget{i-have-antibodies-am-i-now-immune}{%
  \paragraph{I have antibodies. Am I now
  immune?}\label{i-have-antibodies-am-i-now-immune}}

  \begin{itemize}
  \tightlist
  \item
    As of right
    now,\href{https://www.nytimes.com/2020/07/22/health/covid-antibodies-herd-immunity.html?action=click\&pgtype=Article\&state=default\&region=MAIN_CONTENT_3\&context=storylines_faq}{that
    seems likely, for at least several months.} There have been
    frightening accounts of people suffering what seems to be a second
    bout of Covid-19. But experts say these patients may have a
    drawn-out course of infection, with the virus taking a slow toll
    weeks to months after initial exposure. People infected with the
    coronavirus typically
    \href{https://www.nature.com/articles/s41586-020-2456-9}{produce}
    immune molecules called antibodies, which are
    \href{https://www.nytimes.com/2020/05/07/health/coronavirus-antibody-prevalence.html?action=click\&pgtype=Article\&state=default\&region=MAIN_CONTENT_3\&context=storylines_faq}{protective
    proteins made in response to an
    infection}\href{https://www.nytimes.com/2020/05/07/health/coronavirus-antibody-prevalence.html?action=click\&pgtype=Article\&state=default\&region=MAIN_CONTENT_3\&context=storylines_faq}{.
    These antibodies may} last in the body
    \href{https://www.nature.com/articles/s41591-020-0965-6}{only two to
    three months}, which may seem worrisome, but that's perfectly normal
    after an acute infection subsides, said Dr. Michael Mina, an
    immunologist at Harvard University. It may be possible to get the
    coronavirus again, but it's highly unlikely that it would be
    possible in a short window of time from initial infection or make
    people sicker the second time.
  \end{itemize}
\item ~
  \hypertarget{im-a-small-business-owner-can-i-get-relief}{%
  \paragraph{I'm a small-business owner. Can I get
  relief?}\label{im-a-small-business-owner-can-i-get-relief}}

  \begin{itemize}
  \tightlist
  \item
    The
    \href{https://www.nytimes.com/article/small-business-loans-stimulus-grants-freelancers-coronavirus.html?action=click\&pgtype=Article\&state=default\&region=MAIN_CONTENT_3\&context=storylines_faq}{stimulus
    bills enacted in March} offer help for the millions of American
    small businesses. Those eligible for aid are businesses and
    nonprofit organizations with fewer than 500 workers, including sole
    proprietorships, independent contractors and freelancers. Some
    larger companies in some industries are also eligible. The help
    being offered, which is being managed by the Small Business
    Administration, includes the Paycheck Protection Program and the
    Economic Injury Disaster Loan program. But lots of folks have
    \href{https://www.nytimes.com/interactive/2020/05/07/business/small-business-loans-coronavirus.html?action=click\&pgtype=Article\&state=default\&region=MAIN_CONTENT_3\&context=storylines_faq}{not
    yet seen payouts.} Even those who have received help are confused:
    The rules are draconian, and some are stuck sitting on
    \href{https://www.nytimes.com/2020/05/02/business/economy/loans-coronavirus-small-business.html?action=click\&pgtype=Article\&state=default\&region=MAIN_CONTENT_3\&context=storylines_faq}{money
    they don't know how to use.} Many small-business owners are getting
    less than they expected or
    \href{https://www.nytimes.com/2020/06/10/business/Small-business-loans-ppp.html?action=click\&pgtype=Article\&state=default\&region=MAIN_CONTENT_3\&context=storylines_faq}{not
    hearing anything at all.}
  \end{itemize}
\item ~
  \hypertarget{what-are-my-rights-if-i-am-worried-about-going-back-to-work}{%
  \paragraph{What are my rights if I am worried about going back to
  work?}\label{what-are-my-rights-if-i-am-worried-about-going-back-to-work}}

  \begin{itemize}
  \tightlist
  \item
    Employers have to provide
    \href{https://www.osha.gov/SLTC/covid-19/standards.html}{a safe
    workplace} with policies that protect everyone equally.
    \href{https://www.nytimes.com/article/coronavirus-money-unemployment.html?action=click\&pgtype=Article\&state=default\&region=MAIN_CONTENT_3\&context=storylines_faq}{And
    if one of your co-workers tests positive for the coronavirus, the
    C.D.C.} has said that
    \href{https://www.cdc.gov/coronavirus/2019-ncov/community/guidance-business-response.html}{employers
    should tell their employees} -\/- without giving you the sick
    employee's name -\/- that they may have been exposed to the virus.
  \end{itemize}
\item ~
  \hypertarget{should-i-refinance-my-mortgage}{%
  \paragraph{Should I refinance my
  mortgage?}\label{should-i-refinance-my-mortgage}}

  \begin{itemize}
  \tightlist
  \item
    \href{https://www.nytimes.com/article/coronavirus-money-unemployment.html?action=click\&pgtype=Article\&state=default\&region=MAIN_CONTENT_3\&context=storylines_faq}{It
    could be a good idea,} because mortgage rates have
    \href{https://www.nytimes.com/2020/07/16/business/mortgage-rates-below-3-percent.html?action=click\&pgtype=Article\&state=default\&region=MAIN_CONTENT_3\&context=storylines_faq}{never
    been lower.} Refinancing requests have pushed mortgage applications
    to some of the highest levels since 2008, so be prepared to get in
    line. But defaults are also up, so if you're thinking about buying a
    home, be aware that some lenders have tightened their standards.
  \end{itemize}
\item ~
  \hypertarget{what-is-school-going-to-look-like-in-september}{%
  \paragraph{What is school going to look like in
  September?}\label{what-is-school-going-to-look-like-in-september}}

  \begin{itemize}
  \tightlist
  \item
    It is unlikely that many schools will return to a normal schedule
    this fall, requiring the grind of
    \href{https://www.nytimes.com/2020/06/05/us/coronavirus-education-lost-learning.html?action=click\&pgtype=Article\&state=default\&region=MAIN_CONTENT_3\&context=storylines_faq}{online
    learning},
    \href{https://www.nytimes.com/2020/05/29/us/coronavirus-child-care-centers.html?action=click\&pgtype=Article\&state=default\&region=MAIN_CONTENT_3\&context=storylines_faq}{makeshift
    child care} and
    \href{https://www.nytimes.com/2020/06/03/business/economy/coronavirus-working-women.html?action=click\&pgtype=Article\&state=default\&region=MAIN_CONTENT_3\&context=storylines_faq}{stunted
    workdays} to continue. California's two largest public school
    districts --- Los Angeles and San Diego --- said on July 13, that
    \href{https://www.nytimes.com/2020/07/13/us/lausd-san-diego-school-reopening.html?action=click\&pgtype=Article\&state=default\&region=MAIN_CONTENT_3\&context=storylines_faq}{instruction
    will be remote-only in the fall}, citing concerns that surging
    coronavirus infections in their areas pose too dire a risk for
    students and teachers. Together, the two districts enroll some
    825,000 students. They are the largest in the country so far to
    abandon plans for even a partial physical return to classrooms when
    they reopen in August. For other districts, the solution won't be an
    all-or-nothing approach.
    \href{https://bioethics.jhu.edu/research-and-outreach/projects/eschool-initiative/school-policy-tracker/}{Many
    systems}, including the nation's largest, New York City, are
    devising
    \href{https://www.nytimes.com/2020/06/26/us/coronavirus-schools-reopen-fall.html?action=click\&pgtype=Article\&state=default\&region=MAIN_CONTENT_3\&context=storylines_faq}{hybrid
    plans} that involve spending some days in classrooms and other days
    online. There's no national policy on this yet, so check with your
    municipal school system regularly to see what is happening in your
    community.
  \end{itemize}
\end{itemize}

Ms. Mendez received a bill separate from the hospital. The doctors who
cared for her individually charged between \$300 and \$1,800 for each
day. Some days, four different doctors billed her for treatment.

Depending on their insurance plan, patients may still be stuck with
paying co-payments, deductibles and a percentage of the bill --- which
can amount to thousands of dollars, although some plans may limit
out-of-pocket costs, said Jack Hoadley, a health policy researcher at
Georgetown University.

And significantly, some of the conditions imposed with the bailout funds
only apply to patients with insurance.

Hospitals can seek reimbursement from the government for treating
uninsured patients through a different process. But it may turn out that
uninsured patients still receive bills.

Ms. Mendez has submitted the bill to Cigna, her new insurer, and said
that she was led to believe her share of it will be under \$10,000.

Like thousands of other gravely ill Covid-19 patients in New York City,
Ms. Mendez, an office administrator for a Domino's Pizza franchise, had
been deeply sedated and placed on a ventilator to keep her breathing
soon after arriving at the hospital on March 25. She was in the hospital
for 19 nights.

When she awoke, she could not remember her own name or where she was,
she said in an interview.

It was a day or two before her memory returned and her confusion
receded.

When she was discharged, an ambulance took her to her mother's home.

At first her mother tried to keep the bills from her. But then Ms.
Mendez said she began to get phone calls from Mount Sinai asking her how
she intended to pay.

She is hopeful that insurance will cover the vast majority of the
charges. But she is also worried that more bills will arrive.

``I haven't seen anything that says `ambulance' on it,'' she said,
wondering if she was going to be charged for the ride to the hospital.
Then she remembered that she left the hospital by ambulance as well.
``Maybe I'll be charged for both of them.''

Advertisement

\protect\hyperlink{after-bottom}{Continue reading the main story}

\hypertarget{site-index}{%
\subsection{Site Index}\label{site-index}}

\hypertarget{site-information-navigation}{%
\subsection{Site Information
Navigation}\label{site-information-navigation}}

\begin{itemize}
\tightlist
\item
  \href{https://help.nytimes.com/hc/en-us/articles/115014792127-Copyright-notice}{©~2020~The
  New York Times Company}
\end{itemize}

\begin{itemize}
\tightlist
\item
  \href{https://www.nytco.com/}{NYTCo}
\item
  \href{https://help.nytimes.com/hc/en-us/articles/115015385887-Contact-Us}{Contact
  Us}
\item
  \href{https://www.nytco.com/careers/}{Work with us}
\item
  \href{https://nytmediakit.com/}{Advertise}
\item
  \href{http://www.tbrandstudio.com/}{T Brand Studio}
\item
  \href{https://www.nytimes.com/privacy/cookie-policy\#how-do-i-manage-trackers}{Your
  Ad Choices}
\item
  \href{https://www.nytimes.com/privacy}{Privacy}
\item
  \href{https://help.nytimes.com/hc/en-us/articles/115014893428-Terms-of-service}{Terms
  of Service}
\item
  \href{https://help.nytimes.com/hc/en-us/articles/115014893968-Terms-of-sale}{Terms
  of Sale}
\item
  \href{https://spiderbites.nytimes.com}{Site Map}
\item
  \href{https://help.nytimes.com/hc/en-us}{Help}
\item
  \href{https://www.nytimes.com/subscription?campaignId=37WXW}{Subscriptions}
\end{itemize}
