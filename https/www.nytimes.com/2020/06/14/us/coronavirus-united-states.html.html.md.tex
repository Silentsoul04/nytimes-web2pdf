Sections

SEARCH

\protect\hyperlink{site-content}{Skip to
content}\protect\hyperlink{site-index}{Skip to site index}

\href{https://www.nytimes.com/section/us}{U.S.}

\href{https://myaccount.nytimes.com/auth/login?response_type=cookie\&client_id=vi}{}

\href{https://www.nytimes.com/section/todayspaper}{Today's Paper}

\href{/section/us}{U.S.}\textbar{}Coronavirus Cases Spike Across Sun
Belt as Economy Lurches into Motion

\url{https://nyti.ms/37qg1rq}

\begin{itemize}
\item
\item
\item
\item
\item
\item
\end{itemize}

\href{https://www.nytimes.com/news-event/coronavirus?action=click\&pgtype=Article\&state=default\&region=TOP_BANNER\&context=storylines_menu}{The
Coronavirus Outbreak}

\begin{itemize}
\tightlist
\item
  live\href{https://www.nytimes.com/2020/08/01/world/coronavirus-covid-19.html?action=click\&pgtype=Article\&state=default\&region=TOP_BANNER\&context=storylines_menu}{Latest
  Updates}
\item
  \href{https://www.nytimes.com/interactive/2020/us/coronavirus-us-cases.html?action=click\&pgtype=Article\&state=default\&region=TOP_BANNER\&context=storylines_menu}{Maps
  and Cases}
\item
  \href{https://www.nytimes.com/interactive/2020/science/coronavirus-vaccine-tracker.html?action=click\&pgtype=Article\&state=default\&region=TOP_BANNER\&context=storylines_menu}{Vaccine
  Tracker}
\item
  \href{https://www.nytimes.com/interactive/2020/07/29/us/schools-reopening-coronavirus.html?action=click\&pgtype=Article\&state=default\&region=TOP_BANNER\&context=storylines_menu}{What
  School May Look Like}
\item
  \href{https://www.nytimes.com/live/2020/07/31/business/stock-market-today-coronavirus?action=click\&pgtype=Article\&state=default\&region=TOP_BANNER\&context=storylines_menu}{Economy}
\end{itemize}

Advertisement

\protect\hyperlink{after-top}{Continue reading the main story}

Supported by

\protect\hyperlink{after-sponsor}{Continue reading the main story}

\hypertarget{coronavirus-cases-spike-across-sun-belt-as-economy-lurches-into-motion}{%
\section{Coronavirus Cases Spike Across Sun Belt as Economy Lurches into
Motion}\label{coronavirus-cases-spike-across-sun-belt-as-economy-lurches-into-motion}}

Arizona, Texas and Florida are reporting their highest case numbers yet.
As of Saturday, coronavirus cases were climbing in 22 states amid
reopenings.

\includegraphics{https://static01.nyt.com/images/2020/06/12/us/00STATEOFTHEVIRUS4-dallasalt/merlin_173454768_10fe5ae0-207a-4485-a11b-35f4c006b34d-articleLarge.jpg?quality=75\&auto=webp\&disable=upscale}

\href{https://www.nytimes.com/by/julie-bosman}{\includegraphics{https://static01.nyt.com/images/2018/11/09/multimedia/author-julie-bosman/author-julie-bosman-thumbLarge.png}}\href{https://www.nytimes.com/by/mitch-smith}{\includegraphics{https://static01.nyt.com/images/2018/09/10/multimedia/author-mitch-smith/author-mitch-smith-thumbLarge.png}}

By \href{https://www.nytimes.com/by/julie-bosman}{Julie Bosman} and
\href{https://www.nytimes.com/by/mitch-smith}{Mitch Smith}

\begin{itemize}
\item
  Published June 14, 2020Updated June 18, 2020
\item
  \begin{itemize}
  \item
  \item
  \item
  \item
  \item
  \item
  \end{itemize}
\end{itemize}

CHICAGO --- The warning has echoed ominously for weeks from
epidemiologists, small-town mayors and county health officials: Once
states begin to reopen, a surge in
\href{https://www.nytimes.com/news-event/coronavirus}{coronavirus cases}
will follow.

That scenario is now playing out in states
\href{https://www.nytimes.com/interactive/2020/us/coronavirus-us-cases.html}{across
the country}, particularly in the Sun Belt and the West, as thousands of
Americans have been sickened by the virus in new and alarming outbreaks.

Hospitals in Arizona have been urged to activate emergency plans to cope
with a flood of
\href{https://www.nytimes.com/2020/06/15/health/coronavirus-underlying-conditions.html}{coronavirus}
patients. On Saturday, Florida saw its largest single-day count of cases
since the pandemic began. Oregon has failed to contain the spread of the
virus in many places, leading the governor on Thursday to pause what had
been a gradual reopening.

And in Texas, cases are rising swiftly around the largest cities,
including Houston, San Antonio and Dallas.

``I'm very concerned about it,'' said Mayor Eric Johnson of Dallas,
noting that after months of warnings and isolation, many residents had
stopped wearing masks and maintaining social distance out of sheer
fatigue. ``They've been asked for quite some time to not be around
people they love, and that they want to spend time with. Wearing a mask
is not pleasant. And I think people are tired.''

For close to a month,
\href{https://www.nytimes.com/interactive/2020/us/states-reopen-map-coronavirus.html}{much
of the United States} has looked like a nation open or beginning to
open, and increasingly unfettered by restrictions meant to slow the
spread of the coronavirus. With many government limits removed and
people left to make individual choices about precautions, Americans have
gone back to salons and restaurants, crowded into public parks and, in
dozens of cities, joined
\href{https://www.nytimes.com/news-event/george-floyd-protests-minneapolis-new-york-los-angeles}{large
public demonstrations} protesting police misconduct.

Over all, daily coronavirus cases across the United States are
essentially steady, stuck on a plateau. More than two million people
have now contracted the virus in this country,
\href{https://www.nytimes.com/interactive/2020/us/coronavirus-us-cases.html}{according
to a New York Times database}, and every day, about 21,100 new known
cases are reported, not much lower than the numbers from a month ago.
About 800 people die from it each day. Those figures have both dropped
significantly since peaking in April.

But as of Saturday, the daily number of new coronavirus cases was
climbing in 22 states, shifting course from what had been downward
trajectories in many of those places.

The spikes in cases bring leaders in these states to a new crossroads:
Accept the continued rise in infections as an expected cost of reopening
economies or consider slowing the lifting of restrictions aimed at
stopping the spread or even imposing a new set of limits.

In Houston on Thursday,
\href{https://www.houstonchronicle.com/news/houston-texas/houston/article/Hidalgo-to-announce-COVID-19-threat-level-15332940.php}{the
county's top elected official warned} that the community was ``on the
precipice of a disaster'' and urged residents to minimize contact with
others. More than 300 new cases have been identified in that county on
each recent weekday.

But at the city's Galleria mall, there were few signs of concern: People
stood in a tightly spaced line for pretzels at an Auntie Anne's kiosk.
At California Nails, two women sat maskless during pedicures. Signs
urged social distancing, but in crowded walkways outside stores,
shoppers brushed past one another, only inches apart.

\hypertarget{latest-updates-global-coronavirus-outbreak}{%
\section{\texorpdfstring{\href{https://www.nytimes.com/2020/08/01/world/coronavirus-covid-19.html?action=click\&pgtype=Article\&state=default\&region=MAIN_CONTENT_1\&context=storylines_live_updates}{Latest
Updates: Global Coronavirus
Outbreak}}{Latest Updates: Global Coronavirus Outbreak}}\label{latest-updates-global-coronavirus-outbreak}}

Updated 2020-08-01T21:19:55.782Z

\begin{itemize}
\tightlist
\item
  \href{https://www.nytimes.com/2020/08/01/world/coronavirus-covid-19.html?action=click\&pgtype=Article\&state=default\&region=MAIN_CONTENT_1\&context=storylines_live_updates\#link-3ac56579}{Top
  officials work to break impasse over jobless benefit.}
\item
  \href{https://www.nytimes.com/2020/08/01/world/coronavirus-covid-19.html?action=click\&pgtype=Article\&state=default\&region=MAIN_CONTENT_1\&context=storylines_live_updates\#link-8796723}{The
  virus picks up dangerous speed in the Midwest, and in areas that had
  seen success.}
\item
  \href{https://www.nytimes.com/2020/08/01/world/coronavirus-covid-19.html?action=click\&pgtype=Article\&state=default\&region=MAIN_CONTENT_1\&context=storylines_live_updates\#link-25930521}{Thousands
  in Berlin protest Germany's coronavirus measures.}
\end{itemize}

\href{https://www.nytimes.com/2020/08/01/world/coronavirus-covid-19.html?action=click\&pgtype=Article\&state=default\&region=MAIN_CONTENT_1\&context=storylines_live_updates}{See
more updates}

More live coverage:
\href{https://www.nytimes.com/live/2020/07/31/business/stock-market-today-coronavirus?action=click\&pgtype=Article\&state=default\&region=MAIN_CONTENT_1\&context=storylines_live_updates}{Markets}

Throughout most of Florida, the reopening of public life has allowed
bars and movie theaters to operate at half capacity and gyms at full
capacity. On June 5, the state loosened restrictions further, even as
the caseload was beginning to go up.

In Salt Lake City, some people are now behaving as they did before the
coronavirus pandemic, even amid a rise in cases, said Teresa Kehl, a
Utah resident who runs summer soccer camps with her husband.

``We went to a restaurant the other night and none of the employees had
masks on,'' Ms. Kehl said. ``It was kind of shocking.''

Dr. Angela Dunn, the Utah state epidemiologist, has traced the state's
resurgence in the coronavirus to the state's reopenings, which began
before Memorial Day.

``The timing directly correlates with our loosening up restrictions,''
Dr. Dunn said. ``That definitely has something to do with it.''

There is ample reason to tie the latest surge of infections to
relatively early reopenings. Clusters of infections in food processing
facilities, jails and nursing homes continue to create hot spots in some
places, but they do not explain the broader pattern.

Most of the 10 hardest-hit states that have seen rising case levels
started reopening on or before May 8. Louisiana, where cases have
started rising again after extended declines, began reopening on May 15.
Another hard-hit state with growing cases, California, has reopened in a
more incremental way, with continuing restrictions in some regions.

Many states that were slowest to reopen have seen a different
trajectory, at least so far. Most of the 10 states hit hard over all in
the pandemic --- but currently seeing decreasing daily cases of the
virus --- began reopening in mid-May or later. Only two of those states,
Pennsylvania and Indiana, began reopening before May 15. Some of the
others, New Jersey and Michigan, only began reopening in earnest in
June. And two others, New York and Illinois, kept restrictions in their
worst-hit areas while reopening less populous regions.

\includegraphics{https://static01.nyt.com/images/2017/01/29/podcasts/the-daily-album-art/the-daily-album-art-articleInline-v2.jpg?quality=75\&auto=webp\&disable=upscale}

\hypertarget{listen-to-the-daily-what-weve-learned-about-the-coronavirus}{%
\subsubsection{Listen to `The Daily': What We've Learned About the
Coronavirus}\label{listen-to-the-daily-what-weve-learned-about-the-coronavirus}}

Six months into the pandemic, we take stock of where we are, and where
we might be going.

transcript

Back to The Daily

bars

0:00/27:01

-27:01

transcript

\hypertarget{listen-to-the-daily-what-weve-learned-about-the-coronavirus-1}{%
\subsection{Listen to `The Daily': What We've Learned About the
Coronavirus}\label{listen-to-the-daily-what-weve-learned-about-the-coronavirus-1}}

\hypertarget{hosted-by-michael-barbaro-produced-by-michael-simon-johnson-and-annie-brown-with-help-from-alexandra-leigh-young-and-edited-by-theo-balcomb-and-lisa-chow}{%
\subsubsection{Hosted by Michael Barbaro; produced by Michael Simon
Johnson and Annie Brown; with help from Alexandra Leigh Young; and
edited by Theo Balcomb and Lisa
Chow}\label{hosted-by-michael-barbaro-produced-by-michael-simon-johnson-and-annie-brown-with-help-from-alexandra-leigh-young-and-edited-by-theo-balcomb-and-lisa-chow}}

\hypertarget{six-months-into-the-pandemic-we-take-stock-of-where-we-are-and-where-we-might-be-going}{%
\paragraph{Six months into the pandemic, we take stock of where we are,
and where we might be
going.}\label{six-months-into-the-pandemic-we-take-stock-of-where-we-are-and-where-we-might-be-going}}

\begin{itemize}
\item
  michael barbaro\\
  Donald, the pandemic feels different in the U.S. than it did two weeks
  ago, three weeks ago, a month ago. It feels --- and these are highly
  qualitative words, and I know you're probably rolling your eyes --- it
  feels less desperate, and it feels a little less urgent. And I'm not
  even quite sure why that is. So what do you make of that? I mean, does
  that mean we've gotten complacent?
\item
  donald g. mcneil jr.\\
  Yeah. I think some parts of the country are not afraid yet.

  They see this as something that happened to the other. To urban New
  Yorkers, to blacks and Hispanics in the big cities far away.

  But I also see that, even in New York, it's a combination of, ``Well,
  we survived the first wave and it didn't get me.'' Or, ``The people I
  knew who got it, survived.'' And, ``Boy, am I bored, and frustrated in
  lockdown.'' And, ``I want my job back.'' And, ``I want my hanging out
  with my friends back.''
\item
  michael barbaro\\
  Mm-hm.
\item
  donald g. mcneil jr.\\
  And I think we have gotten used to the idea of death --- of absorbing
  a lot of death.
\item
  michael barbaro\\
  From The New York Times, I'm Michael Barbaro. This is ``The Daily.''
\item
  {[}music{]}
\item
  michael barbaro\\
  Today, my colleague, Donald G. McNeil Jr., returns with an update on
  the state of the coronavirus, and what we've learned about the virus
  six months into the pandemic.
\item
  {[}music{]}\\
  It's Monday, June 15.

  Donald, the last time that we spoke to you was in mid-April, when the
  death toll from the coronavirus in the U.S. was nearing 40,000. And
  various states, at that time, were beginning the process of opening
  back up. Slowly, but opening back up. Where is the pandemic now?
\item
  donald g. mcneil jr.\\
  OK. Well, there are 113,000 dead in the U.S., the last time I looked.
  Basically, every state has reopened to some extent in different ways,
  with different phases. In about 21 of those states, cases are going
  up, as was feared. Hospitalizations are going up. So I would see this
  as a very worrying situation.
\item
  michael barbaro\\
  Well, given that --- I mean, in terms of how the United States is
  managing the virus and its spread, are we doing better now than we
  were many, many weeks ago, when we spoke? Or are we doing worse?
\item
  donald g. mcneil jr.\\
  I think we're doing considerably worse in that now we know what the
  danger is, and yet we're still getting an enormous number of
  infections. Now, the places that were hit hard in the first wave ---
  which is New York in the Northeast, but also places far away from
  that, like New Orleans, Seattle and California --- they are mostly
  coming down. They got seriously scared. They had intense lockdowns
  that lasted relatively long. They are wearing masks, they are
  practicing social distancing. And cases are coming down in those
  states. In states where there wasn't a big first wave of the virus,
  where they were more upset by the economic effects of the lockdown ---
  lots of people out of work, lots of people suffering, lots of people
  frustrated in their houses --- those are were the cases, in general,
  are going up. Places like Texas, places like North and South Carolina,
  places like Arizona. And that's dangerous because they're coming out
  of lockdown, and opening society, and encouraging people to be in
  greater contact with each other at a time when the cases are already
  on the upswing. So you're not flattening the curve. You're allowing
  the curve to spike up again.
\item
  michael barbaro\\
  So is it fair to say, at this point in the pandemic, the cities and
  the states that were hardest hit at the beginning seem to have
  enforced the strongest lockdowns, and therefore the virus infection
  rates seem to be going down there. Whereas states that weren't hit as
  hard at the beginning, and perhaps didn't respond as forcefully, are
  now experiencing an uptick in infections instead?
\item
  donald g. mcneil jr.\\
  Yes. In general, that's correct.
\item
  michael barbaro\\
  So, Donald, now that we have lived with this virus for about half a
  year, I want to turn to the things that we have learned about it, and
  that we didn't know when we first started talking to you. And I wonder
  if we can start with transmission.
\item
  donald g. mcneil jr.\\
  OK.
\item
  michael barbaro\\
  I remember you telling me, back in February, the main modes of
  transmission are droplets. You cough or sneeze near me, I catch the
  virus from you. The other was the virus living on surfaces. And I
  would touch that surface, I might touch my eye, I would get infected.
  What have we learned about the actual levels of transmission from
  these two? Did one of them turn out to be a much bigger vector than
  the other?
\item
  donald g. mcneil jr.\\
  Those two are still vectors, although surfaces may be a little less
  important than we feared. But the really important thing we've learned
  is that there is aerosol transmission. Little tiny, tiny droplets, the
  kind that hang in the air inside a room for hours, can hold enough
  virus to transmit this disease. And we emit those droplets not just
  through coughing or sneezing, but just through talking, especially
  loud talking, through laughing, through singing. You know, we're
  unaware of this fine mist of droplets that comes out of our mouth at
  all times. You're only sort of aware of it if you're in the front row
  at a theater and you realize the actresses are spitting on you. But,
  actually, if you and I were to sit three or four feet apart, and we're
  talking, and joking, and laughing, we would actually be spreading a
  cloud of a very fine mist of droplets at each other. So we're learning
  that transmission, particularly in indoor spaces where there is no
  wind, is probably a major spreader of this because everybody's
  vulnerable to it.
\item
  michael barbaro\\
  OK. So staying on this idea of what we've learned --- asymptomatic
  carriers. What have we learned about people who may not even know they
  have the virus? They don't show any symptoms, but they may be
  spreading it. How much has our understanding changed about how
  important asymptomatic carriers are in this pandemic?
\item
  donald g. mcneil jr.\\
  It has changed a lot. The initial estimates out of China were that
  there were very few asymptomatic carriers --- like, 1 percent. That
  has turned out to probably be quite wrong. The C.D.C estimates that
  the number of asymptomatic carriers is about one third.
\item
  michael barbaro\\
  Wow.
\item
  donald g. mcneil jr.\\
  Right. And that changes a lot about how we handle this disease.
\item
  michael barbaro\\
  What do you mean?
\item
  donald g. mcneil jr.\\
  Well, OK --- so temperature checks probably aren't very useful because
  ---
\item
  michael barbaro\\
  You're asymptomatic.
\item
  donald g. mcneil jr.\\
  --- one of the symptoms you feel is fever. If you don't feel fever,
  you can still spread the disease. So you're going to have to detect
  the disease through testing rather than through saying, hey, we've got
  a lot of sick people around here. Because one third of your people are
  not going to be sick, but they're still spreading the disease. So if
  you wanted to open up your office to people --- like I said, fever
  checks wouldn't work. You'd actually have to test everybody, you'd
  have to test them frequently, and you'd have to be able to get the
  results back very quickly. Because you don't want somebody who's
  asymptomatic working in the office for two or three days while you're
  waiting for the results of the test to come back.
\item
  michael barbaro\\
  And how possible is that kind of testing? I mean, what you're
  proposing is every workplace having some sort of mandatory testing
  system to weed out people who may have the virus, and especially those
  who may be asymptomatically infected.
\item
  donald g. mcneil jr.\\
  Yeah. I mean, there's a lot of thinking about testing. In the
  beginning of it, when we only had, you know, 10,000, 20,000 tests, we
  were only testing people who were sick, who had symptoms. It was very
  hard to get a test. Now, we've reached a point where we're on track to
  get to a million a day, I think. When you've reach that level, you can
  use testing for surveillance. That is, you sort of look around the
  country and say, where do we have hot spots? Now, probably, to do that
  right, we need, like, 5 million tests a day, that's what a Harvard
  study suggested. And it means testing in New York City, but also
  testing in Winnemucca, Nev., and every place else. Because you want to
  know wherever the virus is popping up, you want to spot it, and you
  want to test not just the sick but a broad spectrum of people. You
  know, maybe one day, you test all the third graders in the county, or
  something like that. Another day, you test everybody in nursing homes,
  and things like that. Now, that's at the 5 million test level. If you
  go up ---there was a Nobel Prize-winning economist at N.Y.U. who
  proposed that, if we had 30 million tests a day, we could literally
  use this as a way to completely reopen the economy. And that would
  mean everybody who's in contact with other people in an office would
  have to be tested every day, and we'd need rapid results. And it would
  cost, he figured, about 1.5 billion dollars per week. But he said, you
  know what, that's a whole lot less than lockdown has been costing us.
  We could completely reopen the economy if we could test 30 million
  people a day. And we'd save money by having the old economy going
  again. Now, the logistics of doing that is wildly unimaginable. The
  proposal from the N.Y.U. economist has been put in the crazy ideas box
  for now. And yet, people who have really studied this stuff think of
  it as, wow --- crazy, but good. So what are you going to do? You know,
  it depends on what people are willing to commit money to do.
\item
  michael barbaro\\
  OK. And for those who have had the virus, and have recovered --- of
  which there are hundreds of thousands of people --- what are we
  learning about immunity?
\item
  donald g. mcneil jr.\\
  We know that people have antibodies. We know they have IgG antibodies,
  which are the ones that appear later, and usually --- usually ---
  indicate immunity. So we know that people turn up positive on antibody
  tests. And top experts are saying, we think we can assume they're
  immune to the disease, but we're not sure yet. Because we don't really
  know what level of antibodies you need in order to be immune. So
  nobody is saying, OK, that's it, you've passed one test, you're
  immune. That's what people would like to think. All sorts of people
  ask me, should I get an antibody test? And I keep saying, sure, if
  it's going to make you feel better, get an antibody test. But don't
  assume you're immune just because you've got a positive. You probably
  are, but we don't know that yet. It's still too early.
\item
  michael barbaro\\
  So recalling our very last conversation, in which we talked about the
  idea that there might be two classes of people in this pandemic ---
  the immune and the susceptible. We are not yet at a place of our
  understanding of the coronavirus where anyone is truly considered
  immune, and therefore sort of invulnerable and able to wander the
  world differently than the rest of us.
\item
  donald g. mcneil jr.\\
  Scientifically speaking, no, we're not there yet. But practically
  speaking, a lot of people --- even doctors I know --- who've been
  infected and have recovered are behaving as if they're immune. They're
  reasonably confident that they're immune. I mean, they shouldn't let
  down their guard, A, because it's not smart and, B, because it sets a
  bad example. But they probably worry a whole lot less than they did
  before. Now, we do not know how long immunity lasts --- and that's
  going to be another great, big question. And we can't know that until
  some months, or years, have passed because this virus has only been
  around since November. So, probably, the immunity is going to last a
  few years. But we don't know that yet either. These are all unknowns.
  Immunology is complicated.
\item
  michael barbaro\\
  And given everything you just said --- under the current
  circumstances, where will the United States be by the fall when it
  comes to death?
\item
  donald g. mcneil jr.\\
  I'm very worried about the fall for several reasons --- which I can go
  into if you want to.
\item
  michael barbaro\\
  Please.
\item
  donald g. mcneil jr.\\
  OK. We know the virus transmits indoors. And when it gets cold, people
  aren't going to be able to eat outdoors again. They're going to want
  to get into warm spaces. So the possibility for transmission is really
  high. We also know, not just from 1918, but from --- Michael Osterholm
  at Minnesota has looked at eight influenza epidemic since, I think,
  it's 1763. And in each one of them, no matter what time of year the
  virus first hit, winter, spring, summer, or fall, it faded, and then
  came back several months later in a much more lethal wave. And that
  was the phenomenon in 1918. There was a brief, but scary, breakout in
  Haskell County, Kansas, in army camps and stuff. And then, the disease
  mostly disappeared in the United States --- seems to have mostly gone
  overseas and hit the troops in the trenches in Europe. And then, in
  the fall and winter, it came roaring back. And a third of the deaths
  took place in a little tiny period between September and December,
  1918. So I'm very worried that something like that could happen this
  fall and winter, and that we're not mentally prepared for it.
\item
  michael barbaro\\
  So the infection and death rate for the next few months quite likely
  will not be representative of what this virus is capable of. And the
  fall and the winter may be very, very different and much scarier.
\item
  donald g. mcneil jr.\\
  That's possible. And that's what a lot of top public health people are
  worried about --- that we will have transmission. I mean, we're still
  seeing 20,000 new infections per day. And we're at about a little
  under 1,000 new deaths per day. And that's been consistent for a
  couple of months now.
\item
  michael barbaro\\
  And this might be a strange question --- but is that a good number, or
  a bad number?
\item
  donald g. mcneil jr.\\
  That's a terrible number. I mean, 1,000 deaths a day from this? 20,000
  new infections a day? I mean, that's not an epidemic you have under
  control. You know, we don't talk about it that way, but that's a
  rapidly spreading epidemic. Now, we may become complacent about that,
  we may sort of accept that as the new norm. And that may lull us into
  a sense of complacency when fall arrives. And that's a worry. And
  that's why I'm so eager for treatment or vaccine to hurry up, hurry
  up, hurry up.
\item
  michael barbaro\\
  And where are we in that vaccine process now?
\item
  donald g. mcneil jr.\\
  Well, there's 150 or so vaccine candidates being looked at around the
  world. In the United States, we have designated different candidates
  for warp speed. Meaning, testing the vaccines, but simultaneously
  paying companies to build factories to make them so that all the ones
  that turn out to be both safe and effective will have, hopefully,
  millions of doses ready to roll immediately. Because one of the big
  roadblocks to getting vaccine is not just testing the vaccine and
  making sure it works, but then suddenly producing 300 to 600 million
  of doses for this country, depending on if you need one or two doses,
  and seven-plus billion doses for the world. So you want to get a head
  start on the production as much as you can. So we're doing that with a
  number of candidates. And we've never seen anything like this before.
\item
  michael barbaro\\
  Does all that encourage you to think that we might have a vaccine much
  faster than we've ever had a vaccine before? I remember you telling me
  that the fastest we've ever really had a vaccine in production and
  available to people, from start to finish, is close to four years.
\item
  donald g. mcneil jr.\\
  Yes that was the mumps vaccine, and the record is four years. But I
  think we're doing things very, very differently this time. We've got
  multiple candidates. And some of those vaccines --- if what we're
  being told is correct --- are actually going into production even as
  we speak.
\item
  michael barbaro\\
  Wow.
\item
  donald g. mcneil jr.\\
  So that's very encouraging to me.

  But things go wrong when you test vaccines. You get surprises that you
  didn't expect. And so let's hope multiple things don't go wrong.
\item
  {[}music{]}
\item
  michael barbaro\\
  We'll be right back.

  So, Donald, we've talked about the risk of many different activities
  in this moment of the pandemic. We have not talked about something
  that has been going on for several weeks now in the United States,
  which are large-scale protests and demonstrations since the police
  killing of George Floyd. And I wonder what your sense is about the
  risk involved in those protests of spreading the virus?
\item
  donald g. mcneil jr.\\
  Well, I don't worry when there are crowds, outdoors, spaced 6 feet
  apart and wearing masks. I do worry when people are jammed up against
  each other, either confronting a police line, or on a dais while
  they're making speeches, or something like that. I worry about anybody
  who's pushed into the back of a police van. I worry about people
  who've been in cells together. I worry about the funeral ceremonies,
  which all took place indoors. You know, these are all potential
  super-spreader situations.
\item
  michael barbaro\\
  And have we yet seen any uptick in transmission from the U.S.? I know
  it may take several weeks for us to determine that, but have we?
\item
  donald g. mcneil jr.\\
  I mean, if we have spikes here, it's going to be very hard to say, oh,
  that spike came from the protest. Because how do you do contact
  tracing of everybody else who was in a crowd of 10,000 people with
  you? You know, it's easy to do contact tracing on your family and your
  co-workers at the office. It's very hard to do contact tracing on a
  whole crowd of strangers. So we won't necessarily know, when people
  fall ill, that they got infected at the protest march. I mean, each
  individual person may say, well, the only time I've been in
  association with a lot of other people was when I was at a protest
  march. But for some epidemiologist to put all those stories together
  as those people turn up in hospitals, or as their grandmothers turn up
  in hospitals, is going to be very difficult. So we may not see the
  signal we would expect to see, if that makes sense.
\item
  michael barbaro\\
  I'm curious, you know, reflecting on everything that you have just
  told us, I'm curious how you are operating now in the world knowing
  everything you now know, six months into this pandemic? What your
  routines are, what your precautions are. Are you taking the subway?
  Are you always wearing a mask? Are you going to any kind of office?
\item
  donald g. mcneil jr.\\
  I'm working from home. I'm working a lot. I'm lucky in that I'm not
  out of work. But my girlfriend, and a lot of other people I know who
  are out of work, are pent-up, and frustrated, and angry. And some are
  really worried about their income. I always wear a mask when I am
  indoors with other people, as in the grocery store or pharmacy. I
  avoid going indoors with other people, basically, at all costs. If I
  were riding the subway, I would definitely wear a mask. I'm not riding
  the subway. I'm very worried about what's going to happen in New York
  City when a lot of people have to go back to work. Because right now,
  we've got near gridlock on the F.D.R. Drive and stuff sometimes,
  because a lot more people are in cars because they're afraid to go on
  the subway. But at some point, we're going to have to go back on the
  subway. And, frankly, I think the M.T.A. ought to take the windows off
  the buses and subways. I know it's crazy --- and I know it's going to
  be cold in winter, and hot and un-air conditioned in summer. But
  that's the way to get breeze blowing through enclosed spaces. And if
  we want to go back into our offices, we're going to have to find a way
  to have breeze blowing through enclosed spaces so that the virus does
  not hang in the air. We're going to have to rethink our workplaces, or
  stay out of them, because they're going to be too dangerous until we
  have a vaccine.
\item
  michael barbaro\\
  Do you intend to return to an office anytime soon?
\item
  donald g. mcneil jr.\\
  No. I'm 66 years old. You know, I'm reasonably healthy, but I'm also
  in a higher-risk group. So I think me being in an office with a lot of
  members of ``The Daily'' team breathing on the microphones, or
  whatever, would be dangerous for me right now. So no. And I'm sad
  that, you know, I have not seen my granddaughter, except on video. She
  was born on the 4th.
\item
  michael barbaro\\
  Mazel --- mazel tov!
\item
  donald g. mcneil jr.\\
  Thank you. Thank you. And I intend not to hold her until I'm
  vaccinated or immune. My daughter --- the apple did not fall far from
  the tree, and she's just as determined as I am that we should play by
  the rules on that, I think. So it's sad. But I'm taking the long view.
  This increases the chances that both my granddaughter and I will make
  it to her high school graduation. So that's the plan. And if I have to
  sacrifice a little bit of seeing her right now, you know, OK, I'll
  make that sacrifice for both of us.
\item
  michael barbaro\\
  Well, we wish both of you the best. And we're really grateful for your
  time. And thank you very much.
\item
  donald g. mcneil jr.\\
  Thank you for letting me on again.
\item
  {[}music{]}
\item
  michael barbaro\\
  The Times reports that, as infections rise in 22 American states,
  officials there are facing a choice: Accept the increase as the cost
  of reopening their economies, or slow the reopening process and even
  impose new restrictions, however unpopular they may be. In an
  interview published on Sunday, Dr. Anthony Fauci, a White House
  advisor on the pandemic, said that waves of infections would likely
  spike and fall for months, and that he did not expect the U.S. to
  return to normal for another year.
\item
  {[}music{]}\\
  We'll be right back.
\item
  michael barbaro\\
  Here's what else you need to know today. The chief of police in
  Atlanta resigned over the weekend, after an officer she oversees
  killed a 27-year-old black man, Rayshard Brooks. Before he was shot to
  death, Brooks had failed a sobriety test, run from the police and
  grabbed a taser from an arresting officer --- a sequence of events
  that Atlanta's mayor, Keisha Lance Bottoms, said did not warrant his
  death.
\item
  keisha lance bottoms\\
  While there may be debate as to whether this was an appropriate use of
  deadly force, I firmly believe that there is a clear distinction
  between what you can do and what you should. I do not believe that
  this was a justified use of deadly force.
\end{itemize}

michael barbaro

Bottoms immediately ordered that the officer who had killed Brooks be
fired. And the Times reports that protests over the death of George
Floyd have been held in more than 2,000 U.S. cities and towns across all
50 states over the past three weeks. The protests, The Times found,
defied traditional demographic fault lines, occurring not just in
Democratic strongholds, but in rural, conservative and majority white
communities.

\begin{itemize}
\item
  archived recording (protest) 1\\
  Black lives!
\item
  archived recording (protest) 2\\
  Matter!
\item
  archived recording (protest) 1\\
  Black lives!
\item
  archived recording (protest) 2\\
  Matter!
\item
  archived recording (protest) 1\\
  Black lives!
\end{itemize}

michael barbaro

The protests continued over the weekend, from Brooklyn to the small town
of Haughton, La.

\begin{itemize}
\item
  archived recording (speaker)\\
  When I say black lives, y'all say matter. Black lives!
\item
  archived recording (crowd)\\
  Matter!
\item
  archived recording (speaker)\\
  Black Lives!
\item
  archived recording (crowd)\\
  Matter!
\end{itemize}

michael barbaro

That's it for ``The Daily.'' I'm Michael Barbaro. See you tomorrow.

\includegraphics{https://static01.nyt.com/images/2020/06/15/autossell/Covid-Funeral-Home-onsiteC/Covid-Funeral-Home-onsiteC-videoSixteenByNineJumbo1600.jpg}

As testing capacity has increased, so has the number of cases being
counted, and officials in places like Arizona and Florida say the
increase in cases may be explained, at least partly, by the growing
availability of tests.

Dr. Anthony S. Fauci, the country's top infectious disease expert, said
Friday in an
\href{https://abcnews.go.com/Politics/fauci-tells-abcs-powerhouse-politics-attending-rallies-protests/story?id=71219338}{interview
with ABC News} that it was important to look both at case numbers and
the percentage of positive tests to understand whether upticks in cases
reflected broader transmission in American cities.

``If you test more, you will likely pick up more infections,'' Dr. Fauci
said. He added, ``Once you see that the percentage is higher, then
you've really got to be careful, because then you really are seeing
additional infections that you weren't seeing before.''

But epidemiologists said that even taking into account a rise in
testing, the increase in confirmed cases in Sun Belt states suggested
increased transmissions. Other measures, such as the percentage of
positive tests and hospitalizations, reflect that worsening outlook. In
Florida more than 4.5 percent of those who tested between May 31 and
June 6 had the virus, compared with about 2.3 percent of people who
sought tests in mid-May. Earlier in the pandemic, the percent of people
testing positive in Florida was higher, but that was during a period
when testing was far more limited. Similar rates in Arizona and Texas
have also risen in recent weeks.

In Arizona, more than 1,400 people who were believed to have the virus
were hospitalized on Friday, up from 755 a month earlier and higher than
at any other point in the pandemic. In Texas, the 2,166 coronavirus
patients hospitalized on Friday were the most yet in that state.

For states with growing coronavirus outbreaks, some officials have
arrived at the same conclusion: The rise in infections is unfortunate
but inevitable.

``We are not going to be able to stop the spread,'' said Dr. Cara
Christ, the Arizona state health director. ``And so we can't stop living
as well.''

\includegraphics{https://static01.nyt.com/images/2020/06/12/us/00STATEOFTHEVIRUS-azmall/merlin_173351403_1c8240dd-4b85-485f-b595-265a6f20ad67-articleLarge.jpg?quality=75\&auto=webp\&disable=upscale}

But the outbreaks have also prompted frantic and repeated pleas to the
public, asking that people wear masks and practice social distancing to
limit transmission of the virus. On Thursday, Pat Gerard, the chairwoman
of the Board of County Commissioners in Pinellas County, Fla., raised
the specter of another clampdown on businesses to contain the latest
outbreaks.

\href{https://www.nytimes.com/news-event/coronavirus?action=click\&pgtype=Article\&state=default\&region=MAIN_CONTENT_3\&context=storylines_faq}{}

\hypertarget{the-coronavirus-outbreak-}{%
\subsubsection{The Coronavirus Outbreak
›}\label{the-coronavirus-outbreak-}}

\hypertarget{frequently-asked-questions}{%
\paragraph{Frequently Asked
Questions}\label{frequently-asked-questions}}

Updated July 27, 2020

\begin{itemize}
\item ~
  \hypertarget{should-i-refinance-my-mortgage}{%
  \paragraph{Should I refinance my
  mortgage?}\label{should-i-refinance-my-mortgage}}

  \begin{itemize}
  \tightlist
  \item
    \href{https://www.nytimes.com/article/coronavirus-money-unemployment.html?action=click\&pgtype=Article\&state=default\&region=MAIN_CONTENT_3\&context=storylines_faq}{It
    could be a good idea,} because mortgage rates have
    \href{https://www.nytimes.com/2020/07/16/business/mortgage-rates-below-3-percent.html?action=click\&pgtype=Article\&state=default\&region=MAIN_CONTENT_3\&context=storylines_faq}{never
    been lower.} Refinancing requests have pushed mortgage applications
    to some of the highest levels since 2008, so be prepared to get in
    line. But defaults are also up, so if you're thinking about buying a
    home, be aware that some lenders have tightened their standards.
  \end{itemize}
\item ~
  \hypertarget{what-is-school-going-to-look-like-in-september}{%
  \paragraph{What is school going to look like in
  September?}\label{what-is-school-going-to-look-like-in-september}}

  \begin{itemize}
  \tightlist
  \item
    It is unlikely that many schools will return to a normal schedule
    this fall, requiring the grind of
    \href{https://www.nytimes.com/2020/06/05/us/coronavirus-education-lost-learning.html?action=click\&pgtype=Article\&state=default\&region=MAIN_CONTENT_3\&context=storylines_faq}{online
    learning},
    \href{https://www.nytimes.com/2020/05/29/us/coronavirus-child-care-centers.html?action=click\&pgtype=Article\&state=default\&region=MAIN_CONTENT_3\&context=storylines_faq}{makeshift
    child care} and
    \href{https://www.nytimes.com/2020/06/03/business/economy/coronavirus-working-women.html?action=click\&pgtype=Article\&state=default\&region=MAIN_CONTENT_3\&context=storylines_faq}{stunted
    workdays} to continue. California's two largest public school
    districts --- Los Angeles and San Diego --- said on July 13, that
    \href{https://www.nytimes.com/2020/07/13/us/lausd-san-diego-school-reopening.html?action=click\&pgtype=Article\&state=default\&region=MAIN_CONTENT_3\&context=storylines_faq}{instruction
    will be remote-only in the fall}, citing concerns that surging
    coronavirus infections in their areas pose too dire a risk for
    students and teachers. Together, the two districts enroll some
    825,000 students. They are the largest in the country so far to
    abandon plans for even a partial physical return to classrooms when
    they reopen in August. For other districts, the solution won't be an
    all-or-nothing approach.
    \href{https://bioethics.jhu.edu/research-and-outreach/projects/eschool-initiative/school-policy-tracker/}{Many
    systems}, including the nation's largest, New York City, are
    devising
    \href{https://www.nytimes.com/2020/06/26/us/coronavirus-schools-reopen-fall.html?action=click\&pgtype=Article\&state=default\&region=MAIN_CONTENT_3\&context=storylines_faq}{hybrid
    plans} that involve spending some days in classrooms and other days
    online. There's no national policy on this yet, so check with your
    municipal school system regularly to see what is happening in your
    community.
  \end{itemize}
\item ~
  \hypertarget{is-the-coronavirus-airborne}{%
  \paragraph{Is the coronavirus
  airborne?}\label{is-the-coronavirus-airborne}}

  \begin{itemize}
  \tightlist
  \item
    The coronavirus
    \href{https://www.nytimes.com/2020/07/04/health/239-experts-with-one-big-claim-the-coronavirus-is-airborne.html?action=click\&pgtype=Article\&state=default\&region=MAIN_CONTENT_3\&context=storylines_faq}{can
    stay aloft for hours in tiny droplets in stagnant air}, infecting
    people as they inhale, mounting scientific evidence suggests. This
    risk is highest in crowded indoor spaces with poor ventilation, and
    may help explain super-spreading events reported in meatpacking
    plants, churches and restaurants.
    \href{https://www.nytimes.com/2020/07/06/health/coronavirus-airborne-aerosols.html?action=click\&pgtype=Article\&state=default\&region=MAIN_CONTENT_3\&context=storylines_faq}{It's
    unclear how often the virus is spread} via these tiny droplets, or
    aerosols, compared with larger droplets that are expelled when a
    sick person coughs or sneezes, or transmitted through contact with
    contaminated surfaces, said Linsey Marr, an aerosol expert at
    Virginia Tech. Aerosols are released even when a person without
    symptoms exhales, talks or sings, according to Dr. Marr and more
    than 200 other experts, who
    \href{https://academic.oup.com/cid/article/doi/10.1093/cid/ciaa939/5867798}{have
    outlined the evidence in an open letter to the World Health
    Organization}.
  \end{itemize}
\item ~
  \hypertarget{what-are-the-symptoms-of-coronavirus}{%
  \paragraph{What are the symptoms of
  coronavirus?}\label{what-are-the-symptoms-of-coronavirus}}

  \begin{itemize}
  \tightlist
  \item
    Common symptoms
    \href{https://www.nytimes.com/article/symptoms-coronavirus.html?action=click\&pgtype=Article\&state=default\&region=MAIN_CONTENT_3\&context=storylines_faq}{include
    fever, a dry cough, fatigue and difficulty breathing or shortness of
    breath.} Some of these symptoms overlap with those of the flu,
    making detection difficult, but runny noses and stuffy sinuses are
    less common.
    \href{https://www.nytimes.com/2020/04/27/health/coronavirus-symptoms-cdc.html?action=click\&pgtype=Article\&state=default\&region=MAIN_CONTENT_3\&context=storylines_faq}{The
    C.D.C. has also} added chills, muscle pain, sore throat, headache
    and a new loss of the sense of taste or smell as symptoms to look
    out for. Most people fall ill five to seven days after exposure, but
    symptoms may appear in as few as two days or as many as 14 days.
  \end{itemize}
\item ~
  \hypertarget{does-asymptomatic-transmission-of-covid-19-happen}{%
  \paragraph{Does asymptomatic transmission of Covid-19
  happen?}\label{does-asymptomatic-transmission-of-covid-19-happen}}

  \begin{itemize}
  \tightlist
  \item
    So far, the evidence seems to show it does. A widely cited
    \href{https://www.nature.com/articles/s41591-020-0869-5}{paper}
    published in April suggests that people are most infectious about
    two days before the onset of coronavirus symptoms and estimated that
    44 percent of new infections were a result of transmission from
    people who were not yet showing symptoms. Recently, a top expert at
    the World Health Organization stated that transmission of the
    coronavirus by people who did not have symptoms was ``very rare,''
    \href{https://www.nytimes.com/2020/06/09/world/coronavirus-updates.html?action=click\&pgtype=Article\&state=default\&region=MAIN_CONTENT_3\&context=storylines_faq\#link-1f302e21}{but
    she later walked back that statement.}
  \end{itemize}
\end{itemize}

``I think it's only a matter of time before the public sees those
numbers and starts emailing us that we need to shut down again,'' Ms.
Gerard said during a board meeting.

For many business owners, the continued uncertainty about the path and
the intensity of the pandemic has been vexing.

In Arizona, Gov. Doug Ducey, a Republican, moved energetically to reopen
the state in May, and places like swimming pools, gyms and Little League
fields have opened in recent weeks. Arizona officials reported more than
1,600 cases in a day for the first time on Friday.

Carla Logan, the owner of a bistro near downtown Phoenix, said that she
was trying to save her business while making sense of Arizona's rising
number of cases. Theories by some
\href{https://www.nytimes.com/interactive/2020/05/21/opinion/coronavirus-warm-weather-summer-infections.html}{scientists}
that the virus might diminish amid
\href{https://directorsblog.nih.gov/2020/06/02/will-warm-weather-slow-spread-of-novel-coronavirus/}{warmer
weather} were fading fast.

``We were hoping and praying the Arizona heat would kill the virus, but
that didn't happen,'' she said. ``A second shutdown for us would be
catastrophic.''

In Florida, the number of new coronavirus cases has topped 1,000 for all
but one of the past seven days. Most of the state began to reopen on May
4, though South Florida is still under tighter restrictions. Miami's
beaches only reopened on Wednesday.

Gov. Ron DeSantis, a Republican, has attributed the uptick to more
widespread testing. Retail stores, including Publix supermarkets and
Home Depot, now offer tests at a handful of locations, and state-run
sites allow anybody to get tested, regardless of age or symptoms,
without a doctor's prescription. Even then, demand at some sites is low,
and more than half of the available tests a day do not get used, the
governor said.

In some ways, life in the state feels like it is getting back to normal.
The Kennedy Space Center hosted the SpaceX launch. Jacksonville has been
eager to host the Republican National Convention. Orlando has allowed
the filming of professional wrestling and expects to soon welcome the
National Basketball Association and Major League Soccer.

Image

People took advantage of the opening of South Beach on Wednesday in
Miami Beach.Credit...Cliff Hawkins/Getty Images

Despite the uptick in cases in some states, there are also states that
have been reopened for weeks where the number of new known virus cases
has slowed. Pennsylvania, Indiana and Colorado --- which all began
reopening in late April or early May --- have seen hopeful signs.

It is possible that the full effect of reopening may be hidden from
view. Certain states and counties are testing less than others. And
reopening has looked different in different places, partly depending on
varying habits of residents.

In places where masks are standard and people are adhering to social
distancing --- both recommended by public health experts ---
transmission may be slower.

In Douglas County, Kan., home to 122,000 people, only 82 cases of the
coronavirus have been identified, an exceedingly low number for a place
of its size.

The county, which includes the college town of Lawrence, kept in place
restrictions on businesses even after statewide mandates were dropped.
Residents have continued to wear masks and stay far apart, said Dan
Partridge, the director of the local public health agency.

``The worry I have is that fatigue will set in and compliance will
slip,'' Mr. Partridge said.

Epidemiologists point to another factor that could result in even more
coronavirus outbreaks in the coming days: the widespread demonstrations
across the country, where protesters are packed shoulder-to-shoulder,
often without masks.

Minneapolis, where protests erupted after the death of
\href{https://www.nytimes.com/article/george-floyd-who-is.html}{George
Floyd}, is being watched especially closely, though health officials say
that it is too soon to know what effect the demonstrations had on the
virus.

A small number of Minnesota National Guard members mobilized for the
protests have tested positive, and new testing sites have been
established for demonstrators.

At least 30 cases nationally have been linked to protests, including 10
National Guard members and one police officer in Nebraska who have been
infected. Contact tracers in Chicago and elsewhere have begun asking
people who are positive for the coronavirus whether they have attended
protests.

Julie Bosman reported from Chicago, and Mitch Smith from Overland Park,
Kan. Reporting was contributed by Manny Fernandez from Houston, Patricia
Mazzei from Miami, Simon Romero from Albuquerque, Amy Harmon from New
York, and David Montgomery from Austin, Texas.

Advertisement

\protect\hyperlink{after-bottom}{Continue reading the main story}

\hypertarget{site-index}{%
\subsection{Site Index}\label{site-index}}

\hypertarget{site-information-navigation}{%
\subsection{Site Information
Navigation}\label{site-information-navigation}}

\begin{itemize}
\tightlist
\item
  \href{https://help.nytimes.com/hc/en-us/articles/115014792127-Copyright-notice}{©~2020~The
  New York Times Company}
\end{itemize}

\begin{itemize}
\tightlist
\item
  \href{https://www.nytco.com/}{NYTCo}
\item
  \href{https://help.nytimes.com/hc/en-us/articles/115015385887-Contact-Us}{Contact
  Us}
\item
  \href{https://www.nytco.com/careers/}{Work with us}
\item
  \href{https://nytmediakit.com/}{Advertise}
\item
  \href{http://www.tbrandstudio.com/}{T Brand Studio}
\item
  \href{https://www.nytimes.com/privacy/cookie-policy\#how-do-i-manage-trackers}{Your
  Ad Choices}
\item
  \href{https://www.nytimes.com/privacy}{Privacy}
\item
  \href{https://help.nytimes.com/hc/en-us/articles/115014893428-Terms-of-service}{Terms
  of Service}
\item
  \href{https://help.nytimes.com/hc/en-us/articles/115014893968-Terms-of-sale}{Terms
  of Sale}
\item
  \href{https://spiderbites.nytimes.com}{Site Map}
\item
  \href{https://help.nytimes.com/hc/en-us}{Help}
\item
  \href{https://www.nytimes.com/subscription?campaignId=37WXW}{Subscriptions}
\end{itemize}
