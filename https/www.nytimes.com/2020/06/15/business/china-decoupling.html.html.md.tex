Sections

SEARCH

\protect\hyperlink{site-content}{Skip to
content}\protect\hyperlink{site-index}{Skip to site index}

\href{https://www.nytimes.com/section/business}{Business}

\href{https://myaccount.nytimes.com/auth/login?response_type=cookie\&client_id=vi}{}

\href{https://www.nytimes.com/section/todayspaper}{Today's Paper}

\href{/section/business}{Business}\textbar{}Break the China Habit?
Lobsters, Lights and Toilets Show How Hard It Is

\url{https://nyti.ms/2MZ49TL}

\begin{itemize}
\item
\item
\item
\item
\item
\item
\end{itemize}

\href{https://www.nytimes.com/news-event/coronavirus?action=click\&pgtype=Article\&state=default\&region=TOP_BANNER\&context=storylines_menu}{The
Coronavirus Outbreak}

\begin{itemize}
\tightlist
\item
  live\href{https://www.nytimes.com/2020/08/03/world/coronavirus-covid-19.html?action=click\&pgtype=Article\&state=default\&region=TOP_BANNER\&context=storylines_menu}{Latest
  Updates}
\item
  \href{https://www.nytimes.com/interactive/2020/us/coronavirus-us-cases.html?action=click\&pgtype=Article\&state=default\&region=TOP_BANNER\&context=storylines_menu}{Maps
  and Cases}
\item
  \href{https://www.nytimes.com/interactive/2020/science/coronavirus-vaccine-tracker.html?action=click\&pgtype=Article\&state=default\&region=TOP_BANNER\&context=storylines_menu}{Vaccine
  Tracker}
\item
  \href{https://www.nytimes.com/2020/08/02/us/covid-college-reopening.html?action=click\&pgtype=Article\&state=default\&region=TOP_BANNER\&context=storylines_menu}{College
  Reopening}
\item
  \href{https://www.nytimes.com/live/2020/08/03/business/stock-market-today-coronavirus?action=click\&pgtype=Article\&state=default\&region=TOP_BANNER\&context=storylines_menu}{Economy}
\end{itemize}

Advertisement

\protect\hyperlink{after-top}{Continue reading the main story}

Supported by

\protect\hyperlink{after-sponsor}{Continue reading the main story}

\hypertarget{break-the-china-habit-lobsters-lights-and-toilets-show-how-hard-it-is}{%
\section{Break the China Habit? Lobsters, Lights and Toilets Show How
Hard It
Is}\label{break-the-china-habit-lobsters-lights-and-toilets-show-how-hard-it-is}}

The risks of relying economically on the Asian superpower have never
seemed clearer. But as the world tries to get moving again, it needs
China more than ever.

\includegraphics{https://static01.nyt.com/images/2020/06/15/business/15ALTJPchina-reliance1-print/merlin_172059801_3c4fd5de-da7d-4baf-910e-874f248ed4e4-articleLarge.jpg?quality=75\&auto=webp\&disable=upscale}

\href{https://www.nytimes.com/by/damien-cave}{\includegraphics{https://static01.nyt.com/images/2018/10/08/multimedia/author-damien-cave/author-damien-cave-thumbLarge.png}}\href{https://www.nytimes.com/by/motoko-rich}{\includegraphics{https://static01.nyt.com/images/2018/10/15/multimedia/author-motoko-rich/author-motoko-rich-thumbLarge.png}}\href{https://www.nytimes.com/by/jack-ewing}{\includegraphics{https://static01.nyt.com/images/2018/07/18/multimedia/author-jack-ewing/author-jack-ewing-thumbLarge.png}}

By \href{https://www.nytimes.com/by/damien-cave}{Damien Cave},
\href{https://www.nytimes.com/by/motoko-rich}{Motoko Rich} and
\href{https://www.nytimes.com/by/jack-ewing}{Jack Ewing}

\begin{itemize}
\item
  Published June 15, 2020Updated June 17, 2020
\item
  \begin{itemize}
  \item
  \item
  \item
  \item
  \item
  \item
  \end{itemize}
\end{itemize}

\href{https://cn.nytimes.com/business/20200615/china-decoupling/}{阅读简体中文版}\href{https://cn.nytimes.com/business/20200615/china-decoupling/zh-hant/}{閱讀繁體中文版}

As the coronavirus pandemic amplifies longstanding concerns over the
world's economic dependence on
\href{https://www.nytimes.com/2020/06/17/world/asia/China-DNA-surveillance.html}{China},
many countries are trying to reduce their exposure to Beijing's brand of
business.

Japan has set aside \$2.2 billion to help companies
\href{https://www.bloomberg.com/news/articles/2020-04-08/japan-to-fund-firms-to-shift-production-out-of-china}{shift
production out of China}. European trade ministers have emphasized the
need to diversify supply chains. Several countries, including Australia
and Germany, have moved to keep China, among others, from buying
businesses weakened by lockdowns. Hawks in the Trump administration also
continue to press for an economic ``decoupling'' from Beijing.

But outside government circles, in the companies where the decisions
about manufacturing and sales are actually made, the calculations are
more complex.

China is a hard habit to break.

Even after its early mishandling of the coronavirus disrupted the
country's ability to make and buy the world's products, further exposing
the faults of its authoritarian system and leading it to ratchet up its
\href{https://www.nytimes.com/2020/06/07/world/asia/china-coronavirus.html}{propaganda
war}, China's economic power makes it the last best hope for avoiding a
protracted global downturn.

``When this all started, we were thinking, Where else can we go?'' said
Fedele Camarda, a third-generation lobster fisherman in Western
Australia, which sends most of its catch to China. ``Then the rest of
the world was also compromised by the coronavirus, and China is the one
getting back on its feet.''

``Although they're just one market,'' he added, ``they're one very big
market.''

To understand how businesses are responding to the shifting dynamics and
risks, The New York Times profiled three companies in three countries
that are heavily reliant on China. Their experiences vary, but they are
all trying to work out just how much of a breakup with China is needed
--- or whether they can afford one.

\hypertarget{beg-to-return-australias-lobster-boats}{%
\subsection{Beg to Return: Australia's Lobster
Boats}\label{beg-to-return-australias-lobster-boats}}

\includegraphics{https://static01.nyt.com/images/2020/06/15/business/00JPchina-reliance2-print/merlin_172059750_9fe7dbcd-35fe-4b02-8ca6-eae71c2cd3e8-articleLarge.jpg?quality=75\&auto=webp\&disable=upscale}

When Mr. Camarda fished for lobster off Australia's west coast in the
1990s, his catch ended up on plates in a variety of countries.

Fresh crays, as the lobsters are known, went to Japan. Canned lobster
meat went to the United States. The rest was sold inside Australia or to
its nearest neighbors.

But starting around 2000, China began paying more for live lobsters, and
ordering more. That led to a near-total reliance on that market and a
sense of complacency: By the beginning of this year, 95 percent of
Australia's spiny lobsters were being shipped to sellers and restaurants
in China.

\hypertarget{latest-updates-economy}{%
\section{\texorpdfstring{\href{https://www.nytimes.com/live/2020/08/03/business/stock-market-today-coronavirus?action=click\&pgtype=Article\&state=default\&region=MAIN_CONTENT_1\&context=storylines_live_updates}{Latest
Updates:
Economy}}{Latest Updates: Economy}}\label{latest-updates-economy}}

\href{https://www.nytimes.com/live/2020/08/03/business/stock-market-today-coronavirus?action=click\&pgtype=Article\&state=default\&region=MAIN_CONTENT_1\&context=storylines_live_updates\#the-chicago-fed-president-says-its-up-to-congress-to-save-the-economy}{10h
ago}

\href{https://www.nytimes.com/live/2020/08/03/business/stock-market-today-coronavirus?action=click\&pgtype=Article\&state=default\&region=MAIN_CONTENT_1\&context=storylines_live_updates\#the-chicago-fed-president-says-its-up-to-congress-to-save-the-economy}{The
Chicago Fed president says it's up to Congress to save the economy.}

\href{https://www.nytimes.com/live/2020/08/03/business/stock-market-today-coronavirus?action=click\&pgtype=Article\&state=default\&region=MAIN_CONTENT_1\&context=storylines_live_updates\#faa-says-boeing-has-effectively-mitigated-defects-in-the-737-max}{11h
ago}

\href{https://www.nytimes.com/live/2020/08/03/business/stock-market-today-coronavirus?action=click\&pgtype=Article\&state=default\&region=MAIN_CONTENT_1\&context=storylines_live_updates\#faa-says-boeing-has-effectively-mitigated-defects-in-the-737-max}{F.A.A.
says Boeing has `effectively mitigated' defects in the 737 Max.}

\href{https://www.nytimes.com/live/2020/08/03/business/stock-market-today-coronavirus?action=click\&pgtype=Article\&state=default\&region=MAIN_CONTENT_1\&context=storylines_live_updates\#small-businesses-got-emergency-loans-but-not-what-they-expected}{13h
ago}

\href{https://www.nytimes.com/live/2020/08/03/business/stock-market-today-coronavirus?action=click\&pgtype=Article\&state=default\&region=MAIN_CONTENT_1\&context=storylines_live_updates\#small-businesses-got-emergency-loans-but-not-what-they-expected}{Small
businesses got emergency loans, but not what they expected.}

\href{https://www.nytimes.com/live/2020/08/03/business/stock-market-today-coronavirus?action=click\&pgtype=Article\&state=default\&region=MAIN_CONTENT_1\&context=storylines_live_updates}{See
more updates}

More live coverage:
\href{https://www.nytimes.com/2020/08/03/world/coronavirus-covid-19.html?action=click\&pgtype=Article\&state=default\&region=MAIN_CONTENT_1\&context=storylines_live_updates}{Global}

``We all talked about different strategies to overcome the problem, to
not be so reliant on China,'' Mr. Camarda said. ``We just didn't get
around to it.''

And they still haven't, even after the need for diversification hit like
a hammer on Jan. 25.

That's when China, in the midst of its outbreak, stopped buying.
Officials shut down the wet markets that sell fresh meat, vegetables and
seafood, forcing the entire fleet of lobster boats up and down
Australia's west coast --- all 234 --- to stop fishing. More than 2,000
people found themselves without work.

Australia's lobster processors tried to quickly diversify, calling
buyers in every country they had ever worked with, reaching back to
contacts from decades earlier. The industry association pleaded with the
Australian government for help: requesting a larger quota for the year,
an extension of the season and more freedom to sell directly to the
public, all of which were approved by fisheries managers.

But none of it did much good for Mr. Camarda. While certain food exports
to China from other parts of the world increased ---
\href{https://www.poultryworld.net/Meat/Articles/2020/3/Brazilian-chicken-exports-to-China-grow-59-despite-Covid19-556245E/}{chicken
meat from Brazil}, for example --- only a few boats went out in
February, March and April, pulling in very little.

Mr. Camarda returned to the water only about a month ago. Orders to his
company, Neptune 3, are starting to come in again from China, at prices
that are roughly half what they were in January. The orders aren't
anywhere near as large, either, but the industry has coalesced around
trying to rebuild its ties with China, rather than looking elsewhere.

``Even if prices are low and the amount of product is down, we need to
find a way to service that market, because providing that market is what
works for us,'' said Matt Taylor, the chief executive of Western Rock
Lobster, the industry's professional association.

As of about a month ago, there was still one major challenge: shipping.
Supply chains had been scrambled, as passenger planes that carry much of
the world's cargo have been idled and shipping has decreased. So once
again the Australian government stepped in, this time with around \$70
million to subsidize charter flights for seafood exports.

Despite calls for greater self-sufficiency, diversification and
\href{https://www.skynews.com.au/details/_6147890749001}{sovereignty},
as well as moves by China that have hurt barley and beef exports,
Australia is not running away from the Chinese market. It is subsidizing
efforts to get back in.

\hypertarget{no-savior-germanys-china-optimism-wanes}{%
\subsection{No Savior: Germany's China Optimism
Wanes}\label{no-savior-germanys-china-optimism-wanes}}

Image

The lighting manufacturer Osram and other German companies are
rethinking their supply chains.Credit...Andreas Gebert/Reuters

The last time German industry faced a severe downturn, relief came from
China. The country's explosive growth and hunger for Western technology
helped German exporters bounce back quickly from the deep recession a
decade ago.

``In 2008, there were two markets that I ran to: China and the Middle
East,'' said Olaf Berlien, chief executive of Osram, one of the world's
largest lighting companies, which is based in Munich.

But he does not expect Chinese sales to save German industry again.

``China is still a market,'' Mr. Berlien said, ``but it's not a growth
market.''

Osram had turned bearish on China even before the coronavirus forced the
country into quarantine. Car sales were down in 2019 after years of
double-digit growth, largely because of the trade war with the United
States.

The problem is that there is no other market to take China's place as an
engine of world growth. India has potential, but is too disorganized,
Mr. Berlien said. Middle Eastern countries like Saudi Arabia and Qatar
are no longer as wealthy now that oil prices have collapsed.

Osram's diminished expectations for China reflect a deepening skepticism
across Europe about the benefits of turning to the Asian superpower in
times of need. Phil Hogan, the European Union trade commissioner, echoed
the concerns of officials in Germany and France when he
\href{https://ec.europa.eu/commission/commissioners/2019-2024/hogan/announcements/introductory-statement-commissioner-phil-hogan-informal-meeting-eu-trade-ministers_en}{called
in April} for a discussion ``on what it means to be strategically
autonomous.''

Osram, which provides lights for cars and other uses, didn't need the
nudge. It has four factories in China, Mr. Berlien said, but the company
manufactures its more sophisticated products in Malaysia, Germany and
the United States because of China's lack of protection for intellectual
property.

``China is no longer the workbench of the world,'' he said.

Mr. Berlien said that his company and others in Germany had learned from
past crises to insure themselves against supply chain disruptions, by
taking steps like having at least two suppliers of every component or
raw material.

He added that while Osram had no plans to reduce its presence in China,
the coronavirus crisis would prompt companies to look harder for
suppliers closer to home.

``What we are all learning, and I talk to a lot of managers and C.E.O.s
in Germany, is that we all have to rethink our logistics and supply
chains,'' Mr. Berlien said.

``They were very fragmented and very vulnerable,'' he added. ``Because
of the price pressure that we are all under, we took the cheapest
provider wherever in the world it might have been. We undervalued the
provider who was just around the corner.''

\hypertarget{stay-the-course-japans-luxury-toilets}{%
\subsection{Stay the Course: Japan's Luxury
Toilets}\label{stay-the-course-japans-luxury-toilets}}

Image

A Toto toilet factory in Kitakyushu, Japan. China accounted for half of
Toto's overseas sales last year.Credit...Sakura Murakami/Reuters

Toto makes what China's nouveau riche really want: electronic bidet
toilets with heated seats, warm water jets, pleasingly shaped ceramic
bowls and automated lids.

The company, Japan's largest toilet maker, opened its Beijing office in
1985, and its reliance on China has grown along with the country's rise.
China accounted for half of Toto's overseas sales last year, and it has
seven factories in the country.

But even after China's lockdown closed Toto's assembly lines in January
and February, causing delays and lost revenue, the company never
considered withdrawing.

For one thing, it's a huge market with a high rate of homeownership and
rising disposable incomes. For another, many of its workers have the
kinds of technical skills that Toto needs.

``China is close to Japan, and it has the power of a lot of people,''
said Sonoko Abe, a Toto spokeswoman.

In daily meetings, executives discussed ``how we can adjust to the
situation,'' Ms. Abe said. Although the company has plants in Thailand
and Vietnam, it did not try to shift production, but instead relies on a
pipeline of stored inventory.

Many other Japanese companies, even when there are incentives to look
elsewhere, are stepping away from China only slowly, if at all.

The Japanese mask maker Iris Ohyama, for example, which has factories in
Dalian and Suzhou that produce goods for both the Chinese and Japanese
markets. It is drawing on some of the government's funding to open new
factory lines in Japan to accommodate the domestic market, and is
exploring options in France and the United States.

But it has no plans to stop manufacturing in China. **** ``We think the
Chinese market is very important in the long run,'' said Atsuko Kido, a
spokeswoman.

It is also important right now: The International Monetary Fund has
reported that China will be one of the few countries to see economic
growth in 2020, while the U.S. economy is expected to contract by about
6 percent and the eurozone by 7.5 percent.

Kathy Matsui, chief Japan equity strategist at Goldman Sachs in Tokyo,
said that in a time of severe economic pressure, even those who oppose
China's politics feel that they need the country's economy to prosper.

``We are all interconnected,'' she said. ``So it's vital that China
continues to grow for pretty much every major economy around the
world.''

Makiko Inoue contributed reporting.

Advertisement

\protect\hyperlink{after-bottom}{Continue reading the main story}

\hypertarget{site-index}{%
\subsection{Site Index}\label{site-index}}

\hypertarget{site-information-navigation}{%
\subsection{Site Information
Navigation}\label{site-information-navigation}}

\begin{itemize}
\tightlist
\item
  \href{https://help.nytimes.com/hc/en-us/articles/115014792127-Copyright-notice}{©~2020~The
  New York Times Company}
\end{itemize}

\begin{itemize}
\tightlist
\item
  \href{https://www.nytco.com/}{NYTCo}
\item
  \href{https://help.nytimes.com/hc/en-us/articles/115015385887-Contact-Us}{Contact
  Us}
\item
  \href{https://www.nytco.com/careers/}{Work with us}
\item
  \href{https://nytmediakit.com/}{Advertise}
\item
  \href{http://www.tbrandstudio.com/}{T Brand Studio}
\item
  \href{https://www.nytimes.com/privacy/cookie-policy\#how-do-i-manage-trackers}{Your
  Ad Choices}
\item
  \href{https://www.nytimes.com/privacy}{Privacy}
\item
  \href{https://help.nytimes.com/hc/en-us/articles/115014893428-Terms-of-service}{Terms
  of Service}
\item
  \href{https://help.nytimes.com/hc/en-us/articles/115014893968-Terms-of-sale}{Terms
  of Sale}
\item
  \href{https://spiderbites.nytimes.com}{Site Map}
\item
  \href{https://help.nytimes.com/hc/en-us}{Help}
\item
  \href{https://www.nytimes.com/subscription?campaignId=37WXW}{Subscriptions}
\end{itemize}
