Sections

SEARCH

\protect\hyperlink{site-content}{Skip to
content}\protect\hyperlink{site-index}{Skip to site index}

\href{https://www.nytimes.com/section/world/asia}{Asia Pacific}

\href{https://myaccount.nytimes.com/auth/login?response_type=cookie\&client_id=vi}{}

\href{https://www.nytimes.com/section/todayspaper}{Today's Paper}

\href{/section/world/asia}{Asia Pacific}\textbar{}Pakistan's Lockdown
Ended a Month Ago. Now Hospital Signs Read `Full.'

\url{https://nyti.ms/37tMg94}

\begin{itemize}
\item
\item
\item
\item
\item
\end{itemize}

\href{https://www.nytimes.com/news-event/coronavirus?action=click\&pgtype=Article\&state=default\&region=TOP_BANNER\&context=storylines_menu}{The
Coronavirus Outbreak}

\begin{itemize}
\tightlist
\item
  live\href{https://www.nytimes.com/2020/08/04/world/coronavirus-cases.html?action=click\&pgtype=Article\&state=default\&region=TOP_BANNER\&context=storylines_menu}{Latest
  Updates}
\item
  \href{https://www.nytimes.com/interactive/2020/us/coronavirus-us-cases.html?action=click\&pgtype=Article\&state=default\&region=TOP_BANNER\&context=storylines_menu}{Maps
  and Cases}
\item
  \href{https://www.nytimes.com/interactive/2020/science/coronavirus-vaccine-tracker.html?action=click\&pgtype=Article\&state=default\&region=TOP_BANNER\&context=storylines_menu}{Vaccine
  Tracker}
\item
  \href{https://www.nytimes.com/2020/08/02/us/covid-college-reopening.html?action=click\&pgtype=Article\&state=default\&region=TOP_BANNER\&context=storylines_menu}{College
  Reopening}
\item
  \href{https://www.nytimes.com/live/2020/08/04/business/stock-market-today-coronavirus?action=click\&pgtype=Article\&state=default\&region=TOP_BANNER\&context=storylines_menu}{Economy}
\end{itemize}

Advertisement

\protect\hyperlink{after-top}{Continue reading the main story}

Supported by

\protect\hyperlink{after-sponsor}{Continue reading the main story}

\hypertarget{pakistans-lockdown-ended-a-month-ago-now-hospital-signs-read-full}{%
\section{Pakistan's Lockdown Ended a Month Ago. Now Hospital Signs Read
`Full.'}\label{pakistans-lockdown-ended-a-month-ago-now-hospital-signs-read-full}}

Medical workers are falling ill in Pakistan at alarming rates as the
country registers at least 100,000 new coronavirus cases since the
lockdown was lifted.

\includegraphics{https://static01.nyt.com/images/2020/06/15/world/15virus-pakistan-1/merlin_173305998_585bd227-6245-4cf4-a945-4ee459864927-articleLarge.jpg?quality=75\&auto=webp\&disable=upscale}

By Zia ur-Rehman, \href{https://www.nytimes.com/by/salman-masood}{Salman
Masood} and \href{https://www.nytimes.com/by/maria-abi-habib}{Maria
Abi-Habib}

\begin{itemize}
\item
  Published June 15, 2020Updated June 26, 2020
\item
  \begin{itemize}
  \item
  \item
  \item
  \item
  \item
  \end{itemize}
\end{itemize}

KARACHI --- Pakistanis stricken by the coronavirus are being turned away
from hospitals that have simply closed their gates and put up signs
reading ``full house.'' Doctors and nurses are falling ill at alarming
rates, and are also coming under physical assault from desperate and
angry families.

When
\href{https://www.nytimes.com/2020/06/29/world/asia/pakistan-stock-exchange-shooting.html}{Pakistan}'s
government lifted its lockdown on May 9, it warned that the already
impoverished country could no longer withstand the shutdown needed to
mitigate the pandemic's spread. But now left unshackled, the virus is
meting out devastation in other ways, and panic is rising.

Before reopening, Pakistan had recorded about 25,000 infections. A month
later, the country recorded an additional 100,000 cases --- almost
certainly an undercount --- and the pandemic shows no signs of abating.
At least 2,356 people have died of Covid-19, according to official
figures released Thursday.

Pakistan is now reporting so many new cases that it is among the World
Health Organization's top 10 countries where the virus is on the rise.
The W.H.O. wrote a \href{https://www.dawn.com/news/1562494}{letter
criticizing the government's} efforts on June 7 and recommended that
lockdown be reimposed, stating that Pakistan did not meet any of the
criteria needed to lift it.

Medical professionals now expect the virus to peak in July or August and
infect up to 900,000, adding further strain to an already shaky health
care system some warn may collapse.

But government officials have ruled out the possibility of a further
lockdown and dismissed the recommendations by the W.H.O.

\includegraphics{https://static01.nyt.com/images/2020/06/15/world/15virus-pakistan-2/merlin_171005490_f3f854f0-d80f-4e9a-beac-1af4617426fb-articleLarge.jpg?quality=75\&auto=webp\&disable=upscale}

On a recent day in the sprawling port city of Karachi, Ali Hussain and
his brother shuttled between public hospitals, looking for help and
receiving none. Mr. Hussain's older brother had a severe cough and fever
but had been unable to get a coronavirus test for days.

``We cannot afford the private hospitals, they are charging tens of
thousands rupees,'' said Mr. Hussain, who earned 20,000 rupees per
month, about \$121, working at a textile mill before the lockdown.

Like many others, the Hussain family is suffering not only because of
the coronavirus itself but also the economic devastation the pandemic
has wrought. Mr. Hussain said he and his brother could barely afford to
feed themselves since they lost their jobs in March, let alone pay for
private care.

\hypertarget{latest-updates-global-coronavirus-outbreak}{%
\section{\texorpdfstring{\href{https://www.nytimes.com/2020/08/04/world/coronavirus-cases.html?action=click\&pgtype=Article\&state=default\&region=MAIN_CONTENT_1\&context=storylines_live_updates}{Latest
Updates: Global Coronavirus
Outbreak}}{Latest Updates: Global Coronavirus Outbreak}}\label{latest-updates-global-coronavirus-outbreak}}

Updated 2020-08-05T06:48:23.151Z

\begin{itemize}
\tightlist
\item
  \href{https://www.nytimes.com/2020/08/04/world/coronavirus-cases.html?action=click\&pgtype=Article\&state=default\&region=MAIN_CONTENT_1\&context=storylines_live_updates\#link-762df92}{As
  talks drag on, McConnell signals openness to jobless aid extension,
  and negotiators agree on a deadline.}
\item
  \href{https://www.nytimes.com/2020/08/04/world/coronavirus-cases.html?action=click\&pgtype=Article\&state=default\&region=MAIN_CONTENT_1\&context=storylines_live_updates\#link-1228a480}{Novavax
  sees encouraging results from two studies of its experimental
  vaccine.}
\item
  \href{https://www.nytimes.com/2020/08/04/world/coronavirus-cases.html?action=click\&pgtype=Article\&state=default\&region=MAIN_CONTENT_1\&context=storylines_live_updates\#link-794484ed}{Mississippians
  must now wear masks in public, governor says.}
\end{itemize}

\href{https://www.nytimes.com/2020/08/04/world/coronavirus-cases.html?action=click\&pgtype=Article\&state=default\&region=MAIN_CONTENT_1\&context=storylines_live_updates}{See
more updates}

More live coverage:
\href{https://www.nytimes.com/live/2020/08/04/business/stock-market-today-coronavirus?action=click\&pgtype=Article\&state=default\&region=MAIN_CONTENT_1\&context=storylines_live_updates}{Markets}

``We are completely broke and we do not know what to do,'' Mr. Hussain
lamented.

The World Bank projects that Pakistan's economy will contract by 0.2
percent next fiscal year. Up to 18 million of the country's 74 million
jobs could be lost, according to the Pakistan Institute of Development
Economics, an independent research firm set up by the government.

More immediately, Pakistan's struggling health care sector is in deep
crisis.

Image

The police arresting doctors in Quetta in April. The doctors were
demanding facilities and prevention kits to deal with coronavirus
patients.Credit...Arshad Butt/Associated Press

Only a third of Karachi's 600 beds in intensive care wards are available
to treat coronavirus patients in the city's private and public
hospitals, for a population of about 20 million, according to local
health officials. According to the W.H.O., only 751 ventilators are
dedicated to the pandemic in Pakistan, the world's fifth most populous
country, with some 200 million people.

Health care workers admit privately that they are referring patients
like Mr. Hussain's brother to other hospitals they know are at or over
capacity because they fear being attacked by desperate families. Medical
workers across Pakistan are being assaulted on a near-daily basis for
not being able to admit patients or having to tell families that their
loved ones had died.

``Our hospitals are completely exhausted,'' said one doctor, who asked
for his name to be withheld because he is a government employee.

Late last month, a family attacked the staff of one Karachi hospital
with knives and iron rods after doctors declared their relative dead,
rampaging through the emergency ward. On May 14, the emergency
department of another major government hospital in Karachi was ransacked
after health care workers refused to give over the body of their loved
one, warning the family could contract the virus by handling the remains
without using any precautions.

After several similar episodes, employees say that many hospitals are
now handing over the bodies of coronavirus victims to their families
anyway, worried more about the violent backlash than the pandemic's
spread.

The anger reflects the grief and panic that is setting in across the
country, and also an erosion of trust between the state and its
citizens.

\href{https://www.nytimes.com/2020/06/26/world/asia/pakistan-imran-khan-bin-laden-martyr.html}{Prime
Minister Imran Khan} and other officials have frequently dismissed the
virus as a common flu, then rushed to urge people to stay home before
dismissing the severity of the pandemic again. Unfounded rumors have
spread on social media that the government is inflating coronavirus
numbers to milk the international community for more aid money, secretly
leaving patients to die of other causes.

Image

Protesters in Karachi last week demanding the reopening of schools that
were closed to help contain the coronavirus.Credit...Fareed
Khan/Associated Press

The already low morale among health care workers has plummeted further
since the lockdown was lifted. In March, doctors and nurses
\href{https://www.nytimes.com/2020/03/26/world/asia/pakistan-coronavirus-tablighi-jamaat.html}{threatened
to walk off the job and some called in sick}, refusing to work if the
government did not provide them with personal protective equipment. Some
had to spend up to half of their salaries to buy their own masks, prices
skyrocketing as panicked citizens hoarded supplies.

\href{https://www.nytimes.com/news-event/coronavirus?action=click\&pgtype=Article\&state=default\&region=MAIN_CONTENT_3\&context=storylines_faq}{}

\hypertarget{the-coronavirus-outbreak-}{%
\subsubsection{The Coronavirus Outbreak
›}\label{the-coronavirus-outbreak-}}

\hypertarget{frequently-asked-questions}{%
\paragraph{Frequently Asked
Questions}\label{frequently-asked-questions}}

Updated August 4, 2020

\begin{itemize}
\item ~
  \hypertarget{i-have-antibodies-am-i-now-immune}{%
  \paragraph{I have antibodies. Am I now
  immune?}\label{i-have-antibodies-am-i-now-immune}}

  \begin{itemize}
  \tightlist
  \item
    As of right
    now,\href{https://www.nytimes.com/2020/07/22/health/covid-antibodies-herd-immunity.html?action=click\&pgtype=Article\&state=default\&region=MAIN_CONTENT_3\&context=storylines_faq}{that
    seems likely, for at least several months.} There have been
    frightening accounts of people suffering what seems to be a second
    bout of Covid-19. But experts say these patients may have a
    drawn-out course of infection, with the virus taking a slow toll
    weeks to months after initial exposure. People infected with the
    coronavirus typically
    \href{https://www.nature.com/articles/s41586-020-2456-9}{produce}
    immune molecules called antibodies, which are
    \href{https://www.nytimes.com/2020/05/07/health/coronavirus-antibody-prevalence.html?action=click\&pgtype=Article\&state=default\&region=MAIN_CONTENT_3\&context=storylines_faq}{protective
    proteins made in response to an
    infection}\href{https://www.nytimes.com/2020/05/07/health/coronavirus-antibody-prevalence.html?action=click\&pgtype=Article\&state=default\&region=MAIN_CONTENT_3\&context=storylines_faq}{.
    These antibodies may} last in the body
    \href{https://www.nature.com/articles/s41591-020-0965-6}{only two to
    three months}, which may seem worrisome, but that's perfectly normal
    after an acute infection subsides, said Dr. Michael Mina, an
    immunologist at Harvard University. It may be possible to get the
    coronavirus again, but it's highly unlikely that it would be
    possible in a short window of time from initial infection or make
    people sicker the second time.
  \end{itemize}
\item ~
  \hypertarget{im-a-small-business-owner-can-i-get-relief}{%
  \paragraph{I'm a small-business owner. Can I get
  relief?}\label{im-a-small-business-owner-can-i-get-relief}}

  \begin{itemize}
  \tightlist
  \item
    The
    \href{https://www.nytimes.com/article/small-business-loans-stimulus-grants-freelancers-coronavirus.html?action=click\&pgtype=Article\&state=default\&region=MAIN_CONTENT_3\&context=storylines_faq}{stimulus
    bills enacted in March} offer help for the millions of American
    small businesses. Those eligible for aid are businesses and
    nonprofit organizations with fewer than 500 workers, including sole
    proprietorships, independent contractors and freelancers. Some
    larger companies in some industries are also eligible. The help
    being offered, which is being managed by the Small Business
    Administration, includes the Paycheck Protection Program and the
    Economic Injury Disaster Loan program. But lots of folks have
    \href{https://www.nytimes.com/interactive/2020/05/07/business/small-business-loans-coronavirus.html?action=click\&pgtype=Article\&state=default\&region=MAIN_CONTENT_3\&context=storylines_faq}{not
    yet seen payouts.} Even those who have received help are confused:
    The rules are draconian, and some are stuck sitting on
    \href{https://www.nytimes.com/2020/05/02/business/economy/loans-coronavirus-small-business.html?action=click\&pgtype=Article\&state=default\&region=MAIN_CONTENT_3\&context=storylines_faq}{money
    they don't know how to use.} Many small-business owners are getting
    less than they expected or
    \href{https://www.nytimes.com/2020/06/10/business/Small-business-loans-ppp.html?action=click\&pgtype=Article\&state=default\&region=MAIN_CONTENT_3\&context=storylines_faq}{not
    hearing anything at all.}
  \end{itemize}
\item ~
  \hypertarget{what-are-my-rights-if-i-am-worried-about-going-back-to-work}{%
  \paragraph{What are my rights if I am worried about going back to
  work?}\label{what-are-my-rights-if-i-am-worried-about-going-back-to-work}}

  \begin{itemize}
  \tightlist
  \item
    Employers have to provide
    \href{https://www.osha.gov/SLTC/covid-19/standards.html}{a safe
    workplace} with policies that protect everyone equally.
    \href{https://www.nytimes.com/article/coronavirus-money-unemployment.html?action=click\&pgtype=Article\&state=default\&region=MAIN_CONTENT_3\&context=storylines_faq}{And
    if one of your co-workers tests positive for the coronavirus, the
    C.D.C.} has said that
    \href{https://www.cdc.gov/coronavirus/2019-ncov/community/guidance-business-response.html}{employers
    should tell their employees} -\/- without giving you the sick
    employee's name -\/- that they may have been exposed to the virus.
  \end{itemize}
\item ~
  \hypertarget{should-i-refinance-my-mortgage}{%
  \paragraph{Should I refinance my
  mortgage?}\label{should-i-refinance-my-mortgage}}

  \begin{itemize}
  \tightlist
  \item
    \href{https://www.nytimes.com/article/coronavirus-money-unemployment.html?action=click\&pgtype=Article\&state=default\&region=MAIN_CONTENT_3\&context=storylines_faq}{It
    could be a good idea,} because mortgage rates have
    \href{https://www.nytimes.com/2020/07/16/business/mortgage-rates-below-3-percent.html?action=click\&pgtype=Article\&state=default\&region=MAIN_CONTENT_3\&context=storylines_faq}{never
    been lower.} Refinancing requests have pushed mortgage applications
    to some of the highest levels since 2008, so be prepared to get in
    line. But defaults are also up, so if you're thinking about buying a
    home, be aware that some lenders have tightened their standards.
  \end{itemize}
\item ~
  \hypertarget{what-is-school-going-to-look-like-in-september}{%
  \paragraph{What is school going to look like in
  September?}\label{what-is-school-going-to-look-like-in-september}}

  \begin{itemize}
  \tightlist
  \item
    It is unlikely that many schools will return to a normal schedule
    this fall, requiring the grind of
    \href{https://www.nytimes.com/2020/06/05/us/coronavirus-education-lost-learning.html?action=click\&pgtype=Article\&state=default\&region=MAIN_CONTENT_3\&context=storylines_faq}{online
    learning},
    \href{https://www.nytimes.com/2020/05/29/us/coronavirus-child-care-centers.html?action=click\&pgtype=Article\&state=default\&region=MAIN_CONTENT_3\&context=storylines_faq}{makeshift
    child care} and
    \href{https://www.nytimes.com/2020/06/03/business/economy/coronavirus-working-women.html?action=click\&pgtype=Article\&state=default\&region=MAIN_CONTENT_3\&context=storylines_faq}{stunted
    workdays} to continue. California's two largest public school
    districts --- Los Angeles and San Diego --- said on July 13, that
    \href{https://www.nytimes.com/2020/07/13/us/lausd-san-diego-school-reopening.html?action=click\&pgtype=Article\&state=default\&region=MAIN_CONTENT_3\&context=storylines_faq}{instruction
    will be remote-only in the fall}, citing concerns that surging
    coronavirus infections in their areas pose too dire a risk for
    students and teachers. Together, the two districts enroll some
    825,000 students. They are the largest in the country so far to
    abandon plans for even a partial physical return to classrooms when
    they reopen in August. For other districts, the solution won't be an
    all-or-nothing approach.
    \href{https://bioethics.jhu.edu/research-and-outreach/projects/eschool-initiative/school-policy-tracker/}{Many
    systems}, including the nation's largest, New York City, are
    devising
    \href{https://www.nytimes.com/2020/06/26/us/coronavirus-schools-reopen-fall.html?action=click\&pgtype=Article\&state=default\&region=MAIN_CONTENT_3\&context=storylines_faq}{hybrid
    plans} that involve spending some days in classrooms and other days
    online. There's no national policy on this yet, so check with your
    municipal school system regularly to see what is happening in your
    community.
  \end{itemize}
\end{itemize}

So far, at least 35 health care workers have died of the pandemic, the
Pakistan Medical Association said in a statement Thursday. At least
3,600 health care workers are infected with the virus, according to
official figures.

The government ``did not listen to what doctors were saying. Now the
result of this negligence is obvious,'' the Pakistan Medical Association
said in its statement.

In Punjab, the country's most populous province, a doctors' association
claimed earlier this month that 40 percent of the province's medical
staff had tested positive for coronavirus.

``While the pandemic stares us all in the face, the morale of health
care providers has hit rock bottom,'' said Dr. Salman Haseeb Chaudhry,
who represents the Young Doctors Association, at a news conference this
month.

At a protest among health care workers on Tuesday, Shafiq Awan, the
leader of a paramedic association in Karachi, said the government was
not heeding their advice.

Image

Selling face masks in Karachi on Sunday.Credit...Shahzaib Akber/EPA, via
Shutterstock

``We need protective gear, not salutes and praises. If we start dying or
are unable to work, who will treat patients?'' Mr. Awan asked.

Under withering criticism, Prime Minister Khan hit back on Thursday,
saying that his government had responded adequately to the pandemic.

Mr. Khan was at first reluctant to impose a lockdown, stating in early
March that the country's economy could not weather the fallout. By the
end of that month, the country's powerful military sidelined Mr. Khan to
shut down the country.

Both the government and military came under
\href{https://www.nytimes.com/2020/04/23/world/asia/pakistan-coronavirus-ramadan.html}{immense
pressure from Pakistan's powerful Islamists} to loosen the lockdown
during Ramadan, the holy month of fasting that started in April and
ended last month. After just a few weeks, the lockdown was lifted.

``We are a low middle-income country, with two-thirds of the population
dependent on daily incomes,'' Dr. Zafar Mirza, the de facto health
minister, said Wednesday.

``We have to make tough policy choices to strike a balance between lives
and livelihoods.''

Advertisement

\protect\hyperlink{after-bottom}{Continue reading the main story}

\hypertarget{site-index}{%
\subsection{Site Index}\label{site-index}}

\hypertarget{site-information-navigation}{%
\subsection{Site Information
Navigation}\label{site-information-navigation}}

\begin{itemize}
\tightlist
\item
  \href{https://help.nytimes.com/hc/en-us/articles/115014792127-Copyright-notice}{©~2020~The
  New York Times Company}
\end{itemize}

\begin{itemize}
\tightlist
\item
  \href{https://www.nytco.com/}{NYTCo}
\item
  \href{https://help.nytimes.com/hc/en-us/articles/115015385887-Contact-Us}{Contact
  Us}
\item
  \href{https://www.nytco.com/careers/}{Work with us}
\item
  \href{https://nytmediakit.com/}{Advertise}
\item
  \href{http://www.tbrandstudio.com/}{T Brand Studio}
\item
  \href{https://www.nytimes.com/privacy/cookie-policy\#how-do-i-manage-trackers}{Your
  Ad Choices}
\item
  \href{https://www.nytimes.com/privacy}{Privacy}
\item
  \href{https://help.nytimes.com/hc/en-us/articles/115014893428-Terms-of-service}{Terms
  of Service}
\item
  \href{https://help.nytimes.com/hc/en-us/articles/115014893968-Terms-of-sale}{Terms
  of Sale}
\item
  \href{https://spiderbites.nytimes.com}{Site Map}
\item
  \href{https://help.nytimes.com/hc/en-us}{Help}
\item
  \href{https://www.nytimes.com/subscription?campaignId=37WXW}{Subscriptions}
\end{itemize}
