Sections

SEARCH

\protect\hyperlink{site-content}{Skip to
content}\protect\hyperlink{site-index}{Skip to site index}

\href{https://www.nytimes.com/section/parenting}{Parenting}

\href{https://myaccount.nytimes.com/auth/login?response_type=cookie\&client_id=vi}{}

\href{https://www.nytimes.com/section/todayspaper}{Today's Paper}

\href{/section/parenting}{Parenting}\textbar{}When Impulse Buys Make You
Feel Safe

\url{https://nyti.ms/3e5LOQS}

\begin{itemize}
\item
\item
\item
\item
\item
\end{itemize}

\href{https://www.nytimes.com/spotlight/at-home?action=click\&pgtype=Article\&state=default\&region=TOP_BANNER\&context=at_home_menu}{At
Home}

\begin{itemize}
\tightlist
\item
  \href{https://www.nytimes.com/2020/08/03/well/family/the-benefits-of-talking-to-strangers.html?action=click\&pgtype=Article\&state=default\&region=TOP_BANNER\&context=at_home_menu}{Talk:
  To Strangers}
\item
  \href{https://www.nytimes.com/2020/08/01/at-home/coronavirus-make-pizza-on-a-grill.html?action=click\&pgtype=Article\&state=default\&region=TOP_BANNER\&context=at_home_menu}{Make:
  Grilled Pizza}
\item
  \href{https://www.nytimes.com/2020/07/31/arts/television/goldbergs-abc-stream.html?action=click\&pgtype=Article\&state=default\&region=TOP_BANNER\&context=at_home_menu}{Watch:
  'The Goldbergs'}
\item
  \href{https://www.nytimes.com/interactive/2020/at-home/even-more-reporters-editors-diaries-lists-recommendations.html?action=click\&pgtype=Article\&state=default\&region=TOP_BANNER\&context=at_home_menu}{Explore:
  Reporters' Google Docs}
\end{itemize}

Advertisement

\protect\hyperlink{after-top}{Continue reading the main story}

Supported by

\protect\hyperlink{after-sponsor}{Continue reading the main story}

\hypertarget{when-impulse-buys-make-you-feel-safe}{%
\section{When Impulse Buys Make You Feel
Safe}\label{when-impulse-buys-make-you-feel-safe}}

A toddler-sized vacuum can't fix the world. But it can make my kid
smile, and help soothe my uncertainty.

\href{https://www.nytimes.com/by/kaitlyn-greenidge}{\includegraphics{https://static01.nyt.com/images/2018/04/05/opinion/kaitlyn-greenidge/kaitlyn-greenidge-thumbLarge-v2.jpg}}

By \href{https://www.nytimes.com/by/kaitlyn-greenidge}{Kaitlyn
Greenidge}

\begin{itemize}
\item
  June 17, 2020
\item
  \begin{itemize}
  \item
  \item
  \item
  \item
  \item
  \end{itemize}
\end{itemize}

\includegraphics{https://static01.nyt.com/images/2020/06/17/multimedia/17parenting-NL-vacuum/17parenting-NL-vacuum-articleLarge.jpg?quality=75\&auto=webp\&disable=upscale}

\emph{I'm taking a break this week, so I asked Kaitlyn Greenidge,}
\href{https://www.nytimes.com/2020/04/16/parenting/baby/work-conference-life-balance.html}{\emph{an
NYT Parenting contributor}} \emph{and the author of
``}\href{https://www.workman.com/products/we-love-you-charlie-freeman}{\emph{We
Love You, Charlie Freeman,}}\emph{''} \emph{to step in for me. Read her
previous newsletter, about narrating the world for her
daughter,}\href{https://www.nytimes.com/2020/02/26/parenting/making-awkward-small-talk-with-my-baby.html}{\emph{here}}\emph{.
--- Jessica Grose, lead editor, NYT Parenting}

I bought the toddler-sized vacuum cleaner at 3 a.m. in early June. I
felt slightly giddy when I pressed the button.

I'd just spent the past four hours scrolling Twitter, watching as police
officers
\href{https://www.nytimes.com/2020/06/05/us/police-violence-george-floyd.html}{injured
protesters}, reading the vitriol trolls spew, stopping every so often
for the more beautiful images ---
\href{https://www.newsweek.com/black-texas-cowboys-horseback-protest-george-floyds-death-viral-video-1508378}{the
black cowboys in Texas} and the ballroom dancers doing death drops in
the middle of a march and the Amish carrying Black Lives Matter signs.

I'd drunk in all the chaos, and I was jittery and sad and scared. My
daughter was asleep beside me, and everyone in the house was asleep,
too. I had no one to talk to about any of it at that moment. So I bought
the toy vacuum cleaner for a little release.

I knew I shouldn't do it. I knew consuming a child's hard-plastic toy
that is probably going to end up at the bottom of the ocean in 15 years
was a terrible response to all of those feelings. But it was an impulse
that has been irresistible to me in these months of uncertainty.

Since March, so many packages have come to the house in Massachusetts,
where my daughter and I are quarantining with my sisters, nieces,
brother-in-law and mother. My mom ordered something from Amazon nearly
every day. My sister did, too. One of my nieces only emerged from her
room for the mail check. She is just 11, but was engaged in a
long-running, cat-and-mouse game with an off-brand earbud website. Every
few days, the company sent her non-Apple earbuds that didn't work, and
every few days she sent them back and requested a replacement. The
company was not aware that they were playing this game with a
sixth-grader who had infinite patience and still trusted that those in
power would do the right thing.

Purchasing nonessentials is always fraught for me. I grew up poor, when
the miscalculation of overspending by \$20 could mean the lights were
out for a week or the car was repossessed.

When you are poor, everyone has advice on what you can do to not be
poor, but weirdly, none of it ever comes around to ``your employer
should pay you a living wage.'' Instead, there are many people who wish
to tell you that if you just thought better about how to spend that
\$20, it wouldn't matter if you were chronically underpaid.

So, as an adult, even small purchases can cause a panic attack. When my
daughter was born, I was between regular paying gigs. I remember sobbing
as I bought a smoothie at our local juice bar when my daughter was a few
weeks old. I was one month away from a recurring paycheck with a
comfortable amount of savings in the bank, but I was certain that that
\$6 would send my family into financial ruin.

And for a smoothie! What a cliché of a millennial parent I would be. I
wouldn't be able to live the embarrassment down.

I had hoped adulthood, relative financial stability and parenthood would
cure me of this anxiety. I did not want to pass it on to my daughter or
have her live in the tense atmosphere of it.

But then quarantine and protests and all of a sudden it felt like my
anxiety around purchases was justified. I have never bought more things
on a whim than during this time: baby-sized tool kits, baby-sized
musical instruments and so many novelty onesies.

It's about control, of course. Life feels normal when I remind myself I
can still buy things that will make my daughter laugh or things that
will make her look cute. I can't say what our life will look like next
year at this time, whether
\href{https://www.nytimes.com/2020/05/09/business/economy/coronavirus-unemployment.html}{the
record unemployment rates} will come for our family. I \emph{can} say
that a toy truck will make her happy today.

The craziest thing we've bought during this spending frenzy is a pool.
Not a big one. It is only 3 feet deep and 10 feet long. It happened
because my sister and I were talking about what we would do with our
kids during this Covid summer, when the Y was closed and we feared the
beaches might be closed, too.

In general, our quarantine house is a surprisingly harmonious set-up,
but even our close family bonds would be stretched to the limit on the
first hot, muggy day of summer. A pool, then, my sister suggested.

``Absolutely not,'' I said. ``The property values. The housing
insurance. It's not worth it.''

``You're right,'' my sister said. Then she and my mom bought the
above-ground pool when I left the room to feed my daughter.

``It was only \$700,'' my sister said. ``If the adults split the cost,
it's not that much.''

I could feel the old wave of money anxiety coming, countered by this new
wave of uncertainty for the future. I thought of the first hot day
together. I imagined my daughter, who runs hot and always feels sweaty
even on a 60-degree day, clinging to me, and the only relief being an
electric fan.

``It will be OK,'' my sister said.

I spent the next night searching for pool floats. A sloth-shaped one
will ship to me in two weeks, I am told.

\emph{P.S.}
\href{https://www.nytimes.com/spotlight/parenting-kids-coronavirus}{\emph{Click
here to read all NYT Parenting coverage on coronavirus}}\emph{. Follow
us on Instagram}
\href{https://nl.nytimes.com/f/a/KbjXDcX6H0j8UEI7pPcixg~~/AAAAAQA~/RgRfwNx_P0TIaHR0cHM6Ly93d3cuaW5zdGFncmFtLmNvbS9ueXRwYXJlbnRpbmcvP3RlPTEmbmw9bnl0LXBhcmVudGluZyZlbWM9ZWRpdF9wdGdfMjAxOTExMjc_Y2FtcGFpZ25faWQ9MTE4Jmluc3RhbmNlX2lkPTE0MTI0JnNlZ21lbnRfaWQ9MTkxMzImdXNlcl9pZD04NWJmMjRhZjE2MTk0YTlkZjgxMGQ3OTZhMzU4NmVlZCZyZWdpX2lkPTg5OTQ5NDMxMjAxOTExMjdXA255dEIKABt_V95d7BxQnFIbbWVsb255Y2UubWNhZmVlQG55dGltZXMuY29tWAQAAAAA}{\emph{@NYTParenting}}\emph{.
Join}
\href{https://nl.nytimes.com/f/a/vsa5Ga3bcTyDwVsY509i_w~~/AAAAAQA~/RgRfwNx_P0THaHR0cHM6Ly93d3cuZmFjZWJvb2suY29tL255dHBhcmVudGluZy8_dGU9MSZubD1ueXQtcGFyZW50aW5nJmVtYz1lZGl0X3B0Z18yMDE5MTEyNz9jYW1wYWlnbl9pZD0xMTgmaW5zdGFuY2VfaWQ9MTQxMjQmc2VnbWVudF9pZD0xOTEzMiZ1c2VyX2lkPTg1YmYyNGFmMTYxOTRhOWRmODEwZDc5NmEzNTg2ZWVkJnJlZ2lfaWQ9ODk5NDk0MzEyMDE5MTEyN1cDbnl0QgoAG39X3l3sHFCcUhttZWxvbnljZS5tY2FmZWVAbnl0aW1lcy5jb21YBAAAAAA~}{\emph{us
on Facebook}}\emph{. Find}
\href{https://nl.nytimes.com/f/a/swlUZzCVQVD_-CDgazvCtw~~/AAAAAQA~/RgRfwNx_P0TCaHR0cHM6Ly90d2l0dGVyLmNvbS9ueXRwYXJlbnRpbmcvP3RlPTEmbmw9bnl0LXBhcmVudGluZyZlbWM9ZWRpdF9wdGdfMjAxOTExMjc_Y2FtcGFpZ25faWQ9MTE4Jmluc3RhbmNlX2lkPTE0MTI0JnNlZ21lbnRfaWQ9MTkxMzImdXNlcl9pZD04NWJmMjRhZjE2MTk0YTlkZjgxMGQ3OTZhMzU4NmVlZCZyZWdpX2lkPTg5OTQ5NDMxMjAxOTExMjdXA255dEIKABt_V95d7BxQnFIbbWVsb255Y2UubWNhZmVlQG55dGltZXMuY29tWAQAAAAA}{\emph{us
on Twitter}} \emph{for the latest updates. Read last week's newsletter,
about how to manage multigenerational living here.}

\begin{center}\rule{0.5\linewidth}{\linethickness}\end{center}

\hypertarget{want-more-on-parenting-and-money}{%
\subsection{Want More on Parenting and
Money?}\label{want-more-on-parenting-and-money}}

\begin{itemize}
\item
  \href{https://www.nytimes.com/spotlight/price-of-modern-parenting}{Modern
  parenting is expensive}, and parents who are in the ``sandwich
  generation'' caring for young kids and aging relatives at the same
  time,
  \href{https://www.nytimes.com/2020/02/11/parenting/sandwich-generation-costs.html}{feel
  the pain acutely}.
\item
  In August, former NYT Parenting editor Katherine Zoepf wrote about
  renting
  \href{https://www.nytimes.com/2019/08/27/parenting/parents-money-stress.html}{out
  a room in her apartment to make ends meet}.
\item
  In February, NYT Parenting reporter Christina Caron wrote about a
  family with
  \href{https://www.nytimes.com/2020/02/11/parenting/nicu-costs.html}{a
  \$4 million NICU bill}.
\end{itemize}

\begin{center}\rule{0.5\linewidth}{\linethickness}\end{center}

\hypertarget{tiny-victories}{%
\subsection{Tiny Victories}\label{tiny-victories}}

\emph{Parenting can be a grind. Let's celebrate the tiny victories.}

\begin{quote}
Instead of going through the usual bedtime struggle, one night I decided
to only say the word ``pajamas.'' After seven repetitions, my toddler
said, ``OK, mama. I give up.'' And she put on her pajamas. \emph{---
Rebecca Van Sickle, Portland, Ore.}
\end{quote}

\begin{center}\rule{0.5\linewidth}{\linethickness}\end{center}

\emph{If you want a chance to get your Tiny Victory published, find us
on Instagram}
\href{https://www.instagram.com/nytparenting/}{\emph{@NYTparenting}}
\emph{and use the hashtag \#tinyvictories;}
\href{mailto:parenting_submissions@nytimes.com?subject=Tiny\%20Victories}{\emph{email
us}}\emph{; or enter your}
\href{https://www.nytimes.com/2019/03/19/reader-center/parenting-section-tiny-victories.html?module=inline}{\emph{Tiny
Victory at the bottom of this page}}\emph{. Include your full name and
location. Tiny Victories may be edited for clarity and style.}
\emph{Your name, location and comments may be published, but your
contact information will not. By submitting to us, you agree that you
have read, understand and accept the}
\href{https://nyti.ms/2Q9M7i0}{\emph{Reader Submission Terms}} \emph{in
relation to all of the content and other information you send to us.}

Advertisement

\protect\hyperlink{after-bottom}{Continue reading the main story}

\hypertarget{site-index}{%
\subsection{Site Index}\label{site-index}}

\hypertarget{site-information-navigation}{%
\subsection{Site Information
Navigation}\label{site-information-navigation}}

\begin{itemize}
\tightlist
\item
  \href{https://help.nytimes.com/hc/en-us/articles/115014792127-Copyright-notice}{©~2020~The
  New York Times Company}
\end{itemize}

\begin{itemize}
\tightlist
\item
  \href{https://www.nytco.com/}{NYTCo}
\item
  \href{https://help.nytimes.com/hc/en-us/articles/115015385887-Contact-Us}{Contact
  Us}
\item
  \href{https://www.nytco.com/careers/}{Work with us}
\item
  \href{https://nytmediakit.com/}{Advertise}
\item
  \href{http://www.tbrandstudio.com/}{T Brand Studio}
\item
  \href{https://www.nytimes.com/privacy/cookie-policy\#how-do-i-manage-trackers}{Your
  Ad Choices}
\item
  \href{https://www.nytimes.com/privacy}{Privacy}
\item
  \href{https://help.nytimes.com/hc/en-us/articles/115014893428-Terms-of-service}{Terms
  of Service}
\item
  \href{https://help.nytimes.com/hc/en-us/articles/115014893968-Terms-of-sale}{Terms
  of Sale}
\item
  \href{https://spiderbites.nytimes.com}{Site Map}
\item
  \href{https://help.nytimes.com/hc/en-us}{Help}
\item
  \href{https://www.nytimes.com/subscription?campaignId=37WXW}{Subscriptions}
\end{itemize}
