Sections

SEARCH

\protect\hyperlink{site-content}{Skip to
content}\protect\hyperlink{site-index}{Skip to site index}

\href{https://www.nytimes.com/section/us}{U.S.}

\href{https://myaccount.nytimes.com/auth/login?response_type=cookie\&client_id=vi}{}

\href{https://www.nytimes.com/section/todayspaper}{Today's Paper}

\href{/section/us}{U.S.}\textbar{}7 Lessons (and Warnings) From Those
Who Marched With Dr. King

\url{https://nyti.ms/3hxXRbQ}

\begin{itemize}
\item
\item
\item
\item
\item
\end{itemize}

\href{https://www.nytimes.com/news-event/george-floyd-protests-minneapolis-new-york-los-angeles?action=click\&pgtype=Article\&state=default\&region=TOP_BANNER\&context=storylines_menu}{Race
and America}

\begin{itemize}
\tightlist
\item
  \href{https://www.nytimes.com/2020/07/26/us/protests-portland-seattle-trump.html?action=click\&pgtype=Article\&state=default\&region=TOP_BANNER\&context=storylines_menu}{Protesters
  Return to Other Cities}
\item
  \href{https://www.nytimes.com/2020/07/24/us/portland-oregon-protests-white-race.html?action=click\&pgtype=Article\&state=default\&region=TOP_BANNER\&context=storylines_menu}{Portland
  at the Center}
\item
  \href{https://www.nytimes.com/2020/07/23/podcasts/the-daily/portland-protests.html?action=click\&pgtype=Article\&state=default\&region=TOP_BANNER\&context=storylines_menu}{Podcast:
  Showdown in Portland}
\item
  \href{https://www.nytimes.com/interactive/2020/07/16/us/black-lives-matter-protests-louisville-breonna-taylor.html?action=click\&pgtype=Article\&state=default\&region=TOP_BANNER\&context=storylines_menu}{45
  Days in Louisville}
\end{itemize}

Advertisement

\protect\hyperlink{after-top}{Continue reading the main story}

Supported by

\protect\hyperlink{after-sponsor}{Continue reading the main story}

\hypertarget{7-lessons-and-warnings-from-those-who-marched-with-dr-king}{%
\section{7 Lessons (and Warnings) From Those Who Marched With Dr.
King}\label{7-lessons-and-warnings-from-those-who-marched-with-dr-king}}

The tumult and passion of the past weeks have left the surviving
veterans of the civil rights era with trepidation and hope.

\includegraphics{https://static01.nyt.com/images/2020/06/19/us/19UNREST-CIVILRIGHTS-p1/merlin_173140626_209a9e78-205d-44f5-b187-c5dafffb0b74-articleLarge.jpg?quality=75\&auto=webp\&disable=upscale}

\href{https://www.nytimes.com/by/ellen-barry}{\includegraphics{https://static01.nyt.com/images/2018/10/08/multimedia/author-ellen-barry/author-ellen-barry-thumbLarge.png}}

By \href{https://www.nytimes.com/by/ellen-barry}{Ellen Barry}

\begin{itemize}
\item
  Published June 17, 2020Updated June 19, 2020
\item
  \begin{itemize}
  \item
  \item
  \item
  \item
  \item
  \end{itemize}
\end{itemize}

Throughout the past several weeks, as
\href{https://www.nytimes.com/news-event/george-floyd-protests-minneapolis-new-york-los-angeles}{protests
over the killing} of
\href{https://www.nytimes.com/article/george-floyd-who-is.html}{George
Floyd} rippled through America's cities, a 79-year-old retired
schoolteacher has spent her days watching the news in her home in
Albany, Ga., sometimes with tears running down her face.

For Rutha Mae Harris, who once marched and was jailed with the Rev. Dr.
Martin Luther King Jr., it is like revisiting her past.

There have been times when she wondered what her generation had
achieved. But the past weeks --- particularly the sight of kneeling
police officers and throngs of white faces --- have offered some
redemption.

``I love it, I love it, I love it,'' she said. ``It has surprised me,
and it gives me hope. I thought what I had done was in vain.''

For the dwindling cadre of civil rights activists like Ms. Harris who
took to the streets 60 years ago, this is a moment of trepidation and
wonder.

Their activism gave the world images --- the
\href{https://www.nytimes.com/2013/01/21/opinion/good-and-evil-in-birmingham.html?searchResultPosition=4}{snarling
police dogs}of Birmingham, Ala., the
\href{https://www.nytimes.com/video/multimedia/100000003555951/for-selma-marcher-memories-of-chaos-still-fresh-50-years-on.html?searchResultPosition=6}{beatings
of Selma, Ala.} --- that changed the trajectory of race in America. Now
they are watching another movement unfold, familiar but utterly changed.

Dr. King surrounded himself with a variety of thinkers, and in recent
weeks, his allies took different views of the Floyd protests.

But they all marveled at their quicksilver spread. In their time, major
actions were the result of months of planning, punctuated by all-night
arguments over strategy and phone-tree lobbying to get reporters to show
up. Five years passed between
\href{https://www.nytimes.com/2019/02/21/us/remembering-emmett-till-legacy-virtual-reality.html}{Emmett
Till's lynching}and the
\href{https://learning.blogs.nytimes.com/2012/02/01/feb-1-1960-black-students-and-the-greensboro-sit-in/?searchResultPosition=2}{Greensboro,
N.C., sit-ins.} Another year passed between
\href{http://www.library.ncat.edu/resources/archives/four.html\#section-4}{the
sit-ins} and the
\href{https://www.nytimes.com/2011/05/20/opinion/20Lafayette.html}{Freedom
Rides}.

``A movement is different from a demonstration,'' said Taylor Branch, a
\href{https://www.nytimes.com/2006/02/05/books/review/the-whirlwinds-of-revolt.html}{historian
of the civil rights era}.

``It's not automatic --- it's the opposite of automatic,'' he said,
``that a demonstration in the street is going to lead to a movement that
engages enough people, and has a clear enough goal that it has a chance
to become institutionalized, like the Voting Rights Act.''

Dr. King's confidant
\href{https://kinginstitute.stanford.edu/encyclopedia/lafayette-bernard}{Bernard
Lafayette}, 79, could not contain his excitement about recent
demonstrations; he has been offering advice to young activists from his
home in Tuskegee, Ala. Andrew Young, 88, a former mayor of Atlanta, has
\href{https://www.ajc.com/news/opinion/opinion-lives-must-matter-most-serious-time-for-all/7iTWmC6UxZ5s53voGaE1eO/}{vented
his frustration} over
\href{https://www.wsbtv.com/video/local-video/im-thinking-i-want-cry-andrew-young-reacts-violent-protests-atlanta/7BWDE4WNLNAVM35X5UKOCZSUQQ/}{looting
and vandalism}. And
\href{https://www.nytimes.com/2001/01/07/education/algebra-project-bob-moses-empowers-students.html?searchResultPosition=1}{Bob
Moses}, 85, was cautious in his comments, saying the country seemed to
be undergoing an ``awakening.''

``I think that's been its main impact, a kind of revelation about
something that has been going on for over a century, a century and a
half, right under your noses,'' Mr. Moses said. ``But there isn't any
indication of how to fix it.''

\includegraphics{https://static01.nyt.com/images/2020/06/19/us/00UNREST-CIVILRIGHTS-mlkmonument/merlin_173295489_e6ca8426-c24a-4a5e-a06e-f64225c02b80-articleLarge.jpg?quality=75\&auto=webp\&disable=upscale}

Here are some excerpts from those conversations, edited for length and
clarity.

\hypertarget{when-a-police-officer-kneels-with-protesters-pay-attention}{%
\subsubsection{When a police officer kneels with protesters, pay
attention.}\label{when-a-police-officer-kneels-with-protesters-pay-attention}}

\emph{Rutha Mae Harris, 79, was one of the Freedom Singers who toured
the South encouraging black people to register to vote. She has spent
the past week at her home in Albany, Ga., ``glued to MSNBC,'' she said.}

What we did, you know, we started singing. Sometimes the singing worked,
and sometimes it didn't. The marches I was on, we started singing, and
the policemen would drop their billy clubs, and we knew they were no
longer planning to hit us. I am a witness of that.

And I have seen this day, this day in time, policemen walking with the
protesters, hand in hand with the protesters. I was so happy to see
that. We had a little protest here in Georgia, and our police chief was
part of the march. You know, back then, the police chief at that time
was Chief Pritchett. He's the one who arrested all of us, and, of
course, he arrested Martin Luther King.

What we had, it was not equivalent. When you see the cops kneeling, I
just love that. And there are a lot of young white people. I've never
seen that. We had some white people, but not as many. It is a surprise,
and it gives me hope.

\hypertarget{dont-assume-this-moment-will-last}{%
\subsubsection{Don't assume this moment will
last.}\label{dont-assume-this-moment-will-last}}

\emph{Bob Moses, 85, an educator who in the 1960s led a}
\href{https://timesmachine.nytimes.com/timesmachine/1964/11/25/118690271.html?pageNumber=38}{\emph{drive
to register black voters}} **
\href{https://www.nytimes.com/1993/02/21/magazine/mississippi-learning.html?searchResultPosition=4}{\emph{in
Mississippi}}\emph{, has watched the protests from an apartment in
Hollywood, Fla. He said he was moved by a}
\href{https://twitter.com/g0ldie_teee/status/1266929382708465665}{\emph{viral
video clip}} \emph{of three black men from different generations ---
including a 45-year-old and a 16-year-old --- in a shouting match at a
protest in North Carolina, arguing with raw emotion about whether
violence was an appropriate response to systemic racism.}

It's like an awakening: We're trapped. He was trapped, he's 45. You're
trapped, you're just 16. What we've been doing isn't working. What are
we going to do? That level of consciousness really is new. And it's not
just the broader white population that is waking up to some extent, but
also within the African-American population, too.

It may be that the person who killed George Floyd was an aberration. But
the system they were a part of, that protects them and is as American as
apple pie. So waking up to that --- it's not clear whether the country
is capable of waking up to that to its full extent.

\emph{Unlike Ms. Harris, he was skeptical that gestures of solidarity
from the police were meaningful.}

You are talking to an individual policeman in the street, you want him
to express empathy about what is happening, but behind the scenes you
have high politics. The system works to protect the people who are
involved in all of this at different levels, not just the guy who pulls
the trigger and puts the knee on the throat.

It's catharsis for the person asking and for any policeman that
responds. It's what the country has always wanted, to try to solve the
problem at the level of the individual. This individual you know directs
his or her behavior or tones, and the system just keeps rolling on and
producing more atrocities.

It is revelatory that the pressure now is coming from within. It's been
sparked by this one event, but the event really has opened up a
crevasse, so to speak, through which all this history is pouring, like
the Mississippi River onto the Delta. It's pouring into all the streams
of TV, cable news, social media. So that is quite different. And the
question is, can the country handle it?

We don't know. I certainly don't know, at this moment, which way the
country might flip. It can lurch backward as quickly as it can lurch
forward.

Image

Protesters and members of the clergy held hands to form a barrier
between law enforcement standing on Interstate 70 and the larger group
of protesters earlier this month, in the St. Louis suburb of St.
Charles, Mo.~Credit...Whitney Curtis for The New York Times

\hypertarget{white-people-are-now-experiencing-police-violence-firsthand}{%
\subsubsection{White people are now experiencing police violence
firsthand.}\label{white-people-are-now-experiencing-police-violence-firsthand}}

\href{https://www.nytimes.com/1988/07/24/weekinreview/the-nation-in-chicago-outsiders-of-1968-are-insiders-now.html?searchResultPosition=10}{\emph{Don
Rose}}\emph{, 89, a white man who}
\href{https://www.thenation.com/article/archive/don-rose-sixty-years-chicagos-warrior-justice/}{\emph{served
as Dr. King's press secretary in Chicago}}\emph{, and went on to
mobilize protests against the Vietnam War, was exhilarated by the George
Floyd demonstrations. He said video clips and the ability of the
internet to spread messages had pulled white people into the current
movement.}

I wish we had had that. I keep marveling at how wonderful it would have
been, rather than using mimeograph machines.

In those days, when we spoke of police brutality, we weren't often
believed. I often pointed to the behavior of the police in Chicago in
1968 --- that was really the thing that showed a lot of people that
police brutality was a real thing. That was white people's lesson for
what black people had undergone in their own communities.

\emph{He reflected on the violence and looting at some recent protests.}

Of course, violence is very disheartening and fearsome. But the polling
and the reactions of people all around suggests that they certainly
understand what was going on. Obviously no one was supporting the
violence and opportunistic looting. I don't know if it is understood or
forgiven, but it has apparently not caused a white backlash.

The fact that more whites are participating in these marches all over
the country is evidence that over the years, more and more has been
heard. The messages are getting across.

\hypertarget{dont-write-off-anyone-as-an-enemy-persuade-them}{%
\subsubsection{Don't write off anyone as an enemy. Persuade
them.}\label{dont-write-off-anyone-as-an-enemy-persuade-them}}

\emph{Andrew Young, 88, a former mayor of Atlanta and ambassador to the
United Nations, called the wave of protests ``a phenomenal moment,'' but
said they cried out for organization and structure.}

What the difference is, is social media. Not only did we not have social
media, we hardly had phones. That was a blessing, in many ways, because
it took us three or four months in Birmingham to organize. It gave us
time to define what we really thought would work, and how to go about
it. We knew what we wanted. We knew what victory was. That's the only
thing I'm concerned about.

\emph{He offered sharp criticism when initial protests in Atlanta led to
looting and violence.}

I was upset because there were no marshals that were keeping order. We
always made sure, in the organizing community, we tried to keep people
who did not adhere to our values and vision, we asked them to stay out.

\emph{He described a march in St. Augustine, Fla., in 1964, when Ku Klux
Klan members had been deputized by the sheriff to disperse the crowd.}

I didn't know who they were, but I just feel like I can talk to anybody,
so I went over there to try to explain to them why we were marching.
They were shocked that I went up there by myself. It just didn't make
sense, to me, to beat up women and children who only wanted to get the
right to get a hot dog at the lunch counter. So I picked the leaders,
and I was doing a pretty good job of talking to them when someone came
up behind me and hit me with something. I got stomped a little while,
and somebody came up and pulled me up and across the street.

What we were demonstrating was the power of nonviolence. The reason I
had to talk to them is that you don't write people off as the enemy. I
didn't get arrested very much, I usually talked my way through it. When
you enter a confrontation, it is with an intention to move to
reconciliation.

Image

People held up their phones as a man sang during a march in Washington
this month. ``Not only did we not have social media, we hardly had
phones,'' Andrew Young, a former mayor of Atlanta, said of protests
during the civil rights era.Credit...Erin Schaff/The New York Times

\hypertarget{you-may-be-disappointed-we-were}{%
\subsubsection{You may be disappointed. We
were.}\label{you-may-be-disappointed-we-were}}

\emph{Fred Gray, 89, who defended Rosa Parks against charges of
disorderly conduct, still goes to his law office in Tuskegee, Ala.,
every day. He said it was discouraging to see young people fight the
same battle as he and his contemporaries did.}

The same problems we tried to resolve, they have not been solved. I
think that what the Constitution requires, we're still a long way away
from solving the problems we need to have solved. That needs to start
from the top and come all the way down.

What I tried to do was protect and assist people obtaining their
constitutional rights. That's what I tried to do for 65 years. I was not
one of those people who tried to do all of it. My role was to deal with
the legal aspect of it.

We didn't solve it. Several generations later, we have to deal with the
same troubles of racism. I was hopeful 60 years ago that we would solve
them. I've been disappointed so often.

I'm disappointed by the fact that I thought the white power structure,
once they saw what black Americans were capable of, that they could
perform equally. I thought it would change their hearts, but I don't
think the hearts and minds of many people have changed.

\hypertarget{maybe-young-people-now-have-the-urgency-we-had-then}{%
\subsubsection{\texorpdfstring{\textbf{Maybe young people now have the
urgency we had
then.}}{Maybe young people now have the urgency we had then.}}\label{maybe-young-people-now-have-the-urgency-we-had-then}}

\emph{Xernona Clayton, 89, who helped organize marches for Dr. King, has
been monitoring the protests so raptly from her home in Atlanta that, at
times, she has switched on two televisions to follow local and national
news. She was deeply dismayed by the initial outbreak of violence, but
has since been reassured.}

I'm hoping --- I'm a positive thinker --- I believe this day will create
the change we all want.

You can't just hurt people and kill people and wipe out businesses. It's
frightening, you see burning and looting. That's frightening. It scares
some people. But you have to recognize, if change is going to come,
there is pain and suffering, sometimes, that goes with that.

I used to criticize the young people. I thought maybe we, the older
people, had solved the biggest problems --- you got equal treatment,
employment opportunities, civil rights laws, you don't have to drink
from the other fountain. We have made those major changes. I said,
``Maybe we solved their problems, and they don't got the urgency.''

Well, now they got the urgency. Now I think the young people are really
bringing the problem to the fore. They got everybody's attention.

Image

A protester in New York waved an American flag with George Floyd's
plea.Credit...Demetrius Freeman for The New York Times

\hypertarget{organize-organize-organize-and-whatever-it-takes-vote}{%
\subsubsection{Organize, organize, organize. (And, whatever it takes,
vote.)}\label{organize-organize-organize-and-whatever-it-takes-vote}}

\emph{Bernard Lafayette, 79, who, like Mr. Young, accompanied Dr. King
on the 1968 trip to Memphis where he was assassinated, has spent recent
years training young activists in nonviolent social change. He traveled
to}
\href{https://www.nytimes.com/interactive/2014/08/13/us/ferguson-missouri-town-under-siege-after-police-shooting.html}{\emph{Ferguson,
Mo.,}} \emph{to advise protest leaders there, and has spent the past
weeks fielding phone calls from young organizers.}

Oh, I'm very hopeful, but also excited, because I see some very
strategic things happening. The only thing we have to be concerned about
is the sustainability.

I am more or less thinking about strategy, and that's where I'm turning
my energy. They call me on the phone all the time. I get 15 to 20 calls
a day. I answer their questions. Mainly they need training. They need to
build coalitions. I prepare folks to take different roles in the
movement. You can't do everything. People have different roles.

Now what I'm looking for is leadership among the young people. I'm
looking for a new Student Nonviolent Coordinating Committee. The next
thing that we need if we're going to have a movement that is going to
sustain itself --- we need music, OK? Once you get those artists singing
songs about change and the movement, that helps to stimulate people and
bring them together. There is nothing like music to bring people
together.

The other most, most important thing, you got to get people who are
ready to register to vote. You have got to have people in power who
represent you. You've got to be negotiating and talking to the people
who will make decisions. You can't just put it out there and be
screaming in the air. The air can't make the change.

Advertisement

\protect\hyperlink{after-bottom}{Continue reading the main story}

\hypertarget{site-index}{%
\subsection{Site Index}\label{site-index}}

\hypertarget{site-information-navigation}{%
\subsection{Site Information
Navigation}\label{site-information-navigation}}

\begin{itemize}
\tightlist
\item
  \href{https://help.nytimes.com/hc/en-us/articles/115014792127-Copyright-notice}{©~2020~The
  New York Times Company}
\end{itemize}

\begin{itemize}
\tightlist
\item
  \href{https://www.nytco.com/}{NYTCo}
\item
  \href{https://help.nytimes.com/hc/en-us/articles/115015385887-Contact-Us}{Contact
  Us}
\item
  \href{https://www.nytco.com/careers/}{Work with us}
\item
  \href{https://nytmediakit.com/}{Advertise}
\item
  \href{http://www.tbrandstudio.com/}{T Brand Studio}
\item
  \href{https://www.nytimes.com/privacy/cookie-policy\#how-do-i-manage-trackers}{Your
  Ad Choices}
\item
  \href{https://www.nytimes.com/privacy}{Privacy}
\item
  \href{https://help.nytimes.com/hc/en-us/articles/115014893428-Terms-of-service}{Terms
  of Service}
\item
  \href{https://help.nytimes.com/hc/en-us/articles/115014893968-Terms-of-sale}{Terms
  of Sale}
\item
  \href{https://spiderbites.nytimes.com}{Site Map}
\item
  \href{https://help.nytimes.com/hc/en-us}{Help}
\item
  \href{https://www.nytimes.com/subscription?campaignId=37WXW}{Subscriptions}
\end{itemize}
