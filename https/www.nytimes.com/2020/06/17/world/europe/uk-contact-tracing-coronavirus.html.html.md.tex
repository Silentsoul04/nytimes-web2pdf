Sections

SEARCH

\protect\hyperlink{site-content}{Skip to
content}\protect\hyperlink{site-index}{Skip to site index}

\href{https://www.nytimes.com/section/world/europe}{Europe}

\href{https://myaccount.nytimes.com/auth/login?response_type=cookie\&client_id=vi}{}

\href{https://www.nytimes.com/section/todayspaper}{Today's Paper}

\href{/section/world/europe}{Europe}\textbar{}England's `World Beating'
System to Track the Virus Is Anything But

\url{https://nyti.ms/2Bj89f8}

\begin{itemize}
\item
\item
\item
\item
\item
\item
\end{itemize}

\href{https://www.nytimes.com/news-event/coronavirus?action=click\&pgtype=Article\&state=default\&region=TOP_BANNER\&context=storylines_menu}{The
Coronavirus Outbreak}

\begin{itemize}
\tightlist
\item
  live\href{https://www.nytimes.com/2020/08/01/world/coronavirus-covid-19.html?action=click\&pgtype=Article\&state=default\&region=TOP_BANNER\&context=storylines_menu}{Latest
  Updates}
\item
  \href{https://www.nytimes.com/interactive/2020/us/coronavirus-us-cases.html?action=click\&pgtype=Article\&state=default\&region=TOP_BANNER\&context=storylines_menu}{Maps
  and Cases}
\item
  \href{https://www.nytimes.com/interactive/2020/science/coronavirus-vaccine-tracker.html?action=click\&pgtype=Article\&state=default\&region=TOP_BANNER\&context=storylines_menu}{Vaccine
  Tracker}
\item
  \href{https://www.nytimes.com/interactive/2020/07/29/us/schools-reopening-coronavirus.html?action=click\&pgtype=Article\&state=default\&region=TOP_BANNER\&context=storylines_menu}{What
  School May Look Like}
\item
  \href{https://www.nytimes.com/live/2020/07/31/business/stock-market-today-coronavirus?action=click\&pgtype=Article\&state=default\&region=TOP_BANNER\&context=storylines_menu}{Economy}
\end{itemize}

Advertisement

\protect\hyperlink{after-top}{Continue reading the main story}

Supported by

\protect\hyperlink{after-sponsor}{Continue reading the main story}

\hypertarget{englands-world-beating-system-to-track-the-virus-is-anything-but}{%
\section{England's `World Beating' System to Track the Virus Is Anything
But}\label{englands-world-beating-system-to-track-the-virus-is-anything-but}}

Like a lot of the country's pandemic response, contact tracing has been
hampered by inconsistency, with much promised but little delivered.

\includegraphics{https://static01.nyt.com/images/2020/06/17/world/17virus-uk-contacttracing1/merlin_173608458_7e99d7fe-cc43-4f66-b369-22f5fdb5d27c-articleLarge.jpg?quality=75\&auto=webp\&disable=upscale}

\href{https://www.nytimes.com/by/benjamin-mueller}{\includegraphics{https://static01.nyt.com/images/2018/02/20/multimedia/author-benjamin-mueller/author-benjamin-mueller-thumbLarge.jpg}}\href{https://www.nytimes.com/by/jane-bradley}{\includegraphics{https://static01.nyt.com/images/2020/03/04/reader-center/author-jane-bradley/author-jane-bradley-thumbLarge.png}}

By \href{https://www.nytimes.com/by/benjamin-mueller}{Benjamin Mueller}
and \href{https://www.nytimes.com/by/jane-bradley}{Jane Bradley}

\begin{itemize}
\item
  Published June 17, 2020Updated July 3, 2020
\item
  \begin{itemize}
  \item
  \item
  \item
  \item
  \item
  \item
  \end{itemize}
\end{itemize}

LONDON --- Prime Minister
\href{https://www.nytimes.com/2020/06/23/world/europe/uk-coronavirus-reopening.html}{Boris
Johnson} of
\href{https://www.nytimes.com/2020/06/23/world/europe/uk-coronavirus-reopening.html}{Britain}
unveiled last month a
\href{https://www.youtube.com/watch?v=VdlIVAwWmB8}{``world beating''
operation} to track down people who had been exposed to the
\href{https://www.nytimes.com/2020/06/23/world/europe/uk-coronavirus-reopening.html}{coronavirus},
giving the country a chance to climb out of lockdown without losing
sight of where infections were spreading.

As with much of the
\href{https://www.nytimes.com/2020/05/25/world/europe/coronavirus-uk-nursing-homes.html}{government's
response to the pandemic}, however, the results have fallen short of the
promises, jeopardizing the
\href{https://www.nytimes.com/2020/05/11/world/europe/coronavirus-uk-boris-johnson.html}{reopening
of Britain's hobbled economy} and risking a second wave of death in one
of the countries
\href{https://www.nytimes.com/interactive/2020/world/coronavirus-maps.html}{most
debilitated by the virus}.

In almost three weeks since the start of the system in
\href{https://www.nytimes.com/2020/07/03/world/europe/britain-quarantine-us-coronavirus.html}{England},
called
\href{https://www.nhs.uk/conditions/coronavirus-covid-19/testing-and-tracing/nhs-test-and-trace-if-youve-been-in-contact-with-a-person-who-has-coronavirus/}{N.H.S.
Test and Trace}, some contact tracers have failed to reach a single
person, filling their days instead with internet exercise classes and
bookshelf organizing.

Some call handlers, scattered in offices and homes far from the people
they speak with, have mistakenly tried to send patients in England to
testing sites across the sea in Northern Ireland.

And a government minister threatened on a conference call to stop
coordinating with local leaders on the virus-tracking system if they
spoke publicly about its failings, according to three officials briefed
on the call, who spoke on the condition of anonymity for fear of
retribution.

Contact tracing was supposed to be the bridge between lockdown and a
vaccine, enabling the government to pinpoint clusters of infections as
they emerged and to stop infected people from passing on the virus.
Without it, a
\href{https://www.theguardian.com/world/2020/jun/14/who-cautions-against-further-lifting-lockdown-england}{World
Health Organization official said recently}, England would be remiss in
reopening its economy.

But the system, staffed by thousands of poorly trained and low-paid
contact tracers, was rushed out of the gate on May 28 before it was
ready, according to interviews with more than a dozen contact tracers,
public health officials and local government leaders. At the time, the
government was making a barrage of announcements while also trying to
douse a scandal involving
\href{https://www.nytimes.com/2020/05/23/world/europe/dominic-cummings-lockdown.html}{Mr.
Johnson's most senior aide, who had violated lockdown orders}.

The troubled rollout has left public health officials across England
trying to battle a virus they still cannot locate. Test results from
privately run sites, now
\href{https://www.gov.uk/guidance/coronavirus-covid-19-information-for-the-public}{numbering
in the tens of thousands daily}, were not being reported at a local
level as recently as last week, leaders in six councils said. Public
health officials say they catch wind of outbreaks from the news. And
while the virus is cooling off in London, infection rates remain high in
other parts of England,
\href{https://www.theguardian.com/world/2020/jun/07/what-is-coronavirus-r-number-and-is-it-rising-in-uk}{notably
the northwest}.

\includegraphics{https://static01.nyt.com/images/2020/06/17/world/17virus-uk-contacttracing2/merlin_173588823_fa56ba44-b155-45a3-a29b-ed75b026e178-articleLarge.jpg?quality=75\&auto=webp\&disable=upscale}

Other nations in Europe are building their public sectors to support
contact-tracing systems that might be needed for years to come. Germany,
for instance, has hired contact tracers in 375 public health
authorities, with doctors on hand to administer tests.

But in England, where
\href{https://www.nytimes.com/2019/02/24/world/europe/britain-austerity-may-budget.html}{a
decade of austerity} has
\href{https://www.local.gov.uk/sites/default/files/documents/LGA\%20briefing\%20-\%20health\%20and\%20local\%20public\%20health\%20cuts\%20-\%20HoC\%20140519\%20WEB.pdf}{starved
public health departments} of workers who used to regularly track
illnesses, Mr. Johnson has entrusted the job largely to Serco, an
outsourcing giant that was recently obliged to pay the government a
hefty fine for fraud on a previous, unrelated contract. The New York
Times has learned that the contact-tracing contract, awarded in a
secretive procurement process, cost 108 million pounds, or about \$136
million.

\hypertarget{latest-updates-global-coronavirus-outbreak}{%
\section{\texorpdfstring{\href{https://www.nytimes.com/2020/08/01/world/coronavirus-covid-19.html?action=click\&pgtype=Article\&state=default\&region=MAIN_CONTENT_1\&context=storylines_live_updates}{Latest
Updates: Global Coronavirus
Outbreak}}{Latest Updates: Global Coronavirus Outbreak}}\label{latest-updates-global-coronavirus-outbreak}}

Updated 2020-08-02T07:42:09.613Z

\begin{itemize}
\tightlist
\item
  \href{https://www.nytimes.com/2020/08/01/world/coronavirus-covid-19.html?action=click\&pgtype=Article\&state=default\&region=MAIN_CONTENT_1\&context=storylines_live_updates\#link-34047410}{The
  U.S. reels as July cases more than double the total of any other
  month.}
\item
  \href{https://www.nytimes.com/2020/08/01/world/coronavirus-covid-19.html?action=click\&pgtype=Article\&state=default\&region=MAIN_CONTENT_1\&context=storylines_live_updates\#link-780ec966}{Top
  U.S. officials work to break an impasse over the federal jobless
  benefit.}
\item
  \href{https://www.nytimes.com/2020/08/01/world/coronavirus-covid-19.html?action=click\&pgtype=Article\&state=default\&region=MAIN_CONTENT_1\&context=storylines_live_updates\#link-2bc8948}{Its
  outbreak untamed, Melbourne goes into even greater lockdown.}
\end{itemize}

\href{https://www.nytimes.com/2020/08/01/world/coronavirus-covid-19.html?action=click\&pgtype=Article\&state=default\&region=MAIN_CONTENT_1\&context=storylines_live_updates}{See
more updates}

More live coverage:
\href{https://www.nytimes.com/live/2020/07/31/business/stock-market-today-coronavirus?action=click\&pgtype=Article\&state=default\&region=MAIN_CONTENT_1\&context=storylines_live_updates}{Markets}

Allyson Pollock, a professor of public health at Newcastle University,
said, ``The government has dismantled, fragmented and eviscerated so
much of its health service over the last 20 years that it was much more
difficult to get a coordinated system.''

``They're basically trying to build a centralized, parallel, privatized
system,'' she added.

As a result, she said, ``We've had far more deaths than we should have.
And lockdown has had to go on much longer than in other countries
because we've let the virus rip for so long.''

Asked for comment, a spokesman for the Department of Health and Social
Care said that its contact-tracing system was already helping to save
lives by curbing the spread of the illness.

``In the first week, tens of thousands of people have engaged with the
N.H.S. Test and Trace service,'' the spokesman said. ``We are working to
reach more people and making improvements to the service to do that.''

Garry Robinson, Serco's customer services director for Britain, said in
a statement that the company was ``committed to supporting the
government's test and trace program'' and had successfully mobilized
10,500 contact tracers in four weeks, which he called a ``significant
achievement.''

Image

The Chelsea and Westminster Hospital in London.~A decade of austerity in
England has starved public health departments of the workers who used to
regularly track illnesses.Credit...Andrew Testa for The New York Times

The first part of contact tracing involves health professionals calling
people who test positive for the virus and obtaining a list of their
recent contacts. Then, a lower-level tier of workers call those contacts
to ask them to isolate themselves.

But in the first week of virus tracking in England,
\href{https://assets.publishing.service.gov.uk/government/uploads/system/uploads/attachment_data/file/891703/NHS_test_and_trace_bulletin__England__-_28_May_to_3_June_2020.pdf}{government
figures show}, thousands of infected patients were overlooked: Callers
reached 5,407 people with the virus, while missing another 2,710
positive cases that had been transferred into the system --- along with
an unspecified number that had not.

At the same time, contact tracers have waited to be assigned cases that
never came, a problem that officials have ascribed to low numbers of new
cases and infected people submitting their contacts online instead. One
employee, who like others spoke on the condition of anonymity for fear
of being fired, said that most days he watched three films, one after
the next, at a salary of about \$11 per hour.

Local public health officials were asked to make plans by the end of
June for possible tailor-made shutdowns around clusters of infections.
But they say they still have neither the powers to do that nor the
testing data to pinpoint infections.

``We are kind of driving the car while building it,'' said Dominic
Harrison, the director of public health in Blackburn, in northwest
England. ``There are still enormous problems to be resolved.''

The troubled rollout bears the hallmarks of Britain's
\href{https://www.nytimes.com/2020/04/16/world/europe/coronavirus-antibody-test-uk.html}{disastrous
efforts}to respond to the coronavirus: haphazard data, an emphasis on
political theater and a heavy dependence on the private sector. With
\href{https://www.reuters.com/article/us-health-coronavirus-britain-casualties/uk-covid-19-death-toll-tops-47000-as-pressure-heaps-on-pm-johnson-idUSKBN23211E}{deaths
nearing 50,000}, Britain sits alongside the United States and Brazil
among the countries suffering the greatest blows from the coronavirus.

After working to trace contacts in the early days of the pandemic,
Britain largely scrapped that plan by March 12, with
\href{https://assets.publishing.service.gov.uk/government/uploads/system/uploads/attachment_data/file/886989/s0007-spi-m-o-consensus-view-impact-interventions-030220-sage4.pdf}{government
scientists saying it was no longer practical}. Eleven days later,
\href{https://www.nytimes.com/2020/03/23/world/europe/coronavirus-uk-boris-johnson.html}{Mr.
Johnson declared a lockdown}.

The government has denied that contact tracing was ever stopped, and
said that to claim otherwise would be entirely wrong. However, in
internal notes mistakenly forwarded to The New York Times in response to
questions about why it initially ended contact tracing in March,
government officials wrote: ``The answer to this is we basically didn't
have the testing capacity.''

By April, with the death toll soaring,
\href{https://www.theguardian.com/world/2020/apr/17/uk-to-start-coronavirus-contact-tracing-again}{the
government reversed course} and promised to reconstitute the system for
England.

Other nations within the United Kingdom, including Wales and Scotland,
which are in charge of their own contact tracing, appointed public
health officials to run their programs.

For England, however, Mr. Johnson's government contracted Serco and
another company to hire most of its 25,000 contact tracers, despite
\href{https://www.bbc.co.uk/news/business-48853870}{Serco having
recently been fined £19 million} over claims involving a separate
contract that it had charged the government for monitoring convicts who
were dead, jailed or living outside the country.

The government said that Serco was regularly monitored and that no
concerns had been raised about the company before it was awarded the
test and trace contract.

Image

A London Underground train on Monday. Passengers are now required to
wear face masks when traveling on public transport in
England.Credit...Hannah Mckay/Reuters

The government has
\href{https://www.theguardian.com/business/2020/may/04/uk-government-using-crisis-to-transfer-nhs-duties-to-private-sector}{spent
heavily on private companies} in its response to the pandemic: Deloitte,
an accounting firm, manages testing centers; and Palantir, a data-mining
company, has helped organize supplies of protective gear.

\href{https://www.nytimes.com/news-event/coronavirus?action=click\&pgtype=Article\&state=default\&region=MAIN_CONTENT_3\&context=storylines_faq}{}

\hypertarget{the-coronavirus-outbreak-}{%
\subsubsection{The Coronavirus Outbreak
›}\label{the-coronavirus-outbreak-}}

\hypertarget{frequently-asked-questions}{%
\paragraph{Frequently Asked
Questions}\label{frequently-asked-questions}}

Updated July 27, 2020

\begin{itemize}
\item ~
  \hypertarget{should-i-refinance-my-mortgage}{%
  \paragraph{Should I refinance my
  mortgage?}\label{should-i-refinance-my-mortgage}}

  \begin{itemize}
  \tightlist
  \item
    \href{https://www.nytimes.com/article/coronavirus-money-unemployment.html?action=click\&pgtype=Article\&state=default\&region=MAIN_CONTENT_3\&context=storylines_faq}{It
    could be a good idea,} because mortgage rates have
    \href{https://www.nytimes.com/2020/07/16/business/mortgage-rates-below-3-percent.html?action=click\&pgtype=Article\&state=default\&region=MAIN_CONTENT_3\&context=storylines_faq}{never
    been lower.} Refinancing requests have pushed mortgage applications
    to some of the highest levels since 2008, so be prepared to get in
    line. But defaults are also up, so if you're thinking about buying a
    home, be aware that some lenders have tightened their standards.
  \end{itemize}
\item ~
  \hypertarget{what-is-school-going-to-look-like-in-september}{%
  \paragraph{What is school going to look like in
  September?}\label{what-is-school-going-to-look-like-in-september}}

  \begin{itemize}
  \tightlist
  \item
    It is unlikely that many schools will return to a normal schedule
    this fall, requiring the grind of
    \href{https://www.nytimes.com/2020/06/05/us/coronavirus-education-lost-learning.html?action=click\&pgtype=Article\&state=default\&region=MAIN_CONTENT_3\&context=storylines_faq}{online
    learning},
    \href{https://www.nytimes.com/2020/05/29/us/coronavirus-child-care-centers.html?action=click\&pgtype=Article\&state=default\&region=MAIN_CONTENT_3\&context=storylines_faq}{makeshift
    child care} and
    \href{https://www.nytimes.com/2020/06/03/business/economy/coronavirus-working-women.html?action=click\&pgtype=Article\&state=default\&region=MAIN_CONTENT_3\&context=storylines_faq}{stunted
    workdays} to continue. California's two largest public school
    districts --- Los Angeles and San Diego --- said on July 13, that
    \href{https://www.nytimes.com/2020/07/13/us/lausd-san-diego-school-reopening.html?action=click\&pgtype=Article\&state=default\&region=MAIN_CONTENT_3\&context=storylines_faq}{instruction
    will be remote-only in the fall}, citing concerns that surging
    coronavirus infections in their areas pose too dire a risk for
    students and teachers. Together, the two districts enroll some
    825,000 students. They are the largest in the country so far to
    abandon plans for even a partial physical return to classrooms when
    they reopen in August. For other districts, the solution won't be an
    all-or-nothing approach.
    \href{https://bioethics.jhu.edu/research-and-outreach/projects/eschool-initiative/school-policy-tracker/}{Many
    systems}, including the nation's largest, New York City, are
    devising
    \href{https://www.nytimes.com/2020/06/26/us/coronavirus-schools-reopen-fall.html?action=click\&pgtype=Article\&state=default\&region=MAIN_CONTENT_3\&context=storylines_faq}{hybrid
    plans} that involve spending some days in classrooms and other days
    online. There's no national policy on this yet, so check with your
    municipal school system regularly to see what is happening in your
    community.
  \end{itemize}
\item ~
  \hypertarget{is-the-coronavirus-airborne}{%
  \paragraph{Is the coronavirus
  airborne?}\label{is-the-coronavirus-airborne}}

  \begin{itemize}
  \tightlist
  \item
    The coronavirus
    \href{https://www.nytimes.com/2020/07/04/health/239-experts-with-one-big-claim-the-coronavirus-is-airborne.html?action=click\&pgtype=Article\&state=default\&region=MAIN_CONTENT_3\&context=storylines_faq}{can
    stay aloft for hours in tiny droplets in stagnant air}, infecting
    people as they inhale, mounting scientific evidence suggests. This
    risk is highest in crowded indoor spaces with poor ventilation, and
    may help explain super-spreading events reported in meatpacking
    plants, churches and restaurants.
    \href{https://www.nytimes.com/2020/07/06/health/coronavirus-airborne-aerosols.html?action=click\&pgtype=Article\&state=default\&region=MAIN_CONTENT_3\&context=storylines_faq}{It's
    unclear how often the virus is spread} via these tiny droplets, or
    aerosols, compared with larger droplets that are expelled when a
    sick person coughs or sneezes, or transmitted through contact with
    contaminated surfaces, said Linsey Marr, an aerosol expert at
    Virginia Tech. Aerosols are released even when a person without
    symptoms exhales, talks or sings, according to Dr. Marr and more
    than 200 other experts, who
    \href{https://academic.oup.com/cid/article/doi/10.1093/cid/ciaa939/5867798}{have
    outlined the evidence in an open letter to the World Health
    Organization}.
  \end{itemize}
\item ~
  \hypertarget{what-are-the-symptoms-of-coronavirus}{%
  \paragraph{What are the symptoms of
  coronavirus?}\label{what-are-the-symptoms-of-coronavirus}}

  \begin{itemize}
  \tightlist
  \item
    Common symptoms
    \href{https://www.nytimes.com/article/symptoms-coronavirus.html?action=click\&pgtype=Article\&state=default\&region=MAIN_CONTENT_3\&context=storylines_faq}{include
    fever, a dry cough, fatigue and difficulty breathing or shortness of
    breath.} Some of these symptoms overlap with those of the flu,
    making detection difficult, but runny noses and stuffy sinuses are
    less common.
    \href{https://www.nytimes.com/2020/04/27/health/coronavirus-symptoms-cdc.html?action=click\&pgtype=Article\&state=default\&region=MAIN_CONTENT_3\&context=storylines_faq}{The
    C.D.C. has also} added chills, muscle pain, sore throat, headache
    and a new loss of the sense of taste or smell as symptoms to look
    out for. Most people fall ill five to seven days after exposure, but
    symptoms may appear in as few as two days or as many as 14 days.
  \end{itemize}
\item ~
  \hypertarget{does-asymptomatic-transmission-of-covid-19-happen}{%
  \paragraph{Does asymptomatic transmission of Covid-19
  happen?}\label{does-asymptomatic-transmission-of-covid-19-happen}}

  \begin{itemize}
  \tightlist
  \item
    So far, the evidence seems to show it does. A widely cited
    \href{https://www.nature.com/articles/s41591-020-0869-5}{paper}
    published in April suggests that people are most infectious about
    two days before the onset of coronavirus symptoms and estimated that
    44 percent of new infections were a result of transmission from
    people who were not yet showing symptoms. Recently, a top expert at
    the World Health Organization stated that transmission of the
    coronavirus by people who did not have symptoms was ``very rare,''
    \href{https://www.nytimes.com/2020/06/09/world/coronavirus-updates.html?action=click\&pgtype=Article\&state=default\&region=MAIN_CONTENT_3\&context=storylines_faq\#link-1f302e21}{but
    she later walked back that statement.}
  \end{itemize}
\end{itemize}

But it is trying to do contact tracing on the cheap. While some American
states are paying tracers salaries of around \$50,000 a year, many
English tracers said in interviews that they were paid £8.72 an hour,
barely above the minimum wage, a figure equivalent to less than \$24,000
a year. Some of them were teenagers who had never held jobs before.

After answering online ads for generic customer service jobs, they
started work with little or no training. One Serco-employed contact
tracer said that at least a third of his 40 or so colleagues in London
had not received any online training before starting.

``We weren't talked through how a conversation could go or anything,''
said a tracer working in Sheffield, England.

Details of the procurement process, shared by a senior civil servant,
suggest a possible reason for the low pay and sketchy training: Serco
offered to provide the service at an extraordinarily tight profit margin
of less than 5 percent, roughly half the margin of the next cheapest
contender.

The contract was awarded without any real competition, the senior civil
servant said, speaking on the condition of anonymity to describe a
confidential process.

``Serco are pretty much the only people who can stand up a work force in
that time, and love them or hate them, it is about having the numbers,''
the civil servant said.

The virus-tracking system was supposed to be augmented by a smartphone
app that automated some tracing. But the tool, promised initially by
mid-May, has been
\href{https://www.nytimes.com/2020/05/07/world/europe/uk-coronavirus-contact-tracing.html}{shadowed
by fears about technical glitches and data breaches}, and the government
said it was now
\href{https://twitter.com/rowlsmanthorpe/status/1273276669541916674}{trying
to introduce the app before winter}.

Image

The~N.H.S. contact-tracing app was intended to help corral the
coronavirus.Credit...via Agence France-Presse --- Getty Images

Even some of the more experienced, higher-paid contract tracers who
speak to infected people said they were feeling underutilized. Gerry, a
former nurse, said she had expected to begin work as a contact tracer in
early June. Instead, at 10:30 p.m. on May 27, she received an email
telling her the program would begin the next day. The computer system
crashed as thousands of contact tracers tried to log on.

More than two weeks later, she still has not spoken to a single contact.
Other contact tracers complained on a private Facebook group that they
were still waiting for login details two weeks after the start date,
according to screenshots from the group.

Some contact tracers also said they were unaware of any translation
services, a problem that could keep England from tracking the virus
through
\href{https://www.nytimes.com/2020/04/08/world/europe/coronavirus-doctors-immigrants.html}{migrant
and ethnic minority communities}, which have
\href{https://www.nytimes.com/2020/05/07/world/europe/coronavirus-uk-black-britons.html}{suffered
disproportionately}.

``It's a total shambles,'' said Ben Bradshaw, an opposition Labour
lawmaker, who has spoken to government officials about contact tracing.

``Everyone has accepted all the way through this crisis that the
countries that have dealt with it best have always had effective track
and trace systems in place, and that any country wishing to emerge from
lockdown and live with this virus for the foreseeable future will need
an effective track and trace system,'' he said. ``Yet, the history of
this in Britain is a catalog of disasters.''

Image

A memorial in London for the victims of Covid-19 and other illnesses
contracted during the coronavirus outbreak.Credit...Aaron Chown/Press
Association, via Associated Press

Advertisement

\protect\hyperlink{after-bottom}{Continue reading the main story}

\hypertarget{site-index}{%
\subsection{Site Index}\label{site-index}}

\hypertarget{site-information-navigation}{%
\subsection{Site Information
Navigation}\label{site-information-navigation}}

\begin{itemize}
\tightlist
\item
  \href{https://help.nytimes.com/hc/en-us/articles/115014792127-Copyright-notice}{©~2020~The
  New York Times Company}
\end{itemize}

\begin{itemize}
\tightlist
\item
  \href{https://www.nytco.com/}{NYTCo}
\item
  \href{https://help.nytimes.com/hc/en-us/articles/115015385887-Contact-Us}{Contact
  Us}
\item
  \href{https://www.nytco.com/careers/}{Work with us}
\item
  \href{https://nytmediakit.com/}{Advertise}
\item
  \href{http://www.tbrandstudio.com/}{T Brand Studio}
\item
  \href{https://www.nytimes.com/privacy/cookie-policy\#how-do-i-manage-trackers}{Your
  Ad Choices}
\item
  \href{https://www.nytimes.com/privacy}{Privacy}
\item
  \href{https://help.nytimes.com/hc/en-us/articles/115014893428-Terms-of-service}{Terms
  of Service}
\item
  \href{https://help.nytimes.com/hc/en-us/articles/115014893968-Terms-of-sale}{Terms
  of Sale}
\item
  \href{https://spiderbites.nytimes.com}{Site Map}
\item
  \href{https://help.nytimes.com/hc/en-us}{Help}
\item
  \href{https://www.nytimes.com/subscription?campaignId=37WXW}{Subscriptions}
\end{itemize}
