Sections

SEARCH

\protect\hyperlink{site-content}{Skip to
content}\protect\hyperlink{site-index}{Skip to site index}

\href{https://www.nytimes.com/section/arts}{Arts}

\href{https://myaccount.nytimes.com/auth/login?response_type=cookie\&client_id=vi}{}

\href{https://www.nytimes.com/section/todayspaper}{Today's Paper}

\href{/section/arts}{Arts}\textbar{}How to Entertain Your Kids This
Summer? Maybe Don't

\url{https://nyti.ms/2BfK3lD}

\begin{itemize}
\item
\item
\item
\item
\item
\end{itemize}

\href{https://www.nytimes.com/spotlight/at-home?action=click\&pgtype=Article\&state=default\&region=TOP_BANNER\&context=at_home_menu}{At
Home}

\begin{itemize}
\tightlist
\item
  \href{https://www.nytimes.com/2020/07/28/books/time-for-a-literary-road-trip.html?action=click\&pgtype=Article\&state=default\&region=TOP_BANNER\&context=at_home_menu}{Take:
  A Literary Road Trip}
\item
  \href{https://www.nytimes.com/2020/07/29/magazine/bored-with-your-home-cooking-some-smoky-eggplant-will-fix-that.html?action=click\&pgtype=Article\&state=default\&region=TOP_BANNER\&context=at_home_menu}{Cook:
  Smoky Eggplant}
\item
  \href{https://www.nytimes.com/2020/07/27/travel/moose-michigan-isle-royale.html?action=click\&pgtype=Article\&state=default\&region=TOP_BANNER\&context=at_home_menu}{Look
  Out: For Moose}
\item
  \href{https://www.nytimes.com/interactive/2020/at-home/even-more-reporters-editors-diaries-lists-recommendations.html?action=click\&pgtype=Article\&state=default\&region=TOP_BANNER\&context=at_home_menu}{Explore:
  Reporters' Obsessions}
\end{itemize}

Advertisement

\protect\hyperlink{after-top}{Continue reading the main story}

Supported by

\protect\hyperlink{after-sponsor}{Continue reading the main story}

Summer Guide for Families

\hypertarget{how-to-entertain-your-kids-this-summer-maybe-dont}{%
\section{How to Entertain Your Kids This Summer? Maybe
Don't}\label{how-to-entertain-your-kids-this-summer-maybe-dont}}

You can keep your family safe and sane by encouraging old-school play,
embarking on some D.I.Y. projects and, yes, even embracing boredom.

\includegraphics{https://static01.nyt.com/images/2020/06/18/autossell/Screen-Shot-2020-06-16-at-1/Screen-Shot-2020-06-16-at-1-superJumbo.png}

By \href{https://www.nytimes.com/by/alexis-soloski}{Alexis Soloski}

\begin{itemize}
\item
  June 18, 2020
\item
  \begin{itemize}
  \item
  \item
  \item
  \item
  \item
  \end{itemize}
\end{itemize}

A funny thing about summer: It is long. It is also hot. This one comes
in the middle of a global pandemic.

And even in a changed and changing world, I have reserved some mental
energy for panicking about how my kids, husband and I will make it to
September without everyone's brains turning into Haribo gummies. Let me
put it this way: On a recent rainy Saturday, we baked banana bread and
played games. We made lunch together, built a cardboard lantern and
learned about the constellations. It was exhausting. And they still put
down two Disney movies. Three months into school closures, my children
have watched every show. There are no shows left.

And yet, working from home with small children, an ordeal and a
privilege, has been de rigueur since agrarianism got going. Parents
managed it for thousands of years --- without day care, compulsory
schooling or camps. What did children used to do all day? Short answer:
They worked and they played, often with minimal adult supervision.

Unfortunately, as
\href{https://liberalarts.utexas.edu/history/faculty/shm654}{Steven
Mintz}, the author of ``Huck's Raft: A History of American Childhood,''
told me, ``The pandemic has exaggerated and intensified the worst
features of children's play today: adult intrusion; the decline of
physical, outdoor and social play; and mediation by screens.'' Ow.

So, how do we adults ameliorate that while staying safe, employed and
reasonably sane? Here are some ideas.

\hypertarget{go-old-school-very-old}{%
\subsection{Go Old School. Very Old.}\label{go-old-school-very-old}}

In an email, Mintz, a history professor at the University of Texas at
Austin, pointed to Pieter Bruegel the Elder's 1560 painting
``\href{https://artsandculture.google.com/asset/children\%E2\%80\%99s-games-pieter-bruegel-the-elder/CQEeZWQPOI2Yjg?hl=en}{Children's
Games}.'' A canvas to give social-distancing enforcers nightmares, it
shows 100 or so Flemish youths disporting themselves with hoops, stilts,
bubbles, marbles, the occasional pig bladder and the wholesome fun of
beating one another with a scourge. The Flemish parents are elsewhere,
presumably answering emails or cracking open a brown ale.

\includegraphics{https://static01.nyt.com/images/2020/06/19/arts/18kids-culture2/merlin_145228941_83b8701d-3442-4fd8-a2ad-af3f0cdbe6cb-articleLarge.jpg?quality=75\&auto=webp\&disable=upscale}

The painting suggests that a lot of play is social, a difficulty in a
pandemic. But it also insists that the desire for play is innate and
that children will find ways to amuse themselves, especially if you can
supply some rudimentary toys --- kites, cards, blocks, dolls, balls,
paper boats and paper airplanes, a garden hose if you have one, a
half-filled tub. If they have a safe space to play outside (where the
toys are even more analog: sticks, rocks, dirt) and you can work from
your phone while they do it, even better.

This may also be a good time to get away from the idea that play should
be educational or S.T.E.M.-enhancing. ``All play is productive,'' Mintz
said. ``They will learn something from whatever they do.''

\hypertarget{embrace-boredom}{%
\subsection{Embrace Boredom}\label{embrace-boredom}}

Still, children may not want to play on their own or with a sibling, and
you may have conference calls or Twitter threads that beckon. Which
means they will claim boredom, and more than likely they will whine
about it. What should you do? Nothing.

Feeling that we ought to keep kids happy and entertained is a
comparatively modern mind-set and speaks to certain resources and
luxuries. Instead of trying to prevent boredom, maybe welcome it and see
what children do. Tom Hodgkinson, author of
``\href{https://www.penguin.co.uk/books/56063/the-idle-parent/9780141030357.html}{The
Idle Parent},'' suggested ramping up slowly, with an hour or so of
``nothing time'' every day. Maybe less, if your children are very young.
If they resist, he suggested doubling down on tedium --- reading
``Paradise Lost'' or screening a Tarkovsky film --- so that they end up
running into another room and doing something else.

``You could try boring them with your games so they invent something
better,'' he advised. ``Be a really boring mom.''

\hypertarget{a-diy-approach-to-culture}{%
\subsection{A D.I.Y. Approach to
Culture}\label{a-diy-approach-to-culture}}

Normally, around this time of year, my desktop and actual desk are
littered with notes about outdoor theaters, concerts in the park, and
art installations that might hold a child's attention for 10 whole
minutes. Now my calendar looks like new-fallen snow.

If you can't take your kids to cultural events, have your kids bring
culture to you. ``Be like Louisa May Alcott,'' Mintz suggested. The
March girls of ``Little Women'' don't spend a ton of time lobbying for
more ``My Little Pony: Friendship Is Magic'' episodes. Instead they make
up fantasy plays, write newspapers, craft costumes, stage their own
circus, act out stories from Dickens's ``The Pickwick Papers.'' (``The
Pickwick Papers'' are not exciting! And still they make do.) Their
efforts may be painful, but the 20 minutes your children spend preparing
a deeply revisionist ``Frozen 2'' is 20 minutes you can spend doing
something else.

\hypertarget{two-words-free-labor}{%
\subsection{Two Words: Free Labor}\label{two-words-free-labor}}

Housework can also become a form of play, and depending on how well or
poorly your children do it, may be some help. In the 19th century,
Hodgkinson said, ``children were seen as not necessarily a burden on the
household, but a welcome labor force.'' Employ them.

``The thing to remember is that kids want to help, so try to get them in
the habit of doing some of those things,'' Lenore Skenazy, president of
Let Grow, a nonprofit promoting childhood independence, said. ``A
3-year-old separating laundry is quite possible and also quite fun.
Six-year-olds can be making breakfast.'' So, yes, children can cook,
they can clean. If you can take a few extra minutes to gamify the chore
--- Mary Poppins's ``Spoonful of Sugar'' approach --- they may even
enjoy it.

\hypertarget{muddle-through}{%
\subsection{Muddle Through}\label{muddle-through}}

A pandemic isn't forever. Probably. So if it's easier, leave historical
practice aside, give guilt the vacation that you can't take and get
through it. ``Don't think that there's something wrong with you or that
you haven't been the perfect camp counselor and made it a fun and
exciting and rewarding summer for everyone,'' Skenazy said. ``I mean,
just give yourself a break.''

If that break involves a lot of screens, remember that new entertainment
forms and technologies --- from the written word on --- have always
attracted suspicion that they will pulp or corrupt young minds. And most
of us have turned out OK, no matter how many ``Smurfs'' episodes we may
have once absorbed. Video games provide an opportunity to socialize, a
streamed musical is still a musical, a virtual tour of a gallery or
museum isn't the same as wandering the halls yourself, but take what you
can get.

In general, find out what your children like to do and encourage them to
do it. Or go with the obverse: When you have time available, make them
do stuff that you like. In my case, that means
\href{https://www.nytimes.com/2020/04/30/arts/board-games-soothing-virus.html}{playing
board games} and watching
\href{https://www.youtube.com/watch?v=bF7q37SfF4w}{toy theater videos on
YouTube}, plus the occasional
\href{https://www.nytimes.com/2020/05/27/movies/studio-ghibli-hbo-max.html}{Hayao
Miyazaki} movie. Or the more than occasional one.

``Just let them watch a lot of films,'' Hodgkinson said. ``It's
temporary, it's not forever. We really shouldn't be too hard on
ourselves.''

Advertisement

\protect\hyperlink{after-bottom}{Continue reading the main story}

\hypertarget{site-index}{%
\subsection{Site Index}\label{site-index}}

\hypertarget{site-information-navigation}{%
\subsection{Site Information
Navigation}\label{site-information-navigation}}

\begin{itemize}
\tightlist
\item
  \href{https://help.nytimes.com/hc/en-us/articles/115014792127-Copyright-notice}{©~2020~The
  New York Times Company}
\end{itemize}

\begin{itemize}
\tightlist
\item
  \href{https://www.nytco.com/}{NYTCo}
\item
  \href{https://help.nytimes.com/hc/en-us/articles/115015385887-Contact-Us}{Contact
  Us}
\item
  \href{https://www.nytco.com/careers/}{Work with us}
\item
  \href{https://nytmediakit.com/}{Advertise}
\item
  \href{http://www.tbrandstudio.com/}{T Brand Studio}
\item
  \href{https://www.nytimes.com/privacy/cookie-policy\#how-do-i-manage-trackers}{Your
  Ad Choices}
\item
  \href{https://www.nytimes.com/privacy}{Privacy}
\item
  \href{https://help.nytimes.com/hc/en-us/articles/115014893428-Terms-of-service}{Terms
  of Service}
\item
  \href{https://help.nytimes.com/hc/en-us/articles/115014893968-Terms-of-sale}{Terms
  of Sale}
\item
  \href{https://spiderbites.nytimes.com}{Site Map}
\item
  \href{https://help.nytimes.com/hc/en-us}{Help}
\item
  \href{https://www.nytimes.com/subscription?campaignId=37WXW}{Subscriptions}
\end{itemize}
