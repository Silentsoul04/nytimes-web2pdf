Sections

SEARCH

\protect\hyperlink{site-content}{Skip to
content}\protect\hyperlink{site-index}{Skip to site index}

\href{https://www.nytimes.com/section/obituaries}{Obituaries}

\href{https://myaccount.nytimes.com/auth/login?response_type=cookie\&client_id=vi}{}

\href{https://www.nytimes.com/section/todayspaper}{Today's Paper}

\href{/section/obituaries}{Obituaries}\textbar{}Messias Kokama,
Indigenous Leader in the Amazon, Dies at 53

\url{https://nyti.ms/2zIuvX6}

\begin{itemize}
\item
\item
\item
\item
\item
\end{itemize}

\href{https://www.nytimes.com/news-event/coronavirus?action=click\&pgtype=Article\&state=default\&region=TOP_BANNER\&context=storylines_menu}{The
Coronavirus Outbreak}

\begin{itemize}
\tightlist
\item
  live\href{https://www.nytimes.com/2020/08/03/world/coronavirus-covid-19.html?action=click\&pgtype=Article\&state=default\&region=TOP_BANNER\&context=storylines_menu}{Latest
  Updates}
\item
  \href{https://www.nytimes.com/interactive/2020/us/coronavirus-us-cases.html?action=click\&pgtype=Article\&state=default\&region=TOP_BANNER\&context=storylines_menu}{Maps
  and Cases}
\item
  \href{https://www.nytimes.com/interactive/2020/science/coronavirus-vaccine-tracker.html?action=click\&pgtype=Article\&state=default\&region=TOP_BANNER\&context=storylines_menu}{Vaccine
  Tracker}
\item
  \href{https://www.nytimes.com/2020/08/02/us/covid-college-reopening.html?action=click\&pgtype=Article\&state=default\&region=TOP_BANNER\&context=storylines_menu}{College
  Reopening}
\item
  \href{https://www.nytimes.com/live/2020/08/03/business/stock-market-today-coronavirus?action=click\&pgtype=Article\&state=default\&region=TOP_BANNER\&context=storylines_menu}{Economy}
\end{itemize}

Advertisement

\protect\hyperlink{after-top}{Continue reading the main story}

Supported by

\protect\hyperlink{after-sponsor}{Continue reading the main story}

Those We've Lost

\hypertarget{messias-kokama-indigenous-leader-in-the-amazon-dies-at-53}{%
\section{Messias Kokama, Indigenous Leader in the Amazon, Dies at
53}\label{messias-kokama-indigenous-leader-in-the-amazon-dies-at-53}}

Chief Kokama sought better treatment and opportunities for Indigenous
people in Manaus, a major Brazilian city. He died of Covid-19.

\includegraphics{https://static01.nyt.com/images/2020/06/20/obituaries/18Kokama/merlin_173661120_5126d8ef-df48-4f6f-917d-54aa9395b875-articleLarge.jpg?quality=75\&auto=webp\&disable=upscale}

By Michael Astor

\begin{itemize}
\item
  June 18, 2020
\item
  \begin{itemize}
  \item
  \item
  \item
  \item
  \item
  \end{itemize}
\end{itemize}

\emph{This obituary is part of a series about people who have died in
the coronavirus pandemic. Read about others}
\href{https://www.nytimes.com/interactive/2020/obituaries/people-died-coronavirus-obituaries.html}{\emph{here}}\emph{.}

Messias Kokama grew up in a village of Indigenous people in the Amazon
region and traveled nearly a thousand miles down river to the nearest
big city in search of opportunity. What he found instead was
discrimination. As far as officials were concerned, he practically
didn't exist.

Mr. Kokama became an activist, devoting his life to defending Indigenous
rights in a rapidly urbanizing Amazon. He was instrumental in founding
Tribes Park, a rough-hewed collection of cinder block houses where some
700 families from 35 different Amazon tribes are able to maintain their
culture inside Manaus, a sprawling industrial city of 1.7 million in the
heart of the rainforest.

He died at 53 on May 13 at a hospital in the city. The cause was
Covid-19, his daughter Mirian said.

Before Tribes Park, Indigenous people in Manaus were mostly relegated to
the edge of town or sleeping on streets. Because Indigenous Brazilians
are considered wards of the state, social services and other vestiges of
officialdom are generally provided only on indigenous lands by the
Federal Indian Bureau. That began to change when the city officially
recognized Tribes Park in 2018 and paved the streets and provided
electricity and basic health services.

Indigenous leaders had been planning the park for years, but when they
elected Mr. Kokama as chief in 2012, work began in earnest. That meant
navigating a grinding bureaucracy, fending off lawsuits, rebuffing
violent eviction attempts and even battling drug gangs. What's more, Mr.
Kokama had to sort out internal disputes between tribes that were not
traditional allies.

Messias Martins Moreira (who, like many Indigenous Brazilians, used his
tribe's name, Kokama, as his surname) was born on Sept. 19, 1966, in
Taboca, near Brazil's western border with Colombia and Peru. His father,
Abelada Moreira, was a baker, and his mother, Nedina (Martins) Moreira,
worked at home.

Mr. Kokama was 20 when he traveled to Manaus, finding work on a fish
farm. He was shocked to discover the discrimination facing the
Indigenous people there. When the fish farm failed, he went back to his
village and made a living selling household goods, only to return to
Manaus eight years later to press for urban Indigenous rights.

Mr. Kokama, who received his high school diploma at 40, was a strong
advocate of education for Indigenous people and at his death was helping
to build a school in the park. He also pushed for improved health care
as Covid-19 began ravaging the tribes in April.

Along with his daughter, he is survived by his companion, Marilia
Marinho; a son, Miqueias, who has replaced him as chief; another
daughter, Vitoria; and a grandchild.

Mr. Kokama served as a Pentecostal pastor, without renouncing his
Indigenous culture and beliefs.

``We Indigenous can be what we want without losing who we are,'' he
would often say.

\href{https://www.nytimes.com/interactive/2020/obituaries/people-died-coronavirus-obituaries.html?action=click\&pgtype=Article\&state=default\&region=BELOW_MAIN_CONTENT\&context=covid_obits_promo}{}

\hypertarget{those-weve-lost}{%
\section{Those We've Lost}\label{those-weve-lost}}

The coronavirus pandemic has taken an incalculable death toll. This
series is designed to put names and faces to the numbers.

Read more

\includegraphics{https://static01.nyt.com/images/2020/07/30/obituaries/30Pedro/30Pedro-square640.jpg}

\hypertarget{bernaldina-josuxe9-pedro}{%
\section{Bernaldina José Pedro}\label{bernaldina-josuxe9-pedro}}

d. Boa Vista, Brazil

Leader among the Indigenous Macuxi

\includegraphics{https://static01.nyt.com/images/2020/07/31/obituaries/31Swing/merlin_175167783_8913bc90-0d64-43f3-a655-1bb1bf1601c9-square640.jpg}

\hypertarget{john-eric-swing}{%
\section{John Eric Swing}\label{john-eric-swing}}

d. Fountain Valley, Calif.

Champion of Filipino-Americans

\includegraphics{https://static01.nyt.com/images/2020/07/27/obituaries/27Victor/merlin_175001436_38b11f8e-227a-4e2c-9821-7618af9b2524-square640.jpg}

\hypertarget{victor-victor}{%
\section{Victor Victor}\label{victor-victor}}

d. Santo Domingo, Dominican Republic

Beloved musician of the Dominican Republic

\includegraphics{https://static01.nyt.com/images/2020/07/31/obituaries/31Negron/merlin_175160169_516322ae-fd23-4969-b6b2-193ced371105-square640.jpg}

\hypertarget{dr-eddie-negruxf3n}{%
\section{Dr. Eddie Negrón}\label{dr-eddie-negruxf3n}}

d. Fort Walton Beach, Fla.

Internist on Florida's Emerald Coast

\includegraphics{https://static01.nyt.com/images/2020/07/30/obituaries/30Dobson/merlin_175115928_f6b9271c-8f05-4fe1-a38a-5ca4a58f8935-square640.jpg}

\hypertarget{dobby-dobson}{%
\section{Dobby Dobson}\label{dobby-dobson}}

d. Coral Springs, Fla.

Jamaican singer and songwriter

\includegraphics{https://static01.nyt.com/images/2020/08/01/obituaries/28Gonzalez/merlin_175002771_beb57888-3951-409a-ae13-03a94b2e962e-square640.jpg}

\hypertarget{waldemar-gonzalez}{%
\section{Waldemar Gonzalez}\label{waldemar-gonzalez}}

d. White Plains, N.Y.

Teacher and social worker

Advertisement

\protect\hyperlink{after-bottom}{Continue reading the main story}

\hypertarget{site-index}{%
\subsection{Site Index}\label{site-index}}

\hypertarget{site-information-navigation}{%
\subsection{Site Information
Navigation}\label{site-information-navigation}}

\begin{itemize}
\tightlist
\item
  \href{https://help.nytimes.com/hc/en-us/articles/115014792127-Copyright-notice}{©~2020~The
  New York Times Company}
\end{itemize}

\begin{itemize}
\tightlist
\item
  \href{https://www.nytco.com/}{NYTCo}
\item
  \href{https://help.nytimes.com/hc/en-us/articles/115015385887-Contact-Us}{Contact
  Us}
\item
  \href{https://www.nytco.com/careers/}{Work with us}
\item
  \href{https://nytmediakit.com/}{Advertise}
\item
  \href{http://www.tbrandstudio.com/}{T Brand Studio}
\item
  \href{https://www.nytimes.com/privacy/cookie-policy\#how-do-i-manage-trackers}{Your
  Ad Choices}
\item
  \href{https://www.nytimes.com/privacy}{Privacy}
\item
  \href{https://help.nytimes.com/hc/en-us/articles/115014893428-Terms-of-service}{Terms
  of Service}
\item
  \href{https://help.nytimes.com/hc/en-us/articles/115014893968-Terms-of-sale}{Terms
  of Sale}
\item
  \href{https://spiderbites.nytimes.com}{Site Map}
\item
  \href{https://help.nytimes.com/hc/en-us}{Help}
\item
  \href{https://www.nytimes.com/subscription?campaignId=37WXW}{Subscriptions}
\end{itemize}
