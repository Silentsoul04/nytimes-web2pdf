Sections

SEARCH

\protect\hyperlink{site-content}{Skip to
content}\protect\hyperlink{site-index}{Skip to site index}

\href{https://www.nytimes.com/section/opinion/sunday}{Sunday Review}

\href{https://myaccount.nytimes.com/auth/login?response_type=cookie\&client_id=vi}{}

\href{https://www.nytimes.com/section/todayspaper}{Today's Paper}

\href{/section/opinion/sunday}{Sunday Review}\textbar{}Good Cops, Bad
Cops

\href{https://nyti.ms/3hMAaN1}{https://nyti.ms/3hMAaN1}

\begin{itemize}
\item
\item
\item
\item
\item
\item
\end{itemize}

Advertisement

\protect\hyperlink{after-top}{Continue reading the main story}

\href{/section/opinion}{Opinion}

Supported by

\protect\hyperlink{after-sponsor}{Continue reading the main story}

\hypertarget{good-cops-bad-cops}{%
\section{Good Cops, Bad Cops}\label{good-cops-bad-cops}}

On Father's Day, I want to tell my dad something I neglected to mention.

\href{https://www.nytimes.com/by/maureen-dowd}{\includegraphics{https://static01.nyt.com/images/2018/04/02/opinion/maureen-dowd/maureen-dowd-thumbLarge.png}}

By \href{https://www.nytimes.com/by/maureen-dowd}{Maureen Dowd}

Opinion Columnist

\begin{itemize}
\item
  June 20, 2020
\item
  \begin{itemize}
  \item
  \item
  \item
  \item
  \item
  \item
  \end{itemize}
\end{itemize}

\includegraphics{https://static01.nyt.com/images/2020/06/21/opinion/sunday/21Dowd/21Dowd-articleLarge.jpg?quality=75\&auto=webp\&disable=upscale}

I never told my father I was proud of him.

I grew up in the '60s, another era filled with tears and tear gas and
violent clashes about race and class.

I didn't want to be a hippie, but I certainly didn't want to be a
fascist. I was sheltered in my demure blue school uniform and saddle
shoes, watching the world burn.

The National Guard slaughtering students at Kent State. The Chicago
police billy-clubbing yippies at the '68 Democratic convention. Soldiers
in Vietnam getting denounced as ``baby killers,'' and radicals vowing to
``barbecue some pork'' and spill the blood of ``pigs.''

When our school newspaper published an anti-Vietnam War cartoon, the
principal, a nun, dumped all the copies into the incinerator.

As a 16-year-old in 1968, I found it hard to balance hating the Vietnam
War and wanting racial justice with being part of a family, baked in
patriotism, taught to revere uniforms. As Bill Clinton wrote in that
\href{https://www.pbs.org/wgbh/pages/frontline/shows/clinton/etc/draftletter.html}{infamous
1969 letter}, the cool kids were all about ``loathing the military''; I
was making pocket change by ironing my brothers' Coast Guard uniforms,
being careful to make sure the creases were sharp.

I never told classmates about my father's long stretch as a police
detective. I just talked about his second career, after retirement, as a
special assistant to a senator and congressman.

When it was time for the father-daughter lunch at Immaculata, I didn't
sign up. As an Irish immigrant with little formal education, my father
had worked terribly hard to afford that fancy girls' school. But I
didn't tell him about the lunch. I don't know if it was the cop thing or
because he was older and didn't seem that into raising a teenager. (The
day I was born, the other cops at roll call teased him about becoming a
new father at 61.)

As it turned out, one of my dad's closest friends was the speaker at the
lunch and called him to find out why he wasn't there. My dad, hurt,
asked my mom why I didn't want to take him.

And that is something I'm ashamed of.

In the wake of 9/11, I was relieved that people were able to see the
heroic side of police officers and firefighters. Celebrities began
inviting firefighters to their lobby Christmas parties in tony Upper
West Side apartment buildings. But by the next year, that fad was over.

Now come
\href{https://www.nytimes.com/2020/06/12/opinion/sunday/floyd-abolish-defund-police.html}{calls
to abolish the police}. Once I would have tried to blame bad apples. But
the grotesque spectacle of blacks being regularly executed for living
their lives is completely indefensible.

I grew up in the shadow of two powerful patriarchies, the Catholic
Church and the police. Both institutions attracted an element of warped,
sadistic people. Instead of rooting out those dark forces, the
institutions protected them, moving bad priests to another parish and
bad cops to another precinct.

The police and the church are arbiters of right and wrong, yet they let
a poisonous culture grow and conspired to shield those doing wrong and
hurting innocents.

My heart aches for all the good cops --- particularly black cops --- who
are anguished, and for their families in this season when streaming of
N.W.A.'s anti-police anthem is surging; when police shows are getting
axed as ``copaganda'' and
\href{https://www.rollingstone.com/culture/culture-features/olivia-benson-svu-mariska-hargitay-canceled-cops-1014181/}{even
Olivia Benson is canceled}; and when protesters in D.C. hold aloft signs
reading ``EAT THE RICH AND THEIR PIGGIES TOO,'' and ``ACAB'' (All Cops
Are Bastards) is spray-painted all over the pavement.

As we rein in and reimagine how a police force should work, we should
avoid that word ``all.''

I loved my dad, but decades passed before I took the time to learn about
his 40 years on the D.C. force. Finally, my sister gave me his
scrapbook.

There's a picture of him with President Coolidge, getting awarded the
Medal for Bravery. There's
\href{https://www.instagram.com/p/B4DnolcA85U/}{another pic} where he's
in his fedora, guarding F.D.R. at a ballgame as he throws out the first
ball.

My dad caught an escaped killer who hurled a flatiron at him and broke
his nose. He disarmed a wife in a courthouse who shot her cheating
husband and said she regretted only that she hadn't been able to get
``the other woman,'' too.

Over the racist objections of his captain, he wanted to give blood to
save the life of a black man wounded during a series of armed holdups.
The man had fired point-blank at my father, but the gun jammed. Only
then did my dad return fire.

He wasn't big on bloodshed. He transferred out of the homicide
department in short order and threw up after watching my cat have
kittens.

In 1947, he
\href{https://www.nytimes.com/2017/08/19/opinion/sunday/trump-neo-nazis-and-the-klan.html}{faced
down a Ku Klux Klan} leader burning crosses on the lawn of the only
Jewish resident in a Maryland town. In 1954, when he was in charge of
Senate security, he raced over to the House and wrestled the gun from
one of the four Puerto Rican nationalists who shot
\href{https://timesmachine.nytimes.com/timesmachine/1954/03/02/issue.html}{round
after round} onto the House floor, wounding five congressmen. (You can
still see the bullet hole in the table Republicans use to give
speeches.)

Amid the clippings in his scrapbook about his exploits catching
``highwaymen'' are pasted some poems. One about Irish mothers, one about
the evils of whiskey. And several that allude to the fact that police
officers can die at any moment and leave behind their families.

So, on this Father's Day, I'll say what I should have said a long time
ago: I'm proud of you, Dad.

\emph{The Times is committed to publishing}
\href{https://www.nytimes.com/2019/01/31/opinion/letters/letters-to-editor-new-york-times-women.html}{\emph{a
diversity of letters}} \emph{to the editor. We'd like to hear what you
think about this or any of our articles. Here are some}
\href{https://help.nytimes.com/hc/en-us/articles/115014925288-How-to-submit-a-letter-to-the-editor}{\emph{tips}}\emph{.
And here's our email:}
\href{mailto:letters@nytimes.com}{\emph{letters@nytimes.com}}\emph{.}

\emph{Follow The New York Times Opinion section on}
\href{https://www.facebook.com/nytopinion}{\emph{Facebook}}\emph{,}
\href{http://twitter.com/NYTOpinion}{\emph{Twitter (@NYTopinion)}}
\emph{and}
\href{https://www.instagram.com/nytopinion/}{\emph{Instagram}}\emph{.}

Advertisement

\protect\hyperlink{after-bottom}{Continue reading the main story}

\hypertarget{site-index}{%
\subsection{Site Index}\label{site-index}}

\hypertarget{site-information-navigation}{%
\subsection{Site Information
Navigation}\label{site-information-navigation}}

\begin{itemize}
\tightlist
\item
  \href{https://help.nytimes.com/hc/en-us/articles/115014792127-Copyright-notice}{©~2020~The
  New York Times Company}
\end{itemize}

\begin{itemize}
\tightlist
\item
  \href{https://www.nytco.com/}{NYTCo}
\item
  \href{https://help.nytimes.com/hc/en-us/articles/115015385887-Contact-Us}{Contact
  Us}
\item
  \href{https://www.nytco.com/careers/}{Work with us}
\item
  \href{https://nytmediakit.com/}{Advertise}
\item
  \href{http://www.tbrandstudio.com/}{T Brand Studio}
\item
  \href{https://www.nytimes.com/privacy/cookie-policy\#how-do-i-manage-trackers}{Your
  Ad Choices}
\item
  \href{https://www.nytimes.com/privacy}{Privacy}
\item
  \href{https://help.nytimes.com/hc/en-us/articles/115014893428-Terms-of-service}{Terms
  of Service}
\item
  \href{https://help.nytimes.com/hc/en-us/articles/115014893968-Terms-of-sale}{Terms
  of Sale}
\item
  \href{https://spiderbites.nytimes.com}{Site Map}
\item
  \href{https://help.nytimes.com/hc/en-us}{Help}
\item
  \href{https://www.nytimes.com/subscription?campaignId=37WXW}{Subscriptions}
\end{itemize}
