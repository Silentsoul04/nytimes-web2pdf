Sections

SEARCH

\protect\hyperlink{site-content}{Skip to
content}\protect\hyperlink{site-index}{Skip to site index}

\href{https://www.nytimes.com/section/politics}{Politics}

\href{https://myaccount.nytimes.com/auth/login?response_type=cookie\&client_id=vi}{}

\href{https://www.nytimes.com/section/todayspaper}{Today's Paper}

\href{/section/politics}{Politics}\textbar{}Judge Rejects Trump Request
for Order Blocking Bolton's Memoir

\url{https://nyti.ms/37NOgta}

\begin{itemize}
\item
\item
\item
\item
\item
\end{itemize}

Advertisement

\protect\hyperlink{after-top}{Continue reading the main story}

Supported by

\protect\hyperlink{after-sponsor}{Continue reading the main story}

\hypertarget{judge-rejects-trump-request-for-order-blocking-boltons-memoir}{%
\section{Judge Rejects Trump Request for Order Blocking Bolton's
Memoir}\label{judge-rejects-trump-request-for-order-blocking-boltons-memoir}}

But the judge also sharply criticized the former national security
adviser, suggesting his \$2 million book advance may be in jeopardy.

\includegraphics{https://static01.nyt.com/images/2020/06/20/us/politics/20dc-bolton/20dc-bolton-articleLarge.jpg?quality=75\&auto=webp\&disable=upscale}

\href{https://www.nytimes.com/by/charlie-savage}{\includegraphics{https://static01.nyt.com/images/2018/06/12/multimedia/author-charlie-savage/author-charlie-savage-thumbLarge-v2.png}}

By \href{https://www.nytimes.com/by/charlie-savage}{Charlie Savage}

\begin{itemize}
\item
  June 20, 2020
\item
  \begin{itemize}
  \item
  \item
  \item
  \item
  \item
  \end{itemize}
\end{itemize}

WASHINGTON --- President Trump's former national security adviser John
R. Bolton can go forward with the publication of his memoir, a federal
judge ruled on Saturday, rejecting the administration's request for an
order that he try to pull the book back and saying it was too late for
such an order to succeed.

``With hundreds of thousands of copies around the globe --- many in
newsrooms --- the damage is done. There is no restoring the status
quo,'' wrote Judge Royce C. Lamberth of the Federal District Court of
the District of Columbia.

But
\href{https://pacer-documents.s3.amazonaws.com/36/219024/04517891261.pdf}{in
a 10-page opinion}, Judge Lamberth also suggested that Mr. Bolton may be
in jeopardy of forfeiting his \$2 million advance, as the Justice
Department has separately requested --- and that he could be prosecuted
for allowing the book to be published before receiving final notice that
a prepublication review to scrub out classified information was
complete.

``Bolton has gambled with the national security of the United States,''
Judge Lamberth wrote. ``He has exposed his country to harm and himself
to civil (and potentially criminal) liability. But these facts do not
control the motion before the court. The government has failed to
establish that an injunction will prevent irreparable harm.''

The main elements of the book, ``The Room Where It Happened,'' an
unflattering account of Mr. Trump's conduct in office, have
\href{https://www.nytimes.com/2020/06/18/us/politics/trump-bolton.html}{already
been widely reported}.

Mr. Trump has accused Mr. Bolton of lying --- and false information is
not classified. But he has also made clear that he wants the Justice
Department to prosecute his former aide for spilling secrets, a position
he reiterated on Saturday.

The president
\href{https://twitter.com/realDonaldTrump/status/1274353054117236738}{wrote
on Twitter} that Mr. Bolton ``broke the law by releasing Classified
Information (in massive amounts). He must pay a very big price for this,
as others have before him. This should never to happen again!!!''

Still, Mr. Bolton won the first round by defeating the Justice
Department's request for an order that he try to prevent further
dissemination of his book. The department had also claimed that such an
order could bind his publisher, Simon \& Schuster, and bookstores that
already have copies. The book goes on sale next week.

In a statement, Charles J. Cooper, a lawyer for Mr. Bolton, praised
Judge Lamberth's decision but took exception to the judge's strong
suggestion that his client had violated his agreement or published
classified information.

Judge Lamberth issued the ruling after
\href{https://www.nytimes.com/2020/06/19/us/politics/john-bolton-book-hearing.html}{holding
a public hearing on Friday} about the government's request in which he
had strongly signaled that he believed the Justice Department's request
for a temporary restraining order and preliminary injunction had come
too late to ensure that any classified information in the book would
remain secret.

Later on Friday, he held a closed-door hearing with government lawyers
to discuss their contention that classified information remains in the
manuscript --- including an exceptionally restricted kind that could
reveal closely held intelligence sources and methods --- even though the
National Security Council's top official for prepublication review had
told Mr. Bolton that she was satisfied with the edits he had made at her
request.

After her review was complete, the White House never sent a final
approval letter to Mr. Bolton, who told Simon \& Schuster to publish
anyway. But the White House, without telling Mr. Bolton, opened a second
review by a National Security Council official, Michael T. Ellis, who
claimed last week to have found at least six examples of classified
information in the manuscript.

Mr. Ellis had not received training in prepublication review until after
he analyzed Mr. Bolton's manuscript. Mr. Cooper, a lawyer for Mr.
Bolton, has accused the administration of politicizing the process as a
pretext to prevent his client from revealing embarrassing facts about
Mr. Trump.

But other national security officials have also said in declarations to
the court that they think classified information is in the book.

Judge Lamberth will also oversee the part of the lawsuit that seeks to
seize Mr. Bolton's proceeds for writing the book as a penalty for
purportedly breaching the agreements he signed as a condition of
receiving classified information to go through the prepublication review
process. Mr. Cooper has argued that Mr. Bolton lived up to them.

The judge wrote that after viewing classified declarations and
discussing them in the closed hearing, he was ``persuaded that defendant
Bolton likely jeopardized national security by disclosing classified
information in violation of his nondisclosure agreement obligations.''

Judge Lamberth wrote that if Mr. Bolton was dissatisfied with the delay,
he could have sued the government instead of unilaterally publishing. He
said Mr. Bolton had gambled and lost.

``This was Bolton's bet: If he is right and the book does not contain
classified information, he keeps the upside mentioned above; but if he
is wrong, he stands to lose his profits from the book deal, exposes
himself to criminal liability, and imperils national security,'' he
wrote. ``Bolton was wrong.''

But Mr. Cooper disagreed.

``We welcome today's decision by the court denying the government's
attempt to suppress Ambassador Bolton's book,'' he said. ``We
respectfully take issue, however, with the court's preliminary
conclusion at this early stage of the case that Ambassador Bolton did
not comply fully with his contractual prepublication obligation to the
government, and the case will now proceed to development of the full
record on that issue. The full story of these events has yet to be told
--- but it will be.''

The Justice Department's request for an order that Mr. Bolton attempt to
somehow pull back his book had raised broader constitutional alarms. The
Supreme Court has ruled that the First Amendment only rarely permits the
government to block people from printing information. Several news
organizations, including The New York Times, were among outsiders who
submitted briefs urging Judge Lamberth to reject the request.

But the court has upheld the constitutionality of the government using
the prepublication review system to ensure that there is no classified
information in writings by officials who had security clearances. The
court has affirmed the government's seizure of book advances and
royalties from officials who publish books without going through that
process.

The Knight First Amendment Institute at Columbia University has
\href{https://www.nytimes.com/2019/04/02/us/politics/prepublication-censorship-system.html}{filed
a lawsuit challenging that system} on behalf of a group of former
government officials, claiming that it is dysfunctional and puts too
much discretionary power in the hands of reviewing officials who can
abuse that authority to discriminate against politically disfavored
authors.

In a statement, Jameel Jaffer, the executive director of the institute,
praised Judge Lamberth's rejection of the request for an injunction but
criticized his positive discussion of the review system.

``The court was of course right to reject the government's request for a
prior restraint, especially because the injunction the government sought
here was broader than the one the Supreme Court rejected in the Pentagon
Papers case,'' he said. ``In other respects, though, the ruling is a
troubling reaffirmation of broad government power to censor in the name
of national security. The prepublication review system puts far too much
power in the hands of government censors, and reform of this
dysfunctional system is long overdue.''

Advertisement

\protect\hyperlink{after-bottom}{Continue reading the main story}

\hypertarget{site-index}{%
\subsection{Site Index}\label{site-index}}

\hypertarget{site-information-navigation}{%
\subsection{Site Information
Navigation}\label{site-information-navigation}}

\begin{itemize}
\tightlist
\item
  \href{https://help.nytimes.com/hc/en-us/articles/115014792127-Copyright-notice}{©~2020~The
  New York Times Company}
\end{itemize}

\begin{itemize}
\tightlist
\item
  \href{https://www.nytco.com/}{NYTCo}
\item
  \href{https://help.nytimes.com/hc/en-us/articles/115015385887-Contact-Us}{Contact
  Us}
\item
  \href{https://www.nytco.com/careers/}{Work with us}
\item
  \href{https://nytmediakit.com/}{Advertise}
\item
  \href{http://www.tbrandstudio.com/}{T Brand Studio}
\item
  \href{https://www.nytimes.com/privacy/cookie-policy\#how-do-i-manage-trackers}{Your
  Ad Choices}
\item
  \href{https://www.nytimes.com/privacy}{Privacy}
\item
  \href{https://help.nytimes.com/hc/en-us/articles/115014893428-Terms-of-service}{Terms
  of Service}
\item
  \href{https://help.nytimes.com/hc/en-us/articles/115014893968-Terms-of-sale}{Terms
  of Sale}
\item
  \href{https://spiderbites.nytimes.com}{Site Map}
\item
  \href{https://help.nytimes.com/hc/en-us}{Help}
\item
  \href{https://www.nytimes.com/subscription?campaignId=37WXW}{Subscriptions}
\end{itemize}
