Sections

SEARCH

\protect\hyperlink{site-content}{Skip to
content}\protect\hyperlink{site-index}{Skip to site index}

\href{https://myaccount.nytimes.com/auth/login?response_type=cookie\&client_id=vi}{}

\href{https://www.nytimes.com/section/todayspaper}{Today's Paper}

\href{/section/opinion}{Opinion}\textbar{}The Misunderstood, Maligned
Rattlesnake

\href{https://nyti.ms/316nHhs}{https://nyti.ms/316nHhs}

\begin{itemize}
\item
\item
\item
\item
\item
\end{itemize}

Advertisement

\protect\hyperlink{after-top}{Continue reading the main story}

\href{/section/opinion}{Opinion}

Supported by

\protect\hyperlink{after-sponsor}{Continue reading the main story}

\hypertarget{the-misunderstood-maligned-rattlesnake}{%
\section{The Misunderstood, Maligned
Rattlesnake}\label{the-misunderstood-maligned-rattlesnake}}

The beautiful creature in the flower bed was not a threat to us. It was
a gift.

\href{https://www.nytimes.com/by/margaret-renkl}{\includegraphics{https://static01.nyt.com/images/2017/04/08/opinion/margaret-renkl/margaret-renkl-thumbLarge-v2.png}}

By \href{https://www.nytimes.com/by/margaret-renkl}{Margaret Renkl}

Contributing Opinion Writer

\begin{itemize}
\item
  June 22, 2020
\item
  \begin{itemize}
  \item
  \item
  \item
  \item
  \item
  \end{itemize}
\end{itemize}

\includegraphics{https://static01.nyt.com/images/2020/06/22/opinion/22renkl1/merlin_173706246_a67f8a35-473f-41ea-9622-5aa850461077-articleLarge.jpg?quality=75\&auto=webp\&disable=upscale}

NASHVILLE --- The cabin my husband and I borrowed last week was first
built in Kentucky on our friends' family land more than 100 years ago.
The timbers are rough-hewed, and you can still see the bark on some of
the beams as the builders conserved every inch of the trees they felled
from a forest that must have seemed endless. Our friends dismantled the
cabin in 1987 and brought the timbers back to Tennessee, where they used
them to build a new cabin on the Cumberland Plateau. It is now perched
at the edge of a windswept bluff overlooking Lost Cove,
\href{https://www.landtrusttn.org/projects/lost-cove-sewanee-tn/}{one of
the most biodiverse places in the world}, right where the heavens come
together with the earth.

In the five days we spent there to celebrate our anniversary, we walked
in an endless garden --- Queen Anne's lace and forest tickseed, Carolina
horsenettle and narrowleaf vervain, annual fleabane and zigzag
spiderwort and oxeye daisies were all growing wild. The woods were
filled with songbirds: blue jays and goldfinches and tufted titmice and
chickadees and bluebirds and even the
\href{https://www.allaboutbirds.org/guide/Scarlet_Tanager/id}{secretive
scarlet tanager}. Tanagers generally keep to the treetops, but the trees
growing in the soil of Lost Cove reach up to the edge of the bluff. From
our own perch, we had a bird's-eye view of the trees.

We watched a pair of red-tailed hawks teaching their fledglings to hunt.
We listened to \href{https://www.youtube.com/watch?v=NqPPioNKIfo}{a
pileated woodpecker's wild cry} from the top of a dead tree and heard
\href{https://wildambience.com/wildlife-sounds/red-fox/}{a red fox
barking in the dark} as two barred owls called to each other:
\href{https://www.youtube.com/watch?v=y5zc-NHIipw}{\emph{Who, who, who
cooks for you?}} At Lost Cove, the nights are as beautiful as the days.
The fireflies come out to fill the forest just as the stars come out to
fill the skies.

The sound that woke me in the first stirrings of dawn one morning was
the cry of a small animal --- a mouse, perhaps, or a chipmunk --- in the
claws, or jaws, of a predator. It was a piteous sound coming from
directly beneath our window. The creature cried out, just once, and then
was silent.

I've mostly made peace with the fact that the peaceable kingdom is
anything but. All day long and all night long, too, every creature on
that bluff, like every creature deep in the cove itself and every
creature in my suburban yard in Nashville and every creature scurrying
down every city alleyway, is both trying to eat and trying not to be
eaten. An insect-eating scarlet tanager is not inherently less violent
than the owl that eats songbirds. A rabbit is not somehow ``better'' for
eating wildflowers than a fox is for eating rabbits. This is how the
natural world works, and there is no wishing it were otherwise. But
knowing about such suffering is not the same as being a witness to such
suffering, and I did not go back to sleep that early morning.

My ambivalence in this matter of mortality explains why I was both
completely fascinated and completely terrified by the small rattlesnake
my husband found curled up next to the front porch of the cabin later
that day. I was afraid, but I wasn't \emph{only} afraid. I was also a
little bit in love with the magnificent creature that was calmly
surveying us from behind a laurel, making not a sound.

\includegraphics{https://static01.nyt.com/images/2020/06/22/opinion/22renkl2/merlin_173714499_f7d1127e-f064-4dfb-824f-22983606f2af-articleLarge.jpg?quality=75\&auto=webp\&disable=upscale}

Really, what 32nd wedding anniversary would be complete without the
appearance of some perfectly timed memento mori, in this case a deadly
pit viper? Or without an ensuing marital debate?

``It has to be a copperhead,'' my husband said. ``They're all over the
place up here. Rattlesnakes are rare.''

``The markings are all wrong,'' I said. ``It has to be a timber
rattler.''

``It's way too small to be a timber rattler,'' my husband said.

``Rattlesnakes don't start out five feet long,'' I said.

Throughout this lengthy conversation, which I have truncated for the
sake of your sanity, the snake in question did not stir. It was utterly
motionless, so still it provided what my husband believed was
unassailable evidence of his point: ``If this is a rattlesnake, why
isn't it rattling?''

One of his former students settled the question after my husband texted
him a picture of the snake. Jackson Roberts is now a doctoral candidate
in herpetology at Louisiana State University, and he confirmed that we
had in fact been visited by a young timber rattlesnake. ``You were
really lucky to get to see one,'' he told us. ``They're very shy, and
they're becoming more rare as we clear out their habitat. And as people
kill them.''

I asked Mr. Roberts why the snake hadn't deployed its trademark warning
device. ``The rattle is a last-ditch defensive strategy against
predators,'' he said. ``They'd much rather hunker down and wait for
trouble to leave.''

To a rattlesnake, in other words, \emph{we} are the trouble. \emph{We}
are the predators.

Timber rattlesnakes are declining in many states,
\href{https://www.tn.gov/twra/wildlife/reptiles/snakes/timber-rattlesnake.html\#:~:text=The\%20Timber\%20Rattlesnake\%20is\%20is,the\%20end\%20of\%20the\%20tail.}{including
here in Tennessee}, and it's illegal to kill one. It's actually illegal
to kill any snake in Tennessee unless it poses a direct threat to you.
Thing is, there's never any reason to consider a snake a direct threat.
Unless you're the one posing a direct threat to the snake --- if, say,
you're trying to kill it --- a snake will simply sit quietly and wait
for you to go away.

Barely two days after this peaceful rattlesnake entered my ken and
installed itself in my dreams,
\href{https://www.oriannesociety.org/}{the Orianne Society}, a
conservation nonprofit based in Tiger, Ga., started a
\href{https://www.oriannesociety.org/uncategorized/rattlesnakes-protect-educate-conserve/}{new
initiative to celebrate rattlesnakes}. Every day for the month leading
up to World Snake Day on July 16, Orianne is
\href{https://www.youtube.com/channel/UCFTl5TKVlFzTw2I3J-U09Mg}{posting
clips on social media} of chief executive Chris Jenkins talking about
snake biology, safe hiking in rattlesnake country, what to do when you
encounter a snake, etc. --- basically anything that might encourage
people to stop killing snakes.

Image

A timber rattlesnake at a birthing site in the southern
Appalachians.Credit...Heidi Hall, via The Orianne Society

Image

A timber rattlesnake's rattle.Credit...Pete Oxford, via The Orianne
Society

``Rattlesnakes, and snakes in general, are the most misunderstood, the
most maligned, the most persecuted animals on the planet,'' Dr. Jenkins
said in a phone interview last week. ``One of the most important things
we can do for the conservation of any snake, and rattlesnakes in
particular, is education.''

A fear of rattlesnakes is not unfounded. My own cousin's maternal
grandfather died decades ago when he stepped on a rattlesnake too far
out in the woods to get medical help in time to treat the bite. To the
snake, he was a threat. To his family, that didn't matter. The only
thing that mattered to them was that he was dead.

But the truth is that an animal can be dangerous and still pose almost
no threat to people. According to Dr. Jenkins, snakebites are rare, and
up to 50 percent of rattlesnake strikes are ``dry bites'' in which the
snake doesn't actually inject venom. Nevertheless, our culture has
taught us to associate serpents not only with danger but also with evil.

The only antidote to these associations is information. ``Unless you're
actively trying to catch or kill a rattlesnake, the chance of being
bitten is very low,'' said Dr. Jenkins. ``Many more people die every
year from horses --- whether they get thrown off or they get kicked ---
than from snakes. Many more people die from bees and wasps. If you
encounter a rattlesnake, you should be excited. It's a symbol that
you're in a wild place, a special place.''

It's hard to imagine a time when rattlesnakes, no matter how shy and how
peaceful, will be welcomed without fear. But I like to think we can
still ``complicate our perceptions,'' as my friend Erica Wright writes
in her forthcoming book, ``Snake.'' Perhaps we can yet learn, as she
puts it, to ``recognize the grace alongside the fangs and venom.
Complicated. Sublime. Awful and beautiful at once.''

When we checked in the last light of day, our rattlesnake visitor was
still sitting quietly in the flower bed. By morning it was gone,
vanished into the dappled light of the forest or a shady crevice of that
ancient limestone bluff. We never saw it again.

Margaret Renkl is a contributing opinion writer who covers flora, fauna,
politics and culture in the American South. She is the author of the
book ``\href{https://milkweed.org/book/late-migrations}{Late Migrations:
A Natural History of Love and Loss}.''

\emph{The Times is committed to publishing}
\href{https://www.nytimes.com/2019/01/31/opinion/letters/letters-to-editor-new-york-times-women.html}{\emph{a
diversity of letters}} \emph{to the editor. We'd like to hear what you
think about this or any of our articles. Here are some}
\href{https://help.nytimes.com/hc/en-us/articles/115014925288-How-to-submit-a-letter-to-the-editor}{\emph{tips}}\emph{.
And here's our email:}
\href{mailto:letters@nytimes.com}{\emph{letters@nytimes.com}}\emph{.}

\emph{Follow The New York Times Opinion section on}
\href{https://www.facebook.com/nytopinion}{\emph{Facebook}}\emph{,}
\href{http://twitter.com/NYTOpinion}{\emph{Twitter (@NYTopinion)}}
\emph{and}
\href{https://www.instagram.com/nytopinion/}{\emph{Instagram}}\emph{.}

Advertisement

\protect\hyperlink{after-bottom}{Continue reading the main story}

\hypertarget{site-index}{%
\subsection{Site Index}\label{site-index}}

\hypertarget{site-information-navigation}{%
\subsection{Site Information
Navigation}\label{site-information-navigation}}

\begin{itemize}
\tightlist
\item
  \href{https://help.nytimes.com/hc/en-us/articles/115014792127-Copyright-notice}{©~2020~The
  New York Times Company}
\end{itemize}

\begin{itemize}
\tightlist
\item
  \href{https://www.nytco.com/}{NYTCo}
\item
  \href{https://help.nytimes.com/hc/en-us/articles/115015385887-Contact-Us}{Contact
  Us}
\item
  \href{https://www.nytco.com/careers/}{Work with us}
\item
  \href{https://nytmediakit.com/}{Advertise}
\item
  \href{http://www.tbrandstudio.com/}{T Brand Studio}
\item
  \href{https://www.nytimes.com/privacy/cookie-policy\#how-do-i-manage-trackers}{Your
  Ad Choices}
\item
  \href{https://www.nytimes.com/privacy}{Privacy}
\item
  \href{https://help.nytimes.com/hc/en-us/articles/115014893428-Terms-of-service}{Terms
  of Service}
\item
  \href{https://help.nytimes.com/hc/en-us/articles/115014893968-Terms-of-sale}{Terms
  of Sale}
\item
  \href{https://spiderbites.nytimes.com}{Site Map}
\item
  \href{https://help.nytimes.com/hc/en-us}{Help}
\item
  \href{https://www.nytimes.com/subscription?campaignId=37WXW}{Subscriptions}
\end{itemize}
