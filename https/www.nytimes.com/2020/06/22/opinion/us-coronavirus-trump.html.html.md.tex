Sections

SEARCH

\protect\hyperlink{site-content}{Skip to
content}\protect\hyperlink{site-index}{Skip to site index}

\href{https://myaccount.nytimes.com/auth/login?response_type=cookie\&client_id=vi}{}

\href{https://www.nytimes.com/section/todayspaper}{Today's Paper}

\href{/section/opinion}{Opinion}\textbar{}America Is Too Broken to Fight
the Coronavirus

\href{https://nyti.ms/3fSzJPl}{https://nyti.ms/3fSzJPl}

\begin{itemize}
\item
\item
\item
\item
\item
\item
\end{itemize}

Advertisement

\protect\hyperlink{after-top}{Continue reading the main story}

\href{/section/opinion}{Opinion}

Supported by

\protect\hyperlink{after-sponsor}{Continue reading the main story}

\hypertarget{america-is-too-broken-to-fight-the-coronavirus}{%
\section{America Is Too Broken to Fight the
Coronavirus}\label{america-is-too-broken-to-fight-the-coronavirus}}

No other developed country is doing so badly.

\href{https://www.nytimes.com/by/michelle-goldberg}{\includegraphics{https://static01.nyt.com/images/2018/04/02/opinion/michelle-goldberg/michelle-goldberg-thumbLarge.png}}

By \href{https://www.nytimes.com/by/michelle-goldberg}{Michelle
Goldberg}

Opinion Columnist

\begin{itemize}
\item
  June 22, 2020
\item
  \begin{itemize}
  \item
  \item
  \item
  \item
  \item
  \item
  \end{itemize}
\end{itemize}

\includegraphics{https://static01.nyt.com/images/2020/06/22/opinion/22goldbergWeb/merlin_173741667_a9a729a9-6b38-450f-9a96-5e342eb0d1ef-articleLarge.jpg?quality=75\&auto=webp\&disable=upscale}

Graphs of the coronavirus curves in Britain, Canada, Germany and Italy
look like mountains, with steep climbs up and then back down. The one
for America shows a fast climb up to a plateau. For a while, the number
of new cases in the U.S. was at least slowly declining. Now, according
to The Times, it's up a terrifying 22 percent over the last 14 days.

As Politico
\href{https://www.politico.com/news/2020/06/22/united-states-italy-traded-places-coronavirus-333122}{reported
on Monday}, Italy's coronavirus catastrophe once looked to Americans
like a worst-case scenario. Today, it said, ``America's new per capita
cases remain on par with Italy's worst day --- and show signs of rising
further.''

This is what American exceptionalism looks like under Donald Trump. It's
not just that the United States has the highest number of coronavirus
cases and deaths of any country in the world. Republican political
dysfunction has made a coherent campaign to fight the pandemic
impossible.

At the federal level as well as in many states, we're seeing a
combination of the blustering contempt for science that marks the
conservative approach to climate change and the
\href{https://www.nytimes.com/2020/05/05/opinion/coronavirus-deaths.html}{high
tolerance for carnage} that makes American gun culture unique.

The rot starts at the top. At the beginning of the crisis Trump acted as
if he could wish the coronavirus away, and after an interval when he at
least pretended to take it seriously, his administration has resumed a
posture of blithe denial.

The task force led by Mike Pence has
\href{https://www.cnn.com/2020/05/28/politics/donald-trump-coronavirus-task-force/index.html}{been
sidelined}, its members meeting only twice a week. Last Tuesday, the
vice president wrote an
\href{https://www.wsj.com/articles/there-isnt-a-coronavirus-second-wave-11592327890}{op-ed
essay in The Wall Street Journal} about how well things are going: ``We
are winning the fight against the invisible enemy,'' he claimed.

In an interview with Fox News's Sean Hannity last week, Trump said the
virus is
``\href{https://www.foxnews.com/media/trump-hannity-coronavirus-fading-away-tulsa-rally}{fading
away}.'' Speaking to
\href{https://www.wsj.com/articles/trump-talks-juneteenth-john-bolton-economy-in-wsj-interview-11592493771}{The
Journal}, he said that some people might be wearing masks only to show
their disapproval of him and suggested, contrary to all credible public
health guidance, that mask-wearing might increase people's risk of
infection. It's not surprising, then, that many people at his sad
Saturday rally in Tulsa, Okla. --- where coronavirus cases are spiking
--- went maskless.

Just a few weeks ago, panicked about occupying my kids through the
summer in a shut-down New York City, I thought about taking them to stay
with my retired parents in Arizona. Now, as New York gingerly reopens,
Arizona has become a hot spot --- which isn't stopping Trump from
holding a rally at a Phoenix megachurch on Tuesday. Cases are also
soaring in Texas, Florida and several other states. An epidemic that was
once concentrated in blue states is increasingly raging in red ones.

When coronavirus cases started exploding on the East Coast in March,
there were devastating failures by Democratic leaders. New York's
governor, Andrew Cuomo, not only forced nursing homes to take back
residents who'd been hospitalized for the coronavirus, he barred them
from testing the residents to see if they were still infected.

As
\href{https://www.propublica.org/article/fire-through-dry-grass-andrew-cuomo-saw-covid-19-threat-to-nursing-homes-then-he-risked-adding-to-it}{ProPublica
reported}, following Cuomo's order, ``Covid-19 tore through New York
state's nursing facilities, killing more than 6,000 people --- about 6
percent of its more than 100,000 nursing home residents.'' In Florida,
which prohibited such transfers, the virus has so far killed only 1.6
percent of nursing home residents.

Given how Cuomo's errors contributed to New York's catastrophe, it's
hard to say how much credit he deserves for eventually rising to the
occasion. Still, by the time New York's cases got to where Arizona's are
now, he at least understood that the state faced calamity and imposed
the lockdown that helped bring it back from the abyss.

Arizona, Florida and Texas, by contrast, aren't even doing simple things
like mandating mask-wearing. Worse, until last week, the governors of
Arizona and Texas prevented cities from instituting their own such
requirements.

So far, evidence about the role mass protests over police violence
played in
\href{https://www.latimes.com/california/story/2020-06-22/for-third-day-in-a-week-l-a-county-reports-more-than-2-000-new-coronavirus-cases}{coronavirus
spikes} is
\href{https://slate.com/news-and-politics/2020/06/protests-covid-outdoor-masks.html}{mixed},
but liberal support for the demonstrations solidified the conviction
among many conservatives that strict social distancing rules are a
hypocritical tool of social control. The paranoia and resentment that
have long been part of the culture of the modern right are now directed
at those warning about the ongoing dangers of the pandemic.

Across the country, public health workers have faced
\href{https://www.sfgate.com/bayarea/article/CA-health-directors-quit-amid-death-threats-15343863.php}{death
threats}, harassment and
\href{https://www.washingtonpost.com/health/amid-threats-and-political-pushback-public-health-officials-leaving-posts/2020/06/22/6075f7a2-b0cf-11ea-856d-5054296735e5_story.html}{armed
protesters at their homes}. No matter how bad things get in red America,
it's hard to imagine where the political will to contain the virus will
come from.

So while countries with competent leadership haltingly return to normal,
ours will continue to be pummeled. In mid-May, when America's
coronavirus death toll was around 85,000, Trump sycophant Lindsey Graham
said that as long as
\href{https://www.politico.com/news/2020/05/14/white-house-coronavirus-success-259792}{fatalities
didn't go much beyond 120,000}, ``I think you can say you limited the
casualties in this war.''

By The Times's count,
\href{https://www.nytimes.com/interactive/2020/us/coronavirus-us-cases.html\#states}{we
just hit that number}. The war goes on, but Trump has already lost it.

\emph{The Times is committed to publishing}
\href{https://www.nytimes.com/2019/01/31/opinion/letters/letters-to-editor-new-york-times-women.html}{\emph{a
diversity of letters}} \emph{to the editor. We'd like to hear what you
think about this or any of our articles. Here are some}
\href{https://help.nytimes.com/hc/en-us/articles/115014925288-How-to-submit-a-letter-to-the-editor}{\emph{tips}}\emph{.
And here's our email:}
\href{mailto:letters@nytimes.com}{\emph{letters@nytimes.com}}\emph{.}

\emph{Follow The New York Times Opinion section on}
\href{https://www.facebook.com/nytopinion}{\emph{Facebook}}\emph{,}
\href{http://twitter.com/NYTOpinion}{\emph{Twitter (@NYTopinion)}}
\emph{and}
\href{https://www.instagram.com/nytopinion/}{\emph{Instagram}}\emph{.}

Advertisement

\protect\hyperlink{after-bottom}{Continue reading the main story}

\hypertarget{site-index}{%
\subsection{Site Index}\label{site-index}}

\hypertarget{site-information-navigation}{%
\subsection{Site Information
Navigation}\label{site-information-navigation}}

\begin{itemize}
\tightlist
\item
  \href{https://help.nytimes.com/hc/en-us/articles/115014792127-Copyright-notice}{©~2020~The
  New York Times Company}
\end{itemize}

\begin{itemize}
\tightlist
\item
  \href{https://www.nytco.com/}{NYTCo}
\item
  \href{https://help.nytimes.com/hc/en-us/articles/115015385887-Contact-Us}{Contact
  Us}
\item
  \href{https://www.nytco.com/careers/}{Work with us}
\item
  \href{https://nytmediakit.com/}{Advertise}
\item
  \href{http://www.tbrandstudio.com/}{T Brand Studio}
\item
  \href{https://www.nytimes.com/privacy/cookie-policy\#how-do-i-manage-trackers}{Your
  Ad Choices}
\item
  \href{https://www.nytimes.com/privacy}{Privacy}
\item
  \href{https://help.nytimes.com/hc/en-us/articles/115014893428-Terms-of-service}{Terms
  of Service}
\item
  \href{https://help.nytimes.com/hc/en-us/articles/115014893968-Terms-of-sale}{Terms
  of Sale}
\item
  \href{https://spiderbites.nytimes.com}{Site Map}
\item
  \href{https://help.nytimes.com/hc/en-us}{Help}
\item
  \href{https://www.nytimes.com/subscription?campaignId=37WXW}{Subscriptions}
\end{itemize}
