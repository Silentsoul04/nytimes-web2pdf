Sections

SEARCH

\protect\hyperlink{site-content}{Skip to
content}\protect\hyperlink{site-index}{Skip to site index}

\href{https://www.nytimes.com/section/politics}{Politics}

\href{https://myaccount.nytimes.com/auth/login?response_type=cookie\&client_id=vi}{}

\href{https://www.nytimes.com/section/todayspaper}{Today's Paper}

\href{/section/politics}{Politics}\textbar{}The C.I.A.'s Business Is
Secrets, but It Is Recruiting Spies in the Open

\url{https://nyti.ms/2YrBvBs}

\begin{itemize}
\item
\item
\item
\item
\item
\end{itemize}

Advertisement

\protect\hyperlink{after-top}{Continue reading the main story}

Supported by

\protect\hyperlink{after-sponsor}{Continue reading the main story}

\hypertarget{the-cias-business-is-secrets-but-it-is-recruiting-spies-in-the-open}{%
\section{The C.I.A.'s Business Is Secrets, but It Is Recruiting Spies in
the
Open}\label{the-cias-business-is-secrets-but-it-is-recruiting-spies-in-the-open}}

The C.I.A. made its first television recruiting ad, now airing on
streaming services like Hulu, as it tries to build a better, more
diverse spy corps.

\includegraphics{https://static01.nyt.com/images/2020/06/22/us/politics/22dc-intel/merlin_166844961_dccecbb5-e07a-4b6a-95dd-05f30e3cd8d3-articleLarge.jpg?quality=75\&auto=webp\&disable=upscale}

\href{https://www.nytimes.com/by/julian-e-barnes}{\includegraphics{https://static01.nyt.com/images/2019/12/13/reader-center/author-julian-barnes/author-julian-barnes-thumbLarge.png}}

By \href{https://www.nytimes.com/by/julian-e-barnes}{Julian E. Barnes}

\begin{itemize}
\item
  June 22, 2020
\item
  \begin{itemize}
  \item
  \item
  \item
  \item
  \item
  \end{itemize}
\end{itemize}

WASHINGTON --- The C.I.A. has recruited at Ivy League schools, through
Hollywood-produced television programs and even by judging school
science fairs.

But the current era needs a modern recruiting drive, and on Monday, the
C.I.A. unveiled
\href{https://www.youtube.com/watch?v=KvG8c8aVtl8\&feature=youtu.be}{its
first television advertisement}, which is aimed at streaming platforms
like Hulu. The slick, advertising-agency-produced spot has the feel of
clips from the television program ``Homeland'' --- with a dollop of
patriotism.

By some measures, the C.I.A. has little need for recruiting drives.
Every year, thousands of applicants compete for hundreds of spots,
according to current and former officials. In 2019, the agency had its
best recruiting year in a decade. And traditionally it has been easier
for the government to recruit during recessions.

But Gina Haspel, the C.I.A. director, has made recruitment a priority
for her secretive agency, which has to compete against Silicon Valley
for the sharpest minds as it increasingly focuses on hacking and other
digital spying tools. And the agency still must work at bringing in
recruiting classes that reflect the diversity of the United States.

The agency has long produced recruitment videos, some of which are
posted on its
\href{https://www.youtube.com/channel/UClFKF4TkuGWFkkeMYYQoOmg}{YouTube
channel}. It has also made radio and online advertisements. The new ad
is the first meant for a television audience. The C.I.A. declined to say
what agency made it.

The agency is not giving up on recruiting seniors on college campuses.
But advertising on streaming platforms, the agency hopes, could get the
attention of a broader group of potential recruits.

``Americans are consuming streaming content now more than ever, and we
want to be part of what they're seeing,'' said Nicole de Haay, an agency
spokeswoman.

The agency has embraced a variety of marketing tools. It unveiled an
\href{https://www.instagram.com/cia/}{Instagram feed} last year, and it
made its website --- including recruitment information --- available
over the privacy-minded Tor network. It sends its scientists to judge
local science fairs in the Washington region. And it has long offered an
assist to Hollywood productions. (In 2004, Jennifer Garner, then
starring as a fictional C.I.A. agent in the television show ``Alias,''
filmed a \href{https://www.youtube.com/watch?v=eESpdvcyTUw}{recruitment
video for the agency.})

The new advertisement was written with heavy input by current C.I.A.
officers. While it leaves aside the drudgery that is part of any job,
former officials said the ad feels accurate in capturing some of the
most exciting moments of spy craft.

``There is an officer sitting at a desk and they say, `I think I found
something.' That gave me goose bumps,'' said Lisa Maddox, a former
C.I.A. analyst. ``Because I had those moments, and almost everyone I
know had them. When you feel you found that needle in a haystack, you
are so excited. It is a cool moment.''

Ms. Maddox said that recruiting midcareer employees is also critical for
the agency, but those prospects can be hard to reach. The new ad may be
a way to interest them in the agency.

The stereotype of a C.I.A. officer of a white man recruited from Yale
University contains a germ of truth --- you can find plenty of Yale
graduates among current and former agency personnel --- though even as
far back as the agency's founding in 1947, it recruited from 70 colleges
and universities. By the 1980s, William Casey, then the C.I.A. director,
made a priority of improving relations with the nation's colleges as the
agency began a push to diversify its ranks.

``The general supply always exceeds the demand,'' said Nicholas
Dujmovic, a former C.I.A. historian who is now a professor at the
Catholic University of America in Washington. **** ``But there are
skills they are always looking for, so getting those specific
individuals may be tough.''

The new advertisement, which uses actors, not C.I.A. officers, puts a
clear emphasis on diversity. It portrays a black senior official
addressing a class of new recruits and an African-American case officer
doing a secret brush pass to hand off a thumb drive. Officers of East
Asian and South Asian descent are also featured, as well as white
employees.

\href{https://www.cia.gov/library/reports/dls-report.pdf}{A 2015 report
commissioned by the C.I.A.} suggested that over the previous two
decades, the agency had become less diverse, with fewer senior black
officers in the agency.

That report helped prompt John O. Brennan, then the C.I.A. director, to
intensify the agency's recruitment efforts at historically black
colleges and universities.

``As an African-American C.I.A. officer, I can say in those years we
were doing a lot of great work to reach out to various communities,''
said Preston Golson, a former C.I.A. officer. ``But to build a pipeline,
it takes time. You don't see the results until a few years down the
road.''

Mr. Golson himself was recruited in a traditional way, interviewed as a
Harvard undergraduate and offered an internship in the summer of 2001.

Advertisement

\protect\hyperlink{after-bottom}{Continue reading the main story}

\hypertarget{site-index}{%
\subsection{Site Index}\label{site-index}}

\hypertarget{site-information-navigation}{%
\subsection{Site Information
Navigation}\label{site-information-navigation}}

\begin{itemize}
\tightlist
\item
  \href{https://help.nytimes.com/hc/en-us/articles/115014792127-Copyright-notice}{©~2020~The
  New York Times Company}
\end{itemize}

\begin{itemize}
\tightlist
\item
  \href{https://www.nytco.com/}{NYTCo}
\item
  \href{https://help.nytimes.com/hc/en-us/articles/115015385887-Contact-Us}{Contact
  Us}
\item
  \href{https://www.nytco.com/careers/}{Work with us}
\item
  \href{https://nytmediakit.com/}{Advertise}
\item
  \href{http://www.tbrandstudio.com/}{T Brand Studio}
\item
  \href{https://www.nytimes.com/privacy/cookie-policy\#how-do-i-manage-trackers}{Your
  Ad Choices}
\item
  \href{https://www.nytimes.com/privacy}{Privacy}
\item
  \href{https://help.nytimes.com/hc/en-us/articles/115014893428-Terms-of-service}{Terms
  of Service}
\item
  \href{https://help.nytimes.com/hc/en-us/articles/115014893968-Terms-of-sale}{Terms
  of Sale}
\item
  \href{https://spiderbites.nytimes.com}{Site Map}
\item
  \href{https://help.nytimes.com/hc/en-us}{Help}
\item
  \href{https://www.nytimes.com/subscription?campaignId=37WXW}{Subscriptions}
\end{itemize}
