Sections

SEARCH

\protect\hyperlink{site-content}{Skip to
content}\protect\hyperlink{site-index}{Skip to site index}

\href{https://www.nytimes.com/section/obituaries}{Obituaries}

\href{https://myaccount.nytimes.com/auth/login?response_type=cookie\&client_id=vi}{}

\href{https://www.nytimes.com/section/todayspaper}{Today's Paper}

\href{/section/obituaries}{Obituaries}\textbar{}Paulinho Paiakan,
Indigenous Defender of Rainforest, Dies at 67

\url{https://nyti.ms/3dvR2Em}

\begin{itemize}
\item
\item
\item
\item
\item
\end{itemize}

\href{https://www.nytimes.com/news-event/coronavirus?action=click\&pgtype=Article\&state=default\&region=TOP_BANNER\&context=storylines_menu}{The
Coronavirus Outbreak}

\begin{itemize}
\tightlist
\item
  live\href{https://www.nytimes.com/2020/08/03/world/coronavirus-covid-19.html?action=click\&pgtype=Article\&state=default\&region=TOP_BANNER\&context=storylines_menu}{Latest
  Updates}
\item
  \href{https://www.nytimes.com/interactive/2020/us/coronavirus-us-cases.html?action=click\&pgtype=Article\&state=default\&region=TOP_BANNER\&context=storylines_menu}{Maps
  and Cases}
\item
  \href{https://www.nytimes.com/interactive/2020/science/coronavirus-vaccine-tracker.html?action=click\&pgtype=Article\&state=default\&region=TOP_BANNER\&context=storylines_menu}{Vaccine
  Tracker}
\item
  \href{https://www.nytimes.com/2020/08/02/us/covid-college-reopening.html?action=click\&pgtype=Article\&state=default\&region=TOP_BANNER\&context=storylines_menu}{College
  Reopening}
\item
  \href{https://www.nytimes.com/live/2020/08/03/business/stock-market-today-coronavirus?action=click\&pgtype=Article\&state=default\&region=TOP_BANNER\&context=storylines_menu}{Economy}
\end{itemize}

Advertisement

\protect\hyperlink{after-top}{Continue reading the main story}

Supported by

\protect\hyperlink{after-sponsor}{Continue reading the main story}

Those We've Lost

\hypertarget{paulinho-paiakan-indigenous-defender-of-rainforest-dies-at-67}{%
\section{Paulinho Paiakan, Indigenous Defender of Rainforest, Dies at
67}\label{paulinho-paiakan-indigenous-defender-of-rainforest-dies-at-67}}

Mr. Paiakan became celebrated internationally for his campaigns in
Brazil, even appearing once with Sting. He died of Covid-19.

\includegraphics{https://static01.nyt.com/images/2020/06/24/obituaries/23Paiakan/merlin_173799282_b9e998ef-496f-4253-9db0-05415234e95b-articleLarge.jpg?quality=75\&auto=webp\&disable=upscale}

By Michael Astor

\begin{itemize}
\item
  June 23, 2020
\item
  \begin{itemize}
  \item
  \item
  \item
  \item
  \item
  \end{itemize}
\end{itemize}

\emph{This obituary is part of a series about people who have died in
the coronavirus pandemic. Read about others}
\href{https://www.nytimes.com/interactive/2020/obituaries/people-died-coronavirus-obituaries.html}{\emph{here}}\emph{.}

Paulinho Paiakan was working for the Brazilian government helping
surveyors plot the route of the
\href{https://www.youtube.com/watch?v=pDssyllVegw}{Trans-Amazon Highway}
through sometimes hostile Indigenous territory. Then the bulldozers and
paving machines arrived and he understood the destruction they would
bring. He quit his job to become one of the rainforest's staunchest and
best-known defenders.

As an example of an early exploit, Mr. Paiakan in the 1970s organized
hundreds of elaborately painted and feathered warriors to face down
wildcat miners and loggers intent on invading the lands of his Kayapo
people, who live in the Amazon states of Para and Mato Grosso. He was
instrumental in establishing his tribe's 11-million-acre reservation,
where some 9,000 Kayapo continue their traditional way of life with only
minor concessions to the modern world.

Mr. Paiakan later helped convince the government to shelve plans for a
hydroelectric dam in the region. He also took part in a successful
effort to introduce protections for Indigenous people into Brazil's 1988
Constitution, which remain some of the world's strongest.

Mr. Paiakan died on June 17 in a hospital in Redenção, in Brazil's Para
State. He was 67. The cause was Covid-19, his daughter Maial said.

As Brazil's Covid-19 case count has exceeded one million and deaths
there have surpassed 50,000, the country has also lost another
Indigenous Amazon leader to the virus in recent weeks. Messias Kokama,
an advocate of Indigenous people in cities, died on May 13.

Mr. Paiakan achieved a large measure of fame in the 1980s. He traveled
the world to warn of the dangers facing the Amazon and of their impact
on global warming, once appearing alongside the rock star Sting at a
gathering of Indigenous peoples. He extolled the rainforest's
biodiversity and organized a deal for the Kayapo to sell nut oils to the
Body Shop, making them one of the richest Indigenous groups in Brazil.
Ridley Scott was even slated to film a \$40 million biopic about him.

But on the eve of the 1992 Earth Summit in Rio de Janeiro, Mr. Paiakan
and his wife, Irekran,
\href{https://www.nytimes.com/1992/07/05/world/indian-white-rape-case-splits-brazil.html}{were
accused in connection with the rape} of a young woman. Supporters say
the charges were trumped up to silence him. He was eventually convicted
of (nonsexual) assault and sentenced to prison, which he avoided by
fleeing to his native village in the rainforest. His sentence was
commuted in 2006, but by then he had been left destitute and had ceased
to be an effective leader.

``I knew how to organize the people for the struggle, but after the
'90s, I left the organizing to others,'' he said in a television
interview in 2018. ``I used to ask to meet with officials and be well
received, without pepper spray to the face. Today any protest is met
with tear-gas bombs and pepper spray.''

According to his people's practice of using one name, he was born
Bepkaroroti, on April 19, 1953, in the village of Kubekrakej. His mother
was Ikekrote and his father was Tchikiri, the local chief. Paulinho
Paiakan was the name given him by Catholic missionaries who took an
interest in him and brought him to Altamira, a city some 500 miles
north, to study, leading him to become one of the first Kayapo to speak
Portuguese.

He is survived by his wife; four daughters, Maial, O-e, Tania and
Iremao, who were raised in Redenção so they could have a formal
education; and five grandchildren.

His body was flown back to his ancestral lands and buried to the
accompaniment of traditional songs and dancing on June 18, although some
aspects of the ceremony were scaled back to prevent the spread of
infection.

\href{https://www.nytimes.com/interactive/2020/obituaries/people-died-coronavirus-obituaries.html?action=click\&pgtype=Article\&state=default\&region=BELOW_MAIN_CONTENT\&context=covid_obits_promo}{}

\hypertarget{those-weve-lost}{%
\section{Those We've Lost}\label{those-weve-lost}}

The coronavirus pandemic has taken an incalculable death toll. This
series is designed to put names and faces to the numbers.

Read more

\includegraphics{https://static01.nyt.com/images/2020/07/30/obituaries/30Pedro/30Pedro-square640.jpg}

\hypertarget{bernaldina-josuxe9-pedro}{%
\section{Bernaldina José Pedro}\label{bernaldina-josuxe9-pedro}}

d. Boa Vista, Brazil

Leader among the Indigenous Macuxi

\includegraphics{https://static01.nyt.com/images/2020/07/31/obituaries/31Swing/merlin_175167783_8913bc90-0d64-43f3-a655-1bb1bf1601c9-square640.jpg}

\hypertarget{john-eric-swing}{%
\section{John Eric Swing}\label{john-eric-swing}}

d. Fountain Valley, Calif.

Champion of Filipino-Americans

\includegraphics{https://static01.nyt.com/images/2020/07/27/obituaries/27Victor/merlin_175001436_38b11f8e-227a-4e2c-9821-7618af9b2524-square640.jpg}

\hypertarget{victor-victor}{%
\section{Victor Victor}\label{victor-victor}}

d. Santo Domingo, Dominican Republic

Beloved musician of the Dominican Republic

\includegraphics{https://static01.nyt.com/images/2020/07/31/obituaries/31Negron/merlin_175160169_516322ae-fd23-4969-b6b2-193ced371105-square640.jpg}

\hypertarget{dr-eddie-negruxf3n}{%
\section{Dr. Eddie Negrón}\label{dr-eddie-negruxf3n}}

d. Fort Walton Beach, Fla.

Internist on Florida's Emerald Coast

\includegraphics{https://static01.nyt.com/images/2020/07/30/obituaries/30Dobson/merlin_175115928_f6b9271c-8f05-4fe1-a38a-5ca4a58f8935-square640.jpg}

\hypertarget{dobby-dobson}{%
\section{Dobby Dobson}\label{dobby-dobson}}

d. Coral Springs, Fla.

Jamaican singer and songwriter

\includegraphics{https://static01.nyt.com/images/2020/08/01/obituaries/28Gonzalez/merlin_175002771_beb57888-3951-409a-ae13-03a94b2e962e-square640.jpg}

\hypertarget{waldemar-gonzalez}{%
\section{Waldemar Gonzalez}\label{waldemar-gonzalez}}

d. White Plains, N.Y.

Teacher and social worker

Advertisement

\protect\hyperlink{after-bottom}{Continue reading the main story}

\hypertarget{site-index}{%
\subsection{Site Index}\label{site-index}}

\hypertarget{site-information-navigation}{%
\subsection{Site Information
Navigation}\label{site-information-navigation}}

\begin{itemize}
\tightlist
\item
  \href{https://help.nytimes.com/hc/en-us/articles/115014792127-Copyright-notice}{©~2020~The
  New York Times Company}
\end{itemize}

\begin{itemize}
\tightlist
\item
  \href{https://www.nytco.com/}{NYTCo}
\item
  \href{https://help.nytimes.com/hc/en-us/articles/115015385887-Contact-Us}{Contact
  Us}
\item
  \href{https://www.nytco.com/careers/}{Work with us}
\item
  \href{https://nytmediakit.com/}{Advertise}
\item
  \href{http://www.tbrandstudio.com/}{T Brand Studio}
\item
  \href{https://www.nytimes.com/privacy/cookie-policy\#how-do-i-manage-trackers}{Your
  Ad Choices}
\item
  \href{https://www.nytimes.com/privacy}{Privacy}
\item
  \href{https://help.nytimes.com/hc/en-us/articles/115014893428-Terms-of-service}{Terms
  of Service}
\item
  \href{https://help.nytimes.com/hc/en-us/articles/115014893968-Terms-of-sale}{Terms
  of Sale}
\item
  \href{https://spiderbites.nytimes.com}{Site Map}
\item
  \href{https://help.nytimes.com/hc/en-us}{Help}
\item
  \href{https://www.nytimes.com/subscription?campaignId=37WXW}{Subscriptions}
\end{itemize}
