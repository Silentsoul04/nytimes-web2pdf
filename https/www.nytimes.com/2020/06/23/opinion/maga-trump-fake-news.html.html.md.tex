Sections

SEARCH

\protect\hyperlink{site-content}{Skip to
content}\protect\hyperlink{site-index}{Skip to site index}

\href{https://myaccount.nytimes.com/auth/login?response_type=cookie\&client_id=vi}{}

\href{https://www.nytimes.com/section/todayspaper}{Today's Paper}

\href{/section/opinion}{Opinion}\textbar{}The Boy Who Cried Fake News

\href{https://nyti.ms/2VaL1GR}{https://nyti.ms/2VaL1GR}

\begin{itemize}
\item
\item
\item
\item
\item
\item
\end{itemize}

Advertisement

\protect\hyperlink{after-top}{Continue reading the main story}

\href{/section/opinion}{Opinion}

Supported by

\protect\hyperlink{after-sponsor}{Continue reading the main story}

\hypertarget{the-boy-who-cried-fake-news}{%
\section{The Boy Who Cried Fake
News}\label{the-boy-who-cried-fake-news}}

From inside the MAGA gates, Trump can't see how the world has changed.

\href{https://www.nytimes.com/column/jamelle-bouie}{\includegraphics{https://static01.nyt.com/images/2019/01/24/opinion/jamelle-bouie/jamelle-bouie-thumbLarge-v3.png}}

By \href{https://www.nytimes.com/column/jamelle-bouie}{Jamelle Bouie}

Opinion Columnist

\begin{itemize}
\item
  June 23, 2020
\item
  \begin{itemize}
  \item
  \item
  \item
  \item
  \item
  \item
  \end{itemize}
\end{itemize}

\includegraphics{https://static01.nyt.com/images/2020/06/23/opinion/23bouie1/merlin_173771754_1fc97149-836b-489a-a2c6-4021646fc6e2-articleLarge.jpg?quality=75\&auto=webp\&disable=upscale}

\href{https://cn.nytimes.com/opinion/20200624/maga-trump-fake-news/}{阅读简体中文版}\href{https://cn.nytimes.com/opinion/20200624/maga-trump-fake-news/zh-hant/}{閱讀繁體中文版}

If there's anything we've learned in the five years since Donald Trump
came down that escalator, it's that he cannot thrive without a constant
stream of attention, adulation and affirmation. It's why he's obsessed
with cable news and Fox in particular; why his cabinet meetings begin
with almost worshipful praise from each of his appointees; and why he's
constantly touting his sky-high support from other Republicans.

It's also why, on Saturday, he held an indoor rally in the midst of a
respiratory disease pandemic. ``I guarantee you after Saturday, if
everything goes well, he's going to be in a much better mood,'' an
unnamed Trump political adviser
\href{https://www.cnn.com/2020/06/19/politics/trump-turn-around-losing-campaign/index.html}{told
CNN} the day before the event. ``He believes that he needs to be out
there fighting and he feeds off the energy of the crowds.''

The president is plainly unable to handle bad news, or even the idea
that he isn't popular or well-liked. Someone who rejects the idea of
being rejected may, for example,
\href{https://twitter.com/realdonaldtrump/status/1275024974579982336?s=21}{believe
that voter fraud} is the only threat to his re-election. And he's
constructed a bubble, let's call it a safe space, in which he's
insulated from bad news, negative feedback and pretty much any kind of
criticism. The result is that he's unable to respond to a changing
national mood, unable to adjust to a public that wants more leadership
than spectacle.

\begin{quote}
Chatting about Trump's epistemic closure and anything else you want to
talk about! \href{https://t.co/L0xgboygJa}{https://t.co/L0xgboygJa}
\href{https://t.co/sE3jPqWRp5}{https://t.co/sE3jPqWRp5}

--- b-boy bouiebaisse (@jbouie)
\href{https://twitter.com/jbouie/status/1275477105204051969?ref_src=twsrc\%5Etfw}{June
23, 2020}
\end{quote}

We have plenty of evidence that Trump shields himself from anything that
could disrupt the illusion of popularity and success he's constructed
around himself. At the Tulsa rally, he
\href{https://www.politico.com/news/2020/06/22/white-house-testing-coronavirus-333803}{told}
his audience that, when faced with evidence of rising coronavirus
infection rates, he urged his team to reduce the rate of testing. ``When
you do testing to that extent, you're gonna find more people, you're
gonna find more cases. So I said to my people, `Slow the testing down,
please,''' he said. His press secretary says this was a ``comment that
he made in jest,'' but Trump has expressed similar sentiments in the
past. ``If we did very little testing, we wouldn't have the most
cases,''
\href{https://www.whitehouse.gov/briefings-statements/remarks-president-trump-vice-president-pence-meeting-governor-reynolds-iowa/}{he
said in May} during a meeting with Kim Reynolds, the governor of Iowa.
``So, in a way, by doing all of this testing, we make ourselves look
bad.''

Likewise, on the question of his campaign, the president's re-election
staffers know just what he wants to hear. They've either
\href{https://www.thedailybeast.com/trump-world-thrilled-that-their-terrible-poll-numbers-arent-worse}{downplayed
his poor numbers} --- telling him that the polls showing Joe Biden ahead
skew Democratic --- or challenged them outright. After a CNN poll found
him trailing Biden by 14 points, the Trump campaign
\href{https://www.cnn.com/2020/06/10/politics/trump-campaign-cnn-poll/index.html}{sent
a cease-and-desist letter} to Jeff Zucker, the president of the network,
demanding that he retract the poll and apologize for its release. The
poll, read the letter, is a ``stunt'' meant to ``cause voter
suppression, stifle momentum and enthusiasm for the president, and
present a false view generally of the actual support across America for
the president.''

The obvious problem with building a cocoon of praise and sycophancy
around oneself, as any failed authoritarian could explain, is that it
hinders one's ability to respond to conditions on the ground, whether
that's a pandemic or a presidential race. You can't change course if you
refuse to see what's happening in front of you.

Trump rejects his poor ratings on the pandemic and the protests ---
\href{https://abcnews.go.com/Politics/approval-trumps-coronavirus-response-underwater-returns-campaign-trail/story?id=71351241}{58
percent} of Americans disapprove of his handling of the coronavirus, and
\href{https://www.journalism.org/2020/06/12/majorities-of-americans-say-news-coverage-of-george-floyd-protests-has-been-good-trumps-public-message-wrong/}{60
percent} disapprove of his handling of the demonstrations to protest the
death of George Floyd --- and so he continues to do the things that have
placed him in a historically weak position for an incumbent president
seeking re-election.

If Trump were less cloistered, he might know that to improve his
prospects he has to speak to voters on the fence between him and Biden.
He has to address their fears and make a positive case for his
administration. But because he lives within the confines of a gated MAGA
community, Trump has no sense of what the skeptical public wants to
hear. This, too, was apparent at his Tulsa rally, where he spoke at
length about minor controversies --- his ability to drink a glass of
water with one hand, his ability to walk down a ramp --- that are almost
certainly irrelevant to everyone other than himself and his staunchest
supporters.

And it's not just Trump who is closed off from the rest of the world.
Republican officials across the country refuse to believe that the
president is on the path to defeat. ``The more bad things happen in the
country, it just solidifies support for Trump,'' one North Carolina
Republican Party county chairman told Politico in
\href{https://www.politico.com/news/2020/06/15/trump-glide-reelection-republican-officials-316457}{a
story} on the belief, within Republican circles, that ``coronavirus is
on its way out'' and ``polls are unreliable.''

We can't predict what will happen in November. But right now Trump is
losing the presidential race, Democrats are likely to hold the House of
Representatives, and Republicans are at risk of losing the Senate. A
backlash is brewing, and Trump can't see or even sense it. There are
those in his camp, like Mitch McConnell, the Republican majority leader
of the Senate, who can see the writing on the wall. This is presumably
why he's
\href{https://www.cbsnews.com/news/watch-live-mcconnell-tim-scott-senate-republican-police-reform-bill/}{pushing}
a police reform bill --- to give Republicans something to tout in the
fall to those moderate voters who sympathize with protesters.

But it's not clear if the conservative movement as a whole knows what it
has unleashed by hitching its wagon to Donald Trump. Conservatives
thought they were getting
``\href{https://www.whitehouse.gov/briefings-statements/remarks-president-trump-air-force-one-departure-15/}{252
beautiful, brand-new, conservative, wonderful judges}'' and a chance to
cement their political preferences into the constitutional order. What
they may receive instead is a newly energized and increasingly liberal
public that has the numbers to sink that project for at least the near
future, if not much longer.

\emph{The Times is committed to publishing}
\href{https://www.nytimes.com/2019/01/31/opinion/letters/letters-to-editor-new-york-times-women.html}{\emph{a
diversity of letters}} \emph{to the editor. We'd like to hear what you
think about this or any of our articles. Here are some}
\href{https://help.nytimes.com/hc/en-us/articles/115014925288-How-to-submit-a-letter-to-the-editor}{\emph{tips}}\emph{.
And here's our email:}
\href{mailto:letters@nytimes.com}{\emph{letters@nytimes.com}}\emph{.}

\emph{Follow The New York Times Opinion section on}
\href{https://www.facebook.com/nytopinion}{\emph{Facebook}}\emph{,}
\href{http://twitter.com/NYTOpinion}{\emph{Twitter (@NYTopinion)}}
\emph{and}
\href{https://www.instagram.com/nytopinion/}{\emph{Instagram}}\emph{.}

Advertisement

\protect\hyperlink{after-bottom}{Continue reading the main story}

\hypertarget{site-index}{%
\subsection{Site Index}\label{site-index}}

\hypertarget{site-information-navigation}{%
\subsection{Site Information
Navigation}\label{site-information-navigation}}

\begin{itemize}
\tightlist
\item
  \href{https://help.nytimes.com/hc/en-us/articles/115014792127-Copyright-notice}{©~2020~The
  New York Times Company}
\end{itemize}

\begin{itemize}
\tightlist
\item
  \href{https://www.nytco.com/}{NYTCo}
\item
  \href{https://help.nytimes.com/hc/en-us/articles/115015385887-Contact-Us}{Contact
  Us}
\item
  \href{https://www.nytco.com/careers/}{Work with us}
\item
  \href{https://nytmediakit.com/}{Advertise}
\item
  \href{http://www.tbrandstudio.com/}{T Brand Studio}
\item
  \href{https://www.nytimes.com/privacy/cookie-policy\#how-do-i-manage-trackers}{Your
  Ad Choices}
\item
  \href{https://www.nytimes.com/privacy}{Privacy}
\item
  \href{https://help.nytimes.com/hc/en-us/articles/115014893428-Terms-of-service}{Terms
  of Service}
\item
  \href{https://help.nytimes.com/hc/en-us/articles/115014893968-Terms-of-sale}{Terms
  of Sale}
\item
  \href{https://spiderbites.nytimes.com}{Site Map}
\item
  \href{https://help.nytimes.com/hc/en-us}{Help}
\item
  \href{https://www.nytimes.com/subscription?campaignId=37WXW}{Subscriptions}
\end{itemize}
