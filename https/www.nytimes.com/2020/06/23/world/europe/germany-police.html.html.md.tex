Sections

SEARCH

\protect\hyperlink{site-content}{Skip to
content}\protect\hyperlink{site-index}{Skip to site index}

\href{https://www.nytimes.com/section/world/europe}{Europe}

\href{https://myaccount.nytimes.com/auth/login?response_type=cookie\&client_id=vi}{}

\href{https://www.nytimes.com/section/todayspaper}{Today's Paper}

\href{/section/world/europe}{Europe}\textbar{}In Germany, Confronting
Shameful Legacy Is Essential Part of Police Training

\url{https://nyti.ms/3dqbnLi}

\begin{itemize}
\item
\item
\item
\item
\item
\end{itemize}

\href{https://www.nytimes.com/news-event/george-floyd-protests-minneapolis-new-york-los-angeles?action=click\&pgtype=Article\&state=default\&region=TOP_BANNER\&context=storylines_menu}{Race
and America}

\begin{itemize}
\tightlist
\item
  \href{https://www.nytimes.com/2020/07/26/us/protests-portland-seattle-trump.html?action=click\&pgtype=Article\&state=default\&region=TOP_BANNER\&context=storylines_menu}{Protesters
  Return to Other Cities}
\item
  \href{https://www.nytimes.com/2020/07/24/us/portland-oregon-protests-white-race.html?action=click\&pgtype=Article\&state=default\&region=TOP_BANNER\&context=storylines_menu}{Portland
  at the Center}
\item
  \href{https://www.nytimes.com/2020/07/23/podcasts/the-daily/portland-protests.html?action=click\&pgtype=Article\&state=default\&region=TOP_BANNER\&context=storylines_menu}{Podcast:
  Showdown in Portland}
\item
  \href{https://www.nytimes.com/interactive/2020/07/16/us/black-lives-matter-protests-louisville-breonna-taylor.html?action=click\&pgtype=Article\&state=default\&region=TOP_BANNER\&context=storylines_menu}{45
  Days in Louisville}
\end{itemize}

Advertisement

\protect\hyperlink{after-top}{Continue reading the main story}

Supported by

\protect\hyperlink{after-sponsor}{Continue reading the main story}

\hypertarget{in-germany-confronting-shameful-legacy-is-essential-part-of-police-training}{%
\section{In Germany, Confronting Shameful Legacy Is Essential Part of
Police
Training}\label{in-germany-confronting-shameful-legacy-is-essential-part-of-police-training}}

In the postwar era, Germany fundamentally redesigned law enforcement to
prevent past atrocities from ever repeating. Its approach may hold
lessons for police reform everywhere.

\includegraphics{https://static01.nyt.com/images/2020/06/20/world/20unrest-germanpolice01/merlin_170701797_c8da0262-fd65-461d-82d2-c040c1f255e1-articleLarge.jpg?quality=75\&auto=webp\&disable=upscale}

\href{https://www.nytimes.com/by/katrin-bennhold}{\includegraphics{https://static01.nyt.com/images/2018/07/13/multimedia/author-katrin-bennhold/author-katrin-bennhold-thumbLarge.png}}\href{https://www.nytimes.com/by/melissa-eddy}{\includegraphics{https://static01.nyt.com/images/2018/10/09/multimedia/author-melissa-eddy/author-melissa-eddy-thumbLarge.png}}

By \href{https://www.nytimes.com/by/katrin-bennhold}{Katrin Bennhold}
and \href{https://www.nytimes.com/by/melissa-eddy}{Melissa Eddy}

\begin{itemize}
\item
  June 23, 2020
\item
  \begin{itemize}
  \item
  \item
  \item
  \item
  \item
  \end{itemize}
\end{itemize}

BERLIN --- When Inspector Martin Halweg was a young cadet, his class met
a Holocaust survivor who had spent almost four years in a Berlin attic
hiding from the Nazis --- and from police officers like him.

``He described what it felt like running from the police, his fear, his
absolute terror,'' said Mr. Halweg, who was only 16 when he started
training in 1992.

Hearing this firsthand, he said, ``changes you as a person and changes
you as a police officer.''

Visiting a former concentration camp is mandatory for every future
police officer in Berlin. It is one of the ways in which policing was
fundamentally overhauled in Germany after World War II. Cadets are
taught in unsparing detail about the shameful legacy of policing under
the Nazis --- and how it informs the mission and institution of policing
today.

``After the war, we had to start from scratch,'' said Klaus Weinhauer, a
historian and police expert at Bielefeld University. ``The country had
to break with its history --- and so did the police.''

Germans have applied the lessons of their unique and horrid history to
every aspect of their postwar democracy, not least to how they police
their country. Those changes were partly imposed on Germany after the
war and took decades to work their way through attitudes and
institutions. But over time they have become pillars of German identity.

``You cannot compare the history of policing in America to the history
of German policing under the Nazis,'' said Mr. Weinhauer, the historian.

But as Americans debate the need to rethink their own law enforcement in
the wake of George Floyd's death under a white police officer's knee,
Germany's experience may offer insight into what it takes to redesign
institutions to prevent a painful past from repeating itself.

In Germany's case, the greatest preoccupation among the United States,
its Allies and Germans themselves was that the country's police force
never again be militarized, politicized and used as a cudgel by an
authoritarian state.

So they set out to fashion a postwar force with decentralized
responsibility to avoid letting a single agency become too powerful. The
privacy of citizens was rigorously protected, and the police and
military were strictly separated.

Even today, law enforcement in Germany is in the hands of the 16 states,
not the national government, a system that can be cumbersome and
imperfect, especially when dealing with modern-day challenges like
terrorism.

But its safeguards have earned the respect of Germans. Police officers
are required to pass a rigorous multiyear curriculum with history and
Germany's liberal democratic constitution at its core. The bedrock of
public safety in Germany is a strategy of communication and
de-escalation.

\hypertarget{there-is-no-german-fbi}{%
\subsection{There is no `German F.B.I.'}\label{there-is-no-german-fbi}}

\includegraphics{https://static01.nyt.com/images/2020/06/20/world/20unrest-germanpolice06/20unrest-germanpolice06-articleLarge.jpg?quality=75\&auto=webp\&disable=upscale}

Under the Nazis, the police were a central tool of an all-powerful
state. Police officers rounded up political enemies, deported Jews,
guarded ghettos --- and murdered. Some 30 police battalions helped kill
more than a million people on the eastern front.

After World War II, the Western Allies had three priorities in West
Germany, said Mr. Weinhauer, the historian: democratize, denazify and
demilitarize.

The most notorious branch of the police under the Nazis was the Gestapo,
the secret state police, which spied on citizens and also had the power
to arrest. In the Communist East, the Stasi later also combined both
powers into a terrifying instrument of oppression by the state.

Germany's democratic postwar constitution includes a strict separation
of powers: The Office for the Protection of the Constitution, the
country's domestic intelligence agency, cannot make arrests while the
police have limited surveillance powers. The two are barred from
exchanging information outside a dedicated counterterrorism forum.

``There is no such thing as an F.B.I. in Germany,'' Mr. Weinhauer said.

But, he added, ``It took many years until the police forces had learned
to critically reflect on their own involvement in the Holocaust.''

It wasn't until student protests in the 1960s and 1970s that German
society began talking more openly about its Nazi past. In institutions
like the police and the military, it took at least a decade longer.

In 1984, the Berlin police made a visit to the former Sachsenhausen
concentration camp north of the German capital mandatory for cadets. By
the time Inspector Halweg took his oath a decade later, the police
rented out an entire movie theater to show his graduating class
``Schindler's List.''

``It made me very conscious of the oath that I swore on our
constitution, to know what I stand for,'' Inspector Halweg said.

Opportunities to learn continue throughout a police career. The union
for Germany's federal police organizes two annual trips to Israel and
its Holocaust memorial, Yad Vashem.

\hypertarget{demilitarize-and-civilize}{%
\subsection{`Demilitarize and
civilize'}\label{demilitarize-and-civilize}}

Image

Police officers taking part in a training exercise outside a shopping
mall in Berlin in 2019.Credit...Odd Andersen/Agence France-Presse ---
Getty Images

In Georgia, becoming a police officer takes as little as 11 weeks.

``That's completely unthinkable in Germany,'' Inspector Halweg said.

Most U.S. police academies require only a high school diploma or
associate degree and courses rarely run longer than the six months
required in New York City.

Even the more elaborate training courses fall far short of Germany's
minimum standards in terms of entry requirements, length and intensity.

``Before they even start, applicants have to pass personality and
intelligence tests,'' said Margarete Koppers, Berlin's attorney general,
who previously ran the Berlin police force.

Once accepted, training in Germany takes at least two-and-a-half years
at an academy. Cadets are not just taught how to handle a gun but
obliged to take classes in law, ethics and police history. When they
graduate they are rewarded with high trust levels in society and civil
servant status that guarantees decent pay and job security.

In another postwar innovation, German police officers do not handle
minor infractions like parking tickets and noise ordinances, which are
handled by uniformed but unarmed city employees.

``This was an idea of the Allies, they wanted to demilitarize and
civilize police matters,'' said Ralf Poscher, director for the
department of public law at the Max Planck Institute for the Study of
Crime, Security and Law.

More than seven decades later, that early ambition of demilitarization
has morphed into a broad-based strategy of de-escalation that has become
the bedrock of modern German policing.

\hypertarget{a-monopoly-of-force}{%
\subsection{A `monopoly of force'}\label{a-monopoly-of-force}}

Image

Police officers taking their oath of allegiance to the state
constitution in Cologne.Credit...Sascha Steinbach/EPA, via Shutterstock

When Ms. Koppers, Berlin's attorney general, ran the Berlin police
force, she recalled the number of officers who were left traumatized
simply for having to draw their gun.

Gun ownership in Germany is low and shootings are rare.

``Violence is frowned upon in Germany,'' she said. ``Even drawing a gun
can lead to a police officer requesting psychological support.''

The police here have what Germans call ``a monopoly of force.'' Fewer
guns on the streets translate to a lower threat faced by officers, and a
lower number of people killed by police.

German police fatally shot 11 people and injured 34 while on duty in
2018, according to statistics compiled by the German Police Academy in
Münster.

In the United States, with a population four times that of Germany,
1,098 people were killed by police in 2019, according to
\href{https://mappingpoliceviolence.org/}{Mapping Police Violence}. In
Minnesota alone, where Mr. Floyd was killed, police fatally shot 13
people.

\hypertarget{work-in-progress}{%
\subsection{`Work in progress'}\label{work-in-progress}}

Image

German riot policemen in~Hamburg on July 6, 2017.~More than 150
complaints were filed against the police for excessive use of force that
day.~Credit...Omer Messinger/NurPhoto, via Getty Images

Despite the in-depth training, things can still go wrong. In July 2017,
police used pepper spray and billy clubs against demonstrators who had
targeted them with stones and bottles. More than 150 complaints were
filed against the police for excessive use of force, although the
majority of them were later dropped.

``It is a work in progress,'' said Mr. Poscher of the Max Planck
Institute for the Study of Crime, Security and Law.

And sometimes the very remedies to past problems breed new ones.

By creating a system of policing that aims to protect privacy and
prevent the institutional abuses of the Gestapo and the Stasi, German
law enforcement agencies have found that some of the safeguards have
undermined their ability to address challenges like terrorism ---
including far-right terrorism.

When a far-right terrorist network known as the National Socialist
Underground killed nine immigrants over seven years in the early 2000s,
the police assumed that it was other immigrants who had committed the
murders even as the intelligence agency had infiltrated the group.

``It is a challenge,'' said Ms. Koppers, Berlin's attorney general.
``How can we ensure that agencies communicate effectively despite the
deliberate decentralization of powers?''

And despite the active culture of remembrance, the German police are
themselves not immune to racism. Dozens of police officers have been
suspended on suspicion of links to neo-Nazi groups.

A number of migrants have died from excessive violence in police custody
in recent decades. In the wake of mass
\href{https://www.nytimes.com/news-event/george-floyd-protests-minneapolis-new-york-los-angeles?action=click\&pgtype=Article\&state=default\&module=styln-george-floyd\&variant=show\&region=TOP_BANNER\&context=storylines_menu}{protests
against racism and police violence} in the United States, German police
have also come under increased criticism for racial profiling.

Burak Yilmaz, a police academy educator and social worker, said that
since 2015, when hundreds of thousands of mostly Muslim migrants arrived
in Germany, racism had become far more prevalent.

One lesson from German history, Mr. Yilmaz said, is that before
institutions like the police can change a society's values must change.

``The police are a mirror of society,'' he said. ``You cannot turn the
police upside down and leave society as it is.''

\emph{Christopher F. Schuetze contributed reporting.}

Advertisement

\protect\hyperlink{after-bottom}{Continue reading the main story}

\hypertarget{site-index}{%
\subsection{Site Index}\label{site-index}}

\hypertarget{site-information-navigation}{%
\subsection{Site Information
Navigation}\label{site-information-navigation}}

\begin{itemize}
\tightlist
\item
  \href{https://help.nytimes.com/hc/en-us/articles/115014792127-Copyright-notice}{©~2020~The
  New York Times Company}
\end{itemize}

\begin{itemize}
\tightlist
\item
  \href{https://www.nytco.com/}{NYTCo}
\item
  \href{https://help.nytimes.com/hc/en-us/articles/115015385887-Contact-Us}{Contact
  Us}
\item
  \href{https://www.nytco.com/careers/}{Work with us}
\item
  \href{https://nytmediakit.com/}{Advertise}
\item
  \href{http://www.tbrandstudio.com/}{T Brand Studio}
\item
  \href{https://www.nytimes.com/privacy/cookie-policy\#how-do-i-manage-trackers}{Your
  Ad Choices}
\item
  \href{https://www.nytimes.com/privacy}{Privacy}
\item
  \href{https://help.nytimes.com/hc/en-us/articles/115014893428-Terms-of-service}{Terms
  of Service}
\item
  \href{https://help.nytimes.com/hc/en-us/articles/115014893968-Terms-of-sale}{Terms
  of Sale}
\item
  \href{https://spiderbites.nytimes.com}{Site Map}
\item
  \href{https://help.nytimes.com/hc/en-us}{Help}
\item
  \href{https://www.nytimes.com/subscription?campaignId=37WXW}{Subscriptions}
\end{itemize}
