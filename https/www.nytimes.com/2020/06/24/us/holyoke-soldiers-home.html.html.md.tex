Sections

SEARCH

\protect\hyperlink{site-content}{Skip to
content}\protect\hyperlink{site-index}{Skip to site index}

\href{https://www.nytimes.com/section/us}{U.S.}

\href{https://myaccount.nytimes.com/auth/login?response_type=cookie\&client_id=vi}{}

\href{https://www.nytimes.com/section/todayspaper}{Today's Paper}

\href{/section/us}{U.S.}\textbar{}`Total Pandemonium': What Went Wrong
at a Veterans' Home Where 76 Died

\url{https://nyti.ms/2B9yJrB}

\begin{itemize}
\item
\item
\item
\item
\item
\end{itemize}

\href{https://www.nytimes.com/news-event/coronavirus?action=click\&pgtype=Article\&state=default\&region=TOP_BANNER\&context=storylines_menu}{The
Coronavirus Outbreak}

\begin{itemize}
\tightlist
\item
  live\href{https://www.nytimes.com/2020/08/01/world/coronavirus-covid-19.html?action=click\&pgtype=Article\&state=default\&region=TOP_BANNER\&context=storylines_menu}{Latest
  Updates}
\item
  \href{https://www.nytimes.com/interactive/2020/us/coronavirus-us-cases.html?action=click\&pgtype=Article\&state=default\&region=TOP_BANNER\&context=storylines_menu}{Maps
  and Cases}
\item
  \href{https://www.nytimes.com/interactive/2020/science/coronavirus-vaccine-tracker.html?action=click\&pgtype=Article\&state=default\&region=TOP_BANNER\&context=storylines_menu}{Vaccine
  Tracker}
\item
  \href{https://www.nytimes.com/interactive/2020/07/29/us/schools-reopening-coronavirus.html?action=click\&pgtype=Article\&state=default\&region=TOP_BANNER\&context=storylines_menu}{What
  School May Look Like}
\item
  \href{https://www.nytimes.com/live/2020/07/31/business/stock-market-today-coronavirus?action=click\&pgtype=Article\&state=default\&region=TOP_BANNER\&context=storylines_menu}{Economy}
\end{itemize}

Advertisement

\protect\hyperlink{after-top}{Continue reading the main story}

Supported by

\protect\hyperlink{after-sponsor}{Continue reading the main story}

\hypertarget{total-pandemonium-what-went-wrong-at-a-veterans-home-where-76-died}{%
\section{`Total Pandemonium': What Went Wrong at a Veterans' Home Where
76
Died}\label{total-pandemonium-what-went-wrong-at-a-veterans-home-where-76-died}}

An independent report on the coronavirus outbreak at the Holyoke
Soldiers' Home describes ``the opposite of infection control,'' as
administrators combined wards of infected and uninfected patients.

\includegraphics{https://static01.nyt.com/images/2020/06/24/us/24virus-soldiershome/merlin_172465614_5ce4bf18-686d-4e8d-ac72-4eeab0a5a886-articleLarge.jpg?quality=75\&auto=webp\&disable=upscale}

\href{https://www.nytimes.com/by/ellen-barry}{\includegraphics{https://static01.nyt.com/images/2018/10/08/multimedia/author-ellen-barry/author-ellen-barry-thumbLarge.png}}

By \href{https://www.nytimes.com/by/ellen-barry}{Ellen Barry}

\begin{itemize}
\item
  Published June 24, 2020Updated June 26, 2020
\item
  \begin{itemize}
  \item
  \item
  \item
  \item
  \item
  \end{itemize}
\end{itemize}

BOSTON --- An investigation of 76 deaths linked to the coronavirus at
\href{https://www.nytimes.com/2020/05/24/us/they-survived-the-worst-battles-of-world-war-ii-and-died-of-the-virus.html}{a
state-run veterans' home} in Massachusetts paints a picture of a
facility in chaos, as traumatized nurses carried out orders to combine
wards of infected and uninfected men, knowing that the move would prove
deadly to many of their patients.

Workers at the facility, the Holyoke Soldiers' Home, remembered the days
in late March as ``total pandemonium'' and a ``nightmare.''

One social worker told investigators,
\href{https://www.mass.gov/doc/report-to-governor-baker-re-holyoke-soldiers-home/download}{in
a report released on Wednesday}, that she ``felt like it was moving the
concentration camp, we were moving these unknowing veterans off to
die.''

Another recalled sitting in a makeshift ward that was crowded with sick
and dying patients, some unclothed or without masks, and trying to
distract a man who was ``alert and oriented,'' chattering about the
Swedish meatballs his wife used to make.

``It was surreal,'' she said. ``I don't know how the staff over in that
unit, how many of us will ever recover from those images.''

Nursing home deaths have accounted for
\href{https://www.bostonglobe.com/2020/05/27/metro/first-time-state-divulges-death-toll-by-nursing-home-more-than-80-have-20-or-more-covid-19-deaths/}{more
than 60 percent} of the fatalities from the coronavirus in
Massachusetts, a state that prides itself on its health care system.
None of those deaths have received more attention than the cluster at
the Holyoke Soldiers' Home, which housed frail veterans of World War II
and other conflicts.

\hypertarget{latest-updates-global-coronavirus-outbreak}{%
\section{\texorpdfstring{\href{https://www.nytimes.com/2020/08/01/world/coronavirus-covid-19.html?action=click\&pgtype=Article\&state=default\&region=MAIN_CONTENT_1\&context=storylines_live_updates}{Latest
Updates: Global Coronavirus
Outbreak}}{Latest Updates: Global Coronavirus Outbreak}}\label{latest-updates-global-coronavirus-outbreak}}

Updated 2020-08-02T07:42:09.613Z

\begin{itemize}
\tightlist
\item
  \href{https://www.nytimes.com/2020/08/01/world/coronavirus-covid-19.html?action=click\&pgtype=Article\&state=default\&region=MAIN_CONTENT_1\&context=storylines_live_updates\#link-34047410}{The
  U.S. reels as July cases more than double the total of any other
  month.}
\item
  \href{https://www.nytimes.com/2020/08/01/world/coronavirus-covid-19.html?action=click\&pgtype=Article\&state=default\&region=MAIN_CONTENT_1\&context=storylines_live_updates\#link-780ec966}{Top
  U.S. officials work to break an impasse over the federal jobless
  benefit.}
\item
  \href{https://www.nytimes.com/2020/08/01/world/coronavirus-covid-19.html?action=click\&pgtype=Article\&state=default\&region=MAIN_CONTENT_1\&context=storylines_live_updates\#link-2bc8948}{Its
  outbreak untamed, Melbourne goes into even greater lockdown.}
\end{itemize}

\href{https://www.nytimes.com/2020/08/01/world/coronavirus-covid-19.html?action=click\&pgtype=Article\&state=default\&region=MAIN_CONTENT_1\&context=storylines_live_updates}{See
more updates}

More live coverage:
\href{https://www.nytimes.com/live/2020/07/31/business/stock-market-today-coronavirus?action=click\&pgtype=Article\&state=default\&region=MAIN_CONTENT_1\&context=storylines_live_updates}{Markets}

The 174-page independent report, led by the former federal prosecutor
Mark Pearlstein, blasts decisions made by the facility's superintendent,
Bennett Walsh, as ``utterly baffling from an infection-control
perspective.''

The report was especially scathing on the decision to combine crowded
wards. But it catalogs a series of other errors, including failure to
isolate infected veterans, failure to test veterans who had symptoms,
and the rotation of staff members between wards, accelerating the spread
of the virus.

``In short, this was the opposite of infection control: Mr. Walsh and
his team created close to an optimal environment for the spread of
Covid-19,'' the report said.

Gov. Charlie Baker of Massachusetts said on Wednesday that the accounts
in the report were ``one of the most depressing and utterly shameful
descriptions of what was supposed to be a care system that I have ever
heard of.''

The state is acting to fire Mr. Walsh,
\href{https://www.bostonglobe.com/2020/05/02/metro/holyoke-soldiers-home-has-been-overlooked-understaffed-led-by-inexperience/}{a
retired Marine Corps lieutenant colonel} with no previous nursing home
experience, the governor said. A lawyer for Mr. Walsh was not
immediately available for comment.

Mr. Walsh's supervisor, Francisco Urena, resigned from his post as the
state's secretary of veterans' services on Tuesday in anticipation of
the report. Mr. Baker said the secretary was asked to step down.

``Our administration did not do the job we should have done overseeing
Bennett Walsh and the Soldiers' Home,'' Mr. Baker said.

``I'm very sorry,'' Mr. Urena told a reporter for WCVB, a local
television station. ``I tried my best.''

The report suggests that Mr. Walsh, who left the military after a
distinguished 24-year career, was selected by the home's Board of
Trustees in part because his family is politically powerful in western
Massachusetts, and that state officials were not happy with his
management, which was marked by a high rate of staff turnover.

Mr. Walsh, through his lawyer, disputed many elements of the report. He
said it ``contains many baseless accusations that are immaterial to the
issues under considerations,'' and that he would review legal actions.

``It is clear that Mr. Walsh reached out for help when the crisis
erupted,'' said a statement from the lawyer, William Bennett. ``The
failure of the Commonwealth to affirmatively respond to that request
contributed to many of the problems outlined in the report.''

\href{https://www.nytimes.com/news-event/coronavirus?action=click\&pgtype=Article\&state=default\&region=MAIN_CONTENT_3\&context=storylines_faq}{}

\hypertarget{the-coronavirus-outbreak-}{%
\subsubsection{The Coronavirus Outbreak
›}\label{the-coronavirus-outbreak-}}

\hypertarget{frequently-asked-questions}{%
\paragraph{Frequently Asked
Questions}\label{frequently-asked-questions}}

Updated July 27, 2020

\begin{itemize}
\item ~
  \hypertarget{should-i-refinance-my-mortgage}{%
  \paragraph{Should I refinance my
  mortgage?}\label{should-i-refinance-my-mortgage}}

  \begin{itemize}
  \tightlist
  \item
    \href{https://www.nytimes.com/article/coronavirus-money-unemployment.html?action=click\&pgtype=Article\&state=default\&region=MAIN_CONTENT_3\&context=storylines_faq}{It
    could be a good idea,} because mortgage rates have
    \href{https://www.nytimes.com/2020/07/16/business/mortgage-rates-below-3-percent.html?action=click\&pgtype=Article\&state=default\&region=MAIN_CONTENT_3\&context=storylines_faq}{never
    been lower.} Refinancing requests have pushed mortgage applications
    to some of the highest levels since 2008, so be prepared to get in
    line. But defaults are also up, so if you're thinking about buying a
    home, be aware that some lenders have tightened their standards.
  \end{itemize}
\item ~
  \hypertarget{what-is-school-going-to-look-like-in-september}{%
  \paragraph{What is school going to look like in
  September?}\label{what-is-school-going-to-look-like-in-september}}

  \begin{itemize}
  \tightlist
  \item
    It is unlikely that many schools will return to a normal schedule
    this fall, requiring the grind of
    \href{https://www.nytimes.com/2020/06/05/us/coronavirus-education-lost-learning.html?action=click\&pgtype=Article\&state=default\&region=MAIN_CONTENT_3\&context=storylines_faq}{online
    learning},
    \href{https://www.nytimes.com/2020/05/29/us/coronavirus-child-care-centers.html?action=click\&pgtype=Article\&state=default\&region=MAIN_CONTENT_3\&context=storylines_faq}{makeshift
    child care} and
    \href{https://www.nytimes.com/2020/06/03/business/economy/coronavirus-working-women.html?action=click\&pgtype=Article\&state=default\&region=MAIN_CONTENT_3\&context=storylines_faq}{stunted
    workdays} to continue. California's two largest public school
    districts --- Los Angeles and San Diego --- said on July 13, that
    \href{https://www.nytimes.com/2020/07/13/us/lausd-san-diego-school-reopening.html?action=click\&pgtype=Article\&state=default\&region=MAIN_CONTENT_3\&context=storylines_faq}{instruction
    will be remote-only in the fall}, citing concerns that surging
    coronavirus infections in their areas pose too dire a risk for
    students and teachers. Together, the two districts enroll some
    825,000 students. They are the largest in the country so far to
    abandon plans for even a partial physical return to classrooms when
    they reopen in August. For other districts, the solution won't be an
    all-or-nothing approach.
    \href{https://bioethics.jhu.edu/research-and-outreach/projects/eschool-initiative/school-policy-tracker/}{Many
    systems}, including the nation's largest, New York City, are
    devising
    \href{https://www.nytimes.com/2020/06/26/us/coronavirus-schools-reopen-fall.html?action=click\&pgtype=Article\&state=default\&region=MAIN_CONTENT_3\&context=storylines_faq}{hybrid
    plans} that involve spending some days in classrooms and other days
    online. There's no national policy on this yet, so check with your
    municipal school system regularly to see what is happening in your
    community.
  \end{itemize}
\item ~
  \hypertarget{is-the-coronavirus-airborne}{%
  \paragraph{Is the coronavirus
  airborne?}\label{is-the-coronavirus-airborne}}

  \begin{itemize}
  \tightlist
  \item
    The coronavirus
    \href{https://www.nytimes.com/2020/07/04/health/239-experts-with-one-big-claim-the-coronavirus-is-airborne.html?action=click\&pgtype=Article\&state=default\&region=MAIN_CONTENT_3\&context=storylines_faq}{can
    stay aloft for hours in tiny droplets in stagnant air}, infecting
    people as they inhale, mounting scientific evidence suggests. This
    risk is highest in crowded indoor spaces with poor ventilation, and
    may help explain super-spreading events reported in meatpacking
    plants, churches and restaurants.
    \href{https://www.nytimes.com/2020/07/06/health/coronavirus-airborne-aerosols.html?action=click\&pgtype=Article\&state=default\&region=MAIN_CONTENT_3\&context=storylines_faq}{It's
    unclear how often the virus is spread} via these tiny droplets, or
    aerosols, compared with larger droplets that are expelled when a
    sick person coughs or sneezes, or transmitted through contact with
    contaminated surfaces, said Linsey Marr, an aerosol expert at
    Virginia Tech. Aerosols are released even when a person without
    symptoms exhales, talks or sings, according to Dr. Marr and more
    than 200 other experts, who
    \href{https://academic.oup.com/cid/article/doi/10.1093/cid/ciaa939/5867798}{have
    outlined the evidence in an open letter to the World Health
    Organization}.
  \end{itemize}
\item ~
  \hypertarget{what-are-the-symptoms-of-coronavirus}{%
  \paragraph{What are the symptoms of
  coronavirus?}\label{what-are-the-symptoms-of-coronavirus}}

  \begin{itemize}
  \tightlist
  \item
    Common symptoms
    \href{https://www.nytimes.com/article/symptoms-coronavirus.html?action=click\&pgtype=Article\&state=default\&region=MAIN_CONTENT_3\&context=storylines_faq}{include
    fever, a dry cough, fatigue and difficulty breathing or shortness of
    breath.} Some of these symptoms overlap with those of the flu,
    making detection difficult, but runny noses and stuffy sinuses are
    less common.
    \href{https://www.nytimes.com/2020/04/27/health/coronavirus-symptoms-cdc.html?action=click\&pgtype=Article\&state=default\&region=MAIN_CONTENT_3\&context=storylines_faq}{The
    C.D.C. has also} added chills, muscle pain, sore throat, headache
    and a new loss of the sense of taste or smell as symptoms to look
    out for. Most people fall ill five to seven days after exposure, but
    symptoms may appear in as few as two days or as many as 14 days.
  \end{itemize}
\item ~
  \hypertarget{does-asymptomatic-transmission-of-covid-19-happen}{%
  \paragraph{Does asymptomatic transmission of Covid-19
  happen?}\label{does-asymptomatic-transmission-of-covid-19-happen}}

  \begin{itemize}
  \tightlist
  \item
    So far, the evidence seems to show it does. A widely cited
    \href{https://www.nature.com/articles/s41591-020-0869-5}{paper}
    published in April suggests that people are most infectious about
    two days before the onset of coronavirus symptoms and estimated that
    44 percent of new infections were a result of transmission from
    people who were not yet showing symptoms. Recently, a top expert at
    the World Health Organization stated that transmission of the
    coronavirus by people who did not have symptoms was ``very rare,''
    \href{https://www.nytimes.com/2020/06/09/world/coronavirus-updates.html?action=click\&pgtype=Article\&state=default\&region=MAIN_CONTENT_3\&context=storylines_faq\#link-1f302e21}{but
    she later walked back that statement.}
  \end{itemize}
\end{itemize}

Staff members told investigators that they were initially discouraged
from wearing protective equipment, in an effort to conserve a limited
supply, and that they felt ``annoyed, paranoid and fearful for their
lives because they could not find masks,'' the report said.

The most troubling portions of the report describe the weekend of March
28 and 29, when staffing was so short at the home that two wards were
hurriedly combined, a decision one employee described as ``the most
insane thing I ever saw in my entire life.''

A social worker described listening to the chief nursing officer say
``something to the effect that this room will be dead by Sunday, so we
will have more room here.'' Another social worker recalled seeing a
supervisor point to a room and say, ``All this room will be dead by
tomorrow.''

Several staff members told investigators that, in the confusion, some of
the dying men did not receive adequate pain relief medication.

None of the facility's top administrators acknowledged taking part in
the decision to combine the two wards, and its medical director, David
Clinton, told investigators he was not consulted.

``We find this not to be credible, and at the very least, that Dr.
Clinton was aware (or should have been aware) of the move and did
nothing to stop it,'' the report said.

Val Liptak, the interim administrator brought in to manage the crisis,
told investigators that, though she and her team had a ``collective
90-plus years of nursing'' among them, ``none of us have ever seen
anything like this.'' The overcrowded ward, she said, ``looked like a
war zone.''

Among the disturbing revelations in the report was that supervisors had
instructed social workers to call the families of sick veterans and try
to persuade them to change their end-of-life health care preferences, so
that the veterans would not be transferred to a hospital.

One social worker said she stopped making those calls because ``it felt
wrong,'' as she put it, ``in the pit of my belly and heart.''

Advertisement

\protect\hyperlink{after-bottom}{Continue reading the main story}

\hypertarget{site-index}{%
\subsection{Site Index}\label{site-index}}

\hypertarget{site-information-navigation}{%
\subsection{Site Information
Navigation}\label{site-information-navigation}}

\begin{itemize}
\tightlist
\item
  \href{https://help.nytimes.com/hc/en-us/articles/115014792127-Copyright-notice}{©~2020~The
  New York Times Company}
\end{itemize}

\begin{itemize}
\tightlist
\item
  \href{https://www.nytco.com/}{NYTCo}
\item
  \href{https://help.nytimes.com/hc/en-us/articles/115015385887-Contact-Us}{Contact
  Us}
\item
  \href{https://www.nytco.com/careers/}{Work with us}
\item
  \href{https://nytmediakit.com/}{Advertise}
\item
  \href{http://www.tbrandstudio.com/}{T Brand Studio}
\item
  \href{https://www.nytimes.com/privacy/cookie-policy\#how-do-i-manage-trackers}{Your
  Ad Choices}
\item
  \href{https://www.nytimes.com/privacy}{Privacy}
\item
  \href{https://help.nytimes.com/hc/en-us/articles/115014893428-Terms-of-service}{Terms
  of Service}
\item
  \href{https://help.nytimes.com/hc/en-us/articles/115014893968-Terms-of-sale}{Terms
  of Sale}
\item
  \href{https://spiderbites.nytimes.com}{Site Map}
\item
  \href{https://help.nytimes.com/hc/en-us}{Help}
\item
  \href{https://www.nytimes.com/subscription?campaignId=37WXW}{Subscriptions}
\end{itemize}
