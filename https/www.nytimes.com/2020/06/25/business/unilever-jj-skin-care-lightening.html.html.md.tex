Sections

SEARCH

\protect\hyperlink{site-content}{Skip to
content}\protect\hyperlink{site-index}{Skip to site index}

\href{https://www.nytimes.com/section/business}{Business}

\href{https://myaccount.nytimes.com/auth/login?response_type=cookie\&client_id=vi}{}

\href{https://www.nytimes.com/section/todayspaper}{Today's Paper}

\href{/section/business}{Business}\textbar{}Criticism of Skin Lighteners
Brings Retreat by Unilever and Johnson \& Johnson

\url{https://nyti.ms/3eEKQeI}

\begin{itemize}
\item
\item
\item
\item
\item
\end{itemize}

Advertisement

\protect\hyperlink{after-top}{Continue reading the main story}

Supported by

\protect\hyperlink{after-sponsor}{Continue reading the main story}

\hypertarget{criticism-of-skin-lighteners-brings-retreat-by-unilever-and-johnson--johnson}{%
\section{Criticism of Skin Lighteners Brings Retreat by Unilever and
Johnson \&
Johnson}\label{criticism-of-skin-lighteners-brings-retreat-by-unilever-and-johnson--johnson}}

The companies have responded to a new wave of criticism calling beauty
products that advocate lighter skin discriminatory.

\includegraphics{https://static01.nyt.com/images/2020/06/26/business/25UNREST-SKINCARE1-print/25UNREST-SKINCARE-01-articleLarge.jpg?quality=75\&auto=webp\&disable=upscale}

By \href{https://www.nytimes.com/by/priya-arora}{Priya Arora} and
\href{https://www.nytimes.com/by/sapna-maheshwari}{Sapna Maheshwari}

\begin{itemize}
\item
  June 25, 2020
\item
  \begin{itemize}
  \item
  \item
  \item
  \item
  \item
  \end{itemize}
\end{itemize}

As major consumer products companies have rushed to declare their
opposition to racism in response to the national outrage over the
killing of George Floyd, many of them have been accused of openly
promoting a beauty standard rooted in racism and discrimination.

Unilever, Procter \& Gamble, L'Oreal and Johnson \& Johnson --- some of
the world's biggest advertisers --- sell beauty products that advocate
lighter, whiter skin in Africa, Asia and the Middle East. Those products
are not marketed in the United States, but the sales of the skin
lighteners have drawn criticism, especially from South Asians, for
perpetuating colorism --- the term describing the preference for lighter
skin --- in other countries, under popular brand names like Pond's,
Olay, Garnier and Neutrogena, and their own labels like Fair \& Lovely.

The
\href{https://www.nytimes.com/2020/06/13/us/george-floyd-racism-america.html}{backlash}
appears to be forcing action. Unilever
\href{https://www.unilever.com/news/press-releases/2020/unilever-evolves-skin-care-portfolio-to-embrace-a-more-inclusive-vision-of-beauty.html}{said
on Thursday} that it would remove the words ``fair/fairness,
white/whitening, and light/lightening'' from product packaging and
communications and change the name of its Fair \& Lovely brand, a
juggernaut in India that has marketed lighter skin as a path to success
for decades. Unilever has also sold skin lightening products through
Pond's and Vaseline.

\includegraphics{https://static01.nyt.com/images/2020/06/26/business/25unrest-skincare2-print/25unrest-skincare-02-articleLarge.jpg?quality=75\&auto=webp\&disable=upscale}

``We recognize that the use of the words `fair', `white' and `light'
suggest a singular ideal of beauty that we don't think is right, and we
want to address this,'' said Sunny Jain, Unilever's president of beauty
and personal care.

The move followed Johnson \& Johnson's
\href{https://www.nytimes.com/2020/06/19/business/johnson-and-johnson-skin-whitening-cream.html}{announcement}last
week, following questions from
\href{https://www.buzzfeednews.com/article/meghara/skin-lightening-cream-black-lives-matter-companies}{BuzzFeed
News}, that it would no longer sell its Neutrogena Fine Fairness and
Clean \& Clear Fairness lines, which were advertised as dark-spot
reducers but also used to lighten skin.

Procter \& Gamble, which sells similar products under its Olay brand,
declined to comment. L'Oreal, whose
\href{https://www.garnier.in/shop-products/whitening}{Garnier site in
India} includes a ``whitening'' section for men's face wash under the
PowerWhite brand, did not return requests for comment.

The focus on skin lightening products has emerged as U.S. brands reckon
with their use of racial stereotypes involving black Americans on
popular products. The owners of Cream of Wheat, Uncle Ben's rice and
Mrs. Butterworth's syrup
\href{https://www.nytimes.com/2020/06/17/business/aunt-jemima-mrs-butterworth-uncle-ben.html}{said
this month} that they would review how the brands' products are
packaged. That came after Quaker Oats said it would retire its
\href{https://www.nytimes.com/2020/06/17/business/aunt-jemima-racial-stereotype.html}{Aunt
Jemima brand} after acknowledging that its logo, a grinning black woman,
was based on a racial stereotype.

In South Asia, anti-blackness and colorism have origins that predate
colonialism and systemically reinforce differences in caste and class.

\href{https://www.nytimes.com/2020/06/18/dining/padma-lakshmi-taste-the-nation.html?searchResultPosition=1}{Padma
Lakshmi}, the longtime host and executive producer of ``Top Chef,''
recently
\href{https://twitter.com/PadmaLakshmi/status/1270398009692610563}{spoke
out against such products}, and in a phone interview said that colorism
permeated her childhood in India. Once she got to the United States, she
said, ``I never felt that I was on any comparable level to my white
peers because of my skin color, because I got that message in so many
different ways, subtle and overt.''

It came, she said, through ads, magazine covers and even family advice
to stay out of the sun, reinforcing that she was too dark to be
desirable. ``These things also come at you at a very vulnerable time in
your life when you're worried about your appearance,'' she said. ``It's
really a shame to have this added baggage of wanting to be something
you're not.'' Ms. Lakshmi added that it took leaving the United States
for Europe for a time during her modeling career for her to learn her
complexion could be seen as attractive.

The preference for fairness in South Asia is also perpetuated through
Bollywood movies and celebrities. The industry has also long favored
lighter-skinned actors and has employed brownface in several films.

Priyanka Chopra Jonas is among several Bollywood celebrities who have
faced an
\href{https://www.thejuggernaut.com/skin-whitening-industry}{online
backlash} recently over their previous endorsements of fairness creams.
Her recent social media posts calling attention to Mr. Floyd's killing
and the Black Lives Matter movement have prompted people to circulate
images of her advertising fairness creams from Garnier.

Though Ms. Chopra Jonas has mentioned in previous interviews that she
regretted advertising such products, she has not responded to the wave
of criticism this month, and other Bollywood stars who have also
promoted similar products have remained largely silent.

Image

Nina Davuluri, the first Indian-American to win the Miss America pageant
in 2014, has been working to fight colorism.Credit...Mike Coppola/Getty
Images for Opening Act

Skin-lightening products are estimated to be a multibillion-dollar
market, though its precise size is difficult to estimate, particularly
as brands alter the language on their products to less overtly promote
changes in skin tone. Fair \& Lovely, for example, will continue to be
sold, though Unilever noted it had removed ``before-and-after
impressions and shade guides that could indicate a transformation.''

Nina Davuluri, the first Indian-American to win the Miss America pageant
in 2014, has been working to fight colorism for years --- particularly
after she woke up the day after her victory and said she read an Indian
newspaper headline that said: ``Is Miss America too dark to be Miss
India?'' She has been working on a documentary since 2018 that set out
to explore why so many cultures believe lighter skin is better and how
it affects people's lives.

``Colorism is a form of racism --- not all of it, but part of it,'' Ms.
Davuluri said. ``Ultimately, companies are creating these products that
do prey on these archaic notions of colorism and they're also paying
celebrities millions of dollars to advertise for these whitening
products.''

\href{https://www.change.org/p/end-colorism-stop-the-production-of-skin-whitening-products-toxic-messaging?utm_source=share_petition\&utm_medium=custom_url\&recruited_by_id=f6b0b9d0-a7f3-11ea-9da3-0162547e0ce4}{She
created a petition} this month calling on Procter \& Gamble, Unilever,
L'Oreal and Johnson \& Johnson to stop making skin whitening products
and what she deemed racist ads, and instead create inclusive products.

``You have to have accountability to recognize that you can't just say
this in one part of the world --- it really has to be a holistic
standpoint from your entire company,'' she said, referring to the
companies' public statements about equality.

In India, colorism has also long been reinforced by a much older
tradition: matrimonial ads. Alongside categories like education and
caste, skin-tone options like ``fair,'' ``dusky,'' and ``wheatish''
would often appear in newspaper advertisements as parents sought matches
for their children.

\href{http://shaadi.com/}{Shaadi.com}, one of the world's largest
matrimonial sites, recently came under fire after a user, Meghan Nagpal,
discovered the ``skin tone'' filter on the site. The company
\href{https://twitter.com/MissRoshni/status/1270921151687069696}{initially}said
it was simply providing a service many parents wanted, prompting outrage
in a Facebook group for South Asian women. One of the women, Hetal
Lakhani,
\href{https://www.change.org/p/shaadi-com-remove-the-colour-filter-from-matrimonial-website}{started
a petition}, which led to the site
\href{https://twitter.com/ShaadiDotCom/status/1271024481167831040}{taking
down} the filter.

The filter was ``non-functional and barely used,'' the company said in
statement. ``We do not discriminate based on skin color and our member
base is as diverse and pluralistic as the world today is.''

The recent changes give Ms. Lakshmi hope that things are moving in the
right direction.

``Things are getting better,'' she said. ``Ending these creams, stopping
the advertisements, and just not referring to people based on these
kinds of things is going to go a long way. Hopefully, my daughter's
generation will grow up free from the shackles of color prejudice, at
least to some degree.''

Contact Priya Arora at
\href{mailto:priya.arora@nytimes.com}{\nolinkurl{priya.arora@nytimes.com}}
and Sapna Maheshwari at
\href{mailto:sapna@nytimes.com}{\nolinkurl{sapna@nytimes.com}}.

Advertisement

\protect\hyperlink{after-bottom}{Continue reading the main story}

\hypertarget{site-index}{%
\subsection{Site Index}\label{site-index}}

\hypertarget{site-information-navigation}{%
\subsection{Site Information
Navigation}\label{site-information-navigation}}

\begin{itemize}
\tightlist
\item
  \href{https://help.nytimes.com/hc/en-us/articles/115014792127-Copyright-notice}{©~2020~The
  New York Times Company}
\end{itemize}

\begin{itemize}
\tightlist
\item
  \href{https://www.nytco.com/}{NYTCo}
\item
  \href{https://help.nytimes.com/hc/en-us/articles/115015385887-Contact-Us}{Contact
  Us}
\item
  \href{https://www.nytco.com/careers/}{Work with us}
\item
  \href{https://nytmediakit.com/}{Advertise}
\item
  \href{http://www.tbrandstudio.com/}{T Brand Studio}
\item
  \href{https://www.nytimes.com/privacy/cookie-policy\#how-do-i-manage-trackers}{Your
  Ad Choices}
\item
  \href{https://www.nytimes.com/privacy}{Privacy}
\item
  \href{https://help.nytimes.com/hc/en-us/articles/115014893428-Terms-of-service}{Terms
  of Service}
\item
  \href{https://help.nytimes.com/hc/en-us/articles/115014893968-Terms-of-sale}{Terms
  of Sale}
\item
  \href{https://spiderbites.nytimes.com}{Site Map}
\item
  \href{https://help.nytimes.com/hc/en-us}{Help}
\item
  \href{https://www.nytimes.com/subscription?campaignId=37WXW}{Subscriptions}
\end{itemize}
