Sections

SEARCH

\protect\hyperlink{site-content}{Skip to
content}\protect\hyperlink{site-index}{Skip to site index}

\href{https://myaccount.nytimes.com/auth/login?response_type=cookie\&client_id=vi}{}

\href{https://www.nytimes.com/section/todayspaper}{Today's Paper}

\href{/section/opinion}{Opinion}\textbar{}You're Probably Inhaling
Microplastics Right Now

\href{https://nyti.ms/3fVmIVt}{https://nyti.ms/3fVmIVt}

\begin{itemize}
\item
\item
\item
\item
\item
\end{itemize}

Advertisement

\protect\hyperlink{after-top}{Continue reading the main story}

\href{/section/opinion}{Opinion}

Supported by

\protect\hyperlink{after-sponsor}{Continue reading the main story}

\hypertarget{youre-probably-inhaling-microplastics-right-now}{%
\section{You're Probably Inhaling Microplastics Right
Now}\label{youre-probably-inhaling-microplastics-right-now}}

A new study found plentiful evidence of these tiny particles in dust in
the nation's most remote places.

By Janice Brahney

Dr. Brahney is a biogeochemist at Utah State University.

\begin{itemize}
\item
  June 25, 2020
\item
  \begin{itemize}
  \item
  \item
  \item
  \item
  \item
  \end{itemize}
\end{itemize}

\includegraphics{https://static01.nyt.com/images/2020/06/25/opinion/25brahney/25brahney-mediumSquareAt3X.jpg}

LOGAN, Utah --- We weren't looking for what we found.

My research group was trying to determine how much phosphorous was being
carried by wind and rain into some of the most remote regions of the
West and how this nutrient might affect lakes and streams. To do so, we
sampled dust in 11 scattered locations, from Joshua Tree National Park
in California to the Wind River Range in Wyoming.

Back at the lab, peering through microscopes at our samples, we could
see pollen, insect parts and bits of minerals --- all of which would
have made it just another day in the life of a dust scientist. But what
made it different was an unexpected interloper: tiny bits of plastic,
most from synthetic microfibers used for making clothing. They were in
all of our samples. And lots of them.

There was so much microplastic,
\href{https://science.sciencemag.org/content/368/6496/1257}{we
calculated} that up to 6 percent of the dusts in those far-flung
locations are microplastics and that more than 1,000 metric tons are
deposited in those places every year by wind and rain. Some blew in from
nearby cities, but most came from much farther away and represented
decades of plastic waste. Four colleagues and I recently published
\href{https://science.sciencemag.org/content/368/6496/1257}{our
findings} in the journal Science.

This waste has become so ubiquitous that it's now in the air we breathe.
Airborne microplastics don't care what ZIP code you live in. Preventing
a landfill in your community won't limit your exposure. And there are
still many questions. If dust in the Grand Canyon contains
microplastics, how many of these tiny plastic particles are in city
dust? How high will airborne concentrations of microplastics get? What
effect are they having on the environment? Are microplastics more toxic
than other, better-understood sources of air pollution such as natural
and industrial dusts?

We know that inhaled plastics can produce inflammation and lesions in
lungs, and repeated exposure is suspected of leading to respiratory
problems like asthma and cancer. Inhaling microplastics may also
increase exposure to other toxic substances and coatings associated with
plastics and their manufacture.

Natural dust, which include dusts generated by humans, and industrial
dusts can also contain dangerous components, like the pathogen
Coccidioides, a soil-borne fungus that causes
\href{https://www.mayoclinic.org/diseases-conditions/valley-fever/symptoms-causes/syc-20378761}{valley
fever}, which can produce flulike symptoms. Industrial, urban and
agricultural dusts often contain heavy metals as well as synthetic
toxins. Outdoor air pollution causes roughly
\href{https://www.who.int/mediacentre/news/releases/2014/air-pollution/en/}{seven
million premature deaths a year} and is associated with
\href{https://www.nejm.org/doi/full/10.1056/NEJM199312093292401}{pulmonary
diseases}, even when adjusted for underlying risk factors. Those
statistics most likely include some of the effects of plastic. That we
can breathe in microplastics
\href{https://cebp.aacrjournals.org/content/cebp/7/5/419.full.pdf}{has
been known} for decades. We just haven't fully appreciated the scale of
the problem.But as Steve Allen, who does research on microplastics at
the University of Strathclyde in Glasgow, Scotland,
\href{https://www.washingtonpost.com/weather/2020/06/14/national-parks-deep-sea-plastic-pollution-is-showing-up-wherever-scientists-look/}{put
it recently} to The Washington Post, ``It is hard to imagine a sentence
starting with: `The health benefits of breathing airborne microplastic
\ldots.'''

We shouldn't be surprised by these findings. In 2018, about
\href{https://www.statista.com/statistics/282732/global-production-of-plastics-since-1950/}{359
million metric tons} of plastics were produced worldwide. Plastics are
useful, of course, and we need them for medicine, food safety and
technology. But do we really need plastic lawn decorations for every
holiday? The plastic pollution crisis seems to have as much to do with
industry as it does with consumer choices. A
\href{https://advances.sciencemag.org/content/3/7/e1700782.full}{2017
study} in the journal Science Advances estimated that ``if current
production and waste management trends continue, roughly 12 billion
metric tons of plastic waste will be in landfills or in the natural
environment by 2050.''

Movements against plastics pollution have led to bans on plastic straws
and plastic bags, and microbeads in cosmetics. But airborne
microplastics mostly come from clothing, car tires and the fragmentation
of commodities and packaging used briefly and then thrown away,
sometimes decades ago.

Reducing plastic waste means taking aim at consumer comfort and
convenience, and offering sustainable alternatives to plastics for those
on all rungs of the economic ladder.

The path forward to cleaning up this problem is not clear but
undoubtedly will require sweeping and uncomfortable changes. Taking on
this issue requires understanding it, and as our findings underscore,
one thing is clear: We're breathing in microplastics. That can't be
good.

\href{https://qcnr.usu.edu/directory/brahney_janice}{Janice Brahney} is
an assistant professor of watershed sciences at Utah State University,
where she directs the Environmental Biogeochemistry and Paleolimnology
Lab.

\emph{The Times is committed to publishing}
\href{https://www.nytimes.com/2019/01/31/opinion/letters/letters-to-editor-new-york-times-women.html}{\emph{a
diversity of letters}} \emph{to the editor. We'd like to hear what you
think about this or any of our articles. Here are some}
\href{https://help.nytimes.com/hc/en-us/articles/115014925288-How-to-submit-a-letter-to-the-editor}{\emph{tips}}\emph{.
And here's our email:}
\href{mailto:letters@nytimes.com}{\emph{letters@nytimes.com}}\emph{.}

\emph{Follow The New York Times Opinion section on}
\href{https://www.facebook.com/nytopinion}{\emph{Facebook}}\emph{,}
\href{http://twitter.com/NYTOpinion}{\emph{Twitter (@NYTopinion)}}
\emph{and}
\href{https://www.instagram.com/nytopinion/}{\emph{Instagram}}\emph{.}

Advertisement

\protect\hyperlink{after-bottom}{Continue reading the main story}

\hypertarget{site-index}{%
\subsection{Site Index}\label{site-index}}

\hypertarget{site-information-navigation}{%
\subsection{Site Information
Navigation}\label{site-information-navigation}}

\begin{itemize}
\tightlist
\item
  \href{https://help.nytimes.com/hc/en-us/articles/115014792127-Copyright-notice}{©~2020~The
  New York Times Company}
\end{itemize}

\begin{itemize}
\tightlist
\item
  \href{https://www.nytco.com/}{NYTCo}
\item
  \href{https://help.nytimes.com/hc/en-us/articles/115015385887-Contact-Us}{Contact
  Us}
\item
  \href{https://www.nytco.com/careers/}{Work with us}
\item
  \href{https://nytmediakit.com/}{Advertise}
\item
  \href{http://www.tbrandstudio.com/}{T Brand Studio}
\item
  \href{https://www.nytimes.com/privacy/cookie-policy\#how-do-i-manage-trackers}{Your
  Ad Choices}
\item
  \href{https://www.nytimes.com/privacy}{Privacy}
\item
  \href{https://help.nytimes.com/hc/en-us/articles/115014893428-Terms-of-service}{Terms
  of Service}
\item
  \href{https://help.nytimes.com/hc/en-us/articles/115014893968-Terms-of-sale}{Terms
  of Sale}
\item
  \href{https://spiderbites.nytimes.com}{Site Map}
\item
  \href{https://help.nytimes.com/hc/en-us}{Help}
\item
  \href{https://www.nytimes.com/subscription?campaignId=37WXW}{Subscriptions}
\end{itemize}
