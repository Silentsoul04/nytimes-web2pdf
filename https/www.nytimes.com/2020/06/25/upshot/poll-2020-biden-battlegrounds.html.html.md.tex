Sections

SEARCH

\protect\hyperlink{site-content}{Skip to
content}\protect\hyperlink{site-index}{Skip to site index}

\href{https://myaccount.nytimes.com/auth/login?response_type=cookie\&client_id=vi}{}

\href{https://www.nytimes.com/section/todayspaper}{Today's Paper}

\href{/section/upshot}{The Upshot}\textbar{}In Poll, Trump Falls Far
Behind Biden in Six Key Battleground States

\url{https://nyti.ms/2CCmZhz}

\begin{itemize}
\item
\item
\item
\item
\item
\item
\end{itemize}

\begin{itemize}
\item
  \href{https://www.nytimes.com/2020/08/03/us/elections/biden-vs-trump.html?action=click\&pgtype=Article\&state=default\&region=TOP_BANNER\&context=storylines_menu}{Election
  Updates}
\item
  \href{https://www.nytimes.com/article/biden-vice-president-2020.html?action=click\&pgtype=Article\&state=default\&region=TOP_BANNER\&context=storylines_menu}{Biden's
  V.P. Search}
\item
  \href{https://www.nytimes.com/interactive/2020/07/24/us/politics/trump-biden-campaign-donors.html?action=click\&pgtype=Article\&state=default\&region=TOP_BANNER\&context=storylines_menu}{Map
  of Donations}
\item
  \href{https://www.nytimes.com/interactive/2020/us/elections/delegate-count-primary-results.html?action=click\&pgtype=Article\&state=default\&region=TOP_BANNER\&context=storylines_menu}{Delegate
  Count}
\item
  \href{https://www.nytimes.com/interactive/2019/us/politics/2020-presidential-candidates.html?action=click\&pgtype=Article\&state=default\&region=TOP_BANNER\&context=storylines_menu}{The
  Candidates}
\item
  \href{https://www.nytimes.com/newsletters/politics?action=click\&pgtype=Article\&state=default\&region=TOP_BANNER\&context=storylines_menu}{Politics
  Newsletter}
\end{itemize}

Advertisement

\protect\hyperlink{after-top}{Continue reading the main story}

Upshot

Supported by

\protect\hyperlink{after-sponsor}{Continue reading the main story}

\hypertarget{in-poll-trump-falls-far-behind-biden-in-six-key-battleground-states}{%
\section{In Poll, Trump Falls Far Behind Biden in Six Key Battleground
States}\label{in-poll-trump-falls-far-behind-biden-in-six-key-battleground-states}}

Dwindling white support for the president leads to a deficit of at least
six points in each state.

\href{https://www.nytimes.com/by/nate-cohn}{\includegraphics{https://static01.nyt.com/images/2018/06/13/multimedia/author-nate-cohn/author-nate-cohn-thumbLarge.jpg}}

By \href{https://www.nytimes.com/by/nate-cohn}{Nate Cohn}

\begin{itemize}
\item
  Published June 25, 2020Updated July 20, 2020
\item
  \begin{itemize}
  \item
  \item
  \item
  \item
  \item
  \item
  \end{itemize}
\end{itemize}

NYT Upshot/

Siena College poll

\includegraphics{https://static01.nyt.com/newsgraphics/2020/06/15/siena-poll-wave-1/1099537c29f271db65813de9ab77ed10211d2920/headshots/biden-new.png}

\includegraphics{https://static01.nyt.com/newsgraphics/2020/06/15/siena-poll-wave-1/1099537c29f271db65813de9ab77ed10211d2920/headshots/trump-new.png}

\hypertarget{joe-biden-holds-a-strong-lead-among-registered-voters-in-six-battleground-states-carried-by-donald-trump-in-2016}{%
\paragraph{Joe Biden holds a strong lead among registered voters in six
battleground states carried by Donald Trump in
2016.}\label{joe-biden-holds-a-strong-lead-among-registered-voters-in-six-battleground-states-carried-by-donald-trump-in-2016}}

Based on a New York Times/Siena College poll of 3,870 registered voters
from June 8 to June 18.

\href{https://www.nytimes.com/interactive/2020/us/elections/donald-trump.html}{President
Trump} has lost significant ground in the six battleground states that
clinched his Electoral College victory in 2016, according to New York
Times/Siena College surveys, with Joseph R.
\href{https://www.nytimes.com/2020/07/03/upshot/joe-biden-voters-coronavirus.html}{Biden}
Jr. opening double-digit leads in Michigan,
\href{https://www.nytimes.com/2020/07/02/us/politics/pennsylvania-trump-biden.html}{Pennsylvania}
and Wisconsin.

\href{https://www.nytimes.com/2020/07/02/us/politics/pennsylvania-trump-biden.html}{Mr.
Trump's} once-commanding
\href{https://www.nytimes.com/2020/06/29/us/politics/trump-swing-voters.html}{advantage
among white voters} has nearly vanished, a development that would all
but preclude the president's re-election if it persisted. Mr. Biden now
has a 21-point lead among white college graduates, and the president is
losing among white voters in the three Northern battleground states ---
not by much, but he won them by nearly 10 points in 2016.

Four years ago, Mr. Trump's strength in the disproportionately white
working-class battleground states allowed him to win the Electoral
College while losing the popular vote. The surveys indicate that the
president continues to fare better in these relatively white
battleground states than he does nationwide.

A separate Times/Siena survey released on Wednesday found Mr. Biden
leading by
\href{https://www.nytimes.com/2020/06/24/us/politics/trump-biden-poll-nyt-upshot-siena-college.html}{14
points nationwide}, 50 percent to 36 percent.

Mr. Biden would win the presidency with at least 333 electoral votes,
far more than the 270 needed, if he won all six of the states surveyed
and held those won by Hillary Clinton four years ago. Most combinations
of any three of the six states --- which also include Florida, Arizona
and North Carolina --- would suffice.

With a little more than four months to go until the election, there is
still time for the president's political standing to recover, just as it
did on so many occasions four years ago. He maintains a substantial
advantage on the economy, which could become an even more central issue
in what has already been a volatile election cycle. And many of the
undecided voters in these states lean Republican, and may end up
returning to their party's nominee.

But for now, the findings confirm that the president's political
standing has deteriorated sharply since October, when Times/Siena polls
found Mr. Biden ahead by just two percentage points across the same six
states (the average gap is now nine points). Since then, the nation has
faced a series of crises that would pose a grave political challenge to
any president seeking re-election. The
\href{https://www.nytimes.com/2020/07/20/upshot/biden-trump-poll.html}{polls}
suggest that battleground-state voters believe the president has
struggled to meet the moment.

Over all, 42 percent of voters in the battleground states approve of how
Mr. Trump is handling his job as president, while 54 percent disapprove.

These six​ states --- with their mix of major cities, old industrial
hubs, growing suburbs, and even farmland --- together deliver a grim
judgment of Mr. Trump on recent issues that have shaken American life.
His handling of the pandemic and the protests after the death of George
Floyd help explain his erosion across both old and new battlegrounds.

\hypertarget{president-trump-has-the-most-support-among-voters-in-dealing-with-the-economy-the-least-on-issues-connected-to-race}{%
\paragraph{President Trump has the most support among voters in dealing
with the economy, the least on issues connected to
race.}\label{president-trump-has-the-most-support-among-voters-in-dealing-with-the-economy-the-least-on-issues-connected-to-race}}

Battleground voters who approve of Trump's handling of ...

Based on a New York Times/Siena College poll of 3,870 registered voters
from June 8 to June 18.

Allan Larson, 83, a recently retired mechanical engineer in Apache
Junction, Ariz., began to regret his vote for the president shortly
after he took office --- he said Mr. Trump tried to do away with too
many things President Obama had done, and kept firing good people ---
but his handling of the pandemic solidified his views.

\hypertarget{latest-updates-2020-election}{%
\section{\texorpdfstring{\href{https://www.nytimes.com/2020/08/03/us/elections/biden-vs-trump.html?action=click\&pgtype=Article\&state=default\&region=MAIN_CONTENT_1\&context=storylines_live_updates}{Latest
Updates: 2020
Election}}{Latest Updates: 2020 Election}}\label{latest-updates-2020-election}}

Updated 2020-08-03T23:41:33.919Z

\begin{itemize}
\tightlist
\item
  \href{https://www.nytimes.com/2020/08/03/us/elections/biden-vs-trump.html?action=click\&pgtype=Article\&state=default\&region=MAIN_CONTENT_1\&context=storylines_live_updates\#link-6494b448}{Trump
  assails mail-in voting anew, citing delays in declaring a winner in a
  New York congressional primary.}
\item
  \href{https://www.nytimes.com/2020/08/03/us/elections/biden-vs-trump.html?action=click\&pgtype=Article\&state=default\&region=MAIN_CONTENT_1\&context=storylines_live_updates\#link-3de249e6}{Obama
  issues his first slate of 2020 endorsements.}
\item
  \href{https://www.nytimes.com/2020/08/03/us/elections/biden-vs-trump.html?action=click\&pgtype=Article\&state=default\&region=MAIN_CONTENT_1\&context=storylines_live_updates\#link-2340e8b5}{On
  the left and the right, Tuesday's primary contests have party leaders
  paying attention.}
\end{itemize}

\href{https://www.nytimes.com/2020/08/03/us/elections/biden-vs-trump.html?action=click\&pgtype=Article\&state=default\&region=MAIN_CONTENT_1\&context=storylines_live_updates}{See
more updates}

``He's not doing anything about this here virus,'' said Mr. Larson, who
plans to vote for Mr. Biden. ``Just the way he's running things, I don't
think he's doing the job he should do.''

On these recent issues, voter disapproval reflects more than just
general dissatisfaction with the state of the country. It seems to
reflect deeper disagreement with the president's prioritization of the
economy over stopping the spread of coronavirus, and with his focus on
law and order over criminal justice.

A majority of voters, 63 percent, say they would rather back a
presidential candidate who focuses on the cause of protests, even when
the protests go too far, while just 31 percent say they would prefer to
support a candidate who says we need to be tough on demonstrations that
go too far.

Despite double-digit unemployment, 55 percent of voters in these six
states say the federal government's priority should be to limit the
spread of the coronavirus, even if it hurts the economy, while just 35
percent say the federal government's priority should be to restart the
economy. Even the newly unemployed, who would seem to have the most to
gain from a reopened economy, say stopping the coronavirus should be the
government's priority.

A high-profile
\href{https://www.nytimes.com/2020/05/31/us/politics/michigan-trump-election.html}{clash}
with Gov. Gretchen Whitmer of Michigan encapsulates the president's
challenge. Mr. Trump sided with protesters who opposed her stay-at-home
orders, but voters in the state oppose the protests against social
distancing restrictions by 57 percent to 37 percent.

As of now, 59 percent of voters in Michigan disapprove of Mr. Trump's
handling of the coronavirus, the highest level of disapproval in any
battleground state polled. And nearly 40 percent of registered voters
there, including 11 percent of Republicans, say he has treated their
state worse than others in response to the pandemic.

\hypertarget{voters-in-michigan-were-much-more-likely-to-say-that-they-thought-president-trump-treated-their-state-unfairly-in-responding-to-the-coronavirus}{%
\paragraph{Voters in Michigan were much more likely to say that they
thought President Trump treated their state unfairly in responding to
the
coronavirus.}\label{voters-in-michigan-were-much-more-likely-to-say-that-they-thought-president-trump-treated-their-state-unfairly-in-responding-to-the-coronavirus}}

Voters who say Trump treated their state worse than most:

Based on a New York Times/Siena College poll of 3,870 registered voters
from June 8 to June 18.

Mr. Trump's ratings are healthier on the kinds of issues that might have
dominated the election season under more ordinary circumstances. His 56
percent approval rating on the economy, versus 40 percent who
disapprove, is nearly the opposite of his overall job approval rating.
Battleground voters say by a double-digit margin that he would do a
better job on the issue than Mr. Biden, and they also prefer Mr. Trump
to handle relations with China.

There is still time for memories to fade or for the national debate to
return to more favorable turf for the president.

Joe Cook, a 35-year-old bakery manager in Orlando, Fla., voted for Mr.
Trump in 2016 and disapproves of how he has handled the coronavirus
outbreak. He said Mr. Trump shouldn't have let the economy be shut down
during the pandemic, and should have cracked down on rioters.

Nevertheless, he will stick with Mr. Trump because he has run on lower
taxes and less regulation. ``The less government in my life, the
better,'' Mr. Cook said.

For now, though, the president's coalition has suffered serious
defections, eroding the familiar demographic divides of recent
elections.

Mr. Trump retains the support of 86 percent of respondents who said they
voted for him in 2016, down from 92 percent in October.

Mr. Biden, by contrast, has emerged from a contested primary with a
unified Democratic coalition. He wins 93 percent of the voters who
backed Mrs. Clinton four years ago, as well as 92 percent of
self-identified Democrats. Mr. Biden also enjoys a significant advantage
among those who voted for neither Mr. Trump nor Mrs. Clinton in 2016. He
has a 35-point lead among battleground voters who said they backed a
minor-party candidate or wrote in another.

Together, these shifts give Mr. Biden a six-point lead among voters who
participated in the 2016 election, according to voter-file records. The
same voters said they backed Mr. Trump over Mrs. Clinton in 2016 by 2.5
percentage points, slightly better for Mr. Trump than the actual result
of the six states, offering a level of validity to the survey's
findings. Mr. Biden also has a 17-point lead among registered voters who
did not vote in the 2016 race.

Mr. Trump's edge among white voters has dwindled despite national
attention to the kind of racial issues that many analysts believed
propelled his strength among white voters in the first place. If
attitudes about race were vital to Mr. Trump's appeal with white voters,
then a foundation of his strength has been badly shaken.

\href{https://www.nytimes.com/interactive/2020/06/10/upshot/black-lives-matter-attitudes.html}{National
polls suggest} that the Black Lives Matter movement has become
significantly more popular since the 2016 election. The Times/Siena
polls find that white voters in the battleground states support the
recent protests and agree with the movement's major complaints about the
criminal justice system, including that the death of Mr. Floyd is part
of a broader pattern of excessive police violence, and that the criminal
justice system is biased against African-Americans. They disapprove of
how the president is handling both the recent protests and race
relations more generally.

Mr. Biden's gains among white voters have been largest among the young
and college-educated white voters likeliest to back the protesters'
views on racial issues.

Over all in the six states, Mr. Biden holds a 55-34 lead among white
voters with at least a four-year college degree, an 11-point gain from
October. White voters under age 35 now back Mr. Biden by a margin of 50
percent to 31 percent, up from an all-but-tied race in October.

White voters with more conservative attitudes on racial issues appear to
have soured on Mr. Trump in recent months, and yet they have not
embraced Mr. Biden.

\hypertarget{bidens-standing-in-battleground-states-represents-a-major-shift-in-support-from-2016-with-nearly-every-group-of-voters}{%
\paragraph{Biden's standing in battleground states represents a major
shift in support from 2016 with nearly every group of
voters.}\label{bidens-standing-in-battleground-states-represents-a-major-shift-in-support-from-2016-with-nearly-every-group-of-voters}}

gender

Race and education

age

Figures in 2020 are from a New York Times/Siena College poll of 3,870
registered voters from June 8 to June 18. Figures from 2019 are from a
NYT Upshot/Siena College
\href{https://www.nytimes.com/2019/11/04/upshot/trump-biden-warren-polls.html}{poll}
of 3,766 registered voters from Oct. 13 to Oct. 26. Figures from 2016
represent a combination of 7,802 battleground respondents in polls by
The New York Times/Siena College, The New York Times/CBS News, Pew
Research, The Washington Post/ABC News and CNN/ORC in fall 2016.

White voters without a degree, the linchpin of the president's winning
coalition, back Mr. Trump by a 16-point margin in the battlegrounds,
down from a 24-point margin in October and a 26-point one in the final
polls of the last election. Despite that slide, Mr. Biden's support
among white voters without a degree has increased by only one percentage
point since October.

One such voter Mr. Biden has gained is Samantha Spencer, 29, from
Beloit, Wis. ``There's just been so many different things that I've been
like viscerally disgusted by,'' she said. ``I'm a Christian and I know a
lot of people who are also Christians are still sticking with him, but
for my faith I can't justify supporting this garbage anymore.''

Mr. Biden leads among voters 65 and over, reversing a decade-long
Republican advantage. But he has made relatively limited gains among
voters over age 50 since October, including no gains at all among white
voters over age 50 without a college degree.

Their relatively conservative attitudes on race and the protests could
be part of the reason for the president's resilience: White voters in
the battleground states who are 50 and over oppose the recent
demonstrations, and say too many have turned to violent rioting. They
are split on whether discrimination against whites is as big a problem
as discrimination against minorities, and say that riots are a bigger
problem than police treatment of African-Americans by a
10-percentage-point margin.

Perhaps more surprisingly, Mr. Biden has also made few to no gains among
nonwhite voters, despite the national attention on criminal justice and
racism over the last month.

Over all in the battlegrounds, Mr. Biden leads among black voters by 83
percent to 7 percent, up only slightly from October. Hispanic voters
back Mr. Biden by 62-26, also essentially unchanged. Neither lead
exceeds Mrs. Clinton's margin in the final polls from 2016.

Mr. Biden's wide lead is a reflection of the president's weakness rather
than of his own strength. Over all, 55 percent of Mr. Biden's supporters
say their vote is more a vote against Mr. Trump than a vote for Mr.
Biden, while 80 percent of Mr. Trump's supporters say they're mainly
\href{https://www.nytimes.com/2020/06/27/us/politics/trump-biden-protests-polling.html}{voting}
for the president. And Mr. Biden's gains have come without any
improvement in his favorability ratings, even as Mr. Trump's have
plummeted.

But Mr. Biden's standing is nonetheless healthy by most measures. Over
all, 50 percent of battleground voters say they have a favorable view of
him, compared with 47 percent who have an unfavorable view.

It's possible that Mr. Biden will struggle to match his wide lead in the
polls at the ballot box. The battleground voters who don't back either
Mr. Biden or Mr. Trump tend to tilt Republican, whether by party
registration or by affiliation, and 34 percent say they voted for Mr.
Trump in 2016, compared with 20 percent who backed Mrs. Clinton.

Some of these voters may return to the president by the end of the race,
yet at the moment, 56 percent of these voters disapprove of his
performance, while just 29 percent approve.

The results suggest that Mr. Biden still has an open path to a sweeping
victory. Over all, 55 percent of registered voters in the battleground
states said there was at least ``some chance'' they would support Mr.
Biden in the election, including 12 percent of Republicans, 11 percent
of voters who backed Mr. Trump in 2016, and 44 percent of the
Republican-tilting undecided voters.

As for Mr. Trump, 55 percent of registered voters in the battlegrounds
said there was ``not really any chance'' they would vote for him this
November.

\hypertarget{new-york-timessiena-college-polls}{%
\subsubsection{New York Times/Siena College
polls}\label{new-york-timessiena-college-polls}}

We asked thousands of voters across the country about President Trump,
Joseph R. Biden Jr., the coronavirus pandemic, Black Lives Matter and
more.

\begin{itemize}
\item
  \href{https://www.nytimes.com/2020/06/24/us/politics/trump-biden-poll-nyt-upshot-siena-college.html}{\includegraphics{https://static01.nyt.com/newsgraphics/2020/06/23/2020-pollingribbon/3133357315cec7c0cac77287fdc6f8ec7774b74f/thumbs/allstates.jpg}}

  \href{https://www.nytimes.com/2020/06/24/us/politics/trump-biden-poll-nyt-upshot-siena-college.html}{}

  \hypertarget{biden-takes-dominant-lead-as-voters-reject-trump-on-virus-and-race}{%
  \section{Biden Takes Dominant Lead as Voters Reject Trump on Virus and
  Race}\label{biden-takes-dominant-lead-as-voters-reject-trump-on-virus-and-race}}

  June 24, 2020
\item
  \href{https://www.nytimes.com/2020/06/25/upshot/poll-2020-biden-battlegrounds.html}{\includegraphics{https://static01.nyt.com/newsgraphics/2020/06/23/2020-pollingribbon/3133357315cec7c0cac77287fdc6f8ec7774b74f/thumbs/battlestates.jpg}}

  \href{https://www.nytimes.com/2020/06/25/upshot/poll-2020-biden-battlegrounds.html}{}

  \hypertarget{showing-strength-with-white-voters-biden-builds-lead-in-battleground-states}{%
  \section{Showing Strength With White Voters, Biden Builds Lead in
  Battleground
  States}\label{showing-strength-with-white-voters-biden-builds-lead-in-battleground-states}}

  June 25, 2020
\item
  \href{https://www.nytimes.com/2020/06/25/us/politics/trump-senate-republicans-poll.html}{\includegraphics{https://static01.nyt.com/newsgraphics/2020/06/23/2020-pollingribbon/3133357315cec7c0cac77287fdc6f8ec7774b74f/thumbs/altraces.jpg}}

  \href{https://www.nytimes.com/2020/06/25/us/politics/trump-senate-republicans-poll.html}{}

  \hypertarget{trumps-sagging-popularity-drags-down-republican-senate-candidates}{%
  \section{Trump's Sagging Popularity Drags Down Republican Senate
  Candidates}\label{trumps-sagging-popularity-drags-down-republican-senate-candidates}}

  June 25, 2020
\item
  \href{https://www.nytimes.com/2020/06/26/us/politics/biden-vice-president-voters.html}{\includegraphics{https://static01.nyt.com/newsgraphics/2020/06/23/2020-pollingribbon/3133357315cec7c0cac77287fdc6f8ec7774b74f/thumbs/vp.png}}

  \href{https://www.nytimes.com/2020/06/26/us/politics/biden-vice-president-voters.html}{}

  \hypertarget{biden-is-getting-a-lot-of-advice-on-his-vp-heres-what-voters-think}{%
  \section{Biden Is Getting a Lot of Advice on His V.P. Here's What
  Voters
  Think.}\label{biden-is-getting-a-lot-of-advice-on-his-vp-heres-what-voters-think}}

  June 26, 2020
\end{itemize}

\begin{center}\rule{0.5\linewidth}{\linethickness}\end{center}

The Times/Siena poll of 3,870 registered voters in Pennsylvania,
Michigan, Florida, Arizona, Wisconsin and North Carolina was conducted
from June 8 to 18. The margin of sampling error for an individual state
poll ranges from plus-or-minus 4.1 to 4.6 percentage points. The margin
of sampling error on the full battleground sample is plus-or-minus 1.8
percentage points.

Here are the
\href{https://int.nyt.com/data/documenttools/battleground-0625/6ca076db1919b722/full.pdf}{crosstabs}
and
\href{https://int.nyt.com/data/documenttools/nyt-siena-poll-methodology-june-2020/f6f533b4d07f4cbe/full.pdf}{methodology}
for the poll.

\begin{center}\rule{0.5\linewidth}{\linethickness}\end{center}

Claire Cain Miller contributing reporting.

\hypertarget{our-2020-election-guide}{%
\section{Our 2020 Election Guide}\label{our-2020-election-guide}}

Updated July 31, 2020

\begin{itemize}
\item
  \begin{center}\rule{0.5\linewidth}{\linethickness}\end{center}

  \hypertarget{the-latest}{%
  \subsection{The Latest}\label{the-latest}}

  \begin{itemize}
  \tightlist
  \item
    The vice-presidential watch begins in earnest this week.
    \href{https://www.nytimes.com/2020/08/03/us/elections/biden-vs-trump.html?action=click\&pgtype=Article\&state=default\&region=BELOW_MAIN_CONTENT\&context=storylines_guide}{Follow
    the latest updates here.}
  \end{itemize}
\item
  \begin{center}\rule{0.5\linewidth}{\linethickness}\end{center}

  \hypertarget{bidens-vp-search}{%
  \subsection{Biden's V.P. Search}\label{bidens-vp-search}}

  \begin{itemize}
  \tightlist
  \item
    \href{https://www.nytimes.com/article/biden-vice-president-2020.html?action=click\&pgtype=Article\&state=default\&region=BELOW_MAIN_CONTENT\&context=storylines_guide}{Here
    are 13 women} who have been under consideration to be Joe Biden's
    running mate, and why each might be chosen --- and might not be.
  \end{itemize}
\item
  \begin{center}\rule{0.5\linewidth}{\linethickness}\end{center}

  \hypertarget{keep-up-with-our-coverage}{%
  \subsection{Keep Up With Our
  Coverage}\label{keep-up-with-our-coverage}}

  \begin{itemize}
  \tightlist
  \item
    Get an
    \href{https://www.nytimes.com/newsletters/politics?action=click\&pgtype=Article\&state=default\&region=BELOW_MAIN_CONTENT\&context=storylines_guide}{email}
    recapping the day's news
  \end{itemize}

  \begin{itemize}
  \tightlist
  \item
    Download our mobile app on
    \href{https://apps.apple.com/us/app/nytimes/id284862083?ls=1\&mat_click_id=5c79ae7455014fd1bd66b5610c05b8f2-20191112-16948\&referrer=mat_click_id\%3D5c79ae7455014fd1bd66b5610c05b8f2-20191112-16948\%26link_click_id\%3D722930677036718082}{iOS}
    and
    \href{http://a.localytics.com/android?id=com.nytimes.android\&referrer=utm_source\%3Dother_nyt_mobile_web\%26utm_medium\%3DWeb\%2520page\%26utm_term\%3DGeneral\%2520Mobile\%2520Page\%26utm_campaign\%3DNYT\%2520Mobile\%2520General\%2520Page}{Android}
    and turn on Breaking News and Politics alerts
  \end{itemize}
\end{itemize}

Advertisement

\protect\hyperlink{after-bottom}{Continue reading the main story}

\hypertarget{site-index}{%
\subsection{Site Index}\label{site-index}}

\hypertarget{site-information-navigation}{%
\subsection{Site Information
Navigation}\label{site-information-navigation}}

\begin{itemize}
\tightlist
\item
  \href{https://help.nytimes.com/hc/en-us/articles/115014792127-Copyright-notice}{©~2020~The
  New York Times Company}
\end{itemize}

\begin{itemize}
\tightlist
\item
  \href{https://www.nytco.com/}{NYTCo}
\item
  \href{https://help.nytimes.com/hc/en-us/articles/115015385887-Contact-Us}{Contact
  Us}
\item
  \href{https://www.nytco.com/careers/}{Work with us}
\item
  \href{https://nytmediakit.com/}{Advertise}
\item
  \href{http://www.tbrandstudio.com/}{T Brand Studio}
\item
  \href{https://www.nytimes.com/privacy/cookie-policy\#how-do-i-manage-trackers}{Your
  Ad Choices}
\item
  \href{https://www.nytimes.com/privacy}{Privacy}
\item
  \href{https://help.nytimes.com/hc/en-us/articles/115014893428-Terms-of-service}{Terms
  of Service}
\item
  \href{https://help.nytimes.com/hc/en-us/articles/115014893968-Terms-of-sale}{Terms
  of Sale}
\item
  \href{https://spiderbites.nytimes.com}{Site Map}
\item
  \href{https://help.nytimes.com/hc/en-us}{Help}
\item
  \href{https://www.nytimes.com/subscription?campaignId=37WXW}{Subscriptions}
\end{itemize}
