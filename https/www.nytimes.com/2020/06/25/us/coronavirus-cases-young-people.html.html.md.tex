Sections

SEARCH

\protect\hyperlink{site-content}{Skip to
content}\protect\hyperlink{site-index}{Skip to site index}

\href{https://www.nytimes.com/section/us}{U.S.}

\href{https://myaccount.nytimes.com/auth/login?response_type=cookie\&client_id=vi}{}

\href{https://www.nytimes.com/section/todayspaper}{Today's Paper}

\href{/section/us}{U.S.}\textbar{}As Virus Surges, Younger People
Account for `Disturbing' Number of Cases

\url{https://nyti.ms/2YvRmPk}

\begin{itemize}
\item
\item
\item
\item
\item
\item
\end{itemize}

\href{https://www.nytimes.com/news-event/coronavirus?action=click\&pgtype=Article\&state=default\&region=TOP_BANNER\&context=storylines_menu}{The
Coronavirus Outbreak}

\begin{itemize}
\tightlist
\item
  live\href{https://www.nytimes.com/2020/08/04/world/coronavirus-cases.html?action=click\&pgtype=Article\&state=default\&region=TOP_BANNER\&context=storylines_menu}{Latest
  Updates}
\item
  \href{https://www.nytimes.com/interactive/2020/us/coronavirus-us-cases.html?action=click\&pgtype=Article\&state=default\&region=TOP_BANNER\&context=storylines_menu}{Maps
  and Cases}
\item
  \href{https://www.nytimes.com/interactive/2020/science/coronavirus-vaccine-tracker.html?action=click\&pgtype=Article\&state=default\&region=TOP_BANNER\&context=storylines_menu}{Vaccine
  Tracker}
\item
  \href{https://www.nytimes.com/2020/08/02/us/covid-college-reopening.html?action=click\&pgtype=Article\&state=default\&region=TOP_BANNER\&context=storylines_menu}{College
  Reopening}
\item
  \href{https://www.nytimes.com/live/2020/08/04/business/stock-market-today-coronavirus?action=click\&pgtype=Article\&state=default\&region=TOP_BANNER\&context=storylines_menu}{Economy}
\end{itemize}

Advertisement

\protect\hyperlink{after-top}{Continue reading the main story}

Supported by

\protect\hyperlink{after-sponsor}{Continue reading the main story}

\hypertarget{as-virus-surges-younger-people-account-for-disturbing-number-of-cases}{%
\section{As Virus Surges, Younger People Account for `Disturbing' Number
of
Cases}\label{as-virus-surges-younger-people-account-for-disturbing-number-of-cases}}

People in their 20s, 30s and 40s account for a growing proportion of the
cases in many places, raising fears that asymptomatic young people are
helping to fuel the virus's spread.

\includegraphics{https://static01.nyt.com/images/2020/06/25/us/25VIRUS-YOUTH-fla/merlin_173924568_9f1e3875-2994-44cd-9462-06493ff51739-articleLarge.jpg?quality=75\&auto=webp\&disable=upscale}

\href{https://www.nytimes.com/by/julie-bosman}{\includegraphics{https://static01.nyt.com/images/2018/11/09/multimedia/author-julie-bosman/author-julie-bosman-thumbLarge.png}}\href{https://www.nytimes.com/by/sarah-mervosh}{\includegraphics{https://static01.nyt.com/images/2018/07/18/multimedia/author-sarah-mervosh/author-sarah-mervosh-thumbLarge-v3.png}}

By \href{https://www.nytimes.com/by/julie-bosman}{Julie Bosman} and
\href{https://www.nytimes.com/by/sarah-mervosh}{Sarah Mervosh}

\begin{itemize}
\item
  Published June 25, 2020Updated July 23, 2020
\item
  \begin{itemize}
  \item
  \item
  \item
  \item
  \item
  \item
  \end{itemize}
\end{itemize}

CHICAGO --- Younger people are making up a growing percentage of new
coronavirus cases in cities and states where the virus is now surging, a
trend that has alarmed public health officials and prompted renewed
pleas for masks and social distancing.

In Arizona, where
\href{https://www.nytimes.com/2020/06/25/upshot/virus-testing-shortfall-arizona.html}{drive-up
sites are overwhelmed} by people seeking coronavirus tests,
\href{https://www.azdhs.gov/preparedness/epidemiology-disease-control/infectious-disease-epidemiology/covid-19/dashboards/index.php}{people
ages 20 to 44 account for nearly half} of all cases. In Florida, which
breaks records for new cases nearly every day, the median age of
residents testing positive for the virus has dropped to 35, down from 65
in March.

And in
\href{https://www.nytimes.com/2020/06/25/us/texas-coronavirus-cases-reopening.html}{Texas,
where the governor paused} the reopening process on Thursday as
hospitals grow increasingly crowded, young people now account for the
majority of new cases in several urban centers. In Cameron County, which
includes Brownsville and the tourist town of South Padre Island, people
under 40 make up more than half of newly reported cases.

``What is clear is that the proportion of people who are younger appears
to have dramatically changed,'' said Joseph McCormick, a professor of
epidemiology at UTHealth School of Public Health in Brownsville. ``It's
really quite disturbing.''

The pattern is drawing notice from mayors, governors and public health
officials, and comes as a worrisome sign for cities and institutions as
they look to the fall. The rise in cases among younger people could
complicate the plans of leaders who are eager to open schools and
universities, resume athletic events and return to normal life and a
fully functioning economy.

The increases could reflect a simple reality: Since many states have
reopened bars, restaurants and offices, the coronavirus has been allowed
to spread more widely across communities, including to more young
people. But people in their 20s and 30s are also more likely to go out
socializing, experts say, raising concerns that asymptomatic young
people are helping to spread the virus to more vulnerable Americans at a
time when cases are surging dangerously in the South and the West.

Dr. Robert Redfield, the director of the Centers for Disease Control and
Prevention, said on Thursday that younger people have helped fuel the
increase in known coronavirus infections --- and that in the past, many
of those infections went undiagnosed.

``Our best estimate right now is that for every case that was reported,
there actually were 10 other infections,'' he said.

No single answer fully accounts for the surge of cases among young
people, who are less likely to be hospitalized or die from the
coronavirus than older people.

\hypertarget{latest-updates-global-coronavirus-outbreak}{%
\section{\texorpdfstring{\href{https://www.nytimes.com/2020/08/04/world/coronavirus-cases.html?action=click\&pgtype=Article\&state=default\&region=MAIN_CONTENT_1\&context=storylines_live_updates}{Latest
Updates: Global Coronavirus
Outbreak}}{Latest Updates: Global Coronavirus Outbreak}}\label{latest-updates-global-coronavirus-outbreak}}

Updated 2020-08-04T20:57:54.346Z

\begin{itemize}
\tightlist
\item
  \href{https://www.nytimes.com/2020/08/04/world/coronavirus-cases.html?action=click\&pgtype=Article\&state=default\&region=MAIN_CONTENT_1\&context=storylines_live_updates\#link-1228a480}{Novavax
  sees encouraging results from two studies of its experimental
  vaccine.}
\item
  \href{https://www.nytimes.com/2020/08/04/world/coronavirus-cases.html?action=click\&pgtype=Article\&state=default\&region=MAIN_CONTENT_1\&context=storylines_live_updates\#link-4825b93}{Public
  and private schools in Maryland and elsewhere are divided over
  in-person instruction.}
\item
  \href{https://www.nytimes.com/2020/08/04/world/coronavirus-cases.html?action=click\&pgtype=Article\&state=default\&region=MAIN_CONTENT_1\&context=storylines_live_updates\#link-50f7386d}{The
  United Nations calls on policymakers to `plan thoroughly for school
  reopenings.'}
\end{itemize}

\href{https://www.nytimes.com/2020/08/04/world/coronavirus-cases.html?action=click\&pgtype=Article\&state=default\&region=MAIN_CONTENT_1\&context=storylines_live_updates}{See
more updates}

More live coverage:
\href{https://www.nytimes.com/live/2020/08/04/business/stock-market-today-coronavirus?action=click\&pgtype=Article\&state=default\&region=MAIN_CONTENT_1\&context=storylines_live_updates}{Markets}

``Is it the governor's reopening? Is it Memorial Day? Is it the George
Floyd demonstrations? Is it going to the beach?'' said Eric Boerwinkle,
dean of the UTHealth School of Public Health in Houston. ``We don't
really know, but it is probably all of those things that are
contributing.''

The United States recorded 36,975 new cases on Wednesday, a new high
point in daily cases as the country confronted a new stage of the crisis
two months after the previous high in late April. The resurgence is most
immediately threatening states that reopened relatively early in the
South and the West. Alabama, Florida, Oklahoma, South Carolina and Texas
all reported their highest single-day totals this week, as did Montana
and Utah, and cases were rising in 29 states on Thursday.

Adriana Carter, 21, is among the newly infected.

For many weeks this spring, she said, she took steps to limit her
exposure, eating many of her meals at her apartment in San Marcos,
Texas, and wearing a mask when going in and out of stores. At the one
Black Lives Matter protest she attended, most people were in masks.

But after a particularly long week of juggling online summer classes and
her job at an eye clinic, Ms. Carter took a risk one Saturday night in
early June and met a friend at the Square, a popular bar district
downtown. Though they were careful to avoid the most crowded spots, they
chose not to wear masks as they sipped drinks inside and endured the hot
Texas weather.

Days later, her friend woke up feeling ill. Both tested positive for the
virus.

``We were told we could go out to bars,'' she said, adding that she had
been careful to quarantine since she learned that she had been exposed.
``It's very unusual for anyone in their 20s to stay at home all the time
--- not giving any excuses or anything, but I just think we are all just
trying to do the best we can.''

The new cases among young people may appear to be a departure from the
early days of the pandemic when infections in nursing homes were
spiraling out of control, and the virus appeared at higher rates among
older people in New York City.

Experts cautioned that the seemingly new prevalence among young people
may be, in part, a reflection of more widely available testing. But the
growing numbers of people hospitalized in states like North Carolina and
Texas also suggest increased transmission of the virus.

Even now, people younger than 50 are being hospitalized at a far lower
rate than people older than that,
\href{https://www.cdc.gov/coronavirus/2019-ncov/covid-data/covidview/index.html}{according
to C.D.C. data}.

While the effect of the coronavirus on younger people ``may not be
highly associated with hospitalization and death,'' Dr. Redfield said,
``they do act as a transmission connector for individuals that could in
fact be at a higher risk.''

In Florida, which has emerged as a particularly concerning hot spot,
reopened bars have been a source of contagion among young people. The
state shut down the Knight's Pub, a popular bar near the University of
Central Florida in Orlando, after 28 patrons and 13 employees were
infected.

In Miami-Dade County, the number of known coronavirus cases among 18- to
34-year-olds increased fivefold in a month, to more than 1,000, Mayor
Carlos Gimenez said this week.

``They're thinking they're invincible,'' he said, adding that many of
the infected have no symptoms.

They are at higher risk, though, if they are overweight or have diabetes
or other medical conditions, he said. About a third of the coronavirus
patients at the public Jackson Health System were from that age group,
and about half had a high body mass index, Mr. Gimenez said.

Gov. Ron DeSantis described ``a real explosion in new cases'' among
younger people. ``Part of that is just natural,'' he said. ``You kind of
go and you want to be doing things. You want to be out and about. The
folks who are older and would be more vulnerable are being a bit more
careful.''

In fact, some experts believe that a decision by older people to stay
home and exercise caution to avoid the virus may, in part, help explain
why young people appear to be an increasing portion of new cases.

\includegraphics{https://static01.nyt.com/images/2020/06/25/us/25VIRUS-YOUTH-tx/merlin_173920437_0c7f16c1-5d96-4261-83be-3c5e34a66929-articleLarge.jpg?quality=75\&auto=webp\&disable=upscale}

In Dallas County, people between the ages of 18 and 40 have made up 52
percent of newly reported cases since the beginning of June, a jump from
the 38 percent that young people represented in March,
\href{https://www.dallascounty.org/Assets/uploads/docs/hhs/2019-nCoV/COVID-19\%20DCHHS\%20Summary_062320.pdf}{according
to county data}.

\href{https://www.nytimes.com/news-event/coronavirus?action=click\&pgtype=Article\&state=default\&region=MAIN_CONTENT_3\&context=storylines_faq}{}

\hypertarget{the-coronavirus-outbreak-}{%
\subsubsection{The Coronavirus Outbreak
›}\label{the-coronavirus-outbreak-}}

\hypertarget{frequently-asked-questions}{%
\paragraph{Frequently Asked
Questions}\label{frequently-asked-questions}}

Updated August 4, 2020

\begin{itemize}
\item ~
  \hypertarget{i-have-antibodies-am-i-now-immune}{%
  \paragraph{I have antibodies. Am I now
  immune?}\label{i-have-antibodies-am-i-now-immune}}

  \begin{itemize}
  \tightlist
  \item
    As of right
    now,\href{https://www.nytimes.com/2020/07/22/health/covid-antibodies-herd-immunity.html?action=click\&pgtype=Article\&state=default\&region=MAIN_CONTENT_3\&context=storylines_faq}{that
    seems likely, for at least several months.} There have been
    frightening accounts of people suffering what seems to be a second
    bout of Covid-19. But experts say these patients may have a
    drawn-out course of infection, with the virus taking a slow toll
    weeks to months after initial exposure. People infected with the
    coronavirus typically
    \href{https://www.nature.com/articles/s41586-020-2456-9}{produce}
    immune molecules called antibodies, which are
    \href{https://www.nytimes.com/2020/05/07/health/coronavirus-antibody-prevalence.html?action=click\&pgtype=Article\&state=default\&region=MAIN_CONTENT_3\&context=storylines_faq}{protective
    proteins made in response to an
    infection}\href{https://www.nytimes.com/2020/05/07/health/coronavirus-antibody-prevalence.html?action=click\&pgtype=Article\&state=default\&region=MAIN_CONTENT_3\&context=storylines_faq}{.
    These antibodies may} last in the body
    \href{https://www.nature.com/articles/s41591-020-0965-6}{only two to
    three months}, which may seem worrisome, but that's perfectly normal
    after an acute infection subsides, said Dr. Michael Mina, an
    immunologist at Harvard University. It may be possible to get the
    coronavirus again, but it's highly unlikely that it would be
    possible in a short window of time from initial infection or make
    people sicker the second time.
  \end{itemize}
\item ~
  \hypertarget{im-a-small-business-owner-can-i-get-relief}{%
  \paragraph{I'm a small-business owner. Can I get
  relief?}\label{im-a-small-business-owner-can-i-get-relief}}

  \begin{itemize}
  \tightlist
  \item
    The
    \href{https://www.nytimes.com/article/small-business-loans-stimulus-grants-freelancers-coronavirus.html?action=click\&pgtype=Article\&state=default\&region=MAIN_CONTENT_3\&context=storylines_faq}{stimulus
    bills enacted in March} offer help for the millions of American
    small businesses. Those eligible for aid are businesses and
    nonprofit organizations with fewer than 500 workers, including sole
    proprietorships, independent contractors and freelancers. Some
    larger companies in some industries are also eligible. The help
    being offered, which is being managed by the Small Business
    Administration, includes the Paycheck Protection Program and the
    Economic Injury Disaster Loan program. But lots of folks have
    \href{https://www.nytimes.com/interactive/2020/05/07/business/small-business-loans-coronavirus.html?action=click\&pgtype=Article\&state=default\&region=MAIN_CONTENT_3\&context=storylines_faq}{not
    yet seen payouts.} Even those who have received help are confused:
    The rules are draconian, and some are stuck sitting on
    \href{https://www.nytimes.com/2020/05/02/business/economy/loans-coronavirus-small-business.html?action=click\&pgtype=Article\&state=default\&region=MAIN_CONTENT_3\&context=storylines_faq}{money
    they don't know how to use.} Many small-business owners are getting
    less than they expected or
    \href{https://www.nytimes.com/2020/06/10/business/Small-business-loans-ppp.html?action=click\&pgtype=Article\&state=default\&region=MAIN_CONTENT_3\&context=storylines_faq}{not
    hearing anything at all.}
  \end{itemize}
\item ~
  \hypertarget{what-are-my-rights-if-i-am-worried-about-going-back-to-work}{%
  \paragraph{What are my rights if I am worried about going back to
  work?}\label{what-are-my-rights-if-i-am-worried-about-going-back-to-work}}

  \begin{itemize}
  \tightlist
  \item
    Employers have to provide
    \href{https://www.osha.gov/SLTC/covid-19/standards.html}{a safe
    workplace} with policies that protect everyone equally.
    \href{https://www.nytimes.com/article/coronavirus-money-unemployment.html?action=click\&pgtype=Article\&state=default\&region=MAIN_CONTENT_3\&context=storylines_faq}{And
    if one of your co-workers tests positive for the coronavirus, the
    C.D.C.} has said that
    \href{https://www.cdc.gov/coronavirus/2019-ncov/community/guidance-business-response.html}{employers
    should tell their employees} -\/- without giving you the sick
    employee's name -\/- that they may have been exposed to the virus.
  \end{itemize}
\item ~
  \hypertarget{should-i-refinance-my-mortgage}{%
  \paragraph{Should I refinance my
  mortgage?}\label{should-i-refinance-my-mortgage}}

  \begin{itemize}
  \tightlist
  \item
    \href{https://www.nytimes.com/article/coronavirus-money-unemployment.html?action=click\&pgtype=Article\&state=default\&region=MAIN_CONTENT_3\&context=storylines_faq}{It
    could be a good idea,} because mortgage rates have
    \href{https://www.nytimes.com/2020/07/16/business/mortgage-rates-below-3-percent.html?action=click\&pgtype=Article\&state=default\&region=MAIN_CONTENT_3\&context=storylines_faq}{never
    been lower.} Refinancing requests have pushed mortgage applications
    to some of the highest levels since 2008, so be prepared to get in
    line. But defaults are also up, so if you're thinking about buying a
    home, be aware that some lenders have tightened their standards.
  \end{itemize}
\item ~
  \hypertarget{what-is-school-going-to-look-like-in-september}{%
  \paragraph{What is school going to look like in
  September?}\label{what-is-school-going-to-look-like-in-september}}

  \begin{itemize}
  \tightlist
  \item
    It is unlikely that many schools will return to a normal schedule
    this fall, requiring the grind of
    \href{https://www.nytimes.com/2020/06/05/us/coronavirus-education-lost-learning.html?action=click\&pgtype=Article\&state=default\&region=MAIN_CONTENT_3\&context=storylines_faq}{online
    learning},
    \href{https://www.nytimes.com/2020/05/29/us/coronavirus-child-care-centers.html?action=click\&pgtype=Article\&state=default\&region=MAIN_CONTENT_3\&context=storylines_faq}{makeshift
    child care} and
    \href{https://www.nytimes.com/2020/06/03/business/economy/coronavirus-working-women.html?action=click\&pgtype=Article\&state=default\&region=MAIN_CONTENT_3\&context=storylines_faq}{stunted
    workdays} to continue. California's two largest public school
    districts --- Los Angeles and San Diego --- said on July 13, that
    \href{https://www.nytimes.com/2020/07/13/us/lausd-san-diego-school-reopening.html?action=click\&pgtype=Article\&state=default\&region=MAIN_CONTENT_3\&context=storylines_faq}{instruction
    will be remote-only in the fall}, citing concerns that surging
    coronavirus infections in their areas pose too dire a risk for
    students and teachers. Together, the two districts enroll some
    825,000 students. They are the largest in the country so far to
    abandon plans for even a partial physical return to classrooms when
    they reopen in August. For other districts, the solution won't be an
    all-or-nothing approach.
    \href{https://bioethics.jhu.edu/research-and-outreach/projects/eschool-initiative/school-policy-tracker/}{Many
    systems}, including the nation's largest, New York City, are
    devising
    \href{https://www.nytimes.com/2020/06/26/us/coronavirus-schools-reopen-fall.html?action=click\&pgtype=Article\&state=default\&region=MAIN_CONTENT_3\&context=storylines_faq}{hybrid
    plans} that involve spending some days in classrooms and other days
    online. There's no national policy on this yet, so check with your
    municipal school system regularly to see what is happening in your
    community.
  \end{itemize}
\end{itemize}

At the same time, older people have begun to represent a smaller portion
of the total number of people who test positive for the virus. In June,
people over 65 have made up 8 percent of new confirmed cases in Dallas
County, down from 16 percent in March.

The situation is particularly unsettling in Hays County, home to Texas
State University in San Marcos. Coronavirus cases have surged since the
beginning of June, to 2,100 this week, from 371 at the start of the
month. People in their 20s now make up more than half of all known
cases, officials said.

In Arizona, rising infections have set many people on edge, including
some residents in their 20s and 30s.

In the Arcadia neighborhood of Phoenix, Ian Bartczak, who is 31, said he
did not feel comfortable dining out at restaurants and was dismayed to
see crowds of young people squeezing onto patios and bars on a
commercial strip near his home.

``It goes back to, what is a want and what is a need?'' said Mr.
Bartczak, who works for an education technology company. ``Did you have
to go to a big swimming party or El Hefe nightclub with your friends?''

His point of view has created awkwardness with some friends, he said. He
has turned down invitations to go out for sushi, and been puzzled by
friends who chose to visit casinos.

``It's affected some of my relationships because I won't see them or get
kind of angry,'' he said. ``How are you not willing to help the old lady
behind you who could have a poor immune system? Or help lower our cases
so we can increase our economy?''

In Phoenix, Michael Donoghue, an investment analyst who is 33, said he
felt comfortable going out --- carefully --- since he is single and
healthy, lives alone and takes care to avoid close contact with people
who might be at risk, like his 91-year-old grandmother.

Only once since restrictions were lifted in that state has he felt
uncomfortable while out, he said. A bar he visited with friends in
Scottsdale was crowded.

``It just felt like, should we be doing this right now?'' he said.

The resurgence of the virus has echoes of its earliest days in the
United States, as places like California and Washington State, which saw
some of the country's first outbreaks, were seeing new upticks.

In King County, Wash., which includes Seattle, people in their 20s and
30s make up about 45 percent of new coronavirus cases --- a number that
was 25 percent in March, according to Dr. Judith A. Malmgren, an
epidemiologist in Seattle.

She believes the real percentage is even larger than what is being
measured because younger people are less likely to be symptomatic. That
said, she warned that the risk of infecting other people was serious.

``Just because you're in an age group that is less likely to die from
coronavirus,'' she said, ``does not mean that you live alone.''

Julie Bosman reported from Chicago, and Sarah Mervosh from Pittsburgh.
Patricia Mazzei contributed reporting from Miami, and Mitch Smith from
Chicago.

Advertisement

\protect\hyperlink{after-bottom}{Continue reading the main story}

\hypertarget{site-index}{%
\subsection{Site Index}\label{site-index}}

\hypertarget{site-information-navigation}{%
\subsection{Site Information
Navigation}\label{site-information-navigation}}

\begin{itemize}
\tightlist
\item
  \href{https://help.nytimes.com/hc/en-us/articles/115014792127-Copyright-notice}{©~2020~The
  New York Times Company}
\end{itemize}

\begin{itemize}
\tightlist
\item
  \href{https://www.nytco.com/}{NYTCo}
\item
  \href{https://help.nytimes.com/hc/en-us/articles/115015385887-Contact-Us}{Contact
  Us}
\item
  \href{https://www.nytco.com/careers/}{Work with us}
\item
  \href{https://nytmediakit.com/}{Advertise}
\item
  \href{http://www.tbrandstudio.com/}{T Brand Studio}
\item
  \href{https://www.nytimes.com/privacy/cookie-policy\#how-do-i-manage-trackers}{Your
  Ad Choices}
\item
  \href{https://www.nytimes.com/privacy}{Privacy}
\item
  \href{https://help.nytimes.com/hc/en-us/articles/115014893428-Terms-of-service}{Terms
  of Service}
\item
  \href{https://help.nytimes.com/hc/en-us/articles/115014893968-Terms-of-sale}{Terms
  of Sale}
\item
  \href{https://spiderbites.nytimes.com}{Site Map}
\item
  \href{https://help.nytimes.com/hc/en-us}{Help}
\item
  \href{https://www.nytimes.com/subscription?campaignId=37WXW}{Subscriptions}
\end{itemize}
