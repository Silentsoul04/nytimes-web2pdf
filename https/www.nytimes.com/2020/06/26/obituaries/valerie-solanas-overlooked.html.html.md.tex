Sections

SEARCH

\protect\hyperlink{site-content}{Skip to
content}\protect\hyperlink{site-index}{Skip to site index}

\href{https://www.nytimes.com/section/obituaries}{Obituaries}

\href{https://myaccount.nytimes.com/auth/login?response_type=cookie\&client_id=vi}{}

\href{https://www.nytimes.com/section/todayspaper}{Today's Paper}

\href{/section/obituaries}{Obituaries}\textbar{}Overlooked No More:
Valerie Solanas, Radical Feminist Who Shot Andy Warhol

\url{https://nyti.ms/2Nxwl0m}

\begin{itemize}
\item
\item
\item
\item
\item
\item
\end{itemize}

Advertisement

\protect\hyperlink{after-top}{Continue reading the main story}

Supported by

\protect\hyperlink{after-sponsor}{Continue reading the main story}

\hypertarget{overlooked-no-more-valerie-solanas-radical-feminist-who-shot-andy-warhol}{%
\section{Overlooked No More: Valerie Solanas, Radical Feminist Who Shot
Andy
Warhol}\label{overlooked-no-more-valerie-solanas-radical-feminist-who-shot-andy-warhol}}

She made daring arguments in ``SCUM Manifesto,'' her case for a world
without men. But it was her attack on Warhol that came to define her
life.

\includegraphics{https://static01.nyt.com/images/2020/06/29/obituaries/29overlooked-solanas-3/29overlooked-solanas-3-articleLarge-v2.jpg?quality=75\&auto=webp\&disable=upscale}

\href{https://www.nytimes.com/by/bonnie-wertheim}{\includegraphics{https://static01.nyt.com/images/2018/06/13/multimedia/author-bonnie-wertheim/author-bonnie-wertheim-thumbLarge.jpg}}

By \href{https://www.nytimes.com/by/bonnie-wertheim}{Bonnie Wertheim}

\begin{itemize}
\item
  June 26, 2020
\item
  \begin{itemize}
  \item
  \item
  \item
  \item
  \item
  \item
  \end{itemize}
\end{itemize}

\emph{Overlooked is a series of obituaries about remarkable people whose
deaths, beginning in 1851, went unreported in The Times. This month
we're adding the stories of important L.G.B.T.Q. figures.}

On June 3, 1968, Valerie Solanas walked into Andy Warhol's studio, the
Factory, with a gun and a plan to enact vengeance. What happened next
came to define her life and legacy: She fired at Warhol, nearly killing
him. The incident reduced her to a tabloid headline, but also drew
attention to her writing, which is still read in some women's and gender
studies courses today.

Solanas was a radical feminist (though she would say she loathed most
feminists), a pioneering queer theorist (at least according to some) and
the author of
``\href{http://kunsthallezurich.ch/sites/default/files/scum_manifesto.pdf}{SCUM
Manifesto},'' in which she argued for the wholesale extermination of
men.

The manifesto, self-published in 1967, reads as satire, though Solanas
defended it as serious. Its opening line is at once absurd and a call to
arms for the coalition she was forming, the Society for Cutting Up Men:

\begin{quote}
Life in this society being, at best, an utter bore and no aspect of
society being at all relevant to women, there remains to civic-minded,
responsible, thrill-seeking females only to overthrow the government,
eliminate the money system, institute complete automation and destroy
the male sex.
\end{quote}

On the subject of reproduction, she wrote: ``We should produce only
whole, complete beings, not physical defects or deficiencies, including
emotional deficiencies, such as maleness.''

She sold copies in leftist bookstores and on the streets of Greenwich
Village for \$1 (\$2 if the buyer was a man).

The text distilled the anger and yearning that Solanas had exhibited
throughout her life. In college, as a recently-out lesbian, she rallied
against the idea that educated women should be defined as wives and
mothers, even as she acknowledged that, in a society ruled by men, such
fates were probably inevitable. Her ideas about gender and power
calcified in the early 1960s, when she hitchhiked across the country and
back again. She arrived in New York City in 1962 with the start of a
play she was writing and several versions of ``SCUM Manifesto.''

Then, as now, Warhol was one of the most famous artists in America, and
Solanas found her way onto the fringes of his social circle. She shared
with him a copy of her play, ``Up Your Ass'' (1965), with the hope that
he would produce it. Its central character is Bongi Perez, a bantering,
panhandling prostitute who is frequently homeless --- much like Solanas
was herself. Auditions and rehearsals took place in the basement of the
Chelsea Hotel, the bohemian enclave from which Solanas was evicted on
several occasions. Warhol found the manuscript objectionable and
eventually misplaced it, but he did cast her in his erotic film ``I, a
Man'' (1967). (``Up Your Ass'' wouldn't be staged until long after her
death,
\href{https://www.villagevoice.com/2000/01/11/solanas-lost-and-found/}{in
2000 in San Francisco}.)

\includegraphics{https://static01.nyt.com/images/2020/06/29/obituaries/29overlooked-solanas-2/00overlooked-solanas-2-articleLarge.jpg?quality=75\&auto=webp\&disable=upscale}

Solanas then met with
\href{https://www.nytimes.com/1990/07/05/obituaries/maurice-girodias-a-french-publisher-and-an-author-71.html}{Maurice
Girodias}, the iconoclastic French publisher of Olympia Press who
printed the first editions of ``Naked Lunch'' (1959), ``The Story of O''
(1954) and ``Lolita'' (1955), about a deal for a new book.

Over time, Solanas became convinced that Warhol and Girodias were
conspiring to suppress, censor or steal her voice.

On that day in June, when she walked into Warhol's studio, newly located
at 33 Union Square West, Warhol wasn't there. Solanas left and returned
several times, until she spotted him on the sidewalk. Together they rode
the building's elevator up to the sixth floor.

Soon, there were gunshots. Warhol was taken to Columbus Hospital.
Solanas's bullets had punctured his stomach, liver, spleen, esophagus
and lungs. At one point, the doctors pronounced him dead. (He would live
for 19 more years, wearing a surgical corset to support his abdomen.)

That evening, Solanas turned herself in to an officer in Times Square.
``He had too much control over my life,'' she told the officer,
referring to Warhol.

Image

Andy Warhol's surgical scars and the corset that he wore to support his
abdomen after Solanas shot him in 1968, puncturing his stomach, liver,
spleen, esophagus and lungs.Credit...David Montgomery/Getty Images

Valerie Jean Solanas was born on April 9, 1936, in Ventnor City, N.J.,
just off the Atlantic City boardwalk, one of two girls to Louis Solanas,
a bartender, and Dorothy Biondo, a dental assistant. Her parents
separated when Valerie was 4 and divorced in 1947; both remarried. Her
father, she would later say, had sexually abused her from a young age.
Still, she retained a correspondence with him for most of her life.

Valerie was by some accounts a precocious child, but in middle school
she began to show signs of disobedience, skipping class and even
assaulting a teacher. By 15 she had given birth to two children: Linda,
who was raised as her sister, and David, whom she placed for adoption.
At the time, it was not unusual for pregnancies to be concealed by such
means.

In 1954, she enrolled at the University of Maryland, College Park, where
she studied psychology. She then pursued a master's degree in psychology
at the University of Minnesota but dropped out after two semesters
because, she said, she felt that her ideas and research were not very
likely to be funded as well as men's.

She spent the next decade putting her ideas to paper. She moved
frequently as a result of eviction, always with her typewriter in tow.

In 1966, her short story ``A Young Girl's Primer'' appeared in Cavalier,
a Playboy-style magazine that also published Ray Bradbury, Thomas
Pynchon and Stephen King. The tale centers on a woman who sells sex and
conversation for the freedom to be creative. The next year, she began
selling mimeographed copies of ``SCUM'' around the city and seeking a
producer for her play.

The shooting, in June of 1968, brought national attention to her name
and work. The story of the incident was splashed across the front pages
of papers like The Daily News in New York and
\href{https://timesmachine.nytimes.com/timesmachine/1968/06/04/91229352.html?pageNumber=1}{The
New York Times}, which misspelled her name as Solanis. Copies of
``SCUM'' quickly sold out.

Her attack on Warhol fractured mainstream feminist groups, including the
National Organization for Women, whose members were
\href{https://timesmachine.nytimes.com/timesmachine/1968/06/14/88953203.html?pdf_redirect=true\&ip=0\&pageNumber=52}{split
on whether to defend or condemn} her. Those who defended her, including
the writer Ti-Grace Atkinson and the lawyer Flo Kennedy, formed the
bedrock of radical feminism and presented Solanas as a symbol of female
rage. The shooting became wrapped up in a larger narrative on gun
violence when Senator Robert F. Kennedy was shot the next day.

Girodias published an edition of ``SCUM Manifesto'' after the shooting;
Solanas had unwittingly sold him the rights for \$500 the previous year.
Later editions were printed by AK Press and Verso.

During her arraignment, Solanas was charged with attempted murder,
assault and possession of a dangerous weapon.

She was deemed unable to stand trial and was sent for a psychiatric
evaluation at Elmhurst Hospital in Queens, where she received a
diagnosis of paranoid schizophrenia. The evaluators also noted her
intelligence-test scores, which placed her in the 98th percentile.

On June 13, Solanas was ruled insane by the Supreme Court of the State
of New York and spent months in psychiatric hospitals. When she was
released in December, she began calling Warhol, Girodias and others in a
group that she referred to as ``the mob'' with threatening messages.
They led to her arrest in January 1969.

Solanas was held at the Women's House of Detention in Manhattan, then at
Bellevue Hospital, before being sentenced to three years in prison in
June.

After her release, she worked for a year and a half as an editor for
Majority Report: The Women's Liberation Newsletter, a biweekly feminist
publication, and began writing a book, her name as the title. She spent
her final years destitute in Phoenix and living in welfare hotels in San
Francisco.

Image

After shooting Warhol, Solanas turned herself in to a police officer in
Times Square. She would spend the next few years in psychiatric
hospitals and in prison.Credit...Bettmann, via Getty Images

Toward the end of 1987,
\href{https://www.nytimes.com/2014/06/16/arts/ultra-violet-andy-warhol-superstar-dies-at-78.html}{Isabelle
Collin Dufresne}, the Factory ``superstar'' better known as Ultra
Violet, called Solanas to talk about Warhol, who had died that February.

``I keep thinking what a shame it is that she's mad, utterly mad,''
Ultra Violet wrote in her 1988 memoir, ``Famous for 15 Minutes: My Years
With Andy Warhol.'' ``For in the beginning, beyond her overheated
rhetoric, she had a truly revolutionary vision of a better world run by
and for the benefit of women.''

On April 25, 1988, Solanas was found dead in her room at the Bristol
Hotel, in the gritty Tenderloin neighborhood of San Francisco. She was
52. The police report, which also misspelled her name, described the
room as clean, with papers neatly stacked on a desk. Solanas was
kneeling next to the bed, covered in maggots, and had apparently been
dead for five days. The cited cause was pneumonia.

In 1996, her story was theatrically depicted in Mary Harron's film
``\href{https://archive.nytimes.com/www.nytimes.com/library/film/i_shot_andy_warhol.html}{I
Shot Andy Warhol}.'' Lili Taylor was widely praised for her leading
role.

Image

A scene from the 1996 film ``I Shot Andy Warhol.'' Lili Taylor was
widely praised for her role as Solanas.Credit...Samuel Goldwyn Company

Solanas inspired fictional works,
\href{https://www.hollywoodreporter.com/live-feed/american-horror-story-cult-story-beheind-lena-dunhams-episode-1050134\#:~:text=Lena\%20Dunham\%20made\%20her\%20one,Warhol\%20in\%20the\%20late\%201960s.\&text=In\%20the\%20flashback\%2C\%20Dunham\%20inhabited,who\%20had\%20diagnosed\%20paranoid\%20schizophrenia.}{including
an episode of} ``American Horror Story: Cult,'' where she is played by
Lena Dunham,
\href{https://www.nytimes.com/2019/08/21/t-magazine/summer-fall-books-illustrated.html}{and
a 2019 novel} by the Swedish author Sara Stridsberg, ``Valerie,'' which
won the Nordic Council Literature Prize and was longlisted for the Man
Booker Prize. By Stridsberg's account, Solanas was not erratic but
measured, not murderous but tender, not insane but idealistic, even
admirably so.

But it was
\href{https://artsbeat.blogs.nytimes.com/2014/04/23/a-sad-and-remarkable-life-breanne-fahs-talks-about-valerie-solanas/}{with
the 2014 biography} ``Valerie Solanas: The Defiant Life of the Woman Who
Wrote SCUM (and Shot Andy Warhol)'' that a fuller picture of her life
came to light.

In it, the author, Breanne Fahs, writes about an exchange between
Solanas and her friend Jeremiah Newton. Newton asked Solanas if her
manifesto was to be taken literally. ``I don't want to kill all men,''
she replied. But, using an expletive, she added: ``I think males should
be neutered or castrated so they can't mess up any more women's lives.''

Advertisement

\protect\hyperlink{after-bottom}{Continue reading the main story}

\hypertarget{site-index}{%
\subsection{Site Index}\label{site-index}}

\hypertarget{site-information-navigation}{%
\subsection{Site Information
Navigation}\label{site-information-navigation}}

\begin{itemize}
\tightlist
\item
  \href{https://help.nytimes.com/hc/en-us/articles/115014792127-Copyright-notice}{©~2020~The
  New York Times Company}
\end{itemize}

\begin{itemize}
\tightlist
\item
  \href{https://www.nytco.com/}{NYTCo}
\item
  \href{https://help.nytimes.com/hc/en-us/articles/115015385887-Contact-Us}{Contact
  Us}
\item
  \href{https://www.nytco.com/careers/}{Work with us}
\item
  \href{https://nytmediakit.com/}{Advertise}
\item
  \href{http://www.tbrandstudio.com/}{T Brand Studio}
\item
  \href{https://www.nytimes.com/privacy/cookie-policy\#how-do-i-manage-trackers}{Your
  Ad Choices}
\item
  \href{https://www.nytimes.com/privacy}{Privacy}
\item
  \href{https://help.nytimes.com/hc/en-us/articles/115014893428-Terms-of-service}{Terms
  of Service}
\item
  \href{https://help.nytimes.com/hc/en-us/articles/115014893968-Terms-of-sale}{Terms
  of Sale}
\item
  \href{https://spiderbites.nytimes.com}{Site Map}
\item
  \href{https://help.nytimes.com/hc/en-us}{Help}
\item
  \href{https://www.nytimes.com/subscription?campaignId=37WXW}{Subscriptions}
\end{itemize}
