Sections

SEARCH

\protect\hyperlink{site-content}{Skip to
content}\protect\hyperlink{site-index}{Skip to site index}

\href{https://myaccount.nytimes.com/auth/login?response_type=cookie\&client_id=vi}{}

\href{https://www.nytimes.com/section/todayspaper}{Today's Paper}

\href{/section/opinion}{Opinion}\textbar{}Beyond `White Fragility'

\href{https://nyti.ms/2CJnpCZ}{https://nyti.ms/2CJnpCZ}

\begin{itemize}
\item
\item
\item
\item
\item
\item
\end{itemize}

Advertisement

\protect\hyperlink{after-top}{Continue reading the main story}

\href{/section/opinion}{Opinion}

Supported by

\protect\hyperlink{after-sponsor}{Continue reading the main story}

\hypertarget{beyond-white-fragility}{%
\section{Beyond `White Fragility'}\label{beyond-white-fragility}}

If you want to let freedom ring, hammer on economic injustice.

\href{https://www.nytimes.com/column/jamelle-bouie}{\includegraphics{https://static01.nyt.com/images/2019/01/24/opinion/jamelle-bouie/jamelle-bouie-thumbLarge-v3.png}}

By \href{https://www.nytimes.com/column/jamelle-bouie}{Jamelle Bouie}

Opinion Columnist

\begin{itemize}
\item
  June 26, 2020
\item
  \begin{itemize}
  \item
  \item
  \item
  \item
  \item
  \item
  \end{itemize}
\end{itemize}

\includegraphics{https://static01.nyt.com/images/2020/06/29/opinion/29bouie_print1/merlin_173318319_f0cdd4ca-223e-411b-907a-b45f572ed61c-articleLarge.jpg?quality=75\&auto=webp\&disable=upscale}

Since it emerged \href{https://blacklivesmatter.com/herstory/}{seven
years ago} in response to the acquittal of George Zimmerman in the
shooting of Trayvon Martin, the Black Lives Matter movement has produced
a sea change in attitudes, politics and policy.

\href{https://www.pewresearch.org/fact-tank/2016/07/08/how-americans-view-the-black-lives-matter-movement/}{In
2016}, 43 percent of Americans supported Black Lives Matter and its
claims about the criminal justice system;
\href{https://www.pewsocialtrends.org/2020/06/12/amid-protests-majorities-across-racial-and-ethnic-groups-express-support-for-the-black-lives-matter-movement/}{now},
it's up to 67 percent, with 60 percent support among white Americans,
compared with 40 percent four years ago. Whereas Democratic politicians
once
\href{https://www.npr.org/sections/itsallpolitics/2015/07/31/427851451/democratic-candidates-stumble-over-black-lives-matter-movement}{stumbled}
over the issue, now even Republicans are
\href{https://www.theatlantic.com/politics/archive/2020/06/mitt-romney-black-lives-matter/612808/}{falling
over themselves} to say that ``black lives matter.'' And where the
policy conversation was formerly focused on body cameras and chokehold
bans, now mainstream outlets are debating and taking seriously calls to
demilitarize and
\href{https://www.brookings.edu/blog/fixgov/2020/06/19/what-does-defund-the-police-mean-and-does-it-have-merit/}{defund}
police departments or to
\href{https://www.nytimes.com/2020/06/12/opinion/sunday/floyd-abolish-defund-police.html}{abolish}
them outright.

But the Black Lives Matter platform isn't just about criminal justice.
From the start, activists have articulated a
\href{https://m4bl.org/policy-platforms/}{broad, inclusive vision} for
the entire country. This, in fact, has been true of each of the nation's
major movements for racial equality. Among black Americans and their
Radical Republican allies, Reconstruction --- which was still ongoing as
of 150 years ago --- was as much a fight to fundamentally reorder
Southern economic life as it was a struggle for political inclusion. The
struggle against Jim Crow, likewise, was also a struggle for economic
equality and the transformation of society.

``The black revolution is much more than a struggle for the rights of
Negroes,'' the Rev. Martin Luther King Jr. wrote in ``A Testament of
Hope'':

\begin{quote}
It is forcing America to face all its interrelated flaws --- racism,
poverty, militarism and materialism. It is exposing evils that are
rooted deeply in the whole structure of our society. It reveals systemic
rather than superficial flaws and suggests that radical reconstruction
of society itself is the real issue to be faced.
\end{quote}

Our society was built on the racial segmentation of personhood. Some
people were full humans, guaranteed non-enslavement, secured from
expropriation and given the protection of law, and some people ---
blacks, Natives and other nonwhites --- were not. That unequal
distribution of personhood was an economic reality as well. It shaped
your access to employment and capital; determined whether you would be
doomed to the margins of labor or given access to its elevated ranks;
marked who might share in the bounty of capitalist production and who
would most likely be cast out as disposable.

In our society, in other words, the fight for equal personhood can't
help but also be a struggle for economic justice. And what we see, past
and present, is how that fight against the privileges and distinctions
of race can also lay the foundations for a broader assault on the
privileges and distinctions of class.

As soon as the Civil War came to a close, it was clear there could be no
actual freedom for the formerly enslaved without a fundamental
transformation of economic relations. ``We must see that the freedman
are established on the soil, and that they may become proprietors,''
Charles Sumner, the Radical Republican senator from Massachusetts,
\href{https://www.google.com/books/edition/Black_Reconstruction_in_America/DmymDwAAQBAJ?hl=en\&gbpv=1\&dq=black\%20reconstruction\&pg=PA177\&printsec=frontcover\&bsq=\%E2\%80\%9Cwe\%20must\%20see\%20that\%20the\%20freedmen\%E2\%80\%9D}{wrote}
in March 1865. ``The great plantations, which have been so many
nurseries of the rebellion, must be broken up, and the freedmen must
have the pieces.'' Likewise,
\href{https://www.google.com/books/edition/Black_Reconstruction_in_America/DmymDwAAQBAJ?hl=en\&gbpv=1\&dq=black\%20reconstruction\&pg=PA176\&printsec=frontcover\&bsq=\%E2\%80\%9Cspent\%20in\%20vain\%E2\%80\%9D}{said}
the Radical Republican congressman Thaddeus Stevens in September 1865,
``The whole fabric of Southern society must be changed, and never can it
be done if this opportunity is lost.'' The foundations of their
institutions, he continued, ``must be broken up and re-laid, or all of
our blood and treasure have been spent in vain.''

Presidential Reconstruction under Andrew Johnson, a Democrat, would
immediately undermine any means to this end, as he restored defeated
Confederates to citizenship and gave them free rein to impose laws, like
the Black Codes, which sought to reestablish the economic and social
conditions of slavery. But Republicans in Congress were eventually able
to wrest control of Reconstruction from the administration, and just as
importantly, black Americans were actively taking steps to secure their
political freedom against white reactionary opposition. Working through
the Union Army, postwar Union Leagues and the Republican Party, freed
and free blacks worked toward a common goal of political equality. And
once they secured something like it, they set out to try as much as
possible to affect that economic transformation.

``Public schools, hospitals, penitentiaries, and asylums for orphans and
the insane were established for the first time or received increased
funding,'' the historian Eric Foner wrote in
``\href{https://www.harpercollins.com/9780062354518/reconstruction-updated-edition/}{Reconstruction}:
America's Unfinished Revolution, 1863-1877.'' ``South Carolina funded
medical care for poor citizens, and Alabama provided free legal counsel
for indigent defendants.''

For blacks and Radical Republicans, Reconstruction was an attempt to
secure political rights for the sake transforming the entire society.
And its end had as much to do with the reaction of property and capital
owners as it did with racist violence. ``The bargain of 1876,'' W.E.B.
Du Bois
\href{https://www.google.com/books/edition/Black_Reconstruction_in_America/IqDEhQtoYEkC?hl=en\&gbpv=1\&pg=PA563\&printsec=frontcover\&bsq=ceased}{wrote}
in ``Black Reconstruction in America,''

\begin{quote}
was essentially an understanding by which the Federal Government ceased
to sustain the right to vote of half of the laboring population of the
South, and left capital as represented by the old planter class, the new
Northern capitalist, and the capitalist that began to rise out of the
poor whites, with a control of labor greater than in any modern
industrial state in civilized lands.
\end{quote}

Out of that, he continued, ``has arisen in the South an exploitation of
labor unparalleled in modern times, with a government in which all
pretense at party alignment or regard for universal suffrage is given
up.''

Du Bois was writing in the 1930s. A quarter-century later, black
Americans in the South would launch a movement to unravel Jim Crow
repression and economic exploitation. And as that movement progressed
and notched victories against segregation, it became clear that the next
step was to build a coalition against the privileges of class, since the
two were inextricably tied together. The Memphis sanitation workers who
\href{https://www.vox.com/identities/2018/2/12/17004552/mlk-memphis-sanitation-strike-poor-peoples-campaign}{asked}
Martin Luther King Jr. to support their strike in 1968 were black, set
against a white power structure in the city. Their oppression as black
Americans and subjugation as workers were tied together. Unraveling one
could not be accomplished without unraveling the other.

All of this relates back to the relationship between race and
capitalism. To end segregation --- of housing, of schools, of workplaces
--- is to undo one of the major ways in which labor is exploited, caste
established and the ideologies of racial hierarchy sustained. And that,
in turn, opens possibilities for new avenues of advancement. The old
labor slogan
``\href{https://www.press.uillinois.edu/books/catalog/26nmf7cc9780252066214.html}{Negro
and White, Unite and Fight!}'' contains more than a little truth about
the necessary conditions for economic justice. That this unity is fairly
rare in American history is a testament to how often these movements
have ``either advocated, capitulated before, or otherwise failed to
oppose racism at one or more critical junctures in their history,'' as
Robert L. Allen and Pamela P. Allen note in their
\href{https://books.google.com/books/about/Reluctant_Reformers.html?id=Yfu_QgAACAAJ}{1974
study} of racism and social reform movements.

Which brings us back to the present. The activists behind the Black
Lives Matter movement have always connected its aims to working-class,
egalitarian politics. The platform of the Movement for Black Lives, as
it is formally known, includes demands for universal health care,
affordable housing, living wage employment and access to education and
public transportation. Given the extent to which class shapes black
exposure to police violence --- it is poor and working class black
Americans who are most likely to live in neighborhoods marked by
constant police surveillance --- calls to defund and dismantle existing
police departments are a class demand like any other.

But while the movement can't help but be about practical concerns, the
predominating discourse of belief and intention overshadows those
stakes: too much concern with ``white fragility'' and not enough with
wealth inequality. The challenge is to bridge the gap; to show new
supporters that there's far more work to do than changing the way we
police; to channel their sympathy into a deeper understanding of the
problem at hand.

To put a final point of emphasis on the potential of the moment, I'll
leave you with this. In a 1963 pamphlet called
``\href{https://books.google.com/books/about/The_American_revolution.html?id=p2MOAQAAMAAJ}{The
American Revolution}: Pages from a Negro Worker's Notebook,'' the
activist and laborer James Boggs argued for the revolutionary potential
of the black struggle for civil rights. ``The strength of the Negro
cause and its power to shake up the social structure of the nation,''
Boggs wrote, ``comes from the fact that in the Negro struggle all the
questions of human rights and human relationships are posed.'' That is
because it is a struggle for equality ``in production, in consumption,
in the community, in the courts, in the schools, in the universities, in
transportation, in social activity, in government, and indeed in every
sphere of American life.''

\begin{center}\rule{0.5\linewidth}{\linethickness}\end{center}

\includegraphics{https://static01.nyt.com/images/2020/06/28/opinion/28bouie_print2/merlin_173322873_9e85fe75-2534-4098-ae12-c760b1badb05-articleLarge.jpg?quality=75\&auto=webp\&disable=upscale}

\emph{The Times is committed to publishing}
\href{https://www.nytimes.com/2019/01/31/opinion/letters/letters-to-editor-new-york-times-women.html}{\emph{a
diversity of letters}} \emph{to the editor. We'd like to hear what you
think about this or any of our articles. Here are some}
\href{https://help.nytimes.com/hc/en-us/articles/115014925288-How-to-submit-a-letter-to-the-editor}{\emph{tips}}\emph{.
And here's our email:}
\href{mailto:letters@nytimes.com}{\emph{letters@nytimes.com}}\emph{.}

\emph{Follow The New York Times Opinion section on}
\href{https://www.facebook.com/nytopinion}{\emph{Facebook}}\emph{,}
\href{http://twitter.com/NYTOpinion}{\emph{Twitter (@NYTopinion)}}
\emph{and}
\href{https://www.instagram.com/nytopinion/}{\emph{Instagram}}\emph{.}

Advertisement

\protect\hyperlink{after-bottom}{Continue reading the main story}

\hypertarget{site-index}{%
\subsection{Site Index}\label{site-index}}

\hypertarget{site-information-navigation}{%
\subsection{Site Information
Navigation}\label{site-information-navigation}}

\begin{itemize}
\tightlist
\item
  \href{https://help.nytimes.com/hc/en-us/articles/115014792127-Copyright-notice}{©~2020~The
  New York Times Company}
\end{itemize}

\begin{itemize}
\tightlist
\item
  \href{https://www.nytco.com/}{NYTCo}
\item
  \href{https://help.nytimes.com/hc/en-us/articles/115015385887-Contact-Us}{Contact
  Us}
\item
  \href{https://www.nytco.com/careers/}{Work with us}
\item
  \href{https://nytmediakit.com/}{Advertise}
\item
  \href{http://www.tbrandstudio.com/}{T Brand Studio}
\item
  \href{https://www.nytimes.com/privacy/cookie-policy\#how-do-i-manage-trackers}{Your
  Ad Choices}
\item
  \href{https://www.nytimes.com/privacy}{Privacy}
\item
  \href{https://help.nytimes.com/hc/en-us/articles/115014893428-Terms-of-service}{Terms
  of Service}
\item
  \href{https://help.nytimes.com/hc/en-us/articles/115014893968-Terms-of-sale}{Terms
  of Sale}
\item
  \href{https://spiderbites.nytimes.com}{Site Map}
\item
  \href{https://help.nytimes.com/hc/en-us}{Help}
\item
  \href{https://www.nytimes.com/subscription?campaignId=37WXW}{Subscriptions}
\end{itemize}
