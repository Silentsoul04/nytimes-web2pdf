Sections

SEARCH

\protect\hyperlink{site-content}{Skip to
content}\protect\hyperlink{site-index}{Skip to site index}

\href{https://www.nytimes.com/section/science}{Science}

\href{https://myaccount.nytimes.com/auth/login?response_type=cookie\&client_id=vi}{}

\href{https://www.nytimes.com/section/todayspaper}{Today's Paper}

\href{/section/science}{Science}\textbar{}Kenneth Lewes, Who Challenged
Views of Homosexuality, Dies at 76

\url{https://nyti.ms/3eBInl7}

\begin{itemize}
\item
\item
\item
\item
\item
\end{itemize}

\href{https://www.nytimes.com/news-event/coronavirus?action=click\&pgtype=Article\&state=default\&region=TOP_BANNER\&context=storylines_menu}{The
Coronavirus Outbreak}

\begin{itemize}
\tightlist
\item
  live\href{https://www.nytimes.com/2020/08/03/world/coronavirus-covid-19.html?action=click\&pgtype=Article\&state=default\&region=TOP_BANNER\&context=storylines_menu}{Latest
  Updates}
\item
  \href{https://www.nytimes.com/interactive/2020/us/coronavirus-us-cases.html?action=click\&pgtype=Article\&state=default\&region=TOP_BANNER\&context=storylines_menu}{Maps
  and Cases}
\item
  \href{https://www.nytimes.com/interactive/2020/science/coronavirus-vaccine-tracker.html?action=click\&pgtype=Article\&state=default\&region=TOP_BANNER\&context=storylines_menu}{Vaccine
  Tracker}
\item
  \href{https://www.nytimes.com/2020/08/02/us/covid-college-reopening.html?action=click\&pgtype=Article\&state=default\&region=TOP_BANNER\&context=storylines_menu}{College
  Reopening}
\item
  \href{https://www.nytimes.com/live/2020/08/03/business/stock-market-today-coronavirus?action=click\&pgtype=Article\&state=default\&region=TOP_BANNER\&context=storylines_menu}{Economy}
\end{itemize}

Advertisement

\protect\hyperlink{after-top}{Continue reading the main story}

Supported by

\protect\hyperlink{after-sponsor}{Continue reading the main story}

Those We've Lost

\hypertarget{kenneth-lewes-who-challenged-views-of-homosexuality-dies-at-76}{%
\section{Kenneth Lewes, Who Challenged Views of Homosexuality, Dies at
76}\label{kenneth-lewes-who-challenged-views-of-homosexuality-dies-at-76}}

In an influential book, he defied the idea that being gay, as he was, is
an illness, and took on psychiatry's ``history of homophobia.'' He died
of the coronavirus.

\includegraphics{https://static01.nyt.com/images/2020/06/28/obituaries/28Lewes-obit1/26Lewes1-articleLarge.jpg?quality=75\&auto=webp\&disable=upscale}

\href{https://www.nytimes.com/by/sam-roberts}{\includegraphics{https://static01.nyt.com/images/2018/02/20/multimedia/author-sam-roberts/author-sam-roberts-thumbLarge.jpg}}

By \href{https://www.nytimes.com/by/sam-roberts}{Sam Roberts}

\begin{itemize}
\item
  June 26, 2020
\item
  \begin{itemize}
  \item
  \item
  \item
  \item
  \item
  \end{itemize}
\end{itemize}

\emph{This obituary is part of a series about people who have died in
the coronavirus pandemic. Read about others}
\href{https://www.nytimes.com/interactive/2020/obituaries/people-died-coronavirus-obituaries.html}{\emph{here}}\emph{.}

Kenneth Lewes grew up after World War II in a working-class neighborhood
of the northeast Bronx, the son of an immigrant couple who never got
beyond grade school. He guessed even before he entered junior high
school that he was gay.

But it wasn't until he was nearly 50 --- and publishing what would
become a critically acclaimed takedown of post-Freudian psychoanalytic
theories of homosexuality --- that he confided his sexual orientation to
his parents.

``I remember finding my way to the local public library and checking out
books on psychology and human development,'' he said in an interview in
2019 with the Journal of Gay \& Lesbian Mental Health, ``in hopes of
finding some reassurance that my interest in handsome boys was only a
stage that I would soon pass through.''

Dr. Lewes (pronounced LOO-ess) was married at 23 and divorced by 32 ---
the age when he had his first homosexual experience.

``It seemed only natural for me to be out of the closet to my friends,
colleagues and family,'' he said, ``with the important exception of my
parents, who, it had become clear over the years, did not want to hear
anything on that particular subject. I came out to them almost 15 years
later.''

Dr. Lewes died of the new coronavirus on April 17 in a Manhattan
hospital, his partner, Gary Jacobson, said. He was 76.

He is also survived by his sister, Noreen Vasady-Kovacs.

Image

''Credit...

Dr. Lewes's major work,
\href{https://www.nytimes.com/1988/12/11/books/navigating-the-straits-of-oedipus.html}{``The
Psychoanalytic Theory of Male Homosexuality''} (1988), traced the
evolution of the prevailing view that homosexuality was a curable
illness and explored what he called the psychoanalytic establishment's
``century-long history of homophobia.'' (The book's title was changed to
\href{https://catalog.loc.gov/vwebv/search?searchCode=LCCN\&searchArg=2008944236\&searchType=1\&permalink=y}{``Psychoanalysis
and Male Homosexuality''} in later editions.)

Drawing on some 500 primary sources, Dr. Lewes's book, which expanded on
his doctoral dissertation, found that most analysts had adhered to
``popular prejudice'' against gay people and clichés about them. ``Many
analysts,'' he concluded, ``have violated basic norms of decency in
their treatment of homosexuals.''

He said he had been unable to find a single analysis of the subject
written by a psychoanalyst who identified as gay.

In his review of the book in
\href{https://www.nytimes.com/1988/12/11/books/navigating-the-straits-of-oedipus.html}{The
New York Times Book Review}, Richard Green, a professor of psychiatry
and the law at the University of California, Los Angeles, wrote:

``A major fault in the bedrock of analytic theory has been, according to
Mr. Lewes, a monumental misunderstanding of the Oedipus complex, long
considered the rite of passage to normal, healthy heterosexuality. This
misunderstanding created a false dichotomy between heterosexual wellness
and homosexual sickness.''

Dr. Lewes found that the complex could lead to 12 alternative
resolutions, six of them heterosexual and six homosexual. ``All results
of the Oedipus complex are traumatic,'' he wrote, ``and, for similar
reasons, all are `normal.'''

Dr. Green
\href{https://www.nytimes.com/2019/04/17/obituaries/dr-richard-green-dead.html}{(who
died last year)}, one of the earliest critics of psychiatry's
classification of homosexuality as a mental disorder, praised Dr. Lewes
for tracking ``the politicized, moralistic and occasionally objective
evolution of psychoanalytic theories of male homosexuality from the
enlightened flexibility of Freud to the benighted dogmatism of his
disciples,'' adding that ``the history of how the single most
influential school of psychology and psychiatry abused its power and
mishandled the most politically and morally controversial of behaviors''
constituted Dr. Lewes's ``telling impact.''

Kenneth Allen Lewes was born on June 8, 1943, in Charleston, W.Va., to
Joseph and Anne (Harvin) Lewes. His father was an English-born furniture
maker and antique restorer for the National Trust for Historic
Preservation; his mother, born in Czechoslovakia, was a homemaker.

The family moved to New York in 1947. There, intending to become a
mathematician, Kenneth enrolled in the Bronx High School of Science,
graduating in 1960. He then shifted gears, earning a bachelor's degree
in English from Cornell University in 1964 and a doctorate in
Renaissance English literature from Harvard.

At the age of 36, after seven years as a professor of Renaissance
literature at Rutgers University in New Jersey --- and without ever
having taken a psychology course --- he made another transition and
enrolled at the University of Michigan, where he earned a second
doctorate, this time in clinical psychology, in 1982.

His study of English proved surprisingly serviceable, though.

``Literary criticism is generally better equipped to understand the
depth and complexity of symbols,'' Dr. Lewes explained in the 2019
journal interview. ``I vividly remember the first time I conducted a
therapy group in a closed ward of a state mental hospital. I felt
instantly at home. Here were people trying to put into words their
deepest intuitions about life, much as people in the 17th century were
obsessed with theological debate. The people in my therapy group,
however, had not been dead for 300 years.''

He welcomed the greater acceptance of gay men by the psychiatric
profession and society in general in recent decades. But with that
progress he also lamented something lost --- what he called ``the gay
outlaw, the defier and challenger of traditional social values, the
person who insisted that we find our own ways of being in society and
not subscribe to the traditional values and limitations that stunted so
many lives.''

``Instead of him,'' he said, ``we now have the friendly next-door
neighbor, who may have adopted a child or two as well as an obligatory
dog, and who would never think of challenging values that most Americans
assume are timeless and part of nature.''

\href{https://www.nytimes.com/interactive/2020/obituaries/people-died-coronavirus-obituaries.html?action=click\&pgtype=Article\&state=default\&region=BELOW_MAIN_CONTENT\&context=covid_obits_promo}{}

\hypertarget{those-weve-lost}{%
\section{Those We've Lost}\label{those-weve-lost}}

The coronavirus pandemic has taken an incalculable death toll. This
series is designed to put names and faces to the numbers.

Read more

\includegraphics{https://static01.nyt.com/images/2020/07/30/obituaries/30Pedro/30Pedro-square640.jpg}

\hypertarget{bernaldina-josuxe9-pedro}{%
\section{Bernaldina José Pedro}\label{bernaldina-josuxe9-pedro}}

d. Boa Vista, Brazil

Leader among the Indigenous Macuxi

\includegraphics{https://static01.nyt.com/images/2020/07/31/obituaries/31Swing/merlin_175167783_8913bc90-0d64-43f3-a655-1bb1bf1601c9-square640.jpg}

\hypertarget{john-eric-swing}{%
\section{John Eric Swing}\label{john-eric-swing}}

d. Fountain Valley, Calif.

Champion of Filipino-Americans

\includegraphics{https://static01.nyt.com/images/2020/07/27/obituaries/27Victor/merlin_175001436_38b11f8e-227a-4e2c-9821-7618af9b2524-square640.jpg}

\hypertarget{victor-victor}{%
\section{Victor Victor}\label{victor-victor}}

d. Santo Domingo, Dominican Republic

Beloved musician of the Dominican Republic

\includegraphics{https://static01.nyt.com/images/2020/07/31/obituaries/31Negron/merlin_175160169_516322ae-fd23-4969-b6b2-193ced371105-square640.jpg}

\hypertarget{dr-eddie-negruxf3n}{%
\section{Dr. Eddie Negrón}\label{dr-eddie-negruxf3n}}

d. Fort Walton Beach, Fla.

Internist on Florida's Emerald Coast

\includegraphics{https://static01.nyt.com/images/2020/07/30/obituaries/30Dobson/merlin_175115928_f6b9271c-8f05-4fe1-a38a-5ca4a58f8935-square640.jpg}

\hypertarget{dobby-dobson}{%
\section{Dobby Dobson}\label{dobby-dobson}}

d. Coral Springs, Fla.

Jamaican singer and songwriter

\includegraphics{https://static01.nyt.com/images/2020/08/01/obituaries/28Gonzalez/merlin_175002771_beb57888-3951-409a-ae13-03a94b2e962e-square640.jpg}

\hypertarget{waldemar-gonzalez}{%
\section{Waldemar Gonzalez}\label{waldemar-gonzalez}}

d. White Plains, N.Y.

Teacher and social worker

Advertisement

\protect\hyperlink{after-bottom}{Continue reading the main story}

\hypertarget{site-index}{%
\subsection{Site Index}\label{site-index}}

\hypertarget{site-information-navigation}{%
\subsection{Site Information
Navigation}\label{site-information-navigation}}

\begin{itemize}
\tightlist
\item
  \href{https://help.nytimes.com/hc/en-us/articles/115014792127-Copyright-notice}{©~2020~The
  New York Times Company}
\end{itemize}

\begin{itemize}
\tightlist
\item
  \href{https://www.nytco.com/}{NYTCo}
\item
  \href{https://help.nytimes.com/hc/en-us/articles/115015385887-Contact-Us}{Contact
  Us}
\item
  \href{https://www.nytco.com/careers/}{Work with us}
\item
  \href{https://nytmediakit.com/}{Advertise}
\item
  \href{http://www.tbrandstudio.com/}{T Brand Studio}
\item
  \href{https://www.nytimes.com/privacy/cookie-policy\#how-do-i-manage-trackers}{Your
  Ad Choices}
\item
  \href{https://www.nytimes.com/privacy}{Privacy}
\item
  \href{https://help.nytimes.com/hc/en-us/articles/115014893428-Terms-of-service}{Terms
  of Service}
\item
  \href{https://help.nytimes.com/hc/en-us/articles/115014893968-Terms-of-sale}{Terms
  of Sale}
\item
  \href{https://spiderbites.nytimes.com}{Site Map}
\item
  \href{https://help.nytimes.com/hc/en-us}{Help}
\item
  \href{https://www.nytimes.com/subscription?campaignId=37WXW}{Subscriptions}
\end{itemize}
