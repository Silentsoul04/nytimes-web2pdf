Sections

SEARCH

\protect\hyperlink{site-content}{Skip to
content}\protect\hyperlink{site-index}{Skip to site index}

\href{https://www.nytimes.com/section/technology}{Technology}

\href{https://myaccount.nytimes.com/auth/login?response_type=cookie\&client_id=vi}{}

\href{https://www.nytimes.com/section/todayspaper}{Today's Paper}

\href{/section/technology}{Technology}\textbar{}Facebook Adds Labels for
Some Posts as Advertisers Pull Back

\url{https://nyti.ms/3dE3aTM}

\begin{itemize}
\item
\item
\item
\item
\item
\end{itemize}

\begin{itemize}
\item
  \href{https://www.nytimes.com/2020/07/31/us/elections/biden-vs-trump.html?action=click\&pgtype=Article\&state=default\&region=TOP_BANNER\&context=storylines_menu}{Election
  Updates}
\item
  \href{https://www.nytimes.com/article/biden-vice-president-2020.html?action=click\&pgtype=Article\&state=default\&region=TOP_BANNER\&context=storylines_menu}{Biden's
  V.P. Search}
\item
  \href{https://www.nytimes.com/interactive/2020/07/24/us/politics/trump-biden-campaign-donors.html?action=click\&pgtype=Article\&state=default\&region=TOP_BANNER\&context=storylines_menu}{Map
  of Donations}
\item
  \href{https://www.nytimes.com/interactive/2020/us/elections/delegate-count-primary-results.html?action=click\&pgtype=Article\&state=default\&region=TOP_BANNER\&context=storylines_menu}{Delegate
  Count}
\item
  \href{https://www.nytimes.com/interactive/2019/us/politics/2020-presidential-candidates.html?action=click\&pgtype=Article\&state=default\&region=TOP_BANNER\&context=storylines_menu}{The
  Candidates}
\item
  \href{https://www.nytimes.com/newsletters/politics?action=click\&pgtype=Article\&state=default\&region=TOP_BANNER\&context=storylines_menu}{Politics
  Newsletter}
\end{itemize}

Advertisement

\protect\hyperlink{after-top}{Continue reading the main story}

Supported by

\protect\hyperlink{after-sponsor}{Continue reading the main story}

\hypertarget{facebook-adds-labels-for-some-posts-as-advertisers-pull-back}{%
\section{Facebook Adds Labels for Some Posts as Advertisers Pull
Back}\label{facebook-adds-labels-for-some-posts-as-advertisers-pull-back}}

Posts about voting will direct viewers to accurate information, and
violations from important political figures will be marked
``newsworthy.''

\includegraphics{https://static01.nyt.com/images/2020/06/26/business/26facebook/26facebook-articleLarge.jpg?quality=75\&auto=webp\&disable=upscale}

\href{https://www.nytimes.com/by/mike-isaac}{\includegraphics{https://static01.nyt.com/images/2018/02/16/multimedia/author-mike-isaac/author-mike-isaac-thumbLarge.jpg}}\href{https://www.nytimes.com/by/sheera-frenkel}{\includegraphics{https://static01.nyt.com/images/2018/06/14/multimedia/author-sheera-frenkel/author-sheera-frenkel-thumbLarge.png}}

By \href{https://www.nytimes.com/by/mike-isaac}{Mike Isaac} and
\href{https://www.nytimes.com/by/sheera-frenkel}{Sheera Frenkel}

\begin{itemize}
\item
  June 26, 2020
\item
  \begin{itemize}
  \item
  \item
  \item
  \item
  \item
  \end{itemize}
\end{itemize}

SAN FRANCISCO --- Facebook rolled out measures on Friday to add more
context to problematic political posts on its site, as the social
network grappled with a growing outcry from some of its largest
advertisers over the issue of hateful speech.

Facebook said it would attach labels to all posts across its network
that discuss the subject of voting, in a move intended to hamper any
disenfranchisement of voters in the November election. The labels will
direct users to accurate voting information, the company said.

In addition, Facebook said it would expand its policies around hate
speech and prohibit a wider category of hateful language in ads on the
site. A post that violates Facebook's rules but is from an important
political figure, such as President Trump, will get a label saying it
was deemed ``newsworthy'' enough to remain, the company said.

Facebook has been trying to deal with its role in spreading
disinformation and divisive content. The Silicon Valley company has been
under fire for allowing inaccurate or inflammatory posts from Mr. Trump
to remain unaltered on its site, even as
\href{https://www.nytimes.com/2020/05/30/technology/twitter-trump-dorsey.html}{Twitter
has attached fact checks and warnings} to the same content on its
service.

Mark Zuckerberg, Facebook's chief executive, has said that
\href{https://www.nytimes.com/2019/10/17/business/zuckerberg-facebook-free-speech.html}{he
believes in supporting free speech} and that posts from political
leaders should not be policed because they are in the public's interests
to view and read. But critics have said Mr. Zuckerberg is simply
allowing hateful speech to flourish on the social network with few
limits.

In recent weeks, Facebook has faced increasing opposition to its
position on hateful speech from one of its most important constituents:
advertisers, which generate the bulk of its
\href{https://investor.fb.com/investor-news/press-release-details/2020/Facebook-Reports-Fourth-Quarter-and-Full-Year-2019-Results/default.aspx}{\$70.7
billion} in annual revenue. Brands like Eddie Bauer, Ben \& Jerry's and
Magnolia Pictures have announced that they will
\href{https://www.nytimes.com/2020/06/23/business/media/facebook-ad-boycott.html}{cease
buying advertising on Facebook} until it reconsiders its stance.

\hypertarget{latest-updates-2020-election}{%
\section{\texorpdfstring{\href{https://www.nytimes.com/2020/07/31/us/elections/biden-vs-trump.html?action=click\&pgtype=Article\&state=default\&region=MAIN_CONTENT_1\&context=storylines_live_updates}{Latest
Updates: 2020
Election}}{Latest Updates: 2020 Election}}\label{latest-updates-2020-election}}

Updated 2020-08-01T01:26:45.732Z

\begin{itemize}
\tightlist
\item
  \href{https://www.nytimes.com/2020/07/31/us/elections/biden-vs-trump.html?action=click\&pgtype=Article\&state=default\&region=MAIN_CONTENT_1\&context=storylines_live_updates\#link-29fdff45}{Kamala
  Harris, a top vice-presidential contender, confronts double
  standards.}
\item
  \href{https://www.nytimes.com/2020/07/31/us/elections/biden-vs-trump.html?action=click\&pgtype=Article\&state=default\&region=MAIN_CONTENT_1\&context=storylines_live_updates\#link-13ec3d9c}{Karen
  Bass and Susan Rice are rising on Biden's vice-presidential
  shortlist.}
\item
  \href{https://www.nytimes.com/2020/07/31/us/elections/biden-vs-trump.html?action=click\&pgtype=Article\&state=default\&region=MAIN_CONTENT_1\&context=storylines_live_updates\#link-49e9a016}{Trump
  says Russian bounties to kill U.S. troops `never took place.'}
\end{itemize}

\href{https://www.nytimes.com/2020/07/31/us/elections/biden-vs-trump.html?action=click\&pgtype=Article\&state=default\&region=MAIN_CONTENT_1\&context=storylines_live_updates}{See
more updates}

On Friday, more companies said they would pull back from advertising on
Facebook because of hateful speech remaining on the site. They included
Unilever, the British-Dutch maker of consumer goods and one of
Facebook's largest advertisers;
\href{https://www.cnbc.com/2020/06/26/coca-cola-pauses-advertising-on-all-social-media-platforms-globally.html?__source=twitter\%7Cmain}{Coca-Cola},
which is another big advertiser on social media; and
\href{https://twitter.com/levis/status/1276670275732398081?s=21}{Levi
Strauss}, the maker of Levi's and Dockers clothing. On Thursday, Verizon
also said it was
\href{https://www.theverge.com/2020/6/25/21303717/verison-facebook-adl-ad-boycott-instagram-north-face-rei-ben-and-jerrys}{pausing
its advertising} on Facebook.

``The stakes are too high,'' said Steve Lesnard, vice president of
marketing at the North Face, a clothing brand that is participating in
the ad boycott. ``The platform needs to evolve.''

Facebook has also been grappling with an internal uproar over Mr.
Trump's inflammatory posts.
\href{https://www.nytimes.com/2020/06/01/technology/facebook-employee-protest-trump.html}{Employees
staged a virtual walkout} this month in protest of Mr. Zuckerberg's
position of allowing the posts to remain. Some of the company's earliest
workers have also
\href{https://www.nytimes.com/2020/06/03/technology/facebook-trump-employees-letter.html}{implored
the chief executive to change his mind} in an open letter.

Mr. Zuckerberg has refused to budge, though he said he and others on his
policy team will review the company's rules.

Since then, Facebook has made modifications that do not require it to
pull down hateful speech but that give people more options with such
posts. The company said this month that it would allow people in the
United States to
\href{https://www.nytimes.com/2020/06/16/technology/opt-out-political-ads-facebook.html}{opt
out of seeing social-issue, electoral or political ads} from candidates
or political action committees in their Facebook or Instagram feeds, for
example.

On Friday, Mr. Zuckerberg said in a livestreamed address to his
employees, ``I'm committed to making sure Facebook remains a place where
people can use their voice to discuss important issues, because I
believe we can make more progress when we hear each other.''

He added, ``But I also stand against hate, or anything that incites
violence or suppresses voting, and we're committed to removing that no
matter where it comes from.''

He said the definition of hate speech would grow to prohibit ads that
claim ``people from a specific race, ethnicity, national origin,
religious affiliation, caste, sexual orientation, gender identity or
immigration status are a threat to the physical safety, health or
survival of others.'' He said the policy would expand to protect
immigrants, migrants, refugees and asylum seekers ``from ads suggesting
these groups are inferior or expressing contempt, dismissal or disgust
directed at them.''

For posts on voting, the company said it would attach links to what
Facebook calls its ``voter information center,'' an initiative it has
pushed in recent weeks to provide users with more data on elections.

Yael Eisenstat, a visiting fellow at Cornell Tech, who in 2018 headed
the elections integrity team for political ads at Facebook, said the
changes ``are important and good steps.''

``They should have come a long time ago, but clearly there has been an
incredible amount of pressure,'' she said.

She added that it was still an ``open question'' as to whether Facebook
would ``enforce these policies against the less clear-cut posts by the
president that are intentionally sowing distrust in the electoral
process.''

\hypertarget{our-2020-election-guide}{%
\section{Our 2020 Election Guide}\label{our-2020-election-guide}}

Updated July 31, 2020

\begin{itemize}
\item
  \begin{center}\rule{0.5\linewidth}{\linethickness}\end{center}

  \hypertarget{the-latest}{%
  \subsection{The Latest}\label{the-latest}}

  \begin{itemize}
  \tightlist
  \item
    President Trump's assault on the Postal Service is intersecting with
    his attacks on mail-in voting.
    \href{https://www.nytimes.com/2020/07/31/us/politics/trump-usps-mail-delays.html?action=click\&pgtype=Article\&state=default\&region=BELOW_MAIN_CONTENT\&context=storylines_guide}{Voting
    rights groups say it is a recipe for disaster.}
  \end{itemize}
\item
  \begin{center}\rule{0.5\linewidth}{\linethickness}\end{center}

  \hypertarget{bidens-vp-search}{%
  \subsection{Biden's V.P. Search}\label{bidens-vp-search}}

  \begin{itemize}
  \tightlist
  \item
    \href{https://www.nytimes.com/article/biden-vice-president-2020.html?action=click\&pgtype=Article\&state=default\&region=BELOW_MAIN_CONTENT\&context=storylines_guide}{Here
    are 13 women} who have been under consideration to be Joe Biden's
    running mate, and why each might be chosen --- and might not be.
  \end{itemize}
\item
  \begin{center}\rule{0.5\linewidth}{\linethickness}\end{center}

  \hypertarget{keep-up-with-our-coverage}{%
  \subsection{Keep Up With Our
  Coverage}\label{keep-up-with-our-coverage}}

  \begin{itemize}
  \tightlist
  \item
    Get an
    \href{https://www.nytimes.com/newsletters/politics?action=click\&pgtype=Article\&state=default\&region=BELOW_MAIN_CONTENT\&context=storylines_guide}{email}
    recapping the day's news
  \end{itemize}

  \begin{itemize}
  \tightlist
  \item
    Download our mobile app on
    \href{https://apps.apple.com/us/app/nytimes/id284862083?ls=1\&mat_click_id=5c79ae7455014fd1bd66b5610c05b8f2-20191112-16948\&referrer=mat_click_id\%3D5c79ae7455014fd1bd66b5610c05b8f2-20191112-16948\%26link_click_id\%3D722930677036718082}{iOS}
    and
    \href{http://a.localytics.com/android?id=com.nytimes.android\&referrer=utm_source\%3Dother_nyt_mobile_web\%26utm_medium\%3DWeb\%2520page\%26utm_term\%3DGeneral\%2520Mobile\%2520Page\%26utm_campaign\%3DNYT\%2520Mobile\%2520General\%2520Page}{Android}
    and turn on Breaking News and Politics alerts
  \end{itemize}
\end{itemize}

Advertisement

\protect\hyperlink{after-bottom}{Continue reading the main story}

\hypertarget{site-index}{%
\subsection{Site Index}\label{site-index}}

\hypertarget{site-information-navigation}{%
\subsection{Site Information
Navigation}\label{site-information-navigation}}

\begin{itemize}
\tightlist
\item
  \href{https://help.nytimes.com/hc/en-us/articles/115014792127-Copyright-notice}{©~2020~The
  New York Times Company}
\end{itemize}

\begin{itemize}
\tightlist
\item
  \href{https://www.nytco.com/}{NYTCo}
\item
  \href{https://help.nytimes.com/hc/en-us/articles/115015385887-Contact-Us}{Contact
  Us}
\item
  \href{https://www.nytco.com/careers/}{Work with us}
\item
  \href{https://nytmediakit.com/}{Advertise}
\item
  \href{http://www.tbrandstudio.com/}{T Brand Studio}
\item
  \href{https://www.nytimes.com/privacy/cookie-policy\#how-do-i-manage-trackers}{Your
  Ad Choices}
\item
  \href{https://www.nytimes.com/privacy}{Privacy}
\item
  \href{https://help.nytimes.com/hc/en-us/articles/115014893428-Terms-of-service}{Terms
  of Service}
\item
  \href{https://help.nytimes.com/hc/en-us/articles/115014893968-Terms-of-sale}{Terms
  of Sale}
\item
  \href{https://spiderbites.nytimes.com}{Site Map}
\item
  \href{https://help.nytimes.com/hc/en-us}{Help}
\item
  \href{https://www.nytimes.com/subscription?campaignId=37WXW}{Subscriptions}
\end{itemize}
