Sections

SEARCH

\protect\hyperlink{site-content}{Skip to
content}\protect\hyperlink{site-index}{Skip to site index}

\href{https://www.nytimes.com/section/us}{U.S.}

\href{https://myaccount.nytimes.com/auth/login?response_type=cookie\&client_id=vi}{}

\href{https://www.nytimes.com/section/todayspaper}{Today's Paper}

\href{/section/us}{U.S.}\textbar{}Leaders Re-examine U.S. Reopenings as
Coronavirus Cases Hit Another Record

\url{https://nyti.ms/2VqAmIq}

\begin{itemize}
\item
\item
\item
\item
\item
\end{itemize}

\href{https://www.nytimes.com/news-event/coronavirus?action=click\&pgtype=Article\&state=default\&region=TOP_BANNER\&context=storylines_menu}{The
Coronavirus Outbreak}

\begin{itemize}
\tightlist
\item
  live\href{https://www.nytimes.com/2020/08/01/world/coronavirus-covid-19.html?action=click\&pgtype=Article\&state=default\&region=TOP_BANNER\&context=storylines_menu}{Latest
  Updates}
\item
  \href{https://www.nytimes.com/interactive/2020/us/coronavirus-us-cases.html?action=click\&pgtype=Article\&state=default\&region=TOP_BANNER\&context=storylines_menu}{Maps
  and Cases}
\item
  \href{https://www.nytimes.com/interactive/2020/science/coronavirus-vaccine-tracker.html?action=click\&pgtype=Article\&state=default\&region=TOP_BANNER\&context=storylines_menu}{Vaccine
  Tracker}
\item
  \href{https://www.nytimes.com/interactive/2020/07/29/us/schools-reopening-coronavirus.html?action=click\&pgtype=Article\&state=default\&region=TOP_BANNER\&context=storylines_menu}{What
  School May Look Like}
\item
  \href{https://www.nytimes.com/live/2020/07/31/business/stock-market-today-coronavirus?action=click\&pgtype=Article\&state=default\&region=TOP_BANNER\&context=storylines_menu}{Economy}
\end{itemize}

Advertisement

\protect\hyperlink{after-top}{Continue reading the main story}

Supported by

\protect\hyperlink{after-sponsor}{Continue reading the main story}

\hypertarget{leaders-re-examine-us-reopenings-as-coronavirus-cases-hit-another-record}{%
\section{Leaders Re-examine U.S. Reopenings as Coronavirus Cases Hit
Another
Record}\label{leaders-re-examine-us-reopenings-as-coronavirus-cases-hit-another-record}}

Officials nationwide were rethinking their efforts to slow the virus,
which the nation's top infectious disease expert said were ``not
working.''

\includegraphics{https://static01.nyt.com/images/2020/06/28/us/28VIRUS-STATE-icehouse/28VIRUS-STATE-icehouse-articleLarge-v2.jpg?quality=75\&auto=webp\&disable=upscale}

\href{https://www.nytimes.com/by/patricia-mazzei}{\includegraphics{https://static01.nyt.com/images/2018/11/28/multimedia/author-patricia-mazzei/author-patricia-mazzei-thumbLarge.png}}\href{https://www.nytimes.com/by/sarah-mervosh}{\includegraphics{https://static01.nyt.com/images/2018/07/18/multimedia/author-sarah-mervosh/author-sarah-mervosh-thumbLarge-v3.png}}\href{https://www.nytimes.com/by/shawn-hubler}{\includegraphics{https://static01.nyt.com/images/2020/06/05/reader-center/author-shawn-hubler/author-shawn-hubler-thumbLarge.png}}

By \href{https://www.nytimes.com/by/patricia-mazzei}{Patricia Mazzei},
\href{https://www.nytimes.com/by/sarah-mervosh}{Sarah Mervosh} and
\href{https://www.nytimes.com/by/shawn-hubler}{Shawn Hubler}

\begin{itemize}
\item
  June 26, 2020
\item
  \begin{itemize}
  \item
  \item
  \item
  \item
  \item
  \end{itemize}
\end{itemize}

MIAMI --- As coronavirus cases surge across much of the United States,
leaders are urgently rethinking their strategies to curb the spread,
which the nation's top infectious disease expert said on Friday were
``not working.''

For the first time, some governors are backtracking on reopening their
states, issuing new restrictions for parts of the economy that had
resumed.

Leaders in Texas and Florida abruptly set new restrictions on bars, a
reversal that appeared unthinkable just days ago. And Gov. Gavin Newsom
of California told rural Imperial County, where hospitals have been
overwhelmed with patients, that it must reinstate a stay-at-home order,
the most restrictive of requirements.

More than 45,000 new cases were reported on Friday in the United States,
according to a New York Times database. It was the third day in a row
that the country set a daily record during the pandemic. At least six
states --- Florida, Idaho, Kansas, Oregon, South Carolina and Utah ---
hit daily highs on Friday, but even leaders outside of the new hot zones
in the South and West expressed mounting anxiety.

``This is a very dangerous time,'' Gov. Mike DeWine of Ohio said in an
interview on Friday, as cases were trending steadily upward in his state
after appearing to be under control for more than a month. ``I think
what is happening in Texas and Florida and several other states should
be a warning to everyone.''

``We have to be very careful,'' he said.

The stock market responded badly, with the S\&P 500 dropping 2.4
percent. Losses accelerated after the Texas announcement, adding to
investors' concerns that the virus continued to be a threat to the
economy.

The shifting assessments of the nation's handling of the virus stretched
to the highest levels of the federal government, where Dr. Anthony S.
Fauci, the director of the National Institute of Allergy and Infectious
Diseases, made clear that the standard approach to controlling
infectious diseases --- testing sick people, isolating them and tracing
their contacts --- was not working. The failure, he said, was in part
because some infected Americans are asymptomatic and unknowingly
spreading the virus but also because some people exposed to the virus
are reluctant to self-quarantine or have no place to do so.

In a brief interview on Friday, he said officials were having ``intense
discussions'' about a possible shift to ``pool testing,'' in which
samples from many people are tested at once in an effort to quickly find
and isolate the infected.

Dr. Fauci also issued an urgent warning that while coronavirus
infections were spiking mostly in the South, those outbreaks could
spread to other regions.

Even in the face of the alarming news, the White House continued to
praise its own efforts.

``We have made truly remarkable progress in moving our nation forward,''
Vice President Mike Pence said at what has become a rare public briefing
by the coronavirus task force in Washington. ``We've all seen the
encouraging news as we open up.''

\hypertarget{latest-updates-global-coronavirus-outbreak}{%
\section{\texorpdfstring{\href{https://www.nytimes.com/2020/08/01/world/coronavirus-covid-19.html?action=click\&pgtype=Article\&state=default\&region=MAIN_CONTENT_1\&context=storylines_live_updates}{Latest
Updates: Global Coronavirus
Outbreak}}{Latest Updates: Global Coronavirus Outbreak}}\label{latest-updates-global-coronavirus-outbreak}}

Updated 2020-08-02T07:42:09.613Z

\begin{itemize}
\tightlist
\item
  \href{https://www.nytimes.com/2020/08/01/world/coronavirus-covid-19.html?action=click\&pgtype=Article\&state=default\&region=MAIN_CONTENT_1\&context=storylines_live_updates\#link-34047410}{The
  U.S. reels as July cases more than double the total of any other
  month.}
\item
  \href{https://www.nytimes.com/2020/08/01/world/coronavirus-covid-19.html?action=click\&pgtype=Article\&state=default\&region=MAIN_CONTENT_1\&context=storylines_live_updates\#link-780ec966}{Top
  U.S. officials work to break an impasse over the federal jobless
  benefit.}
\item
  \href{https://www.nytimes.com/2020/08/01/world/coronavirus-covid-19.html?action=click\&pgtype=Article\&state=default\&region=MAIN_CONTENT_1\&context=storylines_live_updates\#link-2bc8948}{Its
  outbreak untamed, Melbourne goes into even greater lockdown.}
\end{itemize}

\href{https://www.nytimes.com/2020/08/01/world/coronavirus-covid-19.html?action=click\&pgtype=Article\&state=default\&region=MAIN_CONTENT_1\&context=storylines_live_updates}{See
more updates}

More live coverage:
\href{https://www.nytimes.com/live/2020/07/31/business/stock-market-today-coronavirus?action=click\&pgtype=Article\&state=default\&region=MAIN_CONTENT_1\&context=storylines_live_updates}{Markets}

Mr. Pence did not wear a mask, although the health officials around him
did.

\includegraphics{https://static01.nyt.com/images/2020/06/26/us/26VIRUS-STATE-FLA/merlin_173952390_7120147d-b458-4d7f-8de8-f98c62d4e89c-articleLarge.jpg?quality=75\&auto=webp\&disable=upscale}

The renewed sense of urgency comes as the United States confronts a new,
treacherous phase of the pandemic, no longer defined by a crisis
concentrated in New York City, but by rising cases in many cities and
states. Alabama, Alaska, California, Georgia, Missouri, Nevada, Oklahoma
and Texas also reported their highest single-day totals of new known
cases this week, and the United States set records for daily new cases
on both Wednesday and Thursday. By Friday, new daily cases were rising
in 29 states.

From Miami to Los Angeles, mayors were contemplating slowing or
reversing their plans to return cities to public life. On Friday, San
Francisco announced it was delaying plans to reopen zoos, museums, hair
salons, tattoo parlors and other businesses on Monday, citing a spike in
new cases. ``Our numbers are still low but rising rapidly,'' Mayor
London Breed
\href{https://twitter.com/LondonBreed/status/1276596917770715136}{wrote
on Twitter}, adding, ``I know people are anxious to reopen --- I am too.
But we can't jeopardize the progress we've made.''

Mayor Carlos Gimenez of Miami-Dade County said late Friday that he would
sign an emergency order closing beaches from July 3 to July 7, citing
the surge of cases and fears about mass gatherings during the holiday
weekend. Parks and beaches will be closed to fireworks displays, and
gatherings of more than 50 people, including parades, will be banned.

``The closure may be extended if conditions do not improve,'' he said in
a statement, adding, ``I have decided that the only prudent thing to do
to tamp down this recent uptick is to crack down on recreational
activities that put our overall community at higher risk.''

The decisions in Texas and Florida to revert to stronger restrictions
represented the strongest acknowledgment yet that reopening had not gone
as planned in two of the nation's most populous states, where only days
ago their Republican governors were adamantly resisting calls to close
back down.

On Thursday, Gov. Greg Abbott of Texas placed
\href{https://www.nytimes.com/2020/06/25/us/texas-coronavirus-cases-reopening-Greg-Abbott.html}{the
state's reopening on pause}, while remaining firm that going
``backward'' and closing down businesses was ``the last thing we want to
do.''

But by Friday, he did just that, ordering bars closed and telling
restaurants to limit themselves to 50 percent capacity rather than 75
percent.

``If I could go back and redo anything, it probably would have been to
slow down the opening of bars,'' Mr. Abbott said in an interview with
KVIA-TV in El Paso on Friday evening.

``People go to bars to get close and to drink and to socialize,'' he
said. ``And that's the kind of thing that stokes the spread of the
coronavirus. So sure, in hindsight, it may have been better to slow the
opening of the bar setting.''

Eight weeks ago, Mr. Abbott started a phased-in reopening of Texas, when
the state had reported about 29,000 cases and more than 800 deaths. Bars
had been allowed to open since late May.

New cases and hospitalizations have increased significantly in recent
days in Houston, San Antonio and other large cities. By Friday, Texas
had more than 130,000 known coronavirus cases and more than 2,300
deaths, and the leader of the third-largest county in America --- Harris
County, which is home to Houston --- had deemed the region to be on a
code-red coronavirus threat level.

``We find ourselves careening toward a catastrophic and unsustainable
situation,'' the top elected official in Harris County, Lina Hidalgo,
said at a news conference. She said the current hospitalization rate was
on pace to overwhelm the hospital system ``in the near future.''

Image

Patrons at Big Dean's, a bar and restaurant by the Santa Monica pier,
waited to have their temperatures taken on Thursday. Mayor Eric Garcetti
of Los Angeles said he was urging health officials to come to a
consensus strategy in California.Credit...Bryan Denton for The New York
Times

In Florida, the speed of the virus's growth was dizzying: State
officials reported 8,942 new coronavirus cases on Friday, by far
outpacing its earlier single-day record of 5,508 cases, which had been
set on Wednesday.

Officials announced limits on bars, immediately banning alcohol
consumption on the premises. Bars can still sell food if they are
licensed to do so, but their facilities must remain at 50 percent
capacity.

The return to stricter limits left local officials worried whether
residents would follow the rules, especially now, months into the
crisis.

\href{https://www.nytimes.com/news-event/coronavirus?action=click\&pgtype=Article\&state=default\&region=MAIN_CONTENT_3\&context=storylines_faq}{}

\hypertarget{the-coronavirus-outbreak-}{%
\subsubsection{The Coronavirus Outbreak
›}\label{the-coronavirus-outbreak-}}

\hypertarget{frequently-asked-questions}{%
\paragraph{Frequently Asked
Questions}\label{frequently-asked-questions}}

Updated July 27, 2020

\begin{itemize}
\item ~
  \hypertarget{should-i-refinance-my-mortgage}{%
  \paragraph{Should I refinance my
  mortgage?}\label{should-i-refinance-my-mortgage}}

  \begin{itemize}
  \tightlist
  \item
    \href{https://www.nytimes.com/article/coronavirus-money-unemployment.html?action=click\&pgtype=Article\&state=default\&region=MAIN_CONTENT_3\&context=storylines_faq}{It
    could be a good idea,} because mortgage rates have
    \href{https://www.nytimes.com/2020/07/16/business/mortgage-rates-below-3-percent.html?action=click\&pgtype=Article\&state=default\&region=MAIN_CONTENT_3\&context=storylines_faq}{never
    been lower.} Refinancing requests have pushed mortgage applications
    to some of the highest levels since 2008, so be prepared to get in
    line. But defaults are also up, so if you're thinking about buying a
    home, be aware that some lenders have tightened their standards.
  \end{itemize}
\item ~
  \hypertarget{what-is-school-going-to-look-like-in-september}{%
  \paragraph{What is school going to look like in
  September?}\label{what-is-school-going-to-look-like-in-september}}

  \begin{itemize}
  \tightlist
  \item
    It is unlikely that many schools will return to a normal schedule
    this fall, requiring the grind of
    \href{https://www.nytimes.com/2020/06/05/us/coronavirus-education-lost-learning.html?action=click\&pgtype=Article\&state=default\&region=MAIN_CONTENT_3\&context=storylines_faq}{online
    learning},
    \href{https://www.nytimes.com/2020/05/29/us/coronavirus-child-care-centers.html?action=click\&pgtype=Article\&state=default\&region=MAIN_CONTENT_3\&context=storylines_faq}{makeshift
    child care} and
    \href{https://www.nytimes.com/2020/06/03/business/economy/coronavirus-working-women.html?action=click\&pgtype=Article\&state=default\&region=MAIN_CONTENT_3\&context=storylines_faq}{stunted
    workdays} to continue. California's two largest public school
    districts --- Los Angeles and San Diego --- said on July 13, that
    \href{https://www.nytimes.com/2020/07/13/us/lausd-san-diego-school-reopening.html?action=click\&pgtype=Article\&state=default\&region=MAIN_CONTENT_3\&context=storylines_faq}{instruction
    will be remote-only in the fall}, citing concerns that surging
    coronavirus infections in their areas pose too dire a risk for
    students and teachers. Together, the two districts enroll some
    825,000 students. They are the largest in the country so far to
    abandon plans for even a partial physical return to classrooms when
    they reopen in August. For other districts, the solution won't be an
    all-or-nothing approach.
    \href{https://bioethics.jhu.edu/research-and-outreach/projects/eschool-initiative/school-policy-tracker/}{Many
    systems}, including the nation's largest, New York City, are
    devising
    \href{https://www.nytimes.com/2020/06/26/us/coronavirus-schools-reopen-fall.html?action=click\&pgtype=Article\&state=default\&region=MAIN_CONTENT_3\&context=storylines_faq}{hybrid
    plans} that involve spending some days in classrooms and other days
    online. There's no national policy on this yet, so check with your
    municipal school system regularly to see what is happening in your
    community.
  \end{itemize}
\item ~
  \hypertarget{is-the-coronavirus-airborne}{%
  \paragraph{Is the coronavirus
  airborne?}\label{is-the-coronavirus-airborne}}

  \begin{itemize}
  \tightlist
  \item
    The coronavirus
    \href{https://www.nytimes.com/2020/07/04/health/239-experts-with-one-big-claim-the-coronavirus-is-airborne.html?action=click\&pgtype=Article\&state=default\&region=MAIN_CONTENT_3\&context=storylines_faq}{can
    stay aloft for hours in tiny droplets in stagnant air}, infecting
    people as they inhale, mounting scientific evidence suggests. This
    risk is highest in crowded indoor spaces with poor ventilation, and
    may help explain super-spreading events reported in meatpacking
    plants, churches and restaurants.
    \href{https://www.nytimes.com/2020/07/06/health/coronavirus-airborne-aerosols.html?action=click\&pgtype=Article\&state=default\&region=MAIN_CONTENT_3\&context=storylines_faq}{It's
    unclear how often the virus is spread} via these tiny droplets, or
    aerosols, compared with larger droplets that are expelled when a
    sick person coughs or sneezes, or transmitted through contact with
    contaminated surfaces, said Linsey Marr, an aerosol expert at
    Virginia Tech. Aerosols are released even when a person without
    symptoms exhales, talks or sings, according to Dr. Marr and more
    than 200 other experts, who
    \href{https://academic.oup.com/cid/article/doi/10.1093/cid/ciaa939/5867798}{have
    outlined the evidence in an open letter to the World Health
    Organization}.
  \end{itemize}
\item ~
  \hypertarget{what-are-the-symptoms-of-coronavirus}{%
  \paragraph{What are the symptoms of
  coronavirus?}\label{what-are-the-symptoms-of-coronavirus}}

  \begin{itemize}
  \tightlist
  \item
    Common symptoms
    \href{https://www.nytimes.com/article/symptoms-coronavirus.html?action=click\&pgtype=Article\&state=default\&region=MAIN_CONTENT_3\&context=storylines_faq}{include
    fever, a dry cough, fatigue and difficulty breathing or shortness of
    breath.} Some of these symptoms overlap with those of the flu,
    making detection difficult, but runny noses and stuffy sinuses are
    less common.
    \href{https://www.nytimes.com/2020/04/27/health/coronavirus-symptoms-cdc.html?action=click\&pgtype=Article\&state=default\&region=MAIN_CONTENT_3\&context=storylines_faq}{The
    C.D.C. has also} added chills, muscle pain, sore throat, headache
    and a new loss of the sense of taste or smell as symptoms to look
    out for. Most people fall ill five to seven days after exposure, but
    symptoms may appear in as few as two days or as many as 14 days.
  \end{itemize}
\item ~
  \hypertarget{does-asymptomatic-transmission-of-covid-19-happen}{%
  \paragraph{Does asymptomatic transmission of Covid-19
  happen?}\label{does-asymptomatic-transmission-of-covid-19-happen}}

  \begin{itemize}
  \tightlist
  \item
    So far, the evidence seems to show it does. A widely cited
    \href{https://www.nature.com/articles/s41591-020-0869-5}{paper}
    published in April suggests that people are most infectious about
    two days before the onset of coronavirus symptoms and estimated that
    44 percent of new infections were a result of transmission from
    people who were not yet showing symptoms. Recently, a top expert at
    the World Health Organization stated that transmission of the
    coronavirus by people who did not have symptoms was ``very rare,''
    \href{https://www.nytimes.com/2020/06/09/world/coronavirus-updates.html?action=click\&pgtype=Article\&state=default\&region=MAIN_CONTENT_3\&context=storylines_faq\#link-1f302e21}{but
    she later walked back that statement.}
  \end{itemize}
\end{itemize}

``People are tired of being in a stay-at-home environment, and they're
not going to be compliant,'' said Carlos Migoya, president and chief
executive of the public Jackson Health System in Miami. ``You can't put
the genie back in the bottle. We've got to deal with it being in the
environment.''

Pete Boland, who co-owns the Galley, a restaurant and bar in St.
Petersburg, Fla., was sorting through the details of Florida's latest
order on Friday to determine what the rules will be for establishments
that also serve food.

He had just reopened on Wednesday, following a professional deep
cleaning after some employees fell ill with the virus.

``I don't know if we can continue to do this: open, closing, open,
closing,'' he said. ``You have people who desire to socialize and to
earn and to live and to have some fun in this crazy world.''

In Arizona, Gov. Doug Ducey has held out on setting new limits in his
state, even as cases there surged past 66,000, with an average of 2,750
new cases per day. He warned this week that hospitals were likely to hit
surge capacity soon but he has remained opposed to backtracking on
reopening.

``This is not another executive order to enforce, and it's not about
closing businesses,'' he said this week. ``This is about public
education and personal responsibility.''

Image

Phoenix residents walked in Coronado Park on Wednesday. Gov. Doug Ducey
of Arizona warned this week that hospitals would likely hit surge
capacity, but he has not backtracked on reopenings.Credit...Adriana
Zehbrauskas for The New York Times

Still, shutting down businesses again in Arizona is not out of the
question, Daniel Ruiz, the state's chief operating officer, said in an
interview on Friday.

``We want to treat that like a last resort,'' Mr. Ruiz said. ``It's a
tool in the toolbox, but it's something that we're going to use very
judiciously.''

California, which had the first stay-at-home order in the nation this
spring, has surpassed 200,000 cases, and on Friday, Mr. Newsom announced
new restrictions on Imperial County, which has the state's
\href{https://www.nytimes.com/2020/06/26/us/corona-virus-latinos.html}{highest
rate of infection}. The county has exceeded its hospital capacity so
severely that some 500 patients have had to be moved to beds elsewhere,
and hospitals as far away as the Bay Area have been seeing Imperial
County patients.

``This disease does not take a summer vacation,'' said Mr. Newsom,
noting that at least 15 of California's 58 counties were being monitored
closely as the virus surges.

In Los Angeles County, health officials estimate that every 400th person
may currently be infected. Mayor Eric Garcetti of Los Angeles said he
planned to wait three to five days before deciding whether to pull back
on the city's reopening.

``We're not in the red zone but we're in the yellow zone,'' the mayor
said in an interview on Friday.

From case counts to hospitalizations, he said, the city's metrics are
moving in the wrong direction, in part because of a patchwork of
responses in neighboring areas.

Mr. Garcetti said he would like health officials in the state, the
county and the surrounding region to come to a consensus strategy.

``If you don't move together, there's no point in being the lone
holdout,'' he said. ``If you don't have an entire region working
together, who cares if you keep your gyms closed?''

Patricia Mazzei reported from Miami, Sarah Mervosh from Pittsburgh and
Shawn Hubler from Sacramento. Contributing reporting were Nicholas
Bogel-Burroughs and Giulia McDonnell Nieto del Rio from New York, Julie
Bosman from Chicago, Manny Fernandez from Houston, Frances Robles from
Miami, Michael D. Shear and Sheryl Gay Stolberg from Washington, and
Dave Montgomery from Austin, Texas.

Advertisement

\protect\hyperlink{after-bottom}{Continue reading the main story}

\hypertarget{site-index}{%
\subsection{Site Index}\label{site-index}}

\hypertarget{site-information-navigation}{%
\subsection{Site Information
Navigation}\label{site-information-navigation}}

\begin{itemize}
\tightlist
\item
  \href{https://help.nytimes.com/hc/en-us/articles/115014792127-Copyright-notice}{©~2020~The
  New York Times Company}
\end{itemize}

\begin{itemize}
\tightlist
\item
  \href{https://www.nytco.com/}{NYTCo}
\item
  \href{https://help.nytimes.com/hc/en-us/articles/115015385887-Contact-Us}{Contact
  Us}
\item
  \href{https://www.nytco.com/careers/}{Work with us}
\item
  \href{https://nytmediakit.com/}{Advertise}
\item
  \href{http://www.tbrandstudio.com/}{T Brand Studio}
\item
  \href{https://www.nytimes.com/privacy/cookie-policy\#how-do-i-manage-trackers}{Your
  Ad Choices}
\item
  \href{https://www.nytimes.com/privacy}{Privacy}
\item
  \href{https://help.nytimes.com/hc/en-us/articles/115014893428-Terms-of-service}{Terms
  of Service}
\item
  \href{https://help.nytimes.com/hc/en-us/articles/115014893968-Terms-of-sale}{Terms
  of Sale}
\item
  \href{https://spiderbites.nytimes.com}{Site Map}
\item
  \href{https://help.nytimes.com/hc/en-us}{Help}
\item
  \href{https://www.nytimes.com/subscription?campaignId=37WXW}{Subscriptions}
\end{itemize}
