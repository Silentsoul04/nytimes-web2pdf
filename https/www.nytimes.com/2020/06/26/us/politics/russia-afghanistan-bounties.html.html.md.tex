Sections

SEARCH

\protect\hyperlink{site-content}{Skip to
content}\protect\hyperlink{site-index}{Skip to site index}

\href{https://www.nytimes.com/section/politics}{Politics}

\href{https://myaccount.nytimes.com/auth/login?response_type=cookie\&client_id=vi}{}

\href{https://www.nytimes.com/section/todayspaper}{Today's Paper}

\href{/section/politics}{Politics}\textbar{}Russia Secretly Offered
Afghan Militants Bounties to Kill U.S. Troops, Intelligence Says

\url{https://nyti.ms/3i2hvwS}

\begin{itemize}
\item
\item
\item
\item
\item
\item
\end{itemize}

Advertisement

\protect\hyperlink{after-top}{Continue reading the main story}

Supported by

\protect\hyperlink{after-sponsor}{Continue reading the main story}

\hypertarget{russia-secretly-offered-afghan-militants-bounties-to-kill-us-troops-intelligence-says}{%
\section{Russia Secretly Offered Afghan Militants Bounties to Kill U.S.
Troops, Intelligence
Says}\label{russia-secretly-offered-afghan-militants-bounties-to-kill-us-troops-intelligence-says}}

The Trump administration has been deliberating for months about what to
do about a stunning intelligence assessment.

\includegraphics{https://static01.nyt.com/images/2020/07/05/us/politics/26dc-intel1/26dc-intel1-articleLarge-v2.jpg?quality=75\&auto=webp\&disable=upscale}

\href{https://www.nytimes.com/by/charlie-savage}{\includegraphics{https://static01.nyt.com/images/2018/06/12/multimedia/author-charlie-savage/author-charlie-savage-thumbLarge-v2.png}}\href{https://www.nytimes.com/by/eric-schmitt}{\includegraphics{https://static01.nyt.com/images/2018/06/12/multimedia/author-eric-schmitt/author-eric-schmitt-thumbLarge-v2.png}}\href{https://www.nytimes.com/by/michael-schwirtz}{\includegraphics{https://static01.nyt.com/images/2018/02/20/multimedia/author-michael-schwirtz/author-michael-schwirtz-thumbLarge-v2.jpg}}

By \href{https://www.nytimes.com/by/charlie-savage}{Charlie Savage},
\href{https://www.nytimes.com/by/eric-schmitt}{Eric Schmitt} and
\href{https://www.nytimes.com/by/michael-schwirtz}{Michael Schwirtz}

\begin{itemize}
\item
  Published June 26, 2020Updated July 29, 2020
\item
  \begin{itemize}
  \item
  \item
  \item
  \item
  \item
  \item
  \end{itemize}
\end{itemize}

WASHINGTON --- American intelligence officials have concluded that a
\href{https://www.nytimes.com/2020/07/29/us/politics/trump-putin-bounties.html}{Russian
military intelligence unit secretly offered bounties} to Taliban-linked
militants for killing coalition forces in Afghanistan --- including
targeting American troops --- amid the peace talks to end the
long-running war there, according to officials briefed on the matter.

The United States concluded months ago that the
\href{https://www.nytimes.com/2020/06/29/us/politics/trump-russia-plot-afghanistan.html}{Russian}
unit, which has been linked to assassination attempts and other covert
operations in Europe intended to destabilize the West or take revenge on
turncoats, had covertly offered rewards for successful attacks last
year.

\href{https://www.nytimes.com/2020/06/29/us/politics/trump-russia-plot-afghanistan.html}{Islamist
militants}, or armed criminal elements closely associated with them, are
believed to have collected some bounty money, the officials said. Twenty
Americans were killed in combat in Afghanistan in 2019, but it was not
clear which killings were under suspicion.

The intelligence finding was briefed to President Trump, and the White
House's National Security Council discussed the problem at an
interagency meeting in late March, the officials said. Officials
developed a menu of potential options --- starting with making a
diplomatic complaint to Moscow and a demand that it stop, along with an
escalating series of sanctions and other possible responses, but the
White House has yet to authorize any step, the officials said.

An operation to incentivize the killing of American and other NATO
troops would be a significant and provocative escalation of what
American and Afghan officials have said is Russian support for the
Taliban, and it would be the first time the Russian spy unit was known
to have orchestrated attacks on Western troops.

Any involvement with the Taliban that resulted in the deaths of American
troops would also be a huge escalation of Russia's so-called hybrid war
against the United States, a strategy of destabilizing adversaries
through a combination of such tactics as cyberattacks, the spread of
fake news and covert and deniable military operations.

The Kremlin had not been made aware of the accusations, said Dmitry
Peskov, the press secretary for President Vladimir V. Putin of Russia.
``If someone makes them, we'll respond,'' Mr. Peskov said.

Zabihullah Mujahid, a spokesman for the Taliban, denied that the
insurgents have ``any such relations with any intelligence agency'' and
called the report an attempt to defame them.

``These kinds of deals with the Russian intelligence agency are baseless
--- our target killings and assassinations were ongoing in years before,
and we did it on our own resources,'' he said. ``That changed after our
deal with the Americans, and their lives are secure and we don't attack
them.''

Spokespeople at the National Security Council, the Pentagon, the State
Department and the C.I.A. declined to comment.

The officials familiar with the intelligence did not explain the White
House delay in deciding how to respond to the intelligence about Russia.

While some of his closest advisers, like Secretary of State Mike Pompeo,
have counseled more hawkish policies toward Russia, Mr. Trump has
adopted an accommodating stance toward Moscow.

At a summit in 2018 in Helsinki, Finland,
\href{https://www.nytimes.com/2018/07/16/world/europe/trump-putin-election-intelligence.html}{Mr.
Trump strongly suggested that he believed Mr. Putin's denial} that the
Kremlin interfered in the 2016 presidential election, despite broad
agreement within the American intelligence establishment that it did.
Mr. Trump
\href{https://www.nytimes.com/2017/08/02/world/europe/trump-russia-sanctions.html}{criticized
a bill imposing sanctions on Russia} when he signed it into law after
Congress passed it by veto-proof majorities. And he has repeatedly
\href{https://www.nytimes.com/2019/01/14/us/politics/nato-president-trump.html}{made
statements that undermined the NATO alliance} as a bulwark against
Russian aggression in Europe.

The officials spoke on the condition of anonymity to describe the
delicate intelligence and internal deliberations. They said the
intelligence had been treated as a closely held secret, but the
administration expanded briefings about it this week --- including
sharing information about it with the British government, whose forces
are among those said to have been targeted.

\includegraphics{https://static01.nyt.com/images/2020/06/26/us/politics/26dc-intel2/merlin_173866728_2ea22503-da7b-46c3-bb84-8023ddd0f8d5-articleLarge.jpg?quality=75\&auto=webp\&disable=upscale}

The intelligence assessment is said to be based at least in part on
interrogations of captured Afghan militants and criminals. The officials
did not describe the mechanics of the Russian operation, such as how
targets were picked or how money changed hands. It is also not clear
whether Russian operatives had deployed inside Afghanistan or met with
their Taliban counterparts elsewhere.

The revelations came into focus inside the Trump administration at a
delicate and distracted time. Although officials collected the
intelligence earlier in the year, the interagency meeting at the White
House took place as the
\href{https://www.nytimes.com/interactive/2020/us/coronavirus-us-cases.html}{coronavirus
pandemic was becoming a crisis} and parts of the country were shutting
down.

Moreover, as Mr. Trump seeks re-election in November, he wants to strike
a peace deal with the Taliban to end the Afghanistan war.

Both
\href{https://www.npr.org/sections/thetwo-way/2018/03/26/596933077/top-u-s-commander-in-afghanistan-accuses-russia-of-aiding-taliban}{American
and Afghan officials have previously accused Russia of providing small
arms and other support to the Taliban} that amounts to destabilizing
activity, although Russian government officials have dismissed such
claims as ``idle gossip'' and baseless.

``We share some interests with Russia in Afghanistan, and clearly
they're acting to undermine our interests as well,'' Gen. John W.
Nicholson Jr., the commander of American forces in Afghanistan at the
time, said in a 2018
\href{https://www.bbc.com/news/world-asia-43500299}{interview with the
BBC}.

Though coalition troops suffered
\href{http://icasualties.org/App/AfghanFatalities?page=1\&rows=100}{a
spate of combat casualties last summer and early fall}, only a few have
since been killed. Four Americans were killed in combat in early 2020,
but the Taliban have not attacked American positions since a February
agreement.

American troops have also sharply reduced their movement outside
military bases because of the coronavirus, reducing their exposure to
attack.

While officials were said to be confident about the intelligence that
Russian operatives offered and paid bounties to Afghan militants for
killing Americans, they have greater uncertainty about how high in the
Russian government the covert operation was authorized and what its aim
may be.

Some officials have theorized that the Russians may be seeking revenge
on NATO forces for a
\href{https://www.nytimes.com/2018/05/24/world/middleeast/american-commandos-russian-mercenaries-syria.html}{2018
battle in Syria} in which the American military killed several hundred
pro-Syrian forces, including numerous Russian mercenaries, as they
advanced on an American outpost. Officials have also suggested that the
Russians may have been trying to derail peace talks to keep the United
States bogged down in Afghanistan. But the motivation remains murky.

The officials briefed on the matter said the government had assessed the
operation to be the handiwork of
\href{https://www.nytimes.com/2019/10/08/world/europe/unit-29155-russia-gru.html}{Unit
29155}, an arm of Russia's military intelligence agency, known widely as
the G.R.U. The unit is linked to the March 2018 nerve agent poisoning in
Salisbury, England, of Sergei Skripal, a former G.R.U. officer who had
worked for British intelligence and then defected, and his daughter.

Western intelligence officials say the unit, which has operated for more
than a decade, has been charged by the Kremlin with carrying out a
campaign to destabilize the West through subversion, sabotage and
assassination. In addition to the 2018 poisoning, the unit was behind
\href{https://www.nytimes.com/2019/05/09/world/europe/montenegro-coup-plot-gru.html}{an
attempted coup in Montenegro} in 2016 and
\href{https://www.nytimes.com/2019/12/22/world/europe/bulgaria-russia-assassination-squad.html}{the
poisoning of an arms manufacturer} in Bulgaria a year earlier.

American intelligence officials say the G.R.U. was at the center of
Moscow's covert efforts to interfere in the 2016 presidential election.
In the months before that election, American officials say, two G.R.U.
cyberunits, known as 26165 and 74455, hacked into Democratic Party
servers and then used WikiLeaks to publish embarrassing internal
communications.

In part because those efforts were aimed at helping tilt the election in
Mr. Trump's favor, his handling of issues related to Russia and Mr.
Putin has come under particular scrutiny. The special counsel
investigation found that the Trump campaign welcomed Russia's
intervention and expected to benefit from it, but found insufficient
evidence to establish that his associates had engaged in any criminal
conspiracy with Moscow.

Operations involving Unit 29155 tend to be much more violent than those
involving the cyberunits. Its officers are often decorated military
veterans with years of service, in some cases dating to the Soviet
Union's failed war in Afghanistan in the 1980s. Never before has the
unit been accused of orchestrating attacks on Western soldiers, but
officials briefed on its operations say it has been active in
Afghanistan for many years.

Though Russia declared the Taliban a terrorist organization in 2003,
relations between them have been warming in recent years. Taliban
officials have
\href{https://www.nytimes.com/2019/02/04/world/asia/afghanistan-taliban-russia-talks-russia.html}{traveled
to Moscow for peace talks} with other prominent Afghans, including the
former president, Hamid Karzai. The talks have excluded representatives
from the current Afghan government as well as anyone from the United
States, and at times they have seemed to work at crosscurrents with
American efforts to bring an end to the conflict.

The disclosure comes at a time when Mr. Trump has said he would invite
Mr. Putin to an expanded meeting of the Group of 7 nations,
\href{https://www.nytimes.com/2020/06/01/us/politics/coronavirus-global-competition-russia-china-iran-north-korea.html}{but
tensions between American and Russian militaries are running high}.

In several recent episodes, in international territory and airspace from
off the coast of Alaska to the Black and Mediterranean Seas, combat
planes from each country have scrambled to intercept military aircraft
from the other.

Mujib Mashal contributed reporting from Kabul, Afghanistan.

Advertisement

\protect\hyperlink{after-bottom}{Continue reading the main story}

\hypertarget{site-index}{%
\subsection{Site Index}\label{site-index}}

\hypertarget{site-information-navigation}{%
\subsection{Site Information
Navigation}\label{site-information-navigation}}

\begin{itemize}
\tightlist
\item
  \href{https://help.nytimes.com/hc/en-us/articles/115014792127-Copyright-notice}{©~2020~The
  New York Times Company}
\end{itemize}

\begin{itemize}
\tightlist
\item
  \href{https://www.nytco.com/}{NYTCo}
\item
  \href{https://help.nytimes.com/hc/en-us/articles/115015385887-Contact-Us}{Contact
  Us}
\item
  \href{https://www.nytco.com/careers/}{Work with us}
\item
  \href{https://nytmediakit.com/}{Advertise}
\item
  \href{http://www.tbrandstudio.com/}{T Brand Studio}
\item
  \href{https://www.nytimes.com/privacy/cookie-policy\#how-do-i-manage-trackers}{Your
  Ad Choices}
\item
  \href{https://www.nytimes.com/privacy}{Privacy}
\item
  \href{https://help.nytimes.com/hc/en-us/articles/115014893428-Terms-of-service}{Terms
  of Service}
\item
  \href{https://help.nytimes.com/hc/en-us/articles/115014893968-Terms-of-sale}{Terms
  of Sale}
\item
  \href{https://spiderbites.nytimes.com}{Site Map}
\item
  \href{https://help.nytimes.com/hc/en-us}{Help}
\item
  \href{https://www.nytimes.com/subscription?campaignId=37WXW}{Subscriptions}
\end{itemize}
