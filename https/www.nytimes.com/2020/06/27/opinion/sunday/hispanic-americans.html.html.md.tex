Sections

SEARCH

\protect\hyperlink{site-content}{Skip to
content}\protect\hyperlink{site-index}{Skip to site index}

\href{https://www.nytimes.com/section/opinion/sunday}{Sunday Review}

\href{https://myaccount.nytimes.com/auth/login?response_type=cookie\&client_id=vi}{}

\href{https://www.nytimes.com/section/todayspaper}{Today's Paper}

\href{/section/opinion/sunday}{Sunday Review}\textbar{}Now Is a Time to
Learn From Hispanic Americans

\href{https://nyti.ms/2ZdWCWZ}{https://nyti.ms/2ZdWCWZ}

\begin{itemize}
\item
\item
\item
\item
\item
\item
\end{itemize}

Advertisement

\protect\hyperlink{after-top}{Continue reading the main story}

\href{/section/opinion}{Opinion}

Supported by

\protect\hyperlink{after-sponsor}{Continue reading the main story}

\hypertarget{now-is-a-time-to-learn-from-hispanic-americans}{%
\section{Now Is a Time to Learn From Hispanic
Americans}\label{now-is-a-time-to-learn-from-hispanic-americans}}

The ``Hispanic paradox'' could offer a model for civil society.

\href{https://www.nytimes.com/column/nicholas-kristof}{\includegraphics{https://static01.nyt.com/images/2018/04/03/opinion/nicholas-kristof/nicholas-kristof-thumbLarge-v2.png}}

By \href{https://www.nytimes.com/column/nicholas-kristof}{Nicholas
Kristof}

Opinion Columnist

\begin{itemize}
\item
  June 27, 2020
\item
  \begin{itemize}
  \item
  \item
  \item
  \item
  \item
  \item
  \end{itemize}
\end{itemize}

\includegraphics{https://static01.nyt.com/images/2020/06/28/opinion/28kristof1/merlin_173881950_c67b931f-c837-454a-b08f-49616a096a56-articleLarge.jpg?quality=75\&auto=webp\&disable=upscale}

CORNELIUS, Ore. --- Scholars call it the ``Hispanic Paradox'': Despite
poverty and discrimination, Hispanic Americans live significantly longer
than white or black Americans.

Latinos also
\href{https://www.cdc.gov/nchs/data/series/sr_02/sr02_172.pdf}{appear}
to have lower suicide rates than whites, are less likely to
\href{https://www.niaaa.nih.gov/publications/brochures-and-fact-sheets/alcohol-and-hispanic-community}{drink
alcohol}, are less likely to die
\href{https://www.cdc.gov/mmwr/volumes/69/wr/mm6911a4.htm\#T1_down}{from
drug overdoses} and, at least among immigrants, appear to
\href{https://www.politifact.com/factchecks/2017/aug/03/antonio-villaraigosa/mostly-true-undocumented-immigrants-less-likely-co/}{commit
fewer crimes}.

Researchers have puzzled for decades about why this is. Strong families?
Supportive social networks? Religious faith and active churches? A
hard-driving immigrant work ethic?

It's a paradox because the disadvantaged normally live shorter lives.
Hispanics in the United States endure discrimination, high poverty,
lower rates of health insurance than both whites and blacks --- yet they
enjoy a life expectancy of
\href{https://www.cdc.gov/nchs/data/nvsr/nvsr68/nvsr68_07-508.pdf}{81.8
years}, compared with 78.5 years for whites and 74.9 years for blacks.

This resiliency is now tested by the coronavirus, which has hit Latinos
particularly hard: The Centers for Disease Control and Prevention
\href{https://www.cdc.gov/mmwr/volumes/69/wr/mm6924e2.htm?s_cid=mm6924e2_w}{reported}
this month that 33 percent of Americans testing positive for the
coronavirus have been Hispanic, almost twice their
\href{https://www.census.gov/quickfacts/fact/table/US/RHI725218}{18
percent} share of the population.

I came here to Cornelius, a town west of Portland with a large Latino
population, to gauge the impact of the crisis, and the virus predictably
has struck Hispanics hard. Many are undocumented immigrants and thus
aren't receiving federal relief payments. Yet what struck me, in keeping
with the Hispanic Paradox, was how the community pulled together to ease
the suffering.

Francis, 50, who does not want to be identified by her full name because
she is not documented, lost her job as a receptionist because of
Covid-19, but her 30-year-old daughter and her son-in-law took her in.
``They may think it's weird to have their mother-in-law living with
them, but they're not saying anything,'' she said.

Meanwhile, Francis is volunteering for the community, driving boxes of
food from a Catholic church to needy families. ``My car overheats,'' she
said. ``But I make it work.''

A Brookings Institution study found that since the start of the
pandemic, one in six households in the United States has young children
who aren't getting enough food, so I asked Francis about hunger. She
acknowledged that there must be hungry children but added: ``If people
knew kids were hungry, they would help. The community would step up.''

On the other end of the United States, Latinos in New York City display
a similar resilience. Dr. Carmen Isasi, an epidemiologist at Albert
Einstein College of Medicine who has studied LatinX populations, said
that lately she has seen signs on Spanish-speaking churches offering
food for the needy.

Scholars have been debating the Hispanic Paradox at least since 1974,
when researchers found that the neonatal mortality rate in Texas was
lower for people with Spanish surnames than with English surnames.

Researchers have found another paradox within the paradox:
First-generation Latino immigrants tend to live longest, and their
children --- while better educated and earning more money --- die
earlier. Moreover, Latinos embedded in ethnic enclaves seem to do better
than those who live in heterogeneous neighborhoods.

Part of the explanation may be that what many white Americans think of
as ``traditional American values'' --- an emphasis on faith, family and
community ties --- are disproportionately found among Latino immigrants,
but then fade as their children assimilate.

``If we find that someone needs help, we help them,'' Raúl González
Hernández, who works in a plant nursery and has just recovered from
Covid-19, told me. He said that others had helped him when he arrived
from Michoacán State in Mexico, so he wants to pay it forward ---
particularly if the person needing help is also from Michoacán.

I've been long interested in the Hispanic Paradox because I grew up in a
mostly white farm town in Oregon that has been devastated by lost jobs.
\href{https://www.nytimes.com/2020/01/09/opinion/sunday/deaths-despair-poverty.html}{As
I've written}, a quarter of the kids on my old school bus are dead from
drugs, alcohol, suicide and other ``deaths of despair.''

Latino families in the area have seemed more resilient because of their
greater ``social capital'' --- bonds of family, home region or church.
Instead of being ``criminals, drug dealers, rapists,'' as Donald Trump
\href{https://www.washingtonpost.com/news/fact-checker/wp/2015/07/08/donald-trumps-false-comments-connecting-mexican-immigrants-and-crime/}{alleged}
of Mexican immigrants in 2015, Latino immigrants often seem to be models
of civil society.

``Our community, we rely heavily on each other,'' Petrona
Dominguez-Francisco, who works with a program called
\href{https://www.adelantemujeres.org/}{Adelante Mujeres} that empowers
women, told me.

\includegraphics{https://static01.nyt.com/images/2020/06/28/opinion/28kristof2/merlin_173882100_abfbb1a8-ed82-4ca5-a831-49a752f74d6c-articleLarge.jpg?quality=75\&auto=webp\&disable=upscale}

\href{https://www.pewresearch.org/staff/mark-hugo-lopez/}{Mark Hugo
Lopez}, director of global migration and demography research at the Pew
Research Center, emphasized family ties as part of the basis for the
paradox. ``There's a lot of support in my own family for those who are
facing challenges, such as those who lost their jobs,'' he said.
``That's how Latinos help each other.''

Another element may be faith and church connections. There's
\href{https://www.ncbi.nlm.nih.gov/pmc/articles/PMC4286922/}{some
evidence} that religious beliefs reduce behaviors like drug and alcohol
abuse, risky sexual activity, violence and suicide, and
\href{https://www.hsph.harvard.edu/news/press-releases/religious-upbringing-adult-health/}{a
Harvard study} found that church attendance or daily prayer or
meditation correlates to better health and greater life satisfaction.
Churches also offer a web of services and social connections that can
buffer hardship.

Family and community ties also protect from
\href{https://www.nytimes.com/2019/11/09/opinion/sunday/britain-loneliness-epidemic.html}{a
pandemic of loneliness} in Western countries. One scholar has found that
social isolation is more damaging to health than smoking 15 cigarettes a
day.

This social fabric also isn't a perfect shield from a pandemic. But it
helps, and perhaps there's a lesson in that for all the rest of us.

\emph{The Times is committed to publishing}
\href{https://www.nytimes.com/2019/01/31/opinion/letters/letters-to-editor-new-york-times-women.html}{\emph{a
diversity of letters}} \emph{to the editor. We'd like to hear what you
think about this or any of our articles. Here are some}
\href{https://help.nytimes.com/hc/en-us/articles/115014925288-How-to-submit-a-letter-to-the-editor}{\emph{tips}}\emph{.
And here's our email:}
\href{mailto:letters@nytimes.com}{\emph{letters@nytimes.com}}\emph{.}

Advertisement

\protect\hyperlink{after-bottom}{Continue reading the main story}

\hypertarget{site-index}{%
\subsection{Site Index}\label{site-index}}

\hypertarget{site-information-navigation}{%
\subsection{Site Information
Navigation}\label{site-information-navigation}}

\begin{itemize}
\tightlist
\item
  \href{https://help.nytimes.com/hc/en-us/articles/115014792127-Copyright-notice}{©~2020~The
  New York Times Company}
\end{itemize}

\begin{itemize}
\tightlist
\item
  \href{https://www.nytco.com/}{NYTCo}
\item
  \href{https://help.nytimes.com/hc/en-us/articles/115015385887-Contact-Us}{Contact
  Us}
\item
  \href{https://www.nytco.com/careers/}{Work with us}
\item
  \href{https://nytmediakit.com/}{Advertise}
\item
  \href{http://www.tbrandstudio.com/}{T Brand Studio}
\item
  \href{https://www.nytimes.com/privacy/cookie-policy\#how-do-i-manage-trackers}{Your
  Ad Choices}
\item
  \href{https://www.nytimes.com/privacy}{Privacy}
\item
  \href{https://help.nytimes.com/hc/en-us/articles/115014893428-Terms-of-service}{Terms
  of Service}
\item
  \href{https://help.nytimes.com/hc/en-us/articles/115014893968-Terms-of-sale}{Terms
  of Sale}
\item
  \href{https://spiderbites.nytimes.com}{Site Map}
\item
  \href{https://help.nytimes.com/hc/en-us}{Help}
\item
  \href{https://www.nytimes.com/subscription?campaignId=37WXW}{Subscriptions}
\end{itemize}
