Sections

SEARCH

\protect\hyperlink{site-content}{Skip to
content}\protect\hyperlink{site-index}{Skip to site index}

\href{https://www.nytimes.com/section/opinion/sunday}{Sunday Review}

\href{https://myaccount.nytimes.com/auth/login?response_type=cookie\&client_id=vi}{}

\href{https://www.nytimes.com/section/todayspaper}{Today's Paper}

\href{/section/opinion/sunday}{Sunday Review}\textbar{}Biden's Best Veep
Pick Is Obvious

\href{https://nyti.ms/2YAVJbU}{https://nyti.ms/2YAVJbU}

\begin{itemize}
\item
\item
\item
\item
\item
\item
\end{itemize}

Advertisement

\protect\hyperlink{after-top}{Continue reading the main story}

\href{/section/opinion}{Opinion}

Supported by

\protect\hyperlink{after-sponsor}{Continue reading the main story}

\hypertarget{bidens-best-veep-pick-is-obvious}{%
\section{Biden's Best Veep Pick Is
Obvious}\label{bidens-best-veep-pick-is-obvious}}

She, more than anyone, can get under Trump's skin.

\href{https://www.nytimes.com/by/frank-bruni}{\includegraphics{https://static01.nyt.com/images/2018/04/03/opinion/frank-bruni/frank-bruni-thumbLarge.png}}

By \href{https://www.nytimes.com/by/frank-bruni}{Frank Bruni}

Opinion Columnist

\begin{itemize}
\item
  June 27, 2020
\item
  \begin{itemize}
  \item
  \item
  \item
  \item
  \item
  \item
  \end{itemize}
\end{itemize}

\includegraphics{https://static01.nyt.com/images/2020/06/28/opinion/sunday/28Bruni/28Bruni-articleLarge.jpg?quality=75\&auto=webp\&disable=upscale}

\hypertarget{listen-to-this-op-ed}{%
\subsubsection{Listen to This Op-Ed}\label{listen-to-this-op-ed}}

Audio Recording by Audm

\emph{To hear more audio stories from publishers like The New York
Times, download}
\href{https://www.audm.com/?utm_source=nytmag\&utm_medium=embed\&utm_campaign=left_behind_draper}{**}
\href{https://www.audm.com/?utm_source=nytopinion\&utm_medium=embed\&utm_campaign=biden_best_veep}{\emph{Audm
for iPhone or Android.}}

Whatever his wobbles, Joe Biden has, from the start of his presidential
campaign, got one thing exactly right: The 2020 election \emph{is} a
battle for the soul of America. That's not just a pretty slogan. It's
the stomach-knotting truth --- and it's the frame he should use for
choosing his running mate.

It's why he should pick Senator Tammy Duckworth of Illinois.

She's a paragon of the values that Donald Trump, for all his practice as
a performer, can't even pantomime. She's best described by words that
are musty relics in his venal and vainglorious circle: ``sacrifice,''
``honor,'' ``humility.'' More than any of the many extraordinary women
on
\href{https://www.nytimes.com/article/biden-vice-president-2020.html}{Biden's
list} of potential vice-presidential nominees, she's the anti-Trump, the
antidote to the ugliness he revels in and the cynicism he stokes.

Americans can feel good --- no, wonderful --- about voting for a ticket
with Duckworth on it. And we're beyond hungry for that. We're starving.

That ache transcends all of the other variables that attend Biden's
deliberations as he appraises Elizabeth Warren, Kamala Harris, Val
Demings and others: race, age, experience, exact position on the
spectrum from progressive to moderate.

Duckworth, a former Army lieutenant colonel who lost both of her legs
during combat duty in Iraq, is a choice that makes exquisite emotional
and moral sense. Largely, but not entirely, because of that, she makes
strategic sense, too.

For the uninitiated: Duckworth, 52, is in the fourth year of her first
term in the Senate, before which she served two terms in the House. So
unlike several of the other vice-presidential contenders, she has
ascended to what is conventionally considered the right political
altitude for this next step.

But it's her life story that really makes her stand out. It's the
harrowing chapter in Iraq, yes, but also how she rebounded from it, how
she talks about it. It's her attitude. Her grace.

As my colleague Jennifer Steinhauer explained in
\href{https://www.nytimes.com/2020/06/25/us/politics/tammy-duckworth-vice-president-joe-biden.html}{a
recent profile} of Duckworth in The Times, she didn't just serve in the
Army: She became a helicopter pilot, which isn't a job brimming with
women. And as she flew near Baghdad one day in 2004, her Blackhawk was
struck by a rocket-propelled grenade. The explosion left her near death.

She later received a Purple Heart, but she bristles when she's called a
hero. That designation, she has often said, belongs to her co-pilot, Dan
Milberg, and others who carried her from the wreckage and got her to
safety.

She put it this way when, as part of a
\href{https://www.cbsnews.com/news/note-to-self-senator-tammy-duckworth/}{``Note
to Self''} feature on
``\href{https://www.cbsnews.com/news/note-to-self-senator-tammy-duckworth/}{CBS
This Morning},'' she read aloud a letter that she had written to the
younger Tammy: ``You'll make it out alive completely because of the
grit, sacrifice and outright heroism of \emph{others.} You haven't done
anything to be worthy of their sacrifices, but these heroes will give
you a second chance at life.'' She paused there briefly, fighting back
tears.

To Steinhauer she said, ``I wake up every day thinking, `I am never
going to make Dan regret saving my life.''' Her subsequent advocacy for
veterans, her run for Congress, her election to the Senate: She casts
all of it in terms of gratitude and an obligation to give back.

Tell me how Trump campaigns against that. Tell me how he mocks her ---
which is the only way he knows how to engage with opponents. Or, rather,
tell me how he does so without seeming even more obscene than he already
does and turning off everyone beyond the cultish segment of the
electorate that will never abandon him. Duckworth on the Democratic
ticket is like some psy-ops masterstroke, all the more so because it was
she who nicknamed Trump ``Cadet Bone Spurs.''

I asked her about that on the phone on Thursday, remarking that it was
uncharacteristically acerbic of her. ``This guy's a bully,'' she said.
``And bullies need a taste of their own medicine.''

Warren, too, is terrific at giving Trump that. Her placement on the
Democratic ticket might fire up the progressives who regard Biden
warily. And she could make an excellent governing partner for him.

But mightn't Warren also give moderate voters pause? What about her age?
She's 71. Biden's 77. Can the party of change and modernity, whose last
two presidents were both under 50 when first elected, go with an
all-septuagenarian ticket?

Governing partners don't matter if you don't get to govern. The certain
catastrophe of four more years of Trump demands that Biden choose his
running mate with November at the front, the back, the top and the
bottom of his mind.

Harris also ably prosecutes the case against Trump. But many
progressives have issues with her, and the idea that she'd drive high
turnout among black voters isn't supported by her failed bid for the
Democratic nomination. She lacked support across the board, including
among African-Americans. And in
\href{https://www.nytimes.com/2020/06/26/us/politics/biden-vice-president-voters.html}{a
recent national poll} conducted by The Times and Siena College, more
than four in five voters --- including three in four black voters ---
said that race shouldn't be a factor in Biden's vice-presidential pick.

Duckworth is neither progressive idol nor progressive enemy. That partly
reflects a low policy profile that's among her flaws as a running mate
but could actually work to her advantage, making her difficult to
pigeonhole and open to interpretation. Trump-weary voters can read into
her what they want. And in recent congressional elections, Democrats
have had success among swing voters with candidates who are veterans.

Duckworth certainly can't be dismissed as the same old same old. Her
vice-presidential candidacy would be a trailblazing one, emblematic of a
more diverse and inclusive America. Born in Bangkok to an American
father and a Thai mother, she'd be the first Asian-American and the
first woman of color on the presidential ticket of one of our two major
parties.

\includegraphics{https://static01.nyt.com/images/2020/06/28/opinion/28bruni1/merlin_141176322_9e8d2500-8fe1-406c-ad17-d2212c3989e9-articleLarge.jpg?quality=75\&auto=webp\&disable=upscale}

She was the first United States senator to give birth while in office
and the first to bring her baby onto the Senate floor. You want
relatable? Duckworth has two children under the age of 6. She's a
working mom.

She's not the product of privilege: In fact her family hit such hard
times when she was growing up in Hawaii that at one point she sold
flowers by the side of the road. But she went on to get not only a
college degree but also a master's in international affairs.

Cards on the table: I'm not at all sure that running mates matter much
on Election Day. There's ample evidence that they don't.

But in any given election, they sure as hell might. Biden would be a
fool, given the stakes, not to consider his running mate a victory
clincher or deal breaker and to choose her accordingly.

Duckworth's virtues include everything that I've mentioned plus this:
She projects a combination of confidence and modesty, of toughness and
warmth, that's rare --- and that's a tonic in these toxic times.

I asked her whether she deems Trump a patriot. She said that he wraps
himself in the American flag --- a flag, she noted, that will someday
drape her coffin --- for the wrong reasons.

``I would leap into a burning fire to pull that flag to safety, but I
will fight to the death for your right to burn it,'' she told me. ``The
most patriotic thing you can do is not necessarily putting on the
uniform but speaking truth to power, exercising your First Amendment
rights --- that's what created America, right?''

I asked her how it felt to have her name floated as a possible
vice-presidential nominee.

``It's surreal, right?'' she said, recalling that she was once ``a
hungry kid who fainted in class for lack of nutrition. It's unbelievable
I'm even a U.S. senator.''

``But it's one team, one fight,'' she added, referring to the Democratic
quest to defeat Trump. ``I will work as hard as I can to get Joe Biden
elected because the country needs it. It doesn't matter where I end up
on that team.''

Yes, Senator Duckworth, it does. In the right role, you could help
guarantee the right outcome.

\emph{I invite you to sign up for my free}
\href{https://www.nytimes.com/newsletters/frank-bruni}{\emph{weekly
email newsletter}}\emph{. You can follow me on Twitter
(}\href{https://twitter.com/FrankBruni}{\emph{@FrankBruni}}\emph{).}

\emph{Listen to}
\href{https://www.nytimes.com/column/the-argument}{\emph{``The
Argument'' podcast}} \emph{every Thursday morning, with Ross Douthat,
Michelle Goldberg and me.}

Advertisement

\protect\hyperlink{after-bottom}{Continue reading the main story}

\hypertarget{site-index}{%
\subsection{Site Index}\label{site-index}}

\hypertarget{site-information-navigation}{%
\subsection{Site Information
Navigation}\label{site-information-navigation}}

\begin{itemize}
\tightlist
\item
  \href{https://help.nytimes.com/hc/en-us/articles/115014792127-Copyright-notice}{©~2020~The
  New York Times Company}
\end{itemize}

\begin{itemize}
\tightlist
\item
  \href{https://www.nytco.com/}{NYTCo}
\item
  \href{https://help.nytimes.com/hc/en-us/articles/115015385887-Contact-Us}{Contact
  Us}
\item
  \href{https://www.nytco.com/careers/}{Work with us}
\item
  \href{https://nytmediakit.com/}{Advertise}
\item
  \href{http://www.tbrandstudio.com/}{T Brand Studio}
\item
  \href{https://www.nytimes.com/privacy/cookie-policy\#how-do-i-manage-trackers}{Your
  Ad Choices}
\item
  \href{https://www.nytimes.com/privacy}{Privacy}
\item
  \href{https://help.nytimes.com/hc/en-us/articles/115014893428-Terms-of-service}{Terms
  of Service}
\item
  \href{https://help.nytimes.com/hc/en-us/articles/115014893968-Terms-of-sale}{Terms
  of Sale}
\item
  \href{https://spiderbites.nytimes.com}{Site Map}
\item
  \href{https://help.nytimes.com/hc/en-us}{Help}
\item
  \href{https://www.nytimes.com/subscription?campaignId=37WXW}{Subscriptions}
\end{itemize}
