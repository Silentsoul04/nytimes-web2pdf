Sections

SEARCH

\protect\hyperlink{site-content}{Skip to
content}\protect\hyperlink{site-index}{Skip to site index}

\href{https://www.nytimes.com/section/opinion/sunday}{Sunday Review}

\href{https://myaccount.nytimes.com/auth/login?response_type=cookie\&client_id=vi}{}

\href{https://www.nytimes.com/section/todayspaper}{Today's Paper}

\href{/section/opinion/sunday}{Sunday Review}\textbar{}Waking Up in 2030

\href{https://nyti.ms/387Zj0r}{https://nyti.ms/387Zj0r}

\begin{itemize}
\item
\item
\item
\item
\item
\item
\end{itemize}

Advertisement

\protect\hyperlink{after-top}{Continue reading the main story}

\href{/section/opinion}{Opinion}

Supported by

\protect\hyperlink{after-sponsor}{Continue reading the main story}

\hypertarget{waking-up-in-2030}{%
\section{Waking Up in 2030}\label{waking-up-in-2030}}

The suspended time of the pandemic has put history on fast-forward.

\href{https://www.nytimes.com/by/ross-douthat}{\includegraphics{https://static01.nyt.com/images/2018/04/03/opinion/ross-douthat/ross-douthat-thumbLarge.png}}

By \href{https://www.nytimes.com/by/ross-douthat}{Ross Douthat}

Opinion Columnist

\begin{itemize}
\item
  June 27, 2020
\item
  \begin{itemize}
  \item
  \item
  \item
  \item
  \item
  \item
  \end{itemize}
\end{itemize}

\includegraphics{https://static01.nyt.com/images/2020/06/28/opinion/28douthat1/merlin_173217126_b8ea22c8-a030-4af1-a97c-9007e501f5e6-articleLarge.jpg?quality=75\&auto=webp\&disable=upscale}

There is something peculiar about time during the pandemic. On the one
hand there's a feeling that the normal calendar has simply stopped,
school schedules and sports seasons evaporating, one homebound day
passing much like another. It's a feeling of hiatus, intermission, like
the days between Christmas and the new year, or some extra season
invented by a Renaissance pope to fix a lagging calendar.

Yet at the same time there's a feeling of acceleration, of changes that
might have otherwise dragged out across a decade piling one atop the
other. The George Floyd protests and their electoral consequences, the
transformation of liberal institutions by internal agitation, the
changes happening to cities and corporations and colleges and churches
--- in each case, trends that were working slowly have seemingly speeded
up.

This means that when the coronavirus era finally ends, there will be a
Rip Van Winkle feeling --- a sense of having been asleep and waking to
normality, except that we will have time-traveled and the normality will
resemble the year 2030 as it might have been without the virus, rather
than just a simple turn to 2021 or 2022.

What will this 2030-in-2022 look like? First, certain key cultural
institutions will be increasingly consolidated and concentrated,
academia and journalism especially. In the newspaper industry much of
this process happened already, but Covid is delivering a swifter coup de
grâce to midsize daily newspapers and online start-ups, and handing
advantages to a few national entities (ahem) that they might have
otherwise taken five or 10 more years to gain.

In higher education a similar transformation is being pulled forward:
Colleges were expecting a grim landscape in the later 2020s, because
2010s birthrates were so low, but now a decline in foreign enrollment
and an acceleration of online learning will threaten marginal state
schools and possibly close small liberal-arts colleges much sooner. (The
coronavirus experience is also likely to
\href{https://www.brookings.edu/research/half-a-million-fewer-children-the-coming-covid-baby-bust/}{push
birthrates still lower}, delaying any higher ed recovery by years or
decades more.)

The likely winners will be the prestige schools and big state campuses,
who will have the resources to survive and expand and the name brands to
leverage in new online markets --- though so long as pandemic fears
keeps kids close to home, the state schools may
\href{https://www.nytimes.com/2020/06/22/us/coronavirus-universities-brain-drain.html?searchResultPosition=1}{gain
some ground} at the prestige schools' expense.

In religion, the pandemic may strengthen certain forms of faith, but
that won't save institutional churches from what Fordham's David Gibson
calls
\href{https://religionandpolitics.org/2020/06/23/the-coming-religion-recession/}{a
``religion recession''} caused by falling donations and shrunken
attendance. Smaller churches may suffer most, for the same
tight-margins, high-overhead reasons that restaurants are going under.
But big religious bodies like Roman Catholicism and the Southern
Baptists will probably decline as well, in a hurried-up version of the
decay that awaited them with the next decade's worth of generational
turnover. (Any Catholic diocese that had a 10-year plan for closing or
consolidating schools or parishes, for instance, can expect to do the
same thing but much faster.)

In politics, similarly, what was likely to be a slow-motion leftward
shift, as the less-married, less-religious, more ethnically diverse
younger generation gained more power, is being accelerated nationally by
the catastrophes of the Trump administration, which is putting states in
play for Democrats five or 10 years early.

A political shift is certainly accelerating
\href{https://www.nytimes.com/2020/06/12/opinion/nyt-tom-cotton-oped-liberalism.html}{within
elite institutions}, where the younger generation is trying to establish
a new ideological consensus, a new set of standards and boundaries for
behavior and opinion, that otherwise would have advanced more slowly,
with more contestation, over the next 10 years. (That these institutions
are subject to the consolidating forces described above makes the battle
to control them more important, and the professional stakes more
fraught.)

Finally in corporate America, there may be trends toward both
consolidation and dispersal. The former, because even federal
intervention probably won't prevent small businesses from going under
while bigger businesses ride things out, accelerating the pre-existing
drift toward a less entrepreneurial, more monopolist America.

But the latter, because the remote-work experience, pandemic fears and
possibly-rising crime rates may encourage more companies to abandon the
great consolidated hubs of the digital age, or at least fling more
satellite campuses out to Idaho and Iowa and other lower-cost-of-living
states, dispersing talent back into the heartland for the first time in
two generations.

Of the trends I've described, only this last one seems like a hopeful
sign that post-pandemic America might become less sclerotic, less
decadent than the America of 2019. If one wanted to be especially
optimistic, one could add that maybe --- maybe --- a corporate dispersal
will reduce social stratification, and help create new intellectual,
journalistic and even religious centers.

But overall, the pandemic seems likely to bring us more quickly to a
future of consolidated power, weakened human-scale institutions and
growing ideological conformity. Along with far too many lives, that's
what's likely to be lost in this strange between-time: a decade's worth
of chances to take an off-ramp, choose a different direction, or just
stand athwart 2030 yelling stop.

\emph{The Times is committed to publishing}
\href{https://www.nytimes.com/2019/01/31/opinion/letters/letters-to-editor-new-york-times-women.html}{\emph{a
diversity of letters}} \emph{to the editor. We'd like to hear what you
think about this or any of our articles. Here are some}
\href{https://help.nytimes.com/hc/en-us/articles/115014925288-How-to-submit-a-letter-to-the-editor}{\emph{tips}}\emph{.
And here's our email:}
\href{mailto:letters@nytimes.com}{\emph{letters@nytimes.com}}\emph{.}

\emph{Follow The New York Times Opinion section on}
\href{https://www.facebook.com/nytopinion}{\emph{Facebook}}\emph{,}
\href{http://twitter.com/NYTOpinion}{\emph{Twitter (@NYTOpinion)}}
\emph{and}
\href{https://www.instagram.com/nytopinion/}{\emph{Instagram}}\emph{,
join the Facebook political discussion group,}
\href{https://www.facebook.com/groups/votingwhilefemale/}{\emph{Voting
While Female}}\emph{.}

Advertisement

\protect\hyperlink{after-bottom}{Continue reading the main story}

\hypertarget{site-index}{%
\subsection{Site Index}\label{site-index}}

\hypertarget{site-information-navigation}{%
\subsection{Site Information
Navigation}\label{site-information-navigation}}

\begin{itemize}
\tightlist
\item
  \href{https://help.nytimes.com/hc/en-us/articles/115014792127-Copyright-notice}{©~2020~The
  New York Times Company}
\end{itemize}

\begin{itemize}
\tightlist
\item
  \href{https://www.nytco.com/}{NYTCo}
\item
  \href{https://help.nytimes.com/hc/en-us/articles/115015385887-Contact-Us}{Contact
  Us}
\item
  \href{https://www.nytco.com/careers/}{Work with us}
\item
  \href{https://nytmediakit.com/}{Advertise}
\item
  \href{http://www.tbrandstudio.com/}{T Brand Studio}
\item
  \href{https://www.nytimes.com/privacy/cookie-policy\#how-do-i-manage-trackers}{Your
  Ad Choices}
\item
  \href{https://www.nytimes.com/privacy}{Privacy}
\item
  \href{https://help.nytimes.com/hc/en-us/articles/115014893428-Terms-of-service}{Terms
  of Service}
\item
  \href{https://help.nytimes.com/hc/en-us/articles/115014893968-Terms-of-sale}{Terms
  of Sale}
\item
  \href{https://spiderbites.nytimes.com}{Site Map}
\item
  \href{https://help.nytimes.com/hc/en-us}{Help}
\item
  \href{https://www.nytimes.com/subscription?campaignId=37WXW}{Subscriptions}
\end{itemize}
