Sections

SEARCH

\protect\hyperlink{site-content}{Skip to
content}\protect\hyperlink{site-index}{Skip to site index}

\href{/section/world/europe}{Europe}\textbar{}How the World Missed
Covid-19's Silent Spread

\url{https://nyti.ms/2Vo3AHG}

\begin{itemize}
\item
\item
\item
\item
\item
\item
\end{itemize}

\href{https://www.nytimes.com/news-event/coronavirus?action=click\&pgtype=Article\&state=default\&region=TOP_BANNER\&context=storylines_menu}{The
Coronavirus Outbreak}

\begin{itemize}
\tightlist
\item
  live\href{https://www.nytimes.com/2020/08/01/world/coronavirus-covid-19.html?action=click\&pgtype=Article\&state=default\&region=TOP_BANNER\&context=storylines_menu}{Latest
  Updates}
\item
  \href{https://www.nytimes.com/interactive/2020/us/coronavirus-us-cases.html?action=click\&pgtype=Article\&state=default\&region=TOP_BANNER\&context=storylines_menu}{Maps
  and Cases}
\item
  \href{https://www.nytimes.com/interactive/2020/science/coronavirus-vaccine-tracker.html?action=click\&pgtype=Article\&state=default\&region=TOP_BANNER\&context=storylines_menu}{Vaccine
  Tracker}
\item
  \href{https://www.nytimes.com/interactive/2020/07/29/us/schools-reopening-coronavirus.html?action=click\&pgtype=Article\&state=default\&region=TOP_BANNER\&context=storylines_menu}{What
  School May Look Like}
\item
  \href{https://www.nytimes.com/live/2020/07/31/business/stock-market-today-coronavirus?action=click\&pgtype=Article\&state=default\&region=TOP_BANNER\&context=storylines_menu}{Economy}
\end{itemize}

\includegraphics{https://static01.nyt.com/images/2020/06/22/world/xxasymptomatic/xxasymptomatic-articleLarge-v3.jpg?quality=75\&auto=webp\&disable=upscale}

Behind the Curve

\hypertarget{how-the-world-missed-covid-19s-silent-spread}{%
\section{How the World Missed Covid-19's Silent
Spread}\label{how-the-world-missed-covid-19s-silent-spread}}

Dr. Camilla Rothe's team was among the first to warn about asymptomatic
transmission.Credit...Laetitia Vancon for The New York Times

Supported by

\protect\hyperlink{after-sponsor}{Continue reading the main story}

\href{https://www.nytimes.com/es/2020/06/29/espanol/mundo/coronavirus-asintomaticos.html}{Leer
en español}

Symptomless transmission makes the coronavirus far harder to fight. But
health officials dismissed the risk for months, pushing misleading and
contradictory claims in the face of mounting evidence.

By Matt Apuzzo, Selam Gebrekidan and David D. Kirkpatrick

June 27, 2020

\begin{center}\rule{0.5\linewidth}{\linethickness}\end{center}

MUNICH --- Dr. Camilla Rothe was about to leave for dinner when the
government laboratory called with the surprising test result. Positive.
It was Jan. 27. She had just discovered Germany's first case of the new
coronavirus.

But the diagnosis made no sense. Her patient, a businessman from a
nearby auto parts company, could have been infected by only one person:
a colleague visiting from China. And that colleague should not have been
contagious.

The visitor had seemed perfectly healthy during her stay in Germany. No
coughing or sneezing, no signs of fatigue or fever during two days of
long meetings. She told colleagues that she had started feeling ill
after the flight back to China. Days later, she tested positive for the
coronavirus.

Scientists at the time believed that only people with symptoms could
\href{https://www.nytimes.com/2020/07/21/health/coronavirus-infections-us.html}{spread
the coronavirus}. They assumed it acted like its genetic cousin, SARS.

``People who know much more about coronaviruses than I do were
absolutely sure,'' recalled Dr. Rothe, an infectious disease specialist
at Munich University Hospital.

But if the experts were wrong, if the virus could spread from seemingly
healthy carriers or people who had not yet developed symptoms, the
ramifications were potentially catastrophic. Public-awareness campaigns,
airport screening and stay-home-if-you're sick policies might not stop
it. More aggressive measures might be required --- ordering healthy
people to wear masks, for instance, or restricting international travel.

Dr. Rothe and her colleagues were among the first to warn the world. But
even as evidence accumulated from other scientists, leading health
officials expressed unwavering confidence that symptomless spreading was
not important.

In the days and weeks to come, politicians, public health officials and
rival academics disparaged or ignored the Munich team. Some actively
worked to undermine the warnings at a crucial moment, as the disease was
spreading unnoticed in
\href{https://www.reuters.com/article/us-health-coronavirus-france-church-spec/special-report-five-days-of-worship-that-set-a-virus-time-bomb-in-france-idUSKBN21H0Q2}{French
churches}, Italian soccer stadiums and Austrian ski bars. A cruise ship,
the Diamond Princess, would become a deadly harbinger of symptomless
spreading.

\includegraphics{https://static01.nyt.com/images/2020/06/28/world/28asymptomatic-jump-6/merlin_170540889_773e07a6-e978-4dc4-ab9a-610c9b70b23f-articleLarge.jpg?quality=75\&auto=webp\&disable=upscale}

Image

Officers in protective gear boarded the the Diamond Princess in February
to move a person with the coronavirus to a hospital.Credit...Kim
Kyung-Hoon/Reuters

Interviews with doctors and public health officials in more than a dozen
countries show that for two crucial months --- and in the face of
mounting genetic evidence --- Western health officials and political
leaders played down or denied the risk of symptomless spreading. Leading
health agencies including the World Health Organization and the European
Center for Disease Prevention and Control provided contradictory and
sometimes misleading advice. A crucial public health discussion devolved
into a semantic debate over what to call infected people without clear
symptoms.

The two-month delay was a product of faulty scientific assumptions,
academic rivalries and, perhaps most important, a reluctance to accept
that containing the virus would take drastic measures. The resistance to
emerging evidence was one part of the world's sluggish response to the
virus.

It is impossible to calculate the human toll of that delay, but models
suggest that earlier, aggressive action might have saved tens of
thousands of lives. Countries like Singapore and Australia, which used
testing and contact-tracing and moved swiftly to quarantine seemingly
healthy travelers, fared far better than those that did not.

Image

Enjoying a sunny day at the Louvre in Paris in mid-March.Credit...Dmitry
Kostyukov for The New York Times

Image

Patients awaiting test results in March at a hospital in Brescia, Italy,
one of the first parts of Europe to be hit hard by the
coronavirus.Credit...Alessandro Grassani for The New York Times

It is now widely accepted that seemingly healthy people can spread the
virus, though uncertainty remains over how much they have contributed to
the pandemic. Though estimates vary, models using data from
\href{https://www.nature.com/articles/s41591-020-0869-5}{Hong Kong},
\href{https://www.eurosurveillance.org/content/10.2807/1560-7917.ES.2020.25.17.2000257}{Singapore
and China} suggest that 30 to 60 percent of spreading occurs when people
have no symptoms.

``This was, I think, a very simple truth,'' Dr. Rothe said. ``I was
surprised that it would cause such a storm. I can't explain it.''

Even now, with more than 9 million cases around the world, and
\href{https://www.nytimes.com/interactive/2020/world/coronavirus-maps.html}{a
death toll approaching 500,000}, Covid-19 remains an unsolved riddle. It
is too soon to know whether the worst has passed, or if a second global
wave of infections is about to crash down. But it is clear that an array
of countries, from secretive regimes to overconfident democracies, have
fumbled their response, misjudged the virus and ignored their own
emergency plans.

It is also painfully clear that time was a critical commodity in curbing
the virus --- and that too much of it was wasted.

\hypertarget{she-was-not-ill}{%
\subsection{`She Was Not Ill'}\label{she-was-not-ill}}

On the night of Germany's first positive test, the virus had seemed far
away. Fewer than 100 fatalities had been reported worldwide. Italy,
which would become Europe's ground zero, would not record its first
cases for another three days.

A few reports out of China had already suggested the possibility of
symptomless spreading. But nobody had proved it could happen.

That night, Dr. Rothe tapped out an email to a few dozen doctors and
public health officials.

``Infections can actually be transmitted during the incubation period,''
she wrote.

Three more employees from the auto parts company, Webasto, tested
positive the following day. Their symptoms were so mild that, normally,
it's likely that none would have been flagged for testing, or have
thought to stay at home.

Dr. Rothe decided she had to sound the alarm. Her boss, Dr. Michael
Hoelscher, dashed off an email to The New England Journal of Medicine.
``We believe that this observation is of utmost importance,'' he wrote.

Editors responded immediately. How soon could they see the paper?

Image

Dr. Michael Hoelscher in his office during an interview with a local TV
station.Credit...Laetitia Vancon for The New York Times

Image

Dr. Rothe swabbing a volunteer during a Covid-19 study in a nursing home
in Munich this month.Credit...Laetitia Vancon for The New York Times

The next morning, Jan. 30, public health officials interviewed the
Chinese businesswoman by phone. Hospitalized in Shanghai, she explained
that she'd started feeling sick on the flight home. Looking back, maybe
she'd had some mild aches or fatigue, but she had chalked them up to a
long day of travel.

\hypertarget{latest-updates-global-coronavirus-outbreak}{%
\section{\texorpdfstring{\href{https://www.nytimes.com/2020/08/01/world/coronavirus-covid-19.html?action=click\&pgtype=Article\&state=default\&region=MAIN_CONTENT_1\&context=storylines_live_updates}{Latest
Updates: Global Coronavirus
Outbreak}}{Latest Updates: Global Coronavirus Outbreak}}\label{latest-updates-global-coronavirus-outbreak}}

Updated 2020-08-02T01:29:11.393Z

\begin{itemize}
\tightlist
\item
  \href{https://www.nytimes.com/2020/08/01/world/coronavirus-covid-19.html?action=click\&pgtype=Article\&state=default\&region=MAIN_CONTENT_1\&context=storylines_live_updates\#link-34047410}{The
  U.S. reels as July cases more than double the total of any other
  month.}
\item
  \href{https://www.nytimes.com/2020/08/01/world/coronavirus-covid-19.html?action=click\&pgtype=Article\&state=default\&region=MAIN_CONTENT_1\&context=storylines_live_updates\#link-3ac56579}{Top
  officials work to break impasse over jobless benefit.}
\item
  \href{https://www.nytimes.com/2020/08/01/world/coronavirus-covid-19.html?action=click\&pgtype=Article\&state=default\&region=MAIN_CONTENT_1\&context=storylines_live_updates\#link-25930521}{Thousands
  in Berlin protest Germany's coronavirus measures.}
\end{itemize}

\href{https://www.nytimes.com/2020/08/01/world/coronavirus-covid-19.html?action=click\&pgtype=Article\&state=default\&region=MAIN_CONTENT_1\&context=storylines_live_updates}{See
more updates}

More live coverage:
\href{https://www.nytimes.com/live/2020/07/31/business/stock-market-today-coronavirus?action=click\&pgtype=Article\&state=default\&region=MAIN_CONTENT_1\&context=storylines_live_updates}{Markets}

``From her perspective, she was not ill,'' said Nadine Schian, a Webasto
spokeswoman who was on the call. ``She said, `OK, I felt tired. But I've
been in Germany a lot of times before and I always have jet lag.'''

When the health officials described the call, Dr. Rothe and Dr.
Hoelscher quickly finished and submitted their article. Dr. Rothe did
not talk to the patient herself but said she relied on the health
authority summary.

Within hours,
\href{https://www.nejm.org/doi/full/10.1056/NEJMc2001468}{it was
online}. It was a modest clinical observation at a key time. Just days
earlier, the World Health Organization had said it needed more
information about this very topic.

What the authors did not know, however, was that in a suburb 20 minutes
away, another group of doctors had also been rushing to publish a
report. Neither knew what the other was working on, a seemingly small
academic rift that would have global implications.

\hypertarget{academic-hairsplitting}{%
\subsection{Academic Hairsplitting}\label{academic-hairsplitting}}

The second group was made up of officials with the Bavarian health
authority and Germany's national health agency, known as the Robert Koch
Institute. Inside a suburban office, doctors unfurled mural paper and
traced infection routes using colored pens.

Their team, led by the Bavarian epidemiologist Dr. Merle Böhmer,
submitted an article to The Lancet, another premier medical journal. But
the Munich hospital group had scooped them by three hours. Dr. Böhmer
said her team's article, which went unpublished as a result, had reached
similar conclusions but worded them slightly differently.

Dr. Rothe had written that patients appeared to be contagious before the
onset of \emph{any} symptoms. The government team had written that
patients appeared to be contagious before the onset of \emph{full}
symptoms --- at a time when symptoms were so mild that people might not
even recognize them.

The Chinese woman, for example, had woken up in the middle of the night
feeling jet-lagged. Wanting to be sharp for her meetings, she took a
Chinese medicine called 999 --- containing the equivalent of a Tylenol
tablet --- and went back to bed.

Perhaps that had masked a mild fever? Perhaps her jet lag was actually
fatigue? She had reached for a shawl during a meeting. Maybe that was a
sign of chills?

Image

Dr. Merle Böhmer and her team wrote that patients appeared to be
contagious before showing full symptoms, not before showing any
symptoms.~Credit...Laetitia Vancon for The New York Times

Image

Dr. Hoelscher said he refused to change the wording of Dr. Rothe's
report and to replace her name with those of members of the government
task force.Credit...Laetitia Vancon for The New York Times

After two lengthy phone calls with the woman, doctors at the Robert Koch
Institute were convinced that she had simply failed to recognize her
symptoms. They wrote to the editor of The New England Journal of
Medicine, casting doubt on Dr. Rothe's findings.

Editors there decided that the dispute amounted to hairsplitting. If it
took a lengthy interview to identify symptoms, how could anyone be
expected to do it in the real world?

``The question was whether she had something consistent with Covid-19 or
that anyone would have recognized at the time was Covid-19,'' said Dr.
Eric Rubin, the journal's editor.

``The answer seemed to be no.''

The journal did not publish the letter. But that would not be the end of
it.

That weekend, Andreas Zapf, the head of the Bavarian health authority,
called Dr. Hoelscher of the Munich clinic. ``Look, the people in Berlin
are very angry about your publication,'' Dr. Zapf said, according to Dr.
Hoelscher.

He suggested changing the wording of Dr. Rothe's report and replacing
her name with those of members of the government task force, Dr.
Hoelscher said. He refused.

The health agency would not discuss the phone call.

Until then, Dr. Hoelscher said, their report had seemed straightforward.
Now it was clear: ``Politically, this was a major, major issue.''

\hypertarget{a-complete-tsunami}{%
\subsection{`A Complete Tsunami'}\label{a-complete-tsunami}}

On Monday, Feb. 3, the journal Science ** published
\href{https://www.sciencemag.org/news/2020/02/paper-non-symptomatic-patient-transmitting-coronavirus-wrong}{an
article calling Dr. Rothe's report ``flawed.''} Science reported that
the Robert Koch Institute had written to the New England Journal to
dispute her findings and correct an error.

The Robert Koch Institute declined repeated interview requests over
several weeks and did not answer written questions.

Dr. Rothe's report quickly became a symbol of rushed research.
Scientists said she should have talked to the Chinese patient herself
before publishing, and that the omission had undermined her team's work.
On Twitter, she and her colleagues were disparaged by scientists and
armchair experts alike.

``It broke over us like a complete tsunami,'' Dr. Hoelscher said.

The controversy also overshadowed another crucial development out of
Munich.

The next morning, Dr. Clemens-Martin Wendtner
\href{https://instmikrobiobw.de/aktuelles/ansicht/pressemitteilung}{made
a startling announcement}. Dr. Wendtner was overseeing treatment of
Munich's Covid-19 patients --- there were eight now --- and had taken
swabs from each.

He discovered the virus in the nose and throat at much higher levels,
and far earlier, than had been observed in SARS patients. That meant it
probably could spread before people knew they were sick.

Image

Dr. Clemens-Martin Wendtner's work also suggested the risk that patients
could spread the virus before they realized they had
it.Credit...Laetitia Vancon for The New York Times

Image

Dr. Rothe helping a participant fill out a Covid-19 questionnaire at a
nursing home.Credit...Laetitia Vancon for The New York Times

But the Science story drowned that news out. If Dr. Rothe's paper had
implied that governments might need to do more against Covid-19, the
pushback from the Robert Koch Institute was an implicit defense of the
conventional thinking.

Sweden's public health agency declared that Dr. Rothe's report had
contained major errors. The agency's website said, unequivocally, that
``there is no evidence that people are infectious during the incubation
period'' --- an assertion that would remain online in some form for
months.

French health officials, too, left no room for debate: ``A person is
contagious only when symptoms appear,''
\href{https://www.lemonde.fr/les-decodeurs/article/2020/02/06/coronavirus-une-affiche-du-ministere-ecarte-trop-vite-le-risque-de-contagion-lors-de-l-incubation_6028658_4355770.html}{a
government flyer read}. ``No symptoms = no risk of being contagious.''

As Dr. Rothe and Dr. Hoelscher reeled from the criticism, Japanese
doctors were preparing to board the Diamond Princess cruise ship. A
former passenger had tested positive for coronavirus.

Yet on the ship, parties continued. The infected passenger had been off
the ship for days, after all. And he hadn't reported symptoms while
onboard.

\hypertarget{a-semantic-debate}{%
\subsection{A Semantic Debate}\label{a-semantic-debate}}

Immediately after Dr. Rothe's report, the World Health Organization had
noted that patients might transmit the virus before showing symptoms.
But the organization also underscored a point that it continues to make:
Patients with symptoms are the main drivers of the epidemic.

Once the Science article was published, however, the organization waded
directly into the debate on Dr. Rothe's work. On Tuesday, Feb. 4, Dr.
Sylvie Briand, the agency's chief of infectious disease preparedness,
tweeted a link to the Science article, calling Dr. Rothe's report
flawed.

With that tweet, the W.H.O. focused on a semantic distinction that would
cloud discussion for months: Was the patient asymptomatic, meaning she
would never show symptoms? Or pre-symptomatic, meaning she became sick
later? Or, even more confusing, oligo-symptomatic, meaning that she had
symptoms so mild that she didn't recognize them?

To some doctors, the focus on these arcane distinctions felt like
whistling in the graveyard. A person who feels healthy has no way to
know that she is carrying a virus or is about to become sick. Airport
temperature checks will not catch these people. Neither will asking them
about their symptoms or telling them to stay home when they feel ill.

The W.H.O. later said that the tweet had not been intended as a
criticism.

One group paid little attention to this brewing debate: the Munich-area
doctors working to contain the cluster at the auto parts company. They
spoke daily with potentially sick people, monitoring their symptoms and
tracking their contacts.

Image

Dr. Rothe and her team preparing for the day.Credit...Laetitia Vancon
for The New York Times

Image

Dr. Hoelscher said The New England Journal of Medicine paper had become
a ``major, major'' political issue for him.Credit...Laetitia Vancon for
The New York Times

``For us, it was pretty soon clear that this disease can be transmitted
before symptoms,'' said Dr. Monika Wirth, who tracked contacts in the
nearby county of Fürstenfeldbruck.

Dr. Rothe, though, was shaken. She could not understand why much of the
scientific establishment seemed eager to play down the risk.

``All you need is a pair of eyes,'' she said. ``You don't need
rocket-science virology.''

But she remained confident.

``We will be proven right,'' she told Dr. Hoelscher.

That night, Dr. Rothe received an email from Dr. Michael Libman, an
infectious-disease specialist in Montreal. He thought that criticism of
the paper amounted to semantics. Her paper had convinced him of
something: ``The disease will most likely eventually spread around the
world.''

\hypertarget{political-paralysis}{%
\subsection{Political Paralysis}\label{political-paralysis}}

On Feb. 4, Britain's emergency scientific committee met and, while its
experts did not rule out the possibility of symptomless transmission,
nobody put much stock in Dr. Rothe's paper.

``It was very much a hearsay study,'' said Wendy Barclay, a virologist
and member of the committee, known as the Scientific Advisory Group for
Emergencies. ``In the absence of real robust epidemiology and tracing,
it isn't obvious until you see the data.''

The data would soon arrive, and from an unexpected source. Dr. Böhmer,
from the Bavarian health team, received a startling phone call in the
second week of February.

Virologists had discovered a subtle genetic mutation in the infections
of two patients from the Munich cluster. They had crossed paths for the
briefest of moments, one passing a saltshaker to the other in the
company cafeteria, when neither had symptoms. Their shared mutation made
it clear that one had infected the other.

Dr. Böhmer had been skeptical of symptomless spreading. But now, there
was no doubt: ``It can only be explained with pre-symptomatic
transmission,'' Dr. Böhmer said.

Now it was Dr. Böhmer who sounded the alarm. She said she promptly
shared the finding, and its significance, with the W.H.O. and the
European Center for Disease Prevention and Control.

\href{https://www.nytimes.com/news-event/coronavirus?action=click\&pgtype=Article\&state=default\&region=MAIN_CONTENT_3\&context=storylines_faq}{}

\hypertarget{the-coronavirus-outbreak-}{%
\subsubsection{The Coronavirus Outbreak
›}\label{the-coronavirus-outbreak-}}

\hypertarget{frequently-asked-questions}{%
\paragraph{Frequently Asked
Questions}\label{frequently-asked-questions}}

Updated July 27, 2020

\begin{itemize}
\item ~
  \hypertarget{should-i-refinance-my-mortgage}{%
  \paragraph{Should I refinance my
  mortgage?}\label{should-i-refinance-my-mortgage}}

  \begin{itemize}
  \tightlist
  \item
    \href{https://www.nytimes.com/article/coronavirus-money-unemployment.html?action=click\&pgtype=Article\&state=default\&region=MAIN_CONTENT_3\&context=storylines_faq}{It
    could be a good idea,} because mortgage rates have
    \href{https://www.nytimes.com/2020/07/16/business/mortgage-rates-below-3-percent.html?action=click\&pgtype=Article\&state=default\&region=MAIN_CONTENT_3\&context=storylines_faq}{never
    been lower.} Refinancing requests have pushed mortgage applications
    to some of the highest levels since 2008, so be prepared to get in
    line. But defaults are also up, so if you're thinking about buying a
    home, be aware that some lenders have tightened their standards.
  \end{itemize}
\item ~
  \hypertarget{what-is-school-going-to-look-like-in-september}{%
  \paragraph{What is school going to look like in
  September?}\label{what-is-school-going-to-look-like-in-september}}

  \begin{itemize}
  \tightlist
  \item
    It is unlikely that many schools will return to a normal schedule
    this fall, requiring the grind of
    \href{https://www.nytimes.com/2020/06/05/us/coronavirus-education-lost-learning.html?action=click\&pgtype=Article\&state=default\&region=MAIN_CONTENT_3\&context=storylines_faq}{online
    learning},
    \href{https://www.nytimes.com/2020/05/29/us/coronavirus-child-care-centers.html?action=click\&pgtype=Article\&state=default\&region=MAIN_CONTENT_3\&context=storylines_faq}{makeshift
    child care} and
    \href{https://www.nytimes.com/2020/06/03/business/economy/coronavirus-working-women.html?action=click\&pgtype=Article\&state=default\&region=MAIN_CONTENT_3\&context=storylines_faq}{stunted
    workdays} to continue. California's two largest public school
    districts --- Los Angeles and San Diego --- said on July 13, that
    \href{https://www.nytimes.com/2020/07/13/us/lausd-san-diego-school-reopening.html?action=click\&pgtype=Article\&state=default\&region=MAIN_CONTENT_3\&context=storylines_faq}{instruction
    will be remote-only in the fall}, citing concerns that surging
    coronavirus infections in their areas pose too dire a risk for
    students and teachers. Together, the two districts enroll some
    825,000 students. They are the largest in the country so far to
    abandon plans for even a partial physical return to classrooms when
    they reopen in August. For other districts, the solution won't be an
    all-or-nothing approach.
    \href{https://bioethics.jhu.edu/research-and-outreach/projects/eschool-initiative/school-policy-tracker/}{Many
    systems}, including the nation's largest, New York City, are
    devising
    \href{https://www.nytimes.com/2020/06/26/us/coronavirus-schools-reopen-fall.html?action=click\&pgtype=Article\&state=default\&region=MAIN_CONTENT_3\&context=storylines_faq}{hybrid
    plans} that involve spending some days in classrooms and other days
    online. There's no national policy on this yet, so check with your
    municipal school system regularly to see what is happening in your
    community.
  \end{itemize}
\item ~
  \hypertarget{is-the-coronavirus-airborne}{%
  \paragraph{Is the coronavirus
  airborne?}\label{is-the-coronavirus-airborne}}

  \begin{itemize}
  \tightlist
  \item
    The coronavirus
    \href{https://www.nytimes.com/2020/07/04/health/239-experts-with-one-big-claim-the-coronavirus-is-airborne.html?action=click\&pgtype=Article\&state=default\&region=MAIN_CONTENT_3\&context=storylines_faq}{can
    stay aloft for hours in tiny droplets in stagnant air}, infecting
    people as they inhale, mounting scientific evidence suggests. This
    risk is highest in crowded indoor spaces with poor ventilation, and
    may help explain super-spreading events reported in meatpacking
    plants, churches and restaurants.
    \href{https://www.nytimes.com/2020/07/06/health/coronavirus-airborne-aerosols.html?action=click\&pgtype=Article\&state=default\&region=MAIN_CONTENT_3\&context=storylines_faq}{It's
    unclear how often the virus is spread} via these tiny droplets, or
    aerosols, compared with larger droplets that are expelled when a
    sick person coughs or sneezes, or transmitted through contact with
    contaminated surfaces, said Linsey Marr, an aerosol expert at
    Virginia Tech. Aerosols are released even when a person without
    symptoms exhales, talks or sings, according to Dr. Marr and more
    than 200 other experts, who
    \href{https://academic.oup.com/cid/article/doi/10.1093/cid/ciaa939/5867798}{have
    outlined the evidence in an open letter to the World Health
    Organization}.
  \end{itemize}
\item ~
  \hypertarget{what-are-the-symptoms-of-coronavirus}{%
  \paragraph{What are the symptoms of
  coronavirus?}\label{what-are-the-symptoms-of-coronavirus}}

  \begin{itemize}
  \tightlist
  \item
    Common symptoms
    \href{https://www.nytimes.com/article/symptoms-coronavirus.html?action=click\&pgtype=Article\&state=default\&region=MAIN_CONTENT_3\&context=storylines_faq}{include
    fever, a dry cough, fatigue and difficulty breathing or shortness of
    breath.} Some of these symptoms overlap with those of the flu,
    making detection difficult, but runny noses and stuffy sinuses are
    less common.
    \href{https://www.nytimes.com/2020/04/27/health/coronavirus-symptoms-cdc.html?action=click\&pgtype=Article\&state=default\&region=MAIN_CONTENT_3\&context=storylines_faq}{The
    C.D.C. has also} added chills, muscle pain, sore throat, headache
    and a new loss of the sense of taste or smell as symptoms to look
    out for. Most people fall ill five to seven days after exposure, but
    symptoms may appear in as few as two days or as many as 14 days.
  \end{itemize}
\item ~
  \hypertarget{does-asymptomatic-transmission-of-covid-19-happen}{%
  \paragraph{Does asymptomatic transmission of Covid-19
  happen?}\label{does-asymptomatic-transmission-of-covid-19-happen}}

  \begin{itemize}
  \tightlist
  \item
    So far, the evidence seems to show it does. A widely cited
    \href{https://www.nature.com/articles/s41591-020-0869-5}{paper}
    published in April suggests that people are most infectious about
    two days before the onset of coronavirus symptoms and estimated that
    44 percent of new infections were a result of transmission from
    people who were not yet showing symptoms. Recently, a top expert at
    the World Health Organization stated that transmission of the
    coronavirus by people who did not have symptoms was ``very rare,''
    \href{https://www.nytimes.com/2020/06/09/world/coronavirus-updates.html?action=click\&pgtype=Article\&state=default\&region=MAIN_CONTENT_3\&context=storylines_faq\#link-1f302e21}{but
    she later walked back that statement.}
  \end{itemize}
\end{itemize}

Neither organization included the discovery in its regular reports.

A week after receiving Dr. Böhmer's information, European health
officials were still declaring: ``We are still unsure whether mild or
asymptomatic cases can transmit the virus.'' There was no mention of the
genetic evidence.

Image

Dr. Böhmer had been skeptical of symptomless spreading, but her research
ultimately provided genetic proof that it was
happening.Credit...Laetitita Vancon for The New York Times

Image

``This was a misleading statement by the W.HO.,'' Dr. Wendtner said of
remarks in February by the agency's technical lead about symptomless
spreading.Credit...Laetitia Vancon for The New York Times

W.H.O. officials say the genetic discovery informed their thinking, but
they made no announcement of it. European health officials say the
German information was one early piece of an emerging picture that they
were still piecing together.

The doctors in Munich were increasingly frustrated and confused by the
World Health Organization. First, the group wrongly credited the Chinese
government with alerting the German authorities to the first infection.
Government officials and doctors say the auto parts company itself
sounded the alarm.

Then, the World Health Organization's emergency director, Dr. Michael
Ryan, said on Feb. 27 that the significance of symptomless spreading was
becoming a myth. And Dr. Maria Van Kerkhove, the organization's
technical lead on coronavirus response, suggested it was nothing to
worry about.

``It's rare but possible,'' she
\href{https://www.youtube.com/watch?v=SCgCzYAHusA\&t=22m55s}{said}.
``It's very rare.''

The agency still maintains that people who cough or sneeze are more
contagious than people who don't. But there is no scientific consensus
on how significant this difference is or how it affects the spread of
virus.

And so, with evidence mounting, the Munich team could not understand how
the W.H.O. could be so sure that symptomless spreading was
insignificant.

``At this point, for us it was clear,'' said Dr. Wendtner, the senior
doctor overseeing treatment of the Covid-19 patients. ``This was a
misleading statement by the W.HO.''

\hypertarget{if-this-is-true-were-in-trouble}{%
\subsection{`If This Is True, We're in
Trouble'}\label{if-this-is-true-were-in-trouble}}

The Munich cluster was not the only warning.

The Chinese health authorities had explicitly cautioned that patients
were contagious before showing symptoms. A Japanese bus driver was
infected while transporting seemingly healthy tourists from Wuhan.

And by the middle of February, 355 people aboard the Diamond Princess
cruise ship had tested positive. About a third of the infected
passengers and staff had no symptoms.

But public health officials saw danger in promoting the risk of silent
spreaders. If quarantining sick people and tracing their contacts could
not reliably contain the disease, governments might abandon those
efforts altogether.

In Sweden and Britain, for example, discussion swirled about enduring
the epidemic until the population obtained ``herd immunity.'' Public
health officials worried that might lead to overwhelmed hospitals and
needless deaths.

Image

Diners enjoying a night out in Stockholm in April.Credit...Andres
Kudacki for The New York Times

Image

A crowded train in São Paulo, Brazil, last month.Credit...Victor
Moriyama for The New York Times

Plus, preventing silent spreading required aggressive, widespread
testing that was then impossible for most countries.

``It's not like we had some easy alternative,'' said Dr. Libman, the
Canadian doctor. ``The message was basically: `If this is true, we're in
trouble.'''

European health officials say they were reluctant to acknowledge silent
spreading because the evidence was trickling in and the consequences of
a false alarm would have been severe. ``These reports are seen
everywhere, all over the world,'' said Dr. Josep Jansa, a senior
European Union health official. ``Whatever we put out, there's no way
back.''

Looking back, health officials should have said that, yes, symptomless
spreading was happening and they did not understand how prevalent it
was, said Dr. Agoritsa Baka, a senior European Union doctor.

But doing that, she said, would have amounted to an implicit warning to
countries: What you're doing might not be enough.

\hypertarget{stop-buying-masks}{%
\subsection{`Stop Buying Masks!'}\label{stop-buying-masks}}

While public health officials hesitated, some doctors acted. At a
conference in Seattle in mid-February, Jeffrey Shaman, a Columbia
University professor, said
\href{https://science.sciencemag.org/content/368/6490/489}{his research
suggested} that Covid-19's rapid spread could only be explained if there
were infectious patients with unremarkable symptoms or no symptoms at
all.

In the audience that day was Steven Chu, the Nobel-winning physicist and
former U.S. energy secretary. ``If left to its own devices, this disease
will spread through the whole population,'' he remembers Professor
Shaman warning.

Afterward, Dr. Chu began insisting that healthy colleagues at his
Stanford University laboratory wear masks. Doctors in Cambridge,
England, concluded that asymptomatic transmission was a big source of
infection and advised local health workers and patients to wear masks,
well before the British government acknowledged the risk of silent
spreaders.

The American authorities, faced with a shortage, actively discouraged
the public from buying masks. ``Seriously people --- STOP BUYING
MASKS!'' Surgeon General Jerome M. Adams tweeted on Feb. 29.

By early March, while the World Health Organization continued pressing
the case that symptom-free transmission was rare, science was breaking
in the other direction.

Image

Shoppers wearing masks lined up outside a Costco in Livermore,
Calif.Credit...Max Whittaker for The New York Times

Image

Producing cloth masks in Bangkok.Credit...Adam Dean for The New York
Times

Researchers in Hong Kong
\href{https://www.nature.com/articles/s41591-020-0869-5.pdf}{estimated
that} 44 percent of Covid-19 transmission occurred before symptoms
began, an estimate that was in line with
\href{https://science.sciencemag.org/content/368/6491/eabb6936}{a
British study} that put that number as high as 50 percent.

The Hong Kong study
\href{https://www.nature.com/articles/s41591-020-0869-5.pdf}{concluded}
that people became infectious about two days before their illness
emerged, with a peak on their first day of symptoms. By the time
patients felt the first headache or scratch in the throat, they might
have been spreading the disease for days.

In Belgium, doctors saw that math in action, as Covid-19 tore through
nursing homes, killing nearly 5,000 people.

``We thought that by monitoring symptoms and asking sick people to stay
at home, we would be able to manage the spread,'' said Steven Van Gucht,
the head of Belgium's Covid-19 scientific committee. ``It came in
through people with hardly any symptoms.''

More than 700 people aboard the Diamond Princess were sickened. Fourteen
died. Researchers
\href{https://www.eurosurveillance.org/content/10.2807/1560-7917.ES.2020.25.10.2000180\#html_fulltext}{estimate}
that most of the infection occurred early on, while seemingly healthy
passengers socialized and partied.

Government scientists in Britain
\href{https://assets.publishing.service.gov.uk/government/uploads/system/uploads/attachment_data/file/888804/S0399_Thirtieth_SAGE_meeting_on_Covid-19_.pdf}{concluded}
in late April that 5 to 6 percent of symptomless health care workers
were infected and might have been spreading the virus.

In Munich, Dr. Hoelscher has asked himself many times whether things
would have been different if world leaders had taken the issue seriously
earlier. He compared their response to a rabbit stumbling upon a
poisonous snake.

``We were watching that snake and were somehow paralyzed,'' he said.

\hypertarget{acceptance-or-not}{%
\subsection{Acceptance. Or Not.}\label{acceptance-or-not}}

As the research coalesced in March, European health officials were
convinced.

``OK, this is really a big issue,'' Dr. Baka recalled thinking. ``It
plays a big role in the transmission.''

By the end of the month, the U.S. Centers for Disease Control announced
it was rethinking its policy on masks. It concluded that up to
\href{https://www.nytimes.com/2020/03/31/health/coronavirus-asymptomatic-transmission.html}{25
percent} of patients might have no symptoms.

Since then, the C.D.C., governments around the world and, finally, the
World Health Organization have recommended that people wear masks in
public.

Still, the W.H.O. is sending confusing signals. Earlier this month, Dr.
Van Kerkhove, the technical lead, repeated that transmission from
asymptomatic patients was ``very rare.'' After an outcry from doctors,
the agency said there had been a misunderstanding.

``In all honesty, we don't have a clear picture on this yet,'' Dr. Van
Kerkhove said. She said she had been referring to a few studies showing
limited transmission from asymptomatic patients.

Image

Dr. Rothe at home.Credit...Laetitia Vancon for The New York Times

Image

Dr. Böhmer published a study in The Lancet last month~that found
``substantial'' transmission from people with no symptoms or
exceptionally mild, nonspecific symptoms.Credit...Laetitia Vancon for
The New York Times

Recent internet ads confused the matter even more. A Google search in
mid-June for studies on asymptomatic transmission returned a W.H.O.
advertisement titled: ``People With No Symptoms --- Rarely Spread
Coronavirus.''

Clicking on the link, however, offered a much more nuanced picture:
``Some reports have indicated that people with no symptoms can transmit
the virus. It is not yet known how often it happens.''

After The Times asked about those discrepancies, the organization
removed the advertisements.

Back in Munich, there is little doubt left. Dr. Böhmer, the Bavarian
government doctor, published
\href{https://www.thelancet.com/journals/laninf/article/PIIS1473-3099(20)30314-5/fulltex}{a
study in The Lancet} last month that relied on extensive interviews and
genetic information to methodically track every case in the cluster.

In the months after Dr. Rothe swabbed her first patient, 16 infected
people were identified and caught early. All survived. Aggressive
testing and flawless contact-tracing contained the spread.

Dr. Böhmer's study found ``substantial'' transmission from people with
no symptoms or exceptionally mild, nonspecific symptoms.

Dr. Rothe and her colleagues got a footnote.

Advertisement

\protect\hyperlink{after-bottom}{Continue reading the main story}

\hypertarget{site-index}{%
\subsection{Site Index}\label{site-index}}

\hypertarget{site-information-navigation}{%
\subsection{Site Information
Navigation}\label{site-information-navigation}}

\begin{itemize}
\tightlist
\item
  \href{https://help.nytimes.com/hc/en-us/articles/115014792127-Copyright-notice}{©~2020~The
  New York Times Company}
\end{itemize}

\begin{itemize}
\tightlist
\item
  \href{https://www.nytco.com/}{NYTCo}
\item
  \href{https://help.nytimes.com/hc/en-us/articles/115015385887-Contact-Us}{Contact
  Us}
\item
  \href{https://www.nytco.com/careers/}{Work with us}
\item
  \href{https://nytmediakit.com/}{Advertise}
\item
  \href{http://www.tbrandstudio.com/}{T Brand Studio}
\item
  \href{https://www.nytimes.com/privacy/cookie-policy\#how-do-i-manage-trackers}{Your
  Ad Choices}
\item
  \href{https://www.nytimes.com/privacy}{Privacy}
\item
  \href{https://help.nytimes.com/hc/en-us/articles/115014893428-Terms-of-service}{Terms
  of Service}
\item
  \href{https://help.nytimes.com/hc/en-us/articles/115014893968-Terms-of-sale}{Terms
  of Sale}
\item
  \href{https://spiderbites.nytimes.com}{Site Map}
\item
  \href{https://help.nytimes.com/hc/en-us}{Help}
\item
  \href{https://www.nytimes.com/subscription?campaignId=37WXW}{Subscriptions}
\end{itemize}
