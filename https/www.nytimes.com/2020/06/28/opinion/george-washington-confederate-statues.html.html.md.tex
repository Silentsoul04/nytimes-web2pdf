Sections

SEARCH

\protect\hyperlink{site-content}{Skip to
content}\protect\hyperlink{site-index}{Skip to site index}

\href{https://myaccount.nytimes.com/auth/login?response_type=cookie\&client_id=vi}{}

\href{https://www.nytimes.com/section/todayspaper}{Today's Paper}

\href{/section/opinion}{Opinion}\textbar{}Yes, Even George Washington

\href{https://nyti.ms/2YH8LET}{https://nyti.ms/2YH8LET}

\begin{itemize}
\item
\item
\item
\item
\item
\item
\end{itemize}

Advertisement

\protect\hyperlink{after-top}{Continue reading the main story}

\href{/section/opinion}{Opinion}

Supported by

\protect\hyperlink{after-sponsor}{Continue reading the main story}

\hypertarget{yes-even-george-washington}{%
\section{Yes, Even George Washington}\label{yes-even-george-washington}}

Slavery was a cruel institution that can't be excused by its era.

\href{https://www.nytimes.com/by/charles-m-blow}{\includegraphics{https://static01.nyt.com/images/2018/04/02/opinion/charles-m-blow/charles-m-blow-thumbLarge.png}}

By \href{https://www.nytimes.com/by/charles-m-blow}{Charles M. Blow}

Opinion Columnist

\begin{itemize}
\item
  June 28, 2020
\item
  \begin{itemize}
  \item
  \item
  \item
  \item
  \item
  \item
  \end{itemize}
\end{itemize}

\includegraphics{https://static01.nyt.com/images/2020/06/28/opinion/28Blow/merlin_169899546_198b14d4-f6fe-4442-a8a4-ec0ff86b55bc-articleLarge.jpg?quality=75\&auto=webp\&disable=upscale}

On the issue of American slavery, I am an absolutist: enslavers were
amoral monsters.

The very idea that one group of people believed that they had the right
to own another human being is abhorrent and depraved. The fact that
their control was enforced by violence was barbaric.

People often try to explain this away by saying that the people who
enslaved Africans in this country were simply men and women of their
age, abiding by the mores of the time.

But, that explanation falters. There were also men and women of the time
who found slavery morally reprehensible. The enslavers ignored all this
and used anti-black dehumanization to justify the holding of slaves and
the profiting from slave labor.

People say that some slave owners were kinder than others.

That explanation too is problematic. The withholding of another person's
freedom is itself violent. And the enslaved people who were shipped to
America via the Middle Passage had already endured unspeakably horrific
treatment.

One of the few written accounts of the atrocious conditions on these
ships comes from a man named the Rev. Robert Walsh. The British
government
\href{https://www.thirteen.org/wnet/historyofus/web05/features/source/C04.html}{outlawed
the international slave trade} in 1807, followed by the United States in
1808. The two nations patrolled the seas to prevent people from
continuing to kidnap Africans and bringing them to those countries
illegally. In 1829, one of the patrols spotted such a ship, and what
Walsh saw when he boarded the ship is beyond belief.

The ship had been at sea for 17 days. There were over 500 kidnapped
Africans onboard. Fifty-five had already been thrown overboard.

The Africans were crowded below the main deck. Each deck was only 3 feet
3 inches high. They were packed so tight that they were sitting up
between one another's legs, everyone completely nude.
\href{https://www.thirteen.org/wnet/historyofus/web05/features/source/docs/C04.pdf}{As
Walsh recounted}, ``there was no possibility of their lying down or at
all changing their position by night or day.''

Each had been branded, ``burnt with the red-hot iron,'' on their breast
or arm. Many were children, little girls and little boys.

Not only could light not reach down into the bowels of those ships,
neither could fresh air. As Walsh recounted, ``The heat of these horrid
places was so great and the odor so offensive that it was quite
impossible to enter them, even had there been room.''

These people, these human beings, sat in their own vomit, urine and
feces, and that of others. If another person sat between your legs,
their bowels emptied out on you.

These voyages regularly lasted over a month, meaning many women onboard
experienced menstruation in these conditions.

Many of the enslaved, sick or driven mad, were thrown overboard. Others
simply jumped. In fact, there was so much human flesh going over the
side of those ships that sharks learned to trail them.

This voyage was so horrific that I can only surmise that the men, women
and children who survived it were superhuman, the toughest and the most
resilient our species has to offer.

But of the people who showed up to greet these reeking vessels of human
torture, to bid on its cargo, or to in any way benefit from the trade
and industry that provided the demand for such a supply, I have absolute
contempt.

Some people who are opposed to taking down monuments ask, ``If we start,
where will we stop?'' It might begin with Confederate generals, but all
slave owners could easily become targets. Even George Washington
himself.

To that I say, ``abso-fricking-lutely!''

George Washington enslaved more than 100 human beings, and he signed the
\href{https://www.history.com/topics/black-history/fugitive-slave-acts}{Fugitive
Slave Act of 1793}, authorizing slavers to stalk runaways even in free
states and criminalizing the helping of escaped slaves. When one of the
African people he himself had enslaved escaped, a woman named Ona Maria
Judge, he pursued her relentlessly, sometimes illegally.

Washington would free his slaves in his will, when he no longer had use
for them.

Let me be clear: Those black people enslaved by George Washington and
others, including other founders, were just as much human as I am today.
They love, laugh, cry and hurt just like I do.

When I hear people excuse their enslavement and torture as an artifact
of the times, I'm forced to consider that if slavery were the prevailing
normalcy of this time, my own enslavement would also be a shrug of the
shoulders.

I say that we need to reconsider public monuments in public spaces. No
person's honorifics can erase the horror he or she has inflicted on
others.

Slave owners should not be honored with monuments in public spaces. We
have museums for that, which also provide better context. This is not an
erasure of history, but rather a better appreciation of the horrible
truth of it.

\emph{The Times is committed to publishing}
\href{https://www.nytimes.com/2019/01/31/opinion/letters/letters-to-editor-new-york-times-women.html}{\emph{a
diversity of letters}} \emph{to the editor. We'd like to hear what you
think about this or any of our articles. Here are some}
\href{https://help.nytimes.com/hc/en-us/articles/115014925288-How-to-submit-a-letter-to-the-editor}{\emph{tips}}\emph{.
And here's our email:}
\href{mailto:letters@nytimes.com}{\emph{letters@nytimes.com}}\emph{.}

\emph{Follow The New York Times Opinion section on}
\href{https://www.facebook.com/nytopinion}{\emph{Facebook}} \emph{and}
\href{http://twitter.com/NYTOpinion}{\emph{Twitter
(@NYTopinion)}}\emph{, and}
\href{https://www.instagram.com/nytopinion/}{\emph{Instagram}}\emph{.}

Advertisement

\protect\hyperlink{after-bottom}{Continue reading the main story}

\hypertarget{site-index}{%
\subsection{Site Index}\label{site-index}}

\hypertarget{site-information-navigation}{%
\subsection{Site Information
Navigation}\label{site-information-navigation}}

\begin{itemize}
\tightlist
\item
  \href{https://help.nytimes.com/hc/en-us/articles/115014792127-Copyright-notice}{©~2020~The
  New York Times Company}
\end{itemize}

\begin{itemize}
\tightlist
\item
  \href{https://www.nytco.com/}{NYTCo}
\item
  \href{https://help.nytimes.com/hc/en-us/articles/115015385887-Contact-Us}{Contact
  Us}
\item
  \href{https://www.nytco.com/careers/}{Work with us}
\item
  \href{https://nytmediakit.com/}{Advertise}
\item
  \href{http://www.tbrandstudio.com/}{T Brand Studio}
\item
  \href{https://www.nytimes.com/privacy/cookie-policy\#how-do-i-manage-trackers}{Your
  Ad Choices}
\item
  \href{https://www.nytimes.com/privacy}{Privacy}
\item
  \href{https://help.nytimes.com/hc/en-us/articles/115014893428-Terms-of-service}{Terms
  of Service}
\item
  \href{https://help.nytimes.com/hc/en-us/articles/115014893968-Terms-of-sale}{Terms
  of Sale}
\item
  \href{https://spiderbites.nytimes.com}{Site Map}
\item
  \href{https://help.nytimes.com/hc/en-us}{Help}
\item
  \href{https://www.nytimes.com/subscription?campaignId=37WXW}{Subscriptions}
\end{itemize}
