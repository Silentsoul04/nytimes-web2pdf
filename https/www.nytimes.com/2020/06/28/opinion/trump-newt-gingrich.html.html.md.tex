Sections

SEARCH

\protect\hyperlink{site-content}{Skip to
content}\protect\hyperlink{site-index}{Skip to site index}

\href{https://myaccount.nytimes.com/auth/login?response_type=cookie\&client_id=vi}{}

\href{https://www.nytimes.com/section/todayspaper}{Today's Paper}

\href{/section/opinion}{Opinion}\textbar{}Trump's Napalm Politics? They
Began With Newt

\href{https://nyti.ms/3i7c0gq}{https://nyti.ms/3i7c0gq}

\begin{itemize}
\item
\item
\item
\item
\item
\item
\end{itemize}

Advertisement

\protect\hyperlink{after-top}{Continue reading the main story}

\href{/section/opinion}{Opinion}

Supported by

\protect\hyperlink{after-sponsor}{Continue reading the main story}

\hypertarget{trumps-napalm-politics-they-began-with-newt}{%
\section{Trump's Napalm Politics? They Began With
Newt}\label{trumps-napalm-politics-they-began-with-newt}}

Gingrich wrote the playbook for it all. The nastiness, the contempt for
norms, the transformation of political opponents into enemies.

\href{https://www.nytimes.com/by/jennifer-senior}{\includegraphics{https://static01.nyt.com/images/2018/10/26/opinion/jennifer-senior/jennifer-senior-thumbLarge.png}}

By \href{https://www.nytimes.com/by/jennifer-senior}{Jennifer Senior}

Opinion columnist

\begin{itemize}
\item
  June 28, 2020
\item
  \begin{itemize}
  \item
  \item
  \item
  \item
  \item
  \item
  \end{itemize}
\end{itemize}

\includegraphics{https://static01.nyt.com/images/2020/06/28/opinion/28Senior/merlin_110121425_ff630430-e6d2-4779-b767-9d43850923c1-articleLarge.jpg?quality=75\&auto=webp\&disable=upscale}

Approximately one billion news cycles ago --- which is to say, on June 9
--- a businesswoman named Marjorie Taylor Greene finished first in the
Republican primary in Georgia's deeply conservative 14th Congressional
District, northwest of Atlanta, which means that after a runoff she's
all but assured a seat in the House of Representatives next year.

Unfortunately, she is a cheerful bigot and conspiracy-theory
fluffernutter. She subscribes to
\href{https://www.nytimes.com/2020/06/18/us/politics/qanon-candidates.html?searchResultPosition=1}{QAnon},
the far-right fever dream that says Donald Trump is under siege from a
cabal of deep-state saboteurs, some of whom run a pedophile ring; she
says African-Americans are being held back primarily by ``gangs.''
(She's left behind a contrail of
\href{https://www.youtube.com/watch?v=2rtYok4fdbQ\&feature=youtu.be}{unsavory}
\href{https://www.politico.com/news/2020/06/17/house-republicans-condemn-gop-candidate-racist-videos-325579}{videos}
through cyberspace, if you'd care to Google.)

The House Republican leadership is
\href{https://www.nytimes.com/2020/06/17/us/marjorie-taylor-greene-georgia.html}{trying
to distance} itself from this woman, as if she belongs to some other
party from a faraway galaxy. She doesn't. Her politics are Trumpism
distilled. And Trumpism itself isn't a style and philosophy that began
in 2016, with Trump's election, or even in 2010, with the Tea Party. It
began 40-odd years ago, in Greene's own state, with the election of a
different politician just two districts over.

I'm talking about Newt. You really could argue that today's napalm
politics began with Newt.

The normalization of personal destruction. The contempt for custom. The
media-baiting, the annihilation of bipartisan comity, the delegitimizing
of institutions.

``Gingrich had planted; Trump had reaped,'' writes the Princeton
historian Julian Zelizer in the prologue to his
\href{https://www.nytimes.com/2020/06/24/books/new-july-books.html}{forthcoming
book,} ``Burning Down the House: Newt Gingrich, the Fall of a Speaker,
and the Rise of a New Republican Party.''

I recently read Zelizer's book with morbid fascination. My first real
job in journalism was as a reporter for the The Hill newspaper the year
it launched, in 1994, which happened to be the same year Republicans won
control of the House, overturning four decades of Democratic rule. (I
wrote nothing memorable that day, but I did come up with our banner
headline: ``It's Reigning Republicans.'')

Gingrich became speaker the following January. It was a stunning
development. Previous speakers, no matter how partisan they were, tended
to work, lunch and
\href{https://campaignstops.blogs.nytimes.com/2012/10/05/frenemies-a-love-story/}{even
drink} across the aisle. The only kind of cocktails Gingrich was partial
to were Molotovs.

He conceived of governing as war. Democrats were not merely to be
defeated ideologically. They were to be immolated.

Even as an inexperienced kid, I could see his ascension was bad news.
Looking back, the parallels between then and now couldn't be clearer.

Democrats were devastated that a man with so much malignity and anger in
his heart could suddenly be at the helm; but in Republicans, Gingrich
had a cult.

Gingrich despised the mainstream press, breaking with tradition and
giving valuable real estate over in the Capitol
\href{https://www.washingtonpost.com/archive/lifestyle/1995/01/05/the-talkmeisters/528fe9d1-e4e7-4c51-a0a1-d5f6e2fee610/}{to
conservative, nativist-populist radio hosts} who spoke loudly and
carried a big schtick, just as Trump gives coveted space to the servile
\href{https://www.nytimes.com/article/oann-trump.html}{One America News
Network}.

Gingrich was my introduction to Orwellian newspeak. He had this tic of
starting every other paragraph with ``frankly'' and then telling a lie;
it was his poker tell. Falsehoods and hyperbole came as naturally to him
as smirking. He freely trafficked
\href{https://www.washingtonpost.com/archive/politics/1995/07/26/gingrich-not-convinced-foster-death-was-suicide/4003f01f-83ad-4b41-9b9c-e209344a716c/}{in
conspiracy theories}. His PAC circulated a pamphlet for aspiring
politicians who wished ``to speak like Newt.'' It advised them to repeat
a long list of words to describe Democrats, including \emph{sick,
pathetic, corrupt.}

Like Trump, Gingrich was a thrice-married womanizer who'd somehow
seduced the evangelicals. He too had a skyscraping ego, nursed grudges
as if they were newborns, and lacked impulse control. In 1995, Bill
Clinton made him sit in the back of Air Force One; he responded with a
tantrum and shut down the government, prompting The New York Daily News
to run
\href{https://www.nydailynews.com/blogs/pageviews/newt-gingrich-crybaby-famous-daily-news-cover-explained-blog-entry-1.1637386}{a
cartoon cover} of him in a diaper under the headline ``Cry Baby.''

Gingrich turned the politics of white racial grievance into an art form.
They may have started with Nixon's Southern Strategy, but Gingrich
actually \emph{came} from the South. He intuited the backlash to
globalization, to affirmative action; the culture teemed with stories
about white men under siege. (Including the Michael Douglas movie
``\href{https://www.nytimes.com/1993/02/26/movies/review-film-urban-horrors-all-too-familiar.html}{Falling
Down},'' about a divorced, unemployed defense contractor's descent into
armed madness.) It wasn't long before 1994 became known as
``\href{http://ks/edition/Breaking_the_Political_Glass_Ceiling/XnHB9eQJKrkC}{The
Year of the Angry White Male}.''

Most of Zelizer's book is about Gingrich's Javert-like quest to bring
down the House speaker, Jim Wright, for his shady ethics. (Gingrich
succeeded, only to later be
\href{https://archive.nytimes.com/www.nytimes.com/library/politics/0122gingrich-ethics.html}{reprimanded
and fined} for his own ethical breaches.) Zelizer never mentions
individual parallels to Trump once he starts telling Gingrich's story,
which is clever, because there's no need. They hop off the page like
frogs.

But the one that stands out, the one that goosepimples me even as I
type, is this: Gingrich was the first true reality TV politician. He
understood that the C-Span cameras didn't have to be a passively
recording set of eyes. You could operatically perform for them. Early in
his career, Gingrich staged a coordinated attack on House Democrats that
drew so much fury from Speaker Tip O'Neill it earned him time on the
evening news. ``I'm famous,'' he crowed.

``Conflict equals exposure equals power,'' became one of his favorite
sayings. Which may as well be the motto of reality television. And
Trump.

Assuming she wins in November, Marjorie Taylor Greene will likely be
relegated to the margins of her caucus. But if Gingrich --- and Trump
--- have taught us anything, it's that there's no telling where the last
exit is on the loonytown expressway to extremism; we know only that the
guardrails get lower with each passing mile. ``These are the depths to
which we've descended,'' Zelizer told me in a phone call. ``No one ever
thinks that an outlier will one day be the party's future.''

\emph{The Times is committed to publishing}
\href{https://www.nytimes.com/2019/01/31/opinion/letters/letters-to-editor-new-york-times-women.html}{\emph{a
diversity of letters}} \emph{to the editor. We'd like to hear what you
think about this or any of our articles. Here are some}
\href{https://help.nytimes.com/hc/en-us/articles/115014925288-How-to-submit-a-letter-to-the-editor}{\emph{tips}}\emph{.
And here's our email:}
\href{mailto:letters@nytimes.com}{\emph{letters@nytimes.com}}\emph{.}

\emph{Follow The New York Times Opinion section on}
\href{https://www.facebook.com/nytopinion}{\emph{Facebook}}\emph{,}
\href{http://twitter.com/NYTOpinion}{\emph{Twitter (@NYTopinion)}}
\emph{and}
\href{https://www.instagram.com/nytopinion/}{\emph{Instagram}}\emph{.}

Advertisement

\protect\hyperlink{after-bottom}{Continue reading the main story}

\hypertarget{site-index}{%
\subsection{Site Index}\label{site-index}}

\hypertarget{site-information-navigation}{%
\subsection{Site Information
Navigation}\label{site-information-navigation}}

\begin{itemize}
\tightlist
\item
  \href{https://help.nytimes.com/hc/en-us/articles/115014792127-Copyright-notice}{©~2020~The
  New York Times Company}
\end{itemize}

\begin{itemize}
\tightlist
\item
  \href{https://www.nytco.com/}{NYTCo}
\item
  \href{https://help.nytimes.com/hc/en-us/articles/115015385887-Contact-Us}{Contact
  Us}
\item
  \href{https://www.nytco.com/careers/}{Work with us}
\item
  \href{https://nytmediakit.com/}{Advertise}
\item
  \href{http://www.tbrandstudio.com/}{T Brand Studio}
\item
  \href{https://www.nytimes.com/privacy/cookie-policy\#how-do-i-manage-trackers}{Your
  Ad Choices}
\item
  \href{https://www.nytimes.com/privacy}{Privacy}
\item
  \href{https://help.nytimes.com/hc/en-us/articles/115014893428-Terms-of-service}{Terms
  of Service}
\item
  \href{https://help.nytimes.com/hc/en-us/articles/115014893968-Terms-of-sale}{Terms
  of Sale}
\item
  \href{https://spiderbites.nytimes.com}{Site Map}
\item
  \href{https://help.nytimes.com/hc/en-us}{Help}
\item
  \href{https://www.nytimes.com/subscription?campaignId=37WXW}{Subscriptions}
\end{itemize}
