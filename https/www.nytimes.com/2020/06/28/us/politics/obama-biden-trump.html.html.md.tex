Sections

SEARCH

\protect\hyperlink{site-content}{Skip to
content}\protect\hyperlink{site-index}{Skip to site index}

\href{https://www.nytimes.com/section/politics}{Politics}

\href{https://myaccount.nytimes.com/auth/login?response_type=cookie\&client_id=vi}{}

\href{https://www.nytimes.com/section/todayspaper}{Today's Paper}

\href{/section/politics}{Politics}\textbar{}How the Trump Campaign Is
Drawing Obama Out of Retirement

\url{https://nyti.ms/3eFk1XW}

\begin{itemize}
\item
\item
\item
\item
\item
\item
\end{itemize}

\begin{itemize}
\item
  \href{https://www.nytimes.com/2020/07/31/us/elections/biden-vs-trump.html?action=click\&pgtype=Article\&state=default\&region=TOP_BANNER\&context=storylines_menu}{Election
  Updates}
\item
  \href{https://www.nytimes.com/article/biden-vice-president-2020.html?action=click\&pgtype=Article\&state=default\&region=TOP_BANNER\&context=storylines_menu}{Biden's
  V.P. Search}
\item
  \href{https://www.nytimes.com/interactive/2020/07/24/us/politics/trump-biden-campaign-donors.html?action=click\&pgtype=Article\&state=default\&region=TOP_BANNER\&context=storylines_menu}{Map
  of Donations}
\item
  \href{https://www.nytimes.com/interactive/2020/us/elections/delegate-count-primary-results.html?action=click\&pgtype=Article\&state=default\&region=TOP_BANNER\&context=storylines_menu}{Delegate
  Count}
\item
  \href{https://www.nytimes.com/interactive/2019/us/politics/2020-presidential-candidates.html?action=click\&pgtype=Article\&state=default\&region=TOP_BANNER\&context=storylines_menu}{The
  Candidates}
\item
  \href{https://www.nytimes.com/newsletters/politics?action=click\&pgtype=Article\&state=default\&region=TOP_BANNER\&context=storylines_menu}{Politics
  Newsletter}
\end{itemize}

Advertisement

\protect\hyperlink{after-top}{Continue reading the main story}

Supported by

\protect\hyperlink{after-sponsor}{Continue reading the main story}

\hypertarget{how-the-trump-campaign-is-drawing-obama-out-of-retirement}{%
\section{How the Trump Campaign Is Drawing Obama Out of
Retirement}\label{how-the-trump-campaign-is-drawing-obama-out-of-retirement}}

\includegraphics{https://static01.nyt.com/images/2020/06/30/us/politics/00obama1/merlin_114147145_5babe815-5725-413c-9128-61bf3f9f1a39-articleLarge.jpg?quality=75\&auto=webp\&disable=upscale}

By \href{https://www.nytimes.com/by/glenn-thrush}{Glenn Thrush} and
\href{https://www.nytimes.com/by/elaina-plott}{Elaina Plott}

\begin{itemize}
\item
  Published June 28, 2020Updated July 30, 2020
\item
  \begin{itemize}
  \item
  \item
  \item
  \item
  \item
  \item
  \end{itemize}
\end{itemize}

\href{https://www.nytimes.com/es/2020/06/30/espanol/obama-biden-trump.html}{Leer
en español}

Just after Donald J. Trump was elected president,
\href{https://www.nytimes.com/2020/07/30/us/politics/obama-trump-biden.html}{Barack
Obama} slumped in his chair in the Oval Office and addressed an aide
standing near a conspicuously placed bowl of apples, emblem of a
healthy-snacking policy soon to be swept aside, along with so much else.

\emph{{[}Election 2020:}
\href{https://www.nytimes.com/2020/07/23/us/politics/barack-obama-joe-biden-video.html}{\emph{Joe
Biden and Barack Obama join forces against Trump}}\emph{.{]}}

``I am so done with all of this,'' Mr. Obama said of his job, according
to several people familiar with the exchange.

Yet he knew, even then, that a conventional White House retirement was
not an option. Mr. Obama, 55 at the time, was stuck holding a baton he
had wanted to pass to Hillary Clinton, and saddled with a successor
whose fixation on him, he believed, was rooted in a bizarre personal
animus and the politics of racial backlash exemplified by the birther
lie.

``There is no model for my kind of post-presidency,'' he told the aide.
``I'm clearly renting space inside the guy's head.''

Which is not to say that Mr. Obama was not committed to his pre-Trump
retirement vision --- a placid life that was to consist of writing,
sun-flecked fairways, policy work through his foundation, producing
documentaries with Netflix and family time aplenty at a new \$11.7
million spread on Martha's Vineyard.

Still, more than three years after his exit, the 44th president of the
United States is back on a political battlefield he longed to leave,
drawn into the fight by an enemy, Mr. Trump, who is hellbent on erasing
him, and by a friend,
\href{https://www.nytimes.com/2020/07/23/arts/television/biden-obama-reunion-video.html}{Joseph
R. Biden Jr}., who is equally intent on embracing him.

The stakes of that re-engagement were always going to be high. Mr. Obama
is nothing if not protective of his legacy, especially in the face of
Mr. Trump's many attacks. Yet interviews with more than 50 people in the
former president's orbit portray a conflicted combatant, trying to
balance deep anger at his successor with an instinct to refrain from a
brawl that he fears may dent his popularity and challenge his place in
history.

That calculus, though, may be changing in the wake of George Floyd's
killing by the police in Minneapolis. As America's first black
president, now its first black ex-president, Mr. Obama sees the current
social and racial awakening as an opportunity to elevate a 2020 election
dictated by Mr. Trump's mud-wrestling style into something more
meaningful --- to channel a new, youthful movement toward a political
aim, as he did in 2008.

He is doing so very carefully, characteristically intent on keeping his
cool, his reputation, his political capital and his dreams of a cosseted
retirement intact.

``I don't think he is hesitant. I think he is strategic,'' said Dan
Pfeiffer, a top adviser for over a decade. ``He has always been
strategic about using his voice; it's his most valuable commodity.''

Mr. Obama is also mindful of a cautionary example: Bill Clinton's
attacks against him in 2008 backfired so badly that his wife's campaign
staff had to scale back his appearances.

Many supporters have been pressing him to be more aggressive.

``It would be nice, for a change, if Barack Obama could emerge from his
cave and offer --- no wait, DEMAND --- a way forward,'' the columnist
Drew Magary wrote in a much-shared Medium post in April titled
``\href{https://gen.medium.com/where-the-hell-is-barack-obama-397ce8d7bbe2}{Where
the Hell is Barack Obama?}''

The counterargument: He did his job and deserves to be left alone.

\includegraphics{https://static01.nyt.com/images/2020/06/26/us/politics/00obama2/merlin_115707257_44ee1a9d-c059-4d5a-8d1d-1afcff6e5510-articleLarge.jpg?quality=75\&auto=webp\&disable=upscale}

``Obama has now been out of office for three and a half years, and he is
still facing this kind of scrutiny --- no one is pressuring white
ex-presidents like George W. Bush and Jimmy Carter the same way,'' said
Monique Judge, news editor of the online magazine The Root and author of
a 2018 article arguing that Mr. Obama
\href{https://www.theroot.com/obama-doesn-t-owe-this-country-shit-1826309455}{no
longer owed the country a}
\href{https://www.theroot.com/obama-doesn-t-owe-this-country-shit-1826309455}{thing}.

Mr. Obama's head appears to be somewhere in the middle. He is not
planning to scrap his summer Vineyard vacation and is still anguishing
over the publication date of his long-awaited memoir. But last week he
stepped up his nominally indirect criticism of Mr. Trump's
administration --- decrying a
``\href{https://www.nytimes.com/2020/06/23/us/politics/obama-biden-fundraiser.html}{shambolic,
disorganized, meanspirited approach to governance}'' during an online
Biden fund-raiser. And he made a pledge of sorts, telling Mr. Biden's
supporters: ``Whatever you've done so far is not enough. And I hold
myself and Michelle and our kids to that same standard.''

On Thursday, during an invitation-only Zoom fund-raiser, Mr. Obama
expressed outrage at the president's use of ``kung flu'' and ``China
virus'' to describe the coronavirus. ``I don't want a country in which
the president of the United States is actively trying to promote
anti-Asian sentiment and thinks it's funny. I don't want that. That
still shocks and pisses me off,'' Mr. Obama said, according to a
transcript of his remarks provided by a participant in the event.

Mr. Obama speaks with the former vice president and top campaign aides
frequently, offering suggestions on staffing and messaging. Last month,
he bluntly counseled Mr. Biden to keep his speeches brief, interviews
crisp and slash the length of his tweets, the better to make the
campaign a referendum on Mr. Trump and the economy, according to
Democratic officials.

\hypertarget{latest-updates-2020-election}{%
\section{\texorpdfstring{\href{https://www.nytimes.com/2020/07/31/us/elections/biden-vs-trump.html?action=click\&pgtype=Article\&state=default\&region=MAIN_CONTENT_1\&context=storylines_live_updates}{Latest
Updates: 2020
Election}}{Latest Updates: 2020 Election}}\label{latest-updates-2020-election}}

Updated 2020-08-01T01:26:45.732Z

\begin{itemize}
\tightlist
\item
  \href{https://www.nytimes.com/2020/07/31/us/elections/biden-vs-trump.html?action=click\&pgtype=Article\&state=default\&region=MAIN_CONTENT_1\&context=storylines_live_updates\#link-29fdff45}{Kamala
  Harris, a top vice-presidential contender, confronts double
  standards.}
\item
  \href{https://www.nytimes.com/2020/07/31/us/elections/biden-vs-trump.html?action=click\&pgtype=Article\&state=default\&region=MAIN_CONTENT_1\&context=storylines_live_updates\#link-13ec3d9c}{Karen
  Bass and Susan Rice are rising on Biden's vice-presidential
  shortlist.}
\item
  \href{https://www.nytimes.com/2020/07/31/us/elections/biden-vs-trump.html?action=click\&pgtype=Article\&state=default\&region=MAIN_CONTENT_1\&context=storylines_live_updates\#link-49e9a016}{Trump
  says Russian bounties to kill U.S. troops `never took place.'}
\end{itemize}

\href{https://www.nytimes.com/2020/07/31/us/elections/biden-vs-trump.html?action=click\&pgtype=Article\&state=default\&region=MAIN_CONTENT_1\&context=storylines_live_updates}{See
more updates}

He has taken a particular interest in Mr. Biden's work-in-progress
digital operation, the officials said, enlisting powerful friends, like
the LinkedIn founder Reid Hoffman and the former Google chief executive
Eric Schmidt, to share their expertise, they said.

Yet he continues to slow-walk some requests, especially to headline more
fund-raisers. Some in Mr. Obama's camp suggest he wants to avoid
overshadowing the candidate --- which Mr. Biden's people aren't buying.

``By all means, overshadow us,'' one of them joked.

Image

A meeting at the White House shortly after Donald J. Trump won the
presidential election.Credit...Stephen Crowley/The New York Times

\hypertarget{obama-will-not-be-able-to-rest}{%
\subsection{`Obama Will Not Be Able to
Rest'}\label{obama-will-not-be-able-to-rest}}

From the moment Mr. Trump was elected, Mr. Obama adopted a minimalist
approach: He would critique his policy choices, not the man himself,
following the norm of civility observed by his predecessors, especially
George W. Bush.

But norms are not Mr. Trump's thing. He made it clear from the start
that he wanted to eradicate any trace of Mr. Obama's presence from the
West Wing. ``He had the worst taste,'' Mr. Trump told a visitor in early
2017, showing off his new curtains --- which were not terribly different
from Mr. Obama's, in the view of other people who tramped in and out of
the office during that chaotic period.

The cancellation was more pronounced when it came to policy. One former
White House official recalled Mr. Trump interrupting an early
presentation to make sure one staff proposal was not ``an Obama thing.''

During the transition, in what looks in hindsight like a preview of the
presidency, one Trump aide got the idea of printing out the detailed
checklist of Mr. Obama's campaign promises from the official White House
website to repurpose as a kind of hit list, according to two people
familiar with the effort.

``This is personal for Trump; it is all about President Obama and
demolishing his legacy. It's his obsession,'' said Omarosa Manigault
Newman, an ``Apprentice'' veteran and, until her abrupt departure, one
of the few black officials in Mr. Trump's West Wing. ``President Obama
will not be able to rest as long as Trump is breathing.''

When the two men met for a stilted postelection sit-down in November
2016, the president-elect was polite, so Mr. Obama took the opportunity
to advise him against going scorched-earth on Obamacare. ``Look, you can
take my name off of it; I don't care,'' he said, according to aides.

Mr. Trump nodded noncommittally.

As the transition dragged on, Mr. Obama became increasingly uneasy at
what he saw as the breezy indifference of the new president and his
inexperienced team. Many of them ignored the briefing binders his staff
had painstakingly produced at his direction, former Obama aides
recalled, and instead of focusing on policy or the workings of the West
Wing, they inquired about the quality of tacos in the basement mess or
where to find a good apartment.

As for Mr. Trump, he had ``no idea what he's doing,'' Mr. Obama told an
aide after their Oval Office encounter.

Jared Kushner, Mr. Trump's son-in-law and close adviser, made an equally
indelible impression. During a tour of the building he abruptly
inquired, ``So how many of these people are sticking around?''

The answer was none, his escort replied. (West Wing officials serve at
the president's pleasure, as Mr. Trump would amply illustrate in the
coming months.)

When the Kushner story was relayed to Mr. Obama, aides recalled, he
laughed and repeated it to friends, and even a few journalists, to
illustrate what the country was up against.

A White House spokesman did not deny the account, but suggested Mr.
Kushner might have been talking about security and maintenance personnel
rather than political appointees.

During other conversations with editors he respected, including David
Remnick of The New Yorker and Jeffrey Goldberg of the Atlantic, Mr.
Obama was more ruminative, according to people familiar with the
interactions. At times, he would float some version of this question:
Was there anything he could have done to blunt the Trump backlash?

Image

Mr. Obama delivered his farewell address at McCormick Place in Chicago
on Jan. 10, 2017.Credit...Doug Mills/The New York Times

Mr. Obama eventually came to the conclusion that it was a historic
inevitability, and told people around him the best he could do was ``set
a counterexample.''

Others thought he needed to do more. During the transition, Paulette
Aniskoff, a veteran West Wing aide, began assembling a political
organization of former advisers to help Mr. Obama defend his legacy, aid
other Democrats and plan for his deployment as a surrogate in the 2018
midterms.

He was open to the effort, but his eye was on the exits. ``I'll do what
you want me to do,'' he told Ms. Aniskoff's team, but mandated they
carefully screen out any appearances that would waste time or squander
political capital.

Mr. Obama was, then as now, so determined to avoid uttering the new
president's name that one aide jokingly suggested they refer to him as
``He-Who-Must-Not-Be-Named'' --- Harry Potter's archenemy, Lord
Voldemort.

Mr. Trump had no trouble naming names. In March 2017, he falsely accused
Mr. Obama of personally ordering the surveillance of his campaign
headquarters, tweeting, ``How low has President Obama gone to tapp my
phones during the very sacred election process. This is Nixon/Watergate.
Bad (or sick) guy!''

It was an inflection point of sorts. Mr. Obama told Ms. Aniskoff's team
he would call out his successor by name in the 2018 midterms. But not a
lot.

It was telling how Mr. Obama talked about Mr. Trump that fall: He
referred to him less as a person than as a kind of epidemiological
affliction on the body politic, spread by his Republican enablers.

``It did not start with Donald Trump --- he is a symptom, not the
cause,'' he said in his kickoff speech at the University of Illinois in
September 2018. The American political system, he added, was not
``healthy'' enough to form the ``antibodies'' to fight the contagion of
``racial nationalism.''

The pandemic has, if anything, made him more partial to the comparison.

The virus, he said during his appearance with Mr. Biden last week, ``is
a metaphor'' for so much else.

Image

A golf outing near Dundee, Scotland, after Mr. Obama left
office.Credit...Andrew Milligan/Press Association Images, via Getty
Images

\hypertarget{golf-going-better-than-my-book}{%
\subsection{Golf Going `Better Than My
Book'}\label{golf-going-better-than-my-book}}

Mr. Obama felt one of the best ways to safeguard his legacy was by
writing his book, which he envisioned as both a detailed chronicle of
his presidency and as a serious literary follow-up to his widely praised
1995 memoir, ``Dreams From My Father.''

In late 2016, Mr. Obama's agent, Bob Barnett, began negotiating a
package deal for Mr. Obama's memoir and Michelle Obama's autobiography.
Random House eventually won the bidding war with a record-shattering
\$65 million offer.

The process has been a gilded grind. One former White House official who
checked in with Mr. Obama in mid-2018 was told the project ``was like
doing homework.''

Another associate, who ran into the former president at an event last
year, remarked at how fit he looked. Mr. Obama replied, ``Let's just say
my golf game is going a lot better than my book.''

It was not especially easy for the former president to look on as his
wife's book, ``Becoming,'' was published in 2018 and quickly became an
international blockbuster.

``She had a ghostwriter,'' Mr. Obama told a friend who asked about his
wife's speedy work. ``I am writing every word myself, and that's why
it's taking longer.''

The book's timing remains among the touchiest of topics. Mr. Obama, a
deliberate writer prone to procrastination --- and lengthy digression
--- insisted that there be no set deadline, according to several people
familiar with the process.

In an interview shortly after Mr. Obama left office, one of his closest
advisers had predicted that the book would be out in mid-2019, before
the primary season began in earnest, an option preferred by many working
on the project.

But Mr. Obama did not finish and circulate a draft of between 600 and
800 pages until around New Year's, too late to publish before the
election, according to people familiar with the situation.

He is now seriously considering splitting the project into two volumes,
in the hope of getting some of it into print quickly after the election,
perhaps in time for the Christmas season, several people close to the
process said.

Mr. Obama's other big creative enterprise, a multimillion-dollar 2018
contract with Netflix to produce documentaries and scripted features
with his wife, has been a tonic, and quick work by comparison.

Mr. Obama got a kick out of screening dozens of potential projects and
offered specific suggestions --- scrawled onto the yellow legal pad he
used to write his book --- to directors and writers.
\href{https://www.nytimes.com/2019/04/30/business/media/obama-netflix-shows.html}{His
production firm}, Higher Ground Productions, is run out of a small
bungalow on a Hollywood studio lot once home to Charlie Chaplin's
company, and he spent a day kibbitzing with its small staff during a
visit in November.

One of the first efforts was ``Crip Camp,'' an award-winning documentary
about a summer camp in upstate New York, founded in the early 1970s,
that became a focal point of the disability rights movement.

Mr. Obama saw the project as a vehicle for his vision of grass-roots
political change, and provided feedback during the 18 months the movie
was in production.

``We saw footage that the filmmakers had just begun to cut together and
sent it to the president to look at,'' said Priya Swaminathan, co-head
of Higher Ground. ``He wanted to know how we could help the filmmakers
make this the best telling of the story and they were into the
collaboration. We watched many, many cuts together.''

Image

A speech delivered in Cleveland in 2018 on behalf of Richard Cordray, a
Democrat running for governor.Credit...Maddie McGarvey for The New York
Times

\hypertarget{a-tailor-made-moment}{%
\subsection{A `Tailor-Made' Moment}\label{a-tailor-made-moment}}

Part of what Mr. Obama finds so appealing about filmmaking is that it
allows him to control the narrative. In that respect, the 2020 campaign
has been a disorienting experience: His political career is supposed to
be over, yet he has a semi-starring role in a production he has not
written or directed.

Nowhere has that low-grade frustration been more apparent than in his
complicated relationship with Mr. Biden, who is concurrently covetous of
his support and fiercely determined to win on his own.

Mr. Obama was supportive of Mr. Biden, personally, from the start of the
campaign, but he promised Senator Bernie Sanders, in one of their early
chats, that his public profession of neutrality was genuine and that he
was not working secretly to elect his friend, according to a party
official familiar with the exchange.

Moreover, Mr. Obama has always been cleareyed about his friend's
vulnerabilities, urging Mr. Biden's aides to ensure that he
\href{https://www.nytimes.com/2019/08/16/us/politics/biden-obama-history.html}{not
``embarrass himself'' or ``damage his legacy,''} win or lose.

When a Democratic donor raised the issue of Mr. Biden's age late last
year --- he is 77 --- Mr. Obama acknowledged those concerns, saying, ``I
wasn't even 50 when I got elected, and that job took every ounce of
energy I had,'' according to the person.

Still, he is an enthusiastic supporter, and played a central role in
pushing Mr. Sanders to
\href{https://www.nytimes.com/2020/04/14/us/politics/obama-biden-democratic-primary.html}{``accelerate
the endgame''} that led to Mr. Biden's earlier-than-expected victory in
April. He spent the next few weeks tidying up a few messy political
loose ends, working to improve his chilly relationship with Senator
Elizabeth Warren, who irked him by criticizing his Wall Street speaking
fees as emblematic of the scourge of money in politics, calling it a
\href{https://www.masslive.com/opinion/2017/04/warren_is_right_about_speaking.html}{``snake
that slithers through Washington.''}

He has never seen Mr. Biden's campaign as a proxy war between himself
and Mr. Trump, his aides insist. But he is, nonetheless, tickled by the
lopsided metrics of their competition of late.

Mr. Obama monitors their respective polling numbers closely --- he gets
privately circulated data from the Democratic National Committee --- and
takes pride in the fact that he has millions more Twitter followers than
a president who relies on the platform far more than he does, people
close to him said.

The former president devours online news, scouring The New York Times,
The Washington Post and Atlantic sites on his iPad constantly, and keeps
to his White House night-owl hours, sending texts and story links to
friends between midnight and 2 a.m. Even during the pandemic he does not
sleep late, at least on weekdays, and is often on his Peloton bike by 8
a.m., sending off a new round of texts, often about the latest Trump
outrage.

Mr. Obama was already stepping up his criticism of Mr. Trump before Mr.
Floyd's killing in May. Ms. Aniskoff organized
\href{https://www.nytimes.com/2020/05/09/us/politics/obama-flynn-coronavirus-trump.html}{an
online meeting with 3,000 former administration officials}whose purpose,
in part, was to soft-launch his tougher line. (Democrats close to Mr.
Obama helpfully leaked the recording of his remarks.)

Yet the rising cries for racial justice have lent the 2020 campaign a
coherence for Mr. Obama, a politician most comfortable cloaking his
criticism of an opponent --- be it Mrs. Clinton or Mr. Trump --- in the
language of movement politics.

Image

Pins sold outside East High School in Cleveland, where Mr. Obama spoke
at the Democratic rally in 2018.Credit...Maddie McGarvey for The New
York Times

Mr. Obama's first reaction to the protests, people close to him said,
was anxiety --- that the spasms of rioting would spin out of control and
play into Mr. Trump's narrative of a lawless left.

But peaceful demonstrators took control, igniting a national movement
that challenged Mr. Trump without making him its focal point.

Soon after, in the middle of a strategy call with political aides and
policy experts at his foundation, an excited Mr. Obama pronounced that
``a tailor-made moment'' had arrived.

Mr. Obama has lately been in close contact with his first attorney
general, Eric H. Holder Jr., sharing his outrage over the way the
current attorney general, William P. Barr, personally inspected the
phalanx of federal law enforcement officers who tear-gassed
demonstrators to clear the path for Mr. Trump's walk to a photo op at a
historic church near the White House.

Mr. Holder has few qualms about calling Mr. Trump a racist in the former
president's presence. Mr. Obama has never contradicted him, but he
avoids the term, even in private, preferring a more indirect accusation
of ``racial demagoguery,'' according to several people close to both
men.

His response to the Floyd killing was less about hammering Mr. Trump
than about encouraging young people, who have been slow in embracing Mr.
Biden, to vote. When he chose to speak publicly,
\href{https://www.nytimes.com/2020/06/23/us/politics/obama-biden-fundraiser.html}{it
was to host an online forum highlighting a slate of policing reforms
that went nowhere in Congress in his second term}.

In that sense, the role he is most comfortable occupying is the job he
was once so over.

On June 4, an hour or so before
\href{https://www.nytimes.com/2020/06/04/us/floyd-memorial-funeral.html}{Mr.
Floyd's memorial service} in Minneapolis, the former president called
his brother, Philonise Floyd --- a reprise of the calls he made to
grieving families over his eight years in office.

``I want you to have hope. I want you to know you are not alone. I want
you to know that Michelle and I will do anything you want me to do,''
Mr. Obama said during the emotional 25-minute conversation, according to
the Rev. Al Sharpton, who was on the call. Two other people with
knowledge of the call confirmed its contents.

``That was the first time, I think, that the Floyd family really
experienced solace since he died,'' Mr. Sharpton said in an interview.

\hypertarget{our-2020-election-guide}{%
\section{Our 2020 Election Guide}\label{our-2020-election-guide}}

Updated July 31, 2020

\begin{itemize}
\item
  \begin{center}\rule{0.5\linewidth}{\linethickness}\end{center}

  \hypertarget{the-latest}{%
  \subsection{The Latest}\label{the-latest}}

  \begin{itemize}
  \tightlist
  \item
    President Trump's assault on the Postal Service is intersecting with
    his attacks on mail-in voting.
    \href{https://www.nytimes.com/2020/07/31/us/politics/trump-usps-mail-delays.html?action=click\&pgtype=Article\&state=default\&region=BELOW_MAIN_CONTENT\&context=storylines_guide}{Voting
    rights groups say it is a recipe for disaster.}
  \end{itemize}
\item
  \begin{center}\rule{0.5\linewidth}{\linethickness}\end{center}

  \hypertarget{bidens-vp-search}{%
  \subsection{Biden's V.P. Search}\label{bidens-vp-search}}

  \begin{itemize}
  \tightlist
  \item
    \href{https://www.nytimes.com/article/biden-vice-president-2020.html?action=click\&pgtype=Article\&state=default\&region=BELOW_MAIN_CONTENT\&context=storylines_guide}{Here
    are 13 women} who have been under consideration to be Joe Biden's
    running mate, and why each might be chosen --- and might not be.
  \end{itemize}
\item
  \begin{center}\rule{0.5\linewidth}{\linethickness}\end{center}

  \hypertarget{keep-up-with-our-coverage}{%
  \subsection{Keep Up With Our
  Coverage}\label{keep-up-with-our-coverage}}

  \begin{itemize}
  \tightlist
  \item
    Get an
    \href{https://www.nytimes.com/newsletters/politics?action=click\&pgtype=Article\&state=default\&region=BELOW_MAIN_CONTENT\&context=storylines_guide}{email}
    recapping the day's news
  \end{itemize}

  \begin{itemize}
  \tightlist
  \item
    Download our mobile app on
    \href{https://apps.apple.com/us/app/nytimes/id284862083?ls=1\&mat_click_id=5c79ae7455014fd1bd66b5610c05b8f2-20191112-16948\&referrer=mat_click_id\%3D5c79ae7455014fd1bd66b5610c05b8f2-20191112-16948\%26link_click_id\%3D722930677036718082}{iOS}
    and
    \href{http://a.localytics.com/android?id=com.nytimes.android\&referrer=utm_source\%3Dother_nyt_mobile_web\%26utm_medium\%3DWeb\%2520page\%26utm_term\%3DGeneral\%2520Mobile\%2520Page\%26utm_campaign\%3DNYT\%2520Mobile\%2520General\%2520Page}{Android}
    and turn on Breaking News and Politics alerts
  \end{itemize}
\end{itemize}

Advertisement

\protect\hyperlink{after-bottom}{Continue reading the main story}

\hypertarget{site-index}{%
\subsection{Site Index}\label{site-index}}

\hypertarget{site-information-navigation}{%
\subsection{Site Information
Navigation}\label{site-information-navigation}}

\begin{itemize}
\tightlist
\item
  \href{https://help.nytimes.com/hc/en-us/articles/115014792127-Copyright-notice}{©~2020~The
  New York Times Company}
\end{itemize}

\begin{itemize}
\tightlist
\item
  \href{https://www.nytco.com/}{NYTCo}
\item
  \href{https://help.nytimes.com/hc/en-us/articles/115015385887-Contact-Us}{Contact
  Us}
\item
  \href{https://www.nytco.com/careers/}{Work with us}
\item
  \href{https://nytmediakit.com/}{Advertise}
\item
  \href{http://www.tbrandstudio.com/}{T Brand Studio}
\item
  \href{https://www.nytimes.com/privacy/cookie-policy\#how-do-i-manage-trackers}{Your
  Ad Choices}
\item
  \href{https://www.nytimes.com/privacy}{Privacy}
\item
  \href{https://help.nytimes.com/hc/en-us/articles/115014893428-Terms-of-service}{Terms
  of Service}
\item
  \href{https://help.nytimes.com/hc/en-us/articles/115014893968-Terms-of-sale}{Terms
  of Sale}
\item
  \href{https://spiderbites.nytimes.com}{Site Map}
\item
  \href{https://help.nytimes.com/hc/en-us}{Help}
\item
  \href{https://www.nytimes.com/subscription?campaignId=37WXW}{Subscriptions}
\end{itemize}
