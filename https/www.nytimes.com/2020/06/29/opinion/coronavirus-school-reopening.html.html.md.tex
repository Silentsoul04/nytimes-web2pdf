Sections

SEARCH

\protect\hyperlink{site-content}{Skip to
content}\protect\hyperlink{site-index}{Skip to site index}

\href{https://myaccount.nytimes.com/auth/login?response_type=cookie\&client_id=vi}{}

\href{https://www.nytimes.com/section/todayspaper}{Today's Paper}

\href{/section/opinion}{Opinion}\textbar{}Remote School Is a Nightmare.
Few in Power Care.

\href{https://nyti.ms/2ZliJuv}{https://nyti.ms/2ZliJuv}

\begin{itemize}
\item
\item
\item
\item
\item
\item
\end{itemize}

\href{https://www.nytimes.com/news-event/coronavirus?action=click\&pgtype=Article\&state=default\&region=TOP_BANNER\&context=storylines_menu}{The
Coronavirus Outbreak}

\begin{itemize}
\tightlist
\item
  live\href{https://www.nytimes.com/2020/08/08/world/coronavirus-updates.html?action=click\&pgtype=Article\&state=default\&region=TOP_BANNER\&context=storylines_menu}{Latest
  Updates}
\item
  \href{https://www.nytimes.com/interactive/2020/us/coronavirus-us-cases.html?action=click\&pgtype=Article\&state=default\&region=TOP_BANNER\&context=storylines_menu}{Maps
  and Cases}
\item
  \href{https://www.nytimes.com/interactive/2020/science/coronavirus-vaccine-tracker.html?action=click\&pgtype=Article\&state=default\&region=TOP_BANNER\&context=storylines_menu}{Vaccine
  Tracker}
\item
  \href{https://www.nytimes.com/interactive/2020/world/coronavirus-tips-advice.html?action=click\&pgtype=Article\&state=default\&region=TOP_BANNER\&context=storylines_menu}{F.A.Q.}
\item
  \href{https://www.nytimes.com/live/2020/08/07/business/stock-market-today-coronavirus?action=click\&pgtype=Article\&state=default\&region=TOP_BANNER\&context=storylines_menu}{Markets
  \& Economy}
\end{itemize}

Advertisement

\protect\hyperlink{after-top}{Continue reading the main story}

\href{/section/opinion}{Opinion}

Supported by

\protect\hyperlink{after-sponsor}{Continue reading the main story}

\hypertarget{remote-school-is-a-nightmare-few-in-power-care}{%
\section{Remote School Is a Nightmare. Few in Power
Care.}\label{remote-school-is-a-nightmare-few-in-power-care}}

Government should treat the need to reopen schools as an emergency.

\href{https://www.nytimes.com/by/michelle-goldberg}{\includegraphics{https://static01.nyt.com/images/2018/04/02/opinion/michelle-goldberg/michelle-goldberg-thumbLarge.png}}

By \href{https://www.nytimes.com/by/michelle-goldberg}{Michelle
Goldberg}

Opinion Columnist

\begin{itemize}
\item
  June 29, 2020
\item
  \begin{itemize}
  \item
  \item
  \item
  \item
  \item
  \item
  \end{itemize}
\end{itemize}

\includegraphics{https://static01.nyt.com/images/2020/06/29/opinion/29goldberg1/merlin_171672429_1e894756-4a4e-4339-98a4-84eb5228698d-articleLarge.jpg?quality=75\&auto=webp\&disable=upscale}

Scott Stringer, the comptroller of New York City, has sons who are 7 and
8 years old. Over the last three months, like many parents, he's tried
to navigate what schools are optimistically calling ``remote learning''
while he and his wife both worked from home. It's been, he told me,
``one of the most challenging things I ever had to do in my life.''

So when he hears from parents desperate to understand what's happening
with schools in September, he empathizes. As in many other cities, if
New York public schools reopen, students will likely be in the classroom
only part-time. But no one knows if that means that students will attend
on alternate days, alternate weeks or --- Stringer's preference --- in
half-day shifts.

``Parents have no more information today about what schools will look
like in the fall than they did last March,'' he
\href{https://comptroller.nyc.gov/wp-content/uploads/2020/06/6.26.20-Letter-to-Mayor-de-Blasio-and-Chancellor-Carranza.pdf?utm_source=Media-All\&utm_campaign=5d9c0fee2b-EMAIL_CAMPAIGN_2017_05_31_COPY_01\&utm_medium=email\&utm_term=0_7cd514b03e-5d9c0fee2b-\&utm_source=Media-All\&utm_campaign=5d9c0fee2b-EMAIL_CAMPAIGN_2017_05_31_COPY_01\&utm_medium=email\&utm_term=0_7cd514b03e-5d9c0fee2b-154094945}{wrote
in a letter} to Mayor Bill de Blasio and the New York City schools
chancellor, Richard Carranza, last week.

With expanded unemployment benefits set to expire at the end of July,
many parents will have no choice but to return to work by September.
Even for parents who can work from home, home schooling is often a
crushing burden that's destroying careers,
\href{https://www.nytimes.com/2020/06/23/parenting/parental-burnout-coronavirus.html}{mental
health} and family relationships. And online school has had
\href{https://www.nytimes.com/2020/06/05/us/coronavirus-education-lost-learning.html}{dismal
results}, especially for poor, black and Hispanic students.

\includegraphics{https://static01.nyt.com/images/2020/06/19/autossell/op-boundaries-thumb/op-boundaries-thumb-videoSixteenByNineJumbo1600.jpg}

Yet the nightmarish withdrawal of the key social support underlying
modern parenthood is being presented as a fait accompli, rather than a
worst-case scenario that government is mobilizing to prevent. ``This
school system should be leading the country on figuring out how to bring
our kids back,'' said Stringer. ``And there's no creativity. There's no
energy behind it.''

This isn't just a New York City problem. At every level, government is
failing kids and parents during the pandemic.

The Centers for Disease Control and Prevention says that if schools
reopen, students' desks should be
\href{https://www.cdc.gov/coronavirus/2019-ncov/downloads/php/CDC-Activities-Initiatives-for-COVID-19-Response.pdf}{placed
six feet apart}, which means far fewer kids in most classrooms. But
there's been no crash program to find or build new classroom space, or
to hire more teachers.

Few seem to be exploring the possibility of outdoor classes where
weather allows. Experts I spoke to knew of no plans to scale up child
care for parents who will need it. Randi Weingarten, president of the
American Federation of Teachers, described school districts as
``immobilized'' by lack of funding.

Reopening schools is an excruciating challenge, but more could be done
to rise to it. ``There's a missed creative opportunity to use a
different teaching force,'' said Emily Oster, an economics professor at
Brown University and author of ``Expecting Better'' and ``Cribsheet.''

\hypertarget{the-coronavirus-outbreak}{%
\subsubsection{The Coronavirus
Outbreak}\label{the-coronavirus-outbreak}}

\hypertarget{back-to-school}{%
\paragraph{Back to School}\label{back-to-school}}

Updated Aug. 8, 2020

The latest highlights as the first students return to U.S. schools.

\begin{itemize}
\item
  \begin{itemize}
  \tightlist
  \item
    Health experts say New York State schools are
    \href{https://www.nytimes.com/2020/08/07/health/coronavirus-ny-schools-reopen.html?action=click\&pgtype=Article\&state=default\&region=MAIN_CONTENT_2\&context=storylines_keepup}{in
    a good position to reopen}, and Gov. Andrew M. Cuomo has
    \href{https://www.nytimes.com/2020/08/07/nyregion/cuomo-schools-reopening.html?action=click\&pgtype=Article\&state=default\&region=MAIN_CONTENT_2\&context=storylines_keepup}{cleared
    the way}.
  \item
    Many schools spent the summer focused on reopening classrooms. What
    if they had
    \href{https://www.nytimes.com/2020/08/07/us/remote-learning-fall-2020.html?action=click\&pgtype=Article\&state=default\&region=MAIN_CONTENT_2\&context=storylines_keepup}{focused
    on improving remote learning} instead?
  \item
    A mother in Germany describes how her family
    \href{https://www.nytimes.com/2020/08/07/parenting/germany-schools-reopening-children.html?action=click\&pgtype=Article\&state=default\&region=MAIN_CONTENT_2\&context=storylines_keepup}{coped
    with the anxiety and uncertainty} of going back to school there.
  \item
    A high school freshman tested positive after two days in class. A
    yearbook editor worries about access to sporting events. We spoke to
    students about
    \href{https://www.nytimes.com/2020/08/06/us/coronavirus-students.html?action=click\&pgtype=Article\&state=default\&region=MAIN_CONTENT_2\&context=storylines_keepup}{what
    school is like in the age of Covid-19.}
  \end{itemize}
\end{itemize}

She suggested hiring college-aged people --- who are
\href{http://pewresearch.org/fact-tank/2020/06/11/unemployment-rose-higher-in-three-months-of-covid-19-than-it-did-in-two-years-of-the-great-recession/}{disproportionately
unemployed} --- as something like camp counselors. Kids, kept in pods,
would attend schools for part of the day, then move to a space where
counselors could oversee online learning or recess.

``Those things cost money, but having a bunch of kids lose out on their
learning and having their parents not go to work also costs money,'' she
said.

There's some evidence that young kids don't transmit the coronavirus at
the same rate as adults. In countries where schools have reopened,
\href{https://www.wired.com/story/its-ridiculous-to-treat-schools-like-covid-hot-zones/}{few
outbreaks} have been traced to elementary schools.
\href{https://www.npr.org/2020/06/24/882316641/what-parents-can-learn-from-child-care-centers-that-stayed-open-during-lockdowns}{As
NPR reported}, there have been no reported clusters at the child care
centers that stayed open all over the country this spring to watch the
children of essential workers.

The American Academy of Pediatrics
\href{https://services.aap.org/en/pages/2019-novel-coronavirus-covid-19-infections/clinical-guidance/covid-19-planning-considerations-return-to-in-person-education-in-schools/}{strongly
recommends} that ``all policy considerations for the coming school year
should start with a goal of having students physically present in
school.'' Schools, it says, ``should weigh the benefits of strict
adherence to a six-feet spacing rule between students with the potential
downside if remote learning is the only alternative.''

But many teachers, understandably, aren't willing to jettison C.D.C.
guidelines. So if American kids, unlike those in most other developed
countries, continue to see their education derailed by the coronavirus,
the fault lies primarily with a federal government that puts out safety
standards but won't help schools meet them.

Weingarten tells me that if the Senate doesn't pass the HEROES Act, a
House bill that contains around \$100 billion in support for education,
she thinks many schools, including those in New York City, won't open at
all in September. To open safely, schools are going to need much more
money to buy protective equipment like gloves and masks, retrofit
buildings and hire more teachers and nurses.

Instead, the economic crisis is forcing budget cuts. ``What are states
going to do? What are localities going to do?'' she asks.

My kids go to elementary school in New York City, and I found
Weingarten's words gutting. But she thinks school districts need to
start leveling with parents about what we're facing, unless Republicans
in the Senate can somehow be moved to act.

``At least plan with people so that people can get their heads around
`This is what a school will look like,''' she said. ```This is what the
schedule will be. And if we don't get the money we're on remote.'''
Airlines got a bailout. Parents are on their own.

\emph{The Times is committed to publishing}
\href{https://www.nytimes.com/2019/01/31/opinion/letters/letters-to-editor-new-york-times-women.html}{\emph{a
diversity of letters}} \emph{to the editor. We'd like to hear what you
think about this or any of our articles. Here are some}
\href{https://help.nytimes.com/hc/en-us/articles/115014925288-How-to-submit-a-letter-to-the-editor}{\emph{tips}}\emph{.
And here's our email:}
\href{mailto:letters@nytimes.com}{\emph{letters@nytimes.com}}\emph{.}

\emph{Follow The New York Times Opinion section on}
\href{https://www.facebook.com/nytopinion}{\emph{Facebook}}\emph{,}
\href{http://twitter.com/NYTOpinion}{\emph{Twitter (@NYTopinion)}}
\emph{and}
\href{https://www.instagram.com/nytopinion/}{\emph{Instagram}}\emph{.}

Advertisement

\protect\hyperlink{after-bottom}{Continue reading the main story}

\hypertarget{site-index}{%
\subsection{Site Index}\label{site-index}}

\hypertarget{site-information-navigation}{%
\subsection{Site Information
Navigation}\label{site-information-navigation}}

\begin{itemize}
\tightlist
\item
  \href{https://help.nytimes.com/hc/en-us/articles/115014792127-Copyright-notice}{©~2020~The
  New York Times Company}
\end{itemize}

\begin{itemize}
\tightlist
\item
  \href{https://www.nytco.com/}{NYTCo}
\item
  \href{https://help.nytimes.com/hc/en-us/articles/115015385887-Contact-Us}{Contact
  Us}
\item
  \href{https://www.nytco.com/careers/}{Work with us}
\item
  \href{https://nytmediakit.com/}{Advertise}
\item
  \href{http://www.tbrandstudio.com/}{T Brand Studio}
\item
  \href{https://www.nytimes.com/privacy/cookie-policy\#how-do-i-manage-trackers}{Your
  Ad Choices}
\item
  \href{https://www.nytimes.com/privacy}{Privacy}
\item
  \href{https://help.nytimes.com/hc/en-us/articles/115014893428-Terms-of-service}{Terms
  of Service}
\item
  \href{https://help.nytimes.com/hc/en-us/articles/115014893968-Terms-of-sale}{Terms
  of Sale}
\item
  \href{https://spiderbites.nytimes.com}{Site Map}
\item
  \href{https://help.nytimes.com/hc/en-us}{Help}
\item
  \href{https://www.nytimes.com/subscription?campaignId=37WXW}{Subscriptions}
\end{itemize}
