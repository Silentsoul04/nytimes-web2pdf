Sections

SEARCH

\protect\hyperlink{site-content}{Skip to
content}\protect\hyperlink{site-index}{Skip to site index}

\href{https://www.nytimes.com/section/realestate}{Real Estate}

\href{https://myaccount.nytimes.com/auth/login?response_type=cookie\&client_id=vi}{}

\href{https://www.nytimes.com/section/todayspaper}{Today's Paper}

\href{/section/realestate}{Real Estate}\textbar{}Can Dog Walkers and
Nannies Come Into My Co-op Now?

\url{https://nyti.ms/2YJeBFS}

\begin{itemize}
\item
\item
\item
\item
\item
\item
\end{itemize}

\href{https://www.nytimes.com/spotlight/at-home?action=click\&pgtype=Article\&state=default\&region=TOP_BANNER\&context=at_home_menu}{At
Home}

\begin{itemize}
\tightlist
\item
  \href{https://www.nytimes.com/2020/07/28/books/time-for-a-literary-road-trip.html?action=click\&pgtype=Article\&state=default\&region=TOP_BANNER\&context=at_home_menu}{Take:
  A Literary Road Trip}
\item
  \href{https://www.nytimes.com/2020/07/29/magazine/bored-with-your-home-cooking-some-smoky-eggplant-will-fix-that.html?action=click\&pgtype=Article\&state=default\&region=TOP_BANNER\&context=at_home_menu}{Cook:
  Smoky Eggplant}
\item
  \href{https://www.nytimes.com/2020/07/27/travel/moose-michigan-isle-royale.html?action=click\&pgtype=Article\&state=default\&region=TOP_BANNER\&context=at_home_menu}{Look
  Out: For Moose}
\item
  \href{https://www.nytimes.com/interactive/2020/at-home/even-more-reporters-editors-diaries-lists-recommendations.html?action=click\&pgtype=Article\&state=default\&region=TOP_BANNER\&context=at_home_menu}{Explore:
  Reporters' Obsessions}
\end{itemize}

Advertisement

\protect\hyperlink{after-top}{Continue reading the main story}

Supported by

\protect\hyperlink{after-sponsor}{Continue reading the main story}

Ask Real Estate

\hypertarget{can-dog-walkers-and-nannies-come-into-my-co-op-now}{%
\section{Can Dog Walkers and Nannies Come Into My Co-op
Now?}\label{can-dog-walkers-and-nannies-come-into-my-co-op-now}}

New York City has begun the slow process of reopening, but that doesn't
mean a return to normal. The next phase will look quite different from
the city we knew before the shutdown.

\includegraphics{https://static01.nyt.com/images/2020/06/28/realestate/27Ask-illo/27Ask-illo-articleLarge.jpg?quality=75\&auto=webp\&disable=upscale}

\href{https://www.nytimes.com/by/ronda-kaysen}{\includegraphics{https://static01.nyt.com/images/2018/07/16/multimedia/author-ronda-kaysen/author-ronda-kaysen-thumbLarge-v2.png}}

By \href{https://www.nytimes.com/by/ronda-kaysen}{Ronda Kaysen}

\begin{itemize}
\item
  June 29, 2020
\item
  \begin{itemize}
  \item
  \item
  \item
  \item
  \item
  \item
  \end{itemize}
\end{itemize}

\textbf{Q: Even though New York is now in Phase 2 of its reopening
process, the board of my Manhattan co-op is still following strict rules
about vendors and visitors. All guests have to fill out a form providing
their name and address, and answer personal questions about the reason
for their visit, the state of their health, and even the mode of
transport they used. Housekeepers, dog walkers and nannies still can't
come in at all. Is the board allowed to be this invasive and
restrictive?}

\textbf{A:} New York City has begun the slow process of reopening, but
reopening does not mean a return to normal. This next phase will look
quite different from the city we knew before the shutdown.

The
\href{https://www.governor.ny.gov/sites/governor.ny.gov/files/atoms/files/realestate-masterguidance.pdf}{state's
guidelines} set minimum requirements for buildings, adding that owners
are ``free to provide additional precautions or increased
restrictions.'' The guidelines require buildings to screen all visitors
with a questionnaire that asks about Covid-19 exposure. So your co-op's
form sounds like one that is in line with current rules, even if it may
seem prying.

\begin{center}\rule{0.5\linewidth}{\linethickness}\end{center}

\begin{center}\rule{0.5\linewidth}{\linethickness}\end{center}

Both the state and the
\href{https://rebny.com/content/dam/rebny/Documents/PDF/Resources/CoronavirusResources/Guidelines_Residential\%20Buildings_FINAL.pdf}{Real
Estate Board of New York} recommend restricting nonessential visitors.
Since the recommendations do not clarify whether dog walkers,
housekeepers or nannies are essential, your building has wide latitude
there.

``Yes, things are beginning to open up, but we're living under certain
restrictions,'' said
\href{https://www.armstrongteasdale.com/phyllis-weisberg/}{Phyllis H.
Weisberg}, a real estate lawyer and partner in the New York City office
of the law firm Armstrong Teasdale. ``I've been speaking with lots of
boards and they're wrestling with these issues. They're not
black-and-white issues.''

Some buildings have been letting housekeepers in for weeks, while others
haven't loosened their restrictions at all. ``In most instances boards
have taken a more conservative approach,'' said
\href{https://gumleyhaft.com/about-gumley-haft/daniel-j-wollman-leads-gumleyhaft-property-management-nyc/}{Daniel
J. Wollman}, the chief executive of Gumley Haft, a property management
company, adding that boards can enact policies that are more stringent
than city or state rules.

Ask your board and managing agent to provide you with clearer guidance
and a better sense of the timeline. When do they plan to start letting
vendors back in, and under what conditions? You and your neighbors
should be able to start planning for that future date, so you can inform
your service providers of the new protocols. As people return to work,
they are going to need nannies and dog walkers again --- these services
become increasingly essential.

``Certainly as New York City opens up, you're going to have people going
out and going to work, the buildings will be more open to the outside
world,'' Ms. Weisberg said.

For weekly email updates on residential real estate news,
\href{http://www.nytimes.com/newsletters/realestate/}{sign up here}.
Follow us on Twitter:
\href{https://twitter.com/nytrealestate}{@nytrealestate}.

Advertisement

\protect\hyperlink{after-bottom}{Continue reading the main story}

\hypertarget{site-index}{%
\subsection{Site Index}\label{site-index}}

\hypertarget{site-information-navigation}{%
\subsection{Site Information
Navigation}\label{site-information-navigation}}

\begin{itemize}
\tightlist
\item
  \href{https://help.nytimes.com/hc/en-us/articles/115014792127-Copyright-notice}{©~2020~The
  New York Times Company}
\end{itemize}

\begin{itemize}
\tightlist
\item
  \href{https://www.nytco.com/}{NYTCo}
\item
  \href{https://help.nytimes.com/hc/en-us/articles/115015385887-Contact-Us}{Contact
  Us}
\item
  \href{https://www.nytco.com/careers/}{Work with us}
\item
  \href{https://nytmediakit.com/}{Advertise}
\item
  \href{http://www.tbrandstudio.com/}{T Brand Studio}
\item
  \href{https://www.nytimes.com/privacy/cookie-policy\#how-do-i-manage-trackers}{Your
  Ad Choices}
\item
  \href{https://www.nytimes.com/privacy}{Privacy}
\item
  \href{https://help.nytimes.com/hc/en-us/articles/115014893428-Terms-of-service}{Terms
  of Service}
\item
  \href{https://help.nytimes.com/hc/en-us/articles/115014893968-Terms-of-sale}{Terms
  of Sale}
\item
  \href{https://spiderbites.nytimes.com}{Site Map}
\item
  \href{https://help.nytimes.com/hc/en-us}{Help}
\item
  \href{https://www.nytimes.com/subscription?campaignId=37WXW}{Subscriptions}
\end{itemize}
