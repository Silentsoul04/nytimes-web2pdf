Sections

SEARCH

\protect\hyperlink{site-content}{Skip to
content}\protect\hyperlink{site-index}{Skip to site index}

\href{https://myaccount.nytimes.com/auth/login?response_type=cookie\&client_id=vi}{}

\href{https://www.nytimes.com/section/todayspaper}{Today's Paper}

Five Essential Documentaries, Recommended by David France

\href{https://nyti.ms/2NFLFIm}{https://nyti.ms/2NFLFIm}

\begin{itemize}
\item
\item
\item
\item
\item
\end{itemize}

\href{https://www.nytimes.com/spotlight/at-home?action=click\&pgtype=Article\&state=default\&region=TOP_BANNER\&context=at_home_menu}{At
Home}

\begin{itemize}
\tightlist
\item
  \href{https://www.nytimes.com/2020/08/04/arts/television/sam-jay-netflix-special.html?action=click\&pgtype=Article\&state=default\&region=TOP_BANNER\&context=at_home_menu}{Watch:
  Sam Jay}
\item
  \href{https://www.nytimes.com/interactive/2020/at-home/even-more-reporters-editors-diaries-lists-recommendations.html?action=click\&pgtype=Article\&state=default\&region=TOP_BANNER\&context=at_home_menu}{Peruse:
  Reporters' Google Docs}
\item
  \href{https://www.nytimes.com/2020/08/04/dining/colombian-empanadas-carlos-gaviria.html?action=click\&pgtype=Article\&state=default\&region=TOP_BANNER\&context=at_home_menu}{Make:
  Empanadas}
\item
  \href{https://www.nytimes.com/2020/08/06/arts/design/street-art-nyc-george-floyd.html?action=click\&pgtype=Article\&state=default\&region=TOP_BANNER\&context=at_home_menu}{Explore:
  N.Y.C. Street Art}
\end{itemize}

Advertisement

\protect\hyperlink{after-top}{Continue reading the main story}

Supported by

\protect\hyperlink{after-sponsor}{Continue reading the main story}

\hypertarget{five-essential-documentaries-recommended-by-david-france}{%
\section{Five Essential Documentaries, Recommended by David
France}\label{five-essential-documentaries-recommended-by-david-france}}

With his new movie, ``Welcome to Chechnya,'' premiering on HBO, the
filmmaker shares some of his favorites.

\includegraphics{https://static01.nyt.com/images/2020/06/29/t-magazine/art/david-france-slide-43PP/david-france-slide-43PP-articleLarge.jpg?quality=75\&auto=webp\&disable=upscale}

By \href{https://www.nytimes.com/by/max-berlinger}{Max Berlinger}

\begin{itemize}
\item
  June 29, 2020
\item
  \begin{itemize}
  \item
  \item
  \item
  \item
  \item
  \end{itemize}
\end{itemize}

The connective thread running through the journalist and documentarian
\href{https://www.nytimes.com/interactive/2020/04/13/t-magazine/act-up-aids.html}{David
France}'s work may be queer activism, and yet he doesn't see himself as
an activist. ``I didn't have what it took to be a leader through
difficult times, to find answers and bring people along with me,'' he
told me earlier this month during a call from his apartment in New
York's East Village. ``That was not my skill set.''

His strength, it turned out, was as an observer, someone who functioned
as a megaphone for those on the front line. France chronicled
\href{https://www.nytimes.com/interactive/2020/04/13/t-magazine/act-up-aids.html}{ACT
UP} and other groups that demanded more expansive medical research
during the escalating AIDS crisis of the '80s and '90s, first for
alternative queer publications and later for mainstream ones including
**** Newsweek and New York magazine, before moving into filmmaking in
2012 with
``\href{https://www.nytimes.com/watching/recommendations/watching-film-how-to-survive-a-plague}{How
to Survive a Plague},'' his account, told using archival footage, of the
protest-led battle against H.I.V. ``I've always been interested in
studying the people who are able to step up and launch transformative
activism from the outside,'' he says.

His latest film,
``\href{https://www.hbo.com/documentaries/welcome-to-chechnya/about}{Welcome
to Chechnya},'' premiering June 30 on HBO, builds on this theme. It
follows an underground group of activists who risk their lives to
provide sanctuary and safe passage for L.G.B.T.Q. citizens of the
Russian republic, where gay people are routinely tortured and ****
killed as part of a campaign to supposedly cleanse the nation's
bloodline --- violence that the government has largely shrugged off.
It's the conclusion to what France considers his trilogy, starting with
``How to Survive a Plague,'' which was nominated for the Academy Award
for best documentary, and his 2017 film
``\href{https://www.nytimes.com/watching/titles/the-death-and-life-of-marsha-p-johnson}{The
Death and Life of Marsha P. Johnson},'' about the mysterious passing of
the prominent black trans activist.

``Welcome to Chechnya'' uses a version of
\href{https://www.nytimes.com/2019/11/24/technology/tech-companies-deepfakes.html}{deepfake
technology} to disguise the faces and voices of France's subjects in
order to preserve both their anonymity and humanity --- a technique he
developed after workshopping various ideas (including a Snapchat-like
filter) and asking multiple Hollywood visual-effects experts for advice.
The digital process, a first of its kind, allows us to empathize with
the protagonists --- all of whom are running away from the stability of
family and home toward safety and the unknown --- while keeping their
identities shrouded. ``There's a very specific ask of the audience,''
France says. ``Witness this. Don't let Putin and {[}the Chechnyan leader
Ramzan{]} Kadyrov get away with this. There are still 70 countries
around the world where it's illegal to be gay, and eight where being so
is punishable by death.'' (In the film's chilling final moments, France
notes that the Trump administration has refused these L.G.B.T.Q.
Chechnyan refugees entry to America.)

France, who is currently at work on a film about the novel coronavirus,
shared with T his five essential documentaries, all of which deal with
humanity in its most harrowing and transcendent extremes.

\hypertarget{the-cave-2019}{%
\subsubsection{\texorpdfstring{\href{https://www.nytimes.com/2019/10/17/movies/the-cave-review.html}{`The
Cave' (2019)}}{`The Cave' (2019)}}\label{the-cave-2019}}

``The Syrian filmmaker Feras Fayyad is one of the most soulful
documentary filmmakers working today,'' says France about the director
of this movie, which profiles a female doctor,
\href{https://www.nytimes.com/2020/02/11/world/middleeast/her-dream-of-becoming-a-doctor-turned-into-a-nightmare-and-a-movie.html}{Amani
Ballour}, **** working in dire conditions throughout **** the Syrian
civil war in a makeshift medical facility that gives the film its name.
``It's shot inside an underground hospital targeted by Russian
airstrikes, and Fayyad **** finds generosity, love and beauty where it
is least expected,'' he says.

\begin{center}\rule{0.5\linewidth}{\linethickness}\end{center}

\hypertarget{5-broken-cameras-2012}{%
\subsubsection{\texorpdfstring{\href{https://www.nytimes.com/2012/05/30/movies/5-broken-cameras-shows-life-in-one-palestinian-village.html}{`5
Broken Cameras'
(2012)}}{`5 Broken Cameras' (2012)}}\label{5-broken-cameras-2012}}

``\href{https://www.nytimes.com/2012/01/23/world/middleeast/documentary-from-emad-burnats-camera-competes-at-sundance.html}{Emad
Burnat} is a West Bank farmer who bought a camera to capture his growing
family, but his gaze turned outward when Israelis destroyed his olive
trees to make room for a barrier wall,'' France says. The resulting film
documents the ensuing protests, reflecting upon how the personal and
political collide and **** giving a human face to the ongoing
Israeli-Palestinian conflict. ``Five times his cameras were shot or
smashed, but the documentary evidence nonetheless survived.''

\begin{center}\rule{0.5\linewidth}{\linethickness}\end{center}

\hypertarget{the-work-2017}{%
\subsubsection{\texorpdfstring{\href{https://www.nytimes.com/2017/10/26/movies/the-work-review.html}{`The
Work' (2017)}}{`The Work' (2017)}}\label{the-work-2017}}

``This film was shot inside Folsom Prison, but that's not its point,''
says France. ``It's about the very hard work we all need to undertake if
we hope to reconcile with our past, and with the difficult truths of the
world we inherited.'' In following the San Francisco-based ****
nonprofit group \href{https://insidecircle.org/about-us/}{Inside
Circle}, which provides therapy-like sessions to inmates --- the titular
``work'' --- this searing study, directed by Jairus McLeary, shows how
the prison system can look beyond punishment toward the true roots of
criminal activity. ``It's about how to find a way, together, to create a
better future,'' France adds.

\begin{center}\rule{0.5\linewidth}{\linethickness}\end{center}

\hypertarget{freedom-on-my-mind-1994}{%
\subsubsection{\texorpdfstring{\href{https://www.nytimes.com/1994/06/22/movies/review-film-freedom-on-my-mind-memories-of-a-hot-summer-long-ago.html}{`Freedom
on My Mind'
(1994)}}{`Freedom on My Mind' (1994)}}\label{freedom-on-my-mind-1994}}

As voters' rights are set to become even more of a flashpoint during
this year's elections, this documentary, directed by Marilyn Mulford and
Connie Field, about the 1964 push to sign up black voters in Mississippi
--- during what was known as Freedom Summer --- feels newly relevant.
``We should all go back and watch this Oscar-nominated film about the
brutal voter registration campaign in Mississippi and the dogged
activists who weathered that war,'' France says. ``These are the
shoulders we stand on as we take on the unfinished work of racial
justice in America.''

\begin{center}\rule{0.5\linewidth}{\linethickness}\end{center}

\hypertarget{honeyland-2019}{%
\subsubsection{\texorpdfstring{\href{https://www.nytimes.com/2019/07/25/movies/honeyland-review.html}{`Honeyland'
(2019)}}{`Honeyland' (2019)}}\label{honeyland-2019}}

``There is no more beautiful film about the need for finding harmony
with the natural world,'' France says of this 2019 Academy Award
contender, which doubles as a fable of sorts. Directed by
\href{https://www.nytimes.com/2020/01/24/movies/honeyland-oscars.html}{Tamara
Kotevska and Ljubomir Stefanov}, it tells the story of a Macedonian
beekeeper, Hatidze Muratova, focusing on the ways her routine --- and
nature's delicate balance --- is upended when new neighbors descend.
``Stunningly shot in isolated North Macedonia,'' says France, ``it
chronicles one woman who lives a life fit for a long-ago era when the
disaster of the modern world arrives.''

Advertisement

\protect\hyperlink{after-bottom}{Continue reading the main story}

\hypertarget{site-index}{%
\subsection{Site Index}\label{site-index}}

\hypertarget{site-information-navigation}{%
\subsection{Site Information
Navigation}\label{site-information-navigation}}

\begin{itemize}
\tightlist
\item
  \href{https://help.nytimes.com/hc/en-us/articles/115014792127-Copyright-notice}{©~2020~The
  New York Times Company}
\end{itemize}

\begin{itemize}
\tightlist
\item
  \href{https://www.nytco.com/}{NYTCo}
\item
  \href{https://help.nytimes.com/hc/en-us/articles/115015385887-Contact-Us}{Contact
  Us}
\item
  \href{https://www.nytco.com/careers/}{Work with us}
\item
  \href{https://nytmediakit.com/}{Advertise}
\item
  \href{http://www.tbrandstudio.com/}{T Brand Studio}
\item
  \href{https://www.nytimes.com/privacy/cookie-policy\#how-do-i-manage-trackers}{Your
  Ad Choices}
\item
  \href{https://www.nytimes.com/privacy}{Privacy}
\item
  \href{https://help.nytimes.com/hc/en-us/articles/115014893428-Terms-of-service}{Terms
  of Service}
\item
  \href{https://help.nytimes.com/hc/en-us/articles/115014893968-Terms-of-sale}{Terms
  of Sale}
\item
  \href{https://spiderbites.nytimes.com}{Site Map}
\item
  \href{https://help.nytimes.com/hc/en-us}{Help}
\item
  \href{https://www.nytimes.com/subscription?campaignId=37WXW}{Subscriptions}
\end{itemize}
