Sections

SEARCH

\protect\hyperlink{site-content}{Skip to
content}\protect\hyperlink{site-index}{Skip to site index}

\href{https://www.nytimes.com/section/politics}{Politics}

\href{https://myaccount.nytimes.com/auth/login?response_type=cookie\&client_id=vi}{}

\href{https://www.nytimes.com/section/todayspaper}{Today's Paper}

\href{/section/politics}{Politics}\textbar{}Trump Got Written Briefing
in February on Possible Russian Bounties, Officials Say

\url{https://nyti.ms/3i7QWX4}

\begin{itemize}
\item
\item
\item
\item
\item
\item
\end{itemize}

Advertisement

\protect\hyperlink{after-top}{Continue reading the main story}

Supported by

\protect\hyperlink{after-sponsor}{Continue reading the main story}

\hypertarget{trump-got-written-briefing-in-february-on-possible-russian-bounties-officials-say}{%
\section{Trump Got Written Briefing in February on Possible Russian
Bounties, Officials
Say}\label{trump-got-written-briefing-in-february-on-possible-russian-bounties-officials-say}}

The investigation into Russia's suspected operation is said to focus in
part on the killings of three Marines in a truck bombing last year,
officials said.

\includegraphics{https://static01.nyt.com/images/2020/06/29/us/politics/29dc-intel3/29dc-intel3-articleLarge.jpg?quality=75\&auto=webp\&disable=upscale}

\href{https://www.nytimes.com/by/charlie-savage}{\includegraphics{https://static01.nyt.com/images/2018/06/12/multimedia/author-charlie-savage/author-charlie-savage-thumbLarge-v2.png}}\href{https://www.nytimes.com/by/eric-schmitt}{\includegraphics{https://static01.nyt.com/images/2018/06/12/multimedia/author-eric-schmitt/author-eric-schmitt-thumbLarge-v2.png}}\href{https://www.nytimes.com/by/nicholas-fandos}{\includegraphics{https://static01.nyt.com/images/2018/11/06/multimedia/author-nicholas-fandos/author-nicholas-fandos-thumbLarge-v2.png}}\href{https://www.nytimes.com/by/adam-goldman}{\includegraphics{https://static01.nyt.com/images/2018/07/12/multimedia/author-adam-goldman/author-adam-goldman-thumbLarge.png}}

By \href{https://www.nytimes.com/by/charlie-savage}{Charlie Savage},
\href{https://www.nytimes.com/by/eric-schmitt}{Eric Schmitt},
\href{https://www.nytimes.com/by/nicholas-fandos}{Nicholas Fandos} and
\href{https://www.nytimes.com/by/adam-goldman}{Adam Goldman}

\begin{itemize}
\item
  Published June 29, 2020Updated June 30, 2020
\item
  \begin{itemize}
  \item
  \item
  \item
  \item
  \item
  \item
  \end{itemize}
\end{itemize}

American officials provided a written briefing in late February to
President Trump laying out their conclusion that a
\href{https://www.nytimes.com/2020/06/30/us/politics/russian-bounties-afghanistan-intelligence.html}{Russian}
military intelligence unit offered and
\href{https://www.nytimes.com/2020/06/30/us/politics/russian-bounties-afghanistan-intelligence.html}{paid
bounties to Taliban-linked militants to kill U.S. and coalition troops
in Afghanistan}, two officials familiar with the matter said.

The investigation into the suspected
\href{https://www.nytimes.com/2020/07/01/us/politics/trump-putin-russia-taliban-bounty.html}{Russian}
covert operation to incentivize such killings has focused in part on an
April 2019 car bombing that killed three Marines as one such potential
attack, according to multiple officials familiar with the matter.

The new information emerged as the White House tried on Monday to play
down the intelligence assessment that Russia sought to encourage and
reward killings --- including reiterating a claim that Mr. Trump was
never briefed about the matter and portraying the conclusion as disputed
and dubious.

But that stance clashed with the disclosure by two officials that the
intelligence was included months ago in Mr. Trump's President's Daily
Brief document --- a compilation of the government's latest secrets and
best insights about foreign policy and national security that is
prepared for him to read. One of the officials said the item appeared in
Mr. Trump's brief in late February; the other cited Feb. 27,
specifically.

Moreover, a description of the intelligence assessment that the Russian
unit had carried out the bounties plot was also seen as serious and
solid enough to disseminate more broadly across the intelligence
community in a May 4 article in the C.I.A.'s World Intelligence Review,
a classified compendium commonly referred to as The Wire, two officials
said.

A National Security Council spokesman declined to comment on any
connection between the Marines' deaths and the suspected Russian plot.
The White House press secretary, Kayleigh McEnany, did not answer when
pressed
\href{https://www.whitehouse.gov/briefings-statements/press-briefing-press-secretary-kayleigh-mcenany-062920/}{by
reporters on Monday} whether the intelligence was included in the
written President's Daily Brief, and the National Security Council
spokesman pointed to her comments when asked later about the February
written briefing.

Late Monday, John Ratcliffe, the recently confirmed director of national
intelligence, issued a statement warning that leaks about the matter
were a crime.

``We are still investigating the alleged intelligence referenced in
recent media reporting, and we will brief the president and
congressional leaders at the appropriate time,'' he said. ``This is the
analytic process working the way it should. Unfortunately, unauthorized
disclosures now jeopardize our ability to ever find out the full story
with respect to these allegations.''

The disclosures came amid a growing furor in Washington over the
revelations in recent days that the Trump administration had known for
months about the intelligence conclusion but the White House had
authorized no response to Russia.

\includegraphics{https://static01.nyt.com/images/2020/06/29/us/politics/29dc-intel2/29dc-intel2-articleLarge.jpg?quality=75\&auto=webp\&disable=upscale}

Top Democrats in the House and Senate demanded that all members of
Congress be briefed, and the White House summoned a small group of House
Republicans friendly to the president to begin explaining its position.

The lawmakers emerged saying that they were told the administration was
reviewing reporting about the suspected Russian plot to assess its
credibility. They also said the underlying intelligence was conflicting,
echoing comments from Ms. McEnany that the information in the assessment
had not been ``verified'' because, she said without detail, there were
``dissenting opinions'' among analysts or agencies.

``There was not a consensus among the intelligence community,'' Ms.
McEnany said. ``And, in fact, there were dissenting opinions within the
intelligence community, and it would not be elevated to the president
until it was verified.''

Later Monday, Robert C. O'Brien, Mr. Trump's national security adviser,
\href{https://twitter.com/WHNSC/status/1277804151033004032?s=20}{echoed
her insistence} that the reports were unsubstantiated.

But in denying that Mr. Trump was briefed, administration officials have
been coy about how it is defining that concept and whether it includes
both oral briefings and the President's Daily Brief. ``He was not
personally briefed on the matter,'' Ms. McEnany told reporters when
asked specifically about the written briefing. ``That is all I can share
with you today.''

Mr. Trump is said to often neglect reading that document, preferring
instead to receive an oral briefing summarizing highlights every few
days. Even in those face-to-face meetings, he is
\href{https://www.nytimes.com/2020/05/21/us/politics/presidents-daily-brief-trump.html}{particularly
difficult to brief} on national security matters. He often relies
instead on conservative media and friends for information, current and
former intelligence officials have said.

American intelligence officers and Special Operations forces in
Afghanistan began raising alarms as early as January, and the National
Security Council convened an interagency meeting to discuss the problem
and what to do about it in late March, The New York Times has previously
reported. But despite being presented with options, including a
diplomatic protest and sanctions, the White House authorized no
response.

The administration's explanations on Monday, in public and in private,
appeared to be an attempt to placate lawmakers, particularly Mr. Trump's
fellow Republicans, alarmed by news reports in recent days revealing the
existence of the intelligence assessment and Mr. Trump's insistence he
had not been warned of the suspected Russian plot.

The assessments pointing to a Russian scheme to offer bounties to
Taliban-linked militants and criminals were based on information
collected in raids and interrogations on the ground in Afghanistan,
where American military commanders came to believe Russia was behind the
plot, as well as more sensitive and unspecified intelligence that came
in over time, an American official said.

Officials said there was disagreement among intelligence officials about
the strength of the evidence about the suspected Russian plot and the
evidence linking the attack on the Marines to the suspected Russian
plot, but they did not detail those disputes.

Notably, the National Security Agency, which specializes in hacking and
electronic surveillance, has been more skeptical about interrogations
and other human intelligence, officials said.

Typically, the president is formally briefed when the information has
been vetted and seen as sufficiently credible and important by the
intelligence professionals. Such information would most likely be
included in the President's Daily Brief.

Former officials said that in previous administrations, accusations of
such profound importance --- even if the evidence was not fully
established --- were conveyed to the president. ``We had two threshold
questions: `Does the president need to know this?' and `Why does he need
to know it now?''' said Robert Cardillo, a former senior intelligence
official who briefed President Barack Obama from 2010 to 2014.

David Priess, a former C.I.A. daily intelligence briefer and the author
of ``The President's Book of Secrets: The Untold Story of Intelligence
Briefings to America's Presidents,'' said: ``Many intelligence judgments
in history have not had the consensus of every analyst who worked on it.
That's the nature of intelligence. It's inherently dealing with
uncertainty.''

Both Mr. Cardillo and Mr. Priess said previous presidents received
assessments on issues of potentially vital importance even if they had
dissents from some analysts or agencies. The dissents, they said, were
highlighted for the president to help them understand uncertainties and
the analytic process.

Lawmakers demanded to see the underlying material for themselves.

``This is a time to focus on the two things Congress should be asking
and looking at: No. 1, who knew what, when, and did the commander in
chief know? And if not, how the hell not?'' said Senator Ben Sasse,
Republican of Nebraska and a member of the Senate Intelligence
Committee.

Image

Speaker Nancy Pelosi requested that all members of the House be briefed
on intelligence about a suspected Russian plot against U.S.
troops.Credit...Al Drago for The New York Times

Speaker Nancy Pelosi and Senator Chuck Schumer of New York, the
Democratic leaders of the House and Senate, each requested that all
lawmakers be briefed on the matter and for C.I.A. and other intelligence
officials to explain how Mr. Trump was informed of intelligence
collected about the plot.

The White House began explaining its position directly to lawmakers in a
carefully controlled setting. Mark Meadows, the White House chief of
staff; Mr. Ratcliffe, the director of national intelligence; and Mr.
O'Brien briefed a handful of invited House Republicans. A group of House
Democrats was scheduled to go to the White House on Tuesday morning to
receive a similar briefing.

There was no indication after the session with Republicans whether they
had been told that the information was included in Mr. Trump's written
briefing four months ago. But afterward, two of the Republicans ---
Representatives Liz Cheney of Wyoming and Mac Thornberry of Texas ---
said that they ``remain concerned about Russian activity in Afghanistan,
including reports that they have targeted U.S. forces'' and would need
additional briefings.

``It has been clear for some time that Russia does not wish us well in
Afghanistan,'' they said in a joint statement. ``We believe it is
important to vigorously pursue any information related to Russia or any
other country targeting our forces.''

Other Republicans who attended the briefing were more sanguine. In an
interview, Representative Chris Stewart of Utah said he saw nothing
unusual about the purported decision not to orally inform Mr. Trump,
particularly when the situation did not require the president to take
immediate action.

``It just didn't reach the level of credibility to bring it to the
president's attention,'' he said, adding that military and intelligence
agencies should continue to scrutinize Russia's activities.

The Associated Press
\href{http://apnews.com/a59124b8eb95f6245286ddefe3dd0ffd}{first
reported} that the intelligence community was examining the deaths of
the three Marine reservists:
\href{https://www.nytimes.com/2019/04/09/nyregion/fdny-firefighter-killed-afghanistan.html}{Staff
Sgt. Christopher Slutman}, 43, of Newark, Del.; Cpl. Robert A. Hendriks,
25, of Locust Valley, N.Y.; and Sgt. Benjamin S. Hines, 31, of York, Pa.

They were killed near Bagram Air Base when a vehicle laden with
explosives hit their truck, wounding an Afghan contractor as well. The
huge blast set fire to the truck, engulfing those inside in flames,
while their fellow Marines tried to extricate them, a defense official
said. A brief firefight ensued.

Gen. Zaman Mamozai, the former police chief of Parwan Province, where
Bagram Airfield is, said that the Taliban there hire freelancers from
local criminal networks, often blurring the lines of who carried out
what attacks. He said the Taliban's commanders were only based in two
districts of the province, Seyagird and Shinwari, and from there they
coordinate a more extensive network that largely commissions the
services of criminals.

The Taliban have denied involvement. And a spokesman for President
Vladimir V. Putin of Russia, Dmitry Peskov,
\href{https://www.nbcnews.com/news/us-news/trump-says-no-credible-intel-russia-offered-taliban-bounty-payments-n1232376}{told
NBC News on Monday} that reports of the Russian scheme were incorrect.
He said that ``none of the American representatives have ever raised
this question'' with their Russian counterparts through government or
diplomatic channels.

The Pentagon's chief spokesman, Jonathan Hoffman, declined to comment on
any connection between the Marines' deaths and the suspected Russian
plot. He also declined to say whether or when Defense Secretary Mark T.
Esper was briefed on the intelligence assessment and whether the deaths
of American troops in Afghanistan resulted from the Russian bounties.
But later Monday, Mr. Hoffman issued a statement saying that the Defense
Department was monitoring intelligence on the matter and that it ``has
no corroborating evidence to validate the recent allegations found in
open-source reports.''

Col. DeDe Halfhill, a spokeswoman for Gen. Mark A. Milley, the chairman
of the Joint Chiefs of Staff, also declined to comment on the same
questions.

Reporting was contributed by Thomas Gibbons-Neff, Fahim Abed, Annie
Karni and Emily Cochrane.

Advertisement

\protect\hyperlink{after-bottom}{Continue reading the main story}

\hypertarget{site-index}{%
\subsection{Site Index}\label{site-index}}

\hypertarget{site-information-navigation}{%
\subsection{Site Information
Navigation}\label{site-information-navigation}}

\begin{itemize}
\tightlist
\item
  \href{https://help.nytimes.com/hc/en-us/articles/115014792127-Copyright-notice}{©~2020~The
  New York Times Company}
\end{itemize}

\begin{itemize}
\tightlist
\item
  \href{https://www.nytco.com/}{NYTCo}
\item
  \href{https://help.nytimes.com/hc/en-us/articles/115015385887-Contact-Us}{Contact
  Us}
\item
  \href{https://www.nytco.com/careers/}{Work with us}
\item
  \href{https://nytmediakit.com/}{Advertise}
\item
  \href{http://www.tbrandstudio.com/}{T Brand Studio}
\item
  \href{https://www.nytimes.com/privacy/cookie-policy\#how-do-i-manage-trackers}{Your
  Ad Choices}
\item
  \href{https://www.nytimes.com/privacy}{Privacy}
\item
  \href{https://help.nytimes.com/hc/en-us/articles/115014893428-Terms-of-service}{Terms
  of Service}
\item
  \href{https://help.nytimes.com/hc/en-us/articles/115014893968-Terms-of-sale}{Terms
  of Sale}
\item
  \href{https://spiderbites.nytimes.com}{Site Map}
\item
  \href{https://help.nytimes.com/hc/en-us}{Help}
\item
  \href{https://www.nytimes.com/subscription?campaignId=37WXW}{Subscriptions}
\end{itemize}
