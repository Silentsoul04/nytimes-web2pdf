Sections

SEARCH

\protect\hyperlink{site-content}{Skip to
content}\protect\hyperlink{site-index}{Skip to site index}

\href{https://www.nytimes.com/section/world/africa}{Africa}

\href{https://myaccount.nytimes.com/auth/login?response_type=cookie\&client_id=vi}{}

\href{https://www.nytimes.com/section/todayspaper}{Today's Paper}

\href{/section/world/africa}{Africa}\textbar{}Coronavirus Is Battering
Africa's Growing Middle Class

\url{https://nyti.ms/389dpP3}

\begin{itemize}
\item
\item
\item
\item
\item
\item
\end{itemize}

\href{https://www.nytimes.com/news-event/coronavirus?action=click\&pgtype=Article\&state=default\&region=TOP_BANNER\&context=storylines_menu}{The
Coronavirus Outbreak}

\begin{itemize}
\tightlist
\item
  live\href{https://www.nytimes.com/2020/08/01/world/coronavirus-covid-19.html?action=click\&pgtype=Article\&state=default\&region=TOP_BANNER\&context=storylines_menu}{Latest
  Updates}
\item
  \href{https://www.nytimes.com/interactive/2020/us/coronavirus-us-cases.html?action=click\&pgtype=Article\&state=default\&region=TOP_BANNER\&context=storylines_menu}{Maps
  and Cases}
\item
  \href{https://www.nytimes.com/interactive/2020/science/coronavirus-vaccine-tracker.html?action=click\&pgtype=Article\&state=default\&region=TOP_BANNER\&context=storylines_menu}{Vaccine
  Tracker}
\item
  \href{https://www.nytimes.com/interactive/2020/07/29/us/schools-reopening-coronavirus.html?action=click\&pgtype=Article\&state=default\&region=TOP_BANNER\&context=storylines_menu}{What
  School May Look Like}
\item
  \href{https://www.nytimes.com/live/2020/07/31/business/stock-market-today-coronavirus?action=click\&pgtype=Article\&state=default\&region=TOP_BANNER\&context=storylines_menu}{Economy}
\end{itemize}

Advertisement

\protect\hyperlink{after-top}{Continue reading the main story}

Supported by

\protect\hyperlink{after-sponsor}{Continue reading the main story}

\hypertarget{coronavirus-is-battering-africas-growing-middle-class}{%
\section{Coronavirus Is Battering Africa's Growing Middle
Class}\label{coronavirus-is-battering-africas-growing-middle-class}}

From Kenya to Nigeria, South Africa to Rwanda, the pandemic is
decimating the livelihoods of the once-stable workers who were helping
to drive Africa's economic expansion.

\includegraphics{https://static01.nyt.com/images/2020/06/24/world/00africa-middleclass-top/merlin_173815191_cd3c7eab-0121-42c9-86cc-2cb792d14882-articleLarge.jpg?quality=75\&auto=webp\&disable=upscale}

By \href{https://www.nytimes.com/by/abdi-latif-dahir}{Abdi Latif Dahir}

\begin{itemize}
\item
  Published June 29, 2020Updated July 2, 2020
\item
  \begin{itemize}
  \item
  \item
  \item
  \item
  \item
  \item
  \end{itemize}
\end{itemize}

NAIROBI, Kenya --- James Gichina started out 15 years ago as a driver
shuttling travelers from the airport, worked his way up to safari guide,
and with the help of some bank loans, bought two minivans of his own to
ferry vacationers around.

His clients were, as he is, members of Africa's growing middle class ---
bankers from Nigeria, tech entrepreneurs from South Africa, and fellow
Kenyans who could finally afford trips to enjoy their own country's
beaches and wildlife preserves.

But when the coronavirus pandemic cratered the tourist industry and the
economy, Mr. Gichina removed the seats from his minibus and started
using it to hawk eggs and vegetables. With what he now earns, he said,
he can barely afford to pay rent, buy food or send his 9-year-old son to
school.

``We have been working hard to build better lives,'' Mr. Gichina, 35,
said of his colleagues in the tourist sector. Now, he said, ``We have
nothing.''

\includegraphics{https://static01.nyt.com/images/2020/06/24/world/00africa-middleclass-Ginchina/merlin_173815242_51b4a4aa-9c7d-4fbf-b7ec-0999b759e961-articleLarge.jpg?quality=75\&auto=webp\&disable=upscale}

As the
\href{https://www.nytimes.com/2020/06/16/world/africa/coronavirus-africa.html}{coronavirus
surges in many countries in Africa}, it is threatening to push as many
as 58 million people in the region into extreme poverty,
\href{http://documents1.worldbank.org/curated/en/607011589560173878/pdf/How-Much-Will-Poverty-Rise-in-Sub-Saharan-Africa-in-2020.pdf}{experts
at the World Bank say}. But beyond the devastating consequences for the
continent's most vulnerable people, the pandemic is also whittling away
at one of Africa's signature achievements: the growth of its middle
class.

For the last decade, Africa's middle class has been pivotal to the
educational, political and economic development across the continent.
New business owners and entrepreneurs have created jobs that, in turn,
gave others a leg up as well.

Educated, tech-savvy families and young people with money to spare have
fed the demand for consumer goods, called for
\href{https://afrobarometer.org/sites/default/files/publications/Working\%20paper/Afropaperno150.pdf}{democratic
reforms}, expanded the talent pool at all levels of society, and pushed
for high-quality schools and health care.

About 170 million out of Africa's 1.3 billion people are now classified
as middle class. But about eight million of them could be thrust into
poverty because of the coronavirus and its economic fallout, according
to World Data Lab, a research organization.

It's a setback that may be felt for years to come.

``The tragedy is that because Africa is not growing fast, this collapse
of the middle class could take several years to recover,'' said Homi
Kharas, a senior fellow at the Brookings Institution and the co-founder
of the World Data Lab.

Africa's middle class
\href{https://www.afdb.org/fr/news-and-events/africas-middle-class-triples-to-more-than-310m-over-past-30-years-due-to-economic-growth-and-rising-job-culture-reports-afdb-7986}{tripled
over the past 30 years}, by some estimates, spurred by job opportunities
in sectors like technology, tourism and manufacturing. But now that the
region is facing its
\href{https://www.worldbank.org/en/news/video/2020/04/21/africas-pulse-covid-19-coronavirus-causes-first-recession-in-25-years}{first
recession in 25 years}, millions of educated people living in urban
centers could fall victim to the extreme income inequality that has
defined Africa for decades.

\hypertarget{latest-updates-global-coronavirus-outbreak}{%
\section{\texorpdfstring{\href{https://www.nytimes.com/2020/08/01/world/coronavirus-covid-19.html?action=click\&pgtype=Article\&state=default\&region=MAIN_CONTENT_1\&context=storylines_live_updates}{Latest
Updates: Global Coronavirus
Outbreak}}{Latest Updates: Global Coronavirus Outbreak}}\label{latest-updates-global-coronavirus-outbreak}}

Updated 2020-08-02T05:48:45.291Z

\begin{itemize}
\tightlist
\item
  \href{https://www.nytimes.com/2020/08/01/world/coronavirus-covid-19.html?action=click\&pgtype=Article\&state=default\&region=MAIN_CONTENT_1\&context=storylines_live_updates\#link-34047410}{The
  U.S. reels as July cases more than double the total of any other
  month.}
\item
  \href{https://www.nytimes.com/2020/08/01/world/coronavirus-covid-19.html?action=click\&pgtype=Article\&state=default\&region=MAIN_CONTENT_1\&context=storylines_live_updates\#link-780ec966}{Top
  U.S. officials work to break an impasse over the federal jobless
  benefit.}
\item
  \href{https://www.nytimes.com/2020/08/01/world/coronavirus-covid-19.html?action=click\&pgtype=Article\&state=default\&region=MAIN_CONTENT_1\&context=storylines_live_updates\#link-25930521}{Thousands
  in Berlin protest Germany's coronavirus measures.}
\end{itemize}

\href{https://www.nytimes.com/2020/08/01/world/coronavirus-covid-19.html?action=click\&pgtype=Article\&state=default\&region=MAIN_CONTENT_1\&context=storylines_live_updates}{See
more updates}

More live coverage:
\href{https://www.nytimes.com/live/2020/07/31/business/stock-market-today-coronavirus?action=click\&pgtype=Article\&state=default\&region=MAIN_CONTENT_1\&context=storylines_live_updates}{Markets}

The rising middle class has been ``critical for the future prospects of
African economies as they stimulate long-term growth, social progress,
an inclusive and prosperous society and effective and accountable
governance,'' said Landry Signé, author of ``Unlocking Africa's Business
Potential.'' The coronavirus ``will drastically delay wages and hold
back the dreams of Africa's middle class,'' he said.

Image

The Junction mall in Nairobi.Credit...Khadija Farah for The New York
Times

Governments across
Africa\href{https://www.nytimes.com/2020/03/27/world/africa/south-africa-coronavirus.html}{responded
differently} to the coronavirus, but Kenya was among those that closed
borders, imposed curfews and restricted movement between counties. In
Nairobi, the capital, malls were once touted as a symbol of a rising
middle class. Now their owners are furloughing employees, shuttering
stores and desperately trying to survive the crisis.

When Kenya first announced lockdown restrictions in March, there was
almost no foot traffic at the Junction mall, where Nairobi's middle
class had once gravitated to dine and shop in more than 100 stores.

Eastleigh, a bustling area with dozens of malls, hotels, lodges and
banks, was also put under a total lockdown in early May after a jump in
reported coronavirus cases.

Maryan Bashir, who owns three stores in Eastleigh that sell mattresses
and curtains, said traders like her were already worried about whether
they could still get supplies from China as the pandemic began to affect
imports. But the lockdown left them reeling from lack of customers.

It also cut employment. Out of 12 of her co-workers, only three lived
within the locked-down area and could report to work.

The authorities lifted the curfew from Eastleigh in early June, but Ms.
Bashir said it will be a long time before shop owners like her are able
to make the same profits they made before the pandemic.

``The landlords are still asking for rent,'' she said, ``but if we are
not earning anything, how do we even pay?''

Image

Lockdown restrictions were lifted in the Eastleigh neighborhood of
Nairobi earlier this month. Residents are still struggling to make a
living.Credit...Khadija Farah for The New York Times

The economic fallout of the Covid-19 outbreak is also being felt among
the middle class in Nigeria, Africa's largest economy. Hit by
\href{https://www.nytimes.com/2020/04/22/world/middleeast/oil-price-collapse-coronavirus.html}{low
oil revenues in the pandemic}, the West African nation faces increasing
unemployment rates and a recession that could last until 2021, according
to the International Monetary Fund.

As demand for goods and services crashed, small businesses and
entrepreneurs dependent on cash flow found themselves increasingly in
dire straits.

Biola Kazeem established his sports marketing company, Elev8 Sports
Entertainment, six years ago, marrying his passion for sports with his
college degree in communication. But as
\href{https://www.nytimes.com/article/coronavirus-sports-leagues-returning-canceled.html}{sports
leagues worldwide canceled or postponed events}, Mr. Kazeem said he lost
70 percent of his business and had to put half of his 11-member staff on
leave without pay.

Despite facing financial challenges in the early years, ``nothing
absolutely prepared us for this,'' Mr. Kazeem said in a phone interview
from Lagos.

Image

An empty street market in April in Lagos, Nigeria's largest city. The
government eased the lockdown in Lagos on May 4 but the West African
nation faces increasing unemployment rates and a recession that could
last until 2021.Credit...Yagazie Emezi for The New York Times

In Zimbabwe, which has been
\href{https://www.nytimes.com/2017/03/04/world/africa/zimbabwe-economy-work-force.html}{in
economic free fall} for years, the pandemic and the ensuing restrictions
are threatening the solvency of those who have built a bridge into the
middle class.

For years, Madeline Chiveso's restaurant in downtown Harare, Zimbabwe,
served professionals such as bankers, journalists and engineers flocking
to work. But as infections rose and the restrictions tightened, there
were no customers to serve. She was forced to close the restaurant.

She used to make \$350 a day, and now makes nothing. She is using her
savings to pay bills, she said, jeopardizing her dream of one day owning
her own home.

``The future looks surely uncertain because nobody knows how this would
end,'' said Ms. Chiveso, who is 46 and a single mother of two daughters,
both in college.

Mr. Kharas, of the World Data Lab, defined the middle class in Africa as
households that spend anywhere between \$11 and \$110 per capita per
day.

\href{https://www.nytimes.com/news-event/coronavirus?action=click\&pgtype=Article\&state=default\&region=MAIN_CONTENT_3\&context=storylines_faq}{}

\hypertarget{the-coronavirus-outbreak-}{%
\subsubsection{The Coronavirus Outbreak
›}\label{the-coronavirus-outbreak-}}

\hypertarget{frequently-asked-questions}{%
\paragraph{Frequently Asked
Questions}\label{frequently-asked-questions}}

Updated July 27, 2020

\begin{itemize}
\item ~
  \hypertarget{should-i-refinance-my-mortgage}{%
  \paragraph{Should I refinance my
  mortgage?}\label{should-i-refinance-my-mortgage}}

  \begin{itemize}
  \tightlist
  \item
    \href{https://www.nytimes.com/article/coronavirus-money-unemployment.html?action=click\&pgtype=Article\&state=default\&region=MAIN_CONTENT_3\&context=storylines_faq}{It
    could be a good idea,} because mortgage rates have
    \href{https://www.nytimes.com/2020/07/16/business/mortgage-rates-below-3-percent.html?action=click\&pgtype=Article\&state=default\&region=MAIN_CONTENT_3\&context=storylines_faq}{never
    been lower.} Refinancing requests have pushed mortgage applications
    to some of the highest levels since 2008, so be prepared to get in
    line. But defaults are also up, so if you're thinking about buying a
    home, be aware that some lenders have tightened their standards.
  \end{itemize}
\item ~
  \hypertarget{what-is-school-going-to-look-like-in-september}{%
  \paragraph{What is school going to look like in
  September?}\label{what-is-school-going-to-look-like-in-september}}

  \begin{itemize}
  \tightlist
  \item
    It is unlikely that many schools will return to a normal schedule
    this fall, requiring the grind of
    \href{https://www.nytimes.com/2020/06/05/us/coronavirus-education-lost-learning.html?action=click\&pgtype=Article\&state=default\&region=MAIN_CONTENT_3\&context=storylines_faq}{online
    learning},
    \href{https://www.nytimes.com/2020/05/29/us/coronavirus-child-care-centers.html?action=click\&pgtype=Article\&state=default\&region=MAIN_CONTENT_3\&context=storylines_faq}{makeshift
    child care} and
    \href{https://www.nytimes.com/2020/06/03/business/economy/coronavirus-working-women.html?action=click\&pgtype=Article\&state=default\&region=MAIN_CONTENT_3\&context=storylines_faq}{stunted
    workdays} to continue. California's two largest public school
    districts --- Los Angeles and San Diego --- said on July 13, that
    \href{https://www.nytimes.com/2020/07/13/us/lausd-san-diego-school-reopening.html?action=click\&pgtype=Article\&state=default\&region=MAIN_CONTENT_3\&context=storylines_faq}{instruction
    will be remote-only in the fall}, citing concerns that surging
    coronavirus infections in their areas pose too dire a risk for
    students and teachers. Together, the two districts enroll some
    825,000 students. They are the largest in the country so far to
    abandon plans for even a partial physical return to classrooms when
    they reopen in August. For other districts, the solution won't be an
    all-or-nothing approach.
    \href{https://bioethics.jhu.edu/research-and-outreach/projects/eschool-initiative/school-policy-tracker/}{Many
    systems}, including the nation's largest, New York City, are
    devising
    \href{https://www.nytimes.com/2020/06/26/us/coronavirus-schools-reopen-fall.html?action=click\&pgtype=Article\&state=default\&region=MAIN_CONTENT_3\&context=storylines_faq}{hybrid
    plans} that involve spending some days in classrooms and other days
    online. There's no national policy on this yet, so check with your
    municipal school system regularly to see what is happening in your
    community.
  \end{itemize}
\item ~
  \hypertarget{is-the-coronavirus-airborne}{%
  \paragraph{Is the coronavirus
  airborne?}\label{is-the-coronavirus-airborne}}

  \begin{itemize}
  \tightlist
  \item
    The coronavirus
    \href{https://www.nytimes.com/2020/07/04/health/239-experts-with-one-big-claim-the-coronavirus-is-airborne.html?action=click\&pgtype=Article\&state=default\&region=MAIN_CONTENT_3\&context=storylines_faq}{can
    stay aloft for hours in tiny droplets in stagnant air}, infecting
    people as they inhale, mounting scientific evidence suggests. This
    risk is highest in crowded indoor spaces with poor ventilation, and
    may help explain super-spreading events reported in meatpacking
    plants, churches and restaurants.
    \href{https://www.nytimes.com/2020/07/06/health/coronavirus-airborne-aerosols.html?action=click\&pgtype=Article\&state=default\&region=MAIN_CONTENT_3\&context=storylines_faq}{It's
    unclear how often the virus is spread} via these tiny droplets, or
    aerosols, compared with larger droplets that are expelled when a
    sick person coughs or sneezes, or transmitted through contact with
    contaminated surfaces, said Linsey Marr, an aerosol expert at
    Virginia Tech. Aerosols are released even when a person without
    symptoms exhales, talks or sings, according to Dr. Marr and more
    than 200 other experts, who
    \href{https://academic.oup.com/cid/article/doi/10.1093/cid/ciaa939/5867798}{have
    outlined the evidence in an open letter to the World Health
    Organization}.
  \end{itemize}
\item ~
  \hypertarget{what-are-the-symptoms-of-coronavirus}{%
  \paragraph{What are the symptoms of
  coronavirus?}\label{what-are-the-symptoms-of-coronavirus}}

  \begin{itemize}
  \tightlist
  \item
    Common symptoms
    \href{https://www.nytimes.com/article/symptoms-coronavirus.html?action=click\&pgtype=Article\&state=default\&region=MAIN_CONTENT_3\&context=storylines_faq}{include
    fever, a dry cough, fatigue and difficulty breathing or shortness of
    breath.} Some of these symptoms overlap with those of the flu,
    making detection difficult, but runny noses and stuffy sinuses are
    less common.
    \href{https://www.nytimes.com/2020/04/27/health/coronavirus-symptoms-cdc.html?action=click\&pgtype=Article\&state=default\&region=MAIN_CONTENT_3\&context=storylines_faq}{The
    C.D.C. has also} added chills, muscle pain, sore throat, headache
    and a new loss of the sense of taste or smell as symptoms to look
    out for. Most people fall ill five to seven days after exposure, but
    symptoms may appear in as few as two days or as many as 14 days.
  \end{itemize}
\item ~
  \hypertarget{does-asymptomatic-transmission-of-covid-19-happen}{%
  \paragraph{Does asymptomatic transmission of Covid-19
  happen?}\label{does-asymptomatic-transmission-of-covid-19-happen}}

  \begin{itemize}
  \tightlist
  \item
    So far, the evidence seems to show it does. A widely cited
    \href{https://www.nature.com/articles/s41591-020-0869-5}{paper}
    published in April suggests that people are most infectious about
    two days before the onset of coronavirus symptoms and estimated that
    44 percent of new infections were a result of transmission from
    people who were not yet showing symptoms. Recently, a top expert at
    the World Health Organization stated that transmission of the
    coronavirus by people who did not have symptoms was ``very rare,''
    \href{https://www.nytimes.com/2020/06/09/world/coronavirus-updates.html?action=click\&pgtype=Article\&state=default\&region=MAIN_CONTENT_3\&context=storylines_faq\#link-1f302e21}{but
    she later walked back that statement.}
  \end{itemize}
\end{itemize}

What distinguishes the middle class from the poor, said Razia Khan, the
chief economist for Africa and the Middle East at Standard Chartered
bank, is the ability to earn a steady income. But because of the
pandemic, many more people across Africa are at risk of being ``knocked
back into poverty'' because of lack of jobs, unemployment benefits or
any social safety net, she said.

The pandemic is also posing a threat to nascent industries supported by
governments in Africa in recent years to boost the number of
middle-income earners.

Rwanda, which announced aspirations to become a middle-income nation by
2035,
\href{https://www.nytimes.com/2017/10/12/world/africa/east-africa-rwanda-used-clothing.html}{supported
the local textile and fashion industries} to limit imports of used
clothing from the United States, and boost manufacturing.

Matthew Rugamba, 30, created House of Tayo in 2011, building it into one
of the leading brands in
\href{https://www.nytimes.com/2018/04/04/travel/kigali-rwanda-fashion-designers.html}{Rwanda's
burgeoning fashion scene}. Mr. Rugamba gained enough notice for his
designs to be worn in Hollywood,
\href{https://www.instagram.com/p/BevpsRVjGWH/?utm_source=ig_embed}{at
the premiere of the movie ``Black Panther.''}

Image

Matthew Rugamba's House of Tayo is one of the leading brands in Rwanda's
burgeoning fashion scene.Credit...Chris Schwagga

But as Rwanda enforced one of the toughest lockdowns in Africa, Mr.
Rugamba's store shut its doors, only to open several weeks later to
almost no customers. Even though he's pivoted to making masks and
introduced a delivery service, business has not been the same.

``We were at a point where people value the work that we do,'' Mr.
Rugamba said. But with the pandemic, he said, ``you go through periods
where you are worried that this is something that I have invested nine
years of my life in, and is it going to be there tomorrow?''

More governments are offering financial support and tax breaks to
businesses, and urging proprietors to hold onto their employees even if
they reduce production or services, said Mr. Kharas, of the World Data
Lab.

Economists like Ms. Khan said that emerging markets in Africa, no
strangers to economic shocks, have proven resilient in the past, and
could come out stronger when the pandemic is over.

But that hope is likely a long way off for Mr. Gichina, the safari guide
now selling eggs to survive. He works for the tour company Bonfire
Adventures, which was founded in 2008 by an entrepreneur named Simon
Kabu, specifically to serve Africa's growing middle class.

Once a milk delivery guy and conductor on the matatu minibuses used for
transit in Kenya, Mr. Kabu grew up in Kenya's central highlands to a
mother who was a farmer and a father who was a retired civil servant.

Image

Simon Kabu, the CEO of Bonfire Adventures, at the tour company's now
empty office in Nairobi.Credit...Khadija Farah for The New York Times

But by starting a business that served the ever-growing travel needs of
the middle class, he grew Bonfire Adventures into an award-winning tours
company with 10 offices, 200 permanent staff and 300 drivers and guides.

The coronavirus has gutted all that, pushing Mr. Kabu, 45, to lay off
his employees en masse. The only staff members currently working, he
said, are accountants who are processing refunds for customers unable to
travel.

Mr. Gichina hopes business will resume soon --- especially as he dreads
losing out on the peak wildebeest migration starting in late June, which
usually draws tourists from across the world.

``The banks are pressuring us a lot,'' he said of the urgency to settle
his loans. ``They are saying you have to pay,'' but, he asked, ``where
should we get the money?''

Lynsey Chutel contributed reporting from Johannesburg, South Africa and
Jeffrey Moyo from Harare, Zimbabwe.

Advertisement

\protect\hyperlink{after-bottom}{Continue reading the main story}

\hypertarget{site-index}{%
\subsection{Site Index}\label{site-index}}

\hypertarget{site-information-navigation}{%
\subsection{Site Information
Navigation}\label{site-information-navigation}}

\begin{itemize}
\tightlist
\item
  \href{https://help.nytimes.com/hc/en-us/articles/115014792127-Copyright-notice}{©~2020~The
  New York Times Company}
\end{itemize}

\begin{itemize}
\tightlist
\item
  \href{https://www.nytco.com/}{NYTCo}
\item
  \href{https://help.nytimes.com/hc/en-us/articles/115015385887-Contact-Us}{Contact
  Us}
\item
  \href{https://www.nytco.com/careers/}{Work with us}
\item
  \href{https://nytmediakit.com/}{Advertise}
\item
  \href{http://www.tbrandstudio.com/}{T Brand Studio}
\item
  \href{https://www.nytimes.com/privacy/cookie-policy\#how-do-i-manage-trackers}{Your
  Ad Choices}
\item
  \href{https://www.nytimes.com/privacy}{Privacy}
\item
  \href{https://help.nytimes.com/hc/en-us/articles/115014893428-Terms-of-service}{Terms
  of Service}
\item
  \href{https://help.nytimes.com/hc/en-us/articles/115014893968-Terms-of-sale}{Terms
  of Sale}
\item
  \href{https://spiderbites.nytimes.com}{Site Map}
\item
  \href{https://help.nytimes.com/hc/en-us}{Help}
\item
  \href{https://www.nytimes.com/subscription?campaignId=37WXW}{Subscriptions}
\end{itemize}
