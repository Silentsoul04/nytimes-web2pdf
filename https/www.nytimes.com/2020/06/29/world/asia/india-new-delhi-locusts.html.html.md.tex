Sections

SEARCH

\protect\hyperlink{site-content}{Skip to
content}\protect\hyperlink{site-index}{Skip to site index}

\href{https://www.nytimes.com/section/world/asia}{Asia Pacific}

\href{https://myaccount.nytimes.com/auth/login?response_type=cookie\&client_id=vi}{}

\href{https://www.nytimes.com/section/todayspaper}{Today's Paper}

\href{/section/world/asia}{Asia Pacific}\textbar{}India's Capital Faces
a `Swarmageddon' of Locusts

\url{https://nyti.ms/2COwN8h}

\begin{itemize}
\item
\item
\item
\item
\item
\end{itemize}

Advertisement

\protect\hyperlink{after-top}{Continue reading the main story}

Supported by

\protect\hyperlink{after-sponsor}{Continue reading the main story}

\hypertarget{indias-capital-faces-a-swarmageddon-of-locusts}{%
\section{India's Capital Faces a `Swarmageddon' of
Locusts}\label{indias-capital-faces-a-swarmageddon-of-locusts}}

The country is weathering its worst invasion in more than two decades.
With the monsoon season looming, scientists worry that the worst is yet
to come.

\includegraphics{https://static01.nyt.com/images/2020/06/29/world/29india-locusts-top/merlin_173977668_8bf74943-087f-41ba-ad77-d3ce5d91645e-articleLarge.jpg?quality=75\&auto=webp\&disable=upscale}

\href{https://www.nytimes.com/by/kai-schultz}{\includegraphics{https://static01.nyt.com/images/2019/11/22/reader-center/author-kai-schultz/author-kai-schultz-thumbLarge.png}}\href{https://www.nytimes.com/by/hari-kumar}{\includegraphics{https://static01.nyt.com/images/2019/12/13/reader-center/author-hari-kumar/author-hari-kumar-thumbLarge.png}}

By \href{https://www.nytimes.com/by/kai-schultz}{Kai Schultz} and
\href{https://www.nytimes.com/by/hari-kumar}{Hari Kumar}

\begin{itemize}
\item
  June 29, 2020
\item
  \begin{itemize}
  \item
  \item
  \item
  \item
  \item
  \end{itemize}
\end{itemize}

NEW DELHI --- A miles-long cloud of locusts
\href{https://www.ndtv.com/gurgaon-news/swarms-of-crop-destroying-desert-locusts-reach-gurugram-2253126}{swarmed
India's capital region} over the weekend, flying through metro stations
and playgrounds, invading sugar cane fields and threatening major losses
to the agriculture sector at a time when coronavirus restrictions have
already caused the loss of millions of jobs.

Indian officials have struggled for weeks to contain the country's worst
locust invasion in decades, as the insects have moved from western
regions to the New Delhi area
\href{https://myrepublica.nagariknetwork.com/news/little-damage-to-crops-by-locusts-in-nepal-ministry/}{and
farther east to Nepal} despite efforts to douse crops with pesticides
and
\href{https://www.hindustantimes.com/india-news/in-midnight-operation-jaipur-officials-use-drone-to-kill-locusts/story-SUa82z8JgkCDd3mViICtUI.html}{kill
swarms using drones}. More than a half-dozen Indian states have been
affected.

In a year punctuated by cyclones, heat waves, surging coronavirus
infections and overwhelmed hospitals, scientists warn that the locusts
could push agrarian parts of India to the brink of disaster, severely
disrupting food supplies and slashing earnings for millions of
struggling farmers.

Hari Chand Sharma, a prominent Indian entomologist and agriculture
scientist, said the number of locusts in the country could top a
trillion if the spread were not checked. He blamed foreign nations for
not doing more to stop the insects from traversing large parts of
Africa, Asia and the Middle East this year.

``There were hardly any containment measures taken at all,'' he said,
noting that India had weathered around 20 locust swarms this year, about
10 times the average. He said that coronavirus restrictions may have
played a role in the inaction, but that governments had been lax in the
past, too.

India's locust problem began when millions of the insects flew in from
\href{https://www.nytimes.com/2020/06/29/world/asia/pakistan-stock-exchange-shooting.html}{Pakistan}
and Iran a few months ago. Scientists say unusually warm water in the
Indian Ocean triggered heavy rains over East Africa and the Arabian
Peninsula, creating ideal conditions for desert locusts and leading to
serious infestations in countries like Kenya and Somalia.

\includegraphics{https://static01.nyt.com/images/2020/06/29/world/29india-locusts-3/merlin_173429814_928c112a-c112-47cf-b0a4-843ba43c95d0-articleLarge.jpg?quality=75\&auto=webp\&disable=upscale}

In India, after hungry swarms crossed into the state of Rajasthan,
officials mounted pesticide sprayers on hundreds of tractors in an
effort to save farms. In a single day, a modestly sized swarm can eat as
much food as 35,000 people and travel more than 100 miles.

Strong winds in recent weeks have blown the locusts farther into India,
dispersing the insects across the northern plains. For several hours on
Saturday, thick swarms darkened the skies over the outskirts of New
Delhi and in Gurugram, a neighboring city.

Local officials placed the capital region under ``high alert.''
Residents set off fireworks, banged kitchen utensils and blasted music
from their balconies
\href{https://www.hindustantimes.com/delhi-news/after-locust-swarms-seen-in-delhi-neem-leaves-and-firecrackers-to-chase-pests/story-Rz9CzRaWscDWXbFWUC0EBM.html}{to
chase away the locusts}. The Times of India, a leading newspaper, called
the attack
``\href{https://timesofindia.indiatimes.com/india/swarmageddon-looks-like-this-locusts-fly-into-ncr/articleshow/76668130.cms}{swarmageddon}.''

``The sky was almost invisible,'' said Madhusudan Satija, who was
outside his apartment building in Gurugram when the swarm passed. ``It
was so terrible. They were sticking to the building wall like a thick
layer of wet mud.''

Initial damage assessments were modest. Officials in the state of
Haryana, which includes Gurugram, said
\href{https://www.tribuneindia.com/news/haryana/5-km-long-locust-swarm-haryana-on-alert-105220}{only
a few thousand acres of crops had been damaged}.

But as a new planting season nears, farmers worry that more than 200
million acres of rice, sugar cane, cotton and soybeans could be
decimated. In parts of Rajasthan, more than 60 percent of crops have
been damaged, and a government relief package has covered only a small
fraction of farmers,
\href{https://thewire.in/agriculture/rajasthan-locust-attack-crop-damage-relief}{according
to local news outlets}.

The coronavirus has complicated efforts to stop the locusts. With
confirmed infections topping 500,000 nationwide and many cities still
under partial lockdowns, officials have strained to keep supply chains
open and enforce locust containment measures across state borders.

Opposition politicians have seized on the lapses and accused Prime
Minister Narendra Modi's government of negligence.

Image

Locusts in Bhopal, in~Madhya Pradesh, on June 14.Credit...Sanjeev
Gupta/EPA, via Shutterstock

Randeep Surjewala, a spokesman for the opposition Indian National
Congress party,
\href{https://www.news18.com/news/politics/congress-asks-govt-to-declare-locust-attacks-as-natural-disaster-seeks-aid-for-farmers-2691735.html}{said}
insurance companies were refusing to compensate farmers for their losses
because the central government had not classified locust swarms as
natural disasters.

Mukhtar Abbas Naqvi, a minister in Mr. Modi's cabinet, shot back over
the weekend, chastising the Congress party as using the locust crisis to
``\href{https://www.newindianexpress.com/nation/2020/jun/28/country-tormented-by-locusts-and-losers-union-minister-mukhtar-abbas-naqvi-s-jibe-at-congress-2162588.html}{turn
disaster into anarchy}.''

``In the time of a calamity, there is torment by locusts and losers, and
both should be dealt with strongly,'' he said.

With the monsoon season looming, Dr. Sharma, the entomologist, said the
next few months would be critical. As heavy rains nourish the soil and
the locusts begin to breed, he said, officials will need to move
aggressively against the insects. Even then, their efforts may not be
enough.

``They practically feed on anything,'' Dr. Sharma said. ``First they eat
leaves, and then fruiting bodies like maize, seeds, parts of legumes,
flowers, young fruits. If they still persist, they will damage future
crops of lentils in Rajasthan, and in Haryana and Uttar Pradesh.''

``No crops are left standing when swarms attack,'' he added.

Advertisement

\protect\hyperlink{after-bottom}{Continue reading the main story}

\hypertarget{site-index}{%
\subsection{Site Index}\label{site-index}}

\hypertarget{site-information-navigation}{%
\subsection{Site Information
Navigation}\label{site-information-navigation}}

\begin{itemize}
\tightlist
\item
  \href{https://help.nytimes.com/hc/en-us/articles/115014792127-Copyright-notice}{©~2020~The
  New York Times Company}
\end{itemize}

\begin{itemize}
\tightlist
\item
  \href{https://www.nytco.com/}{NYTCo}
\item
  \href{https://help.nytimes.com/hc/en-us/articles/115015385887-Contact-Us}{Contact
  Us}
\item
  \href{https://www.nytco.com/careers/}{Work with us}
\item
  \href{https://nytmediakit.com/}{Advertise}
\item
  \href{http://www.tbrandstudio.com/}{T Brand Studio}
\item
  \href{https://www.nytimes.com/privacy/cookie-policy\#how-do-i-manage-trackers}{Your
  Ad Choices}
\item
  \href{https://www.nytimes.com/privacy}{Privacy}
\item
  \href{https://help.nytimes.com/hc/en-us/articles/115014893428-Terms-of-service}{Terms
  of Service}
\item
  \href{https://help.nytimes.com/hc/en-us/articles/115014893968-Terms-of-sale}{Terms
  of Sale}
\item
  \href{https://spiderbites.nytimes.com}{Site Map}
\item
  \href{https://help.nytimes.com/hc/en-us}{Help}
\item
  \href{https://www.nytimes.com/subscription?campaignId=37WXW}{Subscriptions}
\end{itemize}
