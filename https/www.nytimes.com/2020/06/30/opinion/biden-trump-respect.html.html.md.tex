Sections

SEARCH

\protect\hyperlink{site-content}{Skip to
content}\protect\hyperlink{site-index}{Skip to site index}

\href{https://myaccount.nytimes.com/auth/login?response_type=cookie\&client_id=vi}{}

\href{https://www.nytimes.com/section/todayspaper}{Today's Paper}

\href{/section/opinion}{Opinion}\textbar{}This Should Be Biden's Bumper
Sticker

\href{https://nyti.ms/2BgFrfB}{https://nyti.ms/2BgFrfB}

\begin{itemize}
\item
\item
\item
\item
\item
\item
\end{itemize}

Advertisement

\protect\hyperlink{after-top}{Continue reading the main story}

\href{/section/opinion}{Opinion}

Supported by

\protect\hyperlink{after-sponsor}{Continue reading the main story}

\hypertarget{this-should-be-bidens-bumper-sticker}{%
\section{This Should Be Biden's Bumper
Sticker}\label{this-should-be-bidens-bumper-sticker}}

He will need a simple, clear message to counter Trump's ``Make America
Great Again'' trope.

\href{https://www.nytimes.com/by/thomas-l-friedman}{\includegraphics{https://static01.nyt.com/images/2018/04/02/opinion/thomas-l-friedman/thomas-l-friedman-thumbLarge.png}}

By \href{https://www.nytimes.com/by/thomas-l-friedman}{Thomas L.
Friedman}

Opinion Columnist

\begin{itemize}
\item
  June 30, 2020
\item
  \begin{itemize}
  \item
  \item
  \item
  \item
  \item
  \item
  \end{itemize}
\end{itemize}

\includegraphics{https://static01.nyt.com/images/2020/07/01/opinion/30friedman1/merlin_173237091_a3dd496d-db1f-40b9-b27a-979c2eb362d7-articleLarge.jpg?quality=75\&auto=webp\&disable=upscale}

I almost --- but not quite --- feel sorry for Donald Trump. He's at war
with two ``invisible enemies'' at once --- the coronavirus and Joe Biden
--- and both remain highly elusive, the pathogen by nature and the
politician by design.

Biden, who made a rare public appearance on Tuesday, has been wise to
stay out of sight. Trump is now in a full-on race to the bottom with
himself, pushing uglier and uglier positions that appeal to smaller and
smaller segments of the American public. Why get in his way?

Of course, eventually Biden will debate the incumbent and will need a
simple, clear message to counter Trump's tired ``Make America Great
Again'' trope.

I have an idea for Biden's bumper sticker.

As I think about what kind of president Biden wants to be and what kind
of president America needs him to be, the slogan that comes to mind was
suggested to me by the environmental innovator Hal Harvey. Harvey didn't
know he was suggesting it; he just happened to sign off a recent email
to me by writing: ``Respect science, respect nature, respect each
other.''

I thought --- wow, that's a perfect message for Biden, and for all of
us. It summarizes so simply the most important values Americans feel
that we've lost in recent years and hope to regain from a post-Trump
presidency.

Biden should highlight his commitment to all three values in every
speech and interview he gives. They draw such a clear, simple and easy
to remember contrast with Trump.

Start with respecting science. Trump's obvious disdain for truth-telling
is annoying when he exaggerates his crowd sizes, his hand sizes, the
size of his bank account or the size of his election victory.

But his disdain for science has become fatal, as we're seeing in this
widening pandemic. Trump has gone from offering quack remedies, like
disinfectant, ultraviolet light and hydroxychloroquine, to mocking
people, including Biden, for adopting the easiest and most
scientifically proven method for limiting the spread of the coronavirus:
wearing a face mask.

The pro-Trump governor of Arizona, where the virus is now spiraling out
of control, at one point actually barred local officials from mandating
that residents wear masks. That's as crazy as when Trump declared, ``If
we stop testing right now, we'd have very few cases, if any.''

Think about that: Stop testing. Then we'll have no knowledge. Then we'll
have no numbers. Then we'll have no virus. Why didn't I think of that?

Stop testing people for drunken driving, and then we'll have no more
drunken drivers. Stop arresting people for shootings, and then the crime
rate will go down.

\includegraphics{https://static01.nyt.com/images/2020/07/01/opinion/30friedman2/merlin_173832828_cc25a706-5471-4614-907d-060bd8181db4-articleLarge.jpg?quality=75\&auto=webp\&disable=upscale}

Attention, fellow Americans, this impugning of scientific methods, this
embrace of conspiracy theories, this undermining of truth and data by
our president and vice president --- this is not happening in other
countries. This is not happening in Germany, France, China, South Korea,
Denmark, Canada, Israel or Japan. This is a form of American
``exceptionalism'' that we never imagined possible.

We're not leading. We're not following. We're lost.

``This is Dark Ages stuff,'' remarked
\href{https://energyinnovation.org/team-member/hal-harvey/}{Harvey,
founder of Energy Innovation}. ``A prime difference between the
Enlightenment and the Dark Ages is respect for knowledge, respect for
science. The whole idea of progress requires objectively looking at
problems, finding and testing solutions, and then spreading and using
the best of them. That's how we grow, that's how we learn, that's how we
prosper.''

Indeed, it is amazing to think that in the year 2020 Biden could
actually run for president with an ad that says: ``I believe in the
Enlightenment, Newtonian physics and the Age of Reason. The other guy
doesn't.''

As for respecting nature, that has two meanings. The first is to respect
the power of nature, which Trump has utterly failed to do. She doesn't
negotiate. You cannot seduce her or sue her. She does whatever
chemistry, biology and physics dictate. Full stop. Which means in a
pandemic that she will just keep infecting people --- relentlessly,
mercilessly, silently and exponentially --- until she runs out of people
to infect or a vaccine or exposure makes enough of us immune. She also
doesn't keep score. She'll make you sick and then blow down your house
with a tornado.

Trump's lack of respect for nature may be a political asset for him with
his base, but it's been a disaster for the country. He has built no
coordinated national strategy against a virus that demands coordination
--- because the virus evolved to exploit any cracks in your personal or
communal immune system, and it pays no heed to the Oklahoma-Texas
borderline.

Respect for nature also means understanding that we live on a hard rock
called planet Earth with a thin cover of oceans and topsoil, enveloped
by a thin layer of atmosphere. Abuse that soil, junk up those oceans
with plastics, distort that atmospheric blanket and we will likely
(further) destroy the perfect Garden of Eden that has been the basis of
all human civilization.

And remember, as bad as this pandemic is, it's just training wheels for
the big, irreversible atmospheric pandemic: climate change.

The latest evidence? See
\href{https://www.nationalgeographic.com/science/2020/06/what-100-degree-day-siberia-means-climate-change/}{National
Geographic online}: ``An extended heat wave that has been baking the
Russian Arctic for months drove the temperature in Verkhoyansk, Russia
--- north of the Arctic Circle --- to 100.4 degrees F on June 20, the
official first day of summer in the Northern Hemisphere.'' That's 100
degrees in the Arctic!

Respect each other? That's not so easy in the midst of our other
pandemic --- a pandemic of incivility. You cannot exaggerate the impact
on the whole civic culture of having a president who has elevated
name-calling, denigration and lying to a central feature of his
presidency, amplified by the White House.

We have social networks whose business model is to elevate and spread
the most enraged voices from the far right and the far left, and
generally bring out the worst in people. Almost every day now some
public figure, or just everyday American, has to apologize for some
inane or hurtful tweet.

But this pandemic of incivility is fed by many sources. We have white
police officers who feel such a sense of impunity that one of them kept
his knee on a Black man's neck for eight minutes and 46 seconds while
people were recording him on their phones.

We have a level of inequality that is so endemic that your
\href{https://www.nytimes.com/interactive/2020/05/13/opinion/inequality-cities-life-expectancy.html}{ZIP
code is now a better predictor of life expectancy} than your genetic
code. Respecting each other means ensuring each other's equal access to
the American dream --- and right now, Black, Hispanic and white
Americans are climbing very different housing, education and health care
ladders, which simply has to be fixed.

And we have a mad gun culture that has way too many young men thinking
respect can come from the barrel of a gun. Minneapolis has witnessed
\href{https://www.startribune.com/a-month-after-floyd-s-death-city-struggles-with-twin-crises/571496482/?refresh=true}{over
100 people shot since the death of George Floyd} on May 25 --- a lot of
it
\href{https://www.kare11.com/article/news/health/coronavirus/minneapolis-dealing-with-multiple-crises/89-125767bb-f565-481a-9d50-a2636ddf309b}{gang-related}.

We have so many important issues to discuss among ourselves right now,
but for that discussion to be productive we can't just go from
justifiable outrage straight to firings, public shamings or disbanding
police departments --- without pausing for respectful dialogue and moral
distinctions.

I don't know what is sufficient to get more people respecting one
another, but I know two things that are necessary. One is a president
who every day models respect rather than denigration. That's Biden's
job.

The other is getting people out of Facebook and into each other's faces
again --- not to shout or denounce, but to listen. It's important what
you learn when you listen. It's even more important \emph{what you say
when you listen}. Listening is a sign of respect. And it is amazing what
people will let you say to them if they first think that you respect
them. That's our job.

\emph{Respect science, respect nature, respect each other. Biden 2020.}

It's the only way to make America great again.

\emph{The Times is committed to publishing}
\href{https://www.nytimes.com/2019/01/31/opinion/letters/letters-to-editor-new-york-times-women.html}{\emph{a
diversity of letters}} \emph{to the editor. We'd like to hear what you
think about this or any of our articles. Here are some}
\href{https://help.nytimes.com/hc/en-us/articles/115014925288-How-to-submit-a-letter-to-the-editor}{\emph{tips}}\emph{.
And here's our email:}
\href{mailto:letters@nytimes.com}{\emph{letters@nytimes.com}}\emph{.}

\emph{Follow The New York Times Opinion section on}
\href{https://www.facebook.com/nytopinion}{\emph{Facebook}}\emph{,}
\href{http://twitter.com/NYTOpinion}{\emph{Twitter (@NYTopinion)}}
\emph{and}
\href{https://www.instagram.com/nytopinion/}{\emph{Instagram}}\emph{.}

Advertisement

\protect\hyperlink{after-bottom}{Continue reading the main story}

\hypertarget{site-index}{%
\subsection{Site Index}\label{site-index}}

\hypertarget{site-information-navigation}{%
\subsection{Site Information
Navigation}\label{site-information-navigation}}

\begin{itemize}
\tightlist
\item
  \href{https://help.nytimes.com/hc/en-us/articles/115014792127-Copyright-notice}{©~2020~The
  New York Times Company}
\end{itemize}

\begin{itemize}
\tightlist
\item
  \href{https://www.nytco.com/}{NYTCo}
\item
  \href{https://help.nytimes.com/hc/en-us/articles/115015385887-Contact-Us}{Contact
  Us}
\item
  \href{https://www.nytco.com/careers/}{Work with us}
\item
  \href{https://nytmediakit.com/}{Advertise}
\item
  \href{http://www.tbrandstudio.com/}{T Brand Studio}
\item
  \href{https://www.nytimes.com/privacy/cookie-policy\#how-do-i-manage-trackers}{Your
  Ad Choices}
\item
  \href{https://www.nytimes.com/privacy}{Privacy}
\item
  \href{https://help.nytimes.com/hc/en-us/articles/115014893428-Terms-of-service}{Terms
  of Service}
\item
  \href{https://help.nytimes.com/hc/en-us/articles/115014893968-Terms-of-sale}{Terms
  of Sale}
\item
  \href{https://spiderbites.nytimes.com}{Site Map}
\item
  \href{https://help.nytimes.com/hc/en-us}{Help}
\item
  \href{https://www.nytimes.com/subscription?campaignId=37WXW}{Subscriptions}
\end{itemize}
