Sections

SEARCH

\protect\hyperlink{site-content}{Skip to
content}\protect\hyperlink{site-index}{Skip to site index}

\href{https://myaccount.nytimes.com/auth/login?response_type=cookie\&client_id=vi}{}

\href{https://www.nytimes.com/section/todayspaper}{Today's Paper}

\href{/section/opinion}{Opinion}\textbar{}Billions in College Aid Hiding
in Plain Sight

\href{https://nyti.ms/3igI5CC}{https://nyti.ms/3igI5CC}

\begin{itemize}
\item
\item
\item
\item
\item
\end{itemize}

Advertisement

\protect\hyperlink{after-top}{Continue reading the main story}

\href{/section/opinion}{Opinion}

Supported by

\protect\hyperlink{after-sponsor}{Continue reading the main story}

\hypertarget{billions-in-college-aid-hiding-in-plain-sight}{%
\section{Billions in College Aid Hiding in Plain
Sight}\label{billions-in-college-aid-hiding-in-plain-sight}}

Students often have little help finding and applying for financial
assistance, and miss out on opportunities for affordable higher
education.

By Charlie Maynard

Mr. Maynard is a co-founder of a free scholarship search service.

\begin{itemize}
\item
  June 30, 2020
\item
  \begin{itemize}
  \item
  \item
  \item
  \item
  \item
  \end{itemize}
\end{itemize}

\includegraphics{https://static01.nyt.com/images/2020/06/30/opinion/30maynard-sub/30maynard-sub-articleLarge.jpg?quality=75\&auto=webp\&disable=upscale}

With college students and graduates burdened with over \$1.5 trillion in
student debt, it's infuriating that applying for financial aid is so
difficult that billions of dollars in
\href{https://www.nerdwallet.com/blog/2018-fafsa-study/}{federal aid} go
unclaimed, leaving many students deprived of opportunity.

To apply for federal aid, students need to fill out a Free Application
for Federal Student Aid, or FAFSA, which has 133 questions, including
difficult ones like what is their parents' total tax-exempt interest
income.

That's just for federal aid, including work-study. Applying for state
aid requires more forms.
\href{https://www.hesc.ny.gov/pay-for-college/financial-aid/types-of-financial-aid.html\#horizontalTab2}{New
York}, for example, has 22 grant or scholarship programs with different
applications.

Added to this, students have to consider
\href{https://nces.ed.gov/fastfacts/display.asp?id=84}{4,400 colleges}
with different financial aid policies and then 10 times that number of
private scholarships with different eligibility requirements and
questions.

A vast majority of students navigate this process alone. They share
their college counselors with
\href{https://www.schoolcounselor.org/asca/media/asca/home/Ratios18-19.pdf}{430
other students on average.} It is therefore no surprise that
\href{https://studentaid.gov/data-center/student/application-volume/fafsa-completion-high-school}{40
percent fall} at the first hurdle and don't even complete their FAFSA.
And no surprise, too, that no students complete all the financial aid
applications available to them.

The coronavirus pandemic only worsens the situation. School counselors
are even harder to reach, a lot of this information is online where
students without a decent computer can't find it, and many families'
ability to pay has been significantly reduced, making financial aid even
more important.

The
\href{https://research.collegeboard.org/trends/student-aid/figures-tables/total-federal-and-nonfederal-loans-type-over-time}{\$100
billion} in student loans issued each year is an unsustainable
consequence, but the most damaging repercussion is the impact on
students' ambitions.

To simplify the process, we developed a free platform,
\href{https://www.goingmerry.com/}{Going Merry}, to help students find
and apply for private scholarships, college financial aid and government
grants, in one place. About 350,000 students and 7,500 counselors use
the service.

Along with applying for assistance, students can see what aid various
colleges offer to students. Stanford University, for example, shows its
commitment to meet the full need to students from historically
underrepresented backgrounds. The University of Virginia highlights its
same commitment to ensure that out-of-state students know about its
financial aid as well as those in-state. Colleges can also feature
particular programs. Centre College in Kentucky, for example, shows its
Grissom Scholars Program, a full-tuition scholarship for
first-generation college students.

A number of sites make it easier to see what aid is available.
\href{https://myintuition.org/}{MyIntuition,} for example, provides an
online calculator to give students a financial aid estimate based on six
simple questions. Another start-up,
\href{https://www.fairopportunityproject.org/}{Fair Opportunity
Project}, provides free college application and financial aid guides to
every public school in America.

Without a simple and transparent process, financial aid applications add
another barrier to education rather than removing one. And they make it
harder for students who need support the most.

First, the FAFSA should be rewritten into simpler English and redesigned
so that critical information isn't hidden in the footnotes. Second, all
states should use the form as their sole application and allocate their
grants based on it. Third, colleges should standardize their financial
aid policies. Finally, scholarship providers should standardize their
essay prompts to match the \href{https://www.commonapp.org/}{Common
App}.

Our society depends on higher education to help our young people achieve
the American dream. Until financial aid applications are simplified,
grants and scholarships will go begging and disadvantaged students who
should attend college will not. There are enough barriers to social
mobility in America. We should not let red tape and endless forms be
among them.

Charlie Maynard is a co-founder and the chief executive of Going Merry,
a free online college scholarship search and application service.

\emph{The Times is committed to publishing}
\href{https://www.nytimes.com/2019/01/31/opinion/letters/letters-to-editor-new-york-times-women.html}{\emph{a
diversity of letters}} \emph{to the editor. We'd like to hear what you
think about this or any of our articles. Here are some}
\href{https://help.nytimes.com/hc/en-us/articles/115014925288-How-to-submit-a-letter-to-the-editor}{\emph{tips}}\emph{.
And here's our email:}
\href{mailto:letters@nytimes.com}{\emph{letters@nytimes.com}}\emph{.}

\emph{Follow The New York Times Opinion section on}
\href{https://www.facebook.com/nytopinion}{\emph{Facebook}}\emph{,}
\href{http://twitter.com/NYTOpinion}{\emph{Twitter (@NYTopinion)}}
\emph{and}
\href{https://www.instagram.com/nytopinion/}{\emph{Instagram}}\emph{.}

Advertisement

\protect\hyperlink{after-bottom}{Continue reading the main story}

\hypertarget{site-index}{%
\subsection{Site Index}\label{site-index}}

\hypertarget{site-information-navigation}{%
\subsection{Site Information
Navigation}\label{site-information-navigation}}

\begin{itemize}
\tightlist
\item
  \href{https://help.nytimes.com/hc/en-us/articles/115014792127-Copyright-notice}{©~2020~The
  New York Times Company}
\end{itemize}

\begin{itemize}
\tightlist
\item
  \href{https://www.nytco.com/}{NYTCo}
\item
  \href{https://help.nytimes.com/hc/en-us/articles/115015385887-Contact-Us}{Contact
  Us}
\item
  \href{https://www.nytco.com/careers/}{Work with us}
\item
  \href{https://nytmediakit.com/}{Advertise}
\item
  \href{http://www.tbrandstudio.com/}{T Brand Studio}
\item
  \href{https://www.nytimes.com/privacy/cookie-policy\#how-do-i-manage-trackers}{Your
  Ad Choices}
\item
  \href{https://www.nytimes.com/privacy}{Privacy}
\item
  \href{https://help.nytimes.com/hc/en-us/articles/115014893428-Terms-of-service}{Terms
  of Service}
\item
  \href{https://help.nytimes.com/hc/en-us/articles/115014893968-Terms-of-sale}{Terms
  of Sale}
\item
  \href{https://spiderbites.nytimes.com}{Site Map}
\item
  \href{https://help.nytimes.com/hc/en-us}{Help}
\item
  \href{https://www.nytimes.com/subscription?campaignId=37WXW}{Subscriptions}
\end{itemize}
