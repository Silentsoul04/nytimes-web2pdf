Sections

SEARCH

\protect\hyperlink{site-content}{Skip to
content}\protect\hyperlink{site-index}{Skip to site index}

\href{https://www.nytimes.com/section/science}{Science}

\href{https://myaccount.nytimes.com/auth/login?response_type=cookie\&client_id=vi}{}

\href{https://www.nytimes.com/section/todayspaper}{Today's Paper}

\href{/section/science}{Science}\textbar{}Most People With Coronavirus
Won't Spread It. Why Do a Few Infect Many?

\url{https://nyti.ms/38cKcme}

\begin{itemize}
\item
\item
\item
\item
\item
\item
\end{itemize}

\href{https://www.nytimes.com/news-event/coronavirus?action=click\&pgtype=Article\&state=default\&region=TOP_BANNER\&context=storylines_menu}{The
Coronavirus Outbreak}

\begin{itemize}
\tightlist
\item
  live\href{https://www.nytimes.com/2020/08/01/world/coronavirus-covid-19.html?action=click\&pgtype=Article\&state=default\&region=TOP_BANNER\&context=storylines_menu}{Latest
  Updates}
\item
  \href{https://www.nytimes.com/interactive/2020/us/coronavirus-us-cases.html?action=click\&pgtype=Article\&state=default\&region=TOP_BANNER\&context=storylines_menu}{Maps
  and Cases}
\item
  \href{https://www.nytimes.com/interactive/2020/science/coronavirus-vaccine-tracker.html?action=click\&pgtype=Article\&state=default\&region=TOP_BANNER\&context=storylines_menu}{Vaccine
  Tracker}
\item
  \href{https://www.nytimes.com/interactive/2020/07/29/us/schools-reopening-coronavirus.html?action=click\&pgtype=Article\&state=default\&region=TOP_BANNER\&context=storylines_menu}{What
  School May Look Like}
\item
  \href{https://www.nytimes.com/live/2020/07/31/business/stock-market-today-coronavirus?action=click\&pgtype=Article\&state=default\&region=TOP_BANNER\&context=storylines_menu}{Economy}
\end{itemize}

Advertisement

\protect\hyperlink{after-top}{Continue reading the main story}

Supported by

\protect\hyperlink{after-sponsor}{Continue reading the main story}

matter

\hypertarget{most-people-with-coronavirus-wont-spread-it-why-do-a-few-infect-many}{%
\section{Most People With Coronavirus Won't Spread It. Why Do a Few
Infect
Many?}\label{most-people-with-coronavirus-wont-spread-it-why-do-a-few-infect-many}}

Growing evidence shows most infected people aren't spreading the virus.
But whether you become a superspreader probably depends more on
circumstance than biology.

\includegraphics{https://static01.nyt.com/images/2020/06/30/science/30virus-superspreaders01/merlin_173670762_ad01cea6-7d80-4092-910d-d48bd8438402-articleLarge.jpg?quality=75\&auto=webp\&disable=upscale}

\href{https://www.nytimes.com/by/carl-zimmer}{\includegraphics{https://static01.nyt.com/images/2018/06/12/multimedia/author-carl-zimmer/author-carl-zimmer-thumbLarge.png}}

By \href{https://www.nytimes.com/by/carl-zimmer}{Carl Zimmer}

\begin{itemize}
\item
  Published June 30, 2020Updated July 23, 2020
\item
  \begin{itemize}
  \item
  \item
  \item
  \item
  \item
  \item
  \end{itemize}
\end{itemize}

\href{https://www.nytimes.com/es/2020/07/03/espanol/el-misterio-de-los-superpropagadores-de-coronavirus.html}{Leer
en español}

Following a birthday party in Texas on May 30, one man
\href{https://apnews.com/d9a6ca7eef083315648003509d07515a}{reportedly}
infected 17 members of his family with the coronavirus.

Reading
\href{https://www.nytimes.com/2020/06/22/us/new-coronavirus-phase.html?action=click\&module=Top\%20Stories\&pgtype=Homepage}{reports}
like these, you might think of the virus as a wildfire, instantly
setting off epidemics wherever it goes. But other reports tell another
story altogether.

In Italy, for example, scientists looked at stored samples of wastewater
for the earliest trace of the virus. Last week they
\href{https://www.thelocal.it/20200619/coronavirus-was-already-in-italy-by-december-waste-water-study-shows}{reported}
that the virus was in Turin and Milan as early as Dec. 18. But two
months would pass before northern Italy's hospitals began filling with
victims of Covid-19. So those December viruses seem to have petered out.

As strange as it may seem, these reports don't contradict each other.
Most infected people don't pass on the
\href{https://www.nytimes.com/2020/07/04/health/239-experts-with-one-big-claim-the-coronavirus-is-airborne.html}{coronavirus}
to someone else. But a small number pass it on to many others in
so-called superspreading events.

``You can think about throwing a match at kindling,'' said Ben Althouse,
principal research scientist at the Institute for Disease Modeling in
Bellevue, Wash. ``You throw one match, it may not light the kindling.
You throw another match, it may not light the kindling. But then one
match hits in the right spot, and all of a sudden the fire goes up.''

Understanding why some matches start fires while many do not will be
crucial to curbing the pandemic, scientists say. ``Otherwise, you're in
the position where you're always one step behind the virus,'' said Adam
Kucharski, an epidemiologist at the London School of Hygiene and
Tropical Medicine.

When the virus first emerged in China, epidemiologists scrambled to
understand how it spread from person to person. One of their first tasks
was to estimate the average number of people each sick person infected,
or what epidemiologists call the reproductive number.

The new coronavirus turned out to have a reproductive number somewhere
between two and three. It's impossible to pin down an exact figure,
since people's behavior can make it easier or harder for the virus to
spread. By going into lockdown, for instance, Massachusetts
\href{https://rt.live/us/MA}{drove its reproductive number} down from
2.2 at the beginning of March to 1 by the end of the month; it's now at
.74.

This averaged figure can also be misleading because it masks the
variability of spread from one person to the next. If nine out of 10
people don't pass on a virus at all, while the 10th passes it to 20
people, the average would still be two.

\hypertarget{latest-updates-global-coronavirus-outbreak}{%
\section{\texorpdfstring{\href{https://www.nytimes.com/2020/08/01/world/coronavirus-covid-19.html?action=click\&pgtype=Article\&state=default\&region=MAIN_CONTENT_1\&context=storylines_live_updates}{Latest
Updates: Global Coronavirus
Outbreak}}{Latest Updates: Global Coronavirus Outbreak}}\label{latest-updates-global-coronavirus-outbreak}}

Updated 2020-08-02T07:42:09.613Z

\begin{itemize}
\tightlist
\item
  \href{https://www.nytimes.com/2020/08/01/world/coronavirus-covid-19.html?action=click\&pgtype=Article\&state=default\&region=MAIN_CONTENT_1\&context=storylines_live_updates\#link-34047410}{The
  U.S. reels as July cases more than double the total of any other
  month.}
\item
  \href{https://www.nytimes.com/2020/08/01/world/coronavirus-covid-19.html?action=click\&pgtype=Article\&state=default\&region=MAIN_CONTENT_1\&context=storylines_live_updates\#link-780ec966}{Top
  U.S. officials work to break an impasse over the federal jobless
  benefit.}
\item
  \href{https://www.nytimes.com/2020/08/01/world/coronavirus-covid-19.html?action=click\&pgtype=Article\&state=default\&region=MAIN_CONTENT_1\&context=storylines_live_updates\#link-2bc8948}{Its
  outbreak untamed, Melbourne goes into even greater lockdown.}
\end{itemize}

\href{https://www.nytimes.com/2020/08/01/world/coronavirus-covid-19.html?action=click\&pgtype=Article\&state=default\&region=MAIN_CONTENT_1\&context=storylines_live_updates}{See
more updates}

More live coverage:
\href{https://www.nytimes.com/live/2020/07/31/business/stock-market-today-coronavirus?action=click\&pgtype=Article\&state=default\&region=MAIN_CONTENT_1\&context=storylines_live_updates}{Markets}

In some diseases, such as influenza and smallpox, a large fraction of
infected people pass on the pathogen to a few more. These diseases tend
to grow steadily and slowly. ``Flu can really plod along,'' said Kristin
Nelson, an assistant professor at Emory University.

But other diseases, like measles and SARS, are prone to sudden flares,
with only a few infected people spreading the disease.

Epidemiologists capture the difference between the flare-ups and the
plodding with something known as the dispersion parameter. It is a
measure of how much variation there is from person to person in
transmitting a pathogen.

But James Lloyd-Smith, a U.C.L.A. disease ecologist who developed the
dispersion parameter 15 years ago, cautioned that just because
scientists can measure it doesn't mean they understand why some diseases
have more superspreading than others. ``We just understand the bits of
it,'' he said.

When Covid-19 broke out, Dr. Kucharski and his colleagues tried to
calculate that number by comparing cases in different countries.

If Covid-19 was like the flu, you'd expect the outbreaks in different
places to be mostly the same size. But Dr. Kucharski and his colleagues
found a wide variation. The best way to explain this pattern, they
found, was that 10 percent of infected people were responsible for 80
percent of new infections. Which meant that most people passed on the
virus to few, if any, others.

Dr. Kucharski and his colleagues published their
\href{https://wellcomeopenresearch.org/articles/5-67}{study} in April as
a preprint, a report that has not been reviewed by other scientists and
published in a scientific journal. Other epidemiologists have calculated
the dispersion parameter with other methods, ending up with similar
estimates.

In Georgia, for example, Dr. Nelson and her colleagues analyzed over
9,500 Covid-19 cases from March to May. They created a model for the
\href{https://www.nytimes.com/2020/07/21/health/coronavirus-infections-us.html}{spread
of the virus} through five counties and estimated how many people each
person infected.

In a
\href{https://www.medrxiv.org/content/10.1101/2020.06.20.20130476v2}{preprint}
published last week, the researchers found many superspreading events.
Just 2 percent of people were responsible for 20 percent of
transmissions.

Now researchers are trying to figure out why so few people spread the
virus to so many. They're trying to answer three questions: Who are the
superspreaders? When does superspreading take place? And where?

As for the first question, doctors have observed that viruses can
multiply to bigger numbers inside some people than others. It's possible
that some people become virus chimneys, blasting out clouds of pathogens
with each breath.

\includegraphics{https://static01.nyt.com/images/2020/06/30/science/30virus-superspreaders03/merlin_172224546_86afedcd-4790-4b80-b39c-88177489c003-articleLarge.jpg?quality=75\&auto=webp\&disable=upscale}

Some people also have more opportunity to get sick, and to then make
other people sick. A bus driver or a nursing home worker may sit at a
hub in the social network, while most people are less likely to come
into contact with others --- especially in a lockdown.

Dr. Nelson suspects the biological differences between people are less
significant. ``I think the circumstances are a lot more important,'' she
said. Dr. Lloyd-Smith agreed. ``I think it's more centered on the
events.''

A lot of transmission seems to happen in a narrow window of time
starting a couple days after infection, even before symptoms emerge. If
people aren't around a lot of people during that window, they can't pass
it along.

\href{https://www.nytimes.com/news-event/coronavirus?action=click\&pgtype=Article\&state=default\&region=MAIN_CONTENT_3\&context=storylines_faq}{}

\hypertarget{the-coronavirus-outbreak-}{%
\subsubsection{The Coronavirus Outbreak
›}\label{the-coronavirus-outbreak-}}

\hypertarget{frequently-asked-questions}{%
\paragraph{Frequently Asked
Questions}\label{frequently-asked-questions}}

Updated July 27, 2020

\begin{itemize}
\item ~
  \hypertarget{should-i-refinance-my-mortgage}{%
  \paragraph{Should I refinance my
  mortgage?}\label{should-i-refinance-my-mortgage}}

  \begin{itemize}
  \tightlist
  \item
    \href{https://www.nytimes.com/article/coronavirus-money-unemployment.html?action=click\&pgtype=Article\&state=default\&region=MAIN_CONTENT_3\&context=storylines_faq}{It
    could be a good idea,} because mortgage rates have
    \href{https://www.nytimes.com/2020/07/16/business/mortgage-rates-below-3-percent.html?action=click\&pgtype=Article\&state=default\&region=MAIN_CONTENT_3\&context=storylines_faq}{never
    been lower.} Refinancing requests have pushed mortgage applications
    to some of the highest levels since 2008, so be prepared to get in
    line. But defaults are also up, so if you're thinking about buying a
    home, be aware that some lenders have tightened their standards.
  \end{itemize}
\item ~
  \hypertarget{what-is-school-going-to-look-like-in-september}{%
  \paragraph{What is school going to look like in
  September?}\label{what-is-school-going-to-look-like-in-september}}

  \begin{itemize}
  \tightlist
  \item
    It is unlikely that many schools will return to a normal schedule
    this fall, requiring the grind of
    \href{https://www.nytimes.com/2020/06/05/us/coronavirus-education-lost-learning.html?action=click\&pgtype=Article\&state=default\&region=MAIN_CONTENT_3\&context=storylines_faq}{online
    learning},
    \href{https://www.nytimes.com/2020/05/29/us/coronavirus-child-care-centers.html?action=click\&pgtype=Article\&state=default\&region=MAIN_CONTENT_3\&context=storylines_faq}{makeshift
    child care} and
    \href{https://www.nytimes.com/2020/06/03/business/economy/coronavirus-working-women.html?action=click\&pgtype=Article\&state=default\&region=MAIN_CONTENT_3\&context=storylines_faq}{stunted
    workdays} to continue. California's two largest public school
    districts --- Los Angeles and San Diego --- said on July 13, that
    \href{https://www.nytimes.com/2020/07/13/us/lausd-san-diego-school-reopening.html?action=click\&pgtype=Article\&state=default\&region=MAIN_CONTENT_3\&context=storylines_faq}{instruction
    will be remote-only in the fall}, citing concerns that surging
    coronavirus infections in their areas pose too dire a risk for
    students and teachers. Together, the two districts enroll some
    825,000 students. They are the largest in the country so far to
    abandon plans for even a partial physical return to classrooms when
    they reopen in August. For other districts, the solution won't be an
    all-or-nothing approach.
    \href{https://bioethics.jhu.edu/research-and-outreach/projects/eschool-initiative/school-policy-tracker/}{Many
    systems}, including the nation's largest, New York City, are
    devising
    \href{https://www.nytimes.com/2020/06/26/us/coronavirus-schools-reopen-fall.html?action=click\&pgtype=Article\&state=default\&region=MAIN_CONTENT_3\&context=storylines_faq}{hybrid
    plans} that involve spending some days in classrooms and other days
    online. There's no national policy on this yet, so check with your
    municipal school system regularly to see what is happening in your
    community.
  \end{itemize}
\item ~
  \hypertarget{is-the-coronavirus-airborne}{%
  \paragraph{Is the coronavirus
  airborne?}\label{is-the-coronavirus-airborne}}

  \begin{itemize}
  \tightlist
  \item
    The coronavirus
    \href{https://www.nytimes.com/2020/07/04/health/239-experts-with-one-big-claim-the-coronavirus-is-airborne.html?action=click\&pgtype=Article\&state=default\&region=MAIN_CONTENT_3\&context=storylines_faq}{can
    stay aloft for hours in tiny droplets in stagnant air}, infecting
    people as they inhale, mounting scientific evidence suggests. This
    risk is highest in crowded indoor spaces with poor ventilation, and
    may help explain super-spreading events reported in meatpacking
    plants, churches and restaurants.
    \href{https://www.nytimes.com/2020/07/06/health/coronavirus-airborne-aerosols.html?action=click\&pgtype=Article\&state=default\&region=MAIN_CONTENT_3\&context=storylines_faq}{It's
    unclear how often the virus is spread} via these tiny droplets, or
    aerosols, compared with larger droplets that are expelled when a
    sick person coughs or sneezes, or transmitted through contact with
    contaminated surfaces, said Linsey Marr, an aerosol expert at
    Virginia Tech. Aerosols are released even when a person without
    symptoms exhales, talks or sings, according to Dr. Marr and more
    than 200 other experts, who
    \href{https://academic.oup.com/cid/article/doi/10.1093/cid/ciaa939/5867798}{have
    outlined the evidence in an open letter to the World Health
    Organization}.
  \end{itemize}
\item ~
  \hypertarget{what-are-the-symptoms-of-coronavirus}{%
  \paragraph{What are the symptoms of
  coronavirus?}\label{what-are-the-symptoms-of-coronavirus}}

  \begin{itemize}
  \tightlist
  \item
    Common symptoms
    \href{https://www.nytimes.com/article/symptoms-coronavirus.html?action=click\&pgtype=Article\&state=default\&region=MAIN_CONTENT_3\&context=storylines_faq}{include
    fever, a dry cough, fatigue and difficulty breathing or shortness of
    breath.} Some of these symptoms overlap with those of the flu,
    making detection difficult, but runny noses and stuffy sinuses are
    less common.
    \href{https://www.nytimes.com/2020/04/27/health/coronavirus-symptoms-cdc.html?action=click\&pgtype=Article\&state=default\&region=MAIN_CONTENT_3\&context=storylines_faq}{The
    C.D.C. has also} added chills, muscle pain, sore throat, headache
    and a new loss of the sense of taste or smell as symptoms to look
    out for. Most people fall ill five to seven days after exposure, but
    symptoms may appear in as few as two days or as many as 14 days.
  \end{itemize}
\item ~
  \hypertarget{does-asymptomatic-transmission-of-covid-19-happen}{%
  \paragraph{Does asymptomatic transmission of Covid-19
  happen?}\label{does-asymptomatic-transmission-of-covid-19-happen}}

  \begin{itemize}
  \tightlist
  \item
    So far, the evidence seems to show it does. A widely cited
    \href{https://www.nature.com/articles/s41591-020-0869-5}{paper}
    published in April suggests that people are most infectious about
    two days before the onset of coronavirus symptoms and estimated that
    44 percent of new infections were a result of transmission from
    people who were not yet showing symptoms. Recently, a top expert at
    the World Health Organization stated that transmission of the
    coronavirus by people who did not have symptoms was ``very rare,''
    \href{https://www.nytimes.com/2020/06/09/world/coronavirus-updates.html?action=click\&pgtype=Article\&state=default\&region=MAIN_CONTENT_3\&context=storylines_faq\#link-1f302e21}{but
    she later walked back that statement.}
  \end{itemize}
\end{itemize}

And certain places seem to lend themselves to superspreading. A busy
bar, for example, is full of people talking loudly. Any one of them
could spew out viruses without ever coughing. And without good
ventilation, the
\href{https://www.nytimes.com/2020/07/04/health/239-experts-with-one-big-claim-the-coronavirus-is-airborne.html}{viruses
can linger in the air for hours}.

A study from Japan this month found
\href{https://wwwnc.cdc.gov/eid/article/26/9/20-2272_article}{clusters
of coronavirus cases} in health care facilities, nursing homes, day care
centers, restaurants, bars, workplaces, and musical events such as live
concerts and karaoke parties.

This pattern of superspreading could explain the puzzling lag in Italy
between the arrival of the virus and the rise of the epidemic. And
geneticists
\href{https://www.nytimes.com/2020/05/27/health/coronavirus-spread-united-states.html}{have
found} a similar lag in other countries: The first viruses to crop up in
a given region don't give rise to the epidemics that come weeks later.

Many countries and states have fought outbreaks with lockdowns, which
have managed to draw down Covid-19's reproductive number. But as
governments move toward reopening, they shouldn't get complacent and
forget the virus's potential for superspreading.

``You can really go from thinking you've got things under control to
having an out-of-control outbreak in a matter of a week,'' Dr.
Lloyd-Smith said.

Singapore's health authorities earned praise early on for holding down
the epidemic by carefully tracing cases of Covid-19. But they didn't
appreciate that huge dormitories where migrant workers lived were prime
spots for superspreading events. Now they are wrestling with a
resurgence of the virus.

On the other hand, knowing that Covid-19 is a superspreading pandemic
could be a good thing. ``It bodes well for control,'' Dr. Nelson said.

Since most transmission happens only in a small number of similar
situations, it may be possible to come up with smart strategies to stop
them from happening. It may be possible to avoid crippling,
across-the-board lockdowns by targeting the superspreading events.

``By curbing the activities in quite a small proportion of our life, we
could actually reduce most of the risk,'' said Dr. Kucharski.

Advertisement

\protect\hyperlink{after-bottom}{Continue reading the main story}

\hypertarget{site-index}{%
\subsection{Site Index}\label{site-index}}

\hypertarget{site-information-navigation}{%
\subsection{Site Information
Navigation}\label{site-information-navigation}}

\begin{itemize}
\tightlist
\item
  \href{https://help.nytimes.com/hc/en-us/articles/115014792127-Copyright-notice}{©~2020~The
  New York Times Company}
\end{itemize}

\begin{itemize}
\tightlist
\item
  \href{https://www.nytco.com/}{NYTCo}
\item
  \href{https://help.nytimes.com/hc/en-us/articles/115015385887-Contact-Us}{Contact
  Us}
\item
  \href{https://www.nytco.com/careers/}{Work with us}
\item
  \href{https://nytmediakit.com/}{Advertise}
\item
  \href{http://www.tbrandstudio.com/}{T Brand Studio}
\item
  \href{https://www.nytimes.com/privacy/cookie-policy\#how-do-i-manage-trackers}{Your
  Ad Choices}
\item
  \href{https://www.nytimes.com/privacy}{Privacy}
\item
  \href{https://help.nytimes.com/hc/en-us/articles/115014893428-Terms-of-service}{Terms
  of Service}
\item
  \href{https://help.nytimes.com/hc/en-us/articles/115014893968-Terms-of-sale}{Terms
  of Sale}
\item
  \href{https://spiderbites.nytimes.com}{Site Map}
\item
  \href{https://help.nytimes.com/hc/en-us}{Help}
\item
  \href{https://www.nytimes.com/subscription?campaignId=37WXW}{Subscriptions}
\end{itemize}
