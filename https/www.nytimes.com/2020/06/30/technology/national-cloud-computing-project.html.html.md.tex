Sections

SEARCH

\protect\hyperlink{site-content}{Skip to
content}\protect\hyperlink{site-index}{Skip to site index}

\href{https://www.nytimes.com/section/technology}{Technology}

\href{https://myaccount.nytimes.com/auth/login?response_type=cookie\&client_id=vi}{}

\href{https://www.nytimes.com/section/todayspaper}{Today's Paper}

\href{/section/technology}{Technology}\textbar{}Universities and Tech
Giants Back National Cloud Computing Project

\url{https://nyti.ms/3g8FqsI}

\begin{itemize}
\item
\item
\item
\item
\item
\end{itemize}

Advertisement

\protect\hyperlink{after-top}{Continue reading the main story}

Supported by

\protect\hyperlink{after-sponsor}{Continue reading the main story}

\hypertarget{universities-and-tech-giants-back-national-cloud-computing-project}{%
\section{Universities and Tech Giants Back National Cloud Computing
Project}\label{universities-and-tech-giants-back-national-cloud-computing-project}}

A proposal to give scientists access to huge data sets and powerful
computers.

\includegraphics{https://static01.nyt.com/images/2020/06/30/business/30cloud/30cloud-articleLarge.jpg?quality=75\&auto=webp\&disable=upscale}

\href{https://www.nytimes.com/by/steve-lohr}{\includegraphics{https://static01.nyt.com/images/2018/02/20/multimedia/author-steve-lohr/author-steve-lohr-thumbLarge.jpg}}

By \href{https://www.nytimes.com/by/steve-lohr}{Steve Lohr}

\begin{itemize}
\item
  June 30, 2020
\item
  \begin{itemize}
  \item
  \item
  \item
  \item
  \item
  \end{itemize}
\end{itemize}

Leading universities and major technology companies agreed on Tuesday to
back a new project intended to give academics and other scientists
access to the computing resources now available mainly to a few tech
giants.

The initiative, the National Research Cloud, has received bipartisan
support in both the House and the Senate. Lawmakers in both houses have
\href{https://eshoo.house.gov/media/press-releases/reps-eshoo-gonzalez-sherrill-introduce-bipartisan-bicameral-legislation-develop}{proposed
bills that would create a task force} of government science leaders,
academics and industry representatives to outline a plan to create and
fund a national research cloud.

This program would give academic scientists access to the cloud data
centers of the tech giants, and to public data sets for research.

Several universities, including Stanford, Carnegie Mellon and Ohio
State, and tech companies including Google, Amazon and IBM backed the
idea as well on Tuesday. The organizations declared their support for
the creation of a research cloud and their willingness to participate in
the project.

The research cloud, though a conceptual blueprint at this stage, is
another sign of the largely effective campaign by universities and tech
companies to persuade the American government to increase government
backing for research into artificial intelligence. The Trump
administration, while cutting research elsewhere, has proposed
\href{https://insight.ieeeusa.org/articles/fy-2021-rd-budget-proposal/}{doubling
federal spending on A.I. research} by 2022.

Fueling the increased government backing is the recognition that A.I.
technology is essential to national security and economic
competitiveness. The national cloud legislation will be proposed as an
amendment to this year's defense budget authorization.

``We have a real challenge in our country from China in terms of what
they are doing with A.I.,'' said Representative Anna G. Eshoo, Democrat
of California, a sponsor of the bill.

Funding for the project, the terms for paying the cloud providers and
what data might be available would be up to the task force and Congress.

``This is a logical first step,'' said Senator Rob Portman, Republican
of Ohio, another sponsor of the proposed law. ``The task force is going
to have to grapple with how you pay for it and how you govern it. But
you shouldn't have to work at Google to have access to this
technology.''

The national research cloud would address a problem that is a byproduct
of impressive progress in recent years. The striking gains made in tasks
like language understanding, computer vision, game playing and
common-sense reasoning have been attained thanks to a branch of A.I.
called deep learning.

That technology increasingly requires immense computing firepower.
\href{https://arxiv.org/pdf/1907.10597.pdf}{A report last year} from the
Allen Institute for Artificial Intelligence, working with data from
OpenAI, another artificial intelligence lab, observed that the volume of
calculations needed to be a leader in advanced A.I. had soared an
estimated 300,000 times in the previous six years. The cost of training
deep learning models, cycling endlessly through troves of data, can be
millions of dollars.

The cost and need for vast computing resources are putting some
cutting-edge A.I. research beyond the reach of academics. Only the tech
giants like Google, Amazon and Microsoft can spend billions a year on
data centers that are often the size of a football field, housing rack
upon rack with hundreds of thousands of computers.

So there has been a brain drain of computer scientists from universities
to the big tech companies, lured by access to their cloud data centers
as well as lucrative pay packages. The worry is that academic research
--- the seed corn of future breakthroughs --- is being shortchanged.

Academic work can be crucial particularly in areas where profits are not
on the immediate horizon. That was the story with deep learning, which
dates to the 1980s. A small band of academics nurtured the field for
years. Only since 2012, with enough computing power and data, did deep
learning really take off.

There have been smaller efforts for university research to tap into the
big tech clouds. But the current concept of an ambitious public-private
partnership for a
\href{https://hai.stanford.edu/blog/national-research-cloud-ensuring-continuation-american-innovation}{National
Research Cloud} came in March from John Etchemendy and Fei-Fei Li,
co-directors of the Stanford Institute for Human-Centered Artificial
Intelligence.

They posted their idea online and sought support from other
universities. The academics then promoted the idea to their political
representatives and industry contacts.

The federal government has long backed major research projects like
particle accelerators for high-energy physics in the 1960s and
supercomputing centers in the 1980s.

But in the past, the government built the labs and facilities. The
research cloud would use the cloud factories of the tech companies.
Academic scientists would be government-subsidized customers of the tech
giants, perhaps at rates below those charged to their business
customers.

Many university researchers say that buying rather than building is the
only sensible path, given the daunting cost of hyper-scale data centers.

``We need to get scientific research on the public cloud,'' said Ed
Lazowska, a professor at the University of Washington. ``We have to
hitch ourselves to that wagon. It's the only way to keep up.''

Advertisement

\protect\hyperlink{after-bottom}{Continue reading the main story}

\hypertarget{site-index}{%
\subsection{Site Index}\label{site-index}}

\hypertarget{site-information-navigation}{%
\subsection{Site Information
Navigation}\label{site-information-navigation}}

\begin{itemize}
\tightlist
\item
  \href{https://help.nytimes.com/hc/en-us/articles/115014792127-Copyright-notice}{©~2020~The
  New York Times Company}
\end{itemize}

\begin{itemize}
\tightlist
\item
  \href{https://www.nytco.com/}{NYTCo}
\item
  \href{https://help.nytimes.com/hc/en-us/articles/115015385887-Contact-Us}{Contact
  Us}
\item
  \href{https://www.nytco.com/careers/}{Work with us}
\item
  \href{https://nytmediakit.com/}{Advertise}
\item
  \href{http://www.tbrandstudio.com/}{T Brand Studio}
\item
  \href{https://www.nytimes.com/privacy/cookie-policy\#how-do-i-manage-trackers}{Your
  Ad Choices}
\item
  \href{https://www.nytimes.com/privacy}{Privacy}
\item
  \href{https://help.nytimes.com/hc/en-us/articles/115014893428-Terms-of-service}{Terms
  of Service}
\item
  \href{https://help.nytimes.com/hc/en-us/articles/115014893968-Terms-of-sale}{Terms
  of Sale}
\item
  \href{https://spiderbites.nytimes.com}{Site Map}
\item
  \href{https://help.nytimes.com/hc/en-us}{Help}
\item
  \href{https://www.nytimes.com/subscription?campaignId=37WXW}{Subscriptions}
\end{itemize}
