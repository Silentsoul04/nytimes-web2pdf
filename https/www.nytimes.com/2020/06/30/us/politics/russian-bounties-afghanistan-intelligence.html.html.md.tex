Sections

SEARCH

\protect\hyperlink{site-content}{Skip to
content}\protect\hyperlink{site-index}{Skip to site index}

\href{https://www.nytimes.com/section/politics}{Politics}

\href{https://myaccount.nytimes.com/auth/login?response_type=cookie\&client_id=vi}{}

\href{https://www.nytimes.com/section/todayspaper}{Today's Paper}

\href{/section/politics}{Politics}\textbar{}Suspicions of Russian
Bounties Were Bolstered by Data on Financial Transfers

\url{https://nyti.ms/2NJfxU9}

\begin{itemize}
\item
\item
\item
\item
\item
\item
\end{itemize}

Advertisement

\protect\hyperlink{after-top}{Continue reading the main story}

Supported by

\protect\hyperlink{after-sponsor}{Continue reading the main story}

\hypertarget{suspicions-of-russian-bounties-were-bolstered-by-data-on-financial-transfers}{%
\section{Suspicions of Russian Bounties Were Bolstered by Data on
Financial
Transfers}\label{suspicions-of-russian-bounties-were-bolstered-by-data-on-financial-transfers}}

Analysts have used other evidence to conclude that the transfers were
most likely part of an effort to offer payments to Taliban-linked
militants to kill American and coalition troops in Afghanistan.

\includegraphics{https://static01.nyt.com/images/2020/06/30/us/politics/30dc-intel/30dc-intel-articleLarge-v3.jpg?quality=75\&auto=webp\&disable=upscale}

\href{https://www.nytimes.com/by/charlie-savage}{\includegraphics{https://static01.nyt.com/images/2018/06/12/multimedia/author-charlie-savage/author-charlie-savage-thumbLarge-v2.png}}\href{https://www.nytimes.com/by/mujib-mashal}{\includegraphics{https://static01.nyt.com/images/2018/10/15/multimedia/author-mujib-mashal/author-mujib-mashal-thumbLarge.png}}\href{https://www.nytimes.com/by/rukmini-callimachi}{\includegraphics{https://static01.nyt.com/images/2018/10/08/multimedia/author-rukmini-callimachi/author-rukmini-callimachi-thumbLarge-v2.png}}\href{https://www.nytimes.com/by/eric-schmitt}{\includegraphics{https://static01.nyt.com/images/2018/06/12/multimedia/author-eric-schmitt/author-eric-schmitt-thumbLarge-v2.png}}\href{https://www.nytimes.com/by/adam-goldman}{\includegraphics{https://static01.nyt.com/images/2018/07/12/multimedia/author-adam-goldman/author-adam-goldman-thumbLarge.png}}

By \href{https://www.nytimes.com/by/charlie-savage}{Charlie Savage},
\href{https://www.nytimes.com/by/mujib-mashal}{Mujib Mashal},
\href{https://www.nytimes.com/by/rukmini-callimachi}{Rukmini
Callimachi}, \href{https://www.nytimes.com/by/eric-schmitt}{Eric
Schmitt} and \href{https://www.nytimes.com/by/adam-goldman}{Adam
Goldman}

\begin{itemize}
\item
  Published June 30, 2020Updated July 29, 2020
\item
  \begin{itemize}
  \item
  \item
  \item
  \item
  \item
  \item
  \end{itemize}
\end{itemize}

American officials intercepted electronic data showing large financial
transfers from a bank account controlled by
\href{https://www.nytimes.com/2020/07/13/world/asia/russia-taliban-afghanistan.html}{Russia's}
military intelligence agency to a
\href{https://www.nytimes.com/2020/07/13/world/asia/russia-taliban-afghanistan.html}{Taliban}-linked
account, evidence that supported their conclusion that
\href{https://www.nytimes.com/2020/07/29/us/politics/trump-putin-bounties.html}{Russia
covertly offered bounties} for killing U.S. and coalition troops in
Afghanistan, according to three officials familiar with the
intelligence.

Though the United States has accused
\href{https://www.nytimes.com/2020/07/01/us/politics/trump-putin-russia-taliban-bounty.html}{Russia}
of providing general support to the Taliban before, analysts concluded
from other intelligence that the transfers were most likely part of a
bounty program that detainees described during interrogations.

Investigators also identified by name numerous
\href{https://www.nytimes.com/2020/07/01/world/asia/afghan-russia-bounty-middleman.html}{Afghans}
in a network linked to the suspected Russian operation, the officials
said --- including, two of them added, a man believed to have served as
an intermediary for distributing some of the funds and who is now
thought to be in Russia.

The intercepts bolstered the findings gleaned from the interrogations,
helping reduce an earlier disagreement among intelligence analysts and
agencies over the reliability of the detainees. The disclosures further
undercut White House officials' claim that the intelligence was too
uncertain to brief President Trump. In fact, the information
\href{https://www.nytimes.com/2020/06/29/us/politics/russian-bounty-trump.html}{was
provided to him} in his daily written brief in late February, two
officials have said.

Afghan officials this week described a sequence of events that
dovetailed with the account of the intelligence. They said that several
businessmen who transfer money through the informal ``hawala'' system
were arrested in Afghanistan over the past six months and were suspected
of being part of a ring of middlemen who operated between the Russian
intelligence agency, known as the G.R.U., and Taliban-linked militants.
The businessmen were arrested in what the officials described as
sweeping raids in the north of Afghanistan as well as in Kabul.

A half-million dollars was seized from the home of one of the men, added
a provincial official. The New York Times had
\href{https://www.nytimes.com/2020/06/28/us/politics/russian-bounties-warnings-trump.html}{previously
reported} that the recovery of an unusually large amount of cash in a
raid was an early piece in the puzzle that investigators put together.

The three American officials who described and confirmed details about
the basis for the intelligence assessment spoke on the condition of
anonymity amid swelling turmoil over the Trump administration's failure
to authorize any response to Russia's suspected proxy targeting of
American troops and playing down of the issue after it came to light
four days ago.

White House and National Security Council officials declined to comment,
as did the Office of the Director of National Intelligence, John
Ratcliffe. They pointed to statements late Monday from Mr. Ratcliffe;
the national security adviser, Robert C. O'Brien; and the Pentagon's top
spokesman, Jonathan Hoffman. All of them said that recent news reports
about Afghanistan remained unsubstantiated.

The White House press secretary, Kayleigh McEnany, berated The Times on
Tuesday after this article was published, saying that reports based on
``selective leaking'' disrupt intelligence gathering. She did not
address or deny the facts about the intelligence assessment, saying she
would not disclose classified information.

On Monday, the administration invited several House Republicans to the
White House to discuss the intelligence. The briefing was mostly carried
out by three Trump administration officials: Mr. Ratcliffe, Mr. O'Brien
and Mark Meadows, the White House chief of staff. Until recently, both
Mr. Meadows and Mr. Ratcliffe were Republican congressmen known for
being outspoken supporters of Mr. Trump.

\includegraphics{https://static01.nyt.com/images/2017/01/29/podcasts/the-daily-album-art/the-daily-album-art-articleInline-v2.jpg?quality=75\&auto=webp\&disable=upscale}

\hypertarget{listen-to-the-daily-a-russian-plot-to-kill-us-soldiers}{%
\subsubsection{Listen to `The Daily': A Russian Plot to Kill U.S.
Soldiers}\label{listen-to-the-daily-a-russian-plot-to-kill-us-soldiers}}

Why intelligence officials suspect Russia of paying militants to kill
American troops in Afghanistan.

transcript

Back to The Daily

bars

0:00/23:45

-23:45

transcript

\hypertarget{listen-to-the-daily-a-russian-plot-to-kill-us-soldiers-1}{%
\subsection{Listen to `The Daily': A Russian Plot to Kill U.S.
Soldiers}\label{listen-to-the-daily-a-russian-plot-to-kill-us-soldiers-1}}

\hypertarget{hosted-by-michael-barbaro-produced-by-austin-mitchell-eric-krupke-and-adizah-eghan-with-help-from-robert-jimison-and-edited-by-lisa-chow}{%
\subsubsection{Hosted by Michael Barbaro; produced by Austin Mitchell,
Eric Krupke and Adizah Eghan, with help from Robert Jimison; and edited
by Lisa
Chow}\label{hosted-by-michael-barbaro-produced-by-austin-mitchell-eric-krupke-and-adizah-eghan-with-help-from-robert-jimison-and-edited-by-lisa-chow}}

\hypertarget{why-intelligence-officials-suspect-russia-of-paying-militants-to-kill-american-troops-in-afghanistan}{%
\paragraph{Why intelligence officials suspect Russia of paying militants
to kill American troops in
Afghanistan.}\label{why-intelligence-officials-suspect-russia-of-paying-militants-to-kill-american-troops-in-afghanistan}}

\begin{itemize}
\item
  michael barbaro\\
  From The New York Times, I'm Michael Barbaro. This is ``The Daily.''
\item
  {[}music{]}\\
  Today: A Times investigation has revealed evidence of a secret Russian
  operation to kill American soldiers in Afghanistan --- and the failure
  of the Trump administration to act on that evidence. I spoke with my
  colleague Eric Schmitt, one of the reporters who broke the original
  story, about what we know now.

  It's Wednesday, July 1.

  Eric, how is it that the U.S. first learned that Russia was up to
  something in Afghanistan?
\item
  eric schmitt\\
  So, Michael, about six months ago or so, U.S. commandos, working with
  Afghan allies, carry out a raid on a Taliban safehouse. And they made
  a remarkable discovery. They found some \$500,000 in American money
  inside this safehouse. Now, to be sure, from time to time when they do
  these kind of raids, you find weapons and you find other kinds of
  things. Even some money. But the military sources that we've talked to
  said they'd never seen such a large haul. I mean, what would these
  guys be doing with \$500,000? How did they get it, and what was it
  going to be used for? So this set off a lot of questions.

  And as they conduct other raids, the commandos, C.I.A., other
  authorities in Afghanistan, they seize the cell phones of different
  fighters --- Taliban fighters --- and they start exploiting that to
  see if there's any clues in the cell phones that might lead them back
  to the source of this money. But perhaps one of most important things
  that happens is when they seize a couple of very important senior
  Taliban and Taliban-related figures. And of course, that's one of the
  first things they want to ask these operatives is do you know anything
  about this money?
\item
  michael barbaro\\
  And what do the militants say?
\item
  eric schmitt\\
  They had a remarkable story to tell. That this was money that they had
  been paid. That they'd been paid by a secretive Russian military
  intelligence unit for the express purpose of killing American, British
  and other coalition forces in Afghanistan.

  But these investigators, they were searching around for other proof
  --- how to link all this together because, of course, how do you
  assess that these Taliban guys weren't telling lies or some kind of
  disinformation?

  And then investigators learned of something else that sealed the deal,
  that seemed to kind of be the glue that pieced all these disparate
  parts together. And that was intercepts. Basically electronic
  intercepts of the financial transactions themselves from this Russian
  military intelligence unit, down to the Afghans on the ground who are
  the intermediaries, who are basically managing this program for them
  there. And then onto the killers themselves before they were
  dispatched to target the American forces there.

  Essentially, it was an electronic paper trail, receipts if you will,
  for services asked and services rendered. This became a very
  compelling argument that the military C.I.A. and other authorities in
  Afghanistan started putting together.
\item
  michael barbaro\\
  And a very serious conclusion, because from what you're describing,
  U.S. intelligence officials are not just putting together a theory
  that this money was offered to Taliban fighters to go after Americans
  --- to basically kill them for hire --- but that money had actually
  been paid out to them, suggesting that such killings had occurred.
\item
  eric schmitt\\
  That's right. This wasn't just in theory, but there was the idea that
  they'd actually recovered some of the proceeds that the Russians had
  paid the Afghans to carry out this mission. So obviously the next task
  was to figure out what deaths may have been actually the result of
  this campaign.
\item
  michael barbaro\\
  And do we have an answer?
\item
  eric schmitt\\
  So the military and the intelligence officers working with Afghan
  officials started looking back over different attacks to see which
  looked suspicious. And their attentions focused on one in particular:
  three Marines who were killed on a patrol just outside of Bagram Air
  Base. They were patrolling on a normal day when a large car bomb
  basically blew up. And this is something that the military is still
  determining, just what the links were, if any, to this program, this
  attack. But it was suspicious, and it may have had the hallmarks of
  this program and some of the receipts tying back to it.
\item
  michael barbaro\\
  And, Eric, in the minds of these intelligence officials who are
  starting to piece this Russian bounty system together, why would
  Russia do this? I mean, why would they pay the Taliban to kill U.S.
  soldiers?
\item
  eric schmitt\\
  Well, Michael, I think you have to go back in the history of the U.S.
  and Russia and Afghanistan, essentially to the very end of the Cold
  War, where in the late 1980s, the C.I.A. secretly armed the Mujahideen
  resistance against the Soviet Union, which had invaded and occupied
  Afghanistan for nearly a decade. And the United States helped
  accelerate the departure of Soviet soldiers from Afghanistan.

  Fast forward to after 9/11 when it's the U.S. that invades
  Afghanistan. Russians want a stable government there. They don't like
  Al Qaeda any more than the United States does. And so for some years,
  there's actually some cooperation between Moscow and Washington.

  Until a few years ago, when President Putin of Russia starts to become
  disillusioned with the U.S. plan in Afghanistan, doesn't believe it's
  going to work, and begins --- behind the backs of the U.S. --- to
  support the Taliban. To provide weapons, arms the Taliban, who are
  still fighting the United States. And so we start to see this break
  where Russia is basically looking for ways to inflict pain on the
  United States, and maybe even accelerate the U.S. departure from
  Afghanistan, just as decades before the U.S. had done to the Soviet
  Union.

  So if you put that framework, where Russia is now looking for a way to
  replace the United States as the power inside of Afghanistan, and
  humiliate the United States at the same time, this bounty program
  starts to make a little bit of sense. If this secretive military
  intelligence unit can put bounties on the heads of American soldiers,
  increase the number of casualties, presumably that would also stir
  unrest back in the United States --- already war weary after two
  decades of conflict in Afghanistan. So the Russian theory is, why not
  just speed that departure along? We take the U.S.`s place and we
  humiliate Washington and President Trump in the process.
\item
  michael barbaro\\
  And I guess the reason why Russia would turn to a middleman, the
  Taliban, on this is because it would never want to attack U.S.
  soldiers on its own in Afghanistan, just the way the U.S. didn't want
  to ever attack Russian soldiers directly in the 1980s.
\item
  eric schmitt\\
  That's right. You hire basically cutouts to do your dirty work, and
  it's very hard for the other side to prove that you're responsible.
  When it's murky like this and you have Afghan intermediaries,
  criminals on the ground, and money's passing back and forth, Russians
  would have plausible deniability to say, oh, perhaps we were just
  supporting them for other aims. There's no evidence that we were
  behind this.
\item
  michael barbaro\\
  But Eric, even so, even with a middleman cutout, as you just called
  it, I have to imagine that this kind of an operation by Russia is very
  risky and represents a pretty significant escalation by Russia.
\item
  eric schmitt\\
  Absolutely, Michael. Any time you have a foreign power, much less one
  like Russia, targeting American service members --- American troops on
  the ground --- that is a very serious thing.
\item
  michael barbaro\\
  Right because this is, in its own way, almost a kind of act of war.
\item
  eric schmitt\\
  Absolutely. That's the way many people would see it. Just because
  Russia might be using intermediaries or henchmen to do this, they're
  the ones responsible. They're the ones setting these killings in
  motion if they've happened. They're the ones that are essentially
  bribing the killers to carry out the attacks, and that's something
  that's very, very serious. And the Pentagon and the White House would
  have to address it.
\item
  {[}music{]}
\item
  michael barbaro\\
  We'll be right back.

  So Eric, in your reporting on this Russian bounty operation, what do
  you learn about how the White House, how the Pentagon decides to
  respond to the conclusion of the intelligence agencies that this
  operation exists?
\item
  eric schmitt\\
  So this assessment that's been put together by the C.I.A. and the
  military special-operations forces in Afghanistan starts to make its
  way up the chain of command into Washington, sometime in late January,
  early February perhaps. And it's very closely held. This is some of
  the most sensitive intelligence in the American government, both
  because of the ramifications if it's true that Russia has put a bounty
  on American soldiers heads, and the political sensitivity that
  anything to do with Russia has with this administration and
  specifically this White House. And that assessment is serious enough
  that it makes its way into what's called the presidential daily
  briefing. This is the compendium of top intelligence and news items.
  It's put together every day for the president to read. President Trump
  is not known to read it very often, very much. He relies more on
  verbal briefings, oral briefings.

  But by February 27, our sources tell us, it was in that document.
  About a month later, at the end of March, the National Security
  Council, the national security arm of the White House, holds its first
  meeting to discuss the intelligence assessment. It's representatives
  from the State Department, from the Pentagon, from the C.I.A., from
  around the government who can weigh in about the impact this might
  have, and most important, the options. How should the United States
  government respond to this? And the options that are discussed at this
  meeting in late March include everything from sending Moscow a stern
  letter --- basically cease and desist or else --- all the way up to
  sanctions, economic sanctions on top of those already imposed on
  Moscow that have been proven effective in damaging their economy.

  We don't know if President Trump was briefed on any of these options.
  But we do know that his administration did not authorize any kind of
  action in response. Nothing has happened so far as a result of this
  assessment.

  So that's the way things stood for many weeks, that this was very
  tightly held information at the most senior levels of the government,
  until late last week ---
\item
  archived recording\\
  This is sort of stunning. Here's the lead.
\end{itemize}

eric schmitt

--- when the Times published a major investigation that basically
spelled out everything we've just been discussing.

\begin{itemize}
\tightlist
\item
  archived recording\\
  A New York Times report alleges Russia offered bounties to the Taliban
  in exchange for killing U.S. forces in Afghanistan.
\end{itemize}

michael barbaro

And what was the immediate reaction to all that information?

\begin{itemize}
\tightlist
\item
  archived recording\\
  Well, look, I'm sick to my stomach over this.
\end{itemize}

eric schmitt

Well, the immediate reaction was one of stunned disbelief ---

\begin{itemize}
\item
  archived recording\\
  Sick to my stomach as a member of Congress, a patriot, but also
  someone who served in Afghanistan ---
\item
  archived recording (nancy pelosi)\\
  This is as bad as it gets.
\end{itemize}

eric schmitt

--- both by Democrats and Republicans in Congress.

\begin{itemize}
\item
  archived recording (nancy pelosi)\\
  And yet the president will not confront the Russians on this score.
\item
  archived recording\\
  Republican Senator Lindsey Graham, he said, quote, ``imperative
  Congress get to the bottom of recent media reports.''
\item
  archived recording (chuck schumer)\\
  Where is President Trump? His number-one job is to protect American
  soldiers.
\end{itemize}

eric schmitt

There was outrage that if indeed this bounty program had existed, what
was the United States government doing about it? How were they
protecting their soldiers, first of all, in Afghanistan? And what steps
were being taken to punish the Russians?

\begin{itemize}
\tightlist
\item
  archived recording (chuck schumer)\\
  He should have a plan. What are we doing? And above all, go after
  Putin.
\end{itemize}

eric schmitt

Because this is at a time when President Trump has continued to carry
out conversations with President Putin. In fact, just a few weeks ago
---

\begin{itemize}
\tightlist
\item
  archived recording (donald trump)\\
  The problem is many of the things that we talk about are about Putin.
\end{itemize}

eric schmitt

--- he invited Russia to join the G8 conference in Washington ---

\begin{itemize}
\tightlist
\item
  archived recording (donald trump)\\
  Then I say have him in the room. Have him in the room.
\end{itemize}

eric schmitt

--- much to the disbelief of European allies and even some of his own
Republican supporters here in the United States.

\begin{itemize}
\tightlist
\item
  archived recording (donald trump)\\
  So we have a G7. He's not there. Half of the meeting is devoted to
  Russia. And if he was there, it would be much easier to solve.
\end{itemize}

eric schmitt

So as this information breaks, it breaks against a backdrop of the
president continuing to enjoy, in his view, very warm relations with
Vladimir Putin in Moscow.

michael barbaro

And how does the White House explain this? I mean not only not
responding to this Russian bounty program, but actually growing closer
to Russia and to Vladimir Putin after our government had reached this
conclusion.

\begin{itemize}
\tightlist
\item
  archived recording (kayleigh mcenany)\\
  Hello, everyone.
\end{itemize}

eric schmitt

The White House's immediate response is ---

\begin{itemize}
\tightlist
\item
  archived recording (kayleigh mcenany)\\
  --- The C.I.A. director, N.S.A., National Security Adviser, and the
  chief of staff can all confirm that neither the president nor the vice
  president were briefed on the alleged Russian bounty intelligence.
\end{itemize}

eric schmitt

--- that President Trump was never briefed on this. He never had a
briefing from the C.I.A. director, from his national security adviser,
from his director of national intelligence, and thus how could he have
made any decision on it?

\begin{itemize}
\tightlist
\item
  archived recording (kayleigh mcenany)\\
  There is no consensus within the intelligence community on these
  allegations. And, in effect, there are dissenting opinions from some
  in the intelligence community with regards to the veracity of what's
  being reported.
\end{itemize}

eric schmitt

The White House press secretary is saying that the reason he wasn't
briefed was because there was no consensus among the intelligence
agencies on what to brief him about.

\begin{itemize}
\tightlist
\item
  archived recording (adam schiff)\\
  We need to get to the bottom of these reports. I'm going to be briefed
  at the White House tomorrow. I'm asking that my entire committee be
  briefed by the intel agencies and ---
\end{itemize}

eric schmitt

Democrats and Republicans both demand briefings from the president's top
advisers on what the intelligence report says.

\begin{itemize}
\tightlist
\item
  archived recording (adam schiff)\\
  But is this another situation where the president either was told and
  just rejects it ---
\end{itemize}

eric schmitt

And what the president knew and when he knew it.

\begin{itemize}
\tightlist
\item
  archived recording (adam schiff)\\
  --- or his people are too scared to tell him because it contradicts
  this narrative of Vladimir Putin being his buddy?
\end{itemize}

eric schmitt

What his aides knew and when they knew it. And if the president really
wasn't briefed, why wasn't he briefed?

michael barbaro

Right, because the thinking is that the president knowing and not acting
is extremely problematic. But the president not knowing is problematic
as well, because what would it say about an administration if the
president was somehow not told this information or did not digest it?

eric schmitt

That's right. There's no good answer for the White House in this. Either
the president was told and he doesn't remember, he wasn't told because
his aides feared what his reaction might be, or he was told and just
dismissed it, because he didn't believe the intelligence because it
involved negative reporting on Russia. You have to remember this has
happened before. You think back to the allegations that Russia meddled
in the 2016 elections.

\begin{itemize}
\tightlist
\item
  archived recording\\
  Just now, President Putin denied having anything to do with the
  election interference in 2016.
\end{itemize}

eric schmitt

And when asked about this at a news conference in Helsinki ---

\begin{itemize}
\tightlist
\item
  archived recording\\
  Would you now, with the whole world watching, tell President Putin,
  would you denounce what happened in 2016, and would you warn him to
  never do it again?
\end{itemize}

eric schmitt

President Trump turned to President Putin.

\begin{itemize}
\tightlist
\item
  archived recording (donald trump)\\
  I have President Putin. He just said it's not Russia. I will say this.
  I don't see any reason why it would be. But I really ---
\end{itemize}

eric schmitt

He said, I believe him over my intelligence agencies. This has a
different feel to it though. So often we've seen the past about some of
the president's utterances and judgments and tweets which have kind of
fallen into partisan camps, and people can say what he really meant or
not. This is something different. This is about soldiers' lives in
Afghanistan. This is about somebody's brother, somebody's husband,
somebody's daughter who are on the front lines in Afghanistan. And
Trump, as the commander in chief, doesn't care enough to take the brief?
Doesn't care enough to read the intelligence about this? Or his aides
don't think he will? Something as sacrosanct as the American soldier in
harm's way in the battlefields of Afghanistan, the White House doesn't
have its back? The president doesn't have their backs? That's something
very, very troubling indeed, if true.

{[}music{]}

michael barbaro

Thank you, Eric.

eric schmitt

Thank you.

michael barbaro

We'll be right back.

Here's what else you need to know today.

\begin{itemize}
\item
  archived recording (elizabeth warren)\\
  Dr. Fauci, based on what you're seeing now, how many Covid-19 deaths
  and infections should America expect before this is all over?
\item
  archived recording (anthony fauci)\\
  I can't make an accurate prediction, but it is going to be very
  disturbing. I will guarantee you that because when you have an
  outbreak ---
\end{itemize}

michael barbaro

During his latest appearance before Congress on Tuesday, Dr. Anthony
Fauci warned lawmakers that the number of new infections in the U.S.
could more than double if current conditions persist.

\begin{itemize}
\tightlist
\item
  archived recording (anthony fauci)\\
  We can't just focus on those areas that are having the surge. It puts
  the entire country at risk. We are now having 40-plus thousand new
  cases a day. I would not be surprised if we go up to 100,000 a day if
  this does not turn around, and so I am very concerned.
\end{itemize}

michael barbaro

His warning comes as a surge of infections in the South and West now
extend to the Midwest, where six states are recording higher infection
rates. Overall, U.S. infections have increased 80 percent over the past
two weeks.

And in a closely watched Senate primary in Kentucky, Amy McGrath, the
moderate choice of the Democratic Party establishment, has narrowly
defeated Charles Booker, a liberal challenger who harnessed growing
public anger over police brutality. McGrath will now face Republican
Senator Mitch McConnell in the fall. In its final weeks, the Kentucky
primary had become a referendum on the future of the Democratic Party,
and whether the outcry over race and policing could influence the
outcome of an election. Despite losing, Booker won nearly 43 percent of
the vote.

{[}music{]}

That's it for ``The Daily.'' I'm Michael Barbaro. See you tomorrow.

That briefing focused on intelligence information that supported the
conclusion that
\href{https://www.nytimes.com/2020/07/01/podcasts/the-daily/russian-bounties-afghanistan.html}{Russia
was running a covert bounty operation} and other information that did
not support it, according to two people familiar with the meeting. For
example, the briefing focused in part on the interrogated detainees'
accounts and the earlier analysts' disagreement over it.

Both people said the intent of the briefing seemed to be to make the
point that the intelligence on the suspected Russian bounty plot was not
clear cut. For example, one of the people said, the White House also
cited some interrogations by Afghan intelligence officials of other
detainees, playing down their credibility by describing them as
low-level.

The administration officials did not mention anything in the House
Republican briefing about intercepted data tracking financial transfers,
both of the people familiar with it said.

Democrats and Senate Republicans were also separately briefed at the
White House on Tuesday morning. Democrats emerged saying that the issue
was clearly not, as Mr. Trump has suggested, a
``\href{https://twitter.com/realdonaldtrump/status/1277431695248183298?lang=en}{hoax}.''
They demanded to hear directly from intelligence officials, rather than
from Mr. Trump's political appointees, but conceded they had not secured
a commitment for such a briefing.

Based on the intelligence they saw, the lawmakers said they were deeply
troubled by Mr. Trump's insistence he did not know about the plot and
his subsequent obfuscation when it became public.

``I find it inexplicable in light of these very public allegations that
the president hasn't come before the country and assured the American
people that he will get to the bottom of whether Russia is putting
bounties on American troops and that he will do everything in his power
to make sure that we protect American troops,'' said Representative Adam
B. Schiff, Democrat of California and the chairman of the House
Intelligence Committee.

He added: ``I do not understand for a moment why the president is not
saying this to the American people right now and is relying on `I don't
know,' `I haven't heard,' `I haven't been briefed.' That is just not
excusable.''

\includegraphics{https://static01.nyt.com/images/2020/06/30/us/politics/30vid-house/30vid-house-videoSixteenByNine3000.jpg}

Mr. Ratcliffe was scheduled to go to Capitol Hill on Wednesday to meet
privately with members of the Senate Intelligence Committee, an official
familiar with the planning said.

The Times
\href{https://www.nytimes.com/2020/06/26/us/politics/russia-afghanistan-bounties.html?action=click\&module=RelatedLinks\&pgtype=Article}{reported
last week} that intelligence officials believed that a unit of the
G.R.U. had offered and paid bounties for killing American troops and
other coalition forces and that the White House had not authorized a
response after the National Security Council convened an interagency
meeting about the problem in late March.

Investigators are said to be focused on at least two deadly attacks on
American soldiers in Afghanistan. One is an April 2019 bombing outside
Bagram Air Base that killed three Marines:
\href{https://www.nytimes.com/2019/04/09/nyregion/fdny-firefighter-killed-afghanistan.html}{Staff
Sgt. Christopher Slutman}, 43, of Newark, Del.; Cpl. Robert A. Hendriks,
25, of Locust Valley, N.Y.; and Sgt. Benjamin S. Hines, 31, of York, Pa.

On Monday, Felicia Arculeo, the mother of Corporal Hendriks, told
\href{https://www.cnbc.com/2020/06/29/mom-of-marine-killed-in-afghanistan-wants-russia-bounty-claim-investigated.html?__source=twitter\%7Cmain}{CNBC}
that she was upset to learn from news reports of the suspicions that her
son's death arose from a Russian bounty operation. She said she wanted
an investigation, adding that ``the parties who are responsible should
be held accountable, if that's even possible.''

Officials did not say which other attack was under scrutiny.

In claiming that the information was not provided to him, Mr. Trump has
also dismissed the intelligence assessment as
``\href{https://twitter.com/realdonaldtrump/status/1277202159109537793?lang=en}{so-called}''
and claimed he was told that it was
``\href{https://twitter.com/realdonaldtrump/status/1277431695248183298?lang=en}{not
credible}.'' The White House subsequently issued statements in the names
of several subordinates denying that he had been briefed.

Ms. McEnany reiterated that claim on Monday and said that the
information had not been elevated to Mr. Trump because there was a
dissenting view about it within the intelligence community.

But she and other administration officials demurred when pressed to say
whether their denials encompassed the president's daily written
briefing, a compendium of the most significant intelligence and analysis
that the intelligence community writes for presidents to read. Mr. Trump
is known to often neglect reading his written briefings.

Intelligence about the suspected Russian plot was included in the
President's Daily Brief
\href{https://www.nytimes.com/2020/06/29/us/politics/russian-bounty-trump.html}{in
late February}, according to two officials, contrasting
\href{https://twitter.com/realDonaldTrump/status/1277202159109537793}{Mr.
Trump's claim on Sunday} that he was never ``briefed or told'' about the
matter.

The information was also considered solid enough to be distributed to
the broader intelligence community in a May 4 article in the C.I.A.'s
World Intelligence Review, commonly called The Wire, according to
several officials.

A spokesman for the Taliban has denied that they accepted Russia-paid
bounties to carry out attacks on Americans and other coalition soldiers,
saying that the group needed no such encouragement for its operations.
But one American official said the focus had been on criminals closely
associated with the Taliban.

In a raid in Kunduz City in the north about six months ago, 13 people
were arrested in a joint operation by American forces and the Afghan
intelligence agency, the National Directorate of Security, according to
Safiullah Amiry, the deputy provincial council chief there. Two of the
main targets of the raid had already fled --- one to Tajikistan and one
to Russia, Mr. Amiry said --- but it was in the Kabul home of one of
them where security forces found a half-million dollars. He said the
Afghan intelligence agency had told him the raids were related to
Russian money being disbursed to militants.

Two former Afghan officials said Monday that members of local criminal
networks had carried out attacks for the Taliban in the past --- not
because they shared the Taliban's ideology or goals, but in exchange for
money.

In Parwan Province, where Bagram Air Base is, the Taliban are known to
have hired local criminals as freelancers, said Gen. Zaman Mamozai, the
former police chief of the province. He said the Taliban's commanders
are based in two districts of the province, Seyagird and Shinwari, and
that from there they coordinate a network that commissions criminals to
carry out attacks.

And Haseeba Efat, a former member of Parwan's provincial council, also
said the Taliban have hired freelancers in Bagram District ---
including, in one case, one of his own distant relatives.

``They agree with these criminals that they won't have monthly salary,
but they will get paid for the work they do when the Taliban need
them,'' Mr. Efat said.

Twenty American service members were killed in combat-related operations
in Afghanistan last year, the most since 2014.

Reporting was contributed by Fahim Abed, Najim Rahim, Helene Cooper and
Nicholas Fandos.

Advertisement

\protect\hyperlink{after-bottom}{Continue reading the main story}

\hypertarget{site-index}{%
\subsection{Site Index}\label{site-index}}

\hypertarget{site-information-navigation}{%
\subsection{Site Information
Navigation}\label{site-information-navigation}}

\begin{itemize}
\tightlist
\item
  \href{https://help.nytimes.com/hc/en-us/articles/115014792127-Copyright-notice}{©~2020~The
  New York Times Company}
\end{itemize}

\begin{itemize}
\tightlist
\item
  \href{https://www.nytco.com/}{NYTCo}
\item
  \href{https://help.nytimes.com/hc/en-us/articles/115015385887-Contact-Us}{Contact
  Us}
\item
  \href{https://www.nytco.com/careers/}{Work with us}
\item
  \href{https://nytmediakit.com/}{Advertise}
\item
  \href{http://www.tbrandstudio.com/}{T Brand Studio}
\item
  \href{https://www.nytimes.com/privacy/cookie-policy\#how-do-i-manage-trackers}{Your
  Ad Choices}
\item
  \href{https://www.nytimes.com/privacy}{Privacy}
\item
  \href{https://help.nytimes.com/hc/en-us/articles/115014893428-Terms-of-service}{Terms
  of Service}
\item
  \href{https://help.nytimes.com/hc/en-us/articles/115014893968-Terms-of-sale}{Terms
  of Sale}
\item
  \href{https://spiderbites.nytimes.com}{Site Map}
\item
  \href{https://help.nytimes.com/hc/en-us}{Help}
\item
  \href{https://www.nytimes.com/subscription?campaignId=37WXW}{Subscriptions}
\end{itemize}
