Sections

SEARCH

\protect\hyperlink{site-content}{Skip to
content}\protect\hyperlink{site-index}{Skip to site index}

\href{https://www.nytimes.com/section/world/asia}{Asia Pacific}

\href{https://myaccount.nytimes.com/auth/login?response_type=cookie\&client_id=vi}{}

\href{https://www.nytimes.com/section/todayspaper}{Today's Paper}

\href{/section/world/asia}{Asia Pacific}\textbar{}Brushing Aside
Opponents, Beijing Imposes Security Law on Hong Kong

\url{https://nyti.ms/2YLdyoH}

\begin{itemize}
\item
\item
\item
\item
\item
\item
\end{itemize}

Advertisement

\protect\hyperlink{after-top}{Continue reading the main story}

Supported by

\protect\hyperlink{after-sponsor}{Continue reading the main story}

News Analysis

\hypertarget{brushing-aside-opponents-beijing-imposes-security-law-on-hong-kong}{%
\section{Brushing Aside Opponents, Beijing Imposes Security Law on Hong
Kong}\label{brushing-aside-opponents-beijing-imposes-security-law-on-hong-kong}}

In passing the rules, China's leaders faced down the democracy movement
in Hong Kong and shrugged off opposition from the Trump administration.

\includegraphics{https://static01.nyt.com/images/2020/06/30/world/30hk-rules-analysis-1/merlin_174058368_3c3933ca-3a48-43fb-a677-a2920d8fa6f3-articleLarge.jpg?quality=75\&auto=webp\&disable=upscale}

\href{https://www.nytimes.com/by/chris-buckley}{\includegraphics{https://static01.nyt.com/images/2018/10/08/multimedia/author-chris-buckley/author-chris-buckley-thumbLarge.png}}\href{https://www.nytimes.com/by/keith-bradsher}{\includegraphics{https://static01.nyt.com/images/2018/10/08/multimedia/author-keith-bradsher/author-keith-bradsher-thumbLarge.png}}

By \href{https://www.nytimes.com/by/chris-buckley}{Chris Buckley} and
\href{https://www.nytimes.com/by/keith-bradsher}{Keith Bradsher}

\begin{itemize}
\item
  June 30, 2020
\item
  \begin{itemize}
  \item
  \item
  \item
  \item
  \item
  \item
  \end{itemize}
\end{itemize}

\href{https://cn.nytimes.com/china/20200701/china-critics-security-law-hong-kong/}{阅读简体中文版}\href{https://cn.nytimes.com/china/20200701/china-critics-security-law-hong-kong/zh-ha}{閱讀繁體中文版}

A year after protesters in Hong Kong jubilantly
\href{https://www.nytimes.com/2019/07/01/world/asia/china-hong-kong-protest.html}{defied
Chinese rule}, the national leader, Xi Jinping, has opened a long-term
counteroffensive in the territory, signing a sweeping
\href{https://www.nytimes.com/2020/06/29/world/asia/china-hong-kong-security-law-rules.html}{new
security law}on Tuesday that sets obedience to Beijing above the former
British colony's civil freedoms.

Conceived in secrecy and passed with intimidating speed, the law has
ignited uncertainty about the future of Hong Kong before any arrests
under its sweeping powers to quash political activity and speech that
challenge Beijing. Chinese officials and policy advisers have described
the security law as part of a ``second return'' for Hong Kong, one, they
suggest, that will scrub away a dangerous residue of Western influence
and liberal values.

The law
released\href{https://www.gld.gov.hk/egazette/pdf/20202444e/es220202444136.pdf}{to
the public} near midnight lays out new crimes for subverting the
government, seeking to ``split'' Hong Kong from China, or ``colluding''
with foreign governments or ``external forces'' to spy or gravely harm
China --- and authorizes life imprisonment for the most serious cases.

``Nobody should underestimate the determination of the central
authorities to defend national security in Hong Kong,'' the main Chinese
government office in the territory said in a
\href{http://hm.people.com.cn/n1/2020/0630/c42272-31765400.html}{statement}
after the rules were approved.

In imposing such expansive and drastic legislation, Communist Party
leaders in Beijing have faced down the pro-democracy movement in Hong
Kong. They have also shrugged off
\href{https://www.nytimes.com/2020/05/27/us/politics/china-hong-kong-pompeo-trade.html}{opposition
from the Trump administration} and other governments, showing Mr. Xi's
determination to
\href{https://www.nytimes.com/2019/10/31/world/asia/hong-kong-protests-china.html}{remake
the territory} on his authoritarian terms.

Some critics have described the law as a potentially fatal blow to the
``one country, two systems'' political framework that preserved Hong
Kong's distinctive status, freedoms and laws after China resumed
sovereignty in 1997. Even before taking effect, the law has created a
chill among the once-defiant activists who defined the protest movement.

``Hong Kong people understand this means the end of the `one country,
two systems' model for the territory, and we are now reduced to being a
city like on the mainland, like Shenzhen or Shanghai,'' said Joseph
Cheng, a longtime political scientist at City University of Hong Kong.
``We will have to behave like the people on the mainland.''

At the least, the new law complicates the delicate, often-convoluted
game that Hong Kong officials and judges have played since China took
back the territory. They have long tried to satisfy Beijing's demands
for loyalty while seeking to assure people in Hong Kong that the
territory's legal system remained insulated from politics, guarding
rights absent in mainland China.

That straddling act has become increasingly unsteady in recent years as
China has
\href{https://www.nytimes.com/2014/09/01/world/asia/hong-kong-elections.html}{applied
growing pressure}on the territory while protesters in Hong Kong have
\href{https://www.nytimes.com/2014/09/29/world/asia/clashes-in-hong-kong.html}{pressed
back}, demanding free elections and greater autonomy.

\includegraphics{https://static01.nyt.com/images/2020/06/30/world/30hk-rules-analysis-2/merlin_173473137_de6de108-5b5d-4be2-85d7-f94dbc5d7934-articleLarge.jpg?quality=75\&auto=webp\&disable=upscale}

Now the security law --- creating a murky realm of police agencies,
crimes defined by Beijing and judges picked by Hong Kong's pro-Beijing
leader --- is likely to make it harder to preserve the city's nebulous
status as a semiautonomous enclave under a Communist Party-run
superpower.

The law sets out plans to build a complex of agencies and offices in
Hong Kong dedicated to enforcing the rules. Those agencies will include
an arm of the Chinese national security apparatus that will have the
power to collect intelligence in Hong Kong, and handle cases when
central authorities decide that the local forces are not up to the job.

``It's the most fundamental change since the handover,'' said Danny
Gittings, an expert on Hong Kong's legal status. ``But that doesn't mean
that the changes will be immediately apparent.''

So far, many companies in Hong Kong
\href{https://www.nytimes.com/2020/06/30/business/china-hong-kong-security-law-business.html}{appear
confident} that commerce and contracts will remain largely untouched by
the law. Hong Kong officials have said that only a small number of
people would be targeted by the rules, and the territory is likely to
preserve some room for criticism of the Communist Party of the kind
forbidden inside mainland China.

``The law will not affect Hong Kong's renowned judicial independence,''
Carrie Lam, the chief executive of Hong Kong, who serves with Beijing's
blessing,
\href{https://www.info.gov.hk/gia/general/202006/30/P2020063000655.htm}{said
in a video speech} to the United Nations Human Rights Council on
Tuesday. ``It will not affect legitimate rights and freedoms of
individuals.''

Still, the law may bite faster and sharper than some expect, including
in education, where the party has warned against Western influence and
dissenting ideas that challenge official Chinese history and values. The
law cites schools as one of the targets for tighter control.

Hong Kong politicians loyal to Beijing and Chinese policy advisers have
called for the rules to be enforced swiftly and vigorously,
extinguishing any possible recurrence of the protests that hit Hong Kong
last year.

Four leading members of Demosisto, a youth opposition organization at
the forefront of protests last year, said on Tuesday that they had quit
the group, citing risks from the law. Two pro-independence groups, Hong
Kong National Front and Studentlocalism, said they would end activities
in the territory.

The law, which was approved unanimously by the Standing Committee of the
National People's Congress, an elite arm of the party-controlled
legislature, went into force an hour before the 23rd anniversary of Hong
Kong's handover to China. The July 1 anniversary has usually been a day
for big street protests in Hong Kong, which have been muted for months.

``If the new security law can succeed in doing what Beijing's rhetoric
anticipates, such protests will be a thing of the past,'' said Suzanne
Pepper, an independent political analyst who has long lived in Hong
Kong. ``Open political debate and dissent that Hong Kong has enjoyed for
the past 20 years will fade into self-censorship.''

Image

The central business district of Hong Kong.~So far, many companies in
Hong Kong appear confident that commerce and contracts will remain
largely untouched by the law.Credit...Lam Yik Fei for The New York Times

Under the law that defines Hong Kong's special status in China, the
territory's authorities were supposed to create their own national
security law. But successive leaders of Hong Kong never pushed the
legislation through, and Chinese leaders have said they had no choice
but to step in and impose a law.

The swift passage has reflected their fury at the pro-democracy protests
that shook Hong Kong for months last year, as well as their shifting
diagnosis of the causes.

During a recent seminar on the security legislation, Zhang Xiaoming, a
top Chinese official who helps oversee Hong Kong policy, suggested that
the territory's basic problem was that its citizens had not been
effectively immersed in party-blessed values, including acceptance that
Hong Kong is an integral part of China, with Beijing setting the terms.

Chinese officials had been saying that the discontent in Hong Kong had
economic roots, and the cure lay in cheaper housing and better jobs.

``Hong Kong's main problem is not an economic one,'' Mr. Zhang
\href{https://www.hmo.gov.cn/gab/bld/zxm/gzdt/202006/t20200608_21923.html}{said}in
the video speech in June. ``It's a political problem, and it's focused
on the fundamental question of what kind of Hong Kong we should build.''

Growing hostility between China and the United States has deepened
Communist Party leaders' worry about Western influence in Hong Kong.
Pro-Beijing politicians in Hong Kong and Chinese state media have
\href{https://www.nytimes.com/2014/10/31/world/asia/dan-garrett-hong-kong-protests.html}{described
protests in the territory} as the handiwork of Western intelligence
operatives trying to topple the Communist Party.

Image

A rally in support of the security law in Hong Kong on
Tuesday.Credit...Lam Yik Fei for The New York Times

``Beijing is particularly wary about Hong Kong being used by the U.S.
and some of its Western allies as a pawn to contain China's rise,'' said
Lau Siu-kai, a former senior Hong Kong government official who is now a
prominent adviser to Beijing.

The Trump administration did not even wait for the official law before
taking steps. The administration on Monday
\href{https://www.nytimes.com/2020/06/29/business/economy/us-halts-high-tech-exports-hong-kong.html}{extended
to Hong Kong} a ban it had long imposed on the sale to mainland China of
advanced technology with potential military applications.

``The United States will not stand idly by while China swallows Hong
Kong into its authoritarian maw,'' Secretary of State Mike Pompeo said
in a statement after the law passed.

A big test of the law lies in the Hong Kong courts, which have a long
tradition of independent decisions. But the law is wired with provisions
that appear designed to ward off attempts by courts and local lawmakers
to hem in its powers.

The Standing Committee of China's National People's Congress holds final
power over how to interpret the rules. And whenever the law comes into
conflict with Hong Kong laws and regulations, the local rules must give
way.

``Beijing will be able to exert influence at every key stage of
fortifying national security, both directly and indirectly through
personnel accountable to it,'' said
\href{https://www.law.hku.hk/academic_staff/cora-chan/}{Cora Chan}, an
associate professor of law at the University of Hong Kong who has
studied
\href{https://www.bloomsburyprofessional.com/uk/chinas-national-security-9781509928156/}{China's
drive for security legislation}.

Still, Hong Kong may not see a deluge of prosecutions under the new law.
In mainland China, the police and prosecutors charge people under state
security crimes relatively rarely, often preferring to imprison
dissidents and other political foes under other, less prominent charges,
such as fraud or stirring up trouble.

In the decade leading up to 2016, the last year for which detailed
statistics are available, Chinese courts finished cases on state
security charges against 8,640 defendants, according the Dui Hua
Foundation, a group based in San Francisco that monitors human rights in
China.

The great majority of the defendants in these mainland Chinese security
trials were members of ethnic minorities, mostly Uighurs and Tibetans,
convicted of promoting ethnic separatism, a broad charge that can be
used against anyone who questions Chinese rule, a forthcoming report
from the foundation shows.

Image

China's leader, Xi Jinping, center, at the National People's Congress in
Beijing last month. Mr. Xi signed off on the law Tuesday.Credit...Kevin
Frayer/Getty Images

Macau, a former Portuguese colony nearby that like Hong Kong is now a
special administrative region of China, adopted a somewhat similar
national security law 11 years ago but has yet to prosecute anyone under
it.

Even without prosecution, the sheer scope of the Hong Kong law may prove
a deterrent.

The new law ordered the Hong Kong government to ensure that media and
internet services adhere to national security priorities, a demand that
could cut into the territory's lively undergrowth of independent civic
groups and news outlets. The law also polices people beyond the borders
of the territory.

The law ``confirmed the central government's comprehensive control over
the security sphere in Hong Kong,'' said Tian Feilong, an associate
professor of law at Beihang University in Beijing who specializes in the
territory's legal system. ``Hong Kong society will have to make major
adjustments in political and cultural life as this law takes effect.''

Advertisement

\protect\hyperlink{after-bottom}{Continue reading the main story}

\hypertarget{site-index}{%
\subsection{Site Index}\label{site-index}}

\hypertarget{site-information-navigation}{%
\subsection{Site Information
Navigation}\label{site-information-navigation}}

\begin{itemize}
\tightlist
\item
  \href{https://help.nytimes.com/hc/en-us/articles/115014792127-Copyright-notice}{©~2020~The
  New York Times Company}
\end{itemize}

\begin{itemize}
\tightlist
\item
  \href{https://www.nytco.com/}{NYTCo}
\item
  \href{https://help.nytimes.com/hc/en-us/articles/115015385887-Contact-Us}{Contact
  Us}
\item
  \href{https://www.nytco.com/careers/}{Work with us}
\item
  \href{https://nytmediakit.com/}{Advertise}
\item
  \href{http://www.tbrandstudio.com/}{T Brand Studio}
\item
  \href{https://www.nytimes.com/privacy/cookie-policy\#how-do-i-manage-trackers}{Your
  Ad Choices}
\item
  \href{https://www.nytimes.com/privacy}{Privacy}
\item
  \href{https://help.nytimes.com/hc/en-us/articles/115014893428-Terms-of-service}{Terms
  of Service}
\item
  \href{https://help.nytimes.com/hc/en-us/articles/115014893968-Terms-of-sale}{Terms
  of Sale}
\item
  \href{https://spiderbites.nytimes.com}{Site Map}
\item
  \href{https://help.nytimes.com/hc/en-us}{Help}
\item
  \href{https://www.nytimes.com/subscription?campaignId=37WXW}{Subscriptions}
\end{itemize}
