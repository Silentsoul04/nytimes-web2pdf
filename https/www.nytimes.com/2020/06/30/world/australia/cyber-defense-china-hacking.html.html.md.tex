Sections

SEARCH

\protect\hyperlink{site-content}{Skip to
content}\protect\hyperlink{site-index}{Skip to site index}

\href{https://www.nytimes.com/section/world/australia}{Australia}

\href{https://myaccount.nytimes.com/auth/login?response_type=cookie\&client_id=vi}{}

\href{https://www.nytimes.com/section/todayspaper}{Today's Paper}

\href{/section/world/australia}{Australia}\textbar{}Australia Spending
Nearly \$1 Billion on Cyberdefense as China Tensions Rise

\url{https://nyti.ms/2YJR87n}

\begin{itemize}
\item
\item
\item
\item
\item
\end{itemize}

Advertisement

\protect\hyperlink{after-top}{Continue reading the main story}

Supported by

\protect\hyperlink{after-sponsor}{Continue reading the main story}

\hypertarget{australia-spending-nearly-1-billion-on-cyberdefense-as-china-tensions-rise}{%
\section{Australia Spending Nearly \$1 Billion on Cyberdefense as China
Tensions
Rise}\label{australia-spending-nearly-1-billion-on-cyberdefense-as-china-tensions-rise}}

Officials promised to recruit at least 500 cyberspies and build on the
country's offensive capabilities to take the online battle overseas.

\includegraphics{https://static01.nyt.com/images/2020/06/30/world/30oz-cyber-1/merlin_172511202_fa9bc323-92eb-46a8-8176-39e51ec38ce8-articleLarge.jpg?quality=75\&auto=webp\&disable=upscale}

\href{https://www.nytimes.com/by/damien-cave}{\includegraphics{https://static01.nyt.com/images/2018/10/08/multimedia/author-damien-cave/author-damien-cave-thumbLarge.png}}

By \href{https://www.nytimes.com/by/damien-cave}{Damien Cave}

\begin{itemize}
\item
  June 30, 2020
\item
  \begin{itemize}
  \item
  \item
  \item
  \item
  \item
  \end{itemize}
\end{itemize}

SYDNEY, Australia --- Confronting a surge of cyberattacks attributed to
the Chinese government, Australia moved to bolster its defenses on
Tuesday, promising to recruit at least 500 cyberspies and build on its
ability to take the battle overseas.

The investment of 1.35 billion Australian dollars (\$930 million) over
the next decade is the largest the country has ever made in cyberweapons
and defenses.

It follows what Prime Minister Scott Morrison has described as a sharp
increase in the frequency, scale and sophistication of online attacks
--- and, more broadly, a
\href{https://www.nytimes.com/2020/06/26/world/australia/politician-home-raid-china-influence.html}{steady
deterioration in relations} between Australia and China.

``The federal government's top priority is protecting our nation's
economy, national security and sovereignty,'' Mr. Morrison said Tuesday.
``Malicious cyberactivity undermines that.''

The new initiative points to growing frustration in Australia with what
current and former intelligence officials have described as a
relentless, increasingly
\href{https://www.nytimes.com/2019/05/20/world/australia/australia-china.html}{aggressive
campaign by China} to spy on, disrupt and threaten the country's
government, vital infrastructure and most important industries.

The full details of attacks that appear to have come from China are
still mostly hidden --- Australian officials remain wary of provoking
Beijing by naming and shaming culprits --- but the public record now
includes several examples of elaborate hacking that has less to do with
theft for profit than growing aggression against a rival government.

In January of last year, for example, hackers
\href{https://www.nytimes.com/2019/02/07/world/australia/cyberattack-parliament-hack.html}{found
their way into the computer systems} of the Australian Parliament. A
year before that, security experts said that tools commonly used by
Chinese hackers had been deployed in attacks on
\href{https://www.smh.com.au/world/north-america/tens-of-thousands-of-australian-firms-could-be-affected-by-chinese-hack-top-official-20181221-p50nl0.html}{Australia's
Defense Department} and
\href{https://www.smh.com.au/politics/federal/chinese-hackers-breach-anu-putting-national-security-at-risk-20180706-p4zq0q.html}{the
Australian National University}.

Two weeks ago, Australian officials said a wide range of political and
private-sector organizations had come under attack by a ``sophisticated
state-based cyberactor'' --- a reference that most cybersecurity experts
took to mean China.

And there are hints that the tools being deployed are increasingly
ambitious and dangerous.

In one attack earlier this year, hackers used a compromised email
account from the Indonesian Embassy in Australia to send a Word document
to a staff member in the office of the top leader in the state of
Western Australia.

The attachment contained an invisible cyberattack tool called Aria-body,
which had never been detected before and had alarming new capabilities.
It allowed hackers to remotely take over a computer, to copy, delete or
create files, and to carry out extensive searches of the device.

A cybersecurity company in Israel
\href{https://www.nytimes.com/2020/05/07/world/asia/china-hacking-military-aria.html}{later
linked} Aria-body to a group of hackers, called Naikon, that has been
traced to the Chinese military.

Peter Jennings, a former defense and intelligence official who heads the
Australian Strategic Policy Institute, said Beijing had leapfrogged
other countries in its cyberabilities and the frequency of its attacks.

``It's just reaching unprecedented heights of activity,'' he said.
``Yes, it's true countries do spy on each other; the problem here is the
all-pervasive nature of what China is doing. In many ways, big and
small, there are hints of bullying and coercion.''

The attacks, while constant, have become more troublesome since
Australia angered China by
\href{https://www.nytimes.com/2020/05/11/world/australia/coronavirus-china-inquiry.html}{calling
for an international inquiry} into the roots of the coronavirus
outbreak. In Beijing, any questioning of
\href{https://www.nytimes.com/2020/04/08/world/asia/coronavirus-china-narrative.html}{the
official narrative} that China defeated the virus as quickly as possible
is seen as an insult.

The rising tensions between the two countries have already affected
trade --- with China cutting imports of barley and beef --- and neither
country has made a public effort to reconcile. China has also tried to
turn the cyberspying accusations
\href{https://www.abc.net.au/news/2020-06-29/chinese-state-media-says-australia-is-stepping-up-spy-activities/12402032?utm_source=sfmc\&utm_medium=email\&utm_content=\&utm_campaign=\%5bnews_sfmc_newsmail_pm_df_!n1\%5d\%3a8935\&user_id=fd273ea9ec95db56c7d3affcb610d4d0e8fcb4982c0e24bcb107bc2167382d75\&WT.tsrc=email\&WT.mc_id=Email\%7c\%5bnews_sfmc_newsmail_pm_df_!n1\%5d\%7c8935ABCNewsmail_topstories_articlelink}{back
on Australia}, with its state media claiming that Beijing disrupted an
Australian operation two years ago.

The response on the cyberfront that Australia outlined on Tuesday starts
with personnel. Roughly a third of the funding will go toward hiring
hundreds of cybersecurity experts to study and share information about
the evolution of emerging threats, and to create countermeasures of
their own.

The Australian Signals Directorate and the Australian Cyber Security
Center will build up their capacity to defend against attacks and their
connections with the companies that run the country's digital networks.

The defense minister, Senator Linda Reynolds, said in a statement that
the investment aimed to create a rapid-response process that would
``prevent malicious cyberactivity from reaching millions of Australians
by blocking known malicious websites and computer viruses at speed.''

Mr. Jennings said the investment was substantial and needed. He added
that it would most likely be a down payment.

``The need for more investment in cybersecurity, both defense and
offense, will keep growing,'' he said. ``This won't be the last
investment, I'm sure.''

Advertisement

\protect\hyperlink{after-bottom}{Continue reading the main story}

\hypertarget{site-index}{%
\subsection{Site Index}\label{site-index}}

\hypertarget{site-information-navigation}{%
\subsection{Site Information
Navigation}\label{site-information-navigation}}

\begin{itemize}
\tightlist
\item
  \href{https://help.nytimes.com/hc/en-us/articles/115014792127-Copyright-notice}{©~2020~The
  New York Times Company}
\end{itemize}

\begin{itemize}
\tightlist
\item
  \href{https://www.nytco.com/}{NYTCo}
\item
  \href{https://help.nytimes.com/hc/en-us/articles/115015385887-Contact-Us}{Contact
  Us}
\item
  \href{https://www.nytco.com/careers/}{Work with us}
\item
  \href{https://nytmediakit.com/}{Advertise}
\item
  \href{http://www.tbrandstudio.com/}{T Brand Studio}
\item
  \href{https://www.nytimes.com/privacy/cookie-policy\#how-do-i-manage-trackers}{Your
  Ad Choices}
\item
  \href{https://www.nytimes.com/privacy}{Privacy}
\item
  \href{https://help.nytimes.com/hc/en-us/articles/115014893428-Terms-of-service}{Terms
  of Service}
\item
  \href{https://help.nytimes.com/hc/en-us/articles/115014893968-Terms-of-sale}{Terms
  of Sale}
\item
  \href{https://spiderbites.nytimes.com}{Site Map}
\item
  \href{https://help.nytimes.com/hc/en-us}{Help}
\item
  \href{https://www.nytimes.com/subscription?campaignId=37WXW}{Subscriptions}
\end{itemize}
