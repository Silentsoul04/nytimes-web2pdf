Sections

SEARCH

\protect\hyperlink{site-content}{Skip to
content}\protect\hyperlink{site-index}{Skip to site index}

\href{https://www.nytimes.com/section/health}{Health}

\href{https://myaccount.nytimes.com/auth/login?response_type=cookie\&client_id=vi}{}

\href{https://www.nytimes.com/section/todayspaper}{Today's Paper}

\href{/section/health}{Health}\textbar{}Here's What Recovery From
Covid-19 Looks Like for Many Survivors

\url{https://nyti.ms/2BYknKw}

\begin{itemize}
\item
\item
\item
\item
\item
\end{itemize}

\href{https://www.nytimes.com/news-event/coronavirus?action=click\&pgtype=Article\&state=default\&region=TOP_BANNER\&context=storylines_menu}{The
Coronavirus Outbreak}

\begin{itemize}
\tightlist
\item
  live\href{https://www.nytimes.com/2020/08/04/world/coronavirus-covid-19.html?action=click\&pgtype=Article\&state=default\&region=TOP_BANNER\&context=storylines_menu}{Latest
  Updates}
\item
  \href{https://www.nytimes.com/interactive/2020/us/coronavirus-us-cases.html?action=click\&pgtype=Article\&state=default\&region=TOP_BANNER\&context=storylines_menu}{Maps
  and Cases}
\item
  \href{https://www.nytimes.com/interactive/2020/science/coronavirus-vaccine-tracker.html?action=click\&pgtype=Article\&state=default\&region=TOP_BANNER\&context=storylines_menu}{Vaccine
  Tracker}
\item
  \href{https://www.nytimes.com/2020/08/02/us/covid-college-reopening.html?action=click\&pgtype=Article\&state=default\&region=TOP_BANNER\&context=storylines_menu}{College
  Reopening}
\item
  \href{https://www.nytimes.com/live/2020/08/03/business/stock-market-today-coronavirus?action=click\&pgtype=Article\&state=default\&region=TOP_BANNER\&context=storylines_menu}{Economy}
\end{itemize}

Advertisement

\protect\hyperlink{after-top}{Continue reading the main story}

Supported by

\protect\hyperlink{after-sponsor}{Continue reading the main story}

\hypertarget{heres-what-recovery-from-covid-19-looks-like-for-many-survivors}{%
\section{Here's What Recovery From Covid-19 Looks Like for Many
Survivors}\label{heres-what-recovery-from-covid-19-looks-like-for-many-survivors}}

Continuing shortness of breath, muscle weakness, flashbacks, mental
fogginess and other symptoms may plague patients for a long time.

\includegraphics{https://static01.nyt.com/images/2020/07/01/health/01VIRUS-RECOVERY-EXPLAINER/01VIRUS-RECOVERY-EXPLAINER-articleLarge-v2.jpg?quality=75\&auto=webp\&disable=upscale}

\href{https://www.nytimes.com/by/pam-belluck}{\includegraphics{https://static01.nyt.com/images/2018/02/16/multimedia/author-pam-belluck/author-pam-belluck-thumbLarge-v2.png}}

By \href{https://www.nytimes.com/by/pam-belluck}{Pam Belluck}

\begin{itemize}
\item
  July 1, 2020
\item
  \begin{itemize}
  \item
  \item
  \item
  \item
  \item
  \end{itemize}
\end{itemize}

\href{https://www.nytimes.com/es/2020/07/02/espanol/ciencia-y-tecnologia/sobrevivientes-coronavirus-recuperacion.html}{Leer
en español}

Hundreds of thousands of seriously ill
\href{https://www.nytimes.com/2020/07/22/health/coronavirus-isolation-testing.html}{coronavirus}
patients who survive and leave the hospital are facing a new and
difficult challenge: recovery. Many are struggling to overcome a range
of troubling residual symptoms, and some problems may persist for
months, years or even the rest of their lives.

Patients who are returning home after being hospitalized for severe
respiratory failure from the virus are confronting physical,
neurological, cognitive and emotional issues.

And they must navigate their recovery process as the pandemic continues,
with all of the stresses and stretched resources that it has brought.

``It's not just, `Oh, I had a terrible time in hospital, but thank
goodness I'm home and everything's back to normal,''' said Dr. David
Putrino, director of rehabilitation innovation at Mount Sinai Health
System in New York City. ``It's, `I just had a terrible time in hospital
and guess what? The world is still burning. I need to address that while
also trying to sort of catch up to what my old life used to be.'''

It is still too early to say how recovery will play out for these
patients. But here is a look at what they are experiencing so far, what
we can learn from former patients with similar medical experiences, and
the challenges that most likely lie ahead.

\hypertarget{what-problems-do-patients-experience-after-leaving-the-hospital}{%
\subsection{What problems do patients experience after leaving the
hospital?}\label{what-problems-do-patients-experience-after-leaving-the-hospital}}

There are many. Patients may leave the hospital with scarring, damage or
inflammation that still needs to heal in the lungs, heart, kidneys,
liver or other organs. This can cause a range of problems, including
urinary and metabolism issues.

Dr. Zijian Chen, the medical director of the new Center for Post-Covid
Care at Mount Sinai Health System, said the biggest physical problem the
center was seeing was shortness of breath, which can be the result of
lung or heart impairments or a blood-clotting problem.

``Some have an intermittent cough that doesn't go away that makes it
hard for them to breathe,'' he said. Some are even on nasal oxygen at
home, but it is not helping them enough.

Some patients who were on ventilators report difficulty swallowing or
speaking above a whisper, a usually temporary result of bruising or
inflammation from a breathing tube that passes through the vocal cords.

Many patients experience muscle weakness after lying in a hospital bed
for so long, said Dr. Dale Needham, a critical care physician at Johns
Hopkins School of Medicine and a leader in the field of intensive care
recovery. As a result, they can have trouble walking, climbing stairs or
lifting objects.

\hypertarget{latest-updates-global-coronavirus-outbreak}{%
\section{\texorpdfstring{\href{https://www.nytimes.com/2020/08/04/world/coronavirus-covid-19.html?action=click\&pgtype=Article\&state=default\&region=MAIN_CONTENT_1\&context=storylines_live_updates}{Latest
Updates: Global Coronavirus
Outbreak}}{Latest Updates: Global Coronavirus Outbreak}}\label{latest-updates-global-coronavirus-outbreak}}

Updated 2020-08-04T09:59:19.194Z

\begin{itemize}
\tightlist
\item
  \href{https://www.nytimes.com/2020/08/04/world/coronavirus-covid-19.html?action=click\&pgtype=Article\&state=default\&region=MAIN_CONTENT_1\&context=storylines_live_updates\#link-6b644638}{`Long
  days, long nights': Washington prepares for a prolonged fight over
  virus relief.}
\item
  \href{https://www.nytimes.com/2020/08/04/world/coronavirus-covid-19.html?action=click\&pgtype=Article\&state=default\&region=MAIN_CONTENT_1\&context=storylines_live_updates\#link-7af9fca0}{Israel's
  rocky reopening of its schools may be a lesson for the U.S.}
\item
  \href{https://www.nytimes.com/2020/08/04/world/coronavirus-covid-19.html?action=click\&pgtype=Article\&state=default\&region=MAIN_CONTENT_1\&context=storylines_live_updates\#link-33bf9168}{Hurricane
  Isaias arrives in North Carolina as officials along the East Coast
  scramble.}
\end{itemize}

\href{https://www.nytimes.com/2020/08/04/world/coronavirus-covid-19.html?action=click\&pgtype=Article\&state=default\&region=MAIN_CONTENT_1\&context=storylines_live_updates}{See
more updates}

More live coverage:
\href{https://www.nytimes.com/live/2020/08/03/business/stock-market-today-coronavirus?action=click\&pgtype=Article\&state=default\&region=MAIN_CONTENT_1\&context=storylines_live_updates}{Markets}

Nerve damage or weakness can also whittle away muscle strength, Dr.
Needham said. Neurological problems can cause other symptoms, too. Dr.
Chen said that Mount Sinai's post-Covid center has referred nearly 40
percent of patients to neurologists for issues like fatigue, confusion
and mental fogginess.

``Some of it is very debilitating,'' he said. ``We have patients who
come in and tell us: `I can't concentrate on work. I've recovered, I
don't have any breathing problems, I don't have chest pain, but I can't
get back to work because I can't concentrate.'''

The center also refers some of these patients for psychological
consults, Dr. Chen said.

``It's really common for patients to have PTSD after going through this
--- nightmares, depression and anxiety because they're having flashbacks
and remembering what happened,'' said Dr. Lauren Ferrante, a pulmonary
and critical care physician at Yale School of Medicine who studies
post-I.C.U. recovery outcomes.

Emotional issues may be heightened for Covid-19 patients because of
their days spent hospitalized without visits from family and friends,
experts say.

``This experience of being extremely sick and extremely alone really
amplifies the trauma,'' said Dr. Putrino, adding that many patients were
contacting his program to ask for telemedicine psychology services.
``They're saying, `Listen, I'm not really myself and I need to speak
with someone.'''

To describe the wide variety of recovery challenges, experts often use
an umbrella term, coined about a decade ago:
\href{https://www.sccm.org/MyICUCare/THRIVE/Post-intensive-Care-Syndrome\#:~:text=Post\%2Dintensive\%20care\%20syndrome\%2C\%20or\%20PICS\%2C\%20is\%20made\%20up,and\%20may\%20affect\%20the\%20family.}{post-intensive
care syndrome} or PICS, which can include any of the physical, cognitive
and emotional symptoms patients encounter.

\hypertarget{what-makes-someone-more-likely-to-face-recovery-challenges}{%
\subsection{What makes someone more likely to face recovery
challenges?}\label{what-makes-someone-more-likely-to-face-recovery-challenges}}

Studies of people hospitalized for respiratory failure from other causes
suggest recovery is more likely to be harder for people who were
\href{https://www.ncbi.nlm.nih.gov/pmc/articles/PMC6026287/}{frail
beforehand} and for people who needed longer hospitalizations, Dr.
Ferrante said.

But many other coronavirus patients --- not just those who are older or
who have other medical conditions --- are spending weeks on ventilators
and weeks more in the hospital after their breathing tubes are removed,
making their recovery hills steeper to climb.

``You have prolonged lengths of stay on a ventilator and in the I.C.U.
that are now longer than we've ever seen before,'' Dr. Ferrante said.
``One worries that this is going to have repercussions for physical
function and that we'll see more people not recovering.''

Another factor that can extend or hamper recovery is a phenomenon called
hospital delirium,
\href{https://www.nytimes.com/2020/06/28/health/coronavirus-delirium-hallucinations.html\#commentsContainer}{a
condition that can involve paranoid hallucinations and anxious
confusion}. It is more likely to occur in patients who undergo prolonged
sedation, have limited social interaction and are unable to move around
--- all common among Covid-19 patients.

Studies, including one by a team at Vanderbilt University Medical
Center, have found that I.C.U. patients
\href{https://www.nejm.org/doi/full/10.1056/NEJMoa1301372}{who
experience hospital delirium are more likely to have cognitive
impairment} in the months after they leave the hospital.

\hypertarget{what-is-the-trajectory-of-recovery}{%
\subsection{What is the trajectory of
recovery?}\label{what-is-the-trajectory-of-recovery}}

Ups and downs are common. ``It's absolutely not a linear process, and
it's very individualized,'' Dr. Needham said.

Perseverance is important. ``What we don't want is for patients to go
home and lie in bed all day,'' Dr. Ferrante said. ``That will not help
with recovery and will probably make things worse.''

Patients and their families should realize that fluctuations in progress
are normal.

``There are going to be days where everything's going right with your
lungs, but your joints are feeling so achy that you can't get up and do
your pulmonary rehab and you have a few setbacks,'' Dr. Putrino said.
``Or your pulmonary care is going OK, but your cognitive fog is causing
you to have anxiety and causing you to spiral, so you need to drop
everything and work with your neuropsychologist intensively.''

``It really does feel like one step forward, two steps back,'' he added,
``and that's OK.''

\hypertarget{how-long-do-these-issues-last}{%
\subsection{How long do these issues
last?}\label{how-long-do-these-issues-last}}

For many people, the lungs are likely to recover, often within months.
But other problems can linger and some people may never make a full
recovery, experts say.

\href{https://www.nytimes.com/news-event/coronavirus?action=click\&pgtype=Article\&state=default\&region=MAIN_CONTENT_3\&context=storylines_faq}{}

\hypertarget{the-coronavirus-outbreak-}{%
\subsubsection{The Coronavirus Outbreak
›}\label{the-coronavirus-outbreak-}}

\hypertarget{frequently-asked-questions}{%
\paragraph{Frequently Asked
Questions}\label{frequently-asked-questions}}

Updated August 3, 2020

\begin{itemize}
\item ~
  \hypertarget{im-a-small-business-owner-can-i-get-relief}{%
  \paragraph{I'm a small-business owner. Can I get
  relief?}\label{im-a-small-business-owner-can-i-get-relief}}

  \begin{itemize}
  \tightlist
  \item
    The
    \href{https://www.nytimes.com/article/small-business-loans-stimulus-grants-freelancers-coronavirus.html?action=click\&pgtype=Article\&state=default\&region=MAIN_CONTENT_3\&context=storylines_faq}{stimulus
    bills enacted in March} offer help for the millions of American
    small businesses. Those eligible for aid are businesses and
    nonprofit organizations with fewer than 500 workers, including sole
    proprietorships, independent contractors and freelancers. Some
    larger companies in some industries are also eligible. The help
    being offered, which is being managed by the Small Business
    Administration, includes the Paycheck Protection Program and the
    Economic Injury Disaster Loan program. But lots of folks have
    \href{https://www.nytimes.com/interactive/2020/05/07/business/small-business-loans-coronavirus.html?action=click\&pgtype=Article\&state=default\&region=MAIN_CONTENT_3\&context=storylines_faq}{not
    yet seen payouts.} Even those who have received help are confused:
    The rules are draconian, and some are stuck sitting on
    \href{https://www.nytimes.com/2020/05/02/business/economy/loans-coronavirus-small-business.html?action=click\&pgtype=Article\&state=default\&region=MAIN_CONTENT_3\&context=storylines_faq}{money
    they don't know how to use.} Many small-business owners are getting
    less than they expected or
    \href{https://www.nytimes.com/2020/06/10/business/Small-business-loans-ppp.html?action=click\&pgtype=Article\&state=default\&region=MAIN_CONTENT_3\&context=storylines_faq}{not
    hearing anything at all.}
  \end{itemize}
\item ~
  \hypertarget{what-are-my-rights-if-i-am-worried-about-going-back-to-work}{%
  \paragraph{What are my rights if I am worried about going back to
  work?}\label{what-are-my-rights-if-i-am-worried-about-going-back-to-work}}

  \begin{itemize}
  \tightlist
  \item
    Employers have to provide
    \href{https://www.osha.gov/SLTC/covid-19/standards.html}{a safe
    workplace} with policies that protect everyone equally.
    \href{https://www.nytimes.com/article/coronavirus-money-unemployment.html?action=click\&pgtype=Article\&state=default\&region=MAIN_CONTENT_3\&context=storylines_faq}{And
    if one of your co-workers tests positive for the coronavirus, the
    C.D.C.} has said that
    \href{https://www.cdc.gov/coronavirus/2019-ncov/community/guidance-business-response.html}{employers
    should tell their employees} -\/- without giving you the sick
    employee's name -\/- that they may have been exposed to the virus.
  \end{itemize}
\item ~
  \hypertarget{should-i-refinance-my-mortgage}{%
  \paragraph{Should I refinance my
  mortgage?}\label{should-i-refinance-my-mortgage}}

  \begin{itemize}
  \tightlist
  \item
    \href{https://www.nytimes.com/article/coronavirus-money-unemployment.html?action=click\&pgtype=Article\&state=default\&region=MAIN_CONTENT_3\&context=storylines_faq}{It
    could be a good idea,} because mortgage rates have
    \href{https://www.nytimes.com/2020/07/16/business/mortgage-rates-below-3-percent.html?action=click\&pgtype=Article\&state=default\&region=MAIN_CONTENT_3\&context=storylines_faq}{never
    been lower.} Refinancing requests have pushed mortgage applications
    to some of the highest levels since 2008, so be prepared to get in
    line. But defaults are also up, so if you're thinking about buying a
    home, be aware that some lenders have tightened their standards.
  \end{itemize}
\item ~
  \hypertarget{what-is-school-going-to-look-like-in-september}{%
  \paragraph{What is school going to look like in
  September?}\label{what-is-school-going-to-look-like-in-september}}

  \begin{itemize}
  \tightlist
  \item
    It is unlikely that many schools will return to a normal schedule
    this fall, requiring the grind of
    \href{https://www.nytimes.com/2020/06/05/us/coronavirus-education-lost-learning.html?action=click\&pgtype=Article\&state=default\&region=MAIN_CONTENT_3\&context=storylines_faq}{online
    learning},
    \href{https://www.nytimes.com/2020/05/29/us/coronavirus-child-care-centers.html?action=click\&pgtype=Article\&state=default\&region=MAIN_CONTENT_3\&context=storylines_faq}{makeshift
    child care} and
    \href{https://www.nytimes.com/2020/06/03/business/economy/coronavirus-working-women.html?action=click\&pgtype=Article\&state=default\&region=MAIN_CONTENT_3\&context=storylines_faq}{stunted
    workdays} to continue. California's two largest public school
    districts --- Los Angeles and San Diego --- said on July 13, that
    \href{https://www.nytimes.com/2020/07/13/us/lausd-san-diego-school-reopening.html?action=click\&pgtype=Article\&state=default\&region=MAIN_CONTENT_3\&context=storylines_faq}{instruction
    will be remote-only in the fall}, citing concerns that surging
    coronavirus infections in their areas pose too dire a risk for
    students and teachers. Together, the two districts enroll some
    825,000 students. They are the largest in the country so far to
    abandon plans for even a partial physical return to classrooms when
    they reopen in August. For other districts, the solution won't be an
    all-or-nothing approach.
    \href{https://bioethics.jhu.edu/research-and-outreach/projects/eschool-initiative/school-policy-tracker/}{Many
    systems}, including the nation's largest, New York City, are
    devising
    \href{https://www.nytimes.com/2020/06/26/us/coronavirus-schools-reopen-fall.html?action=click\&pgtype=Article\&state=default\&region=MAIN_CONTENT_3\&context=storylines_faq}{hybrid
    plans} that involve spending some days in classrooms and other days
    online. There's no national policy on this yet, so check with your
    municipal school system regularly to see what is happening in your
    community.
  \end{itemize}
\item ~
  \hypertarget{is-the-coronavirus-airborne}{%
  \paragraph{Is the coronavirus
  airborne?}\label{is-the-coronavirus-airborne}}

  \begin{itemize}
  \tightlist
  \item
    The coronavirus
    \href{https://www.nytimes.com/2020/07/04/health/239-experts-with-one-big-claim-the-coronavirus-is-airborne.html?action=click\&pgtype=Article\&state=default\&region=MAIN_CONTENT_3\&context=storylines_faq}{can
    stay aloft for hours in tiny droplets in stagnant air}, infecting
    people as they inhale, mounting scientific evidence suggests. This
    risk is highest in crowded indoor spaces with poor ventilation, and
    may help explain super-spreading events reported in meatpacking
    plants, churches and restaurants.
    \href{https://www.nytimes.com/2020/07/06/health/coronavirus-airborne-aerosols.html?action=click\&pgtype=Article\&state=default\&region=MAIN_CONTENT_3\&context=storylines_faq}{It's
    unclear how often the virus is spread} via these tiny droplets, or
    aerosols, compared with larger droplets that are expelled when a
    sick person coughs or sneezes, or transmitted through contact with
    contaminated surfaces, said Linsey Marr, an aerosol expert at
    Virginia Tech. Aerosols are released even when a person without
    symptoms exhales, talks or sings, according to Dr. Marr and more
    than 200 other experts, who
    \href{https://academic.oup.com/cid/article/doi/10.1093/cid/ciaa939/5867798}{have
    outlined the evidence in an open letter to the World Health
    Organization}.
  \end{itemize}
\end{itemize}

One benchmark is a
\href{https://www.nejm.org/doi/full/10.1056/nejmoa1011802}{2011 New
England Journal of Medicine study} of 109 patients in Canada who had
been treated for acute respiratory distress syndrome, or ARDS, the kind
of lung failure that afflicts many Covid-19 patients. Five years later,
most had regained normal or near-normal lung function but still
struggled with persistent physical and emotional issues.

On one crucial test --- how far patients could walk in six minutes ---
their median distance was about 477 yards, only three-quarters of the
distance researchers had predicted. The patients ranged in age from 35
to 57, and while younger patients had a greater rate of physical
recovery than older patients, ``neither group returned to normal
predicted levels of physical function at five years,'' the authors
wrote.

The patients in the study had ARDS from a variety of causes, including
pneumonia, sepsis, pancreatitis or burns. They had a median stay of 49
days in the hospital, including 26 days in the I.C.U. and 24 days on a
ventilator.

\href{https://pubmed.ncbi.nlm.nih.gov/27637716/}{Research led by Dr.
Needham} of Johns Hopkins found that ``patients have prolonged muscle
weakness that lasts months or longer and that muscle weakness is not
just limited to their arms and legs --- it's also their breathing
muscles,'' he said.

Another \href{https://pubmed.ncbi.nlm.nih.gov/32304774/}{study by Dr.
Needham and his colleagues} found that about two-thirds of ARDS patients
had significant fatigue a year later.

Psychological and cognitive symptoms can also linger. About half of the
patients in the 2011 Canadian study reported at least one episode of
``physician-diagnosed depression, anxiety, or both between two and five
years of follow-up.'' And a study of patients treated in the 2003
outbreak of SARS, another type of coronavirus, found that a year later
many had ``\href{https://pubmed.ncbi.nlm.nih.gov/17500304/}{worrying
levels of depression, anxiety, and post-traumatic symptoms}.''

\hypertarget{what-are-the-consequences}{%
\subsection{What are the
consequences?}\label{what-are-the-consequences}}

Among other things, patients may have trouble going back to their jobs.
A team led by Dr. Needham found that nearly one-third of 64 ARDS
patients they followed for five years
\href{https://www.ncbi.nlm.nih.gov/pmc/articles/PMC6002952/}{never
returned to work}.

Some tried but found that they couldn't do their jobs and stopped
working altogether, Dr. Needham said, and others ``had to change their
occupation, specifically for a job that's less challenging and probably
less pay.''

Dr. Chen said he was worried that the long-term consequences of Covid-19
could resemble the chronic health effects of the AIDS epidemic or the
Sept. 11 attack on New York City. ``A new disease that's severe or a
catastrophic event causes symptoms that last a long time,'' he said.
``This is shaping up to be something that may be worse than both of
those.''

There may be ``hundreds of thousands who are going to be afflicted with
these chronic syndromes that may take a long time to heal, and that's
going to be a very big health problem and also a big economic problem if
we don't take care of them,'' Dr. Chen said.

\hypertarget{what-are-hospitals-doing-to-help-patients-when-they-go-home}{%
\subsection{What are hospitals doing to help patients when they go
home?}\label{what-are-hospitals-doing-to-help-patients-when-they-go-home}}

Recovery programs for Covid-19 patients are cropping up at Mount Sinai,
Yale, Johns Hopkins and elsewhere, offering patients telemedicine
consultations and sometimes in-person appointments.

Some patients require medication to help with shortness of breath, heart
problems or blood clotting. Dr. Ferrante said people should check
medications with their doctors because some medicines they were given in
the hospital may not be appropriate for patients to continue at home.

But medication may not be necessary, or may not work, for many issues.
Practicing breathing exercises and using a spirometer, a device that
measures how much air a person can breathe and how quickly, can improve
respiratory issues. Physical therapy can help restore muscle strength,
movement and flexibility. Occupational therapy can help people regain
the ability to do everyday tasks, like grocery shopping and cooking.
Speech therapy can help with swallowing and vocal cord issues.

Physiatrists, doctors who specialize in physical rehabilitation, are
likely to be increasingly in demand, experts say. So are neurologists
and mental health therapists.

``I think the main take-home here is that post-Covid care is complex,''
Dr. Putrino said. ``It's hard enough to rehabilitate someone with a
broken leg where one thing is wrong.''

``But with post-Covid care,'' he said, ``you're dealing with people with
some cognition issues, physical issues, lung issues, heart issues,
kidney issues, trauma --- and all of these things have to be managed
just right.''

Advertisement

\protect\hyperlink{after-bottom}{Continue reading the main story}

\hypertarget{site-index}{%
\subsection{Site Index}\label{site-index}}

\hypertarget{site-information-navigation}{%
\subsection{Site Information
Navigation}\label{site-information-navigation}}

\begin{itemize}
\tightlist
\item
  \href{https://help.nytimes.com/hc/en-us/articles/115014792127-Copyright-notice}{©~2020~The
  New York Times Company}
\end{itemize}

\begin{itemize}
\tightlist
\item
  \href{https://www.nytco.com/}{NYTCo}
\item
  \href{https://help.nytimes.com/hc/en-us/articles/115015385887-Contact-Us}{Contact
  Us}
\item
  \href{https://www.nytco.com/careers/}{Work with us}
\item
  \href{https://nytmediakit.com/}{Advertise}
\item
  \href{http://www.tbrandstudio.com/}{T Brand Studio}
\item
  \href{https://www.nytimes.com/privacy/cookie-policy\#how-do-i-manage-trackers}{Your
  Ad Choices}
\item
  \href{https://www.nytimes.com/privacy}{Privacy}
\item
  \href{https://help.nytimes.com/hc/en-us/articles/115014893428-Terms-of-service}{Terms
  of Service}
\item
  \href{https://help.nytimes.com/hc/en-us/articles/115014893968-Terms-of-sale}{Terms
  of Sale}
\item
  \href{https://spiderbites.nytimes.com}{Site Map}
\item
  \href{https://help.nytimes.com/hc/en-us}{Help}
\item
  \href{https://www.nytimes.com/subscription?campaignId=37WXW}{Subscriptions}
\end{itemize}
