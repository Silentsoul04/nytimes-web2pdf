Sections

SEARCH

\protect\hyperlink{site-content}{Skip to
content}\protect\hyperlink{site-index}{Skip to site index}

\href{https://www.nytimes.com/section/obituaries}{Obituaries}

\href{https://myaccount.nytimes.com/auth/login?response_type=cookie\&client_id=vi}{}

\href{https://www.nytimes.com/section/todayspaper}{Today's Paper}

\href{/section/obituaries}{Obituaries}\textbar{}Overlooked No More: Karl
Heinrich Ulrichs, Pioneering Gay Activist

\url{https://nyti.ms/2BTPWoU}

\begin{itemize}
\item
\item
\item
\item
\item
\end{itemize}

Advertisement

\protect\hyperlink{after-top}{Continue reading the main story}

Supported by

\protect\hyperlink{after-sponsor}{Continue reading the main story}

\hypertarget{overlooked-no-more-karl-heinrich-ulrichs-pioneering-gay-activist}{%
\section{Overlooked No More: Karl Heinrich Ulrichs, Pioneering Gay
Activist}\label{overlooked-no-more-karl-heinrich-ulrichs-pioneering-gay-activist}}

Before the word ``homosexuality'' existed, he argued that same-sex
attraction was innate, and that those who experienced it should be
treated the same as anyone else.

\includegraphics{https://static01.nyt.com/images/2020/07/06/obituaries/06overlooked-ulrich/00overlooked-ulrich-articleLarge.jpg?quality=75\&auto=webp\&disable=upscale}

\href{https://www.nytimes.com/by/liam-stack}{\includegraphics{https://static01.nyt.com/images/2018/10/22/multimedia/author-liam-stack/author-liam-stack-thumbLarge.png}}

By \href{https://www.nytimes.com/by/liam-stack}{Liam Stack}

\begin{itemize}
\item
  July 1, 2020
\item
  \begin{itemize}
  \item
  \item
  \item
  \item
  \item
  \end{itemize}
\end{itemize}

\emph{Overlooked is a series of obituaries about remarkable people whose
deaths, beginning in 1851, went unreported in The Times.}

By the time the lawyer and writer Karl Heinrich Ulrichs took the podium
at a meeting of the Association of German Jurists in 1867, rumors about
his same-sex love affairs --- and the subsequent threat of arrest and
prosecution --- had already cost him his legal career and forced him to
flee his homeland.

Standing in Munich before more than 500 lawyers, officials and academics
--- many of whom jeered as he spoke --- Ulrichs argued for the repeal of
sodomy laws that criminalized sex between men in several of the
German-speaking kingdoms and duchies that existed in the years before
the creation of a unified German state.

``Gentlemen, my proposal is directed toward a revision of the current
penal law,'' he said, according to the historian Robert Beachy in
\href{https://www.nytimes.com/2014/11/02/books/review/gay-berlin-by-robert-beachy.html}{the
2014 book ``Gay Berlin: Birthplace of a Modern Identity.''}

Ulrichs described a ``class of persons'' who faced persecution simply
because ``nature has planted in them a sexual nature that is opposite of
that which is usual."

Same-sex attraction was a deeply taboo topic at the time; the word
``homosexuality'' would not even exist for another two years,
\href{https://www.bbc.com/future/article/20170315-the-invention-of-heterosexuality}{when
it was coined} by the Austro-Hungarian writer Karl-Maria Kertbeny. So
the ideas in Ulrichs's speech --- that such attraction was innate, and
that those who experienced it should be treated the same as anyone else
--- were revolutionary.

His remarks preceded by more than 100 years the Stonewall riots in New
York in 1969, which are widely seen as the start of the modern
L.G.B.T.Q. rights movement.

They helped inspire the rise of the world's
\href{https://www.npr.org/2014/12/17/371424790/between-world-wars-gay-culture-flourished-in-berlin}{first
gay rights movement}, 30 years later in Berlin.

They foreshadowed the imposition of a sodomy law across the German
Empire that would later be
\href{https://encyclopedia.ushmm.org/content/en/article/persecution-of-homosexuals-in-the-third-reich}{used
by the Nazis to target gay men}, thousands of whom were killed in
concentration camps.

And they made history: Ulrichs is believed to have been the first person
to publicly ``come out,'' in the modern sense of the term.

``I think it is reasonable to describe him as the first gay person to
publicly out himself,'' Robert Beachy said in an interview. ``There is
nothing comparable in the historical record. There is just nothing else
like this out there.''

His speech was also deeply unwelcome at the 1867 meeting, where the
audience erupted in shouts of ``Stop!'' and ``Crucify!'' that ultimately
forced Ulrichs off the stage.

For much of Ulrichs's life, same-sex relations were widely seen as a
pathology or as a sin to which any person could succumb if seized by
wickedness. These views still exist in some parts of the world.

Ulrichs helped forge the concepts of gay people as a distinct group and
of sexual identity as an innate human characteristic in a series of
pamphlets he wrote from 1864 to 1879 --- at first under a pseudonym, but
under his own name after he gave his speech at the 1867 conference.

``By publishing these writings I have initiated a scientific discussion
based on facts,'' he wrote in a letter published in 1864 in Deutsche
Allgemeine, a pan-German newspaper.

``Until now the treatment of the subject has been biased, not to mention
contemptuous,'' he added. ``My writings are the voice of a socially
oppressed minority that now claims its rights to be heard.''

His work was widely read by sex researchers. One of them, Richard von
Krafft-Ebing, cited the pamphlets in his pioneering 1886 text,
``Psychopathia Sexualis,'' which described homosexuality as a mental
illness.

In later editions, Krafft-Ebing published letters from men who had read
about Ulrichs in his book. The letters showed that not only did
Ulrichs's pamphlets explore theories about sexuality, but they also
helped foster a sense of community.

``I cannot describe what a salvation it was for me,'' one of the men
wrote, ``to learn that there are many other men who are sexually
constituted the way I am, and that my sexual feeling was not an
aberration but rather a sexual orientation determined by nature.''

Karl Heinrich Ulrichs was born on Aug. 28, 1825, in Aurich, in the
kingdom of Hanover in northwestern Germany, to an upper-middle-class
family that included several Lutheran pastors. He studied Latin and
Greek before beginning his legal studies at the University of Göttingen.

He secured prestigious positions in the Hanoverian Civil Service, but
rumors about his same-sex relationships --- and laws against public
indecency --- led him to resign his post as an assistant judge in 1854.
He became a journalist for Allgemeine Zeitung, a pan-German newspaper
published in Bavaria.

In the years before the invention of the German word ``homosexualität,''
a term that eventually found its way into English and other languages,
Ulrichs's pamphlets provided readers with a morally neutral vocabulary
to describe themselves.

He coined the words ``urnings'' to refer to people we now call gay men,
``urinden'' to refer to people we now call lesbians, ``dionings'' for
people we now call heterosexuals, and ``uranodionism'' for what is today
called bisexuality.

Those terms were inspired by his study of the classics, in particular
the story of Uranus, the god of the heavens, who was portrayed as both
father and mother to the goddess Aphrodite in Plato's ``Symposium.''

The concept of transgender people as distinct from gay, lesbian or
bisexual people did not exist at the time, said Paul B. Preciado, a
transgender philosopher at the Pompidou Center in Paris who has written
about Ulrichs.

Ulrichs's writings, including his pamphlets and a series of letters to
his family, whom he informed of his same-sex desires in 1862, were based
on an understanding of gender and sexuality as fundamentally
interconnected.

For Ulrichs, urnings were a sort of third gender who possessed the
physical body of a man but the inner spirit of a woman, which Preciado
described as ``a female soul confined within a man's body.''

Ulrichs was a German nationalist, Beachy said, and in addition to the
legal emancipation of urnings, his other great political passion was
German unification.

He used his writings to oppose the growing domination of the Kingdom of
Prussia, a military and political powerhouse that seemed determined to
bring the other German states under its control.

He feared that Prussia would succeed in uniting the German states and
would introduce its sodomy law into lands that did not criminalize
same-sex activity, including his native Hanover.

Ulrichs's fears about Prussia proved correct. Prussia annexed Hanover in
1866, and Ulrichs was jailed twice in 1867 for anti-Prussian activities
before he was banished from his homeland.

His personal papers were confiscated, including a list of 150 suspected
urnings in Berlin that was taken to the desk of Otto von Bismarck, who
orchestrated the unification of Germany in 1871.

By 1872, the Prussian sodomy law, also known as Paragraph 175, had been
adopted by all the states of the new German Empire. It was a crushing
blow for Ulrichs.

He published one final pamphlet in 1879 and then crossed the Alps by
foot and settled in Italy, where his public advocacy for urnings ceased.
He spent his remaining years editing a small Latin-language literary
journal. He died on July 14, 1895. He was 69.

Paragraph 175, which criminalized sex between men but did not address
lesbianism, remained in place in some variation for more than 100 years.
It was ultimately repealed in 1994.

In 2017
\href{https://www.nytimes.com/2017/06/23/world/europe/germany-anti-gay-law.html}{the
German Parliament voted unanimously} to void the convictions of roughly
50,000 men who had been prosecuted under the law since World War II and
to compensate thousands who were still alive.

Ulrichs was celebrated by early-20th-century gay activists like Magnus
Hirschfeld, but after the rise of Nazism his contributions to history
were forgotten for decades. Today there are streets named for him in
Berlin, Munich, Hanover and other parts of Germany.

Advertisement

\protect\hyperlink{after-bottom}{Continue reading the main story}

\hypertarget{site-index}{%
\subsection{Site Index}\label{site-index}}

\hypertarget{site-information-navigation}{%
\subsection{Site Information
Navigation}\label{site-information-navigation}}

\begin{itemize}
\tightlist
\item
  \href{https://help.nytimes.com/hc/en-us/articles/115014792127-Copyright-notice}{©~2020~The
  New York Times Company}
\end{itemize}

\begin{itemize}
\tightlist
\item
  \href{https://www.nytco.com/}{NYTCo}
\item
  \href{https://help.nytimes.com/hc/en-us/articles/115015385887-Contact-Us}{Contact
  Us}
\item
  \href{https://www.nytco.com/careers/}{Work with us}
\item
  \href{https://nytmediakit.com/}{Advertise}
\item
  \href{http://www.tbrandstudio.com/}{T Brand Studio}
\item
  \href{https://www.nytimes.com/privacy/cookie-policy\#how-do-i-manage-trackers}{Your
  Ad Choices}
\item
  \href{https://www.nytimes.com/privacy}{Privacy}
\item
  \href{https://help.nytimes.com/hc/en-us/articles/115014893428-Terms-of-service}{Terms
  of Service}
\item
  \href{https://help.nytimes.com/hc/en-us/articles/115014893968-Terms-of-sale}{Terms
  of Sale}
\item
  \href{https://spiderbites.nytimes.com}{Site Map}
\item
  \href{https://help.nytimes.com/hc/en-us}{Help}
\item
  \href{https://www.nytimes.com/subscription?campaignId=37WXW}{Subscriptions}
\end{itemize}
