Sections

SEARCH

\protect\hyperlink{site-content}{Skip to
content}\protect\hyperlink{site-index}{Skip to site index}

\href{https://www.nytimes.com/section/realestate}{Real Estate}

\href{https://myaccount.nytimes.com/auth/login?response_type=cookie\&client_id=vi}{}

\href{https://www.nytimes.com/section/todayspaper}{Today's Paper}

\href{/section/realestate}{Real Estate}\textbar{}Mahopac, N.Y.: A
`Bedroom Community' With an Elegant Past

\url{https://nyti.ms/3gim7NU}

\begin{itemize}
\item
\item
\item
\item
\item
\item
\end{itemize}

\href{https://www.nytimes.com/spotlight/at-home?action=click\&pgtype=Article\&state=default\&region=TOP_BANNER\&context=at_home_menu}{At
Home}

\begin{itemize}
\tightlist
\item
  \href{https://www.nytimes.com/2020/08/03/well/family/the-benefits-of-talking-to-strangers.html?action=click\&pgtype=Article\&state=default\&region=TOP_BANNER\&context=at_home_menu}{Talk:
  To Strangers}
\item
  \href{https://www.nytimes.com/2020/08/01/at-home/coronavirus-make-pizza-on-a-grill.html?action=click\&pgtype=Article\&state=default\&region=TOP_BANNER\&context=at_home_menu}{Make:
  Grilled Pizza}
\item
  \href{https://www.nytimes.com/2020/07/31/arts/television/goldbergs-abc-stream.html?action=click\&pgtype=Article\&state=default\&region=TOP_BANNER\&context=at_home_menu}{Watch:
  'The Goldbergs'}
\item
  \href{https://www.nytimes.com/interactive/2020/at-home/even-more-reporters-editors-diaries-lists-recommendations.html?action=click\&pgtype=Article\&state=default\&region=TOP_BANNER\&context=at_home_menu}{Explore:
  Reporters' Google Docs}
\end{itemize}

Advertisement

\protect\hyperlink{after-top}{Continue reading the main story}

Supported by

\protect\hyperlink{after-sponsor}{Continue reading the main story}

Living in

\hypertarget{mahopac-ny-a-bedroom-community-with-an-elegant-past}{%
\section{Mahopac, N.Y.: A `Bedroom Community' With an Elegant
Past}\label{mahopac-ny-a-bedroom-community-with-an-elegant-past}}

If living in the Putnam County hamlet feels like being on vacation,
that's no accident: It was developed in the 19th century as a summer
resort.

\href{https://www.nytimes.com/slideshow/2020/07/01/realestate/living-in-mahopac-ny.html}{}

\hypertarget{living-in--mahopac-ny}{%
\subsection{Living In ... Mahopac, N.Y.}\label{living-in--mahopac-ny}}

18 Photos

View Slide Show ›

\includegraphics{https://static01.nyt.com/images/2020/07/05/realestate/01LIVING-MAHOPAC-slide-B4FK/01LIVING-MAHOPAC-slide-B4FK-articleLarge.jpg?quality=75\&auto=webp\&disable=upscale}

Tony Cenicola/The New York Times

By Susan Hodara

\begin{itemize}
\item
  Published July 1, 2020Updated July 2, 2020
\item
  \begin{itemize}
  \item
  \item
  \item
  \item
  \item
  \item
  \end{itemize}
\end{itemize}

Life changed for Dean Bender and his wife, Jenny Stasikewich, in the
middle of Lake Mahopac. It was a pleasant Friday evening in September
2018, in the Putnam County, N.Y., hamlet of Mahopac, when they were
invited for a cocktail cruise on a friend's pontoon boat. As they
rounded the lake, the friend, a real estate agent, pointed out glorious
waterfront homes.

``Then we passed one that he called `the humblest house on the lake,'''
Ms. Stasikewich recalled. ``He said it was on the market, the price had
dropped, and did we want to take a look?''

Carmel

Hamlet

TACONIC STATE PKWY.

LONG

POND

SYCAMORE park

Town of CARMEL

PUTNAM COUNTY

Lake Casse

Lake

MacGregor

Lake

Mahopac

Lake

Secor

Mahopac

Public

Library

Mahopac

6

6N

Somers ~

PUTNAM

Mahopac

WESTCHESTER

COUNTY

NEW YORK

New York

1/2 mile

N.Y.C.

By The New York Times

Ms. Stasikewich, 68, is a mosaic artist. Mr. Bender, 70, is a partner at
Thompson \& Bender, a marketing and public relations firm in Briarcliff
Manor, in Westchester County. The couple had spent more than three
decades in the four-bedroom home they owned in Yorktown, N.Y., where
they raised their family. They hadn't been thinking about moving, but
the house on Lake Mahopac intrigued them.

On Saturday, they toured the property with their friend, noting that it
needed work. On Sunday, they brought in another friend, an architect, to
gauge the cost of renovations. By Monday, they had an accepted offer on
the house, for \$512,500.

``We're not impulsive people,'' Mr. Bender said. ``But once we saw the
place and its spectacular location, we knew we had to act fast.''

\includegraphics{https://static01.nyt.com/images/2020/07/05/realestate/01LIVING-MAHOPAC-slide-2LWM/01LIVING-MAHOPAC-slide-2LWM-articleLarge.jpg?quality=75\&auto=webp\&disable=upscale}

With the nearly 600-acre Lake Mahopac as its centerpiece, Mahopac is a
6.4-square-mile census-designated place covering much of the lower half
of the town of Carmel. Its southern border abuts the Westchester town of
\href{https://www.nytimes.com/2019/01/02/realestate/somers-ny-a-close-knit-town-with-plenty-of-shopping-and-great-parks.html}{Somers}
and, to the southwest, a corner of the town of
\href{https://www.nytimes.com/2001/05/13/realestate/if-you-re-thinking-of-living-in-yorktown-a-town-that-values-a-sense-of-country.html}{Yorktown}.
According to census estimates, Mahopac's population is just over 8,400.

Last July, after selling their Yorktown house, Mr. Bender and Ms.
Stasikewich moved into their fully gutted and redesigned home: a
1,500-square foot, two-bedroom cottage, built in 1945 on a quarter acre.
In addition to maximizing their lake views, they added a deck and, in a
twist, when their real estate friend decided to downsize, they acquired
his pontoon boat. ``Living here is like being on vacation,'' Mr. Bender
said.

Perhaps, yet many of Mahopac's residents commute to work in New York
City and other nearby hubs, said Regina Morini, 84, a retired Putnam
County legislator and assistant to the Putnam County Executive, and a
lifelong Mahopac resident (until she moved to Somers last August).
``We're a bedroom community,'' she said.

That wasn't always so. In the mid-1800s, Mahopac was developed as a
summer resort. City residents escaping the heat arrived by train, some
building seasonal cottages around Lake Mahopac and the hamlet's smaller
lakes, others staying in elegant, now-defunct lodgings like the Mahopac
Hotel and the Gregory House. It wasn't until the completion of the Saw
Mill River Parkway in 1954 and the construction of Interstate 684
between 1964 and 1974 that Mahopac became the year-round community it is
today.

Image

795 SOUTH LAKE BOULEVARD \textbar{} A four-bedroom,
three-and-a-half-bathroom house built in 1850, on 0.48 acres with two
docks and a boat slip, listed for \$999,000. 914-962-4900Credit...Tony
Cenicola/The New York Times

\hypertarget{what-youll-find}{%
\subsection{What You'll Find}\label{what-youll-find}}

Mahopac's walkable downtown, lined with mom-and-pop shops, small
businesses and restaurants, stretches about half a mile along the
southern edge of Lake Mahopac. There are two large shopping plazas with
supermarkets on Route 6, a commercial strip that travels north-south
through the hamlet. The rest of Mahopac is residential, its roads, some
narrow and hilly, winding through woods and around the lakes.

Census estimates show there are some 3,400 housing units in Mahopac.
Most are single-family homes, many of them colonials and raised ranches
built in the second half of the 20th century and newer colonials in
subdivisions like Lakeview at Hill Farm and, most recently, Random
Ridge. Surrounding the lakes, vintage bungalows have been upgraded, some
transformed into large waterfront homes.

There are also townhouse communities, including the 75-home Maple Hill
Estates, 89-home Williamsburg Ridge, 100-home Society Hill and 49-home
Hunters Run; a few condominium complexes, including White Sail
Condominiums on Lake Mahopac; and one cooperative complex, Woodcrest
Gardens. There are several small rental buildings.

Image

60 SPRUCETOP DRIVE \textbar{} A four-bedroom, three-and-a-half-bathroom
house, built in 2006 on 2.16 acres, listed for \$699,000.
845-590-6864Credit...Tony Cenicola/The New York Times

\hypertarget{what-youll-pay}{%
\subsection{What You'll Pay}\label{what-youll-pay}}

Joanne Daly, an agent with Coldwell Banker Residential Brokerage, said
prices in Mahopac range from the \$300,000s, for a small starter home,
up to around \$2 million, for a waterfront property on Lake Mahopac.
``Prices tend to be higher for houses around the water,'' she said.

The reopening of the economy following the pandemic lockdown has sparked
a rise in market activity. ``We're seeing an uptick in properties being
listed,'' said Geraldine Finan, an agent with Houlihan Lawrence. ``And
we're seeing multiple offers and bidding wars. Even some of the
higher-end homes that had been sitting for a while are seeing
movement.''

Many buyers are coming from New York City, seeking open space and room
to work remotely. ``The stay-at-home order has forced folks to start
looking at other options outside of apartment living,'' Ms. Daly said.

Data provided by the Hudson Gateway Multiple Listing Service indicated
that as of June 18, there were 101 single-family homes on the market,
from a three-bedroom, 1,808-square-foot gutted colonial, built in 1900
on 0.28 acres and listed for \$100,000, to a five-bedroom,
6,973-square-foot lakefront contemporary house, built in 1983 on 1.3
acres and listed for \$3.5 million. There were two multifamily homes on
the market: a 1,575-square-foot, three-unit property for \$450,000 and a
2,017-square-foot, three-unit property for \$525,000. Four condominiums
were for sale, from a 1,080-square-foot two-bedroom for \$219,000 to an
1,800-square-foot two-bedroom for \$369,900. There were no cooperatives
for sale.

The median sales price for a single-family home during the 12-month
period ending June 18 was \$410,000, up from \$386,000 during the
previous 12 months. For multifamily homes, the median was \$335,000, up
from \$293,500 during the previous 12 months; for condos, the median was
\$318,000, up from \$260,000 in the previous 12 months. There were no
cooperative apartment sales during the 12-month period ending June 18;
during the prior 12 months, the median was \$120,000.

Image

78 REBECCA LANE \textbar{} A five-bedroom, three-and-a-half-bathroom
house with two units, built in 1997 on 1.88 acres, listed for \$620,000.
845-590-6864Credit...Tony Cenicola/The New York Times

\hypertarget{the-vibe}{%
\subsection{The Vibe}\label{the-vibe}}

Mr. Bender described Mahopac as ``easygoing and friendly.'' Ms. Daly,
who has lived there for 28 years, called it a place where ``everybody
looks out for their neighbors.''

Before the pandemic, residents regularly gathered for activities like a
Halloween celebration, a holiday tree-lighting ceremony and, in July,
the Mahopac Volunteer Fire Department's carnival and parade (which is
canceled this year). They might run into one another at restaurants like
Arturo's Tavern or Blu at the Lakehouse; at the Mahopac Golf and Beach
Club or the Putnam County Golf Course, both in Mahopac; or at the
typically bustling (but now curbside-only) Mahopac Public Library, which
overlooks Lake Mahopac. In terms of social action, last month several
hundred community members convened at a rally held by the newly
established Mahopac for Racial Justice group, founded by current and
former Mahopac students of color.

As the weather heats up, so do activities on the water. Residents with
deeded lake rights can enjoy boating, fishing and swimming; several
lakes, like Lake MacGregor, Lake Casse and Lake Secor, have sandy
beaches. Others can swim at the public beach on Long Pond, in the
32-acre Sycamore Park. Mahopac is also home to two public marinas.

Some lakeside houses, including those on either side of Mr. Bender and
Ms. Stasikewich's cottage, belong to part-timers. In the winter, Mr.
Bender said, the hamlet feels more subdued: ``It's a good time for
bird-watching, and the nights are great for stargazing.''

Image

The golf course at the Mahopac Golf and Beach Club, established in 1898
and set on 200 acres on north shore of Lake Mahopac.Credit...Tony
Cenicola/The New York Times

\hypertarget{the-schools}{%
\subsection{The Schools}\label{the-schools}}

Most of Mahopac is served by the Mahopac Central School District, which
also serves small parts of Carmel beyond Mahopac and the adjacent town
of Putnam Valley. A few homes in the southwest corner of Mahopac are
zoned for the Lakeland Central School District.

Leslie Mancuso, president of the board of education, said the Mahopac
Central School District's approximately 4,000 students attend Austin
Road Elementary, Fulmar Road Elementary or Lakeview Elementary for
kindergarten through fifth grade; Mahopac Middle School for sixth,
seventh and eighth grades; and then Mahopac High School. The district
has two alternative public schools, the Mahopac Falls Academy, for
middle-school students, and the Mahopac Academy, for high school. All of
the schools are in the hamlet.

On the
\href{https://data.nysed.gov/profile.php?instid=800000039764}{2019 state
assessments}, 54.8 percent of the district's fourth-graders were
proficient in math and 46.4 percent were proficient in English language
arts; statewide equivalents were 47.7 and 45.4 percent. Mean SAT scores
for Mahopac High's class of 2019 were 557 in math and 554 in
evidence-based reading and writing; statewide means were 533 and 531.

Image

Mahopac Point, a residential area that juts into Lake
Mahopac.Credit...Tony Cenicola/The New York Times

\hypertarget{the-commute}{%
\subsection{The Commute}\label{the-commute}}

For commuters to Manhattan, approximately 55 miles south, the closest
Metro-North Railroad station is Croton Falls, on the Harlem line, about
five miles away. During rush hour, the ride to and from Grand Central
Terminal takes 73 to 94 minutes. At this phase of the reopening, all
fares are considered off-peak; usually, round-trip fares are \$30
off-peak, \$40 peak and \$437 monthly.

Drivers can hop on Interstate 684 in Croton Falls or the Taconic State
Parkway at Shrub Oak, about six miles away. Getting to the city takes a
little over an hour, depending on traffic.

\hypertarget{the-history}{%
\subsection{The History}\label{the-history}}

What is now Mahopac was once inhabited by the Wappinger, members of the
Algonquian tribe of Native Americans. Mahopac is said to be the
Algonquian word for ``lake of the great serpent.'' While most people
pronounce it ``MAY-o-pack,'' longtime residents defer to its roots and
insist it is ``ma-HO-pack.''

``Mohegan, Mohansic, Mahopac,'' Ms. Morini said. ``The accent is on the
second syllable.''

For weekly email updates on residential real estate news,
\href{http://www.nytimes.com/newsletters/realestate/}{sign up here}.
Follow us on Twitter:
\href{https://twitter.com/nytrealestate}{@nytrealestate}.

Advertisement

\protect\hyperlink{after-bottom}{Continue reading the main story}

\hypertarget{site-index}{%
\subsection{Site Index}\label{site-index}}

\hypertarget{site-information-navigation}{%
\subsection{Site Information
Navigation}\label{site-information-navigation}}

\begin{itemize}
\tightlist
\item
  \href{https://help.nytimes.com/hc/en-us/articles/115014792127-Copyright-notice}{©~2020~The
  New York Times Company}
\end{itemize}

\begin{itemize}
\tightlist
\item
  \href{https://www.nytco.com/}{NYTCo}
\item
  \href{https://help.nytimes.com/hc/en-us/articles/115015385887-Contact-Us}{Contact
  Us}
\item
  \href{https://www.nytco.com/careers/}{Work with us}
\item
  \href{https://nytmediakit.com/}{Advertise}
\item
  \href{http://www.tbrandstudio.com/}{T Brand Studio}
\item
  \href{https://www.nytimes.com/privacy/cookie-policy\#how-do-i-manage-trackers}{Your
  Ad Choices}
\item
  \href{https://www.nytimes.com/privacy}{Privacy}
\item
  \href{https://help.nytimes.com/hc/en-us/articles/115014893428-Terms-of-service}{Terms
  of Service}
\item
  \href{https://help.nytimes.com/hc/en-us/articles/115014893968-Terms-of-sale}{Terms
  of Sale}
\item
  \href{https://spiderbites.nytimes.com}{Site Map}
\item
  \href{https://help.nytimes.com/hc/en-us}{Help}
\item
  \href{https://www.nytimes.com/subscription?campaignId=37WXW}{Subscriptions}
\end{itemize}
