Sections

SEARCH

\protect\hyperlink{site-content}{Skip to
content}\protect\hyperlink{site-index}{Skip to site index}

\href{https://www.nytimes.com/section/technology}{Technology}

\href{https://myaccount.nytimes.com/auth/login?response_type=cookie\&client_id=vi}{}

\href{https://www.nytimes.com/section/todayspaper}{Today's Paper}

\href{/section/technology}{Technology}\textbar{}China's Software Stalked
Uighurs Earlier and More Widely, Researchers Learn

\url{https://nyti.ms/2BufEAz}

\begin{itemize}
\item
\item
\item
\item
\item
\item
\end{itemize}

Advertisement

\protect\hyperlink{after-top}{Continue reading the main story}

Supported by

\protect\hyperlink{after-sponsor}{Continue reading the main story}

\hypertarget{chinas-software-stalked-uighurs-earlier-and-more-widely-researchers-learn}{%
\section{China's Software Stalked Uighurs Earlier and More Widely,
Researchers
Learn}\label{chinas-software-stalked-uighurs-earlier-and-more-widely-researchers-learn}}

A new report revealed a broad campaign that targeted Muslims in China
and their diaspora in other countries, beginning as early as 2013.

\includegraphics{https://static01.nyt.com/images/2020/06/23/business/00china-hack-01/00china-hack-01-articleLarge.jpg?quality=75\&auto=webp\&disable=upscale}

\href{https://www.nytimes.com/by/paul-mozur}{\includegraphics{https://static01.nyt.com/images/2018/10/15/multimedia/author-paul-mozur/author-paul-mozur-thumbLarge.png}}\href{https://www.nytimes.com/by/nicole-perlroth}{\includegraphics{https://static01.nyt.com/images/2018/02/20/multimedia/author-nicole-perlroth/author-nicole-perlroth-thumbLarge.jpg}}

By \href{https://www.nytimes.com/by/paul-mozur}{Paul Mozur} and
\href{https://www.nytimes.com/by/nicole-perlroth}{Nicole Perlroth}

\begin{itemize}
\item
  July 1, 2020
\item
  \begin{itemize}
  \item
  \item
  \item
  \item
  \item
  \item
  \end{itemize}
\end{itemize}

\href{https://cn.nytimes.com/technology/20200702/china-uighurs-hackers-malware-hackers-smartphones/}{阅读简体中文版}\href{https://cn.nytimes.com/technology/20200702/china-uighurs-hackers-malware-hackers-smartphones/zh}{閱讀繁體中文版}

TAIPEI, Taiwan --- Before the Chinese police hung high-powered
surveillance cameras and
\href{https://www.nytimes.com/2019/08/31/world/asia/xinjiang-china-uighurs-prisons.html}{locked
up} ethnic minorities
\href{https://apnews.com/269b3de1af34e17c1941a514f78d764c}{by the
hundreds of thousands} in China's western region of Xinjiang, China's
hackers went to work building malware, researchers say.

The Chinese hacking campaign, which researchers at Lookout --- the San
Francisco mobile security firm --- said on Wednesday had begun in
earnest as far back as 2013 and continues to this day, was part of a
broad but often invisible effort to pull in data from the devices that
know people best: their smartphones.

Lookout found links between eight types of malicious software --- some
previously known, others not --- that show how groups connected to
China's government hacked into Android phones used by Xinjiang's largely
Muslim
\href{https://www.nytimes.com/2020/07/06/world/asia/china-xinjiang-uighur-court.html}{Uighur}
population on a scale far larger than had been realized.

The timeline suggests the hacking campaign was an early cornerstone in
China's
\href{https://www.nytimes.com/2019/05/22/world/asia/china-surveillance-xinjiang.html}{Uighur
surveillance efforts} that would later extend to collecting
\href{https://www.nytimes.com/2019/02/21/business/china-xinjiang-uighur-dna-thermo-fisher.html}{blood
samples}, voice prints,
\href{https://www.nytimes.com/2019/12/03/business/china-dna-uighurs-xinjiang.html}{facial
scans} and
\href{https://www.nytimes.com/2019/12/17/technology/china-surveillance.html}{other
personal data} to transform Xinjiang into a
\href{https://www.nytimes.com/2018/09/08/world/asia/china-uighur-muslim-detention-camp.html}{virtual
police state}. It also shows the lengths to which China's minders were
determined to follow Uighurs as they
\href{https://www.dw.com/en/how-china-intimidates-uighurs-abroad-by-threatening-their-families/a-49554977}{fled
China} for as many as 15 other countries.

The tools the hackers assembled hid in special keyboards used by Uighurs
and disguised themselves as commonly used apps in third-party websites.
Some could remotely turn on a phone's microphone, record calls or export
photos, phone locations and conversations on chat apps. Others were
embedded in apps that hosted Uighur-language news, Uighur-targeted
beauty tips, religious texts like the Quran and details of the latest
Muslim cleric arrests.

``Wherever China's Uighurs are going, however far they go, whether it
was Turkey, Indonesia or Syria, the malware followed them there,'' said
Apurva Kumar, a threat intelligence engineer at Lookout who helped
\href{https://blog.lookout.com/multiyear-surveillance-campaigns-discovered-targeting-uyghurs}{unravel
the campaign}. ``It was like watching a predator stalk its prey
throughout the world.''

A decade ago, the People's Liberation Army's hackers were notable not so
much for their sophistication as for the volume of their attacks. But
under threat of American sanctions, President Xi Jinping of China struck
an agreement with President Barack Obama in 2015 to cease hacking
American targets for commercial gain. The agreement stuck
\href{https://www.nytimes.com/2016/06/21/us/politics/china-us-cyber-spying.html}{for
a time}, with a significant drop in
\href{https://www.nytimes.com/2016/06/21/us/politics/china-us-cyber-spying.html}{Chinese
hacks} in the United States.

Last fall, private researchers determined that --- over that same period
--- China had turned its most advanced hacking tools on
\href{https://www.nytimes.com/2019/10/22/technology/china-hackers-ethnic-minorities.html}{its
own people}. In overlapping discoveries, researchers at Google, the
security firm Volexity and the Citizen Lab at the University of
Toronto's Munk School of Public Affairs separately uncovered what
amounted to an advanced Chinese hack against iPhones and Android phones
belonging to Chinese Uighurs and Tibetans throughout the world.

\includegraphics{https://static01.nyt.com/images/2020/06/23/business/00china-hack-02/00china-hack-02-articleLarge.jpg?quality=75\&auto=webp\&disable=upscale}

Google's researchers discovered that hackers had infected websites
frequented by Uighurs --- inside China and in other countries --- with
tools that could hack their iPhones and siphon off their data.

Lookout's latest analysis suggests that China's mobile hacking campaign
was broader and more aggressive than security experts, human rights
activists and spyware victims had realized. But experts on Chinese
surveillance say it should come as no surprise, given the lengths to
which Beijing has gone to monitor Xinjiang.

``We should think about smartphone surveillance being used as a way to
track people's inner life, their everyday behavior, their
trustworthiness,'' said Darren Byler, who studies surveillance of
minority populations at the University of Colorado, Boulder.

In 2015, as Beijing pushed to crack down on sporadic ethnic violence in
Xinjiang, the authorities grew ``desperate'' to track fast-growing
Uighur communications online, Mr. Byler said. Uighurs began to fear that
their online chats discussing Islam or politics were risky. Savvier
Uighurs took to owning a second ``clean phone,'' said Mr. Byler, who
lived in Xinjiang in 2015.

On the streets of Xinjiang, the police began confiscating Uighurs'
phones. Sometimes, they returned them months later with new spyware
installed. Other times, people were handed back entirely different
phones. Officials visiting Uighur villages regularly recorded the serial
numbers used to identify smartphones. They lined the streets with new
hardware that tracked people's phones as they walked past.

The authorities dragged Uighurs off to detention camps for having two
phones or an antiquated phone, arbitrarily dumping a phone, or not
having a phone at all, according to
\href{https://www.nytimes.com/2019/05/22/world/asia/china-surveillance-xinjiang.html}{testimonials}
and
\href{https://www.nytimes.com/interactive/2019/11/16/world/asia/china-xinjiang-documents.html}{government
documents}.

Over that same period, Lookout said China's mobile hacking efforts
accelerated. One type of Chinese malware, known as GoldenEagle after the
words hackers littered throughout their code --- an apparent reference
to the eagles used for hunting in Xinjiang --- was used as early as
2011. But its use picked up in 2015 and 2016. Lookout uncovered more
than 650 versions of GoldenEagle malware and a large number of fake
Uighur apps that function as a sort of Trojan horse to spy on users'
mobile communications.

The malicious apps mimicked so-called virtual private networks, which
are used to set up secure web connections and view prohibited content
inside China. They also targeted apps frequently used by Uighurs for
shopping, video games, music streaming, adult media and travel booking,
as well as specialized Uighur keyboard apps. Some offered Uighurs beauty
and traditional-medicine tips. Others impersonated apps from Twitter,
Facebook, QQ --- the Chinese instant messaging service --- and the
search giant Baidu.

Once downloaded, the apps gave China's hackers a real-time window into
their targets' phone activity. They also gave China's minders the
ability to kill their spyware on command, including when it appeared to
suck up too much battery life. In some cases, Lookout discovered that
all China's hackers needed to do to get data off a target's phone was
send the user an invisible text message. The malware captured a victim's
data and sent it back to the attackers' phone via a text reply, then
deleted any trace of the exchange.

In June 2019, Lookout uncovered Chinese malware buried in an app called
Syrian News. The content was Uighur focused, suggesting China was trying
to bait Uighurs inside Syria into downloading their malware. That
Beijing's hackers would track Uighurs to Syria gave Lookout's
researchers a window into Chinese anxiety over
\href{https://apnews.com/79d6a427b26f4eeab226571956dd256e/AP-Exclusive:-Uighurs-fighting-in-Syria-take-aim-at-China}{Uighur
involvement} in the Syrian civil war. Lookout's researchers found
similarly malicious apps tailored to Uighurs in Kuwait, Turkey,
Indonesia, Malaysia, Afghanistan and Pakistan.

Researchers at other security research groups, like Citizen Lab, had
previously uncovered various pieces of China's mobile hacking campaign
and linked them back to Chinese state hackers. However, Lookout's new
report appears to be the first time researchers were able to piece these
older campaigns with new mobile malware and tie them to the same groups.

``Just how far removed the state is from these operations is always the
open question,'' said Christoph Hebeisen, Lookout's director of security
intelligence. ``It could be that these are patriotic hackers, like the
kind we have seen in Russia. But the targeting of Uighurs, Tibetans, the
diaspora and even Daesh, in one case, suggests otherwise,'' he added,
using another term for the Islamic State.

One clue to the attackers' identities came when Lookout's researchers
found what appeared to be test versions of China's malware on several
smartphones that were clustered in and around the headquarters of the
Chinese defense contractor Xi'an Tianhe Defense Technology.

A large supplier of defense technology, Tianhe sent employees to a major
defense conference in Xinjiang in 2015 to market products that could
monitor crowds. As a surveillance gold rush took over the region, Tianhe
doubled down, establishing a subsidiary in Xinjiang in 2018. The company
did not respond to emails requesting comment.

``That could be an interesting coincidence,'' Mr. Hebeisen said, ``or it
could be the smoking gun.''

Paul Mozur reported from Taipei, and Nicole Perlroth from San Francisco.

Advertisement

\protect\hyperlink{after-bottom}{Continue reading the main story}

\hypertarget{site-index}{%
\subsection{Site Index}\label{site-index}}

\hypertarget{site-information-navigation}{%
\subsection{Site Information
Navigation}\label{site-information-navigation}}

\begin{itemize}
\tightlist
\item
  \href{https://help.nytimes.com/hc/en-us/articles/115014792127-Copyright-notice}{©~2020~The
  New York Times Company}
\end{itemize}

\begin{itemize}
\tightlist
\item
  \href{https://www.nytco.com/}{NYTCo}
\item
  \href{https://help.nytimes.com/hc/en-us/articles/115015385887-Contact-Us}{Contact
  Us}
\item
  \href{https://www.nytco.com/careers/}{Work with us}
\item
  \href{https://nytmediakit.com/}{Advertise}
\item
  \href{http://www.tbrandstudio.com/}{T Brand Studio}
\item
  \href{https://www.nytimes.com/privacy/cookie-policy\#how-do-i-manage-trackers}{Your
  Ad Choices}
\item
  \href{https://www.nytimes.com/privacy}{Privacy}
\item
  \href{https://help.nytimes.com/hc/en-us/articles/115014893428-Terms-of-service}{Terms
  of Service}
\item
  \href{https://help.nytimes.com/hc/en-us/articles/115014893968-Terms-of-sale}{Terms
  of Sale}
\item
  \href{https://spiderbites.nytimes.com}{Site Map}
\item
  \href{https://help.nytimes.com/hc/en-us}{Help}
\item
  \href{https://www.nytimes.com/subscription?campaignId=37WXW}{Subscriptions}
\end{itemize}
