Sections

SEARCH

\protect\hyperlink{site-content}{Skip to
content}\protect\hyperlink{site-index}{Skip to site index}

\href{https://www.nytimes.com/section/world/europe}{Europe}

\href{https://myaccount.nytimes.com/auth/login?response_type=cookie\&client_id=vi}{}

\href{https://www.nytimes.com/section/todayspaper}{Today's Paper}

\href{/section/world/europe}{Europe}\textbar{}Germany Disbands Special
Forces Group Tainted by Far-Right Extremists

\url{https://nyti.ms/38jfW9f}

\begin{itemize}
\item
\item
\item
\item
\item
\end{itemize}

Advertisement

\protect\hyperlink{after-top}{Continue reading the main story}

Supported by

\protect\hyperlink{after-sponsor}{Continue reading the main story}

\hypertarget{germany-disbands-special-forces-group-tainted-by-far-right-extremists}{%
\section{Germany Disbands Special Forces Group Tainted by Far-Right
Extremists}\label{germany-disbands-special-forces-group-tainted-by-far-right-extremists}}

For years, far-right extremists were tolerated inside Germany's most
elite military unit. An underground bunker of explosives has woken the
authorities to an alarming problem.

\includegraphics{https://static01.nyt.com/images/2020/07/01/world/01germany/01germany-articleLarge.jpg?quality=75\&auto=webp\&disable=upscale}

\href{https://www.nytimes.com/by/katrin-bennhold}{\includegraphics{https://static01.nyt.com/images/2018/07/13/multimedia/author-katrin-bennhold/author-katrin-bennhold-thumbLarge.png}}

By \href{https://www.nytimes.com/by/katrin-bennhold}{Katrin Bennhold}

\begin{itemize}
\item
  July 1, 2020
\item
  \begin{itemize}
  \item
  \item
  \item
  \item
  \item
  \end{itemize}
\end{itemize}

BERLIN --- Germany's defense minister announced Wednesday that she would
partially disband the most elite and highly trained special forces in
the country, saying it had been infiltrated by far-right extremism.

The defense minister,
\href{https://www.nytimes.com/2020/02/10/world/europe/annegret-kramp-karrenbauer-resign.html?searchResultPosition=25}{Annegret
Kramp-Karrenbauer}, said one of four fighting companies inside the
special forces had become so infested with far-right extremism that it
would be dissolved. The rest of the special forces unit, known by its
German acronym, KSK, has until the end of October to overhaul its
recruitment, training and leadership practices before being allowed to
rejoin any international military exercises or missions.

``The KSK cannot continue in its current form,'' Ms. Kramp-Karrenbauer
told a news conference, describing ``an unhealthy elitism'' and ``toxic
leadership'' inside the unit, which, she added, had ``developed and
promoted extremist tendencies.''

The announcement came six weeks after investigators discovered a trove
of Nazi memorabilia and an extensive arsenal of stolen ammunition and
explosives on the property of a sergeant major who had served in the KSK
since 2001.

His company is at the center of a long-running controversy over a
notorious party three years ago, where soldiers were reported to have
flashed Hitler salutes and listened to neo-Nazi rock music.

The raid highlighted ``a new quality'' of far-right extremism among
those trained and armed to protect Germany's democracy, Ms.
Kramp-Karrenbauer said. Since then, military leaders and politicians
have rolled out a flurry of initiatives, which critics said were long
overdue.

A committee was formed to report back on far-right extremism in the
special forces and to propose measures to combat it. New legislation was
passed to make it easier to fire far-right soldiers. And, crucially, the
KSK and the rest of the military has been ordered to account for missing
weapons and ammunition.

Some 48,000 rounds of ammunition and 62 kilograms worth of explosives
have gone missing from the special forces, said Gen. Eberhard Zorn,
inspector general of the armed forces and co-author of the report on the
special forces that was presented on Wednesday. The missing weapons and
ammunition have added to concerns that the recent raid was only the tip
of the iceberg.

The explosives in question were used by the KSK to explode building
facades on special missions abroad, General Zorn said. ``This is no
small thing,'' he added. ``It worries me very much.''

It worries others, too.

``Do we have terrorist cells inside our military? I never thought I
would ask that question, but we have to,'' said Patrick Sensburg, a
conservative lawmaker on the intelligence oversight committee and
president of the reservist association.

The commander of the KSK, Gen. Markus Kreitmayr, wrote a three-page
letter to his troops after the recent raid, in which he addressed
far-right soldiers directly: ``You don't deserve our camaraderie!'' he
wrote, urging them to leave the unit on their own. ``If you don't, you
will realize that we will find you and get rid of you!''

Ms. Kramp-Karrenbauer said efforts would now be intensified to determine
whether recent and older cases of extremism were part of a network.

``The probability that it's not just isolated cases but that there are
connections is obvious and has to be fully investigated,'' Ms.
Kramp-Karrenbauer said.

Ms. Kramp-Karrenbauer added that she now wants to better integrate the
KSK into the wider military to increase oversight of the unit. Training
that had been conducted separately from other units would be opened up,
security checks of new recruits would be intensified and the number of
years soldiers could serve in the same company would be capped.

The report presented to the minister by General Zorn concluded that
parts of the KSK existed outside the military chain of command. ``The
KSK, at least in some areas, has become independent in recent years,
under the influence of an unhealthy understanding of elitism by
individual leaders.''

But the failings were not just inside the KSK, the minister said. Across
the military, ammunition and explosives have been allowed to go missing.

Christoph Gramm, the president of military counterintelligence, said his
agency was currently investigating 600 soldiers, 20 of them in the KSK
alone, which has about 1,400 members.

``Elite units such as these have cultural factors that may develop into
susceptibilities,'' Mr. Gramm recently told The New York Times. ``For
example if there is a misguided sense of tolerance.''

``The soldiers have an elitist self-confidence,'' he added. ``They have
special capabilities and skills and a well-developed sense of loyalty.
Such a mind-set can involve risks.''

Ms. Kramp-Karrenbauer, the defense minister, said the military
counterintelligence service, known as the MAD, had failed in its mission
to monitor and detect extremism in recent years.

``The work of the MAD was not satisfactory,'' she said, adding, ``and
it's still not enough.''

The KSK turns 25 next year. Many hope that it will have rooted out its
far-right extremists by then. ``The KSK needs to be our elite for
freedom and democracy,'' said Eva Högl, the parliamentary commissioner
for the armed forces.

But for that to happen, Ms. Högl said, the authorities have to live up
to their recent vows to shine a light in all corners of Germany's
institutions.

Such vows have been made before.

In the early 2000s, members of the National Socialist Underground, a
neo-Nazi terrorist group, killed nine immigrants and a police officer
over seven years. One of the killers was a former soldier. Paid
informants in the domestic intelligence agency helped to hide the
group's leaders and to build its network. When the case finally came to
trial, it emerged that key files had been shredded by the agency.

Ms. Högl was a member of the parliamentary inquiry into what became
known as the N.S.U. scandal. ``Two decades later we still don't know
what the authorities knew,'' she said. ``This time has got to be
different.''

Advertisement

\protect\hyperlink{after-bottom}{Continue reading the main story}

\hypertarget{site-index}{%
\subsection{Site Index}\label{site-index}}

\hypertarget{site-information-navigation}{%
\subsection{Site Information
Navigation}\label{site-information-navigation}}

\begin{itemize}
\tightlist
\item
  \href{https://help.nytimes.com/hc/en-us/articles/115014792127-Copyright-notice}{©~2020~The
  New York Times Company}
\end{itemize}

\begin{itemize}
\tightlist
\item
  \href{https://www.nytco.com/}{NYTCo}
\item
  \href{https://help.nytimes.com/hc/en-us/articles/115015385887-Contact-Us}{Contact
  Us}
\item
  \href{https://www.nytco.com/careers/}{Work with us}
\item
  \href{https://nytmediakit.com/}{Advertise}
\item
  \href{http://www.tbrandstudio.com/}{T Brand Studio}
\item
  \href{https://www.nytimes.com/privacy/cookie-policy\#how-do-i-manage-trackers}{Your
  Ad Choices}
\item
  \href{https://www.nytimes.com/privacy}{Privacy}
\item
  \href{https://help.nytimes.com/hc/en-us/articles/115014893428-Terms-of-service}{Terms
  of Service}
\item
  \href{https://help.nytimes.com/hc/en-us/articles/115014893968-Terms-of-sale}{Terms
  of Sale}
\item
  \href{https://spiderbites.nytimes.com}{Site Map}
\item
  \href{https://help.nytimes.com/hc/en-us}{Help}
\item
  \href{https://www.nytimes.com/subscription?campaignId=37WXW}{Subscriptions}
\end{itemize}
