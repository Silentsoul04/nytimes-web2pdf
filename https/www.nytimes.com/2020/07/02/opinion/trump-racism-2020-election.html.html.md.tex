Sections

SEARCH

\protect\hyperlink{site-content}{Skip to
content}\protect\hyperlink{site-index}{Skip to site index}

\href{https://myaccount.nytimes.com/auth/login?response_type=cookie\&client_id=vi}{}

\href{https://www.nytimes.com/section/todayspaper}{Today's Paper}

\href{/section/opinion}{Opinion}\textbar{}Trump's Re-election Message Is
White Grievance

\href{https://nyti.ms/2VE82m0}{https://nyti.ms/2VE82m0}

\begin{itemize}
\item
\item
\item
\item
\item
\end{itemize}

Advertisement

\protect\hyperlink{after-top}{Continue reading the main story}

\href{/section/opinion}{Opinion}

Supported by

\protect\hyperlink{after-sponsor}{Continue reading the main story}

\hypertarget{trumps-re-election-message-is-white-grievance}{%
\section{Trump's Re-election Message Is White
Grievance}\label{trumps-re-election-message-is-white-grievance}}

Republicans in D.C. just pretend not to see it.

\href{https://www.nytimes.com/by/michelle-goldberg}{\includegraphics{https://static01.nyt.com/images/2018/04/02/opinion/michelle-goldberg/michelle-goldberg-thumbLarge.png}}

By \href{https://www.nytimes.com/by/michelle-goldberg}{Michelle
Goldberg}

Opinion Columnist

\begin{itemize}
\item
  July 2, 2020
\item
  \begin{itemize}
  \item
  \item
  \item
  \item
  \item
  \end{itemize}
\end{itemize}

\includegraphics{https://static01.nyt.com/images/2020/07/03/opinion/03goldbergWeb/03goldbergWeb-articleLarge.jpg?quality=75\&auto=webp\&disable=upscale}

A lot of Republicans are acting puzzled about Donald Trump's re-election
pitch. ``He has no message,'' one Republican source
\href{https://www.reuters.com/article/us-usa-election-trump/fearing-an-election-loss-trump-allies-push-him-to-be-less-polarizing-idUSKBN2427GT}{told
Reuters}. ``He needs to articulate why he wants a second term,'' said
another. Some have expressed hope that Trump would find a way to become
less polarizing, as if polarization were not the raison d'être of his
presidency.

It's hard to know if Republicans like this are truly naïve or if they're
just pretending so they don't have to admit what a foul enterprise
they're part of. Because Trump does indeed have a re-election message, a
stark and obvious one. It is ``white power.''

The president started this week by tweeting out a video that
encapsulates the soul of his movement. In it, a man in The Villages, an
affluent Florida retirement community, shouts, ``White power!'' at
protesters from a golf cart bedecked with Trump signs. ``Thank you to
the great people of The Villages,'' wrote Trump. Only after several
hours and a
\href{https://www.washingtonpost.com/politics/trump-white-power-tweet-set-off-a-scramble-inside-the-white-house--but-no-clear-condemnation/2020/06/29/6fd88c2c-ba21-11ea-8cf5-9c1b8d7f84c6_story.html}{panic
among White House staffers} did the president delete the tweet.

His spokesman claimed he hadn't heard his supporter's extremely clear
words. Trump, naturally, never disavowed them.

And why would he? Republicans might act as if they don't know why
Trump's fans are so unfailingly loyal. Some commentators spent the first
year or two of his presidency dancing around the reason he was elected,
spending so much time probing the ``economic anxiety'' of his base that
the phrase came to stand for a type of willful political blindness.

But Trump understands that he became a significant political figure by
spreading the racist lie that Barack Obama was really born in Kenya. He
launched his history-making presidential bid with a speech calling
Mexican immigrants rapists and adopted a slogan, ``America First,''
previously associated with the raging anti-Semite Charles Lindbergh.
Throughout the 2016 campaign, he won the invaluable prize of earned
media with escalating racist provocations, which his supporters relished
and which captivated cable news.

People voted for Trump for reasons besides racism. There was also
sexism. Some voters were just partisan Republicans, or thought that
reality TV is real and that Trump was as successful as ``The
Apprentice'' made him seem. I once met a young man at a Trump rally
who'd voted for Obama but was worried about the taxes he'd pay when he
inherited his family's car dealership.

Trump, however, seems to grasp that racism is what put him over the top.
It's what made his campaign seem wild and transgressive and hard to look
away from.

Now Trump's poll numbers are cratering, we have double-digit
unemployment and our pandemic-ravaged nation has been rendered an
international pariah. America is faring exactly as well under Trump's
leadership as his casinos, airline and scam university did. It's not
surprising that he's returning to what he knows, and what seemed to work
for him before.

In fact, Trump appears to think his problem is that he hasn't been
racist enough. On Wednesday,
\href{https://www.axios.com/trump-kushner-second-thoughts-408d5a33-725d-442a-88e4-d6ab6742c139.html}{Axios's
Jonathan Swan reported} that Trump regrets listening to his son-in-law
Jared Kushner's ``woke'' ideas --- as a source put it --- including on
criminal justice reform. Instead, he wants to double down on law and
order. ``He truly believes there is a silent majority out there that's
going to come out in droves in November,'' a source told Swan.

And so last week, as if to prod that silent majority, Trump
\href{https://www.washingtonpost.com/nation/2020/06/23/trump-videos-black-violence-protests/}{tweeted
out videos} of Black people assaulting white people. (``Where are the
protesters?'' he asked.) He has made a point of calling the coronavirus
the ``kung flu.'' At a time when even Mississippi is removing
Confederate imagery from its state flag, Trump has thrown himself into
the protection of what he calls ``our heritage.''

He signed an
\href{https://www.vox.com/policy-and-politics/2020/6/27/21305396/trump-confederate-monuments-executive-order}{executive
order} directing federal law enforcement to prosecute people who damage
federal monuments --- threatening them with up to 10 years in prison ---
and withholding funds from municipalities that don't protect statues.
(Whether this latter provision is enforceable is unclear.) He
\href{https://www.nytimes.com/aponline/2020/07/01/us/politics/ap-us-congress-confederate-symbols.html}{said
he'd veto} a \$741 billion defense bill over a provision, written by
Senator Elizabeth Warren of Massachusetts, requiring that military bases
honoring Confederates be renamed. Apoplectic over New York City's plans
to paint the words ``Black Lives Matter'' on Fifth Avenue in front of
Trump Tower, he called the slogan ``a symbol of hate.''

On Tuesday,
\href{https://twitter.com/realDonaldTrump/status/1278136326647406593?s=20}{Trump
tweeted} that he was considering scrapping an Obama-era housing
regulation that required localities to address illegal patterns of
residential segregation. He claimed that the initiative, which his
administration had already put in limbo, was having a ``devastating
impact on these once thriving Suburban areas.''

The message to his white supporters seemed clear enough: Trump is going
to fight to stop people of color from coming to your neighborhood.

The Times reported on the
\href{https://www.nytimes.com/2020/07/01/us/politics/trump-obama-housing-discrimination.html?action=click\&module=Top\%20Stories\&pgtype=Homepage}{president's
rationale}: ``Mr. Trump and his campaign team, already concerned about
his weakness in battleground states, have become increasingly alarmed by
internal polling showing a softening of support among suburban voters.''
Trump sees clearly --- more clearly than most of his party --- that
racism is the main thing he has to offer.

There's good reason to think that he's misjudging these suburban voters.
Polls show that a growing number of them, particularly women, are
repelled by Trump's race-baiting and divisiveness. But Republicans who
complain that the president is undisciplined, that he can't adhere to a
strategy, miss the point: Bigotry has always been the strategy.

The Republicans who support him are yoked to that strategy. Their real
frustration isn't that it's ugly but that it's no longer working.

\emph{The Times is committed to publishing}
\href{https://www.nytimes.com/2019/01/31/opinion/letters/letters-to-editor-new-york-times-women.html}{\emph{a
diversity of letters}} \emph{to the editor. We'd like to hear what you
think about this or any of our articles. Here are some}
\href{https://help.nytimes.com/hc/en-us/articles/115014925288-How-to-submit-a-letter-to-the-editor}{\emph{tips}}\emph{.
And here's our email:}
\href{mailto:letters@nytimes.com}{\emph{letters@nytimes.com}}\emph{.}

\emph{Follow The New York Times Opinion section on}
\href{https://www.facebook.com/nytopinion}{\emph{Facebook}}\emph{,}
\href{http://twitter.com/NYTOpinion}{\emph{Twitter (@NYTopinion)}}
\emph{and}
\href{https://www.instagram.com/nytopinion/}{\emph{Instagram}}\emph{.}

Advertisement

\protect\hyperlink{after-bottom}{Continue reading the main story}

\hypertarget{site-index}{%
\subsection{Site Index}\label{site-index}}

\hypertarget{site-information-navigation}{%
\subsection{Site Information
Navigation}\label{site-information-navigation}}

\begin{itemize}
\tightlist
\item
  \href{https://help.nytimes.com/hc/en-us/articles/115014792127-Copyright-notice}{©~2020~The
  New York Times Company}
\end{itemize}

\begin{itemize}
\tightlist
\item
  \href{https://www.nytco.com/}{NYTCo}
\item
  \href{https://help.nytimes.com/hc/en-us/articles/115015385887-Contact-Us}{Contact
  Us}
\item
  \href{https://www.nytco.com/careers/}{Work with us}
\item
  \href{https://nytmediakit.com/}{Advertise}
\item
  \href{http://www.tbrandstudio.com/}{T Brand Studio}
\item
  \href{https://www.nytimes.com/privacy/cookie-policy\#how-do-i-manage-trackers}{Your
  Ad Choices}
\item
  \href{https://www.nytimes.com/privacy}{Privacy}
\item
  \href{https://help.nytimes.com/hc/en-us/articles/115014893428-Terms-of-service}{Terms
  of Service}
\item
  \href{https://help.nytimes.com/hc/en-us/articles/115014893968-Terms-of-sale}{Terms
  of Sale}
\item
  \href{https://spiderbites.nytimes.com}{Site Map}
\item
  \href{https://help.nytimes.com/hc/en-us}{Help}
\item
  \href{https://www.nytimes.com/subscription?campaignId=37WXW}{Subscriptions}
\end{itemize}
