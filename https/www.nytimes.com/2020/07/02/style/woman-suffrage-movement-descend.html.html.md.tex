Sections

SEARCH

\protect\hyperlink{site-content}{Skip to
content}\protect\hyperlink{site-index}{Skip to site index}

\href{https://www.nytimes.com/section/style}{Style}

\href{https://myaccount.nytimes.com/auth/login?response_type=cookie\&client_id=vi}{}

\href{https://www.nytimes.com/section/todayspaper}{Today's Paper}

\href{/section/style}{Style}\textbar{}My \_\_\_ Was a Suffragist

\href{https://nyti.ms/2VDbwoJ}{https://nyti.ms/2VDbwoJ}

\begin{itemize}
\item
\item
\item
\item
\item
\item
\end{itemize}

Advertisement

\protect\hyperlink{after-top}{Continue reading the main story}

Supported by

\protect\hyperlink{after-sponsor}{Continue reading the main story}

\hypertarget{my-___-was-a-suffragist}{%
\section{My \_\_\_ Was a Suffragist}\label{my-___-was-a-suffragist}}

One hundred years after the 19th Amendment, suffragists' descendants
consider how far we've come and how far we still have to go.

\includegraphics{https://static01.nyt.com/images/2020/07/03/multimedia/03suffrage-blank-17/03suffrage-blank-17-articleLarge.jpg?quality=75\&auto=webp\&disable=upscale}

\href{https://www.nytimes.com/by/jennifer-harlan}{\includegraphics{https://static01.nyt.com/images/2019/09/25/reader-center/author-jennifer-harlan/author-jennifer-harlan-thumbLarge-v2.png}}

By \href{https://www.nytimes.com/by/jennifer-harlan}{Jennifer Harlan}

\begin{itemize}
\item
  Published July 2, 2020Updated Aug. 6, 2020
\item
  \begin{itemize}
  \item
  \item
  \item
  \item
  \item
  \item
  \end{itemize}
\end{itemize}

On Aug. 18, 1920, Tennessee became the 36th state to ratify the 19th
Amendment. Eight days later, ratification was certified by the secretary
of state. The right to vote for women across the United States was
officially enshrined in the Constitution.

The codification of suffrage was the result of nearly a century of
activism, which began even before the Seneca Falls convention in 1848.
From those early years to the formation of the National American Woman
Suffrage Association (NAWSA) in 1890 to the Woman Suffrage Procession in
Washington in 1913, generations of American women and men devoted their
lives to fighting for the vote. The movement was a decades-long game of
democratic Telephone: Of the 68 women who gathered in that town in
upstate New York and declared what was then a radical notion --- that
all men \emph{and} women were created equal --- only one,
\href{https://www.smithsonianmag.com/smart-news/only-one-woman-who-was-seneca-falls-lived-see-women-win-vote-180964044/}{Charlotte
Woodward Pierce}, would live to see their dream become a reality.

And their struggle did not end with the amendment. Well after 1920,
there were many women in the United States, including Native Americans
and Chinese immigrants, who were not able to vote and many more,
particularly African-Americans, for whom it was extremely difficult. One
hundred years later, the country is continuing to grapple with many of
the same questions the suffragists raised, not only who gets to vote but
also what it means to be a citizen and how to ensure that all Americans
are equal in the eyes of the law. And as we mark this centennial, the
generation that came after the suffragists, and the ones that have come
after that, are still in the fight.

``Learning your history is an essential tool, and a call to action,''
said Liza Mickens, a great-great-granddaughter of the suffragist Maggie
Lena Walker. ``I'm honored to be a part of this legacy.'' (Interviews
have been edited for length and clarity.)

\hypertarget{adele-logan-alexander-82-new-york}{%
\subsection{Adele Logan Alexander, 82, New
York}\label{adele-logan-alexander-82-new-york}}

\emph{Granddaughter of Adella Hunt Logan}

Image

The suffragist Adella Hunt Logan in her wedding dress in Atlanta in
1888. A teacher at the Tuskegee Institute in Alabama, she fought for
suffrage, social reform and better health care for black
communities.Credit...Thomas Askew, reproduction by Mark Gulezian

Image

Adella, third from left, with her family at the Tuskegee Institute in
1913, celebrating her 25th anniversary with her husband, Warren Logan,
center. Their son Arthur, standing in front of Warren, was the father of
the historian Adele Logan Alexander.Credit...Arthur P. Bedou,
reproduction by Mark Gulezian

Adella's portrait --- the one that's the cover of my book
``\href{https://yalebooks.yale.edu/book/9780300242607/princess-hither-isles}{Princess
of the Hither Isles: A Black Suffragist's Story from the Jim Crow
South}'' --- hung in my parents' apartment, so she's always been a
visible, physical presence in my life, even though she died when my
father was only 6 years old. And, of course, I was named for her. When I
was in my early 40s, a young historian who was working on a Ph.D.
dissertation about Black women in the suffrage movement was the first
one to show me Adella's writings. I hadn't had that kind of specifics
before, and I got totally hooked on trying to find out more about her as
a suffragist. My father was dead by that time, so I asked my mother,
``Did you know all this?'' And her response was, ``Of course.'' Not only
was Adella involved, but my mother's mother was, too: They were both
light-skinned enough to pass, so the two grandmothers, well before my
parents were born, would go together to white suffrage conferences in
the South, and then come back and share the information they learned
with the Black community.

But even before I knew the details of my family's story, voting and
political involvement were very much a part of my growing up. One of my
first memories is of my hand reaching up for my mother's as she walked
with me to the New York Public Library, down through this back labyrinth
to where the voting booths were. I always knew that this was something
very precious, and not to be taken for granted.

\hypertarget{coline-jenkins-68-greenwich-conn}{%
\subsection{Coline Jenkins, 68, Greenwich,
Conn.}\label{coline-jenkins-68-greenwich-conn}}

\emph{Great-great-granddaughter of Elizabeth Cady Stanton,
great-granddaughter of Harriot Stanton Blatch, granddaughter of Nora
Stanton Blatch de Forest Barney}

Image

Elizabeth Cady Stanton with her daughter Harriot Stanton Blatch and her
granddaughter Nora Stanton Blatch de Forest Barney. All three
generations of Stanton women, pictured here in 1892, fought for the
right to vote.Credit...Coline Jenkins

Image

In 1913, Nora rode around New York on a horse named for the state's
anti-suffrage senator Elihu Root, giving speeches advocating for women's
right to vote. ``When you ride in on a horse, you don't need a podium:
You're already up there,'' her granddaughter Coline Jenkins
said.Credit...via Coline Jenkins

My dad had died when I was little, so it was my mother and my
grandmother, these two very strong women, who raised me. They both had
jobs --- my grandmother Nora was the first woman in the country to get a
degree in civil engineering and my mother was an architect --- and they
were civically active all the time. Nobody sat me down and told me about
Elizabeth or Harriot: I learned vicariously through these two powerful
women in my life. Then when I was around 17 I started visiting some of
the historical sites and reading Elizabeth's memoirs and putting
together the mosaic of values and ideas that have been expressed through
the generations.

If you want to know about democracy, and the tools of democracy, then
learn about the suffrage movement. This was warfare, and the suffragists
used every single weapon available: petitions, lobbying, newspapers,
speeches, marches --- everything except the gun. That's why it's called
the world's greatest bloodless revolution. These women worked their full
lifetimes, and then the next generation did the same, and the next. I'm
very proud to be the daughter of the daughter of the daughter of the
daughter. People talk about the mitochondrial DNA that passes down
through the mother. That's the powerhouse within the cell. And I'm proud
to have that power boost.

\emph{{[}The New York Times is examining the centennial of the 19th
Amendment in many ways, including a book for middle-grade readers called
``}\href{https://www.nytimes.com/2020/07/24/books/finish-the-fight-excerpt.html}{\emph{Finish
the Fight!: The Brave and Revolutionary Women Who Fought for the Right
to Vote}}\emph{.''{]}}

\hypertarget{michelle-duster-56-chicago}{%
\subsection{Michelle Duster, 56,
Chicago}\label{michelle-duster-56-chicago}}

\emph{Great-granddaughter of}
\href{https://www.nytimes.com/interactive/2018/obituaries/overlooked-ida-b-wells.html}{\emph{Ida
B. Wells-Barnett}}

Image

The journalist and suffragist Ida B. Wells-Barnett with her daughters,
Ida and Alfreda, in 1914. The year before, Wells-Barnett marched with
the Illinois delegation in the Woman Suffrage Procession in Washington.
White organizers told her and other Black women to walk at the back, but
she refused.Credit...Special Collections Research Center, University of
Chicago Library

Image

Alfreda was responsible for posthumously publishing Wells-Barnett's
autobiography, ``Crusade for Justice,'' in 1970. ``I was a child at the
time and didn't understand the magnitude of what that meant,'' Alfreda's
granddaughter Michelle Duster said. ``All of the tributes and the
research that are happening now are due to my grandmother's
work.''Credit...Special Collections Research Center, University of
Chicago Library

My grandmother Alfreda was very purposeful when I was growing up about
making sure we knew who her mother was, and passing down her values,
while also not putting pressure on us to live up to anybody else's
legacy. People underestimate how much pressure there is for children
dealing with a parent who's such a public figure, and I think that's why
my grandmother was so determined that we would have our own identities.

At the same time, she was relentless about making sure Ida was not
forgotten. She was the one who found the manuscript for her mother's
incomplete autobiography. She was a widow raising five children on her
own, and she would stay up working on it when the kids were asleep. She
managed to get it
\href{https://press.uchicago.edu/ucp/books/book/chicago/C/bo49856620.html}{published}
in 1970. We now have four generations of my family that have created or
supported some kind of
\href{https://www.simonandschuster.com/books/Ida-B-the-Queen/Michelle-Duster/9781982129811}{work
around Ida's legacy}, and it all started with my grandmother.

I can only imagine that Ida might be slightly disappointed if she saw
where we are now. She spent over 50 years fighting for equality and
justice, and here we are, almost 90 years after she died, still
fighting. We still haven't reached the point where there is true racial
and economic and gender equality in this country. Then again, maybe she
wouldn't be surprised, because she knew that social change takes a long
time.

\hypertarget{sandra-shreve-79-denver}{%
\subsection{Sandra Shreve, 79, Denver}\label{sandra-shreve-79-denver}}

\emph{Great-great-niece of}
\href{https://www.nytimes.com/2018/06/06/obituaries/mary-ann-shadd-cary-abolitionist-overlooked.html}{\emph{Mary
Ann Shadd Cary}}

Image

Born in Delaware, the child of abolitionists, Mary Ann Shadd Cary moved
to Canada and started~The Provincial Freeman, making her~the first Black
woman in North America to publish a newspaper. Circa
1850.Credit...Library and Archives Canada/Mary Ann Shadd Cary
collection/c029977

Most of what I know about Mary came from reading books written by others
and from reading things she wrote herself. She was a woman who would go
all over, selling her newspaper (the first to be published by a Black
woman in North America) and sharing ideas of freedom and liberty. I've
been just fascinated by her and her life and how feisty she was and the
things she was able to accomplish at a time when women were not being
heard.

In 1998, she was inducted into the National Women's Hall of Fame in
Seneca Falls, N.Y. I went to the ceremony with my cousin Dorothy Shadd
Shreve. The evening before, we were all given candles and lined up, 10
abreast, to march down the main road to the hall. Dorothy was in her
90s, but there she was, in her high heels, walking very forcefully down
the street. As we marched arm-in-arm with all these remarkable women,
the people from the town lined up on both sides of the street and
applauded. I remember Dorothy saying, ``And we're still fighting for the
rights of women!'' I'll never forget that experience.

\hypertarget{sarah-plimpton-83-new-york}{%
\subsection{Sarah Plimpton, 83, New
York}\label{sarah-plimpton-83-new-york}}

\emph{Granddaughter of Blanche Ames Ames}

Image

In addition to being an advocate for suffrage, reproductive rights and
equal employment opportunities for women, Blanche Ames Ames, seen here
in 1922, was a talented artist and botanical illustrator.Credit...Ames
Family Papers, Sophia Smith Collection, Smith College

Image

Ames used her skills to create numerous cartoons, such as this one from
1915, to sway public opinion toward the suffragist cause.Credit...Ames
Family Papers, Sophia Smith Collection, Smith College

My grandmother was absolutely wonderful --- just this extraordinary,
vibrant person. I remember, in the early days of television, she would
sit in front of it and argue with whomever she was listening to. She
would never take anything for granted or believe what someone said if
she couldn't verify it herself.

We heard a lot around the dinner table about her struggle to get support
for a women's hospital in Boston, and she and my mother were involved
with Planned Parenthood. But when it came to women and the vote, I think
we all took it for granted. I never doubted that a woman could do
anything. Maybe that was naïve. But she imbued us with this feeling that
you could pursue anything you put your mind to.

\hypertarget{david-steele-ewing-53-nashville}{%
\subsection{David Steele Ewing, 53,
Nashville}\label{david-steele-ewing-53-nashville}}

\emph{Great-great-grandson of Isabella Ewing}

\includegraphics{https://static01.nyt.com/images/2020/08/16/multimedia/03suffrage-blank-05/merlin_173928057_2ebb782d-5c73-40c1-9d05-a64d356a835f-articleLarge.jpg?quality=75\&auto=webp\&disable=upscale}

Image

Isabella's voter registration card from October 1920. She was among the
first African-American women who registered to vote in Tennessee in that
fall's presidential election, the first in which they were able to
participate.Credit...via David Steele Ewing

My father died when I was 2 years old, so I grew up not knowing a lot
about the Ewing family. But at a family reunion about 25 years ago, I
kept hearing about Prince Albert Ewing and Isabella, my
great-great-grandparents. It was a surprise to me to hear about people
who I'm related to, who lived in the city where I've always lived and
who were so involved in the Nashville African-American community.

This was before the days of online genealogy, so I did the old-fashioned
work of going to courthouses and libraries to discover more information.
They were both enslaved, Prince Albert at a place called
\href{https://historictravellersrest.org/}{Travellers Rest} and Isabella
at the \href{https://thehermitage.com/}{Hermitage}. They were married in
1871 and purchased some land near the Hermitage, where they built a
house. That's where I found their voting cards. Prince Albert was a
magistrate in the 1880s. He was elected three times. But Isabella
couldn't vote for him. So it was very important, even though he was no
longer on the bench, that she went to register as soon as she could.

There's all this talk that women were ``given'' the right to vote. Women
were not given the right to vote: They fought for the right to vote.
They organized for the right to vote. They demanded the right to vote.
This didn't happen by magic. They really had to fight for it.

\hypertarget{pamela-michael-71-hudson-nh}{%
\subsection{Pamela Michael, 71, Hudson,
N.H.}\label{pamela-michael-71-hudson-nh}}

\emph{Granddaughter of Frank Tafe and Delia Lefavor Tafe}

Image

Frank Tafe, standing, and his future wife, Delia Lefavor, seated at
right in the truck's passenger seat, at a suffrage parade in Nashua,
N.H., circa 1918. Like many suffragists, the marchers wore all-white
dresses.Credit...via Pamela Michael

My grandmother was quite a gal. She was one of the first women to
graduate from Nashua High School, and she had a career as a secretary at
the Nashua Card, Gummed and Coated Paper Company (later known as the
Nashua Corporation) before she married at 32. She was unique and
progressive for her time, very independent. And my grandfather was very
supportive. Both of my grandparents stood up for what they believed in.

He died before I was born, and she died when I was about 13. My
grandmother never really talked to me about women's rights in much
detail because I was so young. But I've been very active politically, so
it must be in the blood. Speaking out seems to run in the family, and
the women are quite strong. We don't sit back. We're not passive. And
Delia wasn't either.

\hypertarget{liza-mickens-23-richmond-va}{%
\subsection{Liza Mickens, 23, Richmond,
Va.}\label{liza-mickens-23-richmond-va}}

\emph{Great-great-granddaughter of Maggie Lena Walker}

Image

Maggie Lena Walker in her office in Richmond, Va., circa 1910. She was
the first Black female president of a U.S. bank and used her influence
to push for women's suffrage in Virginia.Credit...Courtesy of National
Park Service, Maggie L. Walker National Historic Site

Image

In her diary, Walker marked Nov. 2, 1920 --- Election Day --- as a
holiday. She registered hundreds of Black women to vote after the 19th
Amendment passed, ensuring that they would be able to participate in
that first, historic trip to the polls.Credit...Courtesy of National
Park Service, Maggie L. Walker National Historic Site

My brother and I were brought up telling Maggie Walker's story, and it's
a responsibility I take very seriously. But the family narrative I grew
up telling was focused on the work she did as the first African-American
woman to charter a bank in the United States. It wasn't until recently,
as part of my work campaigning in Virginia for the
\href{https://www.nytimes.com/2020/01/15/us/era-virginia-vote.html}{Equal
Rights Amendment}, that I really learned about her involvement in the
political field. Knowing her, though, it didn't come as a surprise: In
everything she did, her focus was on empowering her community.

She was not necessarily out and marching --- Maggie was partially
disabled because of diabetes --- but she organized. When women did get
the right to vote, she made sure it wasn't just for white women: She
registered hundreds of Black women to vote that first year. She and
other contemporaries also formed a
\href{https://digitalsc.lib.vt.edu/exhibits/show/womens-history-2016/item/4697}{``Lily
Black'' ticket}, in response to the ``lily white'' ticket in the
Republican Party, and in 1921 she became the first Black woman to run
for statewide office in Virginia.

This 100th anniversary is coming at a pivotal time. We're seeing the
call for Black voices to be highlighted in this country, where they've
been silenced for so long. It's so important for young Black women to
have somebody like Maggie to look up to, and being able to use my voice
to share her story is a huge honor.

\hypertarget{william-bellamy-73-laramie-wyo}{%
\subsection{William Bellamy, 73, Laramie,
Wyo.}\label{william-bellamy-73-laramie-wyo}}

\emph{Great-grandson of Mary Godat Bellamy}

Image

In 1910, around the time this portrait was taken, Mary Godat Bellamy was
elected as the first woman in Wyoming's House of Representatives. She
served one term.Credit...via William Bellamy

Image

Bellamy represented Wyoming, the first state in which women could vote,
at the NAWSA convention in Washington in December 1917. She was invited
by Carrie Chapman Catt, then the group's president.Credit...via William
Bellamy

I was 7 when Mary passed away, so I remember her just as great-grandma.
She had a corner store where we would hang out when I was a kid, and she
would read our palms. I always kind of knew what she did --- we all knew
she was the first woman legislator in Wyoming and that she had worked
hard in the Democratic Party for suffrage --- but it wasn't until fairly
recently that I spent more time researching her accomplishments.

The history of political movements is forgotten pretty rapidly after
things are changed one way or the other --- how the movement went
forward, how long sometimes it takes to change a social norm and what
needs to be done to get actual change accomplished. What we learn from
the suffragists is how hard people worked to get something done for
future generations because they believed it was truly the right thing to
do. My great-grandma taught a kind of Jeffersonian attitude: You do what
you can do for society with the gifts you're given. That was how she
looked at life, and that's what she did.

Image

1. Harriot Stanton Blatch, 1911. Library of Congress

2. Maggie Lena Walker, circa 1920s. Courtesy of National Park Service,
Maggie L. Walker National Historic Site

3. Ida B. Wells-Barnett with her family, 1917. Special Collections
Research Center, University of Chicago Library

4. Ida B. Wells-Barnett with her children, 1909. Special Collections
Research Center, University of Chicago Library

5. Mary Godat Bellamy, circa 1910. via William Bellamy

6. Adella Hunt Logan with her family, 1913. Arthur P. Bedou,
reproduction by Mark Gulezian

7. Blanche Ames Ames with her daughter Pauline Ames Plimpton, mother of
Sarah Plimpton, and her husband, Oakes Ames. Undated. Ames Family
Papers, Sophia Smith Collection, Smith College

8. Mary Ann Shadd Cary, circa 1850. Library and Archives Canada/Mary Ann
Shadd Cary collection/c029977

9. Isabella Ewing, 1916. via David Steele Ewing

10. Blanche Ames Ames, 1899. Ames Family Papers, Sophia Smith
Collection, Smith College

11. Maggie Lena Walker with her family, circa 1920. Courtesy of National
Park Service, Maggie L. Walker National Historic Site

12. Nora Stanton Blatch de Forest Barney, Elizabeth Cady Stanton and
Harriot Stanton Blatch, left to right, 1892. via Coline Jenkins

13. Frank Tafe and Delia Lefavor Tafe, circa 1918. via Pamela Michael

Advertisement

\protect\hyperlink{after-bottom}{Continue reading the main story}

\hypertarget{site-index}{%
\subsection{Site Index}\label{site-index}}

\hypertarget{site-information-navigation}{%
\subsection{Site Information
Navigation}\label{site-information-navigation}}

\begin{itemize}
\tightlist
\item
  \href{https://help.nytimes.com/hc/en-us/articles/115014792127-Copyright-notice}{©~2020~The
  New York Times Company}
\end{itemize}

\begin{itemize}
\tightlist
\item
  \href{https://www.nytco.com/}{NYTCo}
\item
  \href{https://help.nytimes.com/hc/en-us/articles/115015385887-Contact-Us}{Contact
  Us}
\item
  \href{https://www.nytco.com/careers/}{Work with us}
\item
  \href{https://nytmediakit.com/}{Advertise}
\item
  \href{http://www.tbrandstudio.com/}{T Brand Studio}
\item
  \href{https://www.nytimes.com/privacy/cookie-policy\#how-do-i-manage-trackers}{Your
  Ad Choices}
\item
  \href{https://www.nytimes.com/privacy}{Privacy}
\item
  \href{https://help.nytimes.com/hc/en-us/articles/115014893428-Terms-of-service}{Terms
  of Service}
\item
  \href{https://help.nytimes.com/hc/en-us/articles/115014893968-Terms-of-sale}{Terms
  of Sale}
\item
  \href{https://spiderbites.nytimes.com}{Site Map}
\item
  \href{https://help.nytimes.com/hc/en-us}{Help}
\item
  \href{https://www.nytimes.com/subscription?campaignId=37WXW}{Subscriptions}
\end{itemize}
