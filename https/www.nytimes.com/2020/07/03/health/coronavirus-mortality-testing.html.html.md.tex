Sections

SEARCH

\protect\hyperlink{site-content}{Skip to
content}\protect\hyperlink{site-index}{Skip to site index}

\href{https://www.nytimes.com/section/health}{Health}

\href{https://myaccount.nytimes.com/auth/login?response_type=cookie\&client_id=vi}{}

\href{https://www.nytimes.com/section/todayspaper}{Today's Paper}

\href{/section/health}{Health}\textbar{}U.S. Coronavirus Cases Are
Rising Sharply, but Deaths Are Still Down

\url{https://nyti.ms/2AsIgtn}

\begin{itemize}
\item
\item
\item
\item
\item
\end{itemize}

\href{https://www.nytimes.com/news-event/coronavirus?action=click\&pgtype=Article\&state=default\&region=TOP_BANNER\&context=storylines_menu}{The
Coronavirus Outbreak}

\begin{itemize}
\tightlist
\item
  live\href{https://www.nytimes.com/2020/08/04/world/coronavirus-cases.html?action=click\&pgtype=Article\&state=default\&region=TOP_BANNER\&context=storylines_menu}{Latest
  Updates}
\item
  \href{https://www.nytimes.com/interactive/2020/us/coronavirus-us-cases.html?action=click\&pgtype=Article\&state=default\&region=TOP_BANNER\&context=storylines_menu}{Maps
  and Cases}
\item
  \href{https://www.nytimes.com/interactive/2020/science/coronavirus-vaccine-tracker.html?action=click\&pgtype=Article\&state=default\&region=TOP_BANNER\&context=storylines_menu}{Vaccine
  Tracker}
\item
  \href{https://www.nytimes.com/2020/08/02/us/covid-college-reopening.html?action=click\&pgtype=Article\&state=default\&region=TOP_BANNER\&context=storylines_menu}{College
  Reopening}
\item
  \href{https://www.nytimes.com/live/2020/08/04/business/stock-market-today-coronavirus?action=click\&pgtype=Article\&state=default\&region=TOP_BANNER\&context=storylines_menu}{Economy}
\end{itemize}

Advertisement

\protect\hyperlink{after-top}{Continue reading the main story}

Supported by

\protect\hyperlink{after-sponsor}{Continue reading the main story}

\hypertarget{us-coronavirus-cases-are-rising-sharply-but-deaths-are-still-down}{%
\section{U.S. Coronavirus Cases Are Rising Sharply, but Deaths Are Still
Down}\label{us-coronavirus-cases-are-rising-sharply-but-deaths-are-still-down}}

This seemingly counterintuitive trend might not last, experts said. But
the nation can still learn from the decline.

\includegraphics{https://static01.nyt.com/images/2020/07/03/science/03virus-deathrates/merlin_174113760_429dffad-438b-44b4-aaa4-2c6113a95001-articleLarge.jpg?quality=75\&auto=webp\&disable=upscale}

By \href{https://www.nytimes.com/by/katherine-j--wu}{Katherine J. Wu}

\begin{itemize}
\item
  Published July 3, 2020Updated July 22, 2020
\item
  \begin{itemize}
  \item
  \item
  \item
  \item
  \item
  \end{itemize}
\end{itemize}

After a minor late-spring lull, the number of confirmed
\href{https://www.nytimes.com/2020/07/22/us/florida-mother-2-children-covid-19.html}{coronavirus}
cases in the United States is
\href{https://www.nytimes.com/interactive/2020/us/coronavirus-us-cases.html}{once
again on the rise}. States like Arizona, Florida and Texas are seeing
some of their highest numbers to date, and as the nation hurtles further
into summer, the surge shows few signs of stopping.

And yet the virus appears to be killing fewer of the people it infects.
In April and May, Covid-19, the disease caused by the coronavirus, led
to as many as 3,000
\href{https://www.nytimes.com/2020/07/22/us/florida-mother-2-children-covid-19.html}{deaths}
per day, and claimed the lives of roughly 7 to 8 percent of Americans
known to have been infected. The number of daily deaths is
\href{https://www.nytimes.com/2020/07/02/briefing/coronavirus-jobs-numbers-joe-biden-your-thursday-briefing.html}{now
closer to 600}, and the death rate is less than 5 percent.

In general, experts see three broad reasons for the downward trend in
the rate of coronavirus deaths: testing, treatment and a shift in whom
the virus is infecting. The relative contribution of these factors is
not yet clear. And because death reports can lag diagnoses by weeks, the
current rise in coronavirus cases could still portend increases in
mortality in the days to come.

\hypertarget{testing-on-the-rise}{%
\subsection{Testing on the rise}\label{testing-on-the-rise}}

Since mid-March, when the coronavirus was declared a national emergency,
diagnostic testing for the coronavirus has risen significantly. More
than 600,000 tests are administered each day in the United States, up
from about 100,000 per day in early spring. Although the nation is still
falling short of the
\href{https://www.kff.org/coronavirus-policy-watch/what-testing-capacity-do-we-need/}{millions
of daily tests that experts have called for}, the increased testing has
identified many more infected individuals with mild or no symptoms,
driving down the overall proportion of patients who die from Covid-19,
said Caitlin Rivers, a senior scholar at the Johns Hopkins Center for
Health Security.

And with more tests available, infections are often identified earlier,
``which allows us to intervene earlier,'' said Saskia Popescu, a
hospital epidemiologist and infectious disease expert in Arizona. Many
treatments seem to work best when given well before people are at
death's door.

As the weeks have worn on, doctors and nurses have also gained a better
handle on how to treat the coronavirus. In several states, emergency
departments are \href{https://covidtracking.com/data}{no longer
overflowing}; between April and June, nationwide hospitalizations
dropped to less than 30,000 from nearly 60,000,
\href{https://covidtracking.com/data/us-daily}{according to the Covid
Tracking Project}. That may have eased the strain on exhausted employees
and limited medical supply chains, including those that keep lifesaving
equipment like ventilators in stock, said Dr. Taison Bell, a physician
specializing in infectious disease and pulmonary and critical care at
the University of Virginia. Under less pressure, hospitals are now
``better able to take care of critically ill patients,'' he said.

\hypertarget{more-and-better-treatments}{%
\subsection{More and better
treatments}\label{more-and-better-treatments}}

Health care workers have also become more knowledgeable about promising
treatments and palliative care options to combat the coronavirus and its
effects. For instance, prone positioning, in which patients are flipped
onto their stomachs, can
\href{https://jamanetwork.com/journals/jamainternalmedicine/fullarticle/2767575}{ease
respiratory distress} by
\href{https://www.nytimes.com/2020/05/13/health/coronavirus-proning-lungs.html}{opening
up the lungs}. Critically ill individuals are also now known to be
vulnerable to
\href{https://www.nytimes.com/2020/05/14/health/coronavirus-strokes.html}{excessive
blood clotting}, and may
\href{https://ashpublications.org/blood/article/135/23/2033/454646/COVID-19-and-its-implications-for-thrombosis-and}{benefit
from blood thinners}. And the steroid
\href{https://www.nytimes.com/2020/06/16/world/europe/dexamethasone-coronavirus-covid.html}{dexamethasone
appears to reduce deaths} among patients with severe Covid-19, although
the data demonstrating this emerged only recently. (Another drug, an
antiviral called remdesivir, seems to
\href{https://www.nytimes.com/2020/05/23/health/coronavirus-remdesivir.html}{speed
recovery}, but does not appear to have notable effects on mortality.)

\hypertarget{latest-updates-global-coronavirus-outbreak}{%
\section{\texorpdfstring{\href{https://www.nytimes.com/2020/08/04/world/coronavirus-cases.html?action=click\&pgtype=Article\&state=default\&region=MAIN_CONTENT_1\&context=storylines_live_updates}{Latest
Updates: Global Coronavirus
Outbreak}}{Latest Updates: Global Coronavirus Outbreak}}\label{latest-updates-global-coronavirus-outbreak}}

Updated 2020-08-04T20:57:54.346Z

\begin{itemize}
\tightlist
\item
  \href{https://www.nytimes.com/2020/08/04/world/coronavirus-cases.html?action=click\&pgtype=Article\&state=default\&region=MAIN_CONTENT_1\&context=storylines_live_updates\#link-1228a480}{Novavax
  sees encouraging results from two studies of its experimental
  vaccine.}
\item
  \href{https://www.nytimes.com/2020/08/04/world/coronavirus-cases.html?action=click\&pgtype=Article\&state=default\&region=MAIN_CONTENT_1\&context=storylines_live_updates\#link-4825b93}{Public
  and private schools in Maryland and elsewhere are divided over
  in-person instruction.}
\item
  \href{https://www.nytimes.com/2020/08/04/world/coronavirus-cases.html?action=click\&pgtype=Article\&state=default\&region=MAIN_CONTENT_1\&context=storylines_live_updates\#link-50f7386d}{The
  United Nations calls on policymakers to `plan thoroughly for school
  reopenings.'}
\end{itemize}

\href{https://www.nytimes.com/2020/08/04/world/coronavirus-cases.html?action=click\&pgtype=Article\&state=default\&region=MAIN_CONTENT_1\&context=storylines_live_updates}{See
more updates}

More live coverage:
\href{https://www.nytimes.com/live/2020/08/04/business/stock-market-today-coronavirus?action=click\&pgtype=Article\&state=default\&region=MAIN_CONTENT_1\&context=storylines_live_updates}{Markets}

``Before, it felt like we were stumbling in the dark,'' Dr. Bell said.
``It feels a little bit better now.''

\hypertarget{a-new-patient-population}{%
\subsection{A new patient population}\label{a-new-patient-population}}

A shifting patient population is probably also altering the disease's
dynamics. Coronavirus-related hospitalizations
\href{https://www.cdc.gov/coronavirus/2019-ncov/covid-data/covidview/index.html}{increase
with age}, and elderly individuals remain some of those hardest hit by
the coronavirus;
\href{https://www.cdc.gov/coronavirus/2019-ncov/need-extra-precautions/older-adults.html}{patients
over 65 account for eight out of 10 deaths from Covid-19}, according to
the Centers for Disease Control and Prevention. But
\href{https://www.nytimes.com/2020/06/25/us/coronavirus-cases-young-people.html}{younger
people now make up a growing proportion of cases}, and they are less
likely to die from the disease. In Arizona, people ages 20 to 44 now
account for nearly half of all cases. In Florida, which just recorded
\href{https://www.reuters.com/article/us-health-coronavirus-usa-florida/florida-shatters-records-with-over-10000-new-covid-19-cases-in-single-day-idUSKBN243299}{more
than 10,000 new cases} in a single day, the median age of residents
testing positive has dropped to 35 from 65. And in Texas, more than half
of those testing positive are
\href{https://txdshs.maps.arcgis.com/apps/opsdashboard/index.html\#/ed483ecd702b4298ab01e8b9cafc8b83}{under
the age of 50}.

Numerous states recently began reopening their economies, which might be
driving some of the youthful bias, said Natalie Dean, an infectious
disease epidemiologist in Florida, where new cases are hitting record
highs. People in their 20s and 30s have returned to bars and beaches;
working-age employees have resumed jobs that cannot be done from home.

``We know that's high-risk,'' Dr. Dean said. ``We're hearing a lot of
reports of clusters being linked to these places'' as they open back up.

At the same time, elderly individuals, as well as those with
\href{https://www.nytimes.com/2020/06/15/health/coronavirus-underlying-conditions.html}{underlying
health conditions thought to exacerbate Covid-19}, may be warier of
exposure, said C. Brandon Ogbunu, a computational biologist and disease
ecologist at Yale University. ``Early on, this disease ripped through
older populations with such aggression,'' he said. ``It's possible
that's where the message was felt the most strongly.''

Moreover, nursing homes and other facilities that harbor vulnerable
populations may be working harder to protect their residents, Dr. Dean
said. In general, ``We now have a better set of tools to keep our
communities safer,'' he said. ``More people are wearing masks. We're
better at sanitizing things.''

Of course, ``Young people don't live in isolation,'' Dr. Bell said. They
are still mingling with older members of the population --- potentially
seeding transmission events that have yet to appear.

\hypertarget{looking-ahead}{%
\subsection{Looking ahead}\label{looking-ahead}}

Experts can't be sure, but behaviors like mask wearing, physical
distancing and hygiene may also be reducing the dose of coronavirus that
people encounter in the population at large, Dr. Dean said. The amount
of virus that individuals carry
\href{https://www.nytimes.com/2020/05/29/health/coronavirus-transmission-dose.html}{may
influence the severity of their symptoms}. But so far, there is no
evidence that this dynamic is contributing to the lower mortality rate
in the United States.

\href{https://www.nytimes.com/news-event/coronavirus?action=click\&pgtype=Article\&state=default\&region=MAIN_CONTENT_3\&context=storylines_faq}{}

\hypertarget{the-coronavirus-outbreak-}{%
\subsubsection{The Coronavirus Outbreak
›}\label{the-coronavirus-outbreak-}}

\hypertarget{frequently-asked-questions}{%
\paragraph{Frequently Asked
Questions}\label{frequently-asked-questions}}

Updated August 4, 2020

\begin{itemize}
\item ~
  \hypertarget{i-have-antibodies-am-i-now-immune}{%
  \paragraph{I have antibodies. Am I now
  immune?}\label{i-have-antibodies-am-i-now-immune}}

  \begin{itemize}
  \tightlist
  \item
    As of right
    now,\href{https://www.nytimes.com/2020/07/22/health/covid-antibodies-herd-immunity.html?action=click\&pgtype=Article\&state=default\&region=MAIN_CONTENT_3\&context=storylines_faq}{that
    seems likely, for at least several months.} There have been
    frightening accounts of people suffering what seems to be a second
    bout of Covid-19. But experts say these patients may have a
    drawn-out course of infection, with the virus taking a slow toll
    weeks to months after initial exposure. People infected with the
    coronavirus typically
    \href{https://www.nature.com/articles/s41586-020-2456-9}{produce}
    immune molecules called antibodies, which are
    \href{https://www.nytimes.com/2020/05/07/health/coronavirus-antibody-prevalence.html?action=click\&pgtype=Article\&state=default\&region=MAIN_CONTENT_3\&context=storylines_faq}{protective
    proteins made in response to an
    infection}\href{https://www.nytimes.com/2020/05/07/health/coronavirus-antibody-prevalence.html?action=click\&pgtype=Article\&state=default\&region=MAIN_CONTENT_3\&context=storylines_faq}{.
    These antibodies may} last in the body
    \href{https://www.nature.com/articles/s41591-020-0965-6}{only two to
    three months}, which may seem worrisome, but that's perfectly normal
    after an acute infection subsides, said Dr. Michael Mina, an
    immunologist at Harvard University. It may be possible to get the
    coronavirus again, but it's highly unlikely that it would be
    possible in a short window of time from initial infection or make
    people sicker the second time.
  \end{itemize}
\item ~
  \hypertarget{im-a-small-business-owner-can-i-get-relief}{%
  \paragraph{I'm a small-business owner. Can I get
  relief?}\label{im-a-small-business-owner-can-i-get-relief}}

  \begin{itemize}
  \tightlist
  \item
    The
    \href{https://www.nytimes.com/article/small-business-loans-stimulus-grants-freelancers-coronavirus.html?action=click\&pgtype=Article\&state=default\&region=MAIN_CONTENT_3\&context=storylines_faq}{stimulus
    bills enacted in March} offer help for the millions of American
    small businesses. Those eligible for aid are businesses and
    nonprofit organizations with fewer than 500 workers, including sole
    proprietorships, independent contractors and freelancers. Some
    larger companies in some industries are also eligible. The help
    being offered, which is being managed by the Small Business
    Administration, includes the Paycheck Protection Program and the
    Economic Injury Disaster Loan program. But lots of folks have
    \href{https://www.nytimes.com/interactive/2020/05/07/business/small-business-loans-coronavirus.html?action=click\&pgtype=Article\&state=default\&region=MAIN_CONTENT_3\&context=storylines_faq}{not
    yet seen payouts.} Even those who have received help are confused:
    The rules are draconian, and some are stuck sitting on
    \href{https://www.nytimes.com/2020/05/02/business/economy/loans-coronavirus-small-business.html?action=click\&pgtype=Article\&state=default\&region=MAIN_CONTENT_3\&context=storylines_faq}{money
    they don't know how to use.} Many small-business owners are getting
    less than they expected or
    \href{https://www.nytimes.com/2020/06/10/business/Small-business-loans-ppp.html?action=click\&pgtype=Article\&state=default\&region=MAIN_CONTENT_3\&context=storylines_faq}{not
    hearing anything at all.}
  \end{itemize}
\item ~
  \hypertarget{what-are-my-rights-if-i-am-worried-about-going-back-to-work}{%
  \paragraph{What are my rights if I am worried about going back to
  work?}\label{what-are-my-rights-if-i-am-worried-about-going-back-to-work}}

  \begin{itemize}
  \tightlist
  \item
    Employers have to provide
    \href{https://www.osha.gov/SLTC/covid-19/standards.html}{a safe
    workplace} with policies that protect everyone equally.
    \href{https://www.nytimes.com/article/coronavirus-money-unemployment.html?action=click\&pgtype=Article\&state=default\&region=MAIN_CONTENT_3\&context=storylines_faq}{And
    if one of your co-workers tests positive for the coronavirus, the
    C.D.C.} has said that
    \href{https://www.cdc.gov/coronavirus/2019-ncov/community/guidance-business-response.html}{employers
    should tell their employees} -\/- without giving you the sick
    employee's name -\/- that they may have been exposed to the virus.
  \end{itemize}
\item ~
  \hypertarget{should-i-refinance-my-mortgage}{%
  \paragraph{Should I refinance my
  mortgage?}\label{should-i-refinance-my-mortgage}}

  \begin{itemize}
  \tightlist
  \item
    \href{https://www.nytimes.com/article/coronavirus-money-unemployment.html?action=click\&pgtype=Article\&state=default\&region=MAIN_CONTENT_3\&context=storylines_faq}{It
    could be a good idea,} because mortgage rates have
    \href{https://www.nytimes.com/2020/07/16/business/mortgage-rates-below-3-percent.html?action=click\&pgtype=Article\&state=default\&region=MAIN_CONTENT_3\&context=storylines_faq}{never
    been lower.} Refinancing requests have pushed mortgage applications
    to some of the highest levels since 2008, so be prepared to get in
    line. But defaults are also up, so if you're thinking about buying a
    home, be aware that some lenders have tightened their standards.
  \end{itemize}
\item ~
  \hypertarget{what-is-school-going-to-look-like-in-september}{%
  \paragraph{What is school going to look like in
  September?}\label{what-is-school-going-to-look-like-in-september}}

  \begin{itemize}
  \tightlist
  \item
    It is unlikely that many schools will return to a normal schedule
    this fall, requiring the grind of
    \href{https://www.nytimes.com/2020/06/05/us/coronavirus-education-lost-learning.html?action=click\&pgtype=Article\&state=default\&region=MAIN_CONTENT_3\&context=storylines_faq}{online
    learning},
    \href{https://www.nytimes.com/2020/05/29/us/coronavirus-child-care-centers.html?action=click\&pgtype=Article\&state=default\&region=MAIN_CONTENT_3\&context=storylines_faq}{makeshift
    child care} and
    \href{https://www.nytimes.com/2020/06/03/business/economy/coronavirus-working-women.html?action=click\&pgtype=Article\&state=default\&region=MAIN_CONTENT_3\&context=storylines_faq}{stunted
    workdays} to continue. California's two largest public school
    districts --- Los Angeles and San Diego --- said on July 13, that
    \href{https://www.nytimes.com/2020/07/13/us/lausd-san-diego-school-reopening.html?action=click\&pgtype=Article\&state=default\&region=MAIN_CONTENT_3\&context=storylines_faq}{instruction
    will be remote-only in the fall}, citing concerns that surging
    coronavirus infections in their areas pose too dire a risk for
    students and teachers. Together, the two districts enroll some
    825,000 students. They are the largest in the country so far to
    abandon plans for even a partial physical return to classrooms when
    they reopen in August. For other districts, the solution won't be an
    all-or-nothing approach.
    \href{https://bioethics.jhu.edu/research-and-outreach/projects/eschool-initiative/school-policy-tracker/}{Many
    systems}, including the nation's largest, New York City, are
    devising
    \href{https://www.nytimes.com/2020/06/26/us/coronavirus-schools-reopen-fall.html?action=click\&pgtype=Article\&state=default\&region=MAIN_CONTENT_3\&context=storylines_faq}{hybrid
    plans} that involve spending some days in classrooms and other days
    online. There's no national policy on this yet, so check with your
    municipal school system regularly to see what is happening in your
    community.
  \end{itemize}
\end{itemize}

There is also no indication that the death rate is lower because the
coronavirus itself has become less deadly, Dr. Ogbunu said. Mutation is
a normal part of any virus's evolutionary trajectory, but these genetic
changes are
\href{https://undark.org/2020/05/14/covid-19-evolution-mutation/}{often
inconsequential}.

Given the recent rise in infections, the dip in coronavirus mortality
will not necessarily last. As more people socialize, those with milder
infections might end up ferrying the pathogen to vulnerable individuals.
As states reopen, local leaders are urging residents to continue
physical distancing and to wear masks. But even tempered by warnings,
moves back toward normalcy could inadvertently signal to people that the
worst is already over, Dr. Popescu said.

Experts are also reluctant to place too much emphasis on falling death
rates. ``We're training a lot of attention on the idea of mortality,''
said Dr. Jennifer Tsai, an emergency medicine physician at Yale
University. Behind that picture, she added, there is a great deal of
suffering. Reports from around the world have painted a sobering
portrait of
\href{https://www.nytimes.com/2020/07/01/health/coronavirus-recovery-survivors.html}{chronic
Covid-19 syndromes}, some of which last for months. Patients may be
saddled with physical and emotional distress that persists long after
the virus has left their bodies.

``Death is not the only outcome,'' Dr. Dean said. And people
marginalized by race, ethnicity and social standing will inevitably
\href{https://www.cdc.gov/coronavirus/2019-ncov/covid-data/covidview/index.html}{bear
more of the disease burden} than others, Dr. Tsai added. ``The risk and
the mortality is going to be passed on to the most vulnerable, no matter
who gets infected first,'' she said.

Recent upswings in coronavirus case numbers leave experts apprehensive
of what's to come. Death, when it occurs, tends to trail infection
\href{https://www.cdc.gov/coronavirus/2019-ncov/hcp/clinical-guidance-management-patients.html}{by
about two to four weeks}. Early on in the pandemic, when testing focused
on patients with worrisome symptoms, the typical lag between case and
death reporting was a week or two. Now that diagnostic testing is more
widespread, that interval has widened.

Two weeks into a new round of coronavirus cases, the United States may
be verging on another wave of deaths. Already, hospitalizations have
begun
\href{https://www.washingtonpost.com/nation/2020/06/23/coronavirus-live-updates-us/}{an
alarming upsurge} in several states.

``I think the next two to three weeks will be very telling,'' Dr.
Popescu said.

\textbf{\emph{{[}}\href{http://on.fb.me/1paTQ1h}{\emph{Like the Science
Times page on Facebook.}}} ****** \emph{\textbar{} Sign up for the}
\textbf{\href{http://nyti.ms/1MbHaRU}{\emph{Science Times
newsletter.}}\emph{{]}}}

Advertisement

\protect\hyperlink{after-bottom}{Continue reading the main story}

\hypertarget{site-index}{%
\subsection{Site Index}\label{site-index}}

\hypertarget{site-information-navigation}{%
\subsection{Site Information
Navigation}\label{site-information-navigation}}

\begin{itemize}
\tightlist
\item
  \href{https://help.nytimes.com/hc/en-us/articles/115014792127-Copyright-notice}{©~2020~The
  New York Times Company}
\end{itemize}

\begin{itemize}
\tightlist
\item
  \href{https://www.nytco.com/}{NYTCo}
\item
  \href{https://help.nytimes.com/hc/en-us/articles/115015385887-Contact-Us}{Contact
  Us}
\item
  \href{https://www.nytco.com/careers/}{Work with us}
\item
  \href{https://nytmediakit.com/}{Advertise}
\item
  \href{http://www.tbrandstudio.com/}{T Brand Studio}
\item
  \href{https://www.nytimes.com/privacy/cookie-policy\#how-do-i-manage-trackers}{Your
  Ad Choices}
\item
  \href{https://www.nytimes.com/privacy}{Privacy}
\item
  \href{https://help.nytimes.com/hc/en-us/articles/115014893428-Terms-of-service}{Terms
  of Service}
\item
  \href{https://help.nytimes.com/hc/en-us/articles/115014893968-Terms-of-sale}{Terms
  of Sale}
\item
  \href{https://spiderbites.nytimes.com}{Site Map}
\item
  \href{https://help.nytimes.com/hc/en-us}{Help}
\item
  \href{https://www.nytimes.com/subscription?campaignId=37WXW}{Subscriptions}
\end{itemize}
