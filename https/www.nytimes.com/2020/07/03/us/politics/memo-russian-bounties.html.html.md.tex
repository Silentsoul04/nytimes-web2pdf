Sections

SEARCH

\protect\hyperlink{site-content}{Skip to
content}\protect\hyperlink{site-index}{Skip to site index}

\href{https://www.nytimes.com/section/politics}{Politics}

\href{https://myaccount.nytimes.com/auth/login?response_type=cookie\&client_id=vi}{}

\href{https://www.nytimes.com/section/todayspaper}{Today's Paper}

\href{/section/politics}{Politics}\textbar{}New Administration Memo
Seeks to Foster Doubts About Suspected Russian Bounties

\url{https://nyti.ms/3gv97EJ}

\begin{itemize}
\item
\item
\item
\item
\item
\end{itemize}

Advertisement

\protect\hyperlink{after-top}{Continue reading the main story}

Supported by

\protect\hyperlink{after-sponsor}{Continue reading the main story}

\hypertarget{new-administration-memo-seeks-to-foster-doubts-about-suspected-russian-bounties}{%
\section{New Administration Memo Seeks to Foster Doubts About Suspected
Russian
Bounties}\label{new-administration-memo-seeks-to-foster-doubts-about-suspected-russian-bounties}}

Criticized for its inaction, the Trump administration commissioned a new
look at a months-old intelligence assessment. It emphasizes gaps.

\includegraphics{https://static01.nyt.com/images/2020/07/03/us/politics/03dc-intel-01/merlin_172428954_12752367-fe61-4b5b-95f3-ae1edc8eadd2-articleLarge.jpg?quality=75\&auto=webp\&disable=upscale}

\href{https://www.nytimes.com/by/charlie-savage}{\includegraphics{https://static01.nyt.com/images/2018/06/12/multimedia/author-charlie-savage/author-charlie-savage-thumbLarge-v2.png}}\href{https://www.nytimes.com/by/eric-schmitt}{\includegraphics{https://static01.nyt.com/images/2018/06/12/multimedia/author-eric-schmitt/author-eric-schmitt-thumbLarge-v2.png}}\href{https://www.nytimes.com/by/rukmini-callimachi}{\includegraphics{https://static01.nyt.com/images/2018/10/08/multimedia/author-rukmini-callimachi/author-rukmini-callimachi-thumbLarge-v2.png}}\href{https://www.nytimes.com/by/adam-goldman}{\includegraphics{https://static01.nyt.com/images/2018/07/12/multimedia/author-adam-goldman/author-adam-goldman-thumbLarge.png}}

By \href{https://www.nytimes.com/by/charlie-savage}{Charlie Savage},
\href{https://www.nytimes.com/by/eric-schmitt}{Eric Schmitt},
\href{https://www.nytimes.com/by/rukmini-callimachi}{Rukmini Callimachi}
and \href{https://www.nytimes.com/by/adam-goldman}{Adam Goldman}

\begin{itemize}
\item
  July 3, 2020
\item
  \begin{itemize}
  \item
  \item
  \item
  \item
  \item
  \end{itemize}
\end{itemize}

A memo produced in recent days by the office of the nation's top
intelligence official acknowledged that the C.I.A. and top
counterterrorism officials have assessed that Russia appears to have
offered bounties to kill American and coalition troops in Afghanistan,
but emphasized uncertainties and gaps in evidence, according to three
officials.

The memo is said to contain no new information, and both its timing and
its stressing of doubts suggested that it was intended to bolster the
Trump administration's attempts to justify its inaction on the
months-old assessment, the officials said. Some former national security
officials said the account of the memo indicated that politics may have
influenced its production.

The National Intelligence Council, which reports to the director of
national intelligence, John Ratcliffe, produced the two-and-a-half page
document, a so-called sense of the community memorandum. Dated July 1,
it appears to have been commissioned after
\href{https://www.nytimes.com/2020/06/26/us/politics/russia-afghanistan-bounties.html}{The
New York Times reported on June 26} that intelligence officials had
assessed months ago that Russia had offered bounties, but the White
House had yet to authorize a response.

The memo said that the C.I.A. and the National Counterterrorism Center
had assessed with medium confidence ---
\href{https://www.dni.gov/files/documents/Newsroom/Reports\%20and\%20Pubs/20071203_release.pdf}{meaning
credibly sourced and plausible}, but falling short of near certainty ---
that a unit of the Russian military intelligence service, known as the
G.R.U., offered the bounties, according to two of the officials briefed
on its contents.

But other parts of the intelligence community --- including the National
Security Agency, which favors electronic surveillance intelligence ---
said they did not have information to support that conclusion at the
same level, therefore expressing lower confidence in the conclusion,
according to the two officials. A third official familiar with the memo
did not describe the precise confidence levels, but also said the
C.I.A.'s was higher than other agencies.

A spokeswoman for Mr. Ratcliffe's office declined to comment. The
officials familiar with the memo described it on the condition of
anonymity.

It is not uncommon for the intelligence council to produce short-notice,
all-source assessments on important topics, especially if agencies'
analyses differ, said Gregory F. Treverton,
\href{https://www.dni.gov/index.php/newsroom/press-releases/press-releases-2014/item/1116-new-national-intelligence-council-chairman-arrives}{the
chairman of the council} from 2014 to 2017. But he voiced concern that
the assessment of the suspected Russian bounty program could be
politicized to fit the White House's characterization of the
intelligence about it.

``I would hope the process still maintains its integrity, but I have
real concerns, given the pressures these analysts are under,'' Mr.
Treverton said in a telephone interview.

\href{https://www.nytimes.com/2014/07/10/world/middleeast/Matthew-G-Olsen-to-Leave-the-National-Counterterrorism-Center.html}{Matthew
G. Olsen}, a former director of the National Counterterrorism Center who
also held other national security posts during both the George W. Bush
and Obama administrations, also said the account of the memo's contents
raised the appearance of potential politicization.

``These products are never definitive, ever --- there's always caveats
and holes and judgments and qualifications,'' Mr. Olsen said. ``The
White House has portrayed it as not verified, but it's never verified,
so that struck me as misrepresentation. It would be very easy, if you
want to take a different spin, to draw those out and amplify the ways
it's inconclusive.''

Mr. Ratcliffe, formerly a Republican congressman known for his outspoken
support for Mr. Trump, was confirmed in late May.

\includegraphics{https://static01.nyt.com/images/2020/07/03/us/politics/03dc-intel-03/03dc-intel-03-articleLarge.jpg?quality=75\&auto=webp\&disable=upscale}

The memo is said to lay out the intelligence that informed the agencies'
conclusions. It declared that the intelligence community knows that
Russian military intelligence officers met with leaders of a
Taliban-linked criminal network and that money was transferred from a
G.R.U. account to the network. After lower-level members of that network
were captured, they told interrogators that the Russians were paying
bounties to encourage the killings of coalition troops, including
Americans.

But, the two officials who discussed the memo in greater detail said, it
stressed that the government lacks direct evidence of what the criminal
network leaders and G.R.U. officials said at face-to-face meetings so it
cannot say with any greater certainty that Russia specifically offered
bounties in return for killings of Western soldiers.

Two suspected leaders of the criminal ring who were believed to have met
with the G.R.U. --- Rahmatullah Azizi, a onetime drug smuggler who grew
wealthy
\href{https://www.nytimes.com/2020/07/01/world/asia/afghan-russia-bounty-middleman.html}{as
a middleman for the Russian spies}, and a second man named Habib Muradi,
according to three officials --- fled to Russia after raids this year
where several of their underlings were captured.

The memo also emphasized that the National Security Agency did not have
surveillance that confirmed what the captured detainees told
interrogators about bounties, according to the officials. The agency did
intercept data of financial transfers that provide circumstantial
support for the detainees' account, but the agency does not have
explicit evidence that the money was bounty payments.

The memo also said that the Defense Intelligence Agency did not have
information directly connecting the suspected operation to the Kremlin,
officials said. But earlier assessments had also said that it was not
clear how far up in the Russian government the bounties were approved.
Intelligence officials suspect that a G.R.U. section known as
\href{https://www.nytimes.com/2019/10/08/world/europe/unit-29155-russia-gru.html}{Unit
29155}, which has been linked to assassination attempts and other covert
operations in Europe intended to destabilize the West or exact revenge
on turncoats, is behind the suspected plot.

The memo was produced as the administration, in response to bipartisan
congressional demands, delivered briefings to lawmakers this week.
Another person familiar with one of the briefings said that lawmakers
were told that the intelligence community had high confidence that
Russia was encouraging Taliban attacks on American and coalition forces
and that the G.R.U. had officers in Afghanistan with links to the
Taliban.

But, the person said, while there was chatter among Afghans about
possible bounties for attacks, American officials were less sure when it
came to trying to link Russians to the acts of specific Taliban
militants or associated criminal units, or showing that the Russians had
actually paid for specific attacks. At one point, about half a million
dollars in cash was seized in a raid on a compound, raising suspicions,
but investigators could not say for sure that it was bounty money.

Image

Intelligence briefers told Congress that it is not clear whether Russian
was behind or paid for one attack that investigators are said to be
focused on: the killing of three Marines last year. American forces shot
at a civilian vehicle in ensuing days.Credit...Rahmat Gul/Associated
Press

The briefers told Congress that it was not clear whether the Russians
were behind or paid for one episode that investigators are said to be
focused on: the killing of three Marines in an April 2019 bombing
outside Bagram Air Base. One official said the new memo said that it
cannot be established with certainty that Russian actions led to that
attack.

The United States has accused Russia of providing support like small
arms to the Taliban for years. After interagency vetting, the
intelligence assessment that Russia's support had escalated into
directly encouraging more attacks on Americans and other coalition
troops was included in Mr. Trump's written daily brief in late February,
officials have said.

Mr. Trump is known to only rarely read his daily briefing, however.
Administration officials have said publicly that he was not ``briefed''
but remained coy about whether the assessment was in his written brief.
In congressional briefings, according to participants, administration
officials have stressed that Mr. Trump was not ``orally'' briefed.

The assessment of the problem also served as the basis of an interagency
meeting in late March convened by the National Security Council, at the
end of which officials were assigned to come up with a menu of potential
responses. The ensuing list started with making a diplomatic complaint
to Russia and escalated into sanctions and other punishments, officials
have said.

But despite receiving that list months ago, the Trump White House has
not authorized action. The administration appeared to have indefinitely
sidelined the issue, the officials said, until The Times article last
week caused an uproar in Congress, prompting a fresh look at it.

Advertisement

\protect\hyperlink{after-bottom}{Continue reading the main story}

\hypertarget{site-index}{%
\subsection{Site Index}\label{site-index}}

\hypertarget{site-information-navigation}{%
\subsection{Site Information
Navigation}\label{site-information-navigation}}

\begin{itemize}
\tightlist
\item
  \href{https://help.nytimes.com/hc/en-us/articles/115014792127-Copyright-notice}{©~2020~The
  New York Times Company}
\end{itemize}

\begin{itemize}
\tightlist
\item
  \href{https://www.nytco.com/}{NYTCo}
\item
  \href{https://help.nytimes.com/hc/en-us/articles/115015385887-Contact-Us}{Contact
  Us}
\item
  \href{https://www.nytco.com/careers/}{Work with us}
\item
  \href{https://nytmediakit.com/}{Advertise}
\item
  \href{http://www.tbrandstudio.com/}{T Brand Studio}
\item
  \href{https://www.nytimes.com/privacy/cookie-policy\#how-do-i-manage-trackers}{Your
  Ad Choices}
\item
  \href{https://www.nytimes.com/privacy}{Privacy}
\item
  \href{https://help.nytimes.com/hc/en-us/articles/115014893428-Terms-of-service}{Terms
  of Service}
\item
  \href{https://help.nytimes.com/hc/en-us/articles/115014893968-Terms-of-sale}{Terms
  of Sale}
\item
  \href{https://spiderbites.nytimes.com}{Site Map}
\item
  \href{https://help.nytimes.com/hc/en-us}{Help}
\item
  \href{https://www.nytimes.com/subscription?campaignId=37WXW}{Subscriptions}
\end{itemize}
