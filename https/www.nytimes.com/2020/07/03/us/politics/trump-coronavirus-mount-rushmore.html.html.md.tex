Sections

SEARCH

\protect\hyperlink{site-content}{Skip to
content}\protect\hyperlink{site-index}{Skip to site index}

\href{https://www.nytimes.com/section/politics}{Politics}

\href{https://myaccount.nytimes.com/auth/login?response_type=cookie\&client_id=vi}{}

\href{https://www.nytimes.com/section/todayspaper}{Today's Paper}

\href{/section/politics}{Politics}\textbar{}Trump Uses Mount Rushmore
Speech to Deliver Divisive Culture War Message

\href{https://nyti.ms/2ZxySNL}{https://nyti.ms/2ZxySNL}

\begin{itemize}
\item
\item
\item
\item
\item
\end{itemize}

\begin{itemize}
\item
  \href{https://www.nytimes.com/2020/08/07/us/elections/biden-vs-trump.html?action=click\&pgtype=Article\&state=default\&region=TOP_BANNER\&context=storylines_menu}{Election
  Updates}
\item
  \href{https://www.nytimes.com/interactive/2020/08/08/us/elections/results-hawaii-primary-elections.html?action=click\&pgtype=Article\&state=default\&region=TOP_BANNER\&context=storylines_menu}{Hawaii
  Results}
\item
  \href{https://www.nytimes.com/article/biden-vice-president-2020.html?action=click\&pgtype=Article\&state=default\&region=TOP_BANNER\&context=storylines_menu}{Biden's
  V.P. Search}
\item
  \href{https://www.nytimes.com/interactive/2019/us/politics/2020-presidential-candidates.html?action=click\&pgtype=Article\&state=default\&region=TOP_BANNER\&context=storylines_menu}{The
  Candidates}
\item
  \href{https://www.nytimes.com/newsletters/politics?action=click\&pgtype=Article\&state=default\&region=TOP_BANNER\&context=storylines_menu}{Politics
  Newsletter}
\end{itemize}

Advertisement

\protect\hyperlink{after-top}{Continue reading the main story}

Supported by

\protect\hyperlink{after-sponsor}{Continue reading the main story}

\hypertarget{trump-uses-mount-rushmore-speech-to-deliver-divisive-culture-war-message}{%
\section{Trump Uses Mount Rushmore Speech to Deliver Divisive Culture
War
Message}\label{trump-uses-mount-rushmore-speech-to-deliver-divisive-culture-war-message}}

Down in the polls and failing to control a raging pandemic, the
president cast himself as waging battle against a ``new far-left
fascism'' that imperils American values and seeks to erase history.

\includegraphics{https://static01.nyt.com/images/2020/07/05/us/politics/06dc-virus-trump-sub/03dc-virus-trump-videoSixteenByNine3000-v5.jpg}

\href{https://www.nytimes.com/by/annie-karni}{\includegraphics{https://static01.nyt.com/images/2019/02/05/multimedia/author-annie-karni/author-annie-karni-thumbLarge.png}}

By \href{https://www.nytimes.com/by/annie-karni}{Annie Karni}

\begin{itemize}
\item
  Published July 3, 2020Updated July 7, 2020
\item
  \begin{itemize}
  \item
  \item
  \item
  \item
  \item
  \end{itemize}
\end{itemize}

WASHINGTON --- Standing in a packed amphitheater in front of Mount
Rushmore for an Independence Day celebration, President Trump delivered
a dark and divisive speech on Friday that cast his struggling effort to
win a second term as a battle against a ``new far-left fascism'' seeking
to wipe out the nation's values and history.

With the coronavirus pandemic raging and his campaign faltering in the
polls, his appearance amounted to a fiery reboot of his re-election
effort, using the holiday and an official presidential address to mount
a full-on culture war against a straw-man version of the left that he
portrayed as inciting mayhem and moving the country toward
totalitarianism.

``Our nation is witnessing a merciless campaign to wipe out our history,
defame our heroes, erase our values and indoctrinate our children,'' Mr.
Trump said, addressing a packed crowd of sign-waving supporters, few of
whom wore masks. ``Angry mobs are trying to tear down statues of our
founders, deface our most sacred memorials and unleash a wave of violent
crime in our cities.''

Mr. Trump barely mentioned the frightening resurgence of the pandemic,
even as the country surpassed 53,000 new cases Friday and health
officials across the nation
\href{https://www.nytimes.com/2020/07/02/us/coronavirus-fourth-of-july.html}{urged
Americans to scale back} their Fourth of July plans.

Instead, appealing unabashedly to his base with ominous language and
imagery, he railed against what he described as a dangerous ``cancel
culture'' intent on toppling monuments and framed himself as a strong
leader who would protect the Second Amendment, law enforcement and the
country's heritage.

The scene at Mount Rushmore was the latest sign of how Mr. Trump
appears, by design or default, increasingly disconnected from the
intense concern among Americans about the health crisis gripping the
country. More than just a partisan rally, it underscored the extent to
which Mr. Trump is appealing to a subset of Americans to
\href{https://www.nytimes.com/2020/07/07/podcasts/the-daily/trump-mount-rushmore-speech.html}{carry
him to a second term} by changing the subject and appealing to fear and
division.

\includegraphics{https://static01.nyt.com/images/2017/01/29/podcasts/the-daily-album-art/the-daily-album-art-articleInline-v2.jpg?quality=75\&auto=webp\&disable=upscale}

\hypertarget{listen-to-the-daily-their-goal-is-the-end-of-america}{%
\subsubsection{Listen to `The Daily': `Their Goal Is the End of
America'}\label{listen-to-the-daily-their-goal-is-the-end-of-america}}

What President Trump's divisive speech at Mount Rushmore reveals about
his re-election campaign.

transcript

Back to The Daily

bars

0:00/22:50

-22:50

transcript

\hypertarget{listen-to-the-daily-their-goal-is-the-end-of-america-1}{%
\subsection{Listen to `The Daily': `Their Goal Is the End of
America'}\label{listen-to-the-daily-their-goal-is-the-end-of-america-1}}

\hypertarget{hosted-by-michael-barbaro-produced-by-rachel-quester-and-luke-vander-ploeg-with-help-from-asthaa-chaturvedi-and-edited-by-mj-davis-lin-and-lisa-chow}{%
\subsubsection{Hosted by Michael Barbaro; produced by Rachel Quester and
Luke Vander Ploeg, with help from Asthaa Chaturvedi; and edited by M.J.
Davis Lin and Lisa
Chow}\label{hosted-by-michael-barbaro-produced-by-rachel-quester-and-luke-vander-ploeg-with-help-from-asthaa-chaturvedi-and-edited-by-mj-davis-lin-and-lisa-chow}}

\hypertarget{what-president-trumps-divisive-speech-at-mount-rushmore-reveals-about-his-re-election-campaign}{%
\paragraph{What President Trump's divisive speech at Mount Rushmore
reveals about his re-election
campaign.}\label{what-president-trumps-divisive-speech-at-mount-rushmore-reveals-about-his-re-election-campaign}}

\begin{itemize}
\item
  michael barbaro\\
  From The New York Times, I'm Michael Barbaro. This is ``The Daily.''

  Today: In a major speech at Mount Rushmore, President Trump said that
  the goal of nationwide protests is not, quote, ``a better America.''
  Their goal, he said, is the end of America. Maggie Haberman on what
  that speech reveals about the president's re-election campaign.

  It's Tuesday July 7.

  Maggie, heading into this July 4 weekend, what was our understanding
  of what this Mount Rushmore speech from President Trump, what it was
  for, what it was intended to do?
\item
  maggie haberman\\
  So Michael, the month of June was pretty calamitous for President
  Trump politically and in terms of his legacy. It began with the
  federal government having protesters forcibly cleared using chemical
  irritants from Lafayette Park across from the White House, so that the
  president could then take a photo op. To mass protests across the
  country. To a huge spike in coronavirus cases in areas of the country
  where it really had not been that prevalent, and where the governors
  in those states were looking toward reopening.

  So the president tried for a reboot of his campaign with a rally in
  Tulsa on June 20. That rally was sparsely attended compared to what
  they had advertised as their likely attendance. And so Mount Rushmore
  and this event was supposed to be the reboot of the failed reboot.
  This was going to be an effort by the president to show he was in
  charge and trying to look toward the general election.
\item
  michael barbaro\\
  And from your reporting, what was this reboot of the reboot going to
  look like in a speech?
\item
  maggie haberman\\
  So the president needs an enemy to fight against. In 2018 during the
  midterms, you saw the president try to galvanize support against a
  looming threat, as he put it, of a caravan that was headed across the
  southern border with Mexico. And this was basically a threat of a
  foreign invasion. And he talked about this a lot and he tweeted about
  it a lot. And the main enemy that the U.S. is dealing with right now
  is the coronavirus, which is spreading rapidly. That's an issue on
  which his polling is pretty bad. And his advisers know it. And another
  force that the country is dealing with right now is police brutality.
  Neither of those are issues that Donald Trump is seen as particularly
  strong on, or areas where he has shown he wants to lead.

  So instead, looking for this enemy, aides described in his speech he
  was going to go after a left-wing culture coming after people who
  don't agree with it. Now the threat is other Americans. The threat is
  people who don't think like you.
\item
  archived recording (donald trump)\\
  Well, thank you very much. And Governor Noem, Secretary Bernhardt, we
  very much appreciate it. Members of Congress, distinguished guests,
  and a very special hello to South Dakota.
\end{itemize}

michael barbaro

OK, let's talk about how this speech actually unfolds. I watched it. I
know you did as well. So I want us to walk through it and pick out a few
key passages that illuminate what he's actually up to here, kind of a
close reading of this speech. So where do you think we should start?

maggie haberman

I would start just understanding what it looked like. He was standing at
this podium, surrounded by flags, in front of this historic monument.

\begin{itemize}
\tightlist
\item
  archived recording (donald trump)\\
  There could be no better place to celebrate America's independence
  than beneath this magnificent, incredible, majestic mountain monument
  to the greatest Americans who have ever lived.
\end{itemize}

maggie haberman

And that was supposed to underscore this current conversation about
monuments and statues around the country.

\begin{itemize}
\tightlist
\item
  archived recording (donald trump)\\
  I am here as your president to proclaim before the country and before
  the world, this monument will never be desecrated. These heroes will
  never be defaced. Their legacy will never, ever be destroyed. Their
  achievements will never be forgotten. And Mount Rushmore will stand
  forever as an eternal tribute to our forefathers and to our freedom.
\end{itemize}

maggie haberman

Much of the conversation has been around Confederate totems, Confederate
statues, the Confederate flag. The president has resisted those
conversations. But even members of his own party have said that it is
time to remove some of those monuments. Where he is drawing the line is
when the conversation moves to George Washington or Thomas Jefferson.
Those are two of the faces on Mount Rushmore. And that's part of why
he's choosing to have this conversation there.

michael barbaro

And what is he saying about that debate around statues to presidents
like that?

maggie haberman

What he is suggesting is that the political left is trying to rewrite
history ---

\begin{itemize}
\tightlist
\item
  archived recording (donald trump)\\
  1776 represented the culmination of thousands of years of Western
  civilization and the triumph of not only spirit, but of wisdom,
  philosophy and reason. And yet ---
\end{itemize}

maggie haberman

--- by calling into question those men, by suggesting that their
legacies need to be thought about again.

\begin{itemize}
\tightlist
\item
  archived recording (donald trump)\\
  As we meet here tonight, there is a growing danger that threatens
  every blessing our ancestors fought so hard for, struggled. They bled
  to secure ---
\end{itemize}

maggie haberman

And the reason that people are saying that their legacies need to be
reconsidered is because they were slaveholders, and that you can't have
an honest conversation about race if you do not acknowledge that.

\begin{itemize}
\tightlist
\item
  archived recording (donald trump)\\
  Our nation is witnessing a merciless campaign to wipe out our history,
  defame our heroes, erase our values and indoctrinate our children.
\end{itemize}

maggie haberman

What he's really trying to do is convince Republicans who are feeling
shaky about him --- and he hopes some independent voters --- that the
protests around the country have gone too far. He is trying to get them
to see it the way he sees it, which is, this isn't just a movement about
the Confederacy. They're coming for our whole history --- ``our.'' They
are coming for the history of white America.

\begin{itemize}
\tightlist
\item
  archived recording (donald trump)\\
  Angry mobs are trying to tear down statues of our founders, deface our
  most sacred memorials and unleash a wave of violent crime in our
  cities.
\end{itemize}

maggie haberman

It is in keeping with what President Trump has done for many, many years
now, which is an `us versus them' approach to his base of older, white
voters.

\begin{itemize}
\tightlist
\item
  archived recording (donald trump)\\
  But no, the American people are strong and proud. And they will not
  allow our country and all of its values, history and culture to be
  taken from them.
\end{itemize}

michael barbaro

OK, what stands out next to you in this speech?

maggie haberman

So the president very quickly went on to talk about how a, quote
unquote, political weapon of the Americans he is talking about in this
speech is ---

\begin{itemize}
\tightlist
\item
  archived recording (donald trump)\\
  Cancel culture.
\end{itemize}

maggie haberman

--- so-called ``cancel culture.''

\begin{itemize}
\tightlist
\item
  archived recording (donald trump)\\
  Driving people from their jobs, shaming dissenters and demanding total
  submission from anyone who disagrees.
\end{itemize}

maggie haberman

He is describing it as anyone who disagrees with certain folks are going
to get chased out of polite society. And that's not really what this.

\begin{itemize}
\tightlist
\item
  archived recording (donald trump)\\
  This is the very definition of totalitarianism. And it is completely
  alien to our culture and to our values. And it has absolutely no place
  in the United States of America.
\end{itemize}

maggie haberman

So in part, this is appealing to a longstanding sense among
conservatives that they are being attacked by the left for their
beliefs. Also notice his emphasis on our values and our culture. He has
used the words culture and values repeatedly to appeal to his base since
2017. This is the thing that he shares with his voters. It certainly is
not geography --- in many cases they're in the Deep South. And he is a
man from Queens. But this sense of our way of life is being taken over
is what he has used time and again to appeal to people.

\begin{itemize}
\item
  archived recording (donald trump)\\
  This attack on our liberty, our magnificent liberty, must be stopped.
  And it will be stopped very quickly.

  We will expose this dangerous movement, protect our nation's children,
  end this radical assault and preserve our beloved American way of
  life. In our schools, our newsrooms, even our corporate boardrooms,
  there is a new far-left fascism that demands absolute allegiance. If
  you do not speak its language, perform its rituals, recite its
  mantras, and follow its commandments, then you will be censored,
  banished, blacklisted, persecuted and punished. It's not going to
  happen to us.
\end{itemize}

michael barbaro

So Maggie, what is the next passage in this speech that strikes you?

maggie haberman

Sure. So keeping up with these themes, the president went on and said
---

\begin{itemize}
\tightlist
\item
  archived recording (donald trump)\\
  This left-wing cultural revolution is designed to overthrow the
  American Revolution.
\end{itemize}

maggie haberman

--- ``This left-wing cultural revolution is designed to overthrow the
American Revolution.'' And then he went on a little bit later to say
their goal is not a better America. Their goal is the end of America.

\begin{itemize}
\tightlist
\item
  archived recording (donald trump)\\
  In so doing, they would destroy the very civilization that rescued
  billions from poverty, disease, violence and hunger, and that lifted
  humanity to new heights of achievement, discovery and progress.
\end{itemize}

maggie haberman

You would think that he was talking about the British the way that he's
describing this ---

\begin{itemize}
\tightlist
\item
  archived recording (donald trump)\\
  In its place, they want power for themselves.
\end{itemize}

maggie haberman

--- as opposed to talking about primarily Black people in this country,
but not only, who have been trying to right historic wrongs. He is
making it sound, once again, as if something is being taken from him and
his supporters.

michael barbaro

In this case, American civilization?

maggie haberman

Yes. And he is trying to drive that home with everything he says.

michael barbaro

I mean, this feels like race baiting.

maggie haberman

I think it more than feels that way, Michael. I would argue it is race
baiting. Look, I don't think that Donald Trump is suddenly a different
person. I think this is who he has been for a very, very long time,
going back decades. But I do think he is getting explicit in what he is
saying, both as protests are growing in the country and as his own poll
numbers are sinking.

\begin{itemize}
\tightlist
\item
  archived recording (donald trump)\\
  But just as patriots did in centuries past, the American people will
  stand in their way. And we will win, and win quickly and with great
  dignity.
\end{itemize}

maggie haberman

He is not explicitly using the words black and white. But he is
explicitly describing one version of America versus another. And that, I
think, is different, along with the fact that we have really not seen a
president before use an Independence Day speech to be so divisive and to
pit Americans in two, the way he is here.

michael barbaro

So Maggie, how does this speech end?

\begin{itemize}
\tightlist
\item
  archived recording (donald trump)\\
  Americans must never lose sight of this miraculous story.
\end{itemize}

maggie haberman

So the president concludes this speech by saying he wants to build ---

\begin{itemize}
\tightlist
\item
  archived recording (donald trump)\\
  I am signing an executive order to establish the National Garden of
  American Heroes.
\end{itemize}

maggie haberman

--- a garden of statues.

\begin{itemize}
\tightlist
\item
  archived recording (donald trump)\\
  A vast, outdoor park that will feature the statues of the greatest
  Americans to ever live.
\end{itemize}

maggie haberman

And in this garden, he wants to put a variety of American figures ---
presidents, artists, sports figures.

\begin{itemize}
\tightlist
\item
  archived recording (donald trump)\\
  From this night ---
\end{itemize}

maggie haberman

And with that ---

\begin{itemize}
\tightlist
\item
  archived recording (donald trump)\\
  --- and from this magnificent place ---
\end{itemize}

maggie haberman

--- the president applauded for himself and for the crowd.

\begin{itemize}
\tightlist
\item
  archived recording (donald trump)\\
  --- God bless your families. God bless our great military. And God
  bless America. Thank you very much.
\end{itemize}

maggie haberman

And there was a fireworks display over Mount Rushmore.

{[}music{]}

michael barbaro

We'll be right back.

Maggie, what most surprises me about this speech, and the fact that it
is supposed to be a reset of a presidential campaign, is that the
message seems to fly in the face of polling that shows Americans don't
agree with this version of how to deal with race.

And I want to read a question that The Times asked voters in six swing
states about these protests. And here was the question: Would you rather
have a candidate who says that we need to be tough on protests that go
too far, or whether they would rather have a candidate who says we need
to focus on the cause of those protests, even when they go too far?

And voters told The Times in those swing states by a 40 percent margin
that they would rather have a candidate who focuses on a cause of the
protests, even when they go too far. So doesn't that suggest that the
president, this speech, this message is profoundly out of sync with the
electorate?

maggie haberman

Look, Michael, you read the polls. I read the polls. They all make clear
that the president is wildly out of step with where the majority of
voters are right now, where conservative voters are, where independent
voters are, where a broad spectrum of voters are. This is a president
who likes to do things his own way. He has ideas that he wants to put
out there, regardless of how much it upsets his advisers, regardless of
how scared senators are about losing their seats because his rhetoric is
making things very hard for them. But he is not where the majority of
Americans are in those polls.

michael barbaro

So Maggie, what is the thinking here? If the president's re-election
campaign has seen those polls that you and I have all seen, do they see
something that we're not seeing? Do they have a theory that extends
beyond these poll numbers?

maggie haberman

Many of the people in the president's campaign believe the direction
that the polls are taking, even if they argue with some of the margins.
Some of the people around the president share with him a belief or
theory, or whatever you want to call it, that people are not being
honest with the pollsters when they talk about how much support they
have for these protests, and that the numbers will come around in
President Trump's favor when we get to the fall.

michael barbaro

And help me understand that. When they say that they don't think the
polls reflect the real support for this movement, what do they mean?

maggie haberman

They think that people are inclined to lie to pollsters on matters of
race. Now there have been campaigns where that has happened. The margins
that we're talking about are so large that it would be really hard to
fathom that. But that is the bet that some of his advisers are making.
Now are they making that on science? Not necessarily. Are they making
that on political research? Only on the margins. For the most part, this
is wishcasting that the president is not doing himself the damage he
seems to be doing.

michael barbaro

Maggie, campaigns tend to be defined by debates in their ranks about
what is the right approach to a moment. So I have to imagine that inside
the Trump campaign there is a debate about whether this is the right
approach to this moment. Is that your sense?

maggie haberman

No, Michael. I don't think there's much of a debate going on.

michael barbaro

Why not?

maggie haberman

Because there's the way the president wants to campaign. And they try to
shape it around that. This is what Donald J. Trump thinks his campaign
message should be. Now, there are areas where his advisers have gotten
him to stick to that script that was written out and say things that
they consider to be less potentially divisive. So for instance, he spoke
broadly about culture and history. But he did not explicitly give a
defense of Confederate statues, which really turns off suburban voters,
in particular suburban women. And his advisers were very pleased with
that, that he stuck to the script and didn't say Confederate.

But then on Monday morning, he tweets support of the Confederate flag
being aired at NASCAR events. So it undoes a lot of what had taken place
before.

michael barbaro

Maggie, when you talk to people in the Trump campaign and you present
them with what seems like a pretty significant dilemma here --- a
president with a message and a national mood that seems very out of sync
with it --- what do they say?

maggie haberman

There is no evidence that this message is going to help the president
win again. There is no evidence that this is a successful approach to
the voters he needs in order to win. But advisers are pretty candid that
he thinks this is how he won last time. And he is convinced he can do it
again.

michael barbaro

Right, so what you're saying is the president is assuming that the
country is more or less exactly where it was in 2016, and that this will
all work out the same way and yield the same result --- an electoral
college victory based on white voters supporting him?

maggie haberman

Correct. The president is of the opinion --- and again, this is not his
campaign. There are people in the campaign who understand this is not
the same electorate. But the president has convinced himself that
nothing has changed, that he can turn Joe Biden into Hillary Clinton.
And so far there is no reason to believe that either of those things is
true.

michael barbaro

So Maggie, at this point, is this the message that you expect the Trump
campaign to be using between now and November? A message of the left
being the enemy and white America needing to be afraid of this movement
seeking racial equality.

maggie haberman

The campaign itself, I think, would like to be delivering a less blunt
instrument version of what the president is saying. But because the
president is able to speak only the way he's comfortable, he will not
change. And so yes, I think this is what you will see for the next few
months.

michael barbaro

Thank you, Maggie.

maggie haberman

Thank you, Michael.

michael barbaro

We'll be right back.

Here's what else you need to know today. A Times analysis found that
Black and Latino Americans have been three times as likely to become
infected with the coronavirus as white Americans, and have been nearly
twice as likely to die from it. The analysis was based on 640,000
infections across nearly 1,000 counties, making it the most far-reaching
study yet of the pandemic's racial disparities. In explaining the higher
infection rate, experts said that Black and Latino people are more
likely to have frontline jobs, rely on public transportation or live in
multigenerational homes, raising the risk of exposure.

And the Supreme Court unanimously ruled that states can require members
of the Electoral College to vote for the presidential candidate that
they had promised to support. 32 states and the District of Columbia
have laws requiring electors to abide by the pledges that they take on
Election Day. But a lower court in Colorado had ruled that they may
disregard those pledges when they actually cast ballots a few weeks
later. The Supreme Court's decision curbs the independence of electors
and limits a potential source of uncertainty in the upcoming
presidential election.

That's it for ``The Daily.'' I'm Michael Barbaro. See you tomorrow.

``Most presidents in history have understood that when they appear at a
national monument, it's usually a moment to act as a unifying chief of
state, not a partisan divider,'' Michael Beschloss, the presidential
historian, said before the speech.

Mr. Trump planned to follow up his trip with a ``Salute to America''
celebration on Saturday on the South Lawn at the White House, marked by
a military flyover and the launch of 10,000 fireworks on the National
Mall.

Mayor Muriel E. Bowser of Washington has warned the gathering violates
federal health guidelines. The Trump administration, which controls the
federal property of the National Mall, pushed for the celebration,
ignoring a mayor whom officials
\href{https://www.nytimes.com/2020/06/05/us/politics/muriel-bowser-trump.html}{view
as a political rival}.

Most politicians, including former Vice President Joseph R. Biden Jr.,
the presumptive Democratic nominee, this year were forgoing any of the
traditional holiday parades and flag-waving appearances. The
\href{https://www.nytimes.com/2020/07/01/business/fourth-of-july-fireworks-displays.html}{vast
majority of fireworks displays} in big cities and small rural towns have
been canceled as new cases reported in the United States have increased
by 90 percent in the past two weeks.

Mr. Trump's itinerary Friday and Saturday, however, had a different
message: The sparkly, booming show must go on at all costs in the
service of the divisive message and powerful images he wants to promote.

``We will not be tyrannized, we will not be demeaned, and we will not be
intimidated by bad, evil people,'' Mr. Trump said, referring to his
political opponents and their supporters.

In response to Mr. Trump's event in South Dakota, Andrew Bates, a
spokesman for Mr. Biden's campaign, said in a statement, ``Our whole
country is suffering through the excruciating costs of having a
negligent, divisive president who doesn't give a damn about anything but
his own gain --- not the sick, not the jobless, not our constitution,
and not our troops in harm's way.''

Under
\href{https://www.nytimes.com/2020/07/01/us/mount-rushmore.html}{the
granite gaze} of Washington, Jefferson, Lincoln and Theodore Roosevelt,
Mr. Trump announced plans to establish what he described as a ``vast
outdoor park that will feature the statues of the greatest Americans to
ever live,'' an apparent repudiation of the growing pressure to remove
statues tied to slavery or colonialism.

As he arrived, Air Force One performed a flyover of Mount Rushmore. His
campaign promoted the stunt online, calling him
\href{https://twitter.com/TeamTrump/status/1279219853790978053}{``the
coolest president ever.''}

In the amphitheater below, few in the packed crowd practiced any social
distancing as people waved signs that referred to CNN as the ``Communist
News Network.'' As he observed a flyover by the Navy's Blue Angels, Mr.
Trump sat on a packed dais with the first lady, Melania Trump, the
national security adviser, Robert O'Brien, and Mark Meadows, the White
House chief of staff, none of whom wore masks.

As the president departed Washington for South Dakota on Friday, at
least five states --- Alabama, Alaska, Kansas, North Carolina and South
Carolina --- reported their highest single day of cases yet. Newly
reported cases of the virus were rising in all but a handful of states,
and many large cities, including Houston, Dallas, Jacksonville and Los
Angeles, were seeing alarming growth.

\includegraphics{https://static01.nyt.com/images/2020/07/04/us/politics/04dc-virus-trump-sub-lede/04dc-virus-trump-sub-lede-articleLarge-v3.jpg?quality=75\&auto=webp\&disable=upscale}

Throughout his presidency, Mr. Trump has tried to bend events to his
will, often using social media to drive home his alternate version of
reality and, thanks to the power of repetition and the loyal support of
his base, sometimes succeeding. But the president's attempt to drive
deeper into the culture wars around a national holiday, during an
intensifying health crisis that will not yield to his tactics, risked
coming across as out of sync with the concerned mood of the country at a
moment when
\href{https://www.nytimes.com/2020/07/02/us/politics/trump-2020-campaign-problems.html}{his
re-election campaign is struggling and unfocused}.

``I don't think it will work, because what he is trying to do is pretend
that the situation is better than it is,'' Mr. Beschloss said.

Mr. Beschloss compared Mr. Trump to Woodrow Wilson, who presided over
the influenza pandemic in 1918 by trying to pretend it was not
happening, and to Herbert Hoover, who in 1932 tried to project that the
Great Depression was not as bad as people were saying.

``People voted him out because they felt he did not understand the
suffering,'' Mr. Beschloss said, referring to Hoover.

Mr. Trump
\href{https://www.nytimes.com/2020/06/26/us/politics/trump-coronavirus.html}{has
consistently played down} the concerns over spikes in new cases, even as
many cities and states have had to slow or reverse their reopenings,
claiming that young people
\href{https://twitter.com/realDonaldTrump/status/1278897430378041344}{``get
better much easier and faster},'' that the death rate is declining and
that the virus will
\href{https://www.nbcnews.com/politics/white-house/trump-says-he-thinks-coronavirus-will-just-disappear-despite-rising-n1232709}{``just
disappear.''}

On Thursday, he lauded his administration's response, referred to the
surge in new cases as ``temporary hot spots'' and focused instead on
what he said was evidence of the economy bouncing back.

``A lot of people would have wilted,''
\href{https://www.whitehouse.gov/briefings-statements/remarks-president-trump-jobs-numbers-report-2/}{Mr.
Trump said at a news conference} where he praised the latest job
numbers. ``We didn't wilt. Our country didn't wilt.''

Despite his rosy outlook, the coronavirus on Friday for the first time
infiltrated Mr. Trump's family circle. The president's elder son, Donald
Trump Jr., and his girlfriend, Kimberly Guilfoyle, had traveled to South
Dakota separately with plans to meet up with the president. But they
left before Mr. Trump's arrival once
\href{https://www.nytimes.com/2020/07/03/us/politics/kimberly-guilfoyle-trump-campaign-coronavirus.html}{Ms.
Guilfoyle tested positive} for the virus and said they planned to cancel
all coming events.

Mr. Trump's show, however, went on without missing a beat. In South
Dakota, Mr. Trump enjoys the backing of Gov. Kristi Noem, a Republican,
who had invited him to make the trip, which amounted to a second attempt
to get his campaign back on track after the
\href{https://www.nytimes.com/2020/06/20/us/politics/tulsa-trump-rally.html}{disappointing
turnout at a rally last month in Tulsa, Okla.}

Image

Demonstrators, some from different tribes in the region, marched through
Keystone on Friday to protest Mr. Trump's visit to Mount
Rushmore.Credit...Andrew Caballero-Reynolds/Agence France-Presse ---
Getty Images

In recent weeks, South Dakota has had one of the country's
\href{https://www.nytimes.com/interactive/2020/us/south-dakota-coronavirus-cases.html}{most
encouraging trend lines}. The state has averaged a few dozen new cases
each day, including 85 announced Friday. There has not been a day with
more than 100 new cases in South Dakota since late May. Ms. Noem said
Friday night that many attendees at Mr. Trump's Fourth of July spectacle
had traveled from out of state to attend.

In Washington, however, officials remain adamantly opposed to the
celebration planned for Saturday, which White House officials defended
as a gathering people could enjoy safely. Administration officials noted
that the celebration in Washington was scaled back from last year's
event,
\href{https://www.nytimes.com/2019/07/02/us/politics/trump-tanks.html}{when
Mr. Trump turned the holiday into a salute to the military}, with tanks
on the streets of the capital and flyovers from Air Force One as well as
aircraft from each branch of the armed forces, as he delivered remarks
from the Lincoln Memorial.

Kayleigh McEnany, the White House press secretary, said this week that
Mr. Trump had recommended following guidelines set by local authorities
only on wearing masks --- not on social distancing overall. ``The C.D.C.
guidelines, I'd also note, say `recommended,' but not required,'' she
said. ``We are very much looking forward to the Fourth of July
celebration.''

This year, the National Park Service said it was taking extra safety
precautions on the National Mall, installing more than 100 hand-washing
stations throughout the area, up from 15 last year. Officials also said
they had 300,000 cloth facial coverings on hand to distribute.

``We are committed to providing the American people with a safe and
spectacular celebration of our nation's birthday in Washington D.C.,
which will honor our military with music, flyovers and fireworks,'' a
spokesman for the park service said. ``We are doing so consistent with
our mission and historical practices, and we hope everyone enjoys the
day's festivities.''

On Friday, Mr. Trump spent the day at his golf course in Sterling, Va.,
before he departed for South Dakota, and White House officials said they
had no safety concerns about the trip.

But the virus had already shown it can infiltrate the administration,
and the White House has experienced the dangers of staging large
gatherings as the pandemic rages. Vice President Mike Pence
\href{https://www.nytimes.com/2020/07/02/us/politics/pence-covid.html}{postponed
a planned trip} this week to Arizona after Secret Service agents set to
accompany him tested positive for the coronavirus or showed symptoms.
And at least eight campaign staff members who helped plan Mr. Trump's
indoor rally last month in Tulsa,
\href{https://www.nytimes.com/2020/06/22/us/politics/trump-campaign-coronavirus-tulsa.html}{have
tested positive}, either before the rally or after attending.

Before the president left for South Dakota on Friday, Trump campaign
aides were circulating on social media
\href{https://twitter.com/abigailmarone/status/1279044148184694787}{a
doctored image} of Mount Rushmore, featuring Mr. Trump's face carved
into the stone next to some of the nation's most revered presidents.

``Mount Rushmore, improved,'' one aide wrote.

Mitch Smith contributed reporting from Chicago.

\hypertarget{our-2020-election-guide}{%
\section{Our 2020 Election Guide}\label{our-2020-election-guide}}

Updated Aug. 8, 2020

\begin{itemize}
\item
  \begin{center}\rule{0.5\linewidth}{\linethickness}\end{center}

  \hypertarget{the-latest}{%
  \subsection{The Latest}\label{the-latest}}

  \begin{itemize}
  \tightlist
  \item
    With 160 lawsuits filed over voting rules and President Trump's
    baseless claims of fraud, Election Day in America
    \href{https://www.nytimes.com/2020/08/08/us/politics/voting-nov-3-election.html?action=click\&pgtype=Article\&state=default\&region=BELOW_MAIN_CONTENT\&context=storylines_guide}{could
    become Election Month}.
  \end{itemize}
\item
  \begin{center}\rule{0.5\linewidth}{\linethickness}\end{center}

  \hypertarget{bidens-vp-search}{%
  \subsection{Biden's V.P. Search}\label{bidens-vp-search}}

  \begin{itemize}
  \tightlist
  \item
    \href{https://www.nytimes.com/article/biden-vice-president-2020.html?action=click\&pgtype=Article\&state=default\&region=BELOW_MAIN_CONTENT\&context=storylines_guide}{Here
    are 13 women} who have been under consideration to be Joe Biden's
    running mate, and why each might be chosen --- and might not be.
  \end{itemize}
\item
  \begin{center}\rule{0.5\linewidth}{\linethickness}\end{center}

  \hypertarget{keep-up-with-our-coverage}{%
  \subsection{Keep Up With Our
  Coverage}\label{keep-up-with-our-coverage}}

  \begin{itemize}
  \tightlist
  \item
    Get an
    \href{https://www.nytimes.com/newsletters/politics?action=click\&pgtype=Article\&state=default\&region=BELOW_MAIN_CONTENT\&context=storylines_guide}{email}
    recapping the day's news
  \end{itemize}

  \begin{itemize}
  \tightlist
  \item
    Download our mobile app on
    \href{https://apps.apple.com/us/app/nytimes/id284862083?ls=1\&mat_click_id=5c79ae7455014fd1bd66b5610c05b8f2-20191112-16948\&referrer=mat_click_id\%3D5c79ae7455014fd1bd66b5610c05b8f2-20191112-16948\%26link_click_id\%3D722930677036718082}{iOS}
    and
    \href{http://a.localytics.com/android?id=com.nytimes.android\&referrer=utm_source\%3Dother_nyt_mobile_web\%26utm_medium\%3DWeb\%2520page\%26utm_term\%3DGeneral\%2520Mobile\%2520Page\%26utm_campaign\%3DNYT\%2520Mobile\%2520General\%2520Page}{Android}
    and turn on Breaking News and Politics alerts
  \end{itemize}
\end{itemize}

Advertisement

\protect\hyperlink{after-bottom}{Continue reading the main story}

\hypertarget{site-index}{%
\subsection{Site Index}\label{site-index}}

\hypertarget{site-information-navigation}{%
\subsection{Site Information
Navigation}\label{site-information-navigation}}

\begin{itemize}
\tightlist
\item
  \href{https://help.nytimes.com/hc/en-us/articles/115014792127-Copyright-notice}{©~2020~The
  New York Times Company}
\end{itemize}

\begin{itemize}
\tightlist
\item
  \href{https://www.nytco.com/}{NYTCo}
\item
  \href{https://help.nytimes.com/hc/en-us/articles/115015385887-Contact-Us}{Contact
  Us}
\item
  \href{https://www.nytco.com/careers/}{Work with us}
\item
  \href{https://nytmediakit.com/}{Advertise}
\item
  \href{http://www.tbrandstudio.com/}{T Brand Studio}
\item
  \href{https://www.nytimes.com/privacy/cookie-policy\#how-do-i-manage-trackers}{Your
  Ad Choices}
\item
  \href{https://www.nytimes.com/privacy}{Privacy}
\item
  \href{https://help.nytimes.com/hc/en-us/articles/115014893428-Terms-of-service}{Terms
  of Service}
\item
  \href{https://help.nytimes.com/hc/en-us/articles/115014893968-Terms-of-sale}{Terms
  of Sale}
\item
  \href{https://spiderbites.nytimes.com}{Site Map}
\item
  \href{https://help.nytimes.com/hc/en-us}{Help}
\item
  \href{https://www.nytimes.com/subscription?campaignId=37WXW}{Subscriptions}
\end{itemize}
