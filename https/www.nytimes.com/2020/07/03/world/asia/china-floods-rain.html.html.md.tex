Sections

SEARCH

\protect\hyperlink{site-content}{Skip to
content}\protect\hyperlink{site-index}{Skip to site index}

\href{https://www.nytimes.com/section/world/asia}{Asia Pacific}

\href{https://myaccount.nytimes.com/auth/login?response_type=cookie\&client_id=vi}{}

\href{https://www.nytimes.com/section/todayspaper}{Today's Paper}

\href{/section/world/asia}{Asia Pacific}\textbar{}Severe Floods in China
Leave Over 106 Dead or Missing

\url{https://nyti.ms/2ZvG6lb}

\begin{itemize}
\item
\item
\item
\item
\item
\end{itemize}

Advertisement

\protect\hyperlink{after-top}{Continue reading the main story}

Supported by

\protect\hyperlink{after-sponsor}{Continue reading the main story}

\hypertarget{severe-floods-in-china-leave-over-106-dead-or-missing}{%
\section{Severe Floods in China Leave Over 106 Dead or
Missing}\label{severe-floods-in-china-leave-over-106-dead-or-missing}}

Unusually intense rainfall has swept away buildings and ruined homes in
southern China, affecting about 15 million residents. More downpours are
forecast for Saturday.

\includegraphics{https://static01.nyt.com/images/2020/07/03/world/03china-floods-1/merlin_174173109_4fc83fd9-eddc-41e6-9d8a-461206c0edcd-articleLarge.jpg?quality=75\&auto=webp\&disable=upscale}

\href{https://www.nytimes.com/by/raymond-zhong}{\includegraphics{https://static01.nyt.com/images/2018/10/15/multimedia/author-raymond-zhong/author-raymond-zhong-thumbLarge.png}}

By \href{https://www.nytimes.com/by/raymond-zhong}{Raymond Zhong}

\begin{itemize}
\item
  July 3, 2020
\item
  \begin{itemize}
  \item
  \item
  \item
  \item
  \item
  \end{itemize}
\end{itemize}

In the
\href{https://mp.weixin.qq.com/s?__biz=MjM5ODg0NzYwMQ==\&mid=2650458989\&idx=1\&sn=cb10b9cee1cfeeee29c3bcbce59af0d7\&chksm=beca1e7389bd976536caa0a37a8c6a3dc3f4163d5075370743b94f45e2b28f711de9b9226f00\&scene=4\#wechat_redirect}{inland
Chinese city of Yichang}, the murky water ran waist-high, stranding
people in their cars and turning streets into canals. Near the
metropolis of Chongqing,
\href{https://mp.weixin.qq.com/s/dYW1K1N6nqmvzLmCAeQiLQ}{angry torrents
of water} swept away country roads. The tourist town of Yangshuo
experienced a cloudburst that an official called
\href{https://www.infzm.com/contents/186093}{a once-in-two-centuries
event}.

Weeks of abnormally intense rains have wrought destruction across
southern China, leaving
\href{https://tv.cctv.com/2020/07/02/VIDEsn6CmjhENykQ0vaWO6vo200702.shtml}{at
least 106 people dead or missing} and affecting 15 million residents in
the worst flooding that parts of the region have seen in decades.

One of the hardest-hit provinces has been Hubei, whose capital, Wuhan,
also had the first emergence of the coronavirus last year. Late last
month,
\href{https://www.weibo.com/1205338040/J8zLu9vIZ?refer_flag=1001030103_\&type=comment}{rescuers
smashed car windows} to free passengers trapped by floodwater
\href{https://mp.weixin.qq.com/s?__biz=MjM5ODg0NzYwMQ==\&mid=2650458989\&idx=1\&sn=cb10b9cee1cfeeee29c3bcbce59af0d7\&chksm=beca1e7389bd976536caa0a37a8c6a3dc3f4163d5075370743b94f45e2b28f711de9b9226f00\&scene=4\#wechat_redirect}{in
Yichang}, a city in Hubei down the Yangtze River from the Three Gorges
Dam, one of the world's largest.

Hubei has had more coronavirus cases than any other part of China. And
people there said the last thing they needed was another devastating
jolt to their lives, their health and their livelihoods.

Beijing

Yellow R.

CHINA

HUBEI

Wuhan

Shanghai

Yichang

Chongqing

Yangtze R.

Yangshuo

TAIWAN

GUANGXI

Hong Kong

South

China Sea

VIETNAM

200 MILES

By The New York Times

``Another problem has arisen before the last one subsided,'' Deng Jin,
25, a recent college graduate from the city of Enshi, lamented recently
\href{https://m.weibo.cn/status/4520707848175659?}{on the social
platform Weibo}. ``Hubei in 2020 is both surreal and difficult.''

Heavy rains this time of year often swell China's rivers and cause its
reservoirs to overflow. This year, however, the battle against the
coronavirus pandemic strained flood preparations, People's Daily, the
official Communist Party newspaper,
\href{http://paper.people.com.cn/rmrb/html/2020-04/17/nw.D110000renmrb_20200417_3-06.htm}{warned
in April}.

The epidemic, combined with the extreme rain, has made dealing with this
year's flooding
\href{http://m.people.cn/n4/2020/0701/c175-14102526.html}{a ``very
formidable'' task}, China's postal agency wrote in a recent memo urging
the authorities to step up their response to the floods, People's Daily
reported this week.

After \href{https://mp.weixin.qq.com/s/a9pJCMb7WM5T52jXpJMHeQ}{31
consecutive days} of official alerts about torrential rain, the
inclement weather shows little sign of letting up. On Friday, the
National Meteorological Center
\href{http://www.nmc.cn/publish/weather-bulletin/index.htm}{forecast
another round of downpours} in China's southwest beginning on Saturday.
Experts are warning of potential landslides and bursts at reservoirs and
dams.

\includegraphics{https://static01.nyt.com/images/2020/07/03/world/03china-floods-2/merlin_173488344_e061ff35-3386-4e77-be19-7c508d802cea-articleLarge.jpg?quality=75\&auto=webp\&disable=upscale}

In China, most small reservoirs were built in the 1960s and '70s and did
not follow high construction standards, said Brandon Meng, a hydraulic
engineer in the southern city of Shenzhen.

``Once there is extreme weather,'' he said, ``it's very easy for them to
be in danger.''

As the rains were becoming intense last month,
\href{https://mp.weixin.qq.com/s?__biz=MjM5NzQwNjcyMQ==\&mid=2651038705\&idx=1\&sn=869571fbb6b9f41381c32e8b6d008d01\&chksm=bd2d729f8a5afb89876d8a3e72274b92cfc357392dd18ca3e9d67ce893e6eec47d988bd82c9e\&scene=4}{some
commentators in China} noted how little attention they were receiving,
both in Chinese news outlets and on social media. Surely, they said, the
confluence of a great plague and great floods should merit wider
interest.

Perhaps people had
\href{https://www.weibo.com/1463665055/J9kyy49Fe?refer_flag=1001030103_\&type=comment}{grown
numb} to hardship. Or perhaps
\href{https://theinitium.com/article/2020629-mainland-invisible-flood/?utm_medium=copy}{China's
government and its censors} did not want to draw more attention to
people's suffering.

Either way, videos and firsthand accounts of the flooding have since
gained wider notice.

Image

Distributing food to people affected by floods at a temporary shelter in
Mianning County, in Sichuan Province, this week.Credit...Agence
France-Presse --- Getty Images

In Yangshuo, a popular travel destination known for its stunning
mountain vistas, an official
\href{https://www.infzm.com/contents/186093}{told the newsmagazine
Southern Weekly} that the area had experienced a once-in-two-centuries
burst of heavy rain on June 7. More than 1,000 hotels and guesthouses
and 5,000 shops were damaged, the authorities told Southern Weekly.

Qin Hui, a retired history professor, was vacationing in Yangshuo when
the rain started coming down in buckets last month. He and his travel
partners tried to evacuate but decided that it was too dangerous. They
were eating breakfast the next morning when they caught a disturbing
sight.

``The swimming pool outside the window suddenly went from clear to
muddy,'' Mr. Qin recounted in
\href{http://qinhui.blog.caixin.com/archives/230173}{an online essay}.
``It turned out to be floodwater coming in from the tube at the bottom
of the pool. Soon after, the murky water flooded out of the pool,
quickly covered the yard and then flowed up the stairs.''

They were trapped in their hotel for two nights until a volunteer team
rescued them.

Image

Wading through floodwaters in Hefei, Anhui Province, on June 27, in a
photo released by the state-run China Daily.Credit...China Daily, via
Reuters

In Chongqing, the city authorities said last month that flooding along
the local section of the Qijiang River, upstream from the Yangtze, was
the worst
\href{https://www.chinadaily.com.cn/a/202006/22/WS5ef09caca310834817254b2c.html}{since
monitoring began in 1940}. About
\href{https://mp.weixin.qq.com/s/dYW1K1N6nqmvzLmCAeQiLQ}{40,000
residents were evacuated}, according to official news outlets.

Chongqing is in a mountainous part of China, and many structures are
built directly into hillsides.
\href{https://mp.weixin.qq.com/s?__biz=MzU4NzExNTgwOQ==\&mid=2247500615\&idx=5\&sn=098259bb066399950b6e7c11dc4950f8\&chksm=fdf27f09ca85f61f6ee1bd541a1376d75d14f048ad610422307a289006682225c7be6ffaadbc\#rd}{A
video} from one district showed brown water gushing out of an upper
window of a residential building, like an artificial waterfall.

Wang Yiwei contributed research.

Advertisement

\protect\hyperlink{after-bottom}{Continue reading the main story}

\hypertarget{site-index}{%
\subsection{Site Index}\label{site-index}}

\hypertarget{site-information-navigation}{%
\subsection{Site Information
Navigation}\label{site-information-navigation}}

\begin{itemize}
\tightlist
\item
  \href{https://help.nytimes.com/hc/en-us/articles/115014792127-Copyright-notice}{©~2020~The
  New York Times Company}
\end{itemize}

\begin{itemize}
\tightlist
\item
  \href{https://www.nytco.com/}{NYTCo}
\item
  \href{https://help.nytimes.com/hc/en-us/articles/115015385887-Contact-Us}{Contact
  Us}
\item
  \href{https://www.nytco.com/careers/}{Work with us}
\item
  \href{https://nytmediakit.com/}{Advertise}
\item
  \href{http://www.tbrandstudio.com/}{T Brand Studio}
\item
  \href{https://www.nytimes.com/privacy/cookie-policy\#how-do-i-manage-trackers}{Your
  Ad Choices}
\item
  \href{https://www.nytimes.com/privacy}{Privacy}
\item
  \href{https://help.nytimes.com/hc/en-us/articles/115014893428-Terms-of-service}{Terms
  of Service}
\item
  \href{https://help.nytimes.com/hc/en-us/articles/115014893968-Terms-of-sale}{Terms
  of Sale}
\item
  \href{https://spiderbites.nytimes.com}{Site Map}
\item
  \href{https://help.nytimes.com/hc/en-us}{Help}
\item
  \href{https://www.nytimes.com/subscription?campaignId=37WXW}{Subscriptions}
\end{itemize}
