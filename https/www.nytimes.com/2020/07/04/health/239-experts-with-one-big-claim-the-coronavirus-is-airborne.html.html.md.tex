Sections

SEARCH

\protect\hyperlink{site-content}{Skip to
content}\protect\hyperlink{site-index}{Skip to site index}

\href{https://www.nytimes.com/section/health}{Health}

\href{https://myaccount.nytimes.com/auth/login?response_type=cookie\&client_id=vi}{}

\href{https://www.nytimes.com/section/todayspaper}{Today's Paper}

\href{/section/health}{Health}\textbar{}239 Experts With One Big Claim:
The Coronavirus Is Airborne

\url{https://nyti.ms/3ix7pEy}

\begin{itemize}
\item
\item
\item
\item
\item
\item
\end{itemize}

\href{https://www.nytimes.com/news-event/coronavirus?action=click\&pgtype=Article\&state=default\&region=TOP_BANNER\&context=storylines_menu}{The
Coronavirus Outbreak}

\begin{itemize}
\tightlist
\item
  live\href{https://www.nytimes.com/2020/08/01/world/coronavirus-covid-19.html?action=click\&pgtype=Article\&state=default\&region=TOP_BANNER\&context=storylines_menu}{Latest
  Updates}
\item
  \href{https://www.nytimes.com/interactive/2020/us/coronavirus-us-cases.html?action=click\&pgtype=Article\&state=default\&region=TOP_BANNER\&context=storylines_menu}{Maps
  and Cases}
\item
  \href{https://www.nytimes.com/interactive/2020/science/coronavirus-vaccine-tracker.html?action=click\&pgtype=Article\&state=default\&region=TOP_BANNER\&context=storylines_menu}{Vaccine
  Tracker}
\item
  \href{https://www.nytimes.com/interactive/2020/07/29/us/schools-reopening-coronavirus.html?action=click\&pgtype=Article\&state=default\&region=TOP_BANNER\&context=storylines_menu}{What
  School May Look Like}
\item
  \href{https://www.nytimes.com/live/2020/07/31/business/stock-market-today-coronavirus?action=click\&pgtype=Article\&state=default\&region=TOP_BANNER\&context=storylines_menu}{Economy}
\end{itemize}

Advertisement

\protect\hyperlink{after-top}{Continue reading the main story}

Supported by

\protect\hyperlink{after-sponsor}{Continue reading the main story}

\hypertarget{239-experts-with-one-big-claim-the-coronavirus-is-airborne}{%
\section{239 Experts With One Big Claim: The Coronavirus Is
Airborne}\label{239-experts-with-one-big-claim-the-coronavirus-is-airborne}}

The W.H.O. has resisted mounting evidence that viral particles floating
indoors are infectious, some scientists say. The agency maintains the
research is still inconclusive.

\includegraphics{https://static01.nyt.com/images/2020/07/04/science/04virus-aerosols3/04virus-aerosols3-articleLarge.jpg?quality=75\&auto=webp\&disable=upscale}

By \href{https://www.nytimes.com/by/apoorva-mandavilli}{Apoorva
Mandavilli}

\begin{itemize}
\item
  Published July 4, 2020Updated July 7, 2020
\item
  \begin{itemize}
  \item
  \item
  \item
  \item
  \item
  \item
  \end{itemize}
\end{itemize}

\href{https://www.nytimes.com/es/2020/07/06/espanol/ciencia-y-tecnologia/coronavirus-transmision-aire.html}{Leer
en español}

The
\href{https://www.nytimes.com/2020/07/04/health/coronavirus-neanderthals.html}{coronavirus}
is finding new victims worldwide, in bars and restaurants, offices,
markets and casinos, giving rise to frightening clusters of infection
that increasingly confirm what many scientists have been saying for
months:
\href{https://www.nytimes.com/2020/07/09/health/virus-aerosols-who.html}{The
virus lingers in the air indoors}, infecting those nearby.

If airborne transmission is a significant factor in the pandemic,
especially in crowded spaces with poor ventilation, the
\href{https://www.nytimes.com/2020/07/06/podcasts/the-daily/coronavirus-science-indoor-infection.html}{consequences
for containment will be significant}. Masks may be needed indoors, even
in socially-distant settings. Health care workers may need N95 masks
that filter out even the smallest respiratory droplets as they care for
coronavirus patients.

Ventilation systems in schools, nursing homes, residences and businesses
may need to minimize recirculating air and add powerful new filters.
Ultraviolet lights may be needed to kill viral particles floating in
tiny droplets indoors.

The World Health Organization has long held that the coronavirus is
spread primarily by large respiratory droplets that, once expelled by
infected people in coughs and sneezes, fall quickly to the floor.

But in an open letter to the W.H.O., 239 scientists in 32 countries
\href{https://academic.oup.com/cid/article/doi/10.1093/cid/ciaa939/5867798}{have
outlined the evidence showing that smaller particles can infect people},
and are calling for the agency to revise its recommendations. The
researchers plan to publish their letter in a scientific journal next
week.

Even in its latest update on the coronavirus, released June 29, the
W.H.O. said airborne transmission of the virus is possible only
\href{https://www.who.int/publications/i/item/WHO-2019-nCoV-IPC-2020.4}{after
medical procedures} that produce aerosols, or droplets smaller than 5
microns. (A micron is equal to one millionth of a meter.)

Proper ventilation and N95 masks are of concern only in those
circumstances, according to the W.H.O. Instead, its infection control
guidance, before and
\href{https://www.who.int/infection-prevention/en/}{during} this
pandemic, has
\href{https://www.who.int/infection-prevention/campaigns/ipc-global-survey-2019/en/}{heavily}
\href{https://www.who.int/infection-prevention/campaigns/clean-hands/5may2019/en/}{promoted}
the importance of \href{https://www.who.int/gpsc/ipc/en/}{handwashing}
as a primary prevention strategy, even though there is limited evidence
for transmission of the virus from surfaces. (The Centers for Disease
Control and Prevention now says surfaces are likely to play only a minor
role.)

Dr. Benedetta Allegranzi, the W.H.O.'s technical lead on infection
control, said the evidence for the virus spreading by air was
unconvincing.

``Especially in the last couple of months, we have been stating several
times that we consider airborne transmission as possible but certainly
not supported by solid or even clear evidence,'' she said. ``There is a
strong debate on this.''

But interviews with nearly 20 scientists --- including a dozen W.H.O.
consultants and several members of the committee that crafted the
guidance --- and internal emails paint a picture of an organization
that, despite good intentions, is out of step with science.

Whether carried aloft by large droplets that zoom through the air after
a sneeze, or by much smaller exhaled droplets that may glide the length
of a room, these experts said, the coronavirus is borne through air and
can infect people when inhaled.

Most of these experts sympathized with the W.H.O.'s growing portfolio
and shrinking budget, and noted the tricky political relationships it
has to manage, especially with the United States and China. They praised
W.H.O. staff for holding daily briefings and tirelessly answering
questions about the pandemic.

\hypertarget{latest-updates-global-coronavirus-outbreak}{%
\section{\texorpdfstring{\href{https://www.nytimes.com/2020/08/01/world/coronavirus-covid-19.html?action=click\&pgtype=Article\&state=default\&region=MAIN_CONTENT_1\&context=storylines_live_updates}{Latest
Updates: Global Coronavirus
Outbreak}}{Latest Updates: Global Coronavirus Outbreak}}\label{latest-updates-global-coronavirus-outbreak}}

Updated 2020-08-02T00:39:11.591Z

\begin{itemize}
\tightlist
\item
  \href{https://www.nytimes.com/2020/08/01/world/coronavirus-covid-19.html?action=click\&pgtype=Article\&state=default\&region=MAIN_CONTENT_1\&context=storylines_live_updates\#link-34047410}{The
  U.S. reels as July cases more than double the total of any other
  month.}
\item
  \href{https://www.nytimes.com/2020/08/01/world/coronavirus-covid-19.html?action=click\&pgtype=Article\&state=default\&region=MAIN_CONTENT_1\&context=storylines_live_updates\#link-3ac56579}{Top
  officials work to break impasse over jobless benefit.}
\item
  \href{https://www.nytimes.com/2020/08/01/world/coronavirus-covid-19.html?action=click\&pgtype=Article\&state=default\&region=MAIN_CONTENT_1\&context=storylines_live_updates\#link-25930521}{Thousands
  in Berlin protest Germany's coronavirus measures.}
\end{itemize}

\href{https://www.nytimes.com/2020/08/01/world/coronavirus-covid-19.html?action=click\&pgtype=Article\&state=default\&region=MAIN_CONTENT_1\&context=storylines_live_updates}{See
more updates}

More live coverage:
\href{https://www.nytimes.com/live/2020/07/31/business/stock-market-today-coronavirus?action=click\&pgtype=Article\&state=default\&region=MAIN_CONTENT_1\&context=storylines_live_updates}{Markets}

But the infection prevention and control committee in particular,
experts said, is bound by a rigid and overly medicalized view of
scientific evidence, is slow and risk-averse in updating its guidance
and allows a few conservative voices to shout down dissent.

``They'll die defending their view,'' said one longstanding W.H.O.
consultant, who did not wish to be identified because of her continuing
work for the organization. Even its staunchest supporters said the
committee should
\href{https://twitter.com/JoyAgnost/status/1263802269658644480}{diversify
its expertise} and relax its criteria for proof, especially in a
fast-moving outbreak.

``I do get frustrated about the issues of airflow and sizing of
particles, absolutely,'' said Mary-Louise McLaws, a committee member and
epidemiologist at the University of New South Wales in Sydney.

``If we started revisiting airflow, we would have to be prepared to
change a lot of what we do,'' she said. ``I think it's a good idea, a
very good idea, but it will cause an enormous shudder through the
infection control society.''

In early April, a group of 36 experts on air quality and aerosols urged
the W.H.O. to consider the growing evidence on airborne transmission of
the coronavirus. The agency responded promptly, calling Lidia Morawska,
the group's leader and a longtime W.H.O. consultant, to arrange a
meeting.

But the discussion was dominated by a few experts who are staunch
supporters of handwashing and felt it must be emphasized over aerosols,
according to some participants, and the committee's advice remained
unchanged.

Dr. Morawska and others pointed to
\href{https://www.nytimes.com/2020/05/12/health/coronavirus-choir.html}{several}
\href{https://www.nytimes.com/2020/04/20/health/airflow-coronavirus-restaurants.html}{incidents}
that indicate
\href{https://news.sky.com/story/coronavirus-circulating-air-may-have-spread-covid-19-to-1-500-german-meat-plant-staff-12014156}{airborne
transmission} of the virus, particularly in poorly ventilated and
crowded indoor spaces. They said the W.H.O. was making an artificial
distinction between tiny aerosols and larger droplets, even though
infected people produce both.

``We've known since 1946 that coughing and talking generate aerosols,''
said Linsey Marr, an expert in airborne transmission of viruses at
Virginia Tech.

Scientists have not been able to grow the coronavirus from aerosols in
the lab. But that doesn't mean aerosols are not infective, Dr. Marr
said: Most of the
\href{https://www.sciencedirect.com/science/article/pii/S0013935120307143?via\%3Dihub}{samples
in those experiments have come from hospital rooms} with good air flow
that would dilute viral levels.

In most buildings, she said, ``the air-exchange rate is usually much
lower, allowing virus to accumulate in the air and pose a greater
risk.''

The W.H.O. also is relying on a dated definition of airborne
transmission, Dr. Marr said. The agency believes an airborne pathogen,
like the measles virus, has to be highly infectious and to travel long
distances.

People generally ``think and talk about airborne transmission profoundly
stupidly,'' said Bill Hanage, an epidemiologist at the Harvard T.H. Chan
School of Public Health.

``We have this notion that airborne transmission means droplets hanging
in the air capable of infecting you many hours later, drifting down
streets, through letter boxes and finding their way into homes
everywhere,'' Dr. Hanage said.

\includegraphics{https://static01.nyt.com/images/2020/07/04/science/04virus-aerosols4/04virus-aerosols4-articleLarge.jpg?quality=75\&auto=webp\&disable=upscale}

Experts all agree that the coronavirus does not behave that way. Dr.
Marr and others said the coronavirus seemed to be most infectious when
people were in prolonged contact at close range, especially indoors, and
even more so in
\href{https://www.nytimes.com/2020/06/30/science/how-coronavirus-spreads.html}{superspreader
events} --- exactly what scientists would expect from aerosol
transmission.

\hypertarget{precautionary-principle}{%
\subsection{Precautionary principle}\label{precautionary-principle}}

The W.H.O. has found itself at odds with groups of scientists more than
once during this pandemic.

The agency lagged behind most of its member nations in
\href{https://www.nytimes.com/2020/06/05/health/coronavirus-masks-who.html}{endorsing
face coverings} for the public. While other organizations, including the
C.D.C., have long since acknowledged the importance of transmission
\href{https://www.nytimes.com/2020/06/27/world/europe/coronavirus-spread-asymptomatic.html}{by
people without symptoms}, the W.H.O. still
\href{https://www.nytimes.com/2020/06/09/health/coronavirus-asymptomatic-world-health-organization.html}{maintains
that asymptomatic transmission is rare}.

``At the country level, a lot of W.H.O. technical staff are scratching
their heads,'' said a consultant at a regional office in Southeast Asia,
who did not wish to be identified because he was worried about losing
his contract. ``This is not giving us credibility.''

The consultant recalled that the W.H.O. staff members in his country
were the only ones to go without masks after the government there
endorsed them.

Many experts said the W.H.O. should embrace what some called a
``precautionary principle'' and others called ``needs and values'' ---
the idea that even without definitive evidence, the agency should assume
the worst of the virus, apply common sense and recommend the best
protection possible.

\href{https://www.nytimes.com/news-event/coronavirus?action=click\&pgtype=Article\&state=default\&region=MAIN_CONTENT_3\&context=storylines_faq}{}

\hypertarget{the-coronavirus-outbreak-}{%
\subsubsection{The Coronavirus Outbreak
›}\label{the-coronavirus-outbreak-}}

\hypertarget{frequently-asked-questions}{%
\paragraph{Frequently Asked
Questions}\label{frequently-asked-questions}}

Updated July 27, 2020

\begin{itemize}
\item ~
  \hypertarget{should-i-refinance-my-mortgage}{%
  \paragraph{Should I refinance my
  mortgage?}\label{should-i-refinance-my-mortgage}}

  \begin{itemize}
  \tightlist
  \item
    \href{https://www.nytimes.com/article/coronavirus-money-unemployment.html?action=click\&pgtype=Article\&state=default\&region=MAIN_CONTENT_3\&context=storylines_faq}{It
    could be a good idea,} because mortgage rates have
    \href{https://www.nytimes.com/2020/07/16/business/mortgage-rates-below-3-percent.html?action=click\&pgtype=Article\&state=default\&region=MAIN_CONTENT_3\&context=storylines_faq}{never
    been lower.} Refinancing requests have pushed mortgage applications
    to some of the highest levels since 2008, so be prepared to get in
    line. But defaults are also up, so if you're thinking about buying a
    home, be aware that some lenders have tightened their standards.
  \end{itemize}
\item ~
  \hypertarget{what-is-school-going-to-look-like-in-september}{%
  \paragraph{What is school going to look like in
  September?}\label{what-is-school-going-to-look-like-in-september}}

  \begin{itemize}
  \tightlist
  \item
    It is unlikely that many schools will return to a normal schedule
    this fall, requiring the grind of
    \href{https://www.nytimes.com/2020/06/05/us/coronavirus-education-lost-learning.html?action=click\&pgtype=Article\&state=default\&region=MAIN_CONTENT_3\&context=storylines_faq}{online
    learning},
    \href{https://www.nytimes.com/2020/05/29/us/coronavirus-child-care-centers.html?action=click\&pgtype=Article\&state=default\&region=MAIN_CONTENT_3\&context=storylines_faq}{makeshift
    child care} and
    \href{https://www.nytimes.com/2020/06/03/business/economy/coronavirus-working-women.html?action=click\&pgtype=Article\&state=default\&region=MAIN_CONTENT_3\&context=storylines_faq}{stunted
    workdays} to continue. California's two largest public school
    districts --- Los Angeles and San Diego --- said on July 13, that
    \href{https://www.nytimes.com/2020/07/13/us/lausd-san-diego-school-reopening.html?action=click\&pgtype=Article\&state=default\&region=MAIN_CONTENT_3\&context=storylines_faq}{instruction
    will be remote-only in the fall}, citing concerns that surging
    coronavirus infections in their areas pose too dire a risk for
    students and teachers. Together, the two districts enroll some
    825,000 students. They are the largest in the country so far to
    abandon plans for even a partial physical return to classrooms when
    they reopen in August. For other districts, the solution won't be an
    all-or-nothing approach.
    \href{https://bioethics.jhu.edu/research-and-outreach/projects/eschool-initiative/school-policy-tracker/}{Many
    systems}, including the nation's largest, New York City, are
    devising
    \href{https://www.nytimes.com/2020/06/26/us/coronavirus-schools-reopen-fall.html?action=click\&pgtype=Article\&state=default\&region=MAIN_CONTENT_3\&context=storylines_faq}{hybrid
    plans} that involve spending some days in classrooms and other days
    online. There's no national policy on this yet, so check with your
    municipal school system regularly to see what is happening in your
    community.
  \end{itemize}
\item ~
  \hypertarget{is-the-coronavirus-airborne}{%
  \paragraph{Is the coronavirus
  airborne?}\label{is-the-coronavirus-airborne}}

  \begin{itemize}
  \tightlist
  \item
    The coronavirus
    \href{https://www.nytimes.com/2020/07/04/health/239-experts-with-one-big-claim-the-coronavirus-is-airborne.html?action=click\&pgtype=Article\&state=default\&region=MAIN_CONTENT_3\&context=storylines_faq}{can
    stay aloft for hours in tiny droplets in stagnant air}, infecting
    people as they inhale, mounting scientific evidence suggests. This
    risk is highest in crowded indoor spaces with poor ventilation, and
    may help explain super-spreading events reported in meatpacking
    plants, churches and restaurants.
    \href{https://www.nytimes.com/2020/07/06/health/coronavirus-airborne-aerosols.html?action=click\&pgtype=Article\&state=default\&region=MAIN_CONTENT_3\&context=storylines_faq}{It's
    unclear how often the virus is spread} via these tiny droplets, or
    aerosols, compared with larger droplets that are expelled when a
    sick person coughs or sneezes, or transmitted through contact with
    contaminated surfaces, said Linsey Marr, an aerosol expert at
    Virginia Tech. Aerosols are released even when a person without
    symptoms exhales, talks or sings, according to Dr. Marr and more
    than 200 other experts, who
    \href{https://academic.oup.com/cid/article/doi/10.1093/cid/ciaa939/5867798}{have
    outlined the evidence in an open letter to the World Health
    Organization}.
  \end{itemize}
\item ~
  \hypertarget{what-are-the-symptoms-of-coronavirus}{%
  \paragraph{What are the symptoms of
  coronavirus?}\label{what-are-the-symptoms-of-coronavirus}}

  \begin{itemize}
  \tightlist
  \item
    Common symptoms
    \href{https://www.nytimes.com/article/symptoms-coronavirus.html?action=click\&pgtype=Article\&state=default\&region=MAIN_CONTENT_3\&context=storylines_faq}{include
    fever, a dry cough, fatigue and difficulty breathing or shortness of
    breath.} Some of these symptoms overlap with those of the flu,
    making detection difficult, but runny noses and stuffy sinuses are
    less common.
    \href{https://www.nytimes.com/2020/04/27/health/coronavirus-symptoms-cdc.html?action=click\&pgtype=Article\&state=default\&region=MAIN_CONTENT_3\&context=storylines_faq}{The
    C.D.C. has also} added chills, muscle pain, sore throat, headache
    and a new loss of the sense of taste or smell as symptoms to look
    out for. Most people fall ill five to seven days after exposure, but
    symptoms may appear in as few as two days or as many as 14 days.
  \end{itemize}
\item ~
  \hypertarget{does-asymptomatic-transmission-of-covid-19-happen}{%
  \paragraph{Does asymptomatic transmission of Covid-19
  happen?}\label{does-asymptomatic-transmission-of-covid-19-happen}}

  \begin{itemize}
  \tightlist
  \item
    So far, the evidence seems to show it does. A widely cited
    \href{https://www.nature.com/articles/s41591-020-0869-5}{paper}
    published in April suggests that people are most infectious about
    two days before the onset of coronavirus symptoms and estimated that
    44 percent of new infections were a result of transmission from
    people who were not yet showing symptoms. Recently, a top expert at
    the World Health Organization stated that transmission of the
    coronavirus by people who did not have symptoms was ``very rare,''
    \href{https://www.nytimes.com/2020/06/09/world/coronavirus-updates.html?action=click\&pgtype=Article\&state=default\&region=MAIN_CONTENT_3\&context=storylines_faq\#link-1f302e21}{but
    she later walked back that statement.}
  \end{itemize}
\end{itemize}

``There is no incontrovertible proof that SARS-CoV-2 travels or is
transmitted significantly by aerosols, but there is absolutely no
evidence that it's not,'' said Dr. Trish Greenhalgh, a primary care
doctor at the University of Oxford in Britain.

``So at the moment we have to make a decision in the face of
uncertainty, and my goodness, it's going to be a disastrous decision if
we get it wrong,'' she said. ``So why not just mask up for a few weeks,
just in case?''

After all, the W.H.O. seems willing to accept without much evidence the
idea that the virus may be transmitted from surfaces, she and other
researchers noted, even as other health agencies have stepped back
emphasizing this route.

``I agree that fomite transmission is not directly demonstrated for this
virus,'' Dr. Allegranzi, the W.H.O.'s technical lead on infection
control, said, referring to objects that may be infectious. ``But it is
well known that other coronaviruses and respiratory viruses are
transmitted, and demonstrated to be transmitted, by contact with
fomite.''

The agency also must consider the needs of all its member nations,
including those with limited resources, and make sure its
recommendations are tempered by ``availability, feasibility, compliance,
resource implications,'' she said.

\textbf{\emph{{[}}\href{http://on.fb.me/1paTQ1h}{\emph{Like the Science
Times page on Facebook.}}} ****** \emph{\textbar{} Sign up for the}
\textbf{\href{http://nyti.ms/1MbHaRU}{\emph{Science Times
newsletter.}}\emph{{]}}}

Aerosols may play some limited role in spreading the virus, said Dr.
Paul Hunter, a member of the infection prevention committee and
professor of medicine at the University of East Anglia in Britain.

But if the W.H.O. were to push for rigorous control measures in the
absence of proof, hospitals in low- and middle-income countries may be
forced to divert scarce resources from other crucial programs.

``That's the balance that an organization like the W.H.O. has to
achieve,'' he said. ``It's the easiest thing in the world to say, `We've
got to follow the precautionary principle,' and ignore the opportunity
costs of that.''

In interviews, other scientists criticized this view as paternalistic.
```We're not going to say what we really think, because we think you
can't deal with it?' I don't think that's right,'' said Don Milton, an
aerosol expert at the University of Maryland.

Even cloth masks, if worn by everyone, can significantly reduce
transmission, and the W.H.O. should say so clearly, he added.

Several experts criticized the W.H.O.'s messaging throughout the
pandemic, saying the staff seems to prize scientific perspective over
clarity.

``What you say is designed to help people understand the nature of a
public health problem,'' said Dr. William Aldis, a longtime W.H.O.
collaborator based in Thailand. ``That's different than just
scientifically describing a disease or a virus.''

The W.H.O. tends to describe ``an absence of evidence as evidence of
absence,'' Dr. Aldis added. In April, for example,
\href{https://www.reuters.com/article/us-health-coronavirus-who-idUSKCN2270FB}{the
W.H.O. said}, ``There is currently no evidence that people who have
recovered from Covid-19 and have antibodies are protected from a second
infection.''

The statement was intended to indicate uncertainty, but the phrasing
stoked unease among the public and earned rebukes from several experts
and journalists. The W.H.O. later walked back its comments.

In a less public instance, the W.H.O. said there was ``no evidence to
suggest'' that people with H.I.V. were at increased risk from the
coronavirus. After Joseph Amon, the director of global health at Drexel
University in Philadelphia who has sat on many agency committees,
pointed out that the phrasing was misleading, the W.H.O. changed it to
say the
\href{https://www.who.int/emergencies/diseases/novel-coronavirus-2019/question-and-answers-hub/q-a-detail/q-a-on-covid-19-hiv-and-antiretrovirals}{level
of risk was ``unknown.''}

But W.H.O. staff and some members said the critics did not give its
committees enough credit.

``Those that may have been frustrated may not be cognizant of how W.H.O.
expert committees work, and they work slowly and deliberately,'' Dr.
McLaws said.

Dr. Soumya Swaminathan, the W.H.O.'s chief scientist, said agency staff
members were trying to evaluate new scientific evidence as fast as
possible, but without sacrificing the quality of their review. She added
that the agency will try to broaden the committees' expertise and
communications to make sure everyone is heard.

``We take it seriously when journalists or scientists or anyone
challenges us and say we can do better than this,'' she said. ``We
definitely want to do better.''

Advertisement

\protect\hyperlink{after-bottom}{Continue reading the main story}

\hypertarget{site-index}{%
\subsection{Site Index}\label{site-index}}

\hypertarget{site-information-navigation}{%
\subsection{Site Information
Navigation}\label{site-information-navigation}}

\begin{itemize}
\tightlist
\item
  \href{https://help.nytimes.com/hc/en-us/articles/115014792127-Copyright-notice}{©~2020~The
  New York Times Company}
\end{itemize}

\begin{itemize}
\tightlist
\item
  \href{https://www.nytco.com/}{NYTCo}
\item
  \href{https://help.nytimes.com/hc/en-us/articles/115015385887-Contact-Us}{Contact
  Us}
\item
  \href{https://www.nytco.com/careers/}{Work with us}
\item
  \href{https://nytmediakit.com/}{Advertise}
\item
  \href{http://www.tbrandstudio.com/}{T Brand Studio}
\item
  \href{https://www.nytimes.com/privacy/cookie-policy\#how-do-i-manage-trackers}{Your
  Ad Choices}
\item
  \href{https://www.nytimes.com/privacy}{Privacy}
\item
  \href{https://help.nytimes.com/hc/en-us/articles/115014893428-Terms-of-service}{Terms
  of Service}
\item
  \href{https://help.nytimes.com/hc/en-us/articles/115014893968-Terms-of-sale}{Terms
  of Sale}
\item
  \href{https://spiderbites.nytimes.com}{Site Map}
\item
  \href{https://help.nytimes.com/hc/en-us}{Help}
\item
  \href{https://www.nytimes.com/subscription?campaignId=37WXW}{Subscriptions}
\end{itemize}
