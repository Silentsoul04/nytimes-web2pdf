Sections

SEARCH

\protect\hyperlink{site-content}{Skip to
content}\protect\hyperlink{site-index}{Skip to site index}

\href{https://www.nytimes.com/section/health}{Health}

\href{https://myaccount.nytimes.com/auth/login?response_type=cookie\&client_id=vi}{}

\href{https://www.nytimes.com/section/todayspaper}{Today's Paper}

\href{/section/health}{Health}\textbar{}DNA Inherited From Neanderthals
May Increase Risk of Covid-19

\url{https://nyti.ms/2YX7Ekv}

\begin{itemize}
\item
\item
\item
\item
\item
\item
\end{itemize}

\href{https://www.nytimes.com/news-event/coronavirus?action=click\&pgtype=Article\&state=default\&region=TOP_BANNER\&context=storylines_menu}{The
Coronavirus Outbreak}

\begin{itemize}
\tightlist
\item
  live\href{https://www.nytimes.com/2020/08/01/world/coronavirus-covid-19.html?action=click\&pgtype=Article\&state=default\&region=TOP_BANNER\&context=storylines_menu}{Latest
  Updates}
\item
  \href{https://www.nytimes.com/interactive/2020/us/coronavirus-us-cases.html?action=click\&pgtype=Article\&state=default\&region=TOP_BANNER\&context=storylines_menu}{Maps
  and Cases}
\item
  \href{https://www.nytimes.com/interactive/2020/science/coronavirus-vaccine-tracker.html?action=click\&pgtype=Article\&state=default\&region=TOP_BANNER\&context=storylines_menu}{Vaccine
  Tracker}
\item
  \href{https://www.nytimes.com/interactive/2020/07/29/us/schools-reopening-coronavirus.html?action=click\&pgtype=Article\&state=default\&region=TOP_BANNER\&context=storylines_menu}{What
  School May Look Like}
\item
  \href{https://www.nytimes.com/live/2020/07/31/business/stock-market-today-coronavirus?action=click\&pgtype=Article\&state=default\&region=TOP_BANNER\&context=storylines_menu}{Economy}
\end{itemize}

Advertisement

\protect\hyperlink{after-top}{Continue reading the main story}

Supported by

\protect\hyperlink{after-sponsor}{Continue reading the main story}

\hypertarget{dna-inherited-from-neanderthals-may-increase-risk-of-covid-19}{%
\section{DNA Inherited From Neanderthals May Increase Risk of
Covid-19}\label{dna-inherited-from-neanderthals-may-increase-risk-of-covid-19}}

The stretch of six genes seems to increase the risk of severe illness
from the coronavirus.

\includegraphics{https://static01.nyt.com/images/2020/07/04/multimedia/04virus-neanderthals-pix/04virus-neanderthals-pix-articleLarge.jpg?quality=75\&auto=webp\&disable=upscale}

\href{https://www.nytimes.com/by/carl-zimmer}{\includegraphics{https://static01.nyt.com/images/2018/06/12/multimedia/author-carl-zimmer/author-carl-zimmer-thumbLarge.png}}

By \href{https://www.nytimes.com/by/carl-zimmer}{Carl Zimmer}

\begin{itemize}
\item
  Published July 4, 2020Updated July 8, 2020
\item
  \begin{itemize}
  \item
  \item
  \item
  \item
  \item
  \item
  \end{itemize}
\end{itemize}

A stretch of DNA
\href{https://www.nytimes.com/2020/06/03/health/coronavirus-blood-type-genetics.html}{linked
to Covid-19} was passed down from Neanderthals 60,000 years ago,
according to a new study.

Scientists don't yet know why this particular segment increases the
\href{https://www.nytimes.com/2020/07/08/health/coronavirus-risk-factors.html}{risk}
of severe illness from the coronavirus. But the
\href{https://www.biorxiv.org/content/10.1101/2020.07.03.186296v1.full.pdf+html}{new
findings}, which were posted online on Friday and have not yet been
published in a scientific journal, show how some clues to modern health
stem from ancient history.

``This interbreeding effect that happened 60,000 years ago is still
having an impact today,'' said Joshua Akey, a geneticist at Princeton
University who was not involved in the new study.

This piece of the genome, which spans six genes on Chromosome 3, has had
a puzzling journey through human history, the study found. The variant
is now common in Bangladesh, where 63 percent of people carry at least
one copy. Across all of South Asia, almost one-third of people have
inherited the segment.

Elsewhere, however, the segment is far less common. Only 8 percent of
Europeans carry it, and just 4 percent have it in East Asia. It is
almost completely absent in Africa.

It's not clear what evolutionary pattern produced this distribution over
the past 60,000 years. ``That's the \$10,000 question,'' said Hugo
Zeberg, a geneticist at the Karolinska Institute in Sweden who was one
of the authors of the new study.

One possibility is that the Neanderthal version is harmful and has been
getting rarer over all. It's also possible that the segment improved
people's health in South Asia, perhaps providing a strong immune
response to viruses in the region.

\hypertarget{latest-updates-global-coronavirus-outbreak}{%
\section{\texorpdfstring{\href{https://www.nytimes.com/2020/08/01/world/coronavirus-covid-19.html?action=click\&pgtype=Article\&state=default\&region=MAIN_CONTENT_1\&context=storylines_live_updates}{Latest
Updates: Global Coronavirus
Outbreak}}{Latest Updates: Global Coronavirus Outbreak}}\label{latest-updates-global-coronavirus-outbreak}}

Updated 2020-08-02T07:14:05.841Z

\begin{itemize}
\tightlist
\item
  \href{https://www.nytimes.com/2020/08/01/world/coronavirus-covid-19.html?action=click\&pgtype=Article\&state=default\&region=MAIN_CONTENT_1\&context=storylines_live_updates\#link-34047410}{The
  U.S. reels as July cases more than double the total of any other
  month.}
\item
  \href{https://www.nytimes.com/2020/08/01/world/coronavirus-covid-19.html?action=click\&pgtype=Article\&state=default\&region=MAIN_CONTENT_1\&context=storylines_live_updates\#link-780ec966}{Top
  U.S. officials work to break an impasse over the federal jobless
  benefit.}
\item
  \href{https://www.nytimes.com/2020/08/01/world/coronavirus-covid-19.html?action=click\&pgtype=Article\&state=default\&region=MAIN_CONTENT_1\&context=storylines_live_updates\#link-2bc8948}{Its
  outbreak untamed, Melbourne goes into even greater lockdown.}
\end{itemize}

\href{https://www.nytimes.com/2020/08/01/world/coronavirus-covid-19.html?action=click\&pgtype=Article\&state=default\&region=MAIN_CONTENT_1\&context=storylines_live_updates}{See
more updates}

More live coverage:
\href{https://www.nytimes.com/live/2020/07/31/business/stock-market-today-coronavirus?action=click\&pgtype=Article\&state=default\&region=MAIN_CONTENT_1\&context=storylines_live_updates}{Markets}

``One should stress that at this point this is pure speculation,'' said
Dr. Zeberg's co-author, Svante Paabo, the director of the Max Planck
Institute for Evolutionary Anthropology in Leipzig, Germany.

Researchers are only beginning to understand why Covid-19 is more
dangerous for some people than others. Older people are more likely to
become severely ill than younger ones. Men are at more risk than women.

Social inequality matters, too. In the United States, Black people are
\href{https://www.cdc.gov/coronavirus/2019-ncov/need-extra-precautions/racial-ethnic-minorities.html}{far
more likely} than white people to become severely ill from the
coronavirus, for example, most likely due in part to the country's
history of
\href{https://www.nytimes.com/2020/04/29/magazine/racial-disparities-covid-19.html}{systemic
racism}. It has left Black people with a high rate of chronic diseases
such as diabetes, as well as living conditions and jobs that may
increase exposure to the virus.

Genes play a role as well. Last month, researchers compared people in
Italy and Spain who became very sick with
\href{https://www.nytimes.com/2020/07/08/health/coronavirus-risk-factors.html}{Covid-19}
to those who had only mild infections. They found two places in the
genome
\href{https://www.nytimes.com/2020/06/03/health/coronavirus-blood-type-genetics.html}{associated
with a greater risk}. One is on Chromosome 9 and includes ABO, a gene
that determines blood type. The other is the Neanderthal segment on
Chromosome 3.

But these genetic findings are being rapidly updated as more people
infected with the coronavirus are studied. Just last week, an
international group of scientists called the
\href{https://www.covid19hg.org/}{Covid-19 Host Genetics Initiative}
released a new set of data downplaying the risk of blood type. ``The
jury is still out on ABO,'' said Mark Daly, a geneticist at Harvard
Medical School who is a member of the initiative.

The new data showed an even stronger link between the disease and the
Chromosome 3 segment. People who carry two copies of the variant are
three times more likely to suffer from severe illness than people who do
not.

After the new batch of data came out on Monday, Dr. Zeberg decided to
find out if the Chromosome 3 segment was passed down from Neanderthals.

About 60,000 years ago, some ancestors of modern humans
\href{https://www.nytimes.com/2020/01/31/science/neanderthal-dna-africa.html}{expanded
out of Africa} and swept across Europe, Asia and Australia. These people
encountered Neanderthals and interbred. Once Neanderthal DNA entered our
gene pool, it spread down through the generations, long after
Neanderthals became extinct.

Most Neanderthal genes turned out to be harmful to modern humans. They
may have been a burden on people's health or made it harder to have
children. As a result, Neanderthal genes became rarer, and many
disappeared from our gene pool.

But some genes appear to have provided an evolutionary edge and have
become quite common. In May, Dr. Zeberg, Dr. Paabo and Dr. Janet Kelso,
also of the Max Planck Institute, discovered that one-third of European
women have a
\href{https://academic.oup.com/mbe/article/doi/10.1093/molbev/msaa119/5841671}{Neanderthal
hormone receptor}. It is associated with increased fertility and fewer
miscarriages.

\href{https://www.nytimes.com/news-event/coronavirus?action=click\&pgtype=Article\&state=default\&region=MAIN_CONTENT_3\&context=storylines_faq}{}

\hypertarget{the-coronavirus-outbreak-}{%
\subsubsection{The Coronavirus Outbreak
›}\label{the-coronavirus-outbreak-}}

\hypertarget{frequently-asked-questions}{%
\paragraph{Frequently Asked
Questions}\label{frequently-asked-questions}}

Updated July 27, 2020

\begin{itemize}
\item ~
  \hypertarget{should-i-refinance-my-mortgage}{%
  \paragraph{Should I refinance my
  mortgage?}\label{should-i-refinance-my-mortgage}}

  \begin{itemize}
  \tightlist
  \item
    \href{https://www.nytimes.com/article/coronavirus-money-unemployment.html?action=click\&pgtype=Article\&state=default\&region=MAIN_CONTENT_3\&context=storylines_faq}{It
    could be a good idea,} because mortgage rates have
    \href{https://www.nytimes.com/2020/07/16/business/mortgage-rates-below-3-percent.html?action=click\&pgtype=Article\&state=default\&region=MAIN_CONTENT_3\&context=storylines_faq}{never
    been lower.} Refinancing requests have pushed mortgage applications
    to some of the highest levels since 2008, so be prepared to get in
    line. But defaults are also up, so if you're thinking about buying a
    home, be aware that some lenders have tightened their standards.
  \end{itemize}
\item ~
  \hypertarget{what-is-school-going-to-look-like-in-september}{%
  \paragraph{What is school going to look like in
  September?}\label{what-is-school-going-to-look-like-in-september}}

  \begin{itemize}
  \tightlist
  \item
    It is unlikely that many schools will return to a normal schedule
    this fall, requiring the grind of
    \href{https://www.nytimes.com/2020/06/05/us/coronavirus-education-lost-learning.html?action=click\&pgtype=Article\&state=default\&region=MAIN_CONTENT_3\&context=storylines_faq}{online
    learning},
    \href{https://www.nytimes.com/2020/05/29/us/coronavirus-child-care-centers.html?action=click\&pgtype=Article\&state=default\&region=MAIN_CONTENT_3\&context=storylines_faq}{makeshift
    child care} and
    \href{https://www.nytimes.com/2020/06/03/business/economy/coronavirus-working-women.html?action=click\&pgtype=Article\&state=default\&region=MAIN_CONTENT_3\&context=storylines_faq}{stunted
    workdays} to continue. California's two largest public school
    districts --- Los Angeles and San Diego --- said on July 13, that
    \href{https://www.nytimes.com/2020/07/13/us/lausd-san-diego-school-reopening.html?action=click\&pgtype=Article\&state=default\&region=MAIN_CONTENT_3\&context=storylines_faq}{instruction
    will be remote-only in the fall}, citing concerns that surging
    coronavirus infections in their areas pose too dire a risk for
    students and teachers. Together, the two districts enroll some
    825,000 students. They are the largest in the country so far to
    abandon plans for even a partial physical return to classrooms when
    they reopen in August. For other districts, the solution won't be an
    all-or-nothing approach.
    \href{https://bioethics.jhu.edu/research-and-outreach/projects/eschool-initiative/school-policy-tracker/}{Many
    systems}, including the nation's largest, New York City, are
    devising
    \href{https://www.nytimes.com/2020/06/26/us/coronavirus-schools-reopen-fall.html?action=click\&pgtype=Article\&state=default\&region=MAIN_CONTENT_3\&context=storylines_faq}{hybrid
    plans} that involve spending some days in classrooms and other days
    online. There's no national policy on this yet, so check with your
    municipal school system regularly to see what is happening in your
    community.
  \end{itemize}
\item ~
  \hypertarget{is-the-coronavirus-airborne}{%
  \paragraph{Is the coronavirus
  airborne?}\label{is-the-coronavirus-airborne}}

  \begin{itemize}
  \tightlist
  \item
    The coronavirus
    \href{https://www.nytimes.com/2020/07/04/health/239-experts-with-one-big-claim-the-coronavirus-is-airborne.html?action=click\&pgtype=Article\&state=default\&region=MAIN_CONTENT_3\&context=storylines_faq}{can
    stay aloft for hours in tiny droplets in stagnant air}, infecting
    people as they inhale, mounting scientific evidence suggests. This
    risk is highest in crowded indoor spaces with poor ventilation, and
    may help explain super-spreading events reported in meatpacking
    plants, churches and restaurants.
    \href{https://www.nytimes.com/2020/07/06/health/coronavirus-airborne-aerosols.html?action=click\&pgtype=Article\&state=default\&region=MAIN_CONTENT_3\&context=storylines_faq}{It's
    unclear how often the virus is spread} via these tiny droplets, or
    aerosols, compared with larger droplets that are expelled when a
    sick person coughs or sneezes, or transmitted through contact with
    contaminated surfaces, said Linsey Marr, an aerosol expert at
    Virginia Tech. Aerosols are released even when a person without
    symptoms exhales, talks or sings, according to Dr. Marr and more
    than 200 other experts, who
    \href{https://academic.oup.com/cid/article/doi/10.1093/cid/ciaa939/5867798}{have
    outlined the evidence in an open letter to the World Health
    Organization}.
  \end{itemize}
\item ~
  \hypertarget{what-are-the-symptoms-of-coronavirus}{%
  \paragraph{What are the symptoms of
  coronavirus?}\label{what-are-the-symptoms-of-coronavirus}}

  \begin{itemize}
  \tightlist
  \item
    Common symptoms
    \href{https://www.nytimes.com/article/symptoms-coronavirus.html?action=click\&pgtype=Article\&state=default\&region=MAIN_CONTENT_3\&context=storylines_faq}{include
    fever, a dry cough, fatigue and difficulty breathing or shortness of
    breath.} Some of these symptoms overlap with those of the flu,
    making detection difficult, but runny noses and stuffy sinuses are
    less common.
    \href{https://www.nytimes.com/2020/04/27/health/coronavirus-symptoms-cdc.html?action=click\&pgtype=Article\&state=default\&region=MAIN_CONTENT_3\&context=storylines_faq}{The
    C.D.C. has also} added chills, muscle pain, sore throat, headache
    and a new loss of the sense of taste or smell as symptoms to look
    out for. Most people fall ill five to seven days after exposure, but
    symptoms may appear in as few as two days or as many as 14 days.
  \end{itemize}
\item ~
  \hypertarget{does-asymptomatic-transmission-of-covid-19-happen}{%
  \paragraph{Does asymptomatic transmission of Covid-19
  happen?}\label{does-asymptomatic-transmission-of-covid-19-happen}}

  \begin{itemize}
  \tightlist
  \item
    So far, the evidence seems to show it does. A widely cited
    \href{https://www.nature.com/articles/s41591-020-0869-5}{paper}
    published in April suggests that people are most infectious about
    two days before the onset of coronavirus symptoms and estimated that
    44 percent of new infections were a result of transmission from
    people who were not yet showing symptoms. Recently, a top expert at
    the World Health Organization stated that transmission of the
    coronavirus by people who did not have symptoms was ``very rare,''
    \href{https://www.nytimes.com/2020/06/09/world/coronavirus-updates.html?action=click\&pgtype=Article\&state=default\&region=MAIN_CONTENT_3\&context=storylines_faq\#link-1f302e21}{but
    she later walked back that statement.}
  \end{itemize}
\end{itemize}

Dr. Zeberg knew that other Neanderthal genes that are common today even
\href{https://www.nytimes.com/2018/10/04/science/neanderthal-genes-viruses.html}{help
us fight viruses}. When modern humans expanded into Asia and Europe,
they may have encountered new viruses against which Neanderthals had
already evolved defenses. We have held onto those genes ever since.

Dr. Zeberg looked at Chromosome 3 in an online database of Neanderthal
genomes. He found that the version that raises people's risk of severe
Covid-19 is the same version found in a Neanderthal who lived in Croatia
50,000 years ago. ``I texted Svante immediately,'' Dr. Zeberg said in an
interview, referring to Dr. Paabo.

Dr. Paabo was on vacation in a cottage in the remote Swedish
countryside. Dr. Zeberg showed up the next day, and they worked day and
night until they posted the study online on Friday.

``It's the most crazy vacation I've ever had in this cottage,'' Dr.
Paabo said.

Tony Capra, a geneticist at Vanderbilt University who was not involved
in the study, thought it was plausible that the Neanderthal chunk of DNA
originally provided a benefit --- perhaps even against other viruses.
``But that was 40,000 years ago, and here we are now,'' he said.

It's possible that an immune response that worked against ancient
viruses has ended up overreacting against the new coronavirus. People
who develop severe cases of Covid-19 typically do so because their
immune systems launch uncontrolled attacks that end up scarring their
lungs and causing inflammation.

Dr. Paabo said the DNA segment may account in part for why people of
Bangladeshi descent are
\href{https://www.theguardian.com/world/2020/jun/19/south-asians-in-uk-most-likely-to-die-of-covid-19-study-finds}{dying
at a high rate} of Covid-19 in the United Kingdom.

It's an open question whether this Neanderthal segment continues to keep
a strong link to Covid-19 as Dr. Zeberg and other researchers study more
patients. And it may take discoveries of the segment in ancient fossils
of modern humans to understand why it became so common in some places
but not others.

But Dr. Zeberg said that the 60,000-year journey of this chunk of DNA in
our species might help explain why it's so dangerous today.

``Its evolutionary history may give us some clues,'' Dr. Zeberg said.

\textbf{\emph{{[}}\href{http://on.fb.me/1paTQ1h}{\emph{Like the Science
Times page on Facebook.}}} ****** \emph{\textbar{} Sign up for the}
\textbf{\href{http://nyti.ms/1MbHaRU}{\emph{Science Times
newsletter.}}\emph{{]}}}

Advertisement

\protect\hyperlink{after-bottom}{Continue reading the main story}

\hypertarget{site-index}{%
\subsection{Site Index}\label{site-index}}

\hypertarget{site-information-navigation}{%
\subsection{Site Information
Navigation}\label{site-information-navigation}}

\begin{itemize}
\tightlist
\item
  \href{https://help.nytimes.com/hc/en-us/articles/115014792127-Copyright-notice}{©~2020~The
  New York Times Company}
\end{itemize}

\begin{itemize}
\tightlist
\item
  \href{https://www.nytco.com/}{NYTCo}
\item
  \href{https://help.nytimes.com/hc/en-us/articles/115015385887-Contact-Us}{Contact
  Us}
\item
  \href{https://www.nytco.com/careers/}{Work with us}
\item
  \href{https://nytmediakit.com/}{Advertise}
\item
  \href{http://www.tbrandstudio.com/}{T Brand Studio}
\item
  \href{https://www.nytimes.com/privacy/cookie-policy\#how-do-i-manage-trackers}{Your
  Ad Choices}
\item
  \href{https://www.nytimes.com/privacy}{Privacy}
\item
  \href{https://help.nytimes.com/hc/en-us/articles/115014893428-Terms-of-service}{Terms
  of Service}
\item
  \href{https://help.nytimes.com/hc/en-us/articles/115014893968-Terms-of-sale}{Terms
  of Sale}
\item
  \href{https://spiderbites.nytimes.com}{Site Map}
\item
  \href{https://help.nytimes.com/hc/en-us}{Help}
\item
  \href{https://www.nytimes.com/subscription?campaignId=37WXW}{Subscriptions}
\end{itemize}
