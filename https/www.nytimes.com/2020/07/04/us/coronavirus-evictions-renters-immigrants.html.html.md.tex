Sections

SEARCH

\protect\hyperlink{site-content}{Skip to
content}\protect\hyperlink{site-index}{Skip to site index}

\href{https://www.nytimes.com/section/us}{U.S.}

\href{https://myaccount.nytimes.com/auth/login?response_type=cookie\&client_id=vi}{}

\href{https://www.nytimes.com/section/todayspaper}{Today's Paper}

\href{/section/us}{U.S.}\textbar{}Sleeping Outside in a Pandemic:
Vulnerable Renters Face Evictions

\url{https://nyti.ms/2NQCF38}

\begin{itemize}
\item
\item
\item
\item
\item
\end{itemize}

\href{https://www.nytimes.com/news-event/coronavirus?action=click\&pgtype=Article\&state=default\&region=TOP_BANNER\&context=storylines_menu}{The
Coronavirus Outbreak}

\begin{itemize}
\tightlist
\item
  live\href{https://www.nytimes.com/2020/08/04/world/coronavirus-cases.html?action=click\&pgtype=Article\&state=default\&region=TOP_BANNER\&context=storylines_menu}{Latest
  Updates}
\item
  \href{https://www.nytimes.com/interactive/2020/us/coronavirus-us-cases.html?action=click\&pgtype=Article\&state=default\&region=TOP_BANNER\&context=storylines_menu}{Maps
  and Cases}
\item
  \href{https://www.nytimes.com/interactive/2020/science/coronavirus-vaccine-tracker.html?action=click\&pgtype=Article\&state=default\&region=TOP_BANNER\&context=storylines_menu}{Vaccine
  Tracker}
\item
  \href{https://www.nytimes.com/2020/08/02/us/covid-college-reopening.html?action=click\&pgtype=Article\&state=default\&region=TOP_BANNER\&context=storylines_menu}{College
  Reopening}
\item
  \href{https://www.nytimes.com/live/2020/08/04/business/stock-market-today-coronavirus?action=click\&pgtype=Article\&state=default\&region=TOP_BANNER\&context=storylines_menu}{Economy}
\end{itemize}

Advertisement

\protect\hyperlink{after-top}{Continue reading the main story}

Supported by

\protect\hyperlink{after-sponsor}{Continue reading the main story}

\hypertarget{sleeping-outside-in-a-pandemic-vulnerable-renters-face-evictions}{%
\section{Sleeping Outside in a Pandemic: Vulnerable Renters Face
Evictions}\label{sleeping-outside-in-a-pandemic-vulnerable-renters-face-evictions}}

As the coronavirus prevents people from going back to work and paying
their rent, evictions have begun, often targeting vulnerable people such
as unauthorized immigrants.

\includegraphics{https://static01.nyt.com/images/2020/07/05/us/05virus-evictions1-print4/merlin_173937597_9f7ce748-c938-4a35-b07a-7659aebb2263-articleLarge.jpg?quality=75\&auto=webp\&disable=upscale}

\href{https://www.nytimes.com/by/caitlin-dickerson}{\includegraphics{https://static01.nyt.com/images/2018/11/06/multimedia/author-caitlin-dickerson/author-caitlin-dickerson-thumbLarge-v2.png}}

By \href{https://www.nytimes.com/by/caitlin-dickerson}{Caitlin
Dickerson}

\begin{itemize}
\item
  Published July 4, 2020Updated July 23, 2020
\item
  \begin{itemize}
  \item
  \item
  \item
  \item
  \item
  \end{itemize}
\end{itemize}

After dark on a warm night this spring, Wilfredo Avila and his
girlfriend Patricia Carillo lumbered toward the park near their
apartment in Washington, D.C., their arms heavy with plastic bags filled
with their clothes.

The couple had only lost their jobs at an auto repair shop and bakery
two weeks earlier. It was enough time to nudge their life already on the
precipice of disaster --- in which any cash they earned was almost
immediately spent on necessities --- over the edge.

A day earlier, they had offered their landlord everything they had: half
of the \$925 they owed for the bedroom they were subletting in the man's
home.

``Be out by tomorrow,'' he told them, ``or we're going to have
problems.''

Mr. Avila, who is from Honduras, knew from watching the news that
evictions had been banned in Washington during the pandemic. He also
knew that calling the authorities could draw attention to his lack of
immigration status. So the couple set off.

They walked for hours, talking about what to do next. Eventually they
sat down on a bench to rest, still with no plan, and nodded off to
sleep.

\includegraphics{https://static01.nyt.com/images/2020/07/05/us/05virus-evictions2-print2/merlin_173937639_3b942a04-8a3f-4339-9b4a-de0ecc403805-articleLarge.jpg?quality=75\&auto=webp\&disable=upscale}

When the nation's economy ground to a halt this spring, economists
warned that
\href{https://www.nytimes.com/2020/05/27/us/coronavirus-evictions-renters.html}{an
avalanche of evictions was looming}. The federal government and many
states rushed to ban them temporarily. They placed moratoriums on
mortgage foreclosures to relieve financial pressure on landlords.

But 20 states, including Louisiana, Texas, Colorado and Wisconsin, have
since lifted their restrictions and
\href{https://evictionlab.org/eviction-tracking/}{researchers have
tracked thousands of recent eviction} filings
\href{https://openjusticeok.shinyapps.io/ok-court-tracker/\#section-tool-by-open-justice-oklahoma}{in
places where data is available}.
\href{https://evictionlab.org/covid-policy-scorecard/}{Eviction bans} in
nine other states and at the federal level are set to expire by the end
of the month. All told, Amherst Capital Management anticipates that
nearly
\href{https://www.multihousingnews.com/post/housing-relief-skips-three-fifth-of-u-s-households/}{28
million households} are at risk of being turned out onto the streets
because of job losses tied to the pandemic.

Even in places with ordinances barring evictions, the protections have
been of little help to immigrants like Mr. Avila and Ms. Carillo, who
fear that complaining to the authorities about their landlord could lead
to a consequence worse than homelessness: deportation.

\hypertarget{latest-updates-global-coronavirus-outbreak}{%
\section{\texorpdfstring{\href{https://www.nytimes.com/2020/08/04/world/coronavirus-cases.html?action=click\&pgtype=Article\&state=default\&region=MAIN_CONTENT_1\&context=storylines_live_updates}{Latest
Updates: Global Coronavirus
Outbreak}}{Latest Updates: Global Coronavirus Outbreak}}\label{latest-updates-global-coronavirus-outbreak}}

Updated 2020-08-04T18:55:17.683Z

\begin{itemize}
\tightlist
\item
  \href{https://www.nytimes.com/2020/08/04/world/coronavirus-cases.html?action=click\&pgtype=Article\&state=default\&region=MAIN_CONTENT_1\&context=storylines_live_updates\#link-4825b93}{Public
  and private schools in Maryland and elsewhere are divided over
  in-person instruction.}
\item
  \href{https://www.nytimes.com/2020/08/04/world/coronavirus-cases.html?action=click\&pgtype=Article\&state=default\&region=MAIN_CONTENT_1\&context=storylines_live_updates\#link-4d1eafa8}{N.Y.C.'s
  health commissioner resigns after clashing with the mayor over the
  virus.}
\item
  \href{https://www.nytimes.com/2020/08/04/world/coronavirus-cases.html?action=click\&pgtype=Article\&state=default\&region=MAIN_CONTENT_1\&context=storylines_live_updates\#link-6b644638}{`Long
  days, long nights': Washington prepares for a prolonged fight over
  virus relief.}
\end{itemize}

\href{https://www.nytimes.com/2020/08/04/world/coronavirus-cases.html?action=click\&pgtype=Article\&state=default\&region=MAIN_CONTENT_1\&context=storylines_live_updates}{See
more updates}

More live coverage:
\href{https://www.nytimes.com/live/2020/08/04/business/stock-market-today-coronavirus?action=click\&pgtype=Article\&state=default\&region=MAIN_CONTENT_1\&context=storylines_live_updates}{Markets}

Immigrant and renter advocates in cities across the country say they are
being inundated with complaints about
\href{https://www.nytimes.com/2020/07/23/business/evictions-moratorium-cares-act.html}{landlords
pressuring tenants} to pay rent money. They say landlords use
harassment, illegal fees for late payments or repairs or simply change
the locks as a way to force out vulnerable renters.

``Today for example, I got here maybe an hour ago and I've already had
three calls,'' said Norieliz Dejesus, sounding stressed one morning in
late May. Ms. Dejesus is a program manager with the organization Chelsea
Collaborative, in Chelsea, Mass.,
\href{https://www.nytimes.com/2020/04/25/us/coronavirus-chelsea-massachusetts.html}{a
hub for incoming migrants} from Eastern Europe and Central America.

``I had one tenant whose landlord wants her out by the end of the
month,'' Ms. Dejesus said, ``The tenant explained the new laws. The
landlord acknowledged the new laws and was like, `I don't care, you have
to leave.'''

Image

Jaimy Gonzalez, left, with family members at their home in Chelsea,
Mass.Credit...Ryan Christopher Jones for The New York Times

Landlords argue that they are unfairly being forced to absorb the brunt
of the financial burden of pandemic job losses. ``Why isn't food free?
Why isn't clothing free? Why aren't all the other necessities of life
free, yet shelter is being made free?'' said Sherwin Belkin, a legal
adviser for the Real Estate Board of New York, which represents property
owners.

The government, he said, should provide vouchers to tenants who cannot
pay rent because of the pandemic, and landlords should be allowed to use
the courts to evict those who still do not pay. ``Something is wrong
when a private industry is being asked to take on its back what is
really a public housing emergency,'' he said.

Jaimy Gonzalez, who shares a four-bedroom apartment with her partner,
three children, her partner's father and brother in Chelsea, provided
text messages from her landlord saying she would have to pay a \$35 fee
for each day that her rent was late.

Ms. Gonzalez, who came to the United States from El Salvador 18 years
ago, had worked as a babysitter before the pandemic. Her partner and his
brother were movers. She said that all the adults in her household have
been out of work since April.

She explained to her landlord that she had no money. Like many other
families who have found themselves suddenly out of work, hers has been
surviving on donated food since the pandemic hit. She learned of a
program funded by the City Council to help people pay rent, but she
could not even afford to print the application in order to apply.

``I was crying morning, afternoon and night thinking about how we were
going to pay the rent, what we were going to do?'' she said.

She held her breath and sent a message to her landlord explaining that
she was doing everything she could to come up with the money, but he
never responded. She said she wakes up every day wondering if he's going
to show up at her door demanding that they leave.

In Washington, where an eviction moratorium is still in place, the
Attorney General's office has collected 165 complaints of illegal
evictions and late fees and sent 38 cease-and-desist letters to
landlords since April 24, when it began keeping data. In one apartment
building in the district, landlords posted signs in Spanish announcing
that tenants who missed rent payments would be evicted immediately,
according to Jennifer Berger, who heads the office's social justice
division.

Ms. Berger said immigrants have lodged the majority of complaints her
office has received. She suspects far more have lost their homes than
those who have come forward to complain.

``There's inherent coercion within the immigrant community because they
live in fear of being deported, so they're afraid to speak out,'' she
said. ``My gut instinct is that there are people who have experienced
this and certainly didn't report it.''

That was the case for Enriqueta, who asked to be identified by her first
name because she came to the United States illegally from Mexico. She
and her husband were working as house cleaners in Austin, Texas, but
lost their jobs in March. April 1 came and went. Within a few days, her
landlord began knocking on the door of her apartment and demanding the
rent.

\href{https://www.nytimes.com/news-event/coronavirus?action=click\&pgtype=Article\&state=default\&region=MAIN_CONTENT_3\&context=storylines_faq}{}

\hypertarget{the-coronavirus-outbreak-}{%
\subsubsection{The Coronavirus Outbreak
›}\label{the-coronavirus-outbreak-}}

\hypertarget{frequently-asked-questions}{%
\paragraph{Frequently Asked
Questions}\label{frequently-asked-questions}}

Updated August 4, 2020

\begin{itemize}
\item ~
  \hypertarget{i-have-antibodies-am-i-now-immune}{%
  \paragraph{I have antibodies. Am I now
  immune?}\label{i-have-antibodies-am-i-now-immune}}

  \begin{itemize}
  \tightlist
  \item
    As of right
    now,\href{https://www.nytimes.com/2020/07/22/health/covid-antibodies-herd-immunity.html?action=click\&pgtype=Article\&state=default\&region=MAIN_CONTENT_3\&context=storylines_faq}{that
    seems likely, for at least several months.} There have been
    frightening accounts of people suffering what seems to be a second
    bout of Covid-19. But experts say these patients may have a
    drawn-out course of infection, with the virus taking a slow toll
    weeks to months after initial exposure. People infected with the
    coronavirus typically
    \href{https://www.nature.com/articles/s41586-020-2456-9}{produce}
    immune molecules called antibodies, which are
    \href{https://www.nytimes.com/2020/05/07/health/coronavirus-antibody-prevalence.html?action=click\&pgtype=Article\&state=default\&region=MAIN_CONTENT_3\&context=storylines_faq}{protective
    proteins made in response to an
    infection}\href{https://www.nytimes.com/2020/05/07/health/coronavirus-antibody-prevalence.html?action=click\&pgtype=Article\&state=default\&region=MAIN_CONTENT_3\&context=storylines_faq}{.
    These antibodies may} last in the body
    \href{https://www.nature.com/articles/s41591-020-0965-6}{only two to
    three months}, which may seem worrisome, but that's perfectly normal
    after an acute infection subsides, said Dr. Michael Mina, an
    immunologist at Harvard University. It may be possible to get the
    coronavirus again, but it's highly unlikely that it would be
    possible in a short window of time from initial infection or make
    people sicker the second time.
  \end{itemize}
\item ~
  \hypertarget{im-a-small-business-owner-can-i-get-relief}{%
  \paragraph{I'm a small-business owner. Can I get
  relief?}\label{im-a-small-business-owner-can-i-get-relief}}

  \begin{itemize}
  \tightlist
  \item
    The
    \href{https://www.nytimes.com/article/small-business-loans-stimulus-grants-freelancers-coronavirus.html?action=click\&pgtype=Article\&state=default\&region=MAIN_CONTENT_3\&context=storylines_faq}{stimulus
    bills enacted in March} offer help for the millions of American
    small businesses. Those eligible for aid are businesses and
    nonprofit organizations with fewer than 500 workers, including sole
    proprietorships, independent contractors and freelancers. Some
    larger companies in some industries are also eligible. The help
    being offered, which is being managed by the Small Business
    Administration, includes the Paycheck Protection Program and the
    Economic Injury Disaster Loan program. But lots of folks have
    \href{https://www.nytimes.com/interactive/2020/05/07/business/small-business-loans-coronavirus.html?action=click\&pgtype=Article\&state=default\&region=MAIN_CONTENT_3\&context=storylines_faq}{not
    yet seen payouts.} Even those who have received help are confused:
    The rules are draconian, and some are stuck sitting on
    \href{https://www.nytimes.com/2020/05/02/business/economy/loans-coronavirus-small-business.html?action=click\&pgtype=Article\&state=default\&region=MAIN_CONTENT_3\&context=storylines_faq}{money
    they don't know how to use.} Many small-business owners are getting
    less than they expected or
    \href{https://www.nytimes.com/2020/06/10/business/Small-business-loans-ppp.html?action=click\&pgtype=Article\&state=default\&region=MAIN_CONTENT_3\&context=storylines_faq}{not
    hearing anything at all.}
  \end{itemize}
\item ~
  \hypertarget{what-are-my-rights-if-i-am-worried-about-going-back-to-work}{%
  \paragraph{What are my rights if I am worried about going back to
  work?}\label{what-are-my-rights-if-i-am-worried-about-going-back-to-work}}

  \begin{itemize}
  \tightlist
  \item
    Employers have to provide
    \href{https://www.osha.gov/SLTC/covid-19/standards.html}{a safe
    workplace} with policies that protect everyone equally.
    \href{https://www.nytimes.com/article/coronavirus-money-unemployment.html?action=click\&pgtype=Article\&state=default\&region=MAIN_CONTENT_3\&context=storylines_faq}{And
    if one of your co-workers tests positive for the coronavirus, the
    C.D.C.} has said that
    \href{https://www.cdc.gov/coronavirus/2019-ncov/community/guidance-business-response.html}{employers
    should tell their employees} -\/- without giving you the sick
    employee's name -\/- that they may have been exposed to the virus.
  \end{itemize}
\item ~
  \hypertarget{should-i-refinance-my-mortgage}{%
  \paragraph{Should I refinance my
  mortgage?}\label{should-i-refinance-my-mortgage}}

  \begin{itemize}
  \tightlist
  \item
    \href{https://www.nytimes.com/article/coronavirus-money-unemployment.html?action=click\&pgtype=Article\&state=default\&region=MAIN_CONTENT_3\&context=storylines_faq}{It
    could be a good idea,} because mortgage rates have
    \href{https://www.nytimes.com/2020/07/16/business/mortgage-rates-below-3-percent.html?action=click\&pgtype=Article\&state=default\&region=MAIN_CONTENT_3\&context=storylines_faq}{never
    been lower.} Refinancing requests have pushed mortgage applications
    to some of the highest levels since 2008, so be prepared to get in
    line. But defaults are also up, so if you're thinking about buying a
    home, be aware that some lenders have tightened their standards.
  \end{itemize}
\item ~
  \hypertarget{what-is-school-going-to-look-like-in-september}{%
  \paragraph{What is school going to look like in
  September?}\label{what-is-school-going-to-look-like-in-september}}

  \begin{itemize}
  \tightlist
  \item
    It is unlikely that many schools will return to a normal schedule
    this fall, requiring the grind of
    \href{https://www.nytimes.com/2020/06/05/us/coronavirus-education-lost-learning.html?action=click\&pgtype=Article\&state=default\&region=MAIN_CONTENT_3\&context=storylines_faq}{online
    learning},
    \href{https://www.nytimes.com/2020/05/29/us/coronavirus-child-care-centers.html?action=click\&pgtype=Article\&state=default\&region=MAIN_CONTENT_3\&context=storylines_faq}{makeshift
    child care} and
    \href{https://www.nytimes.com/2020/06/03/business/economy/coronavirus-working-women.html?action=click\&pgtype=Article\&state=default\&region=MAIN_CONTENT_3\&context=storylines_faq}{stunted
    workdays} to continue. California's two largest public school
    districts --- Los Angeles and San Diego --- said on July 13, that
    \href{https://www.nytimes.com/2020/07/13/us/lausd-san-diego-school-reopening.html?action=click\&pgtype=Article\&state=default\&region=MAIN_CONTENT_3\&context=storylines_faq}{instruction
    will be remote-only in the fall}, citing concerns that surging
    coronavirus infections in their areas pose too dire a risk for
    students and teachers. Together, the two districts enroll some
    825,000 students. They are the largest in the country so far to
    abandon plans for even a partial physical return to classrooms when
    they reopen in August. For other districts, the solution won't be an
    all-or-nothing approach.
    \href{https://bioethics.jhu.edu/research-and-outreach/projects/eschool-initiative/school-policy-tracker/}{Many
    systems}, including the nation's largest, New York City, are
    devising
    \href{https://www.nytimes.com/2020/06/26/us/coronavirus-schools-reopen-fall.html?action=click\&pgtype=Article\&state=default\&region=MAIN_CONTENT_3\&context=storylines_faq}{hybrid
    plans} that involve spending some days in classrooms and other days
    online. There's no national policy on this yet, so check with your
    municipal school system regularly to see what is happening in your
    community.
  \end{itemize}
\end{itemize}

A \$300 bill for water overuse appeared, which she said she found
confusing because the apartment had no dishwasher or washing machine.
She chose not to fight back, fearing that the authorities might try to
separate her from her American-born sons, who are 6 and 8 years old, and
deport her.

Image

Jaimy Gonzalez at her home in Chelsea.~Credit...Ryan Christopher Jones
for The New York Times

Instead, they fled to the home of one of her husband's cousins, who said
they could live temporarily in his living room.
\href{https://www.nytimes.com/2016/02/22/books/evicted-book-review-matthew-desmond.html}{As
is often the case}, the eviction was not an isolated incident, but a
catalyst that she said sent the rest of her family's life into a
tailspin.

Her husband started drinking excessively and then left her. During the
two and a half months she and her sons slept on an air mattress, they
rarely slept through the night, awakened each time someone came into the
house or went to the sink for a glass of water. The boys stopped
attending online school classes because the home where they were staying
did not have internet access. She hardly had time to look for a new job;
she spent all day trying to keep her sons from upsetting their new
temporary landlords.

Even if she found the money, because of her legal status, she did not
have a form of identification she could use to rent another apartment.

``It was a nightmare,'' she said.

Austin city officials learned of their situation earlier this month
through her sons' school and helped them move last week into a bedroom
they are now subletting in a different apartment without a formal lease.
But with no job and businesses in Texas shuttering again because of
coronavirus outbreaks, she said she fears a second eviction could be
coming if she can't keep up with her \$600 rent payments.

Ken Bailey, the fire rescue chief of Travis County, Texas, which
includes Austin, discovered that the handshake deals many unauthorized
immigrants use instead of a formal lease made it difficult for them to
seek help paying their rent during the pandemic.

When Texas began to shut down, Mr. Bailey's department sent
Spanish-language surveys to homes in his jurisdiction. There were a
flood of responses. Most people listed rent payments listed as their
biggest concern.

``A lot of them didn't qualify for governmental assistance or even some
of the nonprofit assistance because they don't have a rental agreement;
they didn't have utilities in their names,'' Mr. Bailey said.

Image

Wilfredo Avila walks into the attic of a friend's home, which is his
temporary living space after he and his girlfriend were evicted from
their previous home during the pandemic.Credit...Ryan Christopher Jones
for The New York Times

Mr. Avila, who slept in a park after being booted from his apartment in
Washington, found a reprieve that feels only temporary. Through a
friend, he and Ms. Carillo met a man who agreed to rent them his attic
for \$800 a month without a lease. The couple has lived there for three
months, but only paid rent twice. So far, their new landlord has been
understanding.

Mr. Avila is eager to get back to work but the auto-body shop does not
have enough business. His only income has come from occasional one-off
jobs cutting grass or cleaning solar panels.

The couple spends most of their time in the long, narrow room with blank
white walls that has become their new home. Their clothes are in messy
piles in a corner, as if folded by someone who used to be tidy but has
given up. Their tooth brushes and toilet paper rest on the back of a
floor cushion they found on the side of a road, their only piece of
furniture except for a bed donated by their new landlord.

The conversation they started in the park about what they are going to
do next has not ended. Now, they talk about how long they will be
allowed to stay in their new home, before they have to set out walking
again to look for the next one.

Advertisement

\protect\hyperlink{after-bottom}{Continue reading the main story}

\hypertarget{site-index}{%
\subsection{Site Index}\label{site-index}}

\hypertarget{site-information-navigation}{%
\subsection{Site Information
Navigation}\label{site-information-navigation}}

\begin{itemize}
\tightlist
\item
  \href{https://help.nytimes.com/hc/en-us/articles/115014792127-Copyright-notice}{©~2020~The
  New York Times Company}
\end{itemize}

\begin{itemize}
\tightlist
\item
  \href{https://www.nytco.com/}{NYTCo}
\item
  \href{https://help.nytimes.com/hc/en-us/articles/115015385887-Contact-Us}{Contact
  Us}
\item
  \href{https://www.nytco.com/careers/}{Work with us}
\item
  \href{https://nytmediakit.com/}{Advertise}
\item
  \href{http://www.tbrandstudio.com/}{T Brand Studio}
\item
  \href{https://www.nytimes.com/privacy/cookie-policy\#how-do-i-manage-trackers}{Your
  Ad Choices}
\item
  \href{https://www.nytimes.com/privacy}{Privacy}
\item
  \href{https://help.nytimes.com/hc/en-us/articles/115014893428-Terms-of-service}{Terms
  of Service}
\item
  \href{https://help.nytimes.com/hc/en-us/articles/115014893968-Terms-of-sale}{Terms
  of Sale}
\item
  \href{https://spiderbites.nytimes.com}{Site Map}
\item
  \href{https://help.nytimes.com/hc/en-us}{Help}
\item
  \href{https://www.nytimes.com/subscription?campaignId=37WXW}{Subscriptions}
\end{itemize}
