Sections

SEARCH

\protect\hyperlink{site-content}{Skip to
content}\protect\hyperlink{site-index}{Skip to site index}

\href{https://www.nytimes.com/section/world/americas}{Americas}

\href{https://myaccount.nytimes.com/auth/login?response_type=cookie\&client_id=vi}{}

\href{https://www.nytimes.com/section/todayspaper}{Today's Paper}

\href{/section/world/americas}{Americas}\textbar{}Caribbean, Struggling
in the Pandemic, Braces for Hurricane Season

\url{https://nyti.ms/2YXYxjF}

\begin{itemize}
\item
\item
\item
\item
\item
\end{itemize}

\href{https://www.nytimes.com/news-event/coronavirus?action=click\&pgtype=Article\&state=default\&region=TOP_BANNER\&context=storylines_menu}{The
Coronavirus Outbreak}

\begin{itemize}
\tightlist
\item
  live\href{https://www.nytimes.com/2020/08/01/world/coronavirus-covid-19.html?action=click\&pgtype=Article\&state=default\&region=TOP_BANNER\&context=storylines_menu}{Latest
  Updates}
\item
  \href{https://www.nytimes.com/interactive/2020/us/coronavirus-us-cases.html?action=click\&pgtype=Article\&state=default\&region=TOP_BANNER\&context=storylines_menu}{Maps
  and Cases}
\item
  \href{https://www.nytimes.com/interactive/2020/science/coronavirus-vaccine-tracker.html?action=click\&pgtype=Article\&state=default\&region=TOP_BANNER\&context=storylines_menu}{Vaccine
  Tracker}
\item
  \href{https://www.nytimes.com/interactive/2020/07/29/us/schools-reopening-coronavirus.html?action=click\&pgtype=Article\&state=default\&region=TOP_BANNER\&context=storylines_menu}{What
  School May Look Like}
\item
  \href{https://www.nytimes.com/live/2020/07/31/business/stock-market-today-coronavirus?action=click\&pgtype=Article\&state=default\&region=TOP_BANNER\&context=storylines_menu}{Economy}
\end{itemize}

Advertisement

\protect\hyperlink{after-top}{Continue reading the main story}

Supported by

\protect\hyperlink{after-sponsor}{Continue reading the main story}

\hypertarget{caribbean-struggling-in-the-pandemic-braces-for-hurricane-season}{%
\section{Caribbean, Struggling in the Pandemic, Braces for Hurricane
Season}\label{caribbean-struggling-in-the-pandemic-braces-for-hurricane-season}}

Scientists predict this hurricane season will be severe. But countries
in the region, with economies burdened by the pandemic and the
devastation of past hurricanes, have not been able to fully prepare.

\includegraphics{https://static01.nyt.com/images/2020/07/05/world/00virus-hurricanes4/merlin_160315398_9e70abf9-b513-401e-9d80-971cd7940bed-articleLarge.jpg?quality=75\&auto=webp\&disable=upscale}

By \href{https://www.nytimes.com/by/kirk-semple}{Kirk Semple}

\begin{itemize}
\item
  Published July 4, 2020Updated July 26, 2020
\item
  \begin{itemize}
  \item
  \item
  \item
  \item
  \item
  \end{itemize}
\end{itemize}

\href{https://www.nytimes.com/es/2020/07/08/espanol/america-latina/huracanes-caribe-coronavirus.html}{Leer
en español}

MEXICO CITY --- Houses with no roofs. Neighborhoods lacking electricity.
Residents who fled still in exile.

Ten months after Hurricane Dorian
\href{https://www.nytimes.com/2019/09/06/world/americas/hurricane-dorian-bahamas.html}{pulverized
the northern Bahamas}, those islands are still struggling to recover,
even as this year's
\href{https://www.nytimes.com/2020/07/26/us/hurricane-douglas-hawaii.html}{hurricane
season} begins. But rebuilding,
\href{https://www.nytimes.com/2019/10/07/world/americas/hurricane-irma-saint-martin.html}{always
a slow process}, has been slowed even further this year by a disaster of
another sort: the coronavirus pandemic.

``That brought rebuilding efforts to a complete halt,'' said Stafford
Symonette, an evangelical pastor whose house on Great Abaco Island was
\href{https://www.nytimes.com/2019/09/06/world/americas/bahamas-abaco-hurricane-damage.html}{severely
damaged} during the hurricane --- and remains that way.

``You still have a lot of people in tents and temporary shelters,'' he
said.

The Bahamas --- like other hurricane-prone countries in the Caribbean
and North Atlantic --- find themselves at the dramatic convergence of a
devastating pandemic and an Atlantic
\href{https://www.nytimes.com/interactive/2020/07/25/us/hurricane-hanna-tracker-map.html}{hurricane}
season that is expected to be more active than normal.

\includegraphics{https://static01.nyt.com/images/2020/07/02/world/00virus-hurricanes3/merlin_160401696_71c0c87a-5142-4e4f-8655-cd2fbec0eb94-articleLarge.jpg?quality=75\&auto=webp\&disable=upscale}

The pandemic has profoundly affected all aspects of hurricane
preparedness and response, and left nations even more vulnerable to the
impacts of storms.

It has complicated rebuilding efforts from past hurricane seasons. It
has crippled national economies in the region,
\href{https://www.nytimes.com/2020/04/02/world/americas/virus-tourism-caribbean.html}{many
of which depend heavily on tourism}. It has forced the reallocation of
diminished government resources --- money and personnel that otherwise
might have been used for hurricane-related work --- to deal with the
public health crisis.

And it has meant that, in the event of a major storm, evacuation centers
and shelters could now turn into dangerous vectors of coronavirus
contagion, driving governments and relief agencies to figure out new
protocols to keep evacuees safe.

These mounting challenges have overwhelmed many of the region's
governments and relief agencies, which are scrambling to prepare for the
next big storm.

``Are we prepared for this hurricane season?'' said Ronald Sanders,
ambassador of Antigua and Barbuda to the United States and to the
Organization of American States. ``The answer is: no. And I don't care
who tells you we are. We haven't been able to dedicate any funds toward
hurricane preparedness this year.''

\hypertarget{latest-updates-global-coronavirus-outbreak}{%
\section{\texorpdfstring{\href{https://www.nytimes.com/2020/08/01/world/coronavirus-covid-19.html?action=click\&pgtype=Article\&state=default\&region=MAIN_CONTENT_1\&context=storylines_live_updates}{Latest
Updates: Global Coronavirus
Outbreak}}{Latest Updates: Global Coronavirus Outbreak}}\label{latest-updates-global-coronavirus-outbreak}}

Updated 2020-08-02T06:58:18.835Z

\begin{itemize}
\tightlist
\item
  \href{https://www.nytimes.com/2020/08/01/world/coronavirus-covid-19.html?action=click\&pgtype=Article\&state=default\&region=MAIN_CONTENT_1\&context=storylines_live_updates\#link-34047410}{The
  U.S. reels as July cases more than double the total of any other
  month.}
\item
  \href{https://www.nytimes.com/2020/08/01/world/coronavirus-covid-19.html?action=click\&pgtype=Article\&state=default\&region=MAIN_CONTENT_1\&context=storylines_live_updates\#link-780ec966}{Top
  U.S. officials work to break an impasse over the federal jobless
  benefit.}
\item
  \href{https://www.nytimes.com/2020/08/01/world/coronavirus-covid-19.html?action=click\&pgtype=Article\&state=default\&region=MAIN_CONTENT_1\&context=storylines_live_updates\#link-2bc8948}{Its
  outbreak untamed, Melbourne goes into even greater lockdown.}
\end{itemize}

\href{https://www.nytimes.com/2020/08/01/world/coronavirus-covid-19.html?action=click\&pgtype=Article\&state=default\&region=MAIN_CONTENT_1\&context=storylines_live_updates}{See
more updates}

More live coverage:
\href{https://www.nytimes.com/live/2020/07/31/business/stock-market-today-coronavirus?action=click\&pgtype=Article\&state=default\&region=MAIN_CONTENT_1\&context=storylines_live_updates}{Markets}

``These countries are struggling and have been for some time,'' he
continued. ``The reality is that we are in dire straits.''

Image

Health workers conducting coronavirus tests in San Cristobal, Dominican
Republic, in June.~Credit...Erika Santelices/Agence France-Presse ---
Getty Images

Weather scientists from the American government
\href{https://www.nytimes.com/2020/05/21/climate/hurricane-season-2020-noaa.html?searchResultPosition=2}{have
predicted} that during this Atlantic storm season, which began on June 1
and runs through Nov. 30, there will be as many as 19 named storms, with
as many as six growing to major hurricane status. An average hurricane
season has 12 named storms and three major hurricanes.

The season has gotten off to a quick start, with four named storms so
far.

The region started the season at a severe economic disadvantage. The
pandemic crushed the tourism industry, a main economic engine for much
of the Caribbean. Hotels were shuttered, cruise ships remained docked,
airplanes were grounded. The Caribbean Development Bank estimated that
regional economic activity may contract by as much as 20 percent this
year.

Mr. Sanders said he worried about what would happen should the region
suffer a repeat of 2017, when
\href{https://www.nytimes.com/2017/09/08/world/americas/caribbean-islands-hurricane-irma-st-martin-barbuda-anguilla.html}{several
major hurricanes}
\href{https://www.nytimes.com/2017/09/19/world/americas/hurricane-maria-caribbean.html}{plowed
through the Caribbean}.

``If that were to happen again this year,'' he said, ``well, I think
these economies will go into complete collapse.''

Image

There have already been four named storms this season. Tropical Storm
Cristobal in the town of Tecoh, near Merida in Yucatan State,
Mexico.Credit...Luis Perez/Agence France-Presse --- Getty Images

The pandemic has also presented a range of public health challenges for
governments and relief groups preparing for hurricanes, including the
need to ensure adequate social distancing during evacuations and in
shelters, and a sufficient supply of personal protective gear for
emergency workers and evacuees.

Health officials are also trying to stockpile medicine and other
supplies and prepare for possible coronavirus outbreaks among evacuees.

``Without a doubt, once we have a natural hazard such as a hurricane,
there will be a greater rate of infection, particularly with respect to
Covid-19, among other diseases that could arise,'' Dr. Laura-Lee
Boodram, an official with the Caribbean Public Health Agency, warned
during a recent panel discussion organized by the Caribbean Tourism
Organization.

The Bahamas has been at a particular disadvantage in its efforts to get
out ahead of this year's hurricane threat.

The coronavirus pandemic swept into the region only a few months after
Dorian,
\href{https://www.nytimes.com/2019/09/02/world/americas/hurricane-dorian-bahamas.html}{one
of the most powerful Atlantic hurricanes on record}, made landfall on
Sept. 1, 2019,
\href{https://www.nytimes.com/2019/09/08/world/americas/bahamas-dead-dorian.html}{killing
scores of people} in the Abaco Islands and Grand Bahama Island,
destroying thousands of structures and
\href{https://www.nytimes.com/2019/09/07/world/americas/bahamas-hurricane-dorian-relief.html}{causing
billions of dollars in damage}.

\href{https://www.nytimes.com/news-event/coronavirus?action=click\&pgtype=Article\&state=default\&region=MAIN_CONTENT_3\&context=storylines_faq}{}

\hypertarget{the-coronavirus-outbreak-}{%
\subsubsection{The Coronavirus Outbreak
›}\label{the-coronavirus-outbreak-}}

\hypertarget{frequently-asked-questions}{%
\paragraph{Frequently Asked
Questions}\label{frequently-asked-questions}}

Updated July 27, 2020

\begin{itemize}
\item ~
  \hypertarget{should-i-refinance-my-mortgage}{%
  \paragraph{Should I refinance my
  mortgage?}\label{should-i-refinance-my-mortgage}}

  \begin{itemize}
  \tightlist
  \item
    \href{https://www.nytimes.com/article/coronavirus-money-unemployment.html?action=click\&pgtype=Article\&state=default\&region=MAIN_CONTENT_3\&context=storylines_faq}{It
    could be a good idea,} because mortgage rates have
    \href{https://www.nytimes.com/2020/07/16/business/mortgage-rates-below-3-percent.html?action=click\&pgtype=Article\&state=default\&region=MAIN_CONTENT_3\&context=storylines_faq}{never
    been lower.} Refinancing requests have pushed mortgage applications
    to some of the highest levels since 2008, so be prepared to get in
    line. But defaults are also up, so if you're thinking about buying a
    home, be aware that some lenders have tightened their standards.
  \end{itemize}
\item ~
  \hypertarget{what-is-school-going-to-look-like-in-september}{%
  \paragraph{What is school going to look like in
  September?}\label{what-is-school-going-to-look-like-in-september}}

  \begin{itemize}
  \tightlist
  \item
    It is unlikely that many schools will return to a normal schedule
    this fall, requiring the grind of
    \href{https://www.nytimes.com/2020/06/05/us/coronavirus-education-lost-learning.html?action=click\&pgtype=Article\&state=default\&region=MAIN_CONTENT_3\&context=storylines_faq}{online
    learning},
    \href{https://www.nytimes.com/2020/05/29/us/coronavirus-child-care-centers.html?action=click\&pgtype=Article\&state=default\&region=MAIN_CONTENT_3\&context=storylines_faq}{makeshift
    child care} and
    \href{https://www.nytimes.com/2020/06/03/business/economy/coronavirus-working-women.html?action=click\&pgtype=Article\&state=default\&region=MAIN_CONTENT_3\&context=storylines_faq}{stunted
    workdays} to continue. California's two largest public school
    districts --- Los Angeles and San Diego --- said on July 13, that
    \href{https://www.nytimes.com/2020/07/13/us/lausd-san-diego-school-reopening.html?action=click\&pgtype=Article\&state=default\&region=MAIN_CONTENT_3\&context=storylines_faq}{instruction
    will be remote-only in the fall}, citing concerns that surging
    coronavirus infections in their areas pose too dire a risk for
    students and teachers. Together, the two districts enroll some
    825,000 students. They are the largest in the country so far to
    abandon plans for even a partial physical return to classrooms when
    they reopen in August. For other districts, the solution won't be an
    all-or-nothing approach.
    \href{https://bioethics.jhu.edu/research-and-outreach/projects/eschool-initiative/school-policy-tracker/}{Many
    systems}, including the nation's largest, New York City, are
    devising
    \href{https://www.nytimes.com/2020/06/26/us/coronavirus-schools-reopen-fall.html?action=click\&pgtype=Article\&state=default\&region=MAIN_CONTENT_3\&context=storylines_faq}{hybrid
    plans} that involve spending some days in classrooms and other days
    online. There's no national policy on this yet, so check with your
    municipal school system regularly to see what is happening in your
    community.
  \end{itemize}
\item ~
  \hypertarget{is-the-coronavirus-airborne}{%
  \paragraph{Is the coronavirus
  airborne?}\label{is-the-coronavirus-airborne}}

  \begin{itemize}
  \tightlist
  \item
    The coronavirus
    \href{https://www.nytimes.com/2020/07/04/health/239-experts-with-one-big-claim-the-coronavirus-is-airborne.html?action=click\&pgtype=Article\&state=default\&region=MAIN_CONTENT_3\&context=storylines_faq}{can
    stay aloft for hours in tiny droplets in stagnant air}, infecting
    people as they inhale, mounting scientific evidence suggests. This
    risk is highest in crowded indoor spaces with poor ventilation, and
    may help explain super-spreading events reported in meatpacking
    plants, churches and restaurants.
    \href{https://www.nytimes.com/2020/07/06/health/coronavirus-airborne-aerosols.html?action=click\&pgtype=Article\&state=default\&region=MAIN_CONTENT_3\&context=storylines_faq}{It's
    unclear how often the virus is spread} via these tiny droplets, or
    aerosols, compared with larger droplets that are expelled when a
    sick person coughs or sneezes, or transmitted through contact with
    contaminated surfaces, said Linsey Marr, an aerosol expert at
    Virginia Tech. Aerosols are released even when a person without
    symptoms exhales, talks or sings, according to Dr. Marr and more
    than 200 other experts, who
    \href{https://academic.oup.com/cid/article/doi/10.1093/cid/ciaa939/5867798}{have
    outlined the evidence in an open letter to the World Health
    Organization}.
  \end{itemize}
\item ~
  \hypertarget{what-are-the-symptoms-of-coronavirus}{%
  \paragraph{What are the symptoms of
  coronavirus?}\label{what-are-the-symptoms-of-coronavirus}}

  \begin{itemize}
  \tightlist
  \item
    Common symptoms
    \href{https://www.nytimes.com/article/symptoms-coronavirus.html?action=click\&pgtype=Article\&state=default\&region=MAIN_CONTENT_3\&context=storylines_faq}{include
    fever, a dry cough, fatigue and difficulty breathing or shortness of
    breath.} Some of these symptoms overlap with those of the flu,
    making detection difficult, but runny noses and stuffy sinuses are
    less common.
    \href{https://www.nytimes.com/2020/04/27/health/coronavirus-symptoms-cdc.html?action=click\&pgtype=Article\&state=default\&region=MAIN_CONTENT_3\&context=storylines_faq}{The
    C.D.C. has also} added chills, muscle pain, sore throat, headache
    and a new loss of the sense of taste or smell as symptoms to look
    out for. Most people fall ill five to seven days after exposure, but
    symptoms may appear in as few as two days or as many as 14 days.
  \end{itemize}
\item ~
  \hypertarget{does-asymptomatic-transmission-of-covid-19-happen}{%
  \paragraph{Does asymptomatic transmission of Covid-19
  happen?}\label{does-asymptomatic-transmission-of-covid-19-happen}}

  \begin{itemize}
  \tightlist
  \item
    So far, the evidence seems to show it does. A widely cited
    \href{https://www.nature.com/articles/s41591-020-0869-5}{paper}
    published in April suggests that people are most infectious about
    two days before the onset of coronavirus symptoms and estimated that
    44 percent of new infections were a result of transmission from
    people who were not yet showing symptoms. Recently, a top expert at
    the World Health Organization stated that transmission of the
    coronavirus by people who did not have symptoms was ``very rare,''
    \href{https://www.nytimes.com/2020/06/09/world/coronavirus-updates.html?action=click\&pgtype=Article\&state=default\&region=MAIN_CONTENT_3\&context=storylines_faq\#link-1f302e21}{but
    she later walked back that statement.}
  \end{itemize}
\end{itemize}

Recovery efforts were fully underway by the time the country recorded
its first coronavirus case on March 16. But less than two weeks later,
with the number of infections mounting, the government had closed the
nation's borders and had begun imposing a series of restrictions on
movement, including curfews, 24-hour lockdowns and a ban on travel
between the archipelago's islands.

Image

A volunteer giving out groceries at a food distribution location in
Nassau, Bahamas, in April.~Credit...Melissa Alcena/Bloomberg

While the measures helped curb the spread of the virus --- the Bahamas
has only 104 confirmed cases so far --- they slowed recovery, delayed
preparations for the new hurricane season and, combined with the global
halt of the tourism industry, further plunged the country into economic
distress.

The Bahamian government said it expects to incur a \$1.3 billion deficit
this fiscal year, equivalent to about 11.6 percent of gross domestic
product and the largest in the history of the Bahamas.

``Any significant storm damage this year would put us in a very serious
spot in terms of our fiscal projections,'' Peter Turnquest, the Bahamas'
deputy prime minister and finance minister, said in an interview this
week.

Among emergency officials' greatest concerns as the hurricane season
unfolds is the insufficient number of storm shelters in parts of the
Bahamas. Many that were damaged during Dorian have yet to be repaired.

The International Organization for Migration said in a report in May
that only 13 of the 25 official shelters on the Abaco Islands and Grand
Bahama were ``usable'' and had only enough capacity for about 2 percent
of the population.

``We certainly pray that there are no storms this year,'' Mr. Turnquest
said.

Image

Customers social distancing in a line for a bank in Nassau in late
April. Many businesses on the island were hurt by the sudden halt in
global tourism due to the coronavirus pandemic.Credit...Melissa
Alcena/Bloomberg

Adding to the uncertainty, the government is now poised to reopen the
country's borders to international visitors. The decision has sowed
anxiety among many Bahamians who fear that it might spur a second wave
of infections across the islands, triggering more lockdowns and border
closures, and further complicating hurricane preparedness and response.

``People are nervous,'' said Steve Pedican, whose house on Great Abaco
Island was severely damaged in the hurricane. ``People don't know what
to expect now.''

When asked what might happen should a major hurricane make landfall on
Great Abaco in the coming months, Mr. Symonette, the evangelical pastor,
went silent for a while, mulling the implications.

``I don't know how we would cope with it if we get another one this
year,'' he finally said. ``Praise God, that he be merciful to us.''

Rachel Knowles contributed reporting from Nassau, Bahamas.

Advertisement

\protect\hyperlink{after-bottom}{Continue reading the main story}

\hypertarget{site-index}{%
\subsection{Site Index}\label{site-index}}

\hypertarget{site-information-navigation}{%
\subsection{Site Information
Navigation}\label{site-information-navigation}}

\begin{itemize}
\tightlist
\item
  \href{https://help.nytimes.com/hc/en-us/articles/115014792127-Copyright-notice}{©~2020~The
  New York Times Company}
\end{itemize}

\begin{itemize}
\tightlist
\item
  \href{https://www.nytco.com/}{NYTCo}
\item
  \href{https://help.nytimes.com/hc/en-us/articles/115015385887-Contact-Us}{Contact
  Us}
\item
  \href{https://www.nytco.com/careers/}{Work with us}
\item
  \href{https://nytmediakit.com/}{Advertise}
\item
  \href{http://www.tbrandstudio.com/}{T Brand Studio}
\item
  \href{https://www.nytimes.com/privacy/cookie-policy\#how-do-i-manage-trackers}{Your
  Ad Choices}
\item
  \href{https://www.nytimes.com/privacy}{Privacy}
\item
  \href{https://help.nytimes.com/hc/en-us/articles/115014893428-Terms-of-service}{Terms
  of Service}
\item
  \href{https://help.nytimes.com/hc/en-us/articles/115014893968-Terms-of-sale}{Terms
  of Sale}
\item
  \href{https://spiderbites.nytimes.com}{Site Map}
\item
  \href{https://help.nytimes.com/hc/en-us}{Help}
\item
  \href{https://www.nytimes.com/subscription?campaignId=37WXW}{Subscriptions}
\end{itemize}
