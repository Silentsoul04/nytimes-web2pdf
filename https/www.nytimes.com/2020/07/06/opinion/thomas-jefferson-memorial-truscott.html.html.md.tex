Sections

SEARCH

\protect\hyperlink{site-content}{Skip to
content}\protect\hyperlink{site-index}{Skip to site index}

\href{https://myaccount.nytimes.com/auth/login?response_type=cookie\&client_id=vi}{}

\href{https://www.nytimes.com/section/todayspaper}{Today's Paper}

\href{/section/opinion}{Opinion}\textbar{}I'm a Direct Descendant of
Thomas Jefferson. Take Down His Memorial.

\href{https://nyti.ms/38wl2iP}{https://nyti.ms/38wl2iP}

\begin{itemize}
\item
\item
\item
\item
\item
\item
\end{itemize}

Advertisement

\protect\hyperlink{after-top}{Continue reading the main story}

\href{/section/opinion}{Opinion}

Supported by

\protect\hyperlink{after-sponsor}{Continue reading the main story}

\hypertarget{im-a-direct-descendant-of-thomas-jefferson-take-down-his-memorial}{%
\section{I'm a Direct Descendant of Thomas Jefferson. Take Down His
Memorial.}\label{im-a-direct-descendant-of-thomas-jefferson-take-down-his-memorial}}

Monticello is shrine enough for a man who wrote that ``all men are
created equal'' and yet never did much to make those words come true.

By Lucian K. Truscott IV

Mr. Truscott is a journalist.

\begin{itemize}
\item
  July 6, 2020
\item
  \begin{itemize}
  \item
  \item
  \item
  \item
  \item
  \item
  \end{itemize}
\end{itemize}

\includegraphics{https://static01.nyt.com/images/2020/07/06/opinion/06truscott/06truscott-articleLarge.jpg?quality=75\&auto=webp\&disable=upscale}

When my brother Frank and I were boys visiting our grandparents at their
home in Virginia, just outside of Washington, we used to heckle our
grandmother until she would drive us into town so we could visit the
Smithsonian museum on the Mall.

As we crossed the Potomac River on the 14th Street Bridge, the Jefferson
Memorial stood off to the left, overlooking the Tidal Basin. I don't
remember ever visiting the memorial, even though it was just a short
walk from the museums. It was located on the Mall, along Jefferson
Drive, naturally.

We were surrounded by the history of Thomas Jefferson when we made those
visits to our grandparents. We would drive down to Charlottesville with
our grandmother to visit our great-aunts and our great-grandmother ---
and they would take us up the mountain to Monticello and drop us off to
play in the house and on the grounds. They treated Monticello like it
was the family home, because in a way it was: They were
great-granddaughters of Jefferson. They had been born and grew up only a
few miles away at a family plantation, called Edgehill.

I guess that's why my brother and I, the great-grandsons, took the
Jefferson Memorial for granted. We had his ancestral home as a
playground. It was where all of our great-grandparents and great-aunts
and great-uncles were buried, and where one day, we were told, we would
be buried, too. We didn't need the Jefferson Memorial. Monticello was
enough.

It's still enough. In fact, as a memorial to Jefferson himself, it's
almost perfect. And that is why his memorial in Washington should be
taken down and replaced. Described by the National Park Service as ``a
shrine to freedom,'' it is anything but.

The memorial is a shrine to a man who during his lifetime owned more
than 600 slaves and had at least six children with one of them, Sally
Hemings. It's a shrine to a man who famously wrote that ``all men are
created equal'' in the Declaration of Independence that founded this
nation --- and yet never did much to make those words come true. Upon
his death, he did not free the people he enslaved, other than those in
the Hemings family, some of whom were his own children. He sold everyone
else to pay off his debts.

In fact, some of his white descendants, including his grandson Thomas
Jefferson Randolph, my great-great-great-great grandfather, fought in
the Civil War in defense of slavery. My great-grandmother lived with him
at Edgehill after she was born there in 1866. That is how close we are
not only to Jefferson but also to slavery. When we visited her as
children, there was only one dead man between my brother and me and
Thomas Jefferson.

I am the sixth-generation great-grandson of a slave owner. My cousins
from the Sally Hemings family are also the great-grandchildren of a
slave owner. But the difference is that our great-grandfather owned
their great-grandmother. My family owned their family. That is the
American history you will not learn when you visit the Jefferson
Memorial. But you will learn it when you visit Monticello: There's now
an exhibit of Sally Hemings's bedroom in her cavelike living quarters in
the south wing, a room my brother and I used to play in when we were
boys.

A tour of Monticello these days will tell you that it was designed by
Jefferson and built by the people he enslaved; it will point out joinery
and furniture built by Sally's brother, John Hemings. Today, there are
displays of rebuilt cabins and barns where those enslaved lived and
worked. At Monticello, you will learn the history of Jefferson, the man
who was president and wrote the Declaration of Independence, and you
will learn the history of Jefferson, the slave owner. Monticello is an
almost perfect memorial, because it reveals him with his moral failings
in full, an imperfect man, a flawed founder.

That's why we don't need the Jefferson Memorial to celebrate him. He
should not be honored with a bronze statue 19 feet tall, surrounded by a
colonnade of white marble. The time to honor the slave-owning founders
of our imperfect union is past. The ground, which should have moved long
ago, has at last shifted beneath us.

And it's time to honor one of our founding mothers, a woman who fought
as an escaped slave to free those still enslaved, who fought as an armed
scout for the Union Army against the Confederacy --- a woman who helped
to bring into being a more perfect union after slavery, a process that
continues to this day. In Jefferson's place, there should be another
statue. It should be of Harriet Tubman.

To see a 19-foot-tall bronze statue of a Black woman, who was a slave
and also a patriot, in place of a white man who enslaved hundreds of men
and women is not erasing history. It's telling the real history of
America.

\hypertarget{what-new-monuments-would-you-like-to-see}{%
\subsection{What new monuments would you like to
see?}\label{what-new-monuments-would-you-like-to-see}}

Lucian K. Truscott IV
(\href{https://twitter.com/LucianKTruscott}{@LucianKTruscott}) is a
novelist and a columnist for Salon.

\emph{The Times is committed to publishing}
\href{https://www.nytimes.com/2019/01/31/opinion/letters/letters-to-editor-new-york-times-women.html}{\emph{a
diversity of letters}} \emph{to the editor. We'd like to hear what you
think about this or any of our articles. Here are some}
\href{https://help.nytimes.com/hc/en-us/articles/115014925288-How-to-submit-a-letter-to-the-editor}{\emph{tips}}\emph{.
And here's our email:}
\href{mailto:letters@nytimes.com}{\emph{letters@nytimes.com}}\emph{.}

\emph{Follow The New York Times Opinion section on}
\href{https://www.facebook.com/nytopinion}{\emph{Facebook}}\emph{,}
\href{http://twitter.com/NYTOpinion}{\emph{Twitter (@NYTopinion)}}
\emph{and}
\href{https://www.instagram.com/nytopinion/}{\emph{Instagram}}\emph{.}

Advertisement

\protect\hyperlink{after-bottom}{Continue reading the main story}

\hypertarget{site-index}{%
\subsection{Site Index}\label{site-index}}

\hypertarget{site-information-navigation}{%
\subsection{Site Information
Navigation}\label{site-information-navigation}}

\begin{itemize}
\tightlist
\item
  \href{https://help.nytimes.com/hc/en-us/articles/115014792127-Copyright-notice}{©~2020~The
  New York Times Company}
\end{itemize}

\begin{itemize}
\tightlist
\item
  \href{https://www.nytco.com/}{NYTCo}
\item
  \href{https://help.nytimes.com/hc/en-us/articles/115015385887-Contact-Us}{Contact
  Us}
\item
  \href{https://www.nytco.com/careers/}{Work with us}
\item
  \href{https://nytmediakit.com/}{Advertise}
\item
  \href{http://www.tbrandstudio.com/}{T Brand Studio}
\item
  \href{https://www.nytimes.com/privacy/cookie-policy\#how-do-i-manage-trackers}{Your
  Ad Choices}
\item
  \href{https://www.nytimes.com/privacy}{Privacy}
\item
  \href{https://help.nytimes.com/hc/en-us/articles/115014893428-Terms-of-service}{Terms
  of Service}
\item
  \href{https://help.nytimes.com/hc/en-us/articles/115014893968-Terms-of-sale}{Terms
  of Sale}
\item
  \href{https://spiderbites.nytimes.com}{Site Map}
\item
  \href{https://help.nytimes.com/hc/en-us}{Help}
\item
  \href{https://www.nytimes.com/subscription?campaignId=37WXW}{Subscriptions}
\end{itemize}
