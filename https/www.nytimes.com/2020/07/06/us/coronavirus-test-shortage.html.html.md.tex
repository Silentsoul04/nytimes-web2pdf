Sections

SEARCH

\protect\hyperlink{site-content}{Skip to
content}\protect\hyperlink{site-index}{Skip to site index}

\href{https://www.nytimes.com/section/us}{U.S.}

\href{https://myaccount.nytimes.com/auth/login?response_type=cookie\&client_id=vi}{}

\href{https://www.nytimes.com/section/todayspaper}{Today's Paper}

\href{/section/us}{U.S.}\textbar{}Months Into Virus Crisis, U.S. Cities
Still Lack Testing Capacity

\url{https://nyti.ms/323iEPp}

\begin{itemize}
\item
\item
\item
\item
\item
\end{itemize}

\href{https://www.nytimes.com/news-event/coronavirus?action=click\&pgtype=Article\&state=default\&region=TOP_BANNER\&context=storylines_menu}{The
Coronavirus Outbreak}

\begin{itemize}
\tightlist
\item
  live\href{https://www.nytimes.com/2020/08/01/world/coronavirus-covid-19.html?action=click\&pgtype=Article\&state=default\&region=TOP_BANNER\&context=storylines_menu}{Latest
  Updates}
\item
  \href{https://www.nytimes.com/interactive/2020/us/coronavirus-us-cases.html?action=click\&pgtype=Article\&state=default\&region=TOP_BANNER\&context=storylines_menu}{Maps
  and Cases}
\item
  \href{https://www.nytimes.com/interactive/2020/science/coronavirus-vaccine-tracker.html?action=click\&pgtype=Article\&state=default\&region=TOP_BANNER\&context=storylines_menu}{Vaccine
  Tracker}
\item
  \href{https://www.nytimes.com/interactive/2020/07/29/us/schools-reopening-coronavirus.html?action=click\&pgtype=Article\&state=default\&region=TOP_BANNER\&context=storylines_menu}{What
  School May Look Like}
\item
  \href{https://www.nytimes.com/live/2020/07/31/business/stock-market-today-coronavirus?action=click\&pgtype=Article\&state=default\&region=TOP_BANNER\&context=storylines_menu}{Economy}
\end{itemize}

Advertisement

\protect\hyperlink{after-top}{Continue reading the main story}

Supported by

\protect\hyperlink{after-sponsor}{Continue reading the main story}

\hypertarget{months-into-virus-crisis-us-cities-still-lack-testing-capacity}{%
\section{Months Into Virus Crisis, U.S. Cities Still Lack Testing
Capacity}\label{months-into-virus-crisis-us-cities-still-lack-testing-capacity}}

With cases surging, some cities are seeing long testing lines and slow
results.

\includegraphics{https://static01.nyt.com/images/2020/07/06/us/06VIRUS-SHORTAGE-phoenix2/merlin_173982183_c5242f80-a986-4b3a-b0f7-ca858526d1fa-articleLarge.jpg?quality=75\&auto=webp\&disable=upscale}

By \href{https://www.nytimes.com/by/sarah-mervosh}{Sarah Mervosh} and
\href{https://www.nytimes.com/by/manny-fernandez}{Manny Fernandez}

\begin{itemize}
\item
  Published July 6, 2020Updated July 15, 2020
\item
  \begin{itemize}
  \item
  \item
  \item
  \item
  \item
  \end{itemize}
\end{itemize}

Lines for
\href{https://www.nytimes.com/2020/07/15/parenting/kids-covid-19-test.html}{coronavirus
tests} have stretched around city blocks and tests ran out altogether in
at least one site on Monday, new evidence that the country is still
struggling to create a sufficient testing system months into its battle
with Covid-19.

At a testing site in New Orleans, a line formed at dawn. But city
officials ran out of tests five minutes after the doors opened at 8
a.m., and many people had to be turned away.

In Phoenix, where temperatures have topped 100 degrees, residents have
waited in cars for as long as eight hours to get tested.

And in San Antonio and other large cities with mounting caseloads of the
virus, officials have reluctantly announced new limits to testing: The
demand has grown too great, they say, so only people showing symptoms
may now be tested --- a return to restrictions that were in place in
many parts of the country during earlier days of the
\href{https://www.nytimes.com/2020/07/15/parenting/kids-covid-19-test.html}{virus}.

``It's terrifying, and clearly an evidence of a failure of the system,''
said Dr. Morgan Katz, an infectious disease expert at Johns Hopkins
Hospital.

In the early months of the nation's outbreak,
\href{https://www.nytimes.com/2020/04/06/health/coronavirus-testing-us.html}{testing
posed a significant problem}, as supplies fell far short and officials
raced to understand how to best handle the virus. Since then, the United
States has \href{https://covidtracking.com/data/us-daily}{vastly ramped
up its testing} capability, conducting nearly 15 million tests in June,
about three times as many as it had in April. But in recent weeks, as
cases have surged in many states, the demand for testing has soared,
surpassing capacity and creating a
\href{https://www.nytimes.com/2020/07/23/health/coronavirus-testing-supply-shortage.html}{new
testing crisis}.

In many cities, officials said a combination of factors was now fueling
the problem: a shortage of certain supplies, backlogs at laboratories
that process the tests, and skyrocketing growth of the virus as cases
climb in almost 40 states and the nation approaches a grim new milestone
of three million total cases.

\href{https://www.nytimes.com/2020/07/06/health/fast-coronavirus-tests.html}{Fast,
widely available testing} is crucial to controlling the virus over the
long term in the United States, experts say, particularly as the country
reopens. With a virus that can spread through asymptomatic people,
screening large numbers of people is seen as essential to identifying
those who are carrying the virus and helping stop them from spreading it
to others.

But the images of long lines at testing sites and complaints from mayors
about the lack of a coordinated, overarching federal testing system have
placed the White House on the defensive.

\hypertarget{latest-updates-global-coronavirus-outbreak}{%
\section{\texorpdfstring{\href{https://www.nytimes.com/2020/08/01/world/coronavirus-covid-19.html?action=click\&pgtype=Article\&state=default\&region=MAIN_CONTENT_1\&context=storylines_live_updates}{Latest
Updates: Global Coronavirus
Outbreak}}{Latest Updates: Global Coronavirus Outbreak}}\label{latest-updates-global-coronavirus-outbreak}}

Updated 2020-08-02T07:42:09.613Z

\begin{itemize}
\tightlist
\item
  \href{https://www.nytimes.com/2020/08/01/world/coronavirus-covid-19.html?action=click\&pgtype=Article\&state=default\&region=MAIN_CONTENT_1\&context=storylines_live_updates\#link-34047410}{The
  U.S. reels as July cases more than double the total of any other
  month.}
\item
  \href{https://www.nytimes.com/2020/08/01/world/coronavirus-covid-19.html?action=click\&pgtype=Article\&state=default\&region=MAIN_CONTENT_1\&context=storylines_live_updates\#link-780ec966}{Top
  U.S. officials work to break an impasse over the federal jobless
  benefit.}
\item
  \href{https://www.nytimes.com/2020/08/01/world/coronavirus-covid-19.html?action=click\&pgtype=Article\&state=default\&region=MAIN_CONTENT_1\&context=storylines_live_updates\#link-2bc8948}{Its
  outbreak untamed, Melbourne goes into even greater lockdown.}
\end{itemize}

\href{https://www.nytimes.com/2020/08/01/world/coronavirus-covid-19.html?action=click\&pgtype=Article\&state=default\&region=MAIN_CONTENT_1\&context=storylines_live_updates}{See
more updates}

More live coverage:
\href{https://www.nytimes.com/live/2020/07/31/business/stock-market-today-coronavirus?action=click\&pgtype=Article\&state=default\&region=MAIN_CONTENT_1\&context=storylines_live_updates}{Markets}

President Trump
\href{https://twitter.com/realDonaldTrump/status/1280205902742781958?s=20}{tweeted}
on Monday that ``our great testing program continues to lead the World,
by FAR!'' Vice President Mike Pence said last week that the country had
so improved its testing capacity that ``we will literally test anyone
who comes into a testing site or comes to their local pharmacy.''

A spokeswoman for the Department of Health and Human Services said
federal officials had been working closely with states to develop and
meet testing goals since early April. So far, she said, the federal
government has distributed about 26 million swabs nationwide, among
other equipment, and was on track to ``meet all the needs for July.''

But testing in the United States has not kept pace with other countries,
notably in Asia, which have been more aggressive. Chinese officials who
were monitoring infections in Wuhan, where the pandemic began, tested
6.5 million people in a matter of days in May.

\includegraphics{https://static01.nyt.com/images/2020/07/06/us/06VIRUS-SHORTAGE-austin/merlin_174146379_47a2634c-b651-44ea-8876-4999c75c16ec-articleLarge.jpg?quality=75\&auto=webp\&disable=upscale}

In Arizona, where reported cases have grown to more than 100,000, a
shortage of testing has alarmed local officials, who say they feel
ill-equipped to help residents on their own.

``The United States of America needs a more robust national testing
strategy,'' Mayor Kate Gallego of Phoenix said in an interview.

Ms. Gallego, a Democrat, said she had been scrambling to lobby for help
from anyone she could think of --- the federal government, private
companies like Walgreens, even a middle school friend who works at a
European testing company. As the crisis has intensified in her state in
recent weeks, she suggested that testing resources could be shifted from
states with decreasing needs to those struggling like hers.

Arizona once had a stockpile of supplies, state officials say, but the
surge in cases since Memorial Day has drained even basic items for
testing, like swabs.

``That really speaks to the national and global supply chain issues,''
said Daniel Ruiz, Arizona's chief operating officer. ``It's not that
these things are in a warehouse ready to be delivered.''

All along, the United States has struggled with issues tied to testing.
In February, the federal government shipped a tainted testing kit to
states, delaying a broader testing strategy and leaving states blind to
a virus that was already beginning to circulate. Later, testing supplies
became a choke point, and states called on the federal government to use
the Defense Production Act to force additional production.

Many places have been able to overcome some of the supply constraints
that defined the earlier days of the outbreak, in part with their own
resources. New York City, once faced with severe shortages as an
epicenter of the virus, is now testing 30,000 people a day, officials
say, an expansion that included the city building its own testing kits
and partnering with private labs.

But even as Gov. Andrew M. Cuomo announced last week that anyone in New
York State who wanted a test could get one, officials in other states
have been left seeking a more robust testing system, and setting new
limits on who can take one.

``We are too fragmented,'' said Dr. Michael Mina, an assistant professor
of epidemiology at Harvard's T.H. Chan School of Public Health. ``We
don't have a good way to load-balance the system.''

Testing delays and shortages have increasingly become a problem in
Texas, where cases are surging.

Cities like San Antonio and Austin have reverted to testing only those
who are showing symptoms as a way to manage the demand and a backlog of
tests.

``We're now focused on the highest priorities,'' Mayor Steve Adler of
Austin said on Monday.

Mr. Adler, a Democrat, said the testing crunch was the result of the
demand for tests statewide, brought on by the uptick in coronavirus
cases after Texas reopened in fast-moving phases starting on May 1. He
attributed the problem in large part to a backlog at laboratories; in
some cases, test results take four to six days, far longer than the 24
hours health experts recommend to most effectively isolate the ill and
track people they have had contact with.

\href{https://www.nytimes.com/news-event/coronavirus?action=click\&pgtype=Article\&state=default\&region=MAIN_CONTENT_3\&context=storylines_faq}{}

\hypertarget{the-coronavirus-outbreak-}{%
\subsubsection{The Coronavirus Outbreak
›}\label{the-coronavirus-outbreak-}}

\hypertarget{frequently-asked-questions}{%
\paragraph{Frequently Asked
Questions}\label{frequently-asked-questions}}

Updated July 27, 2020

\begin{itemize}
\item ~
  \hypertarget{should-i-refinance-my-mortgage}{%
  \paragraph{Should I refinance my
  mortgage?}\label{should-i-refinance-my-mortgage}}

  \begin{itemize}
  \tightlist
  \item
    \href{https://www.nytimes.com/article/coronavirus-money-unemployment.html?action=click\&pgtype=Article\&state=default\&region=MAIN_CONTENT_3\&context=storylines_faq}{It
    could be a good idea,} because mortgage rates have
    \href{https://www.nytimes.com/2020/07/16/business/mortgage-rates-below-3-percent.html?action=click\&pgtype=Article\&state=default\&region=MAIN_CONTENT_3\&context=storylines_faq}{never
    been lower.} Refinancing requests have pushed mortgage applications
    to some of the highest levels since 2008, so be prepared to get in
    line. But defaults are also up, so if you're thinking about buying a
    home, be aware that some lenders have tightened their standards.
  \end{itemize}
\item ~
  \hypertarget{what-is-school-going-to-look-like-in-september}{%
  \paragraph{What is school going to look like in
  September?}\label{what-is-school-going-to-look-like-in-september}}

  \begin{itemize}
  \tightlist
  \item
    It is unlikely that many schools will return to a normal schedule
    this fall, requiring the grind of
    \href{https://www.nytimes.com/2020/06/05/us/coronavirus-education-lost-learning.html?action=click\&pgtype=Article\&state=default\&region=MAIN_CONTENT_3\&context=storylines_faq}{online
    learning},
    \href{https://www.nytimes.com/2020/05/29/us/coronavirus-child-care-centers.html?action=click\&pgtype=Article\&state=default\&region=MAIN_CONTENT_3\&context=storylines_faq}{makeshift
    child care} and
    \href{https://www.nytimes.com/2020/06/03/business/economy/coronavirus-working-women.html?action=click\&pgtype=Article\&state=default\&region=MAIN_CONTENT_3\&context=storylines_faq}{stunted
    workdays} to continue. California's two largest public school
    districts --- Los Angeles and San Diego --- said on July 13, that
    \href{https://www.nytimes.com/2020/07/13/us/lausd-san-diego-school-reopening.html?action=click\&pgtype=Article\&state=default\&region=MAIN_CONTENT_3\&context=storylines_faq}{instruction
    will be remote-only in the fall}, citing concerns that surging
    coronavirus infections in their areas pose too dire a risk for
    students and teachers. Together, the two districts enroll some
    825,000 students. They are the largest in the country so far to
    abandon plans for even a partial physical return to classrooms when
    they reopen in August. For other districts, the solution won't be an
    all-or-nothing approach.
    \href{https://bioethics.jhu.edu/research-and-outreach/projects/eschool-initiative/school-policy-tracker/}{Many
    systems}, including the nation's largest, New York City, are
    devising
    \href{https://www.nytimes.com/2020/06/26/us/coronavirus-schools-reopen-fall.html?action=click\&pgtype=Article\&state=default\&region=MAIN_CONTENT_3\&context=storylines_faq}{hybrid
    plans} that involve spending some days in classrooms and other days
    online. There's no national policy on this yet, so check with your
    municipal school system regularly to see what is happening in your
    community.
  \end{itemize}
\item ~
  \hypertarget{is-the-coronavirus-airborne}{%
  \paragraph{Is the coronavirus
  airborne?}\label{is-the-coronavirus-airborne}}

  \begin{itemize}
  \tightlist
  \item
    The coronavirus
    \href{https://www.nytimes.com/2020/07/04/health/239-experts-with-one-big-claim-the-coronavirus-is-airborne.html?action=click\&pgtype=Article\&state=default\&region=MAIN_CONTENT_3\&context=storylines_faq}{can
    stay aloft for hours in tiny droplets in stagnant air}, infecting
    people as they inhale, mounting scientific evidence suggests. This
    risk is highest in crowded indoor spaces with poor ventilation, and
    may help explain super-spreading events reported in meatpacking
    plants, churches and restaurants.
    \href{https://www.nytimes.com/2020/07/06/health/coronavirus-airborne-aerosols.html?action=click\&pgtype=Article\&state=default\&region=MAIN_CONTENT_3\&context=storylines_faq}{It's
    unclear how often the virus is spread} via these tiny droplets, or
    aerosols, compared with larger droplets that are expelled when a
    sick person coughs or sneezes, or transmitted through contact with
    contaminated surfaces, said Linsey Marr, an aerosol expert at
    Virginia Tech. Aerosols are released even when a person without
    symptoms exhales, talks or sings, according to Dr. Marr and more
    than 200 other experts, who
    \href{https://academic.oup.com/cid/article/doi/10.1093/cid/ciaa939/5867798}{have
    outlined the evidence in an open letter to the World Health
    Organization}.
  \end{itemize}
\item ~
  \hypertarget{what-are-the-symptoms-of-coronavirus}{%
  \paragraph{What are the symptoms of
  coronavirus?}\label{what-are-the-symptoms-of-coronavirus}}

  \begin{itemize}
  \tightlist
  \item
    Common symptoms
    \href{https://www.nytimes.com/article/symptoms-coronavirus.html?action=click\&pgtype=Article\&state=default\&region=MAIN_CONTENT_3\&context=storylines_faq}{include
    fever, a dry cough, fatigue and difficulty breathing or shortness of
    breath.} Some of these symptoms overlap with those of the flu,
    making detection difficult, but runny noses and stuffy sinuses are
    less common.
    \href{https://www.nytimes.com/2020/04/27/health/coronavirus-symptoms-cdc.html?action=click\&pgtype=Article\&state=default\&region=MAIN_CONTENT_3\&context=storylines_faq}{The
    C.D.C. has also} added chills, muscle pain, sore throat, headache
    and a new loss of the sense of taste or smell as symptoms to look
    out for. Most people fall ill five to seven days after exposure, but
    symptoms may appear in as few as two days or as many as 14 days.
  \end{itemize}
\item ~
  \hypertarget{does-asymptomatic-transmission-of-covid-19-happen}{%
  \paragraph{Does asymptomatic transmission of Covid-19
  happen?}\label{does-asymptomatic-transmission-of-covid-19-happen}}

  \begin{itemize}
  \tightlist
  \item
    So far, the evidence seems to show it does. A widely cited
    \href{https://www.nature.com/articles/s41591-020-0869-5}{paper}
    published in April suggests that people are most infectious about
    two days before the onset of coronavirus symptoms and estimated that
    44 percent of new infections were a result of transmission from
    people who were not yet showing symptoms. Recently, a top expert at
    the World Health Organization stated that transmission of the
    coronavirus by people who did not have symptoms was ``very rare,''
    \href{https://www.nytimes.com/2020/06/09/world/coronavirus-updates.html?action=click\&pgtype=Article\&state=default\&region=MAIN_CONTENT_3\&context=storylines_faq\#link-1f302e21}{but
    she later walked back that statement.}
  \end{itemize}
\end{itemize}

Local officials in Austin had not relied on the state when it came to
testing for the most part, the mayor said. And without a national
testing program, he said, city and county officials had to fend for
themselves in the private market.

``Maybe in retrospect if we had thought about this a half-year ago, we
would have set up our own testing capacity,'' Mr. Adler said. ``I don't
know what else we'd do. We were out competing for tests. We were
blocking up as many tests as we could block up on the market.''

Mr. Adler said the testing system needed to be federalized, so that
Austin and other cities would not have to compete for testing labs and
supplies with other cities and other states.

The problem extends far beyond Texas and Arizona, among the hot spots
that have led the country in rising cases in recent weeks.

In Idaho, where cases were also climbing, the state lab was so inundated
that state officials sent a memo to nursing homes and long-term care
facilities, saying the state could no longer meet all their testing
needs. That has left the facilities in a crisis, desperate to find other
labs to process tests for a particularly vulnerable population.

``Everyone is scrambling,'' said Robert Vande Merwe, the executive
director of the Idaho Health Care Association.

Louisiana has also seen testing delays.

Dr. Jennifer Avegno, the director of the New Orleans Health Department,
said the problem her agency was seeing now was different than the one it
experienced in March, when states competed over swabs and test tubes.
Now the problem is a shortage of reagents, she said, which are the
chemical ingredients needed to detect whether the coronavirus is present
in a sample.

The supply chain issues have led officials in New Orleans to reduce the
tests they carry out: At one site on Monday, officials handed out just
150 tickets for testing, which were gone in minutes.

``We are telling everyone to do all the things you are supposed to do,
and if they have any concerns about exposure or close contact or are
feeling sick, there will be a test for you,'' Dr. Avegno said.

``And yet we're starting to have to turn them away,'' she said. ``That
is not what we want to do.''

Image

A locked testing site at Dillard University in New Orleans on Monday
afternoon. All tests available were reserved before 8:15
a.m.Credit...Emily Kask for The New York Times

The urgent demand for tests also was affecting regions outside of those
hardest hit.

In Omaha, a drive-through testing site in the parking lot of a former
grocery store abruptly closed on Saturday.

Lab supplies fell short in the city, partly because they were needed in
communities with bigger outbreaks than in Nebraska, where cases are
prevalent but remain steady.

``We're getting put down on the priority list,'' said Dr. Anne O'Keefe,
senior epidemiologist with the Douglas County Health Department in
Omaha, citing a decision by the manufacturers of high-volume testing
machines to prioritize supplies for those machines for other states
before Nebraska.

The site in Omaha had tested nearly 3,500 people since it opened on June
17. Officials said they did not know when it would reopen.

``We really wanted to provide that extra capacity, to give people better
options,'' Dr. O'Keefe said. ``We're very, very disappointed that we
can't do it.''

Reporting was contributed by Mike Baker, Kimiko de Freytas-Tamura,
Sheryl Gay Stolberg, Zolan Kanno-Youngs, Katie Rogers, Mark Walker and
Elizabeth Williamson.

Advertisement

\protect\hyperlink{after-bottom}{Continue reading the main story}

\hypertarget{site-index}{%
\subsection{Site Index}\label{site-index}}

\hypertarget{site-information-navigation}{%
\subsection{Site Information
Navigation}\label{site-information-navigation}}

\begin{itemize}
\tightlist
\item
  \href{https://help.nytimes.com/hc/en-us/articles/115014792127-Copyright-notice}{©~2020~The
  New York Times Company}
\end{itemize}

\begin{itemize}
\tightlist
\item
  \href{https://www.nytco.com/}{NYTCo}
\item
  \href{https://help.nytimes.com/hc/en-us/articles/115015385887-Contact-Us}{Contact
  Us}
\item
  \href{https://www.nytco.com/careers/}{Work with us}
\item
  \href{https://nytmediakit.com/}{Advertise}
\item
  \href{http://www.tbrandstudio.com/}{T Brand Studio}
\item
  \href{https://www.nytimes.com/privacy/cookie-policy\#how-do-i-manage-trackers}{Your
  Ad Choices}
\item
  \href{https://www.nytimes.com/privacy}{Privacy}
\item
  \href{https://help.nytimes.com/hc/en-us/articles/115014893428-Terms-of-service}{Terms
  of Service}
\item
  \href{https://help.nytimes.com/hc/en-us/articles/115014893968-Terms-of-sale}{Terms
  of Sale}
\item
  \href{https://spiderbites.nytimes.com}{Site Map}
\item
  \href{https://help.nytimes.com/hc/en-us}{Help}
\item
  \href{https://www.nytimes.com/subscription?campaignId=37WXW}{Subscriptions}
\end{itemize}
