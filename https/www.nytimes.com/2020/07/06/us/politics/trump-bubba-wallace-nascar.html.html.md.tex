Sections

SEARCH

\protect\hyperlink{site-content}{Skip to
content}\protect\hyperlink{site-index}{Skip to site index}

\href{https://www.nytimes.com/section/politics}{Politics}

\href{https://myaccount.nytimes.com/auth/login?response_type=cookie\&client_id=vi}{}

\href{https://www.nytimes.com/section/todayspaper}{Today's Paper}

\href{/section/politics}{Politics}\textbar{}Trump Adds to Playbook of
Stoking White Fear and Resentment

\url{https://nyti.ms/3iKs5cm}

\begin{itemize}
\item
\item
\item
\item
\item
\item
\end{itemize}

\begin{itemize}
\item
  \href{https://www.nytimes.com/2020/07/31/us/elections/biden-vs-trump.html?action=click\&pgtype=Article\&state=default\&region=TOP_BANNER\&context=storylines_menu}{Election
  Updates}
\item
  \href{https://www.nytimes.com/article/biden-vice-president-2020.html?action=click\&pgtype=Article\&state=default\&region=TOP_BANNER\&context=storylines_menu}{Biden's
  V.P. Search}
\item
  \href{https://www.nytimes.com/interactive/2020/07/24/us/politics/trump-biden-campaign-donors.html?action=click\&pgtype=Article\&state=default\&region=TOP_BANNER\&context=storylines_menu}{Map
  of Donations}
\item
  \href{https://www.nytimes.com/interactive/2020/us/elections/delegate-count-primary-results.html?action=click\&pgtype=Article\&state=default\&region=TOP_BANNER\&context=storylines_menu}{Delegate
  Count}
\item
  \href{https://www.nytimes.com/interactive/2019/us/politics/2020-presidential-candidates.html?action=click\&pgtype=Article\&state=default\&region=TOP_BANNER\&context=storylines_menu}{The
  Candidates}
\item
  \href{https://www.nytimes.com/newsletters/politics?action=click\&pgtype=Article\&state=default\&region=TOP_BANNER\&context=storylines_menu}{Politics
  Newsletter}
\end{itemize}

Advertisement

\protect\hyperlink{after-top}{Continue reading the main story}

Supported by

\protect\hyperlink{after-sponsor}{Continue reading the main story}

\hypertarget{trump-adds-to-playbook-of-stoking-white-fear-and-resentment}{%
\section{Trump Adds to Playbook of Stoking White Fear and
Resentment}\label{trump-adds-to-playbook-of-stoking-white-fear-and-resentment}}

With a defense of Confederate flags and a false accusation against a
Black NASCAR driver, Bubba Wallace, President Trump is focusing on
racial and cultural flash points to appeal to his base.

\includegraphics{https://static01.nyt.com/images/2020/07/07/us/politics/07a3_conversation/merlin_174247971_86379e04-4c12-469d-a94d-3ae2a278a996-articleLarge.jpg?quality=75\&auto=webp\&disable=upscale}

\href{https://www.nytimes.com/by/maggie-haberman}{\includegraphics{https://static01.nyt.com/images/2018/07/12/multimedia/author-maggie-haberman/author-maggie-haberman-thumbLarge.png}}

By \href{https://www.nytimes.com/by/maggie-haberman}{Maggie Haberman}

\begin{itemize}
\item
  July 6, 2020
\item
  \begin{itemize}
  \item
  \item
  \item
  \item
  \item
  \item
  \end{itemize}
\end{itemize}

\href{https://www.nytimes.com/interactive/2020/us/elections/donald-trump.html}{President
Trump} mounted an explicit defense of the
\href{https://www.nytimes.com/2020/07/17/us/politics/pentagon-trump-confederate-symbols.html}{Confederate
flag} on Monday, suggesting that NASCAR had made a mistake in banning it
from its auto racing events, while falsely accusing a top Black driver,
Darrell Wallace Jr., of perpetrating a hoax involving a noose found in
his garage.

The remarks are part of a pattern. Almost every day in the last two
weeks, Mr. Trump has sought to stoke white fear and resentment,
portraying himself as a protector of an old order that polls show much
of America believes perpetuates entrenched racism and wants to move
beyond.

Two weeks ago, the president
\href{https://www.nytimes.com/2020/06/28/us/politics/trump-white-power-video-racism.html}{retweeted
a video} of a supporter shouting ``white power'' at a retirement
community filled with older people whom he wants to win over. Last week,
\href{https://www.nytimes.com/2020/07/01/us/politics/trump-obama-housing-discrimination.html}{he
wrote that he was reviewing} a fair housing regulation that is aimed at
eliminating racial housing disparities in the suburbs, but that he said
would have a ``devastating impact'' on those communities --- a play to
white suburbanites whose votes would be crucial to his re-election.

On Monday, he also tweeted his displeasure with sports teams that are
reviewing the appropriateness of nicknames that are offensive to Native
Americans, seeking to curry favor with Americans who believe political
correctness has gone too far. He has invoked fear of crime with tweets
about sanctuary cities and crime rates in New York and Chicago, and has
spoken of preserving ``our heritage,'' picking up the language of those
who want to honor the Confederacy.

For many Republicans who are watching
\href{https://www.nytimes.com/2020/06/25/us/politics/trump-senate-republicans-poll.html}{the
president's impact on Senate races} with alarm, his focus on racial and
cultural flash points --- and not on the
\href{https://www.nytimes.com/2020/07/06/world/coronavirus-updates.html}{surge
of the coronavirus} in many states --- is distressing.

``This is part of the same selfish, divide-and-conquer strategy that
helped the president get elected in 2016,'' said Carlos Curbelo, a
former Republican congressman from Florida who has been critical of Mr.
Trump. ``Of course that strategy worked for much of the president's
base, and it certainly benefited him in the past, but it's selfish in
the sense that it is extremely damaging for Republicans in swing states,
in swing districts.''

Mr. Curbelo added, ``It's always been clear, but this is a reminder that
the president looks out for himself first, second and third.''

\hypertarget{latest-updates-2020-election}{%
\section{\texorpdfstring{\href{https://www.nytimes.com/2020/07/31/us/elections/biden-vs-trump.html?action=click\&pgtype=Article\&state=default\&region=MAIN_CONTENT_1\&context=storylines_live_updates}{Latest
Updates: 2020
Election}}{Latest Updates: 2020 Election}}\label{latest-updates-2020-election}}

Updated 2020-08-01T01:26:45.732Z

\begin{itemize}
\tightlist
\item
  \href{https://www.nytimes.com/2020/07/31/us/elections/biden-vs-trump.html?action=click\&pgtype=Article\&state=default\&region=MAIN_CONTENT_1\&context=storylines_live_updates\#link-29fdff45}{Kamala
  Harris, a top vice-presidential contender, confronts double
  standards.}
\item
  \href{https://www.nytimes.com/2020/07/31/us/elections/biden-vs-trump.html?action=click\&pgtype=Article\&state=default\&region=MAIN_CONTENT_1\&context=storylines_live_updates\#link-13ec3d9c}{Karen
  Bass and Susan Rice are rising on Biden's vice-presidential
  shortlist.}
\item
  \href{https://www.nytimes.com/2020/07/31/us/elections/biden-vs-trump.html?action=click\&pgtype=Article\&state=default\&region=MAIN_CONTENT_1\&context=storylines_live_updates\#link-49e9a016}{Trump
  says Russian bounties to kill U.S. troops `never took place.'}
\end{itemize}

\href{https://www.nytimes.com/2020/07/31/us/elections/biden-vs-trump.html?action=click\&pgtype=Article\&state=default\&region=MAIN_CONTENT_1\&context=storylines_live_updates}{See
more updates}

Mr. Trump's inflammatory behavior shows how out of step he is with
\href{https://www.nytimes.com/interactive/2020/06/10/upshot/black-lives-matter-attitudes.html}{shifting
national sentiment on racial justice}, as big corporations, sports
leagues and cultural institutions express greater solidarity with Black
Americans protesting systemic racism. Even some Republicans have been
open to discussions about removing Confederate statues.

While NASCAR and other organizations have moved to retire symbols of the
Confederacy, and
\href{https://www.nytimes.com/2020/06/28/us/mississippi-flag-confederacy.html?searchResultPosition=4}{lawmakers
in Mississippi voted} to bring down the state flag featuring the
Confederate battle emblem, Mr. Trump continues to cast himself as a
defender of the history of the American South, despite its stains of
slavery and oppression. He has called the phrase ``Black Lives Matter''
a ``symbol of hate,'' and he has repeatedly tried to depict pockets of
violence during protests against entrenched racism as representative of
the protest movement as a whole.

Mr. Trump also
\href{https://www.nytimes.com/2020/07/04/us/politics/trump-mt-rushmore.html}{delivered
official speeches} over the weekend that emphasized defending American
historical figures like George Washington and some abolitionists, though
he avoided explicit references to totems of the Confederacy.

But on Monday he was back invoking the Confederacy, with his reference
to NASCAR's ban on Confederate flags, while also attacking Mr. Wallace,
the only Black driver on NASCAR's top circuit.

Mr. Wallace, nicknamed ``Bubba,'' had called for NASCAR to ban the flag
from its events, and the sport agreed to prohibit it from its races and
its properties. At the start of race week at the Talladega Superspeedway
in Alabama last month, a member of Mr. Wallace's racing team
\href{https://www.nytimes.com/2020/06/26/sports/autoracing/nascar-noose-bubba-wallace.html}{found
a noose} hanging in the driver's garage stall and reported it to NASCAR.

``Has @BubbaWallace apologized to all of those great NASCAR drivers \&
officials who came to his aid, stood by his side, \& were willing to
sacrifice everything for him, only to find out that the whole thing was
just another HOAX? That \& Flag decision has caused lowest ratings
EVER!'' Mr. Trump
\href{https://twitter.com/realDonaldTrump/status/1280117571874951170}{posted}
on Twitter on Monday.

\includegraphics{https://static01.nyt.com/images/2020/07/06/us/politics/06trump-flag/merlin_174019152_e177c971-1985-42c0-9688-d2322639b1de-articleLarge.jpg?quality=75\&auto=webp\&disable=upscale}

Kayleigh McEnany, the White House press secretary, offered a contorted
defense of Mr. Trump's tweet about the Confederate flag and Mr. Wallace
during an early afternoon briefing.

She insisted Mr. Trump was being taken out of context, and invoked
Jussie Smollet, the Black television actor known for his role on the TV
series ``Empire,'' who is facing charges that he lied to the authorities
about a hate crime attack that detectives said he had staged last year
in Chicago.

No one has credibly suggested Mr. Wallace manufactured the noose that
was discovered in his garage stall by a colleague. F.B.I. officials
\href{https://www.nytimes.com/2020/06/23/sports/autoracing/bubba-wallace-noose-nascar.html}{later
found} that the knot had been tied into the rope as early as October
2019, well before anyone would have known that Mr. Wallace would be
assigned that stall
\href{https://www.nytimes.com/2020/06/22/sports/autoracing/bubba-wallace-noose-nascar.html}{for
the race}.

Ms. McEnany claimed that the original reports about the incident painted
NASCAR members as ``racist individuals who were roaming around and
engaging in a crime.''

But Mr. Trump received pushback from Senator Lindsey Graham, Republican
of South Carolina and an informal adviser to the president, who said
Monday that he disagreed with Mr. Trump's tweet.

``They're trying to grow the sport,'' Mr. Graham said,
\href{https://twitter.com/mkraju/status/1280165407740104709}{according
to the CNN reporter Manu Raju}, referring to NASCAR's ban on Confederate
flags,
\href{https://www.nytimes.com/2020/06/10/sports/autoracing/nascar-confederate-flags.html?searchResultPosition=1}{which
it announced last month}. ``And I've lived in South Carolina all my life
and if you're in business, the Confederate flag is not a good way to
grow your business.''

Mr. Graham, who is facing a strong challenge from Jaime Harrison, a
Black Democrat, in his re-election bid, said that ``one way you grow the
sport is you take images that divide us and ask that they not be brought
into the venue. That makes sense to me.'' He said that Mr. Wallace did
not have ``anything to apologize for,'' and that his fellow drivers
should be applauded for supporting him.

``I would be looking to celebrate that kind of attitude more than being
worried about it being a hoax,'' Mr. Graham said, according to Mr. Raju.

(Mr. Trump was also wrong in his tweet in characterizing NASCAR's
television audience as having fallen to its ``lowest ratings EVER!'' The
broadcast of Sunday's Brickyard 400 was seen by about 4.3 million
viewers, a 39 percent increase from the average NASCAR race that aired
on NBC last year, according to Nielsen.)

Later on Monday, Mr. Trump added another inflammatory tweet, weighing in
on recent announcements
\href{https://www.nytimes.com/2020/07/03/sports/football/washington-redskins-nickname-nfl.html}{by
the Washington Redskins} of the N.F.L. and
\href{https://www.nytimes.com/2020/07/03/sports/baseball/cleveland-indians-name-change.html}{the
Cleveland Indians} of Major League Baseball that the teams would review
their names. While many Native Americans and other advocates for change
consider the names deeply offensive, Mr. Trump baselessly claimed that
Native Americans would be ``very angry'' about the potential changes.

``They name teams out of STRENGTH, not weakness, but now the Washington
Redskins \& Cleveland Indians, two fabled sports franchises, look like
they are going to be changing their names in order to be politically
correct,'' Mr. Trump
\href{https://www.nytimes.com/2020/07/06/sports/football/washington-team-name-change.html}{tweeted}.
He added a jab at a favorite target, Senator Elizabeth Warren, who
\href{https://www.nytimes.com/2019/08/19/us/politics/elizabeth-warren-native-american.html}{has
apologized} for her past claims of Native ancestry. ``Indians, like
Elizabeth Warren, must be very angry right now,'' the president wrote.

Eleven minutes later, Mr. Trump again referred to the coronavirus as the
``China Virus,'' a phrase that
\href{https://www.nytimes.com/2020/03/18/us/politics/china-virus.html}{critics
say is racist, xenophobic and harmful to Asian-Americans}.

Mr. Trump's tweets came just days after he delivered
\href{https://www.nytimes.com/2020/07/03/us/politics/trump-coronavirus-mount-rushmore.html?action=click\&module=RelatedLinks\&pgtype=Article}{a
divisive speech at Mount Rushmore} in South Dakota as part of the July 4
holiday, in which he denounced Democrats as radical anarchists and said
that children are taught in schools to ``hate'' the United States. In
that address he avoided specifically mentioning anything related to
Confederate monuments.

He talked more generally about efforts to take down statues across the
country, conflating what is primarily an attempt to remove statues of
Confederate generals with others questioning monuments to people like
George Washington and Thomas Jefferson.

``Angry mobs are trying to tear down statues of our founders, deface our
most sacred memorials, and unleash a wave of violent crime in our
cities,'' Mr. Trump said in the speech. ``Many of these people have no
idea why they are doing this, but some know exactly what they are
doing.''

Some of Mr. Trump's advisers have tried to persuade him to focus less
explicitly on statues of Confederate generals, given that he is taking
an unpopular position. But after sticking to the script in his Friday
night speech, he was clear about his support for the Confederate flag in
his tweet on Monday.

Michael M. Grynbaum contributed reporting from New York.

\hypertarget{our-2020-election-guide}{%
\section{Our 2020 Election Guide}\label{our-2020-election-guide}}

Updated July 31, 2020

\begin{itemize}
\item
  \begin{center}\rule{0.5\linewidth}{\linethickness}\end{center}

  \hypertarget{the-latest}{%
  \subsection{The Latest}\label{the-latest}}

  \begin{itemize}
  \tightlist
  \item
    President Trump's assault on the Postal Service is intersecting with
    his attacks on mail-in voting.
    \href{https://www.nytimes.com/2020/07/31/us/politics/trump-usps-mail-delays.html?action=click\&pgtype=Article\&state=default\&region=BELOW_MAIN_CONTENT\&context=storylines_guide}{Voting
    rights groups say it is a recipe for disaster.}
  \end{itemize}
\item
  \begin{center}\rule{0.5\linewidth}{\linethickness}\end{center}

  \hypertarget{bidens-vp-search}{%
  \subsection{Biden's V.P. Search}\label{bidens-vp-search}}

  \begin{itemize}
  \tightlist
  \item
    \href{https://www.nytimes.com/article/biden-vice-president-2020.html?action=click\&pgtype=Article\&state=default\&region=BELOW_MAIN_CONTENT\&context=storylines_guide}{Here
    are 13 women} who have been under consideration to be Joe Biden's
    running mate, and why each might be chosen --- and might not be.
  \end{itemize}
\item
  \begin{center}\rule{0.5\linewidth}{\linethickness}\end{center}

  \hypertarget{keep-up-with-our-coverage}{%
  \subsection{Keep Up With Our
  Coverage}\label{keep-up-with-our-coverage}}

  \begin{itemize}
  \tightlist
  \item
    Get an
    \href{https://www.nytimes.com/newsletters/politics?action=click\&pgtype=Article\&state=default\&region=BELOW_MAIN_CONTENT\&context=storylines_guide}{email}
    recapping the day's news
  \end{itemize}

  \begin{itemize}
  \tightlist
  \item
    Download our mobile app on
    \href{https://apps.apple.com/us/app/nytimes/id284862083?ls=1\&mat_click_id=5c79ae7455014fd1bd66b5610c05b8f2-20191112-16948\&referrer=mat_click_id\%3D5c79ae7455014fd1bd66b5610c05b8f2-20191112-16948\%26link_click_id\%3D722930677036718082}{iOS}
    and
    \href{http://a.localytics.com/android?id=com.nytimes.android\&referrer=utm_source\%3Dother_nyt_mobile_web\%26utm_medium\%3DWeb\%2520page\%26utm_term\%3DGeneral\%2520Mobile\%2520Page\%26utm_campaign\%3DNYT\%2520Mobile\%2520General\%2520Page}{Android}
    and turn on Breaking News and Politics alerts
  \end{itemize}
\end{itemize}

Advertisement

\protect\hyperlink{after-bottom}{Continue reading the main story}

\hypertarget{site-index}{%
\subsection{Site Index}\label{site-index}}

\hypertarget{site-information-navigation}{%
\subsection{Site Information
Navigation}\label{site-information-navigation}}

\begin{itemize}
\tightlist
\item
  \href{https://help.nytimes.com/hc/en-us/articles/115014792127-Copyright-notice}{©~2020~The
  New York Times Company}
\end{itemize}

\begin{itemize}
\tightlist
\item
  \href{https://www.nytco.com/}{NYTCo}
\item
  \href{https://help.nytimes.com/hc/en-us/articles/115015385887-Contact-Us}{Contact
  Us}
\item
  \href{https://www.nytco.com/careers/}{Work with us}
\item
  \href{https://nytmediakit.com/}{Advertise}
\item
  \href{http://www.tbrandstudio.com/}{T Brand Studio}
\item
  \href{https://www.nytimes.com/privacy/cookie-policy\#how-do-i-manage-trackers}{Your
  Ad Choices}
\item
  \href{https://www.nytimes.com/privacy}{Privacy}
\item
  \href{https://help.nytimes.com/hc/en-us/articles/115014893428-Terms-of-service}{Terms
  of Service}
\item
  \href{https://help.nytimes.com/hc/en-us/articles/115014893968-Terms-of-sale}{Terms
  of Sale}
\item
  \href{https://spiderbites.nytimes.com}{Site Map}
\item
  \href{https://help.nytimes.com/hc/en-us}{Help}
\item
  \href{https://www.nytimes.com/subscription?campaignId=37WXW}{Subscriptions}
\end{itemize}
