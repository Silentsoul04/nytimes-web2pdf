Sections

SEARCH

\protect\hyperlink{site-content}{Skip to
content}\protect\hyperlink{site-index}{Skip to site index}

\href{https://myaccount.nytimes.com/auth/login?response_type=cookie\&client_id=vi}{}

\href{https://www.nytimes.com/section/todayspaper}{Today's Paper}

\href{/section/opinion}{Opinion}\textbar{}The Religious Roots of a New
Progressive Era

\href{https://nyti.ms/3gwCrum}{https://nyti.ms/3gwCrum}

\begin{itemize}
\item
\item
\item
\item
\item
\item
\end{itemize}

Advertisement

\protect\hyperlink{after-top}{Continue reading the main story}

\href{/section/opinion}{Opinion}

Supported by

\protect\hyperlink{after-sponsor}{Continue reading the main story}

\hypertarget{the-religious-roots-of-a-new-progressive-era}{%
\section{The Religious Roots of a New Progressive
Era}\label{the-religious-roots-of-a-new-progressive-era}}

Welcome to the post-Protestant Reformation.

\href{https://www.nytimes.com/by/ross-douthat}{\includegraphics{https://static01.nyt.com/images/2018/04/03/opinion/ross-douthat/ross-douthat-thumbLarge.png}}

By \href{https://www.nytimes.com/by/ross-douthat}{Ross Douthat}

Opinion Columnist

\begin{itemize}
\item
  July 7, 2020
\item
  \begin{itemize}
  \item
  \item
  \item
  \item
  \item
  \item
  \end{itemize}
\end{itemize}

\includegraphics{https://static01.nyt.com/images/2020/07/07/opinion/07douthat1/merlin_166618563_099217bf-40c8-40ef-8312-04a68b4ed77b-articleLarge.jpg?quality=75\&auto=webp\&disable=upscale}

``Our form of government has no sense unless it is founded in a deeply
felt religious faith,'' Dwight Eisenhower said in 1952, ``and I don't
care what it is.''

I've always appreciated that line, even though it's usually quoted
somewhat unfairly: If you read the rest of the speech, it's clear
Eisenhower was trying to make an ecumenical point about how multiple
faiths might sustain the doctrine of human equality, not an
indifferentist point about the irrelevance of theology to faith.

Still, taken on its own it's like a koan of the American civic religion,
the faith in faith that reached a zenith under the Eisenhower
presidency: \emph{In God We Trust, and don't sweat the theological
details.}

We've been having a lively debate lately about what the sudden
social-justice ascendancy in American institutions represents, and
whether the new iconoclastic progressivism is just an
\href{https://newrepublic.com/article/158346/willful-blindness-reactionary-liberalism}{organic
development} in liberalism or a
\href{https://www.nytimes.com/2020/06/12/opinion/nyt-tom-cotton-oped-liberalism.html}{post-liberal
successor}. But Ike's koan suggests a different way to think about these
arguments: Instead of seeing today's perturbations as being mostly about
what might come after liberalism, you could see them as a struggle over
what religious worldview should inhabit it, and whether Eisenhower was
right that lots of different faiths could fill the void.

By ``inhabit'' I mean play the role that for most of our history was
played by Mainline Protestantism --- the whole collage of respectable
denominations, Methodists and Lutherans and Episcopalians and
Presbyterians and Baptists, their churches sharing town greens and their
ministers hobnobbing, divided by mild class distinctions as much as by
theological debates, competing amiably for congregants, eyeing Catholics
and Jews and Mormons uneasily and looking down on fundamentalists,
preaching liberty and middle-class morality and assimilation, secure in
their Christianity and their Americanism.

This sketch is all cliché, but the clichés reflect an important,
underremembered reality: For most of our history, American liberal
democracy was a Protestant project, its principles undergirded by
Protestant theological assumptions and its norms shaped and reshaped by
currents in the Mainline churches.

To push a metaphor for a moment --- if the Constitution and the Bill of
Rights were the bones of the house that all Americans inhabited, then
the Protestant Mainline was a combination interior decorator, building
inspector, homeowners' association and zoning committee. Any question
that the liberal order didn't answer, across most of our history, was
answered by Protestant consensus or litigated by intra-Protestant
debate. (What were the limits of religious liberty? Should society
regulate sex, and how? Should society regulate alcohol consumption, and
how? What values should be taught in schools and universities?) And when
the Mainline couldn't come to an agreement, as in the long theological
dispute over slavery and racial equality --- well, then part of the
house burned down and had to be repeatedly reconstructed.

But all that belongs to the past, because in the decades after
Eisenhower, the Mainline suddenly collapsed --- declining numerically
and losing overt influence in all the institutions, elite and local
alike, that it once animated and defined. What took its place, in the
upper echelons on the meritocracy, was an assumption that liberalism
didn't need a religious ghost in its machine, that you could just have a
liberal culture instead of a Protestant culture, and all the important
questions could be worked out through reasoned arguments that required
no theological priors, no Bible-bothering, no authority higher than the
Supreme Court or capital-S Science.

This was a naïve view, and to the extent it was actually operationalized
it generated an arid, soulless liberalism, a meritocracy short on wisdom
and memory, animated by unhappy status-seeking and aspiring only to its
own perpetuation.

But there have also been attempts to replace the Mainline, to infuse a
different deeply felt religious faith into the architecture of American
society. The first was the alliance between conservative Catholics and
evangelicals, the ecumenical ``religious right'' that rose with Ronald
Reagan and peaked with George W. Bush. Its more sophisticated leaders
were very conscious about their ambitions: They imagined themselves to
be forging, through revival and alliance and conversion, a new religious
center while the old Mainline drifted left. And they were successful
enough to inspire periodic panics among their adversaries, dark warnings
of an incipient theocracy.

In the end, though, they failed: Because they didn't win enough converts
or allies in the elite, because they didn't hold enough of their own
younger generation, because the legacy of racism divided them from
African-American and Hispanic churches, because their opposition to the
sexual revolution placed them too far from the political center, because
the Bush presidency ended in disaster. And in the aftermath of that
failure, it appeared that American religion would be defined by
fragmentation and polarization, by potent heresies and weakened
orthodoxies, with the only meaningful spiritual center occupied by pop
gurus like Oprah and Joel Osteen.

Or at least it appeared that way to me; in 2012 I wrote
\href{https://www.simonandschuster.com/books/Bad-Religion/Ross-Douthat/9781439178331}{an
entire book} on the subject. And my analysis applies pretty well to
conservatism in the age of Trump, where prosperity theology and
religious nationalism have gained at Christian orthodoxy's expense, the
official religious right is a client of a heathen president, and the
evangelical-Catholic alliance is rived into countless warring cliques.

But I may have underestimated a different religious tribe --- the direct
heirs of the Protestant Mainline, the ``post-Protestant'' subjects of
Joseph Bottum's ``An Anxious Age: The Post-Protestant Ethic and the
Spirit of America*,''*
\href{https://www.penguinrandomhouse.com/books/16449/an-anxious-age-by-joseph-bottum/}{a
book I commend} to anyone interested in understanding what is happening
to liberalism right now.

Bottum makes two points of particular relevance to our moment. First, he
argues that the Mainline moral sensibility has survived even as Mainline
metaphysical belief has ebbed, and that you can draw a clear line from
the Social Gospel of the late 19th century to the preoccupations of
social justice movements today.

This point was plausible but somewhat abstract when the book came out in
2014. But the palpable spiritual dimension of so much social justice
activism, before and especially after the George Floyd killing --- the
rhetoric of conversion and confession and self-scrutiny, the iconoclasm
and occasional
\href{https://www.sacbee.com/news/local/article244012732.html}{anti-Catholicism},
the idealization of
\href{https://www.tabletmag.com/sections/news/articles/love-and-the-police}{communities
of virtue} and the accusatory frenzy of online witch hunts --- has made
that religious lineage impossible to ignore.

Second, Bottum stresses that it's more useful to think of the
post-Protestants --- the ``poster children,'' he sometimes calls them
--- as an elect rather than an elite, defined more by their education
and their moral sensibility than by their overt wealth or power. They
are not identical to the managerial elite discerned by
\href{https://americanaffairsjournal.org/2017/05/new-class-war/}{other
theorists} of late-modern class hierarchy; instead, they stand adjacent
and somewhat underneath, as adjuncts, consultants, bureaucrats and
activists --- advisers and petitioners and critics rather than formal
leaders, with more economic precarity and moral zeal than those they
criticize or serve.

This point, too, is particularly useful to understanding the new power
struggle within the liberal upper class. In theological terms, we're
watching the post-Protestant elect wrestle power away from the more
secular elite, which long paid lip service to the creed of social
justice but never really evinced true faith.

And that power, once claimed, could be used the way the old Mainline
used its power: not to replace liberal political forms but to infuse
them with a specific set of moral commitments and to establish the terms
on which important cultural debates are held and settled. Who should
have sex with whom, and under what conditions and constraints? Which
religious ideas should be favored, and which dismissed with prejudice?
What conceptions of the country's past should be promoted? Which visions
of the good life taught in schools? What titles or pronouns should
respectable people use? Just as the old denominations once answered
these questions for Americans, their post-Protestant heirs aspire to
answer them today.

If they succeed where the religious right failed, it will be because
post-Protestantism enjoys an intimate relationship with the American
establishment rather than representing an insurgency of outsider groups,
because centrist failures and Trumpian moral squalor removed rivals from
its path, and because its moral message is better suited to what younger
Americans already believe.

If they fail, it will probably be because of three weaknesses: the
absence of a convincing metaphysics to ground post-Protestantism's
zealous moralism; the difficulty of drawing coherence out of its
multiplicity of causes; and the absence of institutional embodiments
that make for deep loyalty and intergenerational transmission.

My guess right now is that these problems will be fatal in the long run
--- that post-Protestantism will burn brighter than the religious right
as a moralistic flame within the liberal order, but then pretty rapidly
burn out. But whether that guess is right, or whether the last 50 years
were just an interregnum between two very different forms of Protestant
establishment --- well, as to that, God knows.

\emph{The Times is committed to publishing}
\href{https://www.nytimes.com/2019/01/31/opinion/letters/letters-to-editor-new-york-times-women.html}{\emph{a
diversity of letters}} \emph{to the editor. We'd like to hear what you
think about this or any of our articles. Here are some}
\href{https://help.nytimes.com/hc/en-us/articles/115014925288-How-to-submit-a-letter-to-the-editor}{\emph{tips}}\emph{.
And here's our email:}
\href{mailto:letters@nytimes.com}{\emph{letters@nytimes.com}}\emph{.}

\emph{Follow The New York Times Opinion section on}
\href{https://www.facebook.com/nytopinion}{\emph{Facebook}}\emph{,}
\href{http://twitter.com/NYTOpinion}{\emph{Twitter (@NYTOpinion)}}
\emph{and}
\href{https://www.instagram.com/nytopinion/}{\emph{Instagram}}\emph{,
join the Facebook political discussion group,}
\href{https://www.facebook.com/groups/votingwhilefemale/}{\emph{Voting
While Female}}\emph{.}

Advertisement

\protect\hyperlink{after-bottom}{Continue reading the main story}

\hypertarget{site-index}{%
\subsection{Site Index}\label{site-index}}

\hypertarget{site-information-navigation}{%
\subsection{Site Information
Navigation}\label{site-information-navigation}}

\begin{itemize}
\tightlist
\item
  \href{https://help.nytimes.com/hc/en-us/articles/115014792127-Copyright-notice}{©~2020~The
  New York Times Company}
\end{itemize}

\begin{itemize}
\tightlist
\item
  \href{https://www.nytco.com/}{NYTCo}
\item
  \href{https://help.nytimes.com/hc/en-us/articles/115015385887-Contact-Us}{Contact
  Us}
\item
  \href{https://www.nytco.com/careers/}{Work with us}
\item
  \href{https://nytmediakit.com/}{Advertise}
\item
  \href{http://www.tbrandstudio.com/}{T Brand Studio}
\item
  \href{https://www.nytimes.com/privacy/cookie-policy\#how-do-i-manage-trackers}{Your
  Ad Choices}
\item
  \href{https://www.nytimes.com/privacy}{Privacy}
\item
  \href{https://help.nytimes.com/hc/en-us/articles/115014893428-Terms-of-service}{Terms
  of Service}
\item
  \href{https://help.nytimes.com/hc/en-us/articles/115014893968-Terms-of-sale}{Terms
  of Sale}
\item
  \href{https://spiderbites.nytimes.com}{Site Map}
\item
  \href{https://help.nytimes.com/hc/en-us}{Help}
\item
  \href{https://www.nytimes.com/subscription?campaignId=37WXW}{Subscriptions}
\end{itemize}
