Sections

SEARCH

\protect\hyperlink{site-content}{Skip to
content}\protect\hyperlink{site-index}{Skip to site index}

\href{https://myaccount.nytimes.com/auth/login?response_type=cookie\&client_id=vi}{}

\href{https://www.nytimes.com/section/todayspaper}{Today's Paper}

\href{/section/opinion}{Opinion}\textbar{}Maybe This Isn't Such a Good
Time to Prosecute a Culture War

\href{https://nyti.ms/2DeKoWF}{https://nyti.ms/2DeKoWF}

\begin{itemize}
\item
\item
\item
\item
\item
\item
\end{itemize}

Advertisement

\protect\hyperlink{after-top}{Continue reading the main story}

\href{/section/opinion}{Opinion}

Supported by

\protect\hyperlink{after-sponsor}{Continue reading the main story}

\hypertarget{maybe-this-isnt-such-a-good-time-to-prosecute-a-culture-war}{%
\section{Maybe This Isn't Such a Good Time to Prosecute a Culture
War}\label{maybe-this-isnt-such-a-good-time-to-prosecute-a-culture-war}}

Trump has gone to the well one time too many.

\href{https://www.nytimes.com/column/jamelle-bouie}{\includegraphics{https://static01.nyt.com/images/2019/01/24/opinion/jamelle-bouie/jamelle-bouie-thumbLarge-v3.png}}

By \href{https://www.nytimes.com/column/jamelle-bouie}{Jamelle Bouie}

Opinion Columnist

\begin{itemize}
\item
  July 7, 2020
\item
  \begin{itemize}
  \item
  \item
  \item
  \item
  \item
  \item
  \end{itemize}
\end{itemize}

\includegraphics{https://static01.nyt.com/images/2020/07/07/opinion/07bouie1/merlin_173773299_951c6a7b-fd35-42c8-872b-3944a0870554-articleLarge.jpg?quality=75\&auto=webp\&disable=upscale}

Donald Trump made his name in Republican Party politics as a
``birther,'' a true believer in --- and an evangelist for --- the racist
conspiracy theory that Barack Obama was a foreign-born, illegitimate
president. Having stoked a wave of white grievance and resentment, Trump
rode it, first to influence --- let's not forget that Mitt Romney came
to receive Trump's \href{https://youtu.be/nmwzGMmGcJw}{endorsement} in
person during the 2012 presidential race --- and then to the summit of
power as president himself.

Now, because of a pandemic Trump refuses to address (``We need to live
with it,''
\href{https://www.nbcnews.com/politics/politics-news/we-need-live-it-white-house-readies-new-message-nation-n1232884}{officials}
in his administration say), his power is at risk. If the election were
held today, Trump would almost certainly lose in a landslide. His sole
good fortune at the moment is that the election won't be held for
another four months, giving him time to close his
\href{https://projects.fivethirtyeight.com/polls/}{10-point gap} with
Joe Biden and turn his campaign around.

But to do that, Trump would have to take responsibility for and respond
to events properly. He would have to show the voting public that he is
capable of presidential leadership. And this, more than anything, is
beyond both his interest and his ability. Trump does not want to govern
and could not do it if he tried.

Instead, as he sees it, the path to re-election lies with the instincts
that brought him to power in the first place. With enough racist
demagogy, Trump seems to think, he'll close the gap with Biden and eke
out another win in the Electoral College. But it is one thing to run a
backlash campaign, as Trump did four years ago, in a growing economy in
which most people aren't acutely worried about their lives and futures.
In that environment, where material needs are mostly met, voters can
afford to either look past racial animus or embrace it as a kind of
luxury political good. When conditions are on the decline, however, they
want actual solutions, and the politics of resentment are, by
themselves, a much harder sell.

Not that Trump isn't trying. In just the last two weeks, he has
\href{https://www.latimes.com/world-nation/story/2020-06-28/trump-tweets-video-with-white-power-chant-then-deletes-it}{re-tweeted}
a video of a supporter in Florida shouting ``white power,''
\href{https://www.politico.com/news/2020/07/01/trump-hud-fair-housing-rule-346996}{threatened}
to scrap an Obama-era fair housing rule meant to break patterns of
segregation (citing its ``devastating impact'' on suburbs), promised to
\href{https://www.npr.org/2020/07/01/885944809/trump-vows-to-veto-defense-bill-if-it-removes-confederate-names-from-military-ba}{veto}
a defense funding bill that would also take the names of Confederate
generals off military bases, and
\href{https://twitter.com/realdonaldtrump/status/1278324680311681024?s=21}{called}
New York City's decision to paint ``Black Lives Matter'' on Fifth Avenue
a ``symbol of hate'' that was ``denigrating'' to this ``luxury avenue.''

Rather than use the Independence Day weekend to make a plea for national
unity --- the usual election-year approach for an incumbent --- Trump
took the holiday as an opportunity to
\href{https://www.whitehouse.gov/briefings-statements/remarks-president-trump-south-dakotas-2020-mount-rushmore-fireworks-celebration-keystone-south-dakota/}{excoriate}
the millions of Americans protesting for racial justice as ``evil''
heralds of a new ``far-left fascism'' who seek ``the end of America.''
Speaking underneath Mount Rushmore on July 3, Trump warned that ``Our
nation is witnessing a merciless campaign to wipe out our history,
defame our heroes, erase our values and indoctrinate our children.''

Trump continued along these lines on Monday with an attack on Bubba
Wallace, the only Black full-time driver in NASCAR. ``Has @BubbaWallace
apologized to all of those great NASCAR drivers \& officials who came to
his aid, stood by his side, \& were willing to sacrifice everything for
him, only to find out that the whole thing was just another HOAX?'', the
president
\href{https://twitter.com/realdonaldtrump/status/1280117571874951170?s=21}{wrote
on Twitter}. ``That \& Flag decision has caused lowest ratings EVER!''

Wallace was one of the leading voices in NASCAR arguing for removing the
Confederate flag from events and banning it from the stands. When a
member of his team discovered a noose in Wallace's stall at Talladega
Superspeedway in Alabama, NASCAR launched an investigation, concluding
that the noose had been in the stall since October of last year. Some
observers, particularly those hostile to the Confederate flag ban,
decided that this meant the noose was a hoax. But NASCAR officials
rejected this view. ``Bubba Wallace and the 43 team had nothing to do
with this,'' Steve Phelps, the president of NASCAR,
\href{https://ftw.usatoday.com/2020/06/nascar-bubba-wallace-noose-talladega-investigation-steve-phelps}{said}.
``Bubba Wallace has done nothing but represent this sport with courage,
class and dignity.''

If conditions now were like those in January --- if unemployment was
still low and there wasn't mass unrest and a deadly pandemic wasn't
continuing to rage out of control --- then the president's rhetoric
might actually work to mobilize his supporters. Part of the story of the
2016 election was the movement, into the Republican coalition,
\href{https://www.nytimes.com/2019/03/29/opinion/sunday/trump-obamacare.html}{of
cross-pressured voters} who opposed conservative anti-government
ideology but were also repelled by immigration, Islam and racial
liberalism. Trump appealed to these voters by pledging support for
policies like Social Security and Medicare while also demonizing racial
and religious minorities.

But just as important as his message was the overall condition of the
economy. It wasn't perfect, but it was good enough. Unemployment was
down, growth was steady and wages were up. The economy wasn't on the
back burner, but it also wasn't the most salient issue of the election.
This gave a candidate like Trump the political space to bring other
issues to the fore. And he took it.

It is possible that Trump would have succeeded under worse economic
conditions; that a crashing economy would have made those
cross-pressured voters even more eager to support a racist, demagogic
candidate. We have something of a comparison point in the 2008 election,
when Sarah Palin brought Trump-like energy to the Republican
presidential ticket, nearly eclipsing John McCain, the presidential
nominee. She drew huge crowds with furious denunciations of Obama that
\href{https://radio.wosu.org/post/how-sarah-palin-paved-way-donald-trump\#stream/0}{centered
on} a sense of him as foreign and un-American. ``I am just so fearful
that this is not a man who sees America the way you and I see America,''
\href{https://books.google.com/books?id=7eKrQrTafusC\&lpg=PA152\&ots=MEzOQJgKyM\&dq=\%22not\%20a\%20man\%20who\%20sees\%20america\%20the\%20way\%20you\%20and\%20i\%20see\%20america\%22\&pg=PA152\#v=onepage\&q=\%22not\%20a\%20man\%20who\%20sees\%20america\%20the\%20way\%20you\%20and\%20i\%20see\%20america\%22\&f=false}{Palin
told} a nearly all-white crowd of supporters a month before the
election.

And yet the kinds of voters Palin tried to appeal to --- the kinds of
voters who would eventually back Trump --- stayed, for the most part,
within the Democratic fold that year. They may have been uncomfortable
with the idea of a Black president, but they were outright opposed to
another four years of Republican economic policy.

Or consider George Wallace, whose politics of cultural rage and racial
resentment resonated with voters at a moment, the late 1960s, of
relative security and prosperity, not decline and desperation. It's not
that demagogues never triumph in bad economic conditions, but that good
times may bring some voters to feel that they can afford to vote their
resentments.

If this is true --- if it takes a decent economy to make voters
conducive to the campaign Trump wants to run --- then he is, at this
moment, speeding down an electoral dead-end. As long as Covid-19 is out
of control, as long as there is mass suffering, sickness and economic
distress, then nothing short of actually doing his job will help him get
ahead. There simply is no substitute for good governance.

Trump can spend the next four months raging against protesters,
defending Confederate monuments and attacking Black celebrities. He can
play the hits for his supporters and whip his most devoted followers
into a frenzy of MAGA enthusiasm. He can turn up the racism dial as much
as he wants and as far as it will go. But if he's looking for approval,
he won't get it.

\emph{The Times is committed to publishing}
\href{https://www.nytimes.com/2019/01/31/opinion/letters/letters-to-editor-new-york-times-women.html}{\emph{a
diversity of letters}} \emph{to the editor. We'd like to hear what you
think about this or any of our articles. Here are some}
\href{https://help.nytimes.com/hc/en-us/articles/115014925288-How-to-submit-a-letter-to-the-editor}{\emph{tips}}\emph{.
And here's our email:}
\href{mailto:letters@nytimes.com}{\emph{letters@nytimes.com}}\emph{.}

\emph{Follow The New York Times Opinion section on}
\href{https://www.facebook.com/nytopinion}{\emph{Facebook}}\emph{,}
\href{http://twitter.com/NYTOpinion}{\emph{Twitter (@NYTopinion)}}
\emph{and}
\href{https://www.instagram.com/nytopinion/}{\emph{Instagram}}\emph{.}

Advertisement

\protect\hyperlink{after-bottom}{Continue reading the main story}

\hypertarget{site-index}{%
\subsection{Site Index}\label{site-index}}

\hypertarget{site-information-navigation}{%
\subsection{Site Information
Navigation}\label{site-information-navigation}}

\begin{itemize}
\tightlist
\item
  \href{https://help.nytimes.com/hc/en-us/articles/115014792127-Copyright-notice}{©~2020~The
  New York Times Company}
\end{itemize}

\begin{itemize}
\tightlist
\item
  \href{https://www.nytco.com/}{NYTCo}
\item
  \href{https://help.nytimes.com/hc/en-us/articles/115015385887-Contact-Us}{Contact
  Us}
\item
  \href{https://www.nytco.com/careers/}{Work with us}
\item
  \href{https://nytmediakit.com/}{Advertise}
\item
  \href{http://www.tbrandstudio.com/}{T Brand Studio}
\item
  \href{https://www.nytimes.com/privacy/cookie-policy\#how-do-i-manage-trackers}{Your
  Ad Choices}
\item
  \href{https://www.nytimes.com/privacy}{Privacy}
\item
  \href{https://help.nytimes.com/hc/en-us/articles/115014893428-Terms-of-service}{Terms
  of Service}
\item
  \href{https://help.nytimes.com/hc/en-us/articles/115014893968-Terms-of-sale}{Terms
  of Sale}
\item
  \href{https://spiderbites.nytimes.com}{Site Map}
\item
  \href{https://help.nytimes.com/hc/en-us}{Help}
\item
  \href{https://www.nytimes.com/subscription?campaignId=37WXW}{Subscriptions}
\end{itemize}
