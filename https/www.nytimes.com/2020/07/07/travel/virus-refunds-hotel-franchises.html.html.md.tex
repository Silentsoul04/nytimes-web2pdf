Sections

SEARCH

\protect\hyperlink{site-content}{Skip to
content}\protect\hyperlink{site-index}{Skip to site index}

\href{https://www.nytimes.com/section/travel}{Travel}

\href{https://myaccount.nytimes.com/auth/login?response_type=cookie\&client_id=vi}{}

\href{https://www.nytimes.com/section/todayspaper}{Today's Paper}

\href{/section/travel}{Travel}\textbar{}Help! I'm Owed a Refund, But the
Hotel Owner Refuses to Budge

\url{https://nyti.ms/3gzHUAK}

\begin{itemize}
\item
\item
\item
\item
\item
\item
\end{itemize}

\href{https://www.nytimes.com/news-event/coronavirus?action=click\&pgtype=Article\&state=default\&region=TOP_BANNER\&context=storylines_menu}{The
Coronavirus Outbreak}

\begin{itemize}
\tightlist
\item
  live\href{https://www.nytimes.com/2020/08/01/world/coronavirus-covid-19.html?action=click\&pgtype=Article\&state=default\&region=TOP_BANNER\&context=storylines_menu}{Latest
  Updates}
\item
  \href{https://www.nytimes.com/interactive/2020/us/coronavirus-us-cases.html?action=click\&pgtype=Article\&state=default\&region=TOP_BANNER\&context=storylines_menu}{Maps
  and Cases}
\item
  \href{https://www.nytimes.com/interactive/2020/science/coronavirus-vaccine-tracker.html?action=click\&pgtype=Article\&state=default\&region=TOP_BANNER\&context=storylines_menu}{Vaccine
  Tracker}
\item
  \href{https://www.nytimes.com/interactive/2020/07/29/us/schools-reopening-coronavirus.html?action=click\&pgtype=Article\&state=default\&region=TOP_BANNER\&context=storylines_menu}{What
  School May Look Like}
\item
  \href{https://www.nytimes.com/live/2020/07/31/business/stock-market-today-coronavirus?action=click\&pgtype=Article\&state=default\&region=TOP_BANNER\&context=storylines_menu}{Economy}
\end{itemize}

Advertisement

\protect\hyperlink{after-top}{Continue reading the main story}

Supported by

\protect\hyperlink{after-sponsor}{Continue reading the main story}

Tripped Up

\hypertarget{help-im-owed-a-refund-but-the-hotel-owner-refuses-to-budge}{%
\section{Help! I'm Owed a Refund, But the Hotel Owner Refuses to
Budge}\label{help-im-owed-a-refund-but-the-hotel-owner-refuses-to-budge}}

Here we are, wondering aloud about the oversight capabilities of hotel
franchises, and what powers they can exert over their thousands of
individual owners. Thanks Covid-19.

\includegraphics{https://static01.nyt.com/images/2020/07/13/travel/13trippedup-sonoma-hotel-rev/07trippedup-sonoma-hotel-rev-articleLarge.jpg?quality=75\&auto=webp\&disable=upscale}

By Sarah Firshein

\begin{itemize}
\item
  July 7, 2020
\item
  \begin{itemize}
  \item
  \item
  \item
  \item
  \item
  \item
  \end{itemize}
\end{itemize}

\hypertarget{dear-tripped-up}{%
\subsubsection{Dear Tripped Up,}\label{dear-tripped-up}}

My wife and I were supposed to attend a wedding in Sonoma this spring,
but the affair was obviously canceled. I canceled our Best Western
reservation well before the 24 hours required by the company's new
coronavirus
\href{https://www.bestwestern.com/en_US/notice/covid-19-response-cancel-policy.html}{cancellation
policy}, but was told that the only option is to postpone our stay for a
year --- as if I jet-set out to wine country every year for a wedding
(and besides, I'm more of a beer guy). I've been messaging back and
forth with Best Western on Facebook; the customer-service people said
the hitch lies with the hotel owner, who is refusing to issue the
refund. What's the point of flexible corporate cancellation policies if
individual hotels aren't required to adhere to them? George

\hypertarget{hi-george}{%
\subsubsection{Hi George,}\label{hi-george}}

The pandemic has brought to light some nerdy issues about the travel
industry --- ones that most of us never needed to think about before.
But here we are, wondering aloud about the oversight capabilities of
hotel franchises, and what powers they can exert over their thousands of
individual owners.

I got several emails about this topic. Nick, another reader, faced a
nearly identical uphill battle while trying to cancel a June reservation
at the Hilton Rome Airport. Hilton's
\href{https://help.hilton.com/s/article/Change-cancellation-policy-extended-to-August-31}{stated
policies} for refunds are flexible, yet the individual hotel owner
refuses to budge.

For starters, the answer is yes: Whenever a company like Best Western or
Hilton announces a flexible corporate cancellation policy, it's meant to
apply to all branded hotels. But Best Western also
\href{https://www.bestwestern.com/en_US/notice/covid-19-response-cancel-policy.html}{states
outright} that a small percentage of individual owners may be eligible
to deny refunds on a small percentage of bookings. And that's what
happened here.

A spokeswoman for Best Western said in a statement that ``During the
COVID-19 pandemic, Best Western Hotels \& Resorts has offered a flexible
cancellation policy to its valued guests. This policy includes that `a
more restrictive cancellation policy may apply to a limited number of
high-demand dates at individual hotels,' which was applicable to this
guest's reservation.''

After reaching out to Hilton, I learned that Nick got trapped by an even
more peculiar loophole --- a bizarre wrench in the pandemic's
ever-expanding toolbox of bizarre wrenches. On March 17, Italy passed
the ``Cura Italia'' decree, a relief measure meant to offset the
economic toll of the pandemic in one of the hardest-hit countries. The
new law gave hotel owners in Italy the option to make the call on
whether to issue refunds or vouchers.

As I've reported before, cash-strapped travel companies have
\href{https://www.nytimes.com/2020/05/12/travel/refunds-or-credits-travelers-and-businesses-face-off.html}{numerous
reasons for retaining non-refundable payments} in the age of the
coronavirus, and it's also not hard to understand why the owner of hotel
in Italy --- at an airport, no less --- would choose that option,
especially when expressly given the greenlight by the Italian
government. Another reader bemoaned a similar issue, also with an
Italian hotel. ``Why is an American who never set foot on Italian soil
subject to a new Italian decree?'' she wondered.

In general, though, hotels have generally been better about
Covid-related cash refunds than airlines, tour operators and cruise
lines. In mid-March, as the world started shutting down, every major
hotel company announced newly flexible cancellation terms, even for
``non-refundable'' or ``advance purchase'' reservations. And even now,
as we move into summer, hotel policies remain pretty flexible. Hilton,
for example, allows guests to cancel any reservation booked through
August without penalty, so long as it's done so at least 24 hours before
the arrival date. It's a strategic move meant to get people to take a
leap, plan travel, book trips.

Hilton's communications department has worked its magic --- a
spokeswoman for the brand has confirmed that Nick's refund is in
process. And in a followup email, you told me that Best Western's
Facebook customer-service team has offered you a gift card, which,
unlike a rebooked stay, can be used at any of the company's hotels.

\href{https://www.nytimes.com/news-event/coronavirus?action=click\&pgtype=Article\&state=default\&region=MAIN_CONTENT_3\&context=storylines_faq}{}

\hypertarget{the-coronavirus-outbreak-}{%
\subsubsection{The Coronavirus Outbreak
›}\label{the-coronavirus-outbreak-}}

\hypertarget{frequently-asked-questions}{%
\paragraph{Frequently Asked
Questions}\label{frequently-asked-questions}}

Updated July 27, 2020

\begin{itemize}
\item ~
  \hypertarget{should-i-refinance-my-mortgage}{%
  \paragraph{Should I refinance my
  mortgage?}\label{should-i-refinance-my-mortgage}}

  \begin{itemize}
  \tightlist
  \item
    \href{https://www.nytimes.com/article/coronavirus-money-unemployment.html?action=click\&pgtype=Article\&state=default\&region=MAIN_CONTENT_3\&context=storylines_faq}{It
    could be a good idea,} because mortgage rates have
    \href{https://www.nytimes.com/2020/07/16/business/mortgage-rates-below-3-percent.html?action=click\&pgtype=Article\&state=default\&region=MAIN_CONTENT_3\&context=storylines_faq}{never
    been lower.} Refinancing requests have pushed mortgage applications
    to some of the highest levels since 2008, so be prepared to get in
    line. But defaults are also up, so if you're thinking about buying a
    home, be aware that some lenders have tightened their standards.
  \end{itemize}
\item ~
  \hypertarget{what-is-school-going-to-look-like-in-september}{%
  \paragraph{What is school going to look like in
  September?}\label{what-is-school-going-to-look-like-in-september}}

  \begin{itemize}
  \tightlist
  \item
    It is unlikely that many schools will return to a normal schedule
    this fall, requiring the grind of
    \href{https://www.nytimes.com/2020/06/05/us/coronavirus-education-lost-learning.html?action=click\&pgtype=Article\&state=default\&region=MAIN_CONTENT_3\&context=storylines_faq}{online
    learning},
    \href{https://www.nytimes.com/2020/05/29/us/coronavirus-child-care-centers.html?action=click\&pgtype=Article\&state=default\&region=MAIN_CONTENT_3\&context=storylines_faq}{makeshift
    child care} and
    \href{https://www.nytimes.com/2020/06/03/business/economy/coronavirus-working-women.html?action=click\&pgtype=Article\&state=default\&region=MAIN_CONTENT_3\&context=storylines_faq}{stunted
    workdays} to continue. California's two largest public school
    districts --- Los Angeles and San Diego --- said on July 13, that
    \href{https://www.nytimes.com/2020/07/13/us/lausd-san-diego-school-reopening.html?action=click\&pgtype=Article\&state=default\&region=MAIN_CONTENT_3\&context=storylines_faq}{instruction
    will be remote-only in the fall}, citing concerns that surging
    coronavirus infections in their areas pose too dire a risk for
    students and teachers. Together, the two districts enroll some
    825,000 students. They are the largest in the country so far to
    abandon plans for even a partial physical return to classrooms when
    they reopen in August. For other districts, the solution won't be an
    all-or-nothing approach.
    \href{https://bioethics.jhu.edu/research-and-outreach/projects/eschool-initiative/school-policy-tracker/}{Many
    systems}, including the nation's largest, New York City, are
    devising
    \href{https://www.nytimes.com/2020/06/26/us/coronavirus-schools-reopen-fall.html?action=click\&pgtype=Article\&state=default\&region=MAIN_CONTENT_3\&context=storylines_faq}{hybrid
    plans} that involve spending some days in classrooms and other days
    online. There's no national policy on this yet, so check with your
    municipal school system regularly to see what is happening in your
    community.
  \end{itemize}
\item ~
  \hypertarget{is-the-coronavirus-airborne}{%
  \paragraph{Is the coronavirus
  airborne?}\label{is-the-coronavirus-airborne}}

  \begin{itemize}
  \tightlist
  \item
    The coronavirus
    \href{https://www.nytimes.com/2020/07/04/health/239-experts-with-one-big-claim-the-coronavirus-is-airborne.html?action=click\&pgtype=Article\&state=default\&region=MAIN_CONTENT_3\&context=storylines_faq}{can
    stay aloft for hours in tiny droplets in stagnant air}, infecting
    people as they inhale, mounting scientific evidence suggests. This
    risk is highest in crowded indoor spaces with poor ventilation, and
    may help explain super-spreading events reported in meatpacking
    plants, churches and restaurants.
    \href{https://www.nytimes.com/2020/07/06/health/coronavirus-airborne-aerosols.html?action=click\&pgtype=Article\&state=default\&region=MAIN_CONTENT_3\&context=storylines_faq}{It's
    unclear how often the virus is spread} via these tiny droplets, or
    aerosols, compared with larger droplets that are expelled when a
    sick person coughs or sneezes, or transmitted through contact with
    contaminated surfaces, said Linsey Marr, an aerosol expert at
    Virginia Tech. Aerosols are released even when a person without
    symptoms exhales, talks or sings, according to Dr. Marr and more
    than 200 other experts, who
    \href{https://academic.oup.com/cid/article/doi/10.1093/cid/ciaa939/5867798}{have
    outlined the evidence in an open letter to the World Health
    Organization}.
  \end{itemize}
\item ~
  \hypertarget{what-are-the-symptoms-of-coronavirus}{%
  \paragraph{What are the symptoms of
  coronavirus?}\label{what-are-the-symptoms-of-coronavirus}}

  \begin{itemize}
  \tightlist
  \item
    Common symptoms
    \href{https://www.nytimes.com/article/symptoms-coronavirus.html?action=click\&pgtype=Article\&state=default\&region=MAIN_CONTENT_3\&context=storylines_faq}{include
    fever, a dry cough, fatigue and difficulty breathing or shortness of
    breath.} Some of these symptoms overlap with those of the flu,
    making detection difficult, but runny noses and stuffy sinuses are
    less common.
    \href{https://www.nytimes.com/2020/04/27/health/coronavirus-symptoms-cdc.html?action=click\&pgtype=Article\&state=default\&region=MAIN_CONTENT_3\&context=storylines_faq}{The
    C.D.C. has also} added chills, muscle pain, sore throat, headache
    and a new loss of the sense of taste or smell as symptoms to look
    out for. Most people fall ill five to seven days after exposure, but
    symptoms may appear in as few as two days or as many as 14 days.
  \end{itemize}
\item ~
  \hypertarget{does-asymptomatic-transmission-of-covid-19-happen}{%
  \paragraph{Does asymptomatic transmission of Covid-19
  happen?}\label{does-asymptomatic-transmission-of-covid-19-happen}}

  \begin{itemize}
  \tightlist
  \item
    So far, the evidence seems to show it does. A widely cited
    \href{https://www.nature.com/articles/s41591-020-0869-5}{paper}
    published in April suggests that people are most infectious about
    two days before the onset of coronavirus symptoms and estimated that
    44 percent of new infections were a result of transmission from
    people who were not yet showing symptoms. Recently, a top expert at
    the World Health Organization stated that transmission of the
    coronavirus by people who did not have symptoms was ``very rare,''
    \href{https://www.nytimes.com/2020/06/09/world/coronavirus-updates.html?action=click\&pgtype=Article\&state=default\&region=MAIN_CONTENT_3\&context=storylines_faq\#link-1f302e21}{but
    she later walked back that statement.}
  \end{itemize}
\end{itemize}

I realize it's not quite the same as a cash refund, but amid an era when
planning (and canceling) travel feels especially difficult and
impersonal, measures like these can feel like a welcome human touch.
Whatever you're drinking --- be it beer or wine --- I'll say cheers to
that.

\emph{The}
\href{https://www.nytimes.com/2020/05/25/travel/coronavirus-refunds-overseas-adventure-travel.html}{\emph{May
25 edition}} \emph{of Tripped Up, about widespread refund issues with a
Boston-based tour operator, drew hundreds of reader responses. One, from
a woman named Lisa, stuck with me: ``I love to travel and am very aware
that many travel services companies are fighting for their lives,'' she
wrote. ``But as a customer, I am increasingly concerned about
shouldering the bankruptcy risk of the tour operator. If they go
bankrupt, those vouchers are worthless. It's a real dilemma, how to be
supportive of travel companies yet not end up losing thousands.''}

\begin{center}\rule{0.5\linewidth}{\linethickness}\end{center}

\href{https://twitter.com/sfirshein?lang=en}{Sarah Firshein} is a
Brooklyn-based writer. If you need advice about a best-laid travel plan
that went awry, \textbf{\href{mailto:travel@nytimes.com}{send an email
to travel@nytimes.com}.}

\begin{center}\rule{0.5\linewidth}{\linethickness}\end{center}

\emph{\textbf{Follow New York Times Travel}}
\emph{on}\href{https://www.instagram.com/nytimestravel/}{\emph{Instagram}}\emph{,}\href{https://twitter.com/nytimestravel}{\emph{Twitter}}
\emph{and}\href{https://www.facebook.com/nytimestravel/}{\emph{Facebook}}\emph{.
And}\href{https://www.nytimes.com/newsletters/traveldispatch}{\emph{sign
up for our weekly Travel Dispatch newsletter}} \emph{to receive expert
tips on traveling smarter and inspiration for your next vacation.}

Advertisement

\protect\hyperlink{after-bottom}{Continue reading the main story}

\hypertarget{site-index}{%
\subsection{Site Index}\label{site-index}}

\hypertarget{site-information-navigation}{%
\subsection{Site Information
Navigation}\label{site-information-navigation}}

\begin{itemize}
\tightlist
\item
  \href{https://help.nytimes.com/hc/en-us/articles/115014792127-Copyright-notice}{©~2020~The
  New York Times Company}
\end{itemize}

\begin{itemize}
\tightlist
\item
  \href{https://www.nytco.com/}{NYTCo}
\item
  \href{https://help.nytimes.com/hc/en-us/articles/115015385887-Contact-Us}{Contact
  Us}
\item
  \href{https://www.nytco.com/careers/}{Work with us}
\item
  \href{https://nytmediakit.com/}{Advertise}
\item
  \href{http://www.tbrandstudio.com/}{T Brand Studio}
\item
  \href{https://www.nytimes.com/privacy/cookie-policy\#how-do-i-manage-trackers}{Your
  Ad Choices}
\item
  \href{https://www.nytimes.com/privacy}{Privacy}
\item
  \href{https://help.nytimes.com/hc/en-us/articles/115014893428-Terms-of-service}{Terms
  of Service}
\item
  \href{https://help.nytimes.com/hc/en-us/articles/115014893968-Terms-of-sale}{Terms
  of Sale}
\item
  \href{https://spiderbites.nytimes.com}{Site Map}
\item
  \href{https://help.nytimes.com/hc/en-us}{Help}
\item
  \href{https://www.nytimes.com/subscription?campaignId=37WXW}{Subscriptions}
\end{itemize}
