Sections

SEARCH

\protect\hyperlink{site-content}{Skip to
content}\protect\hyperlink{site-index}{Skip to site index}

\href{https://myaccount.nytimes.com/auth/login?response_type=cookie\&client_id=vi}{}

\href{https://www.nytimes.com/section/todayspaper}{Today's Paper}

\href{/section/opinion}{Opinion}\textbar{}Sex, Sisters and Dr. Donald

\href{https://nyti.ms/2BUBL3l}{https://nyti.ms/2BUBL3l}

\begin{itemize}
\item
\item
\item
\item
\item
\item
\end{itemize}

Advertisement

\protect\hyperlink{after-top}{Continue reading the main story}

\href{/section/opinion}{Opinion}

Supported by

\protect\hyperlink{after-sponsor}{Continue reading the main story}

\hypertarget{sex-sisters-and-dr-donald}{%
\section{Sex, Sisters and Dr. Donald}\label{sex-sisters-and-dr-donald}}

Who wants to trust their reproductive life to Trump?

\href{https://www.nytimes.com/by/gail-collins}{\includegraphics{https://static01.nyt.com/images/2018/04/03/opinion/gail-collins/gail-collins-thumbLarge.png}}

By \href{https://www.nytimes.com/by/gail-collins}{Gail Collins}

Opinion Columnist

\begin{itemize}
\item
  July 8, 2020
\item
  \begin{itemize}
  \item
  \item
  \item
  \item
  \item
  \item
  \end{itemize}
\end{itemize}

\includegraphics{https://static01.nyt.com/images/2020/07/08/opinion/08collins1/merlin_174349986_51a3375c-57ee-49fd-90af-80e1aadfaaad-articleLarge.jpg?quality=75\&auto=webp\&disable=upscale}

Let's pretend there was an order of nuns with a particular devotion to
the Sacred Heart of Jesus. So much so that the order had, over the
years, decided that any human heart was a holy symbol, and it was
immoral to mess with it, even if you were a physician doing cardiac
surgery.

Following their consciences, these nuns banned heart-related care from
their employees' health policies. That affected thousands of workers,
many of whom did not share their religious convictions. Still, the nuns
noted, their insurance coverage was generous. Except for that one thing.

I suspect you know what I'm setting this up for. The Little Sisters of
the Poor have
\href{https://www.nytimes.com/2020/07/08/us/supreme-court-birth-control-obamacare.html}{won
the latest Supreme Court battle} over contraception. The justices said
they have the right to refuse to include birth control in their
insurance policies. Actually, that was always the case. Under Obama-era
regulations, the federal government took care of the issue when
religious groups had ethical objections.

But the nuns didn't want to let the government know what they weren't
doing. That counted as aiding and abetting the enemy, so they dug in
their heels. No paperwork, no passing along information. And the Trump
administration was happy to help them with the fight. Now, other
employers with religious scruples or simply a yen to save money will
leap on the bandwagon. An estimated 70,000 to 126,000 women will lose
their current free contraceptive coverage.

You have to admit the anticontraception forces were brilliant to get the
Little Sisters of the Poor as their star in court. It sounds a heck of a
lot more sympathetic than the other part of the same decision, Trump v.
Pennsylvania. Or almost any other religious institution. When I was
growing up I went to St. Antoninus Catholic school and I'm sure the nuns
there would have been happy to lend a hand to the anti-birth control
fight --- if anyone wanted a case named after a 15th-century archbishop
of Florence.

``Our life's work and great joy is serving the elderly poor, and we are
so grateful that the contraceptive mandate will no longer steal our
attention from our calling,'' said Mother Loraine Marie Maguire of the
Little Sisters.

``This is not over,'' said Alexis McGill Johnson, the head of Planned
Parenthood.

Well, true that.

When it came to reproduction rights, nobody really knew where Donald
Trump would be going as president. During the campaign, he was asked if
he would be willing to shut down the government to defund Planned
Parenthood. He refused to answer ``because I want to show
unpredictability.''

Yes, Trump went with his strong suit --- nobody really knew what his
principles were. He was pretty clear on abortion --- the religious
right's position ruled. But once he was elected, birth control services
were hit hard, too. The new administration got
\href{https://www.nytimes.com/2017/10/06/us/politics/trump-contraception-birth-control.html}{right
into the fray in 2017}, announcing it was going to let a much wider
range of employers off the hook if they didn't want to cover
contraception in their company health plans.

That was a memorable moment, since it came at about the same time the
House of Representatives passed a bill banning late-term abortions.
Meanwhile, one of said bill's co-sponsors
\href{https://www.nytimes.com/2017/10/05/us/politics/tim-murphy-resigns-abortion-scandal.html}{announced
he was resigning from office} after word came that he had urged his
lover to terminate her pregnancy.

The moral here is that reproduction issues are both very political and
very personal. The fighting has been going on ever since. It's hurt a
lot of women who rely on low-cost services supported by government aid.
Groups like Planned Parenthood
\href{https://www.nytimes.com/2019/08/19/health/planned-parenthood-title-x.html}{refused
to cooperate} with the Trump rule that prohibited doctors from giving
their patients information on abortion availability. A lot of nonprofits
fell by the wayside.
\href{https://www.guttmacher.org/article/2020/02/trump-administrations-domestic-gag-rule-has-slashed-title-x-networks-capacity-half}{Only
about half as many women} can now use the federal government's Title X
family planning programs.

Even under much better circumstances, it'd be unnerving to think of
entrusting your reproductive future to a president who appears to have
about half the medical sophistication of a Barbie Doctor Doll. You
remember, of course, that this is the guy who claimed that testing had
determined out of all the American coronavirus cases, 99 percent were
\href{https://www.washingtonpost.com/politics/2020/07/08/trumps-claim-that-99-percent-covid-9-cases-are-totally-harmless/}{``totally
harmless.''}

He's never thought this issue through with an eye toward anything but
his base. The bottom line is basically whether women should be able to
have sex without risking pregnancy. There are a lot of people, including
conservative Catholics and evangelicals, who say no. There are a lot
more who think that's one of the keys to living a happy, well-planned
life.

Feel free to guess which side most of the women in Trump's life have
been on. This is a guy who likes being unpredictable himself. But he
seems to prefer a certain amount of self-control when it comes to his
mates.

Most Americans believe women should have the right to terminate a
pregnancy, at least in the early months, but the whole idea makes a lot
of people very uncomfortable. However, the country is, in general, a big
fan of contraception. And easy access to birth control is the key for
keeping the abortion rate low. Basically, the president and the Little
Sisters have struck a big blow for unwanted pregnancies.

\emph{The Times is committed to publishing}
\href{https://www.nytimes.com/2019/01/31/opinion/letters/letters-to-editor-new-york-times-women.html}{\emph{a
diversity of letters}} \emph{to the editor. We'd like to hear what you
think about this or any of our articles. Here are some}
\href{https://help.nytimes.com/hc/en-us/articles/115014925288-How-to-submit-a-letter-to-the-editor}{\emph{tips}}\emph{.
And here's our email:}
\href{mailto:letters@nytimes.com}{\emph{letters@nytimes.com}}\emph{.}

\emph{Follow The New York Times Opinion section on}
\href{https://www.facebook.com/nytopinion}{\emph{Facebook}}\emph{,}
\href{http://twitter.com/NYTOpinion}{\emph{Twitter (@NYTopinion)}}
\emph{and}
\href{https://www.instagram.com/nytopinion/}{\emph{Instagram}}\emph{.}

Advertisement

\protect\hyperlink{after-bottom}{Continue reading the main story}

\hypertarget{site-index}{%
\subsection{Site Index}\label{site-index}}

\hypertarget{site-information-navigation}{%
\subsection{Site Information
Navigation}\label{site-information-navigation}}

\begin{itemize}
\tightlist
\item
  \href{https://help.nytimes.com/hc/en-us/articles/115014792127-Copyright-notice}{©~2020~The
  New York Times Company}
\end{itemize}

\begin{itemize}
\tightlist
\item
  \href{https://www.nytco.com/}{NYTCo}
\item
  \href{https://help.nytimes.com/hc/en-us/articles/115015385887-Contact-Us}{Contact
  Us}
\item
  \href{https://www.nytco.com/careers/}{Work with us}
\item
  \href{https://nytmediakit.com/}{Advertise}
\item
  \href{http://www.tbrandstudio.com/}{T Brand Studio}
\item
  \href{https://www.nytimes.com/privacy/cookie-policy\#how-do-i-manage-trackers}{Your
  Ad Choices}
\item
  \href{https://www.nytimes.com/privacy}{Privacy}
\item
  \href{https://help.nytimes.com/hc/en-us/articles/115014893428-Terms-of-service}{Terms
  of Service}
\item
  \href{https://help.nytimes.com/hc/en-us/articles/115014893968-Terms-of-sale}{Terms
  of Sale}
\item
  \href{https://spiderbites.nytimes.com}{Site Map}
\item
  \href{https://help.nytimes.com/hc/en-us}{Help}
\item
  \href{https://www.nytimes.com/subscription?campaignId=37WXW}{Subscriptions}
\end{itemize}
