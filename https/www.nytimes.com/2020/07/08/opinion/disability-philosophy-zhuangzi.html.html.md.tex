Sections

SEARCH

\protect\hyperlink{site-content}{Skip to
content}\protect\hyperlink{site-index}{Skip to site index}

\href{https://myaccount.nytimes.com/auth/login?response_type=cookie\&client_id=vi}{}

\href{https://www.nytimes.com/section/todayspaper}{Today's Paper}

\href{/section/opinion}{Opinion}\textbar{}Was This Ancient Taoist the
First Philosopher of Disability?

\href{https://nyti.ms/2VX7YOp}{https://nyti.ms/2VX7YOp}

\begin{itemize}
\item
\item
\item
\item
\item
\end{itemize}

Advertisement

\protect\hyperlink{after-top}{Continue reading the main story}

\href{/section/opinion}{Opinion}

Supported by

\protect\hyperlink{after-sponsor}{Continue reading the main story}

THE STONE

\hypertarget{was-this-ancient-taoist-the-first-philosopher-of-disability}{%
\section{Was This Ancient Taoist the First Philosopher of
Disability?}\label{was-this-ancient-taoist-the-first-philosopher-of-disability}}

Zhuangzi pushed back against the idea that ``normal'' is good and
difference is bad 2,500 years ago.

By John Altmann and Bryan W. Van Norden

Mr. Altmann is an essayist and author. Mr.
\href{http://www.bryanvannorden.com/}{Van Norden} teaches philosophy at
Vassar College.

\begin{itemize}
\item
  July 8, 2020
\item
  \begin{itemize}
  \item
  \item
  \item
  \item
  \item
  \end{itemize}
\end{itemize}

\includegraphics{https://static01.nyt.com/images/2020/07/08/opinion/08stone/08stone-articleLarge.jpg?quality=75\&auto=webp\&disable=upscale}

In one of his philosophical parables, the Taoist philosopher Zhuangzi
(fourth century B.C.) describes a man he calls Splay-limb Shu. This
man's ``chin is sunk in his belly,'' Zhuangzi writes. ``His shoulders
are above his head, and pinched together so they point to the sky. His
five organs are on top, his thighs tight against his ribs.'' In
Zhuangzi's era as in our own, most people would consider Splay-limb Shu
to be unfortunate.

But Zhuangzi, whose work frequently challenged society's norms, sees
things differently. He notes, for instance, that Shu is in no danger of
being conscripted into the military or pressed into forced labor.
Instead, he lives contentedly in his community, supporting himself by
``plying a needle and taking in laundry.'' Shu, Zhuangzi concludes, is
``able to keep himself alive and to live out the years Heaven gave him''
precisely because he is different from others.

Even today, this insight is striking. Zhuangzi poses the idea that Shu's
difference --- one we would classify today as a disability --- is not a
misfortune, and in doing so challenges an assumption that has existed in
cultures of all kinds for millenniums.

It is hard to pinpoint where this idea --- that it is inherently bad to
be disabled --- originated, but in the West, examples go as far back as
ancient Greece. The linking of virtue and beauty with ``normality''
appears in Plato's account of Socrates' dialogue with Crito, in which
Socrates asserts that ``the good life, the beautiful life and the just
life are the same'' and that life is not ``worth living with a body that
is corrupted and in bad condition.''

Plato's student Aristotle later argued explicitly in ``Politics'' **
that ``no deformed child should be raised,'' but should instead be left
to die of exposure. Islamic, Jewish and Christian philosophers later
found Aristotle's normative conception of human nature congenial to the
mainstream Abrahamic traditions: The ideal form of the human being
exists in the mind of God, who ``created man in his own image'';
differences or variations from this norm are to be considered deviant.
It is not coincidental that the Bible asserts that one may not become a
priest if they are ``a blind man, or a lame \ldots{} or a man that is
brokenfooted, or brokenhanded, or crookbackt, or a dwarf, or that hath a
blemish in his eye.'' (Leviticus 21:18-20 KJV.)

In the Chinese context, though, Zhuangzi is arguing against a Confucian
conception of ``normality'' that, like Aristotelianism, is teleological:
A higher power, Heaven, decrees what ``human nature'' is, and human
nature determines all the normative facts, such as how many limbs a
human should have, standards of physical beauty, tastes in food and
music, and morality. This view implies that to be ``different'' is to be
defective.

We see the target of Zhuangzi's critique in another passage of his
writing, in which Confucius meets an amputee, Shushan No-Toes. Confucius
is at first dismissive of No-Toes, but then, turning to his own
disciples, condescendingly praises No-Toes for doing so well, despite
his disability. Although it is supposed to have occurred 2,500 years
ago, the pattern of the exchange will be familiar to those labeled
``disabled'' today. (See John Altmann's 2016 essay
``\href{https://www.nytimes.com/2016/10/20/opinion/i-dont-want-to-be-inspiring.html}{I
Don't Want to Be `Inspiring.'}'')

But No-Toes explains that from the perspective of the universe, there is
no real distinction between nondisabled and disabled: ``There is nothing
that heaven doesn't cover, nothing that earth doesn't bear up.'' It is
Confucius, No-Toes suggests, who is really ``disabled'' because of his
inability to see past conventional distinctions. The very concept of
disability, then, is ``socially contingent,'' defined by a society's
limitations, not the true worth of an individual --- an argument found
in the work of several contemporary philosophers of disability,
including \href{https://biopoliticalphilosophy.com/}{Shelley Tremain},
\href{http://www.bioethics.net/2020/03/covid-19-triage-and-disability-what-not-to-do/}{Joe
Stramondo},
\href{https://plato.stanford.edu/entries/disability-critical/}{Melinda
Hall} and \href{http://www.raggededgemagazine.com/0501/0501cov.htm}{Cal
Montgomery}.

Zhuangzi understands virtue as manifested by living in accordance with
nature. Corruption occurs, according to Zhuangzi, only when one deviates
from nature's path. If nature determines that a person has one arm*,*
splayed limbs or a hunched back, the person can embrace these changes
and harmonize with them. As Zhuangzi says, ``Virtue {[}takes{]} no
form.''

Zhuangzi is a creative and flexible author, so it is no surprise that
later in the same work, Confucius is ironically appropriated as the
spokesman of Zhuangzi's own position. This Confucius says he wants to
become the disciple of an amputee, ``Royal Nag,'' because he ``looks at
the way things are one {[}or whole{]} and does not see what they're
missing. He looks at losing a foot like shaking off dust.'' Royal Nag
(and Zhuangzi) saw, long before contemporary epistemologists, that
similarity and difference are standpoint dependent: ``Looked at from
their differences, liver and gall are as far apart as the states of Chu
and Yue. Looked at from their sameness, the ten thousand things are all
one.'' In short, the common assumption that it is ``bad'' to be
``disabled'' makes sense only if we project our parochial and
historically contingent human values onto the fabric of the universe.

One response to this critique would be that disabilities are bad, not
because they are violations of the objective teleological structure of
the universe, but because they are inefficient. Those who are
``disabled'' are simply less functional, less able to achieve their
goals, than those who are ``normal.'' This leads easily to the
conclusion that eliminating disabilities would be better, not just for
society but for the disabled themselves. Contemporary technology seems
to have put this almost in our grasp. With the advent of both genetic
screening technologies and Crispr gene editing, we are approaching an
age in which we may be able to design the human body; perhaps soon the
new normal for the American family will be designer babies. We may be
approaching a world in which illness is eradicated, a world of physical
and mental harmony and homogeneity among all peoples. This, many would
argue, is surely the stuff of a utopia --- a ``brave new world.''

The seductiveness of this argument illustrates the danger of the
hegemony of instrumental reasoning --- reasoning employed to find the
most efficient way to a given goal. It is an important aspect of wisdom,
but it also carries the temptation, especially in modern capitalist
society, to reduce all of rationality to means-end efficiency. In some
cases, means-end efficiency results in an inappropriate and inhuman
standard.

To think that we have moved beyond this pitfall would be nice, but we
haven't. It is still very much with us. As the coronavirus pandemic
began to overwhelm medical capacity in the United States in March, the
disability activist and writer Ari Ne'eman
\href{https://www.nytimes.com/2020/03/23/opinion/coronavirus-ventilators-triage-disability.html}{argued}
that the triage guidelines that certain states were putting into use
indicated that it was preferable to let a disabled person die simply
because it would require more resources to keep that person alive. The
principle of granting equal value of human lives, Ne'eman wrote, would
then be ``sacrificed in the name of efficiency.''

We do not mean, in this brief essay, to dismiss all of philosophy
outside of Zhuangzi. The sayings of Confucius include a passage in which
the master is a respectful and congenial host to a blind music master
(``Analects,'' ** 15.42), and the later Confucian tradition includes the
stirring admonition, ``All under Heaven who are tired, crippled,
exhausted, sick, brotherless, childless, widows or widowers --- all are
my siblings who are helpless and have no one else to appeal to.''
Readers of the New Testament will recognize this as a core value in the
teachings of Jesus. In fact, many figures and institutions in the
Abrahamic traditions have been at the forefront of caring for the
disabled, precisely by appealing to the Platonic view that humans'
ultimate value lies in their immaterial souls rather than their
contingent material embodiments.

But in this time of rampant sickness and social inequality, and given
our fundamental duty to extend equal treatment, compassion and care for
others, we think Zhuangzi is an important and insightful guide, a Taoist
gadfly, if you will, to challenge our conventional notions of
flourishing and health. With the 30th anniversary of the Americans With
Disabilities Act approaching, this ancient Chinese Taoist reminds us
that it is the material conditions of a society that determine and
define disability. We have the power to change both those material
conditions and the definition of disability.

\href{https://independent.academia.edu/JohnAltmann}{John Altmann}
(@iron\_intellect) writes about philosophy for general audiences and is
a contributor to the
\href{http://www.opencourtbooks.com/categories/pcp.htm}{Popular Culture
and Philosophy Series} of books.
\href{http://www.bryanvannorden.com/}{Bryan W. Van Norden}
(@bryanvannorden) holds a chair in philosophy at Vassar College and is
the author most recently of
``\href{http://cup.columbia.edu/book/taking-back-philosophy/9780231184373}{Taking
Back Philosophy: A Multicultural Manifesto}.''

\emph{\textbf{Now in print:}}
\emph{``}\href{http://bitly.com/1MW2kN3}{\emph{Modern Ethics in 77
Arguments}}\emph{'' and ``}\href{http://bitly.com/1MW2kN3}{\emph{The
Stone Reader: Modern Philosophy in 133 Arguments}}\emph{,'' with essays
from the series, edited by Peter Catapano and Simon Critchley, published
by Liveright Books.}

\emph{The Times is committed to publishing}
\href{https://www.nytimes.com/2019/01/31/opinion/letters/letters-to-editor-new-york-times-women.html}{\emph{a
diversity of letters}} \emph{to the editor. We'd like to hear what you
think about this or any of our articles. Here are some}
\href{https://help.nytimes.com/hc/en-us/articles/115014925288-How-to-submit-a-letter-to-the-editor}{\emph{tips}}\emph{.
And here's our email:}
\href{mailto:letters@nytimes.com}{\emph{letters@nytimes.com}}\emph{.}

\emph{Follow The New York Times Opinion section on}
\href{https://www.facebook.com/nytopinion}{\emph{Facebook}}\emph{,}
\href{http://twitter.com/NYTOpinion}{\emph{Twitter (@NYTopinion)}}
\emph{and}
\href{https://www.instagram.com/nytopinion/}{\emph{Instagram}}\emph{.}

Advertisement

\protect\hyperlink{after-bottom}{Continue reading the main story}

\hypertarget{site-index}{%
\subsection{Site Index}\label{site-index}}

\hypertarget{site-information-navigation}{%
\subsection{Site Information
Navigation}\label{site-information-navigation}}

\begin{itemize}
\tightlist
\item
  \href{https://help.nytimes.com/hc/en-us/articles/115014792127-Copyright-notice}{©~2020~The
  New York Times Company}
\end{itemize}

\begin{itemize}
\tightlist
\item
  \href{https://www.nytco.com/}{NYTCo}
\item
  \href{https://help.nytimes.com/hc/en-us/articles/115015385887-Contact-Us}{Contact
  Us}
\item
  \href{https://www.nytco.com/careers/}{Work with us}
\item
  \href{https://nytmediakit.com/}{Advertise}
\item
  \href{http://www.tbrandstudio.com/}{T Brand Studio}
\item
  \href{https://www.nytimes.com/privacy/cookie-policy\#how-do-i-manage-trackers}{Your
  Ad Choices}
\item
  \href{https://www.nytimes.com/privacy}{Privacy}
\item
  \href{https://help.nytimes.com/hc/en-us/articles/115014893428-Terms-of-service}{Terms
  of Service}
\item
  \href{https://help.nytimes.com/hc/en-us/articles/115014893968-Terms-of-sale}{Terms
  of Sale}
\item
  \href{https://spiderbites.nytimes.com}{Site Map}
\item
  \href{https://help.nytimes.com/hc/en-us}{Help}
\item
  \href{https://www.nytimes.com/subscription?campaignId=37WXW}{Subscriptions}
\end{itemize}
