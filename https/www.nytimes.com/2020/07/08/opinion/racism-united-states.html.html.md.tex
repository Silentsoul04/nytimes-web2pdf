Sections

SEARCH

\protect\hyperlink{site-content}{Skip to
content}\protect\hyperlink{site-index}{Skip to site index}

\href{https://myaccount.nytimes.com/auth/login?response_type=cookie\&client_id=vi}{}

\href{https://www.nytimes.com/section/todayspaper}{Today's Paper}

\href{/section/opinion}{Opinion}\textbar{}Call a Thing a Thing

\url{https://nyti.ms/2BJmKBv}

\begin{itemize}
\item
\item
\item
\item
\item
\item
\end{itemize}

Advertisement

\protect\hyperlink{after-top}{Continue reading the main story}

\href{/section/opinion}{Opinion}

Supported by

\protect\hyperlink{after-sponsor}{Continue reading the main story}

\hypertarget{call-a-thing-a-thing}{%
\section{Call a Thing a Thing}\label{call-a-thing-a-thing}}

White supremacy is the biggest racial problem this country faces, and
has faced.

\href{https://www.nytimes.com/by/charles-m-blow}{\includegraphics{https://static01.nyt.com/images/2018/04/02/opinion/charles-m-blow/charles-m-blow-thumbLarge.png}}

By \href{https://www.nytimes.com/by/charles-m-blow}{Charles M. Blow}

Opinion Columnist

\begin{itemize}
\item
  July 8, 2020
\item
  \begin{itemize}
  \item
  \item
  \item
  \item
  \item
  \item
  \end{itemize}
\end{itemize}

\includegraphics{https://static01.nyt.com/images/2020/07/08/opinion/08blow1/merlin_173707008_ffe177c8-ba1a-49e8-bbdf-2aaff62665ff-articleLarge.jpg?quality=75\&auto=webp\&disable=upscale}

Now that we are deep into protests over racism, inequality and police
brutality ---~protests that I've come to see as a revisiting of Freedom
Summer ---~ it is clear that Donald Trump sees the activation of white
nationalism and anti-otherness as his path to re-election. We are
engaged in yet another national conversation about race and racism,
privilege and oppression.

But, as is usually the case, the language we used to describe the moment
is lacking. We --- the public and the media, including this newspaper,
including, in the past, this very column --- often use, consciously or
not, language that shields anti-Black white supremacy, rather than to
expose it and hold it accountable.

We use all manner of euphemisms and terms of art to keep from directly
addressing the racial reality in America. This may be some holdover from
a bygone time, but it is now time for it to come to an end.

Take for instance the term ``race relations.'' Polling organizations
like Gallup and the Pew Research Center often ask respondents how they
feel about the state of race relations in the country.

I have never fully understood what this meant. It suggests a
relationship that swings from harmony to disharmony. But that is not the
way race is structured or animated in this country. From the beginning,
the racial dynamics in America have been about power, equality and
access, or the lack thereof.

Protests, and even violence, have erupted when white people felt their
hold on those things was threatened or when Black people --- or
Indigenous people, or Hispanics --- rebelled against those things being
denied.

So what are the relations here? It is a linguistic sidestep that avoids
the true issue: anti-Black and anti-other white supremacy.

It also seems that the way people interpret that question is in direct
proportion to the intensity of revolt that's taking place at a
particular time. Satisfaction with race relations is somewhat correlated
with the silence of the oppressed. When they stop being silent, it
affects the outcome.

After the rise of Black Lives Matter, satisfaction with race relations
suffered a sustained drop.

The same can be said for the term ``racial tension.'' Read your news
carefully and pay close attention to television and your podcasts and
you will hear this phrase repeated. Someone is inflaming racial tensions
or trying to cool them. But again, what does this mean?

Is the act of taking to the streets to demand justice a form of tension?
Again, whenever people object to their oppression, it is framed as
problematic to peaceful coexistence. Furthermore, this tension between
the oppressed and the oppressors has always existed and always will. The
lulls you experience between explosive revolts of the oppressed should
never be mistaken as harmony. They should be taken as rest breaks.

Then there are ever-present terms like ``racial unity'' and ``racial
division.'' America loves to frame race in this country around unity
rather than equality. But, to do so robs the oppressed of legitimate
grievance.

I've never understood the aim of bringing people together in unity
absent the removal of anti-Black white supremacist social and political
frameworks. It is one thing to experience transracial unity with an ally
who is fighting just as hard for your liberation as you are. But it is
literally impossible for me to unify with someone perfectly happy with
the current state of affairs, which included the oppression of people
who look like me.

Most of these phrases suggest a false premise, that white people and
nonwhite ones are operating from equal positions of power in this
society and are simply not getting along or agreeing on issues.

In other words, by implication, they make nonwhite people equally at
fault for the state of race in America, when both history and social
science demonstrate, unequivocally, that this is not true.

It is almost like we are experiencing a Lost Cause revisionism in our
language on the issue of race.

It is time for us to simply call a thing a thing: White supremacy is the
biggest racial problem this country faces, and has faced. It is almost
always the cause of unrest around race. It has been used to slaughter
and destroy, to oppress and imprison. It manifests in every segment of
American life.

It is odd that we are so timid about using it now because the white men
who were the architects of modern white supremacy used it freely.

Mississippi was one of the first states to rewrite its constitution for
the
\href{https://www.jstor.org/stable/25434804?read-now=1\&seq=1}{express
purpose of codifying white supremacy}, and states across the South
followed the Mississippi example.

As one delegate at the
\href{https://archive.org/stream/journalproceedi01convgoog\#page/n273/mode/2up/search/It+is+the+manifest+intention+of+this+Convention+to+secure+to+the+State+of+Mississippi}{Mississippi
constitutional convention of 1890} put it*: ``*It is the manifest
intention of this Convention to secure to the State of Mississippi,
`white supremacy.' ''

One hundred and thirty years on, we are still fighting against this
architecture.

Until we stop playing cute about these facts, until we stop walking
around it like it's not the root, our dialogue will continue to be
hamstrung.

\emph{The Times is committed to publishing}
\href{https://www.nytimes.com/2019/01/31/opinion/letters/letters-to-editor-new-york-times-women.html}{\emph{a
diversity of letters}} \emph{to the editor. We'd like to hear what you
think about this or any of our articles. Here are some}
\href{https://help.nytimes.com/hc/en-us/articles/115014925288-How-to-submit-a-letter-to-the-editor}{\emph{tips}}\emph{.
And here's our email:}
\href{mailto:letters@nytimes.com}{\emph{letters@nytimes.com}}\emph{.}

\emph{Follow The New York Times Opinion section on}
\href{https://www.facebook.com/nytopinion}{\emph{Facebook}} \emph{and}
\href{http://twitter.com/NYTOpinion}{\emph{Twitter
(@NYTopinion)}}\emph{, and}
\href{https://www.instagram.com/nytopinion/}{\emph{Instagram}}\emph{.}

Advertisement

\protect\hyperlink{after-bottom}{Continue reading the main story}

\hypertarget{site-index}{%
\subsection{Site Index}\label{site-index}}

\hypertarget{site-information-navigation}{%
\subsection{Site Information
Navigation}\label{site-information-navigation}}

\begin{itemize}
\tightlist
\item
  \href{https://help.nytimes.com/hc/en-us/articles/115014792127-Copyright-notice}{©~2020~The
  New York Times Company}
\end{itemize}

\begin{itemize}
\tightlist
\item
  \href{https://www.nytco.com/}{NYTCo}
\item
  \href{https://help.nytimes.com/hc/en-us/articles/115015385887-Contact-Us}{Contact
  Us}
\item
  \href{https://www.nytco.com/careers/}{Work with us}
\item
  \href{https://nytmediakit.com/}{Advertise}
\item
  \href{http://www.tbrandstudio.com/}{T Brand Studio}
\item
  \href{https://www.nytimes.com/privacy/cookie-policy\#how-do-i-manage-trackers}{Your
  Ad Choices}
\item
  \href{https://www.nytimes.com/privacy}{Privacy}
\item
  \href{https://help.nytimes.com/hc/en-us/articles/115014893428-Terms-of-service}{Terms
  of Service}
\item
  \href{https://help.nytimes.com/hc/en-us/articles/115014893968-Terms-of-sale}{Terms
  of Sale}
\item
  \href{https://spiderbites.nytimes.com}{Site Map}
\item
  \href{https://help.nytimes.com/hc/en-us}{Help}
\item
  \href{https://www.nytimes.com/subscription?campaignId=37WXW}{Subscriptions}
\end{itemize}
