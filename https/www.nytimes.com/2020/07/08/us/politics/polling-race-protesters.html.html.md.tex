Sections

SEARCH

\protect\hyperlink{site-content}{Skip to
content}\protect\hyperlink{site-index}{Skip to site index}

\href{https://www.nytimes.com/section/politics}{Politics}

\href{https://myaccount.nytimes.com/auth/login?response_type=cookie\&client_id=vi}{}

\href{https://www.nytimes.com/section/todayspaper}{Today's Paper}

\href{/section/politics}{Politics}\textbar{}Some Republicans Have Grown
Wary of Protests Against Racism, Poll Shows

\url{https://nyti.ms/2O9SXo6}

\begin{itemize}
\item
\item
\item
\item
\item
\end{itemize}

\href{https://www.nytimes.com/news-event/george-floyd-protests-minneapolis-new-york-los-angeles?action=click\&pgtype=Article\&state=default\&region=TOP_BANNER\&context=storylines_menu}{Race
and America}

\begin{itemize}
\tightlist
\item
  \href{https://www.nytimes.com/2020/07/26/us/protests-portland-seattle-trump.html?action=click\&pgtype=Article\&state=default\&region=TOP_BANNER\&context=storylines_menu}{Protesters
  Return to Other Cities}
\item
  \href{https://www.nytimes.com/2020/07/24/us/portland-oregon-protests-white-race.html?action=click\&pgtype=Article\&state=default\&region=TOP_BANNER\&context=storylines_menu}{Portland
  at the Center}
\item
  \href{https://www.nytimes.com/2020/07/23/podcasts/the-daily/portland-protests.html?action=click\&pgtype=Article\&state=default\&region=TOP_BANNER\&context=storylines_menu}{Podcast:
  Showdown in Portland}
\item
  \href{https://www.nytimes.com/interactive/2020/07/16/us/black-lives-matter-protests-louisville-breonna-taylor.html?action=click\&pgtype=Article\&state=default\&region=TOP_BANNER\&context=storylines_menu}{45
  Days in Louisville}
\end{itemize}

Advertisement

\protect\hyperlink{after-top}{Continue reading the main story}

Supported by

\protect\hyperlink{after-sponsor}{Continue reading the main story}

\hypertarget{some-republicans-have-grown-wary-of-protests-against-racism-poll-shows}{%
\section{Some Republicans Have Grown Wary of Protests Against Racism,
Poll
Shows}\label{some-republicans-have-grown-wary-of-protests-against-racism-poll-shows}}

A month ago, polls reflected a new consensus around the need for racial
justice. But after weeks of attacks by President Trump, some
Republicans' views have shifted.

\includegraphics{https://static01.nyt.com/images/2020/07/08/us/politics/08polling-race/merlin_174249411_011180d4-f95e-474a-8cfe-aa4b4cf8a4f8-articleLarge.jpg?quality=75\&auto=webp\&disable=upscale}

\href{https://www.nytimes.com/by/giovanni-russonello}{\includegraphics{https://static01.nyt.com/images/2019/04/03/multimedia/author-giovanni-russonello/author-giovanni-russonello-thumbLarge.png}}

By \href{https://www.nytimes.com/by/giovanni-russonello}{Giovanni
Russonello}

\begin{itemize}
\item
  July 8, 2020
\item
  \begin{itemize}
  \item
  \item
  \item
  \item
  \item
  \end{itemize}
\end{itemize}

Most Americans continue to support the nationwide protests against
racial injustice,~but with
\href{https://www.nytimes.com/interactive/2020/us/elections/donald-trump.html}{President
Trump} issuing an
\href{https://www.nytimes.com/2020/07/06/us/politics/trump-bubba-wallace-nascar.html}{ever-more-combative
barrage of attacks}, new polling shows that some Republicans have grown
wary of demonstrators' demands and retreated toward saying that racism
is not in fact a big problem in the United States.

At the start of June, many polls showed the emergence of
\href{https://www.nytimes.com/2020/06/05/us/politics/polling-george-floyd-protests-racism.html}{a
rare consensus} around calls for racial justice and changes to policing,
with a majority of Republicans joining other Americans in saying that
racial discrimination is a big issue for the country.

But a
\href{https://www.monmouth.edu/polling-institute/documents/monmouthpoll_us_070820.pdf/}{Monmouth
University survey} released on Wednesday found that at the end of last
month, just 40 percent of Republicans still said racism was a big
problem, a drop of 15 percentage points from four weeks earlier.

And while close to four in 10 Republicans
\href{https://www.monmouth.edu/polling-institute/documents/monmouthpoll_us_060220.pdf/}{told
Monmouth researchers} at the start of last month that protesters' anger
was justified, that number fell by more than half in the new poll, with
just 15 percent of Republicans saying so. A majority of Democrats and
independents continued to say that the demonstrators' grievances were
fully justified.

Sekou Franklin, a political scientist at Middle Tennessee State
University, said, ``It's very difficult for a conservative Republican
who gravitates toward Trump to then embrace the idea that racism is an
urgent issue that we need to deal with, because Trump is driving a wedge
between those people and the protesters out in the street.''

``But attitudes shift back and forth,'' Dr. Franklin added. ``I think
the concern among many activists and social scientists is with the
moderate observer, who's now more sympathetic to the protesters. Will
that person's attitudes shift back over time?''

Over all, a wide majority of Americans across age, gender and race said
they thought the Black Lives Matter movement had brought attention to
real racial disparities in the country --- and most said they expected
the current protest movement to have a positive impact on race
relations.

Republicans were the only major subgroup to be about evenly split on the
legitimacy and likely effects of the protests. Forty-nine percent of
Republicans told Monmouth researchers that Black Lives Matter had shined
a light on real problems; 47 percent said it hadn't.

But even if many Republicans have been receptive to Mr. Trump's
messaging, that does not always mean they express satisfaction with his
style. Just 43 percent said that his handling of the protests had made
things better, while another 45 percent said it had either made things
worse or had no impact.

Glen Bolger, a longtime Republican pollster, said that he saw some
Republicans' heightened aversion as the natural result of what they are
seeing in the news. ``It's becoming a partisan issue, and that's clearly
a shame, but there's not a lot of sympathy for the people who are
blowing things up or looting stores'' or taking down monuments to the
country's founding fathers, he said.

But Mr. Bolger said Democrats had shown a surprising ability to avoid
being painted as extreme on one particular issue: calls to ``defund the
police.''

This has become the major rallying cry at protests, with activists
pushing for a scaling-down of the police presence in municipalities
across the country and a greater investment in social services. While
some Democrats --- including
\href{https://www.nytimes.com/interactive/2020/us/elections/joe-biden.html}{Joseph
R. Biden Jr.}, the party's presumptive nominee ---~have distanced
themselves from the language, many prominent progressives have proudly
embraced it.

Mr. Trump and other Republicans have seized on this phrase, believing
that most voters will find it alienating and extreme. More than half of
the Trump campaign's television budget from the past week was spent on a
single advertisement depicting an empty police station under a Biden
administration, according to Advertising Analytics.

Over the past seven days, the campaign has spent \$3.1 million on an ad
that shows an answering service in the future responding to a 911 call,
with scenes from eruptive protests taking over the split screen. The ad
appears to illustrate a dystopian nightmare, resembling imagery from Fox
News more than from the Biden campaign platform.

But the Monmouth poll's results, along with
\href{https://www.nytimes.com/2020/07/03/us/politics/polling-defund-the-police.html}{similar
data from other surveys}, don't bode well for this messaging. An
overwhelming share of Americans told Monmouth's interviewers that when
they heard protesters say ``defund the police,'' they understood it as a
demand to change the way police departments operate, not as a push to
eliminate the police altogether.

Seventy-seven percent of all Monmouth respondents --- including
three-quarters of white Americans and two-thirds of Republicans --- said
this, while less than one in five said they thought ``defund the
police'' meant getting rid of police departments.

``I think ultimately that is not as clear-cut a message for Republicans
as we might have hoped, because the Democrats have kind of stepped back
and clarified that,'' Mr. Bolger said.

Dr. Franklin, who identified himself as a longtime supporter of the
movement to defund the police, said he had been surprised at the lack of
``major pushback'' against protesters' cries. ``To the credit of
activists around the country who have been working on the defund
movement,'' he said, ``they've effectively been able to dissect
municipal budgets and make it explicitly clear how much money goes
toward police departments, and how much money can go toward public
health and other first responders.''

With the public now mostly viewing ``defund the police'' as a call to
shift funds, rather than get rid of departments altogether, the terms of
the debate on the left have shifted. ``The defund movement is distinct
from the abolition movement,'' Dr. Franklin said, noting that many
protesters do indeed seek to
\href{https://www.nytimes.com/2019/04/17/magazine/prison-abolition-ruth-wilson-gilmore.html}{abolish
traditional police departments altogether} and replace them with a
system built entirely around models of restorative justice.

Nick Corasaniti contributed reporting.

Advertisement

\protect\hyperlink{after-bottom}{Continue reading the main story}

\hypertarget{site-index}{%
\subsection{Site Index}\label{site-index}}

\hypertarget{site-information-navigation}{%
\subsection{Site Information
Navigation}\label{site-information-navigation}}

\begin{itemize}
\tightlist
\item
  \href{https://help.nytimes.com/hc/en-us/articles/115014792127-Copyright-notice}{©~2020~The
  New York Times Company}
\end{itemize}

\begin{itemize}
\tightlist
\item
  \href{https://www.nytco.com/}{NYTCo}
\item
  \href{https://help.nytimes.com/hc/en-us/articles/115015385887-Contact-Us}{Contact
  Us}
\item
  \href{https://www.nytco.com/careers/}{Work with us}
\item
  \href{https://nytmediakit.com/}{Advertise}
\item
  \href{http://www.tbrandstudio.com/}{T Brand Studio}
\item
  \href{https://www.nytimes.com/privacy/cookie-policy\#how-do-i-manage-trackers}{Your
  Ad Choices}
\item
  \href{https://www.nytimes.com/privacy}{Privacy}
\item
  \href{https://help.nytimes.com/hc/en-us/articles/115014893428-Terms-of-service}{Terms
  of Service}
\item
  \href{https://help.nytimes.com/hc/en-us/articles/115014893968-Terms-of-sale}{Terms
  of Sale}
\item
  \href{https://spiderbites.nytimes.com}{Site Map}
\item
  \href{https://help.nytimes.com/hc/en-us}{Help}
\item
  \href{https://www.nytimes.com/subscription?campaignId=37WXW}{Subscriptions}
\end{itemize}
