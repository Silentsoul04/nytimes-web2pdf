Sections

SEARCH

\protect\hyperlink{site-content}{Skip to
content}\protect\hyperlink{site-index}{Skip to site index}

\href{https://www.nytimes.com/section/world/americas}{Americas}

\href{https://myaccount.nytimes.com/auth/login?response_type=cookie\&client_id=vi}{}

\href{https://www.nytimes.com/section/todayspaper}{Today's Paper}

\href{/section/world/americas}{Americas}\textbar{}Santiago Manuin,
Indigenous Leader in Peru, Dies at 63

\url{https://nyti.ms/2O9zBiW}

\begin{itemize}
\item
\item
\item
\item
\item
\end{itemize}

\href{https://www.nytimes.com/news-event/coronavirus?action=click\&pgtype=Article\&state=default\&region=TOP_BANNER\&context=storylines_menu}{The
Coronavirus Outbreak}

\begin{itemize}
\tightlist
\item
  live\href{https://www.nytimes.com/2020/08/03/world/coronavirus-covid-19.html?action=click\&pgtype=Article\&state=default\&region=TOP_BANNER\&context=storylines_menu}{Latest
  Updates}
\item
  \href{https://www.nytimes.com/interactive/2020/us/coronavirus-us-cases.html?action=click\&pgtype=Article\&state=default\&region=TOP_BANNER\&context=storylines_menu}{Maps
  and Cases}
\item
  \href{https://www.nytimes.com/interactive/2020/science/coronavirus-vaccine-tracker.html?action=click\&pgtype=Article\&state=default\&region=TOP_BANNER\&context=storylines_menu}{Vaccine
  Tracker}
\item
  \href{https://www.nytimes.com/2020/08/02/us/covid-college-reopening.html?action=click\&pgtype=Article\&state=default\&region=TOP_BANNER\&context=storylines_menu}{College
  Reopening}
\item
  \href{https://www.nytimes.com/live/2020/08/03/business/stock-market-today-coronavirus?action=click\&pgtype=Article\&state=default\&region=TOP_BANNER\&context=storylines_menu}{Economy}
\end{itemize}

Advertisement

\protect\hyperlink{after-top}{Continue reading the main story}

Supported by

\protect\hyperlink{after-sponsor}{Continue reading the main story}

Those We've Lost

\hypertarget{santiago-manuin-indigenous-leader-in-peru-dies-at-63}{%
\section{Santiago Manuin, Indigenous Leader in Peru, Dies at
63}\label{santiago-manuin-indigenous-leader-in-peru-dies-at-63}}

Mr. Manuin fought for the rights of Indigenous people and the protection
of the Amazon. He survived a gunshot wound from a 2009 police crackdown,
but died last week of Covid-19.

\includegraphics{https://static01.nyt.com/images/2020/07/18/obituaries/07Manuin-print/merlin_174190794_e297e621-0c2b-4db6-b869-8be38454ef77-articleLarge.jpg?quality=75\&auto=webp\&disable=upscale}

By Mitra Taj

\begin{itemize}
\item
  Published July 8, 2020Updated July 18, 2020
\item
  \begin{itemize}
  \item
  \item
  \item
  \item
  \item
  \end{itemize}
\end{itemize}

\emph{This obituary is part of a series about people who have died in
the coronavirus pandemic. Read about others}
\href{https://www.nytimes.com/interactive/2020/obituaries/people-died-coronavirus-obituaries.html}{\emph{here}}\emph{.}

This was clear to Santiago Manuin: Indigenous tribes of the Amazon had
the right to protect their land and to take part in decisions affecting
them.

He promoted these principles as a prominent Peruvian human rights
activist, and defended them as a longtime leader of the Awajún and
Wampis Indigenous peoples of Peru's northwestern Amazon, expelling armed
rebels from tribal territory and pushing back against a wave of squatter
settlements encouraged by the government.

Mr. Manuin died on July 1 of Covid-19 in a hospital in the coastal city
of Chiclayo after struggling to find adequate health care in his region,
Amazonas, his son, Santiago Jesus Manuin, said. He was 63.

The novel coronavirus that causes the disease has been spreading quickly
through native communities in Peru, which often lack basic health care.

In one of his last statements, Mr. Manuin in April called on tribal
authorities to devise their own strategy for containing Covid-19's
spread, criticizing the government's response to the virus among
Indigenous peoples as inadequate. ``We must face this pandemic like we
have other problems,'' he said. ``We have to be brave.''

President Martín Vizcarra in June acknowledged that the government's
response to them had been slow and promised to do better. The government
started forming Indigenous committees to help address the pandemic this
month.

Mr. Manuin was born in Santa Maria de Nieva near the Marañón River in
northern Peru on July 25, 1956, at a time when his people started having
regular contact with outsiders. His parents worked on rubber plantations
and were mistreated by the owners, stirring his interest in leadership,
his son said.

Mr. Manuin learned to read and write from Jesuit missionaries, and
embraced education and Catholicism as tools to help organize against
injustice. He earned a master's degree in human rights from the
University of Deusto in Spain.

He began gaining recognition, meeting with Pope Francis and winning an
award from Peru's largest human rights group. Mr. Manuin was also known
for having survived a police crackdown on protesters in 2009, one of the
worst in recent Peruvian history.

He was shot as he and other Indigenous protesters blocked a road to
demand the repeal of laws that opened up the Amazon to more logging,
mining and oil drilling. Police moved to disperse the crowds, firing
live ammunition, according to witnesses. At least 33 people died in
clashes, including several police officers slain by protesters.

El Baguazo, as the clashes were known, forced a national reckoning over
Indigenous rights. The laws under fire were repealed, and measures
giving Indigenous people more say in government decisions passed in
following years.

Along with his son Santiago Jesus, Mr. Manuin is survived by his wife,
Justina Mayan, and 13 other children.

``He taught us that we're part of humanity and part of Peru,'' Santiago
Jesus Manuin said, ``and we deserve the same rights and have the same
need to live in peace as others.''

\href{https://www.nytimes.com/interactive/2020/obituaries/people-died-coronavirus-obituaries.html?action=click\&pgtype=Article\&state=default\&region=BELOW_MAIN_CONTENT\&context=covid_obits_promo}{}

\hypertarget{those-weve-lost}{%
\section{Those We've Lost}\label{those-weve-lost}}

The coronavirus pandemic has taken an incalculable death toll. This
series is designed to put names and faces to the numbers.

Read more

\includegraphics{https://static01.nyt.com/images/2020/07/30/obituaries/30Pedro/30Pedro-square640.jpg}

\hypertarget{bernaldina-josuxe9-pedro}{%
\section{Bernaldina José Pedro}\label{bernaldina-josuxe9-pedro}}

d. Boa Vista, Brazil

Leader among the Indigenous Macuxi

\includegraphics{https://static01.nyt.com/images/2020/07/31/obituaries/31Swing/merlin_175167783_8913bc90-0d64-43f3-a655-1bb1bf1601c9-square640.jpg}

\hypertarget{john-eric-swing}{%
\section{John Eric Swing}\label{john-eric-swing}}

d. Fountain Valley, Calif.

Champion of Filipino-Americans

\includegraphics{https://static01.nyt.com/images/2020/07/27/obituaries/27Victor/merlin_175001436_38b11f8e-227a-4e2c-9821-7618af9b2524-square640.jpg}

\hypertarget{victor-victor}{%
\section{Victor Victor}\label{victor-victor}}

d. Santo Domingo, Dominican Republic

Beloved musician of the Dominican Republic

\includegraphics{https://static01.nyt.com/images/2020/07/31/obituaries/31Negron/merlin_175160169_516322ae-fd23-4969-b6b2-193ced371105-square640.jpg}

\hypertarget{dr-eddie-negruxf3n}{%
\section{Dr. Eddie Negrón}\label{dr-eddie-negruxf3n}}

d. Fort Walton Beach, Fla.

Internist on Florida's Emerald Coast

\includegraphics{https://static01.nyt.com/images/2020/07/30/obituaries/30Dobson/merlin_175115928_f6b9271c-8f05-4fe1-a38a-5ca4a58f8935-square640.jpg}

\hypertarget{dobby-dobson}{%
\section{Dobby Dobson}\label{dobby-dobson}}

d. Coral Springs, Fla.

Jamaican singer and songwriter

\includegraphics{https://static01.nyt.com/images/2020/08/01/obituaries/28Gonzalez/merlin_175002771_beb57888-3951-409a-ae13-03a94b2e962e-square640.jpg}

\hypertarget{waldemar-gonzalez}{%
\section{Waldemar Gonzalez}\label{waldemar-gonzalez}}

d. White Plains, N.Y.

Teacher and social worker

Advertisement

\protect\hyperlink{after-bottom}{Continue reading the main story}

\hypertarget{site-index}{%
\subsection{Site Index}\label{site-index}}

\hypertarget{site-information-navigation}{%
\subsection{Site Information
Navigation}\label{site-information-navigation}}

\begin{itemize}
\tightlist
\item
  \href{https://help.nytimes.com/hc/en-us/articles/115014792127-Copyright-notice}{©~2020~The
  New York Times Company}
\end{itemize}

\begin{itemize}
\tightlist
\item
  \href{https://www.nytco.com/}{NYTCo}
\item
  \href{https://help.nytimes.com/hc/en-us/articles/115015385887-Contact-Us}{Contact
  Us}
\item
  \href{https://www.nytco.com/careers/}{Work with us}
\item
  \href{https://nytmediakit.com/}{Advertise}
\item
  \href{http://www.tbrandstudio.com/}{T Brand Studio}
\item
  \href{https://www.nytimes.com/privacy/cookie-policy\#how-do-i-manage-trackers}{Your
  Ad Choices}
\item
  \href{https://www.nytimes.com/privacy}{Privacy}
\item
  \href{https://help.nytimes.com/hc/en-us/articles/115014893428-Terms-of-service}{Terms
  of Service}
\item
  \href{https://help.nytimes.com/hc/en-us/articles/115014893968-Terms-of-sale}{Terms
  of Sale}
\item
  \href{https://spiderbites.nytimes.com}{Site Map}
\item
  \href{https://help.nytimes.com/hc/en-us}{Help}
\item
  \href{https://www.nytimes.com/subscription?campaignId=37WXW}{Subscriptions}
\end{itemize}
