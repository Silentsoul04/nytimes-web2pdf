Sections

SEARCH

\protect\hyperlink{site-content}{Skip to
content}\protect\hyperlink{site-index}{Skip to site index}

\href{https://myaccount.nytimes.com/auth/login?response_type=cookie\&client_id=vi}{}

\href{https://www.nytimes.com/section/todayspaper}{Today's Paper}

\href{/section/opinion}{Opinion}\textbar{}Two Cheers for Liberalism! (Or
Maybe One and a Half)

\href{https://nyti.ms/3egJpSy}{https://nyti.ms/3egJpSy}

\begin{itemize}
\item
\item
\item
\item
\item
\item
\end{itemize}

Advertisement

\protect\hyperlink{after-top}{Continue reading the main story}

\href{/section/opinion}{Opinion}

Supported by

\protect\hyperlink{after-sponsor}{Continue reading the main story}

\hypertarget{two-cheers-for-liberalism-or-maybe-one-and-a-half}{%
\section{Two Cheers for Liberalism! (Or Maybe One and a
Half)}\label{two-cheers-for-liberalism-or-maybe-one-and-a-half}}

Free speech has to rest on a shared morality.

\href{https://www.nytimes.com/by/david-brooks}{\includegraphics{https://static01.nyt.com/images/2018/04/03/opinion/david-brooks/david-brooks-thumbLarge-v2.png}}

By \href{https://www.nytimes.com/by/david-brooks}{David Brooks}

Opinion Columnist

\begin{itemize}
\item
  July 9, 2020
\item
  \begin{itemize}
  \item
  \item
  \item
  \item
  \item
  \item
  \end{itemize}
\end{itemize}

\includegraphics{https://static01.nyt.com/images/2020/07/09/opinion/09brooks1/merlin_173350857_8835adf3-8826-4490-b091-e4ed05a11c5d-articleLarge.jpg?quality=75\&auto=webp\&disable=upscale}

This is a hard, exhausting time. But it's also a pivot point. An
idealistic generation is rising on the scene hungering to fill the
spiritual vacuum their parents left them. There is a palpable desire for
solidarity, to shake off an excessively individualistic culture.

In periods of tumult and confusion many people lose faith in systems of
change. They feel it's necessary to take the extraordinary action to
tear down systems of power. For example, a Senate investigation
\href{https://cityroom.blogs.nytimes.com/2009/08/27/1969-a-year-of-bombings/}{concluded}
that from January 1969 to April 1970 --- a period of tumult similar to
our own --- there were 4,330 bombings in the United States, which killed
43 people --- averaging about nine bombings a day.

Today, thank God, we don't have bombings. But we do have a lot of people
on the right and the left who have lost faith in the institutions of
free speech and open debate --- the basic liberal order. They see that
free speech stuff as a mask elites wear to preserve their power. They
produce what is crudely called the cancel culture, they treat speech as
violence, they attempt to ruin politically discordant people because of
some
\href{https://twitter.com/sapinker/status/1279934082210816003?s=20}{tweets}.

I defend liberalism because I think our core problem is ignorance and
incompetence and not an elite conspiracy. The world right now is
astonishingly complicated, our systems need reform. I don't think one
vantage point can grasp reality or devise solutions. We have to have the
open exchange of views that is the essence of liberalism.

I am a liberal in a classical Enlightenment sense, but I can't give
three cheers for liberalism, or maybe even two. I understand why so
many, and so many younger people, are rejecting it. Liberalism, as it
emerged in the 18th-century Enlightenment, and as it was
institutionalized in America, was based on several false or distorted
ideas.

Liberalism was based on the idea that reason is separate from emotion,
that we need to be dispassionate to see clearly. This is false. Emotions
assign value to things and undergird reason. Because of this error,
liberalism has often devolved into a detached, passionless rationalism.

Liberalism was based on the idea that the choosing individual is the
elemental unit of society. It put great emphasis on individual autonomy.
This is distorted. We're also embedded creatures, members of families,
and groups, shaped by our histories. Liberalism sometimes devolves into
atomization, an alienated society of lonely buffered selves.

Liberalism assumed that people are primarily motivated by self-interest.
This, too, is distorted. People are motivated by both self-interest and
a yearning desire to lead a morally meaningful life. Liberalism often
produces a disenchanted materialist realm.

By itself, liberalism is so thin it can't even defend itself. When young
people passionately demand racial equity, liberalism's response is to
protect free speech. Young people have a dream. Liberalism offers a
neutral process.

Which is why the constitution of liberalism has to be supplemented with
the morality of personalism.

One of the reasons that America is so angry right now is that there is
so much dehumanization. Racism reduces a human being to a skin color.
The first casualty in a culture, political or generational war is the
willingness to see the full humanity of the other. In this moment, some
people seem eager even to dehumanize themselves by reducing themselves
to a simple label and making politics their one identity. ``Speaking as
a. \ldots''

If liberalism left little space for group identity, the current
conversation makes group identity everything and leaves no space for
individual conscience. You get all these absurd generalizations: White
people believe this. Elites believe that.

Personalism is the belief that at the heart of any successful
relationship, any successful organization and any just society, there is
an earnest and ongoing effort to see the full depth and complexity of
each human person.

Shadi Hamid struck a blow for personalism with a Twitter
\href{https://twitter.com/shadihamid/status/1280635678506856448}{thread}
this week: ``As a Muslim, an Arab, and a brown person, it always grated
on me when people would assume things about me merely because of my
`identity,' largely an accident of birth. I cared about being Muslim and
being Arab, and I was proud of my heritage. But that didn't mean that I
stopped being an individual. I was a writer who happened to be Muslim,
not a Muslim who happened to be a writer.''

Personalism is about constructing systems where the whole person is seen
and cultivated --- schools where a child is not just a brain on a stick,
hospitals where patients are not just bodies in beds, cities where cops
see people, not ``perps,'' communities in which each person is seen as a
rich interplay of multiple identities, economic systems that allow
people to realize their full dignity as makers and earners.

Personalism judges each social arrangement by how well it fosters the
kind of relationships that enhance the full complexity and depth of each
soul. This awful year will be somewhat redeemed if we can end it with a
sense of this kind of common morality, and if we can begin the hard work
of reforming our institutions to be in line with it.

\emph{The Times is committed to publishing}
\href{https://www.nytimes.com/2019/01/31/opinion/letters/letters-to-editor-new-york-times-women.html}{\emph{a
diversity of letters}} \emph{to the editor. We'd like to hear what you
think about this or any of our articles. Here are some}
\href{https://help.nytimes.com/hc/en-us/articles/115014925288-How-to-submit-a-letter-to-the-editor}{\emph{tips}}\emph{.
And here's our email:}
\href{mailto:letters@nytimes.com}{\emph{letters@nytimes.com}}\emph{.}

\emph{Follow The New York Times Opinion section on}
\href{https://www.facebook.com/nytopinion}{\emph{Facebook}}\emph{,}
\href{http://twitter.com/NYTOpinion}{\emph{Twitter (@NYTopinion)}}
\emph{and}
\href{https://www.instagram.com/nytopinion/}{\emph{Instagram}}\emph{.}

Advertisement

\protect\hyperlink{after-bottom}{Continue reading the main story}

\hypertarget{site-index}{%
\subsection{Site Index}\label{site-index}}

\hypertarget{site-information-navigation}{%
\subsection{Site Information
Navigation}\label{site-information-navigation}}

\begin{itemize}
\tightlist
\item
  \href{https://help.nytimes.com/hc/en-us/articles/115014792127-Copyright-notice}{©~2020~The
  New York Times Company}
\end{itemize}

\begin{itemize}
\tightlist
\item
  \href{https://www.nytco.com/}{NYTCo}
\item
  \href{https://help.nytimes.com/hc/en-us/articles/115015385887-Contact-Us}{Contact
  Us}
\item
  \href{https://www.nytco.com/careers/}{Work with us}
\item
  \href{https://nytmediakit.com/}{Advertise}
\item
  \href{http://www.tbrandstudio.com/}{T Brand Studio}
\item
  \href{https://www.nytimes.com/privacy/cookie-policy\#how-do-i-manage-trackers}{Your
  Ad Choices}
\item
  \href{https://www.nytimes.com/privacy}{Privacy}
\item
  \href{https://help.nytimes.com/hc/en-us/articles/115014893428-Terms-of-service}{Terms
  of Service}
\item
  \href{https://help.nytimes.com/hc/en-us/articles/115014893968-Terms-of-sale}{Terms
  of Sale}
\item
  \href{https://spiderbites.nytimes.com}{Site Map}
\item
  \href{https://help.nytimes.com/hc/en-us}{Help}
\item
  \href{https://www.nytimes.com/subscription?campaignId=37WXW}{Subscriptions}
\end{itemize}
