Sections

SEARCH

\protect\hyperlink{site-content}{Skip to
content}\protect\hyperlink{site-index}{Skip to site index}

\href{https://www.nytimes.com/section/us}{U.S.}

\href{https://myaccount.nytimes.com/auth/login?response_type=cookie\&client_id=vi}{}

\href{https://www.nytimes.com/section/todayspaper}{Today's Paper}

\href{/section/us}{U.S.}\textbar{}Like `a Bus Accident a Day': Hospitals
Strain Under New Flood of Covid-19 Patients

\url{https://nyti.ms/2ZWjCKm}

\begin{itemize}
\item
\item
\item
\item
\item
\item
\end{itemize}

\href{https://www.nytimes.com/news-event/coronavirus?action=click\&pgtype=Article\&state=default\&region=TOP_BANNER\&context=storylines_menu}{The
Coronavirus Outbreak}

\begin{itemize}
\tightlist
\item
  live\href{https://www.nytimes.com/2020/08/01/world/coronavirus-covid-19.html?action=click\&pgtype=Article\&state=default\&region=TOP_BANNER\&context=storylines_menu}{Latest
  Updates}
\item
  \href{https://www.nytimes.com/interactive/2020/us/coronavirus-us-cases.html?action=click\&pgtype=Article\&state=default\&region=TOP_BANNER\&context=storylines_menu}{Maps
  and Cases}
\item
  \href{https://www.nytimes.com/interactive/2020/science/coronavirus-vaccine-tracker.html?action=click\&pgtype=Article\&state=default\&region=TOP_BANNER\&context=storylines_menu}{Vaccine
  Tracker}
\item
  \href{https://www.nytimes.com/interactive/2020/07/29/us/schools-reopening-coronavirus.html?action=click\&pgtype=Article\&state=default\&region=TOP_BANNER\&context=storylines_menu}{What
  School May Look Like}
\item
  \href{https://www.nytimes.com/live/2020/07/31/business/stock-market-today-coronavirus?action=click\&pgtype=Article\&state=default\&region=TOP_BANNER\&context=storylines_menu}{Economy}
\end{itemize}

Advertisement

\protect\hyperlink{after-top}{Continue reading the main story}

Supported by

\protect\hyperlink{after-sponsor}{Continue reading the main story}

\hypertarget{like-a-bus-accident-a-day-hospitals-strain-under-new-flood-of-covid-19-patients}{%
\section{Like `a Bus Accident a Day': Hospitals Strain Under New Flood
of Covid-19
Patients}\label{like-a-bus-accident-a-day-hospitals-strain-under-new-flood-of-covid-19-patients}}

I.C.U. units are reaching capacity. Nurses are falling sick,
contributing to shortages. The new coronavirus spikes are challenging
hospitals across the United States.

\includegraphics{https://static01.nyt.com/images/2020/07/09/us/09VIRUS-HOSPITALS-jackson/merlin_174017190_17ecdf80-fc0f-4e0c-8e2c-d18ac45cf7cc-articleLarge.jpg?quality=75\&auto=webp\&disable=upscale}

By \href{https://www.nytimes.com/by/kimiko-de-freytas-tamura}{Kimiko de
Freytas-Tamura}, \href{https://www.nytimes.com/by/shawn-hubler}{Shawn
Hubler}, Hailey Fuchs and David Montgomery

\begin{itemize}
\item
  Published July 9, 2020Updated July 11, 2020
\item
  \begin{itemize}
  \item
  \item
  \item
  \item
  \item
  \item
  \end{itemize}
\end{itemize}

TAMPA --- As states across the American South and West grapple with
shortages of vital testing equipment and a key antiviral drug, hospitals
are being flooded with coronavirus patients, forcing them to cancel
elective surgeries and discharge patients early, and doctors worry that
the escalating hospital crunch may last much longer than in earlier-hit
areas like New York.

Even as regular wards are being converted into intensive care units and
long-term care facilities open for patients still too sick to go home,
doctors say they are barely managing.

Hospitals are scrambling to call back nurses and recruit new doctors.
Florida Gov. Ron DeSantis announced he was sending 100 nurses to help
out Jackson Health System in Miami, which said it had already hired 80
extra nurses in the past two weeks. Jackson Memorial, its flagship
hospital, has only 28 I.C.U. beds, out of a total of 234, available.

``When hospitals and health care assistants talk about surge capacity,
they're often talking about a single event,'' said John Sinnott,
chairman of internal medicine at the University of South Florida and
chief epidemiologist at Tampa General Hospital. ``But what we're having
now is the equivalent of a bus accident a day, every day, and it just
keeps adding.''

Florida is struggling with one of the worst outbreaks in the country,
along with Texas, California and Arizona: 43 intensive care units in 21
Florida counties have hit capacity and have no beds available.

In South Carolina, National Guard troops are being called in soon to
help insert intravenous lines and check blood pressure. Roper St.
Francis Healthcare in Charleston saw a 65 percent increase in
coronavirus patients in a single day.

Dr. Christopher McLain, the hospital's chief physician officer, said he
has begun each day on his knees in prayer and often begins meetings the
same way, asking the Lord how to respond to the pandemic. ``We're
already at a severe condition,'' he said.

In Mississippi, five of the state's largest hospitals have already run
out of I.C.U. beds for critical patients, Dr. Thomas Dobbs, the state
health officer, said on Thursday. ``Mississippi hospitals cannot take
care of Mississippi patients,'' he said.

\includegraphics{https://static01.nyt.com/images/2020/07/09/us/09VIRUS-HOSPITALS-houston/merlin_174296814_476c794c-4bf0-42ea-bbfb-61f110227514-articleLarge.jpg?quality=75\&auto=webp\&disable=upscale}

Texas Gov. Greg Abbott on Thursday ordered an increase in hospital bed
capacity for dozens of counties, extending a ban on elective procedures
to new corners of the state in an effort to assist hospitals dealing
with the outbreak.

\hypertarget{latest-updates-global-coronavirus-outbreak}{%
\section{\texorpdfstring{\href{https://www.nytimes.com/2020/08/01/world/coronavirus-covid-19.html?action=click\&pgtype=Article\&state=default\&region=MAIN_CONTENT_1\&context=storylines_live_updates}{Latest
Updates: Global Coronavirus
Outbreak}}{Latest Updates: Global Coronavirus Outbreak}}\label{latest-updates-global-coronavirus-outbreak}}

Updated 2020-08-02T10:04:29.623Z

\begin{itemize}
\tightlist
\item
  \href{https://www.nytimes.com/2020/08/01/world/coronavirus-covid-19.html?action=click\&pgtype=Article\&state=default\&region=MAIN_CONTENT_1\&context=storylines_live_updates\#link-34047410}{The
  U.S. reels as July cases more than double the total of any other
  month.}
\item
  \href{https://www.nytimes.com/2020/08/01/world/coronavirus-covid-19.html?action=click\&pgtype=Article\&state=default\&region=MAIN_CONTENT_1\&context=storylines_live_updates\#link-780ec966}{Top
  U.S. officials work to break an impasse over the federal jobless
  benefit.}
\item
  \href{https://www.nytimes.com/2020/08/01/world/coronavirus-covid-19.html?action=click\&pgtype=Article\&state=default\&region=MAIN_CONTENT_1\&context=storylines_live_updates\#link-2bc8948}{Its
  outbreak untamed, Melbourne goes into even greater lockdown.}
\end{itemize}

\href{https://www.nytimes.com/2020/08/01/world/coronavirus-covid-19.html?action=click\&pgtype=Article\&state=default\&region=MAIN_CONTENT_1\&context=storylines_live_updates}{See
more updates}

More live coverage:
\href{https://www.nytimes.com/live/2020/07/31/business/stock-market-today-coronavirus?action=click\&pgtype=Article\&state=default\&region=MAIN_CONTENT_1\&context=storylines_live_updates}{Markets}

Mr. Abbott directed hospitals to ``postpone surgeries and procedures
that are not immediately, medically necessary.'' The governor had
already done so in hard-hit urban counties that include San Antonio,
Dallas, Houston and Austin.

At the 463-bed hospital operated by Eisenhower Health in Rancho Mirage,
Calif., east of Los Angeles, the coronavirus case count has gone from
less than a dozen in May to 77 this week. Most of the 34 intensive care
beds are full, nearly half of them occupied by people who have been
infected.

``I'm glad some of you are sheltered from what unbridled Covid-19 looks
like. It's a hell show,'' a doctor at the hospital, Dr. Richard Loftus,
posted to a Facebook physician group after a computer trainer at the
hospital turned out to be infected.

Dr. Diego Maselli Caceres at University Hospital in San Antonio, Texas,
said he has watched a sevenfold surge of Covid-19 patients needing
intensive care over the past month, filling up three floors of the
hospital instead of one. His workload has increased to 15 hours a day,
he said.

``You get bombarded with multiple calls at the same time,'' he said,
referring to the ``code blue'' warnings from overhead speakers that send
doctors and nurses rushing to save a patient in distress.

``You hear the calls and you're running from one end to the other, just
like putting out fires, and you're trying to help as much as you can. It
gets overwhelming.''

Hospital bed capacity, including in I.C.U.s, is generally used to gauge
a region's health care infrastructure and the preparedness of its
hospitals to respond to the coronavirus. Data showing I.C.U.s at full or
near capacity have made headlines recently, but health experts say that
attention to capacity does not paint an entirely accurate picture of the
severity of the pandemic.

Regular beds are easily converted into I.C.U. capability, doctors and
hospital experts say. The bigger challenge is having enough nurses who
are qualified to care for such patients and equipment such as
ventilators.

Hospitals can ``pivot enough space,'' said Jay Wolfson, professor of
public health at the University of South Florida. ``The trick is going
to be staffing. If you get people burned out, they get sick, then you
lose critical care personnel.''

At the Medical University of South Carolina in Charleston, emergency
room waiting times can last up to four hours before patients are seen by
a physician. The hospital has set up large white tents outside to allow
for social distancing, but patients are increasingly leaving the site
before their treatment, unwilling to endure the wait.

As physicians and nurses fall ill with the coronavirus, much like their
patients, fewer and fewer staff members have been available to
accommodate the burgeoning number of sick people at their doorstep. Some
emergency room doctors have taken on extra shifts, and the hospital
plans to implement a new system where some doctors will be on-call, even
on their days off, to respond to the surge.

Mohamed Ibrahim Ali, a critical care doctor at Northside Hospital in St.
Petersburg, Fla., one of the hospitals that have no more available
I.C.U. beds, said that the system was clogged up by patients, sent from
nursing homes, who had recovered but had not yet received the all-clear.
Nursing homes at the governor's direction are not accepting residents
back unless they have twice tested negative, a period that could take
days.

Roopa Ganga, an infectious disease specialist at two hospitals near
Tampa, said that they lacked sufficient supplies of remdesivir, the
antiviral drug, forcing her to choose which patients needed it the most.
Patients were also being discharged ``aggressively'' --- perhaps too
early, she said. They sometimes return a few days later, she said, their
symptoms worsened.

\href{https://www.nytimes.com/news-event/coronavirus?action=click\&pgtype=Article\&state=default\&region=MAIN_CONTENT_3\&context=storylines_faq}{}

\hypertarget{the-coronavirus-outbreak-}{%
\subsubsection{The Coronavirus Outbreak
›}\label{the-coronavirus-outbreak-}}

\hypertarget{frequently-asked-questions}{%
\paragraph{Frequently Asked
Questions}\label{frequently-asked-questions}}

Updated July 27, 2020

\begin{itemize}
\item ~
  \hypertarget{should-i-refinance-my-mortgage}{%
  \paragraph{Should I refinance my
  mortgage?}\label{should-i-refinance-my-mortgage}}

  \begin{itemize}
  \tightlist
  \item
    \href{https://www.nytimes.com/article/coronavirus-money-unemployment.html?action=click\&pgtype=Article\&state=default\&region=MAIN_CONTENT_3\&context=storylines_faq}{It
    could be a good idea,} because mortgage rates have
    \href{https://www.nytimes.com/2020/07/16/business/mortgage-rates-below-3-percent.html?action=click\&pgtype=Article\&state=default\&region=MAIN_CONTENT_3\&context=storylines_faq}{never
    been lower.} Refinancing requests have pushed mortgage applications
    to some of the highest levels since 2008, so be prepared to get in
    line. But defaults are also up, so if you're thinking about buying a
    home, be aware that some lenders have tightened their standards.
  \end{itemize}
\item ~
  \hypertarget{what-is-school-going-to-look-like-in-september}{%
  \paragraph{What is school going to look like in
  September?}\label{what-is-school-going-to-look-like-in-september}}

  \begin{itemize}
  \tightlist
  \item
    It is unlikely that many schools will return to a normal schedule
    this fall, requiring the grind of
    \href{https://www.nytimes.com/2020/06/05/us/coronavirus-education-lost-learning.html?action=click\&pgtype=Article\&state=default\&region=MAIN_CONTENT_3\&context=storylines_faq}{online
    learning},
    \href{https://www.nytimes.com/2020/05/29/us/coronavirus-child-care-centers.html?action=click\&pgtype=Article\&state=default\&region=MAIN_CONTENT_3\&context=storylines_faq}{makeshift
    child care} and
    \href{https://www.nytimes.com/2020/06/03/business/economy/coronavirus-working-women.html?action=click\&pgtype=Article\&state=default\&region=MAIN_CONTENT_3\&context=storylines_faq}{stunted
    workdays} to continue. California's two largest public school
    districts --- Los Angeles and San Diego --- said on July 13, that
    \href{https://www.nytimes.com/2020/07/13/us/lausd-san-diego-school-reopening.html?action=click\&pgtype=Article\&state=default\&region=MAIN_CONTENT_3\&context=storylines_faq}{instruction
    will be remote-only in the fall}, citing concerns that surging
    coronavirus infections in their areas pose too dire a risk for
    students and teachers. Together, the two districts enroll some
    825,000 students. They are the largest in the country so far to
    abandon plans for even a partial physical return to classrooms when
    they reopen in August. For other districts, the solution won't be an
    all-or-nothing approach.
    \href{https://bioethics.jhu.edu/research-and-outreach/projects/eschool-initiative/school-policy-tracker/}{Many
    systems}, including the nation's largest, New York City, are
    devising
    \href{https://www.nytimes.com/2020/06/26/us/coronavirus-schools-reopen-fall.html?action=click\&pgtype=Article\&state=default\&region=MAIN_CONTENT_3\&context=storylines_faq}{hybrid
    plans} that involve spending some days in classrooms and other days
    online. There's no national policy on this yet, so check with your
    municipal school system regularly to see what is happening in your
    community.
  \end{itemize}
\item ~
  \hypertarget{is-the-coronavirus-airborne}{%
  \paragraph{Is the coronavirus
  airborne?}\label{is-the-coronavirus-airborne}}

  \begin{itemize}
  \tightlist
  \item
    The coronavirus
    \href{https://www.nytimes.com/2020/07/04/health/239-experts-with-one-big-claim-the-coronavirus-is-airborne.html?action=click\&pgtype=Article\&state=default\&region=MAIN_CONTENT_3\&context=storylines_faq}{can
    stay aloft for hours in tiny droplets in stagnant air}, infecting
    people as they inhale, mounting scientific evidence suggests. This
    risk is highest in crowded indoor spaces with poor ventilation, and
    may help explain super-spreading events reported in meatpacking
    plants, churches and restaurants.
    \href{https://www.nytimes.com/2020/07/06/health/coronavirus-airborne-aerosols.html?action=click\&pgtype=Article\&state=default\&region=MAIN_CONTENT_3\&context=storylines_faq}{It's
    unclear how often the virus is spread} via these tiny droplets, or
    aerosols, compared with larger droplets that are expelled when a
    sick person coughs or sneezes, or transmitted through contact with
    contaminated surfaces, said Linsey Marr, an aerosol expert at
    Virginia Tech. Aerosols are released even when a person without
    symptoms exhales, talks or sings, according to Dr. Marr and more
    than 200 other experts, who
    \href{https://academic.oup.com/cid/article/doi/10.1093/cid/ciaa939/5867798}{have
    outlined the evidence in an open letter to the World Health
    Organization}.
  \end{itemize}
\item ~
  \hypertarget{what-are-the-symptoms-of-coronavirus}{%
  \paragraph{What are the symptoms of
  coronavirus?}\label{what-are-the-symptoms-of-coronavirus}}

  \begin{itemize}
  \tightlist
  \item
    Common symptoms
    \href{https://www.nytimes.com/article/symptoms-coronavirus.html?action=click\&pgtype=Article\&state=default\&region=MAIN_CONTENT_3\&context=storylines_faq}{include
    fever, a dry cough, fatigue and difficulty breathing or shortness of
    breath.} Some of these symptoms overlap with those of the flu,
    making detection difficult, but runny noses and stuffy sinuses are
    less common.
    \href{https://www.nytimes.com/2020/04/27/health/coronavirus-symptoms-cdc.html?action=click\&pgtype=Article\&state=default\&region=MAIN_CONTENT_3\&context=storylines_faq}{The
    C.D.C. has also} added chills, muscle pain, sore throat, headache
    and a new loss of the sense of taste or smell as symptoms to look
    out for. Most people fall ill five to seven days after exposure, but
    symptoms may appear in as few as two days or as many as 14 days.
  \end{itemize}
\item ~
  \hypertarget{does-asymptomatic-transmission-of-covid-19-happen}{%
  \paragraph{Does asymptomatic transmission of Covid-19
  happen?}\label{does-asymptomatic-transmission-of-covid-19-happen}}

  \begin{itemize}
  \tightlist
  \item
    So far, the evidence seems to show it does. A widely cited
    \href{https://www.nature.com/articles/s41591-020-0869-5}{paper}
    published in April suggests that people are most infectious about
    two days before the onset of coronavirus symptoms and estimated that
    44 percent of new infections were a result of transmission from
    people who were not yet showing symptoms. Recently, a top expert at
    the World Health Organization stated that transmission of the
    coronavirus by people who did not have symptoms was ``very rare,''
    \href{https://www.nytimes.com/2020/06/09/world/coronavirus-updates.html?action=click\&pgtype=Article\&state=default\&region=MAIN_CONTENT_3\&context=storylines_faq\#link-1f302e21}{but
    she later walked back that statement.}
  \end{itemize}
\end{itemize}

``About five people came back in one week last week,'' she said. ``That
is making me feel like, you know, you got to slow down.''

Dr. Wolfson, from the University of South Florida, said health
authorities and public officials needed to collaborate better to bypass
regulations that bar out-of-state nurses from working in Florida. New
York's Covid-19 crisis in the spring was aided by a number of health
workers from outside the state.

``It's political inertia,'' he said. ``It takes somebody in a position
of significant political stature to say, `Let's do this.' Florida has
always been afraid that people were going to come into the state and
take their jobs. Now we need all the help we can get.''

Image

A drive-through testing facility in Wimauma, Fla., south of
Tampa.Credit...Zack Wittman for The New York Times

Traveling nurses were brought in to the Eisenhower Health hospital in
California's Riverside County. It was necessary, said Dr. Alan
Williamson, the chief medical officer, because the 3-to-1
nurse-to-patient ratio is much higher even for Covid-19 patients who are
not in the intensive care beds. It has been a challenge, said Dr.
Williamson, because Eisenhower is competing with two other hospitals in
the area that are using the same nurse registries to find help.

Texas is experiencing one of the fastest growing coronavirus caseloads
in the country, with new cases exceeding 10,000 one day this week. At
the Texas Medical Center hospitals in Houston, the average daily rate of
new Covid-19 hospitalizations was 360, nearly double the rate of just
two weeks ago.

``The hospitals are full,'' said Dr. Esmaeil Porsa, president and chief
executive officer of the county's two-hospital public health system,
Harris Health. ``We have been over capacity for a couple of weeks.''

In Corpus Christi, Texas, one of the state's fastest spreading outbreaks
has pushed hospitals to convert floors to treat Covid-19 patients as
they scramble to find extra staff, especially nurses.

``Are we strained? You bet we are,'' said Barbara Canales, the top
official in the surrounding county, adding that hospitals were asking
the state of Texas for help on staffing.

The county, of 360,000 people, saw hospitalizations surge in the last
few weeks, to 300 on Wednesday, up from fewer than 20 in the middle of
June.

Doctors and nurses interviewed said the current spike is unlike anything
they have ever dealt with.

Rick Stern, a veteran oncology nurse who works with the Covid-19
patients at Eisenhower Health, said the job is a constant churn of
gloves, gowns, masks, face shields and heart-wrenching misery.

His first day in the unit, he said, he watched a cancer patient who had
become infected die in the space of 15 hours. At times during this
surge, he said, as many as three patients a day have died on his ward;
he personally has lost three so far.

One of his current patients is 35.

``I've had experience with death,'' he said, ``but this is different.
These people aren't ready to go yet.''

Kimiko de Freytas-Tamura reported from Tampa, Fla., Shawn Hubler from
Sacramento, Calif., Hailey Fuchs from Charleston, S.C., and David
Montgomery from Austin, Texas. J. David Goodman contributed reporting
from Corpus Christi, Texas.

Advertisement

\protect\hyperlink{after-bottom}{Continue reading the main story}

\hypertarget{site-index}{%
\subsection{Site Index}\label{site-index}}

\hypertarget{site-information-navigation}{%
\subsection{Site Information
Navigation}\label{site-information-navigation}}

\begin{itemize}
\tightlist
\item
  \href{https://help.nytimes.com/hc/en-us/articles/115014792127-Copyright-notice}{©~2020~The
  New York Times Company}
\end{itemize}

\begin{itemize}
\tightlist
\item
  \href{https://www.nytco.com/}{NYTCo}
\item
  \href{https://help.nytimes.com/hc/en-us/articles/115015385887-Contact-Us}{Contact
  Us}
\item
  \href{https://www.nytco.com/careers/}{Work with us}
\item
  \href{https://nytmediakit.com/}{Advertise}
\item
  \href{http://www.tbrandstudio.com/}{T Brand Studio}
\item
  \href{https://www.nytimes.com/privacy/cookie-policy\#how-do-i-manage-trackers}{Your
  Ad Choices}
\item
  \href{https://www.nytimes.com/privacy}{Privacy}
\item
  \href{https://help.nytimes.com/hc/en-us/articles/115014893428-Terms-of-service}{Terms
  of Service}
\item
  \href{https://help.nytimes.com/hc/en-us/articles/115014893968-Terms-of-sale}{Terms
  of Sale}
\item
  \href{https://spiderbites.nytimes.com}{Site Map}
\item
  \href{https://help.nytimes.com/hc/en-us}{Help}
\item
  \href{https://www.nytimes.com/subscription?campaignId=37WXW}{Subscriptions}
\end{itemize}
