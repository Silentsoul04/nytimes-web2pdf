Sections

SEARCH

\protect\hyperlink{site-content}{Skip to
content}\protect\hyperlink{site-index}{Skip to site index}

\href{https://www.nytimes.com/section/world/asia}{Asia Pacific}

\href{https://myaccount.nytimes.com/auth/login?response_type=cookie\&client_id=vi}{}

\href{https://www.nytimes.com/section/todayspaper}{Today's Paper}

\href{/section/world/asia}{Asia Pacific}\textbar{}U.S. Imposes Sanctions
on Chinese Officials Over Mass Detention of Muslims

\url{https://nyti.ms/2W1oBs2}

\begin{itemize}
\item
\item
\item
\item
\item
\end{itemize}

Advertisement

\protect\hyperlink{after-top}{Continue reading the main story}

Supported by

\protect\hyperlink{after-sponsor}{Continue reading the main story}

\hypertarget{us-imposes-sanctions-on-chinese-officials-over-mass-detention-of-muslims}{%
\section{U.S. Imposes Sanctions on Chinese Officials Over Mass Detention
of
Muslims}\label{us-imposes-sanctions-on-chinese-officials-over-mass-detention-of-muslims}}

The measure, over human rights abuses against mainly the Uighur ethnic
group, is likely to ratchet up tensions between Washington and Beijing.

\includegraphics{https://static01.nyt.com/images/2020/07/09/us/politics/09dc-china-sanctions/merlin_164255088_49d26b52-e373-4a57-92b7-fb70c6056c67-articleLarge.jpg?quality=75\&auto=webp\&disable=upscale}

By \href{https://www.nytimes.com/by/pranshu-verma}{Pranshu Verma} and
\href{https://www.nytimes.com/by/edward-wong}{Edward Wong}

\begin{itemize}
\item
  July 9, 2020
\item
  \begin{itemize}
  \item
  \item
  \item
  \item
  \item
  \end{itemize}
\end{itemize}

\href{https://cn.nytimes.com/usa/20200710/trump-china-sanctions-uighurs/}{阅读简体中文版}\href{https://cn.nytimes.com/usa/20200710/trump-china-sanctions-uighurs/zh0-}{閱讀繁體中文版}

WASHINGTON --- The Trump administration imposed sanctions on Thursday on
multiple officials from China, including a senior member of the
Communist Party, over human rights abuses against the largely Muslim
Uighur minority, a move that is likely to inflame tensions between
Washington and Beijing.

The targets of the sanctions included Chen Quanguo --- a member of
China's 25-member ruling Politburo and party secretary of the Xinjiang
region --- and is likely to anger top officials in the Communist Party
given his stature. Other officials penalized include Zhu Hailun, a
former deputy party secretary for the region; Wang Mingshan, director of
the Xinjiang Public Security Bureau; and Huo Liujun, a former party
secretary of the bureau. The bureau also faces sanctions.

In recent months, Trump administration officials have criticized Beijing
for its response to the coronavirus pandemic as well as its efforts to
suppress pro-democracy movements in Hong Kong and its mass detention of
Uighurs and other ethnic minorities.

``The United States will not stand idly by as the C.C.P. carries out
human rights abuses targeting Uighurs, ethnic Kazakhs and members of
other minority groups in Xinjiang,'' Secretary of State Mike Pompeo said
in a statement on Thursday, referring to the Chinese Communist Party.

A spokesman of the Chinese foreign ministry, Zhao Lijian, said on Friday
that in response, Beijing would take reciprocal measures against the
relevant U.S. institutions and individuals for ``egregious'' conduct on
Xinjiang-related issues.

The sanctions against Chinese officials were levied under the Global
Magnitsky Human Rights Accountability Act, which was passed in 2016 and
gives the United States the ability to impose human rights penalties on
foreign officials. But the measures appear largely symbolic, as none of
the officials are likely to hold significant assets outside China.

The move also comes after talks first
\href{https://www.nytimes.com/2018/09/10/world/asia/us-china-sanctions-muslim-camps.html}{arose
in 2018} in the Trump administration to punish senior Chinese officials
and companies for the detention of ethnic Uighurs and other minority
Muslims in large internment camps. But those discussions languished as
trade advisers in the administration tried to negotiate an end to the
trade war with Beijing.

For purposes of his re-election campaign, President Trump
\href{https://www.nytimes.com/2019/05/04/world/asia/trump-china-uighurs-trade-deal.html}{was
focused on securing a deal} that would include a commitment by China to
increase its purchases of American agricultural products, according to a
recent book by John R. Bolton, the former national security adviser, and
private accounts by other officials.

Mr. Trump showed no qualms about prioritizing trade talks with China
while ignoring human rights abuses in the country. He even told
President Xi Jinping of China to continue building the internment camps
used to detain Muslims --- ``which Trump thought was exactly the right
thing to do,''
\href{https://www.nytimes.com/2020/06/18/us/politics/trump-china-bolton.html}{according
to Mr. Bolton's book}.

In October 2019, the Trump administration imposed visa restrictions on
some Chinese officials and import controls on certain organizations in
the western region of Xinjiang. While those were the first punishments
imposed by any government in relation to the vast human rights abuses
there, they were fairly weak even though some Americans officials have
advocated harsher measures.

The actions on Thursday target a cluster of officials who played a major
role in devising and enforcing policies in Xinjiang that have detained
hundreds of thousands --- some estimates put it at more than a million
--- members of largely Muslim ethnic minorities in indoctrination camps,
while also smothering those groups under a net of surveillance.

Rayhan Asat, a Uighur lawyer who is a United States resident in
Washington, said sanctions imposed under the Magnitsky Act allowed the
United States to hold Chinese officials accountable for what she called
genocide in Xinjiang. Her younger brother, Ekpar Asat,
\href{https://www.nytimes.com/2020/05/09/us/politics/china-uighurs-arrest.html}{was
detained by security officials} after he returned to Xinjiang in 2016
following a visit to the United States on a State Department cultural
exchange program. He was reportedly sentenced to 15 years in prison on
criminal charges.

``Today's decision sends a clear message to the perpetrators that they
cannot continue to commit the crime of all crimes with impunity, to
victims like my brother Ekpar Asat that they are not forgotten, and to
the bystander countries to follow suit,'' Ms. Asat added.

Mr. Chen, the most prominent of the four officials facing sanctions, has
been the Communist Party secretary of Xinjiang since August 2016. He
oversaw a rise in mass detentions of Uighurs, Kazakhs and other Muslim
minorities, and consequently was named as a potential target of American
sanctions in a congressional act on the Uighur issue signed into law
last month. The New York Times reported last year on government
\href{https://www.nytimes.com/interactive/2019/11/16/world/asia/china-xinjiang-documents.html}{documents
from the Xinjiang region}that described how Mr. Chen, who previously
served as a party chief of Tibet, ordered officials to ``round up
everyone who should be rounded up.''

``Chen Quanguo is truly one of the worst human rights abusers in the
world today, and he cut his repressive teeth in Tibet,'' Matteo Mecacci,
the president of the International Campaign for Tibet, said in response
to the announcement on Thursday. ``By developing a model of intense
security and forced assimilation in the Tibet Autonomous Region, then
implementing and expanding on that model in Xinjiang, Chen has inflicted
untold suffering on millions of Tibetans, Uighurs and other non-Chinese
ethnic groups.''

Another official facing sanctions, Mr. Zhu, led a Communist Party
law-and-order committee in Xinjiang from 2016 until early last year. Mr.
Zhu appears to have played an important role in the mass-detention
drive, urging officials across the region and helping them cope with the
practicalities of rapidly confining hundreds of thousands of people.

In 2017, a directive signed by Mr. Zhu called recent terrorist attacks
in Britain ``a warning and a lesson for us.'' It blamed the British
government's ``excessive emphasis on `human rights above security,' and
inadequate controls on the propagation of extremism on the internet and
in society.''

Mr. Huo and Mr. Wang, the two remaining officials penalized by the
Treasury Department, were senior police officials in Xinjiang who helped
introduce the surveillance programs and technology that have constricted
Uighurs and other minority members, tracking their movements, recording
their visits to sensitive sites like mosques and collecting their DNA
and other biometric information.

Chinese officials have repeatedly defended the indoctrination camps,
which are intended to break down inmates' devotion to Islam, deter any
``separatist'' tendencies and turn people into loyal supporters of the
Communist Party.

Officials have described the camps as humane vocational training centers
that have helped extinguish extremist violence in Xinjiang. Testimony
from former inmates and official records unearthed by researchers and
journalists present a much bleaker picture of the camps, including harsh
conditions, forced labor and the wrenching separation of families.

``The centers are managed as boarding schools where trainees may go home
on a regular basis,'' China's Ministry of Foreign Affairs said in a
\href{https://www.fmprc.gov.cn/mfa_eng/zxxx_662805/t1794581.shtml}{long
statement last week} defending the country's human rights record.
``Trainees' freedom of religious belief is fully respected and protected
at the centers.''

But for some human rights groups that have been fighting for justice for
the Uighur community, the Trump administration's actions on Thursday was
a long-awaited breakthrough.

``A global response is long overdue,'' said Omer Kanat, the executive
director of the Uyghur Human Rights Project. ``This is the beginning of
the end of impunity for the Chinese government.''

Pranshu Verma reported from Washington, and Edward Wong from Lewes, Del.

Advertisement

\protect\hyperlink{after-bottom}{Continue reading the main story}

\hypertarget{site-index}{%
\subsection{Site Index}\label{site-index}}

\hypertarget{site-information-navigation}{%
\subsection{Site Information
Navigation}\label{site-information-navigation}}

\begin{itemize}
\tightlist
\item
  \href{https://help.nytimes.com/hc/en-us/articles/115014792127-Copyright-notice}{©~2020~The
  New York Times Company}
\end{itemize}

\begin{itemize}
\tightlist
\item
  \href{https://www.nytco.com/}{NYTCo}
\item
  \href{https://help.nytimes.com/hc/en-us/articles/115015385887-Contact-Us}{Contact
  Us}
\item
  \href{https://www.nytco.com/careers/}{Work with us}
\item
  \href{https://nytmediakit.com/}{Advertise}
\item
  \href{http://www.tbrandstudio.com/}{T Brand Studio}
\item
  \href{https://www.nytimes.com/privacy/cookie-policy\#how-do-i-manage-trackers}{Your
  Ad Choices}
\item
  \href{https://www.nytimes.com/privacy}{Privacy}
\item
  \href{https://help.nytimes.com/hc/en-us/articles/115014893428-Terms-of-service}{Terms
  of Service}
\item
  \href{https://help.nytimes.com/hc/en-us/articles/115014893968-Terms-of-sale}{Terms
  of Sale}
\item
  \href{https://spiderbites.nytimes.com}{Site Map}
\item
  \href{https://help.nytimes.com/hc/en-us}{Help}
\item
  \href{https://www.nytimes.com/subscription?campaignId=37WXW}{Subscriptions}
\end{itemize}
