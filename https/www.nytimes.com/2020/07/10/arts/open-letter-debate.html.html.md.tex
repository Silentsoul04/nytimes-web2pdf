Sections

SEARCH

\protect\hyperlink{site-content}{Skip to
content}\protect\hyperlink{site-index}{Skip to site index}

\href{https://www.nytimes.com/section/arts}{Arts}

\href{https://myaccount.nytimes.com/auth/login?response_type=cookie\&client_id=vi}{}

\href{https://www.nytimes.com/section/todayspaper}{Today's Paper}

\href{/section/arts}{Arts}\textbar{}An Open Letter on Free Expression
Draws a Counterblast

\url{https://nyti.ms/2OqOwW9}

\begin{itemize}
\item
\item
\item
\item
\item
\end{itemize}

Advertisement

\protect\hyperlink{after-top}{Continue reading the main story}

Supported by

\protect\hyperlink{after-sponsor}{Continue reading the main story}

\hypertarget{an-open-letter-on-free-expression-draws-a-counterblast}{%
\section{An Open Letter on Free Expression Draws a
Counterblast}\label{an-open-letter-on-free-expression-draws-a-counterblast}}

A few days after more than 150 cultural luminaries warned of a growing
``intolerant climate,'' another group responded with a pointed letter of
its own.

\includegraphics{https://static01.nyt.com/images/2020/07/10/arts/10HARPERS-RESPONSE-PROMO/10HARPERS-RESPONSE-PROMO-articleLarge.jpg?quality=75\&auto=webp\&disable=upscale}

\href{https://www.nytimes.com/by/jennifer-schuessler}{\includegraphics{https://static01.nyt.com/images/2018/02/16/multimedia/author-jennifer-schuessler/author-jennifer-schuessler-thumbLarge-v2.png}}

By \href{https://www.nytimes.com/by/jennifer-schuessler}{Jennifer
Schuessler}

\begin{itemize}
\item
  July 10, 2020
\item
  \begin{itemize}
  \item
  \item
  \item
  \item
  \item
  \end{itemize}
\end{itemize}

Three days after an open letter signed by more than 150 cultural
luminaires darkly warning of a growing ``intolerant climate'' stirred
intense response on the internet, another group issued a counterblast on
Friday accusing them of elitism, hypocrisy and complicity in the
bullying they decry.

The first letter, titled ``A Letter on Justice and Open Debate,'' was
\href{https://www.nytimes.com/2020/07/07/arts/harpers-letter.html?searchResultPosition=1}{posted
online} on Tuesday by Harper's Magazine. Signed by prominent figures in
the arts, media and academia, including Margaret Atwood, Wynton Marsalis
and J.K. Rowling, it warned of a growing tide of illiberalism and a
weakening of ``our norms of open debate and toleration of differences in
favor of ideological conformity.''

The response letter, titled
\href{https://theobjective.substack.com/p/a-more-specific-letter-on-justice}{``A
More Specific Letter on Justice and Open Debate,''} chided the Harper's
statement for what it characterized as lofty generalities, as well as
ignoring the realities of who actually gets to be heard. If its more
than 150 signers were far less well-known, that was perhaps part of the
point.

The Harper's letter ``does not deal with the problem of power: who has
it and who does not,'' according to the response, published at The
Objective, a news and commentary site that explores ``how journalism has
interacted with historically ignored communities.''

``Harper's has decided to bestow its platform not to marginalized
people,'' it said, ``but to people who already have large followings and
plenty of opportunities to make their views heard.''

It continued: ``The letter reads as a caustic reaction to a diversifying
industry --- one that's starting to challenge diversifying norms that
have protected bigotry. The writers of the letter use seductive but
nebulous concepts and coded language to obscure the actual meaning
behind their words.''

Almost as soon as it appeared on Tuesday, ``That Letter,'' as Twitter
quickly began calling the Harper's statement, set off rounds of debate
about free speech, privilege and
\href{https://www.vox.com/culture/2019/12/30/20879720/what-is-cancel-culture-explained-history-debate}{the
existence or nonexistence of so-called cancel culture}.

Akela Lacy, a politics reporter at The Intercept who signed and helped
edit the counter-letter, said it grew organically out of a conversation
in a Slack channel called Journalists of Color. Initially, there was
some wariness of feeding what she and others on Twitter wryly referred
to as ``letter discourse.''

``There are so many more important things going on in media right now,''
Ms. Lacy said, citing in particular threats and harassment experienced
by journalists from marginalized groups.

``But the fact is there are a lot of people, particularly Black and
trans, expressing very valid concerns about the climate right now,'' she
said. ``Letting this very lofty position go unanswered didn't feel like
it was benefiting anyone.''

The prominence of the Harper's signers has been a flash point in the
conversation, with some deriding that letter as the whining of
\href{https://www.thedailybeast.com/jk-rowling-and-other-assorted-rich-fools-want-to-cancel-cancel-culture}{``assorted
rich fools,''} as a writer for The Daily Beast put it. The response
letter characterized it as a defense of ``the intellectual freedom of
cis white intellectuals,'' which ``has never been under threat en
masse.''

On Friday, after the response letter was posted, the writer Thomas
Chatterton Williams, who spearheaded the Harper's letter, highlighted
the
\href{https://twitter.com/thomaschattwill/status/1281598079490297861?s=20}{more
than two dozen} Black and other nonwhite intellectuals who signed his
letter.

``You know, just a bunch of privileged solipsistic elites worrying about
problems that don't exist,'' Mr. Williams, who is Black,
\href{https://twitter.com/thomaschattwill/status/1281648311389184000?s=20}{tweeted}.
``So far, haven't seen any of the formerly imprisoned signatories or the
ones who have experienced fatwas cave to the social media backlash,
though,'' he added.

His dig was a reference to the fact that criticism of the Harper's
letter centered as much on who signed it as its content. And within
hours of its publication, some who had signed distanced themselves from
it, saying they would not have joined if they had been aware of some of
the other signers. The inclusion of J.K. Rowling,
\href{https://www.nytimes.com/2020/06/12/style/jk-rowling-transgender-fans.html}{who
has drawn condemnation} for a series of recent comments widely seen as
anti-transgender, drew particular ire.

The new letter included one person,
\href{https://www.nytimes.com/2019/11/13/books/review-black-radical-william-monroe-trotter-kerri-greenidge.html}{the
historian Kerri Greenidge}, who had signed the Harper's letter,
according to emails reviewed by The New York Times, but then asked that
her name be removed, saying on Twitter, ``I do not endorse this @Harpers
letter.''

It also included a number of people signing anonymously, including three
listed as journalists at The New York Times. (The Harper's letter was
signed by four Opinion columnists at The Times, who used their names.)

Ms. Lacy said she was aware of the ``irony'' of an open letter that
included redacted signatures, but said that some people who criticized
the Harper's letter had gotten threats or feared workplace retaliation.

``There's a difference between being canceled in the way Harper's letter
is talking about and actually getting threats of violence,'' she said.

Advertisement

\protect\hyperlink{after-bottom}{Continue reading the main story}

\hypertarget{site-index}{%
\subsection{Site Index}\label{site-index}}

\hypertarget{site-information-navigation}{%
\subsection{Site Information
Navigation}\label{site-information-navigation}}

\begin{itemize}
\tightlist
\item
  \href{https://help.nytimes.com/hc/en-us/articles/115014792127-Copyright-notice}{©~2020~The
  New York Times Company}
\end{itemize}

\begin{itemize}
\tightlist
\item
  \href{https://www.nytco.com/}{NYTCo}
\item
  \href{https://help.nytimes.com/hc/en-us/articles/115015385887-Contact-Us}{Contact
  Us}
\item
  \href{https://www.nytco.com/careers/}{Work with us}
\item
  \href{https://nytmediakit.com/}{Advertise}
\item
  \href{http://www.tbrandstudio.com/}{T Brand Studio}
\item
  \href{https://www.nytimes.com/privacy/cookie-policy\#how-do-i-manage-trackers}{Your
  Ad Choices}
\item
  \href{https://www.nytimes.com/privacy}{Privacy}
\item
  \href{https://help.nytimes.com/hc/en-us/articles/115014893428-Terms-of-service}{Terms
  of Service}
\item
  \href{https://help.nytimes.com/hc/en-us/articles/115014893968-Terms-of-sale}{Terms
  of Sale}
\item
  \href{https://spiderbites.nytimes.com}{Site Map}
\item
  \href{https://help.nytimes.com/hc/en-us}{Help}
\item
  \href{https://www.nytimes.com/subscription?campaignId=37WXW}{Subscriptions}
\end{itemize}
