Sections

SEARCH

\protect\hyperlink{site-content}{Skip to
content}\protect\hyperlink{site-index}{Skip to site index}

\href{https://myaccount.nytimes.com/auth/login?response_type=cookie\&client_id=vi}{}

\href{https://www.nytimes.com/section/todayspaper}{Today's Paper}

\href{/section/opinion}{Opinion}\textbar{}Testing, Testing

\href{https://nyti.ms/2ZX1noe}{https://nyti.ms/2ZX1noe}

\begin{itemize}
\item
\item
\item
\item
\item
\item
\end{itemize}

Advertisement

\protect\hyperlink{after-top}{Continue reading the main story}

\href{/section/opinion}{Opinion}

Supported by

\protect\hyperlink{after-sponsor}{Continue reading the main story}

\hypertarget{testing-testing}{%
\section{Testing, Testing}\label{testing-testing}}

You want a coronavirus test? How about next month?

By Dave Eggers

Mr. Eggers is a novelist and satirist.

\begin{itemize}
\item
  July 10, 2020
\item
  \begin{itemize}
  \item
  \item
  \item
  \item
  \item
  \item
  \end{itemize}
\end{itemize}

\includegraphics{https://static01.nyt.com/images/2020/07/10/opinion/10eggers/10eggers-articleLarge.jpg?quality=75\&auto=webp\&disable=upscale}

\textbf{Q:} I think I have it.

\textbf{A:} Have what?

\textbf{Q:} \emph{It}. I've got extreme fatigue, migraines, chills,
aches, nausea and a fever of 102.

\textbf{A:} Are we talking about coronavirus?

\textbf{Q:} We are. I'm worried. I'm 50. People my age are dying.

\textbf{A:} That does sound concerning. Let's get you tested.

\textbf{Q:} OK, I'm ready.

\textbf{A:} You mean now?

\textbf{Q:} Of course.

\textbf{A:} Oh, you can't do one \emph{now}.

\textbf{Q:} Why not?

\textbf{A:} How's late next week look for you?

\textbf{Q:} Late next week? I'm sick \emph{today.}

\textbf{A:} We have three appointments in mid July. Wait. Those were
just taken. How's your end-of-month?

\textbf{Q:} We're four months into the pandemic. It still takes that
long to get a test?

\textbf{A:} It depends. Looks like Tulsa has a drive-through thingie
tomorrow. Are you anywhere near Tulsa, Okla.?

\textbf{Q:} No.

\textbf{A:} Keystone, S.D.?

\textbf{Q:} No.

\textbf{A:} Well, then it could take longer. Where are you?

\textbf{Q:} San Francisco.

\textbf{A:} Oh, then it'll be a \emph{lot} longer. Let me make sure
\ldots{} Let's see \ldots{} Typing in `San Francisco' \ldots{} Is that
two S's or two C's? No, I got it. Whoa, looks like a lot of people want
tests where you are.

\textbf{Q:} And you don't have enough?

\textbf{A:} Oh, we have plenty of \emph{tests}. We just don't have
\emph{appointments}. You need an appointment to get a test, and the
appointments --- these we don't have.

\textbf{Q:} Until the end of July.

\textbf{A:} Well, see, while we've been talking, those have been taken.
How's early August?

\textbf{Q:} But I won't be sick by then, will I?

\textbf{A:} Let's hope not! If you're sick that long, you'd know you had
the virus for sure.

\textbf{Q:} So if we wait two weeks before testing me, I might not be
sick, and then the test won't work. Shouldn't I know \emph{now} if I
have the virus \emph{now}?

\textbf{A:} Oh, definitely. That would be a big help for you and your
family, I'm betting. Otherwise you might be living in four-dimensional
terror and endlessly self-quarantining for no reason. Are you
quarantining?

\textbf{Q:} I am.

\textbf{A:} Does your head feel like an 80-pound melon being stabbed by
machetes that are serrated and also on fire?

\textbf{Q:} It does.

\textbf{A:} Is your family afraid to be near you? See you?

\textbf{Q:} They are.

\textbf{A:} Good. Now all you need is a test.

\textbf{Q:} But there are no tests.

\textbf{A:} I just told you, there are plenty of \emph{tests}. So many
beautiful tests! Just no appointments.

\textbf{Q:} But you have some in early August?

\textbf{A:} Early August? You know what? To be suresies, let's say
mid-August. And ideally you're still sick. Otherwise there won't be any
point.

\textbf{Q:} So in mid-August, I come in and get a test?

\textbf{A:} Absolutely!

\textbf{Q:} And get a result?

\textbf{A:} Eventually!

\textbf{Q:} It sounded like you just said ``eventually.''

\textbf{A:} Did I say that? That does sound like something I'd say.

\textbf{Q:} How long does it take to get the results?

\textbf{A:} Not long at all! This information is of the utmost urgency.
So we're thinking five days. Or seven. Maybe 10? No more than 12. Two
weeks, tops.

\textbf{Q:} And then you'll call me?

\textbf{A:} Of course we will. Unless we don't. Check the website. Or
the app! The app is pretty sweet. Please don't call us.

\textbf{Q:} And from the app, I get the results?

\textbf{A:} Sure. When they become available.

\textbf{Q:} So I should check the app often?

\textbf{A:} I should say so! But that's just if you're concerned about
your health, your possible death, the fate of your family, and the
global struggle against this plague.

\textbf{Q:} I'll check every 10 minutes. For 12 days?

\textbf{A:} Didn't we say two weeks?

\textbf{Q:} So almost a month until I get an appointment, then 14 days
to get a result. And in the meantime I self-quarantine?

\textbf{A:} Right. And then, sometime in September, you'll know for sure
whether you had Covid-19 in early July. Unless it's a false negative.

\textbf{Q:} Wait. False negative?

\textbf{A:} That's when you have it but the test doesn't show it. So
maybe just assume you have it. And had it. And will always have it.

\textbf{Q:} But if I did have it, I'd have antibodies, right?

\textbf{A:} Absolutely. Maybe. Do you have them?

\textbf{Q:} I don't know yet.

\textbf{A:} You should get a test.

\textbf{Q:} Can I get one?

\textbf{A:} Of course! But you need an appointment.

\textbf{Q:} Can't I get it at the same time as the Covid-19 test? Wait,
why are you laughing?

\textbf{A:} I'm just \ldots{} It's nothing. I mean. It's just that
\ldots{} You know this isn't Denmark, right?

\textbf{Q:} I do know that.

\textbf{A:} OK, because, I mean, I just wanted to make sure. The virus
does funny things to people's brains.

\textbf{Q:} So I need a different appointment?

\textbf{A:} Of course you need a different appointment. How's your
October? The thing with the antibodies is that they don't show up right
away. That's if they show up at all. So it's good to wait.

\textbf{Q:} I'm happy to wait if it means I have antibodies and can't
get the virus again.

\textbf{A:} Oh for sure you can get the virus again! We think. Maybe.

\textbf{Q:} I just wanted some assurance that I'm immune.

\textbf{A:} Assurance? Oh, you won't have that! Of all the tests, the
antibody tests are the least reliable.

\textbf{Q:} They are?

\textbf{A:} Oh, man, some are just plain bad. At the beginning, the
government let anyone make them and sell them without F.D.A. review.
Most of them are terrible. I think Fisher-Price made one.

\textbf{Q:} Can we do one of the \emph{good} tests?

\textbf{A:} We can try! I can make a note to that effect. Let me just
type that in \ldots{} ``Would prefer to get one of \ldots{} good tests.
\ldots'' Got it. When would you like to do it?

\textbf{Q:} You'd mentioned October.

\textbf{A:} October's booked. Can we say November?

\textbf{Q:} And the results?

\textbf{A:} The good thing with the antibody test is you get the results
immediately.

\textbf{Q:} Excellent.

\textbf{A:} But they mean nothing.

\textbf{Q:} Oh.

\textbf{A:} It could be that you're never immune.

\textbf{Q:} Right.

\textbf{A:} And remember that the Covid-19 test you're taking could mean
nothing, too, because it only works if you're in the thick of the
illness. So if you wait till your symptoms are gone before you take the
Covid test, you won't know if you had the virus until you take the
antibody test, which also tells us nothing.

\textbf{Q:} So I'll never know what's causing my fatigue, migraines,
chills, aches, nausea and a fever of 102.

\textbf{A:} You hear anything about this new thing coming out of
Mongolia? Bubonic something? I'm thinking we keep our eyes on that.

Dave Eggers is the author of several books, including ``The Captain and
the Glory.''

\emph{The Times is committed to publishing}
\href{https://www.nytimes.com/2019/01/31/opinion/letters/letters-to-editor-new-york-times-women.html}{\emph{a
diversity of letters}} \emph{to the editor. We'd like to hear what you
think about this or any of our articles. Here are some}
\href{https://help.nytimes.com/hc/en-us/articles/115014925288-How-to-submit-a-letter-to-the-editor}{\emph{tips}}\emph{.
And here's our email:}
\href{mailto:letters@nytimes.com}{\emph{letters@nytimes.com}}\emph{.}

\emph{Follow The New York Times Opinion section on}
\href{https://www.facebook.com/nytopinion}{\emph{Facebook}}\emph{,}
\href{http://twitter.com/NYTOpinion}{\emph{Twitter (@NYTopinion)}}
\emph{and}
\href{https://www.instagram.com/nytopinion/}{\emph{Instagram}}\emph{.}

Advertisement

\protect\hyperlink{after-bottom}{Continue reading the main story}

\hypertarget{site-index}{%
\subsection{Site Index}\label{site-index}}

\hypertarget{site-information-navigation}{%
\subsection{Site Information
Navigation}\label{site-information-navigation}}

\begin{itemize}
\tightlist
\item
  \href{https://help.nytimes.com/hc/en-us/articles/115014792127-Copyright-notice}{©~2020~The
  New York Times Company}
\end{itemize}

\begin{itemize}
\tightlist
\item
  \href{https://www.nytco.com/}{NYTCo}
\item
  \href{https://help.nytimes.com/hc/en-us/articles/115015385887-Contact-Us}{Contact
  Us}
\item
  \href{https://www.nytco.com/careers/}{Work with us}
\item
  \href{https://nytmediakit.com/}{Advertise}
\item
  \href{http://www.tbrandstudio.com/}{T Brand Studio}
\item
  \href{https://www.nytimes.com/privacy/cookie-policy\#how-do-i-manage-trackers}{Your
  Ad Choices}
\item
  \href{https://www.nytimes.com/privacy}{Privacy}
\item
  \href{https://help.nytimes.com/hc/en-us/articles/115014893428-Terms-of-service}{Terms
  of Service}
\item
  \href{https://help.nytimes.com/hc/en-us/articles/115014893968-Terms-of-sale}{Terms
  of Sale}
\item
  \href{https://spiderbites.nytimes.com}{Site Map}
\item
  \href{https://help.nytimes.com/hc/en-us}{Help}
\item
  \href{https://www.nytimes.com/subscription?campaignId=37WXW}{Subscriptions}
\end{itemize}
