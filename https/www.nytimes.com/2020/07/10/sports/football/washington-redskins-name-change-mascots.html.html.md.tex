Sections

SEARCH

\protect\hyperlink{site-content}{Skip to
content}\protect\hyperlink{site-index}{Skip to site index}

\href{https://www.nytimes.com/section/sports/football}{Pro Football}

\href{https://myaccount.nytimes.com/auth/login?response_type=cookie\&client_id=vi}{}

\href{https://www.nytimes.com/section/todayspaper}{Today's Paper}

\href{/section/sports/football}{Pro Football}\textbar{}In Campaign
Against Racism, Team Names Get New Scrutiny

\url{https://nyti.ms/2AQuwJg}

\begin{itemize}
\item
\item
\item
\item
\item
\end{itemize}

Advertisement

\protect\hyperlink{after-top}{Continue reading the main story}

Supported by

\protect\hyperlink{after-sponsor}{Continue reading the main story}

\hypertarget{in-campaign-against-racism-team-names-get-new-scrutiny}{%
\section{In Campaign Against Racism, Team Names Get New
Scrutiny}\label{in-campaign-against-racism-team-names-get-new-scrutiny}}

It is not just the N.F.L.'s Washington team that could get a name
change. A number of schools are also reconsidering nicknames, though
some are resisting any switch.

\includegraphics{https://static01.nyt.com/images/2020/07/13/sports/10unrest-mascotnames-print/merlin_174429429_c430a576-c058-4866-a1d9-3ae0782646fb-articleLarge.jpg?quality=75\&auto=webp\&disable=upscale}

By Gillian R. Brassil, Giulia McDonnell Nieto del Rio,
\href{https://www.nytimes.com/by/billy-witz}{Billy Witz} and
\href{https://www.nytimes.com/by/david-waldstein}{David Waldstein}

\begin{itemize}
\item
  Published July 10, 2020Updated July 12, 2020
\item
  \begin{itemize}
  \item
  \item
  \item
  \item
  \item
  \end{itemize}
\end{itemize}

Guy Jones had been advocating for 21 long years, hoping for the day when
Anderson High School, outside Cincinnati, would drop the nickname he
found vulgar and insulting to his Native American heritage.

A member of the Hunkpapa Lakota nation from the Standing Rock
reservation in South Dakota, he and three white Anderson students in
1999 sought to persuade the school board to change the nickname that
``made me cringe every time I heard it.''

The school board voted down their proposal, and did so in 2003 and in
2018. But last week it finally relented and voted to erase the name for
good, in part, Jones said, because of a wave of momentum fueled by the
Black Lives Matter movement.

``Things are changing,'' he said, ``and I'm glad they are.''

The very next day, Jones heard more news that cheered him out of
Washington, where for years the
\href{https://www.nytimes.com/2020/07/13/sports/football/washington-redskins-new-name.html}{N.F.L.
team that uses the same nickname, Redskins}, had vowed never to change
it.

But in a surprise move, Daniel Snyder, the Washington owner, promised to
conduct a ``thorough review'' of the name. With that, pressure mounted
on other teams to respond. Within hours the Cleveland Indians vowed to
engage with community and Native American leaders on the topic, too.

Activists have waged a long and, at times, frustrating campaign to
persuade teams to change names, logos and mascots. But suddenly, after
so much resistance, a new willingness to reconsider long-cherished names
and logos at the professional, college and high school levels, is giving
fresh impetus to change team names long considered untouchable.

Some of the pressure is coming from corporate sponsors like Nike and
FedEx, whose request for Washington to change the name preceded the
team's announcement. Amazon, Target and Walmart have
also\href{https://www.nytimes.com/2020/07/06/sports/football/washington-team-name-change.html}{moved
to drop the team's merchandise}.

``It's happening in the context of the Black Lives Matter movement and
all that implies of Black people and Indigenous people in our country,''
said Suzan Shown Harjo, a Native American activist, who has been at the
forefront of the movement to change the Washington team's name. ``What's
happening right now is such a broad swath of society.''

\includegraphics{https://static01.nyt.com/images/2020/07/10/sports/10unrest-mascotnames-2/merlin_174427842_a2511fa6-9a75-483e-afed-f8eec05428f6-articleLarge.jpg?quality=75\&auto=webp\&disable=upscale}

As Heather Miller, the executive director of the American Indian Center
in Chicago said, ``This is an especially critical moment right now.''
Philip Yenyo, the executive director of the American Indian Movement of
Ohio, who has been protesting the Cleveland baseball team's name for
almost 30 years, added: ``We need to strike while the iron is hot.''

For decades, groups like these have appealed to professional sports
teams, colleges and high schools to eliminate names and logos that they
say are dehumanizing, disrespectful and racist.

The results have been mixed, and resistance to change has been fierce,
aided recently by President Trump, who scolded protesters in a recent
tweet by saying the
\href{https://www.nytimes.com/2020/07/06/us/politics/trump-bubba-wallace-nascar.html}{nicknames
project strength} and should not be changed.

Raymond Wood II, a Native American and longtime Republican town
councilman in Killingly, Conn., said he was not offended by the
Washington nickname or the name of his hometown high school, the
Killingly Redmen.

``I doubt there was any malice in people's hearts when they formulated
these teams and selected the imagery that they thought would best
represent them,'' he said in a Facebook message. ``If there was, it
backfired.''

Citing arguments like that, holdouts have resisted calls to change,
including the Atlanta Braves, the Chicago Blackhawks and the Kansas City
Chiefs.

The Braves and the Blackhawks both issued statements recently indicating
they would not alter their names, but said they planned to work harder
with Native American groups to promote awareness and respect. They also
reiterated the common refrain of teams with Native names, that they
honor the heritage of those peoples.

``There is absolutely no evidence of that,'' countered Stephanie
Fryberg, a professor of psychology at the University of Michigan and a
member of the Tulalip nation in Washington State. She recently published
a study showing that about two-thirds of Native Americans who frequently
engage in cultural practices are offended by the names and logos, and
also some of the behavior of fans of those teams.

``When you use a person's identity in a sports domain,'' she said, ``and
you allow people to dress in red face and put on headdresses and dance
and chant a Hollywood made-up song that mocks Native tradition and
culture, there is no way to call that honoring.''

Many of the people opposed to Washington's name believe it gives cover
to a host of college and school teams that use it, too. But Anderson's
vote could indicate that is cracking.

In the days since Anderson's school board voted, 4-1, to abandon the
moniker, news reports from around the country point to other schools
considering dropping Native American mascots.

More than 2,200 high schools use Native imagery in their school names
and mascots, according to Mascot DB, a database of team names. That is
600 fewer than once existed --- a trend that was accelerated after a
2014 ruling by the U.S. Trademark Trial and Appeal Board that voided the
Redskins trademark as ``disparaging of Native Americans.'' The ruling
was later
\href{https://www.nytimes.com/2017/06/19/us/politics/supreme-court-trademarks-redskins.html}{overturned}.

It is not only Native Americans names and imagery that cause offense.
Some schools employ names and iconography celebrating the Confederacy,
and with statues to Confederate generals toppling and Mississippi taking
the Confederate bars off its state flag, school names could be next.

Lee-Davis High School in Mechanicsville, Va., is known as the
Confederates, and a local N.A.A.C.P. chapter said it would revive an
effort to change the name.

Image

Neshaminy High School in Langhorne, Pa., has retained its nickname
despite opposition.Credit...Matt Rourke/Associated Press

Avi Hopkins, a Black running back who graduated in 1994, recalled his
anguish when, after scoring touchdowns, cries of, ``Go Confederates,''
would ring in his ears, a needling reminder that the school's name
honored the pillars of the Confederacy, Robert E. Lee and Jefferson
Davis.

``It really broke me in half,'' said Hopkins, who later played at the
Virginia Military Institute. ``I knew that my success was bringing a
positive light to men who negatively impacted my ancestors.''

But much of the momentum to change mascots or team names has focused on
Native American references, particularly the name the Washington team
uses.

It has defenders among schools that use it, too.

Loudon High School near Knoxville, Tenn, is one. Jeff Harig, the
football coach, said it was originally adopted to respectfully reflect
the area's Cherokee Nation heritage and history.

``I would like to think that we embody the positives of the Native
American culture,'' he said. ``So for me personally, I hope we can keep
the Redskins name.''

In Bucks County, Pa., Neshaminy High School also goes by that nickname
(one of its rivals is the Council Rock North Indians). In 2013 Donna
Fann-Boyle, who is Native American, filed a complaint with the
Pennsylvania Human Relations Commission, charging that the nickname was
racist and encouraged harmful behaviors at the school, where her son
graduated in 2016.

When she was revealed publicly as the person who filed the complaint,
people in support of the mascot threatened her and mocked her Native
heritage, she said. Worried that the issue would overshadow her son's
senior year, Ms. Fann-Boyle withdrew the complaint.

In the current climate, however, she hopes the school board may be
willing to reconsider and follow the lead of Washington's N.F.L. team.

``Come on, the world is changing,'' she said of the school board. ``What
are you going to do?''

While colleges have been at the forefront to replace Native American
nicknames --- over the years, Miami University of Ohio, Stanford, Siena
and St. John's all dropped such monikers --- Florida State has held
steadfastly to the name Seminole.

It has asserted that the nickname embraces the culture of the Seminole
nation, and it has garnered the group's blessing by, among other things,
offering a class on Seminole history and incorporating tribal designs
into the trim on their jerseys.

San Diego State University has slow-walked away from Indigenous
iconography over the last 20 years --- all while clinging to its Aztecs
nickname.

``There's no face painting, we don't allow the tomahawk chop,'' said
Ramona Perez, an anthropology professor at San Diego State, who has not
taken a position on the nickname.

But Ozzie Monge, a former San Diego State lecturer who led an
unsuccessful campaign to change the name, suggested the school was
reluctant to tamper much with the branding of teams.

``There's one answer why they haven't changed it,'' he said. ''Sports.''

Advertisement

\protect\hyperlink{after-bottom}{Continue reading the main story}

\hypertarget{site-index}{%
\subsection{Site Index}\label{site-index}}

\hypertarget{site-information-navigation}{%
\subsection{Site Information
Navigation}\label{site-information-navigation}}

\begin{itemize}
\tightlist
\item
  \href{https://help.nytimes.com/hc/en-us/articles/115014792127-Copyright-notice}{©~2020~The
  New York Times Company}
\end{itemize}

\begin{itemize}
\tightlist
\item
  \href{https://www.nytco.com/}{NYTCo}
\item
  \href{https://help.nytimes.com/hc/en-us/articles/115015385887-Contact-Us}{Contact
  Us}
\item
  \href{https://www.nytco.com/careers/}{Work with us}
\item
  \href{https://nytmediakit.com/}{Advertise}
\item
  \href{http://www.tbrandstudio.com/}{T Brand Studio}
\item
  \href{https://www.nytimes.com/privacy/cookie-policy\#how-do-i-manage-trackers}{Your
  Ad Choices}
\item
  \href{https://www.nytimes.com/privacy}{Privacy}
\item
  \href{https://help.nytimes.com/hc/en-us/articles/115014893428-Terms-of-service}{Terms
  of Service}
\item
  \href{https://help.nytimes.com/hc/en-us/articles/115014893968-Terms-of-sale}{Terms
  of Sale}
\item
  \href{https://spiderbites.nytimes.com}{Site Map}
\item
  \href{https://help.nytimes.com/hc/en-us}{Help}
\item
  \href{https://www.nytimes.com/subscription?campaignId=37WXW}{Subscriptions}
\end{itemize}
