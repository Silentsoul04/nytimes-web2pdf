Sections

SEARCH

\protect\hyperlink{site-content}{Skip to
content}\protect\hyperlink{site-index}{Skip to site index}

\href{https://www.nytimes.com/section/travel}{Travel}

\href{https://myaccount.nytimes.com/auth/login?response_type=cookie\&client_id=vi}{}

\href{https://www.nytimes.com/section/todayspaper}{Today's Paper}

\href{/section/travel}{Travel}\textbar{}Thinking of Traveling in the U.
S.? These States Have Travel Restrictions

\url{https://nyti.ms/2W8L3Q5}

\begin{itemize}
\item
\item
\item
\item
\item
\end{itemize}

\href{https://www.nytimes.com/news-event/coronavirus?action=click\&pgtype=Article\&state=default\&region=TOP_BANNER\&context=storylines_menu}{The
Coronavirus Outbreak}

\begin{itemize}
\tightlist
\item
  live\href{https://www.nytimes.com/2020/08/01/world/coronavirus-covid-19.html?action=click\&pgtype=Article\&state=default\&region=TOP_BANNER\&context=storylines_menu}{Latest
  Updates}
\item
  \href{https://www.nytimes.com/interactive/2020/us/coronavirus-us-cases.html?action=click\&pgtype=Article\&state=default\&region=TOP_BANNER\&context=storylines_menu}{Maps
  and Cases}
\item
  \href{https://www.nytimes.com/interactive/2020/science/coronavirus-vaccine-tracker.html?action=click\&pgtype=Article\&state=default\&region=TOP_BANNER\&context=storylines_menu}{Vaccine
  Tracker}
\item
  \href{https://www.nytimes.com/interactive/2020/07/29/us/schools-reopening-coronavirus.html?action=click\&pgtype=Article\&state=default\&region=TOP_BANNER\&context=storylines_menu}{What
  School May Look Like}
\item
  \href{https://www.nytimes.com/live/2020/07/31/business/stock-market-today-coronavirus?action=click\&pgtype=Article\&state=default\&region=TOP_BANNER\&context=storylines_menu}{Economy}
\end{itemize}

Advertisement

\protect\hyperlink{after-top}{Continue reading the main story}

Supported by

\protect\hyperlink{after-sponsor}{Continue reading the main story}

\hypertarget{thinking-of-traveling-in-the-u-s-these-states-have-travel-restrictions}{%
\section{Thinking of Traveling in the U. S.? These States Have Travel
Restrictions}\label{thinking-of-traveling-in-the-u-s-these-states-have-travel-restrictions}}

Nearly half of the states have strict measures in place for visitors,
from mandatory testing to quarantine requirements.

\includegraphics{https://static01.nyt.com/images/2020/07/10/travel/10quarentine/merlin_174085854_8ae7efae-415f-409e-94f7-71b554aa1fb4-articleLarge.jpg?quality=75\&auto=webp\&disable=upscale}

By Karen Schwartz

\begin{itemize}
\item
  Published July 10, 2020Updated July 31, 2020
\item
  \begin{itemize}
  \item
  \item
  \item
  \item
  \item
  \end{itemize}
\end{itemize}

\emph{This list will be updated as states continue to announce changes
to their travel advisories. Are we missing an update? Email us at}
\href{mailto:travelrestrictions@nytimes.com}{\emph{travelrestrictions@nytimes.com}}\emph{.}

In the United States,
\href{https://gasprices.aaa.com/national-average-decreases-as-gas-demand-remains-low/}{gas
prices are down} while the number of coronavirus cases are up, making
the decision of how to vacation during this unprecedented summer a
complex one. Meanwhile, state restrictions on travelers are constantly
evolving, with some requiring mandatory testing and others imposing
\href{https://www.cdc.gov/quarantine/index.html}{quarantine
requirements}.

The Centers for Disease Control and Prevention is continuing to caution
against travel, both internationally and within the United States. For
those who do take a trip,
\href{https://www.cdc.gov/coronavirus/2019-ncov/travelers/travel-in-the-us.html}{the
C.D.C. recommends} people wear a face mask in public, wash hands
frequently, avoid touching their face, keep six feet from others, cover
coughs and sneezes, and use drive-through service and curbside pickup at
restaurants and stores.

Here is a summary of current restrictions in the United States for
leisure travelers, although some people are exempt if they are simply
passing through or not remaining in the state for more than 24 hours.
Many states also have exemptions for essential workers who are on the
job, including health care workers, members of the military and others,
but even
\href{https://www.ncsl.org/research/labor-and-employment/covid-19-essential-workers-in-the-states.aspx}{they
are subject to some restrictions}.

With the number of coronavirus cases surging across the country, check
the areas you plan to visit before you travel. Some municipalities or
counties may have more stringent regulations than issued by their state.

\hypertarget{alabama}{%
\subsubsection{\texorpdfstring{\href{https://alabama.travel/my-trip/staying-safe}{Alabama}}{Alabama}}\label{alabama}}

As of July 28, there were no statewide restrictions in Alabama.

\hypertarget{alaska}{%
\subsubsection{\texorpdfstring{\href{https://covid19.alaska.gov/travelers/}{Alaska}}{Alaska}}\label{alaska}}

People entering Alaska must complete a
\href{https://covid19.alaska.gov/wp-content/uploads/2020/06/06112020-Mandate-10-Travel-declaration-form-Ver-2.0-6-10-f.pdf}{Mandatory
Declaration Form for Interstate Travelers}, and agree to one of the
following conditions:

\begin{itemize}
\item
  Those with proof of a negative test within the previous 72 hours must
  take another test between seven and 14 days after arrival, and
  minimize interactions with others until they receive those new
  results.
\item
  Tourists with a negative result from a test taken in the five days
  before their trip agree to take a second test at the airport upon
  arrival, and then a third test seven to 14 days later. They also agree
  to minimize interaction with others until the third test is back.
\item
  People can also receive a test upon arrival, if one is available, but
  they must self-quarantine until the results are reported. Those who
  refuse to be tested must self-quarantine for 14 days or until the end
  of their stay, whichever is shorter.
\end{itemize}

\hypertarget{arizona}{%
\subsubsection{\texorpdfstring{\href{https://tourism.az.gov/covid-19-updates-2/}{Arizona}}{Arizona}}\label{arizona}}

As of July 28 there were no statewide restrictions in Arizona.

\hypertarget{arkansas}{%
\subsubsection{\texorpdfstring{\href{https://www.arkansas.com/travel-advisory/covid-19}{Arkansas}}{Arkansas}}\label{arkansas}}

As of July 28 there were no statewide restrictions in Arkansas.

\hypertarget{california}{%
\subsubsection{\texorpdfstring{\href{https://www.visitcalifornia.com/latest-covid-19-coronavirus}{California}}{California}}\label{california}}

As of July 28 there were no statewide restrictions in California.

\hypertarget{colorado}{%
\subsubsection{\texorpdfstring{\href{https://covid19.colorado.gov/prepare-protect-yourself/prevent-the-spread/travel}{Colorado}}{Colorado}}\label{colorado}}

As of July 28 there were no statewide restrictions in Colorado.

\hypertarget{connecticut}{%
\subsubsection{\texorpdfstring{\href{https://portal.ct.gov/Coronavirus/Covid-19-Knowledge-Base/Travel-In-or-Out-of-CT}{Connecticut}}{Connecticut}}\label{connecticut}}

Those coming into Connecticut after spending more than 24 hours in a
state or area with a high rate of confirmed infections
\href{https://www.nytimes.com/2020/07/24/nyregion/coronavirus-test-results-nyc.html}{must
self-quarantine} for 14 days. Travelers can avoid quarantine if they
have proof of negative results for a coronavirus test taken in the
previous 72 hours. Those who have been tested but have not received the
results are required to quarantine until negative results are received
and submitted to the state.

The 34 states are Alaska, Alabama, Arizona, Arkansas, California,
Delaware, Florida, Georgia, Idaho, Illinois, Indiana, Iowa, Kansas,
Kentucky, Louisiana, Maryland, Minnesota, Mississippi, Missouri,
Montana, Nebraska, Nevada, New Mexico, North Carolina, North Dakota,
Ohio, Oklahoma, South Carolina, Tennessee, Texas, Utah, Virginia,
Washington and Wisconsin. Visitors or residents returning from Puerto
Rico or Washington, D.C., face the same restrictions.

The rules apply to anyone spending more than 24 hours in Connecticut.
Those who don't quarantine face a fine of up to \$1,000, as do those who
fail to truthfully complete a mandatory health form that must be filled
out by those visiting or returning to Connecticut.

\hypertarget{delaware}{%
\subsubsection{\texorpdfstring{\href{https://www.visitdelaware.com/industry/covid-19-in-delaware/}{Delaware}}{Delaware}}\label{delaware}}

As of July 28 there were no statewide restrictions in Delaware.

\hypertarget{district-of-columbia}{%
\subsubsection{\texorpdfstring{\href{https://washington.org/dc-information/coronavirus-travel-update-washington-dc}{District
of Columbia}}{District of Columbia}}\label{district-of-columbia}}

Visitors who have been to a high-risk state in the previous two weeks
must self-quarantine for 14 days.The order excludes travelers from
Maryland and Virginia, as well as those in the state for less than 24
hours..

The states affected by the order are Arkansas, Arizona, Alabama,
California, Delaware, Florida, Georgia, Idaho, Iowa, Kansas, Louisiana,
Mississippi, Missouri Montana, Nebraska, Nevada, New Mexico, North
Carolina, North Dakota, Ohio, Oklahoma, South Carolina, Tennessee,
Texas, Utah, Washington and Wisconsin.

\hypertarget{florida}{%
\subsubsection{\texorpdfstring{\href{https://floridahealthcovid19.gov/travelers/}{Florida}}{Florida}}\label{florida}}

People from New York, New Jersey and Connecticut must self-quarantine at
their own expense for 14 days when they enter Florida. Violators may be
fined up to \$500 or imprisoned for up to 60 days.

\hypertarget{latest-updates-global-coronavirus-outbreak}{%
\section{\texorpdfstring{\href{https://www.nytimes.com/2020/08/01/world/coronavirus-covid-19.html?action=click\&pgtype=Article\&state=default\&region=MAIN_CONTENT_1\&context=storylines_live_updates}{Latest
Updates: Global Coronavirus
Outbreak}}{Latest Updates: Global Coronavirus Outbreak}}\label{latest-updates-global-coronavirus-outbreak}}

Updated 2020-08-02T05:48:45.291Z

\begin{itemize}
\tightlist
\item
  \href{https://www.nytimes.com/2020/08/01/world/coronavirus-covid-19.html?action=click\&pgtype=Article\&state=default\&region=MAIN_CONTENT_1\&context=storylines_live_updates\#link-34047410}{The
  U.S. reels as July cases more than double the total of any other
  month.}
\item
  \href{https://www.nytimes.com/2020/08/01/world/coronavirus-covid-19.html?action=click\&pgtype=Article\&state=default\&region=MAIN_CONTENT_1\&context=storylines_live_updates\#link-780ec966}{Top
  U.S. officials work to break an impasse over the federal jobless
  benefit.}
\item
  \href{https://www.nytimes.com/2020/08/01/world/coronavirus-covid-19.html?action=click\&pgtype=Article\&state=default\&region=MAIN_CONTENT_1\&context=storylines_live_updates\#link-25930521}{Thousands
  in Berlin protest Germany's coronavirus measures.}
\end{itemize}

\href{https://www.nytimes.com/2020/08/01/world/coronavirus-covid-19.html?action=click\&pgtype=Article\&state=default\&region=MAIN_CONTENT_1\&context=storylines_live_updates}{See
more updates}

More live coverage:
\href{https://www.nytimes.com/live/2020/07/31/business/stock-market-today-coronavirus?action=click\&pgtype=Article\&state=default\&region=MAIN_CONTENT_1\&context=storylines_live_updates}{Markets}

With the number of coronavirus cases and hospitalizations
\href{https://www.nytimes.com/interactive/2020/us/florida-coronavirus-cases.html}{spiking
in parts of Florida}, the mayor of Miami-Dade County in early July
\href{https://www.miamidade.gov/releases/2020-07-06-mayor-order-closures.asp}{ordered}
all short-term vacation rentals in the county closed.

\hypertarget{georgia}{%
\subsubsection{\texorpdfstring{\href{https://dph.georgia.gov/covid-19-travel}{Georgia}}{Georgia}}\label{georgia}}

As of July 28 there were no statewide restrictions in Georgia.

\hypertarget{hawaii}{%
\subsubsection{\texorpdfstring{\href{https://www.hawaiitourismauthority.org/news/alerts/covid-19-novel-coronavirus/}{Hawaii}}{Hawaii}}\label{hawaii}}

All those arriving in Hawaii must isolate for two weeks, or until the
end of their stay, whichever is shorter. Arriving travelers must sign a
form confirming they are aware of the quarantine and that violating it
is a criminal offense that carries up to a \$5,000 fine and up to a year
in prison.

Beginning Sept. 1, travelers can avoid that restriction by showing proof
of a negative coronavirus test taken within 72 hours of their trip.

Forms must also be filled out for inter-island travel, and those with a
temperature of 100.4 or above are not allowed to fly.

\hypertarget{idaho}{%
\subsubsection{\texorpdfstring{\href{https://visitidaho.org/covid-19-travel-alert/}{Idaho}}{Idaho}}\label{idaho}}

Travelers to Boise and other cities in Ada County are encouraged to
self-quarantine for 14 days. Other counties in the state are further
along in their reopening and don't have a similar request.

\hypertarget{illinois}{%
\subsubsection{\texorpdfstring{\href{https://www.dph.illinois.gov/topics-services/diseases-and-conditions/diseases-a-z-list/coronavirus/travel-guidance}{Illinois}}{Illinois}}\label{illinois}}

There are no statewide restrictions, but those entering or returning to
Chicago from Alabama, Arizona, Arkansas, California, Florida, Georgia,
Idaho, Iowa, Kansas, Louisiana, Missouri (as of July 31), Mississippi,
Nebraska (as of July 31), Nevada, North Carolina, North Dakota (as of
July 31), Oklahoma, South Carolina, Tennessee, Texas, Utah and Wisconsin
(as of July 31), are
\href{https://www.chicago.gov/city/en/sites/covid-19/home.html}{required
to self-quarantine for 14 days} from their last contact with those
states. Those violating the order face fines of up to \$500 per day, up
to a maximum of \$7,000.

\hypertarget{indiana}{%
\subsubsection{\texorpdfstring{\href{https://www.coronavirus.in.gov}{Indiana}}{Indiana}}\label{indiana}}

As of July 28 there were no statewide restrictions in Indiana.

\hypertarget{iowa}{%
\subsubsection{\texorpdfstring{\href{https://www.traveliowa.com/aspx/general/dynamicpage.aspx?id=204}{Iowa}}{Iowa}}\label{iowa}}

As of July 28 there were no statewide restrictions in Iowa.

\hypertarget{kansas}{%
\subsubsection{\texorpdfstring{\href{https://www.coronavirus.kdheks.gov/175/Travel-Exposure-Related-Isolation-Quaran}{Kansas}}{Kansas}}\label{kansas}}

Those who visited Florida after June 29 must self-quarantine for 14 days
after entering or returning to Kansas. The same is true for anyone who
visited Arizona between June 17 and July 27.

\hypertarget{kentucky}{%
\subsubsection{\texorpdfstring{\href{https://governor.ky.gov/covid19}{Kentucky}}{Kentucky}}\label{kentucky}}

Travelers who visited states or territories with an infection rate
approaching 15 percent or higher
\href{https://coronavirus.jhu.edu/testing/testing-positivity}{are asked
to self-quarantine for 14 days}.

\href{https://kentucky.gov/Pages/Activity-stream.aspx?n=CHFS\&prId=281}{Those
states}affected are Alabama, Arizona, Florida, Georgia, Idaho, Nevada,
South Carolina and Texas.

\hypertarget{louisiana}{%
\subsubsection{\texorpdfstring{\href{https://louisianatravelassociation.org/covid-19-resources}{Louisiana}}{Louisiana}}\label{louisiana}}

As of July 28 there were no statewide restrictions in Louisiana.

\hypertarget{maine}{%
\subsubsection{\texorpdfstring{\href{https://www.maine.gov/covid19/restartingmaine/keepmainehealthy/faqs}{Maine}}{Maine}}\label{maine}}

Only residents of Vermont, New Hampshire, Connecticut, New York and New
Jersey who stay in commercial lodging in Maine can enter the state
without restriction. Everyone else must either self-quarantine for 14
days, or sign a document stating that they tested negative within the
previous 72 hours. Those in quarantine may leave their hotel or campsite
only for limited outdoor activities, such as hiking, when no other
people are around.

Maine residents who travel out of state to a state not on the exempted
list must also quarantine when they return or test negative for the
virus.

\hypertarget{maryland}{%
\subsubsection{\texorpdfstring{\href{https://www.visitmaryland.org/article/travel-alerts}{Maryland}}{Maryland}}\label{maryland}}

As of July 28 there were no statewide restrictions in Maryland.

\hypertarget{massachusetts}{%
\subsubsection{\texorpdfstring{\href{https://www.mass.gov/info-details/covid-19-updates-and-information}{Massachusetts}}{Massachusetts}}\label{massachusetts}}

\href{https://www.mass.gov/info-details/covid-19-travel-order}{Effective
Aug. 1}, all travelers, including residents of the state who are
returning home, are required to fill out and submit
\href{https://www.mass.gov/forms/massachusetts-travel-form}{an online
health form} and self-quarantine for 14 days, unless they are arriving
from a low-risk state. As of July 21, the states exempt from quarantine
included Connecticut, Hawaii, Maine, New Hampshire, New Jersey, New
York, Rhode Island and Vermont.

Travelers who produce a negative virus test result, administered up to
72 hours before their arrival into the state, can avoid the quarantine.
Those who have taken a coronavirus test before arrival must quarantine
until they receive a negative test result. Travelers who fail to comply
with these policies may be fined \$500 per day.

\hypertarget{michigan}{%
\subsubsection{\texorpdfstring{\href{https://www.michigan.gov/coronavirus/}{Michigan}}{Michigan}}\label{michigan}}

As of July 28 there were no statewide restrictions in Michigan.

\hypertarget{minnesota}{%
\subsubsection{\texorpdfstring{\href{https://www.exploreminnesota.com/info/coronavirus-covid-19-information}{Minnesota}}{Minnesota}}\label{minnesota}}

As of July 28 there were no statewide restrictions in Minnesota.

\hypertarget{mississippi}{%
\subsubsection{\texorpdfstring{\href{https://visitmississippi.org/covid-19-travel-alert/}{Mississippi}}{Mississippi}}\label{mississippi}}

As of July 28 there were no statewide restrictions in Mississippi.

\hypertarget{missouri}{%
\subsubsection{\texorpdfstring{\href{https://www.visitmo.com/travel-updates}{Missouri}}{Missouri}}\label{missouri}}

As of July 28 there were no statewide restrictions in Missouri.

\hypertarget{montana}{%
\subsubsection{\texorpdfstring{\href{https://www.visitmt.com/travel-alerts.html}{Montana}}{Montana}}\label{montana}}

As of July 28 there were no statewide restrictions in Montana.

At Glacier National Park, only the west entrance is open. The Blackfeet
Nation is keeping the park's eastern entrances, which are on tribal
land,
\href{https://www.washingtonpost.com/national/a-closed-border-pandemic-weary-tourists-and-a-big-bottleneck-at-glacier-national-park/2020/07/10/607694f2-c2c0-11ea-b4f6-cb39cd8940fb_story.html}{closed
at least through August}.

\hypertarget{nebraska}{%
\subsubsection{\texorpdfstring{\href{http://dhhs.ne.gov/Pages/COVID-19-Traveler-Recommendations.aspx}{Nebraska}}{Nebraska}}\label{nebraska}}

As of July 28 there were no statewide restrictions in Nebraska.

\hypertarget{nevada}{%
\subsubsection{\texorpdfstring{\href{https://nvhealthresponse.nv.gov/info/travelers-visitors/}{Nevada}}{Nevada}}\label{nevada}}

As of July 28 there were no statewide restrictions in Nevada.

\hypertarget{new-hampshire}{%
\subsubsection{\texorpdfstring{\href{https://www.covidguidance.nh.gov/out-state-visitors}{New
Hampshire}}{New Hampshire}}\label{new-hampshire}}

Those traveling to New Hampshire from non-New England states ``for an
extended period of time'' are asked to self-quarantine for two weeks.

\hypertarget{new-jersey}{%
\subsubsection{\texorpdfstring{\href{https://covid19.nj.gov/faqs/nj-information/general-public/which-states-are-on-the-travel-advisory-list-are-there-travel-restrictions-to-or-from-new-jersey}{New
Jersey}}{New Jersey}}\label{new-jersey}}

Those coming into New Jersey after spending more than 24 hours in a
state or area with a high rate of confirmed infections are advised to
self-quarantine for 14 days, unless they plan to be in the state for
less than 24 hours. The quarantine applies even to those with a recent
negative test. Starting July 27, those travelers are asked to
\href{https://covid19.nj.gov/forms/njtravel}{voluntarily supply contact
information} and details about where they plan on staying.

The 34 states affected by the quarantine advisory are Alaska, Alabama,
Arizona, Arkansas, California, Delaware, Florida, Georgia, Idaho,
Illinois, Indiana, Iowa, Kansas, Kentucky, Louisiana, Maryland,
Minnesota, Mississippi, Missouri, Montana, Nebraska, Nevada, New Mexico,
North Carolina, North Dakota, Ohio, Oklahoma, South Carolina, Tennessee,
Texas, Utah, Virginia, Washington and Wisconsin. Those arriving from
Puerto Rico and Washington, D.C., must also self-quarantine.

\hypertarget{new-mexico}{%
\subsubsection{\texorpdfstring{\href{https://www.newmexico.org/covid-19-traveler-information/}{New
Mexico}}{New Mexico}}\label{new-mexico}}

Upon entering the state, most people, including residents who have
traveled, are required to self-quarantine for 14 days or the duration of
their stay, whichever is shorter.

\href{https://www.nytimes.com/news-event/coronavirus?action=click\&pgtype=Article\&state=default\&region=MAIN_CONTENT_3\&context=storylines_faq}{}

\hypertarget{the-coronavirus-outbreak-}{%
\subsubsection{The Coronavirus Outbreak
›}\label{the-coronavirus-outbreak-}}

\hypertarget{frequently-asked-questions}{%
\paragraph{Frequently Asked
Questions}\label{frequently-asked-questions}}

Updated July 27, 2020

\begin{itemize}
\item ~
  \hypertarget{should-i-refinance-my-mortgage}{%
  \paragraph{Should I refinance my
  mortgage?}\label{should-i-refinance-my-mortgage}}

  \begin{itemize}
  \tightlist
  \item
    \href{https://www.nytimes.com/article/coronavirus-money-unemployment.html?action=click\&pgtype=Article\&state=default\&region=MAIN_CONTENT_3\&context=storylines_faq}{It
    could be a good idea,} because mortgage rates have
    \href{https://www.nytimes.com/2020/07/16/business/mortgage-rates-below-3-percent.html?action=click\&pgtype=Article\&state=default\&region=MAIN_CONTENT_3\&context=storylines_faq}{never
    been lower.} Refinancing requests have pushed mortgage applications
    to some of the highest levels since 2008, so be prepared to get in
    line. But defaults are also up, so if you're thinking about buying a
    home, be aware that some lenders have tightened their standards.
  \end{itemize}
\item ~
  \hypertarget{what-is-school-going-to-look-like-in-september}{%
  \paragraph{What is school going to look like in
  September?}\label{what-is-school-going-to-look-like-in-september}}

  \begin{itemize}
  \tightlist
  \item
    It is unlikely that many schools will return to a normal schedule
    this fall, requiring the grind of
    \href{https://www.nytimes.com/2020/06/05/us/coronavirus-education-lost-learning.html?action=click\&pgtype=Article\&state=default\&region=MAIN_CONTENT_3\&context=storylines_faq}{online
    learning},
    \href{https://www.nytimes.com/2020/05/29/us/coronavirus-child-care-centers.html?action=click\&pgtype=Article\&state=default\&region=MAIN_CONTENT_3\&context=storylines_faq}{makeshift
    child care} and
    \href{https://www.nytimes.com/2020/06/03/business/economy/coronavirus-working-women.html?action=click\&pgtype=Article\&state=default\&region=MAIN_CONTENT_3\&context=storylines_faq}{stunted
    workdays} to continue. California's two largest public school
    districts --- Los Angeles and San Diego --- said on July 13, that
    \href{https://www.nytimes.com/2020/07/13/us/lausd-san-diego-school-reopening.html?action=click\&pgtype=Article\&state=default\&region=MAIN_CONTENT_3\&context=storylines_faq}{instruction
    will be remote-only in the fall}, citing concerns that surging
    coronavirus infections in their areas pose too dire a risk for
    students and teachers. Together, the two districts enroll some
    825,000 students. They are the largest in the country so far to
    abandon plans for even a partial physical return to classrooms when
    they reopen in August. For other districts, the solution won't be an
    all-or-nothing approach.
    \href{https://bioethics.jhu.edu/research-and-outreach/projects/eschool-initiative/school-policy-tracker/}{Many
    systems}, including the nation's largest, New York City, are
    devising
    \href{https://www.nytimes.com/2020/06/26/us/coronavirus-schools-reopen-fall.html?action=click\&pgtype=Article\&state=default\&region=MAIN_CONTENT_3\&context=storylines_faq}{hybrid
    plans} that involve spending some days in classrooms and other days
    online. There's no national policy on this yet, so check with your
    municipal school system regularly to see what is happening in your
    community.
  \end{itemize}
\item ~
  \hypertarget{is-the-coronavirus-airborne}{%
  \paragraph{Is the coronavirus
  airborne?}\label{is-the-coronavirus-airborne}}

  \begin{itemize}
  \tightlist
  \item
    The coronavirus
    \href{https://www.nytimes.com/2020/07/04/health/239-experts-with-one-big-claim-the-coronavirus-is-airborne.html?action=click\&pgtype=Article\&state=default\&region=MAIN_CONTENT_3\&context=storylines_faq}{can
    stay aloft for hours in tiny droplets in stagnant air}, infecting
    people as they inhale, mounting scientific evidence suggests. This
    risk is highest in crowded indoor spaces with poor ventilation, and
    may help explain super-spreading events reported in meatpacking
    plants, churches and restaurants.
    \href{https://www.nytimes.com/2020/07/06/health/coronavirus-airborne-aerosols.html?action=click\&pgtype=Article\&state=default\&region=MAIN_CONTENT_3\&context=storylines_faq}{It's
    unclear how often the virus is spread} via these tiny droplets, or
    aerosols, compared with larger droplets that are expelled when a
    sick person coughs or sneezes, or transmitted through contact with
    contaminated surfaces, said Linsey Marr, an aerosol expert at
    Virginia Tech. Aerosols are released even when a person without
    symptoms exhales, talks or sings, according to Dr. Marr and more
    than 200 other experts, who
    \href{https://academic.oup.com/cid/article/doi/10.1093/cid/ciaa939/5867798}{have
    outlined the evidence in an open letter to the World Health
    Organization}.
  \end{itemize}
\item ~
  \hypertarget{what-are-the-symptoms-of-coronavirus}{%
  \paragraph{What are the symptoms of
  coronavirus?}\label{what-are-the-symptoms-of-coronavirus}}

  \begin{itemize}
  \tightlist
  \item
    Common symptoms
    \href{https://www.nytimes.com/article/symptoms-coronavirus.html?action=click\&pgtype=Article\&state=default\&region=MAIN_CONTENT_3\&context=storylines_faq}{include
    fever, a dry cough, fatigue and difficulty breathing or shortness of
    breath.} Some of these symptoms overlap with those of the flu,
    making detection difficult, but runny noses and stuffy sinuses are
    less common.
    \href{https://www.nytimes.com/2020/04/27/health/coronavirus-symptoms-cdc.html?action=click\&pgtype=Article\&state=default\&region=MAIN_CONTENT_3\&context=storylines_faq}{The
    C.D.C. has also} added chills, muscle pain, sore throat, headache
    and a new loss of the sense of taste or smell as symptoms to look
    out for. Most people fall ill five to seven days after exposure, but
    symptoms may appear in as few as two days or as many as 14 days.
  \end{itemize}
\item ~
  \hypertarget{does-asymptomatic-transmission-of-covid-19-happen}{%
  \paragraph{Does asymptomatic transmission of Covid-19
  happen?}\label{does-asymptomatic-transmission-of-covid-19-happen}}

  \begin{itemize}
  \tightlist
  \item
    So far, the evidence seems to show it does. A widely cited
    \href{https://www.nature.com/articles/s41591-020-0869-5}{paper}
    published in April suggests that people are most infectious about
    two days before the onset of coronavirus symptoms and estimated that
    44 percent of new infections were a result of transmission from
    people who were not yet showing symptoms. Recently, a top expert at
    the World Health Organization stated that transmission of the
    coronavirus by people who did not have symptoms was ``very rare,''
    \href{https://www.nytimes.com/2020/06/09/world/coronavirus-updates.html?action=click\&pgtype=Article\&state=default\&region=MAIN_CONTENT_3\&context=storylines_faq\#link-1f302e21}{but
    she later walked back that statement.}
  \end{itemize}
\end{itemize}

\hypertarget{new-york}{%
\subsubsection{\texorpdfstring{\href{https://coronavirus.health.ny.gov/covid-19-travel-advisory}{New
York}}{New York}}\label{new-york}}

New York requires individuals who have spent more than 24 hours in a
state or area with significant community spread of the coronavirus to
self-quarantine for 14 days.

The 34 states affected by the quarantine order are Alaska, Alabama,
Arizona, Arkansas, California, Delaware, Florida, Georgia, Idaho,
Illinois, Indiana, Iowa, Kansas, Kentucky, Louisiana, Maryland,
Mississippi, Minnesota, Missouri, Montana, Nebraska, Nevada, New Mexico,
North Carolina, North Dakota, Ohio, Oklahoma, South Carolina, Tennessee,
Texas, Utah, Virginia, Washington and Wisconsin. Visitors or residents
returning home from Puerto Rico and Washington, D.C., must also
self-quarantine.

Those arriving at airports in New York must fill out a Health Department
traveler form, or face a possible \$2,000 fine and a mandatory
quarantine order. Failure to comply may also result in their being
ordered to quarantine, regardless of which state they arriving from.
Travelers arriving by air must fill out the form before leaving the
airport, while those arriving by car, train or other modes of
transportation must fill it out online.

\hypertarget{north-carolina}{%
\subsubsection{\texorpdfstring{\href{https://www.nc.gov/covid-19/covid-19-travel-resources}{North
Carolina}}{North Carolina}}\label{north-carolina}}

As of July 28 there were no statewide restrictions in North Carolina.

\hypertarget{north-dakota}{%
\subsubsection{\texorpdfstring{\href{https://www.health.nd.gov/diseases-conditions/coronavirus/travel}{North
Dakota}}{North Dakota}}\label{north-dakota}}

As of July 28 there were no statewide restrictions in North Dakota.

\hypertarget{ohio}{%
\subsubsection{\texorpdfstring{\href{https://coronavirus.ohio.gov/wps/portal/gov/covid-19/home}{Ohio}}{Ohio}}\label{ohio}}

Traveling Ohioans and out-of-state tourists who have visited an area of
high risk, or who have had possible exposure to the coronavirus, are
asked to voluntarily quarantine for 14 days.

As of July 29, Ohio has identified
\href{https://coronavirus.ohio.gov/wps/portal/gov/covid-19/families-and-individuals/COVID-19-Travel-Advisory/COVID-19-Travel-Advisory}{the
following states} as high risk: Alabama, Arizona, Florida, Idaho,
Kansas, Mississippi, and South Carolina.

\hypertarget{oklahoma}{%
\subsubsection{\texorpdfstring{\href{https://coronavirus.health.ok.gov/travel}{Oklahoma}}{Oklahoma}}\label{oklahoma}}

As of July 28 there were no statewide restrictions in Oklahoma.

\hypertarget{oregon}{%
\subsubsection{\texorpdfstring{\href{https://traveloregon.com/travel-alerts/}{Oregon}}{Oregon}}\label{oregon}}

As of July 28 there were no statewide restrictions in Oregon.

\hypertarget{pennsylvania}{%
\subsubsection{\texorpdfstring{\href{https://www.health.pa.gov/topics/disease/coronavirus/Pages/Travelers.aspx}{Pennsylvania}}{Pennsylvania}}\label{pennsylvania}}

The state asks travelers who have visited an area with a Covid-19 surge
to self-quarantine for 14 days. It has identified the affected states
as: Alabama, Arizona, Arkansas, California, Florida, Georgia, Idaho,
Iowa, Kansas, Louisiana, Mississippi, Missouri, Nevada, North Carolina,
Oklahoma, South Carolina, Tennessee, Texas, Utah and Wyoming.

\hypertarget{rhode-island}{%
\subsubsection{\texorpdfstring{\href{https://health.ri.gov/covid/}{Rhode
Island}}{Rhode Island}}\label{rhode-island}}

Those coming to Rhode Island from a state that has a positivity rate for
tests of greater than 5 percent are required to self-quarantine for two
weeks. Alternatively, visitors can provide a negative test for the virus
that was taken within the previous 72 hours. A person who receives a
negative test during their quarantine can stop isolating, although the
state recommends the full two-week quarantine.

\href{https://docs.google.com/spreadsheets/d/e/2PACX-1vSUCk9FlHBoJt5ZO0U6PKTTY7jHH8V4MovED0WiqpTTixdgMSCnUWI25xX5DCmQmtLknzu7Bo0jwY02/pubhtml?gid=0\&single=true}{The
states identified} are Alabama, Arizona, Arkansas, California, Colorado,
Florida, Georgia, Idaho, Indiana, Iowa, Kansas, Kentucky, Louisiana,
Maryland, Mississippi, Missouri, Nebraska, Nevada, North Carolina, North
Dakota, Ohio, Oklahoma, Oregon, Pennsylvania, South Carolina, South
Dakota, Tennessee, Texas, Utah, Virginia, Washington, Wisconsin and
Wyoming. Visitors from Puerto Rico must also quarantine.

\hypertarget{south-carolina}{%
\subsubsection{\texorpdfstring{\href{https://scdhec.gov/infectious-diseases/viruses/coronavirus-disease-2019-covid-19/travelers-covid-19}{South
Carolina}}{South Carolina}}\label{south-carolina}}

The state recommends that people who have visited an area with
widespread or ongoing community transmission of the virus stay home for
14 days from the time they left that region.

\hypertarget{south-dakota}{%
\subsubsection{\texorpdfstring{\href{https://www.travelsouthdakota.com/coronavirus-covid-19}{South
Dakota}}{South Dakota}}\label{south-dakota}}

As of July 28 there were no statewide restrictions in South Dakota.

\hypertarget{tennessee}{%
\subsubsection{\texorpdfstring{\href{https://www.tnvacation.com/articles/tennessee-travel-amid-coronavirus}{Tennessee}}{Tennessee}}\label{tennessee}}

As of July 28 there were no statewide restrictions in Tennessee.

\hypertarget{texas}{%
\subsubsection{\texorpdfstring{\href{https://gov.texas.gov/travel-texas/page/covid19}{Texas}}{Texas}}\label{texas}}

As of July 28 there were no statewide restrictions in Texas.

\hypertarget{utah}{%
\subsubsection{\texorpdfstring{\href{https://www.visitutah.com/plan-your-trip/covid-19/}{Utah}}{Utah}}\label{utah}}

As of July 28 there were no statewide restrictions in Utah.

\hypertarget{vermont}{%
\subsubsection{\texorpdfstring{\href{https://www.healthvermont.gov/response/coronavirus-covid-19/traveling-vermont}{Vermont}}{Vermont}}\label{vermont}}

Visitors from counties in Northeastern states that have similar active
coronavirus rates to Vermont (defined as less than 400 active cases per
million residents) and who travel in a private vehicle do not have to
quarantine. The same is true for Vermont residents who visit those
regions when they return home.

\href{https://accd.vermont.gov/covid-19/restart/cross-state-travel}{These
counties} are in Connecticut, Maine, Massachusetts, New Hampshire, Rhode
Island, New York, Pennsylvania, Ohio, New Jersey, Delaware, Maryland,
Virginia, West Virginia and Washington, D.C.

Most other travelers need to self-quarantine upon arrival in Vermont,
but the state gives travelers a few options. People may self-quarantine
out of state before traveling to Vermont as long as their trip is in a
private vehicle and they make only necessary stops, while wearing a face
mask, social distancing and washing their hands frequently. Those opting
to self-quarantine before their visit to Vermont can either do it for 14
days, or they can shorten it to seven days if they then get a negative
test result.

Those arriving by public transportation or a longer car ride must
self-quarantine for 14 days, or for seven days followed by a negative
test.

\hypertarget{virginia}{%
\subsubsection{\texorpdfstring{\href{https://www.vdh.virginia.gov/coronavirus/frequently-asked-questions/u-s-travelers/}{Virginia}}{Virginia}}\label{virginia}}

As of July 28 there were no statewide restrictions in Virginia.

\hypertarget{washington}{%
\subsubsection{\texorpdfstring{\href{https://www.experiencewa.com/articles/date-coronavirus-travel-advisory}{Washington}}{Washington}}\label{washington}}

As of July 28 there were no statewide restrictions in Washington.

\hypertarget{west-virginia}{%
\subsubsection{\texorpdfstring{\href{https://wvtourism.com/travel-alert/}{West
Virginia}}{West Virginia}}\label{west-virginia}}

As of July 28 there were no statewide restrictions in West Virginia.

\hypertarget{wisconsin}{%
\subsubsection{\texorpdfstring{\href{https://www.dhs.wisconsin.gov/covid-19/travel.htm}{Wisconsin}}{Wisconsin}}\label{wisconsin}}

There is no quarantine request, but the state asks those who have
traveled within the United States and are entering Wisconsin to check
themselves for symptoms of Covid-19 and to stay home as much as possible
for 14 days. Wisconsinites are asked not to travel to summer or rental
homes. Local quarantine restrictions may be in place at the county
level.

\hypertarget{wyoming}{%
\subsubsection{\texorpdfstring{\href{https://health.wyo.gov/publichealth/infectious-disease-epidemiology-unit/disease/novel-coronavirus/covid-19-orders-and-guidance/}{Wyoming}}{Wyoming}}\label{wyoming}}

As of July 28 there were no statewide restrictions in Wyoming.

\begin{center}\rule{0.5\linewidth}{\linethickness}\end{center}

Follow Karen Schwartz on Twitter:
\href{https://twitter.com/wanderwomanisme?lang=en}{@WanderWomanIsMe}

\emph{\textbf{Follow New York Times Travel}}
\emph{on}\href{https://www.instagram.com/nytimestravel/}{\emph{Instagram}}\emph{,}\href{https://twitter.com/nytimestravel}{\emph{Twitter}}
\emph{and}\href{https://www.facebook.com/nytimestravel/}{\emph{Facebook}}\emph{.
And}\href{https://www.nytimes.com/newsletters/traveldispatch}{\emph{sign
up for our weekly Travel Dispatch newsletter}} \emph{to receive expert
tips on traveling smarter and inspiration for your next vacation.}

Advertisement

\protect\hyperlink{after-bottom}{Continue reading the main story}

\hypertarget{site-index}{%
\subsection{Site Index}\label{site-index}}

\hypertarget{site-information-navigation}{%
\subsection{Site Information
Navigation}\label{site-information-navigation}}

\begin{itemize}
\tightlist
\item
  \href{https://help.nytimes.com/hc/en-us/articles/115014792127-Copyright-notice}{©~2020~The
  New York Times Company}
\end{itemize}

\begin{itemize}
\tightlist
\item
  \href{https://www.nytco.com/}{NYTCo}
\item
  \href{https://help.nytimes.com/hc/en-us/articles/115015385887-Contact-Us}{Contact
  Us}
\item
  \href{https://www.nytco.com/careers/}{Work with us}
\item
  \href{https://nytmediakit.com/}{Advertise}
\item
  \href{http://www.tbrandstudio.com/}{T Brand Studio}
\item
  \href{https://www.nytimes.com/privacy/cookie-policy\#how-do-i-manage-trackers}{Your
  Ad Choices}
\item
  \href{https://www.nytimes.com/privacy}{Privacy}
\item
  \href{https://help.nytimes.com/hc/en-us/articles/115014893428-Terms-of-service}{Terms
  of Service}
\item
  \href{https://help.nytimes.com/hc/en-us/articles/115014893968-Terms-of-sale}{Terms
  of Sale}
\item
  \href{https://spiderbites.nytimes.com}{Site Map}
\item
  \href{https://help.nytimes.com/hc/en-us}{Help}
\item
  \href{https://www.nytimes.com/subscription?campaignId=37WXW}{Subscriptions}
\end{itemize}
