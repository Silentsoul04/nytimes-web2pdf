Sections

SEARCH

\protect\hyperlink{site-content}{Skip to
content}\protect\hyperlink{site-index}{Skip to site index}

\href{https://www.nytimes.com/section/world/middleeast}{Middle East}

\href{https://myaccount.nytimes.com/auth/login?response_type=cookie\&client_id=vi}{}

\href{https://www.nytimes.com/section/todayspaper}{Today's Paper}

\href{/section/world/middleeast}{Middle East}\textbar{}Long-Planned and
Bigger Than Thought: Strike on Iran's Nuclear Program

\url{https://nyti.ms/3gOiEXg}

\begin{itemize}
\item
\item
\item
\item
\item
\item
\end{itemize}

Advertisement

\protect\hyperlink{after-top}{Continue reading the main story}

Supported by

\protect\hyperlink{after-sponsor}{Continue reading the main story}

News Analysis

\hypertarget{long-planned-and-bigger-than-thought-strike-on-irans-nuclear-program}{%
\section{Long-Planned and Bigger Than Thought: Strike on Iran's Nuclear
Program}\label{long-planned-and-bigger-than-thought-strike-on-irans-nuclear-program}}

Some officials say that a joint American-Israeli strategy is evolving
--- some might argue regressing --- to a series of short-of-war
clandestine strikes.

\includegraphics{https://static01.nyt.com/images/2020/07/08/us/politics/08dc-iran-nukes1/merlin_161249049_91588784-3513-4481-a82e-72178775a779-articleLarge.jpg?quality=75\&auto=webp\&disable=upscale}

\href{https://www.nytimes.com/by/david-e-sanger}{\includegraphics{https://static01.nyt.com/images/2018/10/03/multimedia/author-david-e-sanger/author-david-e-sanger-thumbLarge.png}}\href{https://www.nytimes.com/by/eric-schmitt}{\includegraphics{https://static01.nyt.com/images/2018/06/12/multimedia/author-eric-schmitt/author-eric-schmitt-thumbLarge-v2.png}}\href{https://www.nytimes.com/by/ronen-bergman}{\includegraphics{https://static01.nyt.com/images/2018/07/16/multimedia/author-ronen-bergman/author-ronen-bergman-thumbLarge.png}}

By \href{https://www.nytimes.com/by/david-e-sanger}{David E. Sanger},
\href{https://www.nytimes.com/by/eric-schmitt}{Eric Schmitt} and
\href{https://www.nytimes.com/by/ronen-bergman}{Ronen Bergman}

\begin{itemize}
\item
  July 10, 2020
\item
  \begin{itemize}
  \item
  \item
  \item
  \item
  \item
  \item
  \end{itemize}
\end{itemize}

As Iran's center for advanced nuclear centrifuges lies in charred ruins
\href{https://www.nytimes.com/2020/07/05/world/middleeast/iran-Natanz-nuclear-damage.html}{after
an explosion}, apparently engineered by Israel, the long-simmering
conflict between the United States and Tehran appears to be escalating
into a potentially dangerous phase likely to play out during the
American presidential election campaign.

New satellite photographs over the stricken facility at Natanz show far
more extensive damage than was clear last week. Two intelligence
officials, updated with the damage assessment for the Natanz site
recently compiled by the United States and Israel, said it could take
the Iranians up to two years to return their nuclear program to the
place it was just before the explosion. An
\href{https://isis-online.org/isis-reports/detail/damage-to-the-iran-centrifuge-assembly-center-icac-at-natanz/}{authoritative
public study} estimates it will be a year or more until Iran's
centrifuge production capacity recovers.

Another major explosion hit the country early Friday morning, lighting
up the sky in a wealthy area of Tehran. It was still unexplained --- but
appeared to come from the direction of a missile base. If it proves to
have been another attack, it will further shake the Iranians by
demonstrating, yet again, that even their best-guarded nuclear and
missile facilities have been infiltrated.

Although Iran has said little of substance about the explosions, Western
officials anticipate some type of retaliation, perhaps against American
or allied forces in Iraq, perhaps a renewal of cyberattacks. In the
past, those have been directed against American financial institutions,
a major Las Vegas casino and a dam in the New York suburbs or, more
recently, the water supply system in Israel,
\href{https://www.nytimes.com/2020/05/19/world/middleeast/israel-iran-cyberattacks.html}{which
its government considers ``critical infrastructure.''}

Officials familiar with the explosion at Natanz compared its complexity
to the sophisticated Stuxnet cyberattack on Iranian nuclear facilities a
decade ago, which had been planned for more than a year. In the case of
last week's episode, the primary theory is that an explosive device was
planted in the heavily-guarded facility, perhaps near a gas line. But
some experts have also floated the possibility that a cyberattack was
used to trigger the gas supply.

Some officials said that a joint American-Israeli strategy was evolving
--- some might argue regressing --- to a series of short-of-war
clandestine strikes, aimed at taking out the most prominent generals of
the Islamic Revolutionary Guards Corps and setting back Iran's nuclear
facilities.

The closest the administration has come to describing its strategy of
more aggressive pushback came in comments last month from Brian H. Hook,
the State Department's special envoy for Iran. ``We have seen
historically,'' he concluded, ``that timidity and weakness invites more
Iranian aggression.''

The next move may be a confrontation over four tankers, now making their
way to Venezuela, which the United States has vowed will not be allowed
to deliver their cargo of Iranian oil in violation of United States
sanctions.

The emerging approach is risky, analysts warn, one that over the long
term may largely serve to drive Iran's nuclear program further
underground, and thus make it harder to detect.

But in the short term, American and Israeli officials are betting that
Iran will limit its retaliation, as it did after an American drone in
January
\href{https://www.nytimes.com/2020/01/11/us/politics/iran-trump.html}{killed
Maj. Gen. Qassim Suleimani}, one of Iran's most important commanders.

While some American officials expressed fears that the killing of
General Suleimani would lead Iran to initiate a war against the United
States, the C.I.A. director, Gina Haspel, reassured them that the
Iranians
\href{https://www.nytimes.com/2020/01/11/us/politics/iran-trump.html}{would
settle on limited missile attacks} against American targets in Iraq ---
which so far has turned out to be correct. Iran's limited response could
be an incentive for further operations against it.

In addition, some American and Israeli officials, and international
security analysts, say that Iran may believe that President Trump will
lose the November election and that his presumptive Democratic rival,
Joseph R. Biden Jr., will want to resurrect some form of
\href{https://www.nytimes.com/2015/07/15/world/middleeast/iran-nuclear-deal-is-reached-after-long-negotiations.html}{the
negotiated settlement that the Obama administration reached with Tehran
five years ago next week.}

\includegraphics{https://static01.nyt.com/images/2020/07/09/us/politics/09dc-iran-nukes2-sub/09dc-iran-nukes2-sub-articleLarge.jpg?quality=75\&auto=webp\&disable=upscale}

``Today, if you are Iran, why compromise with an administration which
may only have a few months left?'' asked Karim Sadjadpour, a senior
fellow at the Carnegie Endowment for International Peace.

But in the short term, he noted, the new offensive has put Iran under
``extreme internal and external pressure,'' as its oil exports continue
to be squeezed and its efforts to revive the nuclear program,
retribution for Mr. Trump's
\href{https://www.nytimes.com/2018/05/08/world/middleeast/trump-iran-nuclear-deal.html}{decision
in May 2018} to abandon the 2015 accord, falter amid sabotage.

``Think about it,'' he said. ``Geographically, Iran is greater in size
than Germany, France and the United Kingdom combined. But they have
never managed to pursue a clandestine nuclear program without getting
caught, or protected their program from sabotage. Are there defectors or
traitors inside the system?''

When the Mossad raided a warehouse in Tehran in January 2018, and
\href{https://www.nytimes.com/2018/07/15/us/politics/iran-israel-mossad-nuclear.html}{emerged
with tens of thousands of pages of nuclear-weapons planning documents}
dating back nearly two decades, it clearly had the help of insiders. The
killing of General Suleimani, the mastermind of Iran's actions in Iraq
and attacks on Americans --- which was also based on intelligence, much
of it given by live agents --- was perhaps Mr. Trump's most aggressive
military move as president.

The Natanz explosion occurred inside the Iran Centrifuge Assembly
Center, where the country was building its most advanced machines,
designed to produce far more nuclear fuel, far faster, than the old
machines used until Iran dismantled most of its facilities in the 2015
accord.

While research on those machines was permitted under the agreement, they
could not be deployed for years --- and Iran's crash effort to mass
produce them was an ambitious effort to show that it could respond to
Mr. Trump's rejection of the deal by speeding up.

\href{https://isis-online.org/isis-reports/detail/damage-to-the-iran-centrifuge-assembly-center-icac-at-natanz/}{A
study by the Institute for Science and International Security} published
Wednesday concluded that while the explosion ``does not eliminate Iran's
ability to deploy advanced centrifuges,'' it was ``a major setback''
that would cost Iran years of development.

Secretary of State Mike Pompeo, who always leaps at any opportunity to
denounce the Iranian government, twice declined on Wednesday to discuss
the issue at a news conference.

But it is hardly a secret inside the State Department that Mr. Pompeo,
who served as Mr. Trump's first C.I.A. director, developed a close
relationship with Yossi Cohen, the director of the Mossad, Israel's
external spy service. The two men talk often, making it difficult to
believe that Mr. Pompeo had no idea about what was coming, if indeed it
was an Israeli operation.

Just as the strike was happening, Mr. Cohen's term was extended for six
months by Prime Minister Benjamin Netanyahu, interpreted by many as a
sign of things to come, since Mr. Cohen is a veteran of Iran operations.
He was a key player in the sophisticated series of cyberstrikes known as
Olympic Games that took out
\href{https://www.nytimes.com/2012/06/01/world/middleeast/obama-ordered-wave-of-cyberattacks-against-iran.html}{nearly
1,000 operating centrifuges at Natanz} --- near the site of last week's
explosion and fire --- a decade ago. And as chief of Mossad, he directed
the covert seizure of the secret nuclear archive.

In some way it feels a bit like a decade ago, when the George W. Bush
administration handed off the cyberoperations to the Obama
administration, part of a broad covert effort to cripple Iran's nuclear
program. At the same time, the Israelis were killing Iranian scientists.
The idea was not only to slow the program, but also to turn the Iranians
against one another, constantly suspecting that there were spies in
their midst.

This time, there are several new elements.

Mr. Trump is an unpredictable player, who has often threatened Iran ---
and just as often pulled back from striking it. And the Iranian leaders
who negotiated the 2015 nuclear deal with President Barack Obama are on
the ropes in Tehran, assailed for having given away too much, only to
discover that Washington was reimposing sanctions.

At the White House, Mr. Trump's top national security advisers are
hardly of one mind over when and how to confront Iran.

Image

Prime Minister Benjamin Netanyahu~ of Israel at the White House in
January with President Trump. Some officials suggested that the joint
Israeli-American strategy on Iran was evolving.Credit...Alyssa Schukar
for The New York Times

Military leaders, including Defense Secretary Mark T. Esper and Gen.
Mark A. Milley, the chairman of the Joint Chiefs of Staff, have been
wary of a sharp military escalation, warning it could further
destabilize the Middle East when Mr. Trump has said he hopes to reduce
the number of American troops in the region.

Pentagon officials nervously cited at least two potential flash points
that could drag American forces into a military clash with Iran or
Iranian-backed proxies in the Persian Gulf region.

One focuses on those oil tankers. Justice Department and F.B.I.
officials
\href{https://www.justice.gov/opa/pr/warrant-and-complaint-seek-seizure-all-iranian-gasoil-aboard-four-tankers-headed-venezuela}{announced
last week} that they had used a counterterrorism statute to obtain a
warrant to seize Iranian oil products aboard the four tankers bound for
Venezuela in violation of American sanctions. Investigators determined
that the fuel cargo aboard the Greek-owned ships were assets of Iran's
Guards Corps, which the Trump administration last year
\href{https://www.nytimes.com/2019/04/08/world/middleeast/trump-iran-revolutionary-guard-corps.html}{designated}
as a terrorist organization. General Suleimani was commander of the
\href{https://www.nytimes.com/2020/01/03/world/middleeast/suleimani-dead.html}{Quds
Force of the Islamic Revolutionary Guards Corps}.

Administration officials said this week that the State, Justice and
Treasury Departments were seeking to work with the Greek government to
halt the shipments, and have the fuel be offloaded. Iran's mission to
the United Nations immediately declared any such seizure would amount to
``piracy.''

Two of the ships are believed to be in the Aegean Sea. But the two
others are steaming in the Gulf of Oman, off the coast of Iran, and are
under close surveillance, an American military official said.

Some American officials worry that if the two tankers comply with the
U.S. court order to give up the fuel, Iranian naval forces could
challenge the transfer to another ship. It is not entirely clear what
United States Navy warships in the area would do if that happened.

Another potential flash point is in Iraq, where Iranian-backed militia
are believed to be responsible for a steadily increasing series of
rocket attacks at the American Embassy in Baghdad and on American and
coalition forces near Baghdad's international airport.

After General Suleimani's death, Tehran and Washington traded modest
strikes in March. But then, tensions appeared to ease --- until early
June.

``We're seeing a beginning of a spike in unprovoked rocket attacks on
Iraqi bases that host U.S. forces in Iraq,'' Gen. Kenneth F. McKenzie
Jr., the head of the military's Central Command,
\href{https://www.centcom.mil/MEDIA/Transcripts/Article/2226655/gen-mckenzie-interview-transcript-aspen-security-forum-june-18-2020/}{said
last month}.

For now, the latest rocket attacks have been more harassing than
harmful.

Advertisement

\protect\hyperlink{after-bottom}{Continue reading the main story}

\hypertarget{site-index}{%
\subsection{Site Index}\label{site-index}}

\hypertarget{site-information-navigation}{%
\subsection{Site Information
Navigation}\label{site-information-navigation}}

\begin{itemize}
\tightlist
\item
  \href{https://help.nytimes.com/hc/en-us/articles/115014792127-Copyright-notice}{©~2020~The
  New York Times Company}
\end{itemize}

\begin{itemize}
\tightlist
\item
  \href{https://www.nytco.com/}{NYTCo}
\item
  \href{https://help.nytimes.com/hc/en-us/articles/115015385887-Contact-Us}{Contact
  Us}
\item
  \href{https://www.nytco.com/careers/}{Work with us}
\item
  \href{https://nytmediakit.com/}{Advertise}
\item
  \href{http://www.tbrandstudio.com/}{T Brand Studio}
\item
  \href{https://www.nytimes.com/privacy/cookie-policy\#how-do-i-manage-trackers}{Your
  Ad Choices}
\item
  \href{https://www.nytimes.com/privacy}{Privacy}
\item
  \href{https://help.nytimes.com/hc/en-us/articles/115014893428-Terms-of-service}{Terms
  of Service}
\item
  \href{https://help.nytimes.com/hc/en-us/articles/115014893968-Terms-of-sale}{Terms
  of Sale}
\item
  \href{https://spiderbites.nytimes.com}{Site Map}
\item
  \href{https://help.nytimes.com/hc/en-us}{Help}
\item
  \href{https://www.nytimes.com/subscription?campaignId=37WXW}{Subscriptions}
\end{itemize}
