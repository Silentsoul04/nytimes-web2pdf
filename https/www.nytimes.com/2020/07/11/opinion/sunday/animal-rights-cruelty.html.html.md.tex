Sections

SEARCH

\protect\hyperlink{site-content}{Skip to
content}\protect\hyperlink{site-index}{Skip to site index}

\href{https://www.nytimes.com/section/opinion/sunday}{Sunday Review}

\href{https://myaccount.nytimes.com/auth/login?response_type=cookie\&client_id=vi}{}

\href{https://www.nytimes.com/section/todayspaper}{Today's Paper}

\href{/section/opinion/sunday}{Sunday Review}\textbar{}The Mistakes That
Will Haunt Our Legacy

\href{https://nyti.ms/3ee4AF2}{https://nyti.ms/3ee4AF2}

\begin{itemize}
\item
\item
\item
\item
\item
\item
\end{itemize}

Advertisement

\protect\hyperlink{after-top}{Continue reading the main story}

\href{/section/opinion}{Opinion}

Supported by

\protect\hyperlink{after-sponsor}{Continue reading the main story}

\hypertarget{the-mistakes-that-will-haunt-our-legacy}{%
\section{The Mistakes That Will Haunt Our
Legacy}\label{the-mistakes-that-will-haunt-our-legacy}}

As we topple statues, let's also search for our own moral blind spots.

\href{https://www.nytimes.com/column/nicholas-kristof}{\includegraphics{https://static01.nyt.com/images/2018/04/03/opinion/nicholas-kristof/nicholas-kristof-thumbLarge-v2.png}}

By \href{https://www.nytimes.com/column/nicholas-kristof}{Nicholas
Kristof}

Opinion Columnist

\begin{itemize}
\item
  July 11, 2020
\item
  \begin{itemize}
  \item
  \item
  \item
  \item
  \item
  \item
  \end{itemize}
\end{itemize}

\includegraphics{https://static01.nyt.com/images/2020/07/12/opinion/sunday/12Kristof/12Kristof-articleLarge.jpg?quality=75\&auto=webp\&disable=upscale}

As we pull down controversial statues and reassess historical figures,
I've been wondering what our great-grandchildren will find bewilderingly
immoral about our own times --- and about us.

Which of today's heroes will be discredited? Which statues toppled? What
will later generations see as our own ethical blind spots?

I believe that one will be our cruelty to animals. Modern society relies
on factory farming to produce protein that is inexpensive and abundant.
But it causes suffering to animals on an incalculable scale.

Over the last 200 years, the world has become far more sensitive to
animal rights. In feudal Europe, a game consisted of nailing a cat to a
post and head-butting it to death; now, growing numbers of states have
passed animal protection laws, McDonald's is moving to cage-free eggs
and there are
\href{https://fedsoc.org/commentary/fedsoc-blog/do-animals-or-humans-claiming-to-represent-them-have-constitutional-standing-to-file-federal-lawsuits-in-the-ninth-circuit-the-answer-is-yes\#:~:text=In\%20the\%20Ninth\%20Circuit\%2C\%20the\%20Answer\%20Is\%20Yes.,-Topics\%3A\&text=You\%20might\%20be\%20a\%20little,file\%20lawsuits\%20in\%20federal\%20cases.}{legal
debates} about whether certain mammals should have standing to sue in
courts.

The upshot is court cases like
\href{https://www.animallaw.info/case/cetacean-community-v-bush}{Cetacean
Community v. Bush}, in which the plaintiffs were whales, dolphins and
porpoises, and
\href{https://www.animallaw.info/pleading/naruto-v-slater-peta}{Naruto},
a Crested Macaque, v. Slater.

Pope Francis
\href{https://www.nytimes.com/2015/09/24/opinion/nicholas-kristof-a-pope-for-all-species.html}{suggests}
that animals go to heaven, and many humans would agree: Paradise would
be diminished without pets.

Yet while we adore our pets and coddle them --- a dog in a wealthy
family may get better medical and dental care than a child in a poor
family --- we as a society often do not extend this empathy to unseen
farm animals, especially poultry.

Some 9.3 billion chickens were slaughtered last year in the United
States --- 28 per American --- and here's
\href{https://www.nytimes.com/2015/03/15/opinion/sunday/nicholas-kristof-to-kill-a-chicken.html}{how
they are typically killed}: Workers shove the chickens' legs into metal
shackles, and the birds are then carried upside down to an electrified
bath that stuns them before a circular saw cuts open their necks and
they are dunked in scalding water.

Even when this system works perfectly, chickens sometimes have legs or
wings broken as they are shackled.
\href{https://youtu.be/IayFKuxqODo}{When the system fails}, they are not
stunned and struggle frantically as they are carried to the saw. The saw
in turn misses many birds --- the Agriculture Department says that
526,000 chickens were not slaughtered correctly last year --- and some
are boiled alive.

A child who plucks out a bird's feathers may be punished, but corporate
executives who torture birds by the billions are showered with stock
options.

Factory farming also diminishes human frontline workers, from struggling
farmers who raise animals to the miserably paid and poorly protected
\href{https://foodispower.org/human-labor-slavery/slaughterhouse-workers/\#:~:text=Slaughtering\%20animals\%20and\%20processing\%20their,facilities\%20employ\%20over\%20500\%2C000\%20workers.}{slaughterhouse
employees} now falling ill from the coronavirus.

In the face of all this, attitudes are changing: Eight percent of young
American adults said in 2018 that they were vegetarians, compared with
just 2 percent of Americans 55 and older.

I became a vegetarian almost two years ago (not a strict one, and I do
eat fish) because my daughter nagged me (``provided moral guidance''
would be a nicer spin), and I suspect that ethical and environmental
considerations --- and the increasing availability of tasty alternatives
to meat --- will lead our descendants to eat less meat, and be baffled
at our casual acceptance of an industrial agricultural model built on
large-scale cruelty.

``One day future generations will look back on our abuse of animals in
factory farms with the same attitude that we have to the cruelties of
the Roman `games' at the Colosseum,'' Peter Singer, a Princeton
University philosopher, told me. ``They will wonder how we could be
blind to the suffering we are so needlessly inflicting on billions of
animals.''

A second area that I think will leave future generations baffled at our
heartlessness is our indifference to suffering in impoverished
countries. More than five million young children will die this year
around the world from diarrhea, malnutrition or other ailments; we let
these children perish essentially because of our own tribalism. They are
not a priority to us.

While I denounced the mistreatment of broiler chickens, it's only fair
to note that about 5 percent of those birds die prematurely. In
contrast, 7.8 percent of children in sub-Saharan Africa die by the age
of 5, according to UNICEF. So heartless agribusiness concerns do a
better job ensuring the survival of baby chicks than the international
community sometimes does for human babies.

A third area where I suspect our descendants will judge us harshly is
climate change. Our generation's denialism will lead to more extreme
weather, more flooded homes, more heat waves --- and resentment that
early-21st-century humans could have been so selfish as to refuse to
take small steps to reduce carbon emissions.

I raised this issue of our moral blind spots in my
\href{http://nytimes.com/kristofemail}{email newsletter} the other day,
and one reader, Brad Marston, a physics professor at Brown University,
put it this way: ``In 100 years our generation may be as poorly regarded
as 19th-century racists are today (or worse), due to our failure to
tackle climate change, leaving a damaged and possibly ruined planet to
future generations.''

So I'm all for re-examining history and removing statues of Confederate
generals. But just as important is our obligation to think deeply about
our own moral myopia today and address it while there is still time.

\emph{The Times is committed to publishing}
\href{https://www.nytimes.com/2019/01/31/opinion/letters/letters-to-editor-new-york-times-women.html}{\emph{a
diversity of letters}} \emph{to the editor. We'd like to hear what you
think about this or any of our articles. Here are some}
\href{https://help.nytimes.com/hc/en-us/articles/115014925288-How-to-submit-a-letter-to-the-editor}{\emph{tips}}\emph{.
And here's our email:}
\href{mailto:letters@nytimes.com}{\emph{letters@nytimes.com}}\emph{.}

Advertisement

\protect\hyperlink{after-bottom}{Continue reading the main story}

\hypertarget{site-index}{%
\subsection{Site Index}\label{site-index}}

\hypertarget{site-information-navigation}{%
\subsection{Site Information
Navigation}\label{site-information-navigation}}

\begin{itemize}
\tightlist
\item
  \href{https://help.nytimes.com/hc/en-us/articles/115014792127-Copyright-notice}{©~2020~The
  New York Times Company}
\end{itemize}

\begin{itemize}
\tightlist
\item
  \href{https://www.nytco.com/}{NYTCo}
\item
  \href{https://help.nytimes.com/hc/en-us/articles/115015385887-Contact-Us}{Contact
  Us}
\item
  \href{https://www.nytco.com/careers/}{Work with us}
\item
  \href{https://nytmediakit.com/}{Advertise}
\item
  \href{http://www.tbrandstudio.com/}{T Brand Studio}
\item
  \href{https://www.nytimes.com/privacy/cookie-policy\#how-do-i-manage-trackers}{Your
  Ad Choices}
\item
  \href{https://www.nytimes.com/privacy}{Privacy}
\item
  \href{https://help.nytimes.com/hc/en-us/articles/115014893428-Terms-of-service}{Terms
  of Service}
\item
  \href{https://help.nytimes.com/hc/en-us/articles/115014893968-Terms-of-sale}{Terms
  of Sale}
\item
  \href{https://spiderbites.nytimes.com}{Site Map}
\item
  \href{https://help.nytimes.com/hc/en-us}{Help}
\item
  \href{https://www.nytimes.com/subscription?campaignId=37WXW}{Subscriptions}
\end{itemize}
