Sections

SEARCH

\protect\hyperlink{site-content}{Skip to
content}\protect\hyperlink{site-index}{Skip to site index}

\href{https://www.nytimes.com/section/politics}{Politics}

\href{https://myaccount.nytimes.com/auth/login?response_type=cookie\&client_id=vi}{}

\href{https://www.nytimes.com/section/todayspaper}{Today's Paper}

\href{/section/politics}{Politics}\textbar{}What Donald Trump's `Access
Hollywood' Weekend Says About 2020

\url{https://nyti.ms/3fmCCsb}

\begin{itemize}
\item
\item
\item
\item
\item
\end{itemize}

\begin{itemize}
\item
  \href{https://www.nytimes.com/2020/07/31/us/elections/biden-vs-trump.html?action=click\&pgtype=Article\&state=default\&region=TOP_BANNER\&context=storylines_menu}{Election
  Updates}
\item
  \href{https://www.nytimes.com/article/biden-vice-president-2020.html?action=click\&pgtype=Article\&state=default\&region=TOP_BANNER\&context=storylines_menu}{Biden's
  V.P. Search}
\item
  \href{https://www.nytimes.com/interactive/2020/07/24/us/politics/trump-biden-campaign-donors.html?action=click\&pgtype=Article\&state=default\&region=TOP_BANNER\&context=storylines_menu}{Map
  of Donations}
\item
  \href{https://www.nytimes.com/interactive/2020/us/elections/delegate-count-primary-results.html?action=click\&pgtype=Article\&state=default\&region=TOP_BANNER\&context=storylines_menu}{Delegate
  Count}
\item
  \href{https://www.nytimes.com/interactive/2019/us/politics/2020-presidential-candidates.html?action=click\&pgtype=Article\&state=default\&region=TOP_BANNER\&context=storylines_menu}{The
  Candidates}
\item
  \href{https://www.nytimes.com/newsletters/politics?action=click\&pgtype=Article\&state=default\&region=TOP_BANNER\&context=storylines_menu}{Politics
  Newsletter}
\end{itemize}

Advertisement

\protect\hyperlink{after-top}{Continue reading the main story}

Supported by

\protect\hyperlink{after-sponsor}{Continue reading the main story}

The Long Run

\hypertarget{what-donald-trumps-access-hollywood-weekend-says-about-2020}{%
\section{What Donald Trump's `Access Hollywood' Weekend Says About
2020}\label{what-donald-trumps-access-hollywood-weekend-says-about-2020}}

On a Friday, the world heard vulgar audio of Mr. Trump boasting about
forcing himself on women. By Sunday night, the episode that was supposed
to doom him had begun to recede.

\includegraphics{https://static01.nyt.com/images/2020/07/09/us/politics/00accesshollywood1/00accesshollywood1-articleLarge.jpg?quality=75\&auto=webp\&disable=upscale}

\href{https://www.nytimes.com/by/matt-flegenheimer}{\includegraphics{https://static01.nyt.com/images/2018/10/02/multimedia/author-matt-flegenheimer/author-matt-flegenheimer-thumbLarge.png}}

By \href{https://www.nytimes.com/by/matt-flegenheimer}{Matt
Flegenheimer}

\begin{itemize}
\item
  July 12, 2020
\item
  \begin{itemize}
  \item
  \item
  \item
  \item
  \item
  \end{itemize}
\end{itemize}

\href{https://www.nytimes.com/es/2020/07/14/espanol/estados-unidos/Donald-trump-eleciones-video-hollywood.html}{Leer
en español}

Donald J. Trump, down and unwilling to get out, saw only one way back
up: Go lower.

Two days had passed since the signal humiliation of his political life
--- the publication of audio in which
\href{https://www.nytimes.com/2016/10/08/us/donald-trump-tape-transcript.html}{Mr.
Trump boasted about forcing himself on women} --- and the candidate was
desperate to redirect the conversation. The result, less than two hours
before an October 2016 debate against Hillary Clinton in St. Louis, was
a gambit so secretive that several concerned parties were left in the
dark.

Campaign advisers told Reince Priebus, the Republican National Committee
chairman who was helping with debate preparations inside the team's
hotel suite, that Mr. Trump had to leave for a perfunctory ``meet and
greet.'' They feared that Mr. Priebus would object if he knew the truth:
Mr. Trump would be appearing on camera with women who had for years
accused Bill Clinton of sexual misconduct --- a brazen attempt to turn
the issue of mistreating women back against the Clinton family.

And those accusers, who had been invited to the debate as surprise Trump
guests but had little warning on the fuller itinerary, seemed unsure
themselves about what awaited them as they were led into a reception
room at the hotel. ``I had no idea what we were going in there for,''
one of them, Juanita Broaddrick, recalled. ``But that doesn't matter. I
would do it all again.''

Before the room's doors opened to the media and the women were revealed,
Stephen K. Bannon, the campaign's chief executive, shared his vision for
the spectacle: ``They're going to rub up on you and be crying,'' he
remembered telling Mr. Trump. ``And you're going to be empathetic.''

Mr. Trump closed his eyes, Mr. Bannon said, tilting his head back ``like
a Roman emperor.''

``I love it,'' the future president ruled.

Four years later, Mr. Trump looks, to all the political world, like a
significant underdog again. His advisers concede that if the election
were held today, he would lose to Joseph R. Biden Jr., the presumptive
Democratic nominee, most likely by a considerable margin.

But as the president road-tests a series of scattershot tactics to
kick-start his struggling campaign ---
\href{https://www.nytimes.com/2020/06/11/us/politics/trump-on-race.html}{race-baiting}
through a national crisis;
\href{https://www.nytimes.com/2020/07/06/us/politics/trump-bubba-wallace-nascar.html}{defending}
symbols of the Confederacy;
\href{https://www.nytimes.com/2020/06/21/health/coronavirus-pandemic-spread-trump.html}{denying}
the objective realities of a pandemic --- allies and adversaries say
their minds have wandered lately to his lowest moment in 2016, the last
time his chances appeared so dire.

The release and aftermath of the so-called ``Access Hollywood'' tape is
at once a reminder of how quickly the contours of an election can change
and of how far Mr. Trump is willing to go to change them. While some
close to Mr. Trump have at times
\href{https://www.nytimes.com/2020/06/17/us/politics/trump-2020-election.html}{questioned}
his focus and resolve in this re-election, his behavior over a weekend
of electoral peril in 2016 supplies a case study in how he can respond
when he feels cornered --- when he suspects he may lose.

``That's his strength,'' said Anthony Scaramucci, the former White House
communications director who has since called for Mr. Trump's defeat.
``When God was handing out shame genes, Trump picked up shameless genes
from that countertop.''

Mr. Trump's capacity for earth-scorching politics, rarely in doubt, has
often been most conspicuous in times of campaign distress. When Ben
Carson surpassed him in some polls of Republican voters in 2015, Mr.
Trump
\href{https://www.nytimes.com/politics/first-draft/2015/10/25/donald-trump-attacks-ben-carson-and-highlights-his-religion}{appeared
to swipe at} his rival's faith. When Ted Cruz proved a resilient primary
foe, Mr. Trump posted an unflattering picture of Mr. Cruz's wife and
\href{https://www.nytimes.com/politics/first-draft/2016/03/23/donald-trump-threatens-ted-cruzs-wife-elicting-angry-retort/}{threatened
to ``spill the beans''} about her, without elaborating.

This is a man who urged a foreign power to investigate Mr. Biden, more
than a year before Election Day 2020, spawning an impeachment inquiry at
home.

In none of those episodes was Mr. Trump confronting the headwinds he
faces now, compelling veterans of 2016 to predict an ugliness in the
coming months that will test the bounds of even the most cynical
strategist's imagination.

They have advised the Biden campaign, sitting comfortably ahead in July,
to brace itself.

``It makes sense for the Biden team to understand why they're winning
today,'' said Robby Mook, Mrs. Clinton's 2016 campaign manager. ``It
makes even more sense for them to think about how they lose.''

Of course, even set against a trove of October surprises through
history, 2016 was something different. In a span of hours on Friday,
Oct. 7, intelligence community leaders
\href{https://www.dhs.gov/news/2016/10/07/joint-statement-department-homeland-security-and-office-director-national}{publicly
accused} Russia of interfering in the election, The Washington Post
\href{https://www.washingtonpost.com/politics/trump-recorded-having-extremely-lewd-conversation-about-women-in-2005/2016/10/07/3b9ce776-8cb4-11e6-bf8a-3d26847eeed4_story.html}{published}
the ``Access Hollywood'' article and WikiLeaks
\href{https://www.nytimes.com/2016/10/12/us/politics/hillary-clinton-emails-wikileaks.html}{began
disseminating} hacked emails from John D. Podesta, Mrs. Clinton's
campaign chairman --- timing that her team did not find coincidental.

The special counsel, Robert S. Mueller III, looked into whether the
release of Mr. Podesta's emails was connected to the ``Access
Hollywood'' tape but did not publicly establish a link to the Trump
campaign. In the end, many Clinton aides believe, nothing that day
affected the election as much as a letter three weeks later from
\href{https://www.nytimes.com/2019/10/12/us/politics/james-comey-trump.html}{James
B. Comey, the F.B.I. director}, reviving the topic of Mrs. Clinton's
private email server.

For admirers of Mr. Trump, these flashbacks register now as a hopeful
memory, a testament to the unpredictability that has long defined his
political arc and might yet again.

\hypertarget{latest-updates-2020-election}{%
\section{\texorpdfstring{\href{https://www.nytimes.com/2020/07/31/us/elections/biden-vs-trump.html?action=click\&pgtype=Article\&state=default\&region=MAIN_CONTENT_1\&context=storylines_live_updates}{Latest
Updates: 2020
Election}}{Latest Updates: 2020 Election}}\label{latest-updates-2020-election}}

Updated 2020-08-01T01:26:45.732Z

\begin{itemize}
\tightlist
\item
  \href{https://www.nytimes.com/2020/07/31/us/elections/biden-vs-trump.html?action=click\&pgtype=Article\&state=default\&region=MAIN_CONTENT_1\&context=storylines_live_updates\#link-29fdff45}{Kamala
  Harris, a top vice-presidential contender, confronts double
  standards.}
\item
  \href{https://www.nytimes.com/2020/07/31/us/elections/biden-vs-trump.html?action=click\&pgtype=Article\&state=default\&region=MAIN_CONTENT_1\&context=storylines_live_updates\#link-13ec3d9c}{Karen
  Bass and Susan Rice are rising on Biden's vice-presidential
  shortlist.}
\item
  \href{https://www.nytimes.com/2020/07/31/us/elections/biden-vs-trump.html?action=click\&pgtype=Article\&state=default\&region=MAIN_CONTENT_1\&context=storylines_live_updates\#link-49e9a016}{Trump
  says Russian bounties to kill U.S. troops `never took place.'}
\end{itemize}

\href{https://www.nytimes.com/2020/07/31/us/elections/biden-vs-trump.html?action=click\&pgtype=Article\&state=default\&region=MAIN_CONTENT_1\&context=storylines_live_updates}{See
more updates}

Today, few of them linger on the details of the weekend when Mr. Trump
seemed all but done: the mass of Republicans urging him to quit; the
taped apology that top advisers have likened to a ``hostage video''; a
public relations calamity so total that even the makers of Tic Tacs, the
breath-fresheners referenced by Mr. Trump on the audio, sent a statement
condemning him.

What supporters do remember is what it felt like to see Mr. Trump fight
back.

``I found that a lot of women reacted to his strength. And I continue to
hear that,'' said Mica Mosbacher, a Republican fund-raiser who sits on
the ``Women for Trump'' 2020 advisory board. ``No one's perfect.''

\includegraphics{https://static01.nyt.com/images/2020/07/09/us/politics/00accesshollywood2/merlin_112849274_780714d4-bfc9-4041-883c-8001fae74070-articleLarge.jpg?quality=75\&auto=webp\&disable=upscale}

\hypertarget{friday}{%
\subsection{Friday}\label{friday}}

Mr. Trump's first instinct, as ever, was defiance: It wasn't me.

Debate practice at Trump Tower had been sidetracked by the low hum of
panic, as an aide, Hope Hicks, handed him a collection of papers --- a
transcript of his vulgar remarks, recorded in 2005 and provided by The
Post, which was seeking comment from the campaign before publication.

\emph{``I just start kissing them. It's like a magnet.''}

\emph{``When you're a star, they let you do it. You can do anything.''}

\emph{``Grab 'em by the pussy. You can do anything.''}

Mr. Trump said these did not sound like things he would say. Advisers
allowed themselves to wonder, briefly, if it had all been a
misunderstanding.

Then the audio file landed. Mr. Trump listened.

``It's me,'' he said.

Across the river in Brooklyn, several Clinton aides had been huddling in
Mr. Mook's office, plotting how best to respond to what they had assumed
would be the story of the day: the intelligence community's assessment
of Russian meddling. A commotion in the headquarters' wider work space
drew them out.

By this point in the race, her staff members had come to view their task
as a kind of teeth-gritting quest, more slog than victory march, pocked
with self-inflicted stumbles and external shocks in relentless measure.

Even ostensible political boosts --- and this certainly looked like one
--- seemed to arrive with a side of nausea.

``Here's the order,'' said Jennifer Palmieri, the campaign's
communications director, recalling her sequence of emotions at the
tape's release: ``Revulsion at what he said, disappointment that no one
was going to care about Russia and dread for Hillary about how unhinged
he was going to become.''

Mrs. Clinton herself was at a hotel in suburban Westchester County,
where she and advisers were holding debate sessions. On the televisions
in a dining area, the group could see cable-news chyrons about the tape.
But initially, nobody could work the sound.

In her 2017 memoir, Mrs. Clinton described an abiding sadness upon
hearing Mr. Trump's words eventually. ``That tape is never going away,''
she wrote. ``It's part of our history now.''

Yet her husband's history, entwined with her own, also made the contents
of the recording especially uncomfortable for the campaign.

Even before ``Access Hollywood,'' her aides had raised the prospect of
Mr. Trump highlighting Mr. Clinton's accusers (and Mrs. Clinton's
posture toward them) to excuse his own misdeeds. And quickly, Mr. Trump
signaled that Mr. Clinton's past would figure in his campaign's
immediate future.

In a statement to reporters as the story went live, Mr. Trump described
his own comments as ``locker room banter'' --- a phrase he came up with
himself, advisers say --- and accused Mr. Clinton of saying ``far worse
to me on the golf course.''

``I apologize,'' Mr. Trump concluded, ``if anyone was offended.''

Some Trump confidantes, including his daughter Ivanka,
\href{https://www.nytimes.com/2017/05/02/us/politics/ivanka-trump.html}{urged
him} to demonstrate less qualified contrition. He agreed to record a
video, to be released hours later, but resisted much of the advice.

The ensuing product --- a surreal
\href{https://www.facebook.com/DonaldTrump/videos/here-is-my-statementive-never-said-im-a-perfect-person-nor-pretended-to-be-someo/10157844642270725/}{90-second
address}, delivered in front of a faux skyline --- was a hodgepodge of
his team's dueling impulses.

He issued a rare admission of outright fault (``I said it, I was wrong,
and I apologize''), framed his life as a tale of growth (``my travels
have also changed me'') and pivoted swiftly to accusing the Clintons of
hypocrisy.

``We will discuss this more in the coming days,'' he pledged. ``See you
at the debate on Sunday.''

Image

Mr. Trump greeted supporters outside of Trump Tower on Saturday, October
8, 2016.~Credit...Spencer Platt/Getty Images

\hypertarget{saturday}{%
\subsection{Saturday}\label{saturday}}

If many top Republicans had gotten their way, Mr. Trump would not have
made it to the debate on Sunday.

Publicly and privately, lawmakers
\href{https://www.pbs.org/newshour/politics/headline-republicans-react-trump-comments-objectifying-women}{were
calling} on him to step aside and allow Mike Pence to lead the ticket.
Party officials projected devastation down-ballot. Others simply could
not stomach associating with the nominee.

``I just felt embarrassed,'' said Carlos Curbelo, a former Republican
congressman who was at the time seeking re-election to a competitive
South Florida seat and had already disavowed Mr. Trump's campaign. ``In
Spanish, we have this concept called `pena ajena.' You're not involved
in the activity, but you feel shame.''

Speaker Paul D. Ryan disinvited Mr. Trump from a Saturday rally at which
they were slated to appear together in Wisconsin. ``There is a bit of an
elephant in the room,'' Mr. Ryan
\href{https://madison.com/wsj/news/local/govt-and-politics/paul-ryan-heckled-on-home-turf-after-donald-trump-mike-pence-scratched-from-gop-event/article_8e0acf67-1f2f-56a4-b42d-e8ebcbd0bc67.html}{told
the crowd}, navigating persistent jeers.

``Where is Trump?'' attendees shouted.

Back in New York, junior aides tracked the defections, exchanging
mutters and fears.

``There's another.''

``Another.''

``Are we ever going to be able to work in Washington again?''

The boss was more puckish, if only to try lightening the mood:
``Certainly has been an interesting 24 hours!'' he
\href{https://twitter.com/realdonaldtrump/status/784767399442653184?lang=en}{tweeted}
on Saturday morning.

He gathered senior staff members inside Trump Tower and asked Mr.
Priebus what he was hearing. What the party chairman was hearing, he
answered, was that Mr. Trump could either drop out or lose in a historic
landslide.

``So,'' Mr. Trump said, ``what's the good news?''

The flourish resonates four years later as a tidy encapsulation of the
Trumpian worldview in a campaign crisis. In recent months, he has been
known to
\href{https://www.nytimes.com/2020/04/29/us/politics/trump-campaign-reelection-polls.html}{lash
out at advisers} who share dispiriting poll numbers, insisting that his
position cannot be so precarious.

People who know him cite this semi-magical thinking as a kind of
political superpower, when harnessed effectively.

``Don't underestimate his personal resiliency,'' Mr. Scaramucci said. He
recalled the president's advice to him once about news-cycle velocity:
``He said, `Yeah, you get negative press. It lasts about a week. And
then it blows over, and they're onto something else, and nobody
cares.'''

On this weekend, though, Trump advisers sensed that little would blow
over on its own.

The idea of deploying Mr. Clinton's accusers had filtered through the
Trump orbit for months, discussed among Mr. Bannon and allies like Aaron
Klein of Breitbart News --- the hard-right, Trump-supporting site that
Mr. Bannon had run --- and long promoted by Roger J. Stone Jr., the
informal Trump adviser and infamous Republican hell-raiser. (On Friday,
Mr. Trump
\href{https://www.nytimes.com/2020/07/10/us/politics/trump-roger-stone-clemency.html}{commuted}
Mr. Stone's sentence on seven felony crimes after he had been convicted
last year of obstructing a congressional investigation into the Trump
2016 campaign and possible ties to Russia.)

Just before ``Billy Bush weekend,'' as Mr. Bannon calls it (a nod to the
``Access Hollywood'' personality on tape with Mr. Trump), three of the
Clinton accusers had been in Washington for interviews with Mr. Klein.

As Republican pleas for Mr. Trump's ouster multiplied, Mr. Bannon
recognized an opportunity. He said he called Mr. Klein, now an adviser
to Prime Minister Benjamin Netanyahu of Israel, and asked how the
Clinton material looked. The answer pleased him. New travel arrangements
were made.

In the meantime, Mr. Trump sought temporary comfort in a familiar balm:
applause.

By 5 p.m. on Saturday, supporters had clustered along Fifth Avenue,
waving signs from the sidewalk. Mr. Trump
\href{https://www.nytimes.com/2016/10/10/us/politics/trump-tower.html}{descended}
to the marbled lobby, joined by his eldest son and his campaign manager,
Kellyanne Conway, and stepped through the glass front door.

He pumped his right fist, to cheers. Fans reached out to graze his suit
jacket.

A reporter asked if he would stay in the race. ``Hundred percent,'' Mr.
Trump replied.

And then he turned back inside, clapping on the way.

Image

After the tape was revealed, Mr. Trump held a news conference with three
women, Kathleen Willey, Juanita Broaddrick and Paula Jones, who had
accused Mr. Clinton of sexual misconduct, and a fourth, Kathy Shelton,
who said that in her youth she was raped by a man whom Mrs. Clinton
represented as a court-appointed defense lawyer in the
1970s.~Credit...Evan Vucci/Associated Press

\hypertarget{sunday}{%
\subsection{Sunday}\label{sunday}}

Mr. Trump seemed to be calling his shot.

``EXCLUSIVE,'' he
\href{https://twitter.com/realdonaldtrump/status/785106800572063744?lang=en}{tweeted}
Sunday morning, sharing a Breitbart link. ``Video Interview: Bill
Clinton Accuser Juanita Broaddrick Relives Brutal Rapes.''

There was little doubt that Mr. Trump would talk about women from the
Clintons' past in St. Louis. But few knew that four women were on their
way themselves: three who had accused Mr. Clinton of sexual misconduct
--- Ms. Broaddrick, Kathleen Willey and Paula Jones --- and a fourth,
Kathy Shelton, who in her youth said she was raped by a man whom Mrs.
Clinton represented as a court-appointed defense lawyer in the 1970s.

Mr. Clinton has long denied wrongdoing in these three instances; Mrs.
Clinton has said she was displeased at the court appointment in the
Shelton case but had little choice but to accept it.

A couple of hours before the debate, Ms. Broaddrick said, she was taken
up a hotel service elevator to meet Mr. Trump privately with Ms. Willey
and Ms. Shelton. Ms. Jones arrived later, in time for the next portion.

``It was delightful,'' Ms. Broaddrick said. ``And then as we start to
leave, Steve Bannon says, `Let's go through this door here.'''

The women were ushered into an adjacent room with a long table,
according to Ms. Broaddrick, who assumed a catered meal was imminent.
They were asked to sit in a row, shoulder to shoulder. Then Mr. Trump
entered and took a seat between them, with two chairs on each side of
him.

``Let them in,'' the candidate instructed.

The doors opened to Mr. Trump's traveling press corps, which the
campaign had brought to the hotel. The reporters appeared confused. Mr.
Bannon beamed.

``These four very courageous women have asked to be here,'' Mr. Trump
\href{https://www.facebook.com/DonaldTrump/videos/10157857037430725/}{said
into the cameras}, ``and it was our honor to help them. And I think
they're each going to make just an individual short statement.''

The guests praised Mr. Trump in succession, at times sharing details of
their claims against the Clintons. Mr. Trump nodded sternly. It was over
in three minutes.

As the gathering broke, reporters called out questions to Mr. Trump
about touching women without consent. Ms. Jones cut in. ``Why don't
y'all go ask Bill Clinton that?'' she said, as Mr. Trump stared forward.
``Go ahead --- ask Hillary, as well.''

The Trump team thrilled at the scene, in part because the often leaky
campaign had successfully kept a secret. The women were awe-struck but
appeared grateful for the megaphone. ``Oh, I was so excited,'' Ms. Jones
said in an interview. ``That felt so good that I got to say that.''

At the debate site, Clinton aides absorbed the production with a mix of
alarm and performative stoicism --- all the more after the Trump
campaign tried to place the women in the seating area for the families
of the candidates, before finding another spot for them in the debate
hall.

Backstage, Mrs. Clinton's advisers told her that Mr. Trump was merely
trying to get in her head.

``Yeah,'' Mrs. Clinton said. ``I got that.''

``The great thing is, it didn't work,'' Ms. Palmieri remembered
replying.

``Nope,'' Mrs. Clinton answered. ``Didn't work.''

Image

Mr. Trump described the language in the video as ``locker-room talk''
during the debate.Credit...Doug Mills/The New York Times

The debate itself, held in a town hall format, was at once stunning and
not surprising in its simmering hostility. The two did not shake hands
at the start. She called him unfit to serve. He suggested that she would
be jailed if he won and often loomed ominously behind her as she spoke.

Mr. Trump
\href{https://www.politico.com/magazine/story/2019/07/10/american-carnage-excerpt-access-hollywood-tape-227269}{has
said} the debate won him the election. At minimum, the evening appeared
to stabilize a campaign that seemed
\href{https://www.huffpost.com/entry/yahoo-64-hours-october-american-politics_n_59d7c567e4b072637c43dd1c}{liable
to capsize} for 48 hours.

He made clear that he saw no reason to step aside or submit to further
public remorse. Most supporters plainly saw no reason to demand as much,
either.

``It's locker-room talk,'' Mr. Trump said,
\href{https://www.nytimes.com/2016/10/10/us/politics/transcript-second-debate.html}{repeating
the formulation} five times onstage, ``and it's one of those things.''

By the time he left St. Louis, the episode that was supposed to doom
him, by bipartisan consensus, had begun to recede.

But the true coda to the weekend --- and perhaps the purest snapshot of
Mr. Trump's ultimate psychology when he feels attacked --- did not come
at the debate. Or the next day. Or even with his election weeks later.

It arrived in the months that followed. As he prepared to take office,
Mr. Trump, validated by his November triumph, began privately floating a
curious theory about the tape's authenticity, an alternate history he
preferred to the real one:

It wasn't him.

Maggie Haberman contributed reporting. Kitty Bennett contributed
research.

\hypertarget{our-2020-election-guide}{%
\section{Our 2020 Election Guide}\label{our-2020-election-guide}}

Updated July 31, 2020

\begin{itemize}
\item
  \begin{center}\rule{0.5\linewidth}{\linethickness}\end{center}

  \hypertarget{the-latest}{%
  \subsection{The Latest}\label{the-latest}}

  \begin{itemize}
  \tightlist
  \item
    President Trump's assault on the Postal Service is intersecting with
    his attacks on mail-in voting.
    \href{https://www.nytimes.com/2020/07/31/us/politics/trump-usps-mail-delays.html?action=click\&pgtype=Article\&state=default\&region=BELOW_MAIN_CONTENT\&context=storylines_guide}{Voting
    rights groups say it is a recipe for disaster.}
  \end{itemize}
\item
  \begin{center}\rule{0.5\linewidth}{\linethickness}\end{center}

  \hypertarget{bidens-vp-search}{%
  \subsection{Biden's V.P. Search}\label{bidens-vp-search}}

  \begin{itemize}
  \tightlist
  \item
    \href{https://www.nytimes.com/article/biden-vice-president-2020.html?action=click\&pgtype=Article\&state=default\&region=BELOW_MAIN_CONTENT\&context=storylines_guide}{Here
    are 13 women} who have been under consideration to be Joe Biden's
    running mate, and why each might be chosen --- and might not be.
  \end{itemize}
\item
  \begin{center}\rule{0.5\linewidth}{\linethickness}\end{center}

  \hypertarget{keep-up-with-our-coverage}{%
  \subsection{Keep Up With Our
  Coverage}\label{keep-up-with-our-coverage}}

  \begin{itemize}
  \tightlist
  \item
    Get an
    \href{https://www.nytimes.com/newsletters/politics?action=click\&pgtype=Article\&state=default\&region=BELOW_MAIN_CONTENT\&context=storylines_guide}{email}
    recapping the day's news
  \end{itemize}

  \begin{itemize}
  \tightlist
  \item
    Download our mobile app on
    \href{https://apps.apple.com/us/app/nytimes/id284862083?ls=1\&mat_click_id=5c79ae7455014fd1bd66b5610c05b8f2-20191112-16948\&referrer=mat_click_id\%3D5c79ae7455014fd1bd66b5610c05b8f2-20191112-16948\%26link_click_id\%3D722930677036718082}{iOS}
    and
    \href{http://a.localytics.com/android?id=com.nytimes.android\&referrer=utm_source\%3Dother_nyt_mobile_web\%26utm_medium\%3DWeb\%2520page\%26utm_term\%3DGeneral\%2520Mobile\%2520Page\%26utm_campaign\%3DNYT\%2520Mobile\%2520General\%2520Page}{Android}
    and turn on Breaking News and Politics alerts
  \end{itemize}
\end{itemize}

Advertisement

\protect\hyperlink{after-bottom}{Continue reading the main story}

\hypertarget{site-index}{%
\subsection{Site Index}\label{site-index}}

\hypertarget{site-information-navigation}{%
\subsection{Site Information
Navigation}\label{site-information-navigation}}

\begin{itemize}
\tightlist
\item
  \href{https://help.nytimes.com/hc/en-us/articles/115014792127-Copyright-notice}{©~2020~The
  New York Times Company}
\end{itemize}

\begin{itemize}
\tightlist
\item
  \href{https://www.nytco.com/}{NYTCo}
\item
  \href{https://help.nytimes.com/hc/en-us/articles/115015385887-Contact-Us}{Contact
  Us}
\item
  \href{https://www.nytco.com/careers/}{Work with us}
\item
  \href{https://nytmediakit.com/}{Advertise}
\item
  \href{http://www.tbrandstudio.com/}{T Brand Studio}
\item
  \href{https://www.nytimes.com/privacy/cookie-policy\#how-do-i-manage-trackers}{Your
  Ad Choices}
\item
  \href{https://www.nytimes.com/privacy}{Privacy}
\item
  \href{https://help.nytimes.com/hc/en-us/articles/115014893428-Terms-of-service}{Terms
  of Service}
\item
  \href{https://help.nytimes.com/hc/en-us/articles/115014893968-Terms-of-sale}{Terms
  of Sale}
\item
  \href{https://spiderbites.nytimes.com}{Site Map}
\item
  \href{https://help.nytimes.com/hc/en-us}{Help}
\item
  \href{https://www.nytimes.com/subscription?campaignId=37WXW}{Subscriptions}
\end{itemize}
