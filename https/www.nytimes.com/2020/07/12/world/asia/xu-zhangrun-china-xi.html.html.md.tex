Sections

SEARCH

\protect\hyperlink{site-content}{Skip to
content}\protect\hyperlink{site-index}{Skip to site index}

\href{https://www.nytimes.com/section/world/asia}{Asia Pacific}

\href{https://myaccount.nytimes.com/auth/login?response_type=cookie\&client_id=vi}{}

\href{https://www.nytimes.com/section/todayspaper}{Today's Paper}

\href{/section/world/asia}{Asia Pacific}\textbar{}Outspoken Chinese
Professor Is Said to Be Released From Detention

\url{https://nyti.ms/3fk4zAE}

\begin{itemize}
\item
\item
\item
\item
\item
\end{itemize}

\href{https://www.nytimes.com/news-event/coronavirus?action=click\&pgtype=Article\&state=default\&region=TOP_BANNER\&context=storylines_menu}{The
Coronavirus Outbreak}

\begin{itemize}
\tightlist
\item
  live\href{https://www.nytimes.com/2020/08/03/world/coronavirus-covid-19.html?action=click\&pgtype=Article\&state=default\&region=TOP_BANNER\&context=storylines_menu}{Latest
  Updates}
\item
  \href{https://www.nytimes.com/interactive/2020/us/coronavirus-us-cases.html?action=click\&pgtype=Article\&state=default\&region=TOP_BANNER\&context=storylines_menu}{Maps
  and Cases}
\item
  \href{https://www.nytimes.com/interactive/2020/science/coronavirus-vaccine-tracker.html?action=click\&pgtype=Article\&state=default\&region=TOP_BANNER\&context=storylines_menu}{Vaccine
  Tracker}
\item
  \href{https://www.nytimes.com/2020/08/02/us/covid-college-reopening.html?action=click\&pgtype=Article\&state=default\&region=TOP_BANNER\&context=storylines_menu}{College
  Reopening}
\item
  \href{https://www.nytimes.com/live/2020/08/03/business/stock-market-today-coronavirus?action=click\&pgtype=Article\&state=default\&region=TOP_BANNER\&context=storylines_menu}{Economy}
\end{itemize}

Advertisement

\protect\hyperlink{after-top}{Continue reading the main story}

Supported by

\protect\hyperlink{after-sponsor}{Continue reading the main story}

\hypertarget{outspoken-chinese-professor-is-said-to-be-released-from-detention}{%
\section{Outspoken Chinese Professor Is Said to Be Released From
Detention}\label{outspoken-chinese-professor-is-said-to-be-released-from-detention}}

Xu Zhangrun, a law professor in Beijing known for criticizing the
Communist Party, was allowed to go home after being detained a week ago,
people familiar with him said.

\includegraphics{https://static01.nyt.com/images/2020/07/12/world/12china-critic/12china-critic-articleLarge.jpg?quality=75\&auto=webp\&disable=upscale}

\href{https://www.nytimes.com/by/chris-buckley}{\includegraphics{https://static01.nyt.com/images/2018/10/08/multimedia/author-chris-buckley/author-chris-buckley-thumbLarge.png}}

By \href{https://www.nytimes.com/by/chris-buckley}{Chris Buckley}

\begin{itemize}
\item
  July 12, 2020
\item
  \begin{itemize}
  \item
  \item
  \item
  \item
  \item
  \end{itemize}
\end{itemize}

An outspoken Chinese law professor who has denounced the Communist
Party's harsh policies under Xi Jinping was released from detention on
Sunday, a week after
\href{https://www.nytimes.com/2020/07/06/world/asia/china-detains-xu-zhangrun-critic.html}{the
police took him away}, two people familiar with the professor said.

The professor, Xu Zhangrun of Tsinghua University in Beijing, has been
one of the few Chinese academics willing to openly and bluntly criticize
the Chinese government. In two
\href{https://www.chinafile.com/reporting-opinion/viewpoint/viral-alarm-when-fury-overcomes-fear}{essays
this year}, he said that official delay and prevarication had stoked the
coronavirus outbreak that emerged in China late last year.

``When decisions lead to policy failure, not only should the course be
corrected, those responsible must acknowledge their mistakes, appeal in
all humility for public forgiveness and be held accountable,'' Professor
Xu wrote in
\href{https://www.chinese-future.org/articles/6j3ywr3swgp3flagdlbk3xd2wkmjcj}{an
essay} in May,
\href{http://chinaheritage.net/journal/remonstrating-with-beijing-xu-zhangruns-advice-to-chinas-national-peoples-congress-21-may-2020/}{according
to a translation} by Geremie R. Barmé, an Australian Sinologist.

Professor Xu, 57, first attracted widespread notice in 2018 for a
\href{http://chinaheritage.net/journal/imminent-fears-immediate-hopes-a-beijing-jeremiad/}{long
essay} that lamented China's increasingly authoritarian turn under Mr.
Xi, the Communist Party leader. Soon after taking power, Mr. Xi
\href{https://www.nytimes.com/2013/08/20/world/asia/chinas-new-leadership-takes-hard-line-in-secret-memo.html}{launched
a drive to discredit} liberal ideas such as universal human rights and
constitutional limits on party power.

Professor Xu was
\href{https://www.nytimes.com/2019/03/26/world/asia/chinese-law-professor-xi.html}{barred
from teaching and research} by Tsinghua University last year, sharply
reducing his income, and he has been warned about his criticisms of the
Chinese government. Even so, he has continued to write trenchant essays
that have been published abroad, and told friends that he accepted the
possibility of detention.

The police took Professor Xu from his home in north Beijing on Monday,
and friends said that officers accused him of using prostitutes while
traveling in southwest China, an accusation the friends called a
malicious slur. The detention prompted widespread criticism abroad,
including from the Trump administration and the
\href{https://eeas.europa.eu/headquarters/headquarters-homepage/82435/china-statement-spokesperson-5th-anniversary-\%E2\%80\%9C709-crackdown\%E2\%80\%9D-human-rights-lawyers-and_en}{European
Union}.

``We are deeply concerned by the PRC's detention of Professor Xu
Zhangrun for criticizing Chinese leaders amid tightening ideological
controls on university campuses in China,'' Morgan Ortagus, a
spokeswoman for the U.S. State Department,
\href{https://twitter.com/statedeptspox/status/1280529327826837504}{said
on Twitter}, referring to the People's Republic of China.

Two people familiar with the professor and his family said that he
returned to his home on Sunday morning. Both people asked that their
names not be used, citing fear of recrimination by the Chinese
authorities.

The police in Changping, the northern district of Beijing where the
professor lives, did not confirm his release, just as they did not
confirm his detention last week.

Advertisement

\protect\hyperlink{after-bottom}{Continue reading the main story}

\hypertarget{site-index}{%
\subsection{Site Index}\label{site-index}}

\hypertarget{site-information-navigation}{%
\subsection{Site Information
Navigation}\label{site-information-navigation}}

\begin{itemize}
\tightlist
\item
  \href{https://help.nytimes.com/hc/en-us/articles/115014792127-Copyright-notice}{©~2020~The
  New York Times Company}
\end{itemize}

\begin{itemize}
\tightlist
\item
  \href{https://www.nytco.com/}{NYTCo}
\item
  \href{https://help.nytimes.com/hc/en-us/articles/115015385887-Contact-Us}{Contact
  Us}
\item
  \href{https://www.nytco.com/careers/}{Work with us}
\item
  \href{https://nytmediakit.com/}{Advertise}
\item
  \href{http://www.tbrandstudio.com/}{T Brand Studio}
\item
  \href{https://www.nytimes.com/privacy/cookie-policy\#how-do-i-manage-trackers}{Your
  Ad Choices}
\item
  \href{https://www.nytimes.com/privacy}{Privacy}
\item
  \href{https://help.nytimes.com/hc/en-us/articles/115014893428-Terms-of-service}{Terms
  of Service}
\item
  \href{https://help.nytimes.com/hc/en-us/articles/115014893968-Terms-of-sale}{Terms
  of Sale}
\item
  \href{https://spiderbites.nytimes.com}{Site Map}
\item
  \href{https://help.nytimes.com/hc/en-us}{Help}
\item
  \href{https://www.nytimes.com/subscription?campaignId=37WXW}{Subscriptions}
\end{itemize}
