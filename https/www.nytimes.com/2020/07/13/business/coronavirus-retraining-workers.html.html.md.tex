Sections

SEARCH

\protect\hyperlink{site-content}{Skip to
content}\protect\hyperlink{site-index}{Skip to site index}

\href{https://www.nytimes.com/section/business}{Business}

\href{https://myaccount.nytimes.com/auth/login?response_type=cookie\&client_id=vi}{}

\href{https://www.nytimes.com/section/todayspaper}{Today's Paper}

\href{/section/business}{Business}\textbar{}The Pandemic Has Accelerated
Demands for a More Skilled Work Force

\url{https://nyti.ms/2ZlL0lW}

\begin{itemize}
\item
\item
\item
\item
\item
\item
\end{itemize}

\href{https://www.nytimes.com/news-event/coronavirus?action=click\&pgtype=Article\&state=default\&region=TOP_BANNER\&context=storylines_menu}{The
Coronavirus Outbreak}

\begin{itemize}
\tightlist
\item
  live\href{https://www.nytimes.com/2020/08/01/world/coronavirus-covid-19.html?action=click\&pgtype=Article\&state=default\&region=TOP_BANNER\&context=storylines_menu}{Latest
  Updates}
\item
  \href{https://www.nytimes.com/interactive/2020/us/coronavirus-us-cases.html?action=click\&pgtype=Article\&state=default\&region=TOP_BANNER\&context=storylines_menu}{Maps
  and Cases}
\item
  \href{https://www.nytimes.com/interactive/2020/science/coronavirus-vaccine-tracker.html?action=click\&pgtype=Article\&state=default\&region=TOP_BANNER\&context=storylines_menu}{Vaccine
  Tracker}
\item
  \href{https://www.nytimes.com/interactive/2020/07/29/us/schools-reopening-coronavirus.html?action=click\&pgtype=Article\&state=default\&region=TOP_BANNER\&context=storylines_menu}{What
  School May Look Like}
\item
  \href{https://www.nytimes.com/live/2020/07/31/business/stock-market-today-coronavirus?action=click\&pgtype=Article\&state=default\&region=TOP_BANNER\&context=storylines_menu}{Economy}
\end{itemize}

Advertisement

\protect\hyperlink{after-top}{Continue reading the main story}

Supported by

\protect\hyperlink{after-sponsor}{Continue reading the main story}

\hypertarget{the-pandemic-has-accelerated-demands-for-a-more-skilled-work-force}{%
\section{The Pandemic Has Accelerated Demands for a More Skilled Work
Force}\label{the-pandemic-has-accelerated-demands-for-a-more-skilled-work-force}}

Even groups that regularly disagree on labor issues said there should be
significant public investment in programs that can upgrade the skills of
American workers.

\includegraphics{https://static01.nyt.com/images/2020/07/09/business/00virus-retrain1/00virus-retrain1-articleLarge-v2.jpg?quality=75\&auto=webp\&disable=upscale}

\href{https://www.nytimes.com/by/steve-lohr}{\includegraphics{https://static01.nyt.com/images/2018/02/20/multimedia/author-steve-lohr/author-steve-lohr-thumbLarge.jpg}}

By \href{https://www.nytimes.com/by/steve-lohr}{Steve Lohr}

\begin{itemize}
\item
  July 13, 2020
\item
  \begin{itemize}
  \item
  \item
  \item
  \item
  \item
  \item
  \end{itemize}
\end{itemize}

Economists, business leaders and labor experts have warned for years
that a coming wave of automation and digital technology would upend the
work force, destroying some jobs while altering how and where work is
done for nearly everyone.

In the past four months,
\href{https://www.nytimes.com/news-event/coronavirus}{the coronavirus
pandemic} has transformed some of those predictions into reality. By
May, half of Americans were working from home, tethered to their
employers via laptops and Wi-Fi, up from 15 percent before the pandemic,
according to \href{https://www.brynjolfsson.com/remotework/}{a recent
study}.

The rapid change is leading to mounting demands --- including from
typically opposing groups, like Republicans and Democrats, and business
executives and labor leaders --- for training programs for millions of
workers. On their own, some of the proposals are modest. But combined
they could cost tens of billions of dollars, in what would be one of the
most ambitious retraining efforts in generations.

``This is the moment when we should make a significant public
investment,'' said David Autor, a labor economist at M.I.T., ``when we
should have a Marshall Plan for ourselves.''

A group of mainly corporate executives and educators advising the Trump
administration on work force policy called for
\href{https://www.commerce.gov/sites/default/files/2020-05/AWPABCalltoActionFINAL051520.pdf}{``immediate
and unprecedented investments in American workers,''}both for training
and help in finding jobs. And even before the pandemic, former Vice
President Joseph R. Biden Jr.
\href{https://twitter.com/JoeBiden/status/1183170484269174784}{had
proposed investing \$50 billion} in work force training.

In Congress, there is
\href{https://www.klobuchar.senate.gov/public/index.cfm/2020/5/klobuchar-sasse-booker-scott-introduce-legislation-to-help-unemployed-american-workers-access-skills-training-programs-during-coronavirus-pandemic}{bipartisan
support} for giving jobless workers a \$4,000 training credit.

The Markle Foundation has proposed
\href{https://www.markle.org/stimulus-american-opportunity}{federally
funded ``opportunity accounts}'' of up to \$15,000 for workers to spend
on training. Union leaders have helped the administration in an effort
to expand federal apprenticeship programs to a wide range of industries.

Past downturns have brought increased government aid for workers and
training programs. But labor experts say they have tended to be policies
that recede once the economy recovers, as happened after the 2008
financial crisis, rather than becoming national priorities.

``The Great Recession was a lost opportunity,'' said Lawrence Katz, an
economist at Harvard University. ``Now, are we going to take this moment
to help low-wage workers move into the middle class and give them skills
to thrive? Or are they just going to go back to low-wage jobs that are
dead ends?''

In the coronavirus economy, companies are adopting more automation, as
they seek to cut costs and increase efficiency. There is debate about
which jobs are most at risk and how soon. But climbing up the skills
ladder is the best way to stay ahead of the automation wave.

Middle-class jobs in today's economy often require some digital skills
but are not considered tech jobs. Data scientists at LinkedIn, a
Microsoft subsidiary, recently mined millions of job listings to
identify
\href{https://blogs.microsoft.com/blog/2020/06/30/microsoft-launches-initiative-to-help-25-million-people-worldwide-acquire-the-digital-skills-needed-in-a-covid-19-economy/}{10
occupations most in demand} in recent years and likely to remain so. The
list included project manager, sales representative, customer service
specialist and graphic designer --- nontech jobs that have been
transformed by technology.

\hypertarget{latest-updates-economy}{%
\section{\texorpdfstring{\href{https://www.nytimes.com/live/2020/07/31/business/stock-market-today-coronavirus?action=click\&pgtype=Article\&state=default\&region=MAIN_CONTENT_1\&context=storylines_live_updates}{Latest
Updates:
Economy}}{Latest Updates: Economy}}\label{latest-updates-economy}}

\href{https://www.nytimes.com/live/2020/07/31/business/stock-market-today-coronavirus?action=click\&pgtype=Article\&state=default\&region=MAIN_CONTENT_1\&context=storylines_live_updates\#kodaks-chief-executive-was-given-stock-options-then-the-share-price-spiked-1000-percent}{34h
ago}

\href{https://www.nytimes.com/live/2020/07/31/business/stock-market-today-coronavirus?action=click\&pgtype=Article\&state=default\&region=MAIN_CONTENT_1\&context=storylines_live_updates\#kodaks-chief-executive-was-given-stock-options-then-the-share-price-spiked-1000-percent}{Kodak's
chief executive was given stock options. Then the share price spiked
1,000 percent.}

\href{https://www.nytimes.com/live/2020/07/31/business/stock-market-today-coronavirus?action=click\&pgtype=Article\&state=default\&region=MAIN_CONTENT_1\&context=storylines_live_updates\#fitch-ratings-downgrades-its-outlook-on-us-debt}{37h
ago}

\href{https://www.nytimes.com/live/2020/07/31/business/stock-market-today-coronavirus?action=click\&pgtype=Article\&state=default\&region=MAIN_CONTENT_1\&context=storylines_live_updates\#fitch-ratings-downgrades-its-outlook-on-us-debt}{Fitch
Ratings downgrades its outlook on U.S. debt.}

\href{https://www.nytimes.com/live/2020/07/31/business/stock-market-today-coronavirus?action=click\&pgtype=Article\&state=default\&region=MAIN_CONTENT_1\&context=storylines_live_updates\#us-sanctions-more-chinese-officials-over-human-rights-violations-as-tensions-flare}{44h
ago}

\href{https://www.nytimes.com/live/2020/07/31/business/stock-market-today-coronavirus?action=click\&pgtype=Article\&state=default\&region=MAIN_CONTENT_1\&context=storylines_live_updates\#us-sanctions-more-chinese-officials-over-human-rights-violations-as-tensions-flare}{U.S.
sanctions more Chinese officials over human rights violations as
tensions flare}

\href{https://www.nytimes.com/live/2020/07/31/business/stock-market-today-coronavirus?action=click\&pgtype=Article\&state=default\&region=MAIN_CONTENT_1\&context=storylines_live_updates}{See
more updates}

More live coverage:
\href{https://www.nytimes.com/2020/08/01/world/coronavirus-covid-19.html?action=click\&pgtype=Article\&state=default\&region=MAIN_CONTENT_1\&context=storylines_live_updates}{Global}

Researchers at the Markle Foundation, in
\href{https://www.markle.org/digitalblindspot}{a report published last
year,} studied the pace of digital tools moving into occupations during
the previous decade. The fastest rates of digitization were in jobs in
retail, warehouses and health care. So a training path might be to help
a home health worker acquire the skills to become a medical technician.

\includegraphics{https://static01.nyt.com/images/2020/07/09/business/00virus-retrain2/merlin_168350511_976d2b20-1e23-4fae-b79f-8a5652e23fea-articleLarge.jpg?quality=75\&auto=webp\&disable=upscale}

Job training in America has often been ineffective, with programs shaped
by local politics and money spent according to the number of people in
courses rather than hiring outcomes. The United States also spends less
than other nations on government employment, training and other labor
services. As a percentage of economic activity, Canada spends three
times as much, Germany about six times more and Scandinavian countries
up to more than 12 times as much, according to the
\href{https://data.oecd.org/socialexp/public-spending-on-labour-markets.htm}{Organization
for Economic Cooperation and Development}.

But there are encouraging pockets of success in America. Some are
nonprofit programs like \href{https://www.yearup.org/}{Year Up},
\href{https://perscholas.org/}{Per Scholas} and
\href{https://questsa.org/}{Project Quest}, which prepare low-income
adults for higher-paying careers in technology, health care, advanced
manufacturing or business. In recent months, these programs have quickly
changed to online teaching and coaching from in-person programming.

In addition, online training on learning networks like
\href{https://www.coursera.org/}{Coursera} and
\href{https://www.udacity.com/}{Udacity} and at digital-only
institutions like the \href{https://www.wgu.edu/}{Western Governors
University}, experts say, can be engines for upgrading the skills of
many workers in America.

Two years ago, Tony Boswell, 50, decided it was time for a career change
after 14 years as a long-haul trucker. He held \$100,000 in loans on his
truck and trailer, and then his engine blew and he put the \$23,000
repair bill on his credit card.

``I figured I had to do something else,'' Mr. Boswell recalled.

At home in Kansas City, Mo., when his truck engine was being rebuilt, he
applied for a Google scholarship to pay for a three-month course in
software development on Udacity. Mr. Boswell got the scholarship and
then another one for a six-month Udacity program to become a website
programmer, saving him the \$1,800 cost of the courses.

Through days and nights of effort, Mr. Boswell mastered the digital
tools of his new field. Today he works for a software development
company in Kansas City. He makes nearly \$60,000 a year, substantially
more than he netted most years as a truck driver. ``I have a life now,''
he said, ``and a future.''

Image

``I have a life now, and a future,'' said Tony Boswell, a former
long-haul trucker who studied to become a software
engineer.Credit...Shawn Brackbill for The New York Times

Nearly 15 percent of people earning bachelor's degrees in nursing in the
United States last year graduated from Western Governors University, an
online nonprofit university focused on working adults who want to learn
new skills. Sarah Williams was one of them.

For Ms. Williams, a 38-year-old working mother of three, the online
program was the only realistic option.

Wrestling with college-level statistics and organic chemistry was
daunting. But she made it through in eight months, with frequent
encouragement and help from her mentor. ``No way I could have done it
alone,'' Ms. Williams said.

The vital role of human assistance, delivered in person or online, by a
mentor or coach is a common thread in skills-building success stories.
Another essential element is employers being engaged as participants in
worker training.

The Trump administration is an enthusiastic proponent of expanding the
apprenticeship concept beyond the construction and building trades. In
March, the Labor Department announced
\href{https://www.dol.gov/newsroom/releases/eta/eta20200310}{Industry
Recognized Apprenticeship Programs} to foster apprenticeships in sectors
like technology and health care. As a first step, it has called on
businesses to form standards-setting bodies for apprenticeships in their
industries.

Ivanka Trump, a chair of the administration's
\href{https://www.commerce.gov/americanworker/american-workforce-policy-advisory-board}{American
Workforce Policy Advisory Board}, is an advocate for broadening
apprenticeship opportunities and making Pell grants, for low-income
students, available for noncollege skills training and job certification
programs.

The administration sees the government's role as working with the
business community and ensuring that the public system of state and
local work force boards, which receive federal funding, are attuned to
hiring demands of the private sector.

``Re-skilling very much needs to come from business,'' said John
Pallasch, assistant secretary for employment and training at the Labor
Department.

Companies, labor experts say, will need to increase their investment in
enhancing the skills of their own workers. Analysts say the overall
trend has been stagnant or declining for years.

The companies that have increased spending recognize training as a
competitive necessity. \href{https://www.exeloncorp.com/}{Exelon}, a
utility holding company, whose operations serve 10 million customers in
the Mid-Atlantic and the Midwest, decided it needed to make better use
of all of the data streaming off its power generators and grid.

In 2018, Exelon began an in-house education program for its managers and
engineers. The goal was to use data to cut costs, improve energy
efficiency and curb pollution. By now, 1,200 Exelon employees have
participated in the program for general knowledge, while more than 100
have gone on to acquire new technical skills.

Jeffrey Swiatek, 41, altered his career path after years as a
maintenance engineer at Exelon. ``The engineering paradigm has
shifted,'' said Mr. Swiatek, who had been trained as a mechanical
engineer. ``We have all this data now.''

Image

The Coursera headquarters in Mountain View, Calif. Experts say workers
can upgrade their skills with online training at learning networks like
Coursera and Udacity.Credit...Jessica Chou for The New York Times

Over several months, Mr. Swiatek took seven courses on Coursera,
learning the coding language Python, as well as the basics of machine
learning and data visualization. He got a half-day off each week, but
still had his regular job. Mostly, he did the coursework on nights and
weekends.

Mr. Swiatek has used his new skills to write software that can help
predict when equipment needs maintenance --- technology that is
projected to save the company \$1 million over eight years. In December,
Mr. Swiatek became a principal quantitative engineer, a promotion with a
raise, broader responsibilities and brighter career prospects.

Exelon calls its approach to training ``co-investing'' --- the company
contributes, but the employee acquires new skills largely on his or her
own time. If Mr. Swiatek departs for another company in a few years, the
investment still probably pays off for Exelon.

But the mobile labor force is one of the central issues in corporate
training: The company that invests in a worker's training may not reap
the benefit. To offset that risk, one proposal is to offer tax credits
to companies for money spent to upgrade the skills of their workers.

``There are various ways to do it, but ultimately you are going to need
public investment,'' said Erik Brynjolfsson, an economist and director
of the Digital Economy Lab at Stanford University. ``Training is not
something that can be just left to the private sector.''

Advertisement

\protect\hyperlink{after-bottom}{Continue reading the main story}

\hypertarget{site-index}{%
\subsection{Site Index}\label{site-index}}

\hypertarget{site-information-navigation}{%
\subsection{Site Information
Navigation}\label{site-information-navigation}}

\begin{itemize}
\tightlist
\item
  \href{https://help.nytimes.com/hc/en-us/articles/115014792127-Copyright-notice}{©~2020~The
  New York Times Company}
\end{itemize}

\begin{itemize}
\tightlist
\item
  \href{https://www.nytco.com/}{NYTCo}
\item
  \href{https://help.nytimes.com/hc/en-us/articles/115015385887-Contact-Us}{Contact
  Us}
\item
  \href{https://www.nytco.com/careers/}{Work with us}
\item
  \href{https://nytmediakit.com/}{Advertise}
\item
  \href{http://www.tbrandstudio.com/}{T Brand Studio}
\item
  \href{https://www.nytimes.com/privacy/cookie-policy\#how-do-i-manage-trackers}{Your
  Ad Choices}
\item
  \href{https://www.nytimes.com/privacy}{Privacy}
\item
  \href{https://help.nytimes.com/hc/en-us/articles/115014893428-Terms-of-service}{Terms
  of Service}
\item
  \href{https://help.nytimes.com/hc/en-us/articles/115014893968-Terms-of-sale}{Terms
  of Sale}
\item
  \href{https://spiderbites.nytimes.com}{Site Map}
\item
  \href{https://help.nytimes.com/hc/en-us}{Help}
\item
  \href{https://www.nytimes.com/subscription?campaignId=37WXW}{Subscriptions}
\end{itemize}
