Sections

SEARCH

\protect\hyperlink{site-content}{Skip to
content}\protect\hyperlink{site-index}{Skip to site index}

\href{https://www.nytimes.com/section/business}{Business}

\href{https://myaccount.nytimes.com/auth/login?response_type=cookie\&client_id=vi}{}

\href{https://www.nytimes.com/section/todayspaper}{Today's Paper}

\href{/section/business}{Business}\textbar{}`I Can't Keep Doing This:'
Small-Business Owners Are Giving Up

\url{https://nyti.ms/3j8LgNc}

\begin{itemize}
\item
\item
\item
\item
\item
\item
\end{itemize}

\href{https://www.nytimes.com/news-event/coronavirus?action=click\&pgtype=Article\&state=default\&region=TOP_BANNER\&context=storylines_menu}{The
Coronavirus Outbreak}

\begin{itemize}
\tightlist
\item
  live\href{https://www.nytimes.com/2020/08/03/world/coronavirus-covid-19.html?action=click\&pgtype=Article\&state=default\&region=TOP_BANNER\&context=storylines_menu}{Latest
  Updates}
\item
  \href{https://www.nytimes.com/interactive/2020/us/coronavirus-us-cases.html?action=click\&pgtype=Article\&state=default\&region=TOP_BANNER\&context=storylines_menu}{Maps
  and Cases}
\item
  \href{https://www.nytimes.com/interactive/2020/science/coronavirus-vaccine-tracker.html?action=click\&pgtype=Article\&state=default\&region=TOP_BANNER\&context=storylines_menu}{Vaccine
  Tracker}
\item
  \href{https://www.nytimes.com/2020/08/02/us/covid-college-reopening.html?action=click\&pgtype=Article\&state=default\&region=TOP_BANNER\&context=storylines_menu}{College
  Reopening}
\item
  \href{https://www.nytimes.com/live/2020/08/03/business/stock-market-today-coronavirus?action=click\&pgtype=Article\&state=default\&region=TOP_BANNER\&context=storylines_menu}{Economy}
\end{itemize}

Advertisement

\protect\hyperlink{after-top}{Continue reading the main story}

Supported by

\protect\hyperlink{after-sponsor}{Continue reading the main story}

\hypertarget{i-cant-keep-doing-this-small-business-owners-are-giving-up}{%
\section{`I Can't Keep Doing This:' Small-Business Owners Are Giving
Up}\label{i-cant-keep-doing-this-small-business-owners-are-giving-up}}

More owners are permanently shutting their doors after new lockdown
orders, realizing that there may be no end in sight to the crisis.

\includegraphics{https://static01.nyt.com/images/2020/07/09/business/00virus-smallbizfate1/merlin_174338130_6ef254c4-fb0e-47b5-8754-c36c5a627958-articleLarge.jpg?quality=75\&auto=webp\&disable=upscale}

\href{https://www.nytimes.com/by/emily-flitter}{\includegraphics{https://static01.nyt.com/images/2019/06/19/reader-center/author-emily-flitter/author-emily-flitter-thumbLarge.png}}

By \href{https://www.nytimes.com/by/emily-flitter}{Emily Flitter}

\begin{itemize}
\item
  July 13, 2020
\item
  \begin{itemize}
  \item
  \item
  \item
  \item
  \item
  \item
  \end{itemize}
\end{itemize}

On the last Friday of June, after Gov. Greg Abbott of Texas said that
bars across the state would have to shut down a second time because
\href{https://www.nytimes.com/news-event/coronavirus}{coronavirus cases}
were skyrocketing, Mick Larkin decided he had had enough.

No matter that Mr. Larkin, an owner of a karaoke club in Wichita Falls,
Texas, had just paid \$1,000 for perishable goods and protective
equipment in anticipation of the weekend rush. No matter that the frozen
margarita machine was full, that 175 plastic syringes with booze-infused
Jell-O were in place, or that there were masks for staff members and
hand sanitizer for guests.

That day, June 26, Mr. Larkin and his partner dumped what they had just
bought into the trash and decided to close their club, Krank It Karaoke,
for good.

``We did everything we were supposed to do,'' Mr. Larkin said. ``When he
shut us down again, and after I put out all that money to meet their
rules, I just said, `I can't keep doing this.'''

It was harrowing enough for small businesses --- the bars, dental care
practices, small law firms, day care centers and other storefronts that
dot the streets and corners of every American town and city --- to have
to shut down after state officials imposed lockdowns in March to contain
the pandemic.

But the
\href{https://www.nytimes.com/2020/06/26/us/coronavirus-florida-texas-bars-closing.html}{resurgence
of the virus}, especially in states such as Texas, Florida and
California that had begun to reopen, has introduced a far darker reality
for many small businesses: Their temporary closures might become
permanent.

Nearly 66,000 businesses have folded since March 1, according to data
from Yelp, which provides a platform for local businesses to advertise
their services and has been tracking announcements of closings posted on
its site. From June 15 to June 29, the most recent period for which data
is available, businesses were closing permanently at a higher rate than
in the previous three months, Yelp found. During the same period,
permanent closures increased by 3 percent overall, accounting for
roughly 14 percent of total closures since March.

\hypertarget{latest-updates-economy}{%
\section{\texorpdfstring{\href{https://www.nytimes.com/live/2020/08/03/business/stock-market-today-coronavirus?action=click\&pgtype=Article\&state=default\&region=MAIN_CONTENT_1\&context=storylines_live_updates}{Latest
Updates:
Economy}}{Latest Updates: Economy}}\label{latest-updates-economy}}

\href{https://www.nytimes.com/live/2020/08/03/business/stock-market-today-coronavirus?action=click\&pgtype=Article\&state=default\&region=MAIN_CONTENT_1\&context=storylines_live_updates\#the-chicago-fed-president-says-its-up-to-congress-to-save-the-economy}{10h
ago}

\href{https://www.nytimes.com/live/2020/08/03/business/stock-market-today-coronavirus?action=click\&pgtype=Article\&state=default\&region=MAIN_CONTENT_1\&context=storylines_live_updates\#the-chicago-fed-president-says-its-up-to-congress-to-save-the-economy}{The
Chicago Fed president says it's up to Congress to save the economy.}

\href{https://www.nytimes.com/live/2020/08/03/business/stock-market-today-coronavirus?action=click\&pgtype=Article\&state=default\&region=MAIN_CONTENT_1\&context=storylines_live_updates\#faa-says-boeing-has-effectively-mitigated-defects-in-the-737-max}{11h
ago}

\href{https://www.nytimes.com/live/2020/08/03/business/stock-market-today-coronavirus?action=click\&pgtype=Article\&state=default\&region=MAIN_CONTENT_1\&context=storylines_live_updates\#faa-says-boeing-has-effectively-mitigated-defects-in-the-737-max}{F.A.A.
says Boeing has `effectively mitigated' defects in the 737 Max.}

\href{https://www.nytimes.com/live/2020/08/03/business/stock-market-today-coronavirus?action=click\&pgtype=Article\&state=default\&region=MAIN_CONTENT_1\&context=storylines_live_updates\#small-businesses-got-emergency-loans-but-not-what-they-expected}{13h
ago}

\href{https://www.nytimes.com/live/2020/08/03/business/stock-market-today-coronavirus?action=click\&pgtype=Article\&state=default\&region=MAIN_CONTENT_1\&context=storylines_live_updates\#small-businesses-got-emergency-loans-but-not-what-they-expected}{Small
businesses got emergency loans, but not what they expected.}

\href{https://www.nytimes.com/live/2020/08/03/business/stock-market-today-coronavirus?action=click\&pgtype=Article\&state=default\&region=MAIN_CONTENT_1\&context=storylines_live_updates}{See
more updates}

More live coverage:
\href{https://www.nytimes.com/2020/08/03/world/coronavirus-covid-19.html?action=click\&pgtype=Article\&state=default\&region=MAIN_CONTENT_1\&context=storylines_live_updates}{Global}

Researchers at Harvard believe the rates of business closures are
\href{https://www.nber.org/papers/w26989.pdf}{likely to be even higher}.
They estimated that nearly 110,000 small businesses across the country
had decided to shut down permanently between early March and early May,
based on data collected in weekly surveys by Alignable, a social media
network for small-business owners.

Christopher Stanton, an associate professor at Harvard Business School
who was one of the researchers, said it was difficult to accurately
gauge how many small businesses were closing because, once they shut
their doors for good, the owners were hard to reach. He added that it
could take up to a year before government officials knew the true toll
the pandemic was taking on small businesses.

At the moment,
\href{https://www.nytimes.com/2020/07/13/world/coronavirus-updates.html?action=click\&module=Top\%20Stories\&pgtype=Homepage\#link-609d3a0e}{39
states continue to record growing numbers of new cases daily}.

It is not clear how many of the businesses Yelp is tracking count as
``small'' --- defined by the Small Business Administration as those with
500 or fewer employees. But the company found that, among the tracked
businesses --- which include restaurants, retailers and other
independent, consumer-facing operations --- retail businesses, led by
beauty supply stores, have been closing at the highest rate since the
pandemic began. Restaurants are the next hardest-hit group.

\includegraphics{https://static01.nyt.com/images/2020/07/09/business/00virus-smallbizfate3/merlin_174392160_a3ce4f20-5931-4d81-ab36-52be61668649-articleLarge.jpg?quality=75\&auto=webp\&disable=upscale}

Small businesses account for 44 percent of all U.S. economic activity,
according to the S.B.A., and closures on such an immense scale could
devastate the country's economic growth. If they were grouped together,
small businesses would be among the country's biggest employers, said
Satyam Khanna, a resident fellow at the Institute for Corporate
Governance and Finance at New York University School of Law who has
\href{https://www.nytimes.com/2020/03/24/opinion/coronavirus-small-businesses.html}{written
about the effects of the pandemic on small businesses}.

So when small businesses close en masse, an entire sector of the economy
suffers, Mr. Khanna said. There is lower cash flow, higher debt and more
unemployment. ``That leads to a big drag on the eventual recovery,'' he
said. ``Because they are such an important source of jobs, losing them
the way we are losing them now is going to make things far worse than
they otherwise need to be.''

Because small businesses depend heavily on foot traffic and operate on
thin margins, they are especially vulnerable to the ripple effects of a
widespread shutdown.

For nearly two decades, Rich Tokheim and his wife sold sports
memorabilia --- hats, T-shirts, coffee mugs and other trinkets --- to
fans in Omaha at their store, The Dugout. Since 2011, The Dugout has
occupied prime real estate across the street from the city's 24,000-seat
baseball stadium, which usually hosts the College World Series each
spring.

The 2020 World Series was canceled in March. In the weeks that came
after, other
\href{https://www.nytimes.com/interactive/2020/06/19/sports/100-days-without-sports.html}{sporting
events were scrapped} --- starting with college sports and extending to
professional leagues that have struggled to relaunch their activities.

Mr. Tokheim, 58, watched his business fall off with growing unease, but
it was only after a friendly chat with a retired college athletic
director in May that the gravity of his situation hit home. He was
already worried about the state of the virus in Nebraska, and whether
there was enough tracking. Then the athletic director predicted that if
college football was canceled for the year, it would be the end of
Division I sports as a whole.

``That really put me in overdrive,'' Mr. Tokheim said. He negotiated an
early exit on his store lease and announced a clearance sale at the
store. The Dugout closed for good on June 30.

The government's Paycheck Protection Program, rolled out in April and
administered by the S.B.A., earmarked \$660 billion of aid for small
businesses, but stipulated that a loan would be forgiven only if most of
it was used to pay employee wages for eight weeks. The rules were later
relaxed, but in a sign of how many small-business owners did not feel
confident that they would be on steady ground by the time repayment was
due, roughly
\href{https://www.nytimes.com/2020/06/30/business/paycheck-protection-program-coronavirus.html}{\$130
billion of aid money} remained untapped when the program ended in June.

Even for those who took a P.P.P. loan, survival is no guarantee. Nick
Muscari, a 38-year-old restaurateur in Lubbock, Texas, received one. His
restaurant, Nick's Sports Grill and Lounge, had been the culmination of
Mr. Muscari's life's work --- his years of toil as a waiter, pizza cook
and manager at restaurants and bars beginning in his teenage years.
Three years ago, he bought out the two partners who helped him start the
restaurant in 2010. He considered it a crowning achievement, but to do
so, he had to borrow money. He still owes a bank \$80,000.

Image

Mr. Muscari still owes a bank \$80,000 on his now-shuttered
restaurant.Credit...Dylan Cole for The New York Times

Image

Nick's Sports Grill and Lounge is now up for rent.Credit...Dylan Cole
for The New York Times

Image

``We never had anybody catch the virus in our establishment,'' Mr.
Muscari said.Credit...Dylan Cole for The New York Times

Mr. Muscari tried to ride out the spring lockdown that temporarily
shuttered his restaurant with the help of the P.P.P. money. But when the
state's second closure order took effect on June 26, he decided to close
for good.

``It had been in the back of our minds, just like, you know, if this
happens again, can we make it?'' Mr. Muscari said. ``We were following
all the rules and people were spread out. We never had anybody catch the
virus in our establishment."

Mr. Muscari, with the business closed and its 30 employees jobless, has
nothing left but his house and his car. He also expects his landlord to
try to sue him for the eight years' worth of rent he is contracted to
pay on his defunct restaurant's space.

Many small businesses are also finding it onerous keep up with
constantly changing local guidelines, while others are deciding that no
matter what their local officials say, it just is not safe to keep
going. Gabriel Gordon, the owner of a tiny but popular barbecue
restaurant in Seal Beach, Calif., decided to close permanently after
studying the restaurant's layout. He had determined that the kitchen
would never be safe for multiple staff members to occupy at once while
the virus was still active in the area.

``It's essentially two hallways that are 11 feet wide,'' Mr. Gordon
said, describing the shape of the restaurant, Beachwood BBQ. ``There are
food trucks that are larger than my kitchen.''

Whatever the specific reasons may be for each closure, Justin Norman,
Yelp's vice president of data science, said that the federal government
should offer small businesses more help. Mr. Norman said Yelp was
concerned about the effects of small-business closures, especially those
owned by people of color, on society. Yelp, however, also has a
financial interest in maintaining a robust small-business environment,
because it relies heavily on advertising by businesses on its platform.

``The time is right now to inject more capital or we may lose them
forever,'' Mr. Norman said. ``It's going to make our economies worse,
it's going to make our communities worse.''

Advertisement

\protect\hyperlink{after-bottom}{Continue reading the main story}

\hypertarget{site-index}{%
\subsection{Site Index}\label{site-index}}

\hypertarget{site-information-navigation}{%
\subsection{Site Information
Navigation}\label{site-information-navigation}}

\begin{itemize}
\tightlist
\item
  \href{https://help.nytimes.com/hc/en-us/articles/115014792127-Copyright-notice}{©~2020~The
  New York Times Company}
\end{itemize}

\begin{itemize}
\tightlist
\item
  \href{https://www.nytco.com/}{NYTCo}
\item
  \href{https://help.nytimes.com/hc/en-us/articles/115015385887-Contact-Us}{Contact
  Us}
\item
  \href{https://www.nytco.com/careers/}{Work with us}
\item
  \href{https://nytmediakit.com/}{Advertise}
\item
  \href{http://www.tbrandstudio.com/}{T Brand Studio}
\item
  \href{https://www.nytimes.com/privacy/cookie-policy\#how-do-i-manage-trackers}{Your
  Ad Choices}
\item
  \href{https://www.nytimes.com/privacy}{Privacy}
\item
  \href{https://help.nytimes.com/hc/en-us/articles/115014893428-Terms-of-service}{Terms
  of Service}
\item
  \href{https://help.nytimes.com/hc/en-us/articles/115014893968-Terms-of-sale}{Terms
  of Sale}
\item
  \href{https://spiderbites.nytimes.com}{Site Map}
\item
  \href{https://help.nytimes.com/hc/en-us}{Help}
\item
  \href{https://www.nytimes.com/subscription?campaignId=37WXW}{Subscriptions}
\end{itemize}
