Sections

SEARCH

\protect\hyperlink{site-content}{Skip to
content}\protect\hyperlink{site-index}{Skip to site index}

\href{https://myaccount.nytimes.com/auth/login?response_type=cookie\&client_id=vi}{}

\href{https://www.nytimes.com/section/todayspaper}{Today's Paper}

\href{/section/opinion}{Opinion}\textbar{}America Drank Away Its
Children's Future

\href{https://nyti.ms/3epXc9y}{https://nyti.ms/3epXc9y}

\begin{itemize}
\item
\item
\item
\item
\item
\item
\end{itemize}

\href{https://www.nytimes.com/news-event/coronavirus?action=click\&pgtype=Article\&state=default\&region=TOP_BANNER\&context=storylines_menu}{The
Coronavirus Outbreak}

\begin{itemize}
\tightlist
\item
  live\href{https://www.nytimes.com/2020/08/08/world/coronavirus-updates.html?action=click\&pgtype=Article\&state=default\&region=TOP_BANNER\&context=storylines_menu}{Latest
  Updates}
\item
  \href{https://www.nytimes.com/interactive/2020/us/coronavirus-us-cases.html?action=click\&pgtype=Article\&state=default\&region=TOP_BANNER\&context=storylines_menu}{Maps
  and Cases}
\item
  \href{https://www.nytimes.com/interactive/2020/science/coronavirus-vaccine-tracker.html?action=click\&pgtype=Article\&state=default\&region=TOP_BANNER\&context=storylines_menu}{Vaccine
  Tracker}
\item
  \href{https://www.nytimes.com/interactive/2020/world/coronavirus-tips-advice.html?action=click\&pgtype=Article\&state=default\&region=TOP_BANNER\&context=storylines_menu}{F.A.Q.}
\item
  \href{https://www.nytimes.com/live/2020/08/07/business/stock-market-today-coronavirus?action=click\&pgtype=Article\&state=default\&region=TOP_BANNER\&context=storylines_menu}{Markets
  \& Economy}
\end{itemize}

Advertisement

\protect\hyperlink{after-top}{Continue reading the main story}

\href{/section/opinion}{Opinion}

Supported by

\protect\hyperlink{after-sponsor}{Continue reading the main story}

\hypertarget{america-drank-away-its-childrens-future}{%
\section{America Drank Away Its Children's
Future}\label{america-drank-away-its-childrens-future}}

As the school year looms, the pandemic is still raging.

\href{https://www.nytimes.com/by/paul-krugman}{\includegraphics{https://static01.nyt.com/images/2018/04/02/opinion/paul-krugman/paul-krugman-thumbLarge.png}}

By \href{https://www.nytimes.com/by/paul-krugman}{Paul Krugman}

Opinion Columnist

\begin{itemize}
\item
  July 13, 2020
\item
  \begin{itemize}
  \item
  \item
  \item
  \item
  \item
  \item
  \end{itemize}
\end{itemize}

\includegraphics{https://static01.nyt.com/images/2020/07/13/opinion/13krugmanWeb/merlin_173962245_f66b1a04-8f75-426f-a0b2-86f3459ff43a-articleLarge.jpg?quality=75\&auto=webp\&disable=upscale}

A brief history of the past four months in America:

Experts: Don't rush to reopen, this isn't over.

Donald Trump: LIBERATE!

Covid-19: Wheee!

Trump officials: Here's our
\href{https://www.nytimes.com/2020/07/12/us/politics/fauci-trump-coronavirus.html}{opposition
research} on Anthony Fauci.

And we're now faced with an agonizing choice: Do we reopen schools,
creating risks of a further viral explosion, or do we keep children
home, with severe negative effects on their learning?

None of this had to happen. Other countries stuck with their lockdowns
long enough to reduce infections to rates much lower than those
prevailing here; Covid-19 death rates per capita in the
\href{https://twitter.com/paulkrugman/status/1282656106762952705}{European
Union} are only a 10th those in the United States --- and falling ---
while ours are rising fast. As a result, they're in a position to reopen
schools fairly safely.

And the experience of the Northeast, the first major epicenter of the
U.S. pandemic, shows that we could have achieved something similar here.
Death rates are
\href{https://covidtracking.com/data/charts/regional-deaths}{way down},
although still higher than in Europe; on Saturday, for the first time
since March, New York City reported
\href{https://www.amny.com/coronavirus/july-11-was-nycs-first-day-without-a-covid-19-death-in-four-months/}{zero}
Covid-19 deaths.

Would a longer lockdown have been economically sustainable? Yes.

It's true that strong social distancing requirements led to high
unemployment and hurt many businesses. But even America, with its
ramshackle social safety net, was able to provide enough disaster relief
--- don't call it stimulus! --- to protect most of its citizens from
severe hardship.

Thanks largely to expanded unemployment benefits, poverty didn't soar
during the lockdown. By some measures it may even have
\href{https://www.nytimes.com/2020/06/21/us/politics/coronavirus-poverty.html}{gone
down}.

True, there were holes in that safety net, and many people did suffer.
But we could have patched those holes. Yes, emergency relief costs a lot
of money, but we can afford it: The federal government has been
borrowing huge sums, but interest rates have remained near historical
lows.

Put it this way: At its most severe, the lockdown seems to have reduced
G.D.P. by a
\href{https://www.newyorkfed.org/research/policy/weekly-economic-index}{little
over 10 percent}. During World War II, America spent more than 30
percent of G.D.P. on
\href{https://eh.net/encyclopedia/the-american-economy-during-world-war-ii/}{defense},
for more than three years. Why couldn't we absorb a much smaller cost
for a few months?

So doing what was necessary to bring the coronavirus under control would
have been annoying, but entirely feasible.

\hypertarget{the-coronavirus-outbreak}{%
\subsubsection{The Coronavirus
Outbreak}\label{the-coronavirus-outbreak}}

\hypertarget{back-to-school}{%
\paragraph{Back to School}\label{back-to-school}}

Updated Aug. 8, 2020

The latest highlights as the first students return to U.S. schools.

\begin{itemize}
\item
  \begin{itemize}
  \tightlist
  \item
    Health experts say New York State schools are
    \href{https://www.nytimes.com/2020/08/07/health/coronavirus-ny-schools-reopen.html?action=click\&pgtype=Article\&state=default\&region=MAIN_CONTENT_2\&context=storylines_keepup}{in
    a good position to reopen}, and Gov. Andrew M. Cuomo has
    \href{https://www.nytimes.com/2020/08/07/nyregion/cuomo-schools-reopening.html?action=click\&pgtype=Article\&state=default\&region=MAIN_CONTENT_2\&context=storylines_keepup}{cleared
    the way}.
  \item
    Many schools spent the summer focused on reopening classrooms. What
    if they had
    \href{https://www.nytimes.com/2020/08/07/us/remote-learning-fall-2020.html?action=click\&pgtype=Article\&state=default\&region=MAIN_CONTENT_2\&context=storylines_keepup}{focused
    on improving remote learning} instead?
  \item
    A mother in Germany describes how her family
    \href{https://www.nytimes.com/2020/08/07/parenting/germany-schools-reopening-children.html?action=click\&pgtype=Article\&state=default\&region=MAIN_CONTENT_2\&context=storylines_keepup}{coped
    with the anxiety and uncertainty} of going back to school there.
  \item
    A high school freshman tested positive after two days in class. A
    yearbook editor worries about access to sporting events. We spoke to
    students about
    \href{https://www.nytimes.com/2020/08/06/us/coronavirus-students.html?action=click\&pgtype=Article\&state=default\&region=MAIN_CONTENT_2\&context=storylines_keepup}{what
    school is like in the age of Covid-19.}
  \end{itemize}
\end{itemize}

But that was the road not taken. Instead, many states not only rushed to
reopen, they reopened stupidly. Instead of being treated as a cheap,
effective way to fight contagion, face masks became a front in the
culture war. Activities that posed an obvious risk of feeding the
pandemic went unchecked: Large gatherings were permitted, bars reopened.

And the cost of those parties and open bars extends beyond the thousands
of Americans who will be killed or suffer permanent health damage as a
result of Covid-19's resurgence. The botched reopening has also
endangered something that, unlike drinking in groups, can't be suspended
without doing long-run damage: in-person education.

Some activities hold up fairly well when moved online. I suspect that
there will be a lot fewer people flying cross-country to stare at
PowerPoints than there were pre-Covid, even once we finally beat this
virus.

Education isn't one of those activities. We now have overwhelming
confirmation of something we already suspected: For many, perhaps most
students there is no substitute for actually being in a classroom.

But rooms full of students are potential Petri dishes, even if the young
are less likely to die from Covid-19 than the old. Other countries have
managed to reopen schools
\href{https://www.nytimes.com/2020/07/11/health/coronavirus-schools-reopen.html}{relatively
safely} --- but they did so with much lower infection rates than
currently prevail in America, and with adequate testing, which we still
\href{https://abcnews.go.com/Health/13-states-now-report-coronavirus-testing-issues-echo/story?id=71698974}{don't
have} in many hot spots.

So we're now facing a terrible, unnecessary dilemma. If we reopen
in-person education, we risk feeding an out-of-control pandemic. If we
don't, we impair the development of millions of American students,
inflicting long-term damage on their lives and careers.

And the reason we're in this position is that states, cheered on by the
Trump administration, rushed to allow large parties and reopen bars. In
a real sense America drank away its children's future.

Now what? At this point there are probably as many
\href{https://twitter.com/youyanggu/status/1282028088658976768}{infected
Americans} as there were in March. So what we should be doing is
admitting that we blew it, and doing a severe lockdown all over again
--- and this time listening to the experts before reopening.
Unfortunately, it's now too late to avoid disrupting education, but the
sooner we deal with this the sooner we can get our society back on
track.

But we don't have the kind of leaders we need. Instead, we have the
likes of Donald Trump and Ron DeSantis, Florida's governor, politicians
who refuse to listen to experts and never admit having been wrong.

So while there have been a few grudging policy adjustments, the main
response we're seeing to colossal policy failure is a hysterical attempt
to shift the blame. Some officials are trying to besmirch Dr. Fauci's
reputation; others are diving into unhinged
\href{https://twitter.com/joshtpm/status/1282343525103939586}{conspiracy
theories}.

As a result, the outlook is grim. This pandemic is going to get worse
before it gets better, and the nation will suffer permanent damage.

\emph{The Times is committed to publishing}
\href{https://www.nytimes.com/2019/01/31/opinion/letters/letters-to-editor-new-york-times-women.html}{\emph{a
diversity of letters}} \emph{to the editor. We'd like to hear what you
think about this or any of our articles. Here are some}
\href{https://help.nytimes.com/hc/en-us/articles/115014925288-How-to-submit-a-letter-to-the-editor}{\emph{tips}}\emph{.
And here's our email:}
\href{mailto:letters@nytimes.com}{\emph{letters@nytimes.com}}\emph{.}

\emph{Follow The New York Times Opinion section on}
\href{https://www.facebook.com/nytopinion}{\emph{Facebook}}\emph{,}
\href{http://twitter.com/NYTOpinion}{\emph{Twitter (@NYTopinion)}}
\emph{and}
\href{https://www.instagram.com/nytopinion/}{\emph{Instagram}}\emph{.}

Advertisement

\protect\hyperlink{after-bottom}{Continue reading the main story}

\hypertarget{site-index}{%
\subsection{Site Index}\label{site-index}}

\hypertarget{site-information-navigation}{%
\subsection{Site Information
Navigation}\label{site-information-navigation}}

\begin{itemize}
\tightlist
\item
  \href{https://help.nytimes.com/hc/en-us/articles/115014792127-Copyright-notice}{©~2020~The
  New York Times Company}
\end{itemize}

\begin{itemize}
\tightlist
\item
  \href{https://www.nytco.com/}{NYTCo}
\item
  \href{https://help.nytimes.com/hc/en-us/articles/115015385887-Contact-Us}{Contact
  Us}
\item
  \href{https://www.nytco.com/careers/}{Work with us}
\item
  \href{https://nytmediakit.com/}{Advertise}
\item
  \href{http://www.tbrandstudio.com/}{T Brand Studio}
\item
  \href{https://www.nytimes.com/privacy/cookie-policy\#how-do-i-manage-trackers}{Your
  Ad Choices}
\item
  \href{https://www.nytimes.com/privacy}{Privacy}
\item
  \href{https://help.nytimes.com/hc/en-us/articles/115014893428-Terms-of-service}{Terms
  of Service}
\item
  \href{https://help.nytimes.com/hc/en-us/articles/115014893968-Terms-of-sale}{Terms
  of Sale}
\item
  \href{https://spiderbites.nytimes.com}{Site Map}
\item
  \href{https://help.nytimes.com/hc/en-us}{Help}
\item
  \href{https://www.nytimes.com/subscription?campaignId=37WXW}{Subscriptions}
\end{itemize}
