Sections

SEARCH

\protect\hyperlink{site-content}{Skip to
content}\protect\hyperlink{site-index}{Skip to site index}

\href{/section/travel}{Travel}\textbar{}Behold Vermont, From Above

\url{https://nyti.ms/3frpLor}

\begin{itemize}
\item
\item
\item
\item
\item
\item
\end{itemize}

\href{https://www.nytimes.com/spotlight/at-home?action=click\&pgtype=Article\&state=default\&region=TOP_BANNER\&context=at_home_menu}{At
Home}

\begin{itemize}
\tightlist
\item
  \href{https://www.nytimes.com/2020/07/28/books/time-for-a-literary-road-trip.html?action=click\&pgtype=Article\&state=default\&region=TOP_BANNER\&context=at_home_menu}{Take:
  A Literary Road Trip}
\item
  \href{https://www.nytimes.com/2020/07/29/magazine/bored-with-your-home-cooking-some-smoky-eggplant-will-fix-that.html?action=click\&pgtype=Article\&state=default\&region=TOP_BANNER\&context=at_home_menu}{Cook:
  Smoky Eggplant}
\item
  \href{https://www.nytimes.com/2020/07/27/travel/moose-michigan-isle-royale.html?action=click\&pgtype=Article\&state=default\&region=TOP_BANNER\&context=at_home_menu}{Look
  Out: For Moose}
\item
  \href{https://www.nytimes.com/interactive/2020/at-home/even-more-reporters-editors-diaries-lists-recommendations.html?action=click\&pgtype=Article\&state=default\&region=TOP_BANNER\&context=at_home_menu}{Explore:
  Reporters' Obsessions}
\end{itemize}

\includegraphics{https://static01.nyt.com/images/2020/07/15/travel/15travel-vermont-4/merlin_173430816_d9ed2a34-18e3-4a6b-9852-303952ed84da-articleLarge.jpg?quality=75\&auto=webp\&disable=upscale}

The World Through a Lens

\hypertarget{behold-vermont-from-above}{%
\section{Behold Vermont, From Above}\label{behold-vermont-from-above}}

These aerial images, stunning in their abstraction, evoke a powerful
sense of transcendence and tranquillity.

An early morning on the Sugar Hill Reservoir in Goshen, Vt.Credit...

Supported by

\protect\hyperlink{after-sponsor}{Continue reading the main story}

Photographs and Text by Caleb Kenna

\begin{itemize}
\item
  July 13, 2020
\item
  \begin{itemize}
  \item
  \item
  \item
  \item
  \item
  \item
  \end{itemize}
\end{itemize}

\href{https://cn.nytimes.com/travel/20200721/vermont-drone-photographs/}{阅读简体中文版}\href{https://cn.nytimes.com/travel/20200721/vermont-drone-photographs/zh-hant/}{閱讀繁體中文版}

\emph{At the onset of the coronavirus pandemic, with travel restrictions
in place worldwide, we launched a new series ---}
\href{https://www.nytimes.com/column/the-world-through-a-lens}{\emph{The
World Through a Lens}} \emph{--- in which photojournalists help
transport you, virtually, to some of our planet's most beautiful and
intriguing places. This week, Caleb Kenna shares a collection of drone
photographs from Vermont.}

\begin{center}\rule{0.5\linewidth}{\linethickness}\end{center}

Ever since I was young, I've loved gazing out the oval windows of
airplanes and daydreaming about the abstract geometric patterns below.

Planes transport us from place to place, from country to country, from
ground level to a bird's-eye view. From the air, familiar landscapes
take on conceptual qualities; we gain fresh perspectives by viewing
hidden patterns.

\includegraphics{https://static01.nyt.com/images/2020/07/15/travel/15travel-vermont-8/13travel-vermont-2-mobileMasterAt3x.jpg}\includegraphics{https://static01.nyt.com/images/2020/07/15/travel/15travel-vermont-6/13travel-vermont-3-mobileMasterAt3x.jpg}\includegraphics{https://static01.nyt.com/images/2020/07/15/travel/15travel-vermont-5/13travel-vermont-16-mobileMasterAt3x.jpg}

Morning light casts a shadow of a maple tree at the University of
Vermont's Morgan Horse Farm in Weybridge.

Fall foliage along Huff Pond Road in Sudbury.

Otter Creek, in Brandon, flows north from Green Mountain National Forest
to Lake Champlain.

I have worked as a freelance photographer for more than twenty years,
traveling Vermont's back roads, making portraits and capturing the
state's iconic landscapes.

Perspective --- alongside light, color and timing --- is a fundamental
building block of photography, and I'm always looking for new ways to
alter mine. Until a few years ago, I hired airplanes --- and hoped for
good weather and a helpful pilot ---~to climb skyward and create aerial
pictures. Nowadays I use a drone.

\includegraphics{https://static01.nyt.com/images/2020/07/13/travel/13travel-vermont-7/merlin_173430537_06919af6-32cf-40ae-8054-489802e8f586-articleLarge.jpg?quality=75\&auto=webp\&disable=upscale}

There are trade-offs, of course. Looking down at the ground virtually
--- through a remote-controlled lens --- isn't a substitute,
experientially, for actually taking to the skies. But it makes me less
reliant on others and is much more environmentally friendly. (It's also
a lot more convenient; I can set up and launch my drone, a DJI Mavic 2
Pro, in about five minutes.)

\includegraphics{https://static01.nyt.com/images/2020/07/13/travel/13travel-vermont-12/13travel-vermont-12-mobileMasterAt3x.jpg}\includegraphics{https://static01.nyt.com/images/2020/07/13/travel/13travel-vermont-9/13travel-vermont-9-mobileMasterAt3x.jpg}\includegraphics{https://static01.nyt.com/images/2020/07/15/travel/15travel-vermont-1/13travel-vermont-13-mobileMasterAt3x.jpg}

Long shadows at a wetland in Brandon.

A soccer field at Middlebury College.

In Shoreham, three crab apple trees provide a break in the springtime
pattern of an apple orchard in bloom.

Using a drone is a natural evolution for a still photographer. On dull
and cloudy days, I can rise above the world and create elevated
photographs full of vibrant colors. On days with great light, I can
capture the long shadows cast in farm fields by lone trees.

Image

Abstract tractor patterns in light snow in Weybridge.

I often look to Alfred Stieglitz's
\href{https://archive.artic.edu/stieglitz/equivalents/}{``Equivalents''
photographs} as a source of inspiration. The series of abstract cloud
studies shot in the 1920s and 30s transcends representations of the
physical world and offers a world of
\href{https://www.nytimes.com/1983/02/13/arts/photography-view-stieglitz-felt-the-pull-of-two-cultures.html}{abstraction
and metaphor}. I'm also influenced by the subsequent work of
\href{https://www.nytimes.com/1995/01/27/arts/photography-review-before-the-artist-became-a-mystic.html}{Minor
White}, a photographer who adopted and expanded on some of Stieglitz's
principles.

\includegraphics{https://static01.nyt.com/images/2020/07/13/travel/13travel-vermont-4/13travel-vermont-4-mobileMasterAt3x.jpg}\includegraphics{https://static01.nyt.com/images/2020/07/13/travel/13travel-vermont-8/13travel-vermont-8-mobileMasterAt3x.jpg}\includegraphics{https://static01.nyt.com/images/2020/07/13/travel/13travel-vermont-5/13travel-vermont-5-mobileMasterAt3x.jpg}

A kayaker on Lake Hortonia, in Sudbury.

Fresh snowfall on an apple orchard in Cornwall.

A pair of trees in a field in Weybridge.

Most of my drone photographs were made around my home in Vermont's
Champlain Valley. (The area is known as the land of milk and honey
because of its many farms and apiaries.) But sometimes I venture farther
afield.

Image

A woman exercises on the track at Middlebury College.

There is a soaring sense of excitement and discovery when ascending over
familiar landscapes. And while the terrain where I fly is often
well-known to me, I can rarely predict what kinds of compositions I'll
walk away with.

\includegraphics{https://static01.nyt.com/images/2020/07/13/travel/13travel-vermont-15/13travel-vermont-15-mobileMasterAt3x.jpg}\includegraphics{https://static01.nyt.com/images/2020/07/13/travel/13travel-vermont-11/13travel-vermont-11-mobileMasterAt3x.jpg}\includegraphics{https://static01.nyt.com/images/2020/07/15/travel/15travel-vermont-3/13travel-vermont-14-mobileMasterAt3x.jpg}

A maple tree in Orwell.

Tractor tracks on a field in Weybridge.

The Nulhegan River in the Silvio O. Conte National Fish and Wildlife
Refuge in Brunswick.

Nor do I always know what my subjects will be. Once, while driving
through the Mettawee Valley, a bucolic setting dotted with small towns
and dairy farms, I pulled off the road next to a corn field and launched
my drone ---~only to spot a beautiful old barn with a slate roof,
completely hidden from my view on the ground.

Image

An old barn --- hidden from view on the ground by the surrounding
cornfields --- in Rupert.

Finding ongoing sources of creative inspiration is a challenge for any
artist, and aerial photography has helped broaden the scope of my work.
More than anything, though, making drone photographs has become a daily
practice for me --- one that often feels like a form of visual
meditation.

\begin{center}\rule{0.5\linewidth}{\linethickness}\end{center}

\href{https://www.calebkenna.com/}{\emph{Caleb Kenna}} \emph{is a
photographer and F.A.A.-certified drone pilot based in Middlebury, Vt.
You can follow his work on}
\href{https://www.instagram.com/calebkenna/}{\emph{Instagram}}
\emph{and}
\href{https://www.facebook.com/CalebKennaPhotography/}{\emph{Facebook}}\emph{.}

\emph{\textbf{Follow New York Times Travel}} \emph{on}
\href{https://www.instagram.com/nytimestravel/}{\emph{Instagram}}\emph{,}
\href{https://twitter.com/nytimestravel}{\emph{Twitter}} \emph{and}
\href{https://www.facebook.com/nytimestravel/}{\emph{Facebook}}\emph{.
And}
\href{https://www.nytimes.com/newsletters/traveldispatch}{\emph{sign up
for our weekly Travel Dispatch newsletter}} \emph{to receive expert tips
on traveling smarter and inspiration for your next vacation.}

Advertisement

\protect\hyperlink{after-bottom}{Continue reading the main story}

\hypertarget{site-index}{%
\subsection{Site Index}\label{site-index}}

\hypertarget{site-information-navigation}{%
\subsection{Site Information
Navigation}\label{site-information-navigation}}

\begin{itemize}
\tightlist
\item
  \href{https://help.nytimes.com/hc/en-us/articles/115014792127-Copyright-notice}{©~2020~The
  New York Times Company}
\end{itemize}

\begin{itemize}
\tightlist
\item
  \href{https://www.nytco.com/}{NYTCo}
\item
  \href{https://help.nytimes.com/hc/en-us/articles/115015385887-Contact-Us}{Contact
  Us}
\item
  \href{https://www.nytco.com/careers/}{Work with us}
\item
  \href{https://nytmediakit.com/}{Advertise}
\item
  \href{http://www.tbrandstudio.com/}{T Brand Studio}
\item
  \href{https://www.nytimes.com/privacy/cookie-policy\#how-do-i-manage-trackers}{Your
  Ad Choices}
\item
  \href{https://www.nytimes.com/privacy}{Privacy}
\item
  \href{https://help.nytimes.com/hc/en-us/articles/115014893428-Terms-of-service}{Terms
  of Service}
\item
  \href{https://help.nytimes.com/hc/en-us/articles/115014893968-Terms-of-sale}{Terms
  of Sale}
\item
  \href{https://spiderbites.nytimes.com}{Site Map}
\item
  \href{https://help.nytimes.com/hc/en-us}{Help}
\item
  \href{https://www.nytimes.com/subscription?campaignId=37WXW}{Subscriptions}
\end{itemize}
