Sections

SEARCH

\protect\hyperlink{site-content}{Skip to
content}\protect\hyperlink{site-index}{Skip to site index}

\href{https://www.nytimes.com/section/politics}{Politics}

\href{https://myaccount.nytimes.com/auth/login?response_type=cookie\&client_id=vi}{}

\href{https://www.nytimes.com/section/todayspaper}{Today's Paper}

\href{/section/politics}{Politics}\textbar{}Supreme Court, 5-4, Lifts
Block on Federal Execution

\url{https://nyti.ms/32cBWSn}

\begin{itemize}
\item
\item
\item
\item
\item
\end{itemize}

Advertisement

\protect\hyperlink{after-top}{Continue reading the main story}

Supported by

\protect\hyperlink{after-sponsor}{Continue reading the main story}

\hypertarget{supreme-court-5-4-lifts-block-on-federal-execution}{%
\section{Supreme Court, 5-4, Lifts Block on Federal
Execution}\label{supreme-court-5-4-lifts-block-on-federal-execution}}

A district court in Washington, D.C., had put the execution of Daniel
Lewis Lee on hold hours before it was to be carried out. The Supreme
Court cleared the way in an unsigned ruling.

\includegraphics{https://static01.nyt.com/images/2020/07/13/us/politics/13dc-virus-execution/merlin_174331752_96f62bd8-c5d6-4b6c-9aa3-806ba441fa94-articleLarge.jpg?quality=75\&auto=webp\&disable=upscale}

By Hailey Fuchs

\begin{itemize}
\item
  Published July 13, 2020Updated July 16, 2020
\item
  \begin{itemize}
  \item
  \item
  \item
  \item
  \item
  \end{itemize}
\end{itemize}

WASHINGTON --- In a 5-4 middle-of-the-night decision, the Supreme Court
cleared the way for the Justice Department to carry out the first
\href{https://www.nytimes.com/2020/07/16/us/politics/wesley-ira-purkey-executed.html}{federal
execution} in more than 17 years, removing a federal judge's hold in a
decision released early on Tuesday.

The federal judge had delayed the execution hours earlier, saying on
Monday that questions about the constitutionality of the lethal
injection procedure the government planned to use had not been fully
litigated.

The Justice Department immediately appealed the ruling by
\href{https://www.dcd.uscourts.gov/content/district-judge-tanya-s-chutkan}{Judge
Tanya S. Chutkan} of the United States District Court in Washington,
D.C. Judge Chutkan
\href{https://ecf.dcd.uscourts.gov/cgi-bin/show_public_doc?2019mc0145-135}{issued
a preliminary injunction} against the planned execution of
\href{https://www.nytimes.com/2020/07/14/us/politics/daniel-lewis-lee-execution-crime.html}{Daniel
Lewis Lee}, citing the ``extreme pain and needless suffering'' that
could result from the lethal injection protocol the government planned
to use.

The Supreme Court's
\href{https://www.supremecourt.gov/opinions/19pdf/20a8_970e.pdf}{unsigned
5-to-4 ruling} said the single drug the government intends to use,
pentobarbital, has been used in over 100 executions ``without incident''
and had been upheld by the Supreme Court and appeals courts.

``The plaintiffs in this case have not made the showing required to
justify last-minute intervention by a federal court,'' the unsigned
order said, quoting from a
\href{https://www.supremecourt.gov/opinions/18pdf/17-8151_1qm2.pdf}{decision
last year}. ```Last-minute stays' like that issued this morning `should
be the extreme exception, not the norm.''

``It is our responsibility `to ensure that method-of-execution
challenges to lawfully issued sentences are resolved fairly and
expeditiously, so that `the question of capital punishment' can remain
with `the people and their representatives, not the courts, to
resolve,''' the order said. ``In keeping with that responsibility, we
vacate the district court's preliminary injunction so that the
plaintiffs' executions may proceed as planned.''

In dissent, Justice Stephen G. Breyer, joined by Justice Ruth Bader
Ginsburg, repeated their
\href{https://www.nytimes.com/2015/11/04/us/politics/death-penalty-opponents-split-over-taking-issue-to-supreme-court.html}{longstanding
doubts} about the constitutionality of the death penalty. ``The
resumption of federal executions promises to provide examples that
illustrate the difficulties of administering the death penalty
consistent with the Constitution,'' he wrote.

In a second dissent, Justice Sonia Sotomayor, joined by Justices
Ginsburg and Elena Kagan, said the court had acted with dangerous haste.

``Today's decision illustrates just how grave the consequences of such
accelerated decision making can be,'' Justice Sotomayor wrote. ``The
court forever deprives respondents of their ability to press a
constitutional challenge to their lethal injections, and prevents lower
courts from reviewing that challenge.''

The back-to-back decisions were the latest in a flurry of court rulings
surrounding the case of Mr. Lee, the first of three federal death row
prisoners scheduled to be executed this week. Mr. Lee was scheduled to
be executed at 4 p.m. on Monday at the federal penitentiary in Terre
Haute, Ind.

Mr. Lee, 47, a former white supremacist who has denounced his ties to
that movement, was set to be executed for his part in the 1996 killing
of a family of three. The Trump administration announced its intention
last summer to resume the federal death penalty after a nearly
two-decade hiatus and to employ a new procedure to carry it out --- a
single widely available drug, pentobarbital --- after several botched
executions by lethal injection renewed scrutiny of capital punishment.

But the government has been fighting off legal challenges to the
single-drug technique. Last month,
\href{https://www.nytimes.com/2020/06/29/us/supreme-court-executions.html}{the
Supreme Court let stand} an appeals court ruling that found the
government was in compliance with
\href{https://www.law.cornell.edu/uscode/text/18/3596}{the Federal Death
Penalty Act of 1994}, which requires executions to be carried out ``in
the manner prescribed by the law of the state in which the sentence is
imposed.''
\href{https://src.bna.com/MZD?_ga=2.258585482.1273884090.1575491003-907374773.1567693399}{Judge
Chutkan had found} the government in violation of the law.

In her ruling on Monday, Judge Chutkan wrote that lethal injection by
pentobarbital could expose the inmates to the risk of flash pulmonary
edema, or the rapid buildup of fluid in the lungs that resembles the
feeling of drowning or asphyxiation. Before the inmates could be put to
death, the judicial system must decide whether the protocol violates
their protections against cruel and unusual punishment, she ruled.

``The government has been trying to plow forward with these executions
despite many unanswered questions about the legality of its new
execution protocol,'' said Shawn Nolan, a lawyer for one of the men
scheduled to be executed this summer, in a statement.

The Justice Department has repeatedly invoked the justice owed to the
murder victims to underscore the need for a speedy execution for the
four men scheduled to die this summer.

Mr. Lee's trial judge, his prosecutor, and several members of the
victims' family pleaded with the government to commute his sentence to
life in prison without the possibility of parole.

Last week, family members of his victims sued the Justice Department,
arguing that traveling to the execution site would put them at risk of
contracting the coronavirus. A district court agreed and granted a
temporary delay in the execution. Late Sunday, the U.S. Court of Appeals
for the Seventh Circuit
\href{https://www.nytimes.com/2020/07/12/us/politics/execution-daniel-lewis-lee.html?searchResultPosition=2}{reversed
that decision} and cleared the way for the execution to proceed.

Mike Ives contributed reporting.

Advertisement

\protect\hyperlink{after-bottom}{Continue reading the main story}

\hypertarget{site-index}{%
\subsection{Site Index}\label{site-index}}

\hypertarget{site-information-navigation}{%
\subsection{Site Information
Navigation}\label{site-information-navigation}}

\begin{itemize}
\tightlist
\item
  \href{https://help.nytimes.com/hc/en-us/articles/115014792127-Copyright-notice}{©~2020~The
  New York Times Company}
\end{itemize}

\begin{itemize}
\tightlist
\item
  \href{https://www.nytco.com/}{NYTCo}
\item
  \href{https://help.nytimes.com/hc/en-us/articles/115015385887-Contact-Us}{Contact
  Us}
\item
  \href{https://www.nytco.com/careers/}{Work with us}
\item
  \href{https://nytmediakit.com/}{Advertise}
\item
  \href{http://www.tbrandstudio.com/}{T Brand Studio}
\item
  \href{https://www.nytimes.com/privacy/cookie-policy\#how-do-i-manage-trackers}{Your
  Ad Choices}
\item
  \href{https://www.nytimes.com/privacy}{Privacy}
\item
  \href{https://help.nytimes.com/hc/en-us/articles/115014893428-Terms-of-service}{Terms
  of Service}
\item
  \href{https://help.nytimes.com/hc/en-us/articles/115014893968-Terms-of-sale}{Terms
  of Sale}
\item
  \href{https://spiderbites.nytimes.com}{Site Map}
\item
  \href{https://help.nytimes.com/hc/en-us}{Help}
\item
  \href{https://www.nytimes.com/subscription?campaignId=37WXW}{Subscriptions}
\end{itemize}
