Sections

SEARCH

\protect\hyperlink{site-content}{Skip to
content}\protect\hyperlink{site-index}{Skip to site index}

\href{https://www.nytimes.com/section/style}{Style}

\href{https://myaccount.nytimes.com/auth/login?response_type=cookie\&client_id=vi}{}

\href{https://www.nytimes.com/section/todayspaper}{Today's Paper}

\href{/section/style}{Style}\textbar{}Disabled Do-It-Yourselfers Lead
Way to Technology Gains

\href{https://nyti.ms/38TIFlv}{https://nyti.ms/38TIFlv}

\begin{itemize}
\item
\item
\item
\item
\item
\item
\end{itemize}

Advertisement

\protect\hyperlink{after-top}{Continue reading the main story}

Supported by

\protect\hyperlink{after-sponsor}{Continue reading the main story}

\hypertarget{disabled-do-it-yourselfers-lead-way-to-technology-gains}{%
\section{Disabled Do-It-Yourselfers Lead Way to Technology
Gains}\label{disabled-do-it-yourselfers-lead-way-to-technology-gains}}

So long to overhyped innovations. Hello to tech that embeds
accessibility into everyday devices.

\includegraphics{https://static01.nyt.com/images/2020/07/14/multimedia/14ADA-TECHNOLOGY/14ADA-TECHNOLOGY-articleLarge.jpg?quality=75\&auto=webp\&disable=upscale}

By David M. Perry

\begin{itemize}
\item
  Published July 14, 2020Updated July 20, 2020
\item
  \begin{itemize}
  \item
  \item
  \item
  \item
  \item
  \item
  \end{itemize}
\end{itemize}

\hypertarget{listen-to-this-article}{%
\subsubsection{Listen to This Article}\label{listen-to-this-article}}

Computer-generated audio recording

Technology is changing the ways that disabled people interact with the
world; perhaps more important, it's also shifting how the world
interacts with disabled people.

As the 30th anniversary of the Americans With Disabilities Act
approaches on July 26, many leaders, designers and scholars in the
disability community say that they aren't excited by stair-climbing
wheelchairs, mechanical exoskeletons or brain-controlled prosthetics.
They are drawn to innovations that embed accessibility into everyday
technologies and the spaces that we all share. Also, they want people to
stop trying to solve problems that don't exist.

Mark Riccobono, who lost his sight to glaucoma as a child and is
\href{https://www.nfb.org/about-us/leadership/presidents-corner/mark-riccobono}{president
of the National Federation of the Blind}, says that blind people
generally love their white canes, a simple and effective piece of
technology. ``A couple times a year someone comes to us and says, `We
have this great new idea for how to replace the cane!' '' he said. ``We
try to be objective, but no. You're trying to solve a problem that's not
a problem.''

Disability technology can be so quotidian that nondisabled users don't
even notice. GPS and spell-check, so ubiquitous for so many people, are
technologies that assist me with dyslexia. Smartphones, where I find my
GPS, may be the most powerful accessibility devices in history,
especially now that voice control offers an alternative to touch screens
for Blind and low-vision users, or people without the manual dexterity
to operate them. (No interface is perfect, however. Some people might
actually want buttons over sleek screens. And affordability remains a
problem.)

As hubs for accessibility programming, though, smartphones drive down
costs. For example, Fred Downs, who lost his left arm when he stepped on
a land mine during the Vietnam War and is now an advocacy director for
Paralyzed Veterans of America, says that in 1980, screenreaders cost up
to \$50,000 a unit and could read one page at a time out loud. Now every
computer, phone and tablet can read nearly any screen. Smartphones
provide navigation, manage hearing aids, run speech apps and can even
drive a wheelchair.

Innovations build off these capabilities, so now, for example,
\href{https://www.accessexplorer.net/mapping/}{companies} are working on
mapping interior spaces to help people navigate them the same way
detailed exterior maps currently do. Those who are disabled
\href{https://www.nytimes.com/2020/05/27/at-home/work-from-home-history.html}{have
long struggled to win the right to work from home}; these days,
technologies like cloud computing and video conferencing are used
everywhere and widely accepted at least for office jobs, especially as
the coronavirus pandemic alters so many workplaces. Disabled employees
who do not wish or are not able to go to an office can now more easily
interact with their colleagues.

Disability-related technologies are not just growing through incremental
adjustments to existing products; transformative ones are on the
horizon. Rory Cooper is director of the
\href{https://www.herl.pitt.edu/}{Human Engineering Research
Laboratories}, sponsored by the University of Pittsburgh and the U.S.
Department of Veterans Affairs. He was paralyzed because of a
spinal-cord injury in 1980 and has used a wheelchair since then. Now he
is improving mobility devices, including wheelchairs and scooters, by
adapting components designed for vehicles and drones. Mr. Cooper says he
can take new batteries, motors and algorithms from other industries and
build ``a much lighter chair with the same capabilities.''

He has developed a waterproof chair that runs on compressed air,
\href{https://www.nytimes.com/2017/12/05/opinion/morgans-wonderland-waterpark-kids-play.html}{originally
for a wheelchair-accessible} water park. Water parks are fun, but more
important, the innovation will make it easier for wheelchair users to go
out in the rain. Meanwhile, makers of self-driving cars are now
consulting not just Blind users, who have long been involved, but people
with myriad other disabilities, including those in wheelchairs, who
would need to be able to roll into the vehicle.

At the New York Public Library's Dimensions lab, Chancey Fleet, who is
Blind, is working with a team to make spatial learning easier for blind
people and to provide access to information --- part of the library's
core mission --- to those who can best get it through touch. Visitors to
the lab in the Heiskell branch of the library in Manhattan are invited
to make 3-D printed objects and tactile graphics, or graphics embedded
with Braille and other textural elements to make their meaning legible
by touch. Ms. Fleet is hoping to end what she calls ``image poverty.''

She says as a Blind child, ``I thought I was someone who didn't have any
aptitude at all in STEM, even though I did well academically.'' But she
later realized her problem was not with science and technology per se.
''Looking back, it seems as though I was a spatial learner,'' she said.
``If the images are there, it turns out that the aptitudes are there.''

Experts in disability and technology, like Ashley Shew, associate
professor at Virginia Tech in the Department of Science, Technology and
Society, says that the best of these projects emerge out of the
DIY-culture so prominent within disabled communities. Too often, the
biggest and most promising innovations may come with hidden barriers,
like cost, maintenance and the need to customize them.

``We've been misled,'' said Ms. Shew, who identifies as multiply
disabled and uses hearing aids and prosthetics. ``The public perception
is very celebratory about new developments,'' but this ``completely
looks over issues of maintenance and wear. People think you're given
this item once and then it's fixed for all eternity.''

Not only are devices like prosthetics and hearing aids often not covered
by insurance, but expert care is hard to find. Ms. Shew, for example,
travels four hours for leg prosthetic care. Meanwhile, too much
technology is designed around a perception of what's normal. For
example, arm prosthetics are often designed with five fingers, a hand,
but Ms. Shew says, ``A lot of arm amputees don't necessarily want'' that
but instead would like a bike-riding arm or a chopping arm.

Maintenance isn't the only ongoing issue for users of
disability-specific technology; intellectual property law can restrict
the ability of users to customize their devices to suit their changing
needs. Ian Smith, a software engineer who is Deaf, has dwarfism and uses
a power wheelchair, points out that too often disabled people are not
permitted to tinker with devices because of trademark issues, negating
what many call the
\href{https://www.eff.org/issues/right-to-repair}{right to repair}.
``You're at the mercy of the manufacturer for upgrades and repairs,'' he
said.

Sara Hendren, who teaches design at Olin College of Engineering in
Massachusetts and is the parent of a child with Down syndrome,
illustrates the benefits of empowering disabled designers in her
forthcoming book,
``\href{https://www.penguinrandomhouse.com/books/561049/what-can-a-body-do-by-sara-hendren/\#:~:text=In\%20a\%20series\%20of\%20vivid,and\%20settings\%20we\%20live\%20with.}{What
Can a Body Do}?'' In it she introduces us to Chris, who was born with
one arm. After being stymied initially in trying to change his infant's
diaper, he ultimately joined felt holsters to soft cords that he could
attach to his shoulder. The baby's feet rest in the felt, secure.

``The result is nothing that would dazzle at some tech expo,'' but it
reveals, Ms. Hendren said in an interview, how the right technology can
make the ``world bend a little bit'' toward the user rather than just
bending the user toward a normative world. Ms. Hendren said that
adaptive technology, the phrase she prefers to the more commonly used
``assistive technology,'' is not about helping, but about shifting both
the body and the world into closer harmony. It's not using tech to make
things seem ``normal.''

Bob Williams, \href{https://communicationfirst.org/mission/}{policy
director at Communication First}, an advocacy group for people like him
with speech-related communication disabilities, has cerebral palsy and
uses a stand-alone device to produce audible speech. It was designed
around 1990, and Mr. Williams is worried about obsolescence. Today, many
nonspeaking individuals can use apps with speech tools built into
tablets, smartphones and computers. ``It's a bridge'' between disabled
and nondisabled people, Mr. Williams says, because everyone can ``relate
to the technology.''

In my family, we've certainly found that to be the case, but not
everyone does. My son, a white Midwesterner who is autistic and has Down
syndrome, uses a speech app called Proloquo2go. There's a default
setting that mimics how he talks, but not everyone finds a voice that is
fitting.

Meryl Alper, assistant professor of communication studies at
Northeastern University, argues in her book
``\href{https://mitpress.mit.edu/books/giving-voice}{Giving Voice},'' **
that this app creates inequality. Not only do many families have trouble
with programming apps like this, but Proloquo2go doesn't ``have a single
speech option in U.S. English in a voice that uses speech samples from
an adult woman of color. The only one that is racialized is Saul, a
`hip-hop' voice.''

Over email, David Niemeijer, the chief executive of AssistiveWare, the
company that makes Proloquo2Go, blames prohibitive costs in making new
voices. He hopes upcoming collaborations among producers of
text-to-speech technologies will lower those costs.

The lack of nonwhite voices in this app is one of many such examples,
says
\href{https://liberalarts.vt.edu/departments-and-schools/department-of-science-technology-and-society/academic-programs/phd-science-and-technology-studies/students/2021/damien-Patrick-williams.html}{Damien
Williams}, a Ph.D. student at Virginia Tech. Mr. Williams says that
disability technology often reflects biases about race, gender and
ideals of what is or should be ``normal.'' There are
\href{https://gizmodo.com/why-cant-this-soap-dispenser-identify-dark-skin-1797931773}{soap
dispensers} that don't recognize black and brown skin, for example, and
automated captioning can't always handle accented English. Mr. Williams
says programmers have to contend with assumptions about differences in
race and class and need to include ``underlying systems that are not
based in outdated ideas about disability.''

For Ms. Shew, the Virginia Tech professor, the best way to ensure that
this transformation continues will require centering the power --- and
the money --- on disabled people as the initiators of innovation. ``The
future of assistive tech should be `cripped,''' a once-pejorative term
that many members of the disability community have reclaimed, she said.
``It should be bent, claimed, reclaimed, reforged, hacked,
owned/controlled, made, swapped and shared by disabled people.''

\begin{center}\rule{0.5\linewidth}{\linethickness}\end{center}

David M. Perry is a journalist and the senior academic adviser in the
History Department at the University of Minnesota.

Advertisement

\protect\hyperlink{after-bottom}{Continue reading the main story}

\hypertarget{site-index}{%
\subsection{Site Index}\label{site-index}}

\hypertarget{site-information-navigation}{%
\subsection{Site Information
Navigation}\label{site-information-navigation}}

\begin{itemize}
\tightlist
\item
  \href{https://help.nytimes.com/hc/en-us/articles/115014792127-Copyright-notice}{©~2020~The
  New York Times Company}
\end{itemize}

\begin{itemize}
\tightlist
\item
  \href{https://www.nytco.com/}{NYTCo}
\item
  \href{https://help.nytimes.com/hc/en-us/articles/115015385887-Contact-Us}{Contact
  Us}
\item
  \href{https://www.nytco.com/careers/}{Work with us}
\item
  \href{https://nytmediakit.com/}{Advertise}
\item
  \href{http://www.tbrandstudio.com/}{T Brand Studio}
\item
  \href{https://www.nytimes.com/privacy/cookie-policy\#how-do-i-manage-trackers}{Your
  Ad Choices}
\item
  \href{https://www.nytimes.com/privacy}{Privacy}
\item
  \href{https://help.nytimes.com/hc/en-us/articles/115014893428-Terms-of-service}{Terms
  of Service}
\item
  \href{https://help.nytimes.com/hc/en-us/articles/115014893968-Terms-of-sale}{Terms
  of Sale}
\item
  \href{https://spiderbites.nytimes.com}{Site Map}
\item
  \href{https://help.nytimes.com/hc/en-us}{Help}
\item
  \href{https://www.nytimes.com/subscription?campaignId=37WXW}{Subscriptions}
\end{itemize}
