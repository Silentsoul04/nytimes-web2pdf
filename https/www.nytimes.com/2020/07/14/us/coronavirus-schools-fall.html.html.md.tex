Sections

SEARCH

\protect\hyperlink{site-content}{Skip to
content}\protect\hyperlink{site-index}{Skip to site index}

\href{https://www.nytimes.com/section/us}{U.S.}

\href{https://myaccount.nytimes.com/auth/login?response_type=cookie\&client_id=vi}{}

\href{https://www.nytimes.com/section/todayspaper}{Today's Paper}

\href{/section/us}{U.S.}\textbar{}Most Big School Districts Aren't Ready
to Reopen. Here's Why.

\url{https://nyti.ms/30dihiq}

\begin{itemize}
\item
\item
\item
\item
\item
\end{itemize}

\href{https://www.nytimes.com/news-event/coronavirus?action=click\&pgtype=Article\&state=default\&region=TOP_BANNER\&context=storylines_menu}{The
Coronavirus Outbreak}

\begin{itemize}
\tightlist
\item
  live\href{https://www.nytimes.com/2020/08/03/world/coronavirus-covid-19.html?action=click\&pgtype=Article\&state=default\&region=TOP_BANNER\&context=storylines_menu}{Latest
  Updates}
\item
  \href{https://www.nytimes.com/interactive/2020/us/coronavirus-us-cases.html?action=click\&pgtype=Article\&state=default\&region=TOP_BANNER\&context=storylines_menu}{Maps
  and Cases}
\item
  \href{https://www.nytimes.com/interactive/2020/science/coronavirus-vaccine-tracker.html?action=click\&pgtype=Article\&state=default\&region=TOP_BANNER\&context=storylines_menu}{Vaccine
  Tracker}
\item
  \href{https://www.nytimes.com/2020/08/02/us/covid-college-reopening.html?action=click\&pgtype=Article\&state=default\&region=TOP_BANNER\&context=storylines_menu}{College
  Reopening}
\item
  \href{https://www.nytimes.com/live/2020/08/03/business/stock-market-today-coronavirus?action=click\&pgtype=Article\&state=default\&region=TOP_BANNER\&context=storylines_menu}{Economy}
\end{itemize}

Advertisement

\protect\hyperlink{after-top}{Continue reading the main story}

Supported by

\protect\hyperlink{after-sponsor}{Continue reading the main story}

\hypertarget{most-big-school-districts-arent-ready-to-reopen-heres-why}{%
\section{Most Big School Districts Aren't Ready to Reopen. Here's
Why.}\label{most-big-school-districts-arent-ready-to-reopen-heres-why}}

All but two of the nation's 10 largest districts exceed a key public
health threshold, according to a New York Times analysis.

\includegraphics{https://static01.nyt.com/images/2020/07/14/us/14virus-schools/merlin_174563640_d9c4929f-c7a1-4342-bdb5-7c191ad5704e-articleLarge.jpg?quality=75\&auto=webp\&disable=upscale}

\href{https://www.nytimes.com/by/dana-goldstein}{\includegraphics{https://static01.nyt.com/images/2018/06/12/multimedia/author-dana-goldstein/author-dana-goldstein-thumbLarge.png}}\href{https://www.nytimes.com/by/eliza-shapiro}{\includegraphics{https://static01.nyt.com/images/2018/12/28/multimedia/author-eliza-shapiro/author-eliza-shapiro-thumbLarge.png}}

By \href{https://www.nytimes.com/by/dana-goldstein}{Dana Goldstein} and
\href{https://www.nytimes.com/by/eliza-shapiro}{Eliza Shapiro}

\begin{itemize}
\item
  Published July 14, 2020Updated July 16, 2020
\item
  \begin{itemize}
  \item
  \item
  \item
  \item
  \item
  \end{itemize}
\end{itemize}

As education leaders decide whether to reopen classrooms in the fall
amid a raging pandemic, many are looking to a standard generally agreed
upon among epidemiologists: To control community spread of the
coronavirus, the average daily infection rate among those who are tested
should not exceed 5 percent.

But of the nation's
\href{https://nces.ed.gov/programs/digest/d17/tables/dt17_215.30.asp}{10
largest school districts}, only New York City and Chicago appear to have
achieved that public health goal, according to a New York Times analysis
of city and county-level data.

Some of the biggest districts, like Miami-Dade County in Florida and
Clark County, Nev., which includes Las Vegas, are in counties that have
recently reported positive test rates more than four times greater than
the 5 percent threshold, the data shows.

The alarming spread of the virus has prompted a growing number of
districts to announce they would rely on online instruction in the fall.
The superintendent of the nation's sixth-largest district, in Broward
County, Fla., on Tuesday recommended full-time remote learning despite
pressure from the state's governor and President Trump. That followed
\href{https://www.nytimes.com/2020/07/13/us/lausd-san-diego-school-reopening.html}{an
announcement on Monday} that California's two largest districts, Los
Angeles and San Diego, will teach 100 percent online.

``I'm just super frustrated and really disappointed that our nation, our
states and our communities have not exercised the discipline that they
need in order to get the coronavirus under control,'' said Robert W.
Runcie, the Broward superintendent. ``Now the futures of our young
people are collateral damage from our inability to take this thing
seriously.''

In recent days, Nashville, Atlanta, Arlington, Va., and Oakland, Calif.,
have also announced plans to start the school year remotely.

The broad national move to keep schools shuttered represents a deepening
crisis for the nation's tens of millions of schoolchildren, who are
already
\href{https://www.nytimes.com/2020/06/05/us/coronavirus-education-lost-learning.html}{falling
behind} academically and socially during the pandemic.

The decisions will also require working parents to continue to carry a
\href{https://www.nytimes.com/2020/07/10/nyregion/nyc-school-daycare-reopening.html}{heavy
burden} of ad hoc child care and home schooling, which is presenting
families with impossible trade-offs.

Many European and Asian nations have been able to
\href{https://www.nytimes.com/2020/07/11/health/coronavirus-schools-reopen.html}{reopen
schools safely} after controlling the spread of the virus using tools
such as widespread mask wearing, testing and contact tracing. Some
American health experts believe that operating schools may be safer than
generally acknowledged, given research suggesting that young children
are less likely than adults to either
\href{https://pubmed.ncbi.nlm.nih.gov/32546824/}{contract} the
coronavirus or to
\href{https://pediatrics.aappublications.org/content/early/2020/07/08/peds.2020-004879}{spread
it}.

\includegraphics{https://static01.nyt.com/images/2020/07/14/us/14virus-schools02/merlin_173143980_2090a5ea-a585-4bbe-b66f-2bbda0c6f3de-articleLarge.jpg?quality=75\&auto=webp\&disable=upscale}

But the fact remains that the United States has failed to control the
spread of the coronavirus, making it difficult to apply the reassuring
news from abroad. Local and state leaders must now decide on the best
course of action between two bad choices: either open school buildings
and take the risk that educators, students and parents become ill, or
keep them shuttered and hinder the development of tens of millions of
children.

``These are like wartime decisions,'' Mr. Runcie said. ``This is
literally like sending people into battle, and without appropriate
tools.''

\hypertarget{latest-updates-global-coronavirus-outbreak}{%
\section{\texorpdfstring{\href{https://www.nytimes.com/2020/08/03/world/coronavirus-covid-19.html?action=click\&pgtype=Article\&state=default\&region=MAIN_CONTENT_1\&context=storylines_live_updates}{Latest
Updates: Global Coronavirus
Outbreak}}{Latest Updates: Global Coronavirus Outbreak}}\label{latest-updates-global-coronavirus-outbreak}}

Updated 2020-08-04T07:05:52.634Z

\begin{itemize}
\tightlist
\item
  \href{https://www.nytimes.com/2020/08/03/world/coronavirus-covid-19.html?action=click\&pgtype=Article\&state=default\&region=MAIN_CONTENT_1\&context=storylines_live_updates\#link-4547638f}{Fauci
  defends Birx after she is criticized by Trump.}
\item
  \href{https://www.nytimes.com/2020/08/03/world/coronavirus-covid-19.html?action=click\&pgtype=Article\&state=default\&region=MAIN_CONTENT_1\&context=storylines_live_updates\#link-15e7f995}{Trump
  derides Democrats as lawmakers and administration officials try to
  break stimulus impasse.}
\item
  \href{https://www.nytimes.com/2020/08/03/world/coronavirus-covid-19.html?action=click\&pgtype=Article\&state=default\&region=MAIN_CONTENT_1\&context=storylines_live_updates\#link-e5a2cda}{The
  deadline for 2020 census counting has been moved up by a month.}
\end{itemize}

\href{https://www.nytimes.com/2020/08/03/world/coronavirus-covid-19.html?action=click\&pgtype=Article\&state=default\&region=MAIN_CONTENT_1\&context=storylines_live_updates}{See
more updates}

More live coverage:
\href{https://www.nytimes.com/live/2020/08/03/business/stock-market-today-coronavirus?action=click\&pgtype=Article\&state=default\&region=MAIN_CONTENT_1\&context=storylines_live_updates}{Markets}

In the United States, districts are increasingly splitting into three
groups: those that plan to teach online only, those that will allow
families to choose between in-person and at-home instruction, and those
offering a hybrid approach, with students spending some days in
classrooms and some learning remotely.

Many large districts fall into the third category, although more are
moving into the first as the virus continues to rage in their regions.

The 5 percent positive test rate was not developed specifically for
schools, but it has emerged as a metric that many districts are
considering when making plans.

The number comes from a general threshold
\href{https://globalepidemics.org/wp-content/uploads/2020/06/key_metrics_and_indicators_v4.pdf}{established
by public health experts}, who say that a positive test rate of less
than 10 percent, and ideally under 3 percent, is generally needed to
control and suppress the spread of the virus in a community.

The \href{https://coronavirus.jhu.edu/testing/testing-positivity}{World
Health Organization} encourages governments to reopen their economies
only if their positivity rates are below 5 percent for at least two
weeks. But the rate is a reliable indicator only when there is
widespread testing, and
\href{https://www.nytimes.com/interactive/2020/us/coronavirus-testing.html}{many
states are still not testing enough.}

This week, Gov. Andrew M. Cuomo of New York, a Democrat, announced that
schools across the state could only reopen in September if they were in
a region where the average daily infection rate was below 5 percent over
a two-week period. None of the state's 10 regions currently have an
infection rate over 2 percent.

Jim Malatras, an aide to the governor, said the state ``wanted to
establish an objective number so schools can plan.''

Image

Teachers in Orange County, Fla., protested outside the school district's
headquarters in Orlando this month, objecting to Gov. Ron DeSantis's
efforts~to have schools open five days a week.Credit...Joe
Burbank/Orlando Sentinel, via Associated Press

In Florida, which has five of the nation's largest school districts ---
Miami-Dade, Broward, Hillsborough, Orange and Palm Beach Counties ---
officials have taken a different approach, aggressively pushing schools
to resume operations.

Last week, the state's education commissioner, Richard Corcoran, who was
nominated by Gov. Ron DeSantis, a Republican, issued an
\href{http://www.fldoe.org/core/fileparse.php/19861/urlt/DOE-2020-EO-06.pdf}{emergency
order} asking districts to reopen ``brick and mortar schools with the
full panoply of services.''

But fully staffing the Broward school system to maintain social
distancing between students and staff members would require at least
\$230 million in new funding, Mr. Runcie said, because of the need to
hire thousands of additional teachers to reduce class sizes to an
average of 14 students.

In California, where case numbers have been soaring, **** reopening
schools has become a moving target. Just two and a half weeks ago, when
Gov. Gavin Newsom, a Democrat, signed the state budget, it included
strong language that discouraged schools from operating exclusively
online.

But as cases climbed, concerns about too much online instruction quickly
morphed into concerns about too little school safety. California is
using the 5 percent positivity threshold as a guideline --- one that has
grown **** increasingly distant in many places. In Los Angeles County,
home to the nation's second-largest school district, the positivity rate
has averaged
\href{http://publichealth.lacounty.gov/media/coronavirus/data/index.htm}{9
percent over the past seven days.}

``We had hoped it wouldn't get to this point,'' said the Los Angeles
schools superintendent, Austin Beutner. ``All of a sudden, in the middle
of June, everything just went through the roof.''

The decision by Los Angeles and San Diego to teach online is expected to
be influential. Several other large districts in the state, including
San Bernardino, Santa Clara and Oakland, will start the year remotely,
and this week the public schools in Pasadena and the entirety of
Stanislaus County in the Central Valley said they would delay in-person
learning at least for the first weeks of August.

Even in Orange County, Calif., where a cluster of conservative officials
has aggressively pushed for reopening, larger districts have been
hearing from teachers' unions and nervously eyeing the local test
positivity rate, which
\href{https://ochca.maps.arcgis.com/apps/opsdashboard/index.html\#/cc4859c8c522496b9f21c451de2fedae}{averaged
14.6 percent} over the last seven days.

\href{https://www.nytimes.com/news-event/coronavirus?action=click\&pgtype=Article\&state=default\&region=MAIN_CONTENT_3\&context=storylines_faq}{}

\hypertarget{the-coronavirus-outbreak-}{%
\subsubsection{The Coronavirus Outbreak
›}\label{the-coronavirus-outbreak-}}

\hypertarget{frequently-asked-questions}{%
\paragraph{Frequently Asked
Questions}\label{frequently-asked-questions}}

Updated August 3, 2020

\begin{itemize}
\item ~
  \hypertarget{im-a-small-business-owner-can-i-get-relief}{%
  \paragraph{I'm a small-business owner. Can I get
  relief?}\label{im-a-small-business-owner-can-i-get-relief}}

  \begin{itemize}
  \tightlist
  \item
    The
    \href{https://www.nytimes.com/article/small-business-loans-stimulus-grants-freelancers-coronavirus.html?action=click\&pgtype=Article\&state=default\&region=MAIN_CONTENT_3\&context=storylines_faq}{stimulus
    bills enacted in March} offer help for the millions of American
    small businesses. Those eligible for aid are businesses and
    nonprofit organizations with fewer than 500 workers, including sole
    proprietorships, independent contractors and freelancers. Some
    larger companies in some industries are also eligible. The help
    being offered, which is being managed by the Small Business
    Administration, includes the Paycheck Protection Program and the
    Economic Injury Disaster Loan program. But lots of folks have
    \href{https://www.nytimes.com/interactive/2020/05/07/business/small-business-loans-coronavirus.html?action=click\&pgtype=Article\&state=default\&region=MAIN_CONTENT_3\&context=storylines_faq}{not
    yet seen payouts.} Even those who have received help are confused:
    The rules are draconian, and some are stuck sitting on
    \href{https://www.nytimes.com/2020/05/02/business/economy/loans-coronavirus-small-business.html?action=click\&pgtype=Article\&state=default\&region=MAIN_CONTENT_3\&context=storylines_faq}{money
    they don't know how to use.} Many small-business owners are getting
    less than they expected or
    \href{https://www.nytimes.com/2020/06/10/business/Small-business-loans-ppp.html?action=click\&pgtype=Article\&state=default\&region=MAIN_CONTENT_3\&context=storylines_faq}{not
    hearing anything at all.}
  \end{itemize}
\item ~
  \hypertarget{what-are-my-rights-if-i-am-worried-about-going-back-to-work}{%
  \paragraph{What are my rights if I am worried about going back to
  work?}\label{what-are-my-rights-if-i-am-worried-about-going-back-to-work}}

  \begin{itemize}
  \tightlist
  \item
    Employers have to provide
    \href{https://www.osha.gov/SLTC/covid-19/standards.html}{a safe
    workplace} with policies that protect everyone equally.
    \href{https://www.nytimes.com/article/coronavirus-money-unemployment.html?action=click\&pgtype=Article\&state=default\&region=MAIN_CONTENT_3\&context=storylines_faq}{And
    if one of your co-workers tests positive for the coronavirus, the
    C.D.C.} has said that
    \href{https://www.cdc.gov/coronavirus/2019-ncov/community/guidance-business-response.html}{employers
    should tell their employees} -\/- without giving you the sick
    employee's name -\/- that they may have been exposed to the virus.
  \end{itemize}
\item ~
  \hypertarget{should-i-refinance-my-mortgage}{%
  \paragraph{Should I refinance my
  mortgage?}\label{should-i-refinance-my-mortgage}}

  \begin{itemize}
  \tightlist
  \item
    \href{https://www.nytimes.com/article/coronavirus-money-unemployment.html?action=click\&pgtype=Article\&state=default\&region=MAIN_CONTENT_3\&context=storylines_faq}{It
    could be a good idea,} because mortgage rates have
    \href{https://www.nytimes.com/2020/07/16/business/mortgage-rates-below-3-percent.html?action=click\&pgtype=Article\&state=default\&region=MAIN_CONTENT_3\&context=storylines_faq}{never
    been lower.} Refinancing requests have pushed mortgage applications
    to some of the highest levels since 2008, so be prepared to get in
    line. But defaults are also up, so if you're thinking about buying a
    home, be aware that some lenders have tightened their standards.
  \end{itemize}
\item ~
  \hypertarget{what-is-school-going-to-look-like-in-september}{%
  \paragraph{What is school going to look like in
  September?}\label{what-is-school-going-to-look-like-in-september}}

  \begin{itemize}
  \tightlist
  \item
    It is unlikely that many schools will return to a normal schedule
    this fall, requiring the grind of
    \href{https://www.nytimes.com/2020/06/05/us/coronavirus-education-lost-learning.html?action=click\&pgtype=Article\&state=default\&region=MAIN_CONTENT_3\&context=storylines_faq}{online
    learning},
    \href{https://www.nytimes.com/2020/05/29/us/coronavirus-child-care-centers.html?action=click\&pgtype=Article\&state=default\&region=MAIN_CONTENT_3\&context=storylines_faq}{makeshift
    child care} and
    \href{https://www.nytimes.com/2020/06/03/business/economy/coronavirus-working-women.html?action=click\&pgtype=Article\&state=default\&region=MAIN_CONTENT_3\&context=storylines_faq}{stunted
    workdays} to continue. California's two largest public school
    districts --- Los Angeles and San Diego --- said on July 13, that
    \href{https://www.nytimes.com/2020/07/13/us/lausd-san-diego-school-reopening.html?action=click\&pgtype=Article\&state=default\&region=MAIN_CONTENT_3\&context=storylines_faq}{instruction
    will be remote-only in the fall}, citing concerns that surging
    coronavirus infections in their areas pose too dire a risk for
    students and teachers. Together, the two districts enroll some
    825,000 students. They are the largest in the country so far to
    abandon plans for even a partial physical return to classrooms when
    they reopen in August. For other districts, the solution won't be an
    all-or-nothing approach.
    \href{https://bioethics.jhu.edu/research-and-outreach/projects/eschool-initiative/school-policy-tracker/}{Many
    systems}, including the nation's largest, New York City, are
    devising
    \href{https://www.nytimes.com/2020/06/26/us/coronavirus-schools-reopen-fall.html?action=click\&pgtype=Article\&state=default\&region=MAIN_CONTENT_3\&context=storylines_faq}{hybrid
    plans} that involve spending some days in classrooms and other days
    online. There's no national policy on this yet, so check with your
    municipal school system regularly to see what is happening in your
    community.
  \end{itemize}
\item ~
  \hypertarget{is-the-coronavirus-airborne}{%
  \paragraph{Is the coronavirus
  airborne?}\label{is-the-coronavirus-airborne}}

  \begin{itemize}
  \tightlist
  \item
    The coronavirus
    \href{https://www.nytimes.com/2020/07/04/health/239-experts-with-one-big-claim-the-coronavirus-is-airborne.html?action=click\&pgtype=Article\&state=default\&region=MAIN_CONTENT_3\&context=storylines_faq}{can
    stay aloft for hours in tiny droplets in stagnant air}, infecting
    people as they inhale, mounting scientific evidence suggests. This
    risk is highest in crowded indoor spaces with poor ventilation, and
    may help explain super-spreading events reported in meatpacking
    plants, churches and restaurants.
    \href{https://www.nytimes.com/2020/07/06/health/coronavirus-airborne-aerosols.html?action=click\&pgtype=Article\&state=default\&region=MAIN_CONTENT_3\&context=storylines_faq}{It's
    unclear how often the virus is spread} via these tiny droplets, or
    aerosols, compared with larger droplets that are expelled when a
    sick person coughs or sneezes, or transmitted through contact with
    contaminated surfaces, said Linsey Marr, an aerosol expert at
    Virginia Tech. Aerosols are released even when a person without
    symptoms exhales, talks or sings, according to Dr. Marr and more
    than 200 other experts, who
    \href{https://academic.oup.com/cid/article/doi/10.1093/cid/ciaa939/5867798}{have
    outlined the evidence in an open letter to the World Health
    Organization}.
  \end{itemize}
\end{itemize}

On Monday, the county's Board of Education voted to recommend that
schools reopen without social distancing and other precautions. But
their recommendation is not binding, and on Tuesday, the Santa Ana
Unified School District, the county's second-largest, announced it
\href{https://www.sausd.us/site/default.aspx?PageType=3\&DomainID=1\&ModuleInstanceID=6157\&ViewID=6446EE88-D30C-497E-9316-3F8874B3E108\&RenderLoc=0\&FlexDataID=96385\&PageID=1}{would
pivot} from a planned hybrid reopening to distance learning.

``While we hope at some point to have our students attend our schools
alongside their classmates and teachers, now is not the time,'' the
superintendent, Jerry Almendarez, said in a statement.

In the Northeast, parents and school leaders face a very different
landscape. New York City, the nation's largest district with some 1.1
million students and 1,800 schools, was the center of the country's
outbreak this spring. Now the city's average positive test rate hovers
around 2 percent --- the lowest among the country's largest school
districts.

Image

New York City schools plan to offer part-time in-person instruction in
the fall.Credit...Gabriela Bhaskar for The New York Times

That leaves New York virtually alone --- with the exception of Chicago,
the third-largest district, where the city had a 5 percent positivity
rate --- in having the virus sufficiently under control to satisfy the
public health threshold. Still, New York City will likely
\href{https://www.nytimes.com/2020/07/08/nyregion/nyc-schools-reopening-plan.html}{offer
in-person instruction only one to three days a week} when the school
year begins in September.

In Texas, Gov. Greg Abbott, a Republican, has made the return to
classrooms a priority, but he
\href{https://www.texastribune.org/2020/07/14/texas-schools-online-pandemic/}{signaled
greater flexibility on Tuesday} amid a steep uptick in virus cases. The
state's guidelines require districts to offer in-person education five
days a week, although parents could choose to have children learn online
only.

The Houston Federation of Teachers, a union that represents 6,500
educators, had blasted the state's plan for reopening campuses as
``unacceptably vague and hardly adequate.'' In a letter sent to the
school district on Sunday, teachers asked to delay classroom instruction
until the area had seen a decline in new cases for at least 14 days and
achieved the positive test rate of less than 5 percent. That is far from
the current landscape in greater Houston, which in recent days had
\href{https://www.tmc.edu/coronavirus-updates/covid-19-testing-trends/}{a
positive test rate of 13 percent}.

``No one wants to be inside the school building more than teachers,''
said Maxie Hollingsworth, a math teacher at a Houston elementary school.
But Ms. Hollingsworth said she was not comfortable returning to her
classroom and risking infection; her daughter has asthma and is at
higher risk of complications from the coronavirus.

``The plain truth to me,'' she said, ``is it is immoral to reopen
schools without the things we need in place.''

Texas officials are closely watching the national landscape, and it is
possible they will modify the five-day-a-week requirement. The Houston
Independent School District, the nation's seventh largest with 209,000
students, is expected to make an announcement on Wednesday about plans
for the academic year.

Rising virus counts in Clark County, Nev., are complicating plans for
the nation's fifth-largest district to provide 326,000 students with two
days per week of in-person learning.

``The No. 1 aspect is safety,'' said Linda Cavazos, the vice president
of the district's board of trustees. ``We may be looking at having to
change the entire thing to distance learning.''

Nery Martinez, who has two teenage children and was laid off from his
job as a bartender at the Caesars Palace casino because of the pandemic,
said he preferred online instruction, despite the financial impact that
supervising his children's learning would have on his family once he
goes back to work.

``Face to face is a lot of risk,'' he said. ``I need to make the money
to pay rent, but I want to be here to protect them.''

Contributing reporting were Shawn Hubler, Dan Levin, Sarah Mervosh and
Mitch Smith.

Advertisement

\protect\hyperlink{after-bottom}{Continue reading the main story}

\hypertarget{site-index}{%
\subsection{Site Index}\label{site-index}}

\hypertarget{site-information-navigation}{%
\subsection{Site Information
Navigation}\label{site-information-navigation}}

\begin{itemize}
\tightlist
\item
  \href{https://help.nytimes.com/hc/en-us/articles/115014792127-Copyright-notice}{©~2020~The
  New York Times Company}
\end{itemize}

\begin{itemize}
\tightlist
\item
  \href{https://www.nytco.com/}{NYTCo}
\item
  \href{https://help.nytimes.com/hc/en-us/articles/115015385887-Contact-Us}{Contact
  Us}
\item
  \href{https://www.nytco.com/careers/}{Work with us}
\item
  \href{https://nytmediakit.com/}{Advertise}
\item
  \href{http://www.tbrandstudio.com/}{T Brand Studio}
\item
  \href{https://www.nytimes.com/privacy/cookie-policy\#how-do-i-manage-trackers}{Your
  Ad Choices}
\item
  \href{https://www.nytimes.com/privacy}{Privacy}
\item
  \href{https://help.nytimes.com/hc/en-us/articles/115014893428-Terms-of-service}{Terms
  of Service}
\item
  \href{https://help.nytimes.com/hc/en-us/articles/115014893968-Terms-of-sale}{Terms
  of Sale}
\item
  \href{https://spiderbites.nytimes.com}{Site Map}
\item
  \href{https://help.nytimes.com/hc/en-us}{Help}
\item
  \href{https://www.nytimes.com/subscription?campaignId=37WXW}{Subscriptions}
\end{itemize}
