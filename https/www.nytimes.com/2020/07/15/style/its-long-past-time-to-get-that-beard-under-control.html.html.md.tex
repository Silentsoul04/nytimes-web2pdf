Sections

SEARCH

\protect\hyperlink{site-content}{Skip to
content}\protect\hyperlink{site-index}{Skip to site index}

\href{https://www.nytimes.com/section/style}{Style}

\href{https://myaccount.nytimes.com/auth/login?response_type=cookie\&client_id=vi}{}

\href{https://www.nytimes.com/section/todayspaper}{Today's Paper}

\href{/section/style}{Style}\textbar{}It's Long Past Time to Get That
Beard Under Control

\url{https://nyti.ms/3j4dh8m}

\begin{itemize}
\item
\item
\item
\item
\item
\end{itemize}

\href{https://www.nytimes.com/spotlight/at-home?action=click\&pgtype=Article\&state=default\&region=TOP_BANNER\&context=at_home_menu}{At
Home}

\begin{itemize}
\tightlist
\item
  \href{https://www.nytimes.com/2020/07/28/books/time-for-a-literary-road-trip.html?action=click\&pgtype=Article\&state=default\&region=TOP_BANNER\&context=at_home_menu}{Take:
  A Literary Road Trip}
\item
  \href{https://www.nytimes.com/2020/07/29/magazine/bored-with-your-home-cooking-some-smoky-eggplant-will-fix-that.html?action=click\&pgtype=Article\&state=default\&region=TOP_BANNER\&context=at_home_menu}{Cook:
  Smoky Eggplant}
\item
  \href{https://www.nytimes.com/2020/07/27/travel/moose-michigan-isle-royale.html?action=click\&pgtype=Article\&state=default\&region=TOP_BANNER\&context=at_home_menu}{Look
  Out: For Moose}
\item
  \href{https://www.nytimes.com/interactive/2020/at-home/even-more-reporters-editors-diaries-lists-recommendations.html?action=click\&pgtype=Article\&state=default\&region=TOP_BANNER\&context=at_home_menu}{Explore:
  Reporters' Obsessions}
\end{itemize}

Advertisement

\protect\hyperlink{after-top}{Continue reading the main story}

Supported by

\protect\hyperlink{after-sponsor}{Continue reading the main story}

Skin Deep

\hypertarget{its-long-past-time-to-get-that-beard-under-control}{%
\section{It's Long Past Time to Get That Beard Under
Control}\label{its-long-past-time-to-get-that-beard-under-control}}

Good thing men are warming up to some new tools.

\includegraphics{https://static01.nyt.com/images/2020/07/14/fashion/14SKIN-BEARDSArt/14SKIN-BEARDSArt-articleLarge.jpg?quality=75\&auto=webp\&disable=upscale}

By Andrew Adam Newman

\begin{itemize}
\item
  July 15, 2020
\item
  \begin{itemize}
  \item
  \item
  \item
  \item
  \item
  \end{itemize}
\end{itemize}

Some men working from home and growing beards for the first time are
coming to the realization that their facial hair is a tangle of waves
and curls. But Matt Vilanova discovered how unmanageable his beard can
be years ago.

``My beard is just naturally very kinky, and it kind of twists and
turns,'' said Mr. Vilanova, 34, a manager at a software company who
lives in Seymour, Conn. ``When I grew a beard a little longer than a
five o'clock shadow, I looked like a Neanderthal.''

A couple of years ago, when his beard was about two inches long, Mr.
Vilanova borrowed --- from his girlfriend and future wife --- a
hair-straightening flatiron, the type with opposing ceramic plates.

Clamping his whiskers in a flatiron hot enough to
\href{https://www.youtube.com/watch?v=IhLEzuH3L40}{pop popcorn} ``was a
very delicate process,'' Mr. Vilanova said. ``To a degree it helped, but
it got very close to my skin. I was like, `I'm going to absolutely
destroy myself if I keep on doing this.'''

A year ago, Mr. Vilanova bought a
\href{https://www.mascbyjeffchastain.com/collections/shop-all/products/kuschelbar\%C2\%AE-hair-and-beard-straightener}{Kuschelbär},
a heated beard-straightening brush made by Masc by Jeff Chastain. It has
heated teeth that emerge from a heated plate, a compact version of the
full-size hair-straightening brushes marketed to women.

``I approached it with some skepticism, but the very first time I used
it, I was like, `This thing is awesome,''' Mr. Vilanova said.
``Everything looks clean and sharp and rigid. You feel like a badass.''

Also a fan of the Kuschelbär: the basketball star Andre Drummond, who
\href{https://www.facebook.com/MASCbyjeffchastain/videos/nba-and-detroitpistons-star-andredrummondd-reached-out-to-me-a-couple-weeks-ago-/598722770629116/}{made
a video of himself}straightening his beard with it.

Sales of beard-straightening brushes are, appropriately enough, heating
up.

Jeff Chastain, who owns a salon in Greenwich Village, introduced the
Kuschelbär (German for ``cuddle bear'') in late 2017, and he says that
it is the original beard-straightening brush. He has sold more than
115,000 of the devices, he said, which he encourages men to use on their
head hair too. He now sells three models, ranging from \$100 for the
original to \$140 for a cordless version.

\href{https://www.aberlite.com/}{Aberlite}, a beard care brand, began
selling a heated straightening brush on Amazon early in 2019. Teng Ma,
who owns the company, said that initially he bought thousands of
no-label women's hair-straightening brushes from Alibaba, the Chinese
e-commerce giant, and rebranded them as beard brushes. Now he sells
three more compact models designed for beards. (Brushes with longer
teeth can glide over shorter beards instead of engaging them.)

Aberlite sold almost 25,000 beard brushes in 2019 and is on pace to sell
as many this year, Mr. Ma said. They cost \$40 to \$90.

Another brand, the Beard Struggle, also bought thousands of full-size
hair-straightening brushes and rebranded them as beard brushes. Then, in
the fall of 2019, it introduced a compact straightening brush designed
for beards,
\href{https://www.thebeardstruggle.com/collections/beard-care-products/products/ulfberht-heated-beard-comb}{the
Vaeringjar}, \$97, with a heated pick on one end to add volume.
According to Faiysal Kothiwala, the founder of the Beard Struggle, the
company has sold about 63,000 of them.

Judd Curtis, 37, lives in Portland, Ore., and works as a tree trimmer
keeping paths clear for power lines. But while Mr. Curtis has no trouble
getting a Douglas fir under control, his beard is another story.

``I have a cowlick on my chin, this one spot where it kind of grows
straight out,'' he said. ``And my beard's just super-bushy, and it kind
of goes all over the place.''

Last July, Mr. Curtis bought a beard-straightening brush on Amazon made
by Arkam, \$40, and left
\href{https://www.amazon.com/gp/customer-reviews/R37IKYCPAE9X8W/ref=cm_cr_getr_d_rvw_ttl?ie=UTF8\&ASIN=B07Q5C9645}{a
five-star review}.

``It now takes me about three minutes in the morning to go from looking
like a crazy hobo to a whimsical woodsman,'' he wrote.

While popular, beard straighteners have not won over everyone.

Greg Berzinsky, 59, an architect in Philadelphia who makes grooming
videos on the popular
\href{https://www.youtube.com/user/TheBeardbrand/featured}{Beardbrand
channel} on YouTube, has tried a beard-straightening brush on his
salt-and-pepper beard. But he prefers styling with a blow dryer and a
rounded brush, which he turns downward and inward as he dries his beard
so it follows his jawline.

Mr. Berzinsky demonstrates this method in
\href{https://www.youtube.com/watch?v=qu5PxhEUMfU\&t=244s}{a video} that
has almost 3.5 million YouTube views and thousands of comments admiring
his beard. (``I don't have daddy issues but \ldots{}'' begins one
comment. ``Sweet Lord \ldots{} Hello Daddy.'')

``It's really to control the bottom fringe of my beard,'' Mr. Berzinsky
said of his approach, adding that he's not straightening his beard as
much as he's adding volume to it, which is helped along by some
Beardbrand \href{https://www.beardbrand.com/products/sea-salt-spray}{Sea
Salt Spray}.

``Facial hair is very different from the hair that grows on top of your
head,'' said Matty Conrad, the founder of throwback barbershops in
British Columbia and a line of men's grooming products,
\href{https://victorybarber.com/}{Victory Barber \& Brand}. ``Not to be
crass or anything, but it's more akin to pubic hair.''

Mr. Conrad said that about 10 percent of his clients with beards are
heat-styling them with a thermal brush or hair dryer, but that's not
what he generally recommends, or the approach he takes with his own wavy
beard.

``I'm used to seeing a certain texture and a certain volume to my
beard,'' Mr. Conrad said. ``As soon as I groom that out, it reduces the
volume, and it makes it really, really straight and almost
overgroomed.''

With barbershops closed, GQ recently featured Mr. Conrad in a
beard-trimming tutorial
\href{https://www.youtube.com/watch?v=nVY_hvAkmSY\&list=PLtJOh1iZIPQFPE0OReKe32vDZ0cinajkt}{video}.
Barbers typically take a buzz-cut approach to beards, the length
determined by the guard attached to the clippers.

But Mr. Conrad takes more of an Afro-trimming approach, focusing on the
silhouette. He uses clippers with no guard going down the sides of his
beard and along his chin --- straight down like a ball would drop, not
following the curve of the face.

At the close of the video, Mr. Conrad's beard is less red carpet than
``Game of Thrones,'' and he wouldn't have it any other way.

``I think there's an inherent coolness to having a bit of a rougher
beard,'' he said. ``If you are a guy that wears a suit every day and is
very polished, then a straightening brush probably makes sense with your
style, but I usually look like I just got off a horse, you know? I like
to embrace that wildness.''

Still, for those who look in a mirror in the morning and see a cave man
staring back, beard-straightening brushes have become indispensable. Mr.
Vilanova never packs his Dopp kit without a straightener. His Kuschelbär
joined him on his honeymoon. It went along for a trip to India and for
many business trips.

His wife likes his beard better after he straightens it but wonders if
he may be getting a bit obsessed.

``If I have to get ready to go somewhere,'' Mr. Vilanova said, ``she
says, `Oh, Matt needs an hour to get ready because he's got to
straighten his beard.'''

Advertisement

\protect\hyperlink{after-bottom}{Continue reading the main story}

\hypertarget{site-index}{%
\subsection{Site Index}\label{site-index}}

\hypertarget{site-information-navigation}{%
\subsection{Site Information
Navigation}\label{site-information-navigation}}

\begin{itemize}
\tightlist
\item
  \href{https://help.nytimes.com/hc/en-us/articles/115014792127-Copyright-notice}{©~2020~The
  New York Times Company}
\end{itemize}

\begin{itemize}
\tightlist
\item
  \href{https://www.nytco.com/}{NYTCo}
\item
  \href{https://help.nytimes.com/hc/en-us/articles/115015385887-Contact-Us}{Contact
  Us}
\item
  \href{https://www.nytco.com/careers/}{Work with us}
\item
  \href{https://nytmediakit.com/}{Advertise}
\item
  \href{http://www.tbrandstudio.com/}{T Brand Studio}
\item
  \href{https://www.nytimes.com/privacy/cookie-policy\#how-do-i-manage-trackers}{Your
  Ad Choices}
\item
  \href{https://www.nytimes.com/privacy}{Privacy}
\item
  \href{https://help.nytimes.com/hc/en-us/articles/115014893428-Terms-of-service}{Terms
  of Service}
\item
  \href{https://help.nytimes.com/hc/en-us/articles/115014893968-Terms-of-sale}{Terms
  of Sale}
\item
  \href{https://spiderbites.nytimes.com}{Site Map}
\item
  \href{https://help.nytimes.com/hc/en-us}{Help}
\item
  \href{https://www.nytimes.com/subscription?campaignId=37WXW}{Subscriptions}
\end{itemize}
