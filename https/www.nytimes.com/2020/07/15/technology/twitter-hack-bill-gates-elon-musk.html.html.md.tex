Sections

SEARCH

\protect\hyperlink{site-content}{Skip to
content}\protect\hyperlink{site-index}{Skip to site index}

\href{https://www.nytimes.com/section/technology}{Technology}

\href{https://myaccount.nytimes.com/auth/login?response_type=cookie\&client_id=vi}{}

\href{https://www.nytimes.com/section/todayspaper}{Today's Paper}

\href{/section/technology}{Technology}\textbar{}A Brazen Online Attack
Targets V.I.P. Twitter Users in a Bitcoin Scam

\url{https://nyti.ms/2ZtyzEO}

\begin{itemize}
\item
\item
\item
\item
\item
\item
\end{itemize}

Advertisement

\protect\hyperlink{after-top}{Continue reading the main story}

Supported by

\protect\hyperlink{after-sponsor}{Continue reading the main story}

\hypertarget{a-brazen-online-attack-targets-vip-twitter-users-in-a-bitcoin-scam}{%
\section{A Brazen Online Attack Targets V.I.P. Twitter Users in a
Bitcoin
Scam}\label{a-brazen-online-attack-targets-vip-twitter-users-in-a-bitcoin-scam}}

In a major show of force, hackers breached some of the site's most
prominent accounts, a Who's Who of Americans in politics, entertainment
and tech.

\includegraphics{https://static01.nyt.com/images/2020/07/15/business/15twitter-musk/merlin_173211669_d34241d2-2334-4677-a144-4646c7925b40-articleLarge.jpg?quality=75\&auto=webp\&disable=upscale}

By \href{https://www.nytimes.com/by/sheera-frenkel}{Sheera Frenkel},
\href{https://www.nytimes.com/by/nathaniel-popper}{Nathaniel Popper},
\href{https://www.nytimes.com/by/kate-conger}{Kate Conger} and
\href{https://www.nytimes.com/by/david-e-sanger}{David E. Sanger}

\begin{itemize}
\item
  Published July 15, 2020Updated July 17, 2020
\item
  \begin{itemize}
  \item
  \item
  \item
  \item
  \item
  \item
  \end{itemize}
\end{itemize}

It was about 4 in the afternoon on Wednesday on the East Coast when
chaos struck online. Dozens of the biggest names in America ---
including Joseph R. Biden Jr., Barack Obama, Kanye West, Bill Gates and
Elon Musk --- posted similar messages on Twitter: Send Bitcoin and the
famous people would send back double your money.

It was all a scam, of course, the result of one of the most brazen
online attacks in memory.

A first wave of attacks hit the Twitter accounts of prominent
cryptocurrency leaders and companies. But soon after, the list of
victims broadened to include a Who's Who of Americans in politics,
entertainment and tech, in a major show of force by the
\href{https://www.nytimes.com/2020/07/17/technology/twitter-hackers-interview.html}{hackers}.

Twitter quickly removed many of the messages, but in some cases similar
tweets were sent again from the same accounts, suggesting that Twitter
was powerless to regain control.

The company eventually disabled broad swaths of its service, including
the ability of verified users to tweet, for a couple of hours as it
scrambled to prevent the scam from spreading further. The company sent a
tweet saying that it was investigating the problem and looking for a
fix. ``You may be unable to Tweet or reset your password while we review
and address this incident,'' the company said in a second tweet. Service
was restored around 8:30 Wednesday night.

Twitter's investigation into the breach revealed that several employees
who had access to internal systems had their accounts compromised in a
``coordinated social engineering attack,'' a spokesman said, referring
to attacks that trick people into giving up their credentials. The
attackers then used Twitter's internal systems to tweet from
high-profile accounts like Mr. Biden's.

``We're looking into what other malicious activity they may have
conducted or information they may have accessed,'' Twitter's spokesman
added. ``We've taken significant steps to limit access to internal
systems and tools while our investigation is ongoing.''

Jack Dorsey, Twitter's chief executive,
\href{https://twitter.com/jack/status/1283571658339397632?s=21}{said in
a post} Wednesday night that it was a ``tough day for us at Twitter. We
all feel terrible this happened. We're diagnosing and will share
everything we can when we have a more complete understanding of exactly
what happened.''

\includegraphics{https://static01.nyt.com/images/2020/07/15/business/15twitter-biden/15twitter-biden-articleLarge-v2.jpg?quality=75\&auto=webp\&disable=upscale}

The hackers did not use their access to take aim at any important
institutions or infrastructure --- instead just asking for Bitcoin. But
the attack was concerning to security experts because it suggested that
the hackers could have easily caused much more havoc.

There was little immediate evidence for who conducted the attack. One of
the most obvious culprits for an attack of this scale, North Korea, has
been documented to have used Bitcoin extensively in the past. But its
nature --- ``effective, but also amateurish'' in the words of one senior
American intelligence official --- led American intelligence agencies to
an initial assessment that this was most likely the work of an
individual hacker, not a state.

Had it been Russia, China, North Korea or Iran, said the official, who
would not speak on the record because they were not authorized to
discuss an intelligence investigation, the effort would have probably
focused on trying to trigger stock market havoc, or perhaps the issuance
of political pronouncements in the name of Mr. Biden or other targets.

Officials also noted that the breach did not affect the account of one
of the most watched and powerful users of Twitter: President Trump. Mr.
Trump's account is under a special kind of lock-and-key after past
incidents, the official noted.

Security experts said that the wide-ranging attacks hinted that the
problem was caused by a security flaw in Twitter's service, not by lax
security measures used by the people who were targeted. Alex Stamos,
director of the Stanford Internet Observatory and the former chief
security officer at Facebook, said there were a range of other theories,
but all suggested that the attackers got inside Twitter's system, rather
than stealing the passwords of individual users.

One American official called that a ``scary possibility'' in a world
where national leaders, sometimes imitating Mr. Trump's techniques, have
adopted Twitter as a primary source of unfiltered communications.

``It could have been much worse. We got lucky that this is what they
decided to do with their power,'' Mr. Stamos said.

The hacker or hackers made some rookie errors. Mr. Stamos said that
because the attackers had sent identical messages from the compromised
accounts, they were easy to detect and delete. The decision to ask for
money through Bitcoin, he added, showed that the attackers were most
likely unable or unwilling to launder money or use their access for a
more sophisticated scam.

The messages were a version of a long-running scam in which hackers pose
as public figures on Twitter, and promise to match or even triple any
funds that are sent to their Bitcoin wallets. But the attacks Wednesday
were the first time that the real accounts of public figures were used
in such a scam.

Bitcoin is a popular vehicle for this type of scam because once a victim
sends money, the design of Bitcoin, with no institution in charge, makes
it essentially impossible to recover the funds.

By Wednesday evening, the Bitcoin wallets promoted in the tweets had
received over 300 transactions and Bitcoin worth over \$100,000,
according to websites that track Bitcoin's public ledger of
transactions,
\href{https://www.nytimes.com/2018/06/27/business/dealbook/blockchains-guide-information.html}{which
is known as the blockchain}.

\hypertarget{118000-in-three-hours}{%
\subsection{\$118,000 in Three Hours}\label{118000-in-three-hours}}

A scam on Twitter was propelled into the mainstream after hackers took
control of several high-profile accounts and directed their followers to
send them Bitcoin with a promise that they would double the amount.

@KimKardashian

@FloydMayweather

\$118,000

Money sent into

Bitcoin wallet

as of 7 p.m.

\$100,000

@MikeBloomberg

@JeffBezos

@kanyewest

@BarackObama

\$50,000

Tweets from

hacked accounts

@WarrenBuffett

@JoeBiden

@ElonMusk

@Uber

Twitter locks down some

accounts shortly after 6 p.m.

@BillGates

@Apple

4 p.m.

5 p.m.

6 p.m.

7 p.m.

@KimKardashian

@FloydMayweather

\$118,000

\$100,000

Money sent into

Bitcoin wallet

as of 7 p.m.

@MikeBloomberg

@JeffBezos

@kanyewest

@BarackObama

\$50,000

Tweets from

hacked

accounts

@WarrenBuffett

@JoeBiden

Twitter locks down

some accounts

shortly after 6 p.m.

@ElonMusk

@Uber

@BillGates

@Apple

4 p.m.

5 p.m.

6 p.m.

7 p.m.

@KimKardashian

@FloydMayweather

\$118,000

Money sent into

Bitcoin wallet

as of 7 p.m.

\$100,000

@MikeBloomberg

Tweets from

hacked accounts

@JeffBezos

@BarackObama

@kanyewest

\$50,000

@JoeBiden

@WarrenBuffett

@ElonMusk

@Uber

Twitter locks down some

accounts shortly after 6 p.m.

@BillGates

@Apple

4 p.m.

5 p.m.

6 p.m.

7 p.m.

Source: \href{https://blockchair.com/}{Blockchair}

Note: All times Eastern. By Matthew Conlen and Lazaro Gamio

Twitter initially handled the attacks by taking down the offending
tweets. A spokesman for the Biden campaign said that Twitter had removed
the tweet promoting the scam and locked down Mr. Biden's account.

But the hackers kept control of many of the accounts, such as those of
Mr. Musk and Mr. West, and sent out new messages as soon as the old ones
were taken down.

As Twitter locked down verified accounts in an attempt to stop the
attack, the company also hampered its function as a real-time news
service. Derrick Snyder, a meteorologist in Kentucky, said in a
\href{https://twitter.com/Derrick_Snyder/status/1283529433689792513}{series
of tweets} that the National Weather Service could not issue warnings on
Twitter about a tornado in Illinois because its account, one that the
company had verified, was shut down.

``What a mess,'' Mr. Snyder wrote. ``There is a tornado warning in
effect.''

Twitter has fallen victim to breaches before. Last August, hackers
compromised the account of Twitter's chief, Mr. Dorsey, and posted
racist messages and bomb threats. His account was taken over after
hackers transferred his phone number to a new SIM card, which stores a
phone's number. The practice, known as SIM-swapping, allowed hackers to
tweet from Mr. Dorsey's account.

In 2017, a rogue worker at the company used their access to Twitter's
systems to briefly
\href{https://www.nytimes.com/2017/11/03/technology/trump-twitter-deleted.html}{delete
President Trump's Twitter account}. The account was restored within
minutes, but the incident raised questions about Twitter's security as
it serves as a megaphone for politicians and celebrities.

And in 2010, Twitter
\href{https://www.ftc.gov/news-events/press-releases/2010/06/twitter-settles-charges-it-failed-protect-consumers-personal}{settled
a complaint} brought by the Federal Trade Commission, in which the
regulator claimed that the company did not do enough to protect users'
personal information. The F.T.C. charged that ``serious lapses'' in
Twitter's security allowed hackers to take control of company systems
and send out phony tweets from high-profile accounts, including Mr.
Obama's. As part of the settlement, Twitter agreed to undergo security
audits for 10 years.

On Wednesday evening, Senator Josh Hawley, a Republican from Missouri,
wrote a letter to Mr. Dorsey asking for information on the attack,
including how many users were compromised.

Shares in the social media company fell 3 percent in after-hours
trading.

Cybersecurity experts said the attack showed how vulnerable social media
remains to attacks.

``This demonstrates a real risk for the elections,'' Mr. Stamos said.
``Twitter has become the most important platform when it comes to
discussion among political elites, and it has real vulnerabilities.''

Advertisement

\protect\hyperlink{after-bottom}{Continue reading the main story}

\hypertarget{site-index}{%
\subsection{Site Index}\label{site-index}}

\hypertarget{site-information-navigation}{%
\subsection{Site Information
Navigation}\label{site-information-navigation}}

\begin{itemize}
\tightlist
\item
  \href{https://help.nytimes.com/hc/en-us/articles/115014792127-Copyright-notice}{©~2020~The
  New York Times Company}
\end{itemize}

\begin{itemize}
\tightlist
\item
  \href{https://www.nytco.com/}{NYTCo}
\item
  \href{https://help.nytimes.com/hc/en-us/articles/115015385887-Contact-Us}{Contact
  Us}
\item
  \href{https://www.nytco.com/careers/}{Work with us}
\item
  \href{https://nytmediakit.com/}{Advertise}
\item
  \href{http://www.tbrandstudio.com/}{T Brand Studio}
\item
  \href{https://www.nytimes.com/privacy/cookie-policy\#how-do-i-manage-trackers}{Your
  Ad Choices}
\item
  \href{https://www.nytimes.com/privacy}{Privacy}
\item
  \href{https://help.nytimes.com/hc/en-us/articles/115014893428-Terms-of-service}{Terms
  of Service}
\item
  \href{https://help.nytimes.com/hc/en-us/articles/115014893968-Terms-of-sale}{Terms
  of Sale}
\item
  \href{https://spiderbites.nytimes.com}{Site Map}
\item
  \href{https://help.nytimes.com/hc/en-us}{Help}
\item
  \href{https://www.nytimes.com/subscription?campaignId=37WXW}{Subscriptions}
\end{itemize}
