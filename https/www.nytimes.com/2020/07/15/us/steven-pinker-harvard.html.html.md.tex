Sections

SEARCH

\protect\hyperlink{site-content}{Skip to
content}\protect\hyperlink{site-index}{Skip to site index}

\href{https://www.nytimes.com/section/us}{U.S.}

\href{https://myaccount.nytimes.com/auth/login?response_type=cookie\&client_id=vi}{}

\href{https://www.nytimes.com/section/todayspaper}{Today's Paper}

\href{/section/us}{U.S.}\textbar{}How a Famous Harvard Professor Became
a Target Over His Tweets

\url{https://nyti.ms/2OqePvg}

\begin{itemize}
\item
\item
\item
\item
\item
\item
\end{itemize}

Advertisement

\protect\hyperlink{after-top}{Continue reading the main story}

Supported by

\protect\hyperlink{after-sponsor}{Continue reading the main story}

\hypertarget{how-a-famous-harvard-professor-became-a-target-over-his-tweets}{%
\section{How a Famous Harvard Professor Became a Target Over His
Tweets}\label{how-a-famous-harvard-professor-became-a-target-over-his-tweets}}

The outcry over free speech and race takes aim at Steven Pinker, the
best-selling author and well-known scholar.

\includegraphics{https://static01.nyt.com/images/2020/07/13/us/13pinker/13pinker-articleLarge-v2.jpg?quality=75\&auto=webp\&disable=upscale}

\href{https://topics.nytimes.com/topics/reference/timestopics/people/p/michael_powell/index.html}{\includegraphics{https://static01.nyt.com/images/2018/02/20/multimedia/author-michael-powell/author-michael-powell-thumbLarge.jpg}}

By
\href{https://topics.nytimes.com/topics/reference/timestopics/people/p/michael_powell/index.html}{Michael
Powell}

\begin{itemize}
\item
  Published July 15, 2020Updated July 22, 2020
\item
  \begin{itemize}
  \item
  \item
  \item
  \item
  \item
  \item
  \end{itemize}
\end{itemize}

Steven Pinker occupies a role that is rare in American life: the
celebrity intellectual. The Harvard professor pops up on outlets from
PBS to the Joe Rogan podcast, translating dense subjects into accessible
ideas with enthusiasm. Bill Gates called
\href{https://stevenpinker.com/publications/enlightenment-now-case-reason-science-humanism-and-progress}{his
most recent book} ``my new favorite book of all time.''

So when more than 550 academics recently
\href{https://docs.google.com/document/u/1/d/17ZqWl5grm_F5Kn_0OarY9Q2jlOnk200PvhM5e3isPvY/mobilebasic?urp=gmail_link}{signed
a letter} seeking to remove him from the list of ``distinguished
fellows'' of the Linguistic Society of America, it drew attention to
their provocative charge: that Professor Pinker minimizes racial
injustices and drowns out the voices of those who suffer sexist and
racist indignities.

But the letter was striking for another reason: It took aim not at
Professor Pinker's scholarly work but at six of his tweets dating back
to 2014, and at a two-word phrase he used in a 2011 book about a
centuries-long decline in violence.

``Dr. Pinker has a history of speaking over genuine grievances and
downplaying injustices, frequently by misrepresenting facts, and at the
exact moments when Black and Brown people are mobilizing against
systemic racism and for crucial changes,'' their letter stated.

The linguists demanded that the society revoke Professor Pinker's status
as a ``distinguished fellow'' and strike his name from its list of media
experts. The society's executive committee declined to do so last week,
stating: ``It is not the mission of the society to control the opinions
of its members, nor their expression.''

But a charge of racial insensitivity carries power in the current
climate, and the letter sounded another shot in the fraught cultural
battles now erupting in academia and publishing.

Also this month, 153 intellectuals and writers --- many of them
political liberals ---
\href{https://www.nytimes.com/2020/07/07/arts/harpers-letter.html}{signed
a letter} in Harper's Magazine that criticized the current intellectual
climate as ``constricted'' and ``intolerant.'' That
\href{https://www.nytimes.com/2020/07/10/arts/open-letter-debate.html}{led
to a fiery response} from opposing liberal and leftist writers, who
accused the Harper's letter writers of elitism and hypocrisy.

In an era of polarizing ideologies, Professor Pinker, a linguist and
social psychologist, is tough to pin down. He is a big supporter of
Democrats, and donated heavily to former President Barack Obama, but he
has denounced what he sees as the close-mindedness of heavily liberal
American universities. He likes to publicly entertain
\href{https://newrepublic.com/article/77727/groups-and-genes}{ideas}
outside the academic mainstream, including the question of innate
differences between the sexes and among different ethnic and racial
groups. And he has suggested that the political left's insistence that
certain subjects are off limits contributed to the rise of the
alt-right.

Reached at his home on Cape Cod, Professor Pinker, 65, noted that as a
tenured faculty member and established author, he could weather the
campaign against him. But he said it could chill junior faculty who hold
views counter to prevailing intellectual currents.

``I have a mind-set that the world is a complex place we are trying to
understand,'' he said. ``There is an inherent value to free speech,
because no one knows the solution to problems a priori.''

He described his critics as ``speech police'' who ``have trawled through
my writings to find offensive lines and adjectives.''

The letter against him focuses mainly on his activity on Twitter, where
he has some 600,000 followers. It points to his 2015 tweet of an
\href{https://www.nytimes.com/2015/10/18/upshot/police-killings-of-blacks-what-the-data-says.html?smid=tw-share}{article}
from The Upshot, the data and analysis-focused team at The New York
Times, which suggested that the high number of police shootings of Black
people may not have been caused by racial bias of individual police
officers, but rather by the larger structural and economic realities
that result in the police having disproportionately high numbers of
encounters with Black residents.

``Data: Police don't shoot blacks disproportionately,''
\href{https://twitter.com/sapinker/status/655389531429064704}{Professor
Pinker tweeted} with a link to the article. ``Problem: Not race, but too
many police shootings.''

The linguists' letter noted that the article made plain that police
killings are a racial problem, and accused Professor Pinker of making
``dishonest claims in order to obfuscate the role of systemic racism in
police violence.''

But the article also suggested that, because every encounter with the
police carries danger of escalation, any racial group interacting with
the police frequently risked becoming victims of police violence, due to
poorly trained officers, armed suspects or overreaction. That appeared
to be the point of Professor Pinker's tweet.

The linguists' letter also accused the professor of engaging in racial
dog whistles when he used the words
``\href{https://twitter.com/sapinker/status/1272150637237751813}{urban
crime}'' and
``\href{https://twitter.com/sapinker/status/1272145748050796544}{urban
violence}'' in other tweets.

But in those tweets, Professor Pinker had linked to the work of scholars
who are widely described as experts on urban crime and urban violence
and its decline.

```Urban' appears to be a usual terminological choice in work in
sociology, political science, law and criminology,''
\href{https://medium.com/@bhpartee/my-response-to-the-pinker-petition-open-letter-to-the-linguistics-community-80e2e4d9dbe2}{wrote}
Jason Merchant, vice provost and a linguistics professor at the
University of Chicago, who defended Professor Pinker.

Another issue, Professor Pinker's critics say, is contained in his 2011
book, ``The Better Angels of Our Nature: Why Violence Has Declined.'' In
a wide-ranging description of crime and urban decay and its effect on
the culture of the 1970s and 1980s, he wrote that ``Bernhard Goetz, a
mild-mannered engineer, became a folk hero for shooting four young
muggers in a New York subway car.''

The linguists' letter took strong issue with the words
``mild-mannered,'' noting that a neighbor later said that Mr. Goetz had
spoken in racist terms of Latinos and Black people. He was not
``mild-mannered'' but rather intent on confrontation, they said.

The origin of the letter remains a mystery. Of 10 signers contacted by
The Times, only one hinted that she knew the identity of the authors.
Many of the linguists proved shy about talking, and since the letter
first surfaced on Twitter on July 3, several prominent linguists have
said their names had been included without their knowledge.

Several department chairs in linguistics and philosophy signed the
letter, including Professor Barry Smith of the University at Buffalo and
Professor Lisa Davidson of New York University. Professor Smith did not
return calls and an email and Professor Davidson declined to comment
when The Times reached out.

The linguists' letter touched only lightly on questions that have proved
storm-tossed for Professor Pinker in the past. In the debate over
whether nature or nurture shapes human behavior, he has leaned toward
nature, arguing that characteristics like psychological traits and
intelligence are to some degree
\href{https://www.nytimes.com/2009/01/11/magazine/11Genome-t.html}{heritable}.

He has also suggested that underrepresentation in the sciences could be
rooted in part in biological differences between men and women. (He
defended Lawrence Summers, the former Harvard president who in 2005
speculated that innate differences between the sexes might in part
explain why fewer women succeed in science and math careers. Mr.
Summers's remark infuriated some female scientists and was among several
controversies that led to his resignation the following year.)

And Professor Pinker has made high-profile blunders, such as when he
provided his expertise on language for the 2007 defense of the financier
Jeffrey Epstein on sex trafficking charges. He has said he did so free
of charge and at the request of a friend, the Harvard law professor Alan
Dershowitz, and regrets it.

The clash may also reflect the fact that Professor Pinker's rosy outlook
--- he argues that the world is becoming a better place, by almost any
measure, from poverty to literacy --- sounds discordant during this
painful moment of national reckoning with the still-ugly scars of racism
and inequality.

The linguists' society, like many academic and nonprofit organizations,
recently released a wide-ranging statement calling for greater diversity
in the field. It also urged linguists to confront how their research
``might reproduce or work against racism.''

John McWhorter, a Columbia University professor of English and
linguistics, cast the Pinker controversy within a moment when, he said,
progressives look suspiciously at anyone who does not embrace the
politics of racial and cultural identity.

``Steve is too big for this kerfuffle to affect him,'' Professor
McWhorter said. ``But it's depressing that an erudite and reasonable
scholar is seen by a lot of intelligent people as an undercover
monster.''

Because this is a fight involving linguists, it features some expected
elements: intense arguments about imprecise wording and sly intellectual
put-downs. Professor Pinker may have inflamed matters when he suggested
in response to the letter that its signers lacked stature. ``I recognize
only one name among the signatories,'' he tweeted. Such an argument,
Byron T. Ahn, a linguistics professor at Princeton, wrote in a tweet of
his own, amounted to ``a kind of indirect ad hominem attack.''

The linguists insisted they were not attempting to censor Professor
Pinker. Rather, they were intent on showing that he had been deceitful
and used racial dog whistles, and thus, was a disreputable
representative for linguistics.

``Any resulting action from this letter may make it clear to Black
scholars that the L.S.A. is sensitive to the impact that tweets of this
sort have on maintaining structures that we should be attempting to
dismantle,'' wrote Professor David Adger of Queen Mary University of
London on his website.

That line of argument left Professor McWhorter, a signer of the letter
in Harper's, exasperated.

``We're in this moment that's like a collective mic drop, and civility
and common sense go out the window,'' he said. ``It's enough to cry
racism or sexism, and that's that.''

Advertisement

\protect\hyperlink{after-bottom}{Continue reading the main story}

\hypertarget{site-index}{%
\subsection{Site Index}\label{site-index}}

\hypertarget{site-information-navigation}{%
\subsection{Site Information
Navigation}\label{site-information-navigation}}

\begin{itemize}
\tightlist
\item
  \href{https://help.nytimes.com/hc/en-us/articles/115014792127-Copyright-notice}{©~2020~The
  New York Times Company}
\end{itemize}

\begin{itemize}
\tightlist
\item
  \href{https://www.nytco.com/}{NYTCo}
\item
  \href{https://help.nytimes.com/hc/en-us/articles/115015385887-Contact-Us}{Contact
  Us}
\item
  \href{https://www.nytco.com/careers/}{Work with us}
\item
  \href{https://nytmediakit.com/}{Advertise}
\item
  \href{http://www.tbrandstudio.com/}{T Brand Studio}
\item
  \href{https://www.nytimes.com/privacy/cookie-policy\#how-do-i-manage-trackers}{Your
  Ad Choices}
\item
  \href{https://www.nytimes.com/privacy}{Privacy}
\item
  \href{https://help.nytimes.com/hc/en-us/articles/115014893428-Terms-of-service}{Terms
  of Service}
\item
  \href{https://help.nytimes.com/hc/en-us/articles/115014893968-Terms-of-sale}{Terms
  of Sale}
\item
  \href{https://spiderbites.nytimes.com}{Site Map}
\item
  \href{https://help.nytimes.com/hc/en-us}{Help}
\item
  \href{https://www.nytimes.com/subscription?campaignId=37WXW}{Subscriptions}
\end{itemize}
