Sections

SEARCH

\protect\hyperlink{site-content}{Skip to
content}\protect\hyperlink{site-index}{Skip to site index}

\href{https://www.nytimes.com/section/world/europe}{Europe}

\href{https://myaccount.nytimes.com/auth/login?response_type=cookie\&client_id=vi}{}

\href{https://www.nytimes.com/section/todayspaper}{Today's Paper}

\href{/section/world/europe}{Europe}\textbar{}For U.K.'s Minority Women,
Economic Toll of Lockdown Lingers

\url{https://nyti.ms/3fwOo3f}

\begin{itemize}
\item
\item
\item
\item
\item
\end{itemize}

\href{https://www.nytimes.com/news-event/coronavirus?action=click\&pgtype=Article\&state=default\&region=TOP_BANNER\&context=storylines_menu}{The
Coronavirus Outbreak}

\begin{itemize}
\tightlist
\item
  live\href{https://www.nytimes.com/2020/08/01/world/coronavirus-covid-19.html?action=click\&pgtype=Article\&state=default\&region=TOP_BANNER\&context=storylines_menu}{Latest
  Updates}
\item
  \href{https://www.nytimes.com/interactive/2020/us/coronavirus-us-cases.html?action=click\&pgtype=Article\&state=default\&region=TOP_BANNER\&context=storylines_menu}{Maps
  and Cases}
\item
  \href{https://www.nytimes.com/interactive/2020/science/coronavirus-vaccine-tracker.html?action=click\&pgtype=Article\&state=default\&region=TOP_BANNER\&context=storylines_menu}{Vaccine
  Tracker}
\item
  \href{https://www.nytimes.com/interactive/2020/07/29/us/schools-reopening-coronavirus.html?action=click\&pgtype=Article\&state=default\&region=TOP_BANNER\&context=storylines_menu}{What
  School May Look Like}
\item
  \href{https://www.nytimes.com/live/2020/07/31/business/stock-market-today-coronavirus?action=click\&pgtype=Article\&state=default\&region=TOP_BANNER\&context=storylines_menu}{Economy}
\end{itemize}

Advertisement

\protect\hyperlink{after-top}{Continue reading the main story}

Supported by

\protect\hyperlink{after-sponsor}{Continue reading the main story}

\hypertarget{for-uks-minority-women-economic-toll-of-lockdown-lingers}{%
\section{For U.K.'s Minority Women, Economic Toll of Lockdown
Lingers}\label{for-uks-minority-women-economic-toll-of-lockdown-lingers}}

Black and other ethnic minority groups have long faced economic and
racial inequality in Britain. As workers return, many are saddled with
debt and working longer hours for less pay.

\includegraphics{https://static01.nyt.com/images/2020/06/30/world/xxvirus-ukwomen/merlin_172717803_8b309efd-e002-4566-9fcb-bef12b9d3d60-articleLarge.jpg?quality=75\&auto=webp\&disable=upscale}

\href{https://www.nytimes.com/by/ceylan-yeginsu}{\includegraphics{https://static01.nyt.com/images/2018/10/17/multimedia/author-ceylan-yeginsu/author-ceylan-yeginsu-thumbLarge-v2.png}}

By \href{https://www.nytimes.com/by/ceylan-yeginsu}{Ceylan Yeginsu}

\begin{itemize}
\item
  July 15, 2020
\item
  \begin{itemize}
  \item
  \item
  \item
  \item
  \item
  \end{itemize}
\end{itemize}

LONDON --- Within 10 days of the
\href{https://www.nytimes.com/2020/03/24/world/europe/britain-coronavirus-lockdown.html}{British
government's lockdown announcement} in late March, one woman lost all
nine of her cleaning jobs. Another was laid off from a laundromat after
she requested a mask, and a live-in nanny was fired for using public
transport on her day off.

In separate interviews, the three women said that they had expected to
confront hardships during the lockdown. But as the economy
\href{https://www.nytimes.com/2020/05/11/world/europe/coronavirus-uk-boris-johnson.html}{starts
to reopen}, they and other women on the lower rungs of the economy say
they are still struggling, weighed down by debts accumulated during the
freeze and often facing pay cuts or forced to do more work for the same
wages.

The three have one other thing in common. They are all women of color, a
group that has long faced economic and racial inequality in Britain and
is now being hit disproportionately by the financial and psychological
impacts of the coronavirus pandemic,
\href{https://www.fawcettsociety.org.uk/Handlers/Download.ashx?IDMF=cae4917f-1df3-4ab8-94e7-550c23bdc9cf}{according
to a recent study} by a group of British universities and women's
charities.

``Covid-19 has brought the harsh realities of pre-existing racial
inequalities into sharp relief, and nowhere is this more manifest than
the disproportionate social and economic impact of Covid-19 on Black and
ethnic minority women,'' said Zubaida Haque, the interim director of the
Runnymede Trust, a London-based organization advocating racial equality.

The main reason that people of color are so vulnerable, experts say, is
that they are more likely to be in precarious employment or to become
unemployed, making it harder for them to qualify for government support
and to protect themselves from the virus.

Minji Paik, a Korean beautician who works in a hair salon in East
London, said she made 15 pounds an hour before the pandemic, about \$19
dollars, plus tips. Now she is making £10 an hour, and has been working
longer shifts because of staff shortages.

``My manager says this is temporary and she will give me more money when
we make money,'' Ms. Paik said. ``But actually, I should be paid more
because I'm working inside and risking my health.''

\href{https://www.gov.uk/government/publications/covid-19-understanding-the-impact-on-bame-communities}{A
government review} of the disparities in the risk and outcomes from the
coronavirus found that death rates have been higher in Black, Asian and
other minority ethnic groups than in white groups. The review found that
Chinese, Indian, Pakistani and other Asians, as well as Caribbeans and
other Black people, had from 10 percent to 50 percent higher risk of
death than white Britons.

``There's often a risk when people start talking about the underlying
causes of death because of the assumption that the reason is related to
genetics or poor diets,'' said Bridget Byrne, director of the
\href{https://www.ethnicity.ac.uk/}{Center on Dynamics of Ethnicity} at
the University of Manchester. ``But actually, you need to look at the
wider process of racism and the structuring of race and deprivation.''

\includegraphics{https://static01.nyt.com/images/2020/06/29/world/xxvirus-ukwomen2/merlin_171379209_b0b24c07-54e9-497d-bd5d-680d583ec2ce-articleLarge.jpg?quality=75\&auto=webp\&disable=upscale}

The precarious nature of the current labor market is a contributing
factor as well, Ms. Byrne said. ``It makes people less willing to voice
their concerns, they worry that if they say, `I don't feel safe, I don't
think I should be coming in,' they will be the first to be laid off.''

\hypertarget{latest-updates-global-coronavirus-outbreak}{%
\section{\texorpdfstring{\href{https://www.nytimes.com/2020/08/01/world/coronavirus-covid-19.html?action=click\&pgtype=Article\&state=default\&region=MAIN_CONTENT_1\&context=storylines_live_updates}{Latest
Updates: Global Coronavirus
Outbreak}}{Latest Updates: Global Coronavirus Outbreak}}\label{latest-updates-global-coronavirus-outbreak}}

Updated 2020-08-02T07:42:09.613Z

\begin{itemize}
\tightlist
\item
  \href{https://www.nytimes.com/2020/08/01/world/coronavirus-covid-19.html?action=click\&pgtype=Article\&state=default\&region=MAIN_CONTENT_1\&context=storylines_live_updates\#link-34047410}{The
  U.S. reels as July cases more than double the total of any other
  month.}
\item
  \href{https://www.nytimes.com/2020/08/01/world/coronavirus-covid-19.html?action=click\&pgtype=Article\&state=default\&region=MAIN_CONTENT_1\&context=storylines_live_updates\#link-780ec966}{Top
  U.S. officials work to break an impasse over the federal jobless
  benefit.}
\item
  \href{https://www.nytimes.com/2020/08/01/world/coronavirus-covid-19.html?action=click\&pgtype=Article\&state=default\&region=MAIN_CONTENT_1\&context=storylines_live_updates\#link-2bc8948}{Its
  outbreak untamed, Melbourne goes into even greater lockdown.}
\end{itemize}

\href{https://www.nytimes.com/2020/08/01/world/coronavirus-covid-19.html?action=click\&pgtype=Article\&state=default\&region=MAIN_CONTENT_1\&context=storylines_live_updates}{See
more updates}

More live coverage:
\href{https://www.nytimes.com/live/2020/07/31/business/stock-market-today-coronavirus?action=click\&pgtype=Article\&state=default\&region=MAIN_CONTENT_1\&context=storylines_live_updates}{Markets}

Candice Brown, 48, a cleaner who is of Jamaican descent, said she had
lost all her clients when
\href{https://www.nytimes.com/2020/03/23/world/europe/coronavirus-uk-boris-johnson.html}{lockdown
measures were imposed in March}.

``They phoned me one by one to say don't come,'' she recalled, referring
to the owners of the nine houses she cleaned each week in the city of
Manchester, in northwestern England. ``Each call was like a bomb,
blasting every bit of my livelihood until I had no work left.''

For two months she tried to navigate the government's financial support
system for those affected by the pandemic, and even borrowed money from
a friend to hire an accountant to help. But eventually, she found out
that she was not eligible for any aid because she lacked the paperwork
to prove her employment history.

``I applied for universal credit,'' she said referring to the
government's income support program. ``But I am still waiting. I haven't
received a penny.''

Even with the easing of the lockdown measures, Ms. Brown has not been
invited back to work because her employers fear that she could contract
and spread the virus by working between multiple households.

``I don't know how much longer I can go on like this,'' she said. ``In
the first month, I was worrying about how to pay my rent and my bills,
now I can't sleep worrying about how to feed my children.''

Image

Deserted streets in Manchester, England, in March. Within 10 days of the
British government's announcement of a lockdown, Candice Brown lost all
nine of her cleaning jobs in the city.Credit...Mary Turner for The New
York Times

A
\href{https://www.fawcettsociety.org.uk/coronavirus-impact-on-bame-women}{survey
published by the Fawcett Society}, a women's rights charity, found that
nearly 43 percent of Black and ethnic minority women believed that they
would be in more debt than before the pandemic, compared with 37 percent
of white women and 34 percent of white men. More than four in 10 of the
women said they would struggle to make ends meet over the next three
months.

Many are tasked with carrying out menial tasks that can be perilous in a
pandemic,
\href{https://www.runnymedetrust.org/uploads/publications/pdfs/2020\%20reports/The\%20Colour\%20of\%20Money\%20Report.pdf}{a
recent study by the Runnymede Trust found}.

Zuhr Rind, 48, a Pakistani laundromat worker in East London, was asked
to work the last shift when the pandemic broke out so that she could
wash the uniforms of the front-of-house employees, who collected laundry
from clients.

\href{https://www.nytimes.com/news-event/coronavirus?action=click\&pgtype=Article\&state=default\&region=MAIN_CONTENT_3\&context=storylines_faq}{}

\hypertarget{the-coronavirus-outbreak-}{%
\subsubsection{The Coronavirus Outbreak
›}\label{the-coronavirus-outbreak-}}

\hypertarget{frequently-asked-questions}{%
\paragraph{Frequently Asked
Questions}\label{frequently-asked-questions}}

Updated July 27, 2020

\begin{itemize}
\item ~
  \hypertarget{should-i-refinance-my-mortgage}{%
  \paragraph{Should I refinance my
  mortgage?}\label{should-i-refinance-my-mortgage}}

  \begin{itemize}
  \tightlist
  \item
    \href{https://www.nytimes.com/article/coronavirus-money-unemployment.html?action=click\&pgtype=Article\&state=default\&region=MAIN_CONTENT_3\&context=storylines_faq}{It
    could be a good idea,} because mortgage rates have
    \href{https://www.nytimes.com/2020/07/16/business/mortgage-rates-below-3-percent.html?action=click\&pgtype=Article\&state=default\&region=MAIN_CONTENT_3\&context=storylines_faq}{never
    been lower.} Refinancing requests have pushed mortgage applications
    to some of the highest levels since 2008, so be prepared to get in
    line. But defaults are also up, so if you're thinking about buying a
    home, be aware that some lenders have tightened their standards.
  \end{itemize}
\item ~
  \hypertarget{what-is-school-going-to-look-like-in-september}{%
  \paragraph{What is school going to look like in
  September?}\label{what-is-school-going-to-look-like-in-september}}

  \begin{itemize}
  \tightlist
  \item
    It is unlikely that many schools will return to a normal schedule
    this fall, requiring the grind of
    \href{https://www.nytimes.com/2020/06/05/us/coronavirus-education-lost-learning.html?action=click\&pgtype=Article\&state=default\&region=MAIN_CONTENT_3\&context=storylines_faq}{online
    learning},
    \href{https://www.nytimes.com/2020/05/29/us/coronavirus-child-care-centers.html?action=click\&pgtype=Article\&state=default\&region=MAIN_CONTENT_3\&context=storylines_faq}{makeshift
    child care} and
    \href{https://www.nytimes.com/2020/06/03/business/economy/coronavirus-working-women.html?action=click\&pgtype=Article\&state=default\&region=MAIN_CONTENT_3\&context=storylines_faq}{stunted
    workdays} to continue. California's two largest public school
    districts --- Los Angeles and San Diego --- said on July 13, that
    \href{https://www.nytimes.com/2020/07/13/us/lausd-san-diego-school-reopening.html?action=click\&pgtype=Article\&state=default\&region=MAIN_CONTENT_3\&context=storylines_faq}{instruction
    will be remote-only in the fall}, citing concerns that surging
    coronavirus infections in their areas pose too dire a risk for
    students and teachers. Together, the two districts enroll some
    825,000 students. They are the largest in the country so far to
    abandon plans for even a partial physical return to classrooms when
    they reopen in August. For other districts, the solution won't be an
    all-or-nothing approach.
    \href{https://bioethics.jhu.edu/research-and-outreach/projects/eschool-initiative/school-policy-tracker/}{Many
    systems}, including the nation's largest, New York City, are
    devising
    \href{https://www.nytimes.com/2020/06/26/us/coronavirus-schools-reopen-fall.html?action=click\&pgtype=Article\&state=default\&region=MAIN_CONTENT_3\&context=storylines_faq}{hybrid
    plans} that involve spending some days in classrooms and other days
    online. There's no national policy on this yet, so check with your
    municipal school system regularly to see what is happening in your
    community.
  \end{itemize}
\item ~
  \hypertarget{is-the-coronavirus-airborne}{%
  \paragraph{Is the coronavirus
  airborne?}\label{is-the-coronavirus-airborne}}

  \begin{itemize}
  \tightlist
  \item
    The coronavirus
    \href{https://www.nytimes.com/2020/07/04/health/239-experts-with-one-big-claim-the-coronavirus-is-airborne.html?action=click\&pgtype=Article\&state=default\&region=MAIN_CONTENT_3\&context=storylines_faq}{can
    stay aloft for hours in tiny droplets in stagnant air}, infecting
    people as they inhale, mounting scientific evidence suggests. This
    risk is highest in crowded indoor spaces with poor ventilation, and
    may help explain super-spreading events reported in meatpacking
    plants, churches and restaurants.
    \href{https://www.nytimes.com/2020/07/06/health/coronavirus-airborne-aerosols.html?action=click\&pgtype=Article\&state=default\&region=MAIN_CONTENT_3\&context=storylines_faq}{It's
    unclear how often the virus is spread} via these tiny droplets, or
    aerosols, compared with larger droplets that are expelled when a
    sick person coughs or sneezes, or transmitted through contact with
    contaminated surfaces, said Linsey Marr, an aerosol expert at
    Virginia Tech. Aerosols are released even when a person without
    symptoms exhales, talks or sings, according to Dr. Marr and more
    than 200 other experts, who
    \href{https://academic.oup.com/cid/article/doi/10.1093/cid/ciaa939/5867798}{have
    outlined the evidence in an open letter to the World Health
    Organization}.
  \end{itemize}
\item ~
  \hypertarget{what-are-the-symptoms-of-coronavirus}{%
  \paragraph{What are the symptoms of
  coronavirus?}\label{what-are-the-symptoms-of-coronavirus}}

  \begin{itemize}
  \tightlist
  \item
    Common symptoms
    \href{https://www.nytimes.com/article/symptoms-coronavirus.html?action=click\&pgtype=Article\&state=default\&region=MAIN_CONTENT_3\&context=storylines_faq}{include
    fever, a dry cough, fatigue and difficulty breathing or shortness of
    breath.} Some of these symptoms overlap with those of the flu,
    making detection difficult, but runny noses and stuffy sinuses are
    less common.
    \href{https://www.nytimes.com/2020/04/27/health/coronavirus-symptoms-cdc.html?action=click\&pgtype=Article\&state=default\&region=MAIN_CONTENT_3\&context=storylines_faq}{The
    C.D.C. has also} added chills, muscle pain, sore throat, headache
    and a new loss of the sense of taste or smell as symptoms to look
    out for. Most people fall ill five to seven days after exposure, but
    symptoms may appear in as few as two days or as many as 14 days.
  \end{itemize}
\item ~
  \hypertarget{does-asymptomatic-transmission-of-covid-19-happen}{%
  \paragraph{Does asymptomatic transmission of Covid-19
  happen?}\label{does-asymptomatic-transmission-of-covid-19-happen}}

  \begin{itemize}
  \tightlist
  \item
    So far, the evidence seems to show it does. A widely cited
    \href{https://www.nature.com/articles/s41591-020-0869-5}{paper}
    published in April suggests that people are most infectious about
    two days before the onset of coronavirus symptoms and estimated that
    44 percent of new infections were a result of transmission from
    people who were not yet showing symptoms. Recently, a top expert at
    the World Health Organization stated that transmission of the
    coronavirus by people who did not have symptoms was ``very rare,''
    \href{https://www.nytimes.com/2020/06/09/world/coronavirus-updates.html?action=click\&pgtype=Article\&state=default\&region=MAIN_CONTENT_3\&context=storylines_faq\#link-1f302e21}{but
    she later walked back that statement.}
  \end{itemize}
\end{itemize}

``I was not happy about it, but what could I do? Work is work and I was
afraid to lose my job if I made an argument,'' she said.

When she asked for a face mask, her manager chided her, she said.

``That is pathetic Z,'' her manager wrote back in a text message she
showed to The New York Times. ``Doctors and nurses do not even have
enough masks and they're still going to work.''

A day later, Ms. Rind said, she was laid off. ``When you have brown
skin, when you have an accent and when you don't have a high education,
you don't have choices,'' she said. ``And this is a very dangerous
situation to be in during Covid.''

The laundromat where she worked did not respond to a request for
comment.

Ms. Rind was recently offered a cleaning job in a hotel but was forced
to turn it down because she lives 40 minutes away and, as a precaution
against the virus, the employer did not want her to take public
transport.

Image

A mural in East London in May. Zuhr Rind, who worked at a laundromat in
the city, said she was laid off shortly after~asking her boss for a face
mask.Credit...Andrew Testa for The New York Times

Black and ethnic minority women also generally have much lower levels of
savings and assets than white Britons, according to the Runnymede Trust
study. So those who lost their jobs in the pandemic have had to seek new
employment straight away, forcing some of them to take lower-paid,
higher-risk posts.

Verona Pollard, an experienced nanny and maternity nurse, has taken up
part-time child care work since she was fired from a full-time nannying
job, after her employer found out she had taken public transport on her
days off.

``She was ruthless about it and wouldn't take me back, even when the
lockdown was lifted,'' she said in a phone interview. `` It's been
brutal since then, I'm just doing odd jobs here and there.''

And in the Fawcett Society's survey, work-related anxiety was highest
among Black and minority ethnic women, with 65.1 percent of women
employed outside the home reporting that they felt apprehensive as a
result of having to go to work during the pandemic.

Ms. Brown, the cleaner in Manchester, said that she was still waiting
for her unemployment benefits to come through and that she had been
borrowing money from a friend and from a former employer to get by. In
recent weeks, she said she has become so stressed and anxious that
rashes have broken out over her body and she has noticed her hair
falling out.

``I promise you, what I am going through now is worse than any virus,''
she said.

Advertisement

\protect\hyperlink{after-bottom}{Continue reading the main story}

\hypertarget{site-index}{%
\subsection{Site Index}\label{site-index}}

\hypertarget{site-information-navigation}{%
\subsection{Site Information
Navigation}\label{site-information-navigation}}

\begin{itemize}
\tightlist
\item
  \href{https://help.nytimes.com/hc/en-us/articles/115014792127-Copyright-notice}{©~2020~The
  New York Times Company}
\end{itemize}

\begin{itemize}
\tightlist
\item
  \href{https://www.nytco.com/}{NYTCo}
\item
  \href{https://help.nytimes.com/hc/en-us/articles/115015385887-Contact-Us}{Contact
  Us}
\item
  \href{https://www.nytco.com/careers/}{Work with us}
\item
  \href{https://nytmediakit.com/}{Advertise}
\item
  \href{http://www.tbrandstudio.com/}{T Brand Studio}
\item
  \href{https://www.nytimes.com/privacy/cookie-policy\#how-do-i-manage-trackers}{Your
  Ad Choices}
\item
  \href{https://www.nytimes.com/privacy}{Privacy}
\item
  \href{https://help.nytimes.com/hc/en-us/articles/115014893428-Terms-of-service}{Terms
  of Service}
\item
  \href{https://help.nytimes.com/hc/en-us/articles/115014893968-Terms-of-sale}{Terms
  of Sale}
\item
  \href{https://spiderbites.nytimes.com}{Site Map}
\item
  \href{https://help.nytimes.com/hc/en-us}{Help}
\item
  \href{https://www.nytimes.com/subscription?campaignId=37WXW}{Subscriptions}
\end{itemize}
