Sections

SEARCH

\protect\hyperlink{site-content}{Skip to
content}\protect\hyperlink{site-index}{Skip to site index}

\href{https://www.nytimes.com/section/arts/music}{Music}

\href{https://myaccount.nytimes.com/auth/login?response_type=cookie\&client_id=vi}{}

\href{https://www.nytimes.com/section/todayspaper}{Today's Paper}

\href{/section/arts/music}{Music}\textbar{}Eddie Gale, Deeply Spiritual
Jazz Trumpeter, Dies at 78

\url{https://nyti.ms/3ewahyc}

\begin{itemize}
\item
\item
\item
\item
\item
\end{itemize}

Advertisement

\protect\hyperlink{after-top}{Continue reading the main story}

Supported by

\protect\hyperlink{after-sponsor}{Continue reading the main story}

\hypertarget{eddie-gale-deeply-spiritual-jazz-trumpeter-dies-at-78}{%
\section{Eddie Gale, Deeply Spiritual Jazz Trumpeter, Dies at
78}\label{eddie-gale-deeply-spiritual-jazz-trumpeter-dies-at-78}}

He drew notice in the 1960s for his provocative playing with Cecil
Taylor and Sun Ra and on his own records. He later devoted most of his
time to teaching.

\includegraphics{https://static01.nyt.com/images/2020/07/17/obituaries/16Gale1-print/merlin_174595962_b6402c43-af14-4373-b7eb-a182f7f0135e-articleLarge.jpg?quality=75\&auto=webp\&disable=upscale}

\href{https://www.nytimes.com/by/john-leland}{\includegraphics{https://static01.nyt.com/images/2018/02/20/multimedia/author-john-leland/author-john-leland-thumbLarge.jpg}}

By \href{https://www.nytimes.com/by/john-leland}{John Leland}

\begin{itemize}
\item
  July 16, 2020
\item
  \begin{itemize}
  \item
  \item
  \item
  \item
  \item
  \end{itemize}
\end{itemize}

Eddie Gale, a spiritually minded jazz trumpeter and educator who
performed with the avant-garde giants Cecil Taylor and Sun Ra, and who
saw the music he made with his own bands as a conduit for communicating
the richness of African-American life, died on July 10 at his home in
Northern California. He was 78.

The cause was prostate cancer, his wife, Georgette Gale, said.

On his recordings as a leader --- including two significant albums for
the Blue Note label in the late 1960s, ``Eddie Gale's Ghetto Music'' and
``Black Rhythm Happening'' --- Mr. Gale drew on the Black church, his
Cub Scout marching band, astrology, street-corner funk and African
polyrhythms to concoct a densely layered urban stew.

``It's his whole life,'' his younger sister Joann Stevens, who sang and
played guitar with him, said in an interview. ``He felt that these are
the things that make the quote, `ghetto,' alive and culturally
enriching. So these are the things he wanted to celebrate and focus on,
even if other people don't.''

Sometimes the music got loud; sometimes it got deeply, deeply funky. And
always it was spiritual.

``It didn't sound like anything that came out before or after,'' the
trumpeter and bandleader Steven Bernstein said. ``Total outlier. It's
6/8 vamps with two bass players and two drummers, unison melodies in the
horns, and then incredible choirs that are bringing blocks of music.''

Mr. Gale moved from New York to California in the 1970s and committed
more of his time to teaching, although he continued to perform
occasionally. His sister said he was still rehearsing his latest band
and playing vigorously as recently as April, when the coronavirus shut
everything down.

Edward Gale Stevens was born in Brooklyn on Aug. 15, 1941, the third of
five children of Edward and Daisy Stevens. His father was a plumber's
assistant; his mother worked in a garment factory.

His first musical influence, he told
\href{https://jazztimes.com/features/profiles/eddie-gale-overdue-ovation/}{JazzTimes}
magazine in 2007, was the Rose Hill Baptist Church and the gospel and
spiritual records his family played. His parents were ``cultural
activists,'' his sister said, with a vast collection 78 r.p.m.
recordings by Louis Armstrong and others.

At the age of 8, he joined the Cub Scouts and took up the bugle. He
later moved to trumpet and took lessons with the jazz great Kenny
Dorham.

Brooklyn, filled with jazz musicians at the time, was a place to learn
the craft. ``In those days, the musicians were available to young
people, if you were really into it,'' Mr. Gale told
\href{https://www.sfgate.com/living/article/JEFFERSON-AWARD-Presented-to-Eddie-Gale-Gale-2487432.php}{The
San Francisco Chronicle} in 2006. ``At some of the jam sessions and at
after-hours clubs, you'd get involved by sitting in. These days, they
don't have that ability to learn directly from the masters.''

Mr. Gale would later work to create formal institutions to pass along
this knowledge.

In the late 1950s and early '60s he was starting to put the pieces
together himself, getting a spot in Sun Ra's Arkestra and meeting John
Coltrane, who let him sit in one night and another time gave him \$35 to
retrieve his trumpet from a pawnshop. ``He was like an uncle or father
figure to me,'' Mr. Gale told JazzTimes.

Mr. Gale was also raising a family. At 18 he married Marlene Manning,
with whom he had five children. The marriage ended in divorce. He also
had a daughter from an earlier relationship.

His two Blue Note albums, released in 1968 and 1969, were well received.
But as the label reorganized it did not renew his contract, and his
subsequent recordings were released only sporadically and were harder to
find.

He poured his energies into teaching, taking a residency at Stanford
University and then starting jazz programs at schools in San Jose, where
he moved in 1972. In 1974, Mayor Norman Y. Mineta named him ``San Jose's
Ambassador of Jazz.''

In 1985 he married Georgette Farley, who worked as a first responder for
counseling services at San Jose State University.

Jazz careers can be fickle things. After the late 1960s, Mr. Gale's work
was less visible, less a promise of a broader revolution to come.
Instead of expanding boundaries in nightclubs, he helped establish local
institutions, including the Evergreen Youth Adult Jazz Society, the
We're Jazzed! Youth/Adult Jazz Festival and the annual Concert for World
Peace and Peace Poetry Contest.

``Eddie believed in doing a lot of projects that had to do with helping
the community,'' Georgette Gale said. ``He coordinated a giveaway of 100
trumpets. And he raised money for the Bay Area Jazz Musicians Self-Help
Healthcare Project.''

He recorded a new version of ``Eddie Gale's Ghetto Music'' in 2018, and
he performed several times with the hip-hop group Boots Riley and the
Coup.

``He loved working with young people, and they were crazy about him,''
Ms. Gale said. At one show with the Coup, he even had a cutting contest
with the group's D.J., Pam the Funktress.

In addition to his wife and his sister Joann, Mr. Gale is survived by
six children: Donna, Marc, Chanel, Djuana, Gwilu and Teyonda; 12
grandchildren; 11 great-grandchildren; another sister, Leticia Peoples;
and a brother, David Stevens. He is also survived by students beyond
anyone's count.

His teaching extended almost to his last days, his sister Joann said.
``Even if he was talking to someone on the phone. His whole mission was
to use jazz as way to educate people about community and Black
culture.''

Advertisement

\protect\hyperlink{after-bottom}{Continue reading the main story}

\hypertarget{site-index}{%
\subsection{Site Index}\label{site-index}}

\hypertarget{site-information-navigation}{%
\subsection{Site Information
Navigation}\label{site-information-navigation}}

\begin{itemize}
\tightlist
\item
  \href{https://help.nytimes.com/hc/en-us/articles/115014792127-Copyright-notice}{©~2020~The
  New York Times Company}
\end{itemize}

\begin{itemize}
\tightlist
\item
  \href{https://www.nytco.com/}{NYTCo}
\item
  \href{https://help.nytimes.com/hc/en-us/articles/115015385887-Contact-Us}{Contact
  Us}
\item
  \href{https://www.nytco.com/careers/}{Work with us}
\item
  \href{https://nytmediakit.com/}{Advertise}
\item
  \href{http://www.tbrandstudio.com/}{T Brand Studio}
\item
  \href{https://www.nytimes.com/privacy/cookie-policy\#how-do-i-manage-trackers}{Your
  Ad Choices}
\item
  \href{https://www.nytimes.com/privacy}{Privacy}
\item
  \href{https://help.nytimes.com/hc/en-us/articles/115014893428-Terms-of-service}{Terms
  of Service}
\item
  \href{https://help.nytimes.com/hc/en-us/articles/115014893968-Terms-of-sale}{Terms
  of Sale}
\item
  \href{https://spiderbites.nytimes.com}{Site Map}
\item
  \href{https://help.nytimes.com/hc/en-us}{Help}
\item
  \href{https://www.nytimes.com/subscription?campaignId=37WXW}{Subscriptions}
\end{itemize}
