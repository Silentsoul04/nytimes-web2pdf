Sections

SEARCH

\protect\hyperlink{site-content}{Skip to
content}\protect\hyperlink{site-index}{Skip to site index}

\href{https://myaccount.nytimes.com/auth/login?response_type=cookie\&client_id=vi}{}

\href{https://www.nytimes.com/section/todayspaper}{Today's Paper}

\href{/section/opinion}{Opinion}\textbar{}Testing Is on the Brink of
Paralysis. That's Very Bad News.

\href{https://nyti.ms/2C7hhof}{https://nyti.ms/2C7hhof}

\begin{itemize}
\item
\item
\item
\item
\item
\end{itemize}

Advertisement

\protect\hyperlink{after-top}{Continue reading the main story}

\href{/section/opinion}{Opinion}

Supported by

\protect\hyperlink{after-sponsor}{Continue reading the main story}

\hypertarget{testing-is-on-the-brink-of-paralysis-thats-very-bad-news}{%
\section{Testing Is on the Brink of Paralysis. That's Very Bad
News.}\label{testing-is-on-the-brink-of-paralysis-thats-very-bad-news}}

Our pandemic fight requires prompt testing results --- and singular
cooperation among the states to achieve them.

By Margaret Bourdeaux, Beth Cameron and Jonathan Zittrain

Drs. Bourdeaux and Cameron are health policy experts. Professor Zittrain
teaches law and computer science.

\begin{itemize}
\item
  July 16, 2020
\item
  \begin{itemize}
  \item
  \item
  \item
  \item
  \item
  \end{itemize}
\end{itemize}

\includegraphics{https://static01.nyt.com/images/2020/07/17/opinion/14zittrain/merlin_174305244_e1b01acb-00fe-41a0-84da-f69670293ced-articleLarge.jpg?quality=75\&auto=webp\&disable=upscale}

As Covid-19 cases
\href{https://coronavirus.jhu.edu/data/new-cases-50-states}{surge} to
their highest levels in dozens of states, the nation's testing effort is
on the brink of paralysis because of widespread delays in getting back
results. And that is very bad news, because even if testing is robust,
the pandemic cannot be controlled without rapid results.

This is the latest failure in our national response to the worst
pandemic in a century. Since the Trump administration has abdicated
responsibility,
\href{https://www.nytimes.com/2020/07/13/us/coronavirus-governors.html}{governors}
must join forces to meet this threat before the
\href{https://perma.cc/UW3Q-9M2X}{cataclysm that Florida is
experiencing} becomes the reality across the country.

Testing should be the governors' first order of business.

Despite President Trump's boast early this month that testing
``\href{https://twitter.com/realDonaldTrump/status/1278897430378041344}{is
so massive and so good},'' the United States' two largest commercial
testing companies,
\href{https://newsroom.questdiagnostics.com/COVIDTestingUpdates}{Quest
Diagnostics} and
\href{https://www.labcorp.com/coronavirus-disease-covid-19/labcorp-newsroom}{LabCorp},
have found themselves
\href{https://www.cnbc.com/2020/07/13/us-coronavirus-surge-leads-to-testing-delays-across-the-nation-quest-diagnostics-says.html}{overwhelmed}
and unable to return results promptly. Delays averaging a week or longer
for all but top-priority hospital patients and symptomatic health care
workers are
\href{https://science.sciencemag.org/content/368/6491/eabb6936}{disastrous
for efforts to slow the spread} of the virus.

Without rapid results, it is impossible to isolate new infections
quickly enough to douse flare-ups before they grow. Slow diagnosis
incapacitates contact tracing, which entails not only isolating those
who test positive but also alerting the infected person's contacts
quickly so they can quarantine, too, and avoid exposing others to the
virus unwittingly.

Among those who waited an absurdly long time for her results was the
mayor of Atlanta, Keisha Lance Bottoms. ``We FINALLY received our test
results taken 8 days before,'' she
\href{https://twitter.com/KeishaBottoms/status/1280824621214896129?s=20}{tweeted}
last week. ``One person in my house was positive then. By the time we
tested again, 1 week later, 3 of us had COVID. If we had known sooner,
we would have immediately quarantined.''

Another complaint came this week from Mr. Trump's former acting chief of
staff, Mick Mulvaney, who wrote in
\href{https://www.cnbc.com/2020/07/13/mick-mulvaney-next-stimulus-bill-should-deal-with-covid-19.html}{an
op-ed commentary for CNBC} that ``my son was tested recently; we had to
wait 5 to 7 days for results.'' Noting, too, that his daughter was told
she didn't qualify for a test, he added, ``That is simply inexcusable at
this point in the pandemic.''

As summer turns to fall, slow and fragmented testing will fatally
undermine the reopening of schools and universities, whose plans are
predicated on quickly identifying outbreaks and suppressing spread.
Testing for millions of students will feed into an already failing
national system.

Vice President Mike Pence's casual invocation of an
``\href{https://thehill.com/homenews/administration/503899-pence-in-call-with-governors-defends-trump-comments-on-coronavirus}{extraordinary
national success in testing}'' in a recent call with governors was
\href{http://perma.cc/3F3D-NS6W}{flatly wrong}, as is the president's
similar
\href{https://twitter.com/realDonaldTrump/status/1280205902742781958?s=20}{trumpeting}
of testing success. These claims contribute to a false sense among the
public that testing may have had early stumbles but is ramping up slowly
but surely.

The reality is that the spread of the virus has vastly outpaced the
expansion of testing capacity. That spread in turn results in more
illness and therefore more tests to process, which further slows down
turnaround time in a vicious cycle. The dedication and patience of
thousands of people waiting in serpentine lines of cars for hours to be
tested are wasted when the results aren't returned quickly enough.

We are at this point because of the absence of a coordinated federal
plan, and, indeed, because of a White House that seems actively hostile
to producing one. The nation's governors and state legislators must
\href{https://perma.cc/TY79-GWLG}{fill the void}.

Unity among the states is not just about neighborliness but also about
self-interest. So long as interstate travel continues, inadequate
testing anywhere threatens public health everywhere, including in places
that have found or developed localized testing capacities and are less
sensitive to the bottlenecks that Quest and LabCorp are experiencing.

The signal difference between federal and state leadership is that the
former can print money and the latter cannot. If states are to step up,
they will need resources: money from Congress without executive branch
holdup, coordination and mutual aid from one another, and cooperation
and expertise from the public itself.

Here's what the governors need to do to bolster the overall testing
capability before the end of the summer, best begun with a summit in the
next two weeks.

Governors must work collectively to fill gaps in their own testing and
contact-tracing programs. The National Governors Association helped in
\href{https://perma.cc/TC92-QCKG}{a similar effort} to curb the spread
of the Zika virus.

In March there was a mad scramble and competition for personal
protective equipment. Now, the allocation of tests and test processing
may end up in another free-for-all. A coordinated approach by all states
would avoid that. Consistent
\href{http://covid-local.org/metrics}{metrics} must be established for
accountability and to identify trigger points that call for rapid policy
responses. Acting in concert can make it easier to undertake tough or
controversial decisions like ordering lockdowns when testing shows
renewed spread.

Governors should also agree to assist in sharing local test processing
capacity, including by university labs, so it is available wherever it
is most needed. Relying largely on two large commercial testing
companies, as we are now, has proved to be a major vulnerability.

For example, the Broad Institute of M.I.T. and Harvard has stepped up in
Massachusetts with
\href{https://covid19-testing.broadinstitute.org/}{more testing
capacity} --- so much so that it is not being fully used. But no
\href{https://www.nytimes.com/2020/05/21/health/coronavirus-testing-lab-capacity.html}{process
is in place} for a doctor in, say, Arizona to prescribe a test that the
Broad will process. That's a problem that governors can help solve. They
can also find ways to subsidize investments by labs to expand capacity,
to help untangle medical insurance complications so tests are covered
and to prompt innovations in testing.

In particular, they should encourage the academic and commercial sectors
to develop, test and produce
\href{https://www.reuters.com/article/us-health-coronavirus-smiths-group/smiths-to-help-make-blood-based-coronavirus-test-in-britain-idUSKBN2491K0}{new},
\href{http://perma.cc/5DUZ-GUBV}{rapid},
\href{http://perma.cc/4V5X-B7CN}{point-of-care} testing. More broadly,
they should recruit data scientists and experts in science communication
ready to lend their skills to a unified effort.

We can't allow the delays at Quest and LabCorp to mark the start of a
downward spiral. Instead, we must marshal a nationwide strategy to place
the United States in the ranks of other countries that are successfully
beating back the pandemic.

Sorting out testing is foundational to slowing the spread of the virus.
From there, governors can build a comprehensive national plan of attack.
Doing so will require new forms of coordinated governance. In the
absence of federal leadership, it's up to governors to step to the fore.

\href{https://www.hks.harvard.edu/about/margaret-bourdeaux}{Margaret
Bourdeaux} is research director of the Program of Global Public Policy
at Harvard Medical School.
\href{https://www.nti.org/about/leadership-and-staff/beth-cameron/}{Beth
Cameron} is the vice president for \href{http://covid-local.org/}{Global
Biological Policy and Programs} at the \href{http://nti.org/}{Nuclear
Threat Initiative}. \href{https://twitter.com/zittrain}{Jonathan
Zittrain} is a professor of law and computer science at Harvard and
co-chair with Dr. Bourdeaux of the
\href{https://cyber.harvard.edu/}{Berkman Klein Center's}
\href{https://cyber.harvard.edu/programs/bkc-policy-practice-digital-pandemic-response}{Digital
Pandemic Response Practice}.

\emph{The Times is committed to publishing}
\href{https://www.nytimes.com/2019/01/31/opinion/letters/letters-to-editor-new-york-times-women.html}{\emph{a
diversity of letters}} \emph{to the editor. We'd like to hear what you
think about this or any of our articles. Here are some}
\href{https://help.nytimes.com/hc/en-us/articles/115014925288-How-to-submit-a-letter-to-the-editor}{\emph{tips}}\emph{.
And here's our email:}
\href{mailto:letters@nytimes.com}{\emph{letters@nytimes.com}}\emph{.}

\emph{Follow The New York Times Opinion section on}
\href{https://www.facebook.com/nytopinion}{\emph{Facebook}}\emph{,}
\href{http://twitter.com/NYTOpinion}{\emph{Twitter (@NYTopinion)}}
\emph{and}
\href{https://www.instagram.com/nytopinion/}{\emph{Instagram}}\emph{.}

Advertisement

\protect\hyperlink{after-bottom}{Continue reading the main story}

\hypertarget{site-index}{%
\subsection{Site Index}\label{site-index}}

\hypertarget{site-information-navigation}{%
\subsection{Site Information
Navigation}\label{site-information-navigation}}

\begin{itemize}
\tightlist
\item
  \href{https://help.nytimes.com/hc/en-us/articles/115014792127-Copyright-notice}{©~2020~The
  New York Times Company}
\end{itemize}

\begin{itemize}
\tightlist
\item
  \href{https://www.nytco.com/}{NYTCo}
\item
  \href{https://help.nytimes.com/hc/en-us/articles/115015385887-Contact-Us}{Contact
  Us}
\item
  \href{https://www.nytco.com/careers/}{Work with us}
\item
  \href{https://nytmediakit.com/}{Advertise}
\item
  \href{http://www.tbrandstudio.com/}{T Brand Studio}
\item
  \href{https://www.nytimes.com/privacy/cookie-policy\#how-do-i-manage-trackers}{Your
  Ad Choices}
\item
  \href{https://www.nytimes.com/privacy}{Privacy}
\item
  \href{https://help.nytimes.com/hc/en-us/articles/115014893428-Terms-of-service}{Terms
  of Service}
\item
  \href{https://help.nytimes.com/hc/en-us/articles/115014893968-Terms-of-sale}{Terms
  of Sale}
\item
  \href{https://spiderbites.nytimes.com}{Site Map}
\item
  \href{https://help.nytimes.com/hc/en-us}{Help}
\item
  \href{https://www.nytimes.com/subscription?campaignId=37WXW}{Subscriptions}
\end{itemize}
