Sections

SEARCH

\protect\hyperlink{site-content}{Skip to
content}\protect\hyperlink{site-index}{Skip to site index}

\href{https://myaccount.nytimes.com/auth/login?response_type=cookie\&client_id=vi}{}

\href{https://www.nytimes.com/section/todayspaper}{Today's Paper}

The T List: Five Things We Recommend This Week

\url{https://nyti.ms/32vs5aa}

\begin{itemize}
\item
\item
\item
\item
\item
\end{itemize}

Advertisement

\protect\hyperlink{after-top}{Continue reading the main story}

Supported by

\protect\hyperlink{after-sponsor}{Continue reading the main story}

\hypertarget{the-t-list-five-things-we-recommend-this-week}{%
\section{The T List: Five Things We Recommend This
Week}\label{the-t-list-five-things-we-recommend-this-week}}

Unisex jerkins, raw vinegars, classic sportswear --- and more.

July 16, 2020

\begin{itemize}
\item
\item
\item
\item
\item
\end{itemize}

\emph{Welcome to the T List, a newsletter from the editors of T
Magazine. Each week, we're sharing things we're eating, wearing,
listening to or coveting now.}
\textbf{\href{https://www.nytimes.com/newsletters/t-list?module=inline}{\emph{Sign
up here}}} \emph{\textbf{to find us in your inbox every Wednesday.}}
\emph{You can always reach us at}
\href{mailto:tlist@nytimes.com}{\emph{tlist@nytimes.com}}\emph{.}

\begin{center}\rule{0.5\linewidth}{\linethickness}\end{center}

Taste This

\hypertarget{new-raw-vinegars-from-brightland}{%
\subsection{New Raw Vinegars From
Brightland}\label{new-raw-vinegars-from-brightland}}

\includegraphics{https://static01.nyt.com/images/2020/07/17/t-magazine/15tmag-tlist-slide-2AZW-print/15tmag-tlist-slide-2AZW-articleLarge.jpg?quality=75\&auto=webp\&disable=upscale}

By Jennifer Conrad

For Aishwarya Iyer, the founder and C.E.O. of the California-based
pantry staple company Brightland, coronavirus-mandated stay-at-home
measures have meant lots of cooking, lots of long walks and lots of time
devoted to developing her company's newest product: vinegar. To start,
there are two limited-release offerings --- Parasol and Rapture --- both
of which are considered raw for retaining a bit of the bacteria that
spark the fermentation process and are thought to promote good
digestion. The former is a champagne vinegar made with chardonnay grapes
and navel and Valencia oranges; consider using it as a shrub in a tart
summer cocktail. The latter is a juicy balsamic vinegar that comes from
zinfandel grapes and blackberries; Iyer likes to drizzle it over grilled
peaches and vanilla ice cream. They also both mix well with olive oil
(try combining Parasol with Brightland's basil-infused oil for a perfect
panzanella dressing). Founded in 2018, the company is committed to
sustainability and social consciousness (this month, 15 percent of
proceeds from its Artist Capsule are going to the NAACP Legal Defense
and Educational Fund). The organic grapes used for the vinegars are
grown in Northern California, and both varieties are double-fermented
and distilled on a family farm on the Central Coast. Fittingly, the
bottles' Alexander Calder-esque labels are meant to evoke a California
sun, and printed on the inside are little messages --- Rapture, for one,
reminds us to ``look for wonder.'' \emph{\$22 per bottle,}
\href{http://brightland.co/}{\emph{brightland.co}}\emph{.}

\begin{center}\rule{0.5\linewidth}{\linethickness}\end{center}

Wear This

\hypertarget{luke-edward-halls-unisex-jerkins}{%
\subsection{Luke Edward Hall's Unisex
Jerkins}\label{luke-edward-halls-unisex-jerkins}}

Image

Left: Luke Edward Hall wearing~The Castle of the Forest Sauvage cream
and green trellis print vintage glazed cotton jerkin. Right:~a cream and
red striped vintage cotton jerkin.Credit...Billal Taright

By Flo Wales Bonner

T Contributor

Having your outfit compared to a pair of curtains isn't usually a
compliment. But for the London-based artist and interior designer Luke
Edward Hall, who studied men's wear design at Central Saint Martins,
furnishing fabrics are as well suited for making clothes as more
traditional materials. And so for his latest project, he is creating a
limited run of unisex jerkins from surplus interior textiles, some left
over from his own design projects and others carefully sourced online.
Named the Castle of the Forest Sauvage, a reference to a land from
Arthurian lore (Hall, who has a love of mythology, reread stories about
the legendary British king during the lockdown), the line takes
inspiration from the jerkin's history as both formal attire and
utilitarian military wear. Hall also looked to his own collection of
Moroccan, Indian and Austrian waistcoats for design cues, eventually
landing on a garment with an unfussy silhouette that lets the wildly
colorful fabrics be the focus. ``There are amazing trellis prints,
medieval village scenes,'' he explains. ``These are patterns not made
for clothes, and that's part of the appeal.'' \emph{Available through
the brand's Instagram page,}
\href{https://www.instagram.com/lukeedwardhall}{\emph{@}}\href{https://www.instagram.com/thecastleoftheforestsauvage/?hl=en}{\emph{thecastleoftheforestsauvage}}\emph{.}

\begin{center}\rule{0.5\linewidth}{\linethickness}\end{center}

See This

\hypertarget{a-florentine-artists-residency-goes-local}{%
\subsection{A Florentine Artist's Residency Goes
Local}\label{a-florentine-artists-residency-goes-local}}

Image

A collection of works in the Palazzo Galli Tasso's loft, its frescoes
newly revealed by renovations. Left: Duccio Maria Gambi's Deep Void vase
sits on Martino di Napoli Rampolla's Marcolone table, with Mattia Papp's
``Atlantis Hall'' hanging on the wall, Sasha Ribera's Poplar stool on
the floor and Bloc Studios' Clelia vase on the pedestal. Right: Lorenzo
Brinati's ``San Giovanni'' hangs above Pietro Franceschini's Bling Bling
ottoman.Credit...Daniel Civetta

By Laura Rysman

T Contributor

A crisis can reveal space for new possibilities, and with Florence
recently emptied of its customary throngs of tourists, Martino di Napoli
Rampolla, **** the founder and creative director of the city's
Numeroventi artist residency, saw a chance to strengthen the community
of Tuscan makers. Earlier this month, he opened the exhibition ``So
Close So Good'' at the residence, which is housed within the stately
16th-century Palazzo Galli Tasso, showcasing the work of 10 local
artists and designers produced during these past months of confinement.
``We realized we can reclaim Florence for ourselves now,'' explains di
Napoli Rampolla. ``The globalized system may be helpful and
remunerative, but it's not what makes a community healthy.'' The works,
which will also be exhibited online beginning July 16, have a
distinctively Florentine feel, with an emphasis on natural materials and
handcraft: The designer and artist Duccio Maria Gambi contributed
slablike sculptures hewn from white onyx, their rough edges highlighted
with spray paint in bold primary colors; **** Bloc Studios, a
marble-focused design practice founded by Sara Ferron Cima, is showing
satin-smooth vases; and the artist Justin Randolph Thompson hand-built a
room-size multimedia installation that layers representations of Black
experiences of both contemporary and colonial-era Italy. ``We finally
see how we can work amongst ourselves,'' Di Napoli Rampolla says,
``instead of looking abroad.'' \emph{``So Close So Good'' is on view at
Numeroventi through Sept. 5, Via Pandolfini 20, Florence, and online
beginning July 16,}
\href{https://www.numeroventi.it/}{\emph{numeroventi.it}}\emph{.}

\begin{center}\rule{0.5\linewidth}{\linethickness}\end{center}

Listen to This

\hypertarget{country-singer-margo-prices-new-album}{%
\subsection{Country Singer Margo Price's New
Album}\label{country-singer-margo-prices-new-album}}

Image

Credit...Bobbi Rich

By Daniel Wagner

This month, having pushed the original date on account of the pandemic,
the singer-songwriter Margo Price released her third studio album,
``That's How Rumors Get Started.'' Instead of touring, she's been
quarantining in Nashville, Tenn., with her family (her husband
contracted the coronavirus but has since recovered). In a bit of a
departure for Price, the album's 10 tracks grapple with restlessness,
career expectations, motherhood and the peripatetic life of a musician.
But while Price may also have mostly traded in the pedal-steel-laden
tracks of her previous two albums for up-tempo melodies Tom Petty would
be proud of, her voice still carries that distinctive golden country
glow for which she's known and loved. Co-produced by Price and her
friend and fellow singer-songwriter Sturgill Simpson --- one of
Nashville's biggest stars, whose chameleonic style makes him an
enigmatic figure to the industry's establishment --- the album sparkles
with gospel singers, iconic guitar lines and soaring, catchy hooks.
Three standout tracks --- ``Hey Child,'' ``Gone to Stay'' and ``Prisoner
Of The Highway'' --- beg to be played on the open road, a place of much
reflection throughout the record. Though that may be well-worn territory
in the canon of country music, Price's lyrics reveal an original
portrait of an artist on the bus passing through.
\href{http://margoprice.net/}{\emph{margoprice.net}}\emph{.}

\begin{center}\rule{0.5\linewidth}{\linethickness}\end{center}

Buy This

\hypertarget{casual-sportswear-inspired-by-princess-diana}{%
\subsection{Casual Sportswear Inspired by Princess
Diana}\label{casual-sportswear-inspired-by-princess-diana}}

Image

Left: Ceres Sport Rib Sport Bra, \$68, and~Sport Rib Hi Legging,~\$108.
Right: Ceres~Sport Jersey Leotard, \$128, and~The Perfect Sweat,
\$148.Credit...Zak Bush

By Crystal Meers

T Contributor

Ceres, a new line of athletic wear created by the stylist and designer
Nina Miner and the yoga teacher Kumi Sawyers, began in a typically Los
Angeles fashion: while the pair were on a hike. The two had met recently
through mutual friends and bonded, as they walked through Rivas Canyon,
over their love for the perfectly broken-in, vintage long johns that
Miner was wearing. As it turned out, they shared a desire to start their
own line of sportswear inspired by the days when Olympians would train
in cotton T-shirts and shorts and Princess Diana would throw a blazer
over her sweatpants --- in other words, a time before skintight black
leggings in synthetic fabrics took over the gym. ``You don't have to be
corseted in. It doesn't have to be so tight in order for it to look
good,'' says Sawyer. The result is a sustainably made 17-piece
collection cut from natural fibers such as cotton and cotton blends in
Kelly green, navy and shades of beige with stylistic nods to both
ballerinas (scoop-back leotards, deep V-neck tops) and boxers
(waffle-knit tanks and leggings). The standout item may be the
high-waisted cotton fleece sweatpants, which the pair swear flatter
every body as they fit slim through the hip and don't have pockets.
``Pockets add bulk, and they're unnecessary because if you're hiking or
running, whatever is in your pockets falls out anyway,'' says Miner, who
designed a fanny pack using fabric remnants from the line's sweatsuits
to solve exactly this problem.
\href{https://www.sportceres.com/password}{\emph{sportceres.com}}\emph{.}

\begin{center}\rule{0.5\linewidth}{\linethickness}\end{center}

From T's Instagram

\hypertarget{thousetour}{%
\subsection{\#THouseTour}\label{thousetour}}

Advertisement

\protect\hyperlink{after-bottom}{Continue reading the main story}

\hypertarget{site-index}{%
\subsection{Site Index}\label{site-index}}

\hypertarget{site-information-navigation}{%
\subsection{Site Information
Navigation}\label{site-information-navigation}}

\begin{itemize}
\tightlist
\item
  \href{https://help.nytimes.com/hc/en-us/articles/115014792127-Copyright-notice}{©~2020~The
  New York Times Company}
\end{itemize}

\begin{itemize}
\tightlist
\item
  \href{https://www.nytco.com/}{NYTCo}
\item
  \href{https://help.nytimes.com/hc/en-us/articles/115015385887-Contact-Us}{Contact
  Us}
\item
  \href{https://www.nytco.com/careers/}{Work with us}
\item
  \href{https://nytmediakit.com/}{Advertise}
\item
  \href{http://www.tbrandstudio.com/}{T Brand Studio}
\item
  \href{https://www.nytimes.com/privacy/cookie-policy\#how-do-i-manage-trackers}{Your
  Ad Choices}
\item
  \href{https://www.nytimes.com/privacy}{Privacy}
\item
  \href{https://help.nytimes.com/hc/en-us/articles/115014893428-Terms-of-service}{Terms
  of Service}
\item
  \href{https://help.nytimes.com/hc/en-us/articles/115014893968-Terms-of-sale}{Terms
  of Sale}
\item
  \href{https://spiderbites.nytimes.com}{Site Map}
\item
  \href{https://help.nytimes.com/hc/en-us}{Help}
\item
  \href{https://www.nytimes.com/subscription?campaignId=37WXW}{Subscriptions}
\end{itemize}
