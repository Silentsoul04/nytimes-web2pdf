Sections

SEARCH

\protect\hyperlink{site-content}{Skip to
content}\protect\hyperlink{site-index}{Skip to site index}

\href{https://www.nytimes.com/section/travel}{Travel}

\href{https://myaccount.nytimes.com/auth/login?response_type=cookie\&client_id=vi}{}

\href{https://www.nytimes.com/section/todayspaper}{Today's Paper}

\href{/section/travel}{Travel}\textbar{}Vacation in the Summer of
Covid-19

\url{https://nyti.ms/3901un0}

\begin{itemize}
\item
\item
\item
\item
\item
\item
\end{itemize}

\href{https://www.nytimes.com/spotlight/at-home?action=click\&pgtype=Article\&state=default\&region=TOP_BANNER\&context=at_home_menu}{At
Home}

\begin{itemize}
\tightlist
\item
  \href{https://www.nytimes.com/2020/08/03/well/family/the-benefits-of-talking-to-strangers.html?action=click\&pgtype=Article\&state=default\&region=TOP_BANNER\&context=at_home_menu}{Talk:
  To Strangers}
\item
  \href{https://www.nytimes.com/2020/08/01/at-home/coronavirus-make-pizza-on-a-grill.html?action=click\&pgtype=Article\&state=default\&region=TOP_BANNER\&context=at_home_menu}{Make:
  Grilled Pizza}
\item
  \href{https://www.nytimes.com/2020/07/31/arts/television/goldbergs-abc-stream.html?action=click\&pgtype=Article\&state=default\&region=TOP_BANNER\&context=at_home_menu}{Watch:
  'The Goldbergs'}
\item
  \href{https://www.nytimes.com/interactive/2020/at-home/even-more-reporters-editors-diaries-lists-recommendations.html?action=click\&pgtype=Article\&state=default\&region=TOP_BANNER\&context=at_home_menu}{Explore:
  Reporters' Google Docs}
\end{itemize}

Advertisement

\protect\hyperlink{after-top}{Continue reading the main story}

Supported by

\protect\hyperlink{after-sponsor}{Continue reading the main story}

\hypertarget{vacation-in-the-summer-of-covid-19}{%
\section{Vacation in the Summer of
Covid-19}\label{vacation-in-the-summer-of-covid-19}}

Traveling during a pandemic requires lots of research, precision
planning and a willingness to play by new and very stringent rules. For
these writers, it still felt good to get away.

\includegraphics{https://static01.nyt.com/images/2020/07/17/travel/17travel-desperatetimes/17travel-desperatetimes-articleLarge.jpg?quality=75\&auto=webp\&disable=upscale}

By \href{https://www.nytimes.com/by/eric-lipton}{Eric Lipton},
Christopher Solomon, Sheila Marikar and
\href{https://www.nytimes.com/by/tariro-mzezewa}{Tariro Mzezewa}

\begin{itemize}
\item
  July 16, 2020
\item
  \begin{itemize}
  \item
  \item
  \item
  \item
  \item
  \item
  \end{itemize}
\end{itemize}

Everyone the world over knows that travel has drastically changed. **
For many, the simple idea of travel is fraught, regardless of the
current restrictions and border closures. But others still feel the need
to get away, drawn to the appeal and respite of new --- or familiar ---
sights and sounds and experiences.

A few of our writers got away, safely, by conducting a great deal of
advance planning, choosing their destinations and activities carefully,
and taking many, many steps to best ensure their health and safety ---
and of those they encountered.

Here are a few chronicles on traveling during the summer of Covid-19.

\emph{If you decide to travel, before you book be sure to look up any
restrictions for your destination.}
\href{http://nytimes.com/2020/07/10/travel/state-travel-restrictions.html}{\emph{Many
states require strict self-quarantine requirements}} \emph{for new
visitors or even returning residents, and the rules are changing by the
day. You might also want to investigate the transmission rate of your
destination and your ability to isolate if necessary while away.
Finally, you should consider if a self-quarantine will be required when
you return home; perhaps yet another reason to vacation close by.}

\begin{center}\rule{0.5\linewidth}{\linethickness}\end{center}

\includegraphics{https://static01.nyt.com/images/2020/07/14/travel/oakImage-1594760565624/oakImage-1594760565624-articleLarge.jpg?quality=75\&auto=webp\&disable=upscale}

SEARSPORT, Maine

\hypertarget{family-beach-vacation}{%
\subsection{Family Beach Vacation}\label{family-beach-vacation}}

First, the flight we planned to take this summer for a family trip was
canceled. Then, day camp for our children was called off. The time had
come to improvise if we wanted a way to get out of our house --- and
still have a memorable summer experience this year.

But there would be a few ground rules, my wife and I decided. It had to
be safe. It had to be affordable. And we would only take a trip if we
could commit as a family to follow the local health rules. Thus began
our grand adventure into the wilds of Maine --- which has a
\href{https://www.maine.gov/governor/mills/sites/maine.gov.governor.mills/files/inline-files/An\%20Order\%20Establishing\%20Quarantine\%20Restrictions\%20On\%20Travelers\%20Arriving\%20in\%20Maine.pdf}{state-ordered}
quarantine for 14 days for visitors from most states.

We picked Maine because we had been there before, and loved the
combination of mountains and the sea, and the people in this defiantly
independent place.

First, we had to get there: It is a 11-hour drive from our home in
Washington, and we had been warned we could not go inside stores once we
got to Maine, so we had to bring more supplies than normal.

That is why we decided to get a rooftop bag
(\href{https://www.amazon.com/gp/product/B072ZHRDMZ/ref=ppx_yo_dt_b_asin_title_o09_s00?ie=UTF8\&psc=1}{\$61
from Amazon}) for our S.U.V. It was big enough to fit all three of our
children, but instead we loaded the luggage. We also splurged on an
inflatable kayak
(\href{https://www.walmart.com/ip/Intex-Explorer-K2-Inflatable-Kayak-with-Oars-and-Hand-Pump/23662871}{\$86})
and a bike rack
(\href{https://www.amazon.com/gp/product/B00AW6XL8K/ref=ppx_yo_dt_b_asin_title_o00_s00?ie=UTF8\&psc=1}{\$174})
and then had several days of debate over what else we might be able to
fit into the car, finally heading out with not a single square foot of
unused space.

We left our house at 5 a.m., while the children were still asleep, and
somehow managed to only make a single stop on the way to Maine,
refueling and getting a curbside delivery from
\href{https://www.reinsdeli.com/Default.aspx}{Rein's New York Style
Deli}, a delicatessen heaven that is right off I-84 in Vernon, Conn.
(Full disclosure: Wanting to avoid public restrooms, we also found a
private spot in the woods.)

How did we manage to drive that far with so few stops? We had downloaded
videos for our kids, ages 7, 6 and 2. We let them drink water, but not
too much. And we just kept going, with my wife and I sharing the
driving.

When we arrived at our house rental in Maine late that afternoon, we
knew immediately we had made the right choice, even if this trip was
going to be very unusual, given the self-quarantine requirement. No one
was enforcing these rules. But we decided to honor them anyway.

The sleepy town of \href{https://searsport.maine.gov/}{Searsport} was
once one of the most important shipbuilding communities in the United
States. All 17 of its shipyards are long gone, leaving behind a main
street with a few restaurants, antique shops and a maritime museum. None
of which we went inside.

But for \$160 a night, we rented a four-bedroom house, with a sprawling,
grassy backyard that faced right out onto Penobscot Bay and the islands
that dotted the horizon between us and the open waters of the Atlantic
Ocean. A small rocky beach, where the birds and crabs were about the
only other company, was down a pathway. Our children ran wild, freed
from months confined to our house in Washington.

We took bike rides, up into the hillsides on empty local roads and out
onto the finger-shaped Sears Island, a paradise of dense Maine woods and
wildflower-filled fields. We floated on the bay in our small boat. We
found an old croquet set in the garage and knocked around the balls.

The daily symphony of Maine summertime weather was on full display: Dawn
this far North starts at 4:15 a.m., with fog in the mornings, cool air
to start the day, a blazing sun that by noon glitters off the top of the
bay's small waves, then a sudden switch to breezy air again by evening.
We would punctuate this sometimes with a small fire in the backyard, and
one night even made s'mores with marshmallows, Hershey's chocolate and
graham crackers. Who needs summer camp? We did it on our own.

We did make several carefully organized day trips during our stay.

\href{https://www.nps.gov/acad/index.htm}{Acadia National Park} was only
an hour away. Given that this was a national park, administered by the
federal government, I called to ask if we could still visit, even if we
were honoring the Maine state quarantine, which also applied to state
parks and beaches. A Park Service employee told me she did not know the
answer, finally a state health department official said it would be OK,
as long as we bought our tickets in advance, meaning we did not have to
come close to employees there and stayed away from any other guests.

So we loaded into our car and headed off, finding a national park that
is typically jampacked in the summer only sparsely populated. When we
took a hike along the Beech Cliff Loop Trail --- me carrying our
2-year-old in a hiking carrier --- we only encountered one other couple
along the way and stayed six feet away.

The oddest part of our trip was this distance. The last time we were in
Maine, one of my favorite parts was getting to know the people, like
Peter Ralston. The
\href{https://www.ralstongallery.com/about}{photographer} lives in
Rockport and has fascinating stories about the years he has spent
photographing Maine's islands, or his early work when he was a friend of
a painter named Andy Wyeth. We also met a young musician and songwriter,
\href{https://www.alexwilder.com/about}{Alex Wilder}, and his family.

Curbside pickup became our life-link to stores and restaurants that
included the famed \href{https://www.youngslobsters.com/}{Young's
Lobster Pound} in Belfast, where I got my first contactless lobster in
my life. (It was still delicious, consumed from our kitchen, with views
of the bay.)

Most businesses were perfectly happy to accommodate out-of-state
quarantineers with these curbside, contactless services. \emph{ERIC
LIPTON}

As of early mid-July, residents of Connecticut, New York, New Jersey,
New Hampshire and Vermont can travel to
\href{https://www.nytimes.com/interactive/2020/us/states-reopen-map-coronavirus.html}{Maine}
without a quarantine. Others are required to self-quarantine for two
weeks, which they can avoid with a
\href{https://www.maine.gov/covid19/restartingmaine/keepmainehealthy/faqs}{negative
coronavirus test} taken within 72 hours before arrival in the state.
(Tests may be taken upon entry of the state, but quarantines are
required until negative results are released.)

\begin{center}\rule{0.5\linewidth}{\linethickness}\end{center}

Image

The writer discovered an Oregon river that was fast and loud and splashy
and forgiving.Credit...Tim Neville

Oregon River

\hypertarget{raft-and-camping-trip}{%
\subsection{Raft and Camping Trip}\label{raft-and-camping-trip}}

It was nearly summer. I was tired of the walls of my house. Pretty sure
the walls were tired of me. In the carport the big blue river raft wore
the look of a dog that waits too long by the door. Enough. I texted my
old friend Tim, a travel writer sidelined by the pandemic. His walls, as
it turned out, were tired of him, too.

But where to go? There was one answer. Away. Away from the relentless
bad news. Away from the unceasing grief. Away from the fear of the
unmasked masses. Into the pines, and onto the water. Back to ``the
rock-bottom facts of ax and wood and fire and frying pans,'' as John
Graves wrote in ``Goodbye to a River,'' my forever vote for the best
book about rivers, and life on rivers.

Raft in tow, I pointed the rig toward
\href{http://www.blm.gov/visit/grande-ronde-wild-scenic-river}{northeast
Oregon}. As the odometer spun up, the towns grew smaller and felt less
menacing. Then the earth opened and the road dropped down the walls of a
steep canyon, and even the small towns disappeared. Better. At the
bottom there was little more than a campground and the Minam Store
selling fishing flies, and a boat launch, and the river, hurrying past.
A deep exhale, as if after a long time underwater.

The Grande Ronde is not well-known to those outside the Northwest. The
river begins in the Blue Mountains of Oregon. For the next 182 miles it
works its way north and east until its confluence with the Snake, in
Washington State. Those who do make the 350-mile drive from Portland,
say, usually come to float a 45-mile stretch of water from Minam to
Troy, a trip that begins on the Wallowa River, until those waters shake
hands with the Grande Ronde about 10 miles downstream.

The Grande Ronde portion is part of the federal Wild \& Scenic Rivers
System, and that designation is deserved. The river's grandfather long
ago wore a canyon through volcanic rock, until today those walls ascend
2,000 feet in places. There is no car access on this stretch. You are on
your own. Which is why you came, after all.

Each spring the tall walls that wear sagebrush and grass briefly flare
green, and the river below is fast and loud and splashy and forgiving to
the novice boater who takes care. There are campsites soft with pine
needles on the inside of every bend, and the feel of warm sun on the
back of the neck after the long winter is as welcome as a hand of a
friend. It all feels like a bit of Montana wilderness, placed down in a
deep crack in the earth.

Tim and I took precautions before meeting. We drove separately, arriving
from different towns. We chose a destination where the only thing
crowded upon arrival was the sky, before an unseasonable deluge. To
shuttle a car between put-in and takeout --- a necessity, for 90 minutes
--- we masked up and rode with the windows down. Once on the river, we
slept in separate tents. We brought a hand-wash station and we scrubbed
with the zeal of surgeons. Most important, though, was what we did
before ever leaving home: We knew the patterns of the other's life. Tim
and I both work from home. We keep our bubbles small. Our risk to the
other, we figured, was acceptably low.

The first morning, we were up early but on the river late, still new
enough at river trips and the work they require, and still impatient in
a city way that leads to wasted time. Finally we pushed off into a cold
spitting rain, the river blown out from the previous night's downpour,
its water turgid and colored. Tossing big dry flies to ravenous trout,
one of our goals, was out the window.

This was a blessing in its way. Not a scrap of agenda remained for us
rafters, except to keep the wet side down. We practiced our fledgling
rowing technique through rapids like Martin's Misery, and we talked, and
we knocked the same old jokes back and forth like a shuttlecock, and we
drank cold beer, and we talked more. Mostly, we tried to forget about
the world above the canyon's rim. And we tried to slow down. Read. Nap
in the hammock strung between ponderosas. Listen to the corkscrew song
of a canyon wren. Watch a young mink play beside the boat. And all the
time, let the fast river carry us down. Which it did. Out of rain, into
sunshine. \emph{CHRISTOPHER SOLOMON}

As of July 8, Wallowa County, home to our float trip, was in Oregon's
Phase 2 opening, allowing more activities. As illnesses have started to
climb again in Oregon, Gov. Kate Brown now requires face masks
statewide, even outdoors, when distancing isn't possible.

\href{https://www.nytimes.com/interactive/2020/us/states-reopen-map-coronavirus.html}{}

\includegraphics{https://static01.nyt.com/images/2020/04/24/us/states-reopen-map-coronavirus-promo-1587778728210/states-reopen-map-coronavirus-promo-1587778728210-articleLarge-v67.png}

\hypertarget{see-how-all-50-states-are-reopening-and-closing-again}{%
\subsection{See How All 50 States Are Reopening (and Closing
Again)}\label{see-how-all-50-states-are-reopening-and-closing-again}}

All 50 states have reopened in some way, though some are pausing their
plans or backtracking amid a rise in cases.

Image

The Santa Ynez Inn in California still hosts a daily, complimentary
happy hour, with some new caveats: plastic wrapped, pre-assembled cheese
and charcuterie plates, single-serve wine ``glasses'' and a reminder to
practice social distancing.Credit...Sheila Marikar

Santa Ynez Valley, California

\hypertarget{jaunt-to-wine-country}{%
\subsection{Jaunt to Wine Country}\label{jaunt-to-wine-country}}

Avowed wine drinkers and avid travelers, my husband and I had
back-burnered the Santa Ynez Valley, **** a wine region 125 miles north
of our Los Angeles home. **** We preferred more exotic destinations:
Mexico, India, Japan. But by June, after three months at home, the
notion of waking up in a different ZIP code felt novel enough to make
some reservations and pack up the car, assuming we could still remember
what to pack. (I forgot a bathing suit, he forgot Advil.)

We took our trip during what turned out to be a brief window of
decreasing virus cases and a gradual reopening of tourist and other
businesses. On the way out of Los Angeles, the city's former sludge of
traffic flowed like water.

The drive from Silver Lake to Malibu, up the 101, took 30 minutes on a
Thursday afternoon. We drove past lettuce farms, lemon trees and a truck
advertising cilantro and watercress. The truck's driver smiled, window
down, face mask around chin. The 101 gave way to State Route 154, with
rolling hills thick with shrub and brush, seemingly devoid of human
intervention.

Before walking into the\href{https://santaynezinn.com/}{Santa Ynez Inn},
a 20-room hotel in the style of a Victorian mansion, we donned our face
masks. The general manager, Julio Penuela, also wore a mask while
checking us in, though the guests behind us did not, standing by the
front door, a good 12 feet away. We arrived shortly before the start of
the daily happy hour.

``We're doing it a little differently because of the pandemic,'' said
Mr. Penuela, gesturing at the plastic wine ``glasses'' and
shrink-wrapped cheese plates. ``We'd usually have more jewelry on
display, too, but we don't want to have things that people can touch.''

Before heading to wine-tasting rooms in the nearby town of Los Alamos,
we walked to \href{https://doscarlitos.com/}{Dos Carlitos}, a Mexican
restaurant up the street. A dozen patrons sat outside, slugging
margaritas and wine between scoops of chips and guacamole.

``You only have to wear your mask if you're moving about,'' a server
told us. That seemed to be the unofficial rule throughout the region. In
an Uber? Mask on. Walking into a tasting room? Mask on. Sitting at a
table? Mask off (one could attempt to taste wine with a mask on, but
that could present some challenges).

Servers stayed valiantly masked while explaining the varietals and
fielding questions. ``We're new at this,'' said Kim van der Linden of
\href{https://www.stolpmanvineyards.com/}{Stolpman Vineyards}, which had
outfitted the lawn of its Los Olivos tasting room with wrought-iron
tables, chairs and umbrellas. ``We used to have everyone inside,
standing along the bar. Obviously, you can't do that now.''

Across the street, a prepaid, 90-minute, private tasting at the
pinot-noir producer \href{https://dragonettecellars.com/}{Dragonette}
came with an unanticipated bonus --- freedom to eat the sandwiches we
bought from Panino, the deli next door, one of the many food options
recommended by tasting room manager Nicholos Luis. (Most wineries
generally do not allow guests to bring in outside food.)

Some tasting rooms in Los Olivos, like Stolpman and Dragonette,
recommended or required advance reservations. Others, like
\href{http://storyofsoilwine.com/}{Story of Soil} and
\href{https://biennacidoestate.com/}{Bien Nacido \& Solomon Hills
Estates}, were able to accommodate walk-ins.

By late afternoon on Friday, the number of people milling about downtown
Los Olivos had thinned out. Judging by the crowd spilling out of the
Italian restaurant \href{https://www.sykitchen.com/}{S.Y. Kitchen} in
Santa Ynez (indoor and outside dining was allowed at the time), some of
them went there. On the phone, the hostess explained that she had no
tables available for three hours.

``It's been busier than it usually is, at this time of year,'' said a
server at Pico, a wine bar and restaurant in Los Alamos. ``People want a
break, they want the country, they want good vibes.''

It seemed, watching people come together, lower their masks and raise
their glasses, that they wanted a level of lightheartedness that often
seems out of reach at home, surrounded by bills and laundry and 24-hour
cable news. We brought back some bottles to help with that. \emph{SHEILA
MARIKAR}

While no statewide travel restrictions are currently in place,
coronavirus cases in California rose in July and
\href{https://www.nytimes.com/interactive/2020/us/california-coronavirus-cases.html}{ordinances
throughout the state} have banned indoor wine tasting. Wineries with the
capacity to host guests outdoors moved their tastings accordingly, but
rules are changing by the day. If you're planning a trip, call the
wineries you intend to visit to find out their policies, and don't
forget your mask.

\begin{center}\rule{0.5\linewidth}{\linethickness}\end{center}

Image

When the writer and her best friend arrived in Kentucky after an 11-hour
drive, they~made a beeline to the home's back porch and enjoyed its
view.Credit...Baylen Campbell

Hazard, Kentucky

\hypertarget{road-trip-to-appalachia}{%
\subsection{Road Trip to Appalachia}\label{road-trip-to-appalachia}}

``You are more than welcome here.''

Those are the six words I should have listened to at the start of
quarantine. They came in a text message from my best friend's mother,
whom I call ``one of my moms'' --- Laura, in Hazard, Ky. When I read the
message, I was packing my carry-on suitcase and preparing to return to
New York from Florida, where I'd been working and spent a few vacation
days.

It was March and I didn't know the scale of what was to come. I told
myself that I didn't want to impose in Kentucky and I didn't want to
potentially expose anyone to the coronavirus in case I was carrying it
after all of my travels. After all, I had been at
\href{https://www.nytimes.com/2020/03/12/travel/coronavirus-disneyworld-theme-parks.html}{a
theme park}, surrounded by sticky-fingered children a week earlier. I
also had many projects to do in my apartment in New York.

The next morning, as I sat in a window seat next to a woman without a
mask, I knew I'd made a mistake. She kept bumping my shoulder when she
nodded off to sleep. Each time she did it, I winced. By the time I
walked out of La Guardia Airport, a single thought was on repeat in my
mind: I should have gone home to Hazard.

When the invitation was extended again two months later, I didn't think
twice. I felt fortunate to have my health, a support system and a job I
could do remotely. I had followed all the rules and remain, to this day,
so grateful to the essential workers. But I was also itching to get out.

While the sirens from ambulances rushing to the hospital four blocks
away from my apartment had slowed, they were still more frequent than
before Covid-19. The fireworks set off daily from 4 p.m. to 5 a.m. for
two weeks made it so I was getting about three hours of sleep every
night. I was exhausted and I now hated New York. After three months
alone in my 400-square-foot apartment, I had completed nearly all the
projects, I'd made pesto with the basil I'd grown,
\href{https://www.nytimes.com/2020/05/29/travel/the-world-in-a-jewelry-box.html}{organized
my jewelry}, baked banana bread and done many virtual happy hours and
workouts. Now loneliness was starting to set in.

I always pack light, so when I walked out of my apartment with my
extra-large suitcase, carry on, backpack and several tote bags promptly
at 6:30 a.m. on a Saturday morning last month, my best friend, Baylen,
was surprised. It was clear I had no intention of coming back to New
York anytime soon.

We planned on being in Hazard for at least a month, so we packed a lot
of clothes, snacks for the road and toys for Baylen's 10-week-old French
bulldog, \href{https://www.instagram.com/hootboyblue/}{Hootenanny}. We
figured we'd stop every few hours on the 11-hour drive for Hoot to do
his business.

Baylen picked up a rental car at Kennedy Airport the night before we
left, and he ``scrubbed it down,'' he told me. As we drove (well, as he
drove, because I don't drive) across the George Washington Bridge and
out of the city, I felt something I never imagined I could feel --- joy
to be in New Jersey. We drove straight through the Garden State and into
Pennsylvania.

In Lebanon, we stopped at a Starbucks, where we put Hoot's food and
water bowls out in the parking lot and played with him. The puppy got
about 100 compliments and I told maskless people who got too close to me
in an effort to pet him that they could not pet him without a mask on.

We drove for another few hours and stopped at Point Lookout in Green
Ridge State Park in Maryland's Allegany County. Hoot handled his
business, we cleaned up after him, snapped some photos and I did what I
hoped I wouldn't need to do: sought out a public restroom. The one at
the visitor center was closed, so I tried to pee in the woods, but there
was a camera and I am afraid of authority, so I got in the car and we
kept going.

About an hour later, somewhere in West Virginia, we stopped for a
bathroom break. I went into a Wendy's that smelled like bleach and had
handfuls of patrons inside and out. The sign outside the bathroom door
said that only one person could enter at a time. After washing my hands
and using a paper towel to open the door to leave the restroom, I doused
my hands in hand sanitizer even though I knew the soap washing was
plenty.

We got to Kentucky at 5 p.m., made a beeline to the porch. We ate the
first homemade meal that we had not cooked ourselves in months, and I
slept for 10 hours. There were neither fireworks nor ambulance sirens.
\emph{TARIRO MZEZEWA}

There are currently no statewide travel restrictions in
\href{https://www.nytimes.com/interactive/2020/us/kentucky-coronavirus-cases.html}{Kentucky}.
Gov. Andy Beshear signed an executive order in July, mandating all
customers in retail facilities​, in​ grocery stores and in several other
businesses to wear a mask when indoors. ​ If people are outside and
can't maintain six feet of distance from others, they also ​must wear a
mask.

\begin{center}\rule{0.5\linewidth}{\linethickness}\end{center}

\emph{\textbf{Follow New York Times Travel}}
\emph{on}\href{https://www.instagram.com/nytimestravel/}{\emph{Instagram}}\emph{,}\href{https://twitter.com/nytimestravel}{\emph{Twitter}}
\emph{and}\href{https://www.facebook.com/nytimestravel/}{\emph{Facebook}}\emph{.
And}\href{https://www.nytimes.com/newsletters/traveldispatch}{\emph{sign
up for our weekly Travel Dispatch newsletter}} \emph{to receive expert
tips on traveling smarter and inspiration for your next vacation.}

Advertisement

\protect\hyperlink{after-bottom}{Continue reading the main story}

\hypertarget{site-index}{%
\subsection{Site Index}\label{site-index}}

\hypertarget{site-information-navigation}{%
\subsection{Site Information
Navigation}\label{site-information-navigation}}

\begin{itemize}
\tightlist
\item
  \href{https://help.nytimes.com/hc/en-us/articles/115014792127-Copyright-notice}{©~2020~The
  New York Times Company}
\end{itemize}

\begin{itemize}
\tightlist
\item
  \href{https://www.nytco.com/}{NYTCo}
\item
  \href{https://help.nytimes.com/hc/en-us/articles/115015385887-Contact-Us}{Contact
  Us}
\item
  \href{https://www.nytco.com/careers/}{Work with us}
\item
  \href{https://nytmediakit.com/}{Advertise}
\item
  \href{http://www.tbrandstudio.com/}{T Brand Studio}
\item
  \href{https://www.nytimes.com/privacy/cookie-policy\#how-do-i-manage-trackers}{Your
  Ad Choices}
\item
  \href{https://www.nytimes.com/privacy}{Privacy}
\item
  \href{https://help.nytimes.com/hc/en-us/articles/115014893428-Terms-of-service}{Terms
  of Service}
\item
  \href{https://help.nytimes.com/hc/en-us/articles/115014893968-Terms-of-sale}{Terms
  of Sale}
\item
  \href{https://spiderbites.nytimes.com}{Site Map}
\item
  \href{https://help.nytimes.com/hc/en-us}{Help}
\item
  \href{https://www.nytimes.com/subscription?campaignId=37WXW}{Subscriptions}
\end{itemize}
