Sections

SEARCH

\protect\hyperlink{site-content}{Skip to
content}\protect\hyperlink{site-index}{Skip to site index}

\href{/section/business}{Business}\textbar{}China's Swimwear Capital
Can't Wait for You to Go Back to the Beach

\url{https://nyti.ms/2WszLq8}

\begin{itemize}
\item
\item
\item
\item
\item
\item
\end{itemize}

\href{https://www.nytimes.com/news-event/coronavirus?action=click\&pgtype=Article\&state=default\&region=TOP_BANNER\&context=storylines_menu}{The
Coronavirus Outbreak}

\begin{itemize}
\tightlist
\item
  live\href{https://www.nytimes.com/2020/08/01/world/coronavirus-covid-19.html?action=click\&pgtype=Article\&state=default\&region=TOP_BANNER\&context=storylines_menu}{Latest
  Updates}
\item
  \href{https://www.nytimes.com/interactive/2020/us/coronavirus-us-cases.html?action=click\&pgtype=Article\&state=default\&region=TOP_BANNER\&context=storylines_menu}{Maps
  and Cases}
\item
  \href{https://www.nytimes.com/interactive/2020/science/coronavirus-vaccine-tracker.html?action=click\&pgtype=Article\&state=default\&region=TOP_BANNER\&context=storylines_menu}{Vaccine
  Tracker}
\item
  \href{https://www.nytimes.com/interactive/2020/07/29/us/schools-reopening-coronavirus.html?action=click\&pgtype=Article\&state=default\&region=TOP_BANNER\&context=storylines_menu}{What
  School May Look Like}
\item
  \href{https://www.nytimes.com/live/2020/07/31/business/stock-market-today-coronavirus?action=click\&pgtype=Article\&state=default\&region=TOP_BANNER\&context=storylines_menu}{Economy}
\end{itemize}

\includegraphics{https://static01.nyt.com/images/2020/07/16/business/00virus-china-swim-1/merlin_174480408_1785d57d-dbb5-48a6-83cf-4ac99846c144-articleLarge.jpg?quality=75\&auto=webp\&disable=upscale}

\hypertarget{chinas-swimwear-capital-cant-wait-for-you-to-go-back-to-the-beach}{%
\section{China's Swimwear Capital Can't Wait for You to Go Back to the
Beach}\label{chinas-swimwear-capital-cant-wait-for-you-to-go-back-to-the-beach}}

When the pandemic shut down pools and resorts around the globe, life
slid to a halt in the seaside town of Xingcheng, a major producer of
bikinis and trunks.

The beach at Xingcheng, a Chinese town that says it makes a quarter of
the world's swimwear.Credit...Giulia Marchi for The New York Times

Supported by

\protect\hyperlink{after-sponsor}{Continue reading the main story}

\href{https://www.nytimes.com/by/raymond-zhong}{\includegraphics{https://static01.nyt.com/images/2018/10/15/multimedia/author-raymond-zhong/author-raymond-zhong-thumbLarge.png}}

By \href{https://www.nytimes.com/by/raymond-zhong}{Raymond Zhong}

\begin{itemize}
\item
  Published July 17, 2020Updated July 18, 2020
\item
  \begin{itemize}
  \item
  \item
  \item
  \item
  \item
  \item
  \end{itemize}
\end{itemize}

\href{https://cn.nytimes.com/business/20200720/china-coronavirus-swimsuits/}{阅读简体中文版}\href{https://cn.nytimes.com/business/20200720/china-coronavirus-swimsuits/zh-hant/}{閱讀繁體中文版}

There may be no place on earth that had been looking forward to summer
more than Xingcheng, a laid-back seaside town dotted with the occasional
high rise.

Hot sun, cold drinks. Long, languorous days at the beach.

But, most important, swimsuits.

Xingcheng (pronounced SHING-chung), an out-of-the-way factory town on
China's northeastern coast, makes swimwear that is exported to the
United States, Germany, Australia and dozens of other countries --- in
total,
\href{https://www.chinadailyhk.com/articles/121/151/137/1557484560000.html?newsId=82464}{a
quarter of the world's swimwear}, it estimates. This year, though, when
China forced its people to stay home to stop the coronavirus,
Xingcheng's production of trunks, bikinis and one-pieces ground to a
halt.

Then, just as China started getting back to work, the epidemic became a
pandemic, and the rest of the world began shutting down. Demand for
Xingcheng's swimsuits dried up. Factories and workshops that reopened
--- masks, disinfectant and temperature checks in place --- had very
little to do.

Some thought about making other stretchy products instead: yoga clothes,
scuba diving suits, wrestling outfits. But that would have meant buying
new material, finding new suppliers, maybe even investing in new
machines. Starting over, basically.

``Nobody was working. Nobody was earning money,'' said Yao Haifu, 42,
who has worked in swimwear factories in Xingcheng for more than a
decade. ``In a word? It was difficult.''

\includegraphics{https://static01.nyt.com/images/2020/07/16/business/00virus-china-swim-3/merlin_174480246_63cfeff5-b47f-4fb7-ad5f-1071b72d8833-articleLarge.jpg?quality=75\&auto=webp\&disable=upscale}

Mr. Yao runs neon-colored cloth through his sewing machine with nimble,
practiced hands. The motor whines; thread dances and shakes as it
unspools and is sucked into the machine. His colleagues at the factory
sit in rows, heaps of half-finished swimsuits by their sides.

The global contraction is hitting all of China's giant export sector
hard. The country's exports were up only 0.5 percent in June from a year
earlier, even as
\href{https://www.nytimes.com/2020/07/15/business/economy/china-coronavirus-economy.html}{the
overall economy rebounded} more strongly. But as Chinese industrial
towns go, Xingcheng may take longer than most to recover.

Across the globe, pools, beaches and water parks are reopening only
cautiously. Travel and tourism are still mostly nonstarters. Perhaps
never in recent history has so little of humankind had any need for new
swimwear.

Image

A swimsuit at Mr. Yao's workshop. Some factories considered making other
stretchy clothing instead, like wrestling outfits.Credit...Giulia Marchi
for The New York Times

Image

Pattern samples. In 2018, Xingcheng produced \$2 billion worth of
swimwear, according to the official Xinhua news agency.Credit...Giulia
Marchi for The New York Times

And so, with a peak season's worth of sales already largely lost,
Xingcheng's factories are scraping by an order at a time, waiting for
world governments to get a grip on the illness. For fear to abate and
economies to mend. For more people to venture
\href{https://www.nytimes.com/2020/05/27/world/europe/italy-beaches-coronavirus-reopening.html}{back
into the water} --- or even just near it, a beverage in hand.

``It's the same abroad and at home --- there's still no spending
power,'' said Hao Jing, a trader who sells swimsuits from Xingcheng to
international buyers.

\hypertarget{latest-updates-economy}{%
\section{\texorpdfstring{\href{https://www.nytimes.com/live/2020/07/31/business/stock-market-today-coronavirus?action=click\&pgtype=Article\&state=default\&region=MAIN_CONTENT_1\&context=storylines_live_updates}{Latest
Updates:
Economy}}{Latest Updates: Economy}}\label{latest-updates-economy}}

\href{https://www.nytimes.com/live/2020/07/31/business/stock-market-today-coronavirus?action=click\&pgtype=Article\&state=default\&region=MAIN_CONTENT_1\&context=storylines_live_updates\#kodaks-chief-executive-was-given-stock-options-then-the-share-price-spiked-1000-percent}{34h
ago}

\href{https://www.nytimes.com/live/2020/07/31/business/stock-market-today-coronavirus?action=click\&pgtype=Article\&state=default\&region=MAIN_CONTENT_1\&context=storylines_live_updates\#kodaks-chief-executive-was-given-stock-options-then-the-share-price-spiked-1000-percent}{Kodak's
chief executive was given stock options. Then the share price spiked
1,000 percent.}

\href{https://www.nytimes.com/live/2020/07/31/business/stock-market-today-coronavirus?action=click\&pgtype=Article\&state=default\&region=MAIN_CONTENT_1\&context=storylines_live_updates\#fitch-ratings-downgrades-its-outlook-on-us-debt}{37h
ago}

\href{https://www.nytimes.com/live/2020/07/31/business/stock-market-today-coronavirus?action=click\&pgtype=Article\&state=default\&region=MAIN_CONTENT_1\&context=storylines_live_updates\#fitch-ratings-downgrades-its-outlook-on-us-debt}{Fitch
Ratings downgrades its outlook on U.S. debt.}

\href{https://www.nytimes.com/live/2020/07/31/business/stock-market-today-coronavirus?action=click\&pgtype=Article\&state=default\&region=MAIN_CONTENT_1\&context=storylines_live_updates\#us-sanctions-more-chinese-officials-over-human-rights-violations-as-tensions-flare}{44h
ago}

\href{https://www.nytimes.com/live/2020/07/31/business/stock-market-today-coronavirus?action=click\&pgtype=Article\&state=default\&region=MAIN_CONTENT_1\&context=storylines_live_updates\#us-sanctions-more-chinese-officials-over-human-rights-violations-as-tensions-flare}{U.S.
sanctions more Chinese officials over human rights violations as
tensions flare}

\href{https://www.nytimes.com/live/2020/07/31/business/stock-market-today-coronavirus?action=click\&pgtype=Article\&state=default\&region=MAIN_CONTENT_1\&context=storylines_live_updates}{See
more updates}

More live coverage:
\href{https://www.nytimes.com/2020/08/01/world/coronavirus-covid-19.html?action=click\&pgtype=Article\&state=default\&region=MAIN_CONTENT_1\&context=storylines_live_updates}{Global}

Xingcheng is not a particularly well-known town even within China. But
it produced \$2 billion worth of swimwear in 2018, according to
\href{http://www.xinhuanet.com/2019-07/05/c_1124716951.htm}{the
government's official Xinhua news agency}. There are 1,200 swimwear
companies in the town, Xinhua says, employing as many as 100,000 people,
or one in five residents.

Beijing

Xingcheng

Bohai Sea

Yellow R.

Yellow

Sea

CHINA

Shanghai

Yangtze R.

TAIWAN

Hong Kong

South

China Sea

200 miles

By The New York Times

This is hardly the most obvious place to find an industry specializing
in beachy coverings.

Xingcheng sits on the Bohai Sea, in a part of the country that most
people in China associate with smog and brutal winters, not short shorts
and floral patterns.

Image

``Bikini small village,'' a Xingcheng precinct for swimsuit companies
and showrooms.Credit...Giulia Marchi for The New York Times

But for people living nearby, Xingcheng is a pleasant enough place to
watch the waves roll in. The summer temperatures are mild. The morning
air pollution wafts away by midday. There are
\href{http://www.weather.com.cn/cityintro/101071401.shtml}{2,600 hours
of sunshine a year} --- not Miami, but not Edinburgh, either.

There is khaki sand and green-blue water and a wide wooden boardwalk
that lights up in the evenings. People set out tables and chairs on the
sand to eat fish and drink beer. A pretty pavilion catches the breeze at
the end of a long walkway that juts into the sea. The factory zone is a
short drive away.

Mr. Yao, whose broad, boyish face is topped with a curly pat of hair,
sews swimsuits in a small factory --- 40 or so workers --- above an auto
repair shop. He is busier than he was a few months ago, when orders
seemed nonexistent. Many evenings, he even works overtime.

Image

``There's still no spending power,'' said Hao Jing, who sells swimsuits
from Xingcheng to international buyersCredit...Giulia Marchi for The New
York Times

Image

The 1,200 local swimwear companies employ as many as 100,000 people,
Xinhua says.Credit...Giulia Marchi for The New York Times

But Mr. Yao's sense is that the garments he is helping to produce are
largely going into warehouses instead of being sold right away. Swimwear
brands are still just stocking up for when customers want to buy again,
he said.

``Once there's demand, they can sell these orders and make up for the
shortfall during this period,'' he said.

For Zhao Yang's company in Xingcheng, which employs around 70 workers
and designers, swimwear orders are starting to trickle in again, though
business is not exactly growing. No new customers had approached him
lately, he said. He is still working his way through a huge stock of
fabric he bought before the Lunar New Year, in anticipation of a busy
spring production season that never was.

Mostly, he is spending time at home with his family, waiting out the
economic dislocation.

``A slower pace of life is not a bad thing,'' Mr. Zhao, 39, said.

Image

Swimwear brands are still doing more stocking up than selling, Mr. Yao
said.Credit...Giulia Marchi for The New York Times

The swimsuit industry in Xingcheng took its
\href{https://www.jiemian.com/article/3126567.html}{first steps in the
1980s}, at the dawn of private enterprise in China.

According to
\href{https://www.weibo.com/p/1005055127456388?is_all=1}{``This City and
Swimwear,''} a book by a local journalist, Han Wenxin, residents of the
nearby village of Beiguancun sewed bathing suits at home and began
selling them at Xingcheng's beaches. In time, factories were built and
merchants started selling farther afield --- in Beijing, in Russia, in
South Africa and beyond.

As business grew, Xingcheng hosted
\href{http://www.zxzxnews.com/Fashion/2019/0903/17326.html}{swimwear
expos} and \href{https://v.qq.com/x/page/z07650idifb.html}{runway shows}
set to thumping club music. If you were in New York's Times Square
\href{https://www.prnewswire.com/news-releases/a-video-clip-telling-about-chinese-swimsuit-shines-at-nyc-times-square-300902370.html}{last
year}, you might have looked up and glimpsed
\href{https://mp.weixin.qq.com/s/B2nlOwy3hv9slQAfvI9rYA}{a short
propaganda film} about Xingcheng's swimwear industry.

Xingcheng sells internationally with the help of people like Ms. Hao,
the trader, who works in the southeastern wholesale hub of Yiwu.

She is spending her days reading news about the pandemic on her phone
but not doing much business. A customer from Dubai expressed interest in
buying some swimsuits but never made a down payment. A buyer in India
hinted at an order but didn't pull the trigger.

Image

Qi Lei, who owns a factory that cuts fabric for swimsuits, said he
worried about the industry's future.Credit...Giulia Marchi for The New
York Times

Qi Lei employs around 10 people in his airy Xingcheng factory, where
industrial machines cut fabric on room-length tables for factories that
stitch the pieces into swimsuits. All around his factory there is cloth
in great bolts, cloth stuffed into giant plastic bags, cloth in a riot
of colors and patterns, from turquoise and snakeskin to tropical flowers
on a chevron background.

Work is finding its way to Mr. Qi --- he is cutting a lot of bikinis, he
said. But he can see that many factories in town are not as lucky. Some
workers are still idling at home.

``Because of the epidemic this year, if you're doing business, you're by
and large losing money,'' he said.

Mr. Qi is proud to help Xingcheng make swimwear for the world. But he
worries about the industry's future. Young people do not want to sew
swimsuits anymore, he said. They are not as willing as their forebears
to grit their teeth and work hard --- to ``eat bitter,'' as people in
China say.

``Pretty much everyone who works in this industry was a young girl 20
years ago,'' Mr. Qi said. ``Now they're getting on in years.''

Image

One of Mr. Qi's workers. If work doesn't pick up, ``I guess I'll just
have to scrimp and make do,'' he said.Credit...Giulia Marchi for The New
York Times

China's factory sector was built over the past four decades on its
people's ability to move quickly and adapt. People from the countryside
poured into cities and towns, ready to take whatever opportunities their
skills afforded them. The demands of the world market changed so rapidly
that it helped not to get too tied up in any one job or industry.

Today, in factory belts around China, the pandemic is testing that
resilience anew. What if summer comes and goes and swimsuit sales still
don't pick up in a major way? What happens to Xingcheng then?

``Supposing there's no work for another year, I guess I'll just have to
scrimp and make do,'' Mr. Qi said. ``I don't have any other ideas.''

Image

A short drive from the factories is the beach, where restaurants like
this one and a boardwalk light up at night.Credit...Giulia Marchi for
The New York Times

Wang Yiwei contributed research.

Advertisement

\protect\hyperlink{after-bottom}{Continue reading the main story}

\hypertarget{site-index}{%
\subsection{Site Index}\label{site-index}}

\hypertarget{site-information-navigation}{%
\subsection{Site Information
Navigation}\label{site-information-navigation}}

\begin{itemize}
\tightlist
\item
  \href{https://help.nytimes.com/hc/en-us/articles/115014792127-Copyright-notice}{©~2020~The
  New York Times Company}
\end{itemize}

\begin{itemize}
\tightlist
\item
  \href{https://www.nytco.com/}{NYTCo}
\item
  \href{https://help.nytimes.com/hc/en-us/articles/115015385887-Contact-Us}{Contact
  Us}
\item
  \href{https://www.nytco.com/careers/}{Work with us}
\item
  \href{https://nytmediakit.com/}{Advertise}
\item
  \href{http://www.tbrandstudio.com/}{T Brand Studio}
\item
  \href{https://www.nytimes.com/privacy/cookie-policy\#how-do-i-manage-trackers}{Your
  Ad Choices}
\item
  \href{https://www.nytimes.com/privacy}{Privacy}
\item
  \href{https://help.nytimes.com/hc/en-us/articles/115014893428-Terms-of-service}{Terms
  of Service}
\item
  \href{https://help.nytimes.com/hc/en-us/articles/115014893968-Terms-of-sale}{Terms
  of Sale}
\item
  \href{https://spiderbites.nytimes.com}{Site Map}
\item
  \href{https://help.nytimes.com/hc/en-us}{Help}
\item
  \href{https://www.nytimes.com/subscription?campaignId=37WXW}{Subscriptions}
\end{itemize}
