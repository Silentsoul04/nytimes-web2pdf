Sections

SEARCH

\protect\hyperlink{site-content}{Skip to
content}\protect\hyperlink{site-index}{Skip to site index}

\href{https://www.nytimes.com/section/us}{U.S.}

\href{https://myaccount.nytimes.com/auth/login?response_type=cookie\&client_id=vi}{}

\href{https://www.nytimes.com/section/todayspaper}{Today's Paper}

\href{/section/us}{U.S.}\textbar{}Federal Agents Unleash Militarized
Crackdown on Portland

\url{https://nyti.ms/398rkoM}

\begin{itemize}
\item
\item
\item
\item
\item
\item
\end{itemize}

\href{https://www.nytimes.com/news-event/george-floyd-protests-minneapolis-new-york-los-angeles?action=click\&pgtype=Article\&state=default\&region=TOP_BANNER\&context=storylines_menu}{Race
and America}

\begin{itemize}
\tightlist
\item
  \href{https://www.nytimes.com/2020/07/26/us/protests-portland-seattle-trump.html?action=click\&pgtype=Article\&state=default\&region=TOP_BANNER\&context=storylines_menu}{Protesters
  Return to Other Cities}
\item
  \href{https://www.nytimes.com/2020/07/24/us/portland-oregon-protests-white-race.html?action=click\&pgtype=Article\&state=default\&region=TOP_BANNER\&context=storylines_menu}{Portland
  at the Center}
\item
  \href{https://www.nytimes.com/2020/07/23/podcasts/the-daily/portland-protests.html?action=click\&pgtype=Article\&state=default\&region=TOP_BANNER\&context=storylines_menu}{Podcast:
  Showdown in Portland}
\item
  \href{https://www.nytimes.com/interactive/2020/07/16/us/black-lives-matter-protests-louisville-breonna-taylor.html?action=click\&pgtype=Article\&state=default\&region=TOP_BANNER\&context=storylines_menu}{45
  Days in Louisville}
\end{itemize}

Advertisement

\protect\hyperlink{after-top}{Continue reading the main story}

Supported by

\protect\hyperlink{after-sponsor}{Continue reading the main story}

\hypertarget{federal-agents-unleash-militarized-crackdown-on-portland}{%
\section{Federal Agents Unleash Militarized Crackdown on
Portland}\label{federal-agents-unleash-militarized-crackdown-on-portland}}

Federal authorities said they would bring order to Portland, Ore., after
weeks of protests there. Local leaders believe the federal presence is
making things worse.

\includegraphics{https://static01.nyt.com/images/2020/07/17/us/17UNREST-PORTLAND1/merlin_174476607_4ba6b1fa-ca04-4341-9773-d38f3b175fe9-articleLarge.jpg?quality=75\&auto=webp\&disable=upscale}

By Sergio Olmos, \href{https://www.nytimes.com/by/mike-baker}{Mike
Baker} and \href{https://www.nytimes.com/by/zolan-kanno-youngs}{Zolan
Kanno-Youngs}

\begin{itemize}
\item
  Published July 17, 2020Updated July 31, 2020
\item
  \begin{itemize}
  \item
  \item
  \item
  \item
  \item
  \item
  \end{itemize}
\end{itemize}

PORTLAND, Ore. --- Federal agents dressed in camouflage and tactical
gear have taken to the streets of
\href{https://www.nytimes.com/2020/07/23/podcasts/the-daily/portland-protests.html}{Portland},
unleashing tear gas, bloodying
\href{https://www.nytimes.com/2020/07/21/us/portland-protests.html}{protesters}
and pulling some people into unmarked vans in what Gov. Kate Brown of
Oregon has called ``a blatant abuse of power.''

The
\href{https://www.nytimes.com/2020/07/17/us/politics/federal-agents-portland-arrests.html}{extraordinary
use of federal force} in recent days, billed as an attempt to tamp down
persistent unrest and protect government property, has infuriated local
leaders who say the agents have stoked tensions. ``This is an attack on
our democracy,'' Mayor Ted Wheeler of Portland said.

Late Friday night, Oregon's attorney general, Ellen Rosenblum, said her
office had opened a criminal investigation into how one protester was
injured near a federal courthouse. She also filed a lawsuit in Federal
District Court accusing the federal agents of engaging in unlawful
tactics and seeking a restraining order.

The strife in Portland, which has had 50 consecutive days of protests,
reflects the growing fault lines in law enforcement as President Trump
threatens an assertive federal role in how cities manage a wave of
national unrest after George Floyd was killed by the Minneapolis police.

One Portland demonstrator, Mark Pettibone, 29, said he had been part of
the protests before four people in camouflage jumped out of an unmarked
van around 2 a.m. Wednesday. They had no obvious markings or
identification, he said, and he had no idea who they were.

``One of the officers said, `It's OK, it's OK,' and just grabbed me and
threw me into the van,'' Mr. Pettibone said. ``Another officer pulled my
beanie down so I couldn't see.''

Mr. Pettibone said that he was terrified --- protesters in the city have
in the past clashed with far-right militia groups also wearing
camouflage and tactical gear --- and that at no point was he told why he
was arrested or detained, or what agency the officers were with. He said
he was held for about two hours before being released.

``It felt like I was being hunted for no reason,'' Mr. Pettibone said.
``It feels like fascism.''

In a statement issued on Friday, Customs and Border Protection described
one case captured on video, saying agents who made an arrest had
information that indicated a suspect had assaulted federal authorities
or damaged property and that they moved him to a safer location for
questioning. The statement, which did not name any suspects, said that
the agents identified themselves but that their names were not displayed
because of ``recent doxxing incidents against law enforcement
personnel.''

The agents in Portland are part of ``rapid deployment teams'' put
together by the Department of Homeland Security after Mr. Trump directed
federal agencies to deploy additional personnel to protect statues,
monuments and federal property during the continuing unrest.

The teams, which include
\href{https://www.nytimes.com/2020/07/10/us/politics/homeland-security-statues-trump.html}{2,000
officials} from Customs and Border Protection, Immigration and Customs
Enforcement, the Transportation Security Administration and the Coast
Guard, are supporting the Federal Protective Service, an agency that
already provides security at federal properties.

Agents have been dispatched to Portland, Seattle and Washington, D.C.,
to guard statues, monuments and federal property, such as the federal
courthouse in Portland, according to homeland security officials.

But the response by the homeland security agents in Portland has
prompted backlash over whether the federal officers are exceeding their
arrest authority and violating the rights of protesters by detaining
demonstrators in the area around the federal courthouse.

The agents have the authority to make arrests if they believe that a
federal crime has been committed. Homeland security has pointed to
dozens of possible crimes in Portland, such as damaging of the federal
courthouse, spray-painting of graffiti on federal property and the
throwing of rocks and bottles at officers.

Law enforcement officials say it is rare for local police departments to
request help from federal authorities --- or for the federal government
to deploy in a city without that consent --- because of the risk of
escalating an already volatile environment.

``The last people you really want are any of these federal officials,''
said Gil Kerlikowske, the former commissioner of Customs and Border
Protection and the former chief of the Seattle Police Department.

Billy J. Williams, the U.S. attorney for the District of Oregon, said in
a statement on Friday that he was asking the Department of Homeland
Security's inspector general to investigate reports of officers
detaining protesters.

Governor Brown said in an interview that she asked the acting homeland
security secretary, Chad F. Wolf, to remove federal officials from the
streets and that he refused. She said the Trump administration appeared
to instead be using the situation for photo-ops to rally his supporters.

``They are provoking confrontation for political purposes,'' Ms. Brown
said.

In early June, the administration deployed an array of federal agents to
cities like San Diego, Buffalo and Las Vegas.

In Washington, tensions were heightened when the Park Police and Secret
Service used chemical agents to disperse a crowd of protesters in
Lafayette Park for a photo opportunity by Mr. Trump. Federal agents
without any insignia also sparked fear and confusion in the
demonstrations, and
\href{https://www.nytimes.com/2020/06/06/us/politics/protests-trump-helicopters-national-guard.html}{military
helicopters} flying below rooftop level sent protesters scurrying for
cover.

Customs and Border Protection also sent drones, helicopters and planes
to
\href{https://www.nytimes.com/2020/06/19/us/politics/george-floyd-protests-surveillance.html}{conduct
surveillance of the protests in 15 cities.}

Mr. Wolf, who arrived in Portland on Thursday, called the protesters a
``violent mob'' of anarchists emboldened by a lack of local enforcement.

Federal officers on the ground in Portland have deployed a range of
forceful tactics: They appeared to fire less-lethal munitions from slits
in the facade of the federal courthouse, one officer walked the street
while swinging a burning ball emitting tear gas, and camouflaged
personnel drove in unmarked vans.

Homeland security officers have been dispatched to help local law
enforcement in the past, but typically when a request was made by local
government or when there was a ``national special security event''
taking place that could be especially vulnerable to terrorism, such as
the U.N. General Assembly or the Super Bowl.

Harry Fones, a homeland security spokesman, did not answer questions
seeking additional details about the tactics of the officers in
Portland, instead referring to
\href{https://www.cbp.gov/newsroom/speeches-and-statements/statement-cbp-response-portland-oregon}{a
Customs and Border Protection statement} that said the federal officers
did display insignia.

Mark Morgan, the acting commissioner of Customs and Border Protection,
\href{https://twitter.com/CBPMarkMorgan/status/1284206662434402305}{said
in a series of tweets} on Friday that the agents from BORTAC, the
equivalent of the agency's SWAT team, would ``continue to arrest the
violent criminals that are destroying federal property \& injuring our
agents/officers in Portland.''

The demonstrations began in the aftermath of Mr. Floyd's death in
Minneapolis, drawing thousands of people to the streets to denounce
police violence and racial injustice. On some nights, protesters would
blanket the Burnside Bridge, each lying face down on the pavement for
\href{https://www.nytimes.com/2020/06/18/us/george-floyd-timing.html}{8
minutes and 46 seconds} in remembrance of Mr. Floyd.

Those mass demonstrations have waned, but hundreds have continued on,
clashing with the police almost nightly. They have set off fireworks,
lit fires and attempted to create an autonomous zone similar to
\href{https://www.nytimes.com/2020/07/01/us/seattle-protest-zone-CHOP-CHAZ-unrest.html}{one
that existed up Interstate 5 in Seattle}. Police officers have responded
with tear gas, although a federal judge has since limited the use of
that tactic, and dozens have been arrested.

The persistent unrest has frustrated city leaders, including Mr.
Wheeler, who has often been a target of protesters. Some Black leaders
in the community have also expressed disappointment, suggesting that the
predominantly white protest crowd was seizing an opportunity and
detracting from the vital efforts needed to reform policing.

City leaders have tried a variety of tactics to calm the tensions. Mr.
Wheeler has pleaded for calm. The city's police chief resigned. City
commissioners have moved to cut some \$16 million from the police
budget.

But the protests have continued.

Mr. Trump has
\href{https://www.nytimes.com/2020/06/02/us/politics/trump-law-enforcement-protests.html}{vowed
to ``dominate'' protesters} and said last week that he had sent homeland
security personnel to Portland because ``the locals couldn't handle
it.''

``It's a pretty wild group, but you have it in very good control,'' he
told Mr. Wolf.

One recent video appeared to show a protester, Donavan La Bella, being
struck in the head by an impact munition while he was holding a speaker
across the street from the federal courthouse, leading to a bloody
scene. His mother has told local media that
\href{https://www.oregonlive.com/news/2020/07/police-shoot-portland-protester-in-head-with-impact-weapon-causing-severe-injuries.html}{he
suffered skull fractures and needed surgery}.

Members of Congress from Oregon have called for an investigation, and
Mr. Williams said the encounter had been referred to the Justice
Department's inspector general for further investigation. The state
attorney general said on Friday that the agency and the Multnomah County
district attorney had opened a criminal investigation.

Kelly Simon, the interim legal director at the American Civil Liberties
Union of Oregon, said that the alarming federal tactics, such as the
unmarked vans, have been used at times to intimidate immigrant
communities, and that she worried the use of the tactics was growing.

``What we're seeing in Portland should concern everybody in this
country,'' Ms. Simon said.

\includegraphics{https://static01.nyt.com/images/2017/01/29/podcasts/the-daily-album-art/the-daily-album-art-articleInline-v2.jpg?quality=75\&auto=webp\&disable=upscale}

\hypertarget{listen-to-the-daily-the-showdown-in-portland}{%
\subsubsection{Listen to `The Daily': The Showdown in
Portland}\label{listen-to-the-daily-the-showdown-in-portland}}

Why have militarized federal authorities been deployed to an American
city?

transcript

Back to The Daily

bars

0:00/30:04

-30:04

transcript

\hypertarget{listen-to-the-daily-the-showdown-in-portland-1}{%
\subsection{Listen to `The Daily': The Showdown in
Portland}\label{listen-to-the-daily-the-showdown-in-portland-1}}

\hypertarget{hosted-by-michael-barbaro-produced-by-andy-mills-and-austin-mitchell-with-help-from-robert-jimison-and-stella-tan-and-edited-by-mj-davis-lin}{%
\subsubsection{Hosted by Michael Barbaro; produced by Andy Mills and
Austin Mitchell; with help from Robert Jimison and Stella Tan; and
edited by M.J. Davis
Lin.}\label{hosted-by-michael-barbaro-produced-by-andy-mills-and-austin-mitchell-with-help-from-robert-jimison-and-stella-tan-and-edited-by-mj-davis-lin}}

\hypertarget{why-have-militarized-federal-authorities-been-deployed-to-an-american-city}{%
\paragraph{Why have militarized federal authorities been deployed to an
American
city?}\label{why-have-militarized-federal-authorities-been-deployed-to-an-american-city}}

\begin{itemize}
\item
  mike baker\\
  This is Mike Baker, a correspondent for The New York Times based in
  the Northwest. It's 2:00 a.m. right now. I'm in downtown Portland
  watching through some clouds of tear gas. There's a group of
  protesters right now. {[}CLEARS THROAT{]} I can feel the tear gas.
  {[}COUGHS{]}

  I am watching here through clouds of tear gas. A group of protesters
  moving down Main Street. They've got their umbrellas out to protect
  themselves. And just down the street is a line of federal officers.
  They're firing --- {[}SOUND OF TEAR GAS FIRING{]} firing tear gas down
  at the crowd. The officers are standing in a long line down the city
  block protecting the federal courthouse.
\end{itemize}

michael barbaro

From the New York Times, I'm Michael Barbaro. This is ``The Daily.''
Today: Inside the volatile situation in Portland, Oregon, and why
federal forces are being deployed to American cities. It's Thursday,
July 23.

Zolan Kanno-Youngs, you cover the Department of Homeland Security for
The Times, the entire universe of federal law enforcement. So where does
the story of what's happening right now in Portland, where does it
start?

zolan kanno-youngs

So I think we have to go back to late May. In late May, as we know,
there were protests sweeping throughout the country. Mass
demonstrations. A majority of those protests involved people who were
demonstrating peacefully. But you did also have instances of people
damaging property, looting, as well as acts of violence. And in Oakland,
you had a situation where an officer with the Federal Protective
Service, an arm of the Department of Homeland Security, who was guarding
a federal courthouse, was actually shot and killed. I should say that
the person who shot and killed him was actually affiliated with a fringe
anti-government movement and wasn't affiliated with the protests. But
that killing did prompt ---

\begin{itemize}
\tightlist
\item
  archived recording\\
  Good afternoon ---
\end{itemize}

zolan kanno-youngs

--- a rare press conference.

\begin{itemize}
\tightlist
\item
  archived recording\\
  The Department of Homeland Security's highest priority is to ensure
  the safety and security of the American people and the Department's
  workforce.
\end{itemize}

zolan kanno-youngs

From the top senior officials from the Department of Homeland Security.

\begin{itemize}
\tightlist
\item
  archived recording\\
  Any loss in the D.H.S. family impacts all of us, and I want the loved
  ones of these brave officers to know that you have the support of the
  department behind you.
\end{itemize}

zolan kanno-youngs

They go out there, and of course, they honor the memory of this officer,
but they also have a message.

\begin{itemize}
\tightlist
\item
  archived recording (kenneth t. cuccinelli ii)\\
  There are currently threats by some to attack police stations and
  federal buildings. That violence not only won't be tolerated, we are
  also committed to ensuring that it won't succeed anywhere. Anywhere.
  And let me be clear ---
\end{itemize}

zolan kanno-youngs

They make it clear that they are going to take action against anybody
that makes a threat or has any sort of action against federal property.
The acting deputy secretary Kenneth Cuccinelli even says ---

\begin{itemize}
\tightlist
\item
  archived recording (kenneth t. cuccinelli ii)\\
  That is an act of domestic terrorism.
\end{itemize}

zolan kanno-youngs

--- that would be an act of domestic terrorism.

\begin{itemize}
\tightlist
\item
  archived recording (kenneth t. cuccinelli ii)\\
  Thank you very much.
\end{itemize}

michael barbaro

And Zolan, why is that phrase significant, domestic terrorism?

zolan kanno-youngs

The reason why this is significant is you have to remember how this
department was created in the wake of the September 11th attacks. This
department was formed in the Bush administration to have a coordinated
effort in the federal government to defend the United States against
national security threats --- directly at that time, foreign terrorism
threats. This was a department that was going to protect the borders of
the United States. And this signaled that the top officials in that
department were turning their attention inward, domestically, to these
protests that are sweeping major cities.

michael barbaro

So what happens after this news conference, which, from what you're
describing, feels like more of a statement than a set of actions?

zolan kanno-youngs

Right. I think at that point, it's a message. The message is we're not
going to tolerate this, right? It's clear. But then things start to move
pretty fast.

Within two days on June 1, we start to see that the department is going
to back up this rhetoric with the concrete action of federal resources.
I remember early in the day, you know, I got a message from a source who
sent me an alert that all Homeland Security investigation special agents
around the Washington, D.C. area got, and it said, you have to be on
standby for any potential unrest later today around the area of
Lafayette Park.

{[}music{]}

So that day, you know, later on, that's where you saw the images of
Secret Service, D.E.A., National Guard, Customs and Border Protection,
Immigration and Customs Enforcement as well. And of course, it was many
of those same federal officials and agents who were stationed outside of
Lafayette Park and would clear out protesters to make room for the
president's photo op.

michael barbaro

So we're now seeing the message delivered at that news conference put
into action on the streets of Washington.

zolan kanno-youngs

That's right. And I mean, if you listen to the senior officials with the
Department of Homeland Security, as well as other officials in the Trump
administration, they would say, look, this federal presence was needed
in Washington. Our agents in front of the White House were being
threatened. And they would also say, well, look, after about a week, the
unrest calmed down.

michael barbaro

So from their perspective, as controversial as some of these actions
were, and as intimidating and unusual as it felt on the ground, this was
working.

zolan kanno-youngs

That's right. That's right. It worked. Their deployment worked if you
were to ask them.

michael barbaro

So what happens next?

zolan kanno-youngs

OK, so over the next few weeks, what really happened is we saw a shift.

\begin{itemize}
\tightlist
\item
  archived recording\\
  A tense standoff with police as protesters tried to tear down a statue
  of former president Andrew Jackson.
\end{itemize}

zolan kanno-youngs

Now we're starting to see protesters and demonstrators honing in and
focusing on statues and memorials.

\begin{itemize}
\tightlist
\item
  archived recording\\
  We're addressing white supremacy finally, and it's just something that
  we grew up with. And it's just been so normalized that the people on
  our money would have owned me.
\end{itemize}

zolan kanno-youngs

Targeting those statues and memorials, sometimes pulling them down,
sometimes defacing them. And you also saw a pretty prompt reaction by
the federal government.

\begin{itemize}
\tightlist
\item
  archived recording (donald trump)\\
  They're bad people. They don't love our country, and they're not
  taking down our monuments.
\end{itemize}

zolan kanno-youngs

So in late June ---

\begin{itemize}
\tightlist
\item
  archived recording (donald trump)\\
  I will have an executive order very shortly.
\end{itemize}

zolan kanno-youngs

--- the president then signs an executive order. The gist of it pretty
much says that the attorney general as well as the acting Secretary of
Homeland Security should direct their resources to defend statues and
monuments and federal property. Just a couple days later, the Department
of Homeland Security then formed a task force, what's known as these
rapid deployment teams. Those teams involve 2,000 officers and agents
that are on standby --- from air marshals with the T.S.A., to tactical
agents with Customs and Border Protection, to special agents with
I.C.E., ready on standby to be deployed throughout the U.S.

michael barbaro

And how unusual is this kind of rapid deployment that you're describing?

zolan kanno-youngs

Well I mean, actually, the department, when it was formed --- and many
former officials with the department would say this as well --- that
flexibility to be able to move different officials around is an
advantage, right? It was actually an intention as well to be able to
have these different agencies support one another. But it's the mission
here, deploying them for monuments and statues, you know, the appearance
of these teams in front of the National Mall and Gettysburg. That's
where many observers, as well as some of the architects of the
department, raised an eyebrow at this.

michael barbaro

Why?

zolan kanno-youngs

This country is grappling with a couple different national emergencies
right now. The Department of Homeland Security also has a huge stake in
the response to the pandemic. We have an election coming up as well. The
department is the agency tasked with cyber security. So it was a
question over priorities.

But for the department, it really comes down to this. Are any of these
people in these crowds committing the federal crime of defacing federal
property? The acting secretary has said that he sees it as his job to
deploy if there is any mere violation of that federal law, whether it be
graffiti on a property or some of the more violent acts that we've seen
in these demonstrations.

And it's that rationale that the department used that weekend, the
weekend of July 4, to start deploying these teams to different cities,
but primarily to Portland.

{[}music{]}

michael barbaro

We'll be right back.

Mike Baker, I just spoke with our colleague Zolan, who explained how
this has all unfolded in Washington over the past few weeks. But you are
actually on the ground in Portland. So help us understand what it has
looked like there during that same period.

mike baker

You know, it began with a similar sort of scene that we saw around the
country.

\begin{itemize}
\tightlist
\item
  archived recording\\
  We matter! We all matter! Black lives matter! {[}CHEERING{]}
\end{itemize}

mike baker

The mass peaceful demonstrations.

\begin{itemize}
\tightlist
\item
  archived recording\\
  {[}CHANTING{]} George Floyd! Say his name! George Floyd! Say his name!
\end{itemize}

mike baker

Thousands of people on the streets. There are really powerful images
here in Portland of crowds covering the entire Burnside Bridge over the
Willamette River, you know, in honor of George Floyd. And at the same
time, you got what we saw and a lot of cities.

\begin{itemize}
\tightlist
\item
  archived recording\\
  Windows shattered, graffiti everywhere.
\end{itemize}

mike baker

Smashing windows of businesses.

\begin{itemize}
\item
  archived recording 1\\
  Well, there's a variety store, the Nike community store, Starbucks got
  hit.
\item
  archived recording 2\\
  You're looking at some pictures that show the fires that were set.
\end{itemize}

mike baker

The first night of protests they broke into the Justice Center and lit
fires. But what's really been different here is the persistence of it.
We're now more than 50 consecutive days into the protests happening
every night.

michael barbaro

Wow, 50 days nonstop.

mike baker

Nonstop, every night.

michael barbaro

And what have these nightly confrontations in Portland looked like?

mike baker

You know, it's all over the place. You know, in some of these
confrontations, many of which you can see in videos online, you can see
these standoffs between protesters and police, where some protesters
will throw water bottles or fireworks.

\begin{itemize}
\tightlist
\item
  archived recording\\
  {[}SOUNDS OF SHATTERING AND CLAMORING{]}
\end{itemize}

mike baker

Videos of them breaking windows of buildings downtown or setting up
barricades in the streets.

Police claim they've had bricks thrown at them, rocks thrown at them.

There've been videos surfacing online of people shooting guns in the
air. One group set a fire in the headquarters of the police union, the
local police union. And throughout much of this time, they made it
really their nightly routine to gather downtown right next to the
federal courthouse.

\begin{itemize}
\tightlist
\item
  archived recording\\
  This is the Portland Police Bureau. This is a civil disturbance, and
  we have declared an unlawful assembly. Leave the area now, or you'll
  be subject to use of force to include crowd control munitions. Leave
  now.
\end{itemize}

mike baker

Police kept coming out, arresting a number of people and responding with
so much tear gas that some of these protesters went to court, sued and
won a judge's order limiting how much this gas could get used.

\begin{itemize}
\tightlist
\item
  archived recording\\
  --- never seen or covered anything like this. The damage and just the
  impact of the statement being made is unprecedented. It's crazy to
  see.
\end{itemize}

mike baker

Just been a persistent issue that they haven't really been able to
resolve.

michael barbaro

And who are the people who are involved in these nightly encounters, as
best you can tell?

mike baker

It's a group with a wide range of backgrounds, ideologies, strategies,
tactics that they've brought. You know, Portland has a history of
anarchist groups. And you can see some of the anarchist symbols on the
streets. You see a lot of people wearing all black clothing, which is
pretty common for those who are part of the Antifa group. And then you
have people who are part of the Black Lives Matter movement chanting the
name of George Floyd and just --- so you really have this huge mix.

michael barbaro

Mike, in your time in Portland, I imagine you're talking to people in
the city about this ongoing problem. What are people you've talked to in
Portland saying about the situation?

mike baker

Yeah, you've got, I mean, it seems like a pretty broad consensus of
people who sympathize with the overall message of the protesters --- the
need for police reform and the need for resolving racial injustices. At
the same time, those same people are, you know, frustrated by what seems
like a line of protest that won't seem to end. Business people I talked
to, who, you know, have had their windows boarded up and then shortened
their hours for safety reasons. And one of them I talked to is
considering, like, maybe it's time to just get out of here because there
doesn't seem to be a resolution ahead.

\begin{itemize}
\tightlist
\item
  archived recording\\
  We are physically and emotionally in pain. I have officers that are
  injured ---
\end{itemize}

mike baker

From police, you hear them saying essentially that they're out of ideas.

\begin{itemize}
\tightlist
\item
  archived recording\\
  We love our community. We want to serve our community and facilitate
  free speech.
\end{itemize}

mike baker

Saying that they're exhausted and in pain, and they're trying to show
that they're part of the community, too. That they aren't some sort of
outside force that's here.

\begin{itemize}
\tightlist
\item
  archived recording\\
  We're at a loss for other solutions right now, and I'm open to any
  community member who's got ideas for other solutions. We all are.
\end{itemize}

michael barbaro

So I have this sense at this point --- correct me if I'm wrong --- that
the police don't quite know how to resolve these nightly encounters. And
these nightly encounters are still happening. And so, is there some
sense of resignation that this is just kind of how it is going to be for
a while?

mike baker

Yeah, I mean, there is certainly no deadline that was going to be coming
up. There's a hope that things were on a better track, that the numbers
that were coming out each night were starting to shrink a little bit,
and that they might be on a pathway to finishing this. And that's when a
deployment of federal officers arrived in town.

michael barbaro

So what happens when those federal officials start showing up and at the
direction of the Department of Homeland Security?

mike baker

Well, I mean, right away, you can see that they're standing out. I mean,
they've shown up here in camouflage fatigues and tactical gear. So just
just visually it's pretty clear that there's an outside force that has
now arrived. And they've come with a pretty aggressive posture.

michael barbaro

And what are these aggressive tactics from the federal forces there
looking like?

\begin{itemize}
\tightlist
\item
  archived recording\\
  {[}BOOM SOUNDS{]}
\end{itemize}

mike baker

Well, some of it, you know, in the streets, you can see a return to a
large amount of tear gas, because these federal officers were not under
the same mandates as local police.

But then there were also tactics you could see coming out in different
videos.

In the first one, you have this protester standing across the street
from the federal courthouse. He's got a boombox over his head, and he's
just cursing at the officers across the street.

\begin{itemize}
\tightlist
\item
  archived recording\\
  {[}EXPLETIVE{]} you!
\end{itemize}

mike baker

All of a sudden, you see him drop to the ground.

He's apparently been shot with some sort of less lethal munition and
really just created a bloody scene right there on the street.

Blood all over the sidewalk, and his family says he had to go to the
hospital for more than a week. In these other videos, you have these
protesters ---

\begin{itemize}
\item
  archived recording 1\\
  What are you doing?
\item
  archived recording 2\\
  I haven't done anything wrong.
\item
  archived recording 3\\
  What is going on? Who are you?
\end{itemize}

mike baker

--- on the streets of Portland, and federal officers again in camouflage
and tactical gear approaching them, grabbing them, and then pulling them
back to unmarked vans ---

\begin{itemize}
\tightlist
\item
  archived recording (interposing voices)\\
  Don't hurt him. No, don't hurt him. He's hurt!
\end{itemize}

mike baker

--- filled with officers in tactical gear.

\begin{itemize}
\item
  archived recording 1\\
  This is an unmarked car. Who is this? Who are you? Where are you
  taking her?
\item
  archived recording 2\\
  You follow us, you will get shot. You understand me?
\item
  archived recording 3\\
  What is happening?
\end{itemize}

michael barbaro

And what is the response to these videos?

mike baker

I mean, you've got outrage from not just the protesters, but from the
same city officials that have been the target of the protesters all
along.

\begin{itemize}
\tightlist
\item
  archived recording\\
  The tactics that the Trump administration are using on the streets of
  Portland are abhorrent. People are being literally scooped off the
  street into unmarked vans, rental cars, apparently.
\end{itemize}

mike baker

The mayor has been villain number one for a lot of these protesters, as
someone who has failed to reform the police department in the ways they
want. And yet here you have him ---

\begin{itemize}
\tightlist
\item
  archived recording (ted wheeler)\\
  It's not helping the situation at all. They're not wanted here. We
  haven't asked them here. In fact, we want them to leave.
\end{itemize}

mike baker

--- ask the federal officers to leave his city. He doesn't want them
here. He doesn't want them on the streets.

\begin{itemize}
\item
  archived recording (ted wheeler)\\
  And what they're doing is they are sharply escalating the situation.
  Their presence here is actually leading to more violence and more
  vandalism.
\item
  archived recording\\
  {[}SOUNDS OF CRASHING AND CLAMORING{]}
\end{itemize}

mike baker

And you have the cycle here of tear gas, things being thrown back and
forth.

Standoffs where protesters are holding umbrellas and shields made out of
pool noodles and plywood. And the officers standing on the other side in
their full tactical gear and helmets and gas masks. And a scene of two
sides and not much a pathway to a resolution in the space between.

michael barbaro

So as of now, it feels like the very thing the federal government is in
Portland to try to tamp down is actually escalating in response.

mike baker

I mean, it's been a significant escalation.

I mean, now we're seeing thousands of people out there. You have people
out there coming out for the first time.

\begin{itemize}
\item
  mike baker\\
  So what was the --- what was your motivation for coming out?
\item
  protestor\\
  I have five grandkids and three daughters, and I don't want to be at
  the end of my life and say that I didn't do anything to make them have
  a better future.
\end{itemize}

mike baker

I caught up with this grandmother from Eugene, Oregon, who was there and
had come up to Portland for the first time and told her family that she
planned to stay on the outskirts to be safe. And then while she was
there, she was motivated to keep moving up. And I caught up with her
again, and she was right at the front of the federal courthouse.

She's a little uneasy watching this unfold.

\begin{itemize}
\tightlist
\item
  protestor\\
  I do feel a little bit unsafe. I don't know what's going to happen.
\end{itemize}

mike baker

Doesn't necessarily agree with the tactics she's watching, but she's
staying there. She feels the need that this is a moment to stand up, to
do something and she needs to be there.

michael barbaro

Zolan, Mike Baker said that the federal presence in Portland has
basically made things worse, not better. And it has really created a
kind of violent feedback loop between the protesters and these federal
officers. And I wonder what you think about that.

zolan kanno-youngs

Well, I mean, whether you listen to the demonstrators, the local
officials there, or the senior officials with the Department of Homeland
Security, it's clear everyone agrees that the federal presence thus far
has not succeeded in terms of bringing an end to the violence that we're
seeing, the unrest that we're seeing at this time. So by that measure,
the goal has not been accomplished. But there is also a question here.
For the Trump administration, is that solely their measure of success?
Is this solely about bringing an end to this unrest? You know, optics do
matter, and the optics of having agents in camouflage gear and tactical
teams in a city led by Democrats, that does send a message.

\begin{itemize}
\tightlist
\item
  archived recording\\
  The radical left-wing mob's agenda, take over our cities ---
\end{itemize}

zolan kanno-youngs

And just a couple days ago, the president's re-election campaign
actually issued a campaign ad.

\begin{itemize}
\tightlist
\item
  archived recording\\
  And Joe Biden stands with them.
\end{itemize}

zolan kanno-youngs

With images that look a lot like that area around the federal courthouse
in Portland, displaying images of unrest and individual acts of
violence.

\begin{itemize}
\tightlist
\item
  archived recording\\
  Violent crime exploding, innocent children fatally shot. Who will be
  there to answer the call when your children aren't safe?
\end{itemize}

zolan kanno-youngs

And at the very end of that ad, they actually lay it out in pretty
direct terms --- text that reads, ``You won't be safe in Joe Biden's
America.''

\begin{itemize}
\tightlist
\item
  archived recording (donald trump)\\
  I'm Donald J. Trump, and I approve this message.
\end{itemize}

zolan kanno-youngs

You're actually seeing the White House kind of double down here.

\begin{itemize}
\tightlist
\item
  archived recording (donald trump)\\
  I can tell you in Portland, they've done a fantastic job. They've been
  ---
\end{itemize}

zolan kanno-youngs

And say, well, look, they're doing a great job in Portland. In fact ---

\begin{itemize}
\tightlist
\item
  archived recording (donald trump)\\
  We're not going to let New York and Chicago and Philadelphia, Detroit,
  and Baltimore and all of these --- Oakland is a mess. We're not going
  to let this happen in our country. All run by liberal Democrats.
\end{itemize}

zolan kanno-youngs

Some of these other cities led by Democrats could use the same kind of
deployment.

\begin{itemize}
\tightlist
\item
  archived recording (donald trump)\\
  This is worse than Afghanistan by far. This is worse than anything
  anyone's ever seen, all run by the same liberal Democrats. And you
  know what, if Biden got in, that would be true for the country. The
  whole country would go to hell. And we're not going to let it go to
  hell.
\end{itemize}

{[}music{]}

michael barbaro

So Zolan, where does this leave us at this point?

zolan kanno-youngs

So it leaves us in this precarious position. We know that on the ground
in Portland, the presence of federal agents and those officers has
increased tension. But to the president, he'd like to see a similar
presence in other cities.

michael barbaro

Zolan, thank you very much.

zolan kanno-youngs

Thanks for having me here. {[}MUSIC PLAYING{]}

michael barbaro

On Wednesday, President Trump announced that he would immediately
dispatch federal law enforcement officers to Chicago.

\begin{itemize}
\tightlist
\item
  archived recording (donald trump)\\
  The F.B.I., A.T.F., D.E.A., U.S. Marshal Service and Homeland Security
  will together be sending hundreds of skilled law enforcement officers
  to Chicago to help drive down violent crime.
\end{itemize}

michael barbaro

In Chicago, Mayor Lori Lightfoot said she would not tolerate the kind of
federal deployment that has played out in Portland.

\begin{itemize}
\tightlist
\item
  archived recording (lori lightfoot)\\
  What we saw the president and the attorney general do in Portland is a
  travesty, and we are not having it in Chicago.
\end{itemize}

michael barbaro

We'll be right back.

Here's what else you need to know today.

\begin{itemize}
\tightlist
\item
  archived recording (mike dewine)\\
  It's essential that we wear masks statewide in Ohio to contain the
  spread of this virus. So therefore, tomorrow at 6 o'clock, tomorrow
  night, our mask order for people who are out in public will be
  extended throughout the state of Ohio.
\end{itemize}

michael barbaro

As the daily death toll from the coronavirus again surpasses 1,000
Americans a day, governors in three more states issued orders requiring
masks: Ohio, Indiana and Minnesota.

\begin{itemize}
\tightlist
\item
  archived recording\\
  The wearing of the mask, plus the social distancing makes a huge, huge
  difference.
\end{itemize}

michael barbaro

The orders came a day after President Trump, who has long resisted
wearing masks, and at times even disparaged them, made his most forceful
call yet for wearing them. And ---

\begin{itemize}
\tightlist
\item
  archived recording (joe biden)\\
  No sitting president has ever done this. Never, never, never. No
  Republican president has done this, no Democratic president. We've
  have racists, and they've existed, and they've tried to get elected
  president. He's the first one that has.
\end{itemize}

michael barbaro

During a campaign event on Wednesday, the presumptive Democratic
nominee, Joe Biden, called President Trump the first racist to be
elected president.

\begin{itemize}
\tightlist
\item
  archived recording (joe biden)\\
  The way he deals with people based on the color of their skin, their
  national origin, where they're from is absolutely sickening.
\end{itemize}

michael barbaro

In response, historians noted that previous presidents owned enslaved
people and were openly racist. And during a news conference, Trump
rejected Biden's characterization.

\begin{itemize}
\item
  archived recording\\
  Would you like to respond to Joe Biden who today described you --- you
  might have heard that --- as the first racist to be elected president?
  Those are his --- that was his words.
\item
  archived recording (donald trump)\\
  I've done things that nobody else --- and I've said this, and I say it
  openly, and not a lot of people dispute it. I've done more for Black
  Americans than anybody with the possible exception of Abraham Lincoln.
  Nobody has even been close.
\end{itemize}

{[}music{]}

michael barbaro

That's it for ``The Daily.'' I'm Michael Barbaro. See you tomorrow.

\includegraphics{https://static01.nyt.com/images/2020/07/17/us/17UNREST-PORTLAND2/merlin_174523377_9d4b64f2-1705-49f9-b5ea-8d2da569189c-articleLarge.jpg?quality=75\&auto=webp\&disable=upscale}

Advertisement

\protect\hyperlink{after-bottom}{Continue reading the main story}

\hypertarget{site-index}{%
\subsection{Site Index}\label{site-index}}

\hypertarget{site-information-navigation}{%
\subsection{Site Information
Navigation}\label{site-information-navigation}}

\begin{itemize}
\tightlist
\item
  \href{https://help.nytimes.com/hc/en-us/articles/115014792127-Copyright-notice}{©~2020~The
  New York Times Company}
\end{itemize}

\begin{itemize}
\tightlist
\item
  \href{https://www.nytco.com/}{NYTCo}
\item
  \href{https://help.nytimes.com/hc/en-us/articles/115015385887-Contact-Us}{Contact
  Us}
\item
  \href{https://www.nytco.com/careers/}{Work with us}
\item
  \href{https://nytmediakit.com/}{Advertise}
\item
  \href{http://www.tbrandstudio.com/}{T Brand Studio}
\item
  \href{https://www.nytimes.com/privacy/cookie-policy\#how-do-i-manage-trackers}{Your
  Ad Choices}
\item
  \href{https://www.nytimes.com/privacy}{Privacy}
\item
  \href{https://help.nytimes.com/hc/en-us/articles/115014893428-Terms-of-service}{Terms
  of Service}
\item
  \href{https://help.nytimes.com/hc/en-us/articles/115014893968-Terms-of-sale}{Terms
  of Sale}
\item
  \href{https://spiderbites.nytimes.com}{Site Map}
\item
  \href{https://help.nytimes.com/hc/en-us}{Help}
\item
  \href{https://www.nytimes.com/subscription?campaignId=37WXW}{Subscriptions}
\end{itemize}
