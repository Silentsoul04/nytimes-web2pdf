Sections

SEARCH

\protect\hyperlink{site-content}{Skip to
content}\protect\hyperlink{site-index}{Skip to site index}

\href{https://www.nytimes.com/section/your-money}{Your Money}

\href{https://myaccount.nytimes.com/auth/login?response_type=cookie\&client_id=vi}{}

\href{https://www.nytimes.com/section/todayspaper}{Today's Paper}

\href{/section/your-money}{Your Money}\textbar{}Are You Eligible for
Food Stamps Now? Maybe, but It's Complex

\url{https://nyti.ms/3hdoQIy}

\begin{itemize}
\item
\item
\item
\item
\item
\end{itemize}

\href{https://www.nytimes.com/news-event/coronavirus?action=click\&pgtype=Article\&state=default\&region=TOP_BANNER\&context=storylines_menu}{The
Coronavirus Outbreak}

\begin{itemize}
\tightlist
\item
  live\href{https://www.nytimes.com/2020/08/01/world/coronavirus-covid-19.html?action=click\&pgtype=Article\&state=default\&region=TOP_BANNER\&context=storylines_menu}{Latest
  Updates}
\item
  \href{https://www.nytimes.com/interactive/2020/us/coronavirus-us-cases.html?action=click\&pgtype=Article\&state=default\&region=TOP_BANNER\&context=storylines_menu}{Maps
  and Cases}
\item
  \href{https://www.nytimes.com/interactive/2020/science/coronavirus-vaccine-tracker.html?action=click\&pgtype=Article\&state=default\&region=TOP_BANNER\&context=storylines_menu}{Vaccine
  Tracker}
\item
  \href{https://www.nytimes.com/interactive/2020/07/29/us/schools-reopening-coronavirus.html?action=click\&pgtype=Article\&state=default\&region=TOP_BANNER\&context=storylines_menu}{What
  School May Look Like}
\item
  \href{https://www.nytimes.com/live/2020/07/31/business/stock-market-today-coronavirus?action=click\&pgtype=Article\&state=default\&region=TOP_BANNER\&context=storylines_menu}{Economy}
\end{itemize}

Advertisement

\protect\hyperlink{after-top}{Continue reading the main story}

Supported by

\protect\hyperlink{after-sponsor}{Continue reading the main story}

Your Money

\hypertarget{are-you-eligible-for-food-stamps-now-maybe-but-its-complex}{%
\section{Are You Eligible for Food Stamps Now? Maybe, but It's
Complex}\label{are-you-eligible-for-food-stamps-now-maybe-but-its-complex}}

States run SNAP, and many students and older people don't realize
they're eligible. Without more federal support, millions more may
qualify.

\includegraphics{https://static01.nyt.com/images/2020/07/17/business/17money/merlin_174674154_12a41568-9815-4fab-a63d-46d8e49ce948-articleLarge.jpg?quality=75\&auto=webp\&disable=upscale}

\href{https://www.nytimes.com/by/ron-lieber}{\includegraphics{https://static01.nyt.com/images/2018/10/22/multimedia/author-ron-lieber/author-ron-lieber-thumbLarge.png}}

By \href{https://www.nytimes.com/by/ron-lieber}{Ron Lieber}

\begin{itemize}
\item
  July 17, 2020
\item
  \begin{itemize}
  \item
  \item
  \item
  \item
  \item
  \end{itemize}
\end{itemize}

The safety net is starting to unravel.

At the end of the month, struggling Americans could lose the extra \$600
per week they've been receiving in unemployment insurance. Some
\href{https://www.nytimes.com/2020/07/11/your-money/coronavirus-eviction-prevention-renters-landlord.html}{eviction
protections} are already expiring.

And as people scramble to afford basic needs, hunger looms.

Tens of millions of Americans are in danger. According to Census Bureau
\href{https://www.census.gov/householdpulsedata}{Pulse Survey} data
released this week,
\href{https://www.census.gov/data-tools/demo/hhp/\#/?measures=FIR}{10.8
percent} of American adults are experiencing some level of food
insecurity. Louisiana, Nevada and Ohio had the highest rates: 17 to 18
percent. Food lines have been a feature of newspaper front pages and
home pages for months now.

And yet there is a program that may be able to help millions of
struggling Americans. One that was underused even before the coronavirus
crisis: food stamps, or as they are known in most places now, the
Supplemental Nutrition Assistance Program.

Policy experts and social services administrators are hoping that
everyone whose income has gone to zero or close to it will at least ask.
``If you've never accessed these benefits before, it may be because of
the way that SNAP in particular has been portrayed or vilified,'' said
Carlos M. Rodriguez, president and chief executive of the Community
FoodBank of New Jersey, which \href{https://cfbnj.org/findfood/}{helps
people} sign up for SNAP. ``People do not understand that this program
is here for them at this exact time.''

SNAP is overseen by the Department of Agriculture, which lays out the
rules. States handle applications and administration, and they have some
leeway with the federal regulations. (And with the terms:
\href{https://mydss.mo.gov/food-assistance/food-stamp-program}{Missouri}
still uses the older ``food stamp'' phrasing.)

As a result, it's possible to offer some general guidelines for
understanding how the program works, but your state has the final word.
The rules are numerous and complicated, but there are exceptions and
waivers that might apply to you --- so don't be deterred.

\hypertarget{latest-updates-global-coronavirus-outbreak}{%
\section{\texorpdfstring{\href{https://www.nytimes.com/2020/08/01/world/coronavirus-covid-19.html?action=click\&pgtype=Article\&state=default\&region=MAIN_CONTENT_1\&context=storylines_live_updates}{Latest
Updates: Global Coronavirus
Outbreak}}{Latest Updates: Global Coronavirus Outbreak}}\label{latest-updates-global-coronavirus-outbreak}}

Updated 2020-08-01T18:42:36.154Z

\begin{itemize}
\tightlist
\item
  \href{https://www.nytimes.com/2020/08/01/world/coronavirus-covid-19.html?action=click\&pgtype=Article\&state=default\&region=MAIN_CONTENT_1\&context=storylines_live_updates\#link-3ac56579}{Top
  officials work to break impasse over jobless benefit.}
\item
  \href{https://www.nytimes.com/2020/08/01/world/coronavirus-covid-19.html?action=click\&pgtype=Article\&state=default\&region=MAIN_CONTENT_1\&context=storylines_live_updates\#link-8796723}{The
  virus picks up dangerous speed in the Midwest, and in areas that had
  seen success.}
\item
  \href{https://www.nytimes.com/2020/08/01/world/coronavirus-covid-19.html?action=click\&pgtype=Article\&state=default\&region=MAIN_CONTENT_1\&context=storylines_live_updates\#link-25930521}{Thousands
  in Berlin protest Germany's coronavirus measures.}
\end{itemize}

\href{https://www.nytimes.com/2020/08/01/world/coronavirus-covid-19.html?action=click\&pgtype=Article\&state=default\&region=MAIN_CONTENT_1\&context=storylines_live_updates}{See
more updates}

More live coverage:
\href{https://www.nytimes.com/live/2020/07/31/business/stock-market-today-coronavirus?action=click\&pgtype=Article\&state=default\&region=MAIN_CONTENT_1\&context=storylines_live_updates}{Markets}

\hypertarget{am-i-eligible}{%
\subsection{Am I eligible?}\label{am-i-eligible}}

In the 2018 fiscal year,
\href{https://fns-prod.azureedge.net/sites/default/files/resource-files/Characteristics2018-Summary.pdf}{39.7
million people} qualified in an average month. To do so, they usually
had to pass both income tests and asset tests, though households with
elderly or disabled people may face less strict rules.

In most places, someone living alone can have a gross monthly income of
no more than \$1,354 and a net income of \$1,041. For a family of four,
the gross income limit is \$2,790 while the net income limit is \$2,146.
The Food and Nutrition Service of the Department of Agriculture lists
these limits and many other rules on its website via a
\href{https://www.fns.usda.gov/snap/recipient/eligibility}{SNAP
frequently asked questions page}.

Net income figures account for deductions that the program allows. Those
deductions include allowances for earnings (to encourage work),
dependent care, certain medical expenses and unusually large housing
costs. Applicants generally have to provide documentation.

Money you receive from unemployment payments may reduce or eliminate
your SNAP eligibility. Still, if unemployment is your only income and
you have few assets, it's worth applying for SNAP to see if you qualify.

The cap on assets is \$2,250, or \$3,500 if a household has someone 60
or older or someone with a disability. Homes and most retirement plan
balances don't count. Vehicles can count, though states have leeway to
set those rules.

\hypertarget{is-there-a-work-requirement}{%
\subsection{Is there a work
requirement?}\label{is-there-a-work-requirement}}

Yes, \href{https://www.fns.usda.gov/snap/work-requirements}{two of
them}.

First, if you're between the ages of 16 and 59, you're supposed to
enroll in relevant state training programs, accept suitable offers of
employment and not quit voluntarily or choose to work less than 30 hours
per week. But there are exceptions, including for people caring for
children under 6 years old or incapacitated adults, and those who have a
physical or mental limitation or are participating regularly in a drug
or alcohol treatment program.

There's another set of rules for people between the ages of 18 and 49
who are both able bodied and have no dependents, including working or
participating in a work program at least 80 hours per month. You can
read more about them on the Department of Agriculture's
\href{https://www.fns.usda.gov/snap/work-requirements-policies}{website}.

Waivers sometimes apply to work rules as well, which is why it's
important to apply for SNAP if you're not sure how your own work
situation applies, instead of just assuming that you're ineligible.

\hypertarget{how-does-the-application-work}{%
\subsection{How does the application
work?}\label{how-does-the-application-work}}

You apply through your state. The Department of Agriculture has a
\href{https://www.fns.usda.gov/snap/state-directory}{map-based
directory} on its website, and the Center on Budget and Policy
Priorities
\href{https://www.cbpp.org/research/food-assistance/snap-state-by-state-data-fact-sheets-and-resources}{has
collected} additional state-by-state information.

For people with no internet access, SNAP's phone number is
1-800-221-5689. There or via the 211 phone service in many areas, you
can likely find a state program's phone number.

Most states have online applications and calculators that screen for
eligibility. The application process usually includes an interview,
which can often happen over the phone. The process is supposed to take
no more than 30 days, and it could take less than a week if your income
or assets are particularly low.

To gain access to benefits, you'll use an electronic benefit transfer
card that works like a debit card in grocery stores. You'll need to be
ready to recertify eligibility from time to time, which can be a major
obstacle for struggling individuals who may also be trying to navigate
uncertain unemployment schedules or commute without a reliable vehicle.

\href{https://www.nytimes.com/news-event/coronavirus?action=click\&pgtype=Article\&state=default\&region=MAIN_CONTENT_3\&context=storylines_faq}{}

\hypertarget{the-coronavirus-outbreak-}{%
\subsubsection{The Coronavirus Outbreak
›}\label{the-coronavirus-outbreak-}}

\hypertarget{frequently-asked-questions}{%
\paragraph{Frequently Asked
Questions}\label{frequently-asked-questions}}

Updated July 27, 2020

\begin{itemize}
\item ~
  \hypertarget{should-i-refinance-my-mortgage}{%
  \paragraph{Should I refinance my
  mortgage?}\label{should-i-refinance-my-mortgage}}

  \begin{itemize}
  \tightlist
  \item
    \href{https://www.nytimes.com/article/coronavirus-money-unemployment.html?action=click\&pgtype=Article\&state=default\&region=MAIN_CONTENT_3\&context=storylines_faq}{It
    could be a good idea,} because mortgage rates have
    \href{https://www.nytimes.com/2020/07/16/business/mortgage-rates-below-3-percent.html?action=click\&pgtype=Article\&state=default\&region=MAIN_CONTENT_3\&context=storylines_faq}{never
    been lower.} Refinancing requests have pushed mortgage applications
    to some of the highest levels since 2008, so be prepared to get in
    line. But defaults are also up, so if you're thinking about buying a
    home, be aware that some lenders have tightened their standards.
  \end{itemize}
\item ~
  \hypertarget{what-is-school-going-to-look-like-in-september}{%
  \paragraph{What is school going to look like in
  September?}\label{what-is-school-going-to-look-like-in-september}}

  \begin{itemize}
  \tightlist
  \item
    It is unlikely that many schools will return to a normal schedule
    this fall, requiring the grind of
    \href{https://www.nytimes.com/2020/06/05/us/coronavirus-education-lost-learning.html?action=click\&pgtype=Article\&state=default\&region=MAIN_CONTENT_3\&context=storylines_faq}{online
    learning},
    \href{https://www.nytimes.com/2020/05/29/us/coronavirus-child-care-centers.html?action=click\&pgtype=Article\&state=default\&region=MAIN_CONTENT_3\&context=storylines_faq}{makeshift
    child care} and
    \href{https://www.nytimes.com/2020/06/03/business/economy/coronavirus-working-women.html?action=click\&pgtype=Article\&state=default\&region=MAIN_CONTENT_3\&context=storylines_faq}{stunted
    workdays} to continue. California's two largest public school
    districts --- Los Angeles and San Diego --- said on July 13, that
    \href{https://www.nytimes.com/2020/07/13/us/lausd-san-diego-school-reopening.html?action=click\&pgtype=Article\&state=default\&region=MAIN_CONTENT_3\&context=storylines_faq}{instruction
    will be remote-only in the fall}, citing concerns that surging
    coronavirus infections in their areas pose too dire a risk for
    students and teachers. Together, the two districts enroll some
    825,000 students. They are the largest in the country so far to
    abandon plans for even a partial physical return to classrooms when
    they reopen in August. For other districts, the solution won't be an
    all-or-nothing approach.
    \href{https://bioethics.jhu.edu/research-and-outreach/projects/eschool-initiative/school-policy-tracker/}{Many
    systems}, including the nation's largest, New York City, are
    devising
    \href{https://www.nytimes.com/2020/06/26/us/coronavirus-schools-reopen-fall.html?action=click\&pgtype=Article\&state=default\&region=MAIN_CONTENT_3\&context=storylines_faq}{hybrid
    plans} that involve spending some days in classrooms and other days
    online. There's no national policy on this yet, so check with your
    municipal school system regularly to see what is happening in your
    community.
  \end{itemize}
\item ~
  \hypertarget{is-the-coronavirus-airborne}{%
  \paragraph{Is the coronavirus
  airborne?}\label{is-the-coronavirus-airborne}}

  \begin{itemize}
  \tightlist
  \item
    The coronavirus
    \href{https://www.nytimes.com/2020/07/04/health/239-experts-with-one-big-claim-the-coronavirus-is-airborne.html?action=click\&pgtype=Article\&state=default\&region=MAIN_CONTENT_3\&context=storylines_faq}{can
    stay aloft for hours in tiny droplets in stagnant air}, infecting
    people as they inhale, mounting scientific evidence suggests. This
    risk is highest in crowded indoor spaces with poor ventilation, and
    may help explain super-spreading events reported in meatpacking
    plants, churches and restaurants.
    \href{https://www.nytimes.com/2020/07/06/health/coronavirus-airborne-aerosols.html?action=click\&pgtype=Article\&state=default\&region=MAIN_CONTENT_3\&context=storylines_faq}{It's
    unclear how often the virus is spread} via these tiny droplets, or
    aerosols, compared with larger droplets that are expelled when a
    sick person coughs or sneezes, or transmitted through contact with
    contaminated surfaces, said Linsey Marr, an aerosol expert at
    Virginia Tech. Aerosols are released even when a person without
    symptoms exhales, talks or sings, according to Dr. Marr and more
    than 200 other experts, who
    \href{https://academic.oup.com/cid/article/doi/10.1093/cid/ciaa939/5867798}{have
    outlined the evidence in an open letter to the World Health
    Organization}.
  \end{itemize}
\item ~
  \hypertarget{what-are-the-symptoms-of-coronavirus}{%
  \paragraph{What are the symptoms of
  coronavirus?}\label{what-are-the-symptoms-of-coronavirus}}

  \begin{itemize}
  \tightlist
  \item
    Common symptoms
    \href{https://www.nytimes.com/article/symptoms-coronavirus.html?action=click\&pgtype=Article\&state=default\&region=MAIN_CONTENT_3\&context=storylines_faq}{include
    fever, a dry cough, fatigue and difficulty breathing or shortness of
    breath.} Some of these symptoms overlap with those of the flu,
    making detection difficult, but runny noses and stuffy sinuses are
    less common.
    \href{https://www.nytimes.com/2020/04/27/health/coronavirus-symptoms-cdc.html?action=click\&pgtype=Article\&state=default\&region=MAIN_CONTENT_3\&context=storylines_faq}{The
    C.D.C. has also} added chills, muscle pain, sore throat, headache
    and a new loss of the sense of taste or smell as symptoms to look
    out for. Most people fall ill five to seven days after exposure, but
    symptoms may appear in as few as two days or as many as 14 days.
  \end{itemize}
\item ~
  \hypertarget{does-asymptomatic-transmission-of-covid-19-happen}{%
  \paragraph{Does asymptomatic transmission of Covid-19
  happen?}\label{does-asymptomatic-transmission-of-covid-19-happen}}

  \begin{itemize}
  \tightlist
  \item
    So far, the evidence seems to show it does. A widely cited
    \href{https://www.nature.com/articles/s41591-020-0869-5}{paper}
    published in April suggests that people are most infectious about
    two days before the onset of coronavirus symptoms and estimated that
    44 percent of new infections were a result of transmission from
    people who were not yet showing symptoms. Recently, a top expert at
    the World Health Organization stated that transmission of the
    coronavirus by people who did not have symptoms was ``very rare,''
    \href{https://www.nytimes.com/2020/06/09/world/coronavirus-updates.html?action=click\&pgtype=Article\&state=default\&region=MAIN_CONTENT_3\&context=storylines_faq\#link-1f302e21}{but
    she later walked back that statement.}
  \end{itemize}
\end{itemize}

``A lot of people roll off at that point,'' said Pamela Herd, a
Georgetown University professor and an expert on the
``\href{https://www.russellsage.org/publications/administrative-burden}{administrative
burdens}'' that keep otherwise eligible people from getting access to
many public programs.

\hypertarget{how-much-money-might-i-get}{%
\subsection{How much money might I
get?}\label{how-much-money-might-i-get}}

People who have less get more, but there are limits and they depend on
your family size.

The \href{https://www.fns.usda.gov/snap/recipient/eligibility}{maximum
monthly allotment} for a one-person household is \$194. For a family of
four, the cap is \$646. Cost-of-living adjustments may change those
amounts in Alaska, Hawaii, Guam and the Virgin Islands.

\hypertarget{are-college-and-graduate-students-eligible}{%
\subsection{Are college and graduate students
eligible?}\label{are-college-and-graduate-students-eligible}}

Sometimes, yes. A 2018 Government Accountability Office
\href{https://www.gao.gov/assets/700/696254.pdf}{report} found that 57
percent of low-income students who seemed potentially eligible for SNAP
(and had at least one other additional factor that suggested they were
food insecure) did not report receiving SNAP benefits. That was about
1.8 million people.

Moreover, investigators found that state SNAP employees and some federal
officials admitted confusion about student eligibility rules.

SNAP rules generally keep students whose parents are supporting them (or
those on a meal plan) from getting benefits. Others who have little
income or assets should consult the Agriculture Department's
\href{https://www.fns.usda.gov/snap/students}{bare-bones guidance} and
inquire further with their state if they think they might qualify. The
\href{https://hope4college.com/about-the-hope-center/}{Hope Center for
College, Community and Justice} at Temple University has
\href{https://hope4college.com/wp-content/uploads/2019/04/Beyond-the-Food-Pantry-Student-Access-to-SNAP.pdf}{a
guide} for colleges and universities that want to help students.

\hypertarget{what-about-social-security-recipients}{%
\subsection{What about Social Security
recipients?}\label{what-about-social-security-recipients}}

It depends. If you're receiving Supplemental Security Income benefits,
you should definitely apply for SNAP. In many instances, someone from a
Social Security office may be able to help.

Some people receiving Social Security retirement benefits may be
eligible for SNAP, too, but as of 2015,
\href{https://www.fns.usda.gov/pressrelease/2015/020215}{fewer than
half} of eligible older Americans were receiving benefits. The
Department of Agriculture has a
\href{https://www.fns.usda.gov/snap/eligibility/elderly-disabled-special-rules}{separate
section} of its website laying out the different eligibility rules for
elderly and disabled people.

\hypertarget{who-can-help-me-sign-up}{%
\subsection{Who can help me sign up?}\label{who-can-help-me-sign-up}}

Carrie R. Welton, director of policy at the Hope Center, a research and
advocacy group, said your first stop should still be the state agency
that determines eligibility. Caseworkers can be both helpful and
empathetic: Ms. Welton recalled her own time on public assistance, when
the person on the other side of the desk started to cry when she
realized that Ms. Welton would need to stop attending college full time
if she hoped to maintain her benefits.

Other organizations may be able to help. Part of Ms. Welton's work
involves translating federal and state policy to help students who may
be eligible for SNAP and other benefits. College financial aid offices
may be able to assist students, too.

Help may also be available at your local food bank (several hundred
colleges and universities have food banks as well). You can find a food
bank near you using the ZIP code tool on
\href{https://www.feedingamerica.org/find-your-local-foodbank}{Feeding
America's website}.

``We're pursuing the initiative to feed the people in the lines but help
shorten them as well,'' said Mr. Rodriguez, the Community FoodBank of
New Jersey president. ``SNAP puts dollars in people's hands to shop the
way you and I do.''

Advertisement

\protect\hyperlink{after-bottom}{Continue reading the main story}

\hypertarget{site-index}{%
\subsection{Site Index}\label{site-index}}

\hypertarget{site-information-navigation}{%
\subsection{Site Information
Navigation}\label{site-information-navigation}}

\begin{itemize}
\tightlist
\item
  \href{https://help.nytimes.com/hc/en-us/articles/115014792127-Copyright-notice}{©~2020~The
  New York Times Company}
\end{itemize}

\begin{itemize}
\tightlist
\item
  \href{https://www.nytco.com/}{NYTCo}
\item
  \href{https://help.nytimes.com/hc/en-us/articles/115015385887-Contact-Us}{Contact
  Us}
\item
  \href{https://www.nytco.com/careers/}{Work with us}
\item
  \href{https://nytmediakit.com/}{Advertise}
\item
  \href{http://www.tbrandstudio.com/}{T Brand Studio}
\item
  \href{https://www.nytimes.com/privacy/cookie-policy\#how-do-i-manage-trackers}{Your
  Ad Choices}
\item
  \href{https://www.nytimes.com/privacy}{Privacy}
\item
  \href{https://help.nytimes.com/hc/en-us/articles/115014893428-Terms-of-service}{Terms
  of Service}
\item
  \href{https://help.nytimes.com/hc/en-us/articles/115014893968-Terms-of-sale}{Terms
  of Sale}
\item
  \href{https://spiderbites.nytimes.com}{Site Map}
\item
  \href{https://help.nytimes.com/hc/en-us}{Help}
\item
  \href{https://www.nytimes.com/subscription?campaignId=37WXW}{Subscriptions}
\end{itemize}
