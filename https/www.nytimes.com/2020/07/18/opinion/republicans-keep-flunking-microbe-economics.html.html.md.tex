Sections

SEARCH

\protect\hyperlink{site-content}{Skip to
content}\protect\hyperlink{site-index}{Skip to site index}

\href{https://myaccount.nytimes.com/auth/login?response_type=cookie\&client_id=vi}{}

\href{https://www.nytimes.com/section/todayspaper}{Today's Paper}

\href{/section/opinion}{Opinion}\textbar{}Republicans Keep Flunking
Microbe Economics

\href{https://nyti.ms/2DWClOI}{https://nyti.ms/2DWClOI}

\begin{itemize}
\item
\item
\item
\item
\item
\item
\end{itemize}

Advertisement

\protect\hyperlink{after-top}{Continue reading the main story}

\href{/section/opinion}{Opinion}

Supported by

\protect\hyperlink{after-sponsor}{Continue reading the main story}

\hypertarget{republicans-keep-flunking-microbe-economics}{%
\section{Republicans Keep Flunking Microbe
Economics}\label{republicans-keep-flunking-microbe-economics}}

Getting other people sick isn't an ``individual choice.''

\href{https://www.nytimes.com/by/paul-krugman}{\includegraphics{https://static01.nyt.com/images/2018/04/02/opinion/paul-krugman/paul-krugman-thumbLarge.png}}

By \href{https://www.nytimes.com/by/paul-krugman}{Paul Krugman}

Opinion Columnist

\begin{itemize}
\item
  July 18, 2020
\item
  \begin{itemize}
  \item
  \item
  \item
  \item
  \item
  \item
  \end{itemize}
\end{itemize}

\includegraphics{https://static01.nyt.com/images/2020/07/18/opinion/18krugman2/merlin_171148875_2896c1c2-c6aa-4245-aaf5-61d093ed9eee-articleLarge.jpg?quality=75\&auto=webp\&disable=upscale}

Governor Ron DeSantis of Florida said something remarkably stupid the
other day. I know, I know: it's probably harder to find a day on which
DeSantis \emph{didn't} say something stupid than a day on which he did.
But this particular piece of thickheadedness, I'd argue, helps us
understand why America's response to the coronavirus has been so
disastrous compared with other wealthy nations.

Florida has, of course, become a Covid-19 epicenter, with soaring case
totals and a daily death toll now consistently exceeding that of the
whole European Union, which has 20 times its population. But DeSantis
won't contemplate any rollback of the state's obviously premature
reopening; he even refuses to close venues that are perfect coronavirus
incubators.

In particular, he insists on letting gyms --- closed spaces full of
people huffing and puffing --- stay open. Why? Because ``if you are in
\href{https://www.palmbeachpost.com/news/20200717/despite-11000-new-coronavirus-cases-desantis-defies-white-house-task-force}{good
shape} you have a very low likelihood of ending up in a significant
condition.''

Actually, this isn't true. Even healthy people can suffer terribly from
Covid-19. And if you've ever actually gone to a gym, you know that not
everyone there is young and fit.

But all this is beside the point. The reason we need to close gyms isn't
to protect the people working out, it's to protect the other people they
might infect. Even gym rats have families, friends, and co-workers; the
guy lifting weights might be OK, but the senior citizens who get sick
because he spent time hanging out in a petri dish might well die.

This should be obvious. Yet five months and almost 140,000 deaths into
this pandemic, many Republicans still can't or won't grasp the point
that choices have consequences beyond those to the individual who makes
them.

Take the insane resistance to wearing masks. Some of this is about
insecure masculinity --- people refusing to take the simplest, cheapest
of precautions because they think it will make them look silly. Some of
it is about culture wars: liberals wear masks, so I won't. But a lot of
it is about fetishization of individual choice.

Many things should be left up to the individual. I may not share your
taste in music or want to do the same things you do with consenting
adults, but such matters aren't legitimately my business.

Other things, however, aren't just about you. The question of whether or
not to dump raw sewage into a public lake isn't something that should be
left up to individual choice. And going to a gym or refusing to wear a
mask during a pandemic is exactly like dumping sewage into a lake: it's
behavior that may be convenient for the people who engage in it, but it
puts others at risk.

Again, this should be obvious. It's common sense; it is also, as it
happens, basic economics. Econ 101 has lots of good things to say about
free markets (probably too many good things, but that's a discussion for
another time), but no rational discussion of economics says that free
markets, left to themselves, can solve the problem of ``externalities''
--- costs that individuals or businesses impose on others who have no
say in the matter. Pollution is the classic example of an externality
that requires government intervention, but spreading a dangerous virus
poses exactly the same issues.

Yet many conservatives seem unable or unwilling to grasp this simple
point. And they seem equally unwilling to grasp a related point --- that
there are some things that must be supplied through public policy rather
than individual initiative. And the most important of these ``public
goods'' is probably scientific knowledge.

Some readers may be aware that Senator Rand Paul --- who proclaims
himself a libertarian --- has been doing a lot of sniping at Dr. Anthony
Fauci. Back in May he
\href{https://www.washingtonpost.com/politics/2020/05/12/fauci-testimony-senate-coronavirus/}{denounced}
Fauci for warning that premature reopening might lead to a surge in new
Covid-19 cases. More recently, apparently undaunted by the fact that
Fauci was right, he
\href{https://www.washingtonpost.com/politics/2020/06/30/we-just-need-some-more-optimism-rand-pauls-crusade-against-anthony-fauci-take-curious-turn/}{demanded}
that Fauci show ``humility'' and display some ``optimism.''

What struck me, however, was the way Paul justified his attacks on
epidemiologists' recommendations: by invoking the free-market doctrines
of Friedrich Hayek. ``Hayek had it right: Only decentralized power and
decision-making, based on millions of individualized situations, can
arrive at what risks and behaviors each individual should choose.''

Whatever you think of Hayek (as you might guess, I'm not a fan), this is
bizarre. Decentralized decision-making can do lots of things, but
establishing scientific truth isn't one of those things. And even
conservatives used to understand both that expertise matters and that
promoting scientific research is a legitimate and necessary role of
government.

But conservatives, and Republicans, have changed. The modern American
right is all about denying that people have any responsibility for each
other, and muzzling experts who try to tell people in power things they
don't want to hear.

And the fact that selfishness and willful ignorance are now guiding
principles for much of our political establishment is a large part of
the reason America is failing the Covid-19 test so spectacularly.

\emph{The Times is committed to publishing}
\href{https://www.nytimes.com/2019/01/31/opinion/letters/letters-to-editor-new-york-times-women.html}{\emph{a
diversity of letters}} \emph{to the editor. We'd like to hear what you
think about this or any of our articles. Here are some}
\href{https://help.nytimes.com/hc/en-us/articles/115014925288-How-to-submit-a-letter-to-the-editor}{\emph{tips}}\emph{.
And here's our email:}
\href{mailto:letters@nytimes.com}{\emph{letters@nytimes.com}}\emph{.}

\emph{Follow The New York Times Opinion section on}
\href{https://www.facebook.com/nytopinion}{\emph{Facebook}}\emph{,}
\href{http://twitter.com/NYTOpinion}{\emph{Twitter (@NYTopinion)}}
\emph{and}
\href{https://www.instagram.com/nytopinion/}{\emph{Instagram}}\emph{.}

Advertisement

\protect\hyperlink{after-bottom}{Continue reading the main story}

\hypertarget{site-index}{%
\subsection{Site Index}\label{site-index}}

\hypertarget{site-information-navigation}{%
\subsection{Site Information
Navigation}\label{site-information-navigation}}

\begin{itemize}
\tightlist
\item
  \href{https://help.nytimes.com/hc/en-us/articles/115014792127-Copyright-notice}{©~2020~The
  New York Times Company}
\end{itemize}

\begin{itemize}
\tightlist
\item
  \href{https://www.nytco.com/}{NYTCo}
\item
  \href{https://help.nytimes.com/hc/en-us/articles/115015385887-Contact-Us}{Contact
  Us}
\item
  \href{https://www.nytco.com/careers/}{Work with us}
\item
  \href{https://nytmediakit.com/}{Advertise}
\item
  \href{http://www.tbrandstudio.com/}{T Brand Studio}
\item
  \href{https://www.nytimes.com/privacy/cookie-policy\#how-do-i-manage-trackers}{Your
  Ad Choices}
\item
  \href{https://www.nytimes.com/privacy}{Privacy}
\item
  \href{https://help.nytimes.com/hc/en-us/articles/115014893428-Terms-of-service}{Terms
  of Service}
\item
  \href{https://help.nytimes.com/hc/en-us/articles/115014893968-Terms-of-sale}{Terms
  of Sale}
\item
  \href{https://spiderbites.nytimes.com}{Site Map}
\item
  \href{https://help.nytimes.com/hc/en-us}{Help}
\item
  \href{https://www.nytimes.com/subscription?campaignId=37WXW}{Subscriptions}
\end{itemize}
