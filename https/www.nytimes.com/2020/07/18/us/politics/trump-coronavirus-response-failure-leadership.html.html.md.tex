Sections

SEARCH

\protect\hyperlink{site-content}{Skip to
content}\protect\hyperlink{site-index}{Skip to site index}

\href{https://www.nytimes.com/section/politics}{Politics}

\href{https://myaccount.nytimes.com/auth/login?response_type=cookie\&client_id=vi}{}

\href{https://www.nytimes.com/section/todayspaper}{Today's Paper}

\href{/section/politics}{Politics}\textbar{}Inside Trump's Failure: The
Rush to Abandon Leadership Role on the Virus

\url{https://nyti.ms/32x1DgH}

\begin{itemize}
\item
\item
\item
\item
\item
\item
\end{itemize}

\href{https://www.nytimes.com/news-event/coronavirus?action=click\&pgtype=Article\&state=default\&region=TOP_BANNER\&context=storylines_menu}{The
Coronavirus Outbreak}

\begin{itemize}
\tightlist
\item
  live\href{https://www.nytimes.com/2020/08/02/world/coronavirus-updates.html?action=click\&pgtype=Article\&state=default\&region=TOP_BANNER\&context=storylines_menu}{Latest
  Updates}
\item
  \href{https://www.nytimes.com/interactive/2020/us/coronavirus-us-cases.html?action=click\&pgtype=Article\&state=default\&region=TOP_BANNER\&context=storylines_menu}{Maps
  and Cases}
\item
  \href{https://www.nytimes.com/interactive/2020/science/coronavirus-vaccine-tracker.html?action=click\&pgtype=Article\&state=default\&region=TOP_BANNER\&context=storylines_menu}{Vaccine
  Tracker}
\item
  \href{https://www.nytimes.com/interactive/2020/07/29/us/schools-reopening-coronavirus.html?action=click\&pgtype=Article\&state=default\&region=TOP_BANNER\&context=storylines_menu}{What
  School May Look Like}
\item
  \href{https://www.nytimes.com/live/2020/07/31/business/stock-market-today-coronavirus?action=click\&pgtype=Article\&state=default\&region=TOP_BANNER\&context=storylines_menu}{Economy}
\end{itemize}

Advertisement

\protect\hyperlink{after-top}{Continue reading the main story}

Supported by

\protect\hyperlink{after-sponsor}{Continue reading the main story}

\hypertarget{inside-trumps-failure-the-rush-to-abandon-leadership-role-on-the-virus}{%
\section{Inside Trump's Failure: The Rush to Abandon Leadership Role on
the
Virus}\label{inside-trumps-failure-the-rush-to-abandon-leadership-role-on-the-virus}}

The roots of the nation's current inability to control the pandemic can
be traced to mid-April, when the White House embraced overly rosy
projections to proclaim victory and move on.

\includegraphics{https://static01.nyt.com/images/2020/07/18/us/politics/18dc-virus-reconstruct1/merlin_174575826_5ad4eed3-8b2b-4c14-9421-dded152c4852-articleLarge.jpg?quality=75\&auto=webp\&disable=upscale}

\href{https://www.nytimes.com/by/michael-d-shear}{\includegraphics{https://static01.nyt.com/images/2018/06/13/multimedia/author-michael-d-shear/author-michael-d-shear-thumbLarge-v2.png}}\href{https://www.nytimes.com/by/noah-weiland}{\includegraphics{https://static01.nyt.com/images/2019/07/23/reader-center/author-noah-weiland/author-noah-weiland-thumbLarge.png}}\href{https://www.nytimes.com/by/eric-lipton}{\includegraphics{https://static01.nyt.com/images/2018/12/06/multimedia/author-eric-lipton/author-eric-lipton-thumbLarge.png}}\href{https://www.nytimes.com/by/maggie-haberman}{\includegraphics{https://static01.nyt.com/images/2018/07/12/multimedia/author-maggie-haberman/author-maggie-haberman-thumbLarge.png}}\href{https://www.nytimes.com/by/david-e-sanger}{\includegraphics{https://static01.nyt.com/images/2018/10/03/multimedia/author-david-e-sanger/author-david-e-sanger-thumbLarge.png}}

By \href{https://www.nytimes.com/by/michael-d-shear}{Michael D. Shear},
\href{https://www.nytimes.com/by/noah-weiland}{Noah Weiland},
\href{https://www.nytimes.com/by/eric-lipton}{Eric Lipton},
\href{https://www.nytimes.com/by/maggie-haberman}{Maggie Haberman} and
\href{https://www.nytimes.com/by/david-e-sanger}{David E. Sanger}

\begin{itemize}
\item
  Published July 18, 2020Updated July 28, 2020
\item
  \begin{itemize}
  \item
  \item
  \item
  \item
  \item
  \item
  \end{itemize}
\end{itemize}

\emph{Follow our live coverage of the}
\href{https://www.nytimes.com/2020/07/27/us/elections/biden-vs-trump.html}{\emph{2020
election between Joe Biden and President Trump}}\emph{.}

WASHINGTON --- Each morning at 8 as the coronavirus crisis was raging in
April, Mark Meadows, the White House chief of staff, convened a small
group of aides to steer the administration through what had become a
public health, economic and political disaster.

Seated around Mr. Meadows's conference table and on a couch in his
office down the hall from the Oval Office, they saw their immediate role
as practical problem solvers. Produce more ventilators. Find more
personal protective equipment. Provide more testing.

But their ultimate goal was to shift responsibility for leading the
fight against the pandemic from the White House to the states. They
referred to this as ``state authority handoff,'' and it was at the heart
of what would become at once a catastrophic policy blunder and an
attempt to escape blame for a crisis that had engulfed the country ---
perhaps one of the greatest failures of presidential leadership in
generations.

Over a critical period beginning in mid-April,
\href{https://www.nytimes.com/2020/07/28/us/politics/donald-fred-trump.html}{President
Trump} and his team convinced themselves that the outbreak was fading,
that they had given state governments all the resources they needed to
contain its remaining ``embers'' and that it was time to ease up on the
lockdown.

In doing so, he was ignoring warnings that the numbers would continue to
drop only if social distancing was kept in place, rushing instead to
restart the economy and tend to his battered re-election hopes.

Casting the decision in ideological terms, Mr. Meadows would tell
people: ``Only in Washington, D.C., do they think that they have the
answer for all of America.''

For scientific affirmation, they turned to
\href{https://www.state.gov/biographies/deborah-l-birx-md/}{Dr. Deborah
L. Birx}, the sole public health professional in the Meadows group. A
highly regarded infectious diseases expert, she was a constant source of
upbeat news for the president and his aides, walking the halls with
charts emphasizing that outbreaks were gradually easing. The country,
she insisted, was likely to resemble Italy, where virus cases
\href{https://www.nytimes.com/interactive/2020/world/europe/italy-coronavirus-cases.html}{declined
steadily} from frightening heights.

On April 11, she told the coronavirus task force in the Situation Room
that the nation was in good shape. Boston and Chicago are two weeks away
from the peak, she cautioned, but the numbers in Detroit and other
hard-hit cities are heading down.

A sharp pivot soon followed, with consequences that continue to plague
the country today as the virus surges anew.

Even as a chorus of state officials and health experts warned that the
pandemic was far from under control, Mr. Trump went, in a matter of
days, from proclaiming that he alone had the authority to decide when
the economy would reopen to pushing that responsibility onto the states.
The government issued detailed reopening guidelines, but almost
immediately, Mr. Trump began criticizing Democratic governors who did
not
\href{https://twitter.com/realDonaldTrump/status/1251169217531056130?s=20}{``liberate''}
their states.

Mr. Trump's bet that the crisis would fade away proved wrong. But an
examination of the shift in April and its aftermath shows that the
approach he embraced was not just a misjudgment. Instead, it was a
deliberate strategy that he would stick doggedly to as evidence mounted
that, in the absence of strong leadership from the White House, the
virus would continue to infect and kill large numbers of Americans.

\includegraphics{https://static01.nyt.com/images/2020/07/19/us/politics/19dc-virus-reconstruct-jump2/merlin_171185484_2fdeb7d2-78a7-4ac2-a868-5f5ce36e798f-articleLarge.jpg?quality=75\&auto=webp\&disable=upscale}

He and his top aides would openly disdain the scientific research into
the disease and the advice of experts on how to contain it, seek to
muzzle more authoritative voices like Dr. Anthony S. Fauci and continue
to distort reality even as it became clear that his hopes for a rapid
rebound in the economy and his electoral prospects were not
materializing.

Mr. Trump had
\href{https://www.nytimes.com/2020/04/11/us/politics/coronavirus-trump-response.html}{missed
or dismissed mounting signals of the impending crisis} in the early
months of the year. Now, interviews with more than two dozen officials
inside the administration and in the states, and a review of emails and
documents, reveal previously unreported details about how the White
House put the nation on its current course during a fateful period this
spring.

\begin{itemize}
\item
  Key elements of the administration's strategy were formulated out of
  sight in Mr. Meadows's daily meetings, by aides who for the most part
  had no experience with public health emergencies and were taking their
  cues from the president. Officials in the West Wing saw the
  better-known White House coronavirus task force as dysfunctional, came
  to view Dr. Fauci as a purveyor of dire warnings but no solutions and
  blamed officials from the Centers for Disease Control and Prevention
  for mishandling the early stages of the virus.
\item
  Dr. Birx was more central than publicly known to the judgment inside
  the West Wing that the virus was on a downward path. Colleagues
  described her as dedicated to public health and working herself to
  exhaustion to get the data right, but her model-based assessment
  nonetheless failed to account for a vital variable: how Mr. Trump's
  rush to urge a return to normal would help undercut the social
  distancing and other measures that were holding down the numbers.
\item
  The president quickly came to feel trapped by his own reopening
  guidelines. States needed declining cases to reopen, or at least a
  declining rate of positive tests. But more testing meant overall cases
  were destined to go up, undercutting the president's push to crank up
  the economy. The result was to intensify Mr. Trump's remarkable public
  campaign against testing, a vivid example of how he often waged war
  with science and his own administration's experts and stated policies.
\item
  Mr. Trump's bizarre public statements, his refusal to wear a mask and
  his pressure on states to get their economies going again left
  governors and other state officials scrambling to deal with a
  leadership vacuum. At one stage, Gov. Gavin Newsom of California was
  told that if he wanted the federal government to help obtain the swabs
  needed to test for the virus, he would have to ask Mr. Trump himself
  --- and thank him.
\item
  Not until early June did White House officials even begin to recognize
  that their assumptions about the course of the pandemic had proved
  wrong. Even now there are internal divisions over how far to go in
  having officials publicly acknowledge the reality of the situation.
\end{itemize}

Judd Deere, a White House spokesman, said the president had imposed
travel restrictions on China early in the pandemic, signed economic
relief measures that have provided Americans with critical assistance
and dealt with other issues including supplies of personal protective
equipment, testing capacity and vaccine development.

``President Trump and his bold actions from the very beginning of this
pandemic stand in stark contrast to the do-nothing Democrats and radical
left who just complain, criticize and condemn anything this president
does to preserve this nation,'' he said.

At
\href{https://www.whitehouse.gov/briefings-statements/remarks-president-trump-vice-president-pence-members-coronavirus-task-force-press-briefing-24/}{a
briefing on April 10}, Mr. Trump predicted that the number of deaths in
the United States from the pandemic would be ``substantially'' fewer
than 100,000. As of Saturday, the death toll stood at 139,186,
\href{https://www.nytimes.com/interactive/2020/07/17/us/coronavirus-deaths.html}{the
pace of new deaths was rising again} and the country, logging a
seven-day average of 65,790 new cases a day, had
\href{https://www.nytimes.com/interactive/2020/world/coronavirus-maps.html}{more
confirmed cases per capita} than any other major industrial nation.

\hypertarget{trumps-choice}{%
\subsection{Trump's Choice}\label{trumps-choice}}

Image

Even as Mr. Trump was acknowledging the need to make tough decisions, he
and his aides would soon be working to do just the
opposite.Credit...Doug Mills/The New York Times

The president had a decision to make.

It was the end of March and his initial, 15-day effort to slow the
spread of the virus by essentially shutting down the country was
expiring in days. Sitting in front of the Resolute Desk in the Oval
Office were Drs. Fauci and Birx, along with other top officials. Days
earlier, Mr. Trump had said he envisioned the country being ``opened up
and raring to go'' by Easter, but now he was on the verge of announcing
that he would keep the country shut down for another 30 days.

``Do you really think we need to do this?'' the president asked Dr.
Fauci. ``Yeah, we really do need to do it,'' Dr. Fauci replied,
explaining again the federal government's role in making sure the virus
did not explode across the country.

Mr. Trump's willingness to go along ---
\href{https://www.nytimes.com/2020/03/30/us/politics/trump-coronavirus.html}{driven
in part by grim television images of bodies piling up at Elmhurst
Hospital Center in New York City} --- was a concession that federal
responsibility was crucial to defeating a virus that did not respect
state boundaries. In a later Rose Garden appearance, he appeared
resigned to continuing the battle.

``Nothing would be worse than declaring victory before the victory is
won,'' Mr. Trump said.

But even as the president was acknowledging the need for tough
decisions, he and his aides would soon be looking to do the opposite ---
build a public case that the federal government had completed its job
and unshackle the president from ownership of the response.

\hypertarget{latest-updates-global-coronavirus-outbreak}{%
\section{\texorpdfstring{\href{https://www.nytimes.com/2020/08/01/world/coronavirus-covid-19.html?action=click\&pgtype=Article\&state=default\&region=MAIN_CONTENT_1\&context=storylines_live_updates}{Latest
Updates: Global Coronavirus
Outbreak}}{Latest Updates: Global Coronavirus Outbreak}}\label{latest-updates-global-coronavirus-outbreak}}

Updated 2020-08-02T17:52:35.962Z

\begin{itemize}
\tightlist
\item
  \href{https://www.nytimes.com/2020/08/01/world/coronavirus-covid-19.html?action=click\&pgtype=Article\&state=default\&region=MAIN_CONTENT_1\&context=storylines_live_updates\#link-34047410}{The
  U.S. reels as July cases more than double the total of any other
  month.}
\item
  \href{https://www.nytimes.com/2020/08/01/world/coronavirus-covid-19.html?action=click\&pgtype=Article\&state=default\&region=MAIN_CONTENT_1\&context=storylines_live_updates\#link-780ec966}{Top
  U.S. officials work to break an impasse over the federal jobless
  benefit.}
\item
  \href{https://www.nytimes.com/2020/08/01/world/coronavirus-covid-19.html?action=click\&pgtype=Article\&state=default\&region=MAIN_CONTENT_1\&context=storylines_live_updates\#link-2bc8948}{Its
  outbreak untamed, Melbourne goes into even greater lockdown.}
\end{itemize}

\href{https://www.nytimes.com/2020/08/01/world/coronavirus-covid-19.html?action=click\&pgtype=Article\&state=default\&region=MAIN_CONTENT_1\&context=storylines_live_updates}{See
more updates}

More live coverage:
\href{https://www.nytimes.com/live/2020/07/31/business/stock-market-today-coronavirus?action=click\&pgtype=Article\&state=default\&region=MAIN_CONTENT_1\&context=storylines_live_updates}{Markets}

The hub of the activity was the working group assembled by Mr. Meadows,
who had just taken over as chief of staff.

Joe Grogan, the domestic policy adviser, had come around to Mr. Trump's
view that the reaction to the virus was overblown, a position shared at
that point by Marc Short, Vice President Mike Pence's chief of staff and
a frequent participant in the meetings. Russell T. Vought, the
president's acting budget director, was there to address the pandemic's
mounting costs.

Chris Liddell, a deputy chief of staff, and Jared Kushner, the
president's senior adviser and son-in-law, acted as the group's
procurement and supply-chain experts.

Hope Hicks, the protector of Mr. Trump's brand, was a regular
participant. Kevin A. Hassett, a top economic adviser, came at times to
help assess the numbers and also participated in a 9 a.m. meeting three
times a week with Mr. Meadows and Treasury Secretary Steven Mnuchin on
the economic aspects of the pandemic.

Then there was Dr. Birx, the response coordinator of the coronavirus
task force. Unlike Dr. Fauci, who only stopped by the White House to
attend meetings, she was given an office near the Situation Room and
freely roamed the West Wing, fully embracing her role as a member of the
president's team.

Image

Key elements of the administration's strategy were formulated out of
sight in daily meetings held by the chief of staff, Mark
Meadows.Credit...Doug Mills/The New York Times

By mid-April, Mr. Trump had grown publicly impatient with the
stay-at-home recommendations he had reluctantly endorsed.
\href{https://www.nytimes.com/2020/04/09/business/economy/unemployment-claim-numbers-coronavirus.html}{Weekly
unemployment claims} made clear the economy was cratering and
\href{https://www.nytimes.com/2020/04/10/us/politics/trump-polls-coronavirus.html}{polling}
was showing his campaign bleeding support. Republican governors were
agitating to lift the lockdown and
\href{https://www.nytimes.com/2020/04/21/us/politics/coronavirus-protests-trump.html}{the
conservative political machinery was mobilizing} to oppose what it saw
as constraints on individual freedom.

At the meetings in Mr. Meadows's office, the issue was clear: How much
longer do we keep this up?

To answer that, they focused on two more questions: Had the virus
peaked? And had the government given the states the tools they needed to
manage the remaining problems?

On the first question, Dr. Birx and Mr. Hassett were optimistic:
Mitigation was working, they insisted, even as many outside experts were
warning that the nation would remain at great risk if it let up on
social distancing and moved prematurely to reopen.

Mr. Meadows thought of himself as a data-driven decision maker, and in
addition to models and infection numbers from the states and the C.D.C.,
they looked at traffic on the New Jersey Turnpike (the volume of cars
coming in and out of New York City was down by 95.2 percent); payroll
and credit card data, and the number of people who were reporting to
have self-quarantined.

If the point was to sustain a monthlong lockdown, the numbers told them,
the administration succeeded. If it was to squelch the virus to
containable levels, later events would show the officials were oblivious
to how widely it was already spreading.

The members of his group believed they had succeeded on the second
question, too, although shortages of protective gear continued in some
places (and would
\href{https://www.nytimes.com/2020/07/08/health/coronavirus-masks-ppe-doc.html}{flare
again} months later).

A one-time
\href{https://www.whitehouse.gov/briefings-statements/remarks-president-trump-vice-president-pence-members-coronavirus-task-force-press-briefing-13/}{anticipated
shortage} of more than 100,000 ventilators had been overcome; now there
was enough of a surplus that the United States could lend them to other
countries. A ban on elective surgeries meant there was plenty of bed
space --- and no more need for the Navy's hospital ships.

The group thought governors should no longer have trouble getting what
they needed for hospitals, doctors and first responders. And they grew
increasingly frustrated by what they saw as politically motivated
complaining about a lack of federal help and the inability of some
states to make effective use of the supplies they were receiving.

Enraged by criticism from New York's Democratic politicians about not
being able to find a shipment of ventilators from the federal
government, Mr. Grogan, the domestic policy chief, angrily told Mr.
Kushner that they should put more ventilators on eighteen-wheelers,
drive them into New York City and invite news helicopters to record it
all --- just to embarrass Gov. Andrew Cuomo and Mayor Bill de Blasio.

Image

Medical staff at Lincoln Hospital in the Bronx in May. At the peak of
the pandemic in New York, the city was facing a potential shortage of
personal protective equipment and ventilators.Credit...Erin Schaff/The
New York Times

On April 14, the country passed what the group saw as a milestone,
administering \href{https://covidtracking.com/data/us-daily}{its three
millionth test}. Inside the West Wing, Mr. Kushner was insistent on that
point: Given their assumption that infections would not surge again
until the fall, there was enough testing ability out there.

Those outside experts who disagreed were largely brushed off. In
mid-April, Dr. Ashish K. Jha, director of the Harvard Global Health
Institute, urged a top administration official to embrace his call for
conducting 500,000 coronavirus tests a day --- far more than was
happening at the time.

The official, Adm. Brett P. Giroir, the administration's testing czar,
who had been delivering upbeat descriptions of the nation's growing
testing capacity, eventually conceded to Dr. Jha that his plan seemed to
be needed. But he made clear the federal government was not prepared to
get there quickly.

``At some point down the road,'' is what Dr. Jha said Admiral Giroir
told him.

``My take is that Jared Kushner believes that this is not something that
the White House should get too involved in,'' Dr. Jha recalled. ``And
then the president believes that it is better left up to the states.''

Their critics notwithstanding, White House officials came to feel that
they had in fact accomplished their job: giving governors the tools they
needed to deal with remaining outbreaks as infections ebbed.

The wind down of the federal government's response would play out over
the next several weeks. The daily briefings with Mr. Trump ended on
April 24. The Meadows team started barring Dr. Fauci from making most
television appearances, lest he go off message and suggest continued
high risk from the virus.

By the beginning of May,
\href{https://www.nytimes.com/2020/05/05/us/politics/coronavirus-task-force-trump.html}{word
leaked} that the daily meetings of the task force itself would be ended,
though Mr. Trump, who had not been told, backpedaled after the coverage
caused an uproar.

On testing, Mr. Trump shifted from stressing that the nation was already
doing more than any other country
\href{https://www.politico.com/news/2020/05/14/trump-coronavirus-testing-high-case-numbers-259524}{to
deriding its importance}. By June the president was regularly making
nonsensical
\href{https://www.youtube.com/watch?v=aN1eptTaWVM}{statements like},
``If we stop testing right now, we'd have very few cases, if any.''

But during the middle weeks of April the president's decision to largely
walk away from an active leadership role --- and give many states
permission to believe the worst of the crisis was behind them --- came
abruptly into public view.

On April 10, Mr. Trump declared that, in his role as something akin to a
``wartime president,'' it would be his decision about whether to reopen
the country. ``That's my
metrics,''\href{https://www.whitehouse.gov/briefings-statements/remarks-president-trump-vice-president-pence-members-coronavirus-task-force-press-briefing-24/}{he
told reporters}, pointing to his own head. ``I would say without
question it's the biggest decision I've ever had to make.''

Three days later, he reiterated his responsibility. ``When somebody is
the president of the United States, the authority is total and that's
the way it's got to be,''
\href{https://www.whitehouse.gov/briefings-statements/remarks-president-trump-vice-president-pence-members-coronavirus-task-force-press-briefing-25/}{he
said.}

The next day, Dr. Birx and Dr. Fauci presented Mr. Trump with a plan for
issuing guidelines to start reopening the country at the end of the
month. Developed largely by Dr. Birx and held closely by her until being
presented to the president --- most task force members did not see them
beforehand --- the guidelines laid out broad, voluntary standards for
states considering how fast to come out of the lockdown.

In political terms, the document's message was that responsibility for
dealing with the pandemic was shifting from Mr. Trump to the states.

On April 16, when Mr. Trump publicly announced the guidelines, he made
the message to the governors explicit.

``You're going to call your own shots,''
\href{https://www.nytimes.com/2020/04/16/us/politics/coronavirus-trump-guidelines.html}{he
said}.

\hypertarget{birxs-influence}{%
\subsection{Birx's Influence}\label{birxs-influence}}

Image

Dr. Birx showing a projected model of national deaths during a
coronavirus task force briefing at the White House in
March.Credit...Erin Schaff/The New York Times

Inside the White House, Dr. Birx was the chief evangelist for the idea
that the threat from the virus was fading.

Unlike Dr. Fauci, Dr. Birx is a strong believer in models that forecast
the course of an outbreak. Dr. Fauci has cautioned that ``models are
only models'' and that real-world outcomes depend on how people respond
to calls for changes in behavior --- to stay home, for example, or wear
masks in public --- sacrifices that required a sense of shared national
responsibility.

In his decades of responding to outbreaks, Dr. Fauci, a voracious reader
of political histories, learned to rely on reports from the ground. Late
at night in his home office this spring, Dr. Fauci, who declined to
comment for this article, dialed health officials in New Orleans, New
York and Chicago, where he heard desperation unrecognizable in the more
sanguine White House meetings.

Dr. Fauci had his own critics, who said he relied on anecdotes and
experience rather than data, and who felt he was not sufficiently
attuned to the devastating economic and social consequences of a
national lockdown.

As the pandemic worsened, Dr. Fauci's darker view of the circumstances
was countered by the reassurances ostensibly offered by Dr. Birx's data.

A renowned AIDS researcher who holds the title of ``ambassador'' as the
State Department's special representative for global health diplomacy,
she had assembled a team of analysts who worked late nights in the White
House complex, feeding her a constant stream of updated data, packaged
in PowerPoint slides emailed to senior officials each day.

There were warnings that the models she studied might not be accurate,
especially in predicting the course of the virus against a backdrop of
evolving political, economic and social factors. Among the models Dr.
Birx relied on most was one produced by researchers at the University of
Washington. But when Mr. Hassett reviewed its performance by looking
back on its predictions from three weeks earlier, it turned out to be
hit or miss.

The authors of the
\href{http://www.healthdata.org/covid/faqs}{University of Washington
model} spoke to Dr. Birx or members of her team almost daily, they said,
and often cautioned that their work was only supposed to offer a
snapshot based on key assumptions, like people continuing to abide by
social distancing until June 1.

``We made clear that to get the epidemic under control and bring it down
to effectively zero transmission required the social distancing mandates
to be in place,'' said Christopher J. L. Murray, the director of the
modeling program. ``April 22 --- somewhere around that period. That's
when the tone shifted. They started to ask questions about what will be
the trajectory and where with the lifting of mandates?''

Some state officials were also alarmed by the administration's use of
the University of Washington model.

\href{https://www.nytimes.com/news-event/coronavirus?action=click\&pgtype=Article\&state=default\&region=MAIN_CONTENT_3\&context=storylines_faq}{}

\hypertarget{the-coronavirus-outbreak-}{%
\subsubsection{The Coronavirus Outbreak
›}\label{the-coronavirus-outbreak-}}

\hypertarget{frequently-asked-questions}{%
\paragraph{Frequently Asked
Questions}\label{frequently-asked-questions}}

Updated July 27, 2020

\begin{itemize}
\item ~
  \hypertarget{should-i-refinance-my-mortgage}{%
  \paragraph{Should I refinance my
  mortgage?}\label{should-i-refinance-my-mortgage}}

  \begin{itemize}
  \tightlist
  \item
    \href{https://www.nytimes.com/article/coronavirus-money-unemployment.html?action=click\&pgtype=Article\&state=default\&region=MAIN_CONTENT_3\&context=storylines_faq}{It
    could be a good idea,} because mortgage rates have
    \href{https://www.nytimes.com/2020/07/16/business/mortgage-rates-below-3-percent.html?action=click\&pgtype=Article\&state=default\&region=MAIN_CONTENT_3\&context=storylines_faq}{never
    been lower.} Refinancing requests have pushed mortgage applications
    to some of the highest levels since 2008, so be prepared to get in
    line. But defaults are also up, so if you're thinking about buying a
    home, be aware that some lenders have tightened their standards.
  \end{itemize}
\item ~
  \hypertarget{what-is-school-going-to-look-like-in-september}{%
  \paragraph{What is school going to look like in
  September?}\label{what-is-school-going-to-look-like-in-september}}

  \begin{itemize}
  \tightlist
  \item
    It is unlikely that many schools will return to a normal schedule
    this fall, requiring the grind of
    \href{https://www.nytimes.com/2020/06/05/us/coronavirus-education-lost-learning.html?action=click\&pgtype=Article\&state=default\&region=MAIN_CONTENT_3\&context=storylines_faq}{online
    learning},
    \href{https://www.nytimes.com/2020/05/29/us/coronavirus-child-care-centers.html?action=click\&pgtype=Article\&state=default\&region=MAIN_CONTENT_3\&context=storylines_faq}{makeshift
    child care} and
    \href{https://www.nytimes.com/2020/06/03/business/economy/coronavirus-working-women.html?action=click\&pgtype=Article\&state=default\&region=MAIN_CONTENT_3\&context=storylines_faq}{stunted
    workdays} to continue. California's two largest public school
    districts --- Los Angeles and San Diego --- said on July 13, that
    \href{https://www.nytimes.com/2020/07/13/us/lausd-san-diego-school-reopening.html?action=click\&pgtype=Article\&state=default\&region=MAIN_CONTENT_3\&context=storylines_faq}{instruction
    will be remote-only in the fall}, citing concerns that surging
    coronavirus infections in their areas pose too dire a risk for
    students and teachers. Together, the two districts enroll some
    825,000 students. They are the largest in the country so far to
    abandon plans for even a partial physical return to classrooms when
    they reopen in August. For other districts, the solution won't be an
    all-or-nothing approach.
    \href{https://bioethics.jhu.edu/research-and-outreach/projects/eschool-initiative/school-policy-tracker/}{Many
    systems}, including the nation's largest, New York City, are
    devising
    \href{https://www.nytimes.com/2020/06/26/us/coronavirus-schools-reopen-fall.html?action=click\&pgtype=Article\&state=default\&region=MAIN_CONTENT_3\&context=storylines_faq}{hybrid
    plans} that involve spending some days in classrooms and other days
    online. There's no national policy on this yet, so check with your
    municipal school system regularly to see what is happening in your
    community.
  \end{itemize}
\item ~
  \hypertarget{is-the-coronavirus-airborne}{%
  \paragraph{Is the coronavirus
  airborne?}\label{is-the-coronavirus-airborne}}

  \begin{itemize}
  \tightlist
  \item
    The coronavirus
    \href{https://www.nytimes.com/2020/07/04/health/239-experts-with-one-big-claim-the-coronavirus-is-airborne.html?action=click\&pgtype=Article\&state=default\&region=MAIN_CONTENT_3\&context=storylines_faq}{can
    stay aloft for hours in tiny droplets in stagnant air}, infecting
    people as they inhale, mounting scientific evidence suggests. This
    risk is highest in crowded indoor spaces with poor ventilation, and
    may help explain super-spreading events reported in meatpacking
    plants, churches and restaurants.
    \href{https://www.nytimes.com/2020/07/06/health/coronavirus-airborne-aerosols.html?action=click\&pgtype=Article\&state=default\&region=MAIN_CONTENT_3\&context=storylines_faq}{It's
    unclear how often the virus is spread} via these tiny droplets, or
    aerosols, compared with larger droplets that are expelled when a
    sick person coughs or sneezes, or transmitted through contact with
    contaminated surfaces, said Linsey Marr, an aerosol expert at
    Virginia Tech. Aerosols are released even when a person without
    symptoms exhales, talks or sings, according to Dr. Marr and more
    than 200 other experts, who
    \href{https://academic.oup.com/cid/article/doi/10.1093/cid/ciaa939/5867798}{have
    outlined the evidence in an open letter to the World Health
    Organization}.
  \end{itemize}
\item ~
  \hypertarget{what-are-the-symptoms-of-coronavirus}{%
  \paragraph{What are the symptoms of
  coronavirus?}\label{what-are-the-symptoms-of-coronavirus}}

  \begin{itemize}
  \tightlist
  \item
    Common symptoms
    \href{https://www.nytimes.com/article/symptoms-coronavirus.html?action=click\&pgtype=Article\&state=default\&region=MAIN_CONTENT_3\&context=storylines_faq}{include
    fever, a dry cough, fatigue and difficulty breathing or shortness of
    breath.} Some of these symptoms overlap with those of the flu,
    making detection difficult, but runny noses and stuffy sinuses are
    less common.
    \href{https://www.nytimes.com/2020/04/27/health/coronavirus-symptoms-cdc.html?action=click\&pgtype=Article\&state=default\&region=MAIN_CONTENT_3\&context=storylines_faq}{The
    C.D.C. has also} added chills, muscle pain, sore throat, headache
    and a new loss of the sense of taste or smell as symptoms to look
    out for. Most people fall ill five to seven days after exposure, but
    symptoms may appear in as few as two days or as many as 14 days.
  \end{itemize}
\item ~
  \hypertarget{does-asymptomatic-transmission-of-covid-19-happen}{%
  \paragraph{Does asymptomatic transmission of Covid-19
  happen?}\label{does-asymptomatic-transmission-of-covid-19-happen}}

  \begin{itemize}
  \tightlist
  \item
    So far, the evidence seems to show it does. A widely cited
    \href{https://www.nature.com/articles/s41591-020-0869-5}{paper}
    published in April suggests that people are most infectious about
    two days before the onset of coronavirus symptoms and estimated that
    44 percent of new infections were a result of transmission from
    people who were not yet showing symptoms. Recently, a top expert at
    the World Health Organization stated that transmission of the
    coronavirus by people who did not have symptoms was ``very rare,''
    \href{https://www.nytimes.com/2020/06/09/world/coronavirus-updates.html?action=click\&pgtype=Article\&state=default\&region=MAIN_CONTENT_3\&context=storylines_faq\#link-1f302e21}{but
    she later walked back that statement.}
  \end{itemize}
\end{itemize}

Colorado health officials
\href{https://documentingcovid19.io/uploads/DHS\%20HHS\%20ventilator\%20usage\%20by\%20state\%20April\%2012.pdf}{wrote
to the administration on April 9}, pleading that the White House not use
the model to allocate supplies to the state, saying its predictions were
rosier than the grim reality they were encountering. (When those
concerns were relayed to her,
\href{https://www.documentcloud.org/documents/6994649-2020-04-13-Colorado-Re-Birx-IHME-Colorado-FOIA.html}{Dr.
Birx replied} that decisions on allocating equipment were based on
factors beyond the one model.)

Dr. Birx declined to be interviewed. A task force official said that she
had only used the University of Washington model in a limited way and
that the White House used ``real data, not modeled data, to understand
the pandemic in the United States.''

The official said the White House ``immediately reacted to the early
signs of community spread'' by working with governors in the affected
states.

But despite the outside warnings and evidence by early May that
\href{https://www.nytimes.com/interactive/2020/us/coronavirus-us-cases.html}{new
infections, while down, remained higher than anticipated}, the White
House never fundamentally re-examined the course it had set in
mid-April.

Dr. Fauci, a friend of Dr. Birx's for 30 years, would describe her as
more political than him, a ``different species.'' More pessimistic by
nature, Dr. Fauci privately warned that the virus was going to be
difficult to control, often commenting that he was the ``skunk at the
garden party.''

Image

White House officials became increasingly disdainful of Dr. Anthony S.
Fauci, the nation's top infectious disease expert.Credit...Doug
Mills/The New York Times

By contrast, Dr. Birx regularly delivered what the new team was hoping
for.

``All metros are stabilizing,'' she would tell them, describing the
virus as having hit its ``peak'' around mid-April. The New York area
accounted for half of the total cases in the country, she said. The
slope was heading in the right direction. ``We're behind the worst of
it.'' She endorsed the idea that the death counts and hospitalization
numbers could be inflated.

For Dr. Birx, Italy's experience was a particularly telling --- and
positive --- comparison. She routinely told colleagues that the United
States was on the same trajectory as Italy, which had huge spikes before
infections and deaths flattened to close to zero.

``She said we were basically going to track Italy,'' one senior adviser
later recalled.

Dr. Birx would roam the halls of the White House, talking to Mr.
Kushner, Ms. Hicks and others, sometimes passing out diagrams to bolster
her case. ``We've hit our peak,'' she would say, and that message would
find its way back to Mr. Trump.

Dr. Birx began using versions of the phrase ``putting out the embers,''
wording that was
\href{https://www.whitehouse.gov/briefings-statements/press-briefing-press-secretary-kayleigh-mcenany-062220/}{later
picked up by the press secretary}, Kayleigh McEnany, and
\href{https://twitter.com/realDonaldTrump/status/1276363261957603328?s=20}{by
Mr. Trump himself}.

By the middle of May, the task force believed that another resurgence
was not likely until the fall, senior administration officials said.

The New York region appeared well on its way to driving new infections
down to levels it could handle --- it was the one area of the country
that did resemble the Italian model. But the models and analysis
embraced by the West Wing failed to account for the weakening adherence
to the lockdowns across the country that began even before Mr. Trump
started urging governors to ``liberate'' their residents from the
methodical guidelines his own government had established.

Later, it was clear that states that
\href{https://www.nytimes.com/interactive/2020/05/07/us/coronavirus-states-reopen-criteria.html}{rushed
to reopen before meeting the criteria in the guidelines} --- like
Arizona, Texas and Alabama --- would have among the worst surges in new
cases.

Dr. Birx's belief that the United States would mirror Italy turned out
to be disastrously wrong. The Italians had been almost entirely
compliant with stay-at-home orders and social distancing, squelching new
infections to negligible levels before the country slowly reopened.
Americans, by contrast, began backing away by late April from what
social distancing efforts they had been making, egged on by Mr. Trump.

The difference was critical. As communities across the United States
raced to reopen, the daily number of new cases barely dropped below
20,000 in early May. The virus was still circulating across the country.

Italy's recovery curve, it turned out, looked nothing like the American
one.

\hypertarget{the-consequences}{%
\subsection{The Consequences}\label{the-consequences}}

Image

A drive-through testing site in Los Angeles last week. The governor of
California was told that if he wanted the federal government to help
obtain the swabs needed to test for the virus, he would have to ask Mr.
Trump himself --- and thank him.Credit...Jenna Schoenefeld for The New
York Times

The real-world consequences of Mr. Trump's abdication of responsibility
rippled across the country.

During a briefing on April 20, Mr. Trump mocked Gov. Larry Hogan of
Maryland, a fellow Republican, for the state's inability to find enough
testing. Dr. Birx displayed maps with dozens of dots indicating labs
that could help.

``He really didn't know about the federal laboratories,'' Mr. Trump
\href{https://www.whitehouse.gov/briefings-statements/remarks-president-trump-vice-president-pence-members-coronavirus-task-force-press-briefing-29/}{told
reporters} with mock astonishment. ``He didn't know about it.''

But when Frances B. Phillips, the state's deputy health secretary,
reached out to one of those dots --- a National Institutes of Health
facility in Maryland --- she was told that they were suffering from the
same shortages as state labs and were not in a position to help.

``It was clear that we were on our own and we need to develop our own
strategy, which is very unlike the kind of federal response in the past
public health emergencies,'' Ms. Phillips recalled.

In California, Mr. Newsom had already experienced firsthand the
complexities of getting help from Washington.

After offering to help acquire 350,000 testing swabs during an early
morning conversation with one of Mr. Newsom's advisers, Mr. Kushner made
it clear that the federal help would hinge on the governor doing him a
favor.

``The governor of California, Gavin Newsom, had to call Donald Trump,
and ask him for the swabs'' recalled the adviser, Bob Kocher, an
Obama-era White House health care official.

Mr. Newsom made the call as requested and then praised Mr. Trump that
same day during a
\href{https://www.facebook.com/CAgovernor/videos/686605895491026/}{news
conference} where he announced the commitment, giving Mr. Trump credit
for the ``substantial increase in supply'' headed to California.

Mayor Francis X. Suarez of Miami, a Republican, said that the White
House approach had only one focus: reopening businesses, instead of
anticipating how cities and states should respond if cases surged again.

``It was all predicated on reduction, open, reduction, open more,
reduction, open,'' he said. ``There was never what happens if there is
an increase after you reopen?''

Other nations had moved aggressively to employ an array of techniques
that Mr. Trump never mobilized on a federal level, including national
testing strategies and contact tracing to track down and isolate people
who had interacted with newly diagnosed patients.

``These things were done in Germany, in Italy, in Greece, Vietnam, in
Singapore, in New Zealand and in China,'' said Andy Slavitt, a former
federal health care official who had been advising the White House.

``They were not secret,'' he said. ``Not mysterious. And these were not
all wealthy countries. They just took accountability for getting it
done. But we did not do that here. There was zero chance here that we
would ever have been in a situation where we would be dealing with
`embers.' ''

\hypertarget{a-new-surge}{%
\subsection{A New Surge}\label{a-new-surge}}

Image

A medical team treating a patient with Covid-19 at Houston Methodist
Hospital this month. Texas was one of the first states to reopen
businesses and is now seeing a surge in virus cases.Credit...Erin
Schaff/The New York Times

By early June, it was clear that the White House had gotten it wrong.

In task force meetings, officials discussed a spike in cases across the
South and whether any bumps in caseloads were caused by crowded protests
over the killing of George Floyd. They briefly considered if it was a
fleeting side effect of Memorial Day gatherings.

They soon realized there was more at play.

Digging into new data from Dr. Birx, they concluded the virus was in
fact spreading with invisible ferocity during the weeks in May when
states were opening up with Mr. Trump's encouragement and many were all
but declaring victory.

With the benefit of hindsight, the head of the Centers for Disease
Control and Prevention, Dr. Robert R. Redfield, acknowledged this week
in
\href{https://jamanetwork.com/journals/jama/pages/conversations-with-dr-bauchner}{a
conversation with the Journal of the American Medical Association} that
administration officials --- himself included --- severely
underestimated infections in April and May. He estimated they were
missing as many as 10 cases each day for every one they were confirming.

The number of new cases has now surged far higher than the previous peak
of more than 36,000 a day in mid-April. On Thursday, there were more
than 75,000 confirmed new cases, a record.

Mr. Trump's disdain for testing continues to affect the country. By the
middle of June, lines stretched for blocks in Phoenix and in Austin,
Texas. And getting results could take a week to 10 days, officials in
Texas said --- effectively inviting the virus to spread uncontrollably.

Dr. Mandy K. Cohen, the top health official in North Carolina, contacted
the Trump administration after a surge in June, asking the government to
quickly open 100 new testing sites in her state, in addition to the 13
it was then operating.

``We will keep those 13 open for another month --- you are welcome,''
Dr. Cohen said, mocking the response she received.

It was a devastating situation, said Mayor Steve Adler of Austin, who
watched as the
\href{https://dshs.texas.gov/coronavirus/additionaldata.aspx}{Covid-19
cases} at intensive care units at area hospitals jumped from three in
mid-May to 185 by early July. Mr. Adler had a simple plea for the White
House.

``When we were trying to get people to wear masks, they would point to
the president and say, well, not something that we need to do,'' he
said.

Mr. Suarez expressed similar frustrations with Mr. Trump's dismissive
approach to mask wearing. ``People follow leaders,'' he said, before
rephrasing his remarks. ``People follow the people who are supposed to
be leaders.''

Advertisement

\protect\hyperlink{after-bottom}{Continue reading the main story}

\hypertarget{site-index}{%
\subsection{Site Index}\label{site-index}}

\hypertarget{site-information-navigation}{%
\subsection{Site Information
Navigation}\label{site-information-navigation}}

\begin{itemize}
\tightlist
\item
  \href{https://help.nytimes.com/hc/en-us/articles/115014792127-Copyright-notice}{©~2020~The
  New York Times Company}
\end{itemize}

\begin{itemize}
\tightlist
\item
  \href{https://www.nytco.com/}{NYTCo}
\item
  \href{https://help.nytimes.com/hc/en-us/articles/115015385887-Contact-Us}{Contact
  Us}
\item
  \href{https://www.nytco.com/careers/}{Work with us}
\item
  \href{https://nytmediakit.com/}{Advertise}
\item
  \href{http://www.tbrandstudio.com/}{T Brand Studio}
\item
  \href{https://www.nytimes.com/privacy/cookie-policy\#how-do-i-manage-trackers}{Your
  Ad Choices}
\item
  \href{https://www.nytimes.com/privacy}{Privacy}
\item
  \href{https://help.nytimes.com/hc/en-us/articles/115014893428-Terms-of-service}{Terms
  of Service}
\item
  \href{https://help.nytimes.com/hc/en-us/articles/115014893968-Terms-of-sale}{Terms
  of Sale}
\item
  \href{https://spiderbites.nytimes.com}{Site Map}
\item
  \href{https://help.nytimes.com/hc/en-us}{Help}
\item
  \href{https://www.nytimes.com/subscription?campaignId=37WXW}{Subscriptions}
\end{itemize}
