\href{/section/arts}{Arts}\textbar{}Building Accessibility Into America,
Literally

\begin{itemize}
\item
\item
\item
\item
\item
\item
\end{itemize}

\includegraphics{https://static01.nyt.com/images/2020/07/26/multimedia/24ADA-Design-02/merlin_174362385_2ddf650a-b8aa-49fc-bce3-b014e8f1f098-articleLarge.jpg?quality=75\&auto=webp\&disable=upscale}

Sections

\protect\hyperlink{site-content}{Skip to
content}\protect\hyperlink{site-index}{Skip to site index}

critic's notebook

\hypertarget{building-accessibility-into-america-literally}{%
\section{Building Accessibility Into America,
Literally}\label{building-accessibility-into-america-literally}}

Thirty years on, the Americans With Disabilities Act has reshaped the
way designers and the public have come to think about equity, civil
rights and American architecture. But it's only a start.

{[}Image description: People descending a spiraling ramp with bright red
walls.{]} Designing for accessibility can be an opportunity, both
creative and economic. It inspired the helical design of the Ed Roberts
Campus in Berkeley, Calif., by Leddy Maytum Stacy
Architects.Credit...Tim Griffith

Supported by

\protect\hyperlink{after-sponsor}{Continue reading the main story}

\href{https://www.nytimes.com/by/michael-kimmelman}{\includegraphics{https://static01.nyt.com/images/2018/02/20/multimedia/author-michael-kimmelman/author-michael-kimmelman-thumbLarge.jpg}}

By \href{https://www.nytimes.com/by/michael-kimmelman}{Michael
Kimmelman}

\begin{itemize}
\item
  Published July 20, 2020Updated July 21, 2020
\item
  \begin{itemize}
  \item
  \item
  \item
  \item
  \item
  \item
  \end{itemize}
\end{itemize}

\hypertarget{listen-to-this-article}{%
\subsubsection{Listen to This Article}\label{listen-to-this-article}}

Computer-generated audio file

The \href{https://vimeo.com/328233990}{Capitol Crawl}, it came to be
called. In March 1990, several dozen activists, cheered on by
supporters, left their canes and wheelchairs and pulled themselves up
the steep stone steps of the United States Capitol.

They wanted to pressure Congress into ratifying the Americans With
Disabilities Act. At the heart of what became a landmark of civil rights
legislation was the elemental role of architecture and design ---
literally building accessibility into cities, products, public spaces
and workplaces, without which equity would remain just talk. Business
leaders predicted doomsday costs if the A.D.A. passed. The New York
Times even published an editorial titled
``\href{https://www.nytimes.com/1989/09/06/opinion/blank-check-for-the-disabled.html}{Blank
Check for the Disabled?}''

Thirty years on, the
\href{https://www.curbed.com/2015/7/23/9937976/how-the-americans-with-disabilities-act-transformed-architecture}{A.D.A.
has reshaped American architecture} and the way designers and the public
have come to think about civil rights and the built world. We take for
granted the ubiquity of entry ramps, Braille signage, push buttons at
front doors, lever handles in lieu of doorknobs, widened public toilets,
and warning tiles on street corners and subway platforms. New
courthouses, schools and museums no longer default to a flight of stairs
out front to express their elevated ideals. The A.D.A. has baked a more
egalitarian aesthetic of forms and spaces into the civic DNA.

But there's still a long way to go.

Last fall, the 22,000-square-foot, \$41.5 million Hunters Point branch
library opened in Queens, N.Y. With a soaring interior of vertiginous
tiers and zigzagging stairs, the project's architectural ambition was
obvious and outsized.
\href{https://www.nytimes.com/2019/09/18/arts/design/hunters-point-community-library.html}{My
review} called it one of the most uplifting public buildings New York
had produced in years.

Disability rights advocates
\href{https://gothamist.com/news/new-41-million-hunters-point-library-has-one-major-flaw}{saw
it} differently. All those stairs and tiers made certain areas
inaccessible to people in wheelchairs, they pointed out. How uplifting
could a public library be if some people --- who expected, deserved and
needed to use it --- felt unwelcome?

\includegraphics{https://static01.nyt.com/images/2020/07/26/multimedia/24ADA-Design-01/merlin_163050270_7c393612-be37-4db7-a8e5-30d8f41345a9-articleLarge.jpg?quality=75\&auto=webp\&disable=upscale}

They were right. I was wrong. City officials insisted that the building
complied with A.D.A. regulations, in effect pointing the finger at the
law itself. As Karen Braitmayer, a Seattle architect and accessibility
consultant, put it, ``that's very definitely neither the spirit nor the
goal'' of the legislation.

The question I wish I'd asked at the time is one that architects and
designers might ask themselves more often today. Bess Williamson, author
of ``Accessible America,''
\href{https://twitter.com/besswww/status/1180128788065198080}{tweeted}
it when the library opened: ``Who sets the priorities?''

A public building has everyone as its client, after all. Does its design
evolve out of a truly collaborative process that engages, upfront, the
diversity of users, including those with disabilities, who know best
what they need and want?

``There is only so much that legislation can ever do,'' Xian Horn, a
disability rights advocate, speaker and teacher in New York, born with
cerebral palsy, said recently. ``The issue goes beyond civil rights.
It's also about hospitality, patronage, a broader vision of
accessibility, and ultimately about doing what's best for the bottom
line.''

With
\href{https://www.cdc.gov/ncbddd/disabilityandhealth/infographic-disability-impacts-all.html}{one
in four American adults living with disabilities}, designing for
accessibility and diversity should hardly be considered a chore or just
a compliance issue. It's an opportunity, both economic and creative, but
one that requires a shift in mind-set. A ramp can be something stuck
onto a building to check off some legal requirement.

Or it can inspire the helical design of the
\href{https://www.edrobertscampus.org/design/}{Ed Roberts Campus} in
Berkeley, Calif., by Leddy Maytum Stacy Architects. Or the serpentine
pathways of the
\href{http://www.weissmanfredi.com/project/brooklyn-botanic-garden-robert-w-wilson-overlook}{Robert
W. Wilson Overlook} that Weiss/Manfredi, the New York architecture firm,
recently devised to wind through the Brooklyn Botanic Garden.

``Architecture, from Vitruvius through Le Corbusier, has mirrored
Western culture, for whom the default user has always been the straight,
white, healthy, tall male,'' said Joel Sanders, a New York-based
architect and Yale professor who runs MIXdesign, a think tank focused on
inclusion. ``Everyone else, including those with mobility or cognitive
issues, tends to become an afterthought, a constraint to creativity, an
added cost.''

Nearly a century ago, tubular steel inspired both Marcel Breuer's
\href{https://www.knoll.com/product/wassily-chair}{Wassily chair} and
new, lighter wheelchairs. Chairs by Charles and Ray Eames, now classics
of midcentury modernism, evolved from a molded
\href{https://hyperallergic.com/328930/leg-splint-shaped-iconic-eames-chair/}{plywood
splint} the couple devised for wounded soldiers during World War II.

``The most beautiful things we design,'' as Mr. Sanders put it, ``are
often the ones whose formal innovation is the product of a social or
cultural need.''

Image

{[}Image description: A path winding around greenery in a garden.{]} At
Brooklyn Botanic Garden, the serpentine pathways of the Robert W. Wilson
Overlook were devised by Weiss/Manfredi, the New York architecture firm,
to wind through the gardenCredit...Steven Severinghaus, via Brooklyn
Botanic Garden

A couple of years ago, the Cooper Hewitt, Smithsonian Design Museum
organized an eye-opening show called
\href{https://www.nytimes.com/2018/01/24/arts/design/cooper-hewitt-access-ability.html}{``Access+Ability''}
highlighting those sorts of designs. It included stylish puffer jackets
with Velcro seams and zip-on sleeves, sold by Target, the retail giant,
and a pair of FlyEase, designed by Nike in response to a letter from a
college-bound teenager with cerebral palsy who asked for sneakers that
didn't look like medical equipment. The sneakers looked fantastic.

A term of art emerged during the 1960s and '70s: universal design.

Today, ``universal design'' can sound a little reductive and creaky,
with its implication of a single norm, as if difference (physical,
cognitive, racial, gender, religious, age, you name it) boils down to a
condition that needs to be compensated for --- as if it can't be
something worth celebrating. There's a
\href{https://www.ncbi.nlm.nih.gov/pmc/articles/PMC6913847/\#:~:text=an\%20inspirational\%20video\%3F-,Sensationalizing\%20Cochlear\%20Implants,\%E2\%80\%9Cmiracle\%20cure\%E2\%80\%9D\%20for\%20deafness.\&text=When\%20the\%20implant\%20is\%20first,sudden\%20flood\%20of\%20sensory\%20inputs.}{debate}
today in the Deaf community about cochlear implants, for example,
because to some detractors the implants imply that deafness is a problem
to be solved, not a culture and condition with its own history and
pride.

``Noncompliant bodies don't need to be `made whole' by designers,'' as
Mr. Sanders told me.

He prefers the term ``inclusive design.''

Call it what you will, examples include text to speech and the homely
curb cut --- that little ramp carved into street corners so wheelchair
users can navigate the six or so vertical inches between pavement and
sidewalk. Mandated by the A.D.A., during the last three decades the curb
cut has made daily life easier for countless shoppers with grocery
carts, teenagers on skateboards, travelers pulling wheeled bags, parents
with strollers and just about everybody else.

``As a person with a mobility disability who grew into adulthood prior
to the passage of the A.D.A.,'' recalled Ms. Braitmayer, the architect
and consultant, ``I spent my early years in a community without curb
cuts on the street corners, without accessible parking at the
neighborhood retail shopping centers, without wheelchair spaces in the
movie theaters. I appreciate every time I can now go to a movie and find
a place to sit without being told that if I sit in the aisle in my
wheelchair, I would be a fire hazard.''

But for architects and designers, abiding by the letter of the A.D.A.
isn't the same as internalizing its civil rights goals, she said. That
is what still remains for architecture --- ``to lift itself into the
next realm,'' was Ms. Braitmayer's phrase.

``People don't only buy Apple products because they're functional,''
added Ms. Horn, the advocate and teacher. ``If you make a phone or a
building or a park or a hotel beautiful and also accessible, it makes
life better for everyone.''

Advertisement

\protect\hyperlink{after-bottom}{Continue reading the main story}

\hypertarget{site-index}{%
\subsection{Site Index}\label{site-index}}

\hypertarget{site-information-navigation}{%
\subsection{Site Information
Navigation}\label{site-information-navigation}}

\begin{itemize}
\tightlist
\item
  \href{https://help.nytimes.com/hc/en-us/articles/115014792127-Copyright-notice}{©~2020~The
  New York Times Company}
\end{itemize}

\begin{itemize}
\tightlist
\item
  \href{https://www.nytco.com/}{NYTCo}
\item
  \href{https://help.nytimes.com/hc/en-us/articles/115015385887-Contact-Us}{Contact
  Us}
\item
  \href{https://www.nytco.com/careers/}{Work with us}
\item
  \href{https://nytmediakit.com/}{Advertise}
\item
  \href{http://www.tbrandstudio.com/}{T Brand Studio}
\item
  \href{https://www.nytimes.com/privacy/cookie-policy\#how-do-i-manage-trackers}{Your
  Ad Choices}
\item
  \href{https://www.nytimes.com/privacy}{Privacy}
\item
  \href{https://help.nytimes.com/hc/en-us/articles/115014893428-Terms-of-service}{Terms
  of Service}
\item
  \href{https://help.nytimes.com/hc/en-us/articles/115014893968-Terms-of-sale}{Terms
  of Sale}
\item
  \href{https://spiderbites.nytimes.com}{Site Map}
\item
  \href{https://help.nytimes.com/hc/en-us}{Help}
\item
  \href{https://www.nytimes.com/subscription?campaignId=37WXW}{Subscriptions}
\end{itemize}
