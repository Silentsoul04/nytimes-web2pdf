Sections

SEARCH

\protect\hyperlink{site-content}{Skip to
content}\protect\hyperlink{site-index}{Skip to site index}

\href{https://www.nytimes.com/section/business/economy}{Economy}

\href{https://myaccount.nytimes.com/auth/login?response_type=cookie\&client_id=vi}{}

\href{https://www.nytimes.com/section/todayspaper}{Today's Paper}

\href{/section/business/economy}{Economy}\textbar{}U.S. Imposes
Sanctions on 11 Chinese Companies Over Human Rights

\url{https://nyti.ms/3hnYmV9}

\begin{itemize}
\item
\item
\item
\item
\item
\end{itemize}

Advertisement

\protect\hyperlink{after-top}{Continue reading the main story}

Supported by

\protect\hyperlink{after-sponsor}{Continue reading the main story}

\hypertarget{us-imposes-sanctions-on-11-chinese-companies-over-human-rights}{%
\section{U.S. Imposes Sanctions on 11 Chinese Companies Over Human
Rights}\label{us-imposes-sanctions-on-11-chinese-companies-over-human-rights}}

The move, which affects suppliers to major international brands such as
Apple, Ralph Lauren and Tommy Hilfiger, could force companies to sever
some ties to China.

\includegraphics{https://static01.nyt.com/images/2020/07/20/business/20dc-china-trade-01/merlin_173817375_1eaad563-5d05-4776-b32e-c6081beed404-articleLarge.jpg?quality=75\&auto=webp\&disable=upscale}

\href{https://www.nytimes.com/by/ana-swanson}{\includegraphics{https://static01.nyt.com/images/2018/12/10/multimedia/author-ana-swanson/author-ana-swanson-thumbLarge.png}}

By \href{https://www.nytimes.com/by/ana-swanson}{Ana Swanson}

\begin{itemize}
\item
  Published July 20, 2020Updated July 22, 2020
\item
  \begin{itemize}
  \item
  \item
  \item
  \item
  \item
  \end{itemize}
\end{itemize}

\href{https://cn.nytimes.com/business/20200721/china-sanctions-uighurs-labor/}{阅读简体中文版}\href{https://cn.nytimes.com/business/20200721/china-sanctions-uighurs-labor/zh-}{閱讀繁體中文版}

WASHINGTON --- The Trump administration on Monday barred 11 new Chinese
companies from purchasing American technology and products without a
special license, saying the firms were complicit in human rights
violations in China's campaign targeting Muslim minorities in the
Xinjiang region.

The list of sanctioned companies includes current and former suppliers
to major international brands such as Apple, Ralph Lauren, Google, HP,
Tommy Hilfiger, Hugo Boss and Muji, according to
\href{https://www.aspi.org.au/report/uyghurs-sale}{a report} by the
Australian Strategic Policy Institute, a think tank established by the
Australian government. The group cited the websites of the sanctioned
Chinese companies, which mentioned their financial relationships with
major American brands.

The administration's announcement could precipitate more efforts by
prominent clothing and technology brands to sever ties with opaque
supply chains that touch on Xinjiang, a major source of cotton,
textiles, petrochemicals and other goods that feed into Chinese
factories.

Human rights groups and journalists have documented a
\href{https://www.nytimes.com/interactive/2019/11/16/world/asia/china-xinjiang-documents.html}{campaign
of mass detentions} carried out by the Chinese government in Xinjiang,
in which one million or more members of Muslim and other minority groups
have been placed into large internment camps intended to increase their
loyalty to the Communist Party. Some of these detainees are
\href{https://www.nytimes.com/2018/12/16/world/asia/xinjiang-china-forced-labor-camps-uighurs.html}{forced
to work in factories} in or near the camps, often processing Xinjiang's
abundant cotton crop into various textiles that may then be funneled
into international supply chains.

A
\href{https://www.nytimes.com/2020/07/19/world/asia/china-mask-forced-labor.html}{Times
video investigation} identified Chinese companies using a contentious
labor program for Muslim Uighurs to satisfy demand for face masks and
other personal protective equipment, some of which ended up in the
United States and other countries.

Nine of the companies that the Trump administration cited on Monday,
including Changji Esquel Textile Co. Ltd., Nanchang O-Film Tech and
Hetian Taida Apparel Co. Ltd., were added to the so-called entity list
for their use of forced labor, the Commerce Department said. Two other
companies, Xinjiang Silk Road BGI and Beijing Liuhe BGI, were added for
conducting genetic analyses that were used to further the repression of
Uighurs and other Muslim minorities in Xinjiang, according to the
\href{https://www.commerce.gov/news/press-releases/2020/07/commerce-department-adds-eleven-chinese-entities-implicated-human}{announcement}.

The blacklist only prevents U.S. companies from selling components or
technologies to Chinese companies without a license, not from purchasing
products. In practice, however, major international brands are unlikely
to continue doing business with any firm named on a government list for
forced labor or other abuses in Xinjiang.

``Beijing actively promotes the reprehensible practice of forced labor
and abusive DNA collection and analysis schemes to repress its
citizens,'' Wilbur Ross, the secretary of commerce, said in a statement.
``This action will ensure that our goods and technologies are not used
in the Chinese Communist Party's despicable offensive against
defenseless Muslim minority populations.''

The move comes amid
\href{https://www.nytimes.com/2020/07/14/world/asia/cold-war-china-us.html}{rising
tensions} between the United States and China, and less than two weeks
after the administration
\href{https://www.nytimes.com/2020/07/09/world/asia/trump-china-sanctions-uighurs.html}{imposed
sanctions} on multiple Chinese officials for aiding in human rights
abuses.

Mr. Trump held off on sanctions over China's treatment of its Uighur
minority for much of 2018 and 2019
\href{https://www.nytimes.com/2019/05/04/world/asia/trump-china-uighurs-trade-deal.html}{in
the interest of closing a trade deal} with China, which he
\href{https://www.nytimes.com/2020/01/15/business/economy/china-trade-deal.html}{signed
in January}. Since then, the Trump administration has become more
critical of China, blaming it for not doing enough to contain the
coronavirus and
\href{https://www.nytimes.com/2020/05/29/us/politics/trump-hong-kong-china-WHO.html}{rebuking
a new security law} that increases Beijing's control over Hong Kong.

The announcement on Monday is the latest step in the administration's
campaign to bar Chinese companies from buying products from American
companies. The United States had previously placed 37 companies on its
entity list for violations related to Xinjiang. The Trump administration
has also sanctioned a variety of Chinese technology companies, including
\href{https://www.nytimes.com/2020/05/15/business/economy/commerce-department-huawei.html}{Huawei},
for national security threats.

One of the companies sanctioned on Monday, Nanchang O-Film Tech, has
said that it manufactured selfie cameras for some models of the iPhone,
as well as other camera and touch screen components for Huawei, Lenovo
and Samsung.

In December 2017, Tim Cook, Apple's chief executive, visited O-Film's
Guangzhou factory, posting a picture of himself on the Chinese social
media platform Weibo, according to the report from the Australian
Strategic Policy Institute.

\includegraphics{https://static01.nyt.com/images/2020/07/20/business/20dc-china-trade-02/merlin_130790091_6f3b2823-8046-41f4-837d-d883ffa8026d-articleLarge.jpg?quality=75\&auto=webp\&disable=upscale}

``Getting a closer look at the remarkable, precision work that goes into
manufacturing the selfie cameras for iPhone 8 and iPhone X at O-Film,''
the post read. According to a O-Film news release that has since been
deleted, Mr. Cook praised the company's ``human approach towards
employees'' and said the workers seemed to be living ``happily,''
according to the ASPI report.

Before that visit, 700 Uighurs were transferred from Xinjiang to work at
an O-Film factory in Nanchang, Jiangxi Province, a move that was
expected to ``gradually alter their ideology'' and increase their
``gratitude toward the Party and contribute to stability,'' the ASPI
report said, citing a Xinjiang newspaper.

It remains unclear whether the government in Xinjiang ultimately
supplied more workers to O-Film. Apple did not immediately respond to a
request for comment. O-Film could not immediately be reached for
comment.

Another company on the list, Hefei Bitland Information Technology Co,
has said on its website that its cooperative partners include Google,
HP, Haier, iFlytek and Lenovo. Another listed company, Changji Esquel
Textile Co. Ltd, also appears to have ties to major international
brands, working with Ralph Lauren, Tommy Hilfiger, Hugo Boss and Muji,
according to the Chinese company's website.

PVH, which owns the Tommy Hilfiger brand, Ralph Lauren, Hugo Boss and a
representative for Muji in the United States, where the brand is
restructuring, did not immediately return requests for comment on
Monday.

\href{https://www.wsj.com/articles/western-companies-get-tangled-in-chinas-muslim-clampdown-11558017472}{The
Wall Street Journal reported} in May 2019 that Esquel had set up three
spinning mills in Xinjiang, and that the company had taken in at least
34 Uighur workers offered by Chinese officials. In a
\href{https://www.esquel.com/news/esquel-opposes-use-forced-labor}{statement}
this April, Esquel denied that it had ever used forced labor and called
the statements ``completely false and deeply upsetting.''

In a letter to Mr. Ross on Monday, Esquel again said it did not and
would never use forced labor, and asked to be removed from the list.

``Where is the evidence that Esquel has ever, in its 25 years of
operations in Xinjiang, used forced labor?'' wrote John Cheh, the chief
executive of Esquel Group. ``No agency of any government nor any
nongovernmental organization has presented such evidence, because it
does not exist. In the lead up to including our Changji mill on the
entity list, no one from the Commerce Department spoke with anyone at
Esquel or we would have gladly provided them with the facts and answered
any questions at that time.''

The companies on the entity list also include KTK Group, which supplies
components for high-speed trains, and Hetian Haolin Hair Accessories Co.
Ltd. On July 1, U.S. Customs and Border Protection seized a shipment of
13 tons of hair products manufactured by Lop County Meixin Hair Product
Co. Ltd. that it suspected were made with human hair originating in
Xinjiang.

Advertisement

\protect\hyperlink{after-bottom}{Continue reading the main story}

\hypertarget{site-index}{%
\subsection{Site Index}\label{site-index}}

\hypertarget{site-information-navigation}{%
\subsection{Site Information
Navigation}\label{site-information-navigation}}

\begin{itemize}
\tightlist
\item
  \href{https://help.nytimes.com/hc/en-us/articles/115014792127-Copyright-notice}{©~2020~The
  New York Times Company}
\end{itemize}

\begin{itemize}
\tightlist
\item
  \href{https://www.nytco.com/}{NYTCo}
\item
  \href{https://help.nytimes.com/hc/en-us/articles/115015385887-Contact-Us}{Contact
  Us}
\item
  \href{https://www.nytco.com/careers/}{Work with us}
\item
  \href{https://nytmediakit.com/}{Advertise}
\item
  \href{http://www.tbrandstudio.com/}{T Brand Studio}
\item
  \href{https://www.nytimes.com/privacy/cookie-policy\#how-do-i-manage-trackers}{Your
  Ad Choices}
\item
  \href{https://www.nytimes.com/privacy}{Privacy}
\item
  \href{https://help.nytimes.com/hc/en-us/articles/115014893428-Terms-of-service}{Terms
  of Service}
\item
  \href{https://help.nytimes.com/hc/en-us/articles/115014893968-Terms-of-sale}{Terms
  of Sale}
\item
  \href{https://spiderbites.nytimes.com}{Site Map}
\item
  \href{https://help.nytimes.com/hc/en-us}{Help}
\item
  \href{https://www.nytimes.com/subscription?campaignId=37WXW}{Subscriptions}
\end{itemize}
