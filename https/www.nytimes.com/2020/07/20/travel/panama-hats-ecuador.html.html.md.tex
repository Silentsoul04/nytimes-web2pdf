Sections

SEARCH

\protect\hyperlink{site-content}{Skip to
content}\protect\hyperlink{site-index}{Skip to site index}

\href{/section/travel}{Travel}\textbar{}A Glimpse Inside the Workshops
of the World's Finest Panama Hat Makers

\url{https://nyti.ms/3fWsJ4o}

\begin{itemize}
\item
\item
\item
\item
\item
\item
\end{itemize}

\includegraphics{https://static01.nyt.com/images/2020/07/21/travel/20travel-panama-01/20travel-panama-promo-articleLarge.jpg?quality=75\&auto=webp\&disable=upscale}

The World Through a Lens

\hypertarget{a-glimpse-inside-the-workshops-of-the-worlds-finest-panama-hat-makers}{%
\section{A Glimpse Inside the Workshops of the World's Finest Panama Hat
Makers}\label{a-glimpse-inside-the-workshops-of-the-worlds-finest-panama-hat-makers}}

Creamy as silk and costlier than gold, a Montecristi superfino Panama
hat is as much a work of art as it is of fashion.

Gabriel Lucas irons a hat at his workshop in Montecristi,
Ecuador.Credit...

Supported by

\protect\hyperlink{after-sponsor}{Continue reading the main story}

Photographs and Text by Roff Smith

\begin{itemize}
\item
  Published July 20, 2020Updated July 22, 2020
\item
  \begin{itemize}
  \item
  \item
  \item
  \item
  \item
  \item
  \end{itemize}
\end{itemize}

\emph{At the onset of the coronavirus pandemic, with travel restrictions
in place worldwide, we launched a new series ---}
\href{https://www.nytimes.com/column/the-world-through-a-lens}{\emph{The
World Through a Lens}} \emph{--- in which photojournalists help
transport you, virtually, to some of our planet's most beautiful and
intriguing places. This week, Roff Smith shares a collection of
photographs from the workshops of hat artisans in Ecuador.}

\begin{center}\rule{0.5\linewidth}{\linethickness}\end{center}

Creamy as silk, costlier by weight than gold, the color of fine old
ivory, a Montecristi superfino Panama hat is as much a work of art as it
is of fashion. The finest specimens have more than 4,000 weaves per
square inch, a weave so fine it takes a jeweler's loupe to count the
rows. And every single one of those weaves is done by hand. No loom is
used --- only dexterous fingers, sharp eyes and Zen-like concentration.

``You cannot allow your mind to wander even for a second,'' says Simón
Espinal, a modest, soft-spoken man who is regarded by his peers as the
greatest living weaver of Panama hats, possibly the greatest ever.
``When you are weaving it is just you and the straw.''

Image

Simón Espinal examines the weave on one of his masterpiece hats.

Image

Mr. Espinal holds one of the slender straws which he will weave into a
Panama hat worth more than its weight in gold.

\includegraphics{https://static01.nyt.com/images/2020/07/20/travel/20travel-panama-21/merlin_174060129_b7033e25-7181-4ce9-9219-8377acfa3bbc-articleLarge.jpg?quality=75\&auto=webp\&disable=upscale}

Mr. Espinal's hats average around 3,000 weaves per square inch --- a
fineness few weavers have ever even approached. His best has just over
4,200 weaves per square inch and took him five months to weave.

Image

Gabriel Lucas replaces a straw in a Panama hat at his workshop in
Montecristi.

The 52-year-old Ecuadorean is one of a dwindling number of elite Panama
hat weavers, nearly all of whom live in Pile, an obscure village tucked
away in the foothills behind Montecristi, a low-slung town about 100
miles up the coast from Guayaquil.

Image

A superfino hat in the process of being woven.

Image

The workshop of Gabriel Lucas, one of the great finishing artisans in
Montecristi.

I became interested in the hats about 15 years ago, quite by accident,
when I read about straw hats that could cost thousands of dollars.
Intrigued, I began researching the hats, made a trip to Ecuador ---
where all true Panama hats are woven --- and discovered this curious,
and gently anachronistic world of the hat weavers of Montecristi.

Image

Patricia Lopez displays the beginnings of a Panama hat.

Although the weaver is the star of the show, the making of a Montecristi
is a collaborative art. After the weaver has finished his or her part,
the raw hat body passes through the hands of a tag-team of specialist
artisans whose titles --- the rematador, the cortador, the apaleador and
the planchador --- lend the making of a Montecristi Panama hat something
of the hot-blooded formality of the bullring. (The term rematador is
drawn directly from bullfighting: There, it is the finisher, one who
``performs some act that will provide an emotional or artistic climax,''
as Hemingway describes it in ``Death in the Afternoon.'')

Image

Straw hanging out to dry. To prepare it for weaving, the straw is
lightly boiled for about a minute, and is then allowed to dry overnight
in the open air.

In Montecristi, the rematador is the specialist weaver who performs the
complicated back weave to seal the brim, thereby bringing to an artistic
close the weaving phase of the hat's creation. After that, the excess
straw is trimmed away by the cortador, who then gives the hat the
closest of shaves with a razor blade to trim away any burrs in the
straw.

``Sometimes, when I am cortador-ing, I come across a straw that has
become discolored or has not been woven correctly,'' says Gabriel Lucas,
one of Montecristi's top finishing artisans, as he performs a delicate
operation on a fine hat that will be worth thousands when it is
finished. ``We call these hijos perdidos --- the lost straws. I have to
carefully cut them out and weave in a new straw to replace it.''

Image

One of the jobs of the finishing artisans is to inspect the hat for any
miswoven or discolored straws. If found, they are cut out and replaced.

Image

The cortador trims the excess straw from the newly woven hat body, then
gives it the finest of shaves with a razor blade to trim away any
prickly bits. Here, the 34-year-old artisan Gabriel Lucas performs the
task at his workshop in Montecristi.

After it has been properly barbered, the hat is pounded with a hardwood
mallet by the apaleador to help bed the fibers, then briskly ironed by
the planchador to give it the right amount of stiffness in preparation
for the final stage: blocking, or the sculpting by hand of the unformed
hat into its recognizable styles: fedora, optimo, plantation.

Image

Gabriel Lucas firmly irons a hat to help the straw hold its structure.

Panama hats are uniquely Ecuadorean, despite their curious misnomer. The
term ``Panama hat'' has been in use since at least the 1830s, and came
about because the hats were often sold in trading posts on the Isthmus
of Panama, which was a shipping crossroads long before the canal was
built. The name was popularized during the California gold rush, when
tens of thousands of prospectors passed through Panama on their way to
the diggings, many of them picking up a hat along the way.

Image

Panama hats are woven from the fibers of the toquilla palm ---
Carludovica palmata.

Image

Immature shoots of the toquilla palm are shucked, and the
fettuccini-like fibers are split again and again to make the fine straw
required for a beautiful hat.

Panama hats became even more firmly fixed in the popular imagination
after the Paris Exposition in 1855, when a Frenchman who had been living
in Panama presented Napoleon III with a finely woven hat. His Highness
loved the hat and wore it everywhere.

Then, as now, celebrities set the tone in the fashion stakes, and nobody
was more A-list than the Emperor of France. Silky fine Panama hats for
spring and summer became de rigueur among the rich and famous. King
Edward VII is said to have instructed his hatter to spare no expense but
get him the finest Panama available. Fabulous sums were paid by him and
others for the best hats. A
\href{https://www.newyorker.com/magazine/1930/07/05/thousand-dollar-hats}{Talk
of The Town article} in The New Yorker from July 1930 describes a
\$1,000 Panama --- around \$16,000 today --- on display at Dobbs hat
store in the city.
\href{https://www.pbs.org/wnet/broadway/stars/florenz-ziegfeld/}{Florenz
Ziegfeld} was discussed as a likely buyer.

Image

The top of a Panama hat is called the plantilla --- this one woven by
Mr. Espinal.

These days, the overwhelming majority of Panama hats are woven in
Cuenca, an attractive town in the Andes whose residents, prompted by the
local government, turned to hat weaving in the mid 1800s, once Panama
hats became popular. These are the hats you find in department stores
and most hat shops. Nice hats, they are woven in a light, simple
``brisa'' weave, which can be turned out swiftly and in commercial
quantities.

Montecristi, on the other hand, is the seat of the art. Locals have been
weaving fine hats out of the fibers of the
\href{https://timesmachine.nytimes.com/timesmachine/1900/09/02/101066082.html?pageNumber=24}{toquilla
palm} for centuries. Here, hat making has remained a cottage industry,
the weavers gathering and preparing their own straw as they have for
generations, weaving their hats in their artistic and time-consuming
``liso'' weave, a pretty herringbone style.

Image

Mr. Espinal focuses on keeping straight the countless strands of straw
as he weaves another of his masterpiece hats.

Their output is necessarily small, and that of the elite weavers in Pile
smaller still. In a good year, Simón Espinal might make three hats.

Lately the government has been urging the weavers in Pile to become more
commercial, to abandon the old ways, not to weave such fine hats --- but
they've refused. ``This,'' says Simón Espinal, ``is a gift from God.''

\begin{center}\rule{0.5\linewidth}{\linethickness}\end{center}

\href{http://www.roffsmithphotography.com/}{\emph{Roff Smith}} \emph{is
a writer and photographer based in England. You can follow his
adventures on Instagram:}
\href{https://www.instagram.com/roffsmith/}{\emph{@roffsmith}}\emph{.}

\emph{\textbf{Follow New York Times Travel}} \emph{on}
\href{https://www.instagram.com/nytimestravel/}{\emph{Instagram}}\emph{,}
\href{https://twitter.com/nytimestravel}{\emph{Twitter}} \emph{and}
\href{https://www.facebook.com/nytimestravel/}{\emph{Facebook}}\emph{.
And}
\href{https://www.nytimes.com/newsletters/traveldispatch}{\emph{sign up
for our weekly Travel Dispatch newsletter}} \emph{to receive expert tips
on traveling smarter and inspiration for your next vacation.}

Advertisement

\protect\hyperlink{after-bottom}{Continue reading the main story}

\hypertarget{site-index}{%
\subsection{Site Index}\label{site-index}}

\hypertarget{site-information-navigation}{%
\subsection{Site Information
Navigation}\label{site-information-navigation}}

\begin{itemize}
\tightlist
\item
  \href{https://help.nytimes.com/hc/en-us/articles/115014792127-Copyright-notice}{©~2020~The
  New York Times Company}
\end{itemize}

\begin{itemize}
\tightlist
\item
  \href{https://www.nytco.com/}{NYTCo}
\item
  \href{https://help.nytimes.com/hc/en-us/articles/115015385887-Contact-Us}{Contact
  Us}
\item
  \href{https://www.nytco.com/careers/}{Work with us}
\item
  \href{https://nytmediakit.com/}{Advertise}
\item
  \href{http://www.tbrandstudio.com/}{T Brand Studio}
\item
  \href{https://www.nytimes.com/privacy/cookie-policy\#how-do-i-manage-trackers}{Your
  Ad Choices}
\item
  \href{https://www.nytimes.com/privacy}{Privacy}
\item
  \href{https://help.nytimes.com/hc/en-us/articles/115014893428-Terms-of-service}{Terms
  of Service}
\item
  \href{https://help.nytimes.com/hc/en-us/articles/115014893968-Terms-of-sale}{Terms
  of Sale}
\item
  \href{https://spiderbites.nytimes.com}{Site Map}
\item
  \href{https://help.nytimes.com/hc/en-us}{Help}
\item
  \href{https://www.nytimes.com/subscription?campaignId=37WXW}{Subscriptions}
\end{itemize}
