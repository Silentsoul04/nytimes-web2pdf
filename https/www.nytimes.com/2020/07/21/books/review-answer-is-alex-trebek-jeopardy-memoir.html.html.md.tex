Sections

SEARCH

\protect\hyperlink{site-content}{Skip to
content}\protect\hyperlink{site-index}{Skip to site index}

\href{https://www.nytimes.com/section/books}{Books}

\href{https://myaccount.nytimes.com/auth/login?response_type=cookie\&client_id=vi}{}

\href{https://www.nytimes.com/section/todayspaper}{Today's Paper}

\href{/section/books}{Books}\textbar{}In Alex Trebek's Reluctant, Moving
Memoir, Life Is All About the Next Question

\url{https://nyti.ms/3eQfiSe}

\begin{itemize}
\item
\item
\item
\item
\item
\end{itemize}

Advertisement

\protect\hyperlink{after-top}{Continue reading the main story}

Supported by

\protect\hyperlink{after-sponsor}{Continue reading the main story}

\href{/column/books-of-the-times}{Books of The Times}

\hypertarget{in-alex-trebeks-reluctant-moving-memoir-life-is-all-about-the-next-question}{%
\section{In Alex Trebek's Reluctant, Moving Memoir, Life Is All About
the Next
Question}\label{in-alex-trebeks-reluctant-moving-memoir-life-is-all-about-the-next-question}}

By \href{https://www.nytimes.com/by/parul-sehgal}{Parul Sehgal}

\begin{itemize}
\item
  July 21, 2020
\item
  \begin{itemize}
  \item
  \item
  \item
  \item
  \item
  \end{itemize}
\end{itemize}

\includegraphics{https://static01.nyt.com/images/2020/07/22/books/21BOOKTREBEK1/21BOOKTREBEK1-articleLarge.png?quality=75\&auto=webp\&disable=upscale}

Buy Book ▾

\begin{itemize}
\tightlist
\item
  \href{https://www.amazon.com/gp/search?index=books\&tag=NYTBSREV-20\&field-keywords=The+Answer+Is...+Alex+Trebek}{Amazon}
\item
  \href{https://du-gae-books-dot-nyt-du-prd.appspot.com/buy?title=The+Answer+Is...\&author=Alex+Trebek}{Apple
  Books}
\item
  \href{https://www.anrdoezrs.net/click-7990613-11819508?url=https\%3A\%2F\%2Fwww.barnesandnoble.com\%2Fw\%2F\%3Fean\%3D9781982157999}{Barnes
  and Noble}
\item
  \href{https://www.anrdoezrs.net/click-7990613-35140?url=https\%3A\%2F\%2Fwww.booksamillion.com\%2Fp\%2FThe\%2BAnswer\%2BIs...\%2FAlex\%2BTrebek\%2F9781982157999}{Books-A-Million}
\item
  \href{https://bookshop.org/a/3546/9781982157999}{Bookshop}
\item
  \href{https://www.indiebound.org/book/9781982157999?aff=NYT}{Indiebound}
\end{itemize}

When you purchase an independently reviewed book through our site, we
earn an affiliate commission.

Alex Trebek was a man in search of a vice.

It was Los Angeles, in the late '70s. The Canadian quiz show emcee had
been tapped to host a new trivia program, the short-lived ``Wizard of
Odds.'' ``I had the world by the tail,'' he writes in his new memoir,
``The Answer Is\ldots{}'' ``I was the talented newcomer in broadcasting.
I was the bright, fair-haired boy.'' He was also saddled with a few
striking disadvantages, as he saw it. ``I didn't drink, didn't smoke,
didn't do drugs,'' he writes. ``There were no big negatives associated
with me.''

He was too chaste to be trusted --- ``it held me back from becoming one
of the guys.'' He tried cursing. He tried boasting about his drinking
even though he privately preferred one percent milk (sufficiently
sinister, to my mind). In the end, he reconciled himself to that
unnerving wholesomeness and reserve, which have become so integral to
his appeal. The ``Jeopardy!'' champion Ken Jennings
\href{https://www.nytimes.com/2019/03/09/opinion/sunday/alex-trebek-jeopardy-ken-jennings.html}{has
described} Trebek as ``a riddle wrapped in an enigma wrapped in a Perry
Ellis suit.''

It's little wonder that Trebek has written a memoir of consummate
caginess, one of the wariest I've read: a friendly, often funny account
marked by a reluctance so deep that it confers a curious integrity upon
the celebrity tell-all. For years, he resisted personal questions (``Get
a life,'' he'd say in interviews) and resisted writing an autobiography.
Only after the outpouring of support following his announcement last
year that he had pancreatic cancer did he feel he owed something to the
public.

\includegraphics{https://static01.nyt.com/images/2020/07/21/books/21booktrebek3/21booktrebek3-articleLarge.jpg?quality=75\&auto=webp\&disable=upscale}

{[} \emph{Read our recent}
\href{https://www.nytimes.com/2020/07/17/books/alex-trebek-jeopardy-the-answer-is.html}{\emph{profile
of Alex Trebek}}\emph{.} {]}

But everything in proportion, please. ``I'm a second-tier celebrity,''
he insists. ``The biggest reason the show has endured is the comfort
that it brings. Viewers have gotten used to having me there, not so much
as a showbiz personality but as an uncle. I'm part of the family more
than an outside celebrity who comes into your home to entertain you.
They find me comforting and reassuring as opposed to being impressed by
me.''

On this point, Trebek is remarkably direct: Even if he can't quite
understand the public fascination with his life, he knows he means
something significant to the culture, something soothing and in short
supply. He knows he fills a need. For the 36 years hosting ``Jeopardy!''
--- an industry record--- he has been a nostalgic father figure of
sorts, showing up reliably at dinnertime and remaining tantalizingly
aloof. In the autumn of the media patriarchs, he stands practically
alone, untinged by scandal. His authority derives from his defense of
facts, not their distortion.

He takes pride in his work, and in the achievements of the contestants
--- when Jennings was finally ousted after winning 74 games in a row,
Trebek teared up. But he never takes himself seriously; his memoir is a
shameless dad-joke extravaganza, largely at his own expense. He is eager
to talk about his hairpiece (``a damn good one''). He shares silly
photos of himself in all-denim outfits (``wearing the Canadian tuxedo is
my birthright'') and posing in the ``Got Milk?'' campaigns of the 1990s
(``I really do love the stuff''). He recalls the early years of
``Jeopardy!'' with relish, when the prizes for runners-up included ``Lee
Nails, `delicious low-calorie meat' from Mr. Turkey and Tinactin
Antifungal Cream --- use only as directed!''

Alex Trebek loves the troops, he loves his wife, he loves his Dodge Ram.
He really loves his bromides. His kids? Champs. His divorce?
\emph{Amazing}; he and his ex are still good friends.

Image

Trebek as a child, on a pony.Credit...Alex Trebek

Around the margins, a darker story blooms. Trebek was born in Sudbury,
Ontario, in 1940, to Ukrainian immigrants --- warm, loving people, if
ill-suited for each other. His father drank. Trebek's early years were
full of poverty, instability and illness, but he presents them with his
typical cloudless beneficence: ``I don't have a lot of ghosts. I don't
have any bad memories that affect my life. It's all good.'' When he was
7, he fell into a frozen lake and became afflicted with painful
rheumatism. For 12 years he'd wake crying in the night until suddenly
the pain disappeared. ``Go figure,'' he shrugs.

Young Trebek had a rebellious streak. He clashed with the nuns at school
and bounced between jobs. He quit military college when he heard that
buzz cuts were mandatory. ``I had a good head of hair --- a sort of
pompadour with a ducktail in the back,'' he writes. (Photographic
evidence is provided.) ``I'd be damned if I was going to let them shave
it off.''

Trebek might have inspired dread in his teachers and early employers,
but he discovered that his real talent was in projecting calm, in
allowing others to shine. As a host, it has been his proudest quality
--- his ability to buoy an anxious contestant through tone alone.

Facts themselves can confer steadiness. A small aside: I took to
``Jeopardy!'' early, and in high school had a weird, cursory career
competing in televised trivia contests. My teammates and I ---
immigrants all, as it happened --- glutted ourselves on dates and data
with a hunger I couldn't have possibly explained at the time but that
now seems embarrassingly obvious. Facts could be trusted. Facts
consoled. Their patient, dogged acquisition constituted a kind of shy
possession of the world.

Of course, any possession in this life is, at best, temporary. ``My life
has been a quest for knowledge and understanding, and I'm nowhere near
having achieved that. And it doesn't bother me in the least,'' Trebek
cheerfully concludes. He ends the book at home, like of all us, in
quarantine. He is exhausted by cancer treatments, exhausted by
uncertainty but still sublimely calm and grateful. As he's always
advised his contestants to do, he's already looking ahead to the next
question.

Advertisement

\protect\hyperlink{after-bottom}{Continue reading the main story}

\hypertarget{site-index}{%
\subsection{Site Index}\label{site-index}}

\hypertarget{site-information-navigation}{%
\subsection{Site Information
Navigation}\label{site-information-navigation}}

\begin{itemize}
\tightlist
\item
  \href{https://help.nytimes.com/hc/en-us/articles/115014792127-Copyright-notice}{©~2020~The
  New York Times Company}
\end{itemize}

\begin{itemize}
\tightlist
\item
  \href{https://www.nytco.com/}{NYTCo}
\item
  \href{https://help.nytimes.com/hc/en-us/articles/115015385887-Contact-Us}{Contact
  Us}
\item
  \href{https://www.nytco.com/careers/}{Work with us}
\item
  \href{https://nytmediakit.com/}{Advertise}
\item
  \href{http://www.tbrandstudio.com/}{T Brand Studio}
\item
  \href{https://www.nytimes.com/privacy/cookie-policy\#how-do-i-manage-trackers}{Your
  Ad Choices}
\item
  \href{https://www.nytimes.com/privacy}{Privacy}
\item
  \href{https://help.nytimes.com/hc/en-us/articles/115014893428-Terms-of-service}{Terms
  of Service}
\item
  \href{https://help.nytimes.com/hc/en-us/articles/115014893968-Terms-of-sale}{Terms
  of Sale}
\item
  \href{https://spiderbites.nytimes.com}{Site Map}
\item
  \href{https://help.nytimes.com/hc/en-us}{Help}
\item
  \href{https://www.nytimes.com/subscription?campaignId=37WXW}{Subscriptions}
\end{itemize}
