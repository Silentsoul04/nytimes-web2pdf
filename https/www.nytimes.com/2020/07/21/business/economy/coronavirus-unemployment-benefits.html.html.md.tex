\href{/section/business/economy}{Economy}\textbar{}End of \$600
Unemployment Bonus Could Push Millions Past the Brink

\url{https://nyti.ms/2WHLktD}

\begin{itemize}
\item
\item
\item
\item
\item
\end{itemize}

\href{https://www.nytimes.com/news-event/coronavirus?action=click\&pgtype=Article\&state=default\&region=TOP_BANNER\&context=storylines_menu}{The
Coronavirus Outbreak}

\begin{itemize}
\tightlist
\item
  live\href{https://www.nytimes.com/2020/08/01/world/coronavirus-covid-19.html?action=click\&pgtype=Article\&state=default\&region=TOP_BANNER\&context=storylines_menu}{Latest
  Updates}
\item
  \href{https://www.nytimes.com/interactive/2020/us/coronavirus-us-cases.html?action=click\&pgtype=Article\&state=default\&region=TOP_BANNER\&context=storylines_menu}{Maps
  and Cases}
\item
  \href{https://www.nytimes.com/interactive/2020/science/coronavirus-vaccine-tracker.html?action=click\&pgtype=Article\&state=default\&region=TOP_BANNER\&context=storylines_menu}{Vaccine
  Tracker}
\item
  \href{https://www.nytimes.com/interactive/2020/07/29/us/schools-reopening-coronavirus.html?action=click\&pgtype=Article\&state=default\&region=TOP_BANNER\&context=storylines_menu}{What
  School May Look Like}
\item
  \href{https://www.nytimes.com/live/2020/07/31/business/stock-market-today-coronavirus?action=click\&pgtype=Article\&state=default\&region=TOP_BANNER\&context=storylines_menu}{Economy}
\end{itemize}

\includegraphics{https://static01.nyt.com/images/2020/07/22/business/22virus-cliff1/22virus-cliff1-articleLarge.jpg?quality=75\&auto=webp\&disable=upscale}

Sections

\protect\hyperlink{site-content}{Skip to
content}\protect\hyperlink{site-index}{Skip to site index}

\hypertarget{end-of-600-unemployment-bonus-could-push-millions-past-the-brink}{%
\section{End of \$600 Unemployment Bonus Could Push Millions Past the
Brink}\label{end-of-600-unemployment-bonus-could-push-millions-past-the-brink}}

A weekly supplement has helped the jobless to pay their bills and
cushioned the economy. As it expires, Congress will determine what comes
next.

Credit...Rebekka Dunlap

Supported by

\protect\hyperlink{after-sponsor}{Continue reading the main story}

\href{https://www.nytimes.com/by/ben-casselman}{\includegraphics{https://static01.nyt.com/images/2018/11/09/multimedia/author-ben-casselman/author-ben-casselman-thumbLarge.png}}

By \href{https://www.nytimes.com/by/ben-casselman}{Ben Casselman}

\begin{itemize}
\item
  Published July 21, 2020Updated July 24, 2020
\item
  \begin{itemize}
  \item
  \item
  \item
  \item
  \item
  \end{itemize}
\end{itemize}

When millions of Americans began losing their jobs in March, the federal
government stepped in with a life preserver:
\href{https://www.nytimes.com/2020/07/30/business/unemployment-payments-change.html}{\$600
a week in extra unemployment benefits} to allow workers to pay rent and
buy groceries, and to cushion the economy.

With economic conditions
\href{https://www.nytimes.com/2020/07/15/business/economy/economic-recovery-coronavirus-resurgence.html}{again
deteriorating}, that life preserver will disappear within days if
Congress doesn't act to extend it. That could prompt a wave of evictions
and inflict more financial harm on millions of Americans while further
damaging the economy.

Even the threat of a lapse in benefits could prove harmful, economists
warn, by forcing households to make precautionary spending cuts.

The benefits program, Federal Pandemic Unemployment Compensation,
expires at the end of July. But because of a quirk in the calendar,
workers in most states won't qualify for the payments after this week.
Most will be left with regular unemployment benefits, which total only a
few hundred dollars a week in many states.

That means that
\href{https://www.nytimes.com/2020/07/17/business/how-many-are-collecting-unemployment-benefits-its-hard-to-say.html}{more
than 20 million Americans} could soon see their weekly income fall by
half or more at a time when the unemployment rate remains higher than in
any period since World War II.

Economists warn that it isn't just individual recipients who will suffer
if the benefits are cut. The federal payments are injecting billions of
dollars into the economy each week, money that flows to landlords,
grocery stores, retailers and countless other businesses. Ernie
Tedeschi, a former Treasury Department official and an economist at
Evercore ISI Research, has estimated that if the payments ceased, the
U.S. gross domestic product would be 2 percent smaller at the end of
2020 and there would be 1.7 million fewer jobs nationwide.

``These unemployment benefit checks are really doing a large job in
propping up spending by these unemployed households,'' said Joseph
Vavra, a University of Chicago economist who has been studying the
impact of the benefits. If they expire, he said, ``there's a good chance
that what is now an unemployment problem becomes a foreclosure crisis
and eviction crisis.''

\hypertarget{latest-updates-economy}{%
\section{\texorpdfstring{\href{https://www.nytimes.com/live/2020/07/31/business/stock-market-today-coronavirus?action=click\&pgtype=Article\&state=default\&region=MAIN_CONTENT_1\&context=storylines_live_updates}{Latest
Updates:
Economy}}{Latest Updates: Economy}}\label{latest-updates-economy}}

\href{https://www.nytimes.com/live/2020/07/31/business/stock-market-today-coronavirus?action=click\&pgtype=Article\&state=default\&region=MAIN_CONTENT_1\&context=storylines_live_updates\#kodaks-chief-executive-was-given-stock-options-then-the-share-price-spiked-1000-percent}{23h
ago}

\href{https://www.nytimes.com/live/2020/07/31/business/stock-market-today-coronavirus?action=click\&pgtype=Article\&state=default\&region=MAIN_CONTENT_1\&context=storylines_live_updates\#kodaks-chief-executive-was-given-stock-options-then-the-share-price-spiked-1000-percent}{Kodak's
chief executive was given stock options. Then the share price spiked
1,000 percent.}

\href{https://www.nytimes.com/live/2020/07/31/business/stock-market-today-coronavirus?action=click\&pgtype=Article\&state=default\&region=MAIN_CONTENT_1\&context=storylines_live_updates\#fitch-ratings-downgrades-its-outlook-on-us-debt}{26h
ago}

\href{https://www.nytimes.com/live/2020/07/31/business/stock-market-today-coronavirus?action=click\&pgtype=Article\&state=default\&region=MAIN_CONTENT_1\&context=storylines_live_updates\#fitch-ratings-downgrades-its-outlook-on-us-debt}{Fitch
Ratings downgrades its outlook on U.S. debt.}

\href{https://www.nytimes.com/live/2020/07/31/business/stock-market-today-coronavirus?action=click\&pgtype=Article\&state=default\&region=MAIN_CONTENT_1\&context=storylines_live_updates\#us-sanctions-more-chinese-officials-over-human-rights-violations-as-tensions-flare}{32h
ago}

\href{https://www.nytimes.com/live/2020/07/31/business/stock-market-today-coronavirus?action=click\&pgtype=Article\&state=default\&region=MAIN_CONTENT_1\&context=storylines_live_updates\#us-sanctions-more-chinese-officials-over-human-rights-violations-as-tensions-flare}{U.S.
sanctions more Chinese officials over human rights violations as
tensions flare}

\href{https://www.nytimes.com/live/2020/07/31/business/stock-market-today-coronavirus?action=click\&pgtype=Article\&state=default\&region=MAIN_CONTENT_1\&context=storylines_live_updates}{See
more updates}

More live coverage:
\href{https://www.nytimes.com/2020/08/01/world/coronavirus-covid-19.html?action=click\&pgtype=Article\&state=default\&region=MAIN_CONTENT_1\&context=storylines_live_updates}{Global}

Congress returned from recess this week to consider a new relief
package, which
\href{https://www.nytimes.com/2020/07/20/us/politics/congress-coronavirus-aid-package.html}{could
include at least a partial extension} of the extra unemployment
benefits. Senate Republicans and the White House are considering a
roughly \$1 trillion package that would retain the program but scale it
back. Democrats are pressing to continue paying the full \$600 a week.

But Congress seems unlikely to act before benefits lapse. And because of
the antiquated computer systems in many state unemployment offices,
which do the processing, it could take weeks to restart payments. That
means that millions are likely to see their income drop at least
temporarily.

For people depending on the checks, that uncertainty is frustrating.

``I have no idea why Congress would wait until a few days before the
checks are going to run out,'' said Jacob Perlman, a benefits recipient
in Chicago. ``This should have been done a month ago.''

Mr. Perlman, 26, earned \$12 an hour as a housekeeper at a fitness club,
making him one of the millions of Americans earning more on unemployment
than they had on the job. But he is eager to return to work.

``The jobs simply are not there right now,'' he said.

Mr. Perlman's regular benefits from the state of Illinois total \$159 a
week, barely enough to cover his \$500 share of the monthly rent, let
alone food or other expenses. So he is already trying to save as much as
possible.

Decisions like Mr. Perlman's to curtail spending even before the
benefits expire, multiplied across millions of households, are a sort of
uncertainty tax on the broader economy, damping the stimulative effect
of the payments.

``There are people who are on the precipice of financial disaster
here,'' said David Wilcox, a former Federal Reserve official who is an
economist at the Peterson Institute for International Economics. ``We
may think that the odds are that Congress will come to a reasonable
conclusion. But for a person who is on the precipice of financial
disaster, it's very low comfort to be told, `You know, I think there's a
70 percent chance that this is going to work out fine.'''

The risk is particularly acute for Black and Latino workers, who have
been disproportionately affected by job losses and are less likely to
have savings or other assets to fall back on. A
\href{https://www.nber.org/papers/w27552}{recent working paper} from
researchers at the University of Chicago and the JPMorgan Chase
Institute found that Black and Latino households cut spending by far
more than white households when their income drops.

``When 30 percent of your population has no wealth, this has real
implications,'' said William E. Spriggs, a Howard University professor
and the chief economist for the A.F.L.-C.I.O. ``There isn't a piggy
bank. This is it. So when you cut their benefits, their drop in
consumption is going to be huge.''

The extra unemployment payments were part of a multitrillion-dollar
federal response to the pandemic's economic devastation. Congress
expanded eligibility for unemployment benefits and food stamps, sent
\$1,200 checks to most households and offered forgivable loans to
millions of small businesses.

Together, those programs did much to offset the damage: Average personal
income rose in April, the worst month of the crisis to date, and
consumer spending rebounded quickly once federal dollars started flowing
into the economy. Mortgage delinquencies, credit card defaults and other
signs of financial stress rose by less than many forecasters initially
feared.

When Congress created the various programs, it still seemed possible
that the pandemic would have begun to ebb by summer and that the economy
would no longer need as much federal help.

Instead, after falling steadily in May and early June, virus cases are
rising in much of the country, and states are reimposing business
restrictions. Real-time measures suggest that the economic recovery that
began in May has
\href{https://www.nytimes.com/live/2020/07/15/business/stock-market-today-coronavirus\#surging-virus-cases-and-renewed-lockdowns-threaten-economic-recovery}{begun
to lose momentum}, and some economists expect the unemployment rate to
start climbing again.

The threat of an economic stall has led some Republicans in Washington
to embrace more aggressive federal action than they were considering a
few weeks ago. Larry Kudlow, a top economic adviser to President Trump
and a critic of the \$600 payments, said this week that there was ``no
way'' Republicans would allow the benefits to expire entirely. But the
congressional outcome remains unclear.

Some economists, particularly on the right, say there are good reasons
to wind down the payments as the economy improves. But even economists
who have been critical of the extra benefits say it would be a mistake
to cut them off entirely.

``That's a lot of income to just withdraw from the economy really
suddenly,'' said Michael R. Strain, an economist at the conservative
American Enterprise Institute. ``Right now there's no question that the
positive economic effects of those payments are outweighing the negative
economic effects.''

Mr. Strain and many other economists would like to see the benefits
linked to economic conditions, ideally at the state level. That would
allow payments to shrink as local economies improve, while eliminating
the uncertainty that comes with setting a fixed end date and then
waiting to see if Congress extends it.

Progressive economists also favor linking benefits to economic
conditions. But they dismiss concerns about discouraging work when there
are millions more unemployed workers than available jobs. And they argue
that cutting benefits now would set off economic ripples that would lead
to more job losses.

``When they can't pay their rent, now it's the landlord whose business
is hurting,'' said Sharon Parrott, a senior vice president at the
progressive Center on Budget and Policy Priorities. ``Those are all
dollars that are not circulating through the economy.''

Cutting off benefits could also increase the spread of the virus by
forcing people to take jobs in which they might be exposed to it or
expose others.

``When that \$600 goes away, people who live week to week, paycheck to
paycheck, they're suddenly going to be unable to pay basic expenses and
will be desperate for work,'' said Michele Evermore, a senior policy
analyst for the National Employment Law Project.

For now, people like Mr. Perlman, who lost his job at a fitness club,
are left to wonder what comes next.

``I just want security,'' he said. ``That's what I want. I'm not looking
to profit off this. If there was a job out there, I would take it.''

Emily Cochrane contributed reporting.

Advertisement

\protect\hyperlink{after-bottom}{Continue reading the main story}

\hypertarget{site-index}{%
\subsection{Site Index}\label{site-index}}

\hypertarget{site-information-navigation}{%
\subsection{Site Information
Navigation}\label{site-information-navigation}}

\begin{itemize}
\tightlist
\item
  \href{https://help.nytimes.com/hc/en-us/articles/115014792127-Copyright-notice}{©~2020~The
  New York Times Company}
\end{itemize}

\begin{itemize}
\tightlist
\item
  \href{https://www.nytco.com/}{NYTCo}
\item
  \href{https://help.nytimes.com/hc/en-us/articles/115015385887-Contact-Us}{Contact
  Us}
\item
  \href{https://www.nytco.com/careers/}{Work with us}
\item
  \href{https://nytmediakit.com/}{Advertise}
\item
  \href{http://www.tbrandstudio.com/}{T Brand Studio}
\item
  \href{https://www.nytimes.com/privacy/cookie-policy\#how-do-i-manage-trackers}{Your
  Ad Choices}
\item
  \href{https://www.nytimes.com/privacy}{Privacy}
\item
  \href{https://help.nytimes.com/hc/en-us/articles/115014893428-Terms-of-service}{Terms
  of Service}
\item
  \href{https://help.nytimes.com/hc/en-us/articles/115014893968-Terms-of-sale}{Terms
  of Sale}
\item
  \href{https://spiderbites.nytimes.com}{Site Map}
\item
  \href{https://help.nytimes.com/hc/en-us}{Help}
\item
  \href{https://www.nytimes.com/subscription?campaignId=37WXW}{Subscriptions}
\end{itemize}
