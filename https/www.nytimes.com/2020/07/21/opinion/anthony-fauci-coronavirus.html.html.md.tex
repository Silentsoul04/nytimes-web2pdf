Sections

SEARCH

\protect\hyperlink{site-content}{Skip to
content}\protect\hyperlink{site-index}{Skip to site index}

\href{https://myaccount.nytimes.com/auth/login?response_type=cookie\&client_id=vi}{}

\href{https://www.nytimes.com/section/todayspaper}{Today's Paper}

\href{/section/opinion}{Opinion}\textbar{}I Spoke With Anthony Fauci. He
Says His Inbox Isn't Pretty.

\href{https://nyti.ms/2CDrvwy}{https://nyti.ms/2CDrvwy}

\begin{itemize}
\item
\item
\item
\item
\item
\item
\end{itemize}

Advertisement

\protect\hyperlink{after-top}{Continue reading the main story}

\href{/section/opinion}{Opinion}

Supported by

\protect\hyperlink{after-sponsor}{Continue reading the main story}

\hypertarget{i-spoke-with-anthony-fauci-he-says-his-inbox-isnt-pretty}{%
\section{I Spoke With Anthony Fauci. He Says His Inbox Isn't
Pretty.}\label{i-spoke-with-anthony-fauci-he-says-his-inbox-isnt-pretty}}

An interview with the man who has an important message for you, if he
can get it out.

\href{https://www.nytimes.com/by/jennifer-senior}{\includegraphics{https://static01.nyt.com/images/2018/10/26/opinion/jennifer-senior/jennifer-senior-thumbLarge.png}}

By \href{https://www.nytimes.com/by/jennifer-senior}{Jennifer Senior}

Opinion Columnist

\begin{itemize}
\item
  July 21, 2020
\item
  \begin{itemize}
  \item
  \item
  \item
  \item
  \item
  \item
  \end{itemize}
\end{itemize}

\includegraphics{https://static01.nyt.com/images/2020/07/21/opinion/21senior1/merlin_171455547_8ab6013e-5a7c-4899-a744-61d164f5322f-articleLarge.jpg?quality=75\&auto=webp\&disable=upscale}

Americans may have lost faith in their most cherished institutions ---
the presidency, Congress, the media, perhaps even democracy itself ---
but
\href{https://www.nytimes.com/2020/07/17/us/politics/fauci-trump-coronavirus.html}{65
percent} of them still believe in Dr. Anthony Fauci.

This, in spite of the fact that he's practically disappeared from
network and cable television while the pandemic has whipped through the
country with alarming speed (his message of sober realism does not, one
suspects, align well with the wishful thinking of his boss).

This, in spite of the fact that the Trump White House waged a highly
unusual campaign last week to undermine his credibility, with both named
and unnamed administration officials dispatched to impale him like an
hors d'oeuvre. Fauci has been the director of the National Institute of
Allergy and Infectious Diseases since 1984, and he's been the custodian
of a jittery nation's sanity since March 2020.

We had a chance to speak nine hours before the president's first
coronavirus news briefing since April. Here are edited excerpts from our
conversation.

\textbf{Are you going to be at the press briefing this afternoon?}

To be honest with you, I don't know. They haven't really said who's
going to be there. I would assume, but I don't know as a fact if I am
going to be there.

\textbf{Have you spoken with the White House about it?}

No. But that's not unlike them all of a sudden, middle of the day, to
say, ``Be down there at five o'clock.'' So I'm not too --- what's the
right word? --- \emph{surprised} that I haven't heard anything yet.

\textbf{Interesting. That means you weren't involved in the discussions
about relaunching them.}

No.

\textbf{Do you think they're a good idea?}

You know, it depends on how it goes. If they stick to public health and
don't get diverted into other types of discussions, I think it could be
productive.

\textbf{Let's get to the news. Our numbers are surging. And you've just
told The Atlantic that we've got to do a reset, which, of course, makes
perfect sense. But given the reluctance of some governors, businesses
and citizens to abide by the basic rules of social distancing and mask
wearing, is it possible to get this pandemic under control}
\emph{\textbf{without}} \textbf{a federal response?}

It would be better if things were a little more uniform. It just seems
that unfortunately, in some sectors, there's this feeling that there's
opening the country on one end of the spectrum, and public health
measures that suppress things and lock them down on the other.

They should not be opposing forces. The guidelines that we put out a
couple of months ago, those should be followed and appreciated as the
\emph{vehicle} to open the country, as opposed to the \emph{obstacle} to
opening the country.

\textbf{You said it would be nicer if some things were more uniform.
Like what?}

The fundamentals. Wear a mask. Avoid crowds. Close the bars. Bars are
the hot spots --- ---

\textbf{But Americans have already been told this, right? And we still
don't do those things. If you were an executive for the day, what lever
would you pull?}

But Jennifer, would you want me to say something that's directly
contrary to what the president is doing? That's not helpful. Then all of
a sudden you don't hear from me for a while.

\textbf{I definitely don't want anyone weaponizing anything you're
saying.}

I've just been doing this for so long, and I'm trying to do my best to
get the message across without being overtly at odds, OK? The only thing
I can do is to get out there with whatever notoriety or recognition I
have and say, these are the four or five things. Please pay attention to
them. And if we do that, I feel confident that we'll turn this around.

What I've been trying to do is appeal to the younger generation. If you
look at the age average of the new cases that are going on in the South,
it's about 10 to 15 years younger than what we previously saw.

So it's clear what's going on. Young people are saying to themselves:
``Wait a minute. I'm young, I'm healthy. The chances of my getting
seriously ill are very low. And in fact, it is about a 20 to 40 percent
likelihood that I won't have any symptoms at all. So why should I
bother?''

What they're missing is something fundamental: By getting infected
themselves --- even if they never get a symptom --- they are part of the
propagation of a pandemic. They are \emph{fueling} the pandemic. We have
to keep hammering that home, because, as much as they do that, they're
completely relinquishing their societal responsibility.

\textbf{How much faith do you have in people to pivot and change their
behaviors?}

It's disconcerting when you see people are not listening. I could show
you some of the emails and texts I get --- everybody seems to have my
cellphone number --- that are pretty hostile about what I'm doing, as if
I'm encroaching upon their individual liberties.

\textbf{Can you read me one?}

No.

\textbf{Just trying to get a glimpse into your inbox.}

It's not good.

\textbf{What do you think is the most effective way for you to
communicate? Because you're right: You can't stand out there with a
bullhorn and directly contradict the man you work for.}

I'm a pretty good communicator. I have been doing that now with multiple
outbreaks for about 40 years, dating back to the very early years of
H.I.V., I'm just going to continue to use whatever bully pulpit I have.
And, you know, just keep hacking at it.

\textbf{Are you reaching out to individual governors?}

The governors call me frequently. It's not a rare situation where
governors and senators get on the phone with me and in good faith ask,
``What do you think I should be doing? What about this? What should I do
about that?''

\textbf{Have you spoken to Gov. Brian Kemp of Georgia, who opposed a
mandate to wear masks in Atlanta?}

I haven't specifically spoken to Kemp, no.

\textbf{Has Joe Biden reached out to you? Or any of his folks?}

No. I mean I think they know better. That I'm in a sensitive position.

\textbf{Is there a time in recent American history when we as a nation
would have been better able to get this pandemic under control?}

In some respects, we are better off because of the technological
advances. I mean, 20 years ago, we never would have been able to get
candidate vaccines ready to go into Phase 3 trials \emph{literally}
within a few months of the discovery of the new virus. That is
unprecedented.

But there was a time when there was much more faith and confidence in
authority and in government. It's very, very difficult to get the
country to pull together in a real unified way. Maybe the last time that
we ever did that was 9/11.

\textbf{Is there anything about this virus, as a pathogen, that has
surprised you}?

Absolutely! You know, it's extremely unique, and I think that is one of
the reasons why there is such confusion and misunderstanding about the
seriousness of it. Of all the viruses and outbreaks that I have been
involved with over the last four decades, I have never seen a virus in
which the spectrum of seriousness is \emph{so} extreme. This disease
goes from nothing to death! So that has really surprised me.

\textbf{Is there nothing else like this in nature?}

There are extreme differences in certain diseases, but none that have
exploded into pandemic proportions.

\textbf{You've said before that there could be some kind of vaccine by
the end of the year. But at what point will most}
\emph{\textbf{families}} \textbf{be able to get a vaccination?}

I think it's going to be sometime in 2021. I don't know whether that's
going to be the first quarter of 2021, the first half --- it's difficult
to say.

\textbf{But testing still isn't up to scale, and personal protective
equipment wasn't distributed in a timely way. Given that, I fear that
there will be many snafus.}

We don't think that's going to happen, for the simple reason that the
federal government has invested \emph{billions} of dollars directly ---
\emph{directly} --- into the pharmaceutical companies that are making
the vaccine. There are never any guarantees. But I would be surprised,
given all the resources that the federal government has put into these
companies. We are counting on them for delivery.

\textbf{That is the} \emph{\textbf{one}} \textbf{way in which you're
saying there} \emph{\textbf{has}} \textbf{been a federalized response.}

Right. There certainly has.

\textbf{The president called you an alarmist in his interview with Chris
Wallace. And I just want to know: Are you?}

I characterize myself as a realist.

\emph{The Times is committed to publishing}
\href{https://www.nytimes.com/2019/01/31/opinion/letters/letters-to-editor-new-york-times-women.html}{\emph{a
diversity of letters}} \emph{to the editor. We'd like to hear what you
think about this or any of our articles. Here are some}
\href{https://help.nytimes.com/hc/en-us/articles/115014925288-How-to-submit-a-letter-to-the-editor}{\emph{tips}}\emph{.
And here's our email:}
\href{mailto:letters@nytimes.com}{\emph{letters@nytimes.com}}\emph{.}

\emph{Follow The New York Times Opinion section on}
\href{https://www.facebook.com/nytopinion}{\emph{Facebook}}\emph{,}
\href{http://twitter.com/NYTOpinion}{\emph{Twitter (@NYTopinion)}}
\emph{and}
\href{https://www.instagram.com/nytopinion/}{\emph{Instagram}}\emph{.}

Advertisement

\protect\hyperlink{after-bottom}{Continue reading the main story}

\hypertarget{site-index}{%
\subsection{Site Index}\label{site-index}}

\hypertarget{site-information-navigation}{%
\subsection{Site Information
Navigation}\label{site-information-navigation}}

\begin{itemize}
\tightlist
\item
  \href{https://help.nytimes.com/hc/en-us/articles/115014792127-Copyright-notice}{©~2020~The
  New York Times Company}
\end{itemize}

\begin{itemize}
\tightlist
\item
  \href{https://www.nytco.com/}{NYTCo}
\item
  \href{https://help.nytimes.com/hc/en-us/articles/115015385887-Contact-Us}{Contact
  Us}
\item
  \href{https://www.nytco.com/careers/}{Work with us}
\item
  \href{https://nytmediakit.com/}{Advertise}
\item
  \href{http://www.tbrandstudio.com/}{T Brand Studio}
\item
  \href{https://www.nytimes.com/privacy/cookie-policy\#how-do-i-manage-trackers}{Your
  Ad Choices}
\item
  \href{https://www.nytimes.com/privacy}{Privacy}
\item
  \href{https://help.nytimes.com/hc/en-us/articles/115014893428-Terms-of-service}{Terms
  of Service}
\item
  \href{https://help.nytimes.com/hc/en-us/articles/115014893968-Terms-of-sale}{Terms
  of Sale}
\item
  \href{https://spiderbites.nytimes.com}{Site Map}
\item
  \href{https://help.nytimes.com/hc/en-us}{Help}
\item
  \href{https://www.nytimes.com/subscription?campaignId=37WXW}{Subscriptions}
\end{itemize}
