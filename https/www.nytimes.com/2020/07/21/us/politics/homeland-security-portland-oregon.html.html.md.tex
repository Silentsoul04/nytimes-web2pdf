Sections

SEARCH

\protect\hyperlink{site-content}{Skip to
content}\protect\hyperlink{site-index}{Skip to site index}

\href{https://www.nytimes.com/section/politics}{Politics}

\href{https://myaccount.nytimes.com/auth/login?response_type=cookie\&client_id=vi}{}

\href{https://www.nytimes.com/section/todayspaper}{Today's Paper}

\href{/section/politics}{Politics}\textbar{}Homeland Security Leaders on
Defensive Amid Calls to Withdraw From Portland

\url{https://nyti.ms/2WZMVv9}

\begin{itemize}
\item
\item
\item
\item
\item
\end{itemize}

Advertisement

\protect\hyperlink{after-top}{Continue reading the main story}

Supported by

\protect\hyperlink{after-sponsor}{Continue reading the main story}

\hypertarget{homeland-security-leaders-on-defensive-amid-calls-to-withdraw-from-portland}{%
\section{Homeland Security Leaders on Defensive Amid Calls to Withdraw
From
Portland}\label{homeland-security-leaders-on-defensive-amid-calls-to-withdraw-from-portland}}

The department's acting secretary met with reporters and blamed local
Oregon officials amid criticism that he has gone too far to suppress
protests in Portland.

\includegraphics{https://static01.nyt.com/images/2020/07/21/us/politics/21dc-unrest-feds-sub/merlin_174809151_ce370a99-3901-46fe-83c0-d6953101a088-articleLarge.jpg?quality=75\&auto=webp\&disable=upscale}

\href{https://www.nytimes.com/by/zolan-kanno-youngs}{\includegraphics{https://static01.nyt.com/images/2019/12/13/reader-center/author-zolan-kanno-youngs/author-zolan-kanno-youngs-thumbLarge.png}}

By \href{https://www.nytimes.com/by/zolan-kanno-youngs}{Zolan
Kanno-Youngs}

\begin{itemize}
\item
  July 21, 2020
\item
  \begin{itemize}
  \item
  \item
  \item
  \item
  \item
  \end{itemize}
\end{itemize}

WASHINGTON --- Senior officials with the Department of Homeland Security
addressed the increased presence of federal agents in Portland, Ore., in
a press briefing for the first time on Tuesday, defending the tactics of
the agents who have been widely criticized for escalating an already
tense conflict with protesters.

They said agents of the department would remain in the city until the
unrest had subsided.

Chad Wolf, the acting secretary of homeland security, cast blame for the
unrest on Portland politicians who have publicly pleaded that he remove
the agents from the city. But Mr. Wolf said the crackdown --- which has
included personnel from the U.S. Marshals and tactical agents from
Customs and Border Protection and Immigration and Customs Enforcement in
addition to the Federal Protective Service, which was already stationed
in Portland --- was specific to the Pacific Northwest city, distancing
his department from President Trump's commitment this week to send
agents to other major cities, from Oakland to New York.

``Violent anarchists in Portland versus normal city criminal activity
behavior by gangs and criminal element, those are two different
things,'' Mr. Wolf said, adding that the department had recorded 43
arrests in the protests. ``What we have in Portland is very different
than what we see in other cities.''

The Trump administration's plan,
\href{https://www.nytimes.com/2020/07/20/us/politics/trump-chicago-portland-federal-agents.html}{revealed
on Monday}, to send 150 Homeland Security Investigations special agents
to Chicago for 60 days is separate from the deployment of
camouflage-wearing tactical agents in Portland, but the deployment of
federal agents to another major city has stoked concern among local
mayors and governors that the efforts are making the unrest worse.

Mayors throughout the United States have called on the administration to
pull back the agents, and even Tom Ridge, the first homeland security
secretary, criticized the deployment on Tuesday.

``It would be a cold day in hell before I would consent to a unilateral,
uninvited intervention into one of my cities,'' Mr. Ridge said in an
interview with Sirius XM radio. ``And I wish the president would take a
more collaborative approach toward fighting this lawlessness than the
unilateral approach he's taken.''

In the rare news conference, Mr. Wolf said he called the mayor of
Portland and the governor of Oregon this month to work with them to
protect the federal courthouse downtown but was met with resistance. He
accused the officials of turning the conversation ``into a political
issue.''

``We stand ready,'' Mr. Wolf said. ``I'm ready to pull my officers out
of there if the violence stops. Portland is unique. There's no other
city like it right now where we see this violence at federal
courthouses.''

But while the homeland security officials said the **** deployment of
tactical agents who have frequently deployed tear gas and at times
forced protesters into unmarked vehicles **** was needed to combat
``violent criminals,'' some of the demonstrators included mothers locked
in arms outside the courthouse. While some in the crowd have thrown
rocks and bottles at federal officers, others have demonstrated
peacefully.

The governor, the mayor and the protesters have all said that the
homeland security agents and U.S. Marshals had only increased tensions
in the city.

``We didn't ask for these troops in our city. We don't want these troops
in our city, and the tactics they're using are very un-American,'' Mayor
Ted Wheeler said, adding that the agents were forcing demonstrators into
vans without probable cause. ``There's some really serious
constitutional issues here.''

Mr. Wheeler added that many of those detained had not been charged, but
rather released after questioning. ``We have people who have come back
and said I feel like I was kidnapped.''

While the department deployed teams of air marshals, Coast Guard
officials, and tactical agents from Customs and Border Protection and
ICE to various cities after Mr. Trump signed an executive order to
protect monuments, statues and federal property, Mr. Wolf said the teams
are in no city besides Portland at this time. Those teams continue to be
ready for deployment.

Citing a law codified by the Homeland Security Act of 2002 that allows
the secretary to protect federal property, Mr. Wolf also defended agents
who have been accused of placing protesters in unmarked vans without
telling them where they are going.

But the law Mr. Wolf cited, 40 U.S. Code 1315, says homeland security
officials have the right to ``conduct investigations'' away from federal
property. Pressed about the level of probable cause needed to detain
someone away from the courthouse in Portland, Mr. Wolf referred to
Richard Cline, the deputy director of the Federal Protective Service.
Mr. Cline described the detaining of one individual, whom he did not
name, who was put into a van so agents could bring him to a safe place
for questioning.

The officials did not address other accounts from demonstrators of being
detained, put into unmarked vehicles and not being told where they were
going.

``We're not going to allow somebody to walk up to federal property,
assault a federal officer or agent and because they walk off federal
property say we can't go arrest you,'' said Mark Morgan, the acting
commissioner of Customs and Border Protection, who confirmed that
federal agents were using unmarked vehicles but said it was needed to
ensure their safety. **** He also carried with him a ballistic
camouflaged vest displaying the label ``POLICE'' to push back on
accounts that agents in Portland lacked insignia and refused to identify
themselves.

But Defense Secretary Mark T. Esper has told other administration
officials that he has concerns about the military-style camouflage worn
by such agents in recent weeks.

Mr. Esper ``has expressed a concern of this within the administration,''
the chief Pentagon spokesman, Jonathan Hoffman, told reporters on
Tuesday. ``We want a system where people can tell the difference''
between the federal agents who are patrolling streets and military
troops who are not, he added.

Gil Kerlikowske, a Customs and Border Protection chief in the Obama
administration, also said the department was not meeting a standard of
probable cause with the detainments.

``They need the same probable cause that any police officer should have
to stop somebody. It's beyond a reasonable suspicion that this person
has actually committed crime,'' Mr. Kerlikowske said. ``You're not
seeing that in Portland.''

Thomas Gibbons-Neff contributed reporting.

Advertisement

\protect\hyperlink{after-bottom}{Continue reading the main story}

\hypertarget{site-index}{%
\subsection{Site Index}\label{site-index}}

\hypertarget{site-information-navigation}{%
\subsection{Site Information
Navigation}\label{site-information-navigation}}

\begin{itemize}
\tightlist
\item
  \href{https://help.nytimes.com/hc/en-us/articles/115014792127-Copyright-notice}{©~2020~The
  New York Times Company}
\end{itemize}

\begin{itemize}
\tightlist
\item
  \href{https://www.nytco.com/}{NYTCo}
\item
  \href{https://help.nytimes.com/hc/en-us/articles/115015385887-Contact-Us}{Contact
  Us}
\item
  \href{https://www.nytco.com/careers/}{Work with us}
\item
  \href{https://nytmediakit.com/}{Advertise}
\item
  \href{http://www.tbrandstudio.com/}{T Brand Studio}
\item
  \href{https://www.nytimes.com/privacy/cookie-policy\#how-do-i-manage-trackers}{Your
  Ad Choices}
\item
  \href{https://www.nytimes.com/privacy}{Privacy}
\item
  \href{https://help.nytimes.com/hc/en-us/articles/115014893428-Terms-of-service}{Terms
  of Service}
\item
  \href{https://help.nytimes.com/hc/en-us/articles/115014893968-Terms-of-sale}{Terms
  of Sale}
\item
  \href{https://spiderbites.nytimes.com}{Site Map}
\item
  \href{https://help.nytimes.com/hc/en-us}{Help}
\item
  \href{https://www.nytimes.com/subscription?campaignId=37WXW}{Subscriptions}
\end{itemize}
