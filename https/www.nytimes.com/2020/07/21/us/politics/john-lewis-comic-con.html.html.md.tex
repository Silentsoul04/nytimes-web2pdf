Sections

SEARCH

\protect\hyperlink{site-content}{Skip to
content}\protect\hyperlink{site-index}{Skip to site index}

\href{https://www.nytimes.com/section/politics}{Politics}

\href{https://myaccount.nytimes.com/auth/login?response_type=cookie\&client_id=vi}{}

\href{https://www.nytimes.com/section/todayspaper}{Today's Paper}

\href{/section/politics}{Politics}\textbar{}When John Lewis Cosplayed at
Comic-Con as His Younger Self

\url{https://nyti.ms/2ZNk9iO}

\begin{itemize}
\item
\item
\item
\item
\item
\end{itemize}

Advertisement

\protect\hyperlink{after-top}{Continue reading the main story}

Supported by

\protect\hyperlink{after-sponsor}{Continue reading the main story}

\hypertarget{when-john-lewis-cosplayed-at-comic-con-as-his-younger-self}{%
\section{When John Lewis Cosplayed at Comic-Con as His Younger
Self}\label{when-john-lewis-cosplayed-at-comic-con-as-his-younger-self}}

For several years, Mr. Lewis would lead a group of children in a march
across the San Diego Convention Center.

\includegraphics{https://static01.nyt.com/images/2020/07/21/multimedia/21xp-lewis-comiccon/merlin_174708567_7454d5a5-de4c-432f-a826-5d8fd7482eab-articleLarge.jpg?quality=75\&auto=webp\&disable=upscale}

\href{https://www.nytimes.com/by/sandra-e-garcia}{\includegraphics{https://static01.nyt.com/images/2020/07/10/reader-center/author-sandra-e-garcia/author-sandra-e-garcia-thumbLarge.png}}

By \href{https://www.nytimes.com/by/sandra-e-garcia}{Sandra E. Garcia}

\begin{itemize}
\item
  Published July 21, 2020Updated July 25, 2020
\item
  \begin{itemize}
  \item
  \item
  \item
  \item
  \item
  \end{itemize}
\end{itemize}

\href{https://www.nytimes.com/2020/07/25/us/john-lewis-memorial-service.html}{Representative
John Lewis} was looking for an orange to carry in his bag. He wanted to
pack two books, a toothbrush, toothpaste, an apple and an orange inside
his backpack, just like he did in 1965, when he led a vanguard of close
to 600 people across the
\href{http://www.nytimes.com/2020/07/18/us/politics/edmund-pettus-bridge-renamed-john-lewis.html}{Edmund
Pettus Bridge}.

\href{https://www.nytimes.com/2020/07/25/us/photos-john-lewis-memorial.html}{Mr.
Lewis} had already acquired a jacket and backpack that were similar to
the ones he wore half a century earlier on the march from Selma to
Montgomery, Ala., though it took him months and trips to several thrift
shops to find them. The orange was the only missing piece to complete
his costume of himself at the 2015 Comic-Con International in San Diego,
an aide, Andrew Aydin, 36, recalled.

``He went full re-creation,'' Mr. Aydin, who was Mr. Lewis's policy
adviser and digital director, said in an interview.

Photos of Mr. Lewis walking across the convention center, cosplaying as
his younger self --- with the same determined expression he sported in
Selma when he was 25 --- began to circulate on social media after the
congressman
\href{https://www.nytimes.com/2020/07/17/us/john-lewis-dead.html}{died
on Friday}. ``John Lewis was a giant and a moral compass,'' said Senator
Elizabeth Warren, one of tens of thousands of people to share a tweet of
the images from that day.

Mr. Lewis was at Comic-Con that day to promote
``March,''\href{https://www.nytimes.com/2016/11/27/books/review/john-lewis-march.html}{a
three-part graphic novel memoir}he wrote with Mr. Aydin and the artist
Nate Powell. The second book in the trilogy had been released a few
months earlier, and Mr. Lewis was in San Diego to promote it. He would
return again in 2016 and
\href{https://apnews.com/053a3111dd354e1c8522cdc03bc4b755/Civil-rights-icon-leads-march-through-California-Comic-Con}{2017},
and recreated the march both times.

His goal that day in 2015 was to help the children understand that you
don't need super powers to be a hero, Mr. Aydin said.

``He was trying to show them how his faith and his belief in America
fundamentally put him in a position where they would look at him as a
hero,'' Mr. Aydin said.

The congressman said that it was ``another children's march, just like
they called it in Alabama,'' Mr. Aydin said.

Mr. Lewis walked the half-mile distance from a panel room to his booth
hand-in-hand with the children, and others joined in as he walked by. By
the time Mr. Lewis took notice, there were close to 1,000 people
following him, Mr. Aydin said.

```This is almost too much,''' Mr. Aydin remembered Mr. Lewis saying.
``It's his way of saying this is something really extraordinary.''

In a 2015 interview with
\href{https://www.cbsnews.com/news/rep-john-lewis-called-real-life-superhero-fans-inspired-march-comic-books/}{CBS
News}, Mr. Lewis called the moment ``unreal.''

``I walked with little children, wonderful little children. We marched
onto the floor of the convention center. And it was unreal,
unbelievable. And this throng of people just walkin' with us,'' he said.

Image

All three volumes of the ``March'' graphic novels.Credit...Carlos
Gonzalez for The New York Times

Image

Mr. Aydin helped Mr. Lewis put the finishing touches on his
costume.Credit...Carlos Gonzalez for The New York Times

The children were third-graders from Oak Park Elementary in San Diego,
and they were learning about the civil rights movement from Mr. Lewis's
book. Their teacher, Mick Rabin, an avid comic book fan, used it to
teach his students about other figures from the movement. But at
Comic-Con, the students only had eyes for the congressman, even as
people dressed up as Spider-Man and Wonder Woman walked by, Mr. Rabin
said.

``The kids went bonkers,'' Mr. Rabin said. ``They knew who he was
because they had read it in the comic, and he was wearing the same
thing.''

The idea for the book had started nearly a decade earlier, in 2008. When
Mr. Aydin, a lifelong comic book fan, admitted to the office staff that
he was going to Comic-Con after working arduous hours during Mr. Lewis's
re-election campaign, his colleagues laughed --- but not Mr. Lewis.

```Don't laugh,''' Mr. Aydin recalled him saying to the staff, before
reminding them of a 1957 comic book that was popular in the civil rights
movement, ``Martin Luther King and The Montgomery Story.''

That conversation eventually led to the ``March'' trilogy.

``We would stay up and I would interview him,'' Mr. Aydin said about
their writing process. ``I'd ask him questions and he'd fall asleep.''

The first of the series,
``\href{https://www.penguinrandomhouse.com/books/560278/march-book-one-oversized-edition-by-john-lewis/}{March:
Book One},'' published in 2013, became a
\href{https://www.nytimes.com/2016/11/27/books/review/john-lewis-march.html}{New
York Times best seller}. Mr. Lewis cried when he learned that, Mr. Aydin
said. The second, in 2015, won an Eisner Award at Comic-Con, and the
third, published in 2016, won the National Book Award for young people's
literature.

In a 2013 appearance on
``\href{http://www.cc.com/video-clips/ocqoae/the-colbert-report-john-lewis-pt--2}{The
Colbert Report},'' Mr. Lewis talked about how inspired he was by Martin
Luther King's comic book.

``I read it and I reread it, and this book inspired me,'' Mr. Lewis
said. ``He became my hero, my inspiration, my leader. He inspired me to
say no to segregation and racial discrimination.''

Mr. Rabin said Mr. Lewis had a similar effect on his students in 2015.

``My students who walked with him that day were transformed forever,''
Mr. Rabin said. ``They were truly utterly transfixed by that
interaction.''

Advertisement

\protect\hyperlink{after-bottom}{Continue reading the main story}

\hypertarget{site-index}{%
\subsection{Site Index}\label{site-index}}

\hypertarget{site-information-navigation}{%
\subsection{Site Information
Navigation}\label{site-information-navigation}}

\begin{itemize}
\tightlist
\item
  \href{https://help.nytimes.com/hc/en-us/articles/115014792127-Copyright-notice}{©~2020~The
  New York Times Company}
\end{itemize}

\begin{itemize}
\tightlist
\item
  \href{https://www.nytco.com/}{NYTCo}
\item
  \href{https://help.nytimes.com/hc/en-us/articles/115015385887-Contact-Us}{Contact
  Us}
\item
  \href{https://www.nytco.com/careers/}{Work with us}
\item
  \href{https://nytmediakit.com/}{Advertise}
\item
  \href{http://www.tbrandstudio.com/}{T Brand Studio}
\item
  \href{https://www.nytimes.com/privacy/cookie-policy\#how-do-i-manage-trackers}{Your
  Ad Choices}
\item
  \href{https://www.nytimes.com/privacy}{Privacy}
\item
  \href{https://help.nytimes.com/hc/en-us/articles/115014893428-Terms-of-service}{Terms
  of Service}
\item
  \href{https://help.nytimes.com/hc/en-us/articles/115014893968-Terms-of-sale}{Terms
  of Sale}
\item
  \href{https://spiderbites.nytimes.com}{Site Map}
\item
  \href{https://help.nytimes.com/hc/en-us}{Help}
\item
  \href{https://www.nytimes.com/subscription?campaignId=37WXW}{Subscriptions}
\end{itemize}
