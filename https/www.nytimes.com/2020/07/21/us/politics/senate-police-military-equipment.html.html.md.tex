Sections

SEARCH

\protect\hyperlink{site-content}{Skip to
content}\protect\hyperlink{site-index}{Skip to site index}

\href{https://www.nytimes.com/section/politics}{Politics}

\href{https://myaccount.nytimes.com/auth/login?response_type=cookie\&client_id=vi}{}

\href{https://www.nytimes.com/section/todayspaper}{Today's Paper}

\href{/section/politics}{Politics}\textbar{}Senate Kills Broad Curbs on
Military Gear for Police, Thwarting Push to Demilitarize

\url{https://nyti.ms/2CqiB5O}

\begin{itemize}
\item
\item
\item
\item
\item
\end{itemize}

\href{https://www.nytimes.com/news-event/george-floyd-protests-minneapolis-new-york-los-angeles?action=click\&pgtype=Article\&state=default\&region=TOP_BANNER\&context=storylines_menu}{Race
and America}

\begin{itemize}
\tightlist
\item
  \href{https://www.nytimes.com/2020/07/26/us/protests-portland-seattle-trump.html?action=click\&pgtype=Article\&state=default\&region=TOP_BANNER\&context=storylines_menu}{Protesters
  Return to Other Cities}
\item
  \href{https://www.nytimes.com/2020/07/24/us/portland-oregon-protests-white-race.html?action=click\&pgtype=Article\&state=default\&region=TOP_BANNER\&context=storylines_menu}{Portland
  at the Center}
\item
  \href{https://www.nytimes.com/2020/07/23/podcasts/the-daily/portland-protests.html?action=click\&pgtype=Article\&state=default\&region=TOP_BANNER\&context=storylines_menu}{Podcast:
  Showdown in Portland}
\item
  \href{https://www.nytimes.com/interactive/2020/07/16/us/black-lives-matter-protests-louisville-breonna-taylor.html?action=click\&pgtype=Article\&state=default\&region=TOP_BANNER\&context=storylines_menu}{45
  Days in Louisville}
\end{itemize}

Advertisement

\protect\hyperlink{after-top}{Continue reading the main story}

Supported by

\protect\hyperlink{after-sponsor}{Continue reading the main story}

\hypertarget{senate-kills-broad-curbs-on-military-gear-for-police-thwarting-push-to-demilitarize}{%
\section{Senate Kills Broad Curbs on Military Gear for Police, Thwarting
Push to
Demilitarize}\label{senate-kills-broad-curbs-on-military-gear-for-police-thwarting-push-to-demilitarize}}

The Senate adopted a narrower proposal to limit the transfer of some
military equipment to local police departments, but data shows that such
restrictions have done little to reduce the flow.

\includegraphics{https://static01.nyt.com/images/2020/07/01/us/politics/01dc-unrest-equipment/merlin_173012736_7c4acbb9-efb4-4f6b-931a-b4947a61311f-articleLarge.jpg?quality=75\&auto=webp\&disable=upscale}

\href{https://www.nytimes.com/by/catie-edmondson}{\includegraphics{https://static01.nyt.com/images/2019/11/20/us/politics/catie-edmonson-twitter-chatblog/catie-edmonson-twitter-chatblog-thumbLarge.png}}

By \href{https://www.nytimes.com/by/catie-edmondson}{Catie Edmondson}

\begin{itemize}
\item
  July 21, 2020
\item
  \begin{itemize}
  \item
  \item
  \item
  \item
  \item
  \end{itemize}
\end{itemize}

WASHINGTON --- The Senate on Tuesday rejected a bipartisan bid to bar
the Pentagon from transferring a wide range of military-grade weaponry
to local police departments, effectively killing the last remaining
initiative before Congress this year to address the excessive use of
force in law enforcement.

With policing overhaul legislation stalled on Capitol Hill, the measure,
which lawmakers sought to attach to the must-pass annual defense bill,
was a last-ditch attempt to begin to demilitarize law enforcement after
a nationwide uproar to address racial discrimination and distrust
between the police and the communities they serve.

But despite the outcry in favor of sweeping changes, lawmakers declined
to place limitations on some of the most controversial military-grade
equipment provided to local police departments, rejecting a proposal by
Senator Brian Schatz, Democrat of Hawaii, to prohibit such items as tear
gas, grenades and bayonets.

The vote, 51 to 44, which failed to reach the required 60-vote threshold
to pass, underscored how fraught and often fruitless attempts to rein in
the program have become, allowing such matériel to flow to law
enforcement in America's cities and towns with few restrictions.

The Senate did approve a measure that would reinstate some restrictions
originally imposed by the Obama administration and rolled back by
President Trump. That amendment, led by Senator James M. Inhofe,
Republican of Oklahoma and the chairman of the Armed Services Committee,
would prohibit the Pentagon from supplying law enforcement with tracked
combat vehicles, drones that carry weaponry like tear gas and rubber
bullets, and other equipment that the Defense Department has said it
does not currently provide to local police departments. It would also
require agencies that receive the equipment to undergo de-escalation
training.

But the limits are unlikely to decrease the amount of military equipment
that goes to police departments around the country or materially
constrain the type of weapons made available to them. An analysis by The
New York Times shows that despite President Barack Obama's efforts to
rein in the program after the killing of an unarmed Black man by the
police in Ferguson, Mo., in 2014, the restrictions did little to reduce
the amount of weaponry available to local police departments through the
program, known as 1033. Nor did Mr. Trump's move to unravel Mr. Obama's
policies make a significant difference.

``Trump came in and said, `I have undone all the reforms,' which in the
first place hadn't done anything, anyway,'' said Peter Kraska, a
professor at Eastern Kentucky University who has studied police
militarization for decades. ``There's just been this whole political
game done.''

``Consequently,'' he added, ``the spigot has stayed on even
post-Ferguson.''

\hypertarget{military-equipment-transferred-to-police-departments}{%
\subsection{Military Equipment Transferred to Police
Departments}\label{military-equipment-transferred-to-police-departments}}

The Pentagon has supplied state and local law enforcement with military
gear during both the Obama and Trump presidencies. Some gear, deemed
inappropriate for police use, was banned in 2015 by an executive order,
which President Trump revoked in 2017.

Obama

Trump

Banned Items

10,000 items

At least 137

distributed

under Trump

5,000

More than

1,800

returned

under

Obama

2008

2010

2012

2014

2016

2018

2020

10,000 items

5,000

2010

2012

2014

2016

Obama

Trump

2018

2020

Banned Items

More than 1,800 returned under

Obama

At least 137 distributed under

Trump

Obama

Trump

Banned Items

10,000 items

At least 137

distributed

under Trump

5,000

More than

1,800

returned

under

Obama

2010

2012

2014

2016

2018

2020

Note: Items shown are aircraft, weapons, armored vehicles, camouflage,
night vision pieces, sights, armor and weapon training equipment, as
well as parts and accessories for these items. Other equipment dispersed
by the Defense Department's 1033 program is not included. Banned items
included bayonets and grenade launchers.

Source: Defense Logistics Agency

By Eleanor Lutz

Lawmakers in both parties, led by Mr. Schatz, announced their intention
to restrict the program last month, after officers wearing riot gear
were documented in cities across the country using pepper spray and
rubber bullets on demonstrators protesting the killings of unarmed Black
Americans by the police,
\href{https://www.nytimes.com/2020/05/31/us/police-tactics-floyd-protests.html}{often
without warning or seemingly unprovoked}.

``The last month has made clear that weapons of war don't belong in
police departments,'' Mr. Schatz said. ``Our communities are not
battlefields. The American people are not enemy combatants.''

Mr. Inhofe's narrower measure, approved on Tuesday in a 90-to-10 vote,
was an attempt to head off Mr. Schatz's more sweeping restrictions. Mr.
Inhofe argued that the program was an ``effective use of taxpayers'
money,'' but cast his amendment as ``strong oversight of the program.''

``We want to make sure that the wrong kind of equipment doesn't get in
the hands of people who cannot properly use it,'' Mr. Inhofe said.

The program, created by Congress in the early 1990s to offload surplus
military equipment to local law enforcement to fight the war on drugs,
has furnished over \$7.4 billion worth of supplies to police departments
--- mostly mundane items like coffee makers and socks, but also assault
rifles and heavily armored trucks. Proponents argue that the program
gives underfunded police departments access to crucial equipment to
protect their officers that they would not otherwise be able to afford,
and police unions for years have feverishly lobbied against attempts to
curtail it.

It is just one of many federal initiatives that help police departments
obtain weaponry and other equipment, but the program has singularly
captured the attention of lawmakers.

``It just speaks so loudly to a direct causal connection between the
U.S. military and the police,'' said Mr. Kraska, who advised the Obama
administration on the program. ``The U.S. military is sending its war
discards from Afghanistan and Iraq, and bringing them to the streets of
America.''

There has historically been little appetite in either party to legislate
significant changes to the program. Even House Democrats, who included a
measure targeting it in their policing overhaul bill, declined to allow
a vote on adding such language, proposed by Representative Hank Johnson
of Georgia, to their version of the defense bill.

Mr. Johnson has tried without success to place restrictions on the
program since 2013, after he marched in a Christmas parade in his
district and was shocked to see the town's mayor riding in a
military-grade utility vehicle ahead of him. On the heels of the
Ferguson protests, Mr. Johnson said, he hoped his colleagues would seize
the moment and back his bipartisan bill, which would require local
governments to sign off on the equipment before a police department
tried to obtain it. But he found little support.

``I don't think what people saw in Ferguson was a wake-up call,'' Mr.
Johnson said in an interview. ``That picture of police officers with
riot gear and helmets on and assault weapons and military vehicles --- I
don't think people took note of it. I think they just assumed, that's
the way policing is in America.''

Changes to the program instead have largely been mandated by
presidential orders. Struck by images of heavily armed police officers
in armored vehicles confronting unarmed protesters in Ferguson, Mr.
Obama signed an executive order in 2015 prohibiting the transfer of
certain weapons and equipment, including tracked armored vehicles,
bayonets, grenade launchers and camouflage uniforms. Even then,
administration officials resisted more expansive changes, arguing that
the program helped bulk up law enforcement's counterterrorism efforts.
Even so, police unions condemned Mr. Obama's restrictions as a threat to
officer safety.

When Mr. Trump took office he rolled back the curbs,
\href{https://fop.net/CmsDocument/Doc/TrumpFirst100Days.pdf}{fulfilling
a campaign promise} he had made to the Fraternal Order of Police,
\href{https://www.nytimes.com/2020/06/25/us/politics/police-reforms-congress.html}{a
powerful national law enforcement union} that for years had lobbied
against restrictions to the program. Jeff Sessions, then the attorney
general,
\href{https://www.nytimes.com/2017/08/28/us/politics/trump-police-military-surplus-equipment.html?action=click\&module=RelatedLinks\&pgtype=Article}{announced
the move at the union's headquarters} in Nashville. Mr. Trump heralded
it as a significant change to military policy.

``You know, when you wanted to take over and you used military equipment
--- and they were saying you couldn't do it --- you know what I said?
That was my first day: `You can do it,''' Mr. Trump told law enforcement
officers
\href{https://www.nytimes.com/2017/07/28/us/politics/trump-immigration-gang-violence-long-island.html}{in
a 2017 speech}. ``In fact, that stuff is disappearing so fast, we have
none left.''

Some equipment that was banned by the Obama administration has since
made it into the hands of local police officers following Mr. Trump's
rollback. The Cypress-Fairbanks police department in Texas serving a
K-12 school district, for example, obtained 60 bayonet knives through
the program in 2019, according to a Pentagon database. A spokeswoman for
the school said in a statement to The Times that the bayonets ``did not
have functionality and are scheduled to be returned to the military.''

But \href{https://www.rand.org/pubs/research_reports/RR2464.html}{a RAND
Corporation study} found in 2018 that the defense and state officials
running the program ``reported little change in operations or in the
equipment'' that police departments obtained from the program as a
result of the executive order. And Pentagon officials overseeing it
groused that many of the items that the Obama administration prohibited,
like grenade launchers,
\href{https://www.dla.mil/DispositionServices/Offers/Reutilization/LawEnforcement/ProgramFAQs.aspx\#q10}{had
not been distributed through the program} for years, anyway.

A spokeswoman for the Pentagon agency that oversees the program said Mr.
Trump's executive order had ``minimal impact'' on the program's
management, as did Mr. Obama's, adding that many of the 2015 executive
order requirements ``already existed or were codified.''

Advertisement

\protect\hyperlink{after-bottom}{Continue reading the main story}

\hypertarget{site-index}{%
\subsection{Site Index}\label{site-index}}

\hypertarget{site-information-navigation}{%
\subsection{Site Information
Navigation}\label{site-information-navigation}}

\begin{itemize}
\tightlist
\item
  \href{https://help.nytimes.com/hc/en-us/articles/115014792127-Copyright-notice}{©~2020~The
  New York Times Company}
\end{itemize}

\begin{itemize}
\tightlist
\item
  \href{https://www.nytco.com/}{NYTCo}
\item
  \href{https://help.nytimes.com/hc/en-us/articles/115015385887-Contact-Us}{Contact
  Us}
\item
  \href{https://www.nytco.com/careers/}{Work with us}
\item
  \href{https://nytmediakit.com/}{Advertise}
\item
  \href{http://www.tbrandstudio.com/}{T Brand Studio}
\item
  \href{https://www.nytimes.com/privacy/cookie-policy\#how-do-i-manage-trackers}{Your
  Ad Choices}
\item
  \href{https://www.nytimes.com/privacy}{Privacy}
\item
  \href{https://help.nytimes.com/hc/en-us/articles/115014893428-Terms-of-service}{Terms
  of Service}
\item
  \href{https://help.nytimes.com/hc/en-us/articles/115014893968-Terms-of-sale}{Terms
  of Sale}
\item
  \href{https://spiderbites.nytimes.com}{Site Map}
\item
  \href{https://help.nytimes.com/hc/en-us}{Help}
\item
  \href{https://www.nytimes.com/subscription?campaignId=37WXW}{Subscriptions}
\end{itemize}
