Sections

SEARCH

\protect\hyperlink{site-content}{Skip to
content}\protect\hyperlink{site-index}{Skip to site index}

\href{https://www.nytimes.com/section/politics}{Politics}

\href{https://myaccount.nytimes.com/auth/login?response_type=cookie\&client_id=vi}{}

\href{https://www.nytimes.com/section/todayspaper}{Today's Paper}

\href{/section/politics}{Politics}\textbar{}As Trump Pushes Into
Portland, His Campaign Ads Turn Darker

\href{https://nyti.ms/3hgvGxc}{https://nyti.ms/3hgvGxc}

\begin{itemize}
\item
\item
\item
\item
\item
\item
\end{itemize}

\begin{itemize}
\item
  \href{https://www.nytimes.com/2020/08/07/us/elections/biden-vs-trump.html?action=click\&pgtype=Article\&state=default\&region=TOP_BANNER\&context=storylines_menu}{Election
  Updates}
\item
  \href{https://www.nytimes.com/interactive/2020/08/06/us/elections/results-tennessee-primary-elections.html?action=click\&pgtype=Article\&state=default\&region=TOP_BANNER\&context=storylines_menu}{Tennessee
  Results}
\item
  \href{https://www.nytimes.com/article/biden-vice-president-2020.html?action=click\&pgtype=Article\&state=default\&region=TOP_BANNER\&context=storylines_menu}{Biden's
  V.P. Search}
\item
  \href{https://www.nytimes.com/interactive/2019/us/politics/2020-presidential-candidates.html?action=click\&pgtype=Article\&state=default\&region=TOP_BANNER\&context=storylines_menu}{The
  Candidates}
\item
  \href{https://www.nytimes.com/newsletters/politics?action=click\&pgtype=Article\&state=default\&region=TOP_BANNER\&context=storylines_menu}{Politics
  Newsletter}
\end{itemize}

Advertisement

\protect\hyperlink{after-top}{Continue reading the main story}

Supported by

\protect\hyperlink{after-sponsor}{Continue reading the main story}

\hypertarget{as-trump-pushes-into-portland-his-campaign-ads-turn-darker}{%
\section{As Trump Pushes Into Portland, His Campaign Ads Turn
Darker}\label{as-trump-pushes-into-portland-his-campaign-ads-turn-darker}}

The Trump campaign is spending millions on ads that promote a dark and
exaggerated portrayal of Democratic-led cities, a tactic that reinforces
his ``law and order'' campaign message.

\includegraphics{https://static01.nyt.com/images/2020/07/21/us/politics/21trump-ads1/merlin_174794250_8fc7f37e-e2aa-48a9-8341-0b37a7a0fda1-articleLarge.jpg?quality=75\&auto=webp\&disable=upscale}

\href{https://www.nytimes.com/by/maggie-haberman}{\includegraphics{https://static01.nyt.com/images/2018/07/12/multimedia/author-maggie-haberman/author-maggie-haberman-thumbLarge.png}}\href{https://www.nytimes.com/by/nick-corasaniti}{\includegraphics{https://static01.nyt.com/images/2018/06/13/multimedia/author-nick-corasaniti/author-nick-corasaniti-thumbLarge-v2.png}}\href{https://www.nytimes.com/by/annie-karni}{\includegraphics{https://static01.nyt.com/images/2019/02/05/multimedia/author-annie-karni/author-annie-karni-thumbLarge.png}}

By \href{https://www.nytimes.com/by/maggie-haberman}{Maggie Haberman},
\href{https://www.nytimes.com/by/nick-corasaniti}{Nick Corasaniti} and
\href{https://www.nytimes.com/by/annie-karni}{Annie Karni}

\begin{itemize}
\item
  Published July 21, 2020Updated Aug. 7, 2020
\item
  \begin{itemize}
  \item
  \item
  \item
  \item
  \item
  \item
  \end{itemize}
\end{itemize}

\emph{Follow our latest coverage of}
\href{https://www.nytimes.com/2020/08/07/us/elections/biden-vs-trump.html}{\emph{the
Biden vs. Trump 2020 election here}}\emph{.}

As
\href{https://www.nytimes.com/interactive/2020/us/elections/donald-trump.html}{President
Trump} deploys federal agents to Portland, Ore., and threatens to
dispatch more to other cities, his re-election campaign is spending
millions of dollars on several ominous television ads that promote fear
and dovetail with his political message of ``law and order.''

The influx of agents in Portland has led to
\href{https://www.nytimes.com/2020/07/21/us/portland-protests.html}{scenes
of confrontations and chaos} that Mr. Trump and his White House aides
have pointed to as they try to burnish a false narrative about
Democratic elected officials allowing dangerous protesters to create
widespread bedlam.

The Trump campaign is driving home that message with
\href{https://www.youtube.com/watch?v=moZOrq0qL3Q\&feature=youtu.be}{a
new ad} that tries to tie its dark portrayal of Democratic-led cities to
Mr. Trump's main rival,
\href{https://www.nytimes.com/interactive/2020/us/elections/joe-biden.html}{Joseph
R. Biden Jr.} --- with exaggerated images intended to persuade viewers
that lawless anarchy would prevail if Mr. Biden won the presidency. The
ad simulates a break-in at the home of an older woman and ends with her
being attacked while she waits on hold for a 911 call, as shadowy, dark
intruders flicker in the background.

So far, the campaign has spent almost \$20 million over the last 20 days
on that ad and two other similar ones, more than Mr. Biden has spent on
his total television budget in the same time frame, and a relatively
large sum for this stage of the race. Though the ads predate the federal
actions in Portland, they convey a common theme of lawlessness under
Democratic leadership.

The focus of the Trump administration in recent days has been on
Portland, where there have been nightly protests for weeks denouncing
systemic racism in policing. In the last few days, federal agents from
the Department of Homeland Security and U.S. Marshals, traveling in
unmarked cars, have
\href{https://www.nytimes.com/2020/07/17/us/portland-protests.html}{swooped
protesters off the street} without explaining why, in some cases
detaining them and in other cases letting them go because they were not
actually suspects. The protests have increased in size since the arrival
of federal officials.

Mr. Trump's deployment of federal law enforcement is highly unusual: He
is acting in spite of local opposition --- city leaders are not asking
for help --- and his actions go beyond emergency steps taken by some
past American leaders like President George H.W. Bush, who sent troops
to quell Los Angeles in 1992 at the request of California officials.

In Washington on Tuesday, Department of Homeland Security officials held
a news conference for the first time to address the increased federal
deployment in Portland, defending the tactics and training of the
agents. Chad F. Wolf, the acting secretary, said a federal statute
allowed the agents to move away from the courthouse that they had been
told to defend, to investigate crimes against federal property and
officers, even if it resulted in the detaining of a protester.

Another top official, Mark Morgan, disputed claims that the agents
lacked adequate insignia, showing reporters a camouflaged ballistic vest
labeled ``POLICE.'' Mr. Wolf also blamed local officials for the unrest
in Portland. ``I asked the mayor and governor, how long do you plan on
having this continue?'' Mr. Wolf said. ``We stand ready. I'm ready to
pull my officers out of there if the violence stops.''

\includegraphics{https://static01.nyt.com/images/2020/07/21/us/politics/21trump-ads2/merlin_174810570_da0a48a5-f3b3-4701-b4ad-3339bb324ff7-articleLarge.jpg?quality=75\&auto=webp\&disable=upscale}

The president has said he might next deploy federal agents to Chicago,
and has listed other cities where similar enforcement could take place,
including New York but also Philadelphia and Detroit, urban centers in
two battleground states. White House officials said the deployments had
grown out of meetings among administration officials after protests in
Washington, D.C., in late May and early June.

The White House has defended the recent measures.

``By any objective standard, the violence, chaos and anarchy in Portland
is unacceptable, yet Democrats continue to put politics above peace
while this president seeks to restore law and order,'' the White House
press secretary, Kayleigh McEnany, said at a briefing on Tuesday
morning. She listed an array of items she said protesters had hurled at
law enforcement officers.

Trump administration officials and campaign aides have woven together
the protests that began after
\href{https://www.nytimes.com/2020/05/31/us/george-floyd-investigation.html}{the
killing of George Floyd} in May to try to bolster their claim that under
Mr. Biden, the police would be ``defunded.'' While Mr. Biden has
\href{https://www.nytimes.com/2020/06/08/us/politics/biden-defund-the-police.html}{walked
a careful line} and said explicitly that he doesn't support defunding
police departments, the Trump campaign has continued to claim otherwise.

The most recent ad from the Trump campaign, depicting the break-in at a
woman's home, has a singular goal: terrifying the viewer into believing
that claim.

The ad's audio includes a news broadcast that talks about ``Seattle's
pledge to defund its police department,'' referring to another
progressive city with which Mr. Trump has feuded.

The spot hews to Mr. Trump's long-held preference for messages that
promote fear and division, dating to
\href{https://www.nytimes.com/2016/01/05/us/politics/in-first-ad-donald-trump-plays-to-fears-on-immigration-and-isis.html}{the
first ad of his 2016 presidential campaign}, which depicted immigrants
as criminals. The campaign has already spent nearly \$550,000 on its new
ad, which was released on Monday.

Describing his opponents as supporting violence while portraying police
officers in glowing terms has been a mainstay of Mr. Trump's public
discourse since the late 1980s.

Protests around the country have been largely peaceful, with spikes of
conflict usually arising in clashes with law enforcement. While polls
show that
\href{https://www.nytimes.com/interactive/2020/06/10/upshot/black-lives-matter-attitudes.html}{a
majority of voters support the Black Lives Matter movement}, Mr. Trump
and some of his advisers are counting on a backlash, so far nonexistent,
with white voters in the fall that will boost the president's numbers.

``Clearly what they're looking to do here is scare the living hell out
of seniors,'' said Pia Carusone, a Democratic ad maker. But, she said,
the new Trump ad falls short in the realm of believability. ``You're
making the assumption that the voter that you're hoping to convince is
going to relate and think that this could happen. And then you have to
make the leap to blame Biden or the Democrats or whoever. And I think it
fails that first test.''

Stuart Stevens, a Republican strategist who now works with the
anti-Trump group known as the Lincoln Project, said Mr. Trump's team was
focusing on an issue that doesn't rank at the top of voter concerns.

``I'd bet a lot that the actress they hired for this is more worried
about Covid-19 than a phony threat about cops,'' Mr. Stevens said.

Of the \$24 million the Trump campaign has spent over all on television
ads over the past 20 days, roughly \$20 million has gone to ads that
focus solely on the issue of the police. About 70 percent of that \$20
million has been spent on a singular ad that shows a split screen: One
side depicts an empty 911 call center, with an answering service asking
callers to select their emergency, and the other displays violent scenes
from the protests.

The Trump digital apparatus has also been running a torrent of ads
warning of a country in crisis: ``Dangerous MOBS of far-left groups are
running through our streets and causing absolute mayhem,''
\href{https://www.facebook.com/ads/library/?id=281763573095189}{one ad
with 308 variations reads}. ``They are DESTROYING our cities and
rioting.''

The Trump team has spent at least \$2 million in the past two months on
Facebook ads with similar themes, according to Advertising Analytics, an
ad tracking firm.

The ads are on a political track. But for former Homeland Security
officials who served in the first year of the Trump administration,
seeing images of federal forces on the streets of American cities was
distressing.

``People like me, who served a long time, have to look very long and
hard to figure out who these people are,'' said Col. David Lapan, a
retired Marine who served in the Trump administration in 2017 as a
spokesman for the Department of Homeland Security. ``For the average
citizen, it looks like the military is being used to suppress American
citizens. Even if that's not the case, and this is law enforcement, it
creates the impression that the military is being used.''

In a statement on Tuesday evening, Mr. Biden drew a parallel with the
largely peaceful protesters who were
\href{https://www.nytimes.com/2020/06/02/us/politics/trump-walk-lafayette-square.html}{cleared
from a park near the White House} on June 1 by armed law enforcement
officials using chemical irritants before Mr. Trump's photo-op outside a
historic church.

Image

Protesters raised their cellphones and sang in front of the Multnomah
County Justice Center in Portland on Monday. The demonstrations have
grown in size since federal agents arrived.Credit...Mason Trinca for The
New York Times

``They are brutally attacking peaceful protesters,
\href{https://www.nytimes.com/2020/07/20/us/portland-protests-navy-christopher-david.html}{including
a U.S. Navy veteran},'' Mr. Biden said of the force used in Portland.
``Of course the U.S. government has the right and duty to protect
federal property. The Obama-Biden administration protected federal
property across the country without resorting to these egregious tactics
--- and without trying to stoke the fires of division in this country.''
In response, Mr. Trump's campaign accused Mr. Biden of attacking law
enforcement officials.

Tom Ridge, the former governor of Pennsylvania who was the first person
to serve as secretary of Homeland Security, also condemned Mr. Trump's
actions.

''The department was established to protect America from the
ever-present threat of global terrorism,'' Mr. Ridge, a Republican, told
the radio host Michael Smerconish. ``It was not established to be the
president's personal militia.''

Mr. Ridge said it would be a ``cold day in hell'' before he would have
consented as a governor to what is taking place. ``I wish the president
would take a more collaborative approach toward fighting this
lawlessness than the unilateral approach he's taken,'' he said.

Zolan Kanno-Youngs contributed reporting. Jack Begg and Isabella Grullón
Paz contributed research.

\hypertarget{our-2020-election-guide}{%
\section{Our 2020 Election Guide}\label{our-2020-election-guide}}

Updated Aug. 7, 2020

\begin{itemize}
\item
  \begin{center}\rule{0.5\linewidth}{\linethickness}\end{center}

  \hypertarget{the-latest}{%
  \subsection{The Latest}\label{the-latest}}

  \begin{itemize}
  \tightlist
  \item
    \href{https://www.nytimes.com/2020/08/07/us/politics/russia-china-trump-biden-election-interference.html?action=click\&pgtype=Article\&state=default\&region=BELOW_MAIN_CONTENT\&context=storylines_guide}{Russia
    is using a range of techniques to denigrate Joe Biden}, American
    intelligence officials said, declaring that Moscow continues to try
    to interfere in the 2020 campaign to help President Trump.
  \end{itemize}
\item
  \begin{center}\rule{0.5\linewidth}{\linethickness}\end{center}

  \hypertarget{bidens-vp-search}{%
  \subsection{Biden's V.P. Search}\label{bidens-vp-search}}

  \begin{itemize}
  \tightlist
  \item
    \href{https://www.nytimes.com/article/biden-vice-president-2020.html?action=click\&pgtype=Article\&state=default\&region=BELOW_MAIN_CONTENT\&context=storylines_guide}{Here
    are 13 women} who have been under consideration to be Joe Biden's
    running mate, and why each might be chosen --- and might not be.
  \end{itemize}
\item
  \begin{center}\rule{0.5\linewidth}{\linethickness}\end{center}

  \hypertarget{keep-up-with-our-coverage}{%
  \subsection{Keep Up With Our
  Coverage}\label{keep-up-with-our-coverage}}

  \begin{itemize}
  \tightlist
  \item
    Get an
    \href{https://www.nytimes.com/newsletters/politics?action=click\&pgtype=Article\&state=default\&region=BELOW_MAIN_CONTENT\&context=storylines_guide}{email}
    recapping the day's news
  \end{itemize}

  \begin{itemize}
  \tightlist
  \item
    Download our mobile app on
    \href{https://apps.apple.com/us/app/nytimes/id284862083?ls=1\&mat_click_id=5c79ae7455014fd1bd66b5610c05b8f2-20191112-16948\&referrer=mat_click_id\%3D5c79ae7455014fd1bd66b5610c05b8f2-20191112-16948\%26link_click_id\%3D722930677036718082}{iOS}
    and
    \href{http://a.localytics.com/android?id=com.nytimes.android\&referrer=utm_source\%3Dother_nyt_mobile_web\%26utm_medium\%3DWeb\%2520page\%26utm_term\%3DGeneral\%2520Mobile\%2520Page\%26utm_campaign\%3DNYT\%2520Mobile\%2520General\%2520Page}{Android}
    and turn on Breaking News and Politics alerts
  \end{itemize}
\end{itemize}

Advertisement

\protect\hyperlink{after-bottom}{Continue reading the main story}

\hypertarget{site-index}{%
\subsection{Site Index}\label{site-index}}

\hypertarget{site-information-navigation}{%
\subsection{Site Information
Navigation}\label{site-information-navigation}}

\begin{itemize}
\tightlist
\item
  \href{https://help.nytimes.com/hc/en-us/articles/115014792127-Copyright-notice}{©~2020~The
  New York Times Company}
\end{itemize}

\begin{itemize}
\tightlist
\item
  \href{https://www.nytco.com/}{NYTCo}
\item
  \href{https://help.nytimes.com/hc/en-us/articles/115015385887-Contact-Us}{Contact
  Us}
\item
  \href{https://www.nytco.com/careers/}{Work with us}
\item
  \href{https://nytmediakit.com/}{Advertise}
\item
  \href{http://www.tbrandstudio.com/}{T Brand Studio}
\item
  \href{https://www.nytimes.com/privacy/cookie-policy\#how-do-i-manage-trackers}{Your
  Ad Choices}
\item
  \href{https://www.nytimes.com/privacy}{Privacy}
\item
  \href{https://help.nytimes.com/hc/en-us/articles/115014893428-Terms-of-service}{Terms
  of Service}
\item
  \href{https://help.nytimes.com/hc/en-us/articles/115014893968-Terms-of-sale}{Terms
  of Sale}
\item
  \href{https://spiderbites.nytimes.com}{Site Map}
\item
  \href{https://help.nytimes.com/hc/en-us}{Help}
\item
  \href{https://www.nytimes.com/subscription?campaignId=37WXW}{Subscriptions}
\end{itemize}
