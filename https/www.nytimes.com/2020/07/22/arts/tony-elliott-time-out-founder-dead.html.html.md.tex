Sections

SEARCH

\protect\hyperlink{site-content}{Skip to
content}\protect\hyperlink{site-index}{Skip to site index}

\href{https://www.nytimes.com/section/arts}{Arts}

\href{https://myaccount.nytimes.com/auth/login?response_type=cookie\&client_id=vi}{}

\href{https://www.nytimes.com/section/todayspaper}{Today's Paper}

\href{/section/arts}{Arts}\textbar{}Tony Elliott, Whose Time Out Clued
Readers In, Dies at 73

\url{https://nyti.ms/3eTb0JS}

\begin{itemize}
\item
\item
\item
\item
\item
\end{itemize}

Advertisement

\protect\hyperlink{after-top}{Continue reading the main story}

Supported by

\protect\hyperlink{after-sponsor}{Continue reading the main story}

\hypertarget{tony-elliott-whose-time-out-clued-readers-in-dies-at-73}{%
\section{Tony Elliott, Whose Time Out Clued Readers In, Dies at
73}\label{tony-elliott-whose-time-out-clued-readers-in-dies-at-73}}

Mr. Elliott said, ``I had one idea, but it was a good one.'' On it he
built a global publishing empire.

\includegraphics{https://static01.nyt.com/images/2020/07/24/obituaries/21Elliot3/21Elliot3-articleLarge.jpg?quality=75\&auto=webp\&disable=upscale}

\href{https://www.nytimes.com/by/john-leland}{\includegraphics{https://static01.nyt.com/images/2018/02/20/multimedia/author-john-leland/author-john-leland-thumbLarge.jpg}}

By \href{https://www.nytimes.com/by/john-leland}{John Leland}

\begin{itemize}
\item
  Published July 22, 2020Updated July 29, 2020
\item
  \begin{itemize}
  \item
  \item
  \item
  \item
  \item
  \end{itemize}
\end{itemize}

Tony Elliott, who started the Time Out global publishing empire in his
mother's London kitchen in 1968 with a capital investment of 70 pounds
and a simple idea --- tell people where they can see the right movie or
band, or find a haircut or a falafel --- died on July 16 in London. He
was 73.

The cause was lung cancer, his wife, Jane Elliott, confirmed.

From its first issue, in 1968 --- a single poster-size sheet, folded
four times, that functioned as a guide to the local counterculture ---
Mr. Elliott's creation grew into a worldwide enterprise, with businesses
in 327 cities and 58 countries, including close to 50 magazines devoted
to particular cities. Its websites draw 63 million unique visitors per
month, said Julio Bruno, chief executive officer of Time Out Group.

``His thing was, `I had one idea, but it was a good one,''' Ms. Elliott
said.

Mr. Elliott, who left college to start the business, was an accidental
tycoon whose idea arrived at a ripe moment --- when the cultural map was
shifting too quickly for the established news media to keep up, and
people not in the know needed guidance from those who were.

``If you were in the in crowd, you knew,'' Ms. Elliott said. ``If you
weren't, that's what Tony provided.''

Through the hippie, punk and cyberculture eras, the magazines championed
fringe theater, cheap eats, family activities and occasional politics.
They were also among the first in the mainstream press to cover gay
life, and an early column, Meet the Fuzz, listed forthcoming
demonstrations and political activities.

Mr. Elliott felt that the original magazine should be part of the world
it covered; Time Out's first office was in the basement of the home of
Pink Floyd's keyboard player, Rick Wright, Ms. Elliott said.

\includegraphics{https://static01.nyt.com/images/2020/07/24/obituaries/21Elliott2/21Elliott2-articleLarge.jpg?quality=75\&auto=webp\&disable=upscale}

As the magazine expanded to other cities, starting with Time Out New
York in 1995, it maintained the voice of a local insider, anticipating
the internet deluge of information that was just around the corner.

``It was the last hurrah of that need,'' said Cyndi Stivers, the first
editor of Time Out New York, who helped start some of Time Out's other
city magazines. ``You didn't have the internet in your pocket, and there
was not much on the web yet. When we launched, people were grateful.''

The magazines started the careers of mostly young writers, some of whom
got old with Time Out. ``Stephin Merritt wrote `69 Love Songs' when he
was our copy editor,'' Ms. Stivers said, referring to the leader of the
band Magnetic Fields.

Anthony Michael Manton Elliott was born on Jan. 7, 1947, in Reading,
England, to Alan and
\href{https://www.nytimes.com/1994/11/17/obituaries/katherine-elliott-75-aided-health-of-poor.html}{Dr.
Katherine Elliott}. His mother was assistant medical director of the
CIBA Foundation; his father was managing director of a food distribution
company. The family moved to London in Tony's second year.

He attended Stowe School, then went to Keele University in the Midlands
city of Keele, north of London. There he edited a student arts magazine
called Unit, which ran features and interviews with Jimi Hendrix, John
Lennon and Yoko Ono. Returning to London during a school break, he found
that the local listings in the mainstream and alternative press were
thin guides to all that was going on in Swinging London. He felt he
could do better.

``In 1968, he came into the Black Dwarf, a radical magazine I was
editing, and said he loved the paper, and why don't we have a supplement
that is essentially listings?'' said Tariq Ali, a writer and historian
who became a columnist at Time Out. ``I burst out laughing.''

Mr. Elliott's original name for the magazine, abandoned days before it
went to press, was Where It's At. Instead, he borrowed the name Time Out
from a Dave Brubeck album. The initial print run, of 5,000 copies,
rolled off a press owned by the local Communist Party.

He was 21.

At the time, most publications' event listings were simply rewritten
news releases, presented dutifully. Mr. Elliott and his founding
partner, Bob Harris, licensed his staff to be opinionated, funny and
idiosyncratic. He demanded absolute consistency of format, typeface and
style, ``but you could say whatever you wanted,'' Ms. Stivers said.

The London magazine grew. In the early 1980s, a large part of its staff
left in a dispute over pay, forming a rival publication called City
Limits, which lasted until 1991.

Image

Later in his career Mr. Elliott lamented the sameness of new
publications compared with the creative anarchy of the late 1960s, and
the modern ``cycle of building magazines round Prada and Levi's
ads.''Credit...Andy Parsons/Time Out

He married Janet Street-Porter, a journalist, in 1975, divorcing
amicably two years later. He met his second wife, Jane, at the design
company that created Time Out's covers; they married in 1989. Their
oldest son, Rufus, 32, is a ``producer and fixer in the media
industry,'' Ms. Elliott said. Their twins, Bruce and Lawrence, 29, are a
teacher and singer songwriter, respectively.

In addition to his wife the three sons survive him, as does a younger
sister, Rose Elliott.

Once Time Out New York had shown that Mr. Elliott's idea could work
elsewhere, the seat-of-the-pants days were over. City life was
gentrifying, and recommendations for what had been cultural adventures
came to resemble tips for consumption.

Mr. Ali remembered running into his old friend and joking that he had
become too rich. ``He said, `Do you think I'm a sellout?' He was
half-jesting, but half wondering what I thought of him now.''

Mr. Elliott tried backing magazines dedicated to the environment and to
digital culture, but neither got off the ground.

In 2002, a grand old man of publishing, he lamented the sameness of new
publications compared with the creative anarchy of the late 1960s, and
despaired of the modern
``\href{https://www.independent.co.uk/news/media/a-lost-cause-181556.html}{cycle
of building magazines round Prada and Levi's ads}.''

Eight years later, he sold most of his company to a private equity fund,
\href{https://oakleycapital.com/}{Oakley Capital}, remaining active as a
board member.

Mr. Elliott poured his energies instead into British cultural
institutions, including rebuilding the
\href{https://www.roundhouse.org.uk/}{Roundhouse} arts center in London
and serving on the board of
\href{https://www.somersethouse.org.uk/about-somerset-house}{Somerset
House}, a creative community that promotes the work of emerging artists.

In 2017, Queen Elizabeth II appointed him a commander of the Order of
the British Empire, or CBE.

``He packed a lot into his short life,'' Ms. Elliott said.

Advertisement

\protect\hyperlink{after-bottom}{Continue reading the main story}

\hypertarget{site-index}{%
\subsection{Site Index}\label{site-index}}

\hypertarget{site-information-navigation}{%
\subsection{Site Information
Navigation}\label{site-information-navigation}}

\begin{itemize}
\tightlist
\item
  \href{https://help.nytimes.com/hc/en-us/articles/115014792127-Copyright-notice}{©~2020~The
  New York Times Company}
\end{itemize}

\begin{itemize}
\tightlist
\item
  \href{https://www.nytco.com/}{NYTCo}
\item
  \href{https://help.nytimes.com/hc/en-us/articles/115015385887-Contact-Us}{Contact
  Us}
\item
  \href{https://www.nytco.com/careers/}{Work with us}
\item
  \href{https://nytmediakit.com/}{Advertise}
\item
  \href{http://www.tbrandstudio.com/}{T Brand Studio}
\item
  \href{https://www.nytimes.com/privacy/cookie-policy\#how-do-i-manage-trackers}{Your
  Ad Choices}
\item
  \href{https://www.nytimes.com/privacy}{Privacy}
\item
  \href{https://help.nytimes.com/hc/en-us/articles/115014893428-Terms-of-service}{Terms
  of Service}
\item
  \href{https://help.nytimes.com/hc/en-us/articles/115014893968-Terms-of-sale}{Terms
  of Sale}
\item
  \href{https://spiderbites.nytimes.com}{Site Map}
\item
  \href{https://help.nytimes.com/hc/en-us}{Help}
\item
  \href{https://www.nytimes.com/subscription?campaignId=37WXW}{Subscriptions}
\end{itemize}
