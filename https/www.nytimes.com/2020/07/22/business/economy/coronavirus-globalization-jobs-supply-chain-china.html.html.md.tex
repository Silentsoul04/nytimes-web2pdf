Sections

SEARCH

\protect\hyperlink{site-content}{Skip to
content}\protect\hyperlink{site-index}{Skip to site index}

\href{https://www.nytimes.com/section/business/economy}{Economy}

\href{https://myaccount.nytimes.com/auth/login?response_type=cookie\&client_id=vi}{}

\href{https://www.nytimes.com/section/todayspaper}{Today's Paper}

\href{/section/business/economy}{Economy}\textbar{}The Pandemic Isn't
Bringing Back Factory Jobs, at Least Not Yet

\url{https://nyti.ms/2CClSib}

\begin{itemize}
\item
\item
\item
\item
\item
\item
\end{itemize}

\href{https://www.nytimes.com/news-event/coronavirus?action=click\&pgtype=Article\&state=default\&region=TOP_BANNER\&context=storylines_menu}{The
Coronavirus Outbreak}

\begin{itemize}
\tightlist
\item
  live\href{https://www.nytimes.com/2020/08/04/world/coronavirus-covid-19.html?action=click\&pgtype=Article\&state=default\&region=TOP_BANNER\&context=storylines_menu}{Latest
  Updates}
\item
  \href{https://www.nytimes.com/interactive/2020/us/coronavirus-us-cases.html?action=click\&pgtype=Article\&state=default\&region=TOP_BANNER\&context=storylines_menu}{Maps
  and Cases}
\item
  \href{https://www.nytimes.com/interactive/2020/science/coronavirus-vaccine-tracker.html?action=click\&pgtype=Article\&state=default\&region=TOP_BANNER\&context=storylines_menu}{Vaccine
  Tracker}
\item
  \href{https://www.nytimes.com/2020/08/02/us/covid-college-reopening.html?action=click\&pgtype=Article\&state=default\&region=TOP_BANNER\&context=storylines_menu}{College
  Reopening}
\item
  \href{https://www.nytimes.com/live/2020/08/03/business/stock-market-today-coronavirus?action=click\&pgtype=Article\&state=default\&region=TOP_BANNER\&context=storylines_menu}{Economy}
\end{itemize}

Advertisement

\protect\hyperlink{after-top}{Continue reading the main story}

Supported by

\protect\hyperlink{after-sponsor}{Continue reading the main story}

\hypertarget{the-pandemic-isnt-bringing-back-factory-jobs-at-least-not-yet}{%
\section{The Pandemic Isn't Bringing Back Factory Jobs, at Least Not
Yet}\label{the-pandemic-isnt-bringing-back-factory-jobs-at-least-not-yet}}

It's a moment of reckoning for global supply chains. But that doesn't
mean companies are flocking back to the United States.

\includegraphics{https://static01.nyt.com/images/2020/07/19/business/00DC-VIRUS-RESHORING-01/merlin_172720542_481d3124-2fec-4c75-a4e6-cbd3534360e3-articleLarge.jpg?quality=75\&auto=webp\&disable=upscale}

\href{https://www.nytimes.com/by/ana-swanson}{\includegraphics{https://static01.nyt.com/images/2018/12/10/multimedia/author-ana-swanson/author-ana-swanson-thumbLarge.png}}\href{https://www.nytimes.com/by/jim-tankersley}{\includegraphics{https://static01.nyt.com/images/2018/10/19/multimedia/author-jim-tankersley/author-jim-tankersley-thumbLarge.png}}

By \href{https://www.nytimes.com/by/ana-swanson}{Ana Swanson} and
\href{https://www.nytimes.com/by/jim-tankersley}{Jim Tankersley}

\begin{itemize}
\item
  July 22, 2020
\item
  \begin{itemize}
  \item
  \item
  \item
  \item
  \item
  \item
  \end{itemize}
\end{itemize}

\href{https://cn.nytimes.com/business/20200723/coronavirus-globalization-jobs-supply-chain-china/}{阅读简体中文版}\href{https://cn.nytimes.com/business/20200723/coronavirus-globalization-jobs-supply-chain-china/zh-hant/}{閱讀繁體中文版}

WASHINGTON --- For companies with supply chains that snake around the
globe, the crises have just kept coming: First the prolonged and painful
U.S.-China trade war, then a pandemic that snarled shipments, stalled
international travel and shut factory doors.

President Trump and his advisers have seized on the disruptions to make
a familiar case to manufacturers: Come back home.

``The global pandemic has proven once and for all that to be a strong
nation, America must be a manufacturing nation,'' Mr. Trump said at a
Ford factory in Ypsilanti, Mich., on May 21. ``We're bringing it back.''

Mr. Trump has spent much of his presidency trying to cajole
manufacturers to return to the United States, through both tough talk
and policies like tariffs. His advisers have pointed to both the trade
war and the pandemic as evidence that it is just too risky for
multinational companies to rely on other countries, particularly China,
to make their goods.

But those arguments have yet to result in a wave of factories returning
to the United States. Foreign direct investment into the United States
--- which measures spending from internationally owned companies to
start, expand or acquire American businesses ---
\href{https://www.bea.gov/news/2020/new-foreign-direct-investment-united-states-2019}{sank
drastically last year}, to its lowest recorded level since 2006.

Foreign-owned companies invested about half as much in the United States
in 2019 as they did in 2016, the year before Mr. Trump took office.
After increasing in the first two years of Mr. Trump's presidency, the
number of manufacturing jobs
\href{https://www.bls.gov/iag/tgs/iag31-33.htm}{flatlined last year} and
fell sharply with the pandemic. As of June, there were nearly 300,000
fewer factory jobs in the United States than there were when Mr. Trump
was inaugurated.

For all the president's criticisms of global supply chains, the economic
incentive to outsource still prevails. While his trade policy has made
doing business abroad, particularly in China, more uncertain and costly,
higher wages in the United States and the lure of foreign markets mean
that most global businesses are choosing to remain global. Most firms
that shifted out of China to avoid the crossfire of the trade war moved
to other low-cost countries, like Vietnam and Mexico. Other companies
say China is a growth market they cannot afford to lose.

And while the pandemic has prompted
\href{https://www.nytimes.com/2020/03/05/business/coronavirus-globalism.html}{a
broader reassessment} of the risks of global supply chains, it has also
brought about the deepest economic contraction in generations, battering
companies' finances and forcing them to cut back on workers. Executives
are deeply uncertain what demand for their products will look like in
the coming months and years --- hardly the environment to encourage big
investments in new American factories.

The furniture maker La-Z-Boy is one example. The company shifted its
production out of China to Vietnam last year to bypass Mr. Trump's
tariffs on \$360 billion worth of Chinese goods, according to tracking
by Panjiva, a research firm. But on a June 24 earnings call, Kurt L.
Darrow, La-Z-Boy's chief executive, announced that the economic effects
of the pandemic would force the firm to make steep cuts to its work
force, including in the United States.

``While we were pleased to have brought back some 6,000 furloughed
workers, we made the decision to permanently close our Newton, Miss.,
La-Z-Boy branded manufacturing facility and reduce our global work force
by approximately 10 percent,'' Mr. Darrow said.

\hypertarget{latest-updates-economy}{%
\section{\texorpdfstring{\href{https://www.nytimes.com/live/2020/08/04/business/stock-market-today-coronavirus?action=click\&pgtype=Article\&state=default\&region=MAIN_CONTENT_1\&context=storylines_live_updates}{Latest
Updates:
Economy}}{Latest Updates: Economy}}\label{latest-updates-economy}}

\href{https://www.nytimes.com/live/2020/08/04/business/stock-market-today-coronavirus?action=click\&pgtype=Article\&state=default\&region=MAIN_CONTENT_1\&context=storylines_live_updates\#bp-to-step-up-renewable-investment-as-it-reports-a-huge-loss}{9m
ago}

\href{https://www.nytimes.com/live/2020/08/04/business/stock-market-today-coronavirus?action=click\&pgtype=Article\&state=default\&region=MAIN_CONTENT_1\&context=storylines_live_updates\#bp-to-step-up-renewable-investment-as-it-reports-a-huge-loss}{BP
to step up renewable investment as it reports a huge loss.}

\href{https://www.nytimes.com/live/2020/08/04/business/stock-market-today-coronavirus?action=click\&pgtype=Article\&state=default\&region=MAIN_CONTENT_1\&context=storylines_live_updates\#some-caterers-are-finding-creative-ways-to-keep-their-businesses-afloat}{9m
ago}

\href{https://www.nytimes.com/live/2020/08/04/business/stock-market-today-coronavirus?action=click\&pgtype=Article\&state=default\&region=MAIN_CONTENT_1\&context=storylines_live_updates\#some-caterers-are-finding-creative-ways-to-keep-their-businesses-afloat}{Some
caterers are finding creative ways to keep their businesses afloat.}

\href{https://www.nytimes.com/live/2020/08/04/business/stock-market-today-coronavirus?action=click\&pgtype=Article\&state=default\&region=MAIN_CONTENT_1\&context=storylines_live_updates\#the-chicago-fed-president-says-its-up-to-congress-to-save-the-economy}{9m
ago}

\href{https://www.nytimes.com/live/2020/08/04/business/stock-market-today-coronavirus?action=click\&pgtype=Article\&state=default\&region=MAIN_CONTENT_1\&context=storylines_live_updates\#the-chicago-fed-president-says-its-up-to-congress-to-save-the-economy}{The
Chicago Fed president says it's up to Congress to save the economy.}

\href{https://www.nytimes.com/live/2020/08/04/business/stock-market-today-coronavirus?action=click\&pgtype=Article\&state=default\&region=MAIN_CONTENT_1\&context=storylines_live_updates}{See
more updates}

More live coverage:
\href{https://www.nytimes.com/2020/08/04/world/coronavirus-covid-19.html?action=click\&pgtype=Article\&state=default\&region=MAIN_CONTENT_1\&context=storylines_live_updates}{Global}

There could still be a more significant reordering of global factory
activity on the horizon. The one-two punch of the trade war and pandemic
has shaken the confidence of executives and investors; led to shortages
of
\href{https://www.nytimes.com/2020/03/13/business/toilet-paper-shortage.html}{toilet
paper},
\href{https://www.nytimes.com/2020/05/05/business/coronavirus-meat-shortages.html}{meat},
\href{https://www.wsj.com/articles/cios-face-shortages-of-tech-gear-as-coronavirus-forces-shipment-delays-11586338202}{laptops}
and
\href{https://www.gq.com/story/inside-the-great-kettlebell-shortage}{kettle
bells}; and revealed hidden frailties in many companies' business
models.

As factories struggle to reopen with components still in short supply,
some executives are questioning the
\href{https://www.nytimes.com/2020/03/05/business/coronavirus-globalism.html}{just-in-time
supply chains} they use to whisk products around the globe, rather than
keeping warehouses stocked --- and particularly how much they rely on
factories in China, where production moved en masse in previous decades.

Emily J. Blanchard, a professor at the Tuck School of Business at
Dartmouth College who studies global value chains, said many firms were
not thinking ``in such broad and apocalyptic terms'' before the
pandemic.

``Covid has generated this new imagination of worst-case scenarios,''
Professor Blanchard said.

\includegraphics{https://static01.nyt.com/images/2020/07/19/business/00DC-Virus-Reshoring-honeywell/merlin_172222197_b7b03af1-dc76-4091-a220-276fe6dea46f-articleLarge.jpg?quality=75\&auto=webp\&disable=upscale}

Under the pressure of the trade war, some multinational companies have
opened new facilities in the United States, including
\href{https://www.cnbc.com/2019/05/13/williams-sonoma-ceo-says-it-shifted-operations-ahead-of-tariff-hikes.html}{Williams
Sonoma} and
\href{https://www.wsj.com/articles/stanley-to-make-more-craftsman-tools-in-u-s-11557919800}{Stanley
Black \& Decker}. Taiwan Semiconductor Manufacturing Company announced
in May that it would set up a
\href{https://www.nytimes.com/2020/05/14/technology/trump-tsmc-us-chip-facility.html}{new
facility in Arizona}, pending funding. And makers of masks and
protective gear, like Honeywell and 3M, are expanding American
production during the pandemic.

Politicians from both parties are offering proposals to encourage more
manufacturing in the United States, such as more funding for industries
like
\href{https://www.nytimes.com/2020/06/11/business/economy/semiconductors-chips-congress-china.html}{semiconductors}
and
\href{https://www.nytimes.com/2020/03/11/business/economy/coronavirus-china-trump-drugs.html}{pharmaceutical
manufacturing}.

The Trump administration's newly created U.S. International Development
Finance Corporation may offer
\href{https://www.reuters.com/article/us-usa-trade-reshoring-exclusive/exclusive-u-s-development-agency-could-loan-billions-for-reshoring-official-says-idUSKBN23U31F}{tens
of billions of dollars} to help reshore manufacturing of protective
equipment and generic drugs. The administration is also considering
other tax incentives and ``reshoring subsidies,'' potentially as part of
the next stimulus package, to try to lure factories home.

But there is little data to support claims by administration officials
that their trade and tax policies have already encouraged significant
\href{https://www.nytimes.com/2020/05/11/opinion/coronavirus-jobs-offshoring.html}{reshoring
of manufacturing} or
\href{https://www.nytimes.com/2020/06/16/business/economy/trump-trade-tariffs.html}{created
a ``blue-collar boom}.''

Image

Workers assembled face shields at a Hasbro manufacturing facility in
East Longmeadow, Mass., in April.Credit...Adam Glanzman/Bloomberg

U.S. \href{https://fred.stlouisfed.org/graph/?g=sKQ7}{factory output}
declined throughout 2019, as Mr. Trump's trade war intensified, and it
has dropped further this year, suggesting there is no boom in new
American factories. Since peaking in mid-2019, corporate investment has
\href{https://fred.stlouisfed.org/series/PNFI}{declined} for three
consecutive quarters. Total foreign direct investment in manufacturing
was nearly one-third lower in the first three years of Mr. Trump's
tenure than it was in the final three years of President Barack Obama's.

Mr. Trump ostensibly fought his trade war on behalf of American
manufacturing. But economists say it has actually been a drag on most
U.S. factories, by
\href{https://www.federalreserve.gov/econres/feds/files/2019086pap.pdf}{increasing
prices for components and inciting foreign retaliation}. It has also
coincided with a plunge in Chinese investment in the United States to
\href{https://rhg.com/research/two-way-street-us-china-investment-trends-2020-update/}{\$5
billion in 2019}, the lowest level since 2009, according to Rhodium
Group, a research firm.

Some Trump officials and their supporters blame a broader global
economic malaise that has dragged down factories around the world. And
they point to the fact that
\href{https://www.nytimes.com/2020/02/05/business/economy/trump-trade.html}{imports
fell last year} and now account for a slightly smaller share of the
goods consumed by Americans, as a sign of their success.

\href{https://www.prosperousamerica.org/cpa_s_new_manufacturing_reshoring_index_shows_strong_gains_in_2019}{Calculations
by the Coalition for a Prosperous America}, a trade group that supports
the administration's policies, indicate that 30.6 percent of the
manufactured goods Americans consumed in 2019 were imported, down
slightly from 31.2 percent the previous year. For much of the last two
decades, the trend went in the opposite direction.

There are good reasons for some companies to move out of China. Wages
are rising, whittling away at one incentive to manufacture there. And
deep
\href{https://www.nytimes.com/2020/07/14/world/asia/cold-war-china-us.html}{fissures
between the United States and China} have opened in areas like security
and technology, which could lead to more aggressive action by either
side, regardless of who wins the presidential election in November.

Still, more companies leaving China does not necessarily represent a win
for American workers. Like La-Z-Boy, many companies that are moving some
facilities out of China --- including Samsung, Hasbro, Apple, Nintendo
and GoPro --- are relocating to countries where wages are even lower.
While U.S. trade with China
\href{https://www.nytimes.com/2020/02/05/business/economy/trump-trade.html}{fell
sharply} last year, imports from Vietnam, Taiwan and Mexico swelled.

For many companies, making their supply chains more resilient has
actually meant spreading out production around the world, not
concentrating it in the United States, said Chris Rogers, a global trade
and logistics analyst at Panjiva.

``If you want to hedge your risks, you need to stay global,'' he said.

Image

Former Vice President Joseph R. Biden Jr., the presumptive Democratic
presidential nominee, has criticized the tariffs Mr. Trump imposed on
China.Credit...Mark Makela for The New York Times

Michael W. Upchurch, the chief financial officer of Kansas City
Southern, which runs railroads through Mexico and the United States,
said in an earnings call this year that more companies were eyeing
Mexico for new facilities because of the tariffs on China and Mexico's
relatively low wages and proximity to American customers.

``There is a real desire to begin to near-shore, and Mexico's a great
place to do business,'' Mr. Upchurch said. Constructing new factories
would take some time, he said, ``but over the next few years, we would
certainly expect to see benefit.''

For other companies, China still beckons.

Purveyors of consumer products, fast food and automobiles continue to
expand in China, which is home to a rapidly growing consumer market and
the world's greatest concentration of factories. Some firms have
struggled to find factory space or skilled workers outside of China.
\href{https://rhg.com/research/two-way-street-us-china-investment-trends-2020-update/}{Data
from Rhodium Group} show that U.S. foreign direct investment in China
continued to rise in 2019, despite the trade war.

In May, the U.S. engineering giant Honeywell
\href{http://www.xinhuanet.com/english/2020-05/19/c_139070358.htm}{opened
a new headquarters} in Wuhan, the original epicenter of the
\href{https://www.nytimes.com/news-event/coronavirus}{coronavirus
outbreak}. Tesla has announced plans to expand its Shanghai factory,
while Popeyes Louisiana Kitchen, Walmart and Costco are
\href{https://www.wsj.com/articles/neither-coronavirus-nor-trade-tensions-can-stop-u-s-companies-push-into-china-11589880603}{planning
new stores in China}.

Some executives insist that, contrary to popular belief, their
investments in China allow them to employ more workers in the United
States.

Milliken \& Company, a textile maker with headquarters in South Carolina
that employs 8,000 people, still performs the bulk of its manufacturing
in the United States. It resisted a wave of offshoring in the 1980s and
'90s by shifting into niche products, like floor coverings and military
uniforms. But it has opened factories in China in recent years to serve
that market, enabling it to hire more people back home, said Halsey M.
Cook Jr., the company's chief executive.

``I think you'd hear the same thing'' from other major multinational
companies, Mr. Cook said, like John Deere. ``Global supply chains are
complicated. When you've seen one, you've seen one.''

Jeanna Smialek contributed reporting.

Advertisement

\protect\hyperlink{after-bottom}{Continue reading the main story}

\hypertarget{site-index}{%
\subsection{Site Index}\label{site-index}}

\hypertarget{site-information-navigation}{%
\subsection{Site Information
Navigation}\label{site-information-navigation}}

\begin{itemize}
\tightlist
\item
  \href{https://help.nytimes.com/hc/en-us/articles/115014792127-Copyright-notice}{©~2020~The
  New York Times Company}
\end{itemize}

\begin{itemize}
\tightlist
\item
  \href{https://www.nytco.com/}{NYTCo}
\item
  \href{https://help.nytimes.com/hc/en-us/articles/115015385887-Contact-Us}{Contact
  Us}
\item
  \href{https://www.nytco.com/careers/}{Work with us}
\item
  \href{https://nytmediakit.com/}{Advertise}
\item
  \href{http://www.tbrandstudio.com/}{T Brand Studio}
\item
  \href{https://www.nytimes.com/privacy/cookie-policy\#how-do-i-manage-trackers}{Your
  Ad Choices}
\item
  \href{https://www.nytimes.com/privacy}{Privacy}
\item
  \href{https://help.nytimes.com/hc/en-us/articles/115014893428-Terms-of-service}{Terms
  of Service}
\item
  \href{https://help.nytimes.com/hc/en-us/articles/115014893968-Terms-of-sale}{Terms
  of Sale}
\item
  \href{https://spiderbites.nytimes.com}{Site Map}
\item
  \href{https://help.nytimes.com/hc/en-us}{Help}
\item
  \href{https://www.nytimes.com/subscription?campaignId=37WXW}{Subscriptions}
\end{itemize}
