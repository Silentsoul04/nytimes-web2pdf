Sections

SEARCH

\protect\hyperlink{site-content}{Skip to
content}\protect\hyperlink{site-index}{Skip to site index}

\href{https://myaccount.nytimes.com/auth/login?response_type=cookie\&client_id=vi}{}

\href{https://www.nytimes.com/section/todayspaper}{Today's Paper}

\href{/section/opinion}{Opinion}\textbar{}America Looks Hopelessly
Broke. It Isn't.

\href{https://nyti.ms/3jr17GV}{https://nyti.ms/3jr17GV}

\begin{itemize}
\item
\item
\item
\item
\item
\item
\end{itemize}

Advertisement

\protect\hyperlink{after-top}{Continue reading the main story}

\href{/section/opinion}{Opinion}

Supported by

\protect\hyperlink{after-sponsor}{Continue reading the main story}

\hypertarget{america-looks-hopelessly-broke-it-isnt}{%
\section{America Looks Hopelessly Broke. It
Isn't.}\label{america-looks-hopelessly-broke-it-isnt}}

For 40 years, both the left and the right have been unnecessarily
obsessed with deficits, to the detriment of the well-being of citizens.

\href{https://www.nytimes.com/by/farhad-manjoo}{\includegraphics{https://static01.nyt.com/images/2019/01/08/opinion/farhad-manjoo-opinion/farhad-manjoo-opinion-thumbLarge.png}}

By \href{https://www.nytimes.com/by/farhad-manjoo}{Farhad Manjoo}

Opinion Columnist

\begin{itemize}
\item
  July 22, 2020
\item
  \begin{itemize}
  \item
  \item
  \item
  \item
  \item
  \item
  \end{itemize}
\end{itemize}

\includegraphics{https://static01.nyt.com/images/2020/07/22/opinion/22manjooWeb/22manjooWeb-articleLarge.jpg?quality=75\&auto=webp\&disable=upscale}

\hypertarget{listen-to-this-op-ed}{%
\subsubsection{Listen to This Op-Ed}\label{listen-to-this-op-ed}}

Audio Recording by Audm

\emph{To hear more audio stories from publishers like The New York
Times, download}
\href{https://www.audm.com/?utm_source=nytmag\&utm_medium=embed\&utm_campaign=left_behind_draper}{**}
\href{https://www.audm.com/?utm_source=nytopinion\&utm_medium=embed\&utm_campaign=fix_america_spend}{\emph{Audm
for iPhone or Android}}\emph{.}

As a respite from
\href{https://www.nytimes.com/2020/04/22/opinion/coronavirus-parenting-burnout.html}{a
chaotic spring spent under quarantine}, my family booked a weeklong
vacation last month in a cozy, remote house in the California desert.
While the kids cannonballed into the saltwater pool and my wife sped
through several novels, I spent my time in the sun doing exactly the
sort of thing you'd imagine an opinion columnist might do on summer
vacation: I read two hot new books about macroeconomics.

Wait, don't leave! I promise this isn't as dry as it sounds.

The first book was
``\href{https://www.nytimes.com/2020/05/20/books/review-price-of-peace-john-maynard-keynes-zachary-carter.html?auth=login-email\&login=email}{The
Price of Peace,}'' Zachary Carter's incisive biography of the British
economist John Maynard Keynes, which illustrates the awesome power of
economic theory to alter the fates of nations and the lives of millions
of people. The second was
``\href{https://www.nytimes.com/2020/06/09/opinion/us-deficit-coronavirus.html}{The
Deficit Myth},'' in which the economist Stephanie Kelton convincingly
overturns the conventional wisdom that federal budget deficits are
somehow bad for the nation.

I'm
\href{https://www.nytimes.com/2020/05/20/opinion/coronavirus-worst-case.html}{on
record as a doomer}, but in different ways, these tomes sparked the
first real note of optimism I've felt about America's future in quite a
while. Together, they suggest a compelling political, moral and economic
case for the federal government to begin to do, again, what it once saw
as its duty --- to make big, bold and even expensive investments to
improve the lives of Americans, and perhaps of people around the world.

In the last few years, and especially in the hellish last couple of
months,
\href{https://www.theatlantic.com/magazine/archive/2020/06/underlying-conditions/610261/}{the
United States has come to feel like a failed state}. The coronavirus is
spreading, the economy is crumbling, society is fragmenting, our
\href{https://www.businessinsider.com/asce-gives-us-infrastructure-a-d-2017-3\#bridges-c-2}{infrastructure
is falling apart}, health care is inadequate and costly, child care is
impossible, and
\href{https://www.nytimes.com/2019/11/26/health/life-expectancy-rate-usa.html}{life
expectancy is declining}.

The federal government is not only often unwilling to help, but
seemingly incapable of it. To get just about anything done anymore,
Uncle Sam must go hat in hand to the behemoth private companies that now
rule much of our lives.
\href{https://www.nytimes.com/2020/03/18/opinion/coronavirus-trump.html}{Please,
Google,} will you create a coronavirus testing website? Please, Walmart,
will you set up in-person testing sites?

And whenever anyone is brave enough to suggest that the government
itself should provide useful services to Americans --- whether
big-ticket items like health care, child care and college education, or
smaller things like an upgraded electric grid or a national broadband
service --- the first reaction from many on the right and the left is
one of defeat and resignation. ``How will you pay for it?'' they ask.
And, often, the whole conversation stops right there, because with a
\href{https://www.treasurydirect.gov/govt/reports/pd/pd_debttothepenny.htm}{\$26.5
trillion national debt}, America looks hopelessly broke.

It is not. Kelton argues that our government's inability to provide for
citizens isn't due to a lack for money; instead, our leaders lack
political will.

Kelton --- who has worked as an economist for Democrats in the Senate
and as an adviser to Bernie Sanders's presidential campaigns --- is one
of the leading proponents of
\href{https://www.vox.com/future-perfect/2019/4/16/18251646/modern-monetary-theory-new-moment-explained}{Modern
Monetary Theory, or M.M.T}. The theory argues that because the
government is in charge of its own currency, it cannot ``run out'' of
money the way a household or a business can, and it therefore does not
need to raise taxes to fund government spending.

\includegraphics{https://static01.nyt.com/images/2020/07/22/opinion/22manjoo2/merlin_174805722_0dec6a6d-4e22-47c0-ac1c-2087ef64b1c2-articleLarge.jpg?quality=75\&auto=webp\&disable=upscale}

This doesn't mean that the government's resources are infinite, just
that deficits are not a true limit on what's possible. Instead of being
constrained by deficits, Kelton and other M.M.T.ers argue, policymakers
should care about ``real'' measures of economic activity: unemployment
and inflation.

Whatever the deficit, if unemployment is rife, it's an indication that
aggregate demand is low; to boost demand, the government can freely
spend, spend, spend --- and should stop spending only when there is a
danger that it will lead to a rise in prices --- that is, inflation ---
not because deficits will soar. In practice, Kelton and other M.M.T.ers
propose a federal jobs guarantee, in which the government would hire
anyone who needs a job for a set wage. The policy, she argues, would
promote full employment while keeping inflation stable.

M.M.T. is controversial even among left-leaning economists --- Lawrence
H. Summers, who once worked as Barack Obama's director of the National
Economic Council, has called it
\href{https://www.washingtonpost.com/opinions/the-lefts-embrace-of-modern-monetary-theory-is-a-recipe-for-disaster/2019/03/04/6ad88eec-3ea4-11e9-9361-301ffb5bd5e6_story.html}{``a
recipe for disaster''} --- and it's easy for non-economists to get lost
in the
\href{https://crookedtimber.org/2020/06/01/modern-monetary-theory-neither-modern-nor-monetary-nor-mainly-theoretical/}{many}
technical
\href{https://www.bradford-delong.com/2019/03/james-montier-wonders-why-people-hate-mmt-he-is-about-to-discover-that-mmt-hates-him.html}{debates}
surrounding the idea.

But one doesn't need to buy into everything about M.M.T. to see Kelton's
fundamental point --- that in the 40 years since Ronald Reagan won the
White House, both the left and the right have been unnecessarily
obsessed with deficits, to the detriment of the well-being of citizens.

The cruelest example of this mind-set occurred
\href{https://www.nytimes.com/2019/09/18/opinion/obama-2008-financial-crisis.html}{after
the Great Recession in 2008}. At the time, many experts suggested that
an adequate response to the downturn would require the government to
spend a trillion dollars or more to boost demand. Instead, Obama and his
aides, worried about sticker shock, lowballed their stimulus, and
millions of people remained unemployed.

In the decade since that recession, many economists and lawmakers have
\href{https://www.nytimes.com/2020/05/16/business/deficits-virus-economists-trump.html}{grown
less worried about deficits}, because red ink has not led to economic
calamity. That's to the good: Deficits are rarely questioned when
lawmakers are spending on the military or on tax cuts for corporations,
so it's only fair that they aren't constrained by deficits when spending
on things like health care, child care and education.

And right now, in the midst of a pandemic, the economy needs as much
help as it can get. In March, Congress passed and the president signed
the CARES Act, which provided more than \$2 trillion in economic
stimulus. Studies show that it has had a remarkable effect --- despite a
steep increase in unemployment due to the virus, the expansion in aid
\href{https://www.nytimes.com/2020/06/21/us/politics/coronavirus-poverty.html}{prevented
a rise in poverty}.

But
\href{https://www.nytimes.com/2020/06/21/us/politics/coronavirus-poverty.html}{most
of that stimulus will soon come to an end}. Congress is working on
\href{https://www.cnn.com/2020/07/21/politics/stimulus-negotiations-congress-latest/index.html}{another
relief package}, but already lawmakers are fighting about its size:
Democrats in the House
p\href{https://www.nytimes.com/2020/05/15/us/politics/house-simulus-vote.html}{assed
a \$3 trillion bill}; Trump and Senate Republicans are looking at
something closer to
\href{https://www.cbsnews.com/news/coronavirus-relief-bill-phase-4-trump-republicans-1-trillion/}{\$1
trillion}.

Near the end of his Keynes book, Carter writes that Keynesianism ``is
not so much a school of economic thought as a spirit of radical
optimism, unjustified by most of human history and extremely difficult
to conjure up precisely when it is most needed: during the depths of a
depression or amid the fevers of war.''

We are in similarly dire straits now --- and one way we might escape is
to do what Keynes would suggest we do: spend our way toward a better
tomorrow.

\hypertarget{office-hours-with-farhad-manjoo}{%
\subsection{Office Hours With Farhad
Manjoo}\label{office-hours-with-farhad-manjoo}}

\emph{Farhad wants to}
\href{https://www.nytimes.com/2019/05/16/opinion/farhad-office-hours.html?module=inline}{\emph{chat
with readers on the phone}}\emph{. If you're interested in talking to a
New York Times columnist about anything that's on your mind, please fill
out this form. Farhad will select a few readers to call.}

\emph{The Times is committed to publishing}
\href{https://www.nytimes.com/2019/01/31/opinion/letters/letters-to-editor-new-york-times-women.html}{\emph{a
diversity of letters}} \emph{to the editor. We'd like to hear what you
think about this or any of our articles. Here are some}
\href{https://help.nytimes.com/hc/en-us/articles/115014925288-How-to-submit-a-letter-to-the-editor}{\emph{tips}}\emph{.
And here's our email:}
\href{mailto:letters@nytimes.com}{\emph{letters@nytimes.com}}\emph{.}

\emph{Follow The New York Times Opinion section on}
\href{https://www.facebook.com/nytopinion}{\emph{Facebook}}\emph{,}
\href{http://twitter.com/NYTOpinion}{\emph{Twitter (@NYTopinion)}}
\emph{and}
\href{https://www.instagram.com/nytopinion/}{\emph{Instagram}}\emph{.}

Advertisement

\protect\hyperlink{after-bottom}{Continue reading the main story}

\hypertarget{site-index}{%
\subsection{Site Index}\label{site-index}}

\hypertarget{site-information-navigation}{%
\subsection{Site Information
Navigation}\label{site-information-navigation}}

\begin{itemize}
\tightlist
\item
  \href{https://help.nytimes.com/hc/en-us/articles/115014792127-Copyright-notice}{©~2020~The
  New York Times Company}
\end{itemize}

\begin{itemize}
\tightlist
\item
  \href{https://www.nytco.com/}{NYTCo}
\item
  \href{https://help.nytimes.com/hc/en-us/articles/115015385887-Contact-Us}{Contact
  Us}
\item
  \href{https://www.nytco.com/careers/}{Work with us}
\item
  \href{https://nytmediakit.com/}{Advertise}
\item
  \href{http://www.tbrandstudio.com/}{T Brand Studio}
\item
  \href{https://www.nytimes.com/privacy/cookie-policy\#how-do-i-manage-trackers}{Your
  Ad Choices}
\item
  \href{https://www.nytimes.com/privacy}{Privacy}
\item
  \href{https://help.nytimes.com/hc/en-us/articles/115014893428-Terms-of-service}{Terms
  of Service}
\item
  \href{https://help.nytimes.com/hc/en-us/articles/115014893968-Terms-of-sale}{Terms
  of Sale}
\item
  \href{https://spiderbites.nytimes.com}{Site Map}
\item
  \href{https://help.nytimes.com/hc/en-us}{Help}
\item
  \href{https://www.nytimes.com/subscription?campaignId=37WXW}{Subscriptions}
\end{itemize}
