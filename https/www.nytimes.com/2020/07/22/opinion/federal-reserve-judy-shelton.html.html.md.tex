Sections

SEARCH

\protect\hyperlink{site-content}{Skip to
content}\protect\hyperlink{site-index}{Skip to site index}

\href{https://myaccount.nytimes.com/auth/login?response_type=cookie\&client_id=vi}{}

\href{https://www.nytimes.com/section/todayspaper}{Today's Paper}

\href{/section/opinion}{Opinion}\textbar{}God Help Us if Judy Shelton
Joins the Fed

\href{https://nyti.ms/2BkFSFs}{https://nyti.ms/2BkFSFs}

\begin{itemize}
\item
\item
\item
\item
\item
\item
\end{itemize}

Advertisement

\protect\hyperlink{after-top}{Continue reading the main story}

\href{/section/opinion}{Opinion}

Supported by

\protect\hyperlink{after-sponsor}{Continue reading the main story}

\hypertarget{god-help-us-if-judy-shelton-joins-the-fed}{%
\section{God Help Us if Judy Shelton Joins the
Fed}\label{god-help-us-if-judy-shelton-joins-the-fed}}

Trump's latest unqualified nominee to the Federal Reserve Board must be
rejected.

\href{https://www.nytimes.com/topic/person/steven-rattner}{\includegraphics{https://static01.nyt.com/images/2013/10/04/opinion/rattner-contributor/rattner-contributor-thumbLarge-v5.png}}

By \href{https://www.nytimes.com/topic/person/steven-rattner}{Steven
Rattner}

Mr. Rattner served as counselor to the Treasury secretary in the Obama
administration.

\begin{itemize}
\item
  July 22, 2020
\item
  \begin{itemize}
  \item
  \item
  \item
  \item
  \item
  \item
  \end{itemize}
\end{itemize}

\includegraphics{https://static01.nyt.com/images/2020/07/23/opinion/23rattnerWeb/merlin_168878430_3f58db13-fa3e-4532-ab5a-bc3991567210-articleLarge.jpg?quality=75\&auto=webp\&disable=upscale}

Having failed in past attempts to put unqualified ideologues on the
Federal Reserve Board, President Trump is giving it another try --- and
is closer to victory than previously.

The nominee in question --- Judy Shelton, known for taking
long-discredited positions on the monetary system --- makes Mr. Trump's
earlier
\href{https://www.nytimes.com/2019/05/02/business/stephen-moore-fed.html}{rejected
choices} seem almost conventional. Among other heretical stances, she
has supported the abolition of the Federal Reserve itself, putting her
in a position to undermine the very institution she is being nominated
to serve.

``Why do we need a central bank?'' Ms. Shelton
\href{https://www.wsj.com/articles/SB123811225716453243}{asked in} a
Wall Street Journal essay in 2009. She wants monetary policy set by the
price of gold, a long-abandoned approach that would be akin to a Supreme
Court justice embracing the Code of Hammurabi.

Regrettably, after much hesitation and with evident reluctance even from
Republicans, the Senate Banking Committee
\href{https://www.nytimes.com/2020/07/21/business/economy/shelton-federal-reserve-trump-senate.html}{voted
Tuesday to advance} Ms. Shelton's nomination to the full Senate. We
mustn't let her nomination become overshadowed by the many other
daunting challenges we face at the moment. When her name reaches the
full Senate floor, four Republicans must find the courage to join the
Democrats in voting no and rebuffing her appointment.

The Federal Reserve is an indispensable player in managing our economy.
Period. It has also, commendably, remained largely free of partisanship.
The nominees of past presidents, Democrats and Republicans alike, have
chosen to work collegially and without personal agendas to fulfill its
critical mission.

Now, as he has done so often elsewhere in the government, Mr. Trump is
doing his best to politicize this remarkable institution.

For starters, had Ms. Shelton's prescriptions been followed, the Fed's
response to the arrival of the virus would have been disastrously wrong
instead of extraordinarily constructive.

Her view that interest rates should be ``rules based'' would have
prevented the central bank's emergency cuts.

Her past opposition to the Fed buying bonds to help stimulate the
economy --- as it did successfully during the 2008 financial crisis ---
would have prevented the central bank from standing up many of the
rescue programs that are now helping to keep the economy afloat.

Her notion that the Fed must consult with Congress,
\href{https://www.washingtonpost.com/business/2019/11/21/trumps-fed-nominee-judy-shelton-recently-questioned-need-an-independent-central-bank/}{rather
than act independently} as is considered the best practice among
developed countries, would have introduced damaging delays, politics
and, likely, policy misfires as ill-equipped members of Congress tried
to grapple with the intricacies of monetary policy.

Then there's the gold standard, a significant culprit in deepening the
Great Depression and abandoned decades ago by every country in the world
(including the United States in 1973). By rigidly fixing prices to a
single commodity, a gold standard exaggerates economic swings, on
balance for the worse.

Between 1880 and 1933, the United States experienced at least five
full-fledged banking crises; in the past 87 years, we've had two. Though
promoted as smoothing price movements, a gold standard in fact magnifies
them, as a comparison of the pre-Depression period to the post-World War
II era makes clear.

In a 2012 poll,
\href{http://www.igmchicago.org/surveys/gold-standard/}{not one of 40
prominent economists} supported disinterring this misguided policy.

A few other weird ideas from Ms. Shelton: She has questioned the
accuracy of government statistics. She wants a single currency for North
America. (Does she not know how badly the euro has worked?)

On at least two existential issues, Ms. Shelton has shown a willingness
to not let principles stand in the way of career advancement. Until her
confirmation hearing,
\href{https://www.washingtonpost.com/business/2020/02/13/trumps-fed-nominee-judy-shelton-could-be-trouble-key-gop-senators-express-concerns-about-her-outlier-views/}{she
backed} getting rid of federal deposit insurance, a key protection for
individual savers. Her long opposition to low-interest rates
notwithstanding,
\href{https://www.washingtonpost.com/business/2019/06/19/fed-meets-trumps-potential-next-pick-wants-see-lower-rates-fast-possible/}{last
year she flip-flopped} to Mr. Trump's view that low rates are, in fact,
a great idea.

Concern within the Senate Banking Committee was obvious during its
protracted consideration. ``Nobody wants anybody on the Federal Reserve
that has a fatal attraction to nutty ideas,'' John Kennedy, Republican
of Louisiana,
\href{https://www.wsj.com/articles/republican-senator-raises-concerns-over-sheltons-fed-candidacy-11581608467}{said
in February}. But then, like so many Republicans unwilling to cross a
revengeful president, Mr. Kennedy capitulated.

To be sure, one iconoclastic and outspoken member of a seven-person
board (who are part of a 12-member committee that sets interest rates)
may not change the Fed's decisions. But if Mr. Trump wins re-election,
he will have the chance to nominate a new chair of the Fed when Jerome
Powell's term expires in 2022.

Although he appointed Mr. Powell to the chairmanship, at times since
then the president has
\href{https://www.nytimes.com/2019/06/24/business/economy/federal-reserve-trump.html?action=click\&module=Intentional\&pgtype=Article}{taken
to Twitter and other forums} to assail him for raising rates in December
2018 and for taking too long to lower them in 2019.

God help us if the next chair is Ms. Shelton or anyone else with her
views. Senate Republicans must recognize this danger and show some
backbone.

\emph{The Times is committed to publishing}
\href{https://www.nytimes.com/2019/01/31/opinion/letters/letters-to-editor-new-york-times-women.html}{\emph{a
diversity of letters}} \emph{to the editor. We'd like to hear what you
think about this or any of our articles. Here are some}
\href{https://help.nytimes.com/hc/en-us/articles/115014925288-How-to-submit-a-letter-to-the-editor}{\emph{tips}}\emph{.
And here's our email:}
\href{mailto:letters@nytimes.com}{\emph{letters@nytimes.com}}\emph{.}

\emph{Follow The New York Times Opinion section on}
\href{https://www.facebook.com/nytopinion}{\emph{Facebook}}\emph{,}
\href{http://twitter.com/NYTOpinion}{\emph{Twitter (@NYTopinion)}}
\emph{and}
\href{https://www.instagram.com/nytopinion/}{\emph{Instagram}}\emph{.}

Advertisement

\protect\hyperlink{after-bottom}{Continue reading the main story}

\hypertarget{site-index}{%
\subsection{Site Index}\label{site-index}}

\hypertarget{site-information-navigation}{%
\subsection{Site Information
Navigation}\label{site-information-navigation}}

\begin{itemize}
\tightlist
\item
  \href{https://help.nytimes.com/hc/en-us/articles/115014792127-Copyright-notice}{©~2020~The
  New York Times Company}
\end{itemize}

\begin{itemize}
\tightlist
\item
  \href{https://www.nytco.com/}{NYTCo}
\item
  \href{https://help.nytimes.com/hc/en-us/articles/115015385887-Contact-Us}{Contact
  Us}
\item
  \href{https://www.nytco.com/careers/}{Work with us}
\item
  \href{https://nytmediakit.com/}{Advertise}
\item
  \href{http://www.tbrandstudio.com/}{T Brand Studio}
\item
  \href{https://www.nytimes.com/privacy/cookie-policy\#how-do-i-manage-trackers}{Your
  Ad Choices}
\item
  \href{https://www.nytimes.com/privacy}{Privacy}
\item
  \href{https://help.nytimes.com/hc/en-us/articles/115014893428-Terms-of-service}{Terms
  of Service}
\item
  \href{https://help.nytimes.com/hc/en-us/articles/115014893968-Terms-of-sale}{Terms
  of Sale}
\item
  \href{https://spiderbites.nytimes.com}{Site Map}
\item
  \href{https://help.nytimes.com/hc/en-us}{Help}
\item
  \href{https://www.nytimes.com/subscription?campaignId=37WXW}{Subscriptions}
\end{itemize}
