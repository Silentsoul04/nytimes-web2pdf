Sections

SEARCH

\protect\hyperlink{site-content}{Skip to
content}\protect\hyperlink{site-index}{Skip to site index}

\href{https://www.nytimes.com/podcasts/the-daily}{The Daily}

\href{https://myaccount.nytimes.com/auth/login?response_type=cookie\&client_id=vi}{}

\href{https://www.nytimes.com/section/todayspaper}{Today's Paper}

\href{/podcasts/the-daily}{The Daily}\textbar{}The Science of School
Reopenings

\url{https://nyti.ms/2OMEtKY}

\begin{itemize}
\item
\item
\item
\item
\item
\item
\end{itemize}

Advertisement

\protect\hyperlink{after-top}{Continue reading the main story}

transcript

Back to The Daily

bars

0:00/27:24

-27:24

transcript

\hypertarget{the-science-of-school-reopenings}{%
\subsection{The Science of School
Reopenings}\label{the-science-of-school-reopenings}}

\hypertarget{hosted-by-michael-barbaro-produced-by-clare-toeniskoetter-and-alexandra-leigh-young-with-help-from-rachel-quester-and-edited-by-mj-davis-lin-and-lisa-tobin}{%
\subsubsection{Hosted by Michael Barbaro; produced by Clare
Toeniskoetter and Alexandra Leigh Young; with help from Rachel Quester;
and edited by M.J. Davis Lin and Lisa
Tobin}\label{hosted-by-michael-barbaro-produced-by-clare-toeniskoetter-and-alexandra-leigh-young-with-help-from-rachel-quester-and-edited-by-mj-davis-lin-and-lisa-tobin}}

\hypertarget{several-countries-have-found-ways-to-reopen-schools-safely-but-can-the-united-states}{%
\paragraph{Several countries have found ways to reopen schools safely.
But can the United
States?}\label{several-countries-have-found-ways-to-reopen-schools-safely-but-can-the-united-states}}

Wednesday, July 22nd, 2020

\begin{itemize}
\item
  michael barbaro\\
  From The New York Times, I'm Michael Barbaro. This is ``The Daily.''

  Today: Across the world, no country with infection rates as high as
  the U.S. has tried to reopen schools. Pam Belluck on the potential
  risks and rewards of that plan.

  It's Wednesday, July 22.

  Pam, where does the United States officially stand on reopening
  schools in the fall?
\item
  pam belluck\\
  So officially, the Trump administration has been saying in recent
  weeks that it really wants schools to open.
\item
  archived recording\\
  Well, good morning all. The White House Coronavirus Task Force met
  today here at the Department of Education.
\end{itemize}

pam belluck

There was a press conference earlier this month where Vice President
Pence and a string of administration officials were basically saying ---

\begin{itemize}
\tightlist
\item
  archived recording (mike pence)\\
  It's absolutely essential that we get our kids back into classroom for
  in-person learning.
\end{itemize}

pam belluck

--- schools should open.

\begin{itemize}
\tightlist
\item
  archived recording (betsy devos)\\
  Ultimately, it's not a matter of if schools should reopen. It's simply
  a matter of how.
\end{itemize}

pam belluck

They should do so at the beginning of the school year.

\begin{itemize}
\tightlist
\item
  archived recording (betsy devos)\\
  They must fully open. And they must be fully operational.
\end{itemize}

pam belluck

Basically, what they're saying is ---

\begin{itemize}
\tightlist
\item
  archived recording (dr. robert redfield)\\
  What is not the intent of C.D.C.`s guidelines is to be used as a
  rationale to keep schools closed.
\end{itemize}

pam belluck

Health concerns, safety concerns --- none of that should get in the way
of reopening schools this fall.

michael barbaro

Right. And I watched that news conference. And the message was very
clear. And it was very unified. And I'm curious if it had the intended
effect of making school districts across the country say, oh, OK, well,
that's what we'll do.

pam belluck

It really didn't. I think it was alarming for a number of school
districts and certainly for public health experts.

\begin{itemize}
\tightlist
\item
  archived recording (randi weingarten)\\
  This nonsense this week of politicizing schools and politicizing
  health care and the well-being of kids was destructive and reckless.
\end{itemize}

pam belluck

They were saying, hey, we all agree that it's a really good goal to open
schools. But you can't just press a button and say, presto, school's in
session.

\begin{itemize}
\tightlist
\item
  archived recording (randi weingarten)\\
  Don't be reckless with our kids and our teachers.
\end{itemize}

pam belluck

And actually, there's been sort of a turn in the other direction in the
last few days, where, increasingly, the large school districts, anyway,
have said, we really don't think we're going to be equipped to open in
person in September.

michael barbaro

Hm.

\begin{itemize}
\tightlist
\item
  archived recording (randi weingarten)\\
  It's hard to teach kids anyway. This is the hardest thing we've ever
  done in our lives. But we've got to do it together. And what the
  president did was just reckless and, frankly, destructive.
\end{itemize}

michael barbaro

So here, you have the federal government saying, do this and do it now,
and local school districts starting to say, no, we don't think that's a
great idea. And of course that's why we talk to science reporters like
you, Pam. Because you can help us negotiate these two poles.

pam belluck

In theory.

michael barbaro

So I wondered if you can explain what the science is starting to tell us
about this question of reopening.

pam belluck

Yeah. So in a way, there's two types of science here. So there's the
science of why children should be back in school, and then there's the
science of whether and how they should get back to school. And the
science of why they should be back in school is really that it's so
important, particularly for younger children, to have an in-person
educational experience, to be able to interact with peers, to be able to
have face-to-face communication with teachers. And unfortunately, at
least what we think in the short term, the legacy of the pandemic so far
has been that online learning has not been successful for many children,
particularly young children. And also, schools serve really important
functions for children's mental health, for their social development.
They're really kind of the lifeblood of the community in more ways than
just 2 plus 2 equals 4. So that's the first type of science.

The second type of science, the science of whether schools should
reopen, looks at the virus, how it spreads and who can spread it. Now,
this science is preliminary right now. Nothing is 100 percent certain.
But there are three things we can be pretty sure about.

One: Children do not get sick with coronavirus as often as adults.
Number two: When they do get sick, they are much less likely to get
seriously ill. The data seems to show that about 2 percent of kids who
are getting infected are getting very sick, we think. So that's a good
sign. And three: And this is less definitive, but there's growing
evidence that younger children --- say, age 10 and under --- are less
likely to spread the virus than older children are.

michael barbaro

The idea being that little kids don't transmit the disease as often.

pam belluck

Yeah. One study that suggests this was a study that was done in France
in a community where two teachers in a high school got very sick while
school was still in session back in February. And then the researchers
went and tested the students and teachers and staff in that high school
for antibodies to the coronavirus to indicate whether they had been
infected. And they found that about 40 percent of the students and
teachers had been infected with Covid-19. Now, that's a pretty high
rate. And it tells you that the virus was really circulating in that
high school while school was in session.

Then they went to six elementary schools in the community, and they did
the same testing of students and teachers and staff. And they found much
lower rates. Only about 9 percent of kids and about 7 percent of
teachers came back positive with antibodies for the coronavirus.

michael barbaro

Wow, much lower.

pam belluck

Yeah. That's significantly lower. And they didn't find any evidence that
the students who were infected actually infected other people. So this
suggests, researchers think, that little kids are less likely to spread
the coronavirus to other people.

michael barbaro

Which would seem to mean --- although I understand it's just one study,
but to the degree it tells us something, it tells us that perhaps
elementary schools would be a safer choice to reopen than, for example,
middle and high schools.

pam belluck

Exactly. Which is really nice to know, if that's true. Because, of
course, it's the younger students who are much more in need of the
in-person instruction and much less able to handle online instruction
independently. And there's another kind of set of anecdotes and some
data that also builds that whole idea. And some of that comes from the
United States. Daycare centers, a number of them stayed open during the
pandemic, especially for children of essential workers. And so far,
there have been very few outbreaks that have occurred at those daycare
centers.

michael barbaro

Pam, I'm curious what the science tells us about countries that have
actually begun to reopen their schools. What are we seeing so far?

pam belluck

So let's look at it country by country. There are some countries that
have had very successful school reopenings. And there are some countries
that did not do so well. So let's start with the successes. The best
examples are probably Norway and Denmark. In Denmark, they brought only
the younger students back first. They had them eat lunch separately.
They had their desks six feet apart. They had lots of cleaning and
hand-washing. And they had them in small groups. So kids were in groups
of maybe 12 students and one teacher.

michael barbaro

So they kind of created small little cohorts that would limit exposure.

pam belluck

Exactly.

michael barbaro

You only expose one of the 12 people around you, not the entire class.

pam belluck

Exactly. And some people were calling them ``pods,'' some people were
calling them ``bubbles.'' This is a kind of main feature of what public
health experts are suggesting for schools. Because it not only limits
the number of kids that a single kid could infect and the number of
teachers and that kind of thing, but it makes your contact tracing very
easy. If one of those kids get sick, you know all the suspects, you
know, who might have either infected that kid or been infected by that
kid. And you don't have to necessarily close your entire school to deal
with that case or two of Covid-19. You can just say, hey, these 12 kids
from first grade, you'll have to be at home for the next two weeks, but
the rest of the school can go on.

michael barbaro

Mm-hmm. And so what do the infection rates look like inside schools in
Denmark and Norway. Is it working?

pam belluck

It worked out really well. They have had no outbreaks reported in
schools. They have had no increase in their cases in their community.
And they ended up being able to bring their older kids back to school
later on as well. So they are kind of the models.

michael barbaro

OK, so before we get our hopes up, what countries have been less
successful and maybe even failed?

pam belluck

So I think one of the countries that has had some issues has been
Israel, which reopened schools. And you'd think they would be set up
pretty well, because they didn't have a lot of cases in their community.
They started school in early May. They started with classes in small
groups. I think they called them ``capsules.'' But then, within a couple
of weeks, they relaxed the class size restrictions. And that appears to
have been too soon. Because not long after that, they ended up having
outbreaks in something like 130 schools ---

michael barbaro

Wow.

pam belluck

--- and 240-something positive tests among students and teachers. So
they ended up having to tighten things up again.

michael barbaro

So Israel just moved too fast.

pam belluck

Yeah.

michael barbaro

But help me understand something. If kids are not great transmitters,
and kids tend not to get serious infections, what does it mean to have
200 kids in a country get infected in a school? Is that even so
worrisome?

pam belluck

Well, it's a really good question. I think we don't really know the
answer to that fully. But ideally, you want to try to limit your cases
as much as possible, because every kid is going to have contact with
concentric circles of other people. And if they're able to spread it to
just one person, and that person can then spread it to other people who
are a lot more vulnerable, then the risk just increases and increases.
And that's what we don't want to happen.

michael barbaro

Right.

pam belluck

And there were also problems in Sweden. And Sweden is kind of the
example of a country that never closed its schools. And for them, they
didn't take any real precautions in society either. So they had a couple
of teachers and staff members die in schools.

michael barbaro

Wow.

pam belluck

They did end up having to close at least one school, because there were
so many staff members that got sick. And we don't have a lot of really
good data, because they didn't do a lot of testing. So we don't really
know how many students got infected. But we do have some data where it
looks like there was a relatively kind of high rate of children and
teenagers who they did test, who were positive for Covid-19 antibodies.
So at least it suggests that it certainly was present in schools and
could have caused some other infections in their country.

{[}music{]}

michael barbaro

One other pattern I've noticed, Pam, is that with the exception of
Sweden, the countries that you've mentioned that have reopened, they've
all pretty much had the virus under control. And although it's been a
kind of mixed bag, it feels like the results are very much tied to the
fact that these countries didn't have massive outbreaks.

pam belluck

Definitely. I think that is one of the main ingredients that public
health experts say that you need, is to try to get the virus under
control in your community before you throw open the doors of schools.
And here in the United States, we have the administration wanting to
kind of rush into opening schools. But we don't have it under control.
We've had record numbers of cases in recent weeks. And so we've never
really had a country that has tried to do this with the kind of
unbridled, out-of-control spread that we have now in the United States.

michael barbaro

We'll be right back.

So Pam, with those lessons from overseas in mind, and acknowledging the
reality that the Covid-19 crisis in the U.S. is pretty unique and
pervasive, how are schools in the U.S. starting to plan for the fall and
the possibility of reopening?

pam belluck

So the C.D.C. has outlined steps that schools can take. And they include
a lot of the things that we have seen in some of these other countries,
like keeping desks six feet apart, and lots of handwashing, and having
good ventilation, ideally, and having cloth masks that most teachers and
students wear. So we're getting some guidance from the federal health
authorities. But this is the United States. And most school decisions
are made at a very local level. So what you have is this really messy
patchwork of school districts across the country trying to figure out
what they are going to do and what they can do safely.

And they're looking at things like, how big is their school building?
How good is the ventilation in their school building? If they keep
students at home for online learning, how many of their students don't
really have good internet access, and that's not going to work for them?
How vulnerable are their teachers? All sorts of things like that. It is
a very complicated situation. And it's going to be different in every
single community.

michael barbaro

Mm-hmm. I wonder if you can give us some specific examples of how,
within this patchwork, different communities across the country are
approaching this.

pam belluck

Sure. So you have New York City, which is by far the largest school
district in the country. And they came out about a week or two ago
saying that they were going to try a plan that would be called sort of a
hybrid plan. They're going to try to bring students back to school one
to three days a week. And the rest of the time, there will be online
instruction. They're placing some emphasis on bringing back students
with special needs, because those are considered to be the kids who are
most vulnerable and really, really need to have in-person instruction,
as well as the younger kids.

But then you have other big school districts, like Los Angeles and San
Diego and Houston and Atlanta, Nashville. They have all decided just
very recently that they cannot make it work safely in person right away.
So they are at least going to be starting their school year with
exclusively online instruction.

michael barbaro

Pam, when we talk about reopening --- this is true of our conversation
so far --- we tend to talk mostly about students. So I want to talk
about teachers. Inevitably, the risk to an adult teacher of catching the
coronavirus in a school would seem significantly higher than a student.
So how much do all these plans we're talking about take teachers into
account?

pam belluck

Yeah, absolutely. They are definitely at much higher risk. And you have
a lot of teachers who are, you know, in their 50s and 60s who are in
more vulnerable age groups. I know a lot of the districts are surveying
their teachers and finding that a lot of their teachers are very
concerned. And some of them are saying they won't go back into the
classroom unless there are certain precautions taken that they feel make
it safe for them. And then there are other teachers who are very eager
to get back into the classroom, because they really value being able to
teach kids in person.

michael barbaro

When I think about that guidance from the Trump administration that we
started this conversation with, that very emphatic encouragement to
physically reopen schools, it's really interesting. Because embedded in
that is the assumption that teachers would show up and do that work,
which really means asking teachers whether they want to or not, whether
they're reluctant or eager to act a bit like the frontline workers we
think about when we think about nurses or police.

pam belluck

They are the frontline workers, yeah. They are really caught at the
crossroads of this. Because they see the value of going back into the
schools. They want to go back into the schools. They also want to keep
themselves and their families safe. So I think there's a range of
voices. My impression is that school districts are listening to teachers
just as they are other constituencies and, in fact, kind of reminding
their parents that it's not just about the students, but that it is also
about the adults who teach their students.

michael barbaro

So Pam, I know you may not have in front of you a map of the entire
country and a list of every school district. But in general, from what
you can tell, is the United States leaning towards physical reopening,
remote learning, or some very frustrating sense of indecision and flux?
Where's the majority of the nation's school system at this moment?

pam belluck

A week ago, I would have answered that question by saying that I thought
most school districts were going to try to do some type of in-person
instruction. And maybe a lot of them wouldn't get there 100 percent, but
they were at least going to try for half and half.

But in the past week, we have seen these major school districts and
states say, we are not ready. Too many people have Covid-19 in our
state. And we cannot take the risk. And so we have to at least start the
school year the same way we ended the school year, with online
instruction only. I think there's just so much uncertainty about getting
this under control in the larger community. And schools are not islands.
Schools are part of the fabric of the community. What happens in the
community is reflected in the school.

michael barbaro

Mm-hmm. Right. You're saying that the fear is that even the best systems
we could possibly put in place in U.S. schools, those would be
undermined by the prevalence of the virus in the communities where the
schools would be reopening. It may not amount to much if the entire
community around that school is saturated with the coronavirus, which is
pretty much the story of many communities in the U.S. right now.

pam belluck

It is pretty much the story in many communities. So yes, I think it's a
very precarious situation. And most experts would say, get your
community under control and then open your schools slowly,
incrementally, with lots of safeguards in place. And then you'll have a
good formula for keeping things under control.

michael barbaro

So knowing we are not there with this under control, where does that
leave schools, school districts, teachers, parents?

pam belluck

I think you are going to have a lot of very stressed out students,
parents and teachers for at least the beginning of the school year.
Maybe it'll light a fire under communities and get places where they
weren't wearing masks and they weren't social distancing to take that
seriously. What better goal could there be than getting things together
so that your school can open safely?

michael barbaro

Hm. You're saying that, potentially, the best argument for the entire
American society to change its approach to this is so that we can open
schools.

pam belluck

Yeah.

{[}music{]}

You can definitely see an argument that it's much more important to be
able to get your school open than to open your bars, open your bowling
alleys, open your fitness centers.

michael barbaro

Although it hasn't been framed that way.

pam belluck

It has not been framed that way.

michael barbaro

Pam, thank you very much.

pam belluck

Thank you.

\begin{itemize}
\tightlist
\item
  archived recording (donald trump)\\
  Thank you very much. And good afternoon. Today, I want to provide an
  update on our response to the China virus.
\end{itemize}

michael barbaro

Weeks after he ended regular briefings about the coronavirus, President
Trump resumed them on Tuesday with a rare acknowledgment of how serious
the situation has become in the U.S.

\begin{itemize}
\tightlist
\item
  archived recording (donald trump)\\
  It will probably, unfortunately, get worse before it gets better.
  Something I don't like saying about things. But that's the way it is.
\end{itemize}

michael barbaro

Trump appeared without key members of his coronavirus task force, like
Dr. Deborah Birx and Dr. Anthony Fauci, who spoke during his previous
briefings. But the president embraced their advice about wearing masks.

\begin{itemize}
\tightlist
\item
  archived recording (donald trump)\\
  We're asking everybody that, when you are not able to socially
  distance, wear a mask. Get a mask. Whether you like the mask or not,
  they have an impact. They'll have an effect. And we need everything we
  can get.
\end{itemize}

michael barbaro

We'll be right back.

{[}music{]}

Here's what else you need to know today. Senate Republicans outlined
their latest economic relief package on Tuesday, which includes billions
of dollars for schools, direct payments to families and a fresh round of
funding for small businesses hurt by the pandemic. Congress faces
intense pressure to pass a new relief bill, since benefits passed in the
first round of stimulus, like enhanced pay of \$600 a week for those who
lost their jobs, will expire at the end of the month. Senate Republicans
have said that they plan to scale back those payments.

That's it for ``The Daily.'' I'm Michael Barbaro. See you tomorrow.

\href{https://www.nytimes.com/column/the-daily}{\includegraphics{https://static01.nyt.com/images/2017/01/29/podcasts/the-daily-album-art/the-daily-album-art-square320-v4.png}The
Daily}Subscribe:

\begin{itemize}
\tightlist
\item
  \href{https://itunes.apple.com/us/podcast/id1200361736}{Apple
  Podcasts}
\item
  \href{https://www.google.com/podcasts?feed=aHR0cHM6Ly9yc3MuYXJ0MTkuY29tL3RoZS1kYWlseQ\%3D\%3D}{Google
  Podcasts}
\end{itemize}

\hypertarget{the-science-of-school-reopenings-1}{%
\section{The Science of School
Reopenings}\label{the-science-of-school-reopenings-1}}

\hypertarget{several-countries-have-found-ways-to-reopen-schools-safely-but-can-the-united-states-1}{%
\subsection{Several countries have found ways to reopen schools safely.
But can the United
States?}\label{several-countries-have-found-ways-to-reopen-schools-safely-but-can-the-united-states-1}}

Hosted by Michael Barbaro; produced by Clare Toeniskoetter and Alexandra
Leigh Young; with help from Rachel Quester; and edited by M.J. Davis Lin
and Lisa Tobin

Transcript

transcript

Back to The Daily

bars

0:00/27:24

-0:00

transcript

\hypertarget{the-science-of-school-reopenings-2}{%
\subsection{The Science of School
Reopenings}\label{the-science-of-school-reopenings-2}}

\hypertarget{hosted-by-michael-barbaro-produced-by-clare-toeniskoetter-and-alexandra-leigh-young-with-help-from-rachel-quester-and-edited-by-mj-davis-lin-and-lisa-tobin-1}{%
\subsubsection{Hosted by Michael Barbaro; produced by Clare
Toeniskoetter and Alexandra Leigh Young; with help from Rachel Quester;
and edited by M.J. Davis Lin and Lisa
Tobin}\label{hosted-by-michael-barbaro-produced-by-clare-toeniskoetter-and-alexandra-leigh-young-with-help-from-rachel-quester-and-edited-by-mj-davis-lin-and-lisa-tobin-1}}

\hypertarget{several-countries-have-found-ways-to-reopen-schools-safely-but-can-the-united-states-2}{%
\paragraph{Several countries have found ways to reopen schools safely.
But can the United
States?}\label{several-countries-have-found-ways-to-reopen-schools-safely-but-can-the-united-states-2}}

Wednesday, July 22nd, 2020

\begin{itemize}
\item
  michael barbaro\\
  From The New York Times, I'm Michael Barbaro. This is ``The Daily.''

  Today: Across the world, no country with infection rates as high as
  the U.S. has tried to reopen schools. Pam Belluck on the potential
  risks and rewards of that plan.

  It's Wednesday, July 22.

  Pam, where does the United States officially stand on reopening
  schools in the fall?
\item
  pam belluck\\
  So officially, the Trump administration has been saying in recent
  weeks that it really wants schools to open.
\item
  archived recording\\
  Well, good morning all. The White House Coronavirus Task Force met
  today here at the Department of Education.
\end{itemize}

pam belluck

There was a press conference earlier this month where Vice President
Pence and a string of administration officials were basically saying ---

\begin{itemize}
\tightlist
\item
  archived recording (mike pence)\\
  It's absolutely essential that we get our kids back into classroom for
  in-person learning.
\end{itemize}

pam belluck

--- schools should open.

\begin{itemize}
\tightlist
\item
  archived recording (betsy devos)\\
  Ultimately, it's not a matter of if schools should reopen. It's simply
  a matter of how.
\end{itemize}

pam belluck

They should do so at the beginning of the school year.

\begin{itemize}
\tightlist
\item
  archived recording (betsy devos)\\
  They must fully open. And they must be fully operational.
\end{itemize}

pam belluck

Basically, what they're saying is ---

\begin{itemize}
\tightlist
\item
  archived recording (dr. robert redfield)\\
  What is not the intent of C.D.C.`s guidelines is to be used as a
  rationale to keep schools closed.
\end{itemize}

pam belluck

Health concerns, safety concerns --- none of that should get in the way
of reopening schools this fall.

michael barbaro

Right. And I watched that news conference. And the message was very
clear. And it was very unified. And I'm curious if it had the intended
effect of making school districts across the country say, oh, OK, well,
that's what we'll do.

pam belluck

It really didn't. I think it was alarming for a number of school
districts and certainly for public health experts.

\begin{itemize}
\tightlist
\item
  archived recording (randi weingarten)\\
  This nonsense this week of politicizing schools and politicizing
  health care and the well-being of kids was destructive and reckless.
\end{itemize}

pam belluck

They were saying, hey, we all agree that it's a really good goal to open
schools. But you can't just press a button and say, presto, school's in
session.

\begin{itemize}
\tightlist
\item
  archived recording (randi weingarten)\\
  Don't be reckless with our kids and our teachers.
\end{itemize}

pam belluck

And actually, there's been sort of a turn in the other direction in the
last few days, where, increasingly, the large school districts, anyway,
have said, we really don't think we're going to be equipped to open in
person in September.

michael barbaro

Hm.

\begin{itemize}
\tightlist
\item
  archived recording (randi weingarten)\\
  It's hard to teach kids anyway. This is the hardest thing we've ever
  done in our lives. But we've got to do it together. And what the
  president did was just reckless and, frankly, destructive.
\end{itemize}

michael barbaro

So here, you have the federal government saying, do this and do it now,
and local school districts starting to say, no, we don't think that's a
great idea. And of course that's why we talk to science reporters like
you, Pam. Because you can help us negotiate these two poles.

pam belluck

In theory.

michael barbaro

So I wondered if you can explain what the science is starting to tell us
about this question of reopening.

pam belluck

Yeah. So in a way, there's two types of science here. So there's the
science of why children should be back in school, and then there's the
science of whether and how they should get back to school. And the
science of why they should be back in school is really that it's so
important, particularly for younger children, to have an in-person
educational experience, to be able to interact with peers, to be able to
have face-to-face communication with teachers. And unfortunately, at
least what we think in the short term, the legacy of the pandemic so far
has been that online learning has not been successful for many children,
particularly young children. And also, schools serve really important
functions for children's mental health, for their social development.
They're really kind of the lifeblood of the community in more ways than
just 2 plus 2 equals 4. So that's the first type of science.

The second type of science, the science of whether schools should
reopen, looks at the virus, how it spreads and who can spread it. Now,
this science is preliminary right now. Nothing is 100 percent certain.
But there are three things we can be pretty sure about.

One: Children do not get sick with coronavirus as often as adults.
Number two: When they do get sick, they are much less likely to get
seriously ill. The data seems to show that about 2 percent of kids who
are getting infected are getting very sick, we think. So that's a good
sign. And three: And this is less definitive, but there's growing
evidence that younger children --- say, age 10 and under --- are less
likely to spread the virus than older children are.

michael barbaro

The idea being that little kids don't transmit the disease as often.

pam belluck

Yeah. One study that suggests this was a study that was done in France
in a community where two teachers in a high school got very sick while
school was still in session back in February. And then the researchers
went and tested the students and teachers and staff in that high school
for antibodies to the coronavirus to indicate whether they had been
infected. And they found that about 40 percent of the students and
teachers had been infected with Covid-19. Now, that's a pretty high
rate. And it tells you that the virus was really circulating in that
high school while school was in session.

Then they went to six elementary schools in the community, and they did
the same testing of students and teachers and staff. And they found much
lower rates. Only about 9 percent of kids and about 7 percent of
teachers came back positive with antibodies for the coronavirus.

michael barbaro

Wow, much lower.

pam belluck

Yeah. That's significantly lower. And they didn't find any evidence that
the students who were infected actually infected other people. So this
suggests, researchers think, that little kids are less likely to spread
the coronavirus to other people.

michael barbaro

Which would seem to mean --- although I understand it's just one study,
but to the degree it tells us something, it tells us that perhaps
elementary schools would be a safer choice to reopen than, for example,
middle and high schools.

pam belluck

Exactly. Which is really nice to know, if that's true. Because, of
course, it's the younger students who are much more in need of the
in-person instruction and much less able to handle online instruction
independently. And there's another kind of set of anecdotes and some
data that also builds that whole idea. And some of that comes from the
United States. Daycare centers, a number of them stayed open during the
pandemic, especially for children of essential workers. And so far,
there have been very few outbreaks that have occurred at those daycare
centers.

michael barbaro

Pam, I'm curious what the science tells us about countries that have
actually begun to reopen their schools. What are we seeing so far?

pam belluck

So let's look at it country by country. There are some countries that
have had very successful school reopenings. And there are some countries
that did not do so well. So let's start with the successes. The best
examples are probably Norway and Denmark. In Denmark, they brought only
the younger students back first. They had them eat lunch separately.
They had their desks six feet apart. They had lots of cleaning and
hand-washing. And they had them in small groups. So kids were in groups
of maybe 12 students and one teacher.

michael barbaro

So they kind of created small little cohorts that would limit exposure.

pam belluck

Exactly.

michael barbaro

You only expose one of the 12 people around you, not the entire class.

pam belluck

Exactly. And some people were calling them ``pods,'' some people were
calling them ``bubbles.'' This is a kind of main feature of what public
health experts are suggesting for schools. Because it not only limits
the number of kids that a single kid could infect and the number of
teachers and that kind of thing, but it makes your contact tracing very
easy. If one of those kids get sick, you know all the suspects, you
know, who might have either infected that kid or been infected by that
kid. And you don't have to necessarily close your entire school to deal
with that case or two of Covid-19. You can just say, hey, these 12 kids
from first grade, you'll have to be at home for the next two weeks, but
the rest of the school can go on.

michael barbaro

Mm-hmm. And so what do the infection rates look like inside schools in
Denmark and Norway. Is it working?

pam belluck

It worked out really well. They have had no outbreaks reported in
schools. They have had no increase in their cases in their community.
And they ended up being able to bring their older kids back to school
later on as well. So they are kind of the models.

michael barbaro

OK, so before we get our hopes up, what countries have been less
successful and maybe even failed?

pam belluck

So I think one of the countries that has had some issues has been
Israel, which reopened schools. And you'd think they would be set up
pretty well, because they didn't have a lot of cases in their community.
They started school in early May. They started with classes in small
groups. I think they called them ``capsules.'' But then, within a couple
of weeks, they relaxed the class size restrictions. And that appears to
have been too soon. Because not long after that, they ended up having
outbreaks in something like 130 schools ---

michael barbaro

Wow.

pam belluck

--- and 240-something positive tests among students and teachers. So
they ended up having to tighten things up again.

michael barbaro

So Israel just moved too fast.

pam belluck

Yeah.

michael barbaro

But help me understand something. If kids are not great transmitters,
and kids tend not to get serious infections, what does it mean to have
200 kids in a country get infected in a school? Is that even so
worrisome?

pam belluck

Well, it's a really good question. I think we don't really know the
answer to that fully. But ideally, you want to try to limit your cases
as much as possible, because every kid is going to have contact with
concentric circles of other people. And if they're able to spread it to
just one person, and that person can then spread it to other people who
are a lot more vulnerable, then the risk just increases and increases.
And that's what we don't want to happen.

michael barbaro

Right.

pam belluck

And there were also problems in Sweden. And Sweden is kind of the
example of a country that never closed its schools. And for them, they
didn't take any real precautions in society either. So they had a couple
of teachers and staff members die in schools.

michael barbaro

Wow.

pam belluck

They did end up having to close at least one school, because there were
so many staff members that got sick. And we don't have a lot of really
good data, because they didn't do a lot of testing. So we don't really
know how many students got infected. But we do have some data where it
looks like there was a relatively kind of high rate of children and
teenagers who they did test, who were positive for Covid-19 antibodies.
So at least it suggests that it certainly was present in schools and
could have caused some other infections in their country.

{[}music{]}

michael barbaro

One other pattern I've noticed, Pam, is that with the exception of
Sweden, the countries that you've mentioned that have reopened, they've
all pretty much had the virus under control. And although it's been a
kind of mixed bag, it feels like the results are very much tied to the
fact that these countries didn't have massive outbreaks.

pam belluck

Definitely. I think that is one of the main ingredients that public
health experts say that you need, is to try to get the virus under
control in your community before you throw open the doors of schools.
And here in the United States, we have the administration wanting to
kind of rush into opening schools. But we don't have it under control.
We've had record numbers of cases in recent weeks. And so we've never
really had a country that has tried to do this with the kind of
unbridled, out-of-control spread that we have now in the United States.

michael barbaro

We'll be right back.

So Pam, with those lessons from overseas in mind, and acknowledging the
reality that the Covid-19 crisis in the U.S. is pretty unique and
pervasive, how are schools in the U.S. starting to plan for the fall and
the possibility of reopening?

pam belluck

So the C.D.C. has outlined steps that schools can take. And they include
a lot of the things that we have seen in some of these other countries,
like keeping desks six feet apart, and lots of handwashing, and having
good ventilation, ideally, and having cloth masks that most teachers and
students wear. So we're getting some guidance from the federal health
authorities. But this is the United States. And most school decisions
are made at a very local level. So what you have is this really messy
patchwork of school districts across the country trying to figure out
what they are going to do and what they can do safely.

And they're looking at things like, how big is their school building?
How good is the ventilation in their school building? If they keep
students at home for online learning, how many of their students don't
really have good internet access, and that's not going to work for them?
How vulnerable are their teachers? All sorts of things like that. It is
a very complicated situation. And it's going to be different in every
single community.

michael barbaro

Mm-hmm. I wonder if you can give us some specific examples of how,
within this patchwork, different communities across the country are
approaching this.

pam belluck

Sure. So you have New York City, which is by far the largest school
district in the country. And they came out about a week or two ago
saying that they were going to try a plan that would be called sort of a
hybrid plan. They're going to try to bring students back to school one
to three days a week. And the rest of the time, there will be online
instruction. They're placing some emphasis on bringing back students
with special needs, because those are considered to be the kids who are
most vulnerable and really, really need to have in-person instruction,
as well as the younger kids.

But then you have other big school districts, like Los Angeles and San
Diego and Houston and Atlanta, Nashville. They have all decided just
very recently that they cannot make it work safely in person right away.
So they are at least going to be starting their school year with
exclusively online instruction.

michael barbaro

Pam, when we talk about reopening --- this is true of our conversation
so far --- we tend to talk mostly about students. So I want to talk
about teachers. Inevitably, the risk to an adult teacher of catching the
coronavirus in a school would seem significantly higher than a student.
So how much do all these plans we're talking about take teachers into
account?

pam belluck

Yeah, absolutely. They are definitely at much higher risk. And you have
a lot of teachers who are, you know, in their 50s and 60s who are in
more vulnerable age groups. I know a lot of the districts are surveying
their teachers and finding that a lot of their teachers are very
concerned. And some of them are saying they won't go back into the
classroom unless there are certain precautions taken that they feel make
it safe for them. And then there are other teachers who are very eager
to get back into the classroom, because they really value being able to
teach kids in person.

michael barbaro

When I think about that guidance from the Trump administration that we
started this conversation with, that very emphatic encouragement to
physically reopen schools, it's really interesting. Because embedded in
that is the assumption that teachers would show up and do that work,
which really means asking teachers whether they want to or not, whether
they're reluctant or eager to act a bit like the frontline workers we
think about when we think about nurses or police.

pam belluck

They are the frontline workers, yeah. They are really caught at the
crossroads of this. Because they see the value of going back into the
schools. They want to go back into the schools. They also want to keep
themselves and their families safe. So I think there's a range of
voices. My impression is that school districts are listening to teachers
just as they are other constituencies and, in fact, kind of reminding
their parents that it's not just about the students, but that it is also
about the adults who teach their students.

michael barbaro

So Pam, I know you may not have in front of you a map of the entire
country and a list of every school district. But in general, from what
you can tell, is the United States leaning towards physical reopening,
remote learning, or some very frustrating sense of indecision and flux?
Where's the majority of the nation's school system at this moment?

pam belluck

A week ago, I would have answered that question by saying that I thought
most school districts were going to try to do some type of in-person
instruction. And maybe a lot of them wouldn't get there 100 percent, but
they were at least going to try for half and half.

But in the past week, we have seen these major school districts and
states say, we are not ready. Too many people have Covid-19 in our
state. And we cannot take the risk. And so we have to at least start the
school year the same way we ended the school year, with online
instruction only. I think there's just so much uncertainty about getting
this under control in the larger community. And schools are not islands.
Schools are part of the fabric of the community. What happens in the
community is reflected in the school.

michael barbaro

Mm-hmm. Right. You're saying that the fear is that even the best systems
we could possibly put in place in U.S. schools, those would be
undermined by the prevalence of the virus in the communities where the
schools would be reopening. It may not amount to much if the entire
community around that school is saturated with the coronavirus, which is
pretty much the story of many communities in the U.S. right now.

pam belluck

It is pretty much the story in many communities. So yes, I think it's a
very precarious situation. And most experts would say, get your
community under control and then open your schools slowly,
incrementally, with lots of safeguards in place. And then you'll have a
good formula for keeping things under control.

michael barbaro

So knowing we are not there with this under control, where does that
leave schools, school districts, teachers, parents?

pam belluck

I think you are going to have a lot of very stressed out students,
parents and teachers for at least the beginning of the school year.
Maybe it'll light a fire under communities and get places where they
weren't wearing masks and they weren't social distancing to take that
seriously. What better goal could there be than getting things together
so that your school can open safely?

michael barbaro

Hm. You're saying that, potentially, the best argument for the entire
American society to change its approach to this is so that we can open
schools.

pam belluck

Yeah.

{[}music{]}

You can definitely see an argument that it's much more important to be
able to get your school open than to open your bars, open your bowling
alleys, open your fitness centers.

michael barbaro

Although it hasn't been framed that way.

pam belluck

It has not been framed that way.

michael barbaro

Pam, thank you very much.

pam belluck

Thank you.

\begin{itemize}
\tightlist
\item
  archived recording (donald trump)\\
  Thank you very much. And good afternoon. Today, I want to provide an
  update on our response to the China virus.
\end{itemize}

michael barbaro

Weeks after he ended regular briefings about the coronavirus, President
Trump resumed them on Tuesday with a rare acknowledgment of how serious
the situation has become in the U.S.

\begin{itemize}
\tightlist
\item
  archived recording (donald trump)\\
  It will probably, unfortunately, get worse before it gets better.
  Something I don't like saying about things. But that's the way it is.
\end{itemize}

michael barbaro

Trump appeared without key members of his coronavirus task force, like
Dr. Deborah Birx and Dr. Anthony Fauci, who spoke during his previous
briefings. But the president embraced their advice about wearing masks.

\begin{itemize}
\tightlist
\item
  archived recording (donald trump)\\
  We're asking everybody that, when you are not able to socially
  distance, wear a mask. Get a mask. Whether you like the mask or not,
  they have an impact. They'll have an effect. And we need everything we
  can get.
\end{itemize}

michael barbaro

We'll be right back.

{[}music{]}

Here's what else you need to know today. Senate Republicans outlined
their latest economic relief package on Tuesday, which includes billions
of dollars for schools, direct payments to families and a fresh round of
funding for small businesses hurt by the pandemic. Congress faces
intense pressure to pass a new relief bill, since benefits passed in the
first round of stimulus, like enhanced pay of \$600 a week for those who
lost their jobs, will expire at the end of the month. Senate Republicans
have said that they plan to scale back those payments.

That's it for ``The Daily.'' I'm Michael Barbaro. See you tomorrow.

Previous

More episodes ofThe Daily

\href{https://www.nytimes.com/2020/07/31/podcasts/the-daily/vanessa-guillen-military-metoo.html?action=click\&module=audio-series-bar\&region=header\&pgtype=Article}{\includegraphics{https://static01.nyt.com/images/2020/07/12/us/politics/31daily/00dc-army-metoo-thumbLarge.jpg}}

July 31, 2020A \#MeToo Moment in the Military

\href{https://www.nytimes.com/2020/07/30/podcasts/the-daily/congress-facebook-amazon-google-apple.html?action=click\&module=audio-series-bar\&region=header\&pgtype=Article}{\includegraphics{https://static01.nyt.com/images/2020/07/30/reader-center/30daily/merlin_175077825_5ebc931b-baa1-489a-960c-34e4d845e997-thumbLarge.jpg}}

July 30, 2020The Big Tech Hearing

\href{https://www.nytimes.com/2020/07/29/podcasts/the-daily/china-trump-foreign-policy.html?action=click\&module=audio-series-bar\&region=header\&pgtype=Article}{\includegraphics{https://static01.nyt.com/images/2020/07/26/world/29daily/00china-us-clash1-thumbLarge.jpg}}

July 29, 2020~~•~ 28:40Confronting China

\href{https://www.nytimes.com/2020/07/28/podcasts/the-daily/unemployment-benefits-coronavirus.html?action=click\&module=audio-series-bar\&region=header\&pgtype=Article}{\includegraphics{https://static01.nyt.com/images/2020/07/23/business/28daily/23virus-uiexplain1-thumbLarge.jpg}}

July 28, 2020~~•~ 26:13Why \$600 Checks Are Tearing Republicans Apart

\href{https://www.nytimes.com/2020/07/27/podcasts/the-daily/new-york-hospitals-covid.html?action=click\&module=audio-series-bar\&region=header\&pgtype=Article}{\includegraphics{https://static01.nyt.com/images/2020/07/27/world/27daily-hospitals/27daily-hospitals-thumbLarge.jpg}}

July 27, 2020~~•~ 33:28The Mistakes New York Made

\href{https://www.nytimes.com/2020/07/26/podcasts/the-daily/the-accusation-the-sunday-read.html?action=click\&module=audio-series-bar\&region=header\&pgtype=Article}{\includegraphics{https://static01.nyt.com/images/2020/03/22/magazine/26audm-2/22mag-titleix-thumbLarge.jpg}}

July 26, 2020The Sunday Read: `The Accusation'

\href{https://www.nytimes.com/2020/07/24/podcasts/the-daily/mlb-baseball-season-coronavirus.html?action=click\&module=audio-series-bar\&region=header\&pgtype=Article}{\includegraphics{https://static01.nyt.com/images/2020/07/22/sports/24daily/22mlb-previewlede1-thumbLarge.jpg}}

July 24, 2020~~•~ 45:34The Battle for a Baseball Season

\href{https://www.nytimes.com/2020/07/23/podcasts/the-daily/portland-protests.html?action=click\&module=audio-series-bar\&region=header\&pgtype=Article}{\includegraphics{https://static01.nyt.com/images/2020/07/22/us/23daily-image/22portland-tactics02-thumbLarge.jpg}}

July 23, 2020~~•~ 30:04The Showdown in Portland

\href{https://www.nytimes.com/2020/07/22/podcasts/the-daily/school-reopenings-coronavirus.html?action=click\&module=audio-series-bar\&region=header\&pgtype=Article}{\includegraphics{https://static01.nyt.com/images/2020/07/12/science/22daily/00virus-schools-reopen01-thumbLarge.jpg}}

July 22, 2020~~•~ 27:24The Science of School Reopenings

\href{https://www.nytimes.com/2020/07/21/podcasts/the-daily/coronavirus-vaccine.html?action=click\&module=audio-series-bar\&region=header\&pgtype=Article}{\includegraphics{https://static01.nyt.com/images/2020/07/19/science/21daily/00VIRUS-VAX-DOUBTS1-thumbLarge.jpg}}

July 21, 2020~~•~ 29:14The Vaccine Trust Problem

\href{https://www.nytimes.com/2020/07/20/podcasts/the-daily/john-lewis.html?action=click\&module=audio-series-bar\&region=header\&pgtype=Article}{\includegraphics{https://static01.nyt.com/images/2020/01/07/obituaries/20thedaily_lewis/00Lewis-John13-thumbLarge.jpg}}

July 20, 2020~~•~ 38:56The Life and Legacy of John Lewis

\href{https://www.nytimes.com/2020/07/19/podcasts/the-daily/lottery-winner-scam.html?action=click\&module=audio-series-bar\&region=header\&pgtype=Article}{\includegraphics{https://static01.nyt.com/images/2018/05/05/magazine/31audm-image/05mag-lottery-image1-thumbLarge-v4.png}}

July 19, 2020~~•~ 45:27The Sunday Read: `The Man Who Cracked the
Lottery'

\href{https://www.nytimes.com/column/the-daily}{See All Episodes ofThe
Daily}

Next

July 22, 2020

\begin{itemize}
\item
\item
\item
\item
\item
\item
\end{itemize}

\emph{\textbf{Listen and subscribe to our podcast from your mobile
device:}}\\
\textbf{\href{https://itunes.apple.com/us/podcast/the-daily/id1200361736?mt=2}{\emph{Via
Apple Podcasts}}} \emph{\textbf{\textbar{}}}
\textbf{\href{https://open.spotify.com/show/3IM0lmZxpFAY7CwMuv9H4g?si=SfuMSC55R1qprFsRZU3_zw}{\emph{Via
Spotify}}} \emph{\textbf{\textbar{}}}
\textbf{\href{http://www.stitcher.com/podcast/the-new-york-times/the-daily-10}{\emph{Via
Stitcher}}}

Around the world, safely reopening schools remains one of the most
daunting challenges to restarting national economies. While approaches
have been different, no country has tried to reopen schools with
coronavirus infection rates at the level of the United States. Today, we
explore the risks and rewards of the plan to reopen American schools
this fall.

\textbf{On today's episode:}

\begin{itemize}
\tightlist
\item
  \href{https://www.nytimes.com/by/pam-belluck}{Pam Belluck}, a health
  and science writer at The New York Times.
\end{itemize}

\includegraphics{https://static01.nyt.com/images/2020/07/12/science/22daily/merlin_170865825_2993c63a-7bb5-4ae4-853c-2f355b29af24-articleLarge.jpg?quality=75\&auto=webp\&disable=upscale}

\textbf{Background reading:}

\begin{itemize}
\item
  The pressure to bring American students back to classrooms is intense,
  but the calculus is tricky with
  \href{https://www.nytimes.com/2020/07/11/health/coronavirus-schools-reopen.html}{infections
  still out of control in many communities}.
\item
  Local economies might not fully recover until working parents can send
  children to school.
  \href{https://www.nytimes.com/2020/07/06/nyregion/nyc-school-reopening-plan.html}{Here's
  why the plan} to reopen New York City schools is so important.
\end{itemize}

\emph{Tune in, and tell us what you think. Email us at}
\href{mailto:thedaily@nytimes.com}{\emph{thedaily@nytimes.com}}\emph{.
Follow Michael Barbaro on Twitter:}
\href{https://twitter.com/mikiebarb}{\emph{@mikiebarb}}\emph{. And if
you're interested in advertising with ``The Daily,'' write to us at}
\href{mailto:thedaily-ads@nytimes.com}{\emph{thedaily-ads@nytimes.com}}\emph{.}

Pam Belluck contributed reporting.

``The Daily'' is made by Theo Balcomb, Andy Mills, Lisa Tobin, Rachel
Quester, Lynsea Garrison, Annie Brown, Clare Toeniskoetter, Paige
Cowett, Michael Simon Johnson, Brad Fisher, Larissa Anderson, Wendy
Dorr, Chris Wood, Jessica Cheung, Stella Tan, Alexandra Leigh Young,
Jonathan Wolfe, Lisa Chow, Eric Krupke, Marc Georges, Luke Vander Ploeg,
Adizah Eghan, Kelly Prime, Julia Longoria, Sindhu Gnanasambandan, M.J.
Davis Lin, Austin Mitchell, Sayre Quevedo, Neena Pathak, Dan Powell,
Dave Shaw, Sydney Harper, Daniel Guillemette, Hans Buetow, Robert
Jimison, Mike Benoist, Bianca Giaever and Asthaa Chaturvedi. Our theme
music is by Jim Brunberg and Ben Landsverk of Wonderly. Special thanks
to Sam Dolnick, Mikayla Bouchard, Lauren Jackson, Julia Simon, Mahima
Chablani and Nora Keller.

Advertisement

\protect\hyperlink{after-bottom}{Continue reading the main story}

\hypertarget{site-index}{%
\subsection{Site Index}\label{site-index}}

\hypertarget{site-information-navigation}{%
\subsection{Site Information
Navigation}\label{site-information-navigation}}

\begin{itemize}
\tightlist
\item
  \href{https://help.nytimes.com/hc/en-us/articles/115014792127-Copyright-notice}{©~2020~The
  New York Times Company}
\end{itemize}

\begin{itemize}
\tightlist
\item
  \href{https://www.nytco.com/}{NYTCo}
\item
  \href{https://help.nytimes.com/hc/en-us/articles/115015385887-Contact-Us}{Contact
  Us}
\item
  \href{https://www.nytco.com/careers/}{Work with us}
\item
  \href{https://nytmediakit.com/}{Advertise}
\item
  \href{http://www.tbrandstudio.com/}{T Brand Studio}
\item
  \href{https://www.nytimes.com/privacy/cookie-policy\#how-do-i-manage-trackers}{Your
  Ad Choices}
\item
  \href{https://www.nytimes.com/privacy}{Privacy}
\item
  \href{https://help.nytimes.com/hc/en-us/articles/115014893428-Terms-of-service}{Terms
  of Service}
\item
  \href{https://help.nytimes.com/hc/en-us/articles/115014893968-Terms-of-sale}{Terms
  of Sale}
\item
  \href{https://spiderbites.nytimes.com}{Site Map}
\item
  \href{https://help.nytimes.com/hc/en-us}{Help}
\item
  \href{https://www.nytimes.com/subscription?campaignId=37WXW}{Subscriptions}
\end{itemize}
