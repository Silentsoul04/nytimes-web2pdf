Sections

SEARCH

\protect\hyperlink{site-content}{Skip to
content}\protect\hyperlink{site-index}{Skip to site index}

\href{https://www.nytimes.com/section/science}{Science}

\href{https://myaccount.nytimes.com/auth/login?response_type=cookie\&client_id=vi}{}

\href{https://www.nytimes.com/section/todayspaper}{Today's Paper}

\href{/section/science}{Science}\textbar{}Vampire Bats Self-Isolate, Too

\url{https://nyti.ms/30CTqoq}

\begin{itemize}
\item
\item
\item
\item
\item
\end{itemize}

Advertisement

\protect\hyperlink{after-top}{Continue reading the main story}

Supported by

\protect\hyperlink{after-sponsor}{Continue reading the main story}

Trilobites

\hypertarget{vampire-bats-self-isolate-too}{%
\section{Vampire Bats Self-Isolate,
Too}\label{vampire-bats-self-isolate-too}}

When these mammals are ill, they have fewer interactions with family and
friends, a new study suggests. ``It's like us,'' said one researcher.

\includegraphics{https://static01.nyt.com/images/2020/07/21/science/21TB-BATS/merlin_174804618_3761fee8-5224-4e5e-b356-d314a61fa360-articleLarge.jpg?quality=75\&auto=webp\&disable=upscale}

\href{https://www.nytimes.com/by/david-waldstein}{\includegraphics{https://static01.nyt.com/images/2018/02/20/multimedia/author-david-waldstein/author-david-waldstein-thumbLarge.jpg}}

By \href{https://www.nytimes.com/by/david-waldstein}{David Waldstein}

\begin{itemize}
\item
  July 22, 2020
\item
  \begin{itemize}
  \item
  \item
  \item
  \item
  \item
  \end{itemize}
\end{itemize}

Vampire bats, those bloodsucking, flying critters with razor-sharp
teeth, are rather social beings. They love grooming one another and
\href{https://www.nytimes.com/2020/03/19/science/vampire-bats-blood.html}{sharing
food supplies}, which consists of regurgitated blood from some other
unfortunate mammal. These bats also call out to one another when they're
apart from their group.

But when they're ill, they call out less frequently and have fewer
interactions with family and friends, new research suggests.

In 2020, such behavior sounds a lot like social distancing. But the
scientists do not think the bats' self-isolation is intentional.
Publishing their findings
\href{https://royalsocietypublishing.org/doi/10.1098/rsbl.2020.0272}{last
week in Biology Letters}, the researchers believe that when bats are
ill, they just have trouble mustering up the energy to call out.

``It's like us,'' said Sebastian Stockmaier, a doctoral candidate at the
University of Texas, Austin, who led the study. ``When they are sick and
feeling bad, they are not interested in social interactions.''

Mr. Stockmaier and his fellow researchers say it is much like that
miserable lethargy you feel when an illness settles in and all you want
to do is lie in bed.

The researchers found that on average, when vampire bats are feeling
sick, they call out 30 percent less frequently than when they are
healthy. And whether intentional or not, it should have a beneficial
side effect of limiting the spread of whatever pathogen is afflicting
them.

``If they are sick, they groom others less,'' Mr. Stockmaier said, ``and
that will theoretically reduce disease transmission.''

To measure this, the scientists went to the Smithsonian Tropical
Research Institute in Panama, where vampire bats abound. They are
generally found in Central and South America and feed off the blood of
mammals, like cattle and horses.

While many people might recoil from the sight of a vampire bat ---
especially the
\href{https://www.nytimes.com/2016/11/01/science/fangs.html}{terrifying
close-ups of their bared teeth} --- Mr. Stockmaier calls them ``cute.''

Finding, catching and keeping them in captivity is not hard, Mr.
Stockmaier said, ``if you know where to get blood.'' (His team gets all
it needs from local slaughterhouses.)

For the experiment, the scientists injected 18 female bats once with
lipopolysaccharide (LPS), a compound that induces an immune response
similar to a bacterial infection, without actually causing the illness
--- or threat of infection --- in the bat. It usually lasts between 24
and 48 hours. Females were used because they are more social than males,
engaging more often in grooming and communal feeding and maintaining
bonds with their offspring for long periods.

The researchers later injected the same group of female bats with saline
solution as a control. In both cases, they removed the bats from the
larger group --- but within hearing distance --- and recorded and
measured their calls.

They found that, on average, the bats made 30 percent fewer calls, with
15 of 18 recording fewer calls compared with the control group.

In another study, Mr. Stockmaier said, the researchers discovered that
bats injected with LPS produced symptoms of illness, slept more, moved
around less and performed less social grooming. He also noted that
previous studies have shown that many similar animals require eight
times more energy to call out than not to call out.

So, they concluded that it is more likely that the bats are just feeling
too lousy to call out, rather than intentionally stifling themselves as
a naturally selected, personal sacrifice to prevent pathogen
transmission to the group at large.

Mr. Stockmaier laments that ``bats are getting a lot of bad press right
now,'' mainly because it is widely believed that the new coronavirus,
which causes Covid-19, originally jumped from horseshoe bats. He is
quick to point out that is a different species from vampire bats, and
that all of them offer something unique to study.

``I love bats,'' he said. ``I think they are fascinating animals.''

Advertisement

\protect\hyperlink{after-bottom}{Continue reading the main story}

\hypertarget{site-index}{%
\subsection{Site Index}\label{site-index}}

\hypertarget{site-information-navigation}{%
\subsection{Site Information
Navigation}\label{site-information-navigation}}

\begin{itemize}
\tightlist
\item
  \href{https://help.nytimes.com/hc/en-us/articles/115014792127-Copyright-notice}{©~2020~The
  New York Times Company}
\end{itemize}

\begin{itemize}
\tightlist
\item
  \href{https://www.nytco.com/}{NYTCo}
\item
  \href{https://help.nytimes.com/hc/en-us/articles/115015385887-Contact-Us}{Contact
  Us}
\item
  \href{https://www.nytco.com/careers/}{Work with us}
\item
  \href{https://nytmediakit.com/}{Advertise}
\item
  \href{http://www.tbrandstudio.com/}{T Brand Studio}
\item
  \href{https://www.nytimes.com/privacy/cookie-policy\#how-do-i-manage-trackers}{Your
  Ad Choices}
\item
  \href{https://www.nytimes.com/privacy}{Privacy}
\item
  \href{https://help.nytimes.com/hc/en-us/articles/115014893428-Terms-of-service}{Terms
  of Service}
\item
  \href{https://help.nytimes.com/hc/en-us/articles/115014893968-Terms-of-sale}{Terms
  of Sale}
\item
  \href{https://spiderbites.nytimes.com}{Site Map}
\item
  \href{https://help.nytimes.com/hc/en-us}{Help}
\item
  \href{https://www.nytimes.com/subscription?campaignId=37WXW}{Subscriptions}
\end{itemize}
