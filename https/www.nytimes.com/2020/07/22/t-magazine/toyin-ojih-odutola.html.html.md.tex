Sections

SEARCH

\protect\hyperlink{site-content}{Skip to
content}\protect\hyperlink{site-index}{Skip to site index}

\href{https://myaccount.nytimes.com/auth/login?response_type=cookie\&client_id=vi}{}

\href{https://www.nytimes.com/section/todayspaper}{Today's Paper}

A Calming Meditation on Nigeria's Jos Plateau

\url{https://nyti.ms/2CBjFni}

\begin{itemize}
\item
\item
\item
\item
\item
\end{itemize}

Advertisement

\protect\hyperlink{after-top}{Continue reading the main story}

Supported by

\protect\hyperlink{after-sponsor}{Continue reading the main story}

True Believers

\hypertarget{a-calming-meditation-on-nigerias-jos-plateau}{%
\section{A Calming Meditation on Nigeria's Jos
Plateau}\label{a-calming-meditation-on-nigerias-jos-plateau}}

Toyin Ojih Odutola, known for her figurative portraits and vibrant use
of color, shares a new artwork.

\includegraphics{https://static01.nyt.com/images/2020/07/20/t-magazine/art/20tmag-artists-slide-7NNZ/20tmag-artists-slide-7NNZ-articleLarge.jpg?quality=75\&auto=webp\&disable=upscale}

By Toyin Ojih Odutola

\begin{itemize}
\item
  July 22, 2020
\item
  \begin{itemize}
  \item
  \item
  \item
  \item
  \item
  \end{itemize}
\end{itemize}

\emph{In this new series, The Artists, an installment of which will
publish every day this week and regularly thereafter, T will highlight a
recent or little-shown work by a Black artist, along with a few words
from that artist, putting the work into context. Today, we're looking at
a piece by} \href{https://toyinojihodutola.com/}{\emph{Toyin Ojih
Odutola}}\emph{, who is known for her figurative portraits and vibrant
use of color.}

\href{https://www.nytimes.com/issue/t-magazine/2020/07/02/true-believers-art-issue}{\includegraphics{https://static01.nyt.com/newsgraphics/2020/06/29/tmag-art-embeds-new/assets/images/art_issue_gif_special_editon.gif}}

\textbf{Name:} Toyin Ojih Odutola

\textbf{Age:} 35

\textbf{Based in:} Brooklyn, N.Y.

\textbf{Originally from:} Ile-Ife, Nigeria

\textbf{When and where did you make this work?} I was very fortunate to
create a makeshift studio in my apartment to work on smaller drawings
during quarantine in New York, around April and May.

\textbf{Can you describe what's going on in it?} As I look at it now, I
still don't know. I think it's too close. All I can express is how I was
feeling when I made it: a need to draw something calm, contained and
steady. After recently completing {[}the site-specific installation{]}
``\href{https://www.barbican.org.uk/our-story/press-room/toyin-ojih-odutola-a-countervailing-theory}{A
Countervailing Theory},'' which involved researching the Jos Plateau
region of Nigeria, I captured the rock formations famously known there
in a framed portrait within the drawing. The work was made using colored
pencil and graphite, with the Jos Plateau rocks drawn recto verso of the
Dura-Lar film sheet, illustrating a distance in the frontal view with a
reveal in the back. As if to say, ``This place that consumed me for over
a year, which I have yet to visit, is still in my mind, holding court in
a special way.'' I normally don't like to leave works simply untitled,
so it's called ``Untitled (Jos)'' (2020).

\textbf{What inspired you to make this work?} Anxiety and uncertainty
while being in quarantine as the maelstrom of the
\href{https://www.nytimes.com/news-event/coronavirus}{coronavirus} was
taking over. Also, having to contend with a world once known and the
expectations attached to it coming to an end. This feeling of being in
the midst of history-making all around us and not knowing where
everything is heading, like yearning for a place you've never been to
--- needing to see that in some form, somehow.

\textbf{What's the work of art in any medium that changed your life?} It
depends on the project I'm working on. Influences are diverse, they come
and go. If I'm being honest, anything challenging staid perceptions of
the world --- or my own ideas --- through a visual language that is
engaging and questioning you as a viewer always draws me in. I'm
constantly learning and unlearning ways of application and seeing,
encountering new forms, fumbling in my attempts to understand.
Regardless of medium, it's exciting to come across works of art that can
help you expand what is possible.

\hypertarget{true-believers-art-issue}{%
\subsubsection{\texorpdfstring{\href{https://www.nytimes.com/issue/t-magazine/2020/07/02/true-believers-art-issue}{True
Believers Art
Issue}}{True Believers Art Issue}}\label{true-believers-art-issue}}

Advertisement

\protect\hyperlink{after-bottom}{Continue reading the main story}

\hypertarget{site-index}{%
\subsection{Site Index}\label{site-index}}

\hypertarget{site-information-navigation}{%
\subsection{Site Information
Navigation}\label{site-information-navigation}}

\begin{itemize}
\tightlist
\item
  \href{https://help.nytimes.com/hc/en-us/articles/115014792127-Copyright-notice}{©~2020~The
  New York Times Company}
\end{itemize}

\begin{itemize}
\tightlist
\item
  \href{https://www.nytco.com/}{NYTCo}
\item
  \href{https://help.nytimes.com/hc/en-us/articles/115015385887-Contact-Us}{Contact
  Us}
\item
  \href{https://www.nytco.com/careers/}{Work with us}
\item
  \href{https://nytmediakit.com/}{Advertise}
\item
  \href{http://www.tbrandstudio.com/}{T Brand Studio}
\item
  \href{https://www.nytimes.com/privacy/cookie-policy\#how-do-i-manage-trackers}{Your
  Ad Choices}
\item
  \href{https://www.nytimes.com/privacy}{Privacy}
\item
  \href{https://help.nytimes.com/hc/en-us/articles/115014893428-Terms-of-service}{Terms
  of Service}
\item
  \href{https://help.nytimes.com/hc/en-us/articles/115014893968-Terms-of-sale}{Terms
  of Sale}
\item
  \href{https://spiderbites.nytimes.com}{Site Map}
\item
  \href{https://help.nytimes.com/hc/en-us}{Help}
\item
  \href{https://www.nytimes.com/subscription?campaignId=37WXW}{Subscriptions}
\end{itemize}
