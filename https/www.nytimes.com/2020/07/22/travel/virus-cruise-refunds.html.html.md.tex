Sections

SEARCH

\protect\hyperlink{site-content}{Skip to
content}\protect\hyperlink{site-index}{Skip to site index}

\href{https://www.nytimes.com/section/travel}{Travel}

\href{https://myaccount.nytimes.com/auth/login?response_type=cookie\&client_id=vi}{}

\href{https://www.nytimes.com/section/todayspaper}{Today's Paper}

\href{/section/travel}{Travel}\textbar{}Help! My Ship Is Supposedly
Still Sailing, and I Don't Want to Be On Board

\url{https://nyti.ms/2WJAljk}

\begin{itemize}
\item
\item
\item
\item
\item
\item
\end{itemize}

\href{https://www.nytimes.com/news-event/coronavirus?action=click\&pgtype=Article\&state=default\&region=TOP_BANNER\&context=storylines_menu}{The
Coronavirus Outbreak}

\begin{itemize}
\tightlist
\item
  live\href{https://www.nytimes.com/2020/08/01/world/coronavirus-covid-19.html?action=click\&pgtype=Article\&state=default\&region=TOP_BANNER\&context=storylines_menu}{Latest
  Updates}
\item
  \href{https://www.nytimes.com/interactive/2020/us/coronavirus-us-cases.html?action=click\&pgtype=Article\&state=default\&region=TOP_BANNER\&context=storylines_menu}{Maps
  and Cases}
\item
  \href{https://www.nytimes.com/interactive/2020/science/coronavirus-vaccine-tracker.html?action=click\&pgtype=Article\&state=default\&region=TOP_BANNER\&context=storylines_menu}{Vaccine
  Tracker}
\item
  \href{https://www.nytimes.com/interactive/2020/07/29/us/schools-reopening-coronavirus.html?action=click\&pgtype=Article\&state=default\&region=TOP_BANNER\&context=storylines_menu}{What
  School May Look Like}
\item
  \href{https://www.nytimes.com/live/2020/07/31/business/stock-market-today-coronavirus?action=click\&pgtype=Article\&state=default\&region=TOP_BANNER\&context=storylines_menu}{Economy}
\end{itemize}

Advertisement

\protect\hyperlink{after-top}{Continue reading the main story}

Supported by

\protect\hyperlink{after-sponsor}{Continue reading the main story}

Tripped Up

\hypertarget{help-my-ship-is-supposedly-still-sailing-and-i-dont-want-to-be-on-board}{%
\section{Help! My Ship Is Supposedly Still Sailing, and I Don't Want to
Be On
Board}\label{help-my-ship-is-supposedly-still-sailing-and-i-dont-want-to-be-on-board}}

The future of the cruise industry remains very unclear, so it's not
totally unreasonable to be anxious about what next spring will look
like.

\includegraphics{https://static01.nyt.com/images/2020/07/25/travel/25TrippedUp/24TrippedUp-articleLarge.jpg?quality=75\&auto=webp\&disable=upscale}

By Sarah Firshein

\begin{itemize}
\item
  July 22, 2020
\item
  \begin{itemize}
  \item
  \item
  \item
  \item
  \item
  \item
  \end{itemize}
\end{itemize}

\hypertarget{dear-tripped-up}{%
\subsubsection{\texorpdfstring{\textbf{Dear Tripped
Up,}}{Dear Tripped Up,}}\label{dear-tripped-up}}

I'm booked on a \href{https://www.cunard.com/en-us}{Cunard Line}
trans-Atlantic cruise in May 2021. The reservation was made using credit
from an April sailing that was canceled due to Covid-19. Because I
canceled my reservation a week before Cunard itself canceled the
sailing, I was not given the opportunity to get a refund.

But I am almost 80 and I feel it is unfair for me to be locked into
traveling next spring when I am so fearful of the coronavirus. I feel as
though I've been penalized for canceling a voyage that was canceled
anyway. Based on what you know about the cruise industry and its
response to the pandemic, what should I do? Susan

\hypertarget{dear-susan}{%
\subsubsection{Dear Susan,}\label{dear-susan}}

I've gotten emails from several Times readers who share your trepidation
about cruising. Given the rates of
\href{https://www.cunard.com/en-us/contact-us/travel-health-advisories}{coronavirus
infections on ships} and all of the challenges with health and safety
measures, some travelers feel not-so-great about the idea of boarding a
ship anytime soon.

New data from the Centers for Disease Control and Prevention
\href{https://www.nytimes.com/2020/07/16/travel/coronavirus-cruise-ban-extended.html}{paint
an especially bleak picture}: From March 1 to July 10, 80 percent of
ships in the agency's jurisdiction were affected by the coronavirus.
Since March 14, the C.D.C. has devoted more than 38,000 hours to
managing outbreaks on cruise ships.

Although certain small ships
\href{https://www.thechronicleherald.ca/news/world/cruise-ships-return-to-danube-with-strict-safety-rules-on-board-475270/}{have
resumed} operations this summer, most ocean vessels that sail to or from
United States ports are suspended through Sept. 30, according to the
C.D.C.'s extended
\href{https://www.cdc.gov/media/releases/2020/s0716-cruise-ship-no-sail-order.html}{No
Sail Order.} Still, many cruise lines are delaying their relaunch date
even further. Cunard is
\href{https://www.cunard.com/en-us/contact-us/travel-health-advisories}{paused
until November.} Carnival Cruise Line has already
\href{https://www.carnival.com/health-and-sailing-updates?icid=default_butterbar_health\&safety_06032020}{canceled
some sailings} into 2021.

\hypertarget{latest-updates-global-coronavirus-outbreak}{%
\section{\texorpdfstring{\href{https://www.nytimes.com/2020/08/01/world/coronavirus-covid-19.html?action=click\&pgtype=Article\&state=default\&region=MAIN_CONTENT_1\&context=storylines_live_updates}{Latest
Updates: Global Coronavirus
Outbreak}}{Latest Updates: Global Coronavirus Outbreak}}\label{latest-updates-global-coronavirus-outbreak}}

Updated 2020-08-02T07:14:05.841Z

\begin{itemize}
\tightlist
\item
  \href{https://www.nytimes.com/2020/08/01/world/coronavirus-covid-19.html?action=click\&pgtype=Article\&state=default\&region=MAIN_CONTENT_1\&context=storylines_live_updates\#link-34047410}{The
  U.S. reels as July cases more than double the total of any other
  month.}
\item
  \href{https://www.nytimes.com/2020/08/01/world/coronavirus-covid-19.html?action=click\&pgtype=Article\&state=default\&region=MAIN_CONTENT_1\&context=storylines_live_updates\#link-780ec966}{Top
  U.S. officials work to break an impasse over the federal jobless
  benefit.}
\item
  \href{https://www.nytimes.com/2020/08/01/world/coronavirus-covid-19.html?action=click\&pgtype=Article\&state=default\&region=MAIN_CONTENT_1\&context=storylines_live_updates\#link-2bc8948}{Its
  outbreak untamed, Melbourne goes into even greater lockdown.}
\end{itemize}

\href{https://www.nytimes.com/2020/08/01/world/coronavirus-covid-19.html?action=click\&pgtype=Article\&state=default\&region=MAIN_CONTENT_1\&context=storylines_live_updates}{See
more updates}

More live coverage:
\href{https://www.nytimes.com/live/2020/07/31/business/stock-market-today-coronavirus?action=click\&pgtype=Article\&state=default\&region=MAIN_CONTENT_1\&context=storylines_live_updates}{Markets}

So, yes, because cruising is a mess right now, it's not totally
unreasonable to be anxious about what next spring will look like.

To win back consumer confidence, maintain cash flow and assure the
safety of future passengers and crew members, the cruising industry
needs to take massive action. Companies say they are hammering out
health protocols and testing upgrades like ultraviolet technologies and
H.V.A.C. systems. Meanwhile, docked ships and blank passenger manifests
have created a revenue crunch, and certain lines are downsizing their
fleets accordingly. Schedules continue to change; as a result,
customer-service channels are churning in overdrive.

``We've seen lines cancel their sailings in small batches --- usually a
couple of months at a time --- in an effort to process fewer
cancellations at once, but they're still dealing with far more booking
adjustments than they're used to en masse,'' said Colleen McDaniel, the
editor in chief of \href{https://www.cruisecritic.com/}{Cruise Critic},
a major cruise-planning website.

Perhaps fitting for an industry that's so in flux, reader complaints
about cruise refunds and credits have felt especially bizarre. One woman
was told by a cruise line customer-service representative that in order
to get a refund, she would need to stop posting complaints about the
company on Twitter. (She didn't; that's how I found her.) Another was
asked to prepay for a cruise a full 33 months in advance --- highly
unusual for a system that runs on deposits and final payments. (One
cruise-editor friend, upon hearing these anecdotes, deemed them a good
``alarm for the industry'' about how customer-service reps are trained.)

One of the biggest --- and consumer-friendliest --- changes in cruising
can be seen in cancellation policies. While specifics vary by cruise
line, in general, the possibility of a refund shrinks (or disappears)
the closer one gets to the departure date. Pre-pandemic, most lines
allowed changes and cancellations up until 90 days in advance; now, many
allow them as close as a day or two before departure.

The hitch, though, is exactly the question your scenario raised: If you
do cancel a cruise reservation before the cruise line itself cancels the
sailing, can you get your money back? Or are you forced, as you were, to
accept a credit for a future cruise?

Another Times reader, Minhgiao, encountered this issue when her
septuagenarian parents canceled their
\href{https://www.vikingcruises.com/}{Viking Cruises} reservation right
before the cruise line canceled the sailing in early March. That close
to the departure date, they were not offered a refund (though eventually
customer service agreed to give them a voucher).

``This was not only unsympathetic to the situation of a world pandemic,
but also unethical,'' wrote Minhgiao. ``Our parents are old and a lot
could happen in a year so the chance of them actually using the travel
vouchers was unknown.''

\href{https://www.nytimes.com/news-event/coronavirus?action=click\&pgtype=Article\&state=default\&region=MAIN_CONTENT_3\&context=storylines_faq}{}

\hypertarget{the-coronavirus-outbreak-}{%
\subsubsection{The Coronavirus Outbreak
›}\label{the-coronavirus-outbreak-}}

\hypertarget{frequently-asked-questions}{%
\paragraph{Frequently Asked
Questions}\label{frequently-asked-questions}}

Updated July 27, 2020

\begin{itemize}
\item ~
  \hypertarget{should-i-refinance-my-mortgage}{%
  \paragraph{Should I refinance my
  mortgage?}\label{should-i-refinance-my-mortgage}}

  \begin{itemize}
  \tightlist
  \item
    \href{https://www.nytimes.com/article/coronavirus-money-unemployment.html?action=click\&pgtype=Article\&state=default\&region=MAIN_CONTENT_3\&context=storylines_faq}{It
    could be a good idea,} because mortgage rates have
    \href{https://www.nytimes.com/2020/07/16/business/mortgage-rates-below-3-percent.html?action=click\&pgtype=Article\&state=default\&region=MAIN_CONTENT_3\&context=storylines_faq}{never
    been lower.} Refinancing requests have pushed mortgage applications
    to some of the highest levels since 2008, so be prepared to get in
    line. But defaults are also up, so if you're thinking about buying a
    home, be aware that some lenders have tightened their standards.
  \end{itemize}
\item ~
  \hypertarget{what-is-school-going-to-look-like-in-september}{%
  \paragraph{What is school going to look like in
  September?}\label{what-is-school-going-to-look-like-in-september}}

  \begin{itemize}
  \tightlist
  \item
    It is unlikely that many schools will return to a normal schedule
    this fall, requiring the grind of
    \href{https://www.nytimes.com/2020/06/05/us/coronavirus-education-lost-learning.html?action=click\&pgtype=Article\&state=default\&region=MAIN_CONTENT_3\&context=storylines_faq}{online
    learning},
    \href{https://www.nytimes.com/2020/05/29/us/coronavirus-child-care-centers.html?action=click\&pgtype=Article\&state=default\&region=MAIN_CONTENT_3\&context=storylines_faq}{makeshift
    child care} and
    \href{https://www.nytimes.com/2020/06/03/business/economy/coronavirus-working-women.html?action=click\&pgtype=Article\&state=default\&region=MAIN_CONTENT_3\&context=storylines_faq}{stunted
    workdays} to continue. California's two largest public school
    districts --- Los Angeles and San Diego --- said on July 13, that
    \href{https://www.nytimes.com/2020/07/13/us/lausd-san-diego-school-reopening.html?action=click\&pgtype=Article\&state=default\&region=MAIN_CONTENT_3\&context=storylines_faq}{instruction
    will be remote-only in the fall}, citing concerns that surging
    coronavirus infections in their areas pose too dire a risk for
    students and teachers. Together, the two districts enroll some
    825,000 students. They are the largest in the country so far to
    abandon plans for even a partial physical return to classrooms when
    they reopen in August. For other districts, the solution won't be an
    all-or-nothing approach.
    \href{https://bioethics.jhu.edu/research-and-outreach/projects/eschool-initiative/school-policy-tracker/}{Many
    systems}, including the nation's largest, New York City, are
    devising
    \href{https://www.nytimes.com/2020/06/26/us/coronavirus-schools-reopen-fall.html?action=click\&pgtype=Article\&state=default\&region=MAIN_CONTENT_3\&context=storylines_faq}{hybrid
    plans} that involve spending some days in classrooms and other days
    online. There's no national policy on this yet, so check with your
    municipal school system regularly to see what is happening in your
    community.
  \end{itemize}
\item ~
  \hypertarget{is-the-coronavirus-airborne}{%
  \paragraph{Is the coronavirus
  airborne?}\label{is-the-coronavirus-airborne}}

  \begin{itemize}
  \tightlist
  \item
    The coronavirus
    \href{https://www.nytimes.com/2020/07/04/health/239-experts-with-one-big-claim-the-coronavirus-is-airborne.html?action=click\&pgtype=Article\&state=default\&region=MAIN_CONTENT_3\&context=storylines_faq}{can
    stay aloft for hours in tiny droplets in stagnant air}, infecting
    people as they inhale, mounting scientific evidence suggests. This
    risk is highest in crowded indoor spaces with poor ventilation, and
    may help explain super-spreading events reported in meatpacking
    plants, churches and restaurants.
    \href{https://www.nytimes.com/2020/07/06/health/coronavirus-airborne-aerosols.html?action=click\&pgtype=Article\&state=default\&region=MAIN_CONTENT_3\&context=storylines_faq}{It's
    unclear how often the virus is spread} via these tiny droplets, or
    aerosols, compared with larger droplets that are expelled when a
    sick person coughs or sneezes, or transmitted through contact with
    contaminated surfaces, said Linsey Marr, an aerosol expert at
    Virginia Tech. Aerosols are released even when a person without
    symptoms exhales, talks or sings, according to Dr. Marr and more
    than 200 other experts, who
    \href{https://academic.oup.com/cid/article/doi/10.1093/cid/ciaa939/5867798}{have
    outlined the evidence in an open letter to the World Health
    Organization}.
  \end{itemize}
\item ~
  \hypertarget{what-are-the-symptoms-of-coronavirus}{%
  \paragraph{What are the symptoms of
  coronavirus?}\label{what-are-the-symptoms-of-coronavirus}}

  \begin{itemize}
  \tightlist
  \item
    Common symptoms
    \href{https://www.nytimes.com/article/symptoms-coronavirus.html?action=click\&pgtype=Article\&state=default\&region=MAIN_CONTENT_3\&context=storylines_faq}{include
    fever, a dry cough, fatigue and difficulty breathing or shortness of
    breath.} Some of these symptoms overlap with those of the flu,
    making detection difficult, but runny noses and stuffy sinuses are
    less common.
    \href{https://www.nytimes.com/2020/04/27/health/coronavirus-symptoms-cdc.html?action=click\&pgtype=Article\&state=default\&region=MAIN_CONTENT_3\&context=storylines_faq}{The
    C.D.C. has also} added chills, muscle pain, sore throat, headache
    and a new loss of the sense of taste or smell as symptoms to look
    out for. Most people fall ill five to seven days after exposure, but
    symptoms may appear in as few as two days or as many as 14 days.
  \end{itemize}
\item ~
  \hypertarget{does-asymptomatic-transmission-of-covid-19-happen}{%
  \paragraph{Does asymptomatic transmission of Covid-19
  happen?}\label{does-asymptomatic-transmission-of-covid-19-happen}}

  \begin{itemize}
  \tightlist
  \item
    So far, the evidence seems to show it does. A widely cited
    \href{https://www.nature.com/articles/s41591-020-0869-5}{paper}
    published in April suggests that people are most infectious about
    two days before the onset of coronavirus symptoms and estimated that
    44 percent of new infections were a result of transmission from
    people who were not yet showing symptoms. Recently, a top expert at
    the World Health Organization stated that transmission of the
    coronavirus by people who did not have symptoms was ``very rare,''
    \href{https://www.nytimes.com/2020/06/09/world/coronavirus-updates.html?action=click\&pgtype=Article\&state=default\&region=MAIN_CONTENT_3\&context=storylines_faq\#link-1f302e21}{but
    she later walked back that statement.}
  \end{itemize}
\end{itemize}

I reached out to Viking and was able to help get Minhgiao's parents
their cash back (\$11,594 total). But in the last couple of months, as
the pandemic swelled, Viking eased its
\href{https://www.vikingcruises.com/oceans/risk-free-guarantee.html}{cancellation
policy} anyway. Now, guests who book by the end of July can cancel up to
24 hours before departure for either a cash refund or a voucher (less
the standard \$100 cancellation fee). With August nigh, it's likely the
window for risk-free booking will further expand.

Far more typical, said Ms. McDaniel, is what you encountered with
Cunard.

``For most cruises that are canceled by the line, cruisers are able to
receive refunds,'' she said. ``But for travelers canceling on their own,
most lines are only offering compensation in terms of future cruise
credit.''

Flexible booking terms and the likelihood of more canceled sailings in
the future are two of the reasons most cruise experts I know now
recommend waiting as long as possible to bow out of a reservation. There
is no magic number, so what that means depends on the line you're
sailing with, when you book, the length of the itinerary, final-payment
deadlines and other factors. Individual booking and cancellation
policies will continue to change as we head into fall.

That said, May feels like an especially long time to have to wait and
see --- even more so when you aren't looking forward to the trip. The
last thing anyone needs more of right now is dread. Luckily, you're off
the hook: That ship may end up sailing after all, but, when reached by
email, Cunard agreed to convert your cruise credit into a refund.

\href{https://twitter.com/sfirshein?lang=en}{Sarah Firshein} is a
Brooklyn-based writer. If you need advice about a best-laid travel plan
that went awry, \textbf{\href{mailto:travel@nytimes.com}{send an email
to travel@nytimes.com}.}

\begin{center}\rule{0.5\linewidth}{\linethickness}\end{center}

\emph{\textbf{For more Travel coverage follow us on}}
\textbf{\href{https://twitter.com/nytimestravel}{\emph{Twitter}}}
\emph{\textbf{and}}
\textbf{\href{https://www.facebook.com/nytimestravel/}{\emph{Facebook}}\emph{.
And}}
\textbf{\href{https://www.nytimes.com/newsletters/traveldispatch?action=click\&module=inline\&pgtype=Article}{\emph{sign
up for our}}} ******
\textbf{\href{https://www.nytimes.com/newsletters/traveldispatch}{\emph{Travel
Dispatch newsletter}}\emph{: Each week you'll receive tips on traveling
smarter, stories on hot destinations and access to photos from all over
the world.}}

Advertisement

\protect\hyperlink{after-bottom}{Continue reading the main story}

\hypertarget{site-index}{%
\subsection{Site Index}\label{site-index}}

\hypertarget{site-information-navigation}{%
\subsection{Site Information
Navigation}\label{site-information-navigation}}

\begin{itemize}
\tightlist
\item
  \href{https://help.nytimes.com/hc/en-us/articles/115014792127-Copyright-notice}{©~2020~The
  New York Times Company}
\end{itemize}

\begin{itemize}
\tightlist
\item
  \href{https://www.nytco.com/}{NYTCo}
\item
  \href{https://help.nytimes.com/hc/en-us/articles/115015385887-Contact-Us}{Contact
  Us}
\item
  \href{https://www.nytco.com/careers/}{Work with us}
\item
  \href{https://nytmediakit.com/}{Advertise}
\item
  \href{http://www.tbrandstudio.com/}{T Brand Studio}
\item
  \href{https://www.nytimes.com/privacy/cookie-policy\#how-do-i-manage-trackers}{Your
  Ad Choices}
\item
  \href{https://www.nytimes.com/privacy}{Privacy}
\item
  \href{https://help.nytimes.com/hc/en-us/articles/115014893428-Terms-of-service}{Terms
  of Service}
\item
  \href{https://help.nytimes.com/hc/en-us/articles/115014893968-Terms-of-sale}{Terms
  of Sale}
\item
  \href{https://spiderbites.nytimes.com}{Site Map}
\item
  \href{https://help.nytimes.com/hc/en-us}{Help}
\item
  \href{https://www.nytimes.com/subscription?campaignId=37WXW}{Subscriptions}
\end{itemize}
