Sections

SEARCH

\protect\hyperlink{site-content}{Skip to
content}\protect\hyperlink{site-index}{Skip to site index}

\href{https://www.nytimes.com/spotlight/at-home}{At Home}

\href{https://myaccount.nytimes.com/auth/login?response_type=cookie\&client_id=vi}{}

\href{https://www.nytimes.com/section/todayspaper}{Today's Paper}

\href{/spotlight/at-home}{At Home}\textbar{}Suggestions, Distractions
and Diaries. Our Staff Is an Open Book.

\url{https://nyti.ms/3fTRi2d}

\begin{itemize}
\item
\item
\item
\item
\item
\end{itemize}

\href{https://www.nytimes.com/spotlight/at-home?action=click\&pgtype=Article\&state=default\&region=TOP_BANNER\&context=at_home_menu}{At
Home}

\begin{itemize}
\tightlist
\item
  \href{https://www.nytimes.com/2020/07/28/books/time-for-a-literary-road-trip.html?action=click\&pgtype=Article\&state=default\&region=TOP_BANNER\&context=at_home_menu}{Take:
  A Literary Road Trip}
\item
  \href{https://www.nytimes.com/2020/07/29/magazine/bored-with-your-home-cooking-some-smoky-eggplant-will-fix-that.html?action=click\&pgtype=Article\&state=default\&region=TOP_BANNER\&context=at_home_menu}{Cook:
  Smoky Eggplant}
\item
  \href{https://www.nytimes.com/2020/07/27/travel/moose-michigan-isle-royale.html?action=click\&pgtype=Article\&state=default\&region=TOP_BANNER\&context=at_home_menu}{Look
  Out: For Moose}
\item
  \href{https://www.nytimes.com/interactive/2020/at-home/even-more-reporters-editors-diaries-lists-recommendations.html?action=click\&pgtype=Article\&state=default\&region=TOP_BANNER\&context=at_home_menu}{Explore:
  Reporters' Obsessions}
\end{itemize}

Advertisement

\protect\hyperlink{after-top}{Continue reading the main story}

Supported by

\protect\hyperlink{after-sponsor}{Continue reading the main story}

At Home Newsletter

\hypertarget{suggestions-distractions-and-diaries-our-staff-is-an-open-book}{%
\section{Suggestions, Distractions and Diaries. Our Staff Is an Open
Book.}\label{suggestions-distractions-and-diaries-our-staff-is-an-open-book}}

Worry is a drumbeat. Above it, though, there's a melody: a craving for
distraction and joy, for intimacy, for serendipity.

\includegraphics{https://static01.nyt.com/images/2020/08/07/smarter-living/0721google-feature-promo/0721google-feature-promo-articleLarge.png?quality=75\&auto=webp\&disable=upscale}

\href{https://www.nytimes.com/by/sam-sifton}{\includegraphics{https://static01.nyt.com/images/2018/06/21/multimedia/author-sam-sifton/author-sam-sifton-thumbLarge.png}}

By \href{https://www.nytimes.com/by/sam-sifton}{Sam Sifton}

\begin{itemize}
\item
  July 23, 2020
\item
  \begin{itemize}
  \item
  \item
  \item
  \item
  \item
  \end{itemize}
\end{itemize}

Welcome. It's a funny thing, what's happening with us, with this growing
community of ours at home and \href{http://www.nytimes.com/athome}{At
Home}. We're anxious about the coming months, about what happens when
schools start or don't, when the sun stops coming up early and falling
late, when the temperature drops and we can't be outside so much, can't
eat on a sidewalk, can't loll around in a park. Will we be back at work,
will we be still working or looking for work at this tiny desk in the
corner of the bedroom, in the attic, in the living room, in the garage?
Will we still be alone, or alone together? Will we be safer, or less
safe?

That worry's a drumbeat. We attend to it closely. Above it, though,
there's a melody: a craving for distraction and joy, for intimacy, for
serendipity.

Lately we've been finding that in the off-hours work of our colleagues,
and in the notes that they've taken recently on life during the pandemic
about what they've been reading and listening to and watching, about
what they've been doing and how and why. We call these documents
\href{https://www.nytimes.com/interactive/2020/at-home/even-more-reporters-editors-diaries-lists-recommendations.html}{``Notes
From Our Homes to Yours,''} and they're a remarkable collection of
observations and recommendations about living a full and cultured life
during the pandemic. I'd urge you to check them out today.

And when you're done, please visit us
\href{http://www.nytimes.com/athome}{At Home} for more straightforward
advice about how to deal with life right now, how to manage, how to
feed, how to help. We are as always here to serve. Let us know what you
want to know:
\href{mailto:athome@nytimes.com}{\nolinkurl{athome@nytimes.com}}. See
you on Friday.

\begin{center}\rule{0.5\linewidth}{\linethickness}\end{center}

\hypertarget{how-to-deal}{%
\subsection{How to deal.}\label{how-to-deal}}

\includegraphics{https://static01.nyt.com/images/2020/07/26/realestate/21fix1/oakImage-1594923633651-articleLarge.jpg?quality=75\&auto=webp\&disable=upscale}

\begin{itemize}
\item
  For people who are still working from home for the foreseeable future,
  it is time to stop making excuses and get things around the house in
  order. That can mean
  \href{https://www.nytimes.com/interactive/2020/07/20/burst/bedroom-organization-tips.html}{tidying
  your bedroom} or
  \href{https://www.nytimes.com/2020/07/21/realestate/that-home-office-of-yours-it-needs-an-upgrade.html}{upgrading
  your home office} (or setting up a home office if you've been trying
  to survive without one). And if your home has amenities you aren't
  able to use, it's worth asking if you
  \href{https://www.nytimes.com/2020/07/18/realestate/gym-playroom-fees-coronavirus.html}{still
  have to pay for them}.
\item
  Urban centers have proved resilient as centers of innovation, but a
  growing sense that
  \href{https://www.nytimes.com/2020/07/21/business/economy/coronavirus-cities.html}{density
  is a core issue in our new world} could lead to a lack of luster for
  cities once thought of as superstars.
\item
  People often talk about how one day
  \href{https://www.nytimes.com/2020/07/18/at-home/coronavirus-fiction-writing.html}{they'll
  write a short story}. Curtis Sittenfeld, the best-selling author of
  ``Prep'' and ``Rodham,'' thinks the time to start is \emph{now}.
\end{itemize}

\begin{center}\rule{0.5\linewidth}{\linethickness}\end{center}

\hypertarget{what-to-eat}{%
\subsection{What to eat.}\label{what-to-eat}}

Image

You can portion your arjamolho into small bowls if you're using it as a
side dish. Though you might be tempted to eat straight from the serving
bowl.Credit...Pedro Guimarães

\begin{itemize}
\item
  When asked for a selection for the
  ``\href{https://www.nytimes.com/column/one-good-meal?module=inline}{One
  Good Meal}'' series, the textile artist Vanessa Barragão picked
  \href{https://www.nytimes.com/2020/07/17/t-magazine/summer-recipes-arjamolho-soup-vanessa-barragao.html}{arjamolho,
  a chilled tomato soup} that is healthy, flavorful and, in her opinion,
  perfect for summer.
\item
  Marcella Hazan spent years teaching Americans the finer points of
  Italian home cooking. It is easy to picture her as a breakout star of
  quarantine cooking if she'd lived this long, and with
  \href{https://www.nytimes.com/2020/07/20/dining/marcella-hazan-tomato-sauce-recipe.html}{these
  three tomato sauce recipes}, she might still get there.
\item
  And Bill Buford spent more than a decade seeking the heart of French
  cuisine for his new book, ``Dirt.'' But in quarantine, he just wants
  to make
  \href{https://www.nytimes.com/2020/07/21/dining/bill-buford-dirt-book-chicken-recipe.html}{the
  perfect chicken}. Pete Wells dug in on Buford's obsession, and the
  effect it has had on his family and friends. ``When they grew up, the
  boys would almost certainly remember when they didn't have to go to
  school for months and their father kept coming up with new ways to
  poach a chicken.''
\end{itemize}

\begin{center}\rule{0.5\linewidth}{\linethickness}\end{center}

\hypertarget{how-to-pass-the-time}{%
\subsection{How to pass the time.}\label{how-to-pass-the-time}}

Image

\begin{itemize}
\item
  Our friends in Parenting asked a very big question:
  \href{https://www.nytimes.com/2020/07/21/parenting/the-state-of-play.html}{How
  do children play, and why does it matter.} The resulting ``State of
  Play'' series is an ambitious dive into how kids work that can be a
  road map for parents everywhere. Ever wonder
  \href{https://www.nytimes.com/2020/07/21/parenting/dinosaur-kids.html}{how
  a paleontologist would play dinosaurs with your kid}? It's in there.
\item
  When asked to sum up concerts in a pandemic era of livestreams in
  living rooms, Jon Pareles did not hold back:
  ``\href{https://www.nytimes.com/2020/07/21/arts/music/livestreams-intimacy.html}{So
  many good intentions, so little joy.}'' He did, however, take the time
  to put together the
  \href{https://www.nytimes.com/2020/07/21/arts/music/best-quarantine-concerts-livestream.html}{10
  best quarantine concerts} that you can still stream at home.
\item
  And while the thought of traveling right now can be intimidating ---
  that's if you're even allowed to travel --- we can help expand your
  worldview a bit. With our ``The World Through a Lens'' series you can
  travel to Ecuador,
  \href{https://www.nytimes.com/2020/07/20/travel/panama-hats-ecuador.html}{the
  real home of Panama hats}; and through a
  \href{https://www.nytimes.com/2020/07/18/at-home/coronavirus-travel-podcasts.html}{series
  of podcasts} you can chart out plenty of other virtual journeys.
\end{itemize}

\begin{center}\rule{0.5\linewidth}{\linethickness}\end{center}

\hypertarget{like-what-you-see}{%
\subsubsection{Like what you see?}\label{like-what-you-see}}

\href{https://www.nytimes.com/newsletters/at-home}{Sign up} to receive
this newsletter in your inbox! And let us know
\href{https://nyt.qualtrics.com/jfe/form/SV_e9cKGVFtci4CObz}{what you
think} of it. You can always find much more to read, watch and do every
day on \href{https://www.nytimes.com/spotlight/at-home}{At Home}.

Advertisement

\protect\hyperlink{after-bottom}{Continue reading the main story}

\hypertarget{site-index}{%
\subsection{Site Index}\label{site-index}}

\hypertarget{site-information-navigation}{%
\subsection{Site Information
Navigation}\label{site-information-navigation}}

\begin{itemize}
\tightlist
\item
  \href{https://help.nytimes.com/hc/en-us/articles/115014792127-Copyright-notice}{©~2020~The
  New York Times Company}
\end{itemize}

\begin{itemize}
\tightlist
\item
  \href{https://www.nytco.com/}{NYTCo}
\item
  \href{https://help.nytimes.com/hc/en-us/articles/115015385887-Contact-Us}{Contact
  Us}
\item
  \href{https://www.nytco.com/careers/}{Work with us}
\item
  \href{https://nytmediakit.com/}{Advertise}
\item
  \href{http://www.tbrandstudio.com/}{T Brand Studio}
\item
  \href{https://www.nytimes.com/privacy/cookie-policy\#how-do-i-manage-trackers}{Your
  Ad Choices}
\item
  \href{https://www.nytimes.com/privacy}{Privacy}
\item
  \href{https://help.nytimes.com/hc/en-us/articles/115014893428-Terms-of-service}{Terms
  of Service}
\item
  \href{https://help.nytimes.com/hc/en-us/articles/115014893968-Terms-of-sale}{Terms
  of Sale}
\item
  \href{https://spiderbites.nytimes.com}{Site Map}
\item
  \href{https://help.nytimes.com/hc/en-us}{Help}
\item
  \href{https://www.nytimes.com/subscription?campaignId=37WXW}{Subscriptions}
\end{itemize}
