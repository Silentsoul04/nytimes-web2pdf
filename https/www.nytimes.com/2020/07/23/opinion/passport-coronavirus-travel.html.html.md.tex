Sections

SEARCH

\protect\hyperlink{site-content}{Skip to
content}\protect\hyperlink{site-index}{Skip to site index}

\href{https://myaccount.nytimes.com/auth/login?response_type=cookie\&client_id=vi}{}

\href{https://www.nytimes.com/section/todayspaper}{Today's Paper}

\href{/section/opinion}{Opinion}\textbar{}Where an American Passport
Doesn't Work: The World, and Irish Pubs

\href{https://nyti.ms/39m6xON}{https://nyti.ms/39m6xON}

\begin{itemize}
\item
\item
\item
\item
\item
\item
\end{itemize}

Advertisement

\protect\hyperlink{after-top}{Continue reading the main story}

\href{/section/opinion}{Opinion}

Supported by

\protect\hyperlink{after-sponsor}{Continue reading the main story}

\hypertarget{where-an-american-passport-doesnt-work-the-world-and-irish-pubs}{%
\section{Where an American Passport Doesn't Work: The World, and Irish
Pubs}\label{where-an-american-passport-doesnt-work-the-world-and-irish-pubs}}

Will the coronavirus finally show Americans how much privilege they've
enjoyed?

\href{https://www.nytimes.com/by/maeve-higgins}{\includegraphics{https://static01.nyt.com/images/2018/07/12/opinion/maeve-higgins/maeve-higgins-thumbLarge.png}}

By \href{https://www.nytimes.com/by/maeve-higgins}{Maeve Higgins}

Contributing Opinion Writer

\begin{itemize}
\item
  July 23, 2020
\item
  \begin{itemize}
  \item
  \item
  \item
  \item
  \item
  \item
  \end{itemize}
\end{itemize}

\includegraphics{https://static01.nyt.com/images/2020/07/23/opinion/23higgins/23higgins-articleLarge.jpg?quality=75\&auto=webp\&disable=upscale}

You know how countries have slogans, like Thailand is ``The Land of
Smiles''? That seems like a lot of pressure on citizens and tourists
alike. An entire land of constant smiles? Even when you have lower back
pain or you've just been served surprise divorce papers: Smile!

It's been encouraging to see countries throw off this corporate-style
branding in recent times. Did you know that America's last attempt at a
slogan was the 2016 ``It's all within reach''? Of course, we're unable
to say it with a straight face anymore after years of travel bans.
``It's aspirational,'' we murmur beneath our masks, wincing as the
country isolates and dips further and further down in the world's
estimation. Ireland regularly promoted itself as ``The Land of a
Thousand Welcomes'' --- until the coronavirus came along and coughed on
that idea, rendering it high-risk.

The list of countries with borders open to Americans has never been
shorter. But for now, Ireland, unlike many others in Europe, is still
allowing Americans in. It's the welcome part that's missing. American
tourists who don't feel like quarantining and instead hope to drink and
dine at recently and cautiously reopened restaurants and bars are being
soundly
\href{https://www.nytimes.com/2020/07/14/world/europe/Ireland-americans-break-quarantine.html}{turned
away.}

This is of interest to me, an Irish citizen who lives in the United
States, because of my recent trip back. I went to Ireland when the
pandemic started,
\href{https://www.nytimes.com/2020/04/11/opinion/sunday/coronavirus-isolation-ireland.html}{figuring
it would be safer}. It was. In fact, Ireland has
\href{https://www.irishtimes.com/news/health/ireland-has-lowest-current-incidence-of-covid-19-in-western-europe-1.4294642}{one
of the lowest rates of Covid-19 in Europe}. However, I missed my home
and my life and mainly Shake Shack, so I decided to go back to New York.
It wasn't easy, because of the travel ban the United States has put in
place on people coming from Europe. This obstacle was new to me. An
Irish passport is a powerful one that usually admits me to most parts of
the world, and an American visa like the one I have in my passport is an
equally rare and precious thing.

Suddenly, though, doors were being slammed shut and gates locked tight.
I decided to return through Canada. I applied for and got my visa waiver
online within minutes. I flew to London but was not allowed on the
connecting flight to Toronto. It turns out Canada has an extensive
travel ban too; Canadians are just too polite to shout about it. Between
the jigs and the reels, as we say in Brooklyn, I had to come through
Mexico. Not just transit through --- I had to stay there for 14 days,
which I did last month. This itinerary was not my choice and certainly
not logical, but that's what the travel ban did; it forced me to take
two extra long-haul flights, as well as holding me squarely in the
beautiful and resilient Mexico City, which at that time was a hot spot
experiencing record-high levels of infection.

I was there during
\href{https://www.nytimes.com/2020/06/23/world/mexico-earthquake.html}{the
June 23 earthquake}. It was the first one I had experienced, and the
shock waves made my building sway. In the end, I made it back to New
York, flying over the land border that is now all but closed to those
seeking asylum. I quarantined and I'm grateful to be home.

I remain rocked by how something as physically flimsy as a passport and
as artlessly made as a border serve to divide human beings up into two
camps, the powerful and the powerless. Of course, this is not new
information to billions of people in the world. My inconvenient route
home was hardly a taste of the reality lived by most people today. It's
just that now this bad luck is finally going around. Americans are now
barred from visiting
\href{https://www.cnn.com/travel/article/us-international-travel-covid-19/index.html}{most
of the rest of the planet}. Perhaps now that American passports have
stopped working and this nation-state is no longer on top, more of us
will understand the injustice of the entire system.

What to do with this understanding, I'm not quite sure. One option is to
hew ever closer to the bald and horrifying reality that we --- any and
all of us --- are not entitled to any rights because we are human
beings, rather because we happen to be born in such and such a country,
and to act accordingly. To shut down, to prioritize ourselves and
vigorously declare to hell with everybody else. Fine. Although we've
been trying that and it isn't actually working. In fact, it seems pretty
self-destructive.

Borders and bans and even nation-states, along with all of the laws and
violence that hold them together and apart, are relatively new compared
with how long humans have been around. Now that more of us are seeing
and living the constricting and dangerous reality of those artificial
distinctions, surely we need to change them. The person, the human
being, the vulnerable creature no different from any other, that is what
is sacred. Not their paperwork. Americans are learning that now. And
what easier way than being gently turned away from an Irish pub?

Maeve Higgins (\href{https://twitter.com/maevehiggins}{@maevehiggins})
is the author of
``\href{https://www.penguinrandomhouse.com/books/546681/maeve-in-america-by-maeve-higgins/}{Maeve
in America: Essays by a Girl From Somewhere Else}'' and a contributing
Opinion writer.

\emph{The Times is committed to publishing}
\href{https://www.nytimes.com/2019/01/31/opinion/letters/letters-to-editor-new-york-times-women.html}{\emph{a
diversity of letters}} \emph{to the editor. We'd like to hear what you
think about this or any of our articles. Here are some}
\href{https://help.nytimes.com/hc/en-us/articles/115014925288-How-to-submit-a-letter-to-the-editor}{\emph{tips}}\emph{.
And here's our email:}
\href{mailto:letters@nytimes.com}{\emph{letters@nytimes.com}}\emph{.}

\emph{Follow The New York Times Opinion section on}
\href{https://www.facebook.com/nytopinion}{\emph{Facebook}}\emph{,}
\href{http://twitter.com/NYTOpinion}{\emph{Twitter (@NYTopinion)}}
\emph{and}
\href{https://www.instagram.com/nytopinion/}{\emph{Instagram}}\emph{.}

Advertisement

\protect\hyperlink{after-bottom}{Continue reading the main story}

\hypertarget{site-index}{%
\subsection{Site Index}\label{site-index}}

\hypertarget{site-information-navigation}{%
\subsection{Site Information
Navigation}\label{site-information-navigation}}

\begin{itemize}
\tightlist
\item
  \href{https://help.nytimes.com/hc/en-us/articles/115014792127-Copyright-notice}{©~2020~The
  New York Times Company}
\end{itemize}

\begin{itemize}
\tightlist
\item
  \href{https://www.nytco.com/}{NYTCo}
\item
  \href{https://help.nytimes.com/hc/en-us/articles/115015385887-Contact-Us}{Contact
  Us}
\item
  \href{https://www.nytco.com/careers/}{Work with us}
\item
  \href{https://nytmediakit.com/}{Advertise}
\item
  \href{http://www.tbrandstudio.com/}{T Brand Studio}
\item
  \href{https://www.nytimes.com/privacy/cookie-policy\#how-do-i-manage-trackers}{Your
  Ad Choices}
\item
  \href{https://www.nytimes.com/privacy}{Privacy}
\item
  \href{https://help.nytimes.com/hc/en-us/articles/115014893428-Terms-of-service}{Terms
  of Service}
\item
  \href{https://help.nytimes.com/hc/en-us/articles/115014893968-Terms-of-sale}{Terms
  of Sale}
\item
  \href{https://spiderbites.nytimes.com}{Site Map}
\item
  \href{https://help.nytimes.com/hc/en-us}{Help}
\item
  \href{https://www.nytimes.com/subscription?campaignId=37WXW}{Subscriptions}
\end{itemize}
