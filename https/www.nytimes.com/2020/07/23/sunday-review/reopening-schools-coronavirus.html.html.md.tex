\href{/section/opinion/sunday}{Sunday Review}\textbar{}Reopening Schools
Is Way Harder Than It Should Be

\href{https://nyti.ms/2WMTQrm}{https://nyti.ms/2WMTQrm}

\begin{itemize}
\item
\item
\item
\item
\item
\item
\end{itemize}

\href{https://www.nytimes.com/news-event/coronavirus?action=click\&pgtype=Article\&state=default\&region=TOP_BANNER\&context=storylines_menu}{The
Coronavirus Outbreak}

\begin{itemize}
\tightlist
\item
  live\href{https://www.nytimes.com/2020/08/08/world/coronavirus-updates.html?action=click\&pgtype=Article\&state=default\&region=TOP_BANNER\&context=storylines_menu}{Latest
  Updates}
\item
  \href{https://www.nytimes.com/interactive/2020/us/coronavirus-us-cases.html?action=click\&pgtype=Article\&state=default\&region=TOP_BANNER\&context=storylines_menu}{Maps
  and Cases}
\item
  \href{https://www.nytimes.com/interactive/2020/science/coronavirus-vaccine-tracker.html?action=click\&pgtype=Article\&state=default\&region=TOP_BANNER\&context=storylines_menu}{Vaccine
  Tracker}
\item
  \href{https://www.nytimes.com/interactive/2020/world/coronavirus-tips-advice.html?action=click\&pgtype=Article\&state=default\&region=TOP_BANNER\&context=storylines_menu}{F.A.Q.}
\item
  \href{https://www.nytimes.com/live/2020/08/07/business/stock-market-today-coronavirus?action=click\&pgtype=Article\&state=default\&region=TOP_BANNER\&context=storylines_menu}{Markets
  \& Economy}
\end{itemize}

\includegraphics{https://static01.nyt.com/images/2020/07/26/opinion/sunday/26darville-print/26darville-print-articleLarge.jpg?quality=75\&auto=webp\&disable=upscale}

Sections

\protect\hyperlink{site-content}{Skip to
content}\protect\hyperlink{site-index}{Skip to site index}

news analysis

\hypertarget{reopening-schools-is-way-harder-than-it-should-be}{%
\section{Reopening Schools Is Way Harder Than It Should
Be}\label{reopening-schools-is-way-harder-than-it-should-be}}

So is leaving them closed. Now what do we do?

Credit...Photo illustration by Doug Chayka; Photographs from Getty
Images

Supported by

\protect\hyperlink{after-sponsor}{Continue reading the main story}

By Sarah Darville

Ms. Darville is a managing editor at Chalkbeat, a nonprofit news outlet
focused on education, with which this article is being copublished.

\begin{itemize}
\item
  July 23, 2020
\item
  \begin{itemize}
  \item
  \item
  \item
  \item
  \item
  \item
  \end{itemize}
\end{itemize}

\textbf{Of all the American institutions} the pandemic has shut down,
none face pressure to reopen quite like schools do. ** Pediatricians
exhort schools to open their doors wherever possible or risk
developmental harm to kids. Working parents, particularly mothers, are
in crisis, worried about having to leave the work force altogether in
the absence of a place to send their young children each day. And
President Trump is campaigning for
\href{https://www.nytimes.com/2020/08/03/business/how-schools-reopen.html}{schools
to reopen}, threatening to withhold funding if they don't.

The pressure has mounted as school districts have made it clear that
they can do no such thing. Across the country --- including in Phoenix,
Houston and a huge chunk of California, where coronavirus cases are
rapidly rising ---
\href{https://www.chalkbeat.org/2020/7/13/21323338/coronavirus-cases-rise-school-districts-start-year-virtually}{schools
are preparing their students and staffs} for a continuation of the
``remote learning'' that began in the spring. In
\href{https://www.nytimes.com/2020/08/07/nyregion/cuomo-schools-reopening.html}{New
York City} and
\href{https://chicago.chalkbeat.org/2020/7/17/21328453/chicago-students-will-return-to-school-buildings-two-days-a-week-under-tentative-fall-plan}{Chicago},
where the virus is more under control, schools are moving toward a
hybrid option with remote learning some days, in-person school others.
Even in places like
\href{https://detroit.chalkbeat.org/2020/7/14/21325166/detroit-fall-reopening-plan}{Detroit}
and
\href{https://tn.chalkbeat.org/2020/7/6/21315536/temperature-checks-face-masks-required-in-shelby-county-schools-memphis-back-to-school-reentry-plan}{Memphis},
where districts plan to offer in-person school for those who want it,
local leaders could change course if virus cases rise; they also have
yet to figure out what to do if too many worried teachers or students
opt out.

Outrage over schools' inability to fully reopen should not, of course,
be directed at schools themselves, but at the public health failure that
makes it impossible for most of them to do so. The consequences of
closed or half-open schools, meanwhile, are far vaster than the brutal
economic challenge facing working parents and their employers. That's
because schools do much more than provide child care. They provide
education, fundamentally. But as the pandemic has made clear, they also
provide meals, social connection and health services.

Meeting any one of these needs in normal times through a single
institution is a struggle. Add in an out-of-control pandemic that
multiplied the number of children who are not getting enough to eat to
\href{https://www.brookings.edu/blog/up-front/2020/07/09/about-14-million-children-in-the-us-are-not-getting-enough-to-eat/}{14
million}, made in-person teaching a health gamble and threw off the
learning trajectory of every child in America --- all while creating
huge projected budget shortfalls for schools --- and you have a ``train
wreck,'' said David K. Cohen, a visiting professor of education at
Harvard.

Compounding the difficulty is the fact that schools are run locally,
autonomy the Trump administration has taken to new extremes by offering
reopening instructions that amount to, ``good luck.'' As a result, many
of the country's 13,000-plus school districts have been left alone to
navigate everything from finding masks to deciding what safe classrooms
look like --- not to mention how to offer widespread and safe food
distribution and personalized emotional support in the absence of
physical gathering space.

\includegraphics{https://static01.nyt.com/images/2020/07/23/opinion/23darville2/merlin_174847689_b162503f-0042-456f-95aa-ed9bd6b79fe1-articleLarge.jpg?quality=75\&auto=webp\&disable=upscale}

``If you wanted to invent a really weak organization to do all of those
things, it would be schools,'' Mr. Cohen said. ``But the reality is,
schools are what families have. Especially poor families and Black and
brown families.''

So if the first sin was failing to control the pandemic, the second was
letting the virus run wild in a country ill-suited to handle the
cascading consequences. The people left to figure it out are
superintendents, school board members, teachers and parents, for whom
that simple word ``reopen'' actually entails a dizzying array of
interlocking problems. The people who will pay the eventual price are
America's children, for years to come.

\textbf{Let's start with child care}, which translates, at the barest
minimum, to providing every child with a safe place to go so their
parents can work and so that they can learn. For schools to play that
role, they require two basic ingredients: sufficient physical space and
willing and capable adult caregivers. But how much space and how many
adults?

That calculation starts with public health considerations. Exactly
\href{https://www.nytimes.com/2020/07/11/health/coronavirus-schools-reopen.html}{what
part open schools play} in spreading the virus is still unknown, but
\href{https://www.nytimes.com/2020/07/18/health/coronavirus-children-schools.html?action=click\&module=Top\%20Stories\&pgtype=Homepage}{new
research} suggests that kids age 10 to 19 can transmit it at rates
similar to adults. And with case numbers still rising in the United
States, school reopenings in places like Denmark that have contained the
virus aren't fitting guides to what would lie ahead here if districts
heeded Mr. Trump's call to bring students back.

One way to mitigate concern is to enforce physical distancing rules,
along with mandates for masks and strict hygiene. The Centers for
Disease Control and Prevention has suggested schools keep students six
feet apart where feasible, a caveat at the root of school leaders'
confusion. The American Academy of Pediatrics has suggested just three
feet may be OK, especially with masks, citing the challenge of
distancing in schools and the downsides of remote learning.

Skyrocketing cases in some parts of the country have rendered this
conversation moot, with school districts and health departments deciding
they cannot take the risk of opening school doors at all. But even if
other districts decide they can open, superintendents have to take into
account two other variables in addition to space: parents' willingness
to send their children to in-person school and teachers' willingness to
show up and teach them.

These considerations can be in direct tension. The more children who
want to go to school, the more teachers needed to teach them. And both
families and teachers are more likely to want to go if schools take more
precautions, like promising six feet of separation. But the more
precautions schools take, the fewer students they can welcome into each
building. Even more complicated, preferences are likely to change
multiple times throughout the year, as families and educators reassess
the spread of the virus in their community and their personal risk
tolerance.

Jason Kamras, the schools chief in Richmond, Va.,
\href{https://twitter.com/jasonkamras/status/1280870903610556417}{recently
likened} the entire conundrum to ``playing a game of 3-D chess while
standing on one leg in the middle of a hurricane.''

``I have families emailing me that they are desperate for their kids to
be back in school,'' he said. ``And also some families who, if their
kids are not in school, they are going to lose their jobs and lose their
homes. Then I have families emailing me, understandably, saying they
have somebody at home who has a compromised immune system and they're
terrified to send their kids.'' (Richmond schools are now set to
\href{https://richmond.com/news/local/richmond-public-schools-will-have-fully-virtual-learning-in-the-fall-because-of-covid-19/article_bcfe040f-7aa5-5c5b-b93a-feeeee56291e.html}{start
the school year online}.)

Nationally, despite the looming child care challenge, a vast majority of
parents
\href{https://www.chalkbeat.org/2020/7/14/21324873/school-closure-reopening-parents-surveys}{remain
skeptical} of a return to in-person school. In a recent survey, nearly
three-quarters of parents called going back into school buildings a
``large to moderate'' risk for their children, and the numbers were even
higher for Black parents and Hispanic parents.

Teachers are also wary. They know more than most how much students and
their families need school. ``I 100 percent want to be with my kids,''
said Kathleen McGinness-Grimes, a ninth-grade algebra teacher at a Bronx
high school. ``I know that they do better in person.''

But teachers did not sign up to prop up the economy by providing child
care while putting their health and the health of their families at
risk. And romantic portrayals of teaching as a calling obscure the
reality that, vocation or not, teachers are also workers who have
received few assurances about job safety. School districts are still
working out who will be able to work from home, what protective
equipment they can provide, how students will be grouped and how
infections will be handled.

``Honestly, we've had times where there's no soap in bathrooms. We've
had no hot water in the building,'' Ms. McGinness-Grimes said. ``I don't
feel safe.''

In Detroit, the schools superintendent, Nikolai Vitti, wants to offer
in-person schooling five days a week to those who want it, with an
online option for those who don't. He says he thinks that only half of
families will choose physical school, allowing for plenty of social
distancing. But he worries teachers' concerns could make that impossible
--- for reasons he understands, even if they frustrate him.

Imagining the thought process of a teacher, he said, ``If we're in a
non-Covid environment and I have to buy pencils for my children, and my
school doesn't have enough guidance counselors and social workers, and
class sizes are large, then how in the world can I trust school systems
to get this right in a Covid environment?''

One possible compromise: Some districts are considering providing
full-time or nearly full-time in-person school to elementary school
students, for whom at-home supervision is most taxing and, according to
early evidence, virus transmission is least likely.

The uncertainty around space and safety has driven some districts to a
hybrid schedule, where students spend some days in school and other days
learning at home, allowing for far fewer students per classroom. One of
the
\href{https://ny.chalkbeat.org/2020/7/8/21317948/nyc-school-schedule-options}{many
permutations} New York City schools can choose from would have students
going to their schools just five days in a three-week stretch: Tuesday
and Thursday one week, Monday and Thursday the second week, and just
Thursday the third week.

The hybrid approach comes up short for many parents. Debra Morello, who
lives in the Bronx, works for a meat distributor and has sent her
5-year-old daughter to one of New York City's centers for children of
essential workers for the last several months. If those centers go away,
and her daughter is welcome in school buildings only occasionally,
``Where am I going to find money for extra child care?'' she wondered.

New York City's schools chancellor says the city is working on
\href{https://ny.chalkbeat.org/2020/7/16/21327364/nyc-vows-to-open-child-care-part-time-school}{some
limited child-care options}, and figuring out whether
\href{https://www.ny1.com/nyc/all-boroughs/education/2020/07/16/why-schools-chancellor-says-split-school-schedule-is-a-no-go-for-fall}{empty
offices} and vacant buildings can be used as school space. But there are
no real answers yet.

Image

Another view of P.S. 20.Credit...Mark Wickens for The New York Times

Image

New benches outside.Credit...Mark Wickens for The New York Times

The reality is that solutions like converting community or outdoor space
into additional classrooms, or increasing the number of available adults
through a new national child care or tutoring corps, would require a
degree of shared responsibility, coordination and resources that schools
have never been able to count on --- and that hasn't changed with the
pandemic's arrival.

\hypertarget{the-coronavirus-outbreak}{%
\subsubsection{The Coronavirus
Outbreak}\label{the-coronavirus-outbreak}}

\hypertarget{back-to-school}{%
\paragraph{Back to School}\label{back-to-school}}

Updated Aug. 8, 2020

The latest highlights as the first students return to U.S. schools.

\begin{itemize}
\item
  \begin{itemize}
  \tightlist
  \item
    Health experts say New York State schools are
    \href{https://www.nytimes.com/2020/08/07/health/coronavirus-ny-schools-reopen.html?action=click\&pgtype=Article\&state=default\&region=MAIN_CONTENT_2\&context=storylines_keepup}{in
    a good position to reopen}, and Gov. Andrew M. Cuomo has
    \href{https://www.nytimes.com/2020/08/07/nyregion/cuomo-schools-reopening.html?action=click\&pgtype=Article\&state=default\&region=MAIN_CONTENT_2\&context=storylines_keepup}{cleared
    the way}.
  \item
    Many schools spent the summer focused on reopening classrooms. What
    if they had
    \href{https://www.nytimes.com/2020/08/07/us/remote-learning-fall-2020.html?action=click\&pgtype=Article\&state=default\&region=MAIN_CONTENT_2\&context=storylines_keepup}{focused
    on improving remote learning} instead?
  \item
    A mother in Germany describes how her family
    \href{https://www.nytimes.com/2020/08/07/parenting/germany-schools-reopening-children.html?action=click\&pgtype=Article\&state=default\&region=MAIN_CONTENT_2\&context=storylines_keepup}{coped
    with the anxiety and uncertainty} of going back to school there.
  \item
    A high school freshman tested positive after two days in class. A
    yearbook editor worries about access to sporting events. We spoke to
    students about
    \href{https://www.nytimes.com/2020/08/06/us/coronavirus-students.html?action=click\&pgtype=Article\&state=default\&region=MAIN_CONTENT_2\&context=storylines_keepup}{what
    school is like in the age of Covid-19.}
  \end{itemize}
\end{itemize}

The clock is ticking: Even as schools delay their start dates, students
and teachers across much of the country are due back in August.

\textbf{In addition to child care, there is food} --- another resource
schools provide that is both much more necessary and much harder to
deliver because of the pandemic. In normal times, U.S. public schools
provide 30 million free or nearly-free meals a day.

Last spring, when schools closed, states and schools devised emergency
workarounds to ensure that students and their families could still have
access to food. Many set up grab-and-go distribution sites. Some handed
out
\href{https://in.chalkbeat.org/2020/6/25/21303626/indianas-summer-food-program-expands-to-meet-increased-needs-during-the-coronavirus-covid19-pandemic}{a
week's worth of food at a time}. We do not know exactly how many
families schools managed to reach.

In Colorado, where
\href{https://www.chalkbeat.org/pages/newsletters}{Chalkbeat} analyzed
available data, the pattern was clear: While demand was high in some
districts, over all, schools
\href{https://co.chalkbeat.org/2020/5/8/21252520/why-colorado-school-districts-are-serving-fewer-meals-during-covid-19-closures}{gave
out only a small fraction} of what they normally would have. Denver's
public schools served 12 percent of the meals they usually provide, for
example.

One of the biggest impediments to picking up available meals --- the
inconvenience of driving or using public transit to go to a pickup site
during a limited window, risking exposure to the virus along the way ---
is not going away in the fall if schools are closed.

Eventually, the federal government set up a program to send money for
food
\href{https://ny.chalkbeat.org/2020/5/20/21265335/nyc-public-school-food-benefits-coronavirus}{directly
to families} to make up for lost meals, though months later, some
families are
\href{https://newark.chalkbeat.org/2020/7/17/21328249/new-jersey-pandemic-ebt-food-benefits-coronavirus}{still
waiting} for those benefits to arrive. That program came out of one of
several pieces of coronavirus relief legislation passed since March,
which have included financial help that researchers estimate has
\href{https://www.nytimes.com/2020/06/21/us/politics/coronavirus-poverty.html}{kept
poverty rates from rising} sharply.

But the financial stress on families is continuing to mount. In June, a
survey by the Census Bureau asked American adults whether children in
their household ``were not eating enough because we couldn't afford
enough food.'' The results indicate that about 14 million children are
hungry because of financial strain --- more than five times the number
in 2018. That's also about two and a half times the number in 2008, the
peak of food insecurity during the Great Recession, according to Lauren
Bauer, an economics fellow at the Brookings Institution who
\href{https://www.brookings.edu/blog/up-front/2020/07/09/about-14-million-children-in-the-us-are-not-getting-enough-to-eat/}{analyzed
the newest data}.

As with the virus itself, Black and Hispanic families bear the heaviest
burden. Among Black households, one in three families with children
reported some food insecurity for children. That was true of about one
in four Hispanic families with children and one in five families with
children nationwide.

``The numbers we're observing in June are higher than we've ever
observed,'' Ms. Bauer said.

The program replacing school meals with benefits doesn't extend into the
new school year. And rule changes that made food distribution doable
this spring --- allowing schools to give meals to any students, rather
than having to check their eligibility --- are also
\href{https://www.politico.com/news/2020/07/20/millions-of-kids-may-lose-out-on-free-meals-as-they-return-to-school-374587}{set
to expire}. Another relief package could address both. But no matter
what happens, schools will be working to fill in the gaps as a major
source of meals for low-income children.

Beyond providing food, schools also serve as de facto community social
workers, sometimes with an actual social worker or two on staff, and
sometimes not. That was true before the pandemic, when teachers and
administrators could count on speaking face-to-face with children and
their parents on a regular basis.

``At this moment, given the pandemic, we need to create safe ways'' to
continue doing that when ``we can't ensure schools will be safe for kids
and teachers and families,'' said John King, a former education
secretary who now runs the Education Trust, a nonprofit focused on
education and civil rights.

Again, the experience of the spring gives a hint at families' challenges
and the lengths schools will go to help them. In an agricultural
community about 50 miles from Fresno, Calif., the Sundale Union
Elementary School District's social worker spent much of the spring
assisting families applying for federal food benefits. The district
posted local job openings on its Facebook page for parents. The
community center that operates out of the district's single school
opened a food and clothing pantry to families twice a week.

``We did a lot more of that than we normally do,'' Superintendent Terri
Rufert said. ``And it wasn't just the parents normally in our
socioeconomically disadvantaged group.''

Whether or not they open their buildings, schools will also be tasked
with helping students adjust after months of upended schedules, limited
social interaction and families trying to keep it together during a
period of heightened stress.

Our failure to get schools fully open means that meeting students'
mental health needs is even harder. And organizing hybrid schedules or
remote learning may sap energy that schools need to serve students'
continuing needs. ``I think we're still going to be in survival mode as
we move into this next academic year,'' said Lisa Sontag-Padilla, a
behavioral scientist with the RAND Corporation who has studied school
mental health services.

Black and Latino children will deserve particular attention, as their
family members are
\href{https://www.nytimes.com/interactive/2020/07/05/us/coronavirus-latinos-african-americans-cdc-data.html}{disproportionately
likely} to have gotten infected with the virus and had to deal with its
medical, emotional and financial effects. Black students, families and
educators have simultaneously been at the center of a national movement
protesting the violent deaths of Black Americans by police officers.

``These kids have suffered a lot over the last few months,'' Zelatrice
Fowler, who teaches gifted students in Phoenix and is the president of
the Arizona Alliance of Black School Educators, said.

She's not sure schools will be up to the task of helping students move
forward. Already, ``There was a disconnect between teachers and
students, especially when the majority of teachers were white,'' she
said. ``Now, there's a bigger need to really, truly understand the
students and their families.'' (Nationally, about 79 percent of
public-school teachers are white, whereas about half of public-school
students are white.)

The support Ms. Fowler will be able to offer this fall will be through a
computer screen. With case numbers rising in Phoenix, schools there are
all starting online.

\textbf{If taking on the child care, food and the mental health
challenges} facing American children this fall were not enough, there is
also, of course, the matter of making sure those children learn.

Providing any form of education this fall means reckoning with an
extraordinary version of what educators call ``summer slide.'' In normal
times, teachers sometimes lack precise information about each child's
academic starting point, and basics like who and what teachers will
teach can remain a mystery until late summer.

Heading into this school year, these constraints are profound. After
school buildings closed this spring, teachers offered various forms of
substitute education, from paper packets to video classroom gatherings.
Nevertheless, a small but significant share of students went totally
unaccounted for as they struggled to connect to online lessons without
reliable internet, took on child care responsibilities for younger
siblings, or just tuned out without the familiar support of teachers and
counselors. Over all, the
\href{https://www.chalkbeat.org/2020/6/26/21304405/surveys-remote-learning-coronavirus-success-failure-teachers-parents}{best
estimates} from teachers are that six in 10 students were regularly
engaged in their coursework.

Image

P.S. 111 in Queens.Credit...Shannon Stapleton/Reuters

``Children were not properly served academically and social-emotionally
when schools were shut down in the spring,'' Dr. Vitti, the Detroit
superintendent, acknowledged. His push for in-person summer school and
an in-person option for students this fall, he said, was spurred by
parents saying: ``Our children are falling behind, even more than they
already were. What are you going to do about it?''

Remote learning was an especially poor substitute for the needs of
America's seven million students with disabilities. Anna Fridman's
5-year-old twins missed out on seven weekly sessions each of speech,
occupational and physical therapy when their Brooklyn pre-K center
closed. ``As a parent, it's very, very painful to see,''
\href{https://ny.chalkbeat.org/2020/6/17/21295121/remote-learning-special-education-nyc}{she
said}.

In Chicago, 16-year-old Sarah Alli-Brown had planned to spend her entire
junior year preparing for the SAT with daily help from an English
teacher. Instead, the older sister of 9-year-old twins, she found
herself charged with caring for her brothers while her mother worked,
and she
\href{https://www.chalkbeat.org/2020/4/1/21225435/two-brothers-to-care-for-little-classwork-sat-worries-for-this-16-year-old-days-now-feel-like-weeks}{went
weeks without real instruction from her school}.

Many of the impediments that made remote learning a mostly nonlearning
experience this spring are still present today. The most significant is
many students' lack of internet access at home. In Newark, where cost is
the main barrier, one in three households
\href{https://newark.chalkbeat.org/2020/3/20/21196093/newark-students-will-get-laptops-free-internet-during-school-closure}{don't
have internet} at home; in some rural parts of the country, broadband
options
\href{https://in.chalkbeat.org/2020/5/28/21273890/how-the-politics-behind-rural-internet-access-leave-parts-of-indiana-in-the-dark-ages}{don't
exist} at all.

An ever-optimistic group, some educators see potential in their hybrid
or fully remote future, imagining more personalized lessons and deeper
connections with smaller groups of students. And districts are working
on ways to make the fall better than the spring, adding more time during
the day to interact with teachers live over video and bringing back more
of the usual structures, like letter grades, to keep kids on track.

But that's little solace to parents of kids who will be trying to learn
to read on Zoom.

Even if teachers and students can return to classrooms, the fundamentals
that can make school come alive will be challenged by even the most
basic health requirements. ``I like students to be able to talk about
their writing with each other,'' Kristin Roberts, a high school English
teacher in Phoenix, said, ``to ask someone how this sentence sounds, to
pass papers back and forth, to sit next to each other and have
conversations about what makes writing good or what they're noticing in
literature.'' How will any of that happen in a mask from six feet apart?

\textbf{Schools are mustering real creativity} to meet the moment. Arne
Duncan, education secretary for most of President Barack Obama's tenure
and now a managing partner with the Emerson Collective, said he's spoken
to school districts that know they won't be able to open in person and
so are planning to run weekly bus routes to bring their most vulnerable
students to school for check-ins. Other schools are figuring out that if
some students are in the building only on Mondays, Tuesdays and
Thursdays, for example, the school will send them home Thursday with
enough food for the weekend. Others are converting cafeterias into
academic space and creating afternoon and Saturday shifts for their
schools.

These workarounds are also dark, revealing how far we remain from being
able to offer a halfway normal school year for the students who need it
most.

The most important way to help is undoubtedly to do what could have made
our American pandemic what it is for children and families in Finland,
Denmark and even Italy. We have to control the virus. Because whether it
should work this way or not, without schools we have a hobbled economy,
hungry children and exhausted parents.

Making schools functional will also take money, as states are facing
\href{https://www.cbpp.org/research/state-budget-and-tax/states-continue-to-face-large-shortfalls-due-to-covid-19-effects}{projected
shortfalls} totaling more than \$500 billion over the next three years
thanks to the spiraling pandemic. Without federal help, schools will
have to lay off teachers and make other painful cuts in the years ahead.
What school leaders, social scientists, parents and others say is needed
in the short term --- the ability to add space and staff members, offer
in-school tutoring and provide additional child-care options, among
other things --- won't be possible without more funding, either.

Congress has yet to answer calls for additional relief. If it comes, and
if every state enacts real public health measures, schools will have a
shot at turning a catastrophe into a mere crisis. Unless both happen,
schools are likely to spend years trying to meet students' growing needs
with less. They will try their best. It will not be remotely enough.

\href{https://www.chalkbeat.org/authors/sarah-darville}{Sarah Darville}
(\href{https://twitter.com/sarahdarv?lang=en}{@sarahdarv}) is the
managing editor for national news at
\href{https://www.chalkbeat.org/}{Chalkbeat}, a nonprofit news
\href{https://www.chalkbeat.org/pages/newsletters}{outlet} focused on
education, with which this article is being copublished.

\emph{The Times is committed to publishing}
\href{https://www.nytimes.com/2019/01/31/opinion/letters/letters-to-editor-new-york-times-women.html}{\emph{a
diversity of letters}} \emph{to the editor. We'd like to hear what you
think about this or any of our articles. Here are some}
\href{https://help.nytimes.com/hc/en-us/articles/115014925288-How-to-submit-a-letter-to-the-editor}{\emph{tips}}\emph{.
And here's our email:}
\href{mailto:letters@nytimes.com}{\emph{letters@nytimes.com}}\emph{.}

\emph{Follow The New York Times Opinion section on}
\href{https://www.facebook.com/nytopinion}{\emph{Facebook}}\emph{,}
\href{http://twitter.com/NYTOpinion}{\emph{Twitter (@NYTopinion)}}
\emph{and}
\href{https://www.instagram.com/nytopinion/}{\emph{Instagram}}\emph{.}

Advertisement

\protect\hyperlink{after-bottom}{Continue reading the main story}

\hypertarget{site-index}{%
\subsection{Site Index}\label{site-index}}

\hypertarget{site-information-navigation}{%
\subsection{Site Information
Navigation}\label{site-information-navigation}}

\begin{itemize}
\tightlist
\item
  \href{https://help.nytimes.com/hc/en-us/articles/115014792127-Copyright-notice}{©~2020~The
  New York Times Company}
\end{itemize}

\begin{itemize}
\tightlist
\item
  \href{https://www.nytco.com/}{NYTCo}
\item
  \href{https://help.nytimes.com/hc/en-us/articles/115015385887-Contact-Us}{Contact
  Us}
\item
  \href{https://www.nytco.com/careers/}{Work with us}
\item
  \href{https://nytmediakit.com/}{Advertise}
\item
  \href{http://www.tbrandstudio.com/}{T Brand Studio}
\item
  \href{https://www.nytimes.com/privacy/cookie-policy\#how-do-i-manage-trackers}{Your
  Ad Choices}
\item
  \href{https://www.nytimes.com/privacy}{Privacy}
\item
  \href{https://help.nytimes.com/hc/en-us/articles/115014893428-Terms-of-service}{Terms
  of Service}
\item
  \href{https://help.nytimes.com/hc/en-us/articles/115014893968-Terms-of-sale}{Terms
  of Sale}
\item
  \href{https://spiderbites.nytimes.com}{Site Map}
\item
  \href{https://help.nytimes.com/hc/en-us}{Help}
\item
  \href{https://www.nytimes.com/subscription?campaignId=37WXW}{Subscriptions}
\end{itemize}
