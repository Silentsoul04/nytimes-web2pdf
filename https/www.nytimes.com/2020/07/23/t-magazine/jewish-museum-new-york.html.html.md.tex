Sections

SEARCH

\protect\hyperlink{site-content}{Skip to
content}\protect\hyperlink{site-index}{Skip to site index}

\href{https://myaccount.nytimes.com/auth/login?response_type=cookie\&client_id=vi}{}

\href{https://www.nytimes.com/section/todayspaper}{Today's Paper}

How New York's Jewish Museum Anticipated the Avant-Garde

\url{https://nyti.ms/3hxyshN}

\begin{itemize}
\item
\item
\item
\item
\item
\item
\end{itemize}

Advertisement

\protect\hyperlink{after-top}{Continue reading the main story}

Supported by

\protect\hyperlink{after-sponsor}{Continue reading the main story}

True Believers

\hypertarget{how-new-yorks-jewish-museum-anticipated-the-avant-garde}{%
\section{How New York's Jewish Museum Anticipated the
Avant-Garde}\label{how-new-yorks-jewish-museum-anticipated-the-avant-garde}}

A string of adventurous curators made a quasi-religious institution a
leading arbiter of mid-20th-century American art.

\includegraphics{https://static01.nyt.com/images/2020/07/13/t-magazine/art/Jewish-museum-slide-EGXT/Jewish-museum-slide-EGXT-articleLarge.jpg?quality=75\&auto=webp\&disable=upscale}

By Arthur Lubow

\begin{itemize}
\item
  Published July 23, 2020Updated July 28, 2020
\item
  \begin{itemize}
  \item
  \item
  \item
  \item
  \item
  \item
  \end{itemize}
\end{itemize}

IN AMERICA IN the 1960s, the museum that showcased the latest and best
contemporary art was located, predictably enough, in New York, but it
wasn't the
\href{https://www.nytimes.com/2015/09/08/t-magazine/adrian-villar-rojas-guggenheim-marian-goodman.html}{Guggenheim},
the
\href{https://www.nytimes.com/topic/organization/whitney-museum-of-american-art}{Whitney}
or the
\href{https://www.nytimes.com/topic/organization/museum-of-modern-art}{Museum
of Modern Art}. Instead, the epicenter of the shock of the new was the
modestly endowed upstart that art-world insiders affectionately called
``the Jewish.''

Housed on upper Fifth Avenue in a Renaissance Revival mansion with a
Modernist annex, \href{https://thejewishmuseum.org/}{the Jewish Museum}
is where the dealer
\href{https://www.nytimes.com/1999/08/23/arts/leo-castelli-influential-art-dealer-dies-at-91.html}{Leo
Castelli} discovered the work of
\href{https://www.nytimes.com/2019/02/18/t-magazine/jasper-johns.html}{Jasper
Johns}, where the 37-year-old
\href{https://www.nytimes.com/2015/06/03/t-magazine/robert-rauschenberg-endless-combinations.html}{Robert
Rauschenberg} had his initial retrospective and where the 1966
exhibition ``Primary Structures'' caught and popularized the cresting
wave of Minimalist sculpture. It was at the Jewish that New York museum
audiences were introduced to such diverse artists as
\href{https://www.nytimes.com/slideshow/2016/10/27/t-magazine/the-work-and-communities-of-mark-di-suvero.html}{Mark
di Suvero},
\href{https://tmagazine.blogs.nytimes.com/2014/02/06/artists-on-artists-richard-tuttle-on-frederic-remington-purveyor-of-wild-west-nostalgia/}{Richard
Tuttle},
\href{https://www.nytimes.com/1992/10/31/arts/joan-mitchell-abstract-artist-is-dead-at-66.html}{Joan
Mitchell},
\href{https://www.nytimes.com/1980/12/27/archives/tony-smith-68-sculptor-of-minimalist-structures-represented.html}{Tony
Smith},
\href{https://www.nytimes.com/2020/02/13/arts/design/robert-irwin-pace-gallery.html}{Robert
Irwin},
\href{https://www.nytimes.com/2008/11/18/arts/design/18hartigan.html}{Grace
Hartigan},
\href{https://www.nytimes.com/2002/08/16/arts/larry-rivers-artist-with-an-edge-dies-at-78.html}{Larry
Rivers},
\href{https://www.nytimes.com/2015/04/20/t-magazine/sunrise-tour-donald-judd-chinati-marfa.html}{Donald
Judd},
\href{https://www.nytimes.com/2018/11/29/obituaries/robert-morris-dead.html}{Robert
Morris},
\href{https://www.nytimes.com/2014/05/05/arts/design/carl-andre-emerges-to-guide-installation-at-diabeacon.html}{Carl
Andre},
\href{https://www.nytimes.com/1996/12/04/arts/dan-flavin-63-sculptor-of-fluorescent-light-dies.html}{Dan
Flavin} and
\href{https://www.nytimes.com/2000/06/10/arts/george-segal-pop-sculptor-dies-at-75-molded-plaster-people-of-a-ghostly-angst.html}{George
Segal}. And it was there, too, that
\href{https://www.nytimes.com/2011/12/28/arts/helen-frankenthaler-abstract-painter-dies-at-83.html}{Helen
Frankenthaler},
\href{https://www.nytimes.com/2010/01/06/arts/06noland.html}{Kenneth
Noland} and \href{https://www.nytimes.com/topic/person/ad-reinhardt}{Ad
Reinhardt} received their first solo museum shows. ``It was the most
radical place in New York City, the only place you could see seriously
installed contemporary art,'' says the artist
\href{https://www.nytimes.com/2019/11/26/t-magazine/mel-bochner.html}{Mel
Bochner}. ``These shows were groundbreaking and established the
possibility of a future. Important things were being discussed at the
\href{https://www.nytimes.com/topic/organization/jewish-museum-nyc}{Jewish
Museum} that weren't being said elsewhere.''

\href{https://www.nytimes.com/issue/t-magazine/2020/07/02/true-believers-art-issue}{\includegraphics{https://static01.nyt.com/newsgraphics/2020/06/29/tmag-art-embeds-new/assets/images/art_issue_gif_special_editon.gif}}

To understand how a museum controlled by the tradition-bound Jewish
Theological Seminary came to spearhead the artistic avant-garde, it is
necessary to appreciate the social status of Jewish Americans in the
postwar years, the minor position of contemporary art at that time and
the role played by a few inspired, ambitious curators. In 1947, the
Jewish Museum expanded out of the seminary's library annex, which
contained a collection of historic Judaica, and moved into the mansion
given to it by Frieda Schiff Warburg, who was the wife and daughter of
renowned financiers. At the time, the few Jews welcomed as trustees at
elite cultural institutions like the
\href{https://www.nytimes.com/topic/organization/metropolitan-museum-of-art}{Metropolitan
Museum of Art} in New York and the
\href{https://www.nytimes.com/topic/organization/art-institute-of-chicago}{Art
Institute of Chicago} came from wealthy, established German Jewish (as
opposed to Eastern European Jewish) families --- and like the board
members, the art that those museums collected needed to have passed the
test of time. Even the avowedly ``modern'' museums in New York hesitated
to display the achievements of young American artists.

\includegraphics{https://static01.nyt.com/images/2020/07/13/t-magazine/art/Jewish-museum-slide-QJN8/Jewish-museum-slide-QJN8-articleLarge.jpg?quality=75\&auto=webp\&disable=upscale}

Many of the collectors who flocked to the less snobbish and more
affordable precincts of contemporary art were Jewish --- among them,
\href{https://www.nytimes.com/1986/01/03/obituaries/robert-scull-prominent-collector-of-pop-art.html}{Robert}
and
\href{https://www.nytimes.com/2001/09/01/arts/ethel-scull-a-patron-of-pop-and-minimal-art-dies-at-79.html}{Ethel
Scull}, who ran a New York taxi business and had a preternatural eye for
the contemporary, and
\href{https://www.nytimes.com/1994/09/13/obituaries/frederick-weisman-82-leader-in-the-business-and-art-worlds.html}{Frederick}
and
\href{https://www.nytimes.com/1991/10/22/arts/marcia-weisman-collector-73-supporter-of-major-art-museums.html}{Marcia
Weisman}, who lived in Los Angeles and had made a fortune in real
estate. Another enthusiast was
\href{https://www.nytimes.com/2002/10/13/nyregion/vera-g-list-94-is-dead-philanthropist-and-collector.html}{Vera
List}, the wife of
\href{https://www.nytimes.com/1987/09/12/obituaries/albert-a-list-86-industrialist-who-supported-many-causes.html}{Albert
List}, a self-made millionaire industrialist of Romanian Jewish heritage
who was on the board of the Jewish Theological Seminary. When Vera
proposed donating a sculpture, ``The Procession,'' by
\href{http://www.elbertweinberg.com/index.html}{Elbert Weinberg}, to the
Jewish Museum in 1957, the trustees hesitated. ``The Procession'' was a
figurative piece that represented men carrying a Torah and a menorah,
rendered in Modernist angles. Some board members questioned whether the
acquisition was permissible, considering the Second Commandment
prohibition of graven images. Eventually, loath to rebuff the generosity
of a significant benefactor, the board approved the donation.

In 1962, Vera List became the chair of the Jewish Museum's newly
constituted board of governors, which declared its allegiance to ``the
work of younger or otherwise unacknowledged'' artists. The senior staff
resigned in protest, but the shift was less abrupt than it might seem,
because Stephen Kayser, the exiting de facto director, had also
championed contemporary art as a way of attracting visitors. For a
10th-anniversary show in 1957, with advice from the distinguished art
historian
\href{https://www.nytimes.com/1996/03/04/us/meyer-schapiro-91-is-dead-his-work-wove-art-and-life.html}{Meyer
Schapiro}, Kayser mounted the far-reaching and farsighted ``Artists of
the New York School: Second Generation,'' at which Castelli first laid
eyes on Johns's 1955 painting ``Green Target.'' Intrigued, Castelli
visited Johns's studio, reacted with messianic fervor to what he was
shown and in early 1958, presented the show that launched Johns's fame
and changed the course of postwar American art. By sparking the interest
and sculpture donation of List, ``Artists of the New York School'' also
altered the trajectory of the Jewish Museum --- and placed it on a path
on which it would help predict and shape the movement of contemporary
art for years to come.

KAYSER'S DEPARTURE amounted to a change of emphasis and style. A
specialist in Jewish art, he had emigrated from Germany in 1938.
``Stephen Kayser was an old-fashioned gentleman,'' recalls Tom
Freudenheim, who was a curator at the museum in the early '60s. ``He
still had his heavy accent. He was of that generation of people who had
a formality about them.'' His successor,
\href{https://www.nytimes.com/1970/03/01/archives/alan-r-solomon-19201970.html}{Alan
R. Solomon}, was a Harvard-trained academic and curator who had founded
the art gallery at Cornell University. ``Alan Solomon was a quite
sophisticated art historian who, as they say in the art world, happened
to be Jewish,'' Freudenheim continued. ``He had very, very high
standards, and he would absolutely not yield on them.'' As Vera List put
it in an oral history at the \href{https://www.aaa.si.edu/}{Smithsonian
Archives of American Art}, Kayser's shows were ``like a whisper,'' and
when Solomon came in, ``he had much more space, and he wasn't
whispering; he was bellowing.''

Solomon served as director for only two years, from July 1962 to July
1964. Despite the brevity of his career, **** he set the museum on its
course. For his debut exhibition, in March 1963, he presented a
Rauschenberg retrospective of paintings, transfer drawings and the
three-dimensional ``combines'' that melded elements of both painting and
sculpture. He followed with
``\href{https://www.amazon.com/Toward-Abstraction-Heller-Steinberg-Solomon/dp/B000PSXR2A}{Toward
a New Abstraction}'' that summer, which provided an overview of
hard-edge abstract painting and included the shaped canvases of
\href{https://www.nytimes.com/2020/03/18/t-magazine/frank-stella.html}{Frank
Stella} and the monochromes of
\href{https://www.nytimes.com/2018/02/08/t-magazine/ellsworth-kelly-austin-last-work.html}{Ellsworth
Kelly}. His 1964 solo show of the 33-year-old Johns was even more
magisterial than the Rauschenberg exhibition.

Image

Jasper Johns's ``Green Target'' (1955), the painting that launched the
artist's career after it debuted at the Jewish Museum.Credit...Jasper
Johns, ``Green Target,'' 1955, encaustic on newspaper and cloth over
canvas. Digital image © The Museum Of Modern Art/licensed By Scala/Art
Resource, N.Y. © 2019 Jasper Johns/licensed By VAGA at ARS, N.Y.

But it wasn't just \emph{what} he showed that would make Solomon a key
figure in reshaping the mission of the modern art museum --- it was when
and how he showed it. ``That was where Alan was really prescient,''
Freudenheim says. ``You might say he was the father of the midcareer
retrospective.'' Solomon treated his contemporaries like old masters,
applying the most rigorous expectations to shows of artists who were
still in their 30s. Bochner, who worked as a guard at the museum in the
early '60s, was once accosted by Solomon in a gallery that was
displaying prints by
\href{https://www.nytimes.com/topic/person/marc-chagall}{Marc Chagall}.
``He looked at me and said, `That print is crooked,''' Bochner recalls.
``I said, `I don't know, sir, it looks OK to me.' He had a tape measure
in his pocket. He took it out and measured. It was one-sixteenth of an
inch off. He said very imperiously, `Please correct that.' He put the
tape back in his pocket and walked off.''

Acknowledging what he termed in his foreword to the Johns catalog to be
the museum's ``two- fold policy,'' Solomon also staged shows, like ****
the Chagall prints, with a clear Jewish content. He recognized that the
museum was the subsidiary of a seminary. He was even able, in a 1963
show of modern American synagogue architecture, organized by
\href{https://www.nytimes.com/topic/person/richard-meier}{Richard
Meier}, to perform the neat topological trick of combining the
contemporary and the Jewish into one. But the tension between some board
members' interests and Solomon's own lingered. In an often told tale,
the widow of an eminent seminary scholar asked at the Rauschenberg
opening whether the artist was a Jew. Told that he was not, she
exclaimed, ``Thank God!''

Rauschenberg would be chosen to represent the United States at the
Venice Biennale the following year, in a presentation organized by
Solomon and the Jewish, and would win the show's grand prize in
painting. But Solomon spent so much time in Italy that the seminary
board lost confidence in him. Nonetheless, the appointment of
\href{https://www.nytimes.com/2014/08/26/arts/sam-hunter-curator-and-museum-founder-dies-at-91.html}{Sam
Hunter} as his successor in 1965 only affirmed the museum's commitment
to the avant-garde. During his tenure, Hunter organized retrospectives
of Ad Reinhardt, Philip Guston and Max Ernst. His signature achievement,
however, was recruiting from MoMA as a senior curator
\href{https://stories.thejewishmuseum.org/the-jewish-museum-remembers-kynaston-mcshine-1d6741c02815}{Kynaston
McShine}, the witty, opinionated scion of a distinguished Black Trinidad
family, whose exhibitions would herald some of the most important trends
in contemporary art. McShine was at the Jewish **** from 1965 to 1968
(the last year as acting director), a break from his otherwise lifelong
tenure at the Modern. But in that short sojourn, he contributed greatly
to the Jewish's mythology. Inspired by the sizable **** main exhibition
gallery in the new wing, he staged ``Large Scale American Paintings'' in
the summer of 1967, displaying canvases of 23 American artists,
including
\href{https://www.nytimes.com/2016/05/02/t-magazine/art/al-held-brushstroke-drawings.html}{Al
Held}'s truly monumental ``Greek Garden,'' **** a geometric study in
acrylic, which stretches 56 feet long. (In June 1965, the museum had
displayed in the same room
\href{https://www.nytimes.com/2016/03/18/arts/design/james-rosenquist-and-erro-discuss-a-long-friendship-forged-in-pop-art.html}{James
Rosenquist}'s 86-foot-long ``F-111,'' a Pop Art depiction of the titular
bomber plane, and a recent purchase of the Sculls'.) Earlier that year,
McShine mounted the first American museum retrospective of Yves Klein,
already a legend not even five years after his premature death.

Image

Artists and guests in front of Robert Rauschenberg's ``Barge,'' 1962-63,
at the opening of the artist's retrospective at the Jewish Museum in
March 1963. Standing, from left: Sherman Drexler, Claes Oldenburg,
Richard Lippold, Merce Cunningham, Robert Murray, Peter Agostini, Edward
Higgins, Barnett Newman, Robert Rauschenberg, Perle Fine, Alfred Jensen,
Ray Parker, Friedel Dzubas, Ernst Van Leyden, Andy Warhol, Marisol,
James Rosenquist, John Chamberlain and George Segal. Kneeling, from
left: Jon Schueler, Arman, David Slivka, Alfred Leslie, Tania, Frederick
Kiesler, Lee Bontecou, Isamu Noguchi, Salvatore Scarpitta and Allan
Kaprow.Credit...Courtesy of the Jewish Museum, N.Y.

The show for which McShine is best remembered --- and which is one of
the most celebrated exhibitions of the late 20th century --- is
``Primary Structures: Younger American and British Sculptors,'' from
1966. McShine assembled works by East Coast, California and British
sculptors, early in their careers, who shared what we now call a
Minimalist aesthetic. Here for the first time together were artists like
Carl Andre, Donald Judd, Dan Flavin,
\href{https://www.nytimes.com/2018/06/15/t-magazine/mary-heilmann-larry-bell-conversation.html}{Larry
Bell},
\href{https://www.nytimes.com/2018/11/21/t-magazine/female-land-artists.html}{Anne
Truitt}, Ellsworth Kelly, Robert Morris and
\href{https://www.nytimes.com/1988/02/04/obituaries/ronald-bladen-69-sculptor-famed-for-stark-poetic-images.html}{Ronald
Bladen}, who used various materials (painted steel or aluminum, colored
plastic, coated glass) in different ways (structured repetition of
prefabricated units, heroically scaled steel) but shared an interest in
machine-made objects, smooth planes of vibrant color and the removal of
the sculpture from a pedestal. ``It had tremendous impact because it was
really the first show including those artists,'' says the dealer
\href{https://www.nytimes.com/slideshow/2016/10/11/t-magazine/my-life-in-pictures-paula-cooper/s/paula-cooper-slide-010G.html}{Paula
Cooper}, who went on to represent many of the show's contributors at the
eponymous New York gallery that she founded in 1968. ``I think it was
the beginning.''

``Primary Structures'' had its eye trained so thoroughly on the future
that it would take years for its importance to be recognized.
\href{https://www.nytimes.com/2018/02/07/t-magazine/judy-chicago-dinner-party.html}{Judy
Chicago}, then known as Judy Gerowitz, exhibited
``\href{https://www.artsy.net/artwork/judy-chicago-rainbow-pickett}{Rainbow
Pickett},'' **** a sequence of six brightly painted wooden beams that
leaned against the wall. ``I got nowhere with a lot of that big
sculpture,'' she says. ``My male peers would get picked up and be on the
choo-choo train, and I had to constantly start over again. After a
decade and a half of that, I changed direction.''
\href{https://www.paulacoopergallery.com/artists/robert-grosvenor/selected-works}{Robert
Grosvenor}, who installed ``Transoxiana,'' a 31-foot-high V-shaped
sculpture of painted wood and steel that was destroyed after the
exhibition, says the show ``had no impact whatsoever'' on his career.
For Hunter, too, ``Primary Structures'' did little to help his standing
at the museum. He was forced to resign in October 1967.

Hunter's
\href{https://www.nytimes.com/1968/04/03/archives/jewish-museum-finds-its-new-director-brooklynborn-karl-katz-in.html}{replacement},
Karl Katz, who had been a curator at the Israel Museum in Jerusalem, had
a reputation as being ``open to pretty much anything,'' says the curator
Susan Tumarkin Goodman, who in 1970 organized ``Using Walls,'' with a
group of artists --- the roster included
\href{https://www.nytimes.com/2013/02/11/arts/design/richard-artschwager-painter-and-sculptor-dies-at-89.html}{Richard
Artschwager},
\href{https://www.lissongallery.com/artists/lawrence-weiner}{Lawrence
Weiner}, Richard Tuttle, Sol LeWitt and Bochner --- who drew and painted
directly onto **** the museum's walls. But Katz's daredevil spirit would
also indirectly end the museum's improbable run as the primary promoter
of the avant-garde. His 1970 exhibition ``Software: Information
Technology and Its New Meaning for Art'' was a flawed but visionary look
at the impact of computer science on art. Everything in the show --- the
exhibitions, the performing artists --- ran on programmed instructions
or were issued from a prescribed system. Those **** artists included
\href{https://www.nytimes.com/2017/04/28/arts/design/vito-acconci-dead-performance-artist.html}{Vito
Acconci},
\href{https://www.nytimes.com/2020/01/05/arts/john-baldessari-dead.html}{John
Baldessari},
\href{https://www.nytimes.com/2006/04/10/arts/design/allan-kaprow-creator-of-artistic-happenings-dies-at-78.html}{Allan
Kaprow}, \href{https://www.skny.com/artists/joseph-kosuth}{Joseph
Kosuth} and
\href{https://www.nytimes.com/2006/01/31/arts/design/nam-june-paik-73-dies-pioneer-of-video-art-whose-work-broke.html}{Nam
June Paik}, most of whom were still largely unknown to the general
public. The day before ``Software'' opened, Katz gave a tour to the
seminary chancellor, Rabbi Louis Finkelstein, and a representative of
the Smithsonian, which wanted to stage the show and therefore defray
some of its costs. The three men viewed Nicholas Negroponte's
installation, ``Seek,'' in which a computer-controlled claw moved 2,000
metal-coated plastic cubes of a maze navigated by gerbils. Then they
advanced to a video recording by Les Levine. As Katz recalls in his
memoir, all was fine until they got to the footage that depicted the
artist stark naked in the company of two equally unclad women. The rabbi
sputtered in furious disbelief.

``I think, Mr. Katz, that this is the end,'' he said.

And it was. A fire at the Smithsonian, coupled with technical failures
in the challenging show, led to the cancellation of the Smithsonian
showing, and ``Software'' finished at least \$50,000 over budget. The
combination of salaciousness and shortfall was insurmountable, and the
seminary declared it would no longer subsidize a program that was not
``basically Jewish.'' Katz submitted his resignation the next day.

Image

Kynaston McShine at the opening reception of ``Primary Structures'' in
1966.Credit...Courtesy of the Jewish Museum, N.Y.

Image

An installation view of Robert Grosvenor's ``Transoxiana'' (1965, top
right) and Robert Morris's ``Untitled (L-Beams)'' (1965) at the
``Primary Structures'' show.Credit...From left: © 2020 The Estate of
Robert Morris/Artists Rights Society (ARS), N.Y.; © Robert Grosvenor,
courtesy of Paula Cooper Gallery, N.Y. Photo: courtesy of the Jewish
Museum, N.Y.

THAT SHOULD HAVE been the final chapter in the museum's role in the
vanguard. By 1970, Solomon's program was becoming everybody's program.
It was no longer risky for prestigious museums to present contemporary
art because, commercially and critically, contemporary art was becoming
prestigious. Two years after returning to MoMA in 1968, McShine mounted
``Information,'' a survey of conceptual art that featured such Jewish
Museum exhibition alumni as Andre and Daniel Buren. In 1980, the Whitney
purchased Jasper Johns's
``\href{https://whitney.org/collection/works/1060}{Three Flags}'' for
\$1 million, a record at the time for **** a living artist. (A ``Flag''
painting the size of a place mat sold at auction in 2014 for \$36
million.) In a generation, contemporary art had gone from being viewed
with suspicion and slight distaste to being an asset, its acquisition an
announcement of both one's taste and one's wealth. So, too, had the
Jewish's once radical commitment to the midcareer retrospective gone
from daring to de rigueur; the show quickly became a museum-menu staple.

Then there was the fact that Jewish people no longer had to start their
own museums. After decades of exclusion, Jews, including those from
Eastern European backgrounds, began joining the boards of the major arts
institutions. Today, Daniel Brodsky is the chair of the Metropolitan,
Leon Black is the chair of MoMA and Leonard Lauder is the chairman
emeritus and éminence grise on the board at the Whitney.

The culture at large had finally caught up with the one the Jewish
Museum had helped create. And yet, improbably, the Jewish had one last
prediction to make. In 1972, the museum appointed as its director
\href{https://www.nytimes.com/1994/09/09/obituaries/joy-ungerleider-mayerson-74-former-head-of-jewish-museum.html}{Joy
Ungerleider-Mayerson}, a specialist in biblical archaeology. Under her
stewardship, the institution began to concentrate on Jewish themes and
Israeli artists; one of her first tasks as director was to negotiate the
acquisition of 600 ancient artifacts from Israel, and one of the major
shows during her tenure, from 1975, was called ``Jewish Experience in
the Art of the 20th Century.'' The museum went, in other words, from
advancing provocative art to concentrating on identity culture. And
viewed that way, the Jewish Museum was still on the cutting edge. The
\href{https://studiomuseum.org/}{Studio Museum} in Harlem, which
exhibits African-American art, opened in 1968;
\href{https://www.elmuseo.org/}{El Museo del Barrio}, which focuses on
Latino culture, was founded the next year. Today, the questions and
tensions that surround and are informed by identity culture --- who am I
and how do I fit within the society around me? --- also provide an
engine for contemporary art itself.

Under its current director,
\href{https://www.nytimes.com/2019/02/21/t-magazine/jane-mayle-claudia-gould.html}{Claudia
Gould}, who has a background in contemporary art, the museum straddles
its dual legacies. It continues to mount innovative, sometimes
forward-thinking shows, such as 2018's survey of
\href{https://www.nytimes.com/2019/02/14/t-magazine/martha-rosler.html}{Martha
Rosler}, the pioneering feminist and video artist whose politically
charged work from the '60s feels newly relevant today. With exhibitions
like
``\href{https://www.nytimes.com/2018/03/15/t-magazine/marc-camille-chaimowicz.html}{Marc
Camille Chaimowicz: Your Place or Mine},'' **** staged **** that same
year, the museum looked anew at an artist whose work had previously been
considered decorative. It is, however, still grappling with internal and
external debates over whether the museum is ``too Jewish'' or ``not
Jewish enough.'' In 2018, it installed an ongoing exhibition of Jewish
ceremonial objects integrated with artworks from its **** collection.
Some felt the show was facile and unscholarly; others found **** it
adventurous: an effort to interest younger visitors in objects they
might otherwise ignore.

In the summer of 2022, the Jewish Museum will present a survey of the
New York art scene from 1962 to 1964. During these years,
\href{https://www.nytimes.com/2018/05/02/t-magazine/andy-warhol-photo-portraits.html}{Andy
Warhol} opened his Factory studio in Midtown, Donald Judd laid out his
artistic philosophy in an essay called ``Specific Objects'' and Pace
Gallery, now one of the largest in the world, moved from Boston to
Manhattan. And then there was the Jewish Museum itself --- pushing the
idea of what a museum could and should be in different and previously
unheard-of directions.

\hypertarget{true-believers-art-issue}{%
\subsubsection{\texorpdfstring{\href{https://www.nytimes.com/issue/t-magazine/2020/07/02/true-believers-art-issue}{True
Believers Art
Issue}}{True Believers Art Issue}}\label{true-believers-art-issue}}

Advertisement

\protect\hyperlink{after-bottom}{Continue reading the main story}

\hypertarget{site-index}{%
\subsection{Site Index}\label{site-index}}

\hypertarget{site-information-navigation}{%
\subsection{Site Information
Navigation}\label{site-information-navigation}}

\begin{itemize}
\tightlist
\item
  \href{https://help.nytimes.com/hc/en-us/articles/115014792127-Copyright-notice}{©~2020~The
  New York Times Company}
\end{itemize}

\begin{itemize}
\tightlist
\item
  \href{https://www.nytco.com/}{NYTCo}
\item
  \href{https://help.nytimes.com/hc/en-us/articles/115015385887-Contact-Us}{Contact
  Us}
\item
  \href{https://www.nytco.com/careers/}{Work with us}
\item
  \href{https://nytmediakit.com/}{Advertise}
\item
  \href{http://www.tbrandstudio.com/}{T Brand Studio}
\item
  \href{https://www.nytimes.com/privacy/cookie-policy\#how-do-i-manage-trackers}{Your
  Ad Choices}
\item
  \href{https://www.nytimes.com/privacy}{Privacy}
\item
  \href{https://help.nytimes.com/hc/en-us/articles/115014893428-Terms-of-service}{Terms
  of Service}
\item
  \href{https://help.nytimes.com/hc/en-us/articles/115014893968-Terms-of-sale}{Terms
  of Sale}
\item
  \href{https://spiderbites.nytimes.com}{Site Map}
\item
  \href{https://help.nytimes.com/hc/en-us}{Help}
\item
  \href{https://www.nytimes.com/subscription?campaignId=37WXW}{Subscriptions}
\end{itemize}
