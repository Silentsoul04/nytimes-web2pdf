Sections

SEARCH

\protect\hyperlink{site-content}{Skip to
content}\protect\hyperlink{site-index}{Skip to site index}

\href{https://www.nytimes.com/section/politics}{Politics}

\href{https://myaccount.nytimes.com/auth/login?response_type=cookie\&client_id=vi}{}

\href{https://www.nytimes.com/section/todayspaper}{Today's Paper}

\href{/section/politics}{Politics}\textbar{}Popular Chinese-Made Drone
Is Found to Have Security Weakness

\url{https://nyti.ms/2ZT3Pgx}

\begin{itemize}
\item
\item
\item
\item
\item
\item
\end{itemize}

Advertisement

\protect\hyperlink{after-top}{Continue reading the main story}

Supported by

\protect\hyperlink{after-sponsor}{Continue reading the main story}

\hypertarget{popular-chinese-made-drone-is-found-to-have-security-weakness}{%
\section{Popular Chinese-Made Drone Is Found to Have Security
Weakness}\label{popular-chinese-made-drone-is-found-to-have-security-weakness}}

Researchers found a potential vulnerability in an app that helps power
the drones, highlighting U.S. officials' concerns that Beijing could get
access to information about Americans.

\includegraphics{https://static01.nyt.com/images/2020/07/22/us/politics/22dc-drone/merlin_163640748_0360d3cd-0619-48a3-9d9a-3aba62be7222-articleLarge.jpg?quality=75\&auto=webp\&disable=upscale}

By \href{https://www.nytimes.com/by/paul-mozur}{Paul Mozur},
\href{https://www.nytimes.com/by/julian-e-barnes}{Julian E. Barnes} and
Aaron Krolik

\begin{itemize}
\item
  July 23, 2020
\item
  \begin{itemize}
  \item
  \item
  \item
  \item
  \item
  \item
  \end{itemize}
\end{itemize}

\href{https://cn.nytimes.com/usa/20200724/dji-drones-security-vulnerability/}{阅读简体中文版}\href{https://cn.nytimes.com/usa/20200724/dji-drones-security-vulnerability/}{阅读简体中文版}

Cybersecurity researchers revealed on Thursday a newfound vulnerability
in an app that controls the world's most popular consumer drones,
threatening to intensify the growing tensions between China and the
United States.

In two reports, the researchers contended that an app on Google's
Android operating system that powers drones made by China-based Da Jiang
Innovations, or DJI, collects large amounts of personal information that
could be exploited by the Beijing government. Hundreds of thousands of
customers across the world use the app to pilot their rotor-powered,
camera-mounted aircraft.

The world's largest maker of commercial drones, DJI has found itself
increasingly in the cross hairs of the United States government, as have
other successful Chinese companies. The Pentagon has banned the use of
its drones, and in January
\href{https://www.nytimes.com/2020/01/29/technology/interior-chinese-drones.html}{the
Interior Department} decided to continue grounding its fleet of the
company's drones over security fears. DJI said the decision was about
politics, not software vulnerabilities.

For months, U.S. government officials have stepped up warnings about the
Chinese government's potentially exploiting weaknesses in tech products
to force companies there to give up information about American users.
Chinese companies must comply with any government request to turn over
data, according to American officials.

``Every Chinese technology company is required by Chinese law to provide
information they obtain, or information stored on their networks, to
Chinese authorities if requested to do so,'' said William R. Evanina,
director of the National Counterintelligence and Security Center. ``All
Americans should be concerned that their images, biometrics, locational
and other data stored on Chinese apps must be turned over to China's
state security apparatus.''

The drone vulnerability, said American officials, is the kind of
security hole that worries Washington.

The security research firms that documented
it,\href{https://www.synacktiv.com/en/publications/dji-android-go-4-application-security-analysis.html}{Synacktiv,}
based in France,
\href{https://blog.grimm-co.com/2020/07/dji-privacy-analysis-validation.html}{and
GRIMM,}located outside Washington, found that the app not only collected
information from phones but that DJI can also update it without Google
reviewing the changes before they are passed on to consumers. That could
violate Google's Android developer terms of service.

The changes are also difficult for users to review, the researchers
said, and even when the app appears to be closed, it awaits instructions
from afar, they found.

``The phone has access to everything the drone is doing, but the
information we are talking about is phone information,'' said Tiphaine
Romand-Latapie, a Synacktiv engineer. ``We don't see why DJI would need
that data.''

Ms. Romand-Latapie acknowledged that the security vulnerability did not
amount to a backdoor, or a flaw that allowed hackers into a phone.

DJI says its app forces updates on users to stop hobbyists who try to
hack the app to circumvent government-imposed restrictions on where and
how high drone can fly.

``This safety feature in the Android version of one of our recreational
flight control apps blocks anyone from trying to use a hacked version to
override our safety features, such as altitude limits and geofencing,''
Brendan Schulman, a DJI spokesman, said in a statement. ``If a hacked
version is detected, users are prompted to download the official version
from our website.'' He added that the feature was not present in
software used by governments and companies.

Neither \href{https://www.synacktiv.com/en}{Synacktiv} nor
\href{https://www.grimm-co.com/}{GRIMM} disclose their clients, but both
have done work for aerospace companies and drone manufacturers that
could potentially compete with DJI.

A Google spokesman said the company was looking into the claims in the
new reports. Synacktiv did not find the same vulnerability in the drone
maker's iPhone application. Apple's App Store is available in China.

``This research is a good reminder that organizations need to pay
attention to the risks associated with the various technologies they're
using for operations,'' said Christopher Krebs, director of the
Cybersecurity and Infrastructure Security Agency.

Some of the privacy concerns about the drones are common across many
applications that scrape far more information than consumers may
realize. But other potential vulnerabilities outlined by the researchers
come from attempts to straddle the radically different internet
environments in China, where the government can demand user data with
near impunity, and in other places, like the United States, where
broader legal protections exist.

For instance, DJI's direct link to the Android app was most likely
designed as a workaround for Chinese policies that block Google in
China, forcing companies to send Android app updates themselves. App
makers in China must rely on a chaotic and competitive clutch of
websites and app stores to get their products to the consumer. Under
such limitations, updates are not easy, and some companies craft
software that can be upgraded directly when needed.

Much of the technical data that the app collects fits with Chinese
government surveillance practices, which require phones and drones to be
linked to a user's identity.

Such features look more like vulnerabilities in places like the United
States. And with U.S.-China ties at their lowest in decades, Washington
has taken an increasingly dim view of such issues, assuming that if
Beijing can exploit a flaw in technology, it eventually will.

An icon of Chinese innovation, as well as a longtime security concern in
the United States, DJI has struggled to allay worries about the safety
of its drones, which shoot movies, guard power plants, count wildlife
and assist military and the police. For years, it has responded
repeatedly to reports of vulnerabilities with patches and has worked
closely with the U. S. government to quash other fears.

Still, security researchers with Synacktiv said the pattern of problems
in DJI's code and its quickly implemented fixes, which suggested that
the company was already aware of some of the problems but had not fixed
them, were also reason for concern.

``It is the mix of all of that which has made us suspicious,'' said Ms.
Romand-Latapie. ``It makes the application quite dangerous for the user
if they are not aware of what the application is capable of doing.''

Synacktiv did not identify any malicious uploads but simply raised the
prospect that the drone app could be used that way.

A New York Times analysis of the software confirmed the functionality.
An attempt to update the app directly from DJI's servers delivered a
message indicating that the phone The Times used ``did not meet the
qualifications for an update package.''

While the federal government has largely stopped using Chinese-made
drones, state and local governments continue to use them, though they
have the option of using a professional version of the app that has
additional security measures.

Lin Qiqing contributed research.

Advertisement

\protect\hyperlink{after-bottom}{Continue reading the main story}

\hypertarget{site-index}{%
\subsection{Site Index}\label{site-index}}

\hypertarget{site-information-navigation}{%
\subsection{Site Information
Navigation}\label{site-information-navigation}}

\begin{itemize}
\tightlist
\item
  \href{https://help.nytimes.com/hc/en-us/articles/115014792127-Copyright-notice}{©~2020~The
  New York Times Company}
\end{itemize}

\begin{itemize}
\tightlist
\item
  \href{https://www.nytco.com/}{NYTCo}
\item
  \href{https://help.nytimes.com/hc/en-us/articles/115015385887-Contact-Us}{Contact
  Us}
\item
  \href{https://www.nytco.com/careers/}{Work with us}
\item
  \href{https://nytmediakit.com/}{Advertise}
\item
  \href{http://www.tbrandstudio.com/}{T Brand Studio}
\item
  \href{https://www.nytimes.com/privacy/cookie-policy\#how-do-i-manage-trackers}{Your
  Ad Choices}
\item
  \href{https://www.nytimes.com/privacy}{Privacy}
\item
  \href{https://help.nytimes.com/hc/en-us/articles/115014893428-Terms-of-service}{Terms
  of Service}
\item
  \href{https://help.nytimes.com/hc/en-us/articles/115014893968-Terms-of-sale}{Terms
  of Sale}
\item
  \href{https://spiderbites.nytimes.com}{Site Map}
\item
  \href{https://help.nytimes.com/hc/en-us}{Help}
\item
  \href{https://www.nytimes.com/subscription?campaignId=37WXW}{Subscriptions}
\end{itemize}
