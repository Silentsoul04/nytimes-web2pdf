Sections

SEARCH

\protect\hyperlink{site-content}{Skip to
content}\protect\hyperlink{site-index}{Skip to site index}

\href{https://www.nytimes.com/section/us}{U.S.}

\href{https://myaccount.nytimes.com/auth/login?response_type=cookie\&client_id=vi}{}

\href{https://www.nytimes.com/section/todayspaper}{Today's Paper}

\href{/section/us}{U.S.}\textbar{}Feds Sending Tactical Team to Seattle,
Expanding Presence Beyond Portland

\url{https://nyti.ms/32NAXIt}

\begin{itemize}
\item
\item
\item
\item
\item
\end{itemize}

Advertisement

\protect\hyperlink{after-top}{Continue reading the main story}

Supported by

\protect\hyperlink{after-sponsor}{Continue reading the main story}

\hypertarget{feds-sending-tactical-team-to-seattle-expanding-presence-beyond-portland}{%
\section{Feds Sending Tactical Team to Seattle, Expanding Presence
Beyond
Portland}\label{feds-sending-tactical-team-to-seattle-expanding-presence-beyond-portland}}

After outrage over the presence of federal agents in Portland, Ore., the
Trump administration is sending a team to Seattle. Officials say they
will be on standby.

\includegraphics{https://static01.nyt.com/images/2020/07/23/us/23UNREST-SEATTLE-durkan/merlin_174529029_bc10fbf9-be65-4068-87c6-414c528a0e22-articleLarge.jpg?quality=75\&auto=webp\&disable=upscale}

\href{https://www.nytimes.com/by/zolan-kanno-youngs}{\includegraphics{https://static01.nyt.com/images/2019/12/13/reader-center/author-zolan-kanno-youngs/author-zolan-kanno-youngs-thumbLarge.png}}\href{https://www.nytimes.com/by/adam-goldman}{\includegraphics{https://static01.nyt.com/images/2018/07/12/multimedia/author-adam-goldman/author-adam-goldman-thumbLarge.png}}\href{https://www.nytimes.com/by/mike-baker}{\includegraphics{https://static01.nyt.com/images/2020/05/19/reader-center/author-mike-baker/author-mike-baker-thumbLarge.png}}

By \href{https://www.nytimes.com/by/zolan-kanno-youngs}{Zolan
Kanno-Youngs}, \href{https://www.nytimes.com/by/adam-goldman}{Adam
Goldman} and \href{https://www.nytimes.com/by/mike-baker}{Mike Baker}

\begin{itemize}
\item
  Published July 23, 2020Updated July 25, 2020
\item
  \begin{itemize}
  \item
  \item
  \item
  \item
  \item
  \end{itemize}
\end{itemize}

The Trump administration, which has pledged to use the full force of the
government to protect federal property, expanded that effort on Thursday
by sending a team of tactical border officers to stand by for duty in
Seattle.

The Special Response Team being deployed is similar to the tactical
teams currently operating in
\href{https://www.nytimes.com/article/portland-protests-explained-protesters.html}{Portland,
Ore.}, where local officials have vehemently objected to their efforts
to subdue street protests. Seattle officials have also said they do not
want federal agents sent to target protesters.

Agents from the Special Response Team, operated under U.S. Customs and
Border Protection, are typically deployed for intense law enforcement
operations, similar to the agency's BORTAC group that has operated in
Portland.

``The C.B.P. team will be on standby in the area, should they be
required,'' the Federal Protective Service said in a statement about the
Seattle effort.

A spokesman for the agency, who requested anonymity to speak about the
operation, said the border officers were sent to back up the Federal
Protective Service officers charged with protecting federal buildings,
and would only be used if protests expected this weekend escalate out of
control.

The deployment to Seattle came on the same day that the inspector
general of the Justice Department announced an investigation into
tactics used by the federal agents in Portland and in front of Lafayette
Square in Washington, D.C., in early June.

The inspector general of the Department of Homeland Security, Joseph V.
Cuffari, is also conducting an inquiry into the tactics of the agents in
Portland. Mr. Cuffari said in a letter to Democrats that he expected to
examine the authority used to deploy agents to the city after President
Trump signed an executive order directing federal agencies to increase
security at monuments, statues and federal property.

The order prompted the Homeland Security Department to form teams that
were briefly deployed to multiple cities to guard federal property,
including Seattle, for the July 4 weekend. The tactical teams in
Portland have remained at a federal courthouse as tension with
protesters there has heightened.

Representative Bennie G. Thompson, a Democrat from Mississippi who
chairs the House Committee on Homeland Security, said on Thursday that
he would hold a hearing next week to examine the D.H.S. response to the
protests.

``The administration's actions are not only violent and clearly
politically motivated, they are anathema to the rights guaranteed by the
Constitution and a threat to every value for which our Republic
stands,'' Mr. Thompson said.

Seattle's mayor, Jenny Durkan, said in an interview that she spoke
earlier Thursday with Chad F. Wolf, the acting secretary of homeland
security. She said he had assured her that the administration had no
plans to deploy a surge of agents to Seattle and would not do so without
communicating with the city. She had not been alerted to plans to
position the tactical team, but said that the department may be
distinguishing between an active deployment and agents who are on
standby.

Ms. Durkan said she made it clear that the city did not need the help of
federal agents.

``Any deployment here would, in my view, undermine public safety,'' Ms.
Durkan said.

Protests in the Seattle area have quieted somewhat since police this
month
\href{https://www.nytimes.com/2020/07/01/us/seattle-protest-zone-CHOP-CHAZ-unrest.html}{cleared
the so-called Capitol Hill Organized Protest zone}, where demonstrators
had laid claim to several city blocks. But there have been signs that
demonstrations may be ramping back up, including on Thursday, when the
police said a group of protesters broke windows and lit fires in the
Capitol Hill neighborhood.

Alexei Woltornist, a spokesman for the Department of Homeland Security,
said in a statement that the expected presence in Seattle would be
smaller than that in Portland.

``There is no large-scale deployment of personnel to Seattle at this
time. As threats warrant, any large-scale use of law enforcement assets
will involve close coordination with local law enforcement,'' Mr.
Woltornist said. ``There are no other cities across the country that
have the same threats and lack of local law enforcement support as we
are experiencing in Portland.''

The Federal Protective Service said it routinely requested assistance
from other law enforcement agencies when there are threats to federal
properties.

Teams of federal agents from several agencies have been operating in
Portland for much of July, drawing outrage from city officials,
including Mayor Ted Wheeler. The mayor,
\href{https://www.nytimes.com/2020/07/23/us/portland-protest-tear-gas-mayor.html}{who
was hit with tear gas from the federal officers on Wednesday night},
said the presence of the agents had only inflamed tensions in a city
that was working to calm them.

The number of protesters in Portland has swollen in recent days into the
thousands, drawing out people who had not joined the protests earlier
but who were appalled that federal forces would be operating in the city
with such aggressive tactics. Some in the demonstrations have targeted
federal agents with lasers and frozen water bottles, and others set fire
to a police union building.

The agents have repeatedly fired tear gas and various kinds of
crowd-control munitions, leaving at least one protester bloodied. Videos
have shown
\href{https://www.nytimes.com/2020/07/17/us/portland-protests.html}{federal
agents seizing protesters} and guiding them into unmarked vans.

Advertisement

\protect\hyperlink{after-bottom}{Continue reading the main story}

\hypertarget{site-index}{%
\subsection{Site Index}\label{site-index}}

\hypertarget{site-information-navigation}{%
\subsection{Site Information
Navigation}\label{site-information-navigation}}

\begin{itemize}
\tightlist
\item
  \href{https://help.nytimes.com/hc/en-us/articles/115014792127-Copyright-notice}{©~2020~The
  New York Times Company}
\end{itemize}

\begin{itemize}
\tightlist
\item
  \href{https://www.nytco.com/}{NYTCo}
\item
  \href{https://help.nytimes.com/hc/en-us/articles/115015385887-Contact-Us}{Contact
  Us}
\item
  \href{https://www.nytco.com/careers/}{Work with us}
\item
  \href{https://nytmediakit.com/}{Advertise}
\item
  \href{http://www.tbrandstudio.com/}{T Brand Studio}
\item
  \href{https://www.nytimes.com/privacy/cookie-policy\#how-do-i-manage-trackers}{Your
  Ad Choices}
\item
  \href{https://www.nytimes.com/privacy}{Privacy}
\item
  \href{https://help.nytimes.com/hc/en-us/articles/115014893428-Terms-of-service}{Terms
  of Service}
\item
  \href{https://help.nytimes.com/hc/en-us/articles/115014893968-Terms-of-sale}{Terms
  of Sale}
\item
  \href{https://spiderbites.nytimes.com}{Site Map}
\item
  \href{https://help.nytimes.com/hc/en-us}{Help}
\item
  \href{https://www.nytimes.com/subscription?campaignId=37WXW}{Subscriptions}
\end{itemize}
