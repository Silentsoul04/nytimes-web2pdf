Sections

SEARCH

\protect\hyperlink{site-content}{Skip to
content}\protect\hyperlink{site-index}{Skip to site index}

\href{https://www.nytimes.com/section/books}{Books}

\href{https://myaccount.nytimes.com/auth/login?response_type=cookie\&client_id=vi}{}

\href{https://www.nytimes.com/section/todayspaper}{Today's Paper}

\href{/section/books}{Books}\textbar{}Meet the Brave but Overlooked
Women of Color Who Fought for the Vote

\url{https://nyti.ms/3jBeWCU}

\begin{itemize}
\item
\item
\item
\item
\item
\end{itemize}

Advertisement

\protect\hyperlink{after-top}{Continue reading the main story}

Supported by

\protect\hyperlink{after-sponsor}{Continue reading the main story}

\hypertarget{meet-the-brave-but-overlooked-women-of-color-who-fought-for-the-vote}{%
\section{Meet the Brave but Overlooked Women of Color Who Fought for the
Vote}\label{meet-the-brave-but-overlooked-women-of-color-who-fought-for-the-vote}}

In ``Finish the Fight!,'' excerpted here, New York Times journalists
tell the stories of lesser-known figures in the battle to make the 19th
Amendment a reality.

\includegraphics{https://static01.nyt.com/images/2020/07/24/multimedia/24suffrage-book-07/24suffrage-book-07-articleLarge.jpg?quality=75\&auto=webp\&disable=upscale}

\href{https://www.nytimes.com/by/veronica-chambers}{\includegraphics{https://static01.nyt.com/images/2018/12/07/multimedia/author-veronica-chambers/author-veronica-chambers-thumbLarge.png}}\href{https://www.nytimes.com/by/jennifer-schuessler}{\includegraphics{https://static01.nyt.com/images/2018/02/16/multimedia/author-jennifer-schuessler/author-jennifer-schuessler-thumbLarge-v2.png}}\href{https://www.nytimes.com/by/amisha-padnani}{\includegraphics{https://static01.nyt.com/images/2018/06/12/multimedia/author-amy-padnani/author-amy-padnani-thumbLarge-v4.png}}\href{https://www.nytimes.com/by/jennifer-harlan}{\includegraphics{https://static01.nyt.com/images/2019/09/25/reader-center/author-jennifer-harlan/author-jennifer-harlan-thumbLarge-v2.png}}\href{https://www.nytimes.com/by/sandra-e-garcia}{\includegraphics{https://static01.nyt.com/images/2020/07/10/reader-center/author-sandra-e-garcia/author-sandra-e-garcia-thumbLarge.png}}\href{https://www.nytimes.com/by/vivian-wang}{\includegraphics{https://static01.nyt.com/images/2018/06/14/multimedia/author-vivian-wang/author-vivian-wang-thumbLarge-v2.png}}

By \href{https://www.nytimes.com/by/veronica-chambers}{Veronica
Chambers},
\href{https://www.nytimes.com/by/jennifer-schuessler}{Jennifer
Schuessler}, \href{https://www.nytimes.com/by/amisha-padnani}{Amisha
Padnani}, \href{https://www.nytimes.com/by/jennifer-harlan}{Jennifer
Harlan}, \href{https://www.nytimes.com/by/sandra-e-garcia}{Sandra E.
Garcia} and \href{https://www.nytimes.com/by/vivian-wang}{Vivian Wang}

\begin{itemize}
\item
  July 24, 2020
\item
  \begin{itemize}
  \item
  \item
  \item
  \item
  \item
  \end{itemize}
\end{itemize}

\emph{``}\href{https://www.nytco.com/press/hmh-books-media-and-the-new-york-times-collaborate-on-nonfiction-project-highlighting-the-diverse-women-who-fought-for-voting-rights/}{\emph{Finish
the Fight!}}\emph{'' is a history of the American suffrage movement for
middle-grade readers. The following excerpts have been edited and
condensed.}

It took the better part of a century to pass a law saying American women
had the right to vote. Three generations of women, and their male
allies, worked tirelessly to make the 19th Amendment --- which decreed
that states could not discriminate at the polls on the basis of sex ---
a reality. We call the right to vote ``suffrage,'' but for a long time,
that word was a kind of shorthand for women's rights. Without the vote,
suffragists argued, women had little say over their lives and their
futures and certainly much less when it came to the larger political
questions that shaped the nation.

The 19th Amendment is a cornerstone of gender equality in our country,
yet many of us know very little about the way the right to vote was won.
For a long time, the history of the suffrage movement has been told
mainly as the story of a few famous white women, such as Elizabeth Cady
Stanton and Susan B. Anthony. It's true they were among the most
important leaders of the movement in the 19th century.

\emph{{[}}\href{https://www.nytimes.com/2020/07/23/theater/finish-the-fight-suffrage-centennial-performance.html}{\emph{Join
us on Aug. 18 for a new play, based on ``Finish the Fight!'' Read more
here}}\emph{.{]}}

But there were tons more women who helped make suffrage a reality:
African-American women such as the writer and orator Frances Ellen
Watkins Harper, the community organizer Juno Frankie Pierce and the
journalists Josephine St. Pierre Ruffin, Elizabeth Piper Ensley and Ida
B. Wells-Barnett, who championed both suffrage and civil rights; Native
American women such as Susette La Flesche Tibbles and Zitkala-Sa; queer
women like the poet Angelina Weld Grimké and the educator Mary Burrill;
Latina women like Jovita Idár, who protected her family's newspaper and
the rights of Mexican-Americans; and Asian-American women like Mabel
Ping-Hua Lee, who led thousands of marchers in a 1912 suffrage parade in
New York. They all fought for the vote as part of a broader struggle for
equality, but their stories aren't nearly as well known as they should
be.

Shirley Chisholm, who, in a tribute to the suffragists, wore white on
the day in 1968 when she became the first African-American woman elected
to Congress, reportedly said, ``If they don't give you a seat at the
table, bring a folding chair.'' We hope that this book helps set a place
at the table for some of the many incredible women who played their part
in the battle for suffrage and equal rights for women.

\includegraphics{https://static01.nyt.com/images/2020/07/24/multimedia/24suffrage-book-08/24suffrage-book-08-articleLarge.jpg?quality=75\&auto=webp\&disable=upscale}

\hypertarget{mary-church-terrell-and-the-power-of-language}{%
\subsection{Mary Church Terrell and the Power of
Language}\label{mary-church-terrell-and-the-power-of-language}}

Sometimes freedom is a matter of timing. Mary Church Terrell knew that
lesson well. She was born in Memphis in September 1863 --- the middle of
the Civil War. Her parents had been enslaved, but Mary was born free,
and she charted a course of leadership that helped change the lives of
women and men across the nation. She became a suffragist. She fought for
the rights of all people of color. Holding America to the promises of
the Declaration of Independence --- life, liberty and the pursuit of
happiness for all --- became her life's work.

These dreams were supported by her parents. Her father, Robert Church,
was the son of an enslaved woman and a wealthy steamship owner who had
allowed Robert to keep his wages. After Robert gained his freedom, he
invested in real estate and became wealthy.

Mary was accepted at Oberlin College, which was founded by abolitionists
and was one of the first colleges in the United States to admit women
and African-Americans. She would later write in her autobiography, ``A
Colored Woman in a White World,'' that ``it would be difficult for a
colored girl to go through a white school with fewer unpleasant
experiences occasioned by race prejudice than I had.''

Mary had some extraordinary experiences. During her first year, she was
invited to Washington by Blanche K. Bruce, one of the first
African-American senators. He asked Mary to attend the inauguration of
President James A. Garfield as his guest. It was during that trip that
she met the great orator and activist Frederick Douglass. She would
later follow in his footsteps, using her gift for language to speak up
for the causes she believed in.

She also wrote a paper at Oberlin on the topic of suffrage, titled
``Should an Amendment to the Constitution Allowing Women the Ballot Be
Adopted?'' Mary became one of the first Black women to earn a college
degree in the United States, graduating with a bachelor's in classics in
1884.

Image

An illustration from ``Finish the Fight!'' shows the motto of the
National Association of Colored Women, ``lifting as we
climb.''Credit...Finish the Fight! published by HMH/Versify, Art by
Johnalynn Holland, 2020

Image

Mary Church Terrell in an undated portrait.Credit...Library of Congress

Later, after she earned a master's degree, Mary embarked on a two-year
tour of France, Switzerland, Italy and Germany, studying languages and
writing in her diary in French and German.

While Mary traveled the world, the United States grew more unsteady.
Lynching had become a form of domestic terrorism in the years after
slavery. Over decades, thousands of Black men and women were brutally
killed by white mobs, and their murderers were never prosecuted. The
government rarely made arrests in these cases, which only allowed the
number of lynchings to grow.

In 1895, Frederick
\href{https://www.nytimes.com/2019/02/14/obituaries/frederick-douglass-dead-1895.html}{died},
and Mary became the first Black woman appointed to the District of
Columbia Board of Education. She later raised funds and visited schools,
encouraging them to celebrate Douglass Day, a precursor to Black History
Month, in his honor.

In 1896, the Supreme Court delivered its ruling in Plessy v. Ferguson,
which declared segregation permissible under the Constitution, as long
as the segregated facilities and accommodations were ``equal.'' But in
reality, separate was rarely equal. That same year, Mary co-founded and
became the first president of the National Association of Colored Women,
a coalition of more than a hundred local Black women's clubs. The
organization's motto was ``lifting as we climb.''

Around this time, Mary began to champion the cause of suffrage. She
joined the National American Woman Suffrage Association (NAWSA) and was
one of very few Black members. Her years at Oberlin and abroad had made
her comfortable in predominantly white groups, and she took NAWSA to
task for excluding women of color. An inclusive movement, she reasoned,
would grow in both power and perspective.

``Seeking no favors because of our color, nor patronage because of our
needs,'' she said, ``we knock at the bar of justice, asking an equal
chance.''

In 1904, Mary was invited to speak at the International Congress of
Women in Berlin. The cost to attend was considerable, but her husband
encouraged her to go anyway. There she delivered a speech three times
--- in German, French and English. It was called ``The Progress of
Colored Women.''

She reminded her audience that her parents had been enslaved, that her
very being was a testament to how far one could travel on the road to
freedom. ``If anyone had had the courage to predict 50 years ago that a
woman with African blood in her veins would journey from the United
States to Berlin, Germany, to address an International Congress of Women
in the year 1904,'' she told the audience, ``he would either have been
laughed to scorn or he would have been immediately confined in an asylum
for the hopelessly insane.''

Mary knew that freedom for all was never about one battle. No single
great win --- the abolition of slavery, the passage of the 19th
Amendment --- would right the wrongs in a country founded on such
injustices as slavery and the denial of women's rights. But perhaps what
made her life most extraordinary is how much joy she got from each small
victory, how much stamina she displayed in her decades-long career as an
activist. In 1953, the year before Mary died, The Washington Post wrote:
``It may fairly be said of her that when she fought bigotry it was never
with hatred; she met lethargy and prejudice with spirit and
understanding. And she won the hearts as well as the minds of men.''

\hypertarget{mabel-ping-hua-lees-great-parade}{%
\subsection{Mabel Ping-Hua Lee's Great
Parade}\label{mabel-ping-hua-lees-great-parade}}

When Mabel Ping-Hua Lee moved to New York City from China as a child,
around 1905, there were few Chinese immigrants on the East Coast. In
1910, the census reported that there were 5,266 people of Chinese
descent living in the city, many of them in the neighborhood of
Chinatown in Lower Manhattan. It was a new community, and the streets
were alive with delicious smells, bright colors and voices from halfway
around the world.

By 1912, Mabel and her parents were living on Bayard Street in
Chinatown, and they had made a name for themselves. Mabel's father was a
minister who led the First Chinese Baptist Church and was fluent in
English. He was so active in the community that some referred to him as
the neighborhood's unofficial mayor.

Everyone also knew the daughter of the ``mayor,'' and they knew how
smart she was. She attended Erasmus Hall High School in Brooklyn and had
big plans to attend the women-only Barnard College, the sister school to
the then all-male Columbia College. She hoped to return to China one day
to open a school for girls.

Image

Mabel Ping-Hua Lee in the 1920s.Credit...Library of Congress

Image

An illustration from ``Finish the Fight!'' of Mabel on her white horse
during the 1912 suffrage parade in New York.Credit...Finish the Fight!
published by HMH/Versify, Art by Nhung Lê, 2020

Still, there was a limit to how much the community would stand behind
her. And Mabel crossed the line when she got involved with the suffrage
movement. The suffragists were considered radical --- how dare they
fight so steadfastly for equal rights? --- but Mabel believed that
voting was the key that would open every important door for women.

She joined the cause and persuaded her mother to join, too, even though
neither of them would be able to vote because the Chinese Exclusion Act
of 1882 prevented Chinese immigrants from becoming citizens. (It was
repealed in 1943.)

Her mother's participation in suffrage was so controversial that
newspapers even wrote about it: ``Tongues are still wagging in
Chinatown,'' The New York Tribune wrote, because Mabel and her mother
``went to a suffrage meeting.''

In 1912, when Mabel was just a teenager, she led a contingent of Chinese
and Chinese-American women in one of the biggest suffrage parades in
U.S. history. The New York Times
\href{https://timesmachine.nytimes.com/timesmachine/1912/05/05/100533097.html?pageNumber=1}{reported},
``Ten thousand strong, the army of those who believe in the cause of
woman's suffrage marched up Fifth Avenue at sundown yesterday in a
parade the like of which New York never knew before.''

Mabel didn't merely march. She rode a white horse at the start of the
parade, and she wore a three-cornered hat in the colors of the British
suffrage movement: purple to symbolize that the cause of suffrage was
noble; white for purity; and green, the color of spring, as a symbol of
hope. (American suffragists usually substituted the gold of the
sunflowers of Kansas --- where they waged some of their earliest
campaigns --- for green.)

In the fall, Mabel began her studies at Barnard. She majored in history
and philosophy, wrote articles about suffrage and feminism for The
Chinese Students' Monthly magazine and gave a speech, ``The Submerged
Half,'' which encouraged the Chinese immigrant community to promote
girls' education and women's rights. ``The welfare of China and possibly
its very existence as an independent nation depend on rendering tardy
justice to its womankind,'' she said. ``For no nation can ever make real
and lasting progress in civilization unless its women are following
close to its men if not actually abreast with them.''

Charlotte Brooks, the author of ``American Exodus: Second-Generation
Chinese Americans in China, 1901-1949,'' said that Mabel was part of a
generation of young Chinese-Americans who traveled back and forth
between China and the United States and saw the connections between the
two struggles.

``Something a lot of people in the U.S. don't realize is that Mabel's
activism grew out of China's New Culture Movement, which included the
idea that the suppression of women, and the poor treatment of women,
were both holding China back and represented a kind of backwardness,''
she explains.

Mabel went on to get a Ph.D. in economics from Columbia University,
becoming the first Chinese woman to earn a doctorate there. In 1921, she
published ``The Economic History of China: With Special Reference to
Agriculture.''

Mabel eventually took over as the director of her father's church, and
she founded the Chinese Christian Center, a community center on Pell
Street that offered English classes, health services, a kindergarten and
job training. She became an example of what a woman can do when given
the chance to learn and lead. As she wrote of China, ``In the fierce
struggle for existence among the nations, that nation is badly
handicapped which leaves undeveloped one half of its intellectual and
moral resources.'' The same, of course, was true of the United States.

Image

Credit...Finish the Fight! published by HMH/Versify, Art by Steffi
Walthall, 2020

Advertisement

\protect\hyperlink{after-bottom}{Continue reading the main story}

\hypertarget{site-index}{%
\subsection{Site Index}\label{site-index}}

\hypertarget{site-information-navigation}{%
\subsection{Site Information
Navigation}\label{site-information-navigation}}

\begin{itemize}
\tightlist
\item
  \href{https://help.nytimes.com/hc/en-us/articles/115014792127-Copyright-notice}{©~2020~The
  New York Times Company}
\end{itemize}

\begin{itemize}
\tightlist
\item
  \href{https://www.nytco.com/}{NYTCo}
\item
  \href{https://help.nytimes.com/hc/en-us/articles/115015385887-Contact-Us}{Contact
  Us}
\item
  \href{https://www.nytco.com/careers/}{Work with us}
\item
  \href{https://nytmediakit.com/}{Advertise}
\item
  \href{http://www.tbrandstudio.com/}{T Brand Studio}
\item
  \href{https://www.nytimes.com/privacy/cookie-policy\#how-do-i-manage-trackers}{Your
  Ad Choices}
\item
  \href{https://www.nytimes.com/privacy}{Privacy}
\item
  \href{https://help.nytimes.com/hc/en-us/articles/115014893428-Terms-of-service}{Terms
  of Service}
\item
  \href{https://help.nytimes.com/hc/en-us/articles/115014893968-Terms-of-sale}{Terms
  of Sale}
\item
  \href{https://spiderbites.nytimes.com}{Site Map}
\item
  \href{https://help.nytimes.com/hc/en-us}{Help}
\item
  \href{https://www.nytimes.com/subscription?campaignId=37WXW}{Subscriptions}
\end{itemize}
