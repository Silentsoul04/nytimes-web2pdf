Sections

SEARCH

\protect\hyperlink{site-content}{Skip to
content}\protect\hyperlink{site-index}{Skip to site index}

\href{https://www.nytimes.com/section/nyregion}{New York}

\href{https://myaccount.nytimes.com/auth/login?response_type=cookie\&client_id=vi}{}

\href{https://www.nytimes.com/section/todayspaper}{Today's Paper}

\href{/section/nyregion}{New York}\textbar{}Did New York Just Get the
Nation's Most Liberal Legislature?

\url{https://nyti.ms/2OR02KD}

\begin{itemize}
\item
\item
\item
\item
\item
\item
\end{itemize}

Advertisement

\protect\hyperlink{after-top}{Continue reading the main story}

Supported by

\protect\hyperlink{after-sponsor}{Continue reading the main story}

\hypertarget{did-new-york-just-get-the-nations-most-liberal-legislature}{%
\section{Did New York Just Get the Nation's Most Liberal
Legislature?}\label{did-new-york-just-get-the-nations-most-liberal-legislature}}

Among those toppled by progressive insurgents was a Brooklyn assemblyman
who was first elected in 1972.

\includegraphics{https://static01.nyt.com/images/2020/07/24/nyregion/24nylegislature1/merlin_173641374_a7bd63aa-d753-4aeb-ba47-26c86a3ef2a7-articleLarge.jpg?quality=75\&auto=webp\&disable=upscale}

\href{https://www.nytimes.com/by/jesse-mckinley}{\includegraphics{https://static01.nyt.com/images/2018/02/20/multimedia/author-jesse-mckinley/author-jesse-mckinley-thumbLarge.jpg}}\href{https://www.nytimes.com/by/luis-ferre-sadurni}{\includegraphics{https://static01.nyt.com/images/2018/06/22/multimedia/author-luis-ferre-sadurni/author-luis-ferre-sadurni-thumbLarge.png}}

By \href{https://www.nytimes.com/by/jesse-mckinley}{Jesse McKinley} and
\href{https://www.nytimes.com/by/luis-ferre-sadurni}{Luis Ferré-Sadurní}

\begin{itemize}
\item
  Published July 24, 2020Updated Aug. 3, 2020
\item
  \begin{itemize}
  \item
  \item
  \item
  \item
  \item
  \item
  \end{itemize}
\end{itemize}

A slew of progressive challengers upset entrenched incumbents in the New
York Legislature in the recent
\href{https://www.nytimes.com/2020/08/03/nyregion/nyc-mail-ballots-voting.html}{Democratic
primary}, cementing their movement's influence in Albany and making it
likely that the state government will become one of the most liberal in
the nation.

The results, held up for weeks because of delays caused by the
coronavirus outbreak, set up potential clashes between an emboldened
Legislature eager to push the priorities of the left and Gov. Andrew M.
Cuomo, a moderate Democrat who generally favors a get-it-done
philosophy.

The most resonant symbol of the new wave was its defeat of Assemblyman
Joseph R. Lentol of Brooklyn, a Democratic stalwart and chair of the
powerful codes committee, who is serving his 24th term after first being
elected in 1972. Mr. Lentol, 77, conceded on Wednesday
to\href{https://www.emilyforassembly.com/}{Emily Gallagher}, 36, a
community activist in Greenpoint.

Image

Emily Gallagher, shown in a photo from her campaign website. Ms.
Gallagher, a democratic socialist, upset Joseph Lentol, a 24-term
assemblyman.

The primary wins have the newcomers and their legislative allies
dreaming of passing bills on issues like criminal justice reform,
affordable housing and tax increases on the very wealthy, as well as
pressing for greater power in the annual and all-important budget
negotiations, which are usually dominated by Mr. Cuomo.

Many will find common ground with a younger and more diverse crop of
legislators elected in 2018, when Democrats picked up eight seats in the
Senate to capture the majority.

The influence of those progressive lawmakers was first made clear last
year when the Legislature passed measures including changes to the
criminal justice and campaign finance systems; new gun control laws; new
rights for voters, immigrants and victims of violence; and bans on
plastic bags, toxic toys and offshore drilling.

Zohran Mamdani, a 28-year-old housing counselor and democratic socialist
who defeated Assemblywoman Aravella Simotas of Queens, said in an
interview that there was no question that the primary results would
``change the nature of the Assembly.''

``How much?'' he asked. ``That is what we're going to show in the next
year.''

The primary results continued a run of success for insurgent left-wing
candidates for legislative and congressional races, including
\href{https://www.nytimes.com/2020/07/17/nyregion/jamaal-bowman-eliot-engel.html}{the
victory of Jamaal Bowman} over Representative Eliot L. Engel, declared
last week.

It also signaled a high-water mark for the Democratic Socialists of
America (D.S.A.), whose candidates won five primary races and firmly
established themselves as an electoral force in New York City and inside
the Democratic Party.

Their swift rise in the state comes just two years after their most
famous standard-bearer, Representative Alexandria Ocasio-Cortez, stunned
the party by defeating Joseph Crowley, the No. 4 House Democrat at the
time, in a June primary.

Mr. Lentol was actually endorsed by the Working Families Party, a
progressive labor-backed organization, and spent more money than any
other incumbent facing a challenge. He acknowledged that his district in
North Brooklyn, which includes Williamsburg and Greenpoint, had
undergone profound changes in the last decade.

``A lot of millennials who moved into the district didn't know who I
was,'' he said, saying he should have done a better job introducing
himself to voters, a proposition made more difficult by the coronavirus
outbreak.

Mr. Lentol
\href{https://brooklyneagle.com/articles/2017/04/24/lentol-reveals-how-he-got-raise-the-age-passed/}{played
a central role} in the passage of a series of changes in criminal
procedure in recent years, including
\href{https://www.nytimes.com/2017/04/10/nyregion/raise-the-age-new-york.html}{the
2017 law} that raised the age of criminal responsibility to 18, a major
victory for those seeking reforms. Previously, New York had been just
one of two states to treat defendants as young as 16 as adults in
Criminal Court, a system that had been criticized for exposing young
offenders to harsh conditions in jails.

``It was a tough race," he said. ``I gave it everything I had, and she
won.''

Ms. Gallagher, a democratic socialist who did not receive the D.S.A.'s
endorsement and had little institutional backing, said she hoped the
insurgent wins this cycle would inject more transparency into the way
policy is crafted in Albany and encourage legislators to ``lean into''
more liberal legislation.

``I think that for a long time, the political climate has told us that
we had to pretend to be a little bit less radical than we are,'' she
said. ``I think there's probably a lot of people already in the Assembly
who wanted to enact bigger changes than they were enabled to.''

All told, the Assembly could have nearly two dozen new members when the
next Legislature is formally seated in January.

Most if not all of the primary wins over incumbents occurred in safely
Democratic seats, meaning the primary winner is almost assured of
winning a seat in the Legislature in November.

Ms. Simotas, 41, the first Greek-American woman elected to the Assembly,
was a relative newcomer to Albany, having served almost five terms in a
city where some members' tenures date to the 1970s.

After she conceded on Wednesday, Mr. Mamdani, a Muslim who was born in
Uganda, celebrated his victory on Twitter with a two-word summation of
his political base:

``\href{https://twitter.com/ZohranKMamdani/status/1285981158480793606}{Socialism
won,}'' he wrote.

The multiple defeats of incumbents also seemed to demonstrate the
weakened sway of Carl E. Heastie, the speaker of the Assembly, who
oversees the overwhelming Democratic majority in Albany's lower chamber
and steered hundreds of thousands of dollars into the campaign coffers
of the most vulnerable incumbents.

Hank Sheinkopf, a veteran Democratic political consultant, said the
rejection of so many sitting Assembly members was a surprise considering
that Mr. Heastie, elected speaker in 2015, had earned a reputation as
fiercely protective of his conference.

``Speakers are supposed to be all-powerful, and their first priority is
protect their members,'' he said. ``So what happened?''

In a statement, Mike Whyland, a spokesman for Mr. Heastie, suggested
that the losses served as a distraction from the work of legislating.

``The elections are over, and right now the speaker is focused on the
large volume of bills that are moving through the Assembly this week,''
Mr. Whyland said. ``After this week's work, he will be more than happy
to talk about the elections.''

The primary results also marked a significant win for the Working
Families Party (W.F.P.), which backed three challengers who unseated
incumbents, and endorsed 11 candidates who won Democratic primaries for
open seats in the Senate and Assembly.

Among those were Jessica González-Rojas, a Latina community activist who
unseated Michael DenDekker, a six-term legislator from Queens. In the
Bronx, Amanda Septimo, a labor organizer, clinched the nomination for
the Assembly seat of Carmen E. Arroyo, a 26-year incumbent who was
knocked off the ballot after a court ruled she submitted fraudulent
petitions.

In Brooklyn, Marcela Mitaynes, a tenant organizer who received a broad
swath of left-wing endorsements --- including from the D.S.A., the
W.F.P. and Ms. Ocasio-Cortez --- defeated
\href{https://nyassembly.gov/mem/Felix-W-Ortiz}{Félix W. Ortiz}, a
longtime incumbent who serves as the assistant speaker in the Assembly.

Mr. Ortiz, who faced three challengers, said the coronavirus pandemic
took him off the campaign trail and probably affected his chances of
being re-elected.

``I don't think I would have done anything differently, other than,
maybe, I should have probably campaigned more,'' he said. ``But that's
part of life.''

In the 38th Assembly District, which encompasses the Glendale and
Woodhaven neighborhoods of Queens, Jenifer Rajkumar, an Indian-American
newcomer, defeated Michael G. Miller, who was elected in 2009.

At one point it seemed incumbents would stand to
\href{https://www.nytimes.com/2020/06/18/nyregion/ny-progressives-elections-coronavirus.html}{benefit
from the disruption} created by the pandemic, which froze in-person
campaigning and door-knocking operations. But challengers shifted their
approach, investing in digital outreach and phone bank operations and
harnessing the appetite for change sparked by the recent protests over
police brutality.

They also appeared to more successfully target people voting by mail: In
many of the races, challengers pulled ahead of incumbents during the
absentee ballot count.

Christina Greer, an associate professor of political science at Fordham
University, said many incumbents probably depended on a small but
reliable base of supporters in their districts, while challengers tapped
into a larger population of eligible voters who may have been
marginalized or not as civically engaged in past elections.

``It'll be interesting to see who from the old guard will see the
demographics in their district changing, and realize they've got to go a
little bit to the left to ward off challengers in the future,'' she
said.

Some of the primary races exposed rifts within the party's left flank,
pitting challengers against incumbents that many Democrats considered
sufficiently progressive. Liberal groups were not always aligned behind
the same candidates.

In Brooklyn's 57th District, for example, Phara Souffrant Forrest, a
nurse and tenant activist backed by the D.S.A., prevailed over
Assemblyman Walter T. Mosley, who was endorsed by the W.F.P.

For the most part, however, the primary wins underscored the
generational shift unfolding in Albany.

Mr. Lentol, whose father and grandfather were also assemblymen, said he
didn't know what he'd do next after nearly a half-century in Albany.

``They're not ringing my phone yet,'' he said. ``But maybe they don't
know I'm available.''

Advertisement

\protect\hyperlink{after-bottom}{Continue reading the main story}

\hypertarget{site-index}{%
\subsection{Site Index}\label{site-index}}

\hypertarget{site-information-navigation}{%
\subsection{Site Information
Navigation}\label{site-information-navigation}}

\begin{itemize}
\tightlist
\item
  \href{https://help.nytimes.com/hc/en-us/articles/115014792127-Copyright-notice}{©~2020~The
  New York Times Company}
\end{itemize}

\begin{itemize}
\tightlist
\item
  \href{https://www.nytco.com/}{NYTCo}
\item
  \href{https://help.nytimes.com/hc/en-us/articles/115015385887-Contact-Us}{Contact
  Us}
\item
  \href{https://www.nytco.com/careers/}{Work with us}
\item
  \href{https://nytmediakit.com/}{Advertise}
\item
  \href{http://www.tbrandstudio.com/}{T Brand Studio}
\item
  \href{https://www.nytimes.com/privacy/cookie-policy\#how-do-i-manage-trackers}{Your
  Ad Choices}
\item
  \href{https://www.nytimes.com/privacy}{Privacy}
\item
  \href{https://help.nytimes.com/hc/en-us/articles/115014893428-Terms-of-service}{Terms
  of Service}
\item
  \href{https://help.nytimes.com/hc/en-us/articles/115014893968-Terms-of-sale}{Terms
  of Sale}
\item
  \href{https://spiderbites.nytimes.com}{Site Map}
\item
  \href{https://help.nytimes.com/hc/en-us}{Help}
\item
  \href{https://www.nytimes.com/subscription?campaignId=37WXW}{Subscriptions}
\end{itemize}
