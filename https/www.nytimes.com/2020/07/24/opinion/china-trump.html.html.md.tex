Sections

SEARCH

\protect\hyperlink{site-content}{Skip to
content}\protect\hyperlink{site-index}{Skip to site index}

\href{https://myaccount.nytimes.com/auth/login?response_type=cookie\&client_id=vi}{}

\href{https://www.nytimes.com/section/todayspaper}{Today's Paper}

\href{/section/opinion}{Opinion}\textbar{}The Two China Fires

\href{https://nyti.ms/39rvvfG}{https://nyti.ms/39rvvfG}

\begin{itemize}
\item
\item
\item
\item
\item
\item
\end{itemize}

Advertisement

\protect\hyperlink{after-top}{Continue reading the main story}

\href{/section/opinion}{Opinion}

Supported by

\protect\hyperlink{after-sponsor}{Continue reading the main story}

\hypertarget{the-two-china-fires}{%
\section{The Two China Fires}\label{the-two-china-fires}}

Is America prepared for a Cold War with China?

\href{https://www.nytimes.com/by/bret-stephens}{\includegraphics{https://static01.nyt.com/images/2017/08/27/insider/bretstephens/bretstephens-thumbLarge-v6.png}}

By \href{https://www.nytimes.com/by/bret-stephens}{Bret Stephens}

Opinion Columnist

\begin{itemize}
\item
  July 24, 2020
\item
  \begin{itemize}
  \item
  \item
  \item
  \item
  \item
  \item
  \end{itemize}
\end{itemize}

\includegraphics{https://static01.nyt.com/images/2020/07/24/opinion/24stephensWeb/merlin_174829527_7743f228-528d-4285-a3c6-b119c8eded6e-articleLarge.jpg?quality=75\&auto=webp\&disable=upscale}

We'll probably never know exactly what sorts of documents were
incinerated at China's Consulate in Houston in the days before the
United States forced it to close on Friday, after accusing it of being a
hub of espionage. We may also never know what caused this month's
catastrophic fire aboard the U.S.S. Bonhomme Richard, a massive
amphibious assault ship that was being fitted out to double as a small
aircraft carrier, in the port of San Diego.

What we should know is that the two fires are actually one. We are
racing toward a conflict with China we may be ill-prepared to wage.

The closure of the consulate comes on the heels of a quad of bellicose
speeches from top administration officials, collectively amounting to a
declaration of Cold War against China. Robert O'Brien, the national
security adviser, painted China's leadership
\href{https://www.whitehouse.gov/briefings-statements/chinese-communist-partys-ideology-global-ambitions/}{as
unreconstructed Marxist-Leninists}. The F.B.I. director, Christopher
Wray, spoke of China's practice in the art of
\href{https://www.fbi.gov/news/speeches/the-threat-posed-by-the-chinese-government-and-the-chinese-communist-party-to-the-economic-and-national-security-of-the-united-states}{``malign
foreign influence.''} Attorney General Bill Barr accused China of
\href{https://www.justice.gov/opa/speech/transcript-attorney-general-barr-s-remarks-china-policy-gerald-r-ford-presidential-museum}{``economic
blitzkrieg.''} And Secretary of State Mike Pompeo hinted the free world
\href{https://www.state.gov/communist-china-and-the-free-worlds-future/}{may
need a new version of NATO}, this one aimed at Beijing instead of
Moscow.

\includegraphics{https://static01.nyt.com/images/2020/07/24/opinion/24stephens2/merlin_174522222_105c77f0-9d3b-4c6f-af79-279673f25b2d-articleLarge.jpg?quality=75\&auto=webp\&disable=upscale}

Given that the source is Team Trump and the timing is an election year,
it's tempting to dismiss the speeches' warnings as cynical,
hypocritical, political --- and therefore wrong. Why complain about
civil liberties in Hong Kong when we have goon squads in Portland? Why
accuse China of trashing global norms when that's been Trump's ambition
from the beginning? Why characterize Chinese President Xi Jinping as a
linear ideological descendant of Joseph Stalin when, as we know from
John Bolton, Trump was fulsomely praising him and soliciting his help
for his re-election bid?

And why all of this now, when Trump needs enemies both foreign and
domestic to rescue his flagging re-election bid?

But the problem with these questions is that --- however on-point they
are as criticisms of Trump --- they obscure two hard facts a Biden
administration will also confront. The first is that, under Xi, China
has become drastically more repressive at home, more aggressive abroad,
and more shameless about both than at nearly any point since the death
of Mao.

This is not a matter of Beijing reacting badly to Trump (as the early
Obama administration erroneously supposed that bad relations with Russia
were a matter of Moscow reacting badly to George W. Bush). Some of
China's biggest digital heists date to the Obama years --- including the
\href{https://www.lawfareblog.com/why-opm-hack-far-worse-you-imagine}{2015
hack of the Office of Personnel Management}, which gave Beijing the
background security files for nearly 22 million current or former U.S.
government employees and their family members. China's
\href{https://www.nytimes.com/2016/07/13/world/asia/south-china-sea-hague-ruling-philippines.html}{outrageous
and illegal claims} to most of the South China Sea also predate Trump
and will fester long after he's gone.

What stands out now is just how brazen Beijing has become. Take one
detail from Wray's speech: ``We have now reached the point where the
F.B.I. is opening a new China-related counterintelligence case about
every 10 hours,'' he said. In one case, a single scientist, Hongjin Tan,
\href{https://www.wsj.com/articles/chinese-national-sentenced-to-prison-in-1-billion-trade-secret-theft-case-11582839551}{pleaded
guilty} to stealing an estimated \$1 billion in trade secrets from an
Oklahoma-based energy company.

Multiply that hundreds if not thousands of times over, and what you have
is arguably the largest single theft of foreign property since Germany
looted Europe in World War II. Whatever else one might say against the
Trump administration, it isn't lying about China.

But this brings us to the second blunt fact. U.S. power in East Asia is
waning. Trump's decision to withdraw the U.S. from the Trans-Pacific
Partnership --- the single best hedge the U.S. had against Chinese
economic dominance of the region --- may, in hindsight, prove to be his
single worst policy mistake. He has tried to shake down both South Korea
and Japan to pay more for basing U.S. forces: penny ante politics that
only raise doubts about America's reliability as an ally.

And then there's the degraded state of the U.S. Navy, epitomized by the
fire on the Bonhomme Richard (itself the latest in a string of
corruption, leadership,
\href{https://news.usni.org/2020/07/02/navy-removes-ford-carrier-program-manager-citing-performance-over-time}{cost
over-run} and
\href{https://www.navytimes.com/news/your-navy/2019/01/14/worse-than-you-thought-inside-the-secret-fitzgerald-probe-the-navy-doesnt-want-you-to-read/}{competency
scandals} to bedevil the service). Trump came to office with grand plans
to build a 355-ship Navy, up from the current 300. The Pentagon all but
admits
\href{https://breakingdefense.com/2020/02/navy-marines-caught-by-surprise-by-espers-budget-cuts/}{it
has no hope of reaching that goal}. Meanwhile, the Chinese Navy ---
which isn't stretched around the world --- has 335 ships,
\href{https://www.nationaldefensemagazine.org/articles/2020/3/9/eagle-vs-dragon-how-the-us-and-chinese-navies-stack-up}{a
55 percent increase in 15 years},

If the U.S. and the People's Republic were to come to blows after some
incident over some atoll in the South China Sea, are we confident we'd
prevail?

When (fingers crossed) Joe Biden is president, he needn't ask his
cabinet members to deliver philippics against Beijing. But, as George
Kennan once wrote about another regime, he must be prepared to confront
China with ``unalterable counter force at every point where they show
signs of encroaching upon the interests of a
\href{https://www.foreignaffairs.com/articles/russian-federation/1947-07-01/sources-soviet-conduct}{peaceful
and stable world.}''

\emph{The Times is committed to publishing}
\href{https://www.nytimes.com/2019/01/31/opinion/letters/letters-to-editor-new-york-times-women.html}{\emph{a
diversity of letters}} \emph{to the editor. We'd like to hear what you
think about this or any of our articles. Here are some}
\href{https://help.nytimes.com/hc/en-us/articles/115014925288-How-to-submit-a-letter-to-the-editor}{\emph{tips}}\emph{.
And here's our email:}
\href{mailto:letters@nytimes.com}{\emph{letters@nytimes.com}}\emph{.}

\emph{Follow The New York Times Opinion section on}
\href{https://www.facebook.com/nytopinion}{\emph{Facebook}}\emph{,}
\href{http://twitter.com/NYTOpinion}{\emph{Twitter (@NYTopinion)}}
\emph{and}
\href{https://www.instagram.com/nytopinion/}{\emph{Instagram}}\emph{.}

Advertisement

\protect\hyperlink{after-bottom}{Continue reading the main story}

\hypertarget{site-index}{%
\subsection{Site Index}\label{site-index}}

\hypertarget{site-information-navigation}{%
\subsection{Site Information
Navigation}\label{site-information-navigation}}

\begin{itemize}
\tightlist
\item
  \href{https://help.nytimes.com/hc/en-us/articles/115014792127-Copyright-notice}{©~2020~The
  New York Times Company}
\end{itemize}

\begin{itemize}
\tightlist
\item
  \href{https://www.nytco.com/}{NYTCo}
\item
  \href{https://help.nytimes.com/hc/en-us/articles/115015385887-Contact-Us}{Contact
  Us}
\item
  \href{https://www.nytco.com/careers/}{Work with us}
\item
  \href{https://nytmediakit.com/}{Advertise}
\item
  \href{http://www.tbrandstudio.com/}{T Brand Studio}
\item
  \href{https://www.nytimes.com/privacy/cookie-policy\#how-do-i-manage-trackers}{Your
  Ad Choices}
\item
  \href{https://www.nytimes.com/privacy}{Privacy}
\item
  \href{https://help.nytimes.com/hc/en-us/articles/115014893428-Terms-of-service}{Terms
  of Service}
\item
  \href{https://help.nytimes.com/hc/en-us/articles/115014893968-Terms-of-sale}{Terms
  of Sale}
\item
  \href{https://spiderbites.nytimes.com}{Site Map}
\item
  \href{https://help.nytimes.com/hc/en-us}{Help}
\item
  \href{https://www.nytimes.com/subscription?campaignId=37WXW}{Subscriptions}
\end{itemize}
