Sections

SEARCH

\protect\hyperlink{site-content}{Skip to
content}\protect\hyperlink{site-index}{Skip to site index}

\href{https://myaccount.nytimes.com/auth/login?response_type=cookie\&client_id=vi}{}

\href{https://www.nytimes.com/section/todayspaper}{Today's Paper}

\href{/section/opinion}{Opinion}\textbar{}American Catastrophe Through
German Eyes

\href{https://nyti.ms/2CZpTgB}{https://nyti.ms/2CZpTgB}

\begin{itemize}
\item
\item
\item
\item
\item
\item
\end{itemize}

Advertisement

\protect\hyperlink{after-top}{Continue reading the main story}

\href{/section/opinion}{Opinion}

Supported by

\protect\hyperlink{after-sponsor}{Continue reading the main story}

\hypertarget{american-catastrophe-through-german-eyes}{%
\section{American Catastrophe Through German
Eyes}\label{american-catastrophe-through-german-eyes}}

Trump says he wants to protect law-abiding citizens. In 1933, Hitler
issued his `Decree of the Reich President for the Protection of People
and State.'

\href{https://www.nytimes.com/by/roger-cohen}{\includegraphics{https://static01.nyt.com/images/2014/11/01/opinion/cohen-circular/cohen-circular-thumbLarge-v6.png}}

By \href{https://www.nytimes.com/by/roger-cohen}{Roger Cohen}

Opinion Columnist

\begin{itemize}
\item
  July 24, 2020
\item
  \begin{itemize}
  \item
  \item
  \item
  \item
  \item
  \item
  \end{itemize}
\end{itemize}

\includegraphics{https://static01.nyt.com/images/2020/07/24/opinion/24cohen1a/merlin_174900909_47930454-7b4d-4e79-a727-cab2468b369b-articleLarge.jpg?quality=75\&auto=webp\&disable=upscale}

PARIS --- No people has found the American lurch toward authoritarianism
under President Trump more alarming than the Germans. For postwar
Germany, the United States was savior, protector and liberal democratic
model. Now, Germans, in shock, speak of the ``American catastrophe.''

A
\href{https://adage.com/article/media/germanys-leading-newsweekly-decries-trumps-incendiary-approach/2261056}{recent
cover of the weekly magazine Der Spiegel} portrays Trump in the Oval
Office holding a lighted match, with a country ablaze visible through
his window. The headline: ``Der Feuerteufel,'' or, literally, ``the Fire
Devil.''

\includegraphics{https://static01.nyt.com/images/2020/07/24/opinion/24cohen2/merlin_174906717_81b6af5e-e4d6-43ab-83cd-87b56e73d46a-articleLarge.jpg?quality=75\&auto=webp\&disable=upscale}

Germans have a particular relationship to fire. The Reichstag fire of
1933 enabled Hitler and the Nazis to scrap the fragile Weimar democracy
that had brought them to power. Hitler's murderous fantasies could now
become reality. War, Auschwitz and the German catastrophe followed.

I have known many thoughtful German diplomats over the years, including
Michael Steiner, who labored to stop the Balkan wars of the 1990s, and
Wolfgang Ischinger, the former German ambassador to the United States.
It always seemed to me that their particular passion for freedom,
democracy and openness stemmed from the knowledge of how easily these
are lost.

Michael Steinberg, a professor of history at Brown University and the
former president of the American Academy in Berlin, wrote to me this
week:

``The American catastrophe seems to get worse every day, but the events
in Portland have particularly alarmed me as a kind of strategic
experiment for fascism. The playbook from the German fall of democracy
in 1933 seems well in place, including rogue military factions, the
destabilization of cities, etc.''

Steinberg continued, ``The basic comparison involves racism as a
political strategy: a racist imaginary of a pure homeland, with cities
demonized as places of decadence.''

Trump provokes outrage in a cascade designed to blunt alarm. He deadens
reactions through volume and repetition. But something about the recent
use of unmarked cars and camouflage-clad federal agents without clear
identifying insignia detaining protesters shattered any inclination to
shrug.

From the deployment of those federal units in Portland, Oregon's largest
city,
\href{https://www.nytimes.com/2020/07/24/us/portland-oregon-protests-white-race.html}{where
protesters have been demanding racial justice and police
accountability}, it's not a huge leap to the use of paramilitaries (like
the German Freikorps in the 1920s) to buttress a ``Law and Order''
campaign. The Freikorps battled communists. Today, Trump claims to
battle ``anarchists,'' ``terrorists'' and violent leftists. It's the
leitmotif of his quest for a second term.

Perhaps the years I spent covering Argentina in the 1980s, in the
aftermath of the military junta, made me particularly sensitive to the
use of unmarked cars --- in the Argentine case, Ford Falcons --- to grab
left-wing political opponents off the street. They were ``disappeared,''
a word whose lingering psychological devastation I measured in countless
tear-filled rooms. Later I went to Berlin, where there was only one
story: totalitarian tragedy and the labors of democratic salvation.

The Department of Homeland Security's Customs and Border Protection
\href{https://www.google.com/url?q=https://www.reuters.com/article/us-global-race-protests-agents/us-homeland-security-confirms-three-units-sent-paramilitary-officers-to-portland-idUSKCN24M2RL\&sa=D\&ust=1595604281633000\&usg=AFQjCNESMShE-qwX6DEJY8-UmYcYsMRIdg}{confirmed
this week} that it has deployed officers from three paramilitary-style
units to join the federal crackdown in Portland. The Trump
administration, facing lawsuits, has cited post-9/11 legislation
establishing the department to justify its action. Chicago is now among
several cities being targeted as Trump seeks to foment confrontation.

As Tom Ridge, a Republican who was the first head of the Department of
Homeland Security, noted in an interview with the Sirius XM host Michael
Smerconish, the department was
``\href{https://www.siriusxm.com/clips/clip/a9914a41-78d6-4402-8150-b8f14a44945b/715571de-a566-4492-ba13-bbe09f516300}{not
established to be the president's personal militia}.''

In wartime, the Third Geneva Convention, to which the United States is a
party, requires even irregular forces to wear ``a fixed distinctive sign
recognizable at a distance.'' This is critical not only to protecting
civilians but also to ensuring accountability for misconduct.

When paramilitary-style units have no identifying insignia, there is no
transparency, no accountability --- and that means impunity. Democracy
dies. Think of all this as setting the scene for Trump's own ``state of
emergency'' if he does not like the November election result. Social
media is combustible enough for a physical fire to be unnecessary.

The president says he wants to protect law-abiding citizens. In 1933,
after the Reichstag burned, Hitler issued the ``Decree of the Reich
President for the Protection of People and State'' as his means to seize
power.

German horror at Trump has many components. He's the fear-mongering
showman wielding nationalism, racism and violence as if the 20th century
held no lessons. He's the would-be destroyer of the multilateral
institutions that brought European peace and made it possible for
Germans to raise their bowed heads again. He is a fascist in the making.

As
\href{https://www.google.com/url?q=https://newrepublic.com/article/157112/germany-gets-coronavirus\&sa=D\&ust=1595605994586000\&usg=AFQjCNH0rL7W7j6lrXzrAGOt2JzfuSTxVw}{Ian
Beacock argued recently in The New Republic,} Angela Merkel, the German
chancellor, got it right on the virus. Not for her the imagery of war
--- all that talk of the silent, invisible enemy to be vanquished. No,
for her the challenge of the virus has been a lesson in the power of
democracy.

``We are not condemned to accept the spread of this virus as an
inevitable fact of life,''
\href{https://www.bundeskanzlerin.de/bkin-en/news/statement-chancellor-1732302}{she
said.} ``We thrive not because we are forced to do something, but
because we share knowledge and encourage active participation.'' She
went on to say that success largely depends ``on each and every one of
us.''

It worked. Merkel was addressing all democratic citizens, Americans
included. No wonder Trump cannot stand her, a woman trained as a
scientist whose life lesson has been the sacred value of freedom.

\emph{The Times is committed to publishing}
\href{https://www.nytimes.com/2019/01/31/opinion/letters/letters-to-editor-new-york-times-women.html}{\emph{a
diversity of letters}} \emph{to the editor. We'd like to hear what you
think about this or any of our articles. Here are some}
\href{https://help.nytimes.com/hc/en-us/articles/115014925288-How-to-submit-a-letter-to-the-editor}{\emph{tips}}\emph{.
And here's our email:}
\href{mailto:letters@nytimes.com}{\emph{letters@nytimes.com}}\emph{.}

\emph{Follow The New York Times Opinion section on}
\href{https://www.facebook.com/nytopinion}{\emph{Facebook}}\emph{,}
\href{http://twitter.com/NYTOpinion}{\emph{Twitter (@NYTopinion)}}
\emph{and}
\href{https://www.instagram.com/nytopinion/}{\emph{Instagram}}\emph{.}

Advertisement

\protect\hyperlink{after-bottom}{Continue reading the main story}

\hypertarget{site-index}{%
\subsection{Site Index}\label{site-index}}

\hypertarget{site-information-navigation}{%
\subsection{Site Information
Navigation}\label{site-information-navigation}}

\begin{itemize}
\tightlist
\item
  \href{https://help.nytimes.com/hc/en-us/articles/115014792127-Copyright-notice}{©~2020~The
  New York Times Company}
\end{itemize}

\begin{itemize}
\tightlist
\item
  \href{https://www.nytco.com/}{NYTCo}
\item
  \href{https://help.nytimes.com/hc/en-us/articles/115015385887-Contact-Us}{Contact
  Us}
\item
  \href{https://www.nytco.com/careers/}{Work with us}
\item
  \href{https://nytmediakit.com/}{Advertise}
\item
  \href{http://www.tbrandstudio.com/}{T Brand Studio}
\item
  \href{https://www.nytimes.com/privacy/cookie-policy\#how-do-i-manage-trackers}{Your
  Ad Choices}
\item
  \href{https://www.nytimes.com/privacy}{Privacy}
\item
  \href{https://help.nytimes.com/hc/en-us/articles/115014893428-Terms-of-service}{Terms
  of Service}
\item
  \href{https://help.nytimes.com/hc/en-us/articles/115014893968-Terms-of-sale}{Terms
  of Sale}
\item
  \href{https://spiderbites.nytimes.com}{Site Map}
\item
  \href{https://help.nytimes.com/hc/en-us}{Help}
\item
  \href{https://www.nytimes.com/subscription?campaignId=37WXW}{Subscriptions}
\end{itemize}
