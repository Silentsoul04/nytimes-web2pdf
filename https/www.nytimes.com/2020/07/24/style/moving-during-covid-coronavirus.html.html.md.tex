Sections

SEARCH

\protect\hyperlink{site-content}{Skip to
content}\protect\hyperlink{site-index}{Skip to site index}

\href{https://www.nytimes.com/section/style}{Style}

\href{https://myaccount.nytimes.com/auth/login?response_type=cookie\&client_id=vi}{}

\href{https://www.nytimes.com/section/todayspaper}{Today's Paper}

\href{/section/style}{Style}\textbar{}So You're Thinking of Moving
\ldots{}

\url{https://nyti.ms/39nmmVs}

\begin{itemize}
\item
\item
\item
\item
\item
\item
\end{itemize}

\href{https://www.nytimes.com/spotlight/at-home?action=click\&pgtype=Article\&state=default\&region=TOP_BANNER\&context=at_home_menu}{At
Home}

\begin{itemize}
\tightlist
\item
  \href{https://www.nytimes.com/2020/08/03/well/family/the-benefits-of-talking-to-strangers.html?action=click\&pgtype=Article\&state=default\&region=TOP_BANNER\&context=at_home_menu}{Talk:
  To Strangers}
\item
  \href{https://www.nytimes.com/2020/08/01/at-home/coronavirus-make-pizza-on-a-grill.html?action=click\&pgtype=Article\&state=default\&region=TOP_BANNER\&context=at_home_menu}{Make:
  Grilled Pizza}
\item
  \href{https://www.nytimes.com/2020/07/31/arts/television/goldbergs-abc-stream.html?action=click\&pgtype=Article\&state=default\&region=TOP_BANNER\&context=at_home_menu}{Watch:
  'The Goldbergs'}
\item
  \href{https://www.nytimes.com/interactive/2020/at-home/even-more-reporters-editors-diaries-lists-recommendations.html?action=click\&pgtype=Article\&state=default\&region=TOP_BANNER\&context=at_home_menu}{Explore:
  Reporters' Google Docs}
\end{itemize}

Advertisement

\protect\hyperlink{after-top}{Continue reading the main story}

Supported by

\protect\hyperlink{after-sponsor}{Continue reading the main story}

\hypertarget{so-youre-thinking-of-moving-}{%
\section{So You're Thinking of Moving
\ldots{}}\label{so-youre-thinking-of-moving-}}

But you're scared about doing it during a pandemic. Here's our F.A.Q. on
changing homes (and cities) safely.

\includegraphics{https://static01.nyt.com/images/2020/07/21/style/00VIRUS-HOW-TO-MOVE/oakImage-1595347553231-articleLarge.jpg?quality=75\&auto=webp\&disable=upscale}

By Hannah Wise

\begin{itemize}
\item
  July 24, 2020
\item
  \begin{itemize}
  \item
  \item
  \item
  \item
  \item
  \item
  \end{itemize}
\end{itemize}

I am an extreme planner. But I'll admit that the task of planning a
cross-country move during a pandemic with a suppressed immune system
(and a sassy cat named Brünnhilde) was a daunting one.

It was a plan born of necessity; I have
\href{https://www.dallasnews.com/news/healthy-living/2016/05/10/how-a-diagnosis-of-crohn-s-disease-changed-me-at-23/}{Crohn's
disease}, an autoimmune disease that affects the intestinal tract, as
well as psoriasis and psoriatic arthritis --- conditions that are
managed through a rigid medication schedule that suppresses my immune
system, leaving me particularly vulnerable to all manner of infections.

Immunocompromised individuals are at a higher risk of experiencing
severe illness from the virus that causes Covid-19 and can be sick
longer once they are infected, according to the
\href{https://www.cdc.gov/coronavirus/2019-ncov/need-extra-precautions/immunocompromised.html}{Centers
for Disease Control and Prevention}. Which means I spent nearly 100 days
sequestered in my apartment in Brooklyn this spring, leaving only to get
the mail. After weeks of relying on the kindness of friends and delivery
services, I decided to move to Kansas City to have more space to
socially distance and to be closer to family.

I am far from the only person who has moved this year, but because I had
to be particularly neurotic about it, I am sharing my reporting ---
based in part on reader questions --- on how to navigate the experience
safely.

If you have additional suggestions or tips, please share them in the
comments. Good luck on your journeys and WEAR A MASK!

\hypertarget{is-it-safe-and-ethical-to-hire-movers}{%
\subsection{Is it safe and ethical to hire
movers?}\label{is-it-safe-and-ethical-to-hire-movers}}

In most states, moving companies are considered essential businesses and
many have altered their procedures to minimize risk for their employees
and clients.

But it's best to call each company in advance and ask them about their
new coronavirus protocols, because there is no one-size-fits-all
approach to safety. In general, you should be looking for companies that
require employees and customers to wear masks, detail how they practice
social distancing, and can explain what steps they are taking to screen
and protect their employees from becoming sick.

Get multiple quotes --- this goes for pricing, too! --- and compare them
to the cost of renting a truck for yourself.

I chose to pack my own belongings and hire a moving company to do the
heavy lifting. My movers wore masks throughout the process, and the
company required that I have a hand-washing station available on both
ends of the move. Everyone practiced social distancing.

Some readers asked if it is ethical to hire individuals using Craigslist
or TaskRabbit, a platform that lets people pay freelancers for odd jobs
and that
\href{https://support.taskrabbit.com/hc/en-us/articles/360040752692-COVID-19-Updates}{recently
introduced contactless services}. Safety in both cases depends on
talking to the individuals you hire before you hire them. Do what you
can to minimize risks for all involved, and make sure you tip
generously.

\hypertarget{what-personal-protective-equipment-should-i-wear-on-the-road}{%
\subsection{What personal protective equipment should I wear on the
road?}\label{what-personal-protective-equipment-should-i-wear-on-the-road}}

Disposable gloves are helpful for small transactions like getting gas.
But really, just
\href{https://www.nytimes.com/article/coronavirus-facts-history.html\#link-6ec3dc3a}{wear
a mask}, keep the
\href{https://www.cdc.gov/coronavirus/2019-ncov/prevent-getting-sick/prevention.html?CDC_AA_refVal=https\%3A\%2F\%2Fwww.cdc.gov\%2Fcoronavirus\%2F2019-ncov\%2Fprepare\%2Fprevention.html}{hand
sanitizer} close and
\href{https://www.nytimes.com/2020/03/13/world/how-to-wash-your-hands-coronavirus.html}{wash
your hands} as frequently as possible. (And
\href{https://www.nytimes.com/2020/03/05/health/stop-touching-your-face-coronavirus.html}{don't
touch your face}!)

When it comes to sanitizing the surfaces around you (inside a car or
moving van, say), you don't need to go
\href{https://www.nytimes.com/2020/06/22/style/11-things-about-naomi-campbell.html}{full
Naomi Campbell}. The C.D.C. has said since March that contaminated
surfaces are ``not thought to be the main way'' the coronavirus spreads.

Still, cleaning high-touch surfaces regularly is good practice. Make
sure to follow the
\href{https://www.nytimes.com/2020/05/06/well/live/coronavirus-cleaning-cleaners-disinfectants-home.html}{specific
instructions for each cleaner you are using}.

\hypertarget{how-do-i-find-a-new-place-to-live}{%
\subsection{How do I find a new place to
live?}\label{how-do-i-find-a-new-place-to-live}}

Looking for a new apartment or house is a stressful and time-consuming
process in normal times. And right now, you might not be able to see the
property in person before committing to it. Doubly stressful!

Make a list of your must-haves for your new home. Ask friends, family
and colleagues if they have neighborhood recommendations. Then, use your
digital sleuthing skills to learn about the area or properties and
narrow your choices.

Use Google Street View to virtually tour the neighborhood. Find Facebook
Groups or city-specific comment forums that put you in touch with --- or
give you insight into --- the people who live there. Read online reviews
of apartment complexes. I've also used location searches on Instagram to
get a feel for certain apartment communities and their surrounding
neighborhoods.

Regardless of whether you are working with a homeowner or a leasing
agent, insist on
\href{https://www.nytimes.com/2020/05/30/realestate/virtual-tours-renting.html}{taking
a video tour of the specific property} you are interested in. Ask
detailed questions to help paint a clear picture of the property. Don't
rush --- you're making a major decision and should feel comfortable
before signing a lease.

\hypertarget{how-do-i-donate-or-sell-my-clothing-and-furniture}{%
\subsection{How do I donate or sell my clothing and
furniture?}\label{how-do-i-donate-or-sell-my-clothing-and-furniture}}

Many clothing and furniture donation sites have either temporarily
closed following local ordinances, or
\href{https://www.nytimes.com/2020/04/13/style/self-care/donate-clothes-coronavirus.html}{cannot
accept new donations} at this time. So, I sold most of my furniture
through my neighborhood Facebook group.

I cleaned each piece, arranged the pickup and asked that anyone coming
to claim the items wear a mask. Advertising your free stuff on
Craigslist can also work very well.

\hypertarget{is-it-safer-to-fly-drive-or-take-the-train}{%
\subsection{Is it safer to fly, drive or take the
train?}\label{is-it-safer-to-fly-drive-or-take-the-train}}

Given that all travel choices include some risk of infection, the
decision will most likely depend on the distance and cost. My move
required a combination of flying and driving.

Other than the belongings that movers were taking, I needed to haul a
small jungle of plants and Brünnhilde home to Kansas City. So my father
generously volunteered to fly to New York to help with the 19-hour
drive.

My dad took the first flight of the day (so as to be on the cleanest
plane), wore a mask throughout the journey and disinfected high-touch
surfaces as he went. The flight had about 30 passengers onboard. Upon
arriving in New York, he picked up and disinfected a rental van. He
stayed in a hotel to further disinfect himself from any germs from the
flight and brought separate sets of clothes for the road trip.

For those who would travel by train,
\href{https://www.amtrak.com/coronavirus}{Amtrak says} it is offering
private rooms on some routes, limiting tickets to encourage distancing
and changing its cleaning procedures.

Strong personal hygiene and mask wearing is recommended no matter your
travel method.

\hypertarget{where-to-stop-and-what-to-eat-if-youre-driving}{%
\subsection{Where to stop and what to eat if you're
driving}\label{where-to-stop-and-what-to-eat-if-youre-driving}}

``Don't go far, stay in your car,'' Robert Sinclair Jr., a Northeast
regional spokesman for AAA,
\href{https://www.nytimes.com/interactive/2020/world/coronavirus-tips-advice.html?action=click\&pgtype=Article\&state=default\&module=styln-coronavirus-national\&variant=show\&region=TOP_BANNER\&context=storylines_menu}{told
The Times.}

Of course, you'll need to take a break sometimes. Scenic overlooks are a
great place to appreciate nature while stretching your legs. If you're
stopping at a truck stop or gas station, look for an option that seems
less busy.

Sanitize your hands each time you leave the car. When pumping gas, Mr.
Sinclair advised sanitizing your hands after replacing the pump
\emph{and} before you touch the car door handle --- disposable gloves
work here, too.

Sanitize your credit card before and after you hand it to an attendant.
Don't forget to bring snacks for the road. And did we mention you should
wear a mask?

\hypertarget{is-it-safer-to-stay-in-airbnbs-hotels-or-at-a-campground}{%
\subsection{Is it safer to stay in Airbnbs, hotels or at a
campground?}\label{is-it-safer-to-stay-in-airbnbs-hotels-or-at-a-campground}}

Airbnb and many hotel chains have made changes to ensure extra
\href{https://www.nytimes.com/2020/06/03/travel/the-most-important-word-in-the-hospitality-industry-clean.html}{cleanliness}.

\href{https://news.marriott.com/news/2020/04/21/marriott-international-launches-global-cleanliness-council-to-promote-even-higher-standards-of-cleanliness-in-the-age-of-covid-19}{Marriott's}
new standards include use of electrostatic disinfectant sprayers, and
Hilton recently
\href{https://www.hilton.com/en/corporate/cleanstay/}{released a
detailed explanation of how guest rooms} and common areas are cleaned.
(A room seal indicates no one has entered a bedroom since it was
cleaned.) Airbnb has published a new
\href{https://www.airbnb.com/resources/hosting-homes/a/cleaning-guidelines-to-help-prevent-the-spread-of-covid-19-163}{cleaning
guide for hosts} based on current C.D.C. guidelines.

\href{https://www.nytimes.com/2020/05/28/travel/camping-west-coast.html}{As
far as camping goes}, many sites aren't open but that varies state by
state. Check local listings.

Contactless booking and check-in options help ensure social distance.
For my trip, we stayed one night in a hotel, using an electronic
check-in, which loaded the door key onto our phones. We didn't see or
speak to a soul.

\hypertarget{whats-the-best-way-to-move-with-children}{%
\subsection{What's the best way to move with
children?}\label{whats-the-best-way-to-move-with-children}}

My colleagues at
\href{https://www.nytimes.com/section/parenting}{Parenting} have written
extensively about how to help children cope during the pandemic. Moving
to a new home
\href{https://www.nytimes.com/2020/07/13/parenting/moving-tips-kids.html}{may
make your child feel less safe}, which they may manifest as misbehavior,
complaining or crying.

Establishing some systems and routines before, during and after a move
can help kids --- and parents --- better handle the situation.

\hypertarget{whats-the-best-way-to-move-with-a-pet}{%
\subsection{What's the best way to move with a
pet?}\label{whats-the-best-way-to-move-with-a-pet}}

Many airlines require a pet-specific ticket, so make sure to call ahead
to book one if you fly. Your pet will have to come out of the carrier
when going through airport security, but you can request a private
security screening for you and your animal if you're worried about a
Great Escape. If you're driving, call any hotel or places you may stay
to confirm they are pet-friendly --- this usually comes with an
additional fee.

Consult your vet about any documents you may need. Your veterinarian may
recommend a mild sedative to help your pet have a smoother journey.
(Brünnhilde is a pretty chill lady, but she does benefit from
\href{https://www.feliway.com/us/Products/feliway-classic-spray}{calming
cat pheromone spray} to help soothe her travel anxiety.)

Don't forget to pack food, a water bowl and any other supplies they
need.

\hypertarget{other-advice-for-those-with-chronic-illnesses}{%
\subsection{Other advice for those with chronic
illnesses}\label{other-advice-for-those-with-chronic-illnesses}}

When talking to people during the moving process, I found that being
upfront about my compromised immune system was a quick way to make clear
why I was asking about hygiene or accommodations.

I also tried to schedule my moving dates around my medication schedule
to minimize side effects and stress. I suggest working in advance with
your medical team and insurance to refill any prescriptions, or to find
new doctors if you're moving to a new city. (And of course, don't forget
to take medications with you.)

Living through a pandemic is exhausting, and even more so if you are
navigating these times with a chronic illness or other vulnerability.
Don't forget to take breaks and be kind to yourself.

\hypertarget{enjoy-the-journey}{%
\subsection{Enjoy the journey}\label{enjoy-the-journey}}

You're entering a new chapter of your life. Don't forget to enjoy it.
The American countryside is beautiful. Look out the window and take it
in.

Advertisement

\protect\hyperlink{after-bottom}{Continue reading the main story}

\hypertarget{site-index}{%
\subsection{Site Index}\label{site-index}}

\hypertarget{site-information-navigation}{%
\subsection{Site Information
Navigation}\label{site-information-navigation}}

\begin{itemize}
\tightlist
\item
  \href{https://help.nytimes.com/hc/en-us/articles/115014792127-Copyright-notice}{©~2020~The
  New York Times Company}
\end{itemize}

\begin{itemize}
\tightlist
\item
  \href{https://www.nytco.com/}{NYTCo}
\item
  \href{https://help.nytimes.com/hc/en-us/articles/115015385887-Contact-Us}{Contact
  Us}
\item
  \href{https://www.nytco.com/careers/}{Work with us}
\item
  \href{https://nytmediakit.com/}{Advertise}
\item
  \href{http://www.tbrandstudio.com/}{T Brand Studio}
\item
  \href{https://www.nytimes.com/privacy/cookie-policy\#how-do-i-manage-trackers}{Your
  Ad Choices}
\item
  \href{https://www.nytimes.com/privacy}{Privacy}
\item
  \href{https://help.nytimes.com/hc/en-us/articles/115014893428-Terms-of-service}{Terms
  of Service}
\item
  \href{https://help.nytimes.com/hc/en-us/articles/115014893968-Terms-of-sale}{Terms
  of Sale}
\item
  \href{https://spiderbites.nytimes.com}{Site Map}
\item
  \href{https://help.nytimes.com/hc/en-us}{Help}
\item
  \href{https://www.nytimes.com/subscription?campaignId=37WXW}{Subscriptions}
\end{itemize}
