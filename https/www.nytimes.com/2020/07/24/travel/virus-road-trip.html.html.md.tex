Sections

SEARCH

\protect\hyperlink{site-content}{Skip to
content}\protect\hyperlink{site-index}{Skip to site index}

\href{https://www.nytimes.com/section/travel}{Travel}

\href{https://myaccount.nytimes.com/auth/login?response_type=cookie\&client_id=vi}{}

\href{https://www.nytimes.com/section/todayspaper}{Today's Paper}

\href{/section/travel}{Travel}\textbar{}Road Trips are Great. Except for
the Driving.

\url{https://nyti.ms/2Bwy97y}

\begin{itemize}
\item
\item
\item
\item
\item
\end{itemize}

\href{https://www.nytimes.com/news-event/coronavirus?action=click\&pgtype=Article\&state=default\&region=TOP_BANNER\&context=storylines_menu}{The
Coronavirus Outbreak}

\begin{itemize}
\tightlist
\item
  live\href{https://www.nytimes.com/2020/08/01/world/coronavirus-covid-19.html?action=click\&pgtype=Article\&state=default\&region=TOP_BANNER\&context=storylines_menu}{Latest
  Updates}
\item
  \href{https://www.nytimes.com/interactive/2020/us/coronavirus-us-cases.html?action=click\&pgtype=Article\&state=default\&region=TOP_BANNER\&context=storylines_menu}{Maps
  and Cases}
\item
  \href{https://www.nytimes.com/interactive/2020/science/coronavirus-vaccine-tracker.html?action=click\&pgtype=Article\&state=default\&region=TOP_BANNER\&context=storylines_menu}{Vaccine
  Tracker}
\item
  \href{https://www.nytimes.com/interactive/2020/07/29/us/schools-reopening-coronavirus.html?action=click\&pgtype=Article\&state=default\&region=TOP_BANNER\&context=storylines_menu}{What
  School May Look Like}
\item
  \href{https://www.nytimes.com/live/2020/07/31/business/stock-market-today-coronavirus?action=click\&pgtype=Article\&state=default\&region=TOP_BANNER\&context=storylines_menu}{Economy}
\end{itemize}

Advertisement

\protect\hyperlink{after-top}{Continue reading the main story}

Supported by

\protect\hyperlink{after-sponsor}{Continue reading the main story}

\hypertarget{road-trips-are-great-except-for-the-driving}{%
\section{Road Trips are Great. Except for the
Driving.}\label{road-trips-are-great-except-for-the-driving}}

I wasn't looking for an exotic vacation, just a temporary reprieve from
compulsive news-watching and a dose of in-person fun with family and
friends.

\includegraphics{https://static01.nyt.com/images/2020/07/23/travel/23travel-roadtrip/23travel-roadtrip-articleLarge.jpg?quality=75\&auto=webp\&disable=upscale}

By Amy Tara Koch

\begin{itemize}
\item
  July 24, 2020
\item
  \begin{itemize}
  \item
  \item
  \item
  \item
  \item
  \end{itemize}
\end{itemize}

I am in the Catskills in a charming, tucked-away treehouse of an inn. My
room, walking distance to hiking trails, overlooks a waterfall. Morning
coffee and evening vodka-tonic are taken on the deck where the
temperature clocks in at a marvelous 75 degrees. Owls hoot. Birds
chirrup. Wind tickles my legs.

During the coronavirus lockdown in Chicago, I dreamed about getting away
to this leafy utopia. What I did not envision was the hell of crossing
the country by car.

During difficult times, I always plant a light at the end of the tunnel:
a bright and shiny experience to make the tough moments more tolerable
and a positive attitude more attainable. Even for those lucky enough to
keep their jobs and their good health, the pandemic has been a very
difficult time.

To get away, to blunt the anxiety that this disease has wreaked upon us,
I wanted a change of scenery. Since I couldn't get to the Amalfi Coast
and wouldn't board a domestic flight, it was easy to buy into the hoopla
surrounding the All-American Road Trip. I wasn't looking for an exotic
vacation, just a temporary reprieve from compulsive news-watching and a
dose of in-person fun with family and friends.

Once I committed to the getaway, I felt excited for the first time since
mid-March. I was escaping! To multiple destinations! And I was going
alone! My husband, Peter, was busy with work and my teenage kids wanted
to hang out with their friend pod outside by the lake. Fine by me. The
prospect of leaving behind routine and responsibility --- meal planning,
specifically --- was heavenly.

The itinerary was ambitious. Peter would drive with me to Pittsburgh,
where we would stop for the night and pick up my rental car. I would
then continue to Washington, D.C., to visit my sister, on to a friend in
Connecticut, over to the Catskills and end at a lovely retreat in the
Hamptons. All of us had taken the lockdown seriously and agreed to
social distance during my visit.

Would it be too much driving? Peter asked. He knew how badly my back
throbbed after even a quickie flight to New York. I'll just bring
\href{https://www.thermacare.com/heat-wraps/back-pain-therapy}{Thermacare},
I replied breezily, knowing full well that he was right. I was not going
to let an annoying detail like chronic back pain get in my way.

\hypertarget{latest-updates-global-coronavirus-outbreak}{%
\section{\texorpdfstring{\href{https://www.nytimes.com/2020/08/01/world/coronavirus-covid-19.html?action=click\&pgtype=Article\&state=default\&region=MAIN_CONTENT_1\&context=storylines_live_updates}{Latest
Updates: Global Coronavirus
Outbreak}}{Latest Updates: Global Coronavirus Outbreak}}\label{latest-updates-global-coronavirus-outbreak}}

Updated 2020-08-02T00:50:37.907Z

\begin{itemize}
\tightlist
\item
  \href{https://www.nytimes.com/2020/08/01/world/coronavirus-covid-19.html?action=click\&pgtype=Article\&state=default\&region=MAIN_CONTENT_1\&context=storylines_live_updates\#link-34047410}{The
  U.S. reels as July cases more than double the total of any other
  month.}
\item
  \href{https://www.nytimes.com/2020/08/01/world/coronavirus-covid-19.html?action=click\&pgtype=Article\&state=default\&region=MAIN_CONTENT_1\&context=storylines_live_updates\#link-3ac56579}{Top
  officials work to break impasse over jobless benefit.}
\item
  \href{https://www.nytimes.com/2020/08/01/world/coronavirus-covid-19.html?action=click\&pgtype=Article\&state=default\&region=MAIN_CONTENT_1\&context=storylines_live_updates\#link-25930521}{Thousands
  in Berlin protest Germany's coronavirus measures.}
\end{itemize}

\href{https://www.nytimes.com/2020/08/01/world/coronavirus-covid-19.html?action=click\&pgtype=Article\&state=default\&region=MAIN_CONTENT_1\&context=storylines_live_updates}{See
more updates}

More live coverage:
\href{https://www.nytimes.com/live/2020/07/31/business/stock-market-today-coronavirus?action=click\&pgtype=Article\&state=default\&region=MAIN_CONTENT_1\&context=storylines_live_updates}{Markets}

I planned to cruise along scenic byways to a soundtrack of Bob Dylan,
Miles Davis and Journey. I would stop for adorable farm stands and
pastoral picnics. I would stretch in the shade of giant sycamore trees.
Like me, the other travelers would be respectfully clad in masks.

The first whiff of anxiety came as I gathered my hygiene arsenal, a
go-bag filled with gloves, masks, Clorox wipes and multiple purse-sized
Purell bottles. There was a deadly virus out there and I could be
exposed to it. Was I being reckless? I tossed Emergen-C packets and a
quart-sized plastic bag filled with vitamin supplements into the bag. A
strong immunity system was another layer of armor.

\hypertarget{chicago-to-pittsburgh}{%
\subsection{Chicago to Pittsburgh}\label{chicago-to-pittsburgh}}

When it was time for Peter and me to hit the road, my left brain did not
compute that the first 462 miles would be on toll roads with scenery
about as thrilling as a Boca Raton office park. Bathroom breaks were an
even greater obstacle. For coffee-lovers, hours in the car means endless
pit stops. My preference would have been to skedaddle behind a tree. But
even if you did risk pulling over, it turns out that these toll roads
are lined with barriers with few tree-shaded nooks. So on the hour, we
pulled the car into a rest stop and I donned my mask and gloves,
speed-walked into the ladies room, flushed with my foot and sprinted out
of the stall holding my breath.

Somewhere in Indiana, I got the brilliant idea to exit the highway in
search of an iced latte and more glamorous toilet. This detour ended at
Cracker Barrel, which was a nicer option but not worth the 30 minutes we
then spent idling at a broken tollbooth. Lesson learned.

\hypertarget{pittsburgh-to-washington}{%
\subsection{Pittsburgh to Washington}\label{pittsburgh-to-washington}}

After eight hours we made it to Pittsburgh, where I picked up a Toyota
4Runner equipped with an E-Z Pass and not much else. Car rental
companies claim an increase in coronavirus cleaning protocols, but my
car had what looked like blueberry muffin residue caked to the gears and
in the seats. When I pointed this out, the cleaning crew took another
pass. I still wiped every surface down with Clorox, encased the driver's
seat with a \href{https://seatsitters.com/}{seat cover} (I use these on
planes, too), and placed a towel on the passenger seat. I didn't notice
the dank Marlboro scent or broken Bluetooth system until I had driven
away.

This was not part of the plan. Even when I plugged my phone into the car
and pressed ``go'' on the maps app, no audio could be heard from the car
speakers. Would I have to drive the next 246 miles, without voice-guided
navigation? That would not be good for me. I could, however, access
Siri's dulcet-toned directions when the phone was not plugged in. So,
I'd drive with the phone on speaker and deal with a drained battery
every 80 minutes or so. To preserve power, I'd need to swap my classic
rock playlists for local radio. Another crack in my fantasy.

However, this first leg of solo driving wasn't bad, save the music
situation and the undercurrent of anxiety I felt each time I had to use
the bathroom or fuel up. I was off the dull, never-changing Midwestern
roads. Pennsylvania rest stops were shaded and pleasant. Each hour, I
whipped out my elastic workout band to stretch. In four hours, I was at
my sister's house, and the next few days were spent hiking, cooking and
singing karaoke to 1980s songs.

\hypertarget{washington-to-connecticut}{%
\subsection{Washington to Connecticut}\label{washington-to-connecticut}}

Next up: the Connecticut town of Sharon. Google Maps had the 321-mile
leg at five-and-a-half hours, which I rounded down to five hours (I tend
to speed). When I hoisted myself into the driver's seat, I practically
retched. The humidity had intensified the car's rank smell and despite
the burning heat, I had to roll down all of the windows. Thankfully, a
decent rock station helped me deal with an hour's worth of traffic as I
left Washington, but then MapQuest directed me to change highways in
what felt like every few miles. Through Maryland, Delaware, New Jersey
and New York the ride required hyper-attentiveness, something I had in
short supply since my sister's smoke detector had gone off at 2 a.m.

After two hours, even with
\href{https://us.hisamitsu/pain-relief-products}{Salonpas}pain patches
affixed to my shoulders, I felt the telltale spasm at my scapula. It
would inevitably explode into migraine-like waves of pain radiating from
my neck to my tailbone. When it did, I had to pull over.

I washed down three Advil and fished out the tennis ball I use as a
massage tool and jammed it under my shoulder. On top of this pain, my
phone drained every so often, forcing me to plug it in and glance down
at directions while driving, something I don't recommend. This happened
precisely as I hit a busy interchange outside New York City. Major
Deegan? Cross Bronx Expressway? I-87? I-95? One wrong turn and I'd be
caught in an off-ramp cycle for hours. **** Here, I was grateful for the
surprising glut of cars. The traffic gave me just enough time to glance
down and scan Siri's directive.

Finally, I noted signs for Connecticut. Almost there, I told myself.
Just as I started to relax, I saw that I was back in New York. I pulled
over to consult the G.P.S. Had I spaced and made a wrong turn? I hadn't.
Sharon is in the northwest corner of Connecticut so there is a
crisscross situation at the states' borders. I arrived at my friend's
home looking --- and smelling --- as if I'd run a marathon. The drive
had taken seven hours. Thankfully, she had chilled wine at the ready.

\href{https://www.nytimes.com/news-event/coronavirus?action=click\&pgtype=Article\&state=default\&region=MAIN_CONTENT_3\&context=storylines_faq}{}

\hypertarget{the-coronavirus-outbreak-}{%
\subsubsection{The Coronavirus Outbreak
›}\label{the-coronavirus-outbreak-}}

\hypertarget{frequently-asked-questions}{%
\paragraph{Frequently Asked
Questions}\label{frequently-asked-questions}}

Updated July 27, 2020

\begin{itemize}
\item ~
  \hypertarget{should-i-refinance-my-mortgage}{%
  \paragraph{Should I refinance my
  mortgage?}\label{should-i-refinance-my-mortgage}}

  \begin{itemize}
  \tightlist
  \item
    \href{https://www.nytimes.com/article/coronavirus-money-unemployment.html?action=click\&pgtype=Article\&state=default\&region=MAIN_CONTENT_3\&context=storylines_faq}{It
    could be a good idea,} because mortgage rates have
    \href{https://www.nytimes.com/2020/07/16/business/mortgage-rates-below-3-percent.html?action=click\&pgtype=Article\&state=default\&region=MAIN_CONTENT_3\&context=storylines_faq}{never
    been lower.} Refinancing requests have pushed mortgage applications
    to some of the highest levels since 2008, so be prepared to get in
    line. But defaults are also up, so if you're thinking about buying a
    home, be aware that some lenders have tightened their standards.
  \end{itemize}
\item ~
  \hypertarget{what-is-school-going-to-look-like-in-september}{%
  \paragraph{What is school going to look like in
  September?}\label{what-is-school-going-to-look-like-in-september}}

  \begin{itemize}
  \tightlist
  \item
    It is unlikely that many schools will return to a normal schedule
    this fall, requiring the grind of
    \href{https://www.nytimes.com/2020/06/05/us/coronavirus-education-lost-learning.html?action=click\&pgtype=Article\&state=default\&region=MAIN_CONTENT_3\&context=storylines_faq}{online
    learning},
    \href{https://www.nytimes.com/2020/05/29/us/coronavirus-child-care-centers.html?action=click\&pgtype=Article\&state=default\&region=MAIN_CONTENT_3\&context=storylines_faq}{makeshift
    child care} and
    \href{https://www.nytimes.com/2020/06/03/business/economy/coronavirus-working-women.html?action=click\&pgtype=Article\&state=default\&region=MAIN_CONTENT_3\&context=storylines_faq}{stunted
    workdays} to continue. California's two largest public school
    districts --- Los Angeles and San Diego --- said on July 13, that
    \href{https://www.nytimes.com/2020/07/13/us/lausd-san-diego-school-reopening.html?action=click\&pgtype=Article\&state=default\&region=MAIN_CONTENT_3\&context=storylines_faq}{instruction
    will be remote-only in the fall}, citing concerns that surging
    coronavirus infections in their areas pose too dire a risk for
    students and teachers. Together, the two districts enroll some
    825,000 students. They are the largest in the country so far to
    abandon plans for even a partial physical return to classrooms when
    they reopen in August. For other districts, the solution won't be an
    all-or-nothing approach.
    \href{https://bioethics.jhu.edu/research-and-outreach/projects/eschool-initiative/school-policy-tracker/}{Many
    systems}, including the nation's largest, New York City, are
    devising
    \href{https://www.nytimes.com/2020/06/26/us/coronavirus-schools-reopen-fall.html?action=click\&pgtype=Article\&state=default\&region=MAIN_CONTENT_3\&context=storylines_faq}{hybrid
    plans} that involve spending some days in classrooms and other days
    online. There's no national policy on this yet, so check with your
    municipal school system regularly to see what is happening in your
    community.
  \end{itemize}
\item ~
  \hypertarget{is-the-coronavirus-airborne}{%
  \paragraph{Is the coronavirus
  airborne?}\label{is-the-coronavirus-airborne}}

  \begin{itemize}
  \tightlist
  \item
    The coronavirus
    \href{https://www.nytimes.com/2020/07/04/health/239-experts-with-one-big-claim-the-coronavirus-is-airborne.html?action=click\&pgtype=Article\&state=default\&region=MAIN_CONTENT_3\&context=storylines_faq}{can
    stay aloft for hours in tiny droplets in stagnant air}, infecting
    people as they inhale, mounting scientific evidence suggests. This
    risk is highest in crowded indoor spaces with poor ventilation, and
    may help explain super-spreading events reported in meatpacking
    plants, churches and restaurants.
    \href{https://www.nytimes.com/2020/07/06/health/coronavirus-airborne-aerosols.html?action=click\&pgtype=Article\&state=default\&region=MAIN_CONTENT_3\&context=storylines_faq}{It's
    unclear how often the virus is spread} via these tiny droplets, or
    aerosols, compared with larger droplets that are expelled when a
    sick person coughs or sneezes, or transmitted through contact with
    contaminated surfaces, said Linsey Marr, an aerosol expert at
    Virginia Tech. Aerosols are released even when a person without
    symptoms exhales, talks or sings, according to Dr. Marr and more
    than 200 other experts, who
    \href{https://academic.oup.com/cid/article/doi/10.1093/cid/ciaa939/5867798}{have
    outlined the evidence in an open letter to the World Health
    Organization}.
  \end{itemize}
\item ~
  \hypertarget{what-are-the-symptoms-of-coronavirus}{%
  \paragraph{What are the symptoms of
  coronavirus?}\label{what-are-the-symptoms-of-coronavirus}}

  \begin{itemize}
  \tightlist
  \item
    Common symptoms
    \href{https://www.nytimes.com/article/symptoms-coronavirus.html?action=click\&pgtype=Article\&state=default\&region=MAIN_CONTENT_3\&context=storylines_faq}{include
    fever, a dry cough, fatigue and difficulty breathing or shortness of
    breath.} Some of these symptoms overlap with those of the flu,
    making detection difficult, but runny noses and stuffy sinuses are
    less common.
    \href{https://www.nytimes.com/2020/04/27/health/coronavirus-symptoms-cdc.html?action=click\&pgtype=Article\&state=default\&region=MAIN_CONTENT_3\&context=storylines_faq}{The
    C.D.C. has also} added chills, muscle pain, sore throat, headache
    and a new loss of the sense of taste or smell as symptoms to look
    out for. Most people fall ill five to seven days after exposure, but
    symptoms may appear in as few as two days or as many as 14 days.
  \end{itemize}
\item ~
  \hypertarget{does-asymptomatic-transmission-of-covid-19-happen}{%
  \paragraph{Does asymptomatic transmission of Covid-19
  happen?}\label{does-asymptomatic-transmission-of-covid-19-happen}}

  \begin{itemize}
  \tightlist
  \item
    So far, the evidence seems to show it does. A widely cited
    \href{https://www.nature.com/articles/s41591-020-0869-5}{paper}
    published in April suggests that people are most infectious about
    two days before the onset of coronavirus symptoms and estimated that
    44 percent of new infections were a result of transmission from
    people who were not yet showing symptoms. Recently, a top expert at
    the World Health Organization stated that transmission of the
    coronavirus by people who did not have symptoms was ``very rare,''
    \href{https://www.nytimes.com/2020/06/09/world/coronavirus-updates.html?action=click\&pgtype=Article\&state=default\&region=MAIN_CONTENT_3\&context=storylines_faq\#link-1f302e21}{but
    she later walked back that statement.}
  \end{itemize}
\end{itemize}

\hypertarget{connecticut-to-the-catskills}{%
\subsection{Connecticut to the
Catskills}\label{connecticut-to-the-catskills}}

A few days later, relaxed and revived, I got back into the car (this
time I had wisely left the windows open overnight) and headed north to
the Catskills, an easy hour's drive that took me past the bucolic farms
of Dutchess County and into the Hudson Valley. When I got to
\href{https://www.woodstockway.com/}{Woodstock Way Hotel}, it was just
as I'd remembered: a perfect hideaway.

I hiked and dined with my cousins. At the farm of my oldest friend,
Marcey, I had a glorious picnic alongside ripening tomatoes upon
socially distanced blankets. I had swaths of time to read and write.
Things felt almost normal, save for the very odd bits like watching my
martini being shaken by a masked-and-gloved bartender, and the 7 a.m.
wait for coffee with masked, socially distanced locals outside a rural
bakery. As usual, once I began moving around, my back pain receded. I
was tempted to book a massage, but decided that good old yoga and my
tennis ball would suffice.

I axed the Hamptons from my trip. I could not endure the five hours it
would add to my return drive to Chicago. Ever the thoughtful friend,
Marcey snipped a bouquet of lavender, mint and lemon balm and plopped it
in the Toyota's cup holder, a farm-fresh flourish to combat the car's
malodorous funk for my last long drive.

\hypertarget{get-me-home}{%
\subsection{Get Me Home}\label{get-me-home}}

On the day of my departure, I took an early hike through the
\href{https://ashokanrailtrail.com/}{Ashokan Rail Trail}and hit the road
by 11:30 a.m. It drizzled as I was leaving New York. Mother Nature
waited until I was in the Pocono Mountains to send golf-ball-size hail
to crash down with such intensity I thought the windshield might crack.
This was terrifying, even though I was used to driving in blizzards in
Chicago. I flicked on the hazard lights and made my way to the side of
the road. Visibility was zero. I waited for 30 minutes, using the time
to charge the phone and down a bottle of water spiked with Emergen-C,
something I knew would increase my intimacy with rest stops along I-84.

Though scenic, this drive was supremely boring. My thoughts invariably
shifted to the state of the world. Would my daughter Bella be able to
attend college, as planned? What would that look like with pandemic
parameters? Would my other daughter, Brette, have a normal high school
experience with remote learning? Would the virus reach its tentacles
deep within 2021? Was a vaccine forthcoming? Who would win the election
in November?

The speculation and angst was tiring. The Connecticut leg had primed me
for at-the-wheel exhaustion. This time I was prepared. I had picked up a
facial mist spray and spritzed it on my neck and cheeks every few
minutes. This and iced coffee kept me alert, though the stabbing back
pain remained a constant companion.

At 7 p.m., I staggered into the hotel oozing eau de Tiger Balm. I was
thrilled to see my husband after 10 days and relieved to bid adieu to
the driving ordeal. The next day's trip back to Chicago was uneventful.
I mostly reclined and slept.

Was it worth the schlep? Yes. FaceTime and Zoom are not substitutes for
spending quality time with friends and family. I was lucky enough to be
in a position to visit them, while staying within health guidelines,
something I know not everyone can do. The in-person connections --- and
putting the darkness of lockdown in the rear view mirror --- gave me the
reboot I needed. But driving across the country alone is a one and done
experience. Hypervigilance was draining. That night and for the next
three nights, I slept for 10 hours.

Now, should you need to find me anytime soon, try me in Chicago. I'll be
in reading (cocktail in hand) on the deck, or fine-tuning my back at the
acupuncturist.

Amy Tara Koch, based in Chicago, writes about travel, style, food and
parenting.

\begin{center}\rule{0.5\linewidth}{\linethickness}\end{center}

\emph{\textbf{Follow New York Times Travel}}
\emph{on}\href{https://www.instagram.com/nytimestravel/}{\emph{Instagram}}\emph{,}\href{https://twitter.com/nytimestravel}{\emph{Twitter}}
\emph{and}\href{https://www.facebook.com/nytimestravel/}{\emph{Facebook}}\emph{.
And}\href{https://www.nytimes.com/newsletters/traveldispatch}{\emph{sign
up for our weekly Travel Dispatch newsletter}} \emph{to receive expert
tips on traveling smarter and inspiration for your next vacation.}

Advertisement

\protect\hyperlink{after-bottom}{Continue reading the main story}

\hypertarget{site-index}{%
\subsection{Site Index}\label{site-index}}

\hypertarget{site-information-navigation}{%
\subsection{Site Information
Navigation}\label{site-information-navigation}}

\begin{itemize}
\tightlist
\item
  \href{https://help.nytimes.com/hc/en-us/articles/115014792127-Copyright-notice}{©~2020~The
  New York Times Company}
\end{itemize}

\begin{itemize}
\tightlist
\item
  \href{https://www.nytco.com/}{NYTCo}
\item
  \href{https://help.nytimes.com/hc/en-us/articles/115015385887-Contact-Us}{Contact
  Us}
\item
  \href{https://www.nytco.com/careers/}{Work with us}
\item
  \href{https://nytmediakit.com/}{Advertise}
\item
  \href{http://www.tbrandstudio.com/}{T Brand Studio}
\item
  \href{https://www.nytimes.com/privacy/cookie-policy\#how-do-i-manage-trackers}{Your
  Ad Choices}
\item
  \href{https://www.nytimes.com/privacy}{Privacy}
\item
  \href{https://help.nytimes.com/hc/en-us/articles/115014893428-Terms-of-service}{Terms
  of Service}
\item
  \href{https://help.nytimes.com/hc/en-us/articles/115014893968-Terms-of-sale}{Terms
  of Sale}
\item
  \href{https://spiderbites.nytimes.com}{Site Map}
\item
  \href{https://help.nytimes.com/hc/en-us}{Help}
\item
  \href{https://www.nytimes.com/subscription?campaignId=37WXW}{Subscriptions}
\end{itemize}
