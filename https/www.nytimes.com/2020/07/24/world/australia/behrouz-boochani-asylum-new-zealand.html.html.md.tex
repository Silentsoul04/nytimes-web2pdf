Sections

SEARCH

\protect\hyperlink{site-content}{Skip to
content}\protect\hyperlink{site-index}{Skip to site index}

\href{https://www.nytimes.com/section/world/australia}{Australia}

\href{https://myaccount.nytimes.com/auth/login?response_type=cookie\&client_id=vi}{}

\href{https://www.nytimes.com/section/todayspaper}{Today's Paper}

\href{/section/world/australia}{Australia}\textbar{}Refugee and Author
Long Detained by Australia Gets Asylum in New Zealand

\url{https://nyti.ms/2E5W3Yf}

\begin{itemize}
\item
\item
\item
\item
\item
\end{itemize}

Advertisement

\protect\hyperlink{after-top}{Continue reading the main story}

Supported by

\protect\hyperlink{after-sponsor}{Continue reading the main story}

\hypertarget{refugee-and-author-long-detained-by-australia-gets-asylum-in-new-zealand}{%
\section{Refugee and Author Long Detained by Australia Gets Asylum in
New
Zealand}\label{refugee-and-author-long-detained-by-australia-gets-asylum-in-new-zealand}}

Behrouz Boochani, a Kurdish-Iranian exile, said the news showed the vast
differences between the two neighboring countries on human rights.

\includegraphics{https://static01.nyt.com/images/2020/07/24/world/24newzealand-refugee1/merlin_114968585_8ac234ec-8384-4a9d-b44e-4b47958a8e82-articleLarge.jpg?quality=75\&auto=webp\&disable=upscale}

\href{https://www.nytimes.com/by/livia-albeck-ripka}{\includegraphics{https://static01.nyt.com/images/2018/06/12/multimedia/author-livia-albeck-ripka/author-livia-albeck-ripka-thumbLarge.png}}

By \href{https://www.nytimes.com/by/livia-albeck-ripka}{Livia
Albeck-Ripka}

\begin{itemize}
\item
  July 24, 2020
\item
  \begin{itemize}
  \item
  \item
  \item
  \item
  \item
  \end{itemize}
\end{itemize}

CAIRNS, Australia --- Behrouz Boochani, a Kurdish-Iranian refugee and
writer who documented abuses under Australia's tough immigration
policies during his yearslong detention on a remote Pacific island, has
been granted asylum in New Zealand.

Mr. Boochani has spent the past several months in Christchurch, where he
applied for refugee status after
\href{https://www.nytimes.com/2019/11/14/world/australia/behrouz-boochani-refugee.html}{being
given a temporary visa} to attend a writers' festival last November.

On Thursday --- which was also Mr. Boochani's 37th birthday, and exactly
seven years since he was first detained on Manus Island in Papua New
Guinea --- he received official notification that his application for a
one-year working visa in New Zealand had been successful.

Mr. Boochani said he planned to apply for permanent residency, which is
a pathway to citizenship. ``It's like the end of a chapter of my life,''
he said, adding that his news was bittersweet because he feared for
those still being detained by the Australian authorities.

Under the Australian government's strict policies for those who attempt
to reach the country by boat, thousands of people, many from the Middle
East and Africa,
\href{https://www.nytimes.com/interactive/2017/11/18/world/australia/manus-island-australia-detainees.html}{have
been detained indefinitely on Manus} and the island nation of Nauru.
``The policy exists, and so that's why it's really difficult to fully
enjoy this,'' he said.

Mr. Boochani said that the vast differences in the way he had been
treated by the Australian and New Zealand authorities revealed a chasm
in their leadership on human rights.

``We ask people in the international community to look at this country,
to look at Australia and what they have done and what they are doing,''
he said, adding that he hoped others in detention would also eventually
be given asylum. ``They must release them.''

\includegraphics{https://static01.nyt.com/images/2020/07/24/world/24newzealand-refugee2/merlin_164638395_87e1d118-b630-4ae3-b1fe-939ff3aadf5e-articleLarge.jpg?quality=75\&auto=webp\&disable=upscale}

Mr. Boochani, who formerly worked as a journalist with the
Kurdish-language magazine Werya,
\href{https://www.nytimes.com/2017/02/13/insider/manus-island-refugee-australia.html}{fled
Iran} in 2013 after the police arrested several of his colleagues and
raided his office. He spent a few months in Indonesia before trying to
travel to Australia by boat, but he was intercepted by the Australian
Navy and sent to Manus Island.

There, he documented human rights abuses against himself and others,
raising awareness of the squalid conditions and
\href{https://www.nytimes.com/2018/11/05/world/australia/nauru-island-asylum-refugees-children-suicide.html}{deteriorating
mental health} of the men he lived with --- many of whom had fled
persecution in their home countries.

Mr. Boochani's posts on social media, which detailed self-harm and
suicides by detainees, as well as inadequate access to health care,
helped expose policies that had largely been obscured by extremely
limited access to the camps for journalists and activists.

In early 2019, Mr. Boochani was
\href{https://www.nytimes.com/2019/01/31/world/australia/behrouz-boochani-victorian-prize-manus-island.html}{awarded
Australia's highest-paying literary prize} for his book ``No Friend but
the Mountains,'' which was written entirely via WhatsApp. It further
detailed abuses in the camps, cementing him as a voice for those who had
largely been silenced by Australia's strict policies. He could not
attend the awards ceremony because of his detainment.

In November, he received a visitor visa to New Zealand in order to
attend the WORD literary festival in Christchurch. He has been living
there since, working as a researcher with the University of Canterbury.

Advertisement

\protect\hyperlink{after-bottom}{Continue reading the main story}

\hypertarget{site-index}{%
\subsection{Site Index}\label{site-index}}

\hypertarget{site-information-navigation}{%
\subsection{Site Information
Navigation}\label{site-information-navigation}}

\begin{itemize}
\tightlist
\item
  \href{https://help.nytimes.com/hc/en-us/articles/115014792127-Copyright-notice}{©~2020~The
  New York Times Company}
\end{itemize}

\begin{itemize}
\tightlist
\item
  \href{https://www.nytco.com/}{NYTCo}
\item
  \href{https://help.nytimes.com/hc/en-us/articles/115015385887-Contact-Us}{Contact
  Us}
\item
  \href{https://www.nytco.com/careers/}{Work with us}
\item
  \href{https://nytmediakit.com/}{Advertise}
\item
  \href{http://www.tbrandstudio.com/}{T Brand Studio}
\item
  \href{https://www.nytimes.com/privacy/cookie-policy\#how-do-i-manage-trackers}{Your
  Ad Choices}
\item
  \href{https://www.nytimes.com/privacy}{Privacy}
\item
  \href{https://help.nytimes.com/hc/en-us/articles/115014893428-Terms-of-service}{Terms
  of Service}
\item
  \href{https://help.nytimes.com/hc/en-us/articles/115014893968-Terms-of-sale}{Terms
  of Sale}
\item
  \href{https://spiderbites.nytimes.com}{Site Map}
\item
  \href{https://help.nytimes.com/hc/en-us}{Help}
\item
  \href{https://www.nytimes.com/subscription?campaignId=37WXW}{Subscriptions}
\end{itemize}
