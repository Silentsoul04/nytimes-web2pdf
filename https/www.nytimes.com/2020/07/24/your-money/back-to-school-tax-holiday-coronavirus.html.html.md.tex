Sections

SEARCH

\protect\hyperlink{site-content}{Skip to
content}\protect\hyperlink{site-index}{Skip to site index}

\href{https://www.nytimes.com/section/your-money}{Your Money}

\href{https://myaccount.nytimes.com/auth/login?response_type=cookie\&client_id=vi}{}

\href{https://www.nytimes.com/section/todayspaper}{Today's Paper}

\href{/section/your-money}{Your Money}\textbar{}16 States Go Ahead With
`Back to School' Sales Tax Holidays

\url{https://nyti.ms/30LqHOa}

\begin{itemize}
\item
\item
\item
\item
\item
\end{itemize}

\href{https://www.nytimes.com/news-event/coronavirus?action=click\&pgtype=Article\&state=default\&region=TOP_BANNER\&context=storylines_menu}{The
Coronavirus Outbreak}

\begin{itemize}
\tightlist
\item
  live\href{https://www.nytimes.com/2020/08/01/world/coronavirus-covid-19.html?action=click\&pgtype=Article\&state=default\&region=TOP_BANNER\&context=storylines_menu}{Latest
  Updates}
\item
  \href{https://www.nytimes.com/interactive/2020/us/coronavirus-us-cases.html?action=click\&pgtype=Article\&state=default\&region=TOP_BANNER\&context=storylines_menu}{Maps
  and Cases}
\item
  \href{https://www.nytimes.com/interactive/2020/science/coronavirus-vaccine-tracker.html?action=click\&pgtype=Article\&state=default\&region=TOP_BANNER\&context=storylines_menu}{Vaccine
  Tracker}
\item
  \href{https://www.nytimes.com/interactive/2020/07/29/us/schools-reopening-coronavirus.html?action=click\&pgtype=Article\&state=default\&region=TOP_BANNER\&context=storylines_menu}{What
  School May Look Like}
\item
  \href{https://www.nytimes.com/live/2020/07/31/business/stock-market-today-coronavirus?action=click\&pgtype=Article\&state=default\&region=TOP_BANNER\&context=storylines_menu}{Economy}
\end{itemize}

Advertisement

\protect\hyperlink{after-top}{Continue reading the main story}

Supported by

\protect\hyperlink{after-sponsor}{Continue reading the main story}

Your money adviser

\hypertarget{16-states-go-ahead-with-back-to-school-sales-tax-holidays}{%
\section{16 States Go Ahead With `Back to School' Sales Tax
Holidays}\label{16-states-go-ahead-with-back-to-school-sales-tax-holidays}}

Schools may not reopen, and the ``holiday'' cuts into government
revenue. But during a pandemic, states may be thinking, why not offer a
bit of relief?

\includegraphics{https://static01.nyt.com/images/2020/07/25/business/25adviser/25adviser-articleLarge.jpg?quality=75\&auto=webp\&disable=upscale}

By \href{https://www.nytimes.com/by/ann-carrns}{Ann Carrns}

\begin{itemize}
\item
  July 24, 2020
\item
  \begin{itemize}
  \item
  \item
  \item
  \item
  \item
  \end{itemize}
\end{itemize}

Even though the coronavirus may make ``back to school'' a misnomer, many
states are going ahead with summer sales tax ``holidays'' that give
shoppers a break on back-to-school items.

This year, 16 states are temporarily exempting clothing, shoes,
notebooks and other school supplies, sometimes including computers, from
state, and often local, sales taxes.

Shoppers can save up to 9 percent during the promotions, which typically
last for a weekend but can be longer in some states. A handful of states
also waive taxes on items with other themes, like disaster preparedness,
during the summer promotions or at other times of the year.

At least one state, Tennessee, added an extra tax holiday weekend in
August, focused on restaurant spending. ``This year, we noted that
consumers who have been cooped up at home might enjoy a sales tax
holiday for purchases in restaurants, either dine in or take out,''
State Representative Susan Lynn, chair of the Tennessee House Finance,
Ways and Means Committee, said in an email.

In many states, details around returning to school are still up in the
air. But a survey by the National Retail Federation, a trade group,
found that families are expecting to spend a record \$790, on average,
on back-to-school items this year, particularly on technology. Nearly
two-thirds of families with children in kindergarten through 12th grade
said they expected to buy computers and other electronics, up from about
half last year, because of the potential for digital at-home classes.

``There's a lot of uncertainty around the school year,'' said Katherine
Cullen, the senior director of industry and consumer insights at the
retail federation. ``Customers are budgeting for all possible
scenarios.''

As details become clearer, she said, families may adjust their spending.
Children need new clothes as they grow regardless of whether they learn
at home or in school, but items like backpacks or lunchboxes may not be
necessary for remote lessons.

\hypertarget{latest-updates-global-coronavirus-outbreak}{%
\section{\texorpdfstring{\href{https://www.nytimes.com/2020/08/01/world/coronavirus-covid-19.html?action=click\&pgtype=Article\&state=default\&region=MAIN_CONTENT_1\&context=storylines_live_updates}{Latest
Updates: Global Coronavirus
Outbreak}}{Latest Updates: Global Coronavirus Outbreak}}\label{latest-updates-global-coronavirus-outbreak}}

Updated 2020-08-01T18:42:36.154Z

\begin{itemize}
\tightlist
\item
  \href{https://www.nytimes.com/2020/08/01/world/coronavirus-covid-19.html?action=click\&pgtype=Article\&state=default\&region=MAIN_CONTENT_1\&context=storylines_live_updates\#link-3ac56579}{Top
  officials work to break impasse over jobless benefit.}
\item
  \href{https://www.nytimes.com/2020/08/01/world/coronavirus-covid-19.html?action=click\&pgtype=Article\&state=default\&region=MAIN_CONTENT_1\&context=storylines_live_updates\#link-8796723}{The
  virus picks up dangerous speed in the Midwest, and in areas that had
  seen success.}
\item
  \href{https://www.nytimes.com/2020/08/01/world/coronavirus-covid-19.html?action=click\&pgtype=Article\&state=default\&region=MAIN_CONTENT_1\&context=storylines_live_updates\#link-25930521}{Thousands
  in Berlin protest Germany's coronavirus measures.}
\end{itemize}

\href{https://www.nytimes.com/2020/08/01/world/coronavirus-covid-19.html?action=click\&pgtype=Article\&state=default\&region=MAIN_CONTENT_1\&context=storylines_live_updates}{See
more updates}

More live coverage:
\href{https://www.nytimes.com/live/2020/07/31/business/stock-market-today-coronavirus?action=click\&pgtype=Article\&state=default\&region=MAIN_CONTENT_1\&context=storylines_live_updates}{Markets}

Fewer than half the states with tax holidays include computers on their
tax-exempt menus, and all set limits on the exempt amount, according to
a list compiled by the Federation of Tax Administrators. (The ones that
include computers are Alabama, Florida, Massachusetts, Missouri, New
Mexico, South Carolina and Tennessee.)

While popular with both politicians and shoppers, sales tax holidays are
generally frowned upon by tax policy experts, who say they offer modest
benefits to most consumers while starving states of revenue for needed
services. Some
\href{https://www.federalreserve.gov/econres/notes/feds-notes/effect-of-sales-tax-holidays-on-consumer-spending-20170324.htm}{research}
suggests that the holidays shift the timing of purchases rather than
spur new spending.

This year, state sales tax
\href{https://www.taxpolicycenter.org/taxvox/covid-19-effect-state-sales-tax-receipts-shrank-6-billion-may}{revenues}
appear to be in free fall, according to an analysis by the Tax Policy
Center, a joint initiative of two nonprofit think tanks, the Urban
Institute and the Brookings Institution. Because of stay-at-home orders
and business closings in the pandemic, along with tax payment
extensions, state sales tax revenue in May fell \$6 billion over all, or
21 percent, from a year earlier, the center calculated. In some states,
the decline was more than 30 percent.

The declines are ``unprecedented,'' the report noted, and because of the
pandemic, sales tax revenues are unlikely to rebound to normal anytime
soon.

``At a time of public health and revenue crisis,'' Lucy Dadayan, senior
research associate at the Tax Policy Center, said in an email, ``sales
tax holidays will help some consumers to save very little but at a cost
to governments.''

Recent state financial reports indicate, for instance, that Alabama's
back-to-school holiday cost the state \$8 million, while Oklahoma gave
up \$7.4 million, said Janelle Cammenga, a policy analyst with the
nonprofit Tax Foundation. Iowa's 2015 holiday cost the state \$3.6
million.

While those figures are relatively small, Ms. Cammenga said, ``sales tax
holiday expenditures represent revenue that is lost without gaining its
intended effect of economic stimulation or job growth.''

Representative Lynn said Tennessee's two holidays would cost the state
about \$25 million in lost revenue. (She noted that the holiday had been
factored into the state budget passed in June, and that local
governments would be made ``whole'' for local sales taxes lost during
the tax-exempt periods.)

\href{https://www.nytimes.com/news-event/coronavirus?action=click\&pgtype=Article\&state=default\&region=MAIN_CONTENT_3\&context=storylines_faq}{}

\hypertarget{the-coronavirus-outbreak-}{%
\subsubsection{The Coronavirus Outbreak
›}\label{the-coronavirus-outbreak-}}

\hypertarget{frequently-asked-questions}{%
\paragraph{Frequently Asked
Questions}\label{frequently-asked-questions}}

Updated July 27, 2020

\begin{itemize}
\item ~
  \hypertarget{should-i-refinance-my-mortgage}{%
  \paragraph{Should I refinance my
  mortgage?}\label{should-i-refinance-my-mortgage}}

  \begin{itemize}
  \tightlist
  \item
    \href{https://www.nytimes.com/article/coronavirus-money-unemployment.html?action=click\&pgtype=Article\&state=default\&region=MAIN_CONTENT_3\&context=storylines_faq}{It
    could be a good idea,} because mortgage rates have
    \href{https://www.nytimes.com/2020/07/16/business/mortgage-rates-below-3-percent.html?action=click\&pgtype=Article\&state=default\&region=MAIN_CONTENT_3\&context=storylines_faq}{never
    been lower.} Refinancing requests have pushed mortgage applications
    to some of the highest levels since 2008, so be prepared to get in
    line. But defaults are also up, so if you're thinking about buying a
    home, be aware that some lenders have tightened their standards.
  \end{itemize}
\item ~
  \hypertarget{what-is-school-going-to-look-like-in-september}{%
  \paragraph{What is school going to look like in
  September?}\label{what-is-school-going-to-look-like-in-september}}

  \begin{itemize}
  \tightlist
  \item
    It is unlikely that many schools will return to a normal schedule
    this fall, requiring the grind of
    \href{https://www.nytimes.com/2020/06/05/us/coronavirus-education-lost-learning.html?action=click\&pgtype=Article\&state=default\&region=MAIN_CONTENT_3\&context=storylines_faq}{online
    learning},
    \href{https://www.nytimes.com/2020/05/29/us/coronavirus-child-care-centers.html?action=click\&pgtype=Article\&state=default\&region=MAIN_CONTENT_3\&context=storylines_faq}{makeshift
    child care} and
    \href{https://www.nytimes.com/2020/06/03/business/economy/coronavirus-working-women.html?action=click\&pgtype=Article\&state=default\&region=MAIN_CONTENT_3\&context=storylines_faq}{stunted
    workdays} to continue. California's two largest public school
    districts --- Los Angeles and San Diego --- said on July 13, that
    \href{https://www.nytimes.com/2020/07/13/us/lausd-san-diego-school-reopening.html?action=click\&pgtype=Article\&state=default\&region=MAIN_CONTENT_3\&context=storylines_faq}{instruction
    will be remote-only in the fall}, citing concerns that surging
    coronavirus infections in their areas pose too dire a risk for
    students and teachers. Together, the two districts enroll some
    825,000 students. They are the largest in the country so far to
    abandon plans for even a partial physical return to classrooms when
    they reopen in August. For other districts, the solution won't be an
    all-or-nothing approach.
    \href{https://bioethics.jhu.edu/research-and-outreach/projects/eschool-initiative/school-policy-tracker/}{Many
    systems}, including the nation's largest, New York City, are
    devising
    \href{https://www.nytimes.com/2020/06/26/us/coronavirus-schools-reopen-fall.html?action=click\&pgtype=Article\&state=default\&region=MAIN_CONTENT_3\&context=storylines_faq}{hybrid
    plans} that involve spending some days in classrooms and other days
    online. There's no national policy on this yet, so check with your
    municipal school system regularly to see what is happening in your
    community.
  \end{itemize}
\item ~
  \hypertarget{is-the-coronavirus-airborne}{%
  \paragraph{Is the coronavirus
  airborne?}\label{is-the-coronavirus-airborne}}

  \begin{itemize}
  \tightlist
  \item
    The coronavirus
    \href{https://www.nytimes.com/2020/07/04/health/239-experts-with-one-big-claim-the-coronavirus-is-airborne.html?action=click\&pgtype=Article\&state=default\&region=MAIN_CONTENT_3\&context=storylines_faq}{can
    stay aloft for hours in tiny droplets in stagnant air}, infecting
    people as they inhale, mounting scientific evidence suggests. This
    risk is highest in crowded indoor spaces with poor ventilation, and
    may help explain super-spreading events reported in meatpacking
    plants, churches and restaurants.
    \href{https://www.nytimes.com/2020/07/06/health/coronavirus-airborne-aerosols.html?action=click\&pgtype=Article\&state=default\&region=MAIN_CONTENT_3\&context=storylines_faq}{It's
    unclear how often the virus is spread} via these tiny droplets, or
    aerosols, compared with larger droplets that are expelled when a
    sick person coughs or sneezes, or transmitted through contact with
    contaminated surfaces, said Linsey Marr, an aerosol expert at
    Virginia Tech. Aerosols are released even when a person without
    symptoms exhales, talks or sings, according to Dr. Marr and more
    than 200 other experts, who
    \href{https://academic.oup.com/cid/article/doi/10.1093/cid/ciaa939/5867798}{have
    outlined the evidence in an open letter to the World Health
    Organization}.
  \end{itemize}
\item ~
  \hypertarget{what-are-the-symptoms-of-coronavirus}{%
  \paragraph{What are the symptoms of
  coronavirus?}\label{what-are-the-symptoms-of-coronavirus}}

  \begin{itemize}
  \tightlist
  \item
    Common symptoms
    \href{https://www.nytimes.com/article/symptoms-coronavirus.html?action=click\&pgtype=Article\&state=default\&region=MAIN_CONTENT_3\&context=storylines_faq}{include
    fever, a dry cough, fatigue and difficulty breathing or shortness of
    breath.} Some of these symptoms overlap with those of the flu,
    making detection difficult, but runny noses and stuffy sinuses are
    less common.
    \href{https://www.nytimes.com/2020/04/27/health/coronavirus-symptoms-cdc.html?action=click\&pgtype=Article\&state=default\&region=MAIN_CONTENT_3\&context=storylines_faq}{The
    C.D.C. has also} added chills, muscle pain, sore throat, headache
    and a new loss of the sense of taste or smell as symptoms to look
    out for. Most people fall ill five to seven days after exposure, but
    symptoms may appear in as few as two days or as many as 14 days.
  \end{itemize}
\item ~
  \hypertarget{does-asymptomatic-transmission-of-covid-19-happen}{%
  \paragraph{Does asymptomatic transmission of Covid-19
  happen?}\label{does-asymptomatic-transmission-of-covid-19-happen}}

  \begin{itemize}
  \tightlist
  \item
    So far, the evidence seems to show it does. A widely cited
    \href{https://www.nature.com/articles/s41591-020-0869-5}{paper}
    published in April suggests that people are most infectious about
    two days before the onset of coronavirus symptoms and estimated that
    44 percent of new infections were a result of transmission from
    people who were not yet showing symptoms. Recently, a top expert at
    the World Health Organization stated that transmission of the
    coronavirus by people who did not have symptoms was ``very rare,''
    \href{https://www.nytimes.com/2020/06/09/world/coronavirus-updates.html?action=click\&pgtype=Article\&state=default\&region=MAIN_CONTENT_3\&context=storylines_faq\#link-1f302e21}{but
    she later walked back that statement.}
  \end{itemize}
\end{itemize}

Because lost revenue from the tax holidays is typically a small
proportion of state budgets, elected officials may see the promotions as
an easy way to gain public favor without doing too much fiscal damage,
said David Brunori, a specialist in state and local taxation at tax and
auditing consultant RSM and a research professor at George Washington
University Law School.

This year, he said, the thinking among state legislators may be that
people have been suffering with virus-related shutdowns, civil unrest
and now heat waves, so why not offer a bit of relief. ``They're `feel
good' measures,'' he said.

Still, the drain on state revenue, combined with the incentive to crowd
into stores looking for bargains when coronavirus cases are surging,
suggests that 2020 may have been a good year for states to skip the tax
holidays, said Dylan Grundman, senior state policy analyst at the
Institute on Taxation and Economic Policy. ``They should have called
them off this year,'' he said.

Here are questions and answers about sales tax holidays:

\textbf{How much can I save during a sales tax holiday?}

State sales taxes range from about 4 to 7 percent but can be as high as
9 percent when additional local option sales taxes are included. Most
sales tax holidays include state and local taxes, but some exclude local
taxes or make them optional, reducing the savings.

Your savings may also be limited by a dollar cap on purchases, whether
it is based on the cost of an individual item or on the total receipt.

\textbf{I'm not comfortable shopping in stores because of the pandemic.
Can I get the tax break by shopping online?}

Generally, online purchases are eligible for the tax break, tax experts
say. But that option may not help lower-income workers, who may lack
internet access or the flexibility to be home from work to receive
deliveries, Mr. Grundman said.

\textbf{What states are holding sales tax holidays this year?}

States holding sales tax promotions in 2020, according to the
\href{https://www.taxadmin.org/2020-sales-tax-holiday}{Federation of Tax
Administrators}, are Alabama, Arkansas, Connecticut, Florida, Iowa,
Maryland, Massachusetts, Mississippi, Missouri, New Mexico, Ohio,
Oklahoma, South Carolina, Tennessee, Texas and Virginia.

Alaska, Delaware, Montana, New Hampshire and Oregon don't charge
statewide sales taxes in the first place. And other states may exempt
clothing and food from sales tax, at least up to certain limits, year
round.

Advertisement

\protect\hyperlink{after-bottom}{Continue reading the main story}

\hypertarget{site-index}{%
\subsection{Site Index}\label{site-index}}

\hypertarget{site-information-navigation}{%
\subsection{Site Information
Navigation}\label{site-information-navigation}}

\begin{itemize}
\tightlist
\item
  \href{https://help.nytimes.com/hc/en-us/articles/115014792127-Copyright-notice}{©~2020~The
  New York Times Company}
\end{itemize}

\begin{itemize}
\tightlist
\item
  \href{https://www.nytco.com/}{NYTCo}
\item
  \href{https://help.nytimes.com/hc/en-us/articles/115015385887-Contact-Us}{Contact
  Us}
\item
  \href{https://www.nytco.com/careers/}{Work with us}
\item
  \href{https://nytmediakit.com/}{Advertise}
\item
  \href{http://www.tbrandstudio.com/}{T Brand Studio}
\item
  \href{https://www.nytimes.com/privacy/cookie-policy\#how-do-i-manage-trackers}{Your
  Ad Choices}
\item
  \href{https://www.nytimes.com/privacy}{Privacy}
\item
  \href{https://help.nytimes.com/hc/en-us/articles/115014893428-Terms-of-service}{Terms
  of Service}
\item
  \href{https://help.nytimes.com/hc/en-us/articles/115014893968-Terms-of-sale}{Terms
  of Sale}
\item
  \href{https://spiderbites.nytimes.com}{Site Map}
\item
  \href{https://help.nytimes.com/hc/en-us}{Help}
\item
  \href{https://www.nytimes.com/subscription?campaignId=37WXW}{Subscriptions}
\end{itemize}
