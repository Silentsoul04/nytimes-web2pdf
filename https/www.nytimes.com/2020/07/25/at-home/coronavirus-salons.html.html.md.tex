Sections

SEARCH

\protect\hyperlink{site-content}{Skip to
content}\protect\hyperlink{site-index}{Skip to site index}

\href{https://www.nytimes.com/spotlight/at-home}{At Home}

\href{https://myaccount.nytimes.com/auth/login?response_type=cookie\&client_id=vi}{}

\href{https://www.nytimes.com/section/todayspaper}{Today's Paper}

\href{/spotlight/at-home}{At Home}\textbar{}Shed Your Quarantine Skin
(and Hair and Nails), Safely

\url{https://nyti.ms/3fYmgWY}

\begin{itemize}
\item
\item
\item
\item
\item
\end{itemize}

\href{https://www.nytimes.com/spotlight/at-home?action=click\&pgtype=Article\&state=default\&region=TOP_BANNER\&context=at_home_menu}{At
Home}

\begin{itemize}
\tightlist
\item
  \href{https://www.nytimes.com/2020/07/28/books/time-for-a-literary-road-trip.html?action=click\&pgtype=Article\&state=default\&region=TOP_BANNER\&context=at_home_menu}{Take:
  A Literary Road Trip}
\item
  \href{https://www.nytimes.com/2020/07/29/magazine/bored-with-your-home-cooking-some-smoky-eggplant-will-fix-that.html?action=click\&pgtype=Article\&state=default\&region=TOP_BANNER\&context=at_home_menu}{Cook:
  Smoky Eggplant}
\item
  \href{https://www.nytimes.com/2020/07/27/travel/moose-michigan-isle-royale.html?action=click\&pgtype=Article\&state=default\&region=TOP_BANNER\&context=at_home_menu}{Look
  Out: For Moose}
\item
  \href{https://www.nytimes.com/interactive/2020/at-home/even-more-reporters-editors-diaries-lists-recommendations.html?action=click\&pgtype=Article\&state=default\&region=TOP_BANNER\&context=at_home_menu}{Explore:
  Reporters' Obsessions}
\end{itemize}

Advertisement

\protect\hyperlink{after-top}{Continue reading the main story}

Supported by

\protect\hyperlink{after-sponsor}{Continue reading the main story}

\hypertarget{shed-your-quarantine-skin-and-hair-and-nails-safely}{%
\section{Shed Your Quarantine Skin (and Hair and Nails),
Safely}\label{shed-your-quarantine-skin-and-hair-and-nails-safely}}

If you venture out for a treatment or trim, you might encounter
plexiglass barriers, tons of cleaning supplies, fewer clients at a time
and higher prices.

\includegraphics{https://static01.nyt.com/images/2020/07/26/multimedia/26ah-beauty/26ah-beauty-articleLarge.jpg?quality=75\&auto=webp\&disable=upscale}

By \href{https://www.nytimes.com/by/katherine-cusumano}{Katherine
Cusumano}

\begin{itemize}
\item
  July 25, 2020
\item
  \begin{itemize}
  \item
  \item
  \item
  \item
  \item
  \end{itemize}
\end{itemize}

As areas around the country reopen following months of lockdowns, people
are emerging from their homes with shaggy hair, translucent skin and
claw-like fingernails --- and then rushing to make self-care
appointments. ``The bathhouse has been booked out every day that we've
been open,'' said Andrew Nehlig, the owner of Sauna House, in Asheville,
N.C., which reopened last month.

You might be overdue for some general maintenance, or perhaps you need
to undo some
\href{https://www.nytimes.com/2020/04/02/t-magazine/home-hair-care-tips-coronavirus.html}{do-it-yourself}
quarantine beauty treatments that went awry. Or you might want to catch
up with your stylist or technician. ``There's sort of a shrink
relationship,'' said Jane Hong, the chief executive of the Manhattan
nail salon and retailer Paintbox. ``This is why we're here on earth, not
to live in isolation but to help one another, support one another and
speak to one another.'' Perhaps your look affirms your very sense of
self: Khane Kutzwell, whose Brooklyn barber shop, Camera Ready Kutz,
\href{https://www.nytimes.com/2020/03/11/nyregion/nyc-queer-black-barbershops.html}{primarily}
serves the L.G.B.T.Q. community, noted that hair styling ``is a
super-duper big thing'' for some of her queer and transgender clients.

Should you decide to venture out, you may be wondering how to stay safer
during a haircut, wax or manicure. ``It's reducing the risk, not
eliminating the risk,'' said Wafaa El-Sadr, a professor of epidemiology
and medicine at Columbia University's Mailman School of Public Health.
``Nobody can tell you it's safe to do x, y and z. We can make x, y and z
as safe as possible.''

\hypertarget{check-the-regulations-in-your-area}{%
\subsection{Check the regulations in your
area.}\label{check-the-regulations-in-your-area}}

Local safety guidelines, as well as measures adopted by salons and spas,
can help mitigate your chance of contracting or spreading Covid-19. In
some cities, certain services, like facials and facial waxing, are
\href{https://www.governor.ny.gov/sites/governor.ny.gov/files/atoms/files/Personal_Care_Summary_Guidelines.pdf}{unavailable}.
Communal facilities --- like steam rooms, saunas and baths --- might be
\href{https://www.northjersey.com/story/news/coronavirus/2020/06/12/nj-reopening-plan-spas-tanning-salons-can-reopen-june-22-murphy-says/3175833001/}{closed},
or their capacities dramatically reduced. Plexiglass
\href{https://www.cdc.gov/coronavirus/2019-ncov/community/organizations/nail-salon-employers.html}{barriers}
may separate you from the receptionist, people in neighboring chairs and
even your manicurist; ventilation systems may pump filtered air into the
room. The amenities you're used to --- a cup of tea or snacks --- may
have disappeared. There will be cleaning supplies everywhere. (``We
could kill pretty much anything that lives,'' said Gabrielle Ophals, a
co-founder of the Manhattan spa Haven.) And everyone will be wearing a
mask.

The first thing to ask yourself, according to Celine Gounder, an
infectious disease expert and former assistant health commissioner for
New York City, is whether there's still widespread community
transmission in your area. If there is, she said, ``then I think as with
anything, whether it's school reopenings or nonessential services ---
and to me, this is a nonessential service --- those need to be shut down
until you can get your community transmission under control.'' Some
areas that forged ahead with reopenings
\href{https://www.nytimes.com/interactive/2020/us/states-reopen-map-coronavirus.html}{are}
pausing or even rolling back those plans; in Los Angeles County,
personal care facilities that began operating again in mid-June are
\href{http://www.ph.lacounty.gov/media/Coronavirus/docs/protocols/Reopening_PersonalCare.pdf}{now}
limited to treating clients outdoors.

\hypertarget{evaluate-the-risk-to-yourself-and-others}{%
\subsection{Evaluate the risk to yourself and
others.}\label{evaluate-the-risk-to-yourself-and-others}}

If a treatment is available to you, reflect on how essential it is, and
how much you'll be exposing yourself if you receive it. ``Step back one
step and think, `Do I need to do this or do I want to do it?''' Dr.
El-Sadr said. **** When it comes to assessing the relative risk of
different appointments, she explained, there are two primary variables:
the duration of your treatment and your distance from the person
providing your service. Time can vary widely (think of a polish change
compared with a full Mani-Pedi), and you should try to limit the length
of your appointment, but one thing is constant: It's nearly impossible
to maintain a six-foot distance. Consider, too, how many other people
will be in the room; a private therapy is inevitably safer than one that
places you in a room with several other people.

The risk also increases for you --- and the person providing your
service --- if you have to remove your mask or they have to touch your
face. So you should forego lip and eyebrow waxes or threading, facials
and professional makeup artistry. And Shari Lipner, a dermatologist at
Weill-Cornell Medical Center who specializes in nail disorders,
recommends skipping the cuticle trim that usually accompanies a
manicure, since cuticles help seal off the nail beds from
microorganisms.

\hypertarget{book-an-appointment}{%
\subsection{Book an appointment.}\label{book-an-appointment}}

In the past, you may have dipped out during your lunch break for a
manicure or bikini wax, but, these days, there's comfort to be found in
planning ahead. Plus, some salons and spas aren't yet taking walk-in
clients and will turn you away unless they have a stylist, technician or
massage therapist available at that exact moment.

Before the pandemic, Sauna House's clientele primarily consisted of
walk-ins; now, it operates entirely by appointment. ``We created a whole
new business,'' Mr. Nehlig said. He reduced the capacity of the
bathhouse from 28 people to eight, established a time limit inside and
increased the time between massage appointments from 15 minutes to 30
minutes. These measures help maintain social distancing and allow a
buffer period for sanitizing.

As a result, bookings are both more necessary --- and harder to come by.
You can also find out if the facility takes private appointments, or
whether your stylist is making house calls.

\hypertarget{review-the-businesss-guidelines}{%
\subsection{Review the business's
guidelines.}\label{review-the-businesss-guidelines}}

Start by checking the salon's or spa's website, which may outline its
rules and regulations. This makes the process safer and more seamless,
and reduces frustration on both ends. Plan for the waiting room to be
closed, and some
\href{https://www.nytimes.com/2020/06/24/style/coronavirus-public-bathrooms.html}{restrooms}
might be restricted to employee use, so think twice about guzzling a
liter of water before you arrive.

If you have questions, reach out by phone or email to clarify the
procedures. Dr. El-Sadr recommends making sure that the shop is
operating at a reduced capacity, with ample space among chairs, and that
everyone is wearing a mask. You may also be asked to complete a health
screening online or over the phone before your visit and to have your
temperature taken upon arrival.

``The most important thing is, if you have any symptoms, don't go,'' Dr.
Gounder said. ``Be honest about it.'' Some businesses have waived
cancellation fees to encourage you to stay home if you feel the
slightest bit under the weather.

\hypertarget{bring-your-own-supplies-especially-your-mask}{%
\subsection{Bring your own supplies, especially your
mask.}\label{bring-your-own-supplies-especially-your-mask}}

Carry hand sanitizer and disinfectant wipes. Some nail salons
\href{https://www.glamour.com/story/nail-salon-coronavirus-safety}{might
ask} that you bring your own tools, if you have them --- a practice that
Dr. Lipner also suggests. (Discard or sanitize them after your
manicure.) Most of all, don't forget your
\href{https://www.nytimes.com/2020/05/22/at-home/best-face-masks-fashion-coronavirus.html}{mask},
and consider bringing an extra in case it gets wet or dirty during the
course of your treatment. If you're going for a haircut, ensure you have
a well-fitting mask that goes behind your ears, rather than around the
back of your head.

Wearing a mask is, as Dr. Gounder put it, ``the No. 1, 2 and 3 most
important thing that a client can be doing'' to reduce the risk of
contracting or spreading Covid-19. A
\href{https://www.cdc.gov/mmwr/volumes/69/wr/mm6928e2.htm}{recent,
widely cited report} by the Centers for Disease Control and Prevention
underscored this: In Springfield, Mo., two hair stylists continued
coming to work, days after they began to feel sick. Together, they
exposed 139 people to the virus --- none of whom reported symptoms in
the two weeks after. Around 98 percent of the clients interviewed by the
C.D.C. said they were wearing a mask during their appointments.

\hypertarget{be-patient--and-tip-handsomely}{%
\subsection{Be patient --- and tip
handsomely.}\label{be-patient--and-tip-handsomely}}

Stylists, barbers, estheticians and other personal care professionals
are on the front lines of the pandemic. They work face-to-face with ---
and, often, within a couple inches of --- their clients. ``The vast
majority of people working in these industries are women, and many of
them women of color,'' Dr. Gounder said. (Her new podcast, ``Epidemic,''
recently aired an episode on the effect of the pandemic among beauty
professionals.) ``So that definitely contributes to exacerbation of the
disparities we've seen in terms of who's affected by Covid.''

The economic toll of the coronavirus shutdowns has been especially steep
among the small businesses that closed from March into June and July. So
as shops reopen, don't be surprised to see higher prices, and be
prepared to tip to excess.

As everyone gets accustomed to this new normal, try compassion. (``We're
nervous, too!'' read the website of one Austin nail salon before
reopening in June.) And if you remain uncomfortable with going out for
the sake of your hair, nails or skin, there are other ways to support
your friends in the beauty industry. Shoot them a Venmo or buy a gift
card for a future appointment --- and
\href{https://www.nytimes.com/2020/04/23/style/self-care/quarantine-cut-your-own-bangs-coronavirus.html}{trim
your own bangs}.

Advertisement

\protect\hyperlink{after-bottom}{Continue reading the main story}

\hypertarget{site-index}{%
\subsection{Site Index}\label{site-index}}

\hypertarget{site-information-navigation}{%
\subsection{Site Information
Navigation}\label{site-information-navigation}}

\begin{itemize}
\tightlist
\item
  \href{https://help.nytimes.com/hc/en-us/articles/115014792127-Copyright-notice}{©~2020~The
  New York Times Company}
\end{itemize}

\begin{itemize}
\tightlist
\item
  \href{https://www.nytco.com/}{NYTCo}
\item
  \href{https://help.nytimes.com/hc/en-us/articles/115015385887-Contact-Us}{Contact
  Us}
\item
  \href{https://www.nytco.com/careers/}{Work with us}
\item
  \href{https://nytmediakit.com/}{Advertise}
\item
  \href{http://www.tbrandstudio.com/}{T Brand Studio}
\item
  \href{https://www.nytimes.com/privacy/cookie-policy\#how-do-i-manage-trackers}{Your
  Ad Choices}
\item
  \href{https://www.nytimes.com/privacy}{Privacy}
\item
  \href{https://help.nytimes.com/hc/en-us/articles/115014893428-Terms-of-service}{Terms
  of Service}
\item
  \href{https://help.nytimes.com/hc/en-us/articles/115014893968-Terms-of-sale}{Terms
  of Sale}
\item
  \href{https://spiderbites.nytimes.com}{Site Map}
\item
  \href{https://help.nytimes.com/hc/en-us}{Help}
\item
  \href{https://www.nytimes.com/subscription?campaignId=37WXW}{Subscriptions}
\end{itemize}
