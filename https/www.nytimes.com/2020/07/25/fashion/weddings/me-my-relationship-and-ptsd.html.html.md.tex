Sections

SEARCH

\protect\hyperlink{site-content}{Skip to
content}\protect\hyperlink{site-index}{Skip to site index}

\href{https://www.nytimes.com/section/fashion/weddings}{Love}

\href{https://myaccount.nytimes.com/auth/login?response_type=cookie\&client_id=vi}{}

\href{https://www.nytimes.com/section/todayspaper}{Today's Paper}

\href{/section/fashion/weddings}{Love}\textbar{}Me, My Relationship and
PTSD

\url{https://nyti.ms/32QJTgk}

\begin{itemize}
\item
\item
\item
\item
\item
\end{itemize}

Advertisement

\protect\hyperlink{after-top}{Continue reading the main story}

Supported by

\protect\hyperlink{after-sponsor}{Continue reading the main story}

First Person

\hypertarget{me-my-relationship-and-ptsd}{%
\section{Me, My Relationship and
PTSD}\label{me-my-relationship-and-ptsd}}

Past and present find their place as a couple wades through what their
future holds. And it's all OK.

\includegraphics{https://static01.nyt.com/images/2020/07/10/fashion/00NOT-ENGAGED/00NOT-ENGAGED-articleLarge.jpg?quality=75\&auto=webp\&disable=upscale}

By Caira Conner

\begin{itemize}
\item
  July 25, 2020
\item
  \begin{itemize}
  \item
  \item
  \item
  \item
  \item
  \end{itemize}
\end{itemize}

Sam and I began the conversation partly in jest. His co-worker had just
eloped in Hawaii, and as we scrolled through their photos I gave him an
elbow to the ribs and said in a singsong voice, ``Well, maybe we should
go to Hawaii, too!''

Later we spoke about it in more thoughtful tones, and as it turned out,
neither of us had ever been and we both had always wanted to go to
Hawaii. I raised my eyebrows and widened my grin. ``I think we should do
it.'' ``Not because it's time to get married,'' I added, loudly. ``But
because we have the perfect opportunity to do it.''

It's true, we had extra vacation days, and I was a full-time freelancer
with an unexpected financial reserve. How much more serendipitous could
the circumstances be?

Before long, we had two airline tickets to fly into Kauai on Jan. 1, and
a foolproof excuse to get out of the dreaded ``What are you doing for
News Year Eve?'' repertoire. My eyes lit up when I told people about the
plans we'd made, but I didn't tell them the whole story. We weren't
going to get married on this trip. We weren't going to get engaged,
either.

Sam and I had been set up by mutual work friends who felt our mid-30s
skepticism, love of tennis and sensitive digestive systems were a match
made in heaven. Six months after our first date, I moved into his
apartment. But for the last eight before the Hawaii conversation, what
we had been dealing with behind the scenes of the natural rhythms of a
developing relationship was my post-traumatic stress disorder.

\emph{{[}}\href{https://www.nytimes.com/newsletters/love-letter?module=inline}{\emph{Sign
up for Love Letter and always get the latest in Modern Love, weddings,
and relationships in the news by email.}}\emph{{]}}

\hypertarget{a-postcard-from-the-past}{%
\subsubsection{A Postcard From the
Past}\label{a-postcard-from-the-past}}

My biological father returned to my life on an early spring day in 2019
as though he'd been there the entire time. He hadn't. One afternoon I
went to check my mail and there was a postcard with the words, ``Hello,
what's up, never hear from you'' printed on the back. I blinked and
looked closer, to make sure it could actually be addressed to me.

It had been 10 years since I'd seen him last, and 32 since the abuse had
taken place. I shoved the postcard back into the mailbox and went
outside. I walked fast and hard up the sidewalk, wondering if all the
terrible things I knew to be true somehow didn't exist anymore --- if
the passage of time had voided them out entirely.

A week later, I received a phone call that he'd had a heart attack in
his backyard. The week following, he was dead.

The emotional breakdown that followed was a crumble at first. It was
little, confused pieces every day, and huge, gasping sobs every night.
It was four months into my relationship with Sam, and he spent his days
at work and his evenings holding my hand on the couch. He believed it
would pass. I wasn't so sure. I didn't understand how the death of
someone I didn't know could cause such intolerable internal pain.

``This isn't how I want to be, you know.''

``I know,'' he said.

Our one-year anniversary came and went quietly that November. I thought
about how PTSD had now been a part of our relationship for longer than
it had not. I thought about how nice it would be if our weeknights were
about lazing on the couch, talking through our days, or having an
occasional argument over the dishes. I wondered if my emotional
volatility would ever cool off. I wondered if our relationship could
make it through this stress.

\hypertarget{would-hawaii-be-my-getaway}{%
\subsubsection{Would Hawaii Be My
Getaway?}\label{would-hawaii-be-my-getaway}}

On Jan. 1, 2020, we touched down at Lihue International Airport with jet
lag and weary happiness. It was the first day of a new year and what
better time to leave the past behind? My thoughts chirped along as we
walked to baggage claim, the warmth and relief sweeping in like a wave.
``This feels like where I need to be right now,'' I thought. We joined
throngs of sunburned tourists waiting for happy hour to start at a
restaurant patio, which was open on the holiday. A waitress appeared at
our table just in time with a basket of warm, salty edamame. We drank
chilled seltzer with bitters out of ruby red tumblers, and smiled at
each other across the table.

``We made it,'' I said.

On our fourth morning of the trip, I looked at Sam across the backyard
patio table and burst into tears.

``I wish we were planning a wedding, or thinking about having a baby. I
wish that's where we were instead,'' I sobbed. The words came out faster
than I could breathe. ``I hate that I'm still here, after a year,
reading books about complex PTSD and ruining our lives.''

I'd been officially diagnosed by the psychiatrist I started seeing after
the death of my father. The term ``complex'' gave definition to the
feeling-states I now experienced out of context, outside of the time
frame in which they first occurred, decades earlier. She explained the
waves of sadness that rose every day, like the tide, were emotional
flashbacks. They rode in with a sense of despair, and utter
hopelessness, and sometimes it felt like I was drowning in their
aftermath.

It was sunrise in Kauai, but I was frozen in time in 2019, standing in
the hallway of my apartment building, holding a postcard with the
delicacy of a hand grenade. I was on the couch in a child psychiatrist's
office in 1989, being forced to talk about something that happened that
I wasn't ready to share.

``I thought I would feel better here, you know? Because at least it
would be in Hawaii.''

For as many times as I'd reminded myself this trip wasn't about riding
off into the sunset, I was still surprised that the bellyache of
sadness, with its deep, wrenching grip, had come with me.

``Nothing has been ruined. This is just where we are now, and it's OK,''
he said.

We sat quietly and after a while, a chicken wandered in to peck at the
grass in the backyard we shared with the neighbors. The whistle of a
cardinal sounded in the distance. I took a few deep breaths as the
stillness crept in.

``At least the weather here is nicer,'' I said, and my face cleared. Sam
smiled. He still believed it would pass.

We returned home to our shared apartment in Brooklyn, and I began
research on other types of therapy that would help with trauma
processing.

It's where I was, for now.

\emph{Continue following our fashion and lifestyle coverage on Facebook
(}\href{https://www.facebook.com/nytimesstyles}{\emph{Styles}}
\emph{and} \href{https://www.facebook.com/modernlove}{\emph{Modern
Love}}\emph{), Twitter
(}\href{https://twitter.com/nytstyles}{\emph{Styles}}\emph{,}
\href{https://twitter.com/nytfashion}{\emph{Fashion}} \emph{and}
\href{https://twitter.com/nytimesvows}{\emph{Weddings}}\emph{) and}
\href{https://instagram.com/nytimesfashion}{\emph{Instagram}}\emph{.}

Advertisement

\protect\hyperlink{after-bottom}{Continue reading the main story}

\hypertarget{site-index}{%
\subsection{Site Index}\label{site-index}}

\hypertarget{site-information-navigation}{%
\subsection{Site Information
Navigation}\label{site-information-navigation}}

\begin{itemize}
\tightlist
\item
  \href{https://help.nytimes.com/hc/en-us/articles/115014792127-Copyright-notice}{©~2020~The
  New York Times Company}
\end{itemize}

\begin{itemize}
\tightlist
\item
  \href{https://www.nytco.com/}{NYTCo}
\item
  \href{https://help.nytimes.com/hc/en-us/articles/115015385887-Contact-Us}{Contact
  Us}
\item
  \href{https://www.nytco.com/careers/}{Work with us}
\item
  \href{https://nytmediakit.com/}{Advertise}
\item
  \href{http://www.tbrandstudio.com/}{T Brand Studio}
\item
  \href{https://www.nytimes.com/privacy/cookie-policy\#how-do-i-manage-trackers}{Your
  Ad Choices}
\item
  \href{https://www.nytimes.com/privacy}{Privacy}
\item
  \href{https://help.nytimes.com/hc/en-us/articles/115014893428-Terms-of-service}{Terms
  of Service}
\item
  \href{https://help.nytimes.com/hc/en-us/articles/115014893968-Terms-of-sale}{Terms
  of Sale}
\item
  \href{https://spiderbites.nytimes.com}{Site Map}
\item
  \href{https://help.nytimes.com/hc/en-us}{Help}
\item
  \href{https://www.nytimes.com/subscription?campaignId=37WXW}{Subscriptions}
\end{itemize}
