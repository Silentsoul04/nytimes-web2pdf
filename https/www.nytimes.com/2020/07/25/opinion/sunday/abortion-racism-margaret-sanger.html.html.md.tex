Sections

SEARCH

\protect\hyperlink{site-content}{Skip to
content}\protect\hyperlink{site-index}{Skip to site index}

\href{https://www.nytimes.com/section/opinion/sunday}{Sunday Review}

\href{https://myaccount.nytimes.com/auth/login?response_type=cookie\&client_id=vi}{}

\href{https://www.nytimes.com/section/todayspaper}{Today's Paper}

\href{/section/opinion/sunday}{Sunday Review}\textbar{}The Ghost of
Margaret Sanger

\href{https://nyti.ms/32WuCuA}{https://nyti.ms/32WuCuA}

\begin{itemize}
\item
\item
\item
\item
\item
\item
\end{itemize}

Advertisement

\protect\hyperlink{after-top}{Continue reading the main story}

\href{/section/opinion}{Opinion}

Supported by

\protect\hyperlink{after-sponsor}{Continue reading the main story}

\hypertarget{the-ghost-of-margaret-sanger}{%
\section{The Ghost of Margaret
Sanger}\label{the-ghost-of-margaret-sanger}}

Planned Parenthood and the complexities of anti-racism.

\href{https://www.nytimes.com/by/ross-douthat}{\includegraphics{https://static01.nyt.com/images/2018/04/03/opinion/ross-douthat/ross-douthat-thumbLarge.png}}

By \href{https://www.nytimes.com/by/ross-douthat}{Ross Douthat}

Opinion Columnist

\begin{itemize}
\item
  July 25, 2020
\item
  \begin{itemize}
  \item
  \item
  \item
  \item
  \item
  \item
  \end{itemize}
\end{itemize}

\includegraphics{https://static01.nyt.com/images/2020/07/26/opinion/26douthat1/merlin_174917709_758689de-793c-4864-baff-5c962a0ed7f5-articleLarge.jpg?quality=75\&auto=webp\&disable=upscale}

This week, Planned Parenthood of Greater New York announced that
\href{https://www.cnn.com/2020/07/22/us/margaret-sanger-planned-parenthood-trnd/index.html}{it
would remove Margaret Sanger's name} from its Manhattan Health Center.
The grounds were Sanger's eugenic ideas and alliances, which for years
have been highlighted by anti-abortion advocates and minimized by her
admirers. Under the pressures of the current moment, apparently, that
minimization isn't sustainable any more.

This is an interesting shift from just a year ago, when Clarence Thomas
faced a wave of
\href{https://www.washingtonpost.com/history/2019/05/31/clarence-thomas-tried-link-abortion-eugenics-seven-historians-told-post-hes-wrong/?utm_term=.0c5106b1bcd7}{media
scorn} when he took note of Sanger's eugenic sympathies. But Thomas was
citing Sanger's writings to suggest that abortion in America
\emph{today} reflects a kind of structural racism --- an inherited
tendency, which persists even without racist intent, for pro-abortion
policies to reduce minority births more than white births. Whereas the
removal of Sanger's name, presumably, was intended to drive home the
opposite point --- to establish a clear separation between past and
present, between racism then and abortion rights today.

But the difficulty is that according to current thinking on how
structural racism lingers and what anti-racism requires, Thomas still
seems to have a reasonable case.

That thinking emphasizes, first, the persistent influence of
formerly-institutionalized racism even in the absence of conscious
racists, and second, the importance of assessing every policy based on
its effects on racial equality. ``There is no such thing as a nonracist
or race-neutral policy,'' writes the best-selling theorist Ibram X.
Kendi. ``Every policy in every institution in every community in every
nation is producing or sustaining either racial inequity or equity.''

Now apply these frameworks to the history of Planned Parenthood. The
organization had eugenic ideas close to its root, and while Sanger
herself was pro-contraception rather than pro-abortion, her successors
championed both abortion rights and
\href{https://www.hup.harvard.edu/catalog.php?isbn=9780674034600}{global
population control policies} that were racist by any reasonable
definition.

Then when abortion was legalized in the United States, with Planned
Parenthood's strong support, its initial effect was a sharp decline in
minority births. According to the Wellesley economist Phillip Levine,
white births dipped only slightly after legalization, while the nonwhite
birthrate dropped by 15 percent. Fifty years later, the abortion rate is
\href{https://www.theatlantic.com/health/archive/2014/09/abortions-racial-gap/380251/}{five
times} higher for African-Americans than for whites.

So in this story, a worldview with racist antecedents wins a major
policy victory that immediately has a disproportionate effect on
minority birthrates. And then there is the further twist that over the
longer run, Roe v. Wade and the sexual revolution probably changed
family structure as well, as George Akerlof and (future Fed chair) Janet
Yellen
\href{https://www.brookings.edu/research/an-analysis-of-out-of-wedlock-births-in-the-united-states/}{argued
in a 1996 paper}, by creating a wider space for men to expect sex
without commitment and to behave irresponsibly toward pregnant woman:
``By making the birth of the child the physical choice of the mother,''
they wrote, ``the sexual revolution has made marriage and child support
a social choice of the father.''

Like the abortion rate itself, this trend --- the long rise of
fatherlessness --- has been steeper in poor and vulnerable communities.
So it, too, has helped to sustain racial inequality, by reserving to
\href{https://douthat.blogs.nytimes.com/2014/01/29/social-liberalism-as-class-warfare/}{the
whiter upper classes} the socioeconomic advantages that two-parent
families enjoy.

Keep following this logic, and you might conclude that if Planned
Parenthood really took anti-racism seriously it would repent of its
support for abortion, and devote itself exclusively to helping support
African-American pregnancies instead.

Are you convinced? I expect not. Maybe you think the decline of the
two-parent family is strictly about de-industrialization. Maybe you
believe the benefits of abortion access for minority women outweigh
whatever power lower birthrates cost the African-American community writ
large.

Maybe you think the nuclear family was itself a form of white or Western
oppression, and any anti-racism that requires its revival isn't worthy
of the name. (This appears to be the position of the official
\href{https://blacklivesmatter.com/what-we-believe/}{Black Lives
Matters} organization.) Or maybe you simply think abortion is an
absolute human right, which must be defended even if, as policy, it
appears to have a disparate racial impact.

Each of these claims could spin out another column in response. For now,
I just want the skeptical reader to consider, through the case of
Planned Parenthood's history and abortion's social consequences, just
how complicated the questions opened up by concepts like structural
racism and the racism/anti-racism binary can become.

Followed rigorously to their conclusions, they may lead to surprising or
inconvenient ideological conclusions, to intersectional dilemmas no
doctrine can resolve, or just to a deep uncertainty about the best path
to racial redress.

Or they might even lead to a creeping sense that Clarence Thomas has a
point: that at the very moment that America finally granted
African-Americans full citizenship, it also embarked on a separate
social revolution, whose most ruthless feature --- the belief that
equality and liberty require removing protections from unborn human life
--- left a specific stamp on the African-American experience, just as
the most ruthless features of our history always do.

\emph{The Times is committed to publishing}
\href{https://www.nytimes.com/2019/01/31/opinion/letters/letters-to-editor-new-york-times-women.html}{\emph{a
diversity of letters}} \emph{to the editor. We'd like to hear what you
think about this or any of our articles. Here are some}
\href{https://help.nytimes.com/hc/en-us/articles/115014925288-How-to-submit-a-letter-to-the-editor}{\emph{tips}}\emph{.
And here's our email:}
\href{mailto:letters@nytimes.com}{\emph{letters@nytimes.com}}\emph{.}

\emph{Follow The New York Times Opinion section on}
\href{https://www.facebook.com/nytopinion}{\emph{Facebook}}\emph{,}
\href{http://twitter.com/NYTOpinion}{\emph{Twitter (@NYTOpinion)}}
\emph{and}
\href{https://www.instagram.com/nytopinion/}{\emph{Instagram}}\emph{,
join the Facebook political discussion group,}
\href{https://www.facebook.com/groups/votingwhilefemale/}{\emph{Voting
While Female}}\emph{.}

Advertisement

\protect\hyperlink{after-bottom}{Continue reading the main story}

\hypertarget{site-index}{%
\subsection{Site Index}\label{site-index}}

\hypertarget{site-information-navigation}{%
\subsection{Site Information
Navigation}\label{site-information-navigation}}

\begin{itemize}
\tightlist
\item
  \href{https://help.nytimes.com/hc/en-us/articles/115014792127-Copyright-notice}{©~2020~The
  New York Times Company}
\end{itemize}

\begin{itemize}
\tightlist
\item
  \href{https://www.nytco.com/}{NYTCo}
\item
  \href{https://help.nytimes.com/hc/en-us/articles/115015385887-Contact-Us}{Contact
  Us}
\item
  \href{https://www.nytco.com/careers/}{Work with us}
\item
  \href{https://nytmediakit.com/}{Advertise}
\item
  \href{http://www.tbrandstudio.com/}{T Brand Studio}
\item
  \href{https://www.nytimes.com/privacy/cookie-policy\#how-do-i-manage-trackers}{Your
  Ad Choices}
\item
  \href{https://www.nytimes.com/privacy}{Privacy}
\item
  \href{https://help.nytimes.com/hc/en-us/articles/115014893428-Terms-of-service}{Terms
  of Service}
\item
  \href{https://help.nytimes.com/hc/en-us/articles/115014893968-Terms-of-sale}{Terms
  of Sale}
\item
  \href{https://spiderbites.nytimes.com}{Site Map}
\item
  \href{https://help.nytimes.com/hc/en-us}{Help}
\item
  \href{https://www.nytimes.com/subscription?campaignId=37WXW}{Subscriptions}
\end{itemize}
