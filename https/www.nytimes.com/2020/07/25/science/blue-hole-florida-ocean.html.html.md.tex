Sections

SEARCH

\protect\hyperlink{site-content}{Skip to
content}\protect\hyperlink{site-index}{Skip to site index}

\href{https://www.nytimes.com/section/science}{Science}

\href{https://myaccount.nytimes.com/auth/login?response_type=cookie\&client_id=vi}{}

\href{https://www.nytimes.com/section/todayspaper}{Today's Paper}

\href{/section/science}{Science}\textbar{}What's in This Deep `Blue
Hole' Off Florida? They're Working on It

\url{https://nyti.ms/3fUuYFz}

\begin{itemize}
\item
\item
\item
\item
\item
\item
\end{itemize}

Advertisement

\protect\hyperlink{after-top}{Continue reading the main story}

Supported by

\protect\hyperlink{after-sponsor}{Continue reading the main story}

\hypertarget{whats-in-this-deep-blue-hole-off-florida-theyre-working-on-it}{%
\section{What's in This Deep `Blue Hole' Off Florida? They're Working on
It}\label{whats-in-this-deep-blue-hole-off-florida-theyre-working-on-it}}

There are hundreds, perhaps thousands, of such holes on the ocean floor.
Scientists are planning a mission to the one they call the ``Green
Banana.''

\includegraphics{https://static01.nyt.com/images/2020/07/25/autossell/25xp-bluehole-image/25xp-bluehole-image-videoSixteenByNineJumbo1600.png}

By \href{https://www.nytimes.com/by/heather-murphy}{Heather Murphy}

\begin{itemize}
\item
  Published July 25, 2020Updated Aug. 3, 2020
\item
  \begin{itemize}
  \item
  \item
  \item
  \item
  \item
  \item
  \end{itemize}
\end{itemize}

Sprinkled across the ocean floor, invisible from the surface, are
hundreds --- or maybe thousands --- of sink holes. These ``blue holes,''
as scientists call them, do not swallow up everything incapable of
fighting their gravitational force, like their
\href{https://www.nytimes.com/2019/04/10/science/what-is-black-hole.html}{black
hole cousins}. But to those who study them, they are still nearly as
intriguing.

This week, one particular blue hole --- the Green Banana --- has
captured the imagination of many a land dweller. Headline after headline
has offered a
\href{https://www.google.com/search?client=firefox-b-1-d\&q=blue+hole}{variation
on the same} theme: Scientists are
\href{https://abcnews.go.com/US/scientists-investigate-blue-oceanic-mystery-off-florida-coast/story?id=71919756}{flocking}
to a mysterious blue hole.
\href{https://www.sciencetimes.com/articles/26577/20200723/fishermen-divers-mysterious-blue-hole-florida.htm}{One
publication} asked:``What Could It Be?''

What it \emph{is} is the Green Banana, one of the deepest blue holes
ever discovered, according to Jim Culter, a senior scientist at Mote
Marine Laboratory, and it's on the verge of being studied in the most
comprehensive way yet.

Scientists will venture into the Green Banana's depths next month, where
they hope to answer longstanding questions about whether the sink hole
--- which extends around 275 feet, like an inverted, hourglass-shaped
20-story building, anchored in the ocean floor --- connects to other
sink holes and whether freshwater flows within.

The scientists leading the mission to the sink hole, which begins 155
feet below the ocean's surface around 50 miles offshore from St.
Petersburg, agree that the name, the Green Banana, sounds like it should
be a bar in Key West. According to Larry Borden, a longtime commercial
fisherman and boat captain who has known about the Green Banana for
decades, the name emerged in the mid 1970s after a boat captain saw a
green banana skin floating by a known ``spring,'' as fishermen referred
to the underwater sink holes back then.

They were called springs, Mr. Borden said he had heard, because in the
1530s, when the Spanish explorer Hernando de Soto was hanging out in the
area, there was fresh water streaming out of the holes. Spear fishermen
who gathered near the holes centuries later, including Mr. Borden,
wondered how deep they were.

Eventually, Mr. Borden told a diver friend, Curt Bowen, about the Green
Banana, and in 1993, Mr. Bowen became one of the first people to dive to
the bottom and to map the blue hole. An
\href{http://www.advanceddivermagazine.com/ADMEZINE/GreenBanana.pdf}{article
in Advanced Diver Magazine}, which Mr. Bowen owns, posited that there
were so many sink holes on the floor of that Gulf of Mexico that if it
were possible to drain it, ``it would probably look like Swiss cheese.''

\includegraphics{https://static01.nyt.com/images/2020/07/25/multimedia/25xp-bluehole-pix2/merlin_174913872_8e0d8ae9-ac6c-4914-8e9d-e06bfb04befc-articleLarge.jpg?quality=75\&auto=webp\&disable=upscale}

Nearly 30 years later, scientists still don't know just how porous it
is. Mr. Culter said that scientists have verified about 20 underwater
sink holes on the West coast of Florida alone, but there are probably
twice that number. Part of what makes them hard to count is also what
makes them so intriguing, said Emily Hall, a scientist at Mote Marine
Laboratory leading the mission: They are difficult to spot from above.

``You're in the middle of the Gulf of Mexico and you don't see anything
all around,'' Dr. Hall said. And then after diving for quite some time,
``This hole opens up, and it's booming with life.''

There seems to be something about the unusual seawater chemistry in blue
holes that is particularly good at facilitating life. Pools of fish,
oodles of sponges and an array of plants are common within these
``oases,'' as Dr. Hall calls them. The water inside is also often
\href{https://www.nytimes.com/2019/11/01/science/blue-holes-hurricanes.html}{atypically
clear}, which is part of why they are beloved by divers.

One reason that so little is known about them, according to the
\href{https://oceanexplorer.noaa.gov/explorations/20blue-holes/welcome.html}{National
Oceanic and Atmospheric Administration}, is that their entry points are
often narrow --- before they broaden out --- making it impossible for an
automated submersible to enter.

In next month's mission, which NOAA is funding, the plan is to carefully
lower a 600-pound lander inside. Together the lander, which is shaped
like a triangular prism, and divers will collect water and sediment
samples and complete a biological survey, Dr. Hall said.

``The excitement comes from the idea that this is exploration --- we
don't know what we will see down there biologically and chemically,''
she said. ``We have an idea. But every time we go down there we find
something new.''

The team recently explored a nearby blue hole, around 350 feet deep,
known as ``Amberjack.'' They were surprised to discover two dead
smalltooth sawfish, an endangered species, at the bottom.

Advertisement

\protect\hyperlink{after-bottom}{Continue reading the main story}

\hypertarget{site-index}{%
\subsection{Site Index}\label{site-index}}

\hypertarget{site-information-navigation}{%
\subsection{Site Information
Navigation}\label{site-information-navigation}}

\begin{itemize}
\tightlist
\item
  \href{https://help.nytimes.com/hc/en-us/articles/115014792127-Copyright-notice}{©~2020~The
  New York Times Company}
\end{itemize}

\begin{itemize}
\tightlist
\item
  \href{https://www.nytco.com/}{NYTCo}
\item
  \href{https://help.nytimes.com/hc/en-us/articles/115015385887-Contact-Us}{Contact
  Us}
\item
  \href{https://www.nytco.com/careers/}{Work with us}
\item
  \href{https://nytmediakit.com/}{Advertise}
\item
  \href{http://www.tbrandstudio.com/}{T Brand Studio}
\item
  \href{https://www.nytimes.com/privacy/cookie-policy\#how-do-i-manage-trackers}{Your
  Ad Choices}
\item
  \href{https://www.nytimes.com/privacy}{Privacy}
\item
  \href{https://help.nytimes.com/hc/en-us/articles/115014893428-Terms-of-service}{Terms
  of Service}
\item
  \href{https://help.nytimes.com/hc/en-us/articles/115014893968-Terms-of-sale}{Terms
  of Sale}
\item
  \href{https://spiderbites.nytimes.com}{Site Map}
\item
  \href{https://help.nytimes.com/hc/en-us}{Help}
\item
  \href{https://www.nytimes.com/subscription?campaignId=37WXW}{Subscriptions}
\end{itemize}
