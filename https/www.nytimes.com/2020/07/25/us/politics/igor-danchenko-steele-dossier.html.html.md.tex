Sections

SEARCH

\protect\hyperlink{site-content}{Skip to
content}\protect\hyperlink{site-index}{Skip to site index}

\href{https://www.nytimes.com/section/politics}{Politics}

\href{https://myaccount.nytimes.com/auth/login?response_type=cookie\&client_id=vi}{}

\href{https://www.nytimes.com/section/todayspaper}{Today's Paper}

\href{/section/politics}{Politics}\textbar{}The F.B.I. Pledged to Keep a
Source Anonymous. Trump Allies Aided His Unmasking.

\url{https://nyti.ms/2WU7EjD}

\begin{itemize}
\item
\item
\item
\item
\item
\end{itemize}

Advertisement

\protect\hyperlink{after-top}{Continue reading the main story}

Supported by

\protect\hyperlink{after-sponsor}{Continue reading the main story}

\hypertarget{the-fbi-pledged-to-keep-a-source-anonymous-trump-allies-aided-his-unmasking}{%
\section{The F.B.I. Pledged to Keep a Source Anonymous. Trump Allies
Aided His
Unmasking.}\label{the-fbi-pledged-to-keep-a-source-anonymous-trump-allies-aided-his-unmasking}}

After a Russia expert who had collected research on Donald Trump for a
disputed dossier agreed to tell the F.B.I. what he knew about it, law
enforcement officials declassified a road map to identifying him.

\includegraphics{https://static01.nyt.com/images/2020/07/23/us/politics/23dc-fbi1/merlin_174843396_504052f0-7a46-4160-a442-09f3ac47eb82-articleLarge.jpg?quality=75\&auto=webp\&disable=upscale}

\href{https://www.nytimes.com/by/adam-goldman}{\includegraphics{https://static01.nyt.com/images/2018/07/12/multimedia/author-adam-goldman/author-adam-goldman-thumbLarge.png}}\href{https://www.nytimes.com/by/charlie-savage}{\includegraphics{https://static01.nyt.com/images/2018/06/12/multimedia/author-charlie-savage/author-charlie-savage-thumbLarge-v2.png}}

By \href{https://www.nytimes.com/by/adam-goldman}{Adam Goldman} and
\href{https://www.nytimes.com/by/charlie-savage}{Charlie Savage}

\begin{itemize}
\item
  July 25, 2020
\item
  \begin{itemize}
  \item
  \item
  \item
  \item
  \item
  \end{itemize}
\end{itemize}

WASHINGTON --- Not long after
\href{https://www.nytimes.com/2017/01/11/us/politics/donald-trump-russia-intelligence.html}{the
early 2017 publication of a notorious dossier about President Trump}
jolted Washington, an expert in Russian politics told the F.B.I. he had
been one of its key sources, drawing on his contacts to deliver
information that would make up some of the most salacious and unproven
assertions in the document.

The F.B.I. had approached the expert, a man named Igor Danchenko, as it
vetted the dossier's claims. He agreed to tell investigators what he
knew with an important condition, people familiar with the matter said
--- that the F.B.I. keep his identity secret so he could protect
himself, his sources and his family and friends in Russia.

But his hope of remaining anonymous evaporated last week after Attorney
General William P. Barr directed the F.B.I. to declassify
\href{https://www.judiciary.senate.gov/imo/media/doc/February\%209,\%202017\%20Electronic\%20Communication.pdf}{a
redacted report about its three-day interview of Mr. Danchenko in 2017}
and hand it over to Senator Lindsey Graham, Republican of South Carolina
and chairman of the Senate Judiciary Committee. Mr. Graham promptly
\href{https://www.judiciary.senate.gov/press/rep/releases/judiciary-committee-releases-declassified-documents-that-substantially-undercut-steele-dossier-page-fisa-warrants}{made
the interview summary public while calling the entire Russia
investigation ``corrupt.''}

The report blacked out Mr. Danchenko's name and other identifying
information. But within two days, a post on a newly created blog
\href{https://ifoundthepss.blogspot.com/}{entitled ``I Found the Primary
Subsource''} identified him, citing clues left visible in the F.B.I.
document. A \href{https://twitter.com/Hmmm57474203}{pseudonymous Twitter
account} created in May then promoted the existence of the blog. And the
next day, RT, the Kremlin-owned, English-language news and propaganda
outlet,
\href{https://www.rt.com/usa/495342-russiagate-steele-dossier-source/}{published
an article amplifying Mr. Danchenko's identification}.

The decision by Justice Department and F.B.I. leaders to divulge such a
report was highly unusual and created the risk it would help identify a
person who had confidentially provided information to agents, even if
officials did not intend to provide such a road map. The move comes at a
time when Mr. Barr, who is to testify before lawmakers on Tuesday, has
repeatedly been accused of abusing his powers to help Mr. Trump
politically.

Former law enforcement officials said the outing will make it harder for
F.B.I. agents to gain the trust of people they need to cooperate in
future and unrelated investigations.

``These things have to remain very closely held because you put
witnesses at risk,'' said James W. McJunkin, a former F.B.I. assistant
director for counterterrorism. ``To release sensitive information
unnecessarily that could jeopardize someone's life is egregious.''

A lawyer for Mr. Danchenko, Mark E. Schamel, said that because his
client's name had already been exposed, he would not ask The New York
Times to withhold it. He acknowledged that ``Igor Danchenko has been
identified as one of the sources who provided data and analysis'' to
Christopher Steele, the British former spy who compiled the dossier and
whose last name has become shorthand for it.

Mr. Danchenko's identity is noteworthy because it further calls into
question the credibility of the dossier. By turning to Mr. Danchenko as
his primary source to gather possible dirt on Mr. Trump involving
Russia, Mr. Steele was relying not on someone with a history of working
with Russian intelligence operatives or bringing to light their covert
activities but instead a researcher focused on analyzing business and
political risks in Russia.

Spokespeople at both the F.B.I. and the Justice Department declined to
comment. An email sent to an address listed on the blog was not
returned.

Mr. Trump's supporters on Capitol Hill have long sought access to
Justice Department and F.B.I. documents about the Russia investigation.
The F.B.I. director, Christopher A. Wray, told lawmakers
\href{https://hankjohnson.house.gov/media-center/press-releases/rep-johnson-questions-fbi-director-judiciary-hearing}{in
late 2017} that the bureau was wary of turning over records related to
its effort to verify the Steele dossier to Congress. ``We are dealing
with very, very dicey questions of sources and methods, which is the
lifeblood of foreign intelligence and our liaison relationships with our
foreign partners,'' he said.

But since his confirmation early last year, Mr. Barr and other Trump
appointees have approved a wave of extraordinary declassifications that
the president's allies, including Mr. Graham, have used to attack the
Russia inquiry.

Mr. Graham said he had asked the F.B.I. to declassify the interview
report after it was described in an inspector general report last year
because he wanted the public to read it. He stressed that he did not
know the identity of Mr. Steele's source and said he did not know
whether the F.B.I. released identifying information it should have
protected, saying the bureau had appeared to be ``painstaking'' in
redacting such details.

``I don't know how he was exposed,'' Mr. Graham said in an interview on
Friday. ``I didn't see anything in the memo exposing who he was. I mean,
you can believe these websites if you want to --- I don't know. I know
this: It's important for the country to understand what happened here.''

In addition to their political implications, the documents have at times
revealed the closely held secrets that Mr. Wray feared jeopardizing:
sources of information and the methods used for gathering it.

\includegraphics{https://static01.nyt.com/images/2020/07/23/us/politics/23dc-fbi2/merlin_172896705_fda5852f-6152-44af-82c0-9d2eb7449d06-articleLarge.jpg?quality=75\&auto=webp\&disable=upscale}

\href{https://www.judiciary.senate.gov/imo/media/doc/2020-04-24\%20Submission\%20SJC\%20SSCI.pdf}{Transcripts
of recordings} released in April resulted in
\href{https://dailycaller.com/2020/05/06/george-papadopoulos-fbi-informant-transcript/}{the
identification of a confidential F.B.I. informant} who had agree to wear
a wire when talking to George Papadopoulos, a former Trump adviser who
was convicted of lying to the F.B.I. Other
\href{https://int.nyt.com/data/documenthelper/6976-flynn-kislyak-transcripts/cd9e96e708a9b0c8ba58/optimized/full.pdf\#page=1}{released
transcripts} of a Russian diplomat's conversations with former national
security adviser Michael T. Flynn
\href{https://www.nytimes.com/2020/05/29/us/politics/flynn-russian-ambassador-transcripts.html}{revealed
that the bureau was able to monitor the phone line}of the Russian
Embassy in Washington even before a call connected with Mr. Flynn's
voice mail.

The unmaskings from the release of the F.B.I. report have already
spiraled beyond Mr. Danchenko. Building on the knowledge of his
identity, another Twitter user
\href{https://twitter.com/FOOL_NELSON/status/1285347075048251392}{named}
a likely source for Mr. Danchenko. Online sleuths were trying to
identify others from his network who were cited but not named in the
Steele dossier.

The release of Mr. Danchenko's interview summary likely put him and
other sources in Russia's sights, said Senator Mark Warner of Virginia,
the top Democrat on the Senate Intelligence Committee.

``Under Attorney General Barr, the levers of the Department of Justice
continue to be weaponized in defense of the president's political
agenda, even at the expense of national security,'' said Mr. Warner, who
did not confirm that Mr. Danchenko was Mr. Steele's primary source or
discuss his committee's own investigation into Russian election
interference. ``I'm deeply concerned by this release. There is no doubt
that the Russians are poring over it to see if they can identify this
individual or other sources.''

Mr. Danchenko also cooperated with the intelligence committee on
condition of confidentiality, according to two people familiar with its
investigation.

Some posts on the blog that revealed Mr. Danchenko's name are dated
before Mr. Graham released the interview report, but the Twitter user
who promoted the blog said he or she had backdated the posts to change
their order.

Born in Ukraine, Mr. Danchenko, 42, is a Russian-trained lawyer who
earned degrees at the University of Louisville and Georgetown
University, according to LinkedIn. He was a senior research analyst from
2005 to 2010 at the Brookings Institution, where he co-wrote a research
paper showing that, as a student, President Vladimir V. Putin of Russia
\href{https://www.washingtonpost.com/news/answer-sheet/wp/2014/03/18/russias-plagiarism-problem-even-putin-has-done-it/}{appeared
to have plagiarized part of his dissertation}.

According to his interview with the F.B.I., Mr. Steele contacted Mr.
Danchenko around March 2016 and assigned him to ask people he knew in
Russia and Ukraine about connections, including any ties to corruption,
between a pro-Russian government in Ukraine and the veteran Republican
strategist Paul Manafort. Mr. Steele did not explain why, but Mr.
Manafort joined the Trump campaign around that time and was later
promoted to its chairman. He
\href{https://www.nytimes.com/2019/03/13/us/politics/paul-manafort-sentencing.html}{was
convicted} in 2018 of tax and bank fraud and other charges that grew out
of the Russia investigation.

Mr. Steele later expanded Mr. Danchenko's assignment to look for any
compromising information about Mr. Trump.

By Jan. 13, 2017, the F.B.I. had identified Mr. Danchenko, who soon
agreed to answer investigators' questions in exchange
\href{https://www.justice.gov/archives/jm/criminal-resource-manual-719-informal-immunity-distinguished-formal-immunity}{for
immunity.}

The F.B.I. told a court it found Mr. Danchenko ``truthful and
cooperative,'' according to the report by the Justice Department
inspector general, Michael E. Horowitz, although a supervisory F.B.I.
intelligence analyst said Mr. Danchenko may have minimized aspects of
what he told Mr. Steele.

Mr. Graham said he wanted the public to be able to see for itself how
the interview report ``clearly shows that the dossier was not reliable
and they continued to use it anyway.''

Mr. Danchenko did nothing wrong in accepting a paid assignment to gather
allegations about Mr. Trump's ties to Russia and conveying them to Mr.
Steele's research firm, Orbis Business Intelligence, said Mr. Schamel,
who attended his client's F.B.I. debriefings but whose name was redacted
from the report about them.

``Mr. Danchenko is a highly respected senior research analyst; he is
neither an author nor editor for any of the final reports produced by
Orbis,'' Mr. Schamel said. ``Mr. Danchenko stands by his data analysis
and research and will leave it to others to evaluate and interpret any
broader story with regard to Orbis's final report.''

The Steele dossier was deeply flawed. For example, it included a claim
that Mr. Trump's former lawyer Michael D. Cohen had met with a Russian
intelligence officer in Prague to discuss collusion with the campaign.
The report by the special counsel who took over the Russia
investigation, Robert S. Mueller III,
\href{https://www.nytimes.com/interactive/2019/04/18/us/politics/mueller-report-document.html\#g-page-351}{found
that Mr. Cohen never traveled to Prague}.

And Mr. Danchenko's statements to the F.B.I. contradicted parts of the
dossier, suggesting that Mr. Steele may have exaggerated the soundness
of other allegations, making what Mr. Danchenko portrayed as rumor and
speculation sound more solid.

The Steele dossier played no role in the F.B.I.'s opening of the Russia
investigation in July 2016, and Mr. Mueller did not rely on it for his
report.

But its flaws have taken on outsized political significance, as Mr.
Trump's allies have sought to conflate it with the larger effort to
understand Russia's covert efforts to tilt the 2016 election in his
favor and whether any Trump campaign associates conspired in that
effort. Mr. Mueller laid out extensive details about Russia's covert
operation and contacts with Trump campaign associates, but found
insufficient evidence to bring any conspiracy charges.

The dossier did play an important role in a narrow part of the F.B.I.'s
early Russia investigation: the wiretapping of Carter Page, a former
Trump campaign adviser with close ties to Russian officials, which began
in October 2016 and was extended three times in 2017. The Justice
Department's applications for court orders authorizing the wiretap
relied in part on information from the dossier in making the case that
investigators had reason to believe that Mr. Page might be working with
Russians.

Mr. Page was never charged, and Mr. Mueller's report only briefly
discussed him. Mr. Horowitz scathingly portrayed the wiretap
applications as riddled with errors and omissions.

Mr. Danchenko provided information to Mr. Steele that figured into one
of the biggest flaws with those applications. Mr. Horowitz first brought
to public light that when the F.B.I. interviewed Mr. Steele's primary
source --- who turned out to be Mr. Danchenko --- his account was
inconsistent with important aspects of the dossier.

But law enforcement officials recycled the same language derived from
the dossier in their final two applications for court orders to continue
wiretapping Mr. Page. They also told a court they had spoken to Mr.
Steele's primary source but without revealing that his statements raised
questions about the dossier's credibility, which Mr. Horowitz said was
misleading.

After the inspector general report, the F.B.I. conceded to the court
that it should not have sought the last two renewals.

The disclosure of Mr. Danchenko's identity --- which the inspector
general report concealed --- also brought into focus another
questionable statement in the wiretap applications. Mr. Horowitz wrote
that the last two applications described Mr. Steele's source as
``Russian-based.'' Though Mr. Danchenko visited Moscow while gathering
information for Mr. Steele, he lives in the United States.

A criminal prosecutor appointed by Mr. Barr to scrutinize the Russia
investigation, John H. Durham, the U.S. attorney in Connecticut, has
also focused on the dossier and asked questions about Mr. Danchenko,
according to people familiar with aspects of his inquiry. Mr. Schamel
said he had not been contacted by Mr. Durham or his investigators.

Nicholas Fandos contributed reporting.

Advertisement

\protect\hyperlink{after-bottom}{Continue reading the main story}

\hypertarget{site-index}{%
\subsection{Site Index}\label{site-index}}

\hypertarget{site-information-navigation}{%
\subsection{Site Information
Navigation}\label{site-information-navigation}}

\begin{itemize}
\tightlist
\item
  \href{https://help.nytimes.com/hc/en-us/articles/115014792127-Copyright-notice}{©~2020~The
  New York Times Company}
\end{itemize}

\begin{itemize}
\tightlist
\item
  \href{https://www.nytco.com/}{NYTCo}
\item
  \href{https://help.nytimes.com/hc/en-us/articles/115015385887-Contact-Us}{Contact
  Us}
\item
  \href{https://www.nytco.com/careers/}{Work with us}
\item
  \href{https://nytmediakit.com/}{Advertise}
\item
  \href{http://www.tbrandstudio.com/}{T Brand Studio}
\item
  \href{https://www.nytimes.com/privacy/cookie-policy\#how-do-i-manage-trackers}{Your
  Ad Choices}
\item
  \href{https://www.nytimes.com/privacy}{Privacy}
\item
  \href{https://help.nytimes.com/hc/en-us/articles/115014893428-Terms-of-service}{Terms
  of Service}
\item
  \href{https://help.nytimes.com/hc/en-us/articles/115014893968-Terms-of-sale}{Terms
  of Sale}
\item
  \href{https://spiderbites.nytimes.com}{Site Map}
\item
  \href{https://help.nytimes.com/hc/en-us}{Help}
\item
  \href{https://www.nytimes.com/subscription?campaignId=37WXW}{Subscriptions}
\end{itemize}
