Sections

SEARCH

\protect\hyperlink{site-content}{Skip to
content}\protect\hyperlink{site-index}{Skip to site index}

\href{https://www.nytimes.com/section/world/asia}{Asia Pacific}

\href{https://myaccount.nytimes.com/auth/login?response_type=cookie\&client_id=vi}{}

\href{https://www.nytimes.com/section/todayspaper}{Today's Paper}

\href{/section/world/asia}{Asia Pacific}\textbar{}These Monkeys Were
Once Revered. Now They Are Taking Over.

\url{https://nyti.ms/3hMkU27}

\begin{itemize}
\item
\item
\item
\item
\item
\end{itemize}

\href{https://www.nytimes.com/news-event/coronavirus?action=click\&pgtype=Article\&state=default\&region=TOP_BANNER\&context=storylines_menu}{The
Coronavirus Outbreak}

\begin{itemize}
\tightlist
\item
  live\href{https://www.nytimes.com/2020/08/03/world/coronavirus-covid-19.html?action=click\&pgtype=Article\&state=default\&region=TOP_BANNER\&context=storylines_menu}{Latest
  Updates}
\item
  \href{https://www.nytimes.com/interactive/2020/us/coronavirus-us-cases.html?action=click\&pgtype=Article\&state=default\&region=TOP_BANNER\&context=storylines_menu}{Maps
  and Cases}
\item
  \href{https://www.nytimes.com/interactive/2020/science/coronavirus-vaccine-tracker.html?action=click\&pgtype=Article\&state=default\&region=TOP_BANNER\&context=storylines_menu}{Vaccine
  Tracker}
\item
  \href{https://www.nytimes.com/2020/08/02/us/covid-college-reopening.html?action=click\&pgtype=Article\&state=default\&region=TOP_BANNER\&context=storylines_menu}{College
  Reopening}
\item
  \href{https://www.nytimes.com/live/2020/08/03/business/stock-market-today-coronavirus?action=click\&pgtype=Article\&state=default\&region=TOP_BANNER\&context=storylines_menu}{Economy}
\end{itemize}

Advertisement

\protect\hyperlink{after-top}{Continue reading the main story}

Supported by

\protect\hyperlink{after-sponsor}{Continue reading the main story}

Thailand Dispatch

\hypertarget{these-monkeys-were-once-revered-now-they-are-taking-over}{%
\section{These Monkeys Were Once Revered. Now They Are Taking
Over.}\label{these-monkeys-were-once-revered-now-they-are-taking-over}}

The monkeys of Lopburi, Thailand, were once a draw for tourists and
pilgrims who would feed them. But with few recent visitors, the monkeys
are getting hungry --- and aggressive.

\includegraphics{https://static01.nyt.com/images/2020/07/26/world/26monkeys-dispatch3/merlin_174060759_d9359faa-e0dc-459a-8585-a098d9018353-articleLarge.jpg?quality=75\&auto=webp\&disable=upscale}

\href{https://www.nytimes.com/by/hannah-beech}{\includegraphics{https://static01.nyt.com/images/2018/10/08/multimedia/author-hannah-beech/author-hannah-beech-thumbLarge.png}}

By \href{https://www.nytimes.com/by/hannah-beech}{Hannah Beech}

\begin{itemize}
\item
  July 25, 2020
\item
  \begin{itemize}
  \item
  \item
  \item
  \item
  \item
  \end{itemize}
\end{itemize}

LOPBURI, Thailand --- The customers waiting outside a bank in Lopburi,
Thailand, left their jewelry at home and kept other treasures out of
sight. But danger lurked anyway.

In broad daylight, they watched a thief steal an iced tea and a vandal
brazenly attack a motorcycle seat. One woman quit her place in the line,
when a stalker crept up and threatened to bite her.

With a sigh, a police officer brandished a slingshot, and the monkeys
scattered. Less than a minute later, they were back.

Lopburi, a onetime capital of a Siamese kingdom and a repository of
ancient architecture, is a city under siege. Crab-eating macaques, a
Southeast Asian species with piercing eyes and curious natures, have
spilled out of the temples where they were once revered and taken over
the heart of the old town.

Their growing population, at least 8,400 in the area with most
concentrated in a few city blocks, has decimated parts of the local
economy. With territorial troupes of macaques roaming the neighborhood,
dozens of businesses --- including a music school, gold shop, barber,
cellphone store and movie theater --- have been forced to close in
recent years.

The coronavirus pandemic has added to the chaos. The frolicking monkeys
drew droves of tourists as well as Buddhist faithful, who believe
feeding the animals is a meritorious deed. Their favorite offerings
included coconut yogurt, strawberry soda and brightly colored snack
packs. Now the macaques don't understand where that source of sustenance
has gone. And they are hungry.

\includegraphics{https://static01.nyt.com/images/2020/07/26/world/26monkeys-dispatch1/merlin_174060687_ac420efb-4bb8-4e0c-9ebe-727d5f66ca2e-articleLarge.jpg?quality=75\&auto=webp\&disable=upscale}

Image

Stuffed crocodile toys scared the monkeys off for a little while, but
they quickly caught on.Credit...Adam Dean for The New York Times

Over the years, the monkeys moved into abandoned buildings, trashing
display cases and rattling the bars installed to keep them out. Unless
security guards are vigilant, they rip antennas and windshield wipers
off parked cars.

Dangling earrings, sunglasses and plastic bags that look like they may
have food in them are irresistible to the monkeys. And in the areas of
the city most densely packed with the animals, many residents live in
fear of the next sneak attack.

\hypertarget{latest-updates-global-coronavirus-outbreak}{%
\section{\texorpdfstring{\href{https://www.nytimes.com/2020/08/03/world/coronavirus-covid-19.html?action=click\&pgtype=Article\&state=default\&region=MAIN_CONTENT_1\&context=storylines_live_updates}{Latest
Updates: Global Coronavirus
Outbreak}}{Latest Updates: Global Coronavirus Outbreak}}\label{latest-updates-global-coronavirus-outbreak}}

Updated 2020-08-04T05:55:16.339Z

\begin{itemize}
\tightlist
\item
  \href{https://www.nytimes.com/2020/08/03/world/coronavirus-covid-19.html?action=click\&pgtype=Article\&state=default\&region=MAIN_CONTENT_1\&context=storylines_live_updates\#link-4547638f}{Fauci
  defends Birx after she is criticized by Trump.}
\item
  \href{https://www.nytimes.com/2020/08/03/world/coronavirus-covid-19.html?action=click\&pgtype=Article\&state=default\&region=MAIN_CONTENT_1\&context=storylines_live_updates\#link-15e7f995}{Trump
  derides Democrats as lawmakers and administration officials try to
  break stimulus impasse.}
\item
  \href{https://www.nytimes.com/2020/08/03/world/coronavirus-covid-19.html?action=click\&pgtype=Article\&state=default\&region=MAIN_CONTENT_1\&context=storylines_live_updates\#link-e5a2cda}{The
  deadline for 2020 census counting has been moved up by a month.}
\end{itemize}

\href{https://www.nytimes.com/2020/08/03/world/coronavirus-covid-19.html?action=click\&pgtype=Article\&state=default\&region=MAIN_CONTENT_1\&context=storylines_live_updates}{See
more updates}

More live coverage:
\href{https://www.nytimes.com/live/2020/08/03/business/stock-market-today-coronavirus?action=click\&pgtype=Article\&state=default\&region=MAIN_CONTENT_1\&context=storylines_live_updates}{Markets}

But in a Buddhist-majority culture in which culling monkeys would
disturb spiritual sensibilities, local officials and residents have few
options to fend off the gangs of macaques. Besides, in the past, the
monkeys drew tourists to Lopburi. Without them, the economy might suffer
even more.

At a hardware store across the street from the ruins of a 13th-century
Hindu temple, oversized stuffed animals in the shape of crocodiles and
tigers peer out at the street where the monkey traffic outpaces that of
pedestrians. The plush toys were meant to scare away the monkeys and it
worked for a couple months. But the macaques soon figured out that they
weren't real, said Yupa Srisanguan, the shop owner.

``It has never been this bad,'' Ms. Yupa said, as a young male macaque
wandered into her store, intent on chewing the loops of rubber hose
hanging from the ceiling. ``We're not against the monkeys, but it's
difficult when people are afraid of being bitten when they come to our
store.''

Image

Culling the monkeys would disturb local traditions, so officials and
residents have few options to fend off the gangs of
macaques.Credit...Adam Dean for The New York Times

Image

Compared to the monkeys of the forest, their urban counterparts have
less muscle and are more susceptible to hypertension.Credit...Adam Dean
for The New York Times

When she was a little girl, Ms. Yupa, 70, said, the monkeys were fewer,
bigger and healthier, their fur shiny and thick. They kept to the
temples, as well as the ruins of the ancient Khmer civilization that
once held sway over this part of central Thailand.

But with an influx of monkey-enchanted visitors, some foreign, came an
easy and often unhealthy font of food. Along with bananas and citrus,
the macaques feasted on junk. Their fur thinned. Some went bald. Without
having to worry about their next meal, the monkeys, which can give birth
twice a year, had more time for other pursuits. The population exploded.

Compared with the monkeys of the forest, their urban counterparts have
less muscle and are more susceptible to hypertension and blood disease,
said Narongporn Doodduem, the director of a regional office of the
Wildlife Conservation Department.

``The monkeys are never hungry,'' he said, ``just like children who eat
too much KFC.''

As traffic recently piled up at a light in old town Lopburi, Nirad
Pholngeun, a police officer, kept his slingshot at the ready. He has
been stationed at this street corner for five years and has watched the
growing monkey population with alarm.

A truck, presumably from out of town, idled at the light, its flatbed
filled with crates of fruit for the market. A monkey spotted the
produce, wove through the traffic, leapt onto the truck and held aloft a
juicy dragon fruit. The one expeditionary macaque drew dozens more. By
the time the light turned green, the crates were cleared and the gorging
began.

Image

Enjoying a snack found in the back of a pickup truck.Credit...Adam Dean
for The New York Times

Image

Nirad Pholngeun, a police officer, miming using a slingshot to scare
away the monkeys.Credit...Adam Dean for The New York Times

Throughout the mayhem, Mr. Nirad raised his slingshot but there was
little a police officer could do against so many macaques. His battle
tactic was a charade anyway. The slingshot held no projectiles.

``It's hopeless,'' he said. ``Within a blink of an eye there are more
monkeys. So many babies.''

Local wildlife officials have begun sterilizing the monkeys en masse to
control their numbers. More than 300 animals underwent surgery last
month, and 200 more will be sterilized in August.

Capturing the monkeys for the operations is a major undertaking, said
Mr. Narongporn, the wildlife official. On the first day of the June
campaign, the monkey catchers wore camouflage-printed uniforms and lured
the animals into cages with food. But by the second day, the monkeys
knew to avoid them. The monkey catchers had to switch to wearing shorts
and floral shirts, pretending they were holidaymakers.

\href{https://www.nytimes.com/news-event/coronavirus?action=click\&pgtype=Article\&state=default\&region=MAIN_CONTENT_3\&context=storylines_faq}{}

\hypertarget{the-coronavirus-outbreak-}{%
\subsubsection{The Coronavirus Outbreak
›}\label{the-coronavirus-outbreak-}}

\hypertarget{frequently-asked-questions}{%
\paragraph{Frequently Asked
Questions}\label{frequently-asked-questions}}

Updated August 3, 2020

\begin{itemize}
\item ~
  \hypertarget{im-a-small-business-owner-can-i-get-relief}{%
  \paragraph{I'm a small-business owner. Can I get
  relief?}\label{im-a-small-business-owner-can-i-get-relief}}

  \begin{itemize}
  \tightlist
  \item
    The
    \href{https://www.nytimes.com/article/small-business-loans-stimulus-grants-freelancers-coronavirus.html?action=click\&pgtype=Article\&state=default\&region=MAIN_CONTENT_3\&context=storylines_faq}{stimulus
    bills enacted in March} offer help for the millions of American
    small businesses. Those eligible for aid are businesses and
    nonprofit organizations with fewer than 500 workers, including sole
    proprietorships, independent contractors and freelancers. Some
    larger companies in some industries are also eligible. The help
    being offered, which is being managed by the Small Business
    Administration, includes the Paycheck Protection Program and the
    Economic Injury Disaster Loan program. But lots of folks have
    \href{https://www.nytimes.com/interactive/2020/05/07/business/small-business-loans-coronavirus.html?action=click\&pgtype=Article\&state=default\&region=MAIN_CONTENT_3\&context=storylines_faq}{not
    yet seen payouts.} Even those who have received help are confused:
    The rules are draconian, and some are stuck sitting on
    \href{https://www.nytimes.com/2020/05/02/business/economy/loans-coronavirus-small-business.html?action=click\&pgtype=Article\&state=default\&region=MAIN_CONTENT_3\&context=storylines_faq}{money
    they don't know how to use.} Many small-business owners are getting
    less than they expected or
    \href{https://www.nytimes.com/2020/06/10/business/Small-business-loans-ppp.html?action=click\&pgtype=Article\&state=default\&region=MAIN_CONTENT_3\&context=storylines_faq}{not
    hearing anything at all.}
  \end{itemize}
\item ~
  \hypertarget{what-are-my-rights-if-i-am-worried-about-going-back-to-work}{%
  \paragraph{What are my rights if I am worried about going back to
  work?}\label{what-are-my-rights-if-i-am-worried-about-going-back-to-work}}

  \begin{itemize}
  \tightlist
  \item
    Employers have to provide
    \href{https://www.osha.gov/SLTC/covid-19/standards.html}{a safe
    workplace} with policies that protect everyone equally.
    \href{https://www.nytimes.com/article/coronavirus-money-unemployment.html?action=click\&pgtype=Article\&state=default\&region=MAIN_CONTENT_3\&context=storylines_faq}{And
    if one of your co-workers tests positive for the coronavirus, the
    C.D.C.} has said that
    \href{https://www.cdc.gov/coronavirus/2019-ncov/community/guidance-business-response.html}{employers
    should tell their employees} -\/- without giving you the sick
    employee's name -\/- that they may have been exposed to the virus.
  \end{itemize}
\item ~
  \hypertarget{should-i-refinance-my-mortgage}{%
  \paragraph{Should I refinance my
  mortgage?}\label{should-i-refinance-my-mortgage}}

  \begin{itemize}
  \tightlist
  \item
    \href{https://www.nytimes.com/article/coronavirus-money-unemployment.html?action=click\&pgtype=Article\&state=default\&region=MAIN_CONTENT_3\&context=storylines_faq}{It
    could be a good idea,} because mortgage rates have
    \href{https://www.nytimes.com/2020/07/16/business/mortgage-rates-below-3-percent.html?action=click\&pgtype=Article\&state=default\&region=MAIN_CONTENT_3\&context=storylines_faq}{never
    been lower.} Refinancing requests have pushed mortgage applications
    to some of the highest levels since 2008, so be prepared to get in
    line. But defaults are also up, so if you're thinking about buying a
    home, be aware that some lenders have tightened their standards.
  \end{itemize}
\item ~
  \hypertarget{what-is-school-going-to-look-like-in-september}{%
  \paragraph{What is school going to look like in
  September?}\label{what-is-school-going-to-look-like-in-september}}

  \begin{itemize}
  \tightlist
  \item
    It is unlikely that many schools will return to a normal schedule
    this fall, requiring the grind of
    \href{https://www.nytimes.com/2020/06/05/us/coronavirus-education-lost-learning.html?action=click\&pgtype=Article\&state=default\&region=MAIN_CONTENT_3\&context=storylines_faq}{online
    learning},
    \href{https://www.nytimes.com/2020/05/29/us/coronavirus-child-care-centers.html?action=click\&pgtype=Article\&state=default\&region=MAIN_CONTENT_3\&context=storylines_faq}{makeshift
    child care} and
    \href{https://www.nytimes.com/2020/06/03/business/economy/coronavirus-working-women.html?action=click\&pgtype=Article\&state=default\&region=MAIN_CONTENT_3\&context=storylines_faq}{stunted
    workdays} to continue. California's two largest public school
    districts --- Los Angeles and San Diego --- said on July 13, that
    \href{https://www.nytimes.com/2020/07/13/us/lausd-san-diego-school-reopening.html?action=click\&pgtype=Article\&state=default\&region=MAIN_CONTENT_3\&context=storylines_faq}{instruction
    will be remote-only in the fall}, citing concerns that surging
    coronavirus infections in their areas pose too dire a risk for
    students and teachers. Together, the two districts enroll some
    825,000 students. They are the largest in the country so far to
    abandon plans for even a partial physical return to classrooms when
    they reopen in August. For other districts, the solution won't be an
    all-or-nothing approach.
    \href{https://bioethics.jhu.edu/research-and-outreach/projects/eschool-initiative/school-policy-tracker/}{Many
    systems}, including the nation's largest, New York City, are
    devising
    \href{https://www.nytimes.com/2020/06/26/us/coronavirus-schools-reopen-fall.html?action=click\&pgtype=Article\&state=default\&region=MAIN_CONTENT_3\&context=storylines_faq}{hybrid
    plans} that involve spending some days in classrooms and other days
    online. There's no national policy on this yet, so check with your
    municipal school system regularly to see what is happening in your
    community.
  \end{itemize}
\item ~
  \hypertarget{is-the-coronavirus-airborne}{%
  \paragraph{Is the coronavirus
  airborne?}\label{is-the-coronavirus-airborne}}

  \begin{itemize}
  \tightlist
  \item
    The coronavirus
    \href{https://www.nytimes.com/2020/07/04/health/239-experts-with-one-big-claim-the-coronavirus-is-airborne.html?action=click\&pgtype=Article\&state=default\&region=MAIN_CONTENT_3\&context=storylines_faq}{can
    stay aloft for hours in tiny droplets in stagnant air}, infecting
    people as they inhale, mounting scientific evidence suggests. This
    risk is highest in crowded indoor spaces with poor ventilation, and
    may help explain super-spreading events reported in meatpacking
    plants, churches and restaurants.
    \href{https://www.nytimes.com/2020/07/06/health/coronavirus-airborne-aerosols.html?action=click\&pgtype=Article\&state=default\&region=MAIN_CONTENT_3\&context=storylines_faq}{It's
    unclear how often the virus is spread} via these tiny droplets, or
    aerosols, compared with larger droplets that are expelled when a
    sick person coughs or sneezes, or transmitted through contact with
    contaminated surfaces, said Linsey Marr, an aerosol expert at
    Virginia Tech. Aerosols are released even when a person without
    symptoms exhales, talks or sings, according to Dr. Marr and more
    than 200 other experts, who
    \href{https://academic.oup.com/cid/article/doi/10.1093/cid/ciaa939/5867798}{have
    outlined the evidence in an open letter to the World Health
    Organization}.
  \end{itemize}
\end{itemize}

``The monkeys are smart,'' Mr. Narongporn said. ``They remember.''

With the coronavirus dissuading many tourists and Buddhist pilgrims from
visiting Lopburi, local residents have taken to feeding the monkeys
themselves.

``We can't let them starve,'' said Itiphat Tansitikulphati, the owner of
the Muang Thong Hotel.

Image

Local wildlife officials have begun sterilizing the monkeys en masse to
control their numbers. Credit...Adam Dean for The New York Times

Image

With the coronavirus preventing many tourists and Buddhist pilgrims from
visiting Lopburi, local residents have taken to feeding the monkeys
themselves.Credit...Adam Dean for The New York Times

Every day, an old female monkey calls on his hotel, waiting politely for
her meal to be served. Banana cake is her favorite, but plain fruit will
do, too.

``A long time ago, a lot of Lopburi was forest so we are taking the land
from the monkeys,'' Mr. Itiphat said.

A third generation hotelier, he has relinquished the top floor to the
monkeys, who have wrecked it with the zeal of drunken partygoers,
ripping up wooden planks and shredding corrugated metal.

An electric fence protects the ground floor of the hotel. But even
before the coronavirus hit, visitors, many of whom were traveling
businesspeople, were scared off by the marauding monkeys, Mr. Itiphat
said. His hotel barely survives.

``The balance between humans and monkeys is off,'' he said. ``It hurts
business.''

Nearby, Patiphan Tantiwong runs a general store on the main street of
Lopburi. He has given in to the macaques. A plump male sat on bags of
dog food, sipping a yogurt drink. A clutch of youngsters swarmed a
counter waiting for handouts of cookies. There were monkeys among the
piles of batteries and monkeys among the light bulbs.

The babies squeaked and played a form of macaque peekaboo.

``The monkeys were here before us,'' Mr. Patiphan said, as a juvenile
tugged on the hem of his trousers demanding a treat. ``We have to adapt
to them, not the other way around.''

Image

At a general store on main street, the monkeys have practically taken
over.~Credit...Adam Dean for The New York Times

Muktita Suhartono contributed reporting.

Advertisement

\protect\hyperlink{after-bottom}{Continue reading the main story}

\hypertarget{site-index}{%
\subsection{Site Index}\label{site-index}}

\hypertarget{site-information-navigation}{%
\subsection{Site Information
Navigation}\label{site-information-navigation}}

\begin{itemize}
\tightlist
\item
  \href{https://help.nytimes.com/hc/en-us/articles/115014792127-Copyright-notice}{©~2020~The
  New York Times Company}
\end{itemize}

\begin{itemize}
\tightlist
\item
  \href{https://www.nytco.com/}{NYTCo}
\item
  \href{https://help.nytimes.com/hc/en-us/articles/115015385887-Contact-Us}{Contact
  Us}
\item
  \href{https://www.nytco.com/careers/}{Work with us}
\item
  \href{https://nytmediakit.com/}{Advertise}
\item
  \href{http://www.tbrandstudio.com/}{T Brand Studio}
\item
  \href{https://www.nytimes.com/privacy/cookie-policy\#how-do-i-manage-trackers}{Your
  Ad Choices}
\item
  \href{https://www.nytimes.com/privacy}{Privacy}
\item
  \href{https://help.nytimes.com/hc/en-us/articles/115014893428-Terms-of-service}{Terms
  of Service}
\item
  \href{https://help.nytimes.com/hc/en-us/articles/115014893968-Terms-of-sale}{Terms
  of Sale}
\item
  \href{https://spiderbites.nytimes.com}{Site Map}
\item
  \href{https://help.nytimes.com/hc/en-us}{Help}
\item
  \href{https://www.nytimes.com/subscription?campaignId=37WXW}{Subscriptions}
\end{itemize}
