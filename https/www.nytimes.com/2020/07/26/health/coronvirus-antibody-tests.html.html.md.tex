Sections

SEARCH

\protect\hyperlink{site-content}{Skip to
content}\protect\hyperlink{site-index}{Skip to site index}

\href{https://www.nytimes.com/section/health}{Health}

\href{https://myaccount.nytimes.com/auth/login?response_type=cookie\&client_id=vi}{}

\href{https://www.nytimes.com/section/todayspaper}{Today's Paper}

\href{/section/health}{Health}\textbar{}Your Coronavirus Antibodies Are
Disappearing. Should You Care?

\url{https://nyti.ms/30U14v0}

\begin{itemize}
\item
\item
\item
\item
\item
\item
\end{itemize}

\href{https://www.nytimes.com/news-event/coronavirus?action=click\&pgtype=Article\&state=default\&region=TOP_BANNER\&context=storylines_menu}{The
Coronavirus Outbreak}

\begin{itemize}
\tightlist
\item
  live\href{https://www.nytimes.com/2020/08/01/world/coronavirus-covid-19.html?action=click\&pgtype=Article\&state=default\&region=TOP_BANNER\&context=storylines_menu}{Latest
  Updates}
\item
  \href{https://www.nytimes.com/interactive/2020/us/coronavirus-us-cases.html?action=click\&pgtype=Article\&state=default\&region=TOP_BANNER\&context=storylines_menu}{Maps
  and Cases}
\item
  \href{https://www.nytimes.com/interactive/2020/science/coronavirus-vaccine-tracker.html?action=click\&pgtype=Article\&state=default\&region=TOP_BANNER\&context=storylines_menu}{Vaccine
  Tracker}
\item
  \href{https://www.nytimes.com/interactive/2020/07/29/us/schools-reopening-coronavirus.html?action=click\&pgtype=Article\&state=default\&region=TOP_BANNER\&context=storylines_menu}{What
  School May Look Like}
\item
  \href{https://www.nytimes.com/live/2020/07/31/business/stock-market-today-coronavirus?action=click\&pgtype=Article\&state=default\&region=TOP_BANNER\&context=storylines_menu}{Economy}
\end{itemize}

Advertisement

\protect\hyperlink{after-top}{Continue reading the main story}

Supported by

\protect\hyperlink{after-sponsor}{Continue reading the main story}

\hypertarget{your-coronavirus-antibodies-are-disappearing-should-you-care}{%
\section{Your Coronavirus Antibodies Are Disappearing. Should You
Care?}\label{your-coronavirus-antibodies-are-disappearing-should-you-care}}

Declining antibody levels do not mean less immunity, experts say.
Besides, two widely used tests may detect the wrong antibodies.

\includegraphics{https://static01.nyt.com/images/2020/07/26/us/politics/26virus-antibodies/merlin_174661485_e2c83e91-d733-4091-8cb8-ad86bbc8fd6a-articleLarge.jpg?quality=75\&auto=webp\&disable=upscale}

By \href{https://www.nytimes.com/by/apoorva-mandavilli}{Apoorva
Mandavilli}

\begin{itemize}
\item
  Published July 26, 2020Updated July 27, 2020
\item
  \begin{itemize}
  \item
  \item
  \item
  \item
  \item
  \item
  \end{itemize}
\end{itemize}

Your blood carries the memory of every pathogen you've ever encountered.
If you've been infected with the coronavirus, your body most likely
remembers that, too.

Antibodies are the legacy of that encounter. Why, then, have so many
people stricken by the virus discovered that they don't seem to have
antibodies?

Blame the tests.

Most commercial antibody tests offer crude yes-no answers. The tests are
\href{https://www.nytimes.com/2020/04/24/health/coronavirus-antibody-tests.html}{notorious
for delivering} false positives --- results indicating that someone has
antibodies when he or she does not.

But the volume of coronavirus antibodies drops sharply once the acute
illness ends. Now it is increasingly clear that these tests may also
produce false-negative results, missing antibodies to the coronavirus
that are present at low levels.

Moreover, some tests --- including those made by Abbott and Roche and
offered by Quest Diagnostics and LabCorp --- are designed to detect a
subtype of antibodies that doesn't confer immunity and may wane even
faster than the kind that can destroy the virus.

What that means is that declining antibodies, as shown by commercial
tests, don't necessarily mean declining immunity, several experts said.
Long-term surveys of antibodies, intended to assess how widely the
coronavirus has spread, may also underestimate the true prevalence.

``We're learning a lot about how antibodies change over time,'' said Dr.
Fiona Havers, a medical epidemiologist who has led such surveys for the
Centers for Disease Control and Prevention.

If the narrative on immunity to the coronavirus has seemed to shift
constantly, it's in part because the virus was a stranger to scientists.
But it's increasingly clear that this virus behaves much like any other.

This is how immunity to viruses generally works: The initial encounter
with a pathogen --- typically in childhood --- surprises the body. The
resulting illness can be mild or severe, depending on the dose of the
virus and the child's health, access to health care and genetics.

A mild illness may trigger production of only a few antibodies, and a
severe one many more. The vast majority of people who become infected
with the coronavirus have few to no symptoms, many experts believe, and
those people may produce a milder immune response.

\hypertarget{latest-updates-global-coronavirus-outbreak}{%
\section{\texorpdfstring{\href{https://www.nytimes.com/2020/08/01/world/coronavirus-covid-19.html?action=click\&pgtype=Article\&state=default\&region=MAIN_CONTENT_1\&context=storylines_live_updates}{Latest
Updates: Global Coronavirus
Outbreak}}{Latest Updates: Global Coronavirus Outbreak}}\label{latest-updates-global-coronavirus-outbreak}}

Updated 2020-08-02T06:58:18.835Z

\begin{itemize}
\tightlist
\item
  \href{https://www.nytimes.com/2020/08/01/world/coronavirus-covid-19.html?action=click\&pgtype=Article\&state=default\&region=MAIN_CONTENT_1\&context=storylines_live_updates\#link-34047410}{The
  U.S. reels as July cases more than double the total of any other
  month.}
\item
  \href{https://www.nytimes.com/2020/08/01/world/coronavirus-covid-19.html?action=click\&pgtype=Article\&state=default\&region=MAIN_CONTENT_1\&context=storylines_live_updates\#link-780ec966}{Top
  U.S. officials work to break an impasse over the federal jobless
  benefit.}
\item
  \href{https://www.nytimes.com/2020/08/01/world/coronavirus-covid-19.html?action=click\&pgtype=Article\&state=default\&region=MAIN_CONTENT_1\&context=storylines_live_updates\#link-2bc8948}{Its
  outbreak untamed, Melbourne goes into even greater lockdown.}
\end{itemize}

\href{https://www.nytimes.com/2020/08/01/world/coronavirus-covid-19.html?action=click\&pgtype=Article\&state=default\&region=MAIN_CONTENT_1\&context=storylines_live_updates}{See
more updates}

More live coverage:
\href{https://www.nytimes.com/live/2020/07/31/business/stock-market-today-coronavirus?action=click\&pgtype=Article\&state=default\&region=MAIN_CONTENT_1\&context=storylines_live_updates}{Markets}

But even a minor infection is often enough to teach the body to
recognize the intruder.

After the battle ends, balloon-like cells that live in the bone marrow
steadily pump out a small number of specialized assassins. The next time
--- and every time after that --- that the body comes across the virus,
those cells can mass-produce antibodies within hours.

The mnemonic response grows stronger with every encounter. It's one of
the great miracles of the human body.

``Whatever your level is today, if you get infected, your antibody
titers are going to go way up,'' said Dr. Michael Mina, an immunologist
at Harvard University, referring to the levels of antibodies in the
blood. ``The virus will never even have a chance the second time
around.''

A single drop of blood contains billions of antibodies, all lying in
wait for their specific targets. Sometimes, as may be the case for
antibodies to the coronavirus, there are too few to get a positive
signal on a test --- but that does not mean the person tested has no
immunity to the virus.

``Even if their antibodies wane below the limits of detection of our
instruments, it doesn't mean their `memory' is gone,'' Dr. Mina said.

A small number of people may not produce any antibodies to the
coronavirus. But even in that unlikely event, they will have so-called
cellular immunity, which includes T cells that learn to identify and
destroy the virus. Virtually everyone infected with the coronavirus
seems to develop T-cell responses, according to several recent studies.

``This means that even if the antibody titer is low, those people who
are previously infected may have a good enough T-cell response that can
provide protection,'' said Akiko Iwasaki, an immunologist at Yale
University.

T cells are harder to detect and to study, however, so when it comes to
immunity, antibodies have received all of the attention. The coronavirus
carries several antigens --- proteins or pieces of a protein --- that
can provoke the body into producing antibodies.

The most powerful antibodies recognize a piece of the coronavirus's
spike protein, the receptor binding domain, or R.B.D. That is the part
of the virus that docks onto human cells. Antibodies that recognize the
R.B.D. can neutralize the virus and prevent infection.

But the Roche and Abbott tests that are now widely available --- and
several others
\href{https://www.fda.gov/medical-devices/emergency-situations-medical-devices/eua-authorized-serology-test-performance}{authorized
by the Food and Drug Administration} --- instead look for antibodies to
a protein called the nucleocapsid, or N, that is bound up with the
virus's genetic material.

Some scientists were stunned to hear of this choice.

``God, I did not realize that --- that's crazy,'' said Angela Rasmussen,
a virologist at Columbia University in New York. ``It's kind of puzzling
to design a test that's not looking for what's thought to be the major
antigen.''

The N protein is plentiful in the blood, and testing for antibodies to
it produces a swifter, brighter signal than testing for antibodies to
the spike protein. Because antibody tests are used to detect past
infection, however, manufacturers are not required to prove that the
antibodies their tests seek are those that actually confer protection
against the virus.

Officials at the Food and Drug Administration did not respond to
requests for comment on whether the two tests target the appropriate
antibodies.

There's another wrinkle to the story. Some
\href{https://www.medrxiv.org/content/10.1101/2020.07.14.20153536v2?\%253fcollection=}{reports}
now suggest that
\href{https://www.nature.com/articles/s41591-020-0965-6}{antibodies
to}the viral nucleocapsid may
\href{https://www.medrxiv.org/content/10.1101/2020.07.16.20155663v2}{decline
faster} than those to R.B.D. or to the entire spike --- the really
effective ones.

``The majority of people are getting tested for anti-N antibody, which
does tend to wane more rapidly --- and so, you know, it may be not the
most suitable test for looking at neutralizing capacity,'' Dr. Iwasaki
said.

\href{https://www.nytimes.com/news-event/coronavirus?action=click\&pgtype=Article\&state=default\&region=MAIN_CONTENT_3\&context=storylines_faq}{}

\hypertarget{the-coronavirus-outbreak-}{%
\subsubsection{The Coronavirus Outbreak
›}\label{the-coronavirus-outbreak-}}

\hypertarget{frequently-asked-questions}{%
\paragraph{Frequently Asked
Questions}\label{frequently-asked-questions}}

Updated July 27, 2020

\begin{itemize}
\item ~
  \hypertarget{should-i-refinance-my-mortgage}{%
  \paragraph{Should I refinance my
  mortgage?}\label{should-i-refinance-my-mortgage}}

  \begin{itemize}
  \tightlist
  \item
    \href{https://www.nytimes.com/article/coronavirus-money-unemployment.html?action=click\&pgtype=Article\&state=default\&region=MAIN_CONTENT_3\&context=storylines_faq}{It
    could be a good idea,} because mortgage rates have
    \href{https://www.nytimes.com/2020/07/16/business/mortgage-rates-below-3-percent.html?action=click\&pgtype=Article\&state=default\&region=MAIN_CONTENT_3\&context=storylines_faq}{never
    been lower.} Refinancing requests have pushed mortgage applications
    to some of the highest levels since 2008, so be prepared to get in
    line. But defaults are also up, so if you're thinking about buying a
    home, be aware that some lenders have tightened their standards.
  \end{itemize}
\item ~
  \hypertarget{what-is-school-going-to-look-like-in-september}{%
  \paragraph{What is school going to look like in
  September?}\label{what-is-school-going-to-look-like-in-september}}

  \begin{itemize}
  \tightlist
  \item
    It is unlikely that many schools will return to a normal schedule
    this fall, requiring the grind of
    \href{https://www.nytimes.com/2020/06/05/us/coronavirus-education-lost-learning.html?action=click\&pgtype=Article\&state=default\&region=MAIN_CONTENT_3\&context=storylines_faq}{online
    learning},
    \href{https://www.nytimes.com/2020/05/29/us/coronavirus-child-care-centers.html?action=click\&pgtype=Article\&state=default\&region=MAIN_CONTENT_3\&context=storylines_faq}{makeshift
    child care} and
    \href{https://www.nytimes.com/2020/06/03/business/economy/coronavirus-working-women.html?action=click\&pgtype=Article\&state=default\&region=MAIN_CONTENT_3\&context=storylines_faq}{stunted
    workdays} to continue. California's two largest public school
    districts --- Los Angeles and San Diego --- said on July 13, that
    \href{https://www.nytimes.com/2020/07/13/us/lausd-san-diego-school-reopening.html?action=click\&pgtype=Article\&state=default\&region=MAIN_CONTENT_3\&context=storylines_faq}{instruction
    will be remote-only in the fall}, citing concerns that surging
    coronavirus infections in their areas pose too dire a risk for
    students and teachers. Together, the two districts enroll some
    825,000 students. They are the largest in the country so far to
    abandon plans for even a partial physical return to classrooms when
    they reopen in August. For other districts, the solution won't be an
    all-or-nothing approach.
    \href{https://bioethics.jhu.edu/research-and-outreach/projects/eschool-initiative/school-policy-tracker/}{Many
    systems}, including the nation's largest, New York City, are
    devising
    \href{https://www.nytimes.com/2020/06/26/us/coronavirus-schools-reopen-fall.html?action=click\&pgtype=Article\&state=default\&region=MAIN_CONTENT_3\&context=storylines_faq}{hybrid
    plans} that involve spending some days in classrooms and other days
    online. There's no national policy on this yet, so check with your
    municipal school system regularly to see what is happening in your
    community.
  \end{itemize}
\item ~
  \hypertarget{is-the-coronavirus-airborne}{%
  \paragraph{Is the coronavirus
  airborne?}\label{is-the-coronavirus-airborne}}

  \begin{itemize}
  \tightlist
  \item
    The coronavirus
    \href{https://www.nytimes.com/2020/07/04/health/239-experts-with-one-big-claim-the-coronavirus-is-airborne.html?action=click\&pgtype=Article\&state=default\&region=MAIN_CONTENT_3\&context=storylines_faq}{can
    stay aloft for hours in tiny droplets in stagnant air}, infecting
    people as they inhale, mounting scientific evidence suggests. This
    risk is highest in crowded indoor spaces with poor ventilation, and
    may help explain super-spreading events reported in meatpacking
    plants, churches and restaurants.
    \href{https://www.nytimes.com/2020/07/06/health/coronavirus-airborne-aerosols.html?action=click\&pgtype=Article\&state=default\&region=MAIN_CONTENT_3\&context=storylines_faq}{It's
    unclear how often the virus is spread} via these tiny droplets, or
    aerosols, compared with larger droplets that are expelled when a
    sick person coughs or sneezes, or transmitted through contact with
    contaminated surfaces, said Linsey Marr, an aerosol expert at
    Virginia Tech. Aerosols are released even when a person without
    symptoms exhales, talks or sings, according to Dr. Marr and more
    than 200 other experts, who
    \href{https://academic.oup.com/cid/article/doi/10.1093/cid/ciaa939/5867798}{have
    outlined the evidence in an open letter to the World Health
    Organization}.
  \end{itemize}
\item ~
  \hypertarget{what-are-the-symptoms-of-coronavirus}{%
  \paragraph{What are the symptoms of
  coronavirus?}\label{what-are-the-symptoms-of-coronavirus}}

  \begin{itemize}
  \tightlist
  \item
    Common symptoms
    \href{https://www.nytimes.com/article/symptoms-coronavirus.html?action=click\&pgtype=Article\&state=default\&region=MAIN_CONTENT_3\&context=storylines_faq}{include
    fever, a dry cough, fatigue and difficulty breathing or shortness of
    breath.} Some of these symptoms overlap with those of the flu,
    making detection difficult, but runny noses and stuffy sinuses are
    less common.
    \href{https://www.nytimes.com/2020/04/27/health/coronavirus-symptoms-cdc.html?action=click\&pgtype=Article\&state=default\&region=MAIN_CONTENT_3\&context=storylines_faq}{The
    C.D.C. has also} added chills, muscle pain, sore throat, headache
    and a new loss of the sense of taste or smell as symptoms to look
    out for. Most people fall ill five to seven days after exposure, but
    symptoms may appear in as few as two days or as many as 14 days.
  \end{itemize}
\item ~
  \hypertarget{does-asymptomatic-transmission-of-covid-19-happen}{%
  \paragraph{Does asymptomatic transmission of Covid-19
  happen?}\label{does-asymptomatic-transmission-of-covid-19-happen}}

  \begin{itemize}
  \tightlist
  \item
    So far, the evidence seems to show it does. A widely cited
    \href{https://www.nature.com/articles/s41591-020-0869-5}{paper}
    published in April suggests that people are most infectious about
    two days before the onset of coronavirus symptoms and estimated that
    44 percent of new infections were a result of transmission from
    people who were not yet showing symptoms. Recently, a top expert at
    the World Health Organization stated that transmission of the
    coronavirus by people who did not have symptoms was ``very rare,''
    \href{https://www.nytimes.com/2020/06/09/world/coronavirus-updates.html?action=click\&pgtype=Article\&state=default\&region=MAIN_CONTENT_3\&context=storylines_faq\#link-1f302e21}{but
    she later walked back that statement.}
  \end{itemize}
\end{itemize}

In the United States, millions of people have taken the Roche and Abbott
tests. LabCorp alone has performed more than two million antibody tests
made by the two manufacturers.

Quest relies on tests made by Abbott, Ortho Clinical and Euroimmun.
Quest declined to reveal what proportion of the 2.7 million tests it has
deployed so far were made by Abbott.

Dr. Jonathan Berz, a physician in Boston, tested positive for the virus
in early April but felt fine, apart from a sore throat. His wife was
sicker, and despite several negative diagnostic tests, she remained ill
for weeks.

``Initially, we felt as a family that, `Oh wow, we got sick,
unfortunately,''' Dr. Berz said. ```But the good side of that is that
we're going to have immunity.'''

In early June, the couple and their two children took Abbott antibody
tests processed by Quest. All four turned up negative. Even though Dr.
Berz knew that immunity is complex and that T cells also play a role, he
was disappointed.

As a doctor in a Covid-19 clinic, he had always acted as though he was
at risk for infection. But after seeing the antibody results, he said,
``my level of anxiety just increased.''

A spokeswoman at Abbott said the test had 100 percent sensitivity 17
days after symptoms began but did not provide information about
sensitivity beyond that time.

Dr. Beatus Ofenloch-Haehnle, who heads immunoassay research at Roche,
defended the company's antibody test. His team has tracked N antibodies
in 130 people who had mild to no symptoms and has not yet seen a
decline, he said.

``There is some fluctuation, but no waning at all,'' he said. ``We have
a lot of data, and we do not rely anymore on theory.'' The N antibody
can be a decent proxy for immunity, Dr. Ofenloch-Haehnle added.

He also pointed to a study by Public Health England that suggested that
the Abbott and Roche tests seemed to perform well
\href{https://assets.publishing.service.gov.uk/government/uploads/system/uploads/attachment_data/file/898437/Evaluation__of_sensitivity_and_specificity_of_4_commercially_available_SARS-CoV-2_antibody_immunoassays.pdf}{up
to 73 days after symptom onset}. ``I think we should be careful to jump
to conclusions too soon,'' he said.

Other experts also urged caution. Without more information about what
antibody testing results mean, they said, people should do as Dr. Berz
did: Act as though they do not have immunity.

There is no definitive information as yet on what levels of antibodies
are needed for immunity or how long that protection might last. ``I
think we're getting closer and closer to that knowledge,'' Dr. Iwasaki
said.

Advertisement

\protect\hyperlink{after-bottom}{Continue reading the main story}

\hypertarget{site-index}{%
\subsection{Site Index}\label{site-index}}

\hypertarget{site-information-navigation}{%
\subsection{Site Information
Navigation}\label{site-information-navigation}}

\begin{itemize}
\tightlist
\item
  \href{https://help.nytimes.com/hc/en-us/articles/115014792127-Copyright-notice}{©~2020~The
  New York Times Company}
\end{itemize}

\begin{itemize}
\tightlist
\item
  \href{https://www.nytco.com/}{NYTCo}
\item
  \href{https://help.nytimes.com/hc/en-us/articles/115015385887-Contact-Us}{Contact
  Us}
\item
  \href{https://www.nytco.com/careers/}{Work with us}
\item
  \href{https://nytmediakit.com/}{Advertise}
\item
  \href{http://www.tbrandstudio.com/}{T Brand Studio}
\item
  \href{https://www.nytimes.com/privacy/cookie-policy\#how-do-i-manage-trackers}{Your
  Ad Choices}
\item
  \href{https://www.nytimes.com/privacy}{Privacy}
\item
  \href{https://help.nytimes.com/hc/en-us/articles/115014893428-Terms-of-service}{Terms
  of Service}
\item
  \href{https://help.nytimes.com/hc/en-us/articles/115014893968-Terms-of-sale}{Terms
  of Sale}
\item
  \href{https://spiderbites.nytimes.com}{Site Map}
\item
  \href{https://help.nytimes.com/hc/en-us}{Help}
\item
  \href{https://www.nytimes.com/subscription?campaignId=37WXW}{Subscriptions}
\end{itemize}
