Sections

SEARCH

\protect\hyperlink{site-content}{Skip to
content}\protect\hyperlink{site-index}{Skip to site index}

\href{https://www.nytimes.com/section/health}{Health}

\href{https://myaccount.nytimes.com/auth/login?response_type=cookie\&client_id=vi}{}

\href{https://www.nytimes.com/section/todayspaper}{Today's Paper}

\href{/section/health}{Health}\textbar{}Your Coronavirus Antibodies Are
Disappearing. Should You Care?

\url{https://nyti.ms/30U14v0}

\begin{itemize}
\item
\item
\item
\item
\item
\item
\end{itemize}

\href{https://www.nytimes.com/news-event/coronavirus?action=click\&pgtype=Article\&state=default\&region=TOP_BANNER\&context=storylines_menu}{The
Coronavirus Outbreak}

\begin{itemize}
\tightlist
\item
  live\href{https://www.nytimes.com/2020/08/04/world/coronavirus-cases.html?action=click\&pgtype=Article\&state=default\&region=TOP_BANNER\&context=storylines_menu}{Latest
  Updates}
\item
  \href{https://www.nytimes.com/interactive/2020/us/coronavirus-us-cases.html?action=click\&pgtype=Article\&state=default\&region=TOP_BANNER\&context=storylines_menu}{Maps
  and Cases}
\item
  \href{https://www.nytimes.com/interactive/2020/science/coronavirus-vaccine-tracker.html?action=click\&pgtype=Article\&state=default\&region=TOP_BANNER\&context=storylines_menu}{Vaccine
  Tracker}
\item
  \href{https://www.nytimes.com/2020/08/02/us/covid-college-reopening.html?action=click\&pgtype=Article\&state=default\&region=TOP_BANNER\&context=storylines_menu}{College
  Reopening}
\item
  \href{https://www.nytimes.com/live/2020/08/04/business/stock-market-today-coronavirus?action=click\&pgtype=Article\&state=default\&region=TOP_BANNER\&context=storylines_menu}{Economy}
\end{itemize}

Advertisement

\protect\hyperlink{after-top}{Continue reading the main story}

Supported by

\protect\hyperlink{after-sponsor}{Continue reading the main story}

\hypertarget{your-coronavirus-antibodies-are-disappearing-should-you-care}{%
\section{Your Coronavirus Antibodies Are Disappearing. Should You
Care?}\label{your-coronavirus-antibodies-are-disappearing-should-you-care}}

Declining antibody levels do not mean less immunity, experts say.
Besides, two widely used tests may detect the wrong antibodies.

\includegraphics{https://static01.nyt.com/images/2020/07/26/us/politics/26virus-antibodies/merlin_174661485_e2c83e91-d733-4091-8cb8-ad86bbc8fd6a-articleLarge.jpg?quality=75\&auto=webp\&disable=upscale}

By \href{https://www.nytimes.com/by/apoorva-mandavilli}{Apoorva
Mandavilli}

\begin{itemize}
\item
  Published July 26, 2020Updated July 27, 2020
\item
  \begin{itemize}
  \item
  \item
  \item
  \item
  \item
  \item
  \end{itemize}
\end{itemize}

Your blood carries the memory of every pathogen you've ever encountered.
If you've been infected with the coronavirus, your body most likely
remembers that, too.

Antibodies are the legacy of that encounter. Why, then, have so many
people stricken by the virus discovered that they don't seem to have
antibodies?

Blame the tests.

Most commercial antibody tests offer crude yes-no answers. The tests are
\href{https://www.nytimes.com/2020/04/24/health/coronavirus-antibody-tests.html}{notorious
for delivering} false positives --- results indicating that someone has
antibodies when he or she does not.

But the volume of coronavirus antibodies drops sharply once the acute
illness ends. Now it is increasingly clear that these tests may also
produce false-negative results, missing antibodies to the coronavirus
that are present at low levels.

Moreover, some tests --- including those made by Abbott and Roche and
offered by Quest Diagnostics and LabCorp --- are designed to detect a
subtype of antibodies that doesn't confer immunity and may wane even
faster than the kind that can destroy the virus.

What that means is that declining antibodies, as shown by commercial
tests, don't necessarily mean declining immunity, several experts said.
Long-term surveys of antibodies, intended to assess how widely the
coronavirus has spread, may also underestimate the true prevalence.

``We're learning a lot about how antibodies change over time,'' said Dr.
Fiona Havers, a medical epidemiologist who has led such surveys for the
Centers for Disease Control and Prevention.

If the narrative on immunity to the coronavirus has seemed to shift
constantly, it's in part because the virus was a stranger to scientists.
But it's increasingly clear that this virus behaves much like any other.

This is how immunity to viruses generally works: The initial encounter
with a pathogen --- typically in childhood --- surprises the body. The
resulting illness can be mild or severe, depending on the dose of the
virus and the child's health, access to health care and genetics.

A mild illness may trigger production of only a few antibodies, and a
severe one many more. The vast majority of people who become infected
with the coronavirus have few to no symptoms, many experts believe, and
those people may produce a milder immune response.

\hypertarget{latest-updates-global-coronavirus-outbreak}{%
\section{\texorpdfstring{\href{https://www.nytimes.com/2020/08/04/world/coronavirus-cases.html?action=click\&pgtype=Article\&state=default\&region=MAIN_CONTENT_1\&context=storylines_live_updates}{Latest
Updates: Global Coronavirus
Outbreak}}{Latest Updates: Global Coronavirus Outbreak}}\label{latest-updates-global-coronavirus-outbreak}}

Updated 2020-08-05T07:58:24.076Z

\begin{itemize}
\tightlist
\item
  \href{https://www.nytimes.com/2020/08/04/world/coronavirus-cases.html?action=click\&pgtype=Article\&state=default\&region=MAIN_CONTENT_1\&context=storylines_live_updates\#link-762df92}{As
  talks drag on, McConnell signals openness to jobless aid extension,
  and negotiators agree on a deadline.}
\item
  \href{https://www.nytimes.com/2020/08/04/world/coronavirus-cases.html?action=click\&pgtype=Article\&state=default\&region=MAIN_CONTENT_1\&context=storylines_live_updates\#link-1228a480}{Novavax
  sees encouraging results from two studies of its experimental
  vaccine.}
\item
  \href{https://www.nytimes.com/2020/08/04/world/coronavirus-cases.html?action=click\&pgtype=Article\&state=default\&region=MAIN_CONTENT_1\&context=storylines_live_updates\#link-794484ed}{Mississippians
  must now wear masks in public, governor says.}
\end{itemize}

\href{https://www.nytimes.com/2020/08/04/world/coronavirus-cases.html?action=click\&pgtype=Article\&state=default\&region=MAIN_CONTENT_1\&context=storylines_live_updates}{See
more updates}

More live coverage:
\href{https://www.nytimes.com/live/2020/08/04/business/stock-market-today-coronavirus?action=click\&pgtype=Article\&state=default\&region=MAIN_CONTENT_1\&context=storylines_live_updates}{Markets}

But even a minor infection is often enough to teach the body to
recognize the intruder.

After the battle ends, balloon-like cells that live in the bone marrow
steadily pump out a small number of specialized assassins. The next time
--- and every time after that --- that the body comes across the virus,
those cells can mass-produce antibodies within hours.

The mnemonic response grows stronger with every encounter. It's one of
the great miracles of the human body.

``Whatever your level is today, if you get infected, your antibody
titers are going to go way up,'' said Dr. Michael Mina, an immunologist
at Harvard University, referring to the levels of antibodies in the
blood. ``The virus will never even have a chance the second time
around.''

A single drop of blood contains billions of antibodies, all lying in
wait for their specific targets. Sometimes, as may be the case for
antibodies to the coronavirus, there are too few to get a positive
signal on a test --- but that does not mean the person tested has no
immunity to the virus.

``Even if their antibodies wane below the limits of detection of our
instruments, it doesn't mean their `memory' is gone,'' Dr. Mina said.

A small number of people may not produce any antibodies to the
coronavirus. But even in that unlikely event, they will have so-called
cellular immunity, which includes T cells that learn to identify and
destroy the virus. Virtually everyone infected with the coronavirus
seems to develop T-cell responses, according to several recent studies.

``This means that even if the antibody titer is low, those people who
are previously infected may have a good enough T-cell response that can
provide protection,'' said Akiko Iwasaki, an immunologist at Yale
University.

T cells are harder to detect and to study, however, so when it comes to
immunity, antibodies have received all of the attention. The coronavirus
carries several antigens --- proteins or pieces of a protein --- that
can provoke the body into producing antibodies.

The most powerful antibodies recognize a piece of the coronavirus's
spike protein, the receptor binding domain, or R.B.D. That is the part
of the virus that docks onto human cells. Antibodies that recognize the
R.B.D. can neutralize the virus and prevent infection.

But the Roche and Abbott tests that are now widely available --- and
several others
\href{https://www.fda.gov/medical-devices/emergency-situations-medical-devices/eua-authorized-serology-test-performance}{authorized
by the Food and Drug Administration} --- instead look for antibodies to
a protein called the nucleocapsid, or N, that is bound up with the
virus's genetic material.

Some scientists were stunned to hear of this choice.

``God, I did not realize that --- that's crazy,'' said Angela Rasmussen,
a virologist at Columbia University in New York. ``It's kind of puzzling
to design a test that's not looking for what's thought to be the major
antigen.''

The N protein is plentiful in the blood, and testing for antibodies to
it produces a swifter, brighter signal than testing for antibodies to
the spike protein. Because antibody tests are used to detect past
infection, however, manufacturers are not required to prove that the
antibodies their tests seek are those that actually confer protection
against the virus.

Officials at the Food and Drug Administration did not respond to
requests for comment on whether the two tests target the appropriate
antibodies.

There's another wrinkle to the story. Some
\href{https://www.medrxiv.org/content/10.1101/2020.07.14.20153536v2?\%253fcollection=}{reports}
now suggest that
\href{https://www.nature.com/articles/s41591-020-0965-6}{antibodies
to}the viral nucleocapsid may
\href{https://www.medrxiv.org/content/10.1101/2020.07.16.20155663v2}{decline
faster} than those to R.B.D. or to the entire spike --- the really
effective ones.

``The majority of people are getting tested for anti-N antibody, which
does tend to wane more rapidly --- and so, you know, it may be not the
most suitable test for looking at neutralizing capacity,'' Dr. Iwasaki
said.

\href{https://www.nytimes.com/news-event/coronavirus?action=click\&pgtype=Article\&state=default\&region=MAIN_CONTENT_3\&context=storylines_faq}{}

\hypertarget{the-coronavirus-outbreak-}{%
\subsubsection{The Coronavirus Outbreak
›}\label{the-coronavirus-outbreak-}}

\hypertarget{frequently-asked-questions}{%
\paragraph{Frequently Asked
Questions}\label{frequently-asked-questions}}

Updated August 4, 2020

\begin{itemize}
\item ~
  \hypertarget{i-have-antibodies-am-i-now-immune}{%
  \paragraph{I have antibodies. Am I now
  immune?}\label{i-have-antibodies-am-i-now-immune}}

  \begin{itemize}
  \tightlist
  \item
    As of right
    now,\href{https://www.nytimes.com/2020/07/22/health/covid-antibodies-herd-immunity.html?action=click\&pgtype=Article\&state=default\&region=MAIN_CONTENT_3\&context=storylines_faq}{that
    seems likely, for at least several months.} There have been
    frightening accounts of people suffering what seems to be a second
    bout of Covid-19. But experts say these patients may have a
    drawn-out course of infection, with the virus taking a slow toll
    weeks to months after initial exposure. People infected with the
    coronavirus typically
    \href{https://www.nature.com/articles/s41586-020-2456-9}{produce}
    immune molecules called antibodies, which are
    \href{https://www.nytimes.com/2020/05/07/health/coronavirus-antibody-prevalence.html?action=click\&pgtype=Article\&state=default\&region=MAIN_CONTENT_3\&context=storylines_faq}{protective
    proteins made in response to an
    infection}\href{https://www.nytimes.com/2020/05/07/health/coronavirus-antibody-prevalence.html?action=click\&pgtype=Article\&state=default\&region=MAIN_CONTENT_3\&context=storylines_faq}{.
    These antibodies may} last in the body
    \href{https://www.nature.com/articles/s41591-020-0965-6}{only two to
    three months}, which may seem worrisome, but that's perfectly normal
    after an acute infection subsides, said Dr. Michael Mina, an
    immunologist at Harvard University. It may be possible to get the
    coronavirus again, but it's highly unlikely that it would be
    possible in a short window of time from initial infection or make
    people sicker the second time.
  \end{itemize}
\item ~
  \hypertarget{im-a-small-business-owner-can-i-get-relief}{%
  \paragraph{I'm a small-business owner. Can I get
  relief?}\label{im-a-small-business-owner-can-i-get-relief}}

  \begin{itemize}
  \tightlist
  \item
    The
    \href{https://www.nytimes.com/article/small-business-loans-stimulus-grants-freelancers-coronavirus.html?action=click\&pgtype=Article\&state=default\&region=MAIN_CONTENT_3\&context=storylines_faq}{stimulus
    bills enacted in March} offer help for the millions of American
    small businesses. Those eligible for aid are businesses and
    nonprofit organizations with fewer than 500 workers, including sole
    proprietorships, independent contractors and freelancers. Some
    larger companies in some industries are also eligible. The help
    being offered, which is being managed by the Small Business
    Administration, includes the Paycheck Protection Program and the
    Economic Injury Disaster Loan program. But lots of folks have
    \href{https://www.nytimes.com/interactive/2020/05/07/business/small-business-loans-coronavirus.html?action=click\&pgtype=Article\&state=default\&region=MAIN_CONTENT_3\&context=storylines_faq}{not
    yet seen payouts.} Even those who have received help are confused:
    The rules are draconian, and some are stuck sitting on
    \href{https://www.nytimes.com/2020/05/02/business/economy/loans-coronavirus-small-business.html?action=click\&pgtype=Article\&state=default\&region=MAIN_CONTENT_3\&context=storylines_faq}{money
    they don't know how to use.} Many small-business owners are getting
    less than they expected or
    \href{https://www.nytimes.com/2020/06/10/business/Small-business-loans-ppp.html?action=click\&pgtype=Article\&state=default\&region=MAIN_CONTENT_3\&context=storylines_faq}{not
    hearing anything at all.}
  \end{itemize}
\item ~
  \hypertarget{what-are-my-rights-if-i-am-worried-about-going-back-to-work}{%
  \paragraph{What are my rights if I am worried about going back to
  work?}\label{what-are-my-rights-if-i-am-worried-about-going-back-to-work}}

  \begin{itemize}
  \tightlist
  \item
    Employers have to provide
    \href{https://www.osha.gov/SLTC/covid-19/standards.html}{a safe
    workplace} with policies that protect everyone equally.
    \href{https://www.nytimes.com/article/coronavirus-money-unemployment.html?action=click\&pgtype=Article\&state=default\&region=MAIN_CONTENT_3\&context=storylines_faq}{And
    if one of your co-workers tests positive for the coronavirus, the
    C.D.C.} has said that
    \href{https://www.cdc.gov/coronavirus/2019-ncov/community/guidance-business-response.html}{employers
    should tell their employees} -\/- without giving you the sick
    employee's name -\/- that they may have been exposed to the virus.
  \end{itemize}
\item ~
  \hypertarget{should-i-refinance-my-mortgage}{%
  \paragraph{Should I refinance my
  mortgage?}\label{should-i-refinance-my-mortgage}}

  \begin{itemize}
  \tightlist
  \item
    \href{https://www.nytimes.com/article/coronavirus-money-unemployment.html?action=click\&pgtype=Article\&state=default\&region=MAIN_CONTENT_3\&context=storylines_faq}{It
    could be a good idea,} because mortgage rates have
    \href{https://www.nytimes.com/2020/07/16/business/mortgage-rates-below-3-percent.html?action=click\&pgtype=Article\&state=default\&region=MAIN_CONTENT_3\&context=storylines_faq}{never
    been lower.} Refinancing requests have pushed mortgage applications
    to some of the highest levels since 2008, so be prepared to get in
    line. But defaults are also up, so if you're thinking about buying a
    home, be aware that some lenders have tightened their standards.
  \end{itemize}
\item ~
  \hypertarget{what-is-school-going-to-look-like-in-september}{%
  \paragraph{What is school going to look like in
  September?}\label{what-is-school-going-to-look-like-in-september}}

  \begin{itemize}
  \tightlist
  \item
    It is unlikely that many schools will return to a normal schedule
    this fall, requiring the grind of
    \href{https://www.nytimes.com/2020/06/05/us/coronavirus-education-lost-learning.html?action=click\&pgtype=Article\&state=default\&region=MAIN_CONTENT_3\&context=storylines_faq}{online
    learning},
    \href{https://www.nytimes.com/2020/05/29/us/coronavirus-child-care-centers.html?action=click\&pgtype=Article\&state=default\&region=MAIN_CONTENT_3\&context=storylines_faq}{makeshift
    child care} and
    \href{https://www.nytimes.com/2020/06/03/business/economy/coronavirus-working-women.html?action=click\&pgtype=Article\&state=default\&region=MAIN_CONTENT_3\&context=storylines_faq}{stunted
    workdays} to continue. California's two largest public school
    districts --- Los Angeles and San Diego --- said on July 13, that
    \href{https://www.nytimes.com/2020/07/13/us/lausd-san-diego-school-reopening.html?action=click\&pgtype=Article\&state=default\&region=MAIN_CONTENT_3\&context=storylines_faq}{instruction
    will be remote-only in the fall}, citing concerns that surging
    coronavirus infections in their areas pose too dire a risk for
    students and teachers. Together, the two districts enroll some
    825,000 students. They are the largest in the country so far to
    abandon plans for even a partial physical return to classrooms when
    they reopen in August. For other districts, the solution won't be an
    all-or-nothing approach.
    \href{https://bioethics.jhu.edu/research-and-outreach/projects/eschool-initiative/school-policy-tracker/}{Many
    systems}, including the nation's largest, New York City, are
    devising
    \href{https://www.nytimes.com/2020/06/26/us/coronavirus-schools-reopen-fall.html?action=click\&pgtype=Article\&state=default\&region=MAIN_CONTENT_3\&context=storylines_faq}{hybrid
    plans} that involve spending some days in classrooms and other days
    online. There's no national policy on this yet, so check with your
    municipal school system regularly to see what is happening in your
    community.
  \end{itemize}
\end{itemize}

In the United States, millions of people have taken the Roche and Abbott
tests. LabCorp alone has performed more than two million antibody tests
made by the two manufacturers.

Quest relies on tests made by Abbott, Ortho Clinical and Euroimmun.
Quest declined to reveal what proportion of the 2.7 million tests it has
deployed so far were made by Abbott.

Dr. Jonathan Berz, a physician in Boston, tested positive for the virus
in early April but felt fine, apart from a sore throat. His wife was
sicker, and despite several negative diagnostic tests, she remained ill
for weeks.

``Initially, we felt as a family that, `Oh wow, we got sick,
unfortunately,''' Dr. Berz said. ```But the good side of that is that
we're going to have immunity.'''

In early June, the couple and their two children took Abbott antibody
tests processed by Quest. All four turned up negative. Even though Dr.
Berz knew that immunity is complex and that T cells also play a role, he
was disappointed.

As a doctor in a Covid-19 clinic, he had always acted as though he was
at risk for infection. But after seeing the antibody results, he said,
``my level of anxiety just increased.''

A spokeswoman at Abbott said the test had 100 percent sensitivity 17
days after symptoms began but did not provide information about
sensitivity beyond that time.

Dr. Beatus Ofenloch-Haehnle, who heads immunoassay research at Roche,
defended the company's antibody test. His team has tracked N antibodies
in 130 people who had mild to no symptoms and has not yet seen a
decline, he said.

``There is some fluctuation, but no waning at all,'' he said. ``We have
a lot of data, and we do not rely anymore on theory.'' The N antibody
can be a decent proxy for immunity, Dr. Ofenloch-Haehnle added.

He also pointed to a study by Public Health England that suggested that
the Abbott and Roche tests seemed to perform well
\href{https://assets.publishing.service.gov.uk/government/uploads/system/uploads/attachment_data/file/898437/Evaluation__of_sensitivity_and_specificity_of_4_commercially_available_SARS-CoV-2_antibody_immunoassays.pdf}{up
to 73 days after symptom onset}. ``I think we should be careful to jump
to conclusions too soon,'' he said.

Other experts also urged caution. Without more information about what
antibody testing results mean, they said, people should do as Dr. Berz
did: Act as though they do not have immunity.

There is no definitive information as yet on what levels of antibodies
are needed for immunity or how long that protection might last. ``I
think we're getting closer and closer to that knowledge,'' Dr. Iwasaki
said.

Advertisement

\protect\hyperlink{after-bottom}{Continue reading the main story}

\hypertarget{site-index}{%
\subsection{Site Index}\label{site-index}}

\hypertarget{site-information-navigation}{%
\subsection{Site Information
Navigation}\label{site-information-navigation}}

\begin{itemize}
\tightlist
\item
  \href{https://help.nytimes.com/hc/en-us/articles/115014792127-Copyright-notice}{©~2020~The
  New York Times Company}
\end{itemize}

\begin{itemize}
\tightlist
\item
  \href{https://www.nytco.com/}{NYTCo}
\item
  \href{https://help.nytimes.com/hc/en-us/articles/115015385887-Contact-Us}{Contact
  Us}
\item
  \href{https://www.nytco.com/careers/}{Work with us}
\item
  \href{https://nytmediakit.com/}{Advertise}
\item
  \href{http://www.tbrandstudio.com/}{T Brand Studio}
\item
  \href{https://www.nytimes.com/privacy/cookie-policy\#how-do-i-manage-trackers}{Your
  Ad Choices}
\item
  \href{https://www.nytimes.com/privacy}{Privacy}
\item
  \href{https://help.nytimes.com/hc/en-us/articles/115014893428-Terms-of-service}{Terms
  of Service}
\item
  \href{https://help.nytimes.com/hc/en-us/articles/115014893968-Terms-of-sale}{Terms
  of Sale}
\item
  \href{https://spiderbites.nytimes.com}{Site Map}
\item
  \href{https://help.nytimes.com/hc/en-us}{Help}
\item
  \href{https://www.nytimes.com/subscription?campaignId=37WXW}{Subscriptions}
\end{itemize}
