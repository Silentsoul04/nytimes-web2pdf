Sections

SEARCH

\protect\hyperlink{site-content}{Skip to
content}\protect\hyperlink{site-index}{Skip to site index}

\href{https://www.nytimes.com/section/us}{U.S.}

\href{https://myaccount.nytimes.com/auth/login?response_type=cookie\&client_id=vi}{}

\href{https://www.nytimes.com/section/todayspaper}{Today's Paper}

\href{/section/us}{U.S.}\textbar{}Garrett Foster Brought His Gun to
Austin Protests. Then He Was Shot Dead.

\url{https://nyti.ms/3jFOAj5}

\begin{itemize}
\item
\item
\item
\item
\item
\end{itemize}

\href{https://www.nytimes.com/news-event/george-floyd-protests-minneapolis-new-york-los-angeles?action=click\&pgtype=Article\&state=default\&region=TOP_BANNER\&context=storylines_menu}{Race
and America}

\begin{itemize}
\tightlist
\item
  \href{https://www.nytimes.com/2020/07/26/us/protests-portland-seattle-trump.html?action=click\&pgtype=Article\&state=default\&region=TOP_BANNER\&context=storylines_menu}{Protesters
  Return to Other Cities}
\item
  \href{https://www.nytimes.com/2020/07/24/us/portland-oregon-protests-white-race.html?action=click\&pgtype=Article\&state=default\&region=TOP_BANNER\&context=storylines_menu}{Portland
  at the Center}
\item
  \href{https://www.nytimes.com/2020/07/23/podcasts/the-daily/portland-protests.html?action=click\&pgtype=Article\&state=default\&region=TOP_BANNER\&context=storylines_menu}{Podcast:
  Showdown in Portland}
\item
  \href{https://www.nytimes.com/interactive/2020/07/16/us/black-lives-matter-protests-louisville-breonna-taylor.html?action=click\&pgtype=Article\&state=default\&region=TOP_BANNER\&context=storylines_menu}{45
  Days in Louisville}
\end{itemize}

Advertisement

\protect\hyperlink{after-top}{Continue reading the main story}

Supported by

\protect\hyperlink{after-sponsor}{Continue reading the main story}

\hypertarget{garrett-foster-brought-his-gun-to-austin-protests-then-he-was-shot-dead}{%
\section{Garrett Foster Brought His Gun to Austin Protests. Then He Was
Shot
Dead.}\label{garrett-foster-brought-his-gun-to-austin-protests-then-he-was-shot-dead}}

The police in Austin, Texas, have not identified the motorist who
fatally shot a protester after driving his car in the direction of
marchers.

\includegraphics{https://static01.nyt.com/images/2020/08/24/multimedia/24protest-austin-shooting/24protest-austin-shooting-videoSixteenByNine3000.jpg}

By David Montgomery and
\href{https://www.nytimes.com/by/manny-fernandez}{Manny Fernandez}

\begin{itemize}
\item
  July 26, 2020
\item
  \begin{itemize}
  \item
  \item
  \item
  \item
  \item
  \end{itemize}
\end{itemize}

AUSTIN, Texas --- It was not unusual for Garrett Foster to be at a
protest against police brutality on a Saturday night. And it was not out
of character for him to be armed as he marched.

Mr. Foster was carrying an AK-47 rifle as he joined a Black Lives Matter
demonstration blocks from the State Capitol in Austin, Texas. Gun-rights
supporters on both the left and the right often carry rifles at protests
in Texas, a state whose liberal gun laws allow it.

Mr. Foster, wearing a black bandanna and a baseball cap, bumped into an
independent journalist at the march on Saturday, and he spoke
matter-of-factly about the weapon that was draped on a strap in front of
him.

``They don't let us march in the streets anymore, so I got to practice
some of our rights,'' Mr. Foster told the journalist, Hiram Gilberto
Garcia, who was broadcasting the interview
\href{https://www.pscp.tv/w/1YqJDpVaEPDJV}{live on Periscope}. ``If I
use it against the cops, I'm dead,'' he conceded.

Later that night, Mr. Foster was fatally shot, but not by the police.
The authorities said he was killed by a motorist who had a confrontation
with protesters.

The police and witnesses said the man in the car turned it aggressively
toward the marchers, and Mr. Foster then approached it. The driver
opened fire, shooting Mr. Foster three times. He was rushed to a
hospital and was later pronounced dead.

Austin's police chief, Brian Manley, told reporters on Sunday that as
the motorist turned, a crowd of protesters surrounded the vehicle, and
some struck the car. The driver, whose name has not been released, then
opened fire from inside the car as Mr. Foster approached. Another person
in the crowd pulled out a handgun and shot at the vehicle as it sped
away.

Minutes after the shooting, the driver called 911 and said he had been
involved in a shooting and had driven away from the scene, Chief Manley
said. The caller told dispatchers he had shot someone who had approached
the driver's window and pointed a rifle at him.

``His account is that Mr. Foster pointed the weapon directly at him and
he fired his handgun at Mr. Foster,'' the chief said of the driver.

Both the driver and the other person who fired a weapon were detained
and interviewed by detectives. Both had state-issued handgun licenses
and have been released as the investigation continues, Chief Manley
said.

The shooting stunned a capital city where demonstrations and marches are
a proud and commonplace tradition. A
\href{https://www.gofundme.com/f/official-garrett-foster-memorial-fund}{GoFundMe
page} to help Mr. Foster's relatives with his funeral expenses had
already raised nearly \$100,000 by Sunday evening.

And while Mayor Steve Adler and other officials expressed their
condolences on Sunday, at least one police leader criticized Mr. Foster.

On Twitter, Kenneth Casaday, the president of the Austin police
officers' union, retweeted a video clip of Mr. Foster explaining to Mr.
Garcia, the independent journalist, why he brought his rifle. In the
clip, Mr. Foster is heard using curse words to talk about ``all the
people that hate us,'' but are too afraid to ``stop and actually do
anything about it.''

\href{https://twitter.com/KennethCasaday/status/1287373267318300674}{In
his tweet}, Mr. Casaday wrote: ``This is the guy that lost his life last
night. He was looking for confrontation and he found it.''

Mr. Garcia, who has filmed numerous Austin demonstrations in recent
weeks, captured the chaotic moments of the shooting
\href{https://www.facebook.com/watch/live/?v=295346775139805\&ref=watch_permalink}{live
on video}. Protesters are seen marching through an intersection when a
car blares its horn. Marchers appear to converge around the car as a man
calls out, ``Everybody back up.'' At that instant, five shots ring out,
followed shortly by several more loud bangs that echo through the
downtown streets.

Mr. Foster, who had served in the military, was armed, but he was not
seeking out trouble at the march, relatives and witnesses told
reporters. At the time of the shooting, Mr. Foster was pushing his
fiancée through the intersection in her wheelchair.

Mr. Foster and his fiancée, Whitney Mitchell, had been taking part in
protests against police brutality in Austin daily since the killing of
George Floyd in Minneapolis. Mr. Foster is white, and Ms. Mitchell, who
is a quadruple amputee, is African-American. She was not injured in the
shooting.

``He was doing it because he feels really strongly about justice and
he's very heavily against police brutality, and he wanted to support his
fiancée,'' Mr. Foster's mother, Sheila Foster, said in an interview with
``Good Morning America,'' adding that she was not surprised he was armed
while at the march.

``He does have a license to carry, and he would've felt the need to
protect himself,'' Ms. Foster said.

In Texas, it is lawful to carry rifles, shotguns and other so-called
long guns on the street without a permit, as long as the weapons are not
brandished in a threatening manner; state-issued licenses are required
only to carry handguns.

The presence of Mr. Foster's weapon could play a key role in the case if
the driver claims that he shot Mr. Foster out of fear for his life, a
defense allowed under the so-called ``stand your ground'' law in the
state.

The shooting reignited a long-running debate in Texas about the ``open
carry'' movement, in which many men and women carry their rifles and
other weapons in public places.

Gun-control supporters say the movement that encourages such displays
seeks to intimidate the police and the public, while gun-rights
activists defend it as a celebration of their Second Amendment rights.

In a 2016 attack on police officers at a downtown Dallas demonstration,
several marchers carried AR-15s and other military-style rifles, and
local officials said their presence
\href{https://www.nytimes.com/2016/07/11/us/texas-open-carry-laws-blurred-lines-between-suspects-and-marchers.html}{created
confusion} for police officers. A single gunman, Micah Johnson, a former
Army reservist, killed five officers.

``There are multiple layers to this tragedy, but adding guns to any
emotional and potentially volatile situation can, and too often does,
lead to deadly violence,'' Ed Scruggs, the board president of Texas Gun
Sense, a gun legislation reform group, said in a statement about the
Austin shooting.

C.J. Grisham, founder and president of the gun-rights organization Open
Carry Texas, defended the practice of bringing rifles to rallies and
marches, particularly after
\href{https://www.nytimes.com/2020/07/09/us/bloomington-vauhxx-booker-car-protesters.html}{numerous
attacks} around the country in which motorists have driven their cars
into demonstrations and injured or killed protesters.

``Protesters are under attack from a wide variety of people,'' Mr.
Grisham said. ``It's unfortunate these days that if you're going to
exercise your First Amendment rights, you probably need to be exercising
your Second Amendment rights as well.''

The shooting occurred shortly before 10 p.m. James Sasinowski, 24, a
witness, said it seemed the driver was trying to turn a corner and did
not want to wait for marchers to pass.

``The driver intentionally and aggressively accelerated into a crowd of
people,'' Mr. Sasinowski said. ``We were not aggravating him at all. He
incited the violence.''

Michael Capochiano, another witness, had a slightly different account of
what happened. He said he was marching with other demonstrators when he
saw a motorist honk his horn and turn toward the crowd, forcing people
to scatter.

``You could hear the wheels squealing from hitting the accelerator so
fast,'' said Mr. Capochiano, 53, a restaurant accountant. ``I'm a little
surprised that nobody got hit.''

The car came to a stop after turning from Fourth Street onto Congress
Avenue and appeared to strike a traffic pylon. As people shouted angrily
at the driver, Mr. Foster walked toward the car, with the muzzle of his
rifle pointed downward, he said.

``He was not aiming the gun or doing anything aggressive with the gun,''
Mr. Capochiano said. ``I'm not sure if there was much of an exchange of
words. It wasn't like there was any sort of verbal altercations. He
wasn't charging at the car.''

David Montgomery reported from Austin and Manny Fernandez from Houston.
Bryan Pietsch contributed reporting from Andover, Minn.

Advertisement

\protect\hyperlink{after-bottom}{Continue reading the main story}

\hypertarget{site-index}{%
\subsection{Site Index}\label{site-index}}

\hypertarget{site-information-navigation}{%
\subsection{Site Information
Navigation}\label{site-information-navigation}}

\begin{itemize}
\tightlist
\item
  \href{https://help.nytimes.com/hc/en-us/articles/115014792127-Copyright-notice}{©~2020~The
  New York Times Company}
\end{itemize}

\begin{itemize}
\tightlist
\item
  \href{https://www.nytco.com/}{NYTCo}
\item
  \href{https://help.nytimes.com/hc/en-us/articles/115015385887-Contact-Us}{Contact
  Us}
\item
  \href{https://www.nytco.com/careers/}{Work with us}
\item
  \href{https://nytmediakit.com/}{Advertise}
\item
  \href{http://www.tbrandstudio.com/}{T Brand Studio}
\item
  \href{https://www.nytimes.com/privacy/cookie-policy\#how-do-i-manage-trackers}{Your
  Ad Choices}
\item
  \href{https://www.nytimes.com/privacy}{Privacy}
\item
  \href{https://help.nytimes.com/hc/en-us/articles/115014893428-Terms-of-service}{Terms
  of Service}
\item
  \href{https://help.nytimes.com/hc/en-us/articles/115014893968-Terms-of-sale}{Terms
  of Sale}
\item
  \href{https://spiderbites.nytimes.com}{Site Map}
\item
  \href{https://help.nytimes.com/hc/en-us}{Help}
\item
  \href{https://www.nytimes.com/subscription?campaignId=37WXW}{Subscriptions}
\end{itemize}
