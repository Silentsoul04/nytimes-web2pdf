Sections

SEARCH

\protect\hyperlink{site-content}{Skip to
content}\protect\hyperlink{site-index}{Skip to site index}

\href{https://www.nytimes.com/section/us}{U.S.}

\href{https://myaccount.nytimes.com/auth/login?response_type=cookie\&client_id=vi}{}

\href{https://www.nytimes.com/section/todayspaper}{Today's Paper}

\href{/section/us}{U.S.}\textbar{}`You Do the Right Things, and Still
You Get It'

\url{https://nyti.ms/3g1o1CP}

\begin{itemize}
\item
\item
\item
\item
\item
\item
\end{itemize}

\href{https://www.nytimes.com/news-event/coronavirus?action=click\&pgtype=Article\&state=default\&region=TOP_BANNER\&context=storylines_menu}{The
Coronavirus Outbreak}

\begin{itemize}
\tightlist
\item
  live\href{https://www.nytimes.com/2020/08/01/world/coronavirus-covid-19.html?action=click\&pgtype=Article\&state=default\&region=TOP_BANNER\&context=storylines_menu}{Latest
  Updates}
\item
  \href{https://www.nytimes.com/interactive/2020/us/coronavirus-us-cases.html?action=click\&pgtype=Article\&state=default\&region=TOP_BANNER\&context=storylines_menu}{Maps
  and Cases}
\item
  \href{https://www.nytimes.com/interactive/2020/science/coronavirus-vaccine-tracker.html?action=click\&pgtype=Article\&state=default\&region=TOP_BANNER\&context=storylines_menu}{Vaccine
  Tracker}
\item
  \href{https://www.nytimes.com/interactive/2020/07/29/us/schools-reopening-coronavirus.html?action=click\&pgtype=Article\&state=default\&region=TOP_BANNER\&context=storylines_menu}{What
  School May Look Like}
\item
  \href{https://www.nytimes.com/live/2020/07/31/business/stock-market-today-coronavirus?action=click\&pgtype=Article\&state=default\&region=TOP_BANNER\&context=storylines_menu}{Economy}
\end{itemize}

Advertisement

\protect\hyperlink{after-top}{Continue reading the main story}

Supported by

\protect\hyperlink{after-sponsor}{Continue reading the main story}

\hypertarget{you-do-the-right-things-and-still-you-get-it}{%
\section{`You Do the Right Things, and Still You Get
It'}\label{you-do-the-right-things-and-still-you-get-it}}

A Texas family tried to ward off the virus. But as cases in the state
soared and debates about masks and distancing raged, there was only so
much they could control.

\includegraphics{https://static01.nyt.com/images/2020/07/27/multimedia/00virus-family/merlin_174540594_42136cfc-2c99-4023-beca-b1d821583575-videoSixteenByNine3000.jpg}

\href{https://www.nytimes.com/by/sheri-fink}{\includegraphics{https://static01.nyt.com/images/2018/08/24/multimedia/author-sheri-fink/author-sheri-fink-thumbLarge.png}}

By \href{https://www.nytimes.com/by/sheri-fink}{Sheri Fink}

\begin{itemize}
\item
  Published July 26, 2020Updated July 27, 2020
\item
  \begin{itemize}
  \item
  \item
  \item
  \item
  \item
  \item
  \end{itemize}
\end{itemize}

HOUSTON --- Elaine Roberts, a longtime bagger at a supermarket, tried to
be so careful. She put on gloves and stopped riding the bus to work,
instead relying on her father to drive her to keep their family safe.
She wore masks --- in space-themed fabrics stitched by her sister --- as
she stacked products on shelves, helped people to their cars and
retrieved carts from the parking lot.

But many of the customers at the Randalls store in a Houston suburb did
not wear them, she noticed, even as coronavirus cases in the state began
rising in early June. Gov. Greg Abbott, who had pushed to reopen
businesses in Texas, was refusing to make masks mandatory and for weeks
had blocked local officials from enforcing any mask requirements. The
grocery store only **** posted signs asking shoppers to wear them.

Ms. Roberts, 35, who has autism and lives with her parents, got sick
first, sneezing and coughing. Then her father, Paul, and mother, Sheryl,
who had been so cautious after the pandemic struck that their rare
ventures out were mostly for bird-watching in a nearly empty park, were
hospitalized with breathing problems.

Their cases were unusual: Sheryl Roberts, a sunny retired nurse,
experienced severe psychiatric symptoms that made doctors fear she was
suicidal, possibly an effect of the disease and medicines to treat it.
She is recovering, but her husband is critically ill, on a ventilator,
with failing kidneys and a mysterious paralysis that has afflicted about
a dozen others at Houston Methodist Hospital.

While no one can be certain how Elaine Roberts was infected, her older
sister, Sidra Roman, blamed grocery customers who she felt had put her
family in danger.

\includegraphics{https://static01.nyt.com/images/2020/07/24/multimedia/00virus-family-2/merlin_174742029_3bd5634c-6f92-4a3c-91f7-31d400e3423c-articleLarge.jpg?quality=75\&auto=webp\&disable=upscale}

``Wearing a piece of cloth, it's a little uncomfortable,'' she said.
``It's a lot less uncomfortable than ventilators, dialysis lines, all of
those things that have had to happen to my father. And it's not
necessarily you that's going to get sick and get hurt.''

``Whoever came to the grocery store and didn't wear a mask,'' she added,
``doesn't know this is going on.''

What happened to the Robertses is in many ways the story of Texas, one
of the nation's hot spots as coronavirus cases mount and deaths climb.
For weeks, politicians were divided over keeping the economy open,
citizens were polarized about wearing masks, **** doctors were warning
that careless behavior could imperil others, and families were put at
risk by their young.

Mr. Roberts, 67, is among the patients now packing intensive care units
across Texas and other parts of the Sun Belt. The surge in virus cases
here that took off in June first appeared to involve mostly younger
adults, causing milder illnesses doctors believed would respond to new
treatments. But the chain of infections that began with people under 40
--- many who socialized at bars or parties without masks or distancing
--- moved to essential workers like Ms. Roberts, and then to their
relatives.

``We thought this might be different, maybe with some of the things
we've learned,'' Dr. Pat Herlihy, chief of critical care at Baylor St.
Luke's Medical Center, said last week. But, he went on, ``We're right
there now with super, super sick people.''

The same is likely to befall hospitals in other areas where cases are
rising; Houston was among the cities at the leading edge of the summer
wave, and critical illnesses often lag new infections by weeks.

Nearly 11,000 confirmed coronavirus patients were in Texas hospitals as
of Wednesday, the last day for which complete data were available. It
was a record high, according to the state health department, five times
as many as the peak in the spring.

\hypertarget{latest-updates-global-coronavirus-outbreak}{%
\section{\texorpdfstring{\href{https://www.nytimes.com/2020/08/01/world/coronavirus-covid-19.html?action=click\&pgtype=Article\&state=default\&region=MAIN_CONTENT_1\&context=storylines_live_updates}{Latest
Updates: Global Coronavirus
Outbreak}}{Latest Updates: Global Coronavirus Outbreak}}\label{latest-updates-global-coronavirus-outbreak}}

Updated 2020-08-02T10:04:29.623Z

\begin{itemize}
\tightlist
\item
  \href{https://www.nytimes.com/2020/08/01/world/coronavirus-covid-19.html?action=click\&pgtype=Article\&state=default\&region=MAIN_CONTENT_1\&context=storylines_live_updates\#link-34047410}{The
  U.S. reels as July cases more than double the total of any other
  month.}
\item
  \href{https://www.nytimes.com/2020/08/01/world/coronavirus-covid-19.html?action=click\&pgtype=Article\&state=default\&region=MAIN_CONTENT_1\&context=storylines_live_updates\#link-780ec966}{Top
  U.S. officials work to break an impasse over the federal jobless
  benefit.}
\item
  \href{https://www.nytimes.com/2020/08/01/world/coronavirus-covid-19.html?action=click\&pgtype=Article\&state=default\&region=MAIN_CONTENT_1\&context=storylines_live_updates\#link-2bc8948}{Its
  outbreak untamed, Melbourne goes into even greater lockdown.}
\end{itemize}

\href{https://www.nytimes.com/2020/08/01/world/coronavirus-covid-19.html?action=click\&pgtype=Article\&state=default\&region=MAIN_CONTENT_1\&context=storylines_live_updates}{See
more updates}

More live coverage:
\href{https://www.nytimes.com/live/2020/07/31/business/stock-market-today-coronavirus?action=click\&pgtype=Article\&state=default\&region=MAIN_CONTENT_1\&context=storylines_live_updates}{Markets}

At Houston Methodist, the city's largest hospital, beds were filled
disproportionately with Hispanic patients and with multiple members of
families. There were people who believed they were invulnerable to the
virus and others, like the Robertses, who knew that they were not.
Coronavirus deaths across Methodist's hospital system have multiplied,
as they have elsewhere: 31 in May, 47 in June and 144 in the first three
weeks of July.

Administrators have created I.C.U. after I.C.U. to tend to the growing
number of severely ill patients who often require weeks of
resource-intensive treatment. In recent days, doctors were told to stop
offering a remedy used as a last resort --- treatment with a heart-lung
machine --- to any more patients because staffing was too stretched.

With patients on ventilators awaiting beds in I.C.U.s, physicians have
been pressed to move patients through as quickly as possible, including
urging families to make decisions about removing life support when there
is little chance of recovery.

Image

Dr. Faisal Masud, the head of critical care at Houston Methodist
Hospital.Credit...Erin Schaff/The New York Times

Dr. Herlihy and Dr. Faisal Masud, the head of critical care at the
Methodist hospital system, said that because so many patients were so
severely sick, they had been forced to turn away some transfers from
other institutions.

``I get desperate calls, desperate emails,'' Dr. Masud said. ``I have to
make the call as to who can come and not come in. That's a huge burden,
because in my heart, with my saying no, they will more than likely end
up dying.''

\hypertarget{masks-were-kind-of-50-50}{%
\subsection{Masks Were `Kind of 50-50'}\label{masks-were-kind-of-50-50}}

Elaine Roberts began working at the Randalls grocery store in Bellaire,
part of a larger chain, when she was 16. Nearly two decades later, she
is one of its longest-tenured employees.

Diagnosed in childhood with a form of autism that she says has made
learning difficult, she didn't speak until she was 8, when the words
came in a burst during a Disney World trip. But her parents have raised
her to be as independent as possible.

She completed a four-year vocational program after high school and
applied to countless other jobs over the years, to no avail. Outgoing
and chatty, she has a boyfriend whom she's known since elementary school
and a circle of good friends. She loves old television comedies and the
color pink.

``She's so sweet and very caring and will do anything you ask of her,''
said her manager, Cindy Fletcher.

To protect against the virus, Ms. Fletcher said, the store devotes many
hours a week to cleaning, and employees are asked to stay home if they
have viral symptoms.

Until late June, the company did not require patrons to wear face masks.
Postings asked customers to put them on, but ``it wasn't anything we had
to enforce,'' Ms. Fletcher said. ``It was kind of 50-50,'' she added,
with ``younger customers not as much.''

Image

Elaine Roberts began working at Randalls in Bellaire, Texas, when she
was 16.Credit...Erin Schaff/The New York Times

Image

Ms. Roberts isolated at home alone when her parents were both
hospitalized.Credit...Erin Schaff/The New York Times

Ms. Roberts had no choice about coming into close contact with shoppers,
whether they wore masks or not. ``I ended up sacking their groceries,''
she said. ``I couldn't say anything to them about it. I didn't want to
be bossy.''

Public health officials acknowledge that masks and social distancing are
not complete defenses against the virus, but
\href{https://www.ucsf.edu/news/2020/06/417906/still-confused-about-masks-heres-science-behind-how-face-masks-prevent}{studies
suggest} they can have a significant impact in protecting others. In
Harris County, which includes Houston, a local order directing
businesses to require people to wear masks went into effect on June 22
after the governor relented. A sign went up at the Randalls entrance
saying that masks were mandatory. Compliance, Ms. Fletcher says, has
been good.

But it was too late for the **** Roberts family.

Paul Roberts, a former musician and carpenter turned computer programmer
for NASA, now works at a software company. He and his wife, a retired
Methodist Hospital nurse who calls herself a ``glass half full person,''
ran an online fanzine and attended Comic Con events years ago. For
years, they gathered weekly with their two daughters, son-in-law and
now-7-year-old grandson for jigsaw puzzles and fierce games of Uno.

``They are amazing nerds,'' Ms. Roman, 38, said of her parents.

Sheryl Roberts, 65, understood the perils of the pandemic --- she had
diabetes, asthma and heart disease, which could put her at higher risk.
Her husband had chronic lung disease and a stent to open a blocked
coronary artery.

``We have been so careful, so very careful, and stayed away from
people,'' Ms. Roberts said. Her husband began working from home in the
spring when Washington State, New York and then other areas around the
country were hit hard. Mr. Roberts occasionally made a supermarket run
during ``senior'' hour; the couple's only ``big, hot date'' in recent
months, Ms. Roberts said, was to view wildflowers from their car.

Their younger daughter was diligent as well. But then she came back from
work sneezing one day in **** mid-June and thought it was allergies.
Soon she had a cough, fever, headaches and diarrhea, and lost her senses
of taste and smell, telltale symptoms of the coronavirus.

``She told me, `I don't know what's going on, Mom, but I wore a mask, I
wore gloves, I washed my hands,''' Ms. Roberts said. ``You do the right
things, and still you get it.''

Elaine Roberts, who tested positive for the coronavirus, did not become
seriously ill. But for her parents, it would be much worse.

\hypertarget{daughter-sister-caretaker}{%
\subsection{Daughter, Sister,
Caretaker}\label{daughter-sister-caretaker}}

Mr. Roberts and his wife started sneezing, then coughing, just like
their daughter, and developed fevers and severe body aches. Then he got
``awfully sick, awfully quickly,'' Sheryl Roberts recalled. He became
confused on June 22. Alarmed, she tested his oxygen level. It was low,
and she called her older daughter to take him to an emergency care
center, the second visit in two days.

Before he left, his wife asked him to make a promise.

Image

Sheryl Roberts at Houston Methodist earlier this month. She had severe
psychiatric symptoms, which may have been triggered by the illness and
medications she was given to treat it.Credit...Erin Schaff/The New York
Times

``He and I made a deal,'' she recalled. ``He was going to get well, and
I was going to do the same. We were going to live through this.'' But a
few days later, his lungs ravaged by the virus, Mr. Roberts was put on a
ventilator. ``He cratered,'' his wife said.

\href{https://www.nytimes.com/news-event/coronavirus?action=click\&pgtype=Article\&state=default\&region=MAIN_CONTENT_3\&context=storylines_faq}{}

\hypertarget{the-coronavirus-outbreak-}{%
\subsubsection{The Coronavirus Outbreak
›}\label{the-coronavirus-outbreak-}}

\hypertarget{frequently-asked-questions}{%
\paragraph{Frequently Asked
Questions}\label{frequently-asked-questions}}

Updated July 27, 2020

\begin{itemize}
\item ~
  \hypertarget{should-i-refinance-my-mortgage}{%
  \paragraph{Should I refinance my
  mortgage?}\label{should-i-refinance-my-mortgage}}

  \begin{itemize}
  \tightlist
  \item
    \href{https://www.nytimes.com/article/coronavirus-money-unemployment.html?action=click\&pgtype=Article\&state=default\&region=MAIN_CONTENT_3\&context=storylines_faq}{It
    could be a good idea,} because mortgage rates have
    \href{https://www.nytimes.com/2020/07/16/business/mortgage-rates-below-3-percent.html?action=click\&pgtype=Article\&state=default\&region=MAIN_CONTENT_3\&context=storylines_faq}{never
    been lower.} Refinancing requests have pushed mortgage applications
    to some of the highest levels since 2008, so be prepared to get in
    line. But defaults are also up, so if you're thinking about buying a
    home, be aware that some lenders have tightened their standards.
  \end{itemize}
\item ~
  \hypertarget{what-is-school-going-to-look-like-in-september}{%
  \paragraph{What is school going to look like in
  September?}\label{what-is-school-going-to-look-like-in-september}}

  \begin{itemize}
  \tightlist
  \item
    It is unlikely that many schools will return to a normal schedule
    this fall, requiring the grind of
    \href{https://www.nytimes.com/2020/06/05/us/coronavirus-education-lost-learning.html?action=click\&pgtype=Article\&state=default\&region=MAIN_CONTENT_3\&context=storylines_faq}{online
    learning},
    \href{https://www.nytimes.com/2020/05/29/us/coronavirus-child-care-centers.html?action=click\&pgtype=Article\&state=default\&region=MAIN_CONTENT_3\&context=storylines_faq}{makeshift
    child care} and
    \href{https://www.nytimes.com/2020/06/03/business/economy/coronavirus-working-women.html?action=click\&pgtype=Article\&state=default\&region=MAIN_CONTENT_3\&context=storylines_faq}{stunted
    workdays} to continue. California's two largest public school
    districts --- Los Angeles and San Diego --- said on July 13, that
    \href{https://www.nytimes.com/2020/07/13/us/lausd-san-diego-school-reopening.html?action=click\&pgtype=Article\&state=default\&region=MAIN_CONTENT_3\&context=storylines_faq}{instruction
    will be remote-only in the fall}, citing concerns that surging
    coronavirus infections in their areas pose too dire a risk for
    students and teachers. Together, the two districts enroll some
    825,000 students. They are the largest in the country so far to
    abandon plans for even a partial physical return to classrooms when
    they reopen in August. For other districts, the solution won't be an
    all-or-nothing approach.
    \href{https://bioethics.jhu.edu/research-and-outreach/projects/eschool-initiative/school-policy-tracker/}{Many
    systems}, including the nation's largest, New York City, are
    devising
    \href{https://www.nytimes.com/2020/06/26/us/coronavirus-schools-reopen-fall.html?action=click\&pgtype=Article\&state=default\&region=MAIN_CONTENT_3\&context=storylines_faq}{hybrid
    plans} that involve spending some days in classrooms and other days
    online. There's no national policy on this yet, so check with your
    municipal school system regularly to see what is happening in your
    community.
  \end{itemize}
\item ~
  \hypertarget{is-the-coronavirus-airborne}{%
  \paragraph{Is the coronavirus
  airborne?}\label{is-the-coronavirus-airborne}}

  \begin{itemize}
  \tightlist
  \item
    The coronavirus
    \href{https://www.nytimes.com/2020/07/04/health/239-experts-with-one-big-claim-the-coronavirus-is-airborne.html?action=click\&pgtype=Article\&state=default\&region=MAIN_CONTENT_3\&context=storylines_faq}{can
    stay aloft for hours in tiny droplets in stagnant air}, infecting
    people as they inhale, mounting scientific evidence suggests. This
    risk is highest in crowded indoor spaces with poor ventilation, and
    may help explain super-spreading events reported in meatpacking
    plants, churches and restaurants.
    \href{https://www.nytimes.com/2020/07/06/health/coronavirus-airborne-aerosols.html?action=click\&pgtype=Article\&state=default\&region=MAIN_CONTENT_3\&context=storylines_faq}{It's
    unclear how often the virus is spread} via these tiny droplets, or
    aerosols, compared with larger droplets that are expelled when a
    sick person coughs or sneezes, or transmitted through contact with
    contaminated surfaces, said Linsey Marr, an aerosol expert at
    Virginia Tech. Aerosols are released even when a person without
    symptoms exhales, talks or sings, according to Dr. Marr and more
    than 200 other experts, who
    \href{https://academic.oup.com/cid/article/doi/10.1093/cid/ciaa939/5867798}{have
    outlined the evidence in an open letter to the World Health
    Organization}.
  \end{itemize}
\item ~
  \hypertarget{what-are-the-symptoms-of-coronavirus}{%
  \paragraph{What are the symptoms of
  coronavirus?}\label{what-are-the-symptoms-of-coronavirus}}

  \begin{itemize}
  \tightlist
  \item
    Common symptoms
    \href{https://www.nytimes.com/article/symptoms-coronavirus.html?action=click\&pgtype=Article\&state=default\&region=MAIN_CONTENT_3\&context=storylines_faq}{include
    fever, a dry cough, fatigue and difficulty breathing or shortness of
    breath.} Some of these symptoms overlap with those of the flu,
    making detection difficult, but runny noses and stuffy sinuses are
    less common.
    \href{https://www.nytimes.com/2020/04/27/health/coronavirus-symptoms-cdc.html?action=click\&pgtype=Article\&state=default\&region=MAIN_CONTENT_3\&context=storylines_faq}{The
    C.D.C. has also} added chills, muscle pain, sore throat, headache
    and a new loss of the sense of taste or smell as symptoms to look
    out for. Most people fall ill five to seven days after exposure, but
    symptoms may appear in as few as two days or as many as 14 days.
  \end{itemize}
\item ~
  \hypertarget{does-asymptomatic-transmission-of-covid-19-happen}{%
  \paragraph{Does asymptomatic transmission of Covid-19
  happen?}\label{does-asymptomatic-transmission-of-covid-19-happen}}

  \begin{itemize}
  \tightlist
  \item
    So far, the evidence seems to show it does. A widely cited
    \href{https://www.nature.com/articles/s41591-020-0869-5}{paper}
    published in April suggests that people are most infectious about
    two days before the onset of coronavirus symptoms and estimated that
    44 percent of new infections were a result of transmission from
    people who were not yet showing symptoms. Recently, a top expert at
    the World Health Organization stated that transmission of the
    coronavirus by people who did not have symptoms was ``very rare,''
    \href{https://www.nytimes.com/2020/06/09/world/coronavirus-updates.html?action=click\&pgtype=Article\&state=default\&region=MAIN_CONTENT_3\&context=storylines_faq\#link-1f302e21}{but
    she later walked back that statement.}
  \end{itemize}
\end{itemize}

Within a week, she, too, was admitted to Methodist after becoming short
of breath.

Neither daughter could see their parents: Methodist, like many other
hospitals around the country, blocked visitors to contain the virus's
spread. The couple were isolated in separate buildings, and could not
communicate with each other. Mr. Roberts was gravely ill, and his wife's
condition was deteriorating. Ms. Roman, an oil industry engineer, tried
to fill the gap.

``I've known for a very long time that when the time comes, I get to
step up,'' said Ms. Roman, 38. ``I have to take care of my parents. I
have to take care of my sister. I just didn't expect it all to converge
at once.''

After about a week in the hospital, there was a crisis: Ms. Roberts
became delirious and repeatedly pulled the tubing that supplied oxygen
out from under her nose. Doctors put restraints on her, stationed a
sitter outside her room and called Ms. Roman to say they thought her
mother's turmoil might be a result of medication side effects combined
with her illness.

Ms. Roman called her sister in tears. ``I said, `I'm scared, Lainie, I'm
scared.' She said, `I am, too.'''

After Ms. Roberts's steroid dose was cut, the symptoms resolved over a
couple of days. **** ``They said that I had said that I was going to
kill myself,'' Ms. Roberts recalled the doctors telling her. ``This is
not me.''

Her breathing gradually improved, and she did not need a ventilator. A
few days later, she said she was keeping herself going by imagining a
trip on her bucket list: taking her husband to see macaws in the Amazon.

Image

Dr. Al-Saadi and a resident physician preparing to replace Mr. Roberts's
dialysis catheter.Credit...Erin Schaff/The New York Times

Doctors called Ms. Roman with updates on her father and requests to give
consent for procedures, including a catheter for emergency kidney
dialysis. He received steroids, which work against inflammation, and
experimental medications. Mr. Roberts was put under deep sedation and
given drugs to paralyze him so the ventilator could work more
effectively.

There were some glimmers of hope --- Mr. Roberts's lungs seemed to be
healing --- but whenever the medical team reduced the sedation over the
next few days, his blood pressure rose and his heart raced, signs of
agitation. On July 9, Dr. Mukhtar Al-Saadi called Ms. Roman with an
update. ``It was very difficult for us to wake him up meaningfully to
see if he can breathe on his own,'' the doctor said.

\hypertarget{believe-me-its-real}{%
\subsection{`Believe Me, It's Real'}\label{believe-me-its-real}}

Last week, after she was discharged and just about to be wheeled out of
the hospital, Ms. Roberts received a terrifying call. Her husband was
still not waking up or moving, and doctors believed a massive stroke or
another neurological problem was the likely reason. Ms. Roberts and her
daughters gathered that night, discussing the difficult decisions they
might have to make.

Image

Extended intravenous tubes connect Mr. Roberts to his medications
outside the door.Credit...Erin Schaff/The New York Times

``Do we just let him go, if he's brain-dead?'' Sheryl Roberts said they
wondered. As they considered what the ``very bright, very proud man''
would want, Ms. Roman said, the three women wept.

A brain scan the next day showed that he had not had a stroke, but
additional studies were delayed to avoid exposing the few available
technicians to the virus. On Friday, Dr. R. Glenn Smith, a neurology
attending physician, performed neuromuscular testing that indicated
severe damage to Mr. Roberts's nerve coverings.

About a dozen other patients at the hospital have developed a paralysis
or profound weakness that doctors believe may be a complication of the
virus, according to Dr. Smith. Doctors had already begun treating Mr.
Roberts with a medication used for Guillain-Barre syndrome, a similar
paralyzing disorder that occurs rarely after some viral infections.

They don't know how much function he will be able to regain; he has
begun showing some limited progress. On Tuesday a staff member brought a
tablet into Mr. Roberts's room and made a video connection. ``He nodded,
I chatted,'' Ms. Roberts said. ``He blew me a kiss.''

While her husband waits for a bed in a long-term acute care unit to
begin rehabilitation, he remains on a ventilator. Even if there are no
more challenges, his recovery will take months, Dr. Smith said.

Image

Mr. Roberts is still being treated at the hospital while the family
awaits an opening in a long-term acute care unit.Credit...Erin
Schaff/The New York Times

"It's going to be slow,'' Ms. Roman said. ``It's not going to be easy.''
But, she added, ``it seems like he's still Dad upstairs so I'll take
it.''

The family's ordeal has made her mother more outspoken about the toll of
the pandemic. The misinformation and confusion about the virus that she
sees on social media scares her, she said. ``The ignorance kills me.
`It's really not that bad, it's not really fatal.'''

She said she now responds to such statements. ``I'm always happy to show
right up and say, `You know, I just lived through it --- believe me,
it's real.'''

She still requires oxygen, and Elaine Roberts is taking care of her,
cooking meals, helping her shower and maintaining her breathing device.
When her parents were both gone, she assumed new household tasks. ``My
youngest has proved to me she's far more capable of things than I ever
dreamed,'' Ms. Roberts said. ``I'm so proud of her.''

On Monday, Elaine Roberts has a coronavirus test scheduled. If it is
negative, she hopes to go back to work at Randalls.

Advertisement

\protect\hyperlink{after-bottom}{Continue reading the main story}

\hypertarget{site-index}{%
\subsection{Site Index}\label{site-index}}

\hypertarget{site-information-navigation}{%
\subsection{Site Information
Navigation}\label{site-information-navigation}}

\begin{itemize}
\tightlist
\item
  \href{https://help.nytimes.com/hc/en-us/articles/115014792127-Copyright-notice}{©~2020~The
  New York Times Company}
\end{itemize}

\begin{itemize}
\tightlist
\item
  \href{https://www.nytco.com/}{NYTCo}
\item
  \href{https://help.nytimes.com/hc/en-us/articles/115015385887-Contact-Us}{Contact
  Us}
\item
  \href{https://www.nytco.com/careers/}{Work with us}
\item
  \href{https://nytmediakit.com/}{Advertise}
\item
  \href{http://www.tbrandstudio.com/}{T Brand Studio}
\item
  \href{https://www.nytimes.com/privacy/cookie-policy\#how-do-i-manage-trackers}{Your
  Ad Choices}
\item
  \href{https://www.nytimes.com/privacy}{Privacy}
\item
  \href{https://help.nytimes.com/hc/en-us/articles/115014893428-Terms-of-service}{Terms
  of Service}
\item
  \href{https://help.nytimes.com/hc/en-us/articles/115014893968-Terms-of-sale}{Terms
  of Sale}
\item
  \href{https://spiderbites.nytimes.com}{Site Map}
\item
  \href{https://help.nytimes.com/hc/en-us}{Help}
\item
  \href{https://www.nytimes.com/subscription?campaignId=37WXW}{Subscriptions}
\end{itemize}
