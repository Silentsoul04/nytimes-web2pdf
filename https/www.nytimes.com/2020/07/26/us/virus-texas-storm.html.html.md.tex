Sections

SEARCH

\protect\hyperlink{site-content}{Skip to
content}\protect\hyperlink{site-index}{Skip to site index}

\href{https://www.nytimes.com/section/us}{U.S.}

\href{https://myaccount.nytimes.com/auth/login?response_type=cookie\&client_id=vi}{}

\href{https://www.nytimes.com/section/todayspaper}{Today's Paper}

\href{/section/us}{U.S.}\textbar{}Hurricane's Choice for Texans: Shelter
From the Virus or the Storm

\url{https://nyti.ms/30TqhW6}

\begin{itemize}
\item
\item
\item
\item
\item
\end{itemize}

\href{https://www.nytimes.com/news-event/coronavirus?action=click\&pgtype=Article\&state=default\&region=TOP_BANNER\&context=storylines_menu}{The
Coronavirus Outbreak}

\begin{itemize}
\tightlist
\item
  live\href{https://www.nytimes.com/2020/08/01/world/coronavirus-covid-19.html?action=click\&pgtype=Article\&state=default\&region=TOP_BANNER\&context=storylines_menu}{Latest
  Updates}
\item
  \href{https://www.nytimes.com/interactive/2020/us/coronavirus-us-cases.html?action=click\&pgtype=Article\&state=default\&region=TOP_BANNER\&context=storylines_menu}{Maps
  and Cases}
\item
  \href{https://www.nytimes.com/interactive/2020/science/coronavirus-vaccine-tracker.html?action=click\&pgtype=Article\&state=default\&region=TOP_BANNER\&context=storylines_menu}{Vaccine
  Tracker}
\item
  \href{https://www.nytimes.com/interactive/2020/07/29/us/schools-reopening-coronavirus.html?action=click\&pgtype=Article\&state=default\&region=TOP_BANNER\&context=storylines_menu}{What
  School May Look Like}
\item
  \href{https://www.nytimes.com/live/2020/07/31/business/stock-market-today-coronavirus?action=click\&pgtype=Article\&state=default\&region=TOP_BANNER\&context=storylines_menu}{Economy}
\end{itemize}

Advertisement

\protect\hyperlink{after-top}{Continue reading the main story}

Supported by

\protect\hyperlink{after-sponsor}{Continue reading the main story}

\hypertarget{hurricanes-choice-for-texans-shelter-from-the-virus-or-the-storm}{%
\section{Hurricane's Choice for Texans: Shelter From the Virus or the
Storm}\label{hurricanes-choice-for-texans-shelter-from-the-virus-or-the-storm}}

Hurricane Hanna prompted officials and residents alike to rethink how
and where to ride out a dangerous storm during a pandemic.

\includegraphics{https://static01.nyt.com/images/2020/07/26/us/26virus-texasstorm-1/merlin_174981702_376b70bf-a409-4c37-988a-4df716664853-articleLarge.jpg?quality=75\&auto=webp\&disable=upscale}

By \href{https://www.nytimes.com/by/edgar-sandoval}{Edgar Sandoval}

\begin{itemize}
\item
  Published July 26, 2020Updated July 29, 2020
\item
  \begin{itemize}
  \item
  \item
  \item
  \item
  \item
  \end{itemize}
\end{itemize}

CORPUS CHRISTI, Texas ---~Bartt Howe's boat was his refuge from the
pandemic. Battling diabetes and H.I.V., he knew that catching the
coronavirus as well could kill him, so he had been living alone on the
docked boat for three months.

Then Hurricane Hanna began to slam the Texas coast on Saturday, forcing
Mr. Howe to trade one deadly menace for another: To avoid injury or
death in the hurricane, he had to risk infection ashore.

``I had managed to stay safe all this time, but the storm kicked me out
of my boat,'' he said with a hint of resignation. ``Now here I am, back
on land, on borrowed time.''

Corpus Christi, about 160 miles north of the Texas-Mexico border, was
already wrestling with a worsening virus outbreak when Hanna, a Category
1 hurricane, made landfall at about 5 p.m. Saturday. Residents like Mr.
Howe and area officials have had to figure out --- quickly --- how to
cope with two dueling crises, each complicating the response to the
other.

Much of the surge in Covid-19 cases in the region
\href{https://www.nytimes.com/2020/07/11/us/coronavirus-texas-corpus-christi.html}{has
been attributed to visitors from bigger cities like Houston} who flocked
to the area's beaches when case counts were still low, officials said.
More than 10,000 people in Nueces County, which includes Corpus Christi,
have now
\href{https://www.nytimes.com/interactive/2020/us/texas-coronavirus-cases.html}{been
infected with the virus}, and at least 140 people have died.

``When I saw that the hurricane was headed our way, I thought, we have
enough problems,'' said the mayor of Corpus Christi, Joe McComb.

\hypertarget{latest-updates-global-coronavirus-outbreak}{%
\section{\texorpdfstring{\href{https://www.nytimes.com/2020/08/01/world/coronavirus-covid-19.html?action=click\&pgtype=Article\&state=default\&region=MAIN_CONTENT_1\&context=storylines_live_updates}{Latest
Updates: Global Coronavirus
Outbreak}}{Latest Updates: Global Coronavirus Outbreak}}\label{latest-updates-global-coronavirus-outbreak}}

Updated 2020-08-02T10:04:29.623Z

\begin{itemize}
\tightlist
\item
  \href{https://www.nytimes.com/2020/08/01/world/coronavirus-covid-19.html?action=click\&pgtype=Article\&state=default\&region=MAIN_CONTENT_1\&context=storylines_live_updates\#link-34047410}{The
  U.S. reels as July cases more than double the total of any other
  month.}
\item
  \href{https://www.nytimes.com/2020/08/01/world/coronavirus-covid-19.html?action=click\&pgtype=Article\&state=default\&region=MAIN_CONTENT_1\&context=storylines_live_updates\#link-780ec966}{Top
  U.S. officials work to break an impasse over the federal jobless
  benefit.}
\item
  \href{https://www.nytimes.com/2020/08/01/world/coronavirus-covid-19.html?action=click\&pgtype=Article\&state=default\&region=MAIN_CONTENT_1\&context=storylines_live_updates\#link-2bc8948}{Its
  outbreak untamed, Melbourne goes into even greater lockdown.}
\end{itemize}

\href{https://www.nytimes.com/2020/08/01/world/coronavirus-covid-19.html?action=click\&pgtype=Article\&state=default\&region=MAIN_CONTENT_1\&context=storylines_live_updates}{See
more updates}

More live coverage:
\href{https://www.nytimes.com/live/2020/07/31/business/stock-market-today-coronavirus?action=click\&pgtype=Article\&state=default\&region=MAIN_CONTENT_1\&context=storylines_live_updates}{Markets}

In ordinary times, city officials would ask people in seaside and
flood-prone areas to evacuate and seek shelter with relatives or in
emergency shelters ---~places where people share bathrooms and tight
quarters, health officials said. But fear of contagion has thrown old
protocols out the window.

So how do you keep people safe through a hurricane when the coronavirus
shows no signs of abating? ``Very carefully,'' said Annette Rodriguez,
the county public health director.

Hanna has not displaced many people, as hurricanes go. Only about 10
people sought shelter in the county Saturday night, Ms. Rodriguez said.
The relatively low-stakes storm allowed area officials to assess how to
help people evacuate safely while diminishing the spread of the virus.

``Having two events tied together, it is just a huge challenge,'' Ms.
Rodriguez said. ``It was definitely a good trial run.''

\includegraphics{https://static01.nyt.com/images/2020/07/26/us/26virus-texasstorm3/26virus-texasstorm3-articleLarge.jpg?quality=75\&auto=webp\&disable=upscale}

When the next hurricane comes, officials plan to start evacuation
procedures five days in advance, instead of the standard three, Ms.
Rodriguez said. Officials also plan to seal off alternate seats on buses
transporting evacuees to safer areas, and increase the number of trips
the buses make to compensate.

Once evacuees reach a shelter --- usually a high school gymnasium ---
officials would ensure that they wear masks and keep their distance from
one another, Ms. Rodriguez said, adding that shelters ``would have a
separate section for anyone sick.''

``We don't want to leave anybody behind,'' she said. ``But we also want
to stop the virus from spreading.''

Officials acknowledged that the guidance they gave the public could be
confusing and at times conflicting.

``Our message was, take care of yourself, but if you need to seek
shelter, bring a mask or three,'' Mr. McComb said. ``You have to protect
yourself from a storm, but at the same time, we don't need more cases of
the coronavirus.''

Residents woke on Sunday to a battered region. The Red Cross reported
some severe flooding in coastal areas, widespread power outages and
property damage, including roofs blown off houses, but no severe
injuries.

\href{https://www.nytimes.com/news-event/coronavirus?action=click\&pgtype=Article\&state=default\&region=MAIN_CONTENT_3\&context=storylines_faq}{}

\hypertarget{the-coronavirus-outbreak-}{%
\subsubsection{The Coronavirus Outbreak
›}\label{the-coronavirus-outbreak-}}

\hypertarget{frequently-asked-questions}{%
\paragraph{Frequently Asked
Questions}\label{frequently-asked-questions}}

Updated July 27, 2020

\begin{itemize}
\item ~
  \hypertarget{should-i-refinance-my-mortgage}{%
  \paragraph{Should I refinance my
  mortgage?}\label{should-i-refinance-my-mortgage}}

  \begin{itemize}
  \tightlist
  \item
    \href{https://www.nytimes.com/article/coronavirus-money-unemployment.html?action=click\&pgtype=Article\&state=default\&region=MAIN_CONTENT_3\&context=storylines_faq}{It
    could be a good idea,} because mortgage rates have
    \href{https://www.nytimes.com/2020/07/16/business/mortgage-rates-below-3-percent.html?action=click\&pgtype=Article\&state=default\&region=MAIN_CONTENT_3\&context=storylines_faq}{never
    been lower.} Refinancing requests have pushed mortgage applications
    to some of the highest levels since 2008, so be prepared to get in
    line. But defaults are also up, so if you're thinking about buying a
    home, be aware that some lenders have tightened their standards.
  \end{itemize}
\item ~
  \hypertarget{what-is-school-going-to-look-like-in-september}{%
  \paragraph{What is school going to look like in
  September?}\label{what-is-school-going-to-look-like-in-september}}

  \begin{itemize}
  \tightlist
  \item
    It is unlikely that many schools will return to a normal schedule
    this fall, requiring the grind of
    \href{https://www.nytimes.com/2020/06/05/us/coronavirus-education-lost-learning.html?action=click\&pgtype=Article\&state=default\&region=MAIN_CONTENT_3\&context=storylines_faq}{online
    learning},
    \href{https://www.nytimes.com/2020/05/29/us/coronavirus-child-care-centers.html?action=click\&pgtype=Article\&state=default\&region=MAIN_CONTENT_3\&context=storylines_faq}{makeshift
    child care} and
    \href{https://www.nytimes.com/2020/06/03/business/economy/coronavirus-working-women.html?action=click\&pgtype=Article\&state=default\&region=MAIN_CONTENT_3\&context=storylines_faq}{stunted
    workdays} to continue. California's two largest public school
    districts --- Los Angeles and San Diego --- said on July 13, that
    \href{https://www.nytimes.com/2020/07/13/us/lausd-san-diego-school-reopening.html?action=click\&pgtype=Article\&state=default\&region=MAIN_CONTENT_3\&context=storylines_faq}{instruction
    will be remote-only in the fall}, citing concerns that surging
    coronavirus infections in their areas pose too dire a risk for
    students and teachers. Together, the two districts enroll some
    825,000 students. They are the largest in the country so far to
    abandon plans for even a partial physical return to classrooms when
    they reopen in August. For other districts, the solution won't be an
    all-or-nothing approach.
    \href{https://bioethics.jhu.edu/research-and-outreach/projects/eschool-initiative/school-policy-tracker/}{Many
    systems}, including the nation's largest, New York City, are
    devising
    \href{https://www.nytimes.com/2020/06/26/us/coronavirus-schools-reopen-fall.html?action=click\&pgtype=Article\&state=default\&region=MAIN_CONTENT_3\&context=storylines_faq}{hybrid
    plans} that involve spending some days in classrooms and other days
    online. There's no national policy on this yet, so check with your
    municipal school system regularly to see what is happening in your
    community.
  \end{itemize}
\item ~
  \hypertarget{is-the-coronavirus-airborne}{%
  \paragraph{Is the coronavirus
  airborne?}\label{is-the-coronavirus-airborne}}

  \begin{itemize}
  \tightlist
  \item
    The coronavirus
    \href{https://www.nytimes.com/2020/07/04/health/239-experts-with-one-big-claim-the-coronavirus-is-airborne.html?action=click\&pgtype=Article\&state=default\&region=MAIN_CONTENT_3\&context=storylines_faq}{can
    stay aloft for hours in tiny droplets in stagnant air}, infecting
    people as they inhale, mounting scientific evidence suggests. This
    risk is highest in crowded indoor spaces with poor ventilation, and
    may help explain super-spreading events reported in meatpacking
    plants, churches and restaurants.
    \href{https://www.nytimes.com/2020/07/06/health/coronavirus-airborne-aerosols.html?action=click\&pgtype=Article\&state=default\&region=MAIN_CONTENT_3\&context=storylines_faq}{It's
    unclear how often the virus is spread} via these tiny droplets, or
    aerosols, compared with larger droplets that are expelled when a
    sick person coughs or sneezes, or transmitted through contact with
    contaminated surfaces, said Linsey Marr, an aerosol expert at
    Virginia Tech. Aerosols are released even when a person without
    symptoms exhales, talks or sings, according to Dr. Marr and more
    than 200 other experts, who
    \href{https://academic.oup.com/cid/article/doi/10.1093/cid/ciaa939/5867798}{have
    outlined the evidence in an open letter to the World Health
    Organization}.
  \end{itemize}
\item ~
  \hypertarget{what-are-the-symptoms-of-coronavirus}{%
  \paragraph{What are the symptoms of
  coronavirus?}\label{what-are-the-symptoms-of-coronavirus}}

  \begin{itemize}
  \tightlist
  \item
    Common symptoms
    \href{https://www.nytimes.com/article/symptoms-coronavirus.html?action=click\&pgtype=Article\&state=default\&region=MAIN_CONTENT_3\&context=storylines_faq}{include
    fever, a dry cough, fatigue and difficulty breathing or shortness of
    breath.} Some of these symptoms overlap with those of the flu,
    making detection difficult, but runny noses and stuffy sinuses are
    less common.
    \href{https://www.nytimes.com/2020/04/27/health/coronavirus-symptoms-cdc.html?action=click\&pgtype=Article\&state=default\&region=MAIN_CONTENT_3\&context=storylines_faq}{The
    C.D.C. has also} added chills, muscle pain, sore throat, headache
    and a new loss of the sense of taste or smell as symptoms to look
    out for. Most people fall ill five to seven days after exposure, but
    symptoms may appear in as few as two days or as many as 14 days.
  \end{itemize}
\item ~
  \hypertarget{does-asymptomatic-transmission-of-covid-19-happen}{%
  \paragraph{Does asymptomatic transmission of Covid-19
  happen?}\label{does-asymptomatic-transmission-of-covid-19-happen}}

  \begin{itemize}
  \tightlist
  \item
    So far, the evidence seems to show it does. A widely cited
    \href{https://www.nature.com/articles/s41591-020-0869-5}{paper}
    published in April suggests that people are most infectious about
    two days before the onset of coronavirus symptoms and estimated that
    44 percent of new infections were a result of transmission from
    people who were not yet showing symptoms. Recently, a top expert at
    the World Health Organization stated that transmission of the
    coronavirus by people who did not have symptoms was ``very rare,''
    \href{https://www.nytimes.com/2020/06/09/world/coronavirus-updates.html?action=click\&pgtype=Article\&state=default\&region=MAIN_CONTENT_3\&context=storylines_faq\#link-1f302e21}{but
    she later walked back that statement.}
  \end{itemize}
\end{itemize}

Mr. Howe, 49, was one of the handful of people who sought shelter
Saturday night at a high school in Kingsville, a rural town about 45
miles from Corpus Christi. He returned to Harbor El Sol Marina in Corpus
Christi Bay Sunday afternoon to check on his beloved 27-foot boat, Sera
Sera, a name he found ironic now.

``What will be, will be,'' he said. ``And that's how it is.''

As the storm bore down on Saturday, he initially intended to ride it out
on board, he said, even as his marina neighbors fled the aggressive sea
surges. By 11:30 p.m., though, the waves were hammering the boat hard.

``I couldn't even see the sea wall ---~that's how bad it got,'' he said
on Sunday, a green mask covering his mouth and nose.

``I knew that if I got it, I would die,'' he said of the virus. ``But I
had no choice. I knew it was time to go.''

Image

Kendra Travino, standing, and Raymond Maddox surveyed the wreckage on
Sunday at a marina in Corpus Christi, Texas, where their houseboat was
destroyed by the storm.Credit...Tamir Kalifa for The New York Times

A fellow boater who managed to reach him by phone alerted emergency
responders to his situation, and they threw a rope his way and got him
to safety, Mr. Howe recalled.

Johnny Heath, 42, joined Mr. Howe at the marina Sunday afternoon and
spotted his own boat, the Classified Cat, floating behind some dock
debris and mangled vessels. It was missing a few walls, and water had
damaged most of its furnishings, he could tell from where he stood.

``Everything is ruined,'' Mr. Heath said. ``I was working on that boat,
was trying to make it electric. I was planning on living there for the
rest of my life. Now look at it.

``But at least I'm still alive,'' he said with a shrug.

John Nolan, 55, who owns two boats in the marina, said he'd had a brush
with the virus months ago, and had recovered quickly from a mild fever.
His economic losses from the storm will take a lot longer, he said.

``They are in bad shape, but still floating, better than others,'' Mr.
Nolan said of his boats. Not far away, a few more worried residents
picked personal belongings from a pile of debris. A woman found a pair
of tennis shoes and a man grabbed a helmet.

A woman who Mr. Howe knows only as Sheila called to him, ``Hey, Bartt!
You made it out alive!''

``I did ---~I survived,'' he said, smiling behind his mask. ``At least,
this time.''

Advertisement

\protect\hyperlink{after-bottom}{Continue reading the main story}

\hypertarget{site-index}{%
\subsection{Site Index}\label{site-index}}

\hypertarget{site-information-navigation}{%
\subsection{Site Information
Navigation}\label{site-information-navigation}}

\begin{itemize}
\tightlist
\item
  \href{https://help.nytimes.com/hc/en-us/articles/115014792127-Copyright-notice}{©~2020~The
  New York Times Company}
\end{itemize}

\begin{itemize}
\tightlist
\item
  \href{https://www.nytco.com/}{NYTCo}
\item
  \href{https://help.nytimes.com/hc/en-us/articles/115015385887-Contact-Us}{Contact
  Us}
\item
  \href{https://www.nytco.com/careers/}{Work with us}
\item
  \href{https://nytmediakit.com/}{Advertise}
\item
  \href{http://www.tbrandstudio.com/}{T Brand Studio}
\item
  \href{https://www.nytimes.com/privacy/cookie-policy\#how-do-i-manage-trackers}{Your
  Ad Choices}
\item
  \href{https://www.nytimes.com/privacy}{Privacy}
\item
  \href{https://help.nytimes.com/hc/en-us/articles/115014893428-Terms-of-service}{Terms
  of Service}
\item
  \href{https://help.nytimes.com/hc/en-us/articles/115014893968-Terms-of-sale}{Terms
  of Sale}
\item
  \href{https://spiderbites.nytimes.com}{Site Map}
\item
  \href{https://help.nytimes.com/hc/en-us}{Help}
\item
  \href{https://www.nytimes.com/subscription?campaignId=37WXW}{Subscriptions}
\end{itemize}
