Sections

SEARCH

\protect\hyperlink{site-content}{Skip to
content}\protect\hyperlink{site-index}{Skip to site index}

\href{https://www.nytimes.com/section/world/europe}{Europe}

\href{https://myaccount.nytimes.com/auth/login?response_type=cookie\&client_id=vi}{}

\href{https://www.nytimes.com/section/todayspaper}{Today's Paper}

\href{/section/world/europe}{Europe}\textbar{}For French-Algerian
Families, Virus Disrupts Cherished Summer Ritual

\url{https://nyti.ms/3hBGO80}

\begin{itemize}
\item
\item
\item
\item
\item
\item
\end{itemize}

\href{https://www.nytimes.com/news-event/coronavirus?action=click\&pgtype=Article\&state=default\&region=TOP_BANNER\&context=storylines_menu}{The
Coronavirus Outbreak}

\begin{itemize}
\tightlist
\item
  live\href{https://www.nytimes.com/2020/08/01/world/coronavirus-covid-19.html?action=click\&pgtype=Article\&state=default\&region=TOP_BANNER\&context=storylines_menu}{Latest
  Updates}
\item
  \href{https://www.nytimes.com/interactive/2020/us/coronavirus-us-cases.html?action=click\&pgtype=Article\&state=default\&region=TOP_BANNER\&context=storylines_menu}{Maps
  and Cases}
\item
  \href{https://www.nytimes.com/interactive/2020/science/coronavirus-vaccine-tracker.html?action=click\&pgtype=Article\&state=default\&region=TOP_BANNER\&context=storylines_menu}{Vaccine
  Tracker}
\item
  \href{https://www.nytimes.com/interactive/2020/07/29/us/schools-reopening-coronavirus.html?action=click\&pgtype=Article\&state=default\&region=TOP_BANNER\&context=storylines_menu}{What
  School May Look Like}
\item
  \href{https://www.nytimes.com/live/2020/07/31/business/stock-market-today-coronavirus?action=click\&pgtype=Article\&state=default\&region=TOP_BANNER\&context=storylines_menu}{Economy}
\end{itemize}

Advertisement

\protect\hyperlink{after-top}{Continue reading the main story}

Supported by

\protect\hyperlink{after-sponsor}{Continue reading the main story}

TOULOUSE Dispatch

\hypertarget{for-french-algerian-families-virus-disrupts-cherished-summer-ritual}{%
\section{For French-Algerian Families, Virus Disrupts Cherished Summer
Ritual}\label{for-french-algerian-families-virus-disrupts-cherished-summer-ritual}}

Holidays in Algeria are a cornerstone of the cross-cultural identity of
many French people with roots there. This year, they are stuck at home,
and the pain is acute: ``It's sacred for us to leave.''

\includegraphics{https://static01.nyt.com/images/2020/07/22/world/00france-holiday-dispatch/00toulouse-holiday-dispatch-articleLarge.jpg?quality=75\&auto=webp\&disable=upscale}

By \href{https://www.nytimes.com/by/constant-meheut}{Constant Méheut}

\begin{itemize}
\item
  July 26, 2020
\item
  \begin{itemize}
  \item
  \item
  \item
  \item
  \item
  \item
  \end{itemize}
\end{itemize}

TOULOUSE, France --- Sitting around a table strewn with steaming cups of
mint tea, a dozen women were sharing memories of their summer holidays
in their homeland, Algeria.

Malika Haï recalled sweltering days spent with her cousins near the
beaches. Samia Tran described the cheerful family dinners around
traditional dishes.

And Zohra Benkebane, almost an hour into the conversation, was the first
to burst into tears.

``We all have a lump in our throats,'' Ms. Tran said as she hugged her
sobbing friend. ``It's too hard. We need to go home.''

For many French citizens of Algerian descent whose families migrated
across the Mediterranean in the second half of the 20th century, summer
holidays in Algeria are a deep-rooted tradition. Every year thousands of
people venture off toward what they commonly call the ``bled*''* --- a
word derived from Arabic that refers to the countryside.

``Leaving for the bled ** is a form of holiday routine,'' said Jennifer
Bidet, a sociologist at the Paris Descartes University who estimated,
based on official statistics, that 82 percent of French people of
Algerian origin had spent at least one holiday in Algeria during
childhood, while 34 percent returned every year.

\includegraphics{https://static01.nyt.com/images/2020/07/22/world/00france-holiday-dispatch2/merlin_174805632_4b117448-4da5-4f6e-a49b-4daa4972daf7-articleLarge.jpg?quality=75\&auto=webp\&disable=upscale}

But with the Covid-19 pandemic still raging, Algeria is keeping its
borders tightly closed until further notice. That effectively forbids
vacations that had become a cornerstone of the cross-cultural identity
of many French-Algerian families, much to their dismay.

``Holidays in the bled are a cultural bridge,'' said Mustapha
Benzitouni, a 45-year-old French-Algerian. ``It allows people to
rediscover an identity through their parents, through their belonging to
a people, through their belonging to a culture.''

Perhaps nowhere has the Algerian travel ban been felt more acutely than
in Toulouse, a city of about 500,000 people in southwestern France that
was shaped by waves of immigration.

Hundreds of Toulouse families of Algerian descent are now stranded at
home, unable to afford, or simply unwilling, to spend summer vacations
anywhere but Algeria.

\hypertarget{latest-updates-global-coronavirus-outbreak}{%
\section{\texorpdfstring{\href{https://www.nytimes.com/2020/08/01/world/coronavirus-covid-19.html?action=click\&pgtype=Article\&state=default\&region=MAIN_CONTENT_1\&context=storylines_live_updates}{Latest
Updates: Global Coronavirus
Outbreak}}{Latest Updates: Global Coronavirus Outbreak}}\label{latest-updates-global-coronavirus-outbreak}}

Updated 2020-08-01T18:42:36.154Z

\begin{itemize}
\tightlist
\item
  \href{https://www.nytimes.com/2020/08/01/world/coronavirus-covid-19.html?action=click\&pgtype=Article\&state=default\&region=MAIN_CONTENT_1\&context=storylines_live_updates\#link-3ac56579}{Top
  officials work to break impasse over jobless benefit.}
\item
  \href{https://www.nytimes.com/2020/08/01/world/coronavirus-covid-19.html?action=click\&pgtype=Article\&state=default\&region=MAIN_CONTENT_1\&context=storylines_live_updates\#link-8796723}{The
  virus picks up dangerous speed in the Midwest, and in areas that had
  seen success.}
\item
  \href{https://www.nytimes.com/2020/08/01/world/coronavirus-covid-19.html?action=click\&pgtype=Article\&state=default\&region=MAIN_CONTENT_1\&context=storylines_live_updates\#link-25930521}{Thousands
  in Berlin protest Germany's coronavirus measures.}
\end{itemize}

\href{https://www.nytimes.com/2020/08/01/world/coronavirus-covid-19.html?action=click\&pgtype=Article\&state=default\&region=MAIN_CONTENT_1\&context=storylines_live_updates}{See
more updates}

More live coverage:
\href{https://www.nytimes.com/live/2020/07/31/business/stock-market-today-coronavirus?action=click\&pgtype=Article\&state=default\&region=MAIN_CONTENT_1\&context=storylines_live_updates}{Markets}

``It's sacred for us to leave,'' said Ms. Haï, 58, who, like many
Algerians of her generation, mixed Arabic and French when speaking.
``During a normal summer, in July and August, the neighborhood goes
completely empty.''

The neighborhood to which Ms. Haï referred is Le Mirail, an impoverished
area outside the city center that is plagued by drug trafficking and
where about 30,000 people live in dreary apartment blocks. A large
majority of the residents come from Algeria, with other families from
Morocco and Tunisia, who also often visit their homelands in France's
former North African colonies in July and August.

Image

Le Mirail is a neighborhood outside downtown Toulouse; it's home to
about 30,000 people, largely from France's former North African
colonies.Credit...Dmitry Kostyukov for The New York Times

Unlike Algeria, Morocco and Tunisia have recently reopened their borders
to tourists and their citizens living abroad, meaning some of Le
Mirail's residents can go ahead with their summer plans.

For Algerians, though, the travel ban means parents need to come up with
alternative plans for idle children.

``Spending the summer here is impossible, there's hardly anything to
do,'' said Djelloul Zitouni, 38, a father of three who was playing with
his children on a small playground nestled in the middle of Le Mirail.

Like every year, Mr. Zitouni --- who migrated to France from Algeria 14
years ago to work as a driver --- had planned on spending August in his
hometown, the coastal city of Oran. ``Eleven months of hard work for one
month of dreams,'' he said.

This year Mr. Zitouni said he would try to ``get by with the children,''
taking them to the local swimming pool and a few days at the seaside.

Image

``Spending the summer here is impossible,'' said Djelloul Zitouni, with
his children at a playground in Le Mirail.Credit...Dmitry Kostyukov for
The New York Times

Worried that bored teenagers could lead to trouble, local community
groups and the authorities have tried to alleviate the doldrums by
organizing activities in Le Mirail.

On a recent afternoon, dozens of families, mostly of Algerian heritage,
gathered on large plots of grass bordering a small lake in the
neighborhood to take part in painting workshops, water games and dance
classes.

But Soraya Amalou, a volunteer, had no illusions that these activities
could make up for the loss of a genuine summer escape. ``Spending
holidays here means no holidays. In this neighborhood, you suffer from
tiny apartments, from insecurity, you suffer from everything,'' she
said.

By contrast, summer vacations in Algeria, which Ms. Haï likened to a
``breath of fresh air,'' are much anticipated all year long, and the
rituals leading up to the trip --- from the tickets booked well in
advance to the suitcases filled with presents for the cousins --- have
shaped several migrant generations.

Image

Local community groups have tried to alleviate the summer doldrums by
organizing activities for children.Credit...Dmitry Kostyukov for The New
York Times

The French colonization of Algeria, which lasted from 1830 to 1962,
forged lasting, yet complex, ties between the two nations, which
Benjamin Stora, a historian of Algeria, described as a ``very special
relationship of both hatred and fascination.''

\href{https://www.nytimes.com/news-event/coronavirus?action=click\&pgtype=Article\&state=default\&region=MAIN_CONTENT_3\&context=storylines_faq}{}

\hypertarget{the-coronavirus-outbreak-}{%
\subsubsection{The Coronavirus Outbreak
›}\label{the-coronavirus-outbreak-}}

\hypertarget{frequently-asked-questions}{%
\paragraph{Frequently Asked
Questions}\label{frequently-asked-questions}}

Updated July 27, 2020

\begin{itemize}
\item ~
  \hypertarget{should-i-refinance-my-mortgage}{%
  \paragraph{Should I refinance my
  mortgage?}\label{should-i-refinance-my-mortgage}}

  \begin{itemize}
  \tightlist
  \item
    \href{https://www.nytimes.com/article/coronavirus-money-unemployment.html?action=click\&pgtype=Article\&state=default\&region=MAIN_CONTENT_3\&context=storylines_faq}{It
    could be a good idea,} because mortgage rates have
    \href{https://www.nytimes.com/2020/07/16/business/mortgage-rates-below-3-percent.html?action=click\&pgtype=Article\&state=default\&region=MAIN_CONTENT_3\&context=storylines_faq}{never
    been lower.} Refinancing requests have pushed mortgage applications
    to some of the highest levels since 2008, so be prepared to get in
    line. But defaults are also up, so if you're thinking about buying a
    home, be aware that some lenders have tightened their standards.
  \end{itemize}
\item ~
  \hypertarget{what-is-school-going-to-look-like-in-september}{%
  \paragraph{What is school going to look like in
  September?}\label{what-is-school-going-to-look-like-in-september}}

  \begin{itemize}
  \tightlist
  \item
    It is unlikely that many schools will return to a normal schedule
    this fall, requiring the grind of
    \href{https://www.nytimes.com/2020/06/05/us/coronavirus-education-lost-learning.html?action=click\&pgtype=Article\&state=default\&region=MAIN_CONTENT_3\&context=storylines_faq}{online
    learning},
    \href{https://www.nytimes.com/2020/05/29/us/coronavirus-child-care-centers.html?action=click\&pgtype=Article\&state=default\&region=MAIN_CONTENT_3\&context=storylines_faq}{makeshift
    child care} and
    \href{https://www.nytimes.com/2020/06/03/business/economy/coronavirus-working-women.html?action=click\&pgtype=Article\&state=default\&region=MAIN_CONTENT_3\&context=storylines_faq}{stunted
    workdays} to continue. California's two largest public school
    districts --- Los Angeles and San Diego --- said on July 13, that
    \href{https://www.nytimes.com/2020/07/13/us/lausd-san-diego-school-reopening.html?action=click\&pgtype=Article\&state=default\&region=MAIN_CONTENT_3\&context=storylines_faq}{instruction
    will be remote-only in the fall}, citing concerns that surging
    coronavirus infections in their areas pose too dire a risk for
    students and teachers. Together, the two districts enroll some
    825,000 students. They are the largest in the country so far to
    abandon plans for even a partial physical return to classrooms when
    they reopen in August. For other districts, the solution won't be an
    all-or-nothing approach.
    \href{https://bioethics.jhu.edu/research-and-outreach/projects/eschool-initiative/school-policy-tracker/}{Many
    systems}, including the nation's largest, New York City, are
    devising
    \href{https://www.nytimes.com/2020/06/26/us/coronavirus-schools-reopen-fall.html?action=click\&pgtype=Article\&state=default\&region=MAIN_CONTENT_3\&context=storylines_faq}{hybrid
    plans} that involve spending some days in classrooms and other days
    online. There's no national policy on this yet, so check with your
    municipal school system regularly to see what is happening in your
    community.
  \end{itemize}
\item ~
  \hypertarget{is-the-coronavirus-airborne}{%
  \paragraph{Is the coronavirus
  airborne?}\label{is-the-coronavirus-airborne}}

  \begin{itemize}
  \tightlist
  \item
    The coronavirus
    \href{https://www.nytimes.com/2020/07/04/health/239-experts-with-one-big-claim-the-coronavirus-is-airborne.html?action=click\&pgtype=Article\&state=default\&region=MAIN_CONTENT_3\&context=storylines_faq}{can
    stay aloft for hours in tiny droplets in stagnant air}, infecting
    people as they inhale, mounting scientific evidence suggests. This
    risk is highest in crowded indoor spaces with poor ventilation, and
    may help explain super-spreading events reported in meatpacking
    plants, churches and restaurants.
    \href{https://www.nytimes.com/2020/07/06/health/coronavirus-airborne-aerosols.html?action=click\&pgtype=Article\&state=default\&region=MAIN_CONTENT_3\&context=storylines_faq}{It's
    unclear how often the virus is spread} via these tiny droplets, or
    aerosols, compared with larger droplets that are expelled when a
    sick person coughs or sneezes, or transmitted through contact with
    contaminated surfaces, said Linsey Marr, an aerosol expert at
    Virginia Tech. Aerosols are released even when a person without
    symptoms exhales, talks or sings, according to Dr. Marr and more
    than 200 other experts, who
    \href{https://academic.oup.com/cid/article/doi/10.1093/cid/ciaa939/5867798}{have
    outlined the evidence in an open letter to the World Health
    Organization}.
  \end{itemize}
\item ~
  \hypertarget{what-are-the-symptoms-of-coronavirus}{%
  \paragraph{What are the symptoms of
  coronavirus?}\label{what-are-the-symptoms-of-coronavirus}}

  \begin{itemize}
  \tightlist
  \item
    Common symptoms
    \href{https://www.nytimes.com/article/symptoms-coronavirus.html?action=click\&pgtype=Article\&state=default\&region=MAIN_CONTENT_3\&context=storylines_faq}{include
    fever, a dry cough, fatigue and difficulty breathing or shortness of
    breath.} Some of these symptoms overlap with those of the flu,
    making detection difficult, but runny noses and stuffy sinuses are
    less common.
    \href{https://www.nytimes.com/2020/04/27/health/coronavirus-symptoms-cdc.html?action=click\&pgtype=Article\&state=default\&region=MAIN_CONTENT_3\&context=storylines_faq}{The
    C.D.C. has also} added chills, muscle pain, sore throat, headache
    and a new loss of the sense of taste or smell as symptoms to look
    out for. Most people fall ill five to seven days after exposure, but
    symptoms may appear in as few as two days or as many as 14 days.
  \end{itemize}
\item ~
  \hypertarget{does-asymptomatic-transmission-of-covid-19-happen}{%
  \paragraph{Does asymptomatic transmission of Covid-19
  happen?}\label{does-asymptomatic-transmission-of-covid-19-happen}}

  \begin{itemize}
  \tightlist
  \item
    So far, the evidence seems to show it does. A widely cited
    \href{https://www.nature.com/articles/s41591-020-0869-5}{paper}
    published in April suggests that people are most infectious about
    two days before the onset of coronavirus symptoms and estimated that
    44 percent of new infections were a result of transmission from
    people who were not yet showing symptoms. Recently, a top expert at
    the World Health Organization stated that transmission of the
    coronavirus by people who did not have symptoms was ``very rare,''
    \href{https://www.nytimes.com/2020/06/09/world/coronavirus-updates.html?action=click\&pgtype=Article\&state=default\&region=MAIN_CONTENT_3\&context=storylines_faq\#link-1f302e21}{but
    she later walked back that statement.}
  \end{itemize}
\end{itemize}

Mr. Stora said that ``returning to the bled'' was a way for
French-Algerians to ``reconnect with a national filiation.''

But while French-Algerians can be made to feel like they don't fully fit
in with France, they also ``are badly regarded in Algeria,'' Mr. Stora
said, where they are seen as French citizens whose Algerian heritage is
but a detail.

``They treat us like French bourgeois and raise prices as soon as we
arrive,'' said Ahmed Adjelout, 72, who was waiting in a travel agency in
downtown Toulouse in the hope of rescheduling his July 22 flight to
Oran, which had just been canceled.

Mr. Adjelout, a retiree with a beret thrust upon his head, recalled how
he would be called ``an emigrant, a stranger'' by Algerians whenever he
returned to the country he left in 1967.

``The paradox,'' Mr. Adjelout added, ``is that in Algeria, we're seen as
French and in France, we're seen as Algerians.''

Image

Central Toulouse, which lies about 365 miles south of Paris. Waves of
immigration have shaped Toulouse.Credit...Dmitry Kostyukov for The New
York Times

This ambiguous situation ---
\href{https://www.nytimes.com/2015/08/16/world/africa/france-algeria-immigration-discrimination-racism.html}{straddling
two cultures and belonging to neither} --- can make building an identity
a challenge for the estimated 2.5 million people of Algerian descent in
France, especially for second- and third-generation immigrants for whom
Algeria is merely a summer getaway.

``It's tricky to deal with both sides, the French and the Algerian, no
culture really welcomes us,'' said Fatiha Zelmat, whose mother, Naouel
Matti, has taken her to
\href{https://www.nytimes.com/2019/05/05/world/africa/algeria-casbah-preservation-plan.html}{the
ancient stone alleys of Algiers}, the Algerian capital, every summer
since she was born --- except this year.

Ms. Zelmat, 21, said she had fond memories of her time in Algeria, but
she also condemned a more conservative culture that forbids women from
smoking or wearing shorts.

``I have mixed feelings about Algeria,'' she said.

Ms. Bidet, the sociologist, said that for some young people, spending
holidays in Algeria --- where they can afford activities that would be
far beyond their means in France --- is an opportunity to escape the
poor social status to which they are normally relegated.

But, she noted, this reversal of social hierarchies is only temporary
and does nothing to address the problems of integration they face in
France.

Ms. Matti, who covered her hair with an elegant white veil, said that
young people of Algerian descent like her daughter were integrated
neither in Algeria nor in France, where they often grow up in deeply
socially segregated neighborhoods like Le Mirail.

``Our children stop going to Algeria because they don't feel they belong
there,'' Ms. Matti said. ``But where will they go instead?''

Image

Le Mirail is home to many people with origins in North Africa. ``It's
tricky to deal with both sides, the French and the Algerian, no culture
really welcomes us,'' one young woman said.Credit...Dmitry Kostyukov for
The New York Times

Advertisement

\protect\hyperlink{after-bottom}{Continue reading the main story}

\hypertarget{site-index}{%
\subsection{Site Index}\label{site-index}}

\hypertarget{site-information-navigation}{%
\subsection{Site Information
Navigation}\label{site-information-navigation}}

\begin{itemize}
\tightlist
\item
  \href{https://help.nytimes.com/hc/en-us/articles/115014792127-Copyright-notice}{©~2020~The
  New York Times Company}
\end{itemize}

\begin{itemize}
\tightlist
\item
  \href{https://www.nytco.com/}{NYTCo}
\item
  \href{https://help.nytimes.com/hc/en-us/articles/115015385887-Contact-Us}{Contact
  Us}
\item
  \href{https://www.nytco.com/careers/}{Work with us}
\item
  \href{https://nytmediakit.com/}{Advertise}
\item
  \href{http://www.tbrandstudio.com/}{T Brand Studio}
\item
  \href{https://www.nytimes.com/privacy/cookie-policy\#how-do-i-manage-trackers}{Your
  Ad Choices}
\item
  \href{https://www.nytimes.com/privacy}{Privacy}
\item
  \href{https://help.nytimes.com/hc/en-us/articles/115014893428-Terms-of-service}{Terms
  of Service}
\item
  \href{https://help.nytimes.com/hc/en-us/articles/115014893968-Terms-of-sale}{Terms
  of Sale}
\item
  \href{https://spiderbites.nytimes.com}{Site Map}
\item
  \href{https://help.nytimes.com/hc/en-us}{Help}
\item
  \href{https://www.nytimes.com/subscription?campaignId=37WXW}{Subscriptions}
\end{itemize}
