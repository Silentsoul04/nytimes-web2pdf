\href{/section/business}{Business}\textbar{}Jeff Bezos Cast in a Role He
Never Wanted: Amazon's D.C. Defender

\url{https://nyti.ms/3f6Z9bg}

\begin{itemize}
\item
\item
\item
\item
\item
\item
\end{itemize}

\includegraphics{https://static01.nyt.com/images/2020/07/23/business/00bezos4/merlin_164620407_5cc85779-3229-41f4-aaa8-ed480bf14102-articleLarge.jpg?quality=75\&auto=webp\&disable=upscale}

Sections

\protect\hyperlink{site-content}{Skip to
content}\protect\hyperlink{site-index}{Skip to site index}

\hypertarget{jeff-bezos-cast-in-a-role-he-never-wanted-amazons-dc-defender}{%
\section{Jeff Bezos Cast in a Role He Never Wanted: Amazon's D.C.
Defender}\label{jeff-bezos-cast-in-a-role-he-never-wanted-amazons-dc-defender}}

The chief executive, who testifies before Congress for the first time on
Wednesday, had taken a hands-off approach with lawmakers in Washington.

Jeff Bezos and his portrait, by Robert McCurdy, at the National Portrait
Gallery.Credit...Jared Soares for The New York Times

Supported by

\protect\hyperlink{after-sponsor}{Continue reading the main story}

By \href{https://www.nytimes.com/by/david-mccabe}{David McCabe} and
\href{https://www.nytimes.com/by/karen-weise}{Karen Weise}

\begin{itemize}
\item
  Published July 27, 2020Updated July 29, 2020, 11:44 a.m. ET
\item
  \begin{itemize}
  \item
  \item
  \item
  \item
  \item
  \item
  \end{itemize}
\end{itemize}

WASHINGTON --- Last September, Jeff Bezos, Amazon's chief executive,
rattled off terrifying statistics about the warming planet from the
storied National Press Club, two blocks from the White House. Then he
said he had something exciting to announce.

But when he uncovered a towering sign with the news, Amazon's name was
nowhere in sight. Instead, the sign
\href{https://www.nytimes.com/2019/09/19/technology/amazon-carbon-neutral.html}{introduced
the Climate Pledge}, a project to reduce carbon emissions from
companies. Yes, Amazon would be the first, and at the time only,
signatory. But this was a bigger push, Mr. Bezos said.

It was Amazon news, couched as something grander.

The event reflected Mr. Bezos' approach to the nation's capital. He has
jumped at opportunities to cast himself as a statesman --- the savior of
The Washington Post, who holds court among the country's elite. At the
same time, he has eschewed the day-to-day grind of bolstering Amazon's
influence with policymakers.

But that changes on Wednesday, when Mr. Bezos testifies before Congress
for the first time. He will be joined by the chief executives of
Alphabet, Apple and Facebook as part of lawmakers' investigations into
the power of the largest tech companies. He is expected to face an
onslaught of critiques, with questions as varied as Amazon's labor
conditions and market power and his status as the richest person in the
world.

It's the kind of appearance Mr. Bezos had steadfastly avoided.

``It's not traditional lobbying,'' Steve Case, the America Online
co-founder, said of how Mr. Bezos, whom he considers an old friend, had
approached Washington until now. ``It is much more of a longer-term
relationship-building --- a little bit of a reputation-building ---
effort that has to be sustained over decades.''

Amazon declined to comment on Mr. Bezos.

He arrived in Washington with a splash in 2013, when he
\href{https://www.nytimes.com/2013/08/06/business/media/a-mogul-gets-a-landmark-in-the-capital.html}{bought
The Washington Post} from its longtime owners for \$250 million and gave
the paper new life. In 2016, Mr. Bezos bought the biggest home in the
city, a 27,000-square-foot manse that used to be a museum in the
Kalorama neighborhood, where former President Barack Obama and other
political leaders live.

\includegraphics{https://static01.nyt.com/images/2020/07/23/business/00bezos2/merlin_122260895_01941d48-6020-41b8-ab23-7583c7318f32-articleLarge.jpg?quality=75\&auto=webp\&disable=upscale}

While Mr. Bezos' presence in the city grew, so did Amazon's, as it began
pouring money into the traditional modes of influencing policymakers. It
spent \$16.8 million on federal lobbying in 2019, up from less than \$10
million in 2015, according to \href{https://www.opensecrets.org/}{the
Center for Responsive Politics}. Last year, it gave \$11.1 million to
think tanks and associations, more than twice as much as the previous
year, according to its
\href{https://s2.q4cdn.com/299287126/files/doc_downloads/political_expenditures_statement/2018-Political-Expenditures-Statement.pdf}{disclosures}.
In 2018, it selected Crystal City, Va., a Metro ride away from
Washington, as the site of its second headquarters.

Mr. Bezos occasionally appeared in support of the company's efforts. In
2017, for example, he was interviewed by the head of the Internet
Association, a lobbying group that represents Amazon and other tech
giants, at its annual gala.

But as he does with many parts of Amazon, Mr. Bezos took a hands-off
approach with its policy and communications group, which has grown to
more than 800 employees globally. He'd come through Washington for the
annual Amazon board meeting, with a few quiet visits sprinkled
throughout the year.

He has avoided high-profile meetings with his company's sharpest
critics, like the one Mark Zuckerberg, Facebook's chief executive, held
\href{https://www.nytimes.com/2020/07/07/technology/facebook-ad-boycott-civil-rights.html}{a
few weeks ago} with organizers of an ad boycott of his company. Mr.
Bezos has not made a habit of glad-handing worried lawmakers, the way
Sundar Pichai, who runs Alphabet, Google's parent company,
\href{https://www.nytimes.com/2018/09/28/technology/google-pichai-congress-testify.html}{did
in 2018}. And
\href{https://www.nytimes.com/2019/11/20/us/politics/trump-texas-apple-factory.html}{unlike
Tim Cook} of Apple, Mr. Bezos has not developed a close relationship
with President Trump.

The work of Amazon's political relations was left to other executives.
In 2013, when Mr. Obama toured an Amazon warehouse, it was Dave Clark, a
rising star at the company, who showed him around. In more recent years,
Jay Carney, Mr. Obama's former press secretary, has become the face of
Amazon's interactions with lawmakers.

Mr. Carney was the one who called Gov. Andrew M. Cuomo of New York to
say Amazon was backing out on its commitment to place a second
headquarters in the state after facing a backlash from local activists
and politicians. And he managed the crisis when Senator Bernie Sanders,
the progressive independent from Vermont, pushed the company to raise
its minimum wage.

When Mr. Trump was still a long-shot candidate, Mr. Bezos
\href{https://twitter.com/JeffBezos/status/674008204838199297}{tweeted}
that he wanted to ``\#sendDonaldtospace.'' But since Mr. Trump's
election, Mr. Bezos has remained quiet even as the president attacked
The Washington Post, stating, without providing evidence, that the paper
was doing Amazon's bidding. The newspaper is owned privately by Mr.
Bezos, not Amazon.

Image

Mr. Bezos discussing climate change last year at the National Press Club
in Washington, a statesmanlike role he prefers to the political fray
he'll face in Congress on Wednesday.Credit...Emma Howells for The New
York Times

``Jeff kind of shrugs his shoulders and says it kind of goes with the
territory,'' Mr. Case said. ``I'm sure he doesn't like it, but he takes
it.''

By 2018, Washingtonian magazine
\href{https://www.washingtonian.com/2018/04/22/inside-jeff-bezos-dc-life/}{reported}
that Mr. Bezos had ``quietly become a freewheeling D.C. socialite''
alongside a photo illustration that showed him towering over the
Washington Monument. Washington Life --- which breathlessly tracks the
area's wealthy residents --- named him
\href{https://washingtonlife.com/2018/05/10/business-real-estate-jeff-bezos/}{one
of the 100 most powerful people} in the city. In November, he received
an award at the Smithsonian Institution's National Portrait Gallery
gala, which had commissioned his portrait for its collection.

Mr. Bezos' celebrity has also increased in recent years. Last year, he
announced that he and his wife,
\href{https://www.nytimes.com/2019/05/28/us/mackenzie-bezos-charity.html}{MacKenzie
Bezos, were divorcing}, which was followed days later by a National
Enquirer exposé of an extramarital affair with Lauren Sanchez, a former
television host. Then
\href{https://www.nytimes.com/2019/02/07/technology/jeff-bezos-sanchez-enquirer.html}{he
accused} the tabloid of ``extortion and blackmail,'' saying it had
threatened to publish lewd photos unless he said the outlet, which is
close to the White House, was not politically motivated in reporting on
his relationships.

In January, he finally debuted his mansion, hosting prominent figures in
politics and business. The invitations, sent from an email address at
The Post, were signed simply ``Jeff.''

Mingling in the home's downstairs area and terraced backyard, the guests
included administration figures like Ivanka Trump and her husband, Jared
Kushner; corporate titans like Jamie Dimon, the chief executive of
JPMorgan Chase; and cultural celebrities like the actor Ben Stiller.

Senator Mitt Romney, a Republican from Utah, was there as well. ``It's
very much consistent with its original design and is tastefully done,''
he said of the house a few weeks later.

Mr. Romney said he had spoken only briefly with Mr. Bezos at the party
to thank him for his hospitality but said he had gotten to talk with
another notable guest, Bill Gates, about climate change and nuclear
power.

``So it was most enjoyable,'' Mr. Romney said.

The environment on the Hill this week is likely to be far less
hospitable. Mr. Bezos' wealth has grown by more than \$50 billion in
recent months, just as unemployment has skyrocketed during the pandemic,
making him an avatar for inequality. Questions about Amazon's dominance
have also grown louder, as more Americans have been forced to shop
online because of the coronavirus. Warehouse workers have said that
Amazon is putting them at risk of contracting the virus in the company's
pursuit of speedy deliveries.

Image

Guests at a party at Mr. Bezos' Washington home in January
included~Ivanka Trump and Jared Kushner.Credit...Anna Moneymaker for The
New York Times

Even as the concerns of politicians became more pronounced, Amazon
resisted sending Mr. Bezos before Congress. The company agreed to send
him after lawmakers threatened to subpoena his testimony.

``No one is above the law, no matter how rich or powerful,''
\href{https://twitter.com/davidcicilline/status/1261432733982773251}{Representative
David Cicilline}, the Rhode Island Democrat who leads the Judiciary
Committee's antitrust subcommittee, said in a May tweet.

Mr. Case said lawmakers should not expect Mr. Bezos to get rattled. He
recalled when Mr. Bezos appeared onstage two years ago at the Economic
Club of Washington, D.C., with David Rubenstein, a private equity
magnate. Mr. Bezos expounded on a variety of topics, including his
just-announced \$2 billion fund to support education and the homeless.

Mr. Case, who shared a table at the event with Mr. Bezos' parents, said
that many people in the room did not know Mr. Bezos, but that they had
left impressed. Mr. Bezos bounced between clearly prepared talking
points and ``unplugged Jeff just being Jeff,'' Mr. Case said. ``His best
ambassador is himself.''

David McCabe reported from Washington, and Karen Weise from Seattle.

Advertisement

\protect\hyperlink{after-bottom}{Continue reading the main story}

\hypertarget{site-index}{%
\subsection{Site Index}\label{site-index}}

\hypertarget{site-information-navigation}{%
\subsection{Site Information
Navigation}\label{site-information-navigation}}

\begin{itemize}
\tightlist
\item
  \href{https://help.nytimes.com/hc/en-us/articles/115014792127-Copyright-notice}{©~2020~The
  New York Times Company}
\end{itemize}

\begin{itemize}
\tightlist
\item
  \href{https://www.nytco.com/}{NYTCo}
\item
  \href{https://help.nytimes.com/hc/en-us/articles/115015385887-Contact-Us}{Contact
  Us}
\item
  \href{https://www.nytco.com/careers/}{Work with us}
\item
  \href{https://nytmediakit.com/}{Advertise}
\item
  \href{http://www.tbrandstudio.com/}{T Brand Studio}
\item
  \href{https://www.nytimes.com/privacy/cookie-policy\#how-do-i-manage-trackers}{Your
  Ad Choices}
\item
  \href{https://www.nytimes.com/privacy}{Privacy}
\item
  \href{https://help.nytimes.com/hc/en-us/articles/115014893428-Terms-of-service}{Terms
  of Service}
\item
  \href{https://help.nytimes.com/hc/en-us/articles/115014893968-Terms-of-sale}{Terms
  of Sale}
\item
  \href{https://spiderbites.nytimes.com}{Site Map}
\item
  \href{https://help.nytimes.com/hc/en-us}{Help}
\item
  \href{https://www.nytimes.com/subscription?campaignId=37WXW}{Subscriptions}
\end{itemize}
