Sections

SEARCH

\protect\hyperlink{site-content}{Skip to
content}\protect\hyperlink{site-index}{Skip to site index}

\href{https://www.nytimes.com/section/nyregion}{New York}

\href{https://myaccount.nytimes.com/auth/login?response_type=cookie\&client_id=vi}{}

\href{https://www.nytimes.com/section/todayspaper}{Today's Paper}

\href{/section/nyregion}{New York}\textbar{}Trump Again Tries to Block
Subpoena for Taxes, Calling It `Wildly Overbroad'

\url{https://nyti.ms/2CWnJhY}

\begin{itemize}
\item
\item
\item
\item
\item
\end{itemize}

Advertisement

\protect\hyperlink{after-top}{Continue reading the main story}

Supported by

\protect\hyperlink{after-sponsor}{Continue reading the main story}

\hypertarget{trump-again-tries-to-block-subpoena-for-taxes-calling-it-wildly-overbroad}{%
\section{Trump Again Tries to Block Subpoena for Taxes, Calling It
`Wildly
Overbroad'}\label{trump-again-tries-to-block-subpoena-for-taxes-calling-it-wildly-overbroad}}

The president mounted his most forceful and detailed legal attack yet on
the subpoena for his tax returns from the Manhattan district attorney.

\includegraphics{https://static01.nyt.com/images/2020/07/27/nyregion/27nytrump/merlin_174883950_e60d81b0-29a2-405d-923f-1a4e6fc430ab-articleLarge.jpg?quality=75\&auto=webp\&disable=upscale}

\href{https://www.nytimes.com/by/benjamin-weiser}{\includegraphics{https://static01.nyt.com/images/2018/07/16/multimedia/author-benjamin-weiser/author-benjamin-weiser-thumbLarge.png}}\href{https://www.nytimes.com/by/william-k-rashbaum}{\includegraphics{https://static01.nyt.com/images/2018/06/13/multimedia/author-william-k-rashbaum/author-william-k-rashbaum-thumbLarge.jpg}}

By \href{https://www.nytimes.com/by/benjamin-weiser}{Benjamin Weiser}
and \href{https://www.nytimes.com/by/william-k-rashbaum}{William K.
Rashbaum}

\begin{itemize}
\item
  July 27, 2020
\item
  \begin{itemize}
  \item
  \item
  \item
  \item
  \item
  \end{itemize}
\end{itemize}

President Trump on Monday mounted his most forceful and detailed legal
attack yet on the
\href{https://www.nytimes.com/2020/08/03/nyregion/donald-trump-taxes-cyrus-vance.html}{subpoena
for his tax returns by the Manhattan district attorney}, arguing the
request was ``wildly overbroad'' and ``issued in bad faith,'' a new
court filing shows.

Mr. Trump's lawyers asked a federal judge in Manhattan to declare that
the subpoena from the district attorney, Cyrus R. Vance Jr., a Democrat,
was ``invalid and unenforceable.''

They also asked that the judge issue an order barring Mr. Vance from
``taking any action to enforce'' the subpoena --- which sought years of
tax and other financial records from his accountants --- and that he
block
\href{https://www.nytimes.com/2020/07/28/us/politics/donald-fred-trump.html}{Mr.
Trump's} accounting firm, Mazars USA, from turning over any of the
information.

``The Mazars subpoena is so sweeping that it amounts to an unguided and
unlawful fishing expedition into the President's personal financial and
business dealings,'' the lawyers wrote.

Mr. Trump's arguments came just weeks after
\href{https://www.nytimes.com/2020/07/09/us/trump-taxes-supreme-court.html?action=click\&module=RelatedLinks\&pgtype=Article}{the
Supreme Court cleared the way for Manhattan prosecutors to seek his
financial records}, in a decision that was seen as a major defeat for
Mr. Trump and a statement on the limits of presidential power.

Mr. Vance had subpoenaed Mr. Trump's accounting firm last August for
\href{https://www.nytimes.com/2019/09/16/nyregion/trump-tax-returns-cy-vance.html}{eight
years of his personal tax returns and those of his family business} as
part of an investigation into hush-money payments to Stormy Daniels, an
adult-film actress who said she had an affair with Mr. Trump.

The president, who has denied the affair, has fought the subpoena for
almost a year,
\href{https://www.nytimes.com/2019/09/19/nyregion/trump-tax-returns-lawsuit.html}{arguing
that a sitting president is immune from state criminal investigations}.

The Supreme Court rejected Mr. Trump's position on immunity, but it said
he could return to the lower court, where his legal battle began, and
raise new objections to the subpoena. The filing on Monday focused on
the subpoena itself, rather than the broader legal issues that were
before the Supreme Court.

Mr. Vance's office, which is scheduled to respond to Mr. Trump's latest
filing on Monday, has accused Mr. Trump of
\href{https://www.nytimes.com/2020/07/16/nyregion/donald-trump-taxes-cyrus-vance.html}{intentionally
dragging out the subpoena fight to effectively shield himself from
criminal investigation}, and obtain the kind of immunity to which the
Supreme Court said he was not entitled.

``What the president's lawyers are seeking here is delay,'' Carey R.
Dunne, a senior lawyer in Mr. Vance's office, told the lower court
judge, Victor Marrero, in a hearing on July 16. ``I think that's the
entire strategy here.''

Mr. Dunne said that the longer Mr. Trump fought the case, the greater
the likelihood that the statute of limitations would expire for any
possible crimes that might have been committed.

``Let's not let delay kill this case,'' Mr. Dunne told Judge Marrero at
the hearing.

Jay Sekulow, one of Mr. Trump's lawyers, denied the accusation at the
time. ``Our strategy seeks due process,'' he said in an email.

Mr. Vance's prosecutors have argued that Judge Marrero has already
decided most of the issues Mr. Trump has raised. Last October, the judge
wrote a
\href{https://www.nytimes.com/2019/10/07/nyregion/trump-taxes-lawsuit-vance.html}{75-page
opinion that rejected the president's argument that he was immune from
all investigations}.

In that ruling, Mr. Vance's office has argued that Judge Marrero found
there was no demonstrated bad faith or harassment in Mr. Vance's
decision to issue the subpoena, and that the judge rejected Mr. Trump's
claim that there was evidence of any motive other than enforcement of
the law.

Mr. Vance's office has been looking into whether any New York State laws
were broken in connection with the hush-money payments arranged in 2016
for Ms. Daniels and another woman by Michael D. Cohen, the president's
former lawyer and fixer. Mr. Cohen later pleaded guilty to federal
campaign finance violations for his role in the payments, and was
sentenced to a three years in prison.

Mr. Trump's lawyers, in the new court filing, said they had initially
cooperated with the district attorney's investigation, turning over
hundreds of documents in response to an earlier subpoena the prosecutors
had issued to the Trump Organization.

But the president's lawyers said they balked when they learned, during
negotiations over the scope of the first subpoena, that the prosecutors
believed it also covered Mr. Trump's tax returns.

Mr. Trump's lawyers said that Mr. Vance's office then retaliated by
issuing a new subpoena to the accounting firm in an effort to
``circumvent the president.''

In arguing that the subpoena was overly broad, Mr. Trump's lawyers said
that it demanded ``voluminous documents related to every facet of the
business and financial affairs of the President and numerous associated
entities --- from the banal to the complex, from drafts and memoranda to
formal records, from source documents to summaries.''

``Simply put,'' they added, ``it asks for everything.''

Even if the subpoena is ultimately enforced and Mr. Vance's office
obtains the records, they are unlikely to become public anytime soon.
The records would be covered by grand jury secrecy rules and might only
emerge if charges were later filed and they were introduced in a trial.

Advertisement

\protect\hyperlink{after-bottom}{Continue reading the main story}

\hypertarget{site-index}{%
\subsection{Site Index}\label{site-index}}

\hypertarget{site-information-navigation}{%
\subsection{Site Information
Navigation}\label{site-information-navigation}}

\begin{itemize}
\tightlist
\item
  \href{https://help.nytimes.com/hc/en-us/articles/115014792127-Copyright-notice}{©~2020~The
  New York Times Company}
\end{itemize}

\begin{itemize}
\tightlist
\item
  \href{https://www.nytco.com/}{NYTCo}
\item
  \href{https://help.nytimes.com/hc/en-us/articles/115015385887-Contact-Us}{Contact
  Us}
\item
  \href{https://www.nytco.com/careers/}{Work with us}
\item
  \href{https://nytmediakit.com/}{Advertise}
\item
  \href{http://www.tbrandstudio.com/}{T Brand Studio}
\item
  \href{https://www.nytimes.com/privacy/cookie-policy\#how-do-i-manage-trackers}{Your
  Ad Choices}
\item
  \href{https://www.nytimes.com/privacy}{Privacy}
\item
  \href{https://help.nytimes.com/hc/en-us/articles/115014893428-Terms-of-service}{Terms
  of Service}
\item
  \href{https://help.nytimes.com/hc/en-us/articles/115014893968-Terms-of-sale}{Terms
  of Sale}
\item
  \href{https://spiderbites.nytimes.com}{Site Map}
\item
  \href{https://help.nytimes.com/hc/en-us}{Help}
\item
  \href{https://www.nytimes.com/subscription?campaignId=37WXW}{Subscriptions}
\end{itemize}
