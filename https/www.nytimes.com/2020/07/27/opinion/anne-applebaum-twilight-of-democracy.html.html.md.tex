Sections

SEARCH

\protect\hyperlink{site-content}{Skip to
content}\protect\hyperlink{site-index}{Skip to site index}

\href{https://myaccount.nytimes.com/auth/login?response_type=cookie\&client_id=vi}{}

\href{https://www.nytimes.com/section/todayspaper}{Today's Paper}

\href{/section/opinion}{Opinion}\textbar{}Twilight of the Liberal Right

\href{https://nyti.ms/307aiF0}{https://nyti.ms/307aiF0}

\begin{itemize}
\item
\item
\item
\item
\item
\item
\end{itemize}

Advertisement

\protect\hyperlink{after-top}{Continue reading the main story}

\href{/section/opinion}{Opinion}

Supported by

\protect\hyperlink{after-sponsor}{Continue reading the main story}

\hypertarget{twilight-of-the-liberal-right}{%
\section{Twilight of the Liberal
Right}\label{twilight-of-the-liberal-right}}

Conservatism always contained the seeds of authoritarianism.

\href{https://www.nytimes.com/by/michelle-goldberg}{\includegraphics{https://static01.nyt.com/images/2018/04/02/opinion/michelle-goldberg/michelle-goldberg-thumbLarge.png}}

By \href{https://www.nytimes.com/by/michelle-goldberg}{Michelle
Goldberg}

Opinion Columnist

\begin{itemize}
\item
  July 27, 2020
\item
  \begin{itemize}
  \item
  \item
  \item
  \item
  \item
  \item
  \end{itemize}
\end{itemize}

\includegraphics{https://static01.nyt.com/images/2020/07/27/opinion/27goldbergWeb/27goldbergWeb-articleLarge.jpg?quality=75\&auto=webp\&disable=upscale}

\hypertarget{listen-to-this-op-ed}{%
\subsubsection{Listen to This Op-Ed}\label{listen-to-this-op-ed}}

Audio Recording by Audm

\emph{To hear more audio stories from publishers like The New York
Times, download}
\href{https://www.audm.com/?utm_source=nytmag\&utm_medium=embed\&utm_campaign=left_behind_draper}{**}
\href{https://www.audm.com/?utm_source=nytopinion\&utm_medium=embed\&utm_campaign=twilight_liberal_right}{\emph{Audm
for iPhone or Android}}\emph{.}

Anne Applebaum's new book, ``Twilight of Democracy: The Seductive Lure
of Authoritarianism,'' begins cinematically, with a party she threw at a
Polish manor house to mark the dawn of the new millennium.

Applebaum's husband was then the deputy foreign minister in Poland's
center-right government; she was a right-leaning journalist who would go
on to write a Pulitzer Prize-winning history of the Soviet gulag. Many
of the guests came from the cosmopolitan anti-Communist intelligentsia.
About half of them, she writes, no longer speak to the other half.

In ``Twilight of Democracy,'' Applebaum tries to understand why so many
of her old friends --- conservatives who once fancied themselves
champions of democracy and classical liberalism --- have become paranoid
right-wing populists. ``Were some of our friends always closet
authoritarians?'' she asks. ``Or have the people with whom we clinked
glasses in the first minutes of the new millennium somehow changed over
the subsequent two decades?''

To Applebaum, today's right, in both America and Europe, ``has little in
common with most of the political movements that have been so described
since the Second World War.'' Until recently, she writes, the right was
``dedicated not just to representative democracy, but to religious
tolerance, independent judiciaries, free press and speech, economic
integration, international institutions, the trans-Atlantic alliance and
a political idea of `the West.''' What happened?

Like Applebaum, I'm astonished to see erstwhile Cold Warriors abase
themselves before Vladimir Putin. But I think she's working from a
mistaken premise about what once constituted conservatism. Liberal
democracy per se was never the animating passion of the trans-Atlantic
right --- anti-Communism was. When the threat of Communist expansion
disappeared, so did most of the right's commitment to a set of values
that, it's now evident, were purely instrumental.

Reading Applebaum's book, I kept thinking of an
\href{https://www.thenation.com/article/archive/exclusive-lee-atwaters-infamous-1981-interview-southern-strategy/}{infamous
1981 interview} given by the Republican campaign consultant Lee Atwater.
In the 1950s, Atwater said, Southern conservatives would just repeat a
vile racial slur. By 1968, ``that hurts you, backfires,'' he said. ``So
you say stuff like forced busing, states' rights, and all that stuff.''
From there, right-wing politics grew even more abstract, so ``now you're
talking about cutting taxes, and all these things you're talking about
are totally economic things and a byproduct of them is, blacks get hurt
worse than whites,'' he said.

There were always some American conservatives who really were in it for
laissez-faire economics. But it's now clear that those conservatives
were wrong about their movement's animating passion. So too with those
on the center-right who thought their comrades were opposed to
authoritarianism on principle.

Back when the idea of a President Trump still seemed an absurdist
impossibility, the political theorist Corey Robin wrote, in his 2011
book ``The Reactionary Mind,'' about the recurring argument that
conservatism had slipped its sober mooring to become populist and
radical.

He saw this as a misunderstanding of the right. In his view, reaction
has always had a revolutionary edge. Conservatism, he wrote, seeks to
``make privilege popular, to transform a tottering old regime into a
dynamic, ideologically coherent movement of the masses.'' Seen this way,
corrupt autocratic populists like Trump and Viktor Orban of Hungary fit
quite neatly into the tradition Applebaum was once part of.

In her book, Applebaum explores the purported ideological evolution of
the Fox News host and Trump sycophant Laura Ingraham, an anti-immigrant
demagogue who has three adopted immigrant children. In the 1990s,
Applebaum associated Ingraham with a ``kind of post-Cold War optimism,''
an American conservatism that was ``energetic, reformist and generous.''

But it's hard to see what was ever reformist, never mind generous, about
Ingraham. She first came to
\href{https://www.nytimes.com/1984/07/16/us/dartmouth-group-privacy-battle-concord-nh-july-15-ap-student-reporter-s-taping.html}{public
notice} as the editor of a conservative college newspaper who sent an
undercover reporter to a meeting of a gay student group and published
attendees' intimate revelations.

Many adults, of course, transcend their college selves, but Ingraham
never seemed to. It was 2003, not 2016, that Ingraham complained about
``police departments, hospitals, courts, schools and government
agencies'' that ``now prefer hiring multilingual employees owing to the
number of illegal and non-English-speaking immigrants in the
community.'' Her conversion to Trumpism doesn't require much
explanation.

I'm genuinely
\href{https://www.nytimes.com/2019/10/28/opinion/trump-human-scum-tweet.html}{grateful
for the moral courage} and concrete political work of anti-Trump
conservatives. It can't be easy to break with the politics and the
people that have defined one's life. I'm aware, too, that the left has
its own ingrained pathologies; Applebaum's center-right views were
shaped by the lived reality of Soviet Communism.

``Twilight of Democracy'' is certainly worth reading. Applebaum has a
keen understanding of how conspiracism and corruption intertwine to
suffocate democracy. Her description of Poland's Law and Justice
government, which has ``put a fantasy at the heart of government
policy,'' helps illuminate the role Trump's obsession with the ``deep
state'' has played in our own rolling catastrophe.

But there's no mystery in the right's surrender to authoritarianism,
because for many of the people Applebaum describes, it wasn't a
surrender at all. It was a liberation.

\emph{The Times is committed to publishing}
\href{https://www.nytimes.com/2019/01/31/opinion/letters/letters-to-editor-new-york-times-women.html}{\emph{a
diversity of letters}} \emph{to the editor. We'd like to hear what you
think about this or any of our articles. Here are some}
\href{https://help.nytimes.com/hc/en-us/articles/115014925288-How-to-submit-a-letter-to-the-editor}{\emph{tips}}\emph{.
And here's our email:}
\href{mailto:letters@nytimes.com}{\emph{letters@nytimes.com}}\emph{.}

\emph{Follow The New York Times Opinion section on}
\href{https://www.facebook.com/nytopinion}{\emph{Facebook}}\emph{,}
\href{http://twitter.com/NYTOpinion}{\emph{Twitter (@NYTopinion)}}
\emph{and}
\href{https://www.instagram.com/nytopinion/}{\emph{Instagram}}\emph{.}

Advertisement

\protect\hyperlink{after-bottom}{Continue reading the main story}

\hypertarget{site-index}{%
\subsection{Site Index}\label{site-index}}

\hypertarget{site-information-navigation}{%
\subsection{Site Information
Navigation}\label{site-information-navigation}}

\begin{itemize}
\tightlist
\item
  \href{https://help.nytimes.com/hc/en-us/articles/115014792127-Copyright-notice}{©~2020~The
  New York Times Company}
\end{itemize}

\begin{itemize}
\tightlist
\item
  \href{https://www.nytco.com/}{NYTCo}
\item
  \href{https://help.nytimes.com/hc/en-us/articles/115015385887-Contact-Us}{Contact
  Us}
\item
  \href{https://www.nytco.com/careers/}{Work with us}
\item
  \href{https://nytmediakit.com/}{Advertise}
\item
  \href{http://www.tbrandstudio.com/}{T Brand Studio}
\item
  \href{https://www.nytimes.com/privacy/cookie-policy\#how-do-i-manage-trackers}{Your
  Ad Choices}
\item
  \href{https://www.nytimes.com/privacy}{Privacy}
\item
  \href{https://help.nytimes.com/hc/en-us/articles/115014893428-Terms-of-service}{Terms
  of Service}
\item
  \href{https://help.nytimes.com/hc/en-us/articles/115014893968-Terms-of-sale}{Terms
  of Sale}
\item
  \href{https://spiderbites.nytimes.com}{Site Map}
\item
  \href{https://help.nytimes.com/hc/en-us}{Help}
\item
  \href{https://www.nytimes.com/subscription?campaignId=37WXW}{Subscriptions}
\end{itemize}
