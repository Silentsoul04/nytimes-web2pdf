Sections

SEARCH

\protect\hyperlink{site-content}{Skip to
content}\protect\hyperlink{site-index}{Skip to site index}

\href{https://myaccount.nytimes.com/auth/login?response_type=cookie\&client_id=vi}{}

\href{https://www.nytimes.com/section/todayspaper}{Today's Paper}

\href{/section/opinion}{Opinion}\textbar{}What Will a Post-Trump G.O.P.
Look Like?

\href{https://nyti.ms/3jFfV4Z}{https://nyti.ms/3jFfV4Z}

\begin{itemize}
\item
\item
\item
\item
\item
\item
\end{itemize}

Advertisement

\protect\hyperlink{after-top}{Continue reading the main story}

\href{/section/opinion}{Opinion}

Supported by

\protect\hyperlink{after-sponsor}{Continue reading the main story}

\hypertarget{what-will-a-post-trump-gop-look-like}{%
\section{What Will a Post-Trump G.O.P. Look
Like?}\label{what-will-a-post-trump-gop-look-like}}

And consider, what will it take for the Republican Party to begin to
heal itself?

\href{https://www.nytimes.com/by/bret-stephens}{\includegraphics{https://static01.nyt.com/images/2017/08/27/insider/bretstephens/bretstephens-thumbLarge-v6.png}}

By \href{https://www.nytimes.com/by/bret-stephens}{Bret Stephens}

Opinion Columnist

\begin{itemize}
\item
  July 27, 2020
\item
  \begin{itemize}
  \item
  \item
  \item
  \item
  \item
  \item
  \end{itemize}
\end{itemize}

\includegraphics{https://static01.nyt.com/images/2020/07/28/opinion/27stephensWeb/27stephensWeb-articleLarge.jpg?quality=75\&auto=webp\&disable=upscale}

If Donald Trump stages another come-from-behind victory in November ---
helped, in all likelihood,
\href{https://www.nytimes.com/2020/07/26/us/protests-portland-seattle-trump.html?action=click\&module=Top\%20Stories\&pgtype=Homepage}{by
the collapse of public order in American cities} --- the Republican
Party will become an oddity for the Trump Organization: the only entity
it owns but does not brand. Not only will Trump remain in office for
another term, but the Trumpers will also dominate the G.O.P. for another
generation.

Look for Tom Cotton to be the likely nominee in 2024 (with --- why not?
--- Laura Ingraham as his running mate).

And if Trump loses? Then the future of the party will be up for grabs.
It's time to start thinking about who can grab it, who should, and who
will.

Much depends on the margin of defeat. If it's razor thin and comes down
to a vote-count dispute in a single state, as it did in Florida in 2000,
Trump will almost surely allege fraud, claim victory and set off a
constitutional crisis. As Ohio State law professor Edward Foley noted
last year
\href{https://lawecommons.luc.edu/cgi/viewcontent.cgi?article=2719\&context=luclj}{in
a must-read law review article}, a state like Pennsylvania could send
competing certificates of electoral votes to Congress. Interpretive
ambiguities in the 12th Amendment and the Electoral Count Act of 1887
could deadlock the House and the Senate. We could have two self-declared
presidents on the eve of next year's inauguration.

Who controls the nuclear football in that event is a question someone
needs to start thinking about right now.

But let's assume Trump loses narrowly but indisputably. In that case,
the Trump family will do what it can to retain control of the G.O.P.

Tommy Hicks Jr.,
\href{https://www.buzzfeednews.com/article/tariniparti/trump-tommy-hicks-rnc-co-chair}{the
current Republican National Committee co-chairman}, is one possible
candidate to move up to become chairman, and run the R.N.C., but the
likelier choice is Hicks's good friend Donald Trump Jr. The Trumpers
will make the argument that NeverTrumpers cost them the election and are
thus responsible for everything bad that might happen in a Biden
administration, from crime on the streets to liberal Supreme Court picks
to some future Benghazi-type episode.

Something unpleasant might come of this. It tends to happen whenever a
large mass of conformists convince themselves that they've been betrayed
by a nonconforming minority in their midst.

Then there's the third scenario: An overwhelming and humiliating Trump
defeat, on the order of George H.W. Bush's 168 to 370 electoral vote
loss to Bill Clinton in 1992.

The infighting will begin the moment Florida, North Carolina or any
other must-win state for Trump is called for Joe Biden. It will pit two
main camps against each other. On the right, it will be the What Were We
Thinking? side of the party. On the further right, the Trump Didn't Go
Far Enough side. Think of it as a cage match between Marco Rubio and
Tucker Carlson for the soul of the G.O.P.

Both sides will recognize that Trump was a uniquely incompetent
executive who --- as in his business dealings --- always proved his own
worst enemy, always squandered his luck, never learned from his
mistakes, never grew in office. Both sides will want to wash their hands
of the soon-to-be-former president, his obnoxious relatives, their
intellectual vacuity and their self-dealing ways. And both will have to
tread carefully around a wounded and bitter man who, like a minefield
laid for some long-ago war, still has the power to kill anyone who
missteps.

That's where agreement ends. The What Were We Thinking? Republicans will
want to hurry the party back to some version of what it was when Paul
Ryan was its star. They'll want to pretend that Trump never happened.
They will organize a task force composed of former party worthies to
write an election post-mortem,
\href{https://online.wsj.com/public/resources/documents/RNCreport03182013.pdf}{akin
to what then-G.O.P. chair Reince Priebus did after 2012,} emphasizing
the need to repair relations with minorities, women and younger voters.
They'll talk up the virtues of Republicans as reformers and
problem-solvers, not Know-Nothings and culture warriors.

The Didn't Go Far Enough camp will make the opposite case. They'll note
that Trump never built the wall, never got U.S. troops out of the Middle
East, never drained the swamp of Beltway corruption, ended NAFTA in name
only, did Wall Street's bidding at Main Street's expense, and ``owned
the libs'' on Twitter while losing the broader battle of ideas. This
camp will seek a new champion: Trump plus a brain.

These are two deeply unattractive versions of the party of Lincoln, one
feckless, the other fanatical. Even so, all who care about the health of
American democracy should hold their noses and hope the feckless side
prevails.

As with the Democrats after Jimmy Carter's defeat in 1980, it will
probably take more than one electoral shellacking for
conservative-leaning voters to appreciate the scale of disaster that
Trump's presidency inflicted on the party and the country. It will
probably also take more than one defeat for the party to learn that
electoral contests should still be waged, and won, near the center of
the ideological spectrum, not the fringe.

But everything has to start somewhere. A decisive Trump loss in November
isn't a sufficient condition for the G.O.P. to begin to heal itself.
It's still a beginning.

\emph{The Times is committed to publishing}
\href{https://www.nytimes.com/2019/01/31/opinion/letters/letters-to-editor-new-york-times-women.html}{\emph{a
diversity of letters}} \emph{to the editor. We'd like to hear what you
think about this or any of our articles. Here are some}
\href{https://help.nytimes.com/hc/en-us/articles/115014925288-How-to-submit-a-letter-to-the-editor}{\emph{tips}}\emph{.
And here's our email:}
\href{mailto:letters@nytimes.com}{\emph{letters@nytimes.com}}\emph{.}

\emph{Follow The New York Times Opinion section on}
\href{https://www.facebook.com/nytopinion}{\emph{Facebook}}\emph{,}
\href{http://twitter.com/NYTOpinion}{\emph{Twitter (@NYTopinion)}}
\emph{and}
\href{https://www.instagram.com/nytopinion/}{\emph{Instagram}}\emph{.}

Advertisement

\protect\hyperlink{after-bottom}{Continue reading the main story}

\hypertarget{site-index}{%
\subsection{Site Index}\label{site-index}}

\hypertarget{site-information-navigation}{%
\subsection{Site Information
Navigation}\label{site-information-navigation}}

\begin{itemize}
\tightlist
\item
  \href{https://help.nytimes.com/hc/en-us/articles/115014792127-Copyright-notice}{©~2020~The
  New York Times Company}
\end{itemize}

\begin{itemize}
\tightlist
\item
  \href{https://www.nytco.com/}{NYTCo}
\item
  \href{https://help.nytimes.com/hc/en-us/articles/115015385887-Contact-Us}{Contact
  Us}
\item
  \href{https://www.nytco.com/careers/}{Work with us}
\item
  \href{https://nytmediakit.com/}{Advertise}
\item
  \href{http://www.tbrandstudio.com/}{T Brand Studio}
\item
  \href{https://www.nytimes.com/privacy/cookie-policy\#how-do-i-manage-trackers}{Your
  Ad Choices}
\item
  \href{https://www.nytimes.com/privacy}{Privacy}
\item
  \href{https://help.nytimes.com/hc/en-us/articles/115014893428-Terms-of-service}{Terms
  of Service}
\item
  \href{https://help.nytimes.com/hc/en-us/articles/115014893968-Terms-of-sale}{Terms
  of Sale}
\item
  \href{https://spiderbites.nytimes.com}{Site Map}
\item
  \href{https://help.nytimes.com/hc/en-us}{Help}
\item
  \href{https://www.nytimes.com/subscription?campaignId=37WXW}{Subscriptions}
\end{itemize}
