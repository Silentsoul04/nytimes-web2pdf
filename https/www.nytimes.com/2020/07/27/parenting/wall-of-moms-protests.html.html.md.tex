Sections

SEARCH

\protect\hyperlink{site-content}{Skip to
content}\protect\hyperlink{site-index}{Skip to site index}

\href{https://www.nytimes.com/section/parenting}{Parenting}

\href{https://myaccount.nytimes.com/auth/login?response_type=cookie\&client_id=vi}{}

\href{https://www.nytimes.com/section/todayspaper}{Today's Paper}

\href{/section/parenting}{Parenting}\textbar{}`The Moms Are Here': `Wall
of Moms' Groups Mobilize Nationwide

\url{https://nyti.ms/2EguAmL}

\begin{itemize}
\item
\item
\item
\item
\item
\end{itemize}

\href{https://www.nytimes.com/news-event/george-floyd-protests-minneapolis-new-york-los-angeles?action=click\&pgtype=Article\&state=default\&region=TOP_BANNER\&context=storylines_menu}{Race
and America}

\begin{itemize}
\tightlist
\item
  \href{https://www.nytimes.com/2020/07/26/us/protests-portland-seattle-trump.html?action=click\&pgtype=Article\&state=default\&region=TOP_BANNER\&context=storylines_menu}{Protesters
  Return to Other Cities}
\item
  \href{https://www.nytimes.com/2020/07/24/us/portland-oregon-protests-white-race.html?action=click\&pgtype=Article\&state=default\&region=TOP_BANNER\&context=storylines_menu}{Portland
  at the Center}
\item
  \href{https://www.nytimes.com/2020/07/23/podcasts/the-daily/portland-protests.html?action=click\&pgtype=Article\&state=default\&region=TOP_BANNER\&context=storylines_menu}{Podcast:
  Showdown in Portland}
\item
  \href{https://www.nytimes.com/interactive/2020/07/16/us/black-lives-matter-protests-louisville-breonna-taylor.html?action=click\&pgtype=Article\&state=default\&region=TOP_BANNER\&context=storylines_menu}{45
  Days in Louisville}
\end{itemize}

Advertisement

\protect\hyperlink{after-top}{Continue reading the main story}

Supported by

\protect\hyperlink{after-sponsor}{Continue reading the main story}

\hypertarget{the-moms-are-here-wall-of-moms-groups-mobilize-nationwide}{%
\section{`The Moms Are Here': `Wall of Moms' Groups Mobilize
Nationwide}\label{the-moms-are-here-wall-of-moms-groups-mobilize-nationwide}}

The movement that started with a few dozen moms in Portland now has
offshoots in cities across the country.

\includegraphics{https://static01.nyt.com/images/2020/07/27/multimedia/27parenting-wall-o-moms-1/merlin_174952425_2ca0ae25-162a-4dd2-a4d6-b5124ef8802a-articleLarge.jpg?quality=75\&auto=webp\&disable=upscale}

By \href{https://www.nytimes.com/by/dani-blum}{Dani Blum}

\begin{itemize}
\item
  Published July 27, 2020Updated July 29, 2020, 10:00 a.m. ET
\item
  \begin{itemize}
  \item
  \item
  \item
  \item
  \item
  \end{itemize}
\end{itemize}

In the flurry of videos and social media posts that have emerged from
the
\href{https://www.nytimes.com/video/us/100000007243995/portland-protests-federal-government.html}{protests
in Portland,}Ore., activist moms are everywhere. They sing lullabies.
They link arm--in--arm, forming a human barricade between protesters and
federal agents. Some wear respirators, gas masks and helmets. Some hand
out sunflowers.

On one night of protests last week, they chanted, ``Feds stay clear! The
moms are here!'' On another, they repeated the word, ``Mama,'' over and
over, echoing a final plea from
\href{https://www.nytimes.com/2020/05/31/us/george-floyd-investigation.html}{George
Floyd}, who was killed in police custody in May.

The ``Wall of Moms,'' as the group calls itself, formed after Beverley
Barnum, who goes by ``Bev,'' 35, a mother of two in Portland, scrolled
through social media posts one night in bed and saw videos of federal
agents placing protesters in unmarked vehicles. Through a Facebook
group, she rallied a few dozen moms who then showed up at a
demonstration on the night of July 18.

Since then, the Wall of Moms has continued to protest nightly in
Portland, with hundreds of women dressed in yellow to identify
themselves as participants turning out. A Wall of Dads has also joined
the front lines of the protests, many carrying leaf blowers to redirect
the tear gas that federal agents have deployed.

\includegraphics{https://static01.nyt.com/images/2020/07/27/multimedia/27-parenting-wall-o-moms-2/merlin_174794100_83a0d1c9-2b1a-4373-9929-ae8a6c6dca37-articleLarge.jpg?quality=75\&auto=webp\&disable=upscale}

More recently, new chapters of Wall of Moms collectives have mobilized
across the country, with several turning out at demonstrations on
Saturday. A group of about 50 Wall of Moms participants marched in
Seattle as clashes between police and
protesters\href{https://www.nytimes.com/2020/07/25/us/protests-seattle-portland.html}{intensified},
said Christine Edgar, who helped organize the local chapter. One of
those arrested, Sonia Alexander, 46, a mother of two, said she was taken
to the emergency room after a flash-bang grenade exploded near her leg.

A delegation in Oakland, Calif., waved large peace signs and marched at
the front of a demonstration; one mom carried a sign that read,
``Schedule: Bath time, Bed time, Fight fascists, Defend Black lives,
Repeat.'' In Aurora, Colo., on Saturday, the Wall of Moms held arms and
flowers at a protest in honor
of\href{https://www.nytimes.com/article/who-was-elijah-mcclain.html}{Elijah
McClain}, a 23-year-old who died last summer after police in Aurora
restrained him with a chokehold. At Saturday's demonstration, a person
was shot and
\href{https://www.denverpost.com/2020/07/25/elijah-mcclain-protest-aurora-saturday/}{wounded
after a car drove through the crowd}.

Wall of Moms groups in Missouri, North Carolina, Alabama, Texas, Chicago
and Maryland are reaching out to local activists and plotting their next
steps, organizers from each group said in interviews.

Gia Gilk, 45, a mother in Albuquerque, N.M., started a Facebook group to
organize a local Wall of Moms chapter last week, thinking she would
attract 30 or 40 members. Within 24 hours, she said, almost 3,000 moms
had signed up. ``I've never done anything like this before,'' said Gilk,
about coordinating the group. ``I just think it's time for us to finally
stand up.''

The Wall of Moms exists to protect and amplify protesters, organizers
say. An official online
\href{https://thewallofmoms.com/wall-of-mom-chapters}{``tool
kit''}(designed by the group in Portland) for starting a Wall of Moms
chapter, stresses that groups should reach out to local Black Lives
Matter and racial justice organizations. Wall of Moms members are
directed to take cues from local activists: to not speak at protests
unless they're asked, and to donate any funds raised to Black-led
organizations.

\hypertarget{wall-of-moms-marchers-are-mostly-white-and-many-are-first-time-protesters}{%
\subsection{Wall of Moms marchers are mostly white, and many are
first-time
protesters.}\label{wall-of-moms-marchers-are-mostly-white-and-many-are-first-time-protesters}}

The Wall of Moms groups consist of predominantly white women who have
garnered a swell of attention that Black mothers protesting in Portland
for months did not receive, participants and organizers said in
interviews. That attention is not lost on the participants nor the
organizations they partner with, some said. ``Black moms are leading
this,'' said Jennifer Kristiansen, 37, a lawyer and Wall of Moms member
who was arrested during a Portland demonstration. ``Moms didn't just
show up a couple nights ago. Black moms have always been there.''

Image

Norma Lewis holds a flower while forming a ``wall of moms'' during a
Black Lives Matter protest in Portland, Ore.~Credit...Noah
Berger/Associated Press

For some longtime activists, the Wall of Moms' momentum demonstrates how
widespread the movement against racism and police brutality has become.
``These moms are realizing people need protection,'' said Nicole
Roussell, 32, who helped organize a protest in Washington, D.C., on
Saturday where Wall of Moms members showed up. ``They're spontaneously
popping up all around the country, days before the protest, and then
coming out --- it just really shows the current movement is getting
broader and wider and deeper.''

In Portland, Wall of Moms has partnered with
\href{https://www.dontshootpdx.org/}{Don't Shoot Portland}, a police
accountability organization. ``Most of them have never protested before.
They felt called,'' said Tai Carpenter, 29, the president of Don't Shoot
Portland. ``A lot of us have been on the front lines for a long time
organizing. To come in at a moment like this, it's crazy. These moms are
seeing it head-on --- it's a different perspective.''

Julianne Jackson, 35, a longtime activist and Portland mother who helped
lead Wall of Moms at a march last week, said the group provides a
powerful symbol. ``When you see a mom get tear-gassed, and they're very
clearly labeled a mom, they're not starting trouble, they're wearing
high-waisted pants and trying to live their life --- when you see that,
it's an incredible sight,'' Jackson said.

\hypertarget{mothers-have-a-legacy-role-in-protest-movements}{%
\subsection{Mothers have a legacy role in protest
movements.}\label{mothers-have-a-legacy-role-in-protest-movements}}

Mothers have long played a critical role in activism in the U.S., but
particularly of late. In 2015 in Chicago, Black mothers founded
\href{https://www.ontheblock.org/about}{Mothers/Men Against Senseless
Killings,} a community group focused on violence prevention and food and
housing insecurity.
\href{https://www.theguardian.com/world/2016/nov/22/mothers-of-the-movement-trayvon-martin-sandra-bland-eric-garner-amadou-diallo-sean-bell}{Mothers
of the Movemen}t, a collective of Black women whose children were killed
in clashes with the police or by gun violence, have traveled across the
country since 2016 to speak about their experiences and push for
legislative change. And in June, Maebel Gebremedhin, 33, a Brooklyn
mother of three, organized a local
\href{https://www.nytimes.com/2020/06/15/parenting/childrens-march-protest-brooklyn.html}{Children's
March}focused on families and kids.

Back in 2013, Collette Flanagan started
\href{https://mothersagainstpolicebrutality.org/about/}{Mothers Against
Police Brutality}, after police in Dallas killed her son Clinton Allen.
While her group is not affiliated with Wall of Moms, Flanagan said in an
interview that she supports them and is ``in awe'' of them. ``The power
of being a mother, whether you have lost a child or not, is that because
you're a mother, you're able to absorb another mother's pain,'' she
said. ``That becomes a very powerful chain of resistance,'' she said.

The Wall of Moms groups are using that bond between mothers in the way
it should be used, she said. ``It can't be penetrated. That's why people
are noticing, because there's nothing else like it.''

That unifying connection is part of what has drawn moms to join the
protest. For the last few nights, Savanna Taylor, 28, of Portland, has
found someone (usually her mom) to watch her 4-year-old son, so she
could join the Wall of Moms in front of the courthouse. Some nights she
arrived back home between 1 a.m. and 3 a.m., after hours of marching and
chanting, after federal agents deployed so much tear gas that some of
the mothers she locked arms with had vomited and wet themselves, she
said. ``Seeing moms in solidarity is what gets people, because they know
we've got kids at home. We're trying to protect everyone's kids as if
they were our own,'' Taylor said.

Advertisement

\protect\hyperlink{after-bottom}{Continue reading the main story}

\hypertarget{site-index}{%
\subsection{Site Index}\label{site-index}}

\hypertarget{site-information-navigation}{%
\subsection{Site Information
Navigation}\label{site-information-navigation}}

\begin{itemize}
\tightlist
\item
  \href{https://help.nytimes.com/hc/en-us/articles/115014792127-Copyright-notice}{©~2020~The
  New York Times Company}
\end{itemize}

\begin{itemize}
\tightlist
\item
  \href{https://www.nytco.com/}{NYTCo}
\item
  \href{https://help.nytimes.com/hc/en-us/articles/115015385887-Contact-Us}{Contact
  Us}
\item
  \href{https://www.nytco.com/careers/}{Work with us}
\item
  \href{https://nytmediakit.com/}{Advertise}
\item
  \href{http://www.tbrandstudio.com/}{T Brand Studio}
\item
  \href{https://www.nytimes.com/privacy/cookie-policy\#how-do-i-manage-trackers}{Your
  Ad Choices}
\item
  \href{https://www.nytimes.com/privacy}{Privacy}
\item
  \href{https://help.nytimes.com/hc/en-us/articles/115014893428-Terms-of-service}{Terms
  of Service}
\item
  \href{https://help.nytimes.com/hc/en-us/articles/115014893968-Terms-of-sale}{Terms
  of Sale}
\item
  \href{https://spiderbites.nytimes.com}{Site Map}
\item
  \href{https://help.nytimes.com/hc/en-us}{Help}
\item
  \href{https://www.nytimes.com/subscription?campaignId=37WXW}{Subscriptions}
\end{itemize}
