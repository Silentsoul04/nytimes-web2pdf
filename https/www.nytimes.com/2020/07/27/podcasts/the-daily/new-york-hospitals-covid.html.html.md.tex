Sections

SEARCH

\protect\hyperlink{site-content}{Skip to
content}\protect\hyperlink{site-index}{Skip to site index}

\href{https://www.nytimes.com/podcasts/the-daily}{The Daily}

\href{https://myaccount.nytimes.com/auth/login?response_type=cookie\&client_id=vi}{}

\href{https://www.nytimes.com/section/todayspaper}{Today's Paper}

\href{/podcasts/the-daily}{The Daily}\textbar{}The Mistakes New York
Made

\url{https://nyti.ms/32ZNeK2}

\begin{itemize}
\item
\item
\item
\item
\item
\item
\end{itemize}

Advertisement

\protect\hyperlink{after-top}{Continue reading the main story}

transcript

Back to The Daily

bars

0:00/33:28

-33:28

transcript

\hypertarget{the-mistakes-new-york-made}{%
\subsection{The Mistakes New York
Made}\label{the-mistakes-new-york-made}}

\hypertarget{hosted-by-michael-barbaro-produced-by-neena-pathak-austin-mitchell-and-andy-mills-and-edited-by-lisa-chow-and-lisa-tobin}{%
\subsubsection{Hosted by Michael Barbaro, produced by Neena Pathak,
Austin Mitchell and Andy Mills, and edited by Lisa Chow and Lisa
Tobin}\label{hosted-by-michael-barbaro-produced-by-neena-pathak-austin-mitchell-and-andy-mills-and-edited-by-lisa-chow-and-lisa-tobin}}

\hypertarget{an-investigation-into-hospitals-during-the-peak-of-the-citys-coronavirus-outbreak-exposed-significant-disparities-in-health-care}{%
\paragraph{An investigation into hospitals during the peak of the city's
coronavirus outbreak exposed significant disparities in health
care.}\label{an-investigation-into-hospitals-during-the-peak-of-the-citys-coronavirus-outbreak-exposed-significant-disparities-in-health-care}}

Monday, July 27th, 2020

\begin{itemize}
\item
  michael barbaro\\
  From The New York Times, I'm Michael Barbaro. This is ``The Daily.''
\item
  {[}music{]}\\
  Today: A Times investigation finds that surviving the coronavirus in
  New York had a lot to do with which hospital a person went to. My
  colleague, investigative reporter Brian Rosenthal, on inequality and
  the pandemic. It's Monday, July 27.
\item
  archived recording (andrew cuomo)\\
  Thank you for being here today. This is an amazing accomplishment.

  Strategy, plan of action all along. Step one: flatten the curve. Step
  two: increase hospital capacity.

  That's what this is all about --- not overwhelming hospital capacity
  and, at the same time, increasing the hospital capacity that we have.
  So if it does exceed those numbers, which it will in most probability,
  that we have the additional capacity to deal with it.
\end{itemize}

michael barbaro

Brian, you have been part of a team investigating how the coronavirus
was handled in New York City. And I'm curious why you undertook this
project. My sense is that New York has done a fairly solid job
flattening the curve over the past few months. So what was your aim?

brian rosenthal

So New York was clearly the first big hotspot for the coronavirus in the
United States. And yes, we did succeed in flattening the curve. But we
also experienced a lot of tragedy along the way. A lot of death and a
lot of heartbreak. And now that the rest of the country is going through
different surges in the virus and different versions of what we went
through in March and April, I think it's really important to look at the
experience in New York --- the successes that we've had, but also the
mistakes that were made. And if you look at what happened in hospitals
in New York in a real close way, you'll see that there were a lot of
mistakes. And as a result, people died.

michael barbaro

And where does that story start in your reporting?

brian rosenthal

When the pandemic began in New York, a team of us on the metro desk
really were trying to follow what was happening. And we realized very
quickly that there was no one story about how this was playing out in
hospitals, because there are 47 different hospitals in New York City.
And each one was having its own experience. So a team of us divided
them. Some of us took the public hospitals. Some of us took more of the
private hospitals. And we started calling doctors, nurses, physician
assistants, all kinds of workers in each of those hospitals.

michael barbaro

And Brian, why does that distinction matter, public and private?

brian rosenthal

Well, the public hospitals are the hospitals that are run by the
government. And they cater mostly to residents who have Medicaid or
Medicare, or don't have any insurance at all. And the private hospitals
are kind of the more elite institutions that we might be familiar with
--- Mount Sinai, N.Y.U. Langone, Columbia, Cornell. And they cater
mostly to wealthier residents with health insurance through their
employer or purchased privately.

michael barbaro

And after you talked to doctors and nurses and staff from all these
different hospitals, both the public and the private, what did you
learn?

brian rosenthal

We found significant differences between the level of care available at
these wealthy private hospitals, mostly in Manhattan, and the public
hospitals and small independent hospitals scattered throughout the other
boroughs. There were differences in basically everything once you walk
in the door.

But the biggest differences were in staffing --- the level of nurses and
doctors and other types of staff that were available on a per patient
basis, as well as the equipment that was available. The age of the
equipment, the type of the equipment, and access to drug trials and
experimental treatments and advanced treatments that cost a lot of money
and may not necessarily always work, but give the patients a fighting
chance. Those are available much more in the private hospitals than the
public hospitals.

michael barbaro

Tell me about the staffing ratios.

{[}music{]}

brian rosenthal

Yeah, so the staffing ratio is very important in whether patients live
or die. Research has shown that. And there are some best practices that
have been established through the years.

If you look at an emergency room, for example, the best practice is that
there should be four patients for every one nurse. So that way, the
nurse is not having too many patients that they are trying to monitor.
And we were able to collect numbers showing the ratios in emergency
rooms at private hospitals versus public hospitals. And you could see
that the ratio is increased at every hospital. But at the private
hospitals, while the ratio went up to one nurse for six or seven
patients, it went up at the public hospitals to one nurse for 10 or 15
or even 20 patients.

michael barbaro

So about twice.

brian rosenthal

Yeah, and in the I.C.U.s, the general ratio is, because the patients are
so severe, it's two patients for every nurse. And again, those ratios
got stretched at every hospital in the city, but in private hospitals,
it would be stretched to three or four patients for every nurse. And in
the public hospitals, it was getting stretched to seven, eight, nine
patients for every nurse, which was obviously very dangerous.

michael barbaro

And what did the staff you talked to say were the consequences in some
of these public hospitals? What did that translate into during the
pandemic?

brian rosenthal

It meant that doctors and nurses have less time to spend with each
patient in public hospitals to see how they were doing, to talk with
them, to run tests, and, perhaps most importantly, just to monitor them.
Almost all of them were on ventilators and really needed to be
constantly monitored. One of the things that we've learned with the
coronavirus is that patients can detoriate very quickly. They can seem
like they're doing fine one minute, and the next minute, they could be
going into cardiac arrest. And at the understaffed public hospitals, we
even heard some cases of patients waking up from medically induced
comas, finding that there were no nurses around, and in their confusion,
actually removing their life supports and dying.

michael barbaro

Wow.

brian rosenthal

It was something that was a pattern, so much of a pattern that at
Elmhurst Hospital --- that overwhelmed hospital that received a lot of
attention --- this happens so often where somebody woke up, confused and
removed their life support because they needed to go to the bathroom.
And they collapsed and they were discovered either in the bathroom or
near the bathroom. Some of the doctors there actually developed a name
for it. They called them ``bathroom codes.'' And in those cases, the
patients were discovered, you know, half an hour later, 45 minutes later
by doctors and nurses who were devastated, because if there had been
staff there monitoring them, they would have been cared for.

michael barbaro

But instead, a nurse was doing the rounds for 15 or 20 other patients.

brian rosenthal

That's right. In every case that we heard about --- at least four cases
at Elmhurt Hospital --- the patients died.

michael barbaro

Mm-hmm. How else do the people you talk to in these hospitals tell you
that staffing impacted mortality?

brian rosenthal

Well, another example is something called ``proning,'' which is quite
simply flipping a patient on their stomach. And it was something that
very quickly, during the pandemic, doctors realized that if they did ---
if they flipped patients on their stomach --- it would help the patient
breathe and could be a useful tool in helping them recover. And so that
was something that was used a lot in New York in private hospitals, but
unfortunately, in public hospitals, there was not the staffing available
to do it.

michael barbaro

Why?

brian rosenthal

Well, it turns out that proning --- just flipping someone on their
stomach --- can actually be quite complicated if they have a bunch of
I.V. lines and tubes running through them. And it can require five or
six people to coordinate all the movements and make sure those lines are
still running while flipping the patient. So it seems very simple. And
the doctors knew that it would help. But in some of those public
hospitals, they were not able to do it because they did not have the
staff available.

michael barbaro

Mm-hmm.

brian rosenthal

One doctor at a small independent hospital told us that out of 10 of the
deaths that he witnessed, he thought two or three of the patients could
have been saved.

michael barbaro

If there had been better staffing.

brian rosenthal

Yeah, if the hospital had the resources of a private hospital.

michael barbaro

Wow. I mean, that's 20 to 30 percent

brian rosenthal

Yeah, I mean, it translates to thousands of people. And we actually
looked at the mortality rates at most of the 47 hospitals in the city.
And in some cases, the mortality rate was three times higher in the
public hospitals in the lower income areas. Some of that mortality
difference could be explained by differences in patient populations ---
you know, underlying health conditions of the patients. But the experts
and the doctors that we talked to said that the quality of care was
definitely a factor in those differences.

michael barbaro

Brian, as horrible as everything you're describing is, it feels like
there's a pretty logical solution to it. Which is taking Covid-19
patients from these overburdened, understaffed public hospitals, and
transferring them to the less burdened, better staffed private
hospitals.

brian rosenthal

You'd think that, yes. And Governor Cuomo even said at the peak of the
pandemic that that was going to happen.

\begin{itemize}
\tightlist
\item
  archived recording (andrew cuomo)\\
  How many beds would you need at the apex? Between 70 and 110,000.
  Right now, we have 53,000 statewide. We have only 36,000 downstate.
  Every hospital by mandate has to add a 50 percent increase. And they
  have all done that. We're setting up extra facilities, which ---
\end{itemize}

michael barbaro

But in the end, it didn't. And why not? Like, what prevents a patient at
Elmhurst Hospital in Queens from being transferred to N.Y.U. Langone,
which happens to be on the east side of Manhattan. It's not that far.

brian rosenthal

Well, Elmhurst is a public hospital. And for decades, they have not
really transferred patients to N.Y.U. Langone. They've transferred
patients to other hospitals within the public system, but they just
don't really work together with the private system.

michael barbaro

So there's no infrastructure set up to make such transfers. And
therefore, they're unlikely to happen.

brian rosenthal

Well, nothing physically prevents a patient from being transferred. But
first of all, the hospital, Elmhurst, may not want to transfer the
patient because there is revenue attached to every patient. Even a
public hospital cares about maximizing its revenue. So the doctor and
the nurse inside the hospital may want very much to transfer a patient
to Langone, but the administrator, the C.E.O. of the hospital, might not
want to do that for financial reasons. So there was a problem on that
end.

And then there's a problem on the other end, because N.Y.U. Langone is a
private hospital. And it wants to treat patients with private health
insurance because that's going to bring the biggest profit. And the
patient coming from Elmhurst, the public hospital, is going to be a
patient without private health insurance. So it's not a patient that
N.Y.U. Langone really wants, anyway. So on both ends, Elmhurst may not
want to transfer the patient, and N.Y.U. Langone might not want to take
the patient.

michael barbaro

So the incentives are not there for this very simple fix to work.

brian rosenthal

That's right. Because the incentive is profit.

michael barbaro

So at the end of the day, were there any transfers between the public
and private hospitals? Any meaningful number of transfers?

brian rosenthal

There were less than 50 ---

michael barbaro

Wow.

brian rosenthal

--- during the whole course of the pandemic, thousands of people in
hospitals. There were less than 50 transfers from public hospitals to
private hospitals.

michael barbaro

That is a genuinely shocking number.

brian rosenthal

Yeah, and again, the transfers were wanted by the doctors and the
nurses. But they didn't end up happening.

michael barbaro

I'm very rarely shocked.

brian rosenthal

Yeah. So that brings us to the other possible solution, which New York
explored and actually put a lot of money into, which was the overflow
hospitals --- makeshift hospitals set up around the city that could take
patients from these overburdened hospitals. But it turns out those
didn't work either.

{[}music{]}

michael barbaro

We'll be right back.

\begin{itemize}
\item
  archived recording 1\\
  Now as we all know, New York is the national epicenter of the
  coronavirus crisis. Now it is all hands on deck there.
\item
  archived recording\\
  And the death toll in New York City from the Covid-19 pandemic has
  climbed to 450, with 26,000 testing positive so far. This is the Naval
  ship Comfort due to arrive in the area on Monday from Virginia. And a
  field hospital ---
\end{itemize}

michael barbaro

Brian, I remember these overflow hospitals really well.

\begin{itemize}
\tightlist
\item
  archived recording (andrew cuomo)\\
  What we're doing here at the Javits Center is constructing four
  emergency hospitals.
\end{itemize}

michael barbaro

I remember Governor Cuomo walking through the Javits Center, this huge
convention center ---

\begin{itemize}
\tightlist
\item
  archived recording (andrew cuomo)\\
  This was never an anticipated use. But you do what you have to do.
  That's the New York way. That's the American way.
\end{itemize}

michael barbaro

--- on the west side of Manhattan, kind of showing off the hundreds of
beds. I remember there being little flowers on the sides of the tables
next to the cots. And I know these were set up in each borough. So what
happened that meant that they didn't actually do their job?

brian rosenthal

Well, let's take the example of the Billie Jean King Tennis Center.

\begin{itemize}
\tightlist
\item
  archived recording\\
  Part of the Billie Jean King National Tennis Center right now is being
  converted into a temporary hospital.
\end{itemize}

brian rosenthal

It's one of the biggest tennis centers in the world. It's where the U.S.
Open is held.

\begin{itemize}
\tightlist
\item
  archived recording\\
  Some patients from nearby Elmhurst Hospital are expected to be
  transferred to the National Tennis Center Hospital.
\end{itemize}

brian rosenthal

It was going to have 470 beds and hundreds of employees that were going
to be available to take patients, specifically from Elmhurst and Queens.

\begin{itemize}
\tightlist
\item
  archived recording\\
  This place will be a lifesaving place. It's going to help take the
  pressure off Elmhurst.
\end{itemize}

brian rosenthal

It was supposed to be a crucial facility. But the first problem that it
had was bureaucracy. There were paperwork requirements. There were all
kinds of orientations that the doctors needed to do, training on the
computer systems, training on the type of equipment that was going to be
there and the paperwork that had to be filled out. And you had doctors
in the middle of the peak of the pandemic, when people were dying,
spending time doing things that had nothing to do with patient care.

Another problem was that the hospital was suffering from a bit of an
identity crisis about which types of patients it was going to treat and
at different points of time, even within the week that it was being set
up.

\begin{itemize}
\tightlist
\item
  archived recording\\
  As of this morning, the complex was not likely to include Covid-19
  patients. The U.S. Open is ---
\end{itemize}

brian rosenthal

City officials were changing their mind about that question.

\begin{itemize}
\tightlist
\item
  archived recording\\
  This facility will be able to take people from Elmhurst, other
  coronavirus patients, bring them over here, relieve some of that
  pressure immediately.
\end{itemize}

brian rosenthal

And they were conveying different directives to other hospitals about
which types of patients they should be transferring to the Billie Jean
King Tennis Center. And they ended up crafting a series of rules that
were very restrictive about the types of patients that could go to
Billie Jean King. There were over 25 different exclusionary criteria,
which is basically disqualifying conditions that if the patient has,
they can't go to Billie Jean King. And one of them was just the fact
that the patient had a fever, which is a hallmark symptom of the
coronavirus.

But at the same time, there were also a series of rules about the types
of patients that they would not see because they were not severe enough.
They were patients that were quarantining with the virus in hotels, and,
in some cases, ended up dying in those hotels. And when employees at
Billie Jean King asked why they couldn't see and care for those
patients, they were told that those patients aren't severe enough to be
at Billie Jean King. So they couldn't see the really severe patients.
They also couldn't see the patients that were not severe. And as a
result, they didn't end up treating much of anybody.

michael barbaro

Hm. So did they see any patients?

brian rosenthal

Well, hold on, because there's another problem, and it relates to
ambulances. So in the peak of the pandemic, if you were at your house
and you called 911, the ambulance that arrived could not take you to
Billie Jean King directly.

michael barbaro

Why not?

brian rosenthal

Well, the city had decided that ambulances would have to first take
patients to a hospital, even if they're overburdened. And that hospital
would triage the patient and then figure out where to send them. So
Billie Jean King was really only taking transfers from other hospitals.
But even the transferring process was blocked by ambulance regulations.
Because there were situations where hospitals wanted to transfer
patients, but there was no ambulance available to transfer them. And
Billie Jean King had its own ambulances on site that could have gone to
the hospital and picked up the patient. But the regular hospitals had
exclusivity agreements with ambulance companies that said that nobody
could pick up their patients. They could only send patients out in their
own ambulances with these companies.

michael barbaro

And so that patient is just going to stay at Elmhurst and not get
transferred to Billie Jean King.

brian rosenthal

Until an ambulance from that company with the exclusive agreement is
available, yes. And that happened, so patients had to wait.

michael barbaro

OK, so back to that question. In the end, how many patients made it into
this Billie Jean King overflow hospital?

brian rosenthal

79.

michael barbaro

Geez.

brian rosenthal

That's 79 throughout the course of the month that the Billie Jean King
Center was open. At any one time, there were maybe 20 or 30 patients
there.

michael barbaro

So what were all the staff, the nurses, the doctors at Billie Jean King
Field Hospital, overflow hospital, what were they doing?

brian rosenthal

Well, in many cases, nothing. You know, I want to be clear, because the
doctors and nurses and other staffers that came to work at Billie Jean
King, they came, in many cases, from around the country. They were
experienced medical professionals. And they really wanted to help. And
they were extremely well paid as well. They were paid, the doctors in
many cases, over \$600 an hour.

michael barbaro

Wow.

brian rosenthal

So they showed up to work ready to help, eager to help, but no patients
came in the door. So I talked to some of them that said that in the peak
of the pandemic, they were just sitting around on their phone all day.
One of the workers at Billie Jean King who I talked with, who is a nurse
practitioner who came up from Baltimore, she said, ``I basically got
paid \$2,000 a day to sit on my phone and look at Facebook. We all felt
guilty. I felt really ashamed, to be honest.''

michael barbaro

Right, because like you said, they came to serve the public in New York.
In particular, a public that was trying to get into overburdened public
hospitals, and here they are, not able to do that because of exclusive
ambulance agreements and kind of bureaucratic nonsense.

brian rosenthal

That's right, yeah. The facility ended up closing in early May after the
peak of the pandemic. There was really no need for it. And ultimately,
for its work in treating 79 patients, so far the city has paid the
contractor about \$52 million. But the bill is actually still coming in.
The total bill might actually be over \$100 million.

michael barbaro

Brian, whenever we talk about inequality, it can feel like a very
out-of-reach set of solutions, right? Because almost by definition, it
is systemic deeply rooted issues. But in the case of hospitals in New
York, the solutions felt very practical and very simple, as you have
laid them out. You know, cancel those exclusive ambulance agreements.
Transfer patients from public to private hospitals. They all seemed
quite within reach.

brian rosenthal

Yeah, I think that's right. And I think it's also important to note that
even while the pandemic was going on, there were plenty of doctors and
other hospital workers who noticed these inequalities and were trying to
fix them. We talked with a number of doctors that actually rotated
between working in the private hospitals and working in the public
hospitals, and were trying to raise alarms, and even hospitals within
the private networks trying to push their bosses to do more to address
inequalities. But the reality was by that point, the inequalities were
so ingrained into the hospital system that there wasn't a whole lot that
they could do.

I think the story of what happened in hospitals in New York, in the
height of the coronavirus pandemic, is really a story about officials
and hospital executives and bureaucrats who accepted these inequalities
in the system long ago, and have obviously known about inequalities for
decades, but chose not to address them and found that they got exposed
in this pandemic.

michael barbaro

But of course, in that case, isn't it the role of government? Isn't it
the role of the mayor of New York City, the governor of New York, to not
accept those kinds of inequalities, and to do everything in their power
to slice through that kind of complacency in the midst of a public
health crisis?

brian rosenthal

Yeah, and I think if you talk to the governor or the mayor, if you had
them sitting here, they would say that they did as much as they could.
And they did certainly spend a lot of money setting up field hospitals
to help and set up a system to help with transfers.

But one thing that I think is very telling is when I called the
governor's office to ask why more patients were not transferred from
overburdened hospitals to private hospitals that had open beds, the
governor's office said that they accommodated every transfer that was
requested by the hospitals. And they felt like that was their job. So
they handled each request, but they were not willing to force hospitals
to transfer. They were not willing to take that more fundamental step in
changing the government's role. And I think it's because they themselves
kind of accepted the reality as it was, that there were going to be
inequalities between different types of hospitals and different types of
patients.

michael barbaro

Right, to say that they processed every request they got for transfers
is to say, like, I caught a couple of the raindrops in this giant storm,
but what about that flood down the street?

brian rosenthal

Right, it's not addressing the more fundamental problem.

michael barbaro

Brian, at the start of our conversation, you mentioned that peak
hospitalizations are now occurring throughout much of the rest of the
country. It's subsided in New York, but it's now happening in Texas.
It's happening in Florida. It's happening in Arizona.

brian rosenthal

Yes.

michael barbaro

I know that your investigation was into the hospitals in New York. But
do we expect that what you saw in New York --- these inequities, these
private-public hospital disparities --- that they are likely to play out
across the rest of the country?

brian rosenthal

There will definitely be disparities in every city in America. I think
the question is whether other cities have learned from New York and are
going to be willing to put in place systems and policies that can help
balance out those inequalities in a more real way than we saw in New
York. And I think that's still to be determined.

michael barbaro

Brian, thank you very much. We appreciate it.

brian rosenthal

Thank you.

{[}music{]}

michael barbaro

On Sunday, The Times reported that the total number of infections in
Florida has now surpassed that of New York, making the state the new
epicenter of the pandemic. Florida has nearly 424,000 reported cases,
compared with about 415,000 cases in New York. We'll be right back.

Here's what else you need to know today.

The Times reports that the presence of federal agents in Portland
galvanized thousands of people to join protests across the country over
the weekend, reviving nationwide protests that had largely dissipated.

\begin{itemize}
\tightlist
\item
  archived recording\\
  Black lives matter! {[}CAR HONKING{]} Black lives matter! {[}CAR
  HONKING{]}
\end{itemize}

michael barbaro

One of the most intense protests was in Seattle, where a demonstration
against police brutality turned violent, after some protesters lit a
detention center on fire, smashed windows and damaged a police building.

In response, police declared the protest a riot, fired flash grenades,
unleashed pepper spray and rushed into crowds, knocking people to the
ground.

That's it for ``The Daily.'' I'm Michael Barbaro. See you tomorrow.

\href{https://www.nytimes.com/column/the-daily}{\includegraphics{https://static01.nyt.com/images/2017/01/29/podcasts/the-daily-album-art/the-daily-album-art-square320-v4.png}The
Daily}Subscribe:

\begin{itemize}
\tightlist
\item
  \href{https://itunes.apple.com/us/podcast/id1200361736}{Apple
  Podcasts}
\item
  \href{https://www.google.com/podcasts?feed=aHR0cHM6Ly9yc3MuYXJ0MTkuY29tL3RoZS1kYWlseQ\%3D\%3D}{Google
  Podcasts}
\end{itemize}

\hypertarget{the-mistakes-new-york-made-1}{%
\section{The Mistakes New York
Made}\label{the-mistakes-new-york-made-1}}

\hypertarget{an-investigation-into-hospitals-during-the-peak-of-the-citys-coronavirus-outbreak-exposed-significant-disparities-in-health-care-1}{%
\subsection{An investigation into hospitals during the peak of the
city's coronavirus outbreak exposed significant disparities in health
care.}\label{an-investigation-into-hospitals-during-the-peak-of-the-citys-coronavirus-outbreak-exposed-significant-disparities-in-health-care-1}}

Hosted by Michael Barbaro, produced by Neena Pathak, Austin Mitchell and
Andy Mills, and edited by Lisa Chow and Lisa Tobin

Transcript

transcript

Back to The Daily

bars

0:00/33:28

-0:00

transcript

\hypertarget{the-mistakes-new-york-made-2}{%
\subsection{The Mistakes New York
Made}\label{the-mistakes-new-york-made-2}}

\hypertarget{hosted-by-michael-barbaro-produced-by-neena-pathak-austin-mitchell-and-andy-mills-and-edited-by-lisa-chow-and-lisa-tobin-1}{%
\subsubsection{Hosted by Michael Barbaro, produced by Neena Pathak,
Austin Mitchell and Andy Mills, and edited by Lisa Chow and Lisa
Tobin}\label{hosted-by-michael-barbaro-produced-by-neena-pathak-austin-mitchell-and-andy-mills-and-edited-by-lisa-chow-and-lisa-tobin-1}}

\hypertarget{an-investigation-into-hospitals-during-the-peak-of-the-citys-coronavirus-outbreak-exposed-significant-disparities-in-health-care-2}{%
\paragraph{An investigation into hospitals during the peak of the city's
coronavirus outbreak exposed significant disparities in health
care.}\label{an-investigation-into-hospitals-during-the-peak-of-the-citys-coronavirus-outbreak-exposed-significant-disparities-in-health-care-2}}

Monday, July 27th, 2020

\begin{itemize}
\item
  michael barbaro\\
  From The New York Times, I'm Michael Barbaro. This is ``The Daily.''
\item
  {[}music{]}\\
  Today: A Times investigation finds that surviving the coronavirus in
  New York had a lot to do with which hospital a person went to. My
  colleague, investigative reporter Brian Rosenthal, on inequality and
  the pandemic. It's Monday, July 27.
\item
  archived recording (andrew cuomo)\\
  Thank you for being here today. This is an amazing accomplishment.

  Strategy, plan of action all along. Step one: flatten the curve. Step
  two: increase hospital capacity.

  That's what this is all about --- not overwhelming hospital capacity
  and, at the same time, increasing the hospital capacity that we have.
  So if it does exceed those numbers, which it will in most probability,
  that we have the additional capacity to deal with it.
\end{itemize}

michael barbaro

Brian, you have been part of a team investigating how the coronavirus
was handled in New York City. And I'm curious why you undertook this
project. My sense is that New York has done a fairly solid job
flattening the curve over the past few months. So what was your aim?

brian rosenthal

So New York was clearly the first big hotspot for the coronavirus in the
United States. And yes, we did succeed in flattening the curve. But we
also experienced a lot of tragedy along the way. A lot of death and a
lot of heartbreak. And now that the rest of the country is going through
different surges in the virus and different versions of what we went
through in March and April, I think it's really important to look at the
experience in New York --- the successes that we've had, but also the
mistakes that were made. And if you look at what happened in hospitals
in New York in a real close way, you'll see that there were a lot of
mistakes. And as a result, people died.

michael barbaro

And where does that story start in your reporting?

brian rosenthal

When the pandemic began in New York, a team of us on the metro desk
really were trying to follow what was happening. And we realized very
quickly that there was no one story about how this was playing out in
hospitals, because there are 47 different hospitals in New York City.
And each one was having its own experience. So a team of us divided
them. Some of us took the public hospitals. Some of us took more of the
private hospitals. And we started calling doctors, nurses, physician
assistants, all kinds of workers in each of those hospitals.

michael barbaro

And Brian, why does that distinction matter, public and private?

brian rosenthal

Well, the public hospitals are the hospitals that are run by the
government. And they cater mostly to residents who have Medicaid or
Medicare, or don't have any insurance at all. And the private hospitals
are kind of the more elite institutions that we might be familiar with
--- Mount Sinai, N.Y.U. Langone, Columbia, Cornell. And they cater
mostly to wealthier residents with health insurance through their
employer or purchased privately.

michael barbaro

And after you talked to doctors and nurses and staff from all these
different hospitals, both the public and the private, what did you
learn?

brian rosenthal

We found significant differences between the level of care available at
these wealthy private hospitals, mostly in Manhattan, and the public
hospitals and small independent hospitals scattered throughout the other
boroughs. There were differences in basically everything once you walk
in the door.

But the biggest differences were in staffing --- the level of nurses and
doctors and other types of staff that were available on a per patient
basis, as well as the equipment that was available. The age of the
equipment, the type of the equipment, and access to drug trials and
experimental treatments and advanced treatments that cost a lot of money
and may not necessarily always work, but give the patients a fighting
chance. Those are available much more in the private hospitals than the
public hospitals.

michael barbaro

Tell me about the staffing ratios.

{[}music{]}

brian rosenthal

Yeah, so the staffing ratio is very important in whether patients live
or die. Research has shown that. And there are some best practices that
have been established through the years.

If you look at an emergency room, for example, the best practice is that
there should be four patients for every one nurse. So that way, the
nurse is not having too many patients that they are trying to monitor.
And we were able to collect numbers showing the ratios in emergency
rooms at private hospitals versus public hospitals. And you could see
that the ratio is increased at every hospital. But at the private
hospitals, while the ratio went up to one nurse for six or seven
patients, it went up at the public hospitals to one nurse for 10 or 15
or even 20 patients.

michael barbaro

So about twice.

brian rosenthal

Yeah, and in the I.C.U.s, the general ratio is, because the patients are
so severe, it's two patients for every nurse. And again, those ratios
got stretched at every hospital in the city, but in private hospitals,
it would be stretched to three or four patients for every nurse. And in
the public hospitals, it was getting stretched to seven, eight, nine
patients for every nurse, which was obviously very dangerous.

michael barbaro

And what did the staff you talked to say were the consequences in some
of these public hospitals? What did that translate into during the
pandemic?

brian rosenthal

It meant that doctors and nurses have less time to spend with each
patient in public hospitals to see how they were doing, to talk with
them, to run tests, and, perhaps most importantly, just to monitor them.
Almost all of them were on ventilators and really needed to be
constantly monitored. One of the things that we've learned with the
coronavirus is that patients can detoriate very quickly. They can seem
like they're doing fine one minute, and the next minute, they could be
going into cardiac arrest. And at the understaffed public hospitals, we
even heard some cases of patients waking up from medically induced
comas, finding that there were no nurses around, and in their confusion,
actually removing their life supports and dying.

michael barbaro

Wow.

brian rosenthal

It was something that was a pattern, so much of a pattern that at
Elmhurst Hospital --- that overwhelmed hospital that received a lot of
attention --- this happens so often where somebody woke up, confused and
removed their life support because they needed to go to the bathroom.
And they collapsed and they were discovered either in the bathroom or
near the bathroom. Some of the doctors there actually developed a name
for it. They called them ``bathroom codes.'' And in those cases, the
patients were discovered, you know, half an hour later, 45 minutes later
by doctors and nurses who were devastated, because if there had been
staff there monitoring them, they would have been cared for.

michael barbaro

But instead, a nurse was doing the rounds for 15 or 20 other patients.

brian rosenthal

That's right. In every case that we heard about --- at least four cases
at Elmhurt Hospital --- the patients died.

michael barbaro

Mm-hmm. How else do the people you talk to in these hospitals tell you
that staffing impacted mortality?

brian rosenthal

Well, another example is something called ``proning,'' which is quite
simply flipping a patient on their stomach. And it was something that
very quickly, during the pandemic, doctors realized that if they did ---
if they flipped patients on their stomach --- it would help the patient
breathe and could be a useful tool in helping them recover. And so that
was something that was used a lot in New York in private hospitals, but
unfortunately, in public hospitals, there was not the staffing available
to do it.

michael barbaro

Why?

brian rosenthal

Well, it turns out that proning --- just flipping someone on their
stomach --- can actually be quite complicated if they have a bunch of
I.V. lines and tubes running through them. And it can require five or
six people to coordinate all the movements and make sure those lines are
still running while flipping the patient. So it seems very simple. And
the doctors knew that it would help. But in some of those public
hospitals, they were not able to do it because they did not have the
staff available.

michael barbaro

Mm-hmm.

brian rosenthal

One doctor at a small independent hospital told us that out of 10 of the
deaths that he witnessed, he thought two or three of the patients could
have been saved.

michael barbaro

If there had been better staffing.

brian rosenthal

Yeah, if the hospital had the resources of a private hospital.

michael barbaro

Wow. I mean, that's 20 to 30 percent

brian rosenthal

Yeah, I mean, it translates to thousands of people. And we actually
looked at the mortality rates at most of the 47 hospitals in the city.
And in some cases, the mortality rate was three times higher in the
public hospitals in the lower income areas. Some of that mortality
difference could be explained by differences in patient populations ---
you know, underlying health conditions of the patients. But the experts
and the doctors that we talked to said that the quality of care was
definitely a factor in those differences.

michael barbaro

Brian, as horrible as everything you're describing is, it feels like
there's a pretty logical solution to it. Which is taking Covid-19
patients from these overburdened, understaffed public hospitals, and
transferring them to the less burdened, better staffed private
hospitals.

brian rosenthal

You'd think that, yes. And Governor Cuomo even said at the peak of the
pandemic that that was going to happen.

\begin{itemize}
\tightlist
\item
  archived recording (andrew cuomo)\\
  How many beds would you need at the apex? Between 70 and 110,000.
  Right now, we have 53,000 statewide. We have only 36,000 downstate.
  Every hospital by mandate has to add a 50 percent increase. And they
  have all done that. We're setting up extra facilities, which ---
\end{itemize}

michael barbaro

But in the end, it didn't. And why not? Like, what prevents a patient at
Elmhurst Hospital in Queens from being transferred to N.Y.U. Langone,
which happens to be on the east side of Manhattan. It's not that far.

brian rosenthal

Well, Elmhurst is a public hospital. And for decades, they have not
really transferred patients to N.Y.U. Langone. They've transferred
patients to other hospitals within the public system, but they just
don't really work together with the private system.

michael barbaro

So there's no infrastructure set up to make such transfers. And
therefore, they're unlikely to happen.

brian rosenthal

Well, nothing physically prevents a patient from being transferred. But
first of all, the hospital, Elmhurst, may not want to transfer the
patient because there is revenue attached to every patient. Even a
public hospital cares about maximizing its revenue. So the doctor and
the nurse inside the hospital may want very much to transfer a patient
to Langone, but the administrator, the C.E.O. of the hospital, might not
want to do that for financial reasons. So there was a problem on that
end.

And then there's a problem on the other end, because N.Y.U. Langone is a
private hospital. And it wants to treat patients with private health
insurance because that's going to bring the biggest profit. And the
patient coming from Elmhurst, the public hospital, is going to be a
patient without private health insurance. So it's not a patient that
N.Y.U. Langone really wants, anyway. So on both ends, Elmhurst may not
want to transfer the patient, and N.Y.U. Langone might not want to take
the patient.

michael barbaro

So the incentives are not there for this very simple fix to work.

brian rosenthal

That's right. Because the incentive is profit.

michael barbaro

So at the end of the day, were there any transfers between the public
and private hospitals? Any meaningful number of transfers?

brian rosenthal

There were less than 50 ---

michael barbaro

Wow.

brian rosenthal

--- during the whole course of the pandemic, thousands of people in
hospitals. There were less than 50 transfers from public hospitals to
private hospitals.

michael barbaro

That is a genuinely shocking number.

brian rosenthal

Yeah, and again, the transfers were wanted by the doctors and the
nurses. But they didn't end up happening.

michael barbaro

I'm very rarely shocked.

brian rosenthal

Yeah. So that brings us to the other possible solution, which New York
explored and actually put a lot of money into, which was the overflow
hospitals --- makeshift hospitals set up around the city that could take
patients from these overburdened hospitals. But it turns out those
didn't work either.

{[}music{]}

michael barbaro

We'll be right back.

\begin{itemize}
\item
  archived recording 1\\
  Now as we all know, New York is the national epicenter of the
  coronavirus crisis. Now it is all hands on deck there.
\item
  archived recording\\
  And the death toll in New York City from the Covid-19 pandemic has
  climbed to 450, with 26,000 testing positive so far. This is the Naval
  ship Comfort due to arrive in the area on Monday from Virginia. And a
  field hospital ---
\end{itemize}

michael barbaro

Brian, I remember these overflow hospitals really well.

\begin{itemize}
\tightlist
\item
  archived recording (andrew cuomo)\\
  What we're doing here at the Javits Center is constructing four
  emergency hospitals.
\end{itemize}

michael barbaro

I remember Governor Cuomo walking through the Javits Center, this huge
convention center ---

\begin{itemize}
\tightlist
\item
  archived recording (andrew cuomo)\\
  This was never an anticipated use. But you do what you have to do.
  That's the New York way. That's the American way.
\end{itemize}

michael barbaro

--- on the west side of Manhattan, kind of showing off the hundreds of
beds. I remember there being little flowers on the sides of the tables
next to the cots. And I know these were set up in each borough. So what
happened that meant that they didn't actually do their job?

brian rosenthal

Well, let's take the example of the Billie Jean King Tennis Center.

\begin{itemize}
\tightlist
\item
  archived recording\\
  Part of the Billie Jean King National Tennis Center right now is being
  converted into a temporary hospital.
\end{itemize}

brian rosenthal

It's one of the biggest tennis centers in the world. It's where the U.S.
Open is held.

\begin{itemize}
\tightlist
\item
  archived recording\\
  Some patients from nearby Elmhurst Hospital are expected to be
  transferred to the National Tennis Center Hospital.
\end{itemize}

brian rosenthal

It was going to have 470 beds and hundreds of employees that were going
to be available to take patients, specifically from Elmhurst and Queens.

\begin{itemize}
\tightlist
\item
  archived recording\\
  This place will be a lifesaving place. It's going to help take the
  pressure off Elmhurst.
\end{itemize}

brian rosenthal

It was supposed to be a crucial facility. But the first problem that it
had was bureaucracy. There were paperwork requirements. There were all
kinds of orientations that the doctors needed to do, training on the
computer systems, training on the type of equipment that was going to be
there and the paperwork that had to be filled out. And you had doctors
in the middle of the peak of the pandemic, when people were dying,
spending time doing things that had nothing to do with patient care.

Another problem was that the hospital was suffering from a bit of an
identity crisis about which types of patients it was going to treat and
at different points of time, even within the week that it was being set
up.

\begin{itemize}
\tightlist
\item
  archived recording\\
  As of this morning, the complex was not likely to include Covid-19
  patients. The U.S. Open is ---
\end{itemize}

brian rosenthal

City officials were changing their mind about that question.

\begin{itemize}
\tightlist
\item
  archived recording\\
  This facility will be able to take people from Elmhurst, other
  coronavirus patients, bring them over here, relieve some of that
  pressure immediately.
\end{itemize}

brian rosenthal

And they were conveying different directives to other hospitals about
which types of patients they should be transferring to the Billie Jean
King Tennis Center. And they ended up crafting a series of rules that
were very restrictive about the types of patients that could go to
Billie Jean King. There were over 25 different exclusionary criteria,
which is basically disqualifying conditions that if the patient has,
they can't go to Billie Jean King. And one of them was just the fact
that the patient had a fever, which is a hallmark symptom of the
coronavirus.

But at the same time, there were also a series of rules about the types
of patients that they would not see because they were not severe enough.
They were patients that were quarantining with the virus in hotels, and,
in some cases, ended up dying in those hotels. And when employees at
Billie Jean King asked why they couldn't see and care for those
patients, they were told that those patients aren't severe enough to be
at Billie Jean King. So they couldn't see the really severe patients.
They also couldn't see the patients that were not severe. And as a
result, they didn't end up treating much of anybody.

michael barbaro

Hm. So did they see any patients?

brian rosenthal

Well, hold on, because there's another problem, and it relates to
ambulances. So in the peak of the pandemic, if you were at your house
and you called 911, the ambulance that arrived could not take you to
Billie Jean King directly.

michael barbaro

Why not?

brian rosenthal

Well, the city had decided that ambulances would have to first take
patients to a hospital, even if they're overburdened. And that hospital
would triage the patient and then figure out where to send them. So
Billie Jean King was really only taking transfers from other hospitals.
But even the transferring process was blocked by ambulance regulations.
Because there were situations where hospitals wanted to transfer
patients, but there was no ambulance available to transfer them. And
Billie Jean King had its own ambulances on site that could have gone to
the hospital and picked up the patient. But the regular hospitals had
exclusivity agreements with ambulance companies that said that nobody
could pick up their patients. They could only send patients out in their
own ambulances with these companies.

michael barbaro

And so that patient is just going to stay at Elmhurst and not get
transferred to Billie Jean King.

brian rosenthal

Until an ambulance from that company with the exclusive agreement is
available, yes. And that happened, so patients had to wait.

michael barbaro

OK, so back to that question. In the end, how many patients made it into
this Billie Jean King overflow hospital?

brian rosenthal

79.

michael barbaro

Geez.

brian rosenthal

That's 79 throughout the course of the month that the Billie Jean King
Center was open. At any one time, there were maybe 20 or 30 patients
there.

michael barbaro

So what were all the staff, the nurses, the doctors at Billie Jean King
Field Hospital, overflow hospital, what were they doing?

brian rosenthal

Well, in many cases, nothing. You know, I want to be clear, because the
doctors and nurses and other staffers that came to work at Billie Jean
King, they came, in many cases, from around the country. They were
experienced medical professionals. And they really wanted to help. And
they were extremely well paid as well. They were paid, the doctors in
many cases, over \$600 an hour.

michael barbaro

Wow.

brian rosenthal

So they showed up to work ready to help, eager to help, but no patients
came in the door. So I talked to some of them that said that in the peak
of the pandemic, they were just sitting around on their phone all day.
One of the workers at Billie Jean King who I talked with, who is a nurse
practitioner who came up from Baltimore, she said, ``I basically got
paid \$2,000 a day to sit on my phone and look at Facebook. We all felt
guilty. I felt really ashamed, to be honest.''

michael barbaro

Right, because like you said, they came to serve the public in New York.
In particular, a public that was trying to get into overburdened public
hospitals, and here they are, not able to do that because of exclusive
ambulance agreements and kind of bureaucratic nonsense.

brian rosenthal

That's right, yeah. The facility ended up closing in early May after the
peak of the pandemic. There was really no need for it. And ultimately,
for its work in treating 79 patients, so far the city has paid the
contractor about \$52 million. But the bill is actually still coming in.
The total bill might actually be over \$100 million.

michael barbaro

Brian, whenever we talk about inequality, it can feel like a very
out-of-reach set of solutions, right? Because almost by definition, it
is systemic deeply rooted issues. But in the case of hospitals in New
York, the solutions felt very practical and very simple, as you have
laid them out. You know, cancel those exclusive ambulance agreements.
Transfer patients from public to private hospitals. They all seemed
quite within reach.

brian rosenthal

Yeah, I think that's right. And I think it's also important to note that
even while the pandemic was going on, there were plenty of doctors and
other hospital workers who noticed these inequalities and were trying to
fix them. We talked with a number of doctors that actually rotated
between working in the private hospitals and working in the public
hospitals, and were trying to raise alarms, and even hospitals within
the private networks trying to push their bosses to do more to address
inequalities. But the reality was by that point, the inequalities were
so ingrained into the hospital system that there wasn't a whole lot that
they could do.

I think the story of what happened in hospitals in New York, in the
height of the coronavirus pandemic, is really a story about officials
and hospital executives and bureaucrats who accepted these inequalities
in the system long ago, and have obviously known about inequalities for
decades, but chose not to address them and found that they got exposed
in this pandemic.

michael barbaro

But of course, in that case, isn't it the role of government? Isn't it
the role of the mayor of New York City, the governor of New York, to not
accept those kinds of inequalities, and to do everything in their power
to slice through that kind of complacency in the midst of a public
health crisis?

brian rosenthal

Yeah, and I think if you talk to the governor or the mayor, if you had
them sitting here, they would say that they did as much as they could.
And they did certainly spend a lot of money setting up field hospitals
to help and set up a system to help with transfers.

But one thing that I think is very telling is when I called the
governor's office to ask why more patients were not transferred from
overburdened hospitals to private hospitals that had open beds, the
governor's office said that they accommodated every transfer that was
requested by the hospitals. And they felt like that was their job. So
they handled each request, but they were not willing to force hospitals
to transfer. They were not willing to take that more fundamental step in
changing the government's role. And I think it's because they themselves
kind of accepted the reality as it was, that there were going to be
inequalities between different types of hospitals and different types of
patients.

michael barbaro

Right, to say that they processed every request they got for transfers
is to say, like, I caught a couple of the raindrops in this giant storm,
but what about that flood down the street?

brian rosenthal

Right, it's not addressing the more fundamental problem.

michael barbaro

Brian, at the start of our conversation, you mentioned that peak
hospitalizations are now occurring throughout much of the rest of the
country. It's subsided in New York, but it's now happening in Texas.
It's happening in Florida. It's happening in Arizona.

brian rosenthal

Yes.

michael barbaro

I know that your investigation was into the hospitals in New York. But
do we expect that what you saw in New York --- these inequities, these
private-public hospital disparities --- that they are likely to play out
across the rest of the country?

brian rosenthal

There will definitely be disparities in every city in America. I think
the question is whether other cities have learned from New York and are
going to be willing to put in place systems and policies that can help
balance out those inequalities in a more real way than we saw in New
York. And I think that's still to be determined.

michael barbaro

Brian, thank you very much. We appreciate it.

brian rosenthal

Thank you.

{[}music{]}

michael barbaro

On Sunday, The Times reported that the total number of infections in
Florida has now surpassed that of New York, making the state the new
epicenter of the pandemic. Florida has nearly 424,000 reported cases,
compared with about 415,000 cases in New York. We'll be right back.

Here's what else you need to know today.

The Times reports that the presence of federal agents in Portland
galvanized thousands of people to join protests across the country over
the weekend, reviving nationwide protests that had largely dissipated.

\begin{itemize}
\tightlist
\item
  archived recording\\
  Black lives matter! {[}CAR HONKING{]} Black lives matter! {[}CAR
  HONKING{]}
\end{itemize}

michael barbaro

One of the most intense protests was in Seattle, where a demonstration
against police brutality turned violent, after some protesters lit a
detention center on fire, smashed windows and damaged a police building.

In response, police declared the protest a riot, fired flash grenades,
unleashed pepper spray and rushed into crowds, knocking people to the
ground.

That's it for ``The Daily.'' I'm Michael Barbaro. See you tomorrow.

Previous

More episodes ofThe Daily

\href{https://www.nytimes.com/2020/08/04/podcasts/the-daily/mail-in-voting-president-trump.html?action=click\&module=audio-series-bar\&region=header\&pgtype=Article}{\includegraphics{https://static01.nyt.com/images/2020/07/30/us/politics/04daily/30trump-election1-thumbLarge.jpg}}

August 4, 2020Is the U.S. Ready to Vote by Mail?

\href{https://www.nytimes.com/2020/08/03/podcasts/the-daily/algorithmic-justice-racism.html?action=click\&module=audio-series-bar\&region=header\&pgtype=Article}{\includegraphics{https://static01.nyt.com/images/2020/06/24/business/03daily/24michigan-arrest1-thumbLarge.jpg}}

August 3, 2020~~•~ 28:13Wrongfully Accused by an Algorithm

\href{https://www.nytimes.com/2020/08/02/podcasts/the-daily/on-female-rage.html?action=click\&module=audio-series-bar\&region=header\&pgtype=Article}{\includegraphics{https://static01.nyt.com/images/2018/01/21/magazine/21mag-femaleanger1-copy/21mag-femaleanger1-thumbLarge.jpg}}

August 2, 2020The Sunday Read: `On Female Rage'

\href{https://www.nytimes.com/2020/07/31/podcasts/the-daily/vanessa-guillen-military-metoo.html?action=click\&module=audio-series-bar\&region=header\&pgtype=Article}{\includegraphics{https://static01.nyt.com/images/2020/07/12/us/politics/31daily/00dc-army-metoo-thumbLarge.jpg}}

July 31, 2020A \#MeToo Moment in the Military

\href{https://www.nytimes.com/2020/07/30/podcasts/the-daily/congress-facebook-amazon-google-apple.html?action=click\&module=audio-series-bar\&region=header\&pgtype=Article}{\includegraphics{https://static01.nyt.com/images/2020/07/30/reader-center/30daily/merlin_175077825_5ebc931b-baa1-489a-960c-34e4d845e997-thumbLarge.jpg}}

July 30, 2020~~•~ 35:19The Big Tech Hearing

\href{https://www.nytimes.com/2020/07/29/podcasts/the-daily/china-trump-foreign-policy.html?action=click\&module=audio-series-bar\&region=header\&pgtype=Article}{\includegraphics{https://static01.nyt.com/images/2020/07/26/world/29daily/00china-us-clash1-thumbLarge.jpg}}

July 29, 2020~~•~ 28:40Confronting China

\href{https://www.nytimes.com/2020/07/28/podcasts/the-daily/unemployment-benefits-coronavirus.html?action=click\&module=audio-series-bar\&region=header\&pgtype=Article}{\includegraphics{https://static01.nyt.com/images/2020/07/23/business/28daily/23virus-uiexplain1-thumbLarge.jpg}}

July 28, 2020~~•~ 26:13Why \$600 Checks Are Tearing Republicans Apart

\href{https://www.nytimes.com/2020/07/27/podcasts/the-daily/new-york-hospitals-covid.html?action=click\&module=audio-series-bar\&region=header\&pgtype=Article}{\includegraphics{https://static01.nyt.com/images/2020/07/27/world/27daily-hospitals/27daily-hospitals-thumbLarge.jpg}}

July 27, 2020~~•~ 33:28The Mistakes New York Made

\href{https://www.nytimes.com/2020/07/26/podcasts/the-daily/the-accusation-the-sunday-read.html?action=click\&module=audio-series-bar\&region=header\&pgtype=Article}{\includegraphics{https://static01.nyt.com/images/2020/03/22/magazine/26audm-2/22mag-titleix-thumbLarge.jpg}}

July 26, 2020The Sunday Read: `The Accusation'

\href{https://www.nytimes.com/2020/07/24/podcasts/the-daily/mlb-baseball-season-coronavirus.html?action=click\&module=audio-series-bar\&region=header\&pgtype=Article}{\includegraphics{https://static01.nyt.com/images/2020/07/22/sports/24daily/22mlb-previewlede1-thumbLarge.jpg}}

July 24, 2020~~•~ 45:34The Battle for a Baseball Season

\href{https://www.nytimes.com/2020/07/23/podcasts/the-daily/portland-protests.html?action=click\&module=audio-series-bar\&region=header\&pgtype=Article}{\includegraphics{https://static01.nyt.com/images/2020/07/22/us/23daily-image/22portland-tactics02-thumbLarge.jpg}}

July 23, 2020~~•~ 30:04The Showdown in Portland

\href{https://www.nytimes.com/2020/07/22/podcasts/the-daily/school-reopenings-coronavirus.html?action=click\&module=audio-series-bar\&region=header\&pgtype=Article}{\includegraphics{https://static01.nyt.com/images/2020/07/12/science/22daily/00virus-schools-reopen01-thumbLarge.jpg}}

July 22, 2020~~•~ 27:24The Science of School Reopenings

\href{https://www.nytimes.com/column/the-daily}{See All Episodes ofThe
Daily}

Next

July 27, 2020

\begin{itemize}
\item
\item
\item
\item
\item
\item
\end{itemize}

\emph{\textbf{Listen and subscribe to our podcast from your mobile
device:}}\\
\textbf{\href{https://itunes.apple.com/us/podcast/the-daily/id1200361736?mt=2}{\emph{Via
Apple Podcasts}}} \emph{\textbf{\textbar{}}}
\textbf{\href{https://open.spotify.com/show/3IM0lmZxpFAY7CwMuv9H4g?si=SfuMSC55R1qprFsRZU3_zw}{\emph{Via
Spotify}}} \emph{\textbf{\textbar{}}}
\textbf{\href{http://www.stitcher.com/podcast/the-new-york-times/the-daily-10}{\emph{Via
Stitcher}}}

A New York Times investigation found that surviving the coronavirus in
New York had a lot to do with which hospital a person went to.

Our investigative reporter Brian M. Rosenthal pulls back the curtain on
inequality and the pandemic in the city.

\textbf{On today's episode:}

\begin{itemize}
\tightlist
\item
  \href{https://www.nytimes.com/by/brian-m-rosenthal}{Brian M.
  Rosenthal}, an investigative reporter on the Metro Desk of The New
  York Times.
\end{itemize}

\includegraphics{https://static01.nyt.com/images/2020/07/27/world/27daily-hospitals/merlin_172404552_f4e79cbc-7bf0-488a-a220-9f945e56e065-articleLarge.jpg?quality=75\&auto=webp\&disable=upscale}

\textbf{Background reading:}

\begin{itemize}
\item
  At the peak of New York's pandemic, patients at some community
  hospitals were
  \href{https://www.nytimes.com/2020/07/01/nyregion/Coronavirus-hospitals.html}{three
  times more likely to die} than were patients at medical centers in the
  wealthiest parts of the city.
\item
  The story of a \$52 million
  \href{https://www.nytimes.com/2020/07/21/nyregion/coronavirus-hospital-usta-queens.html}{temporary
  care facility in New York} illustrates the missteps made at every
  level of government in the race to create more hospital capacity.
\end{itemize}

\emph{Tune in, and tell us what you think. Email us at}
\href{mailto:thedaily@nytimes.com}{\emph{thedaily@nytimes.com}}\emph{.
Follow Michael Barbaro on Twitter:}
\href{https://twitter.com/mikiebarb}{\emph{@mikiebarb}}\emph{. And if
you're interested in advertising with ``The Daily,'' write to us at}
\href{mailto:thedaily-ads@nytimes.com}{\emph{thedaily-ads@nytimes.com}}\emph{.}

Brian M. Rosenthal contributed reporting.

``The Daily'' is made by Theo Balcomb, Andy Mills, Lisa Tobin, Rachel
Quester, Lynsea Garrison, Annie Brown, Clare Toeniskoetter, Paige
Cowett, Michael Simon Johnson, Brad Fisher, Larissa Anderson, Wendy
Dorr, Chris Wood, Jessica Cheung, Stella Tan, Alexandra Leigh Young,
Jonathan Wolfe, Lisa Chow, Eric Krupke, Marc Georges, Luke Vander Ploeg,
Adizah Eghan, Kelly Prime, Julia Longoria, Sindhu Gnanasambandan, M.J.
Davis Lin, Austin Mitchell, Sayre Quevedo, Neena Pathak, Dan Powell,
Dave Shaw, Sydney Harper, Daniel Guillemette, Hans Buetow, Robert
Jimison, Mike Benoist, Bianca Giaever and Asthaa Chaturvedi. Our theme
music is by Jim Brunberg and Ben Landsverk of Wonderly. Special thanks
to Sam Dolnick, Mikayla Bouchard, Lauren Jackson, Julia Simon, Mahima
Chablani and Nora Keller.

Advertisement

\protect\hyperlink{after-bottom}{Continue reading the main story}

\hypertarget{site-index}{%
\subsection{Site Index}\label{site-index}}

\hypertarget{site-information-navigation}{%
\subsection{Site Information
Navigation}\label{site-information-navigation}}

\begin{itemize}
\tightlist
\item
  \href{https://help.nytimes.com/hc/en-us/articles/115014792127-Copyright-notice}{©~2020~The
  New York Times Company}
\end{itemize}

\begin{itemize}
\tightlist
\item
  \href{https://www.nytco.com/}{NYTCo}
\item
  \href{https://help.nytimes.com/hc/en-us/articles/115015385887-Contact-Us}{Contact
  Us}
\item
  \href{https://www.nytco.com/careers/}{Work with us}
\item
  \href{https://nytmediakit.com/}{Advertise}
\item
  \href{http://www.tbrandstudio.com/}{T Brand Studio}
\item
  \href{https://www.nytimes.com/privacy/cookie-policy\#how-do-i-manage-trackers}{Your
  Ad Choices}
\item
  \href{https://www.nytimes.com/privacy}{Privacy}
\item
  \href{https://help.nytimes.com/hc/en-us/articles/115014893428-Terms-of-service}{Terms
  of Service}
\item
  \href{https://help.nytimes.com/hc/en-us/articles/115014893968-Terms-of-sale}{Terms
  of Sale}
\item
  \href{https://spiderbites.nytimes.com}{Site Map}
\item
  \href{https://help.nytimes.com/hc/en-us}{Help}
\item
  \href{https://www.nytimes.com/subscription?campaignId=37WXW}{Subscriptions}
\end{itemize}
