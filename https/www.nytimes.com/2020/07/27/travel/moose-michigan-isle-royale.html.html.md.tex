Sections

SEARCH

\protect\hyperlink{site-content}{Skip to
content}\protect\hyperlink{site-index}{Skip to site index}

\href{/section/travel}{Travel}\textbar{}On the Lookout for Moose on
Michigan's Isle Royale

\url{https://nyti.ms/3f2RquQ}

\begin{itemize}
\item
\item
\item
\item
\item
\item
\end{itemize}

\href{https://www.nytimes.com/spotlight/at-home?action=click\&pgtype=Article\&state=default\&region=TOP_BANNER\&context=at_home_menu}{At
Home}

\begin{itemize}
\tightlist
\item
  \href{https://www.nytimes.com/2020/07/28/books/time-for-a-literary-road-trip.html?action=click\&pgtype=Article\&state=default\&region=TOP_BANNER\&context=at_home_menu}{Take:
  A Literary Road Trip}
\item
  \href{https://www.nytimes.com/2020/07/29/magazine/bored-with-your-home-cooking-some-smoky-eggplant-will-fix-that.html?action=click\&pgtype=Article\&state=default\&region=TOP_BANNER\&context=at_home_menu}{Cook:
  Smoky Eggplant}
\item
  \href{https://www.nytimes.com/2020/07/27/travel/moose-michigan-isle-royale.html?action=click\&pgtype=Article\&state=default\&region=TOP_BANNER\&context=at_home_menu}{Look
  Out: For Moose}
\item
  \href{https://www.nytimes.com/interactive/2020/at-home/even-more-reporters-editors-diaries-lists-recommendations.html?action=click\&pgtype=Article\&state=default\&region=TOP_BANNER\&context=at_home_menu}{Explore:
  Reporters' Obsessions}
\end{itemize}

\includegraphics{https://static01.nyt.com/images/2020/07/27/travel/27travel-michigan-09/27travel-michigan-09-articleLarge.jpg?quality=75\&auto=webp\&disable=upscale}

The World Through a Lens

\hypertarget{on-the-lookout-for-moose-on-michigans-isle-royale}{%
\section{On the Lookout for Moose on Michigan's Isle
Royale}\label{on-the-lookout-for-moose-on-michigans-isle-royale}}

The remote Isle Royale, tucked away in the northern reaches of Lake
Superior, is one of America's least visited national parks.

A bull moose at the Rock Harbor campground.Credit...

Supported by

\protect\hyperlink{after-sponsor}{Continue reading the main story}

Photographs and Text by
\href{https://www.nytimes.com/by/tony-cenicola}{Tony Cenicola}

\begin{itemize}
\item
  Published July 27, 2020Updated July 31, 2020
\item
  \begin{itemize}
  \item
  \item
  \item
  \item
  \item
  \item
  \end{itemize}
\end{itemize}

\emph{At the onset of the coronavirus pandemic, with travel restrictions
in place worldwide, we launched a new series ---}
\href{https://www.nytimes.com/column/the-world-through-a-lens}{\emph{The
World Through a Lens}} \emph{--- in which photojournalists help
transport you, virtually, to some of our planet's most beautiful and
intriguing places. This week, Tony Cenicola, a New York Times staff
photographer, shares a collection of images from a remote island in
Michigan.}

\begin{center}\rule{0.5\linewidth}{\linethickness}\end{center}

Tucked away in the northern reaches of Lake Superior, far closer to both
Ontario and Minnesota than to the Upper Peninsula of Michigan, lies one
of the country's least visited national parks: Isle Royale.

The park --- which consists of the 206-square-mile Isle Royale, along
with hundreds of smaller adjacent islands --- sees very few visitors. In
2018, the year I went, just
\href{https://www.nps.gov/isro/learn/management/statistics.htm}{18,479
people} visited the island portion of the park, the lowest number of any
park in the contiguous 48 states. (Compare that, for example, with Grand
Canyon National Park, which in 2018 drew nearly 6.4 million visitors.)

\includegraphics{https://static01.nyt.com/images/2020/07/27/travel/27travel-michigan-20/merlin_142981839_0a78641b-255c-43aa-903e-20e4bc3e24a2-articleLarge.jpg?quality=75\&auto=webp\&disable=upscale}

Image

The coastline on Raspberry Island.

By the time I planned my trip, the only inn on the island was fully
booked, so camping was my sole option. And I decided to drive from New
York, because it would have been something of a nightmare to get on a
plane with all my photography equipment and camping gear.

100 miles

CANADA

ISLE ROYALE

MINNESOTA

LAKE

SUPERIOR

Houghton

MICHIGAN

Lake

Huron

Lake

MICHIGAN

WISCONSIN

MICHIGAN

By The New York Times

Isle Royale is a six-hour ferry ride from the port in Houghton, a small
city on the Upper Peninsula. Established as a national park in 1940, it
is known for its moose population; in 2018 there were around 1,500 on
the island. (It's also known for its much smaller
\href{https://www.nps.gov/isro/learn/nature/wolf-moose-populations.htm}{wolf
population}, which has fluctuated dramatically in recent years, raising
\href{https://www.nytimes.com/2013/05/09/opinion/save-the-wolves-of-isle-royale-national-park.html}{complicated
questions about conservation}.) On the ferry, my fellow passengers and I
were instructed to keep a safe distance from the moose --- about the
length of a railway car. ``When in doubt, move farther away,'' the
\href{https://www.nps.gov/isro/learn/nature/moose.htm}{National Park
Service advises}.

Image

Rock Harbor Lodge

Image

A shack on a hike near the lodge.

It was late afternoon when I arrived at my campsite for the night, at
the \href{https://www.nps.gov/isro/planyourvisit/rock-harbor.htm}{Rock
Harbor} campground. I wasn't even done setting up my tent when a bull
moose appeared with a full rack of antlers. He was just wandering
through, foraging for food in the underbrush.

I could feel the adrenaline race through my head as I started shooting
pictures of him from no more than 50 feet away. He was in a thick stand
of trees, so I didn't think there was any danger of him charging me. He
stuck around for nearly an hour, and I kept shooting him from behind the
trees.

Image

When viewing moose, the National Park Service recommends that, when in
doubt, move farther away.

My wife and I have something of a running obsession with moose. We have
moose paraphernalia in our house. There's a local road near our home
that we call the ``mooseway'' for no particular reason. (There are no
moose in the area.) Whenever we travel to an area where there's even the
remotest possibility of sighting a moose, we're on high alert.

And because of my minor obsession, seeing one on this trip was my top
priority --- and I felt both excited and relieved that it happened so
quickly.

Image

As the wolf population on Isle Royale declines, the moose population
increases, and vice versa.

Over the course of the hour, more and more people gathered to watch the
moose. He was standing near a vacant campsite, and a handful of people
settled onto a nearby picnic table to watch him. Eventually the moose
picked up his head and looked our way. That was enough to send several
onlookers running away through the woods.

Image

In 2018, the moose population on Isle Royale was around 1,500.

You're only allowed to stay at the Rock Harbor campground for one night,
so the next day I had to break camp and lug all my equipment and camping
gear to a new site three miles away --- no easy feat, since my pack
weighed around 65 pounds.

Image

The view from Mount Franklin.

I ended up hiking around 13 miles that day, through difficult terrain:
wetlands, inland lakes and streams. I spotted turtles basking on logs
and saw evidence of beaver activity.

Image

Evidence of beaver activity.

Image

A group of turtles in a pond between Mount Ojibway and Daisy Farm
campground.

At one point, realizing I didn't have enough water in my quart-size
water bottle, I began picking wild blueberries and placing them in the
bottle. I'd gulp a few down with each sip. It helped extend my water
supply and keep my energy level up.

At 7 p.m., once I was settled into my new campsite, I collapsed, ate the
balance of my blueberries, sipped the remaining water and had a granola
bar. After a few hours of rest, I woke up around 1 a.m. and went out to
photograph the incredible night sky. Mars was shining so brightly it
reflected in Lake Superior.

Image

Mars (the reddish dot) is reflected in Lake Superior.Credit...Tony
Cenicola/The New York Times

The next morning, I trekked to the harbor for breakfast at the inn.
There, I rented a motorized rowboat to tour a few other parts of the
island, including the
\href{http://iri.forest.mtu.edu/Historic_Fisheries/Pages/Rock_Tobin_Harbor/Edisen.htm}{Edisen
Fishery}, a historical fishing camp that shows what life was like here
for commercial fishermen and their families in the 1800s and 1900s,
before the island became a national park.

Image

A handful of the buildings at the Edisen Fishery.

Image

A reproduction of a grave marker for Arthur Lee Scott, a miner who died
in the area in the 1870s.

The motorized rowboat made everything so much easier, and it meant that
I didn't have to hike back to the harbor with all my equipment when
leaving the island. In the end I took a seaplane to get back to the
mainland --- a leisurely conclusion to an otherwise tiring, and
satisfying, trip.

Image

Departing Isle Royale by seaplane.

\href{https://www.nytimes.com/by/tony-cenicola}{\emph{Tony Cenicola}}
\emph{is a photographer for The New York Times.}

\emph{\textbf{Follow New York Times Travel}} \emph{on}
\href{https://www.instagram.com/nytimestravel/}{\emph{Instagram}}\emph{,}
\href{https://twitter.com/nytimestravel}{\emph{Twitter}} \emph{and}
\href{https://www.facebook.com/nytimestravel/}{\emph{Facebook}}\emph{.
And}
\href{https://www.nytimes.com/newsletters/traveldispatch}{\emph{sign up
for our weekly Travel Dispatch newsletter}} \emph{to receive expert tips
on traveling smarter and inspiration for your next vacation.}

Advertisement

\protect\hyperlink{after-bottom}{Continue reading the main story}

\hypertarget{site-index}{%
\subsection{Site Index}\label{site-index}}

\hypertarget{site-information-navigation}{%
\subsection{Site Information
Navigation}\label{site-information-navigation}}

\begin{itemize}
\tightlist
\item
  \href{https://help.nytimes.com/hc/en-us/articles/115014792127-Copyright-notice}{©~2020~The
  New York Times Company}
\end{itemize}

\begin{itemize}
\tightlist
\item
  \href{https://www.nytco.com/}{NYTCo}
\item
  \href{https://help.nytimes.com/hc/en-us/articles/115015385887-Contact-Us}{Contact
  Us}
\item
  \href{https://www.nytco.com/careers/}{Work with us}
\item
  \href{https://nytmediakit.com/}{Advertise}
\item
  \href{http://www.tbrandstudio.com/}{T Brand Studio}
\item
  \href{https://www.nytimes.com/privacy/cookie-policy\#how-do-i-manage-trackers}{Your
  Ad Choices}
\item
  \href{https://www.nytimes.com/privacy}{Privacy}
\item
  \href{https://help.nytimes.com/hc/en-us/articles/115014893428-Terms-of-service}{Terms
  of Service}
\item
  \href{https://help.nytimes.com/hc/en-us/articles/115014893968-Terms-of-sale}{Terms
  of Sale}
\item
  \href{https://spiderbites.nytimes.com}{Site Map}
\item
  \href{https://help.nytimes.com/hc/en-us}{Help}
\item
  \href{https://www.nytimes.com/subscription?campaignId=37WXW}{Subscriptions}
\end{itemize}
