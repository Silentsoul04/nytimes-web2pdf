Sections

SEARCH

\protect\hyperlink{site-content}{Skip to
content}\protect\hyperlink{site-index}{Skip to site index}

\href{https://www.nytimes.com/section/us}{U.S.}

\href{https://myaccount.nytimes.com/auth/login?response_type=cookie\&client_id=vi}{}

\href{https://www.nytimes.com/section/todayspaper}{Today's Paper}

\href{/section/us}{U.S.}\textbar{}Urban Explorers Give Modern Ruins a
Second Life

\url{https://nyti.ms/2WWhbqD}

\begin{itemize}
\item
\item
\item
\item
\item
\end{itemize}

\href{https://www.nytimes.com/spotlight/at-home?action=click\&pgtype=Article\&state=default\&region=TOP_BANNER\&context=at_home_menu}{At
Home}

\begin{itemize}
\tightlist
\item
  \href{https://www.nytimes.com/2020/08/03/well/family/the-benefits-of-talking-to-strangers.html?action=click\&pgtype=Article\&state=default\&region=TOP_BANNER\&context=at_home_menu}{Talk:
  To Strangers}
\item
  \href{https://www.nytimes.com/2020/08/01/at-home/coronavirus-make-pizza-on-a-grill.html?action=click\&pgtype=Article\&state=default\&region=TOP_BANNER\&context=at_home_menu}{Make:
  Grilled Pizza}
\item
  \href{https://www.nytimes.com/2020/07/31/arts/television/goldbergs-abc-stream.html?action=click\&pgtype=Article\&state=default\&region=TOP_BANNER\&context=at_home_menu}{Watch:
  'The Goldbergs'}
\item
  \href{https://www.nytimes.com/interactive/2020/at-home/even-more-reporters-editors-diaries-lists-recommendations.html?action=click\&pgtype=Article\&state=default\&region=TOP_BANNER\&context=at_home_menu}{Explore:
  Reporters' Google Docs}
\end{itemize}

Advertisement

\protect\hyperlink{after-top}{Continue reading the main story}

Supported by

\protect\hyperlink{after-sponsor}{Continue reading the main story}

\hypertarget{urban-explorers-give-modern-ruins-a-second-life}{%
\section{Urban Explorers Give Modern Ruins a Second
Life}\label{urban-explorers-give-modern-ruins-a-second-life}}

In photos and videos, researchers and thrill-seekers celebrate the
allure of abandoned schools, factories, hotels, movie palaces and other
forgotten properties.

\includegraphics{https://static01.nyt.com/images/2020/07/20/multimedia/00xp-abandoned1/00xp-abandoned1-articleLarge.jpg?quality=75\&auto=webp\&disable=upscale}

\href{https://www.nytimes.com/by/christopher-mele}{\includegraphics{https://static01.nyt.com/images/2018/06/13/multimedia/author-christopher-mele/author-christopher-mele-thumbLarge.jpg}}

By \href{https://www.nytimes.com/by/christopher-mele}{Christopher Mele}

\begin{itemize}
\item
  July 27, 2020
\item
  \begin{itemize}
  \item
  \item
  \item
  \item
  \item
  \end{itemize}
\end{itemize}

For Jake Williams, nothing means success like wrack and ruin.

Mr. Williams had studied business marketing in college before
withdrawing and pursuing a full-time career as an
\href{http://www.forbidden-places.net/why.php}{urban explorer},
researching and telling the stories of abandoned properties.

He films his excursions and,
\href{https://www.youtube.com/channel/UC5k3Kc0avyDJ2nG9Kxm9JmQ}{as the
producer of Bright Sun Films}, shares them on YouTube. The subjects of
some of his more popular videos, like a
\href{https://www.youtube.com/watch?v=3gJfEOx_NGA}{former Days Inn
hotel} or an abandoned
\href{https://www.youtube.com/watch?v=Mr1CWRc174o}{Walmart}, are fairly
mundane, but viewers are drawn out of morbid curiosity, he said.

``I think when you see an abandoned place on the side of the road,'' he
said, ``people will ask, `How'd that get there?'''

The urban exploration movement traces its origins to online forums that
allowed ``all these weirdos to connect'' and trade tips on places to
visit, said Matthew Christopher, the founder of the website
\href{https://www.abandonedamerica.us/}{Abandoned America}.

\includegraphics{https://static01.nyt.com/images/2020/07/20/multimedia/00xp-abandoned2/merlin_174766122_5e10f21b-9b5e-46f1-8c77-a43987b4ec21-articleLarge.jpg?quality=75\&auto=webp\&disable=upscale}

Drew Scavello, the creator of
\href{https://www.facebook.com/pg/truthindestructionphotography/photos/}{Truth
In Destruction}, which photographically chronicles abandoned places,
said that when he started urban exploring in 2007, a small number of
people were focused on sites in Boston, Detroit and Philadelphia. Since
then, the movement has grown into a large, loose-knit network that
includes teenagers up to septuagenarians.

Mr. Scavello said he was drawn to photographing former psychiatric
hospitals, which he described as ``overlooked and undervalued'' because
of the stigma attached to mental illness.

In his work, artifacts from bygone eras are not encased in glass or
roped off but are instead readily accessible. For instance, he said,
during a visit to a former state hospital in Iowa, he found
\href{https://warehouse-13-artifact-database.fandom.com/wiki/Walter_Freeman\%27s_Orbitoclast}{an
orbitoclast, a device once used in lobotomies}, in a cabinet.

``It's a much more tangible way to connect to history than going to a
museum and taking a preplanned tour,'' Mr. Scavello said. ``A lot of the
time, it's pretty incredible some of the stuff that gets left behind.''

Mr. Christopher of Abandoned America started photographing and
documenting abandoned spaces after working at a private mental health
institution and learning from patients and staff members about a former
state-run hospital,
\href{https://www.phillymag.com/news/2015/06/28/byberry-mental-institution-survivors/}{Philadelphia
State Hospital, also known as Byberry Hospital}, which closed in 1990.

He said former patients of that hospital ``were warehoused, forgotten
and erased.''

From there, he discovered abandoned schools, factories, hotels and movie
palaces. ``Before you knew it, I was obsessed with it,'' he said.

Image

An abandoned theater in an undisclosed location. Jaime M. Ullinger, an
associate professor of anthropology at Quinnipiac University, described
these kinds of sites as ``liminal,'' or in-between
spaces.Credit...Matthew Christopher/Abandoned America

His talks, books and photographs attract fans and the curious with the
allure of adventure, nostalgia and academic interest.

His work is more than a snapshot of a time gone by; it is also a
commentary on the impact of humans on the environment and the kind of
throwaway culture society has embraced.

Some of the sites he has documented date to a time when the United
States was competing with Europe and trying to show off America's
grandiosity.

``They thought they were building institutions to last centuries but now
it's a quick churn,'' Mr. Christopher said.

That's a view shared by Bryan Weissman and Michael Berindei, who run a
website called \href{https://theproperpeople.com/about/}{The Proper
People}. The name is a nod to a sign posted at a property they once
visited that declared ``Access Prohibited --- Except by the Proper
People.''

``A common theme we try to touch on in our videos is the idea that the
world we live in is becoming more and more disposable,'' Mr. Berindei
said.

He described the remnants of buildings from the 1920s and earlier as
``really grand, heavily ornamented structures that truly impress.''

The builders from those eras probably believed that what they were
constructing ``would be essentially permanent, and so they naturally
injected art, creativity and craftsmanship into them,'' he said.

Mr. Berindei said he appreciated construction from the 1940s and 1950s,
but in the decades that followed, buildings came to be ``thought of as a
good or commodity, rather than a permanent mark on our landscape.''

``The architecture of the past will only become more and more
unbelievable as more of our built world is replaced with prefab, cheaply
constructed junk,'' he said.

Mr. Christopher said documenting abandoned sites dates to at least
\href{https://www.nytimes.com/2007/09/28/arts/design/28pira.html}{Piranesi,
the 18th-century artist who sketched Roman ruins}.

Jaime M. Ullinger, an associate professor of anthropology at Quinnipiac
University in Connecticut, described modern-day abandoned sites as
``liminal,'' or in-between spaces. They don't serve their former
function, but they have not been razed or rehabilitated either, which
makes them inherently interesting.

``It used to be this thing,'' she said. ``Now, it's this thing and it's
not quite anything.''

Mr. Weissman and Mr. Berindei have documented visits to
\href{https://www.youtube.com/watch?v=XTPOkrJLWPM}{former amusement
parks}, \href{https://www.youtube.com/watch?v=QmNyVFibClQ}{malls} and
\href{https://www.youtube.com/watch?v=OUQz62-Ny1g}{hotels}.

They also \href{https://www.youtube.com/watch?v=oz5BamDwRDg}{visited a
gigantic former power plant in Philadelphia} that dates to 1925, a site
they described as ``extremely dangerous.''

A video shows them gingerly walking across a narrow beam over a dark pit
to gain access. Farther inside, a large chunk of concrete dangles
precariously from the ceiling.

They have encountered other hazards in their travels, including the
toxic chemicals known as PCBs, lead paint and mercury (especially at
former power plants) and mold, asbestos and
\href{https://www1.nyc.gov/site/doh/health/health-topics/pigeon.page}{pigeon
droppings}.

Scrapes, cuts and bruises are not uncommon. ``I don't think we're up to
date on our tetanus shots,'' Mr. Berindei said.

Another hazard can be a legal one related to trespassing. Does Mr.
Christopher always seek the permission of the owners of the properties
he visits? ``No,'' he said with a laugh.

Mr. Weissman and Mr. Berindei of The Proper People have had a few
run-ins with law enforcement, but they have never been arrested or
issued a citation. They said any tension eases once they explain the
nature of their work to the authorities.

On a visit to an abandoned power plant in New Orleans, Mr. Weissman and
Mr. Berindei found a colony of people who were relying on generators and
power tools to strip the site of scrap metal to sell to support their
drug habits.

``We talked to a few of them,'' Mr. Berindei said. ``They seem like nice
people. It was just a sad situation.''

Image

A former factory that produced china. ``There is this kind of beauty in
that it's something that is old and no longer in its original
grandeur,'' Professor Ullinger said.Credit...Matthew
Christopher/Abandoned America

Mr. Christopher said he relies on research, networking and luck to find
locations. Sometimes he stumbles onto sites or hears about them from
contacts in the salvage and demolition industries, or from historians
and preservationists.

Some of the creepiest stuff he's seen?
\href{https://twitter.com/abandonedameric/status/1057725225205465089?s=20}{Faceless
CPR dummies},
\href{https://twitter.com/abandonedameric/status/1057737618702692352}{a
rotting taxidermied elk head} and
\href{https://twitter.com/abandonedameric/status/1157330150515761153}{the
morgue at a former children's hospital}.

Mr. Christopher acknowledged that the economic fallout from the
coronavirus pandemic may lead to an increase in the number of abandoned
properties, especially retail centers, but that does not mean he's
looking forward to such an outcome.

``In a way,'' he said, ``it's a little bit like saying to a doctor
during the pandemic, `You will be really busy in the I.C.U.'''

Advertisement

\protect\hyperlink{after-bottom}{Continue reading the main story}

\hypertarget{site-index}{%
\subsection{Site Index}\label{site-index}}

\hypertarget{site-information-navigation}{%
\subsection{Site Information
Navigation}\label{site-information-navigation}}

\begin{itemize}
\tightlist
\item
  \href{https://help.nytimes.com/hc/en-us/articles/115014792127-Copyright-notice}{©~2020~The
  New York Times Company}
\end{itemize}

\begin{itemize}
\tightlist
\item
  \href{https://www.nytco.com/}{NYTCo}
\item
  \href{https://help.nytimes.com/hc/en-us/articles/115015385887-Contact-Us}{Contact
  Us}
\item
  \href{https://www.nytco.com/careers/}{Work with us}
\item
  \href{https://nytmediakit.com/}{Advertise}
\item
  \href{http://www.tbrandstudio.com/}{T Brand Studio}
\item
  \href{https://www.nytimes.com/privacy/cookie-policy\#how-do-i-manage-trackers}{Your
  Ad Choices}
\item
  \href{https://www.nytimes.com/privacy}{Privacy}
\item
  \href{https://help.nytimes.com/hc/en-us/articles/115014893428-Terms-of-service}{Terms
  of Service}
\item
  \href{https://help.nytimes.com/hc/en-us/articles/115014893968-Terms-of-sale}{Terms
  of Sale}
\item
  \href{https://spiderbites.nytimes.com}{Site Map}
\item
  \href{https://help.nytimes.com/hc/en-us}{Help}
\item
  \href{https://www.nytimes.com/subscription?campaignId=37WXW}{Subscriptions}
\end{itemize}
