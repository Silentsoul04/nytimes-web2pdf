Sections

SEARCH

\protect\hyperlink{site-content}{Skip to
content}\protect\hyperlink{site-index}{Skip to site index}

\href{https://www.nytimes.com/section/us}{U.S.}

\href{https://myaccount.nytimes.com/auth/login?response_type=cookie\&client_id=vi}{}

\href{https://www.nytimes.com/section/todayspaper}{Today's Paper}

\href{/section/us}{U.S.}\textbar{}Hoping to Understand the Virus,
Everyone Is Parsing a Mountain of Data

\url{https://nyti.ms/2P1h6xj}

\begin{itemize}
\item
\item
\item
\item
\item
\item
\end{itemize}

\href{https://www.nytimes.com/news-event/coronavirus?action=click\&pgtype=Article\&state=default\&region=TOP_BANNER\&context=storylines_menu}{The
Coronavirus Outbreak}

\begin{itemize}
\tightlist
\item
  live\href{https://www.nytimes.com/2020/08/01/world/coronavirus-covid-19.html?action=click\&pgtype=Article\&state=default\&region=TOP_BANNER\&context=storylines_menu}{Latest
  Updates}
\item
  \href{https://www.nytimes.com/interactive/2020/us/coronavirus-us-cases.html?action=click\&pgtype=Article\&state=default\&region=TOP_BANNER\&context=storylines_menu}{Maps
  and Cases}
\item
  \href{https://www.nytimes.com/interactive/2020/science/coronavirus-vaccine-tracker.html?action=click\&pgtype=Article\&state=default\&region=TOP_BANNER\&context=storylines_menu}{Vaccine
  Tracker}
\item
  \href{https://www.nytimes.com/interactive/2020/07/29/us/schools-reopening-coronavirus.html?action=click\&pgtype=Article\&state=default\&region=TOP_BANNER\&context=storylines_menu}{What
  School May Look Like}
\item
  \href{https://www.nytimes.com/live/2020/07/31/business/stock-market-today-coronavirus?action=click\&pgtype=Article\&state=default\&region=TOP_BANNER\&context=storylines_menu}{Economy}
\end{itemize}

Advertisement

\protect\hyperlink{after-top}{Continue reading the main story}

Supported by

\protect\hyperlink{after-sponsor}{Continue reading the main story}

\hypertarget{hoping-to-understand-the-virus-everyone-is-parsing-a-mountain-of-data}{%
\section{Hoping to Understand the Virus, Everyone Is Parsing a Mountain
of
Data}\label{hoping-to-understand-the-virus-everyone-is-parsing-a-mountain-of-data}}

The coronavirus pandemic has left ordinary people debating case counts,
positivity rates and hospitalization numbers in hopes of understanding
the virus's path.

\includegraphics{https://static01.nyt.com/images/2020/07/24/us/00VIRUS-DATA-test/merlin_174873537_7130ec15-4146-45e1-bcf0-7cacd41ca2b1-articleLarge.jpg?quality=75\&auto=webp\&disable=upscale}

\href{https://www.nytimes.com/by/julie-bosman}{\includegraphics{https://static01.nyt.com/images/2018/11/09/multimedia/author-julie-bosman/author-julie-bosman-thumbLarge.png}}

By \href{https://www.nytimes.com/by/julie-bosman}{Julie Bosman}

\begin{itemize}
\item
  Published July 27, 2020Updated July 30, 2020
\item
  \begin{itemize}
  \item
  \item
  \item
  \item
  \item
  \item
  \end{itemize}
\end{itemize}

CHICAGO --- The latest count of new coronavirus cases was jarring: Some
1,500 virus cases were identified three consecutive days last week in
Illinois, and fears of a resurgence in the state even led the mayor of
Chicago to shut down bars all over town on Friday.

But at the same moment, there were other, hopeful data points that
seemed to tell a different story entirely. Deaths from the virus
statewide are one-tenth what they were at their peak in May. And the
positivity rate of new coronavirus tests in Illinois is about half that
of neighboring states.

``There are so many numbers flying around,'' said Dr. Allison Arwady,
the commissioner of the Chicago health department. ``It's hard for
people to know what's the most important thing to follow.''

This is a pandemic that has been told in harrowing stories from
hospitals, factories, nursing homes and meatpacking plants. But as the
crisis stretches on, it is also unfolding in an increasingly complex
spread of numbers.

Six months since the first cases were detected in the United States,
more people have been infected by far than in any other country, and the
daily rundown of national numbers on Friday was a reminder of a mounting
emergency: more than 73,500 new cases, 1,100 deaths and 939,838 tests,
as well as 59,670 people currently hospitalized for the virus.

Americans now have access to an expanding set of data to help them
interpret the coronavirus pandemic. They are closely tracking the number
of sick and dead. They can read daily case counts in their cities and
states, the percentage of positive tests, the number of people
hospitalized and the weekly change in cases. It is possible to look on
the Illinois Department of Public Health
\href{http://www.dph.illinois.gov/covid19/hospitalization-utilization}{website}
and learn how many hospital beds exist statewide, how many ventilators
are available in Peoria and how many intensive-care unit beds are free
in Champaign.

\includegraphics{https://static01.nyt.com/images/2020/07/24/us/00VIRUS-DATA-arwady/merlin_174679425_4c82a6b9-ecf4-4692-8be9-ddae8cc0570b-articleLarge.jpg?quality=75\&auto=webp\&disable=upscale}

Sophisticated data-gathering operations by
\href{https://www.nytimes.com/interactive/2020/us/coronavirus-us-cases.html}{newspapers},
research universities and \href{https://covidtracking.com/}{volunteers}
have sprung up in response to the pandemic, monitoring and collecting
coronavirus metrics around the clock. Elected officials who were not
particularly well versed in public health or infectious disease when
2020 began now sound a little like epidemiologists, spending their days
steeped in data and making policy decisions based on the figures before
them.

``Everybody's tracking this virus in a way that they've never done with
any other infectious disease,'' said Dr. Amesh A. Adalja, a physician
and senior scholar at the Johns Hopkins Center for Health Security who
has treated coronavirus patients. ``For some people, it's helped them
understand what is happening. For other people, it's been misinterpreted
and not very helpful.''

\hypertarget{latest-updates-global-coronavirus-outbreak}{%
\section{\texorpdfstring{\href{https://www.nytimes.com/2020/08/01/world/coronavirus-covid-19.html?action=click\&pgtype=Article\&state=default\&region=MAIN_CONTENT_1\&context=storylines_live_updates}{Latest
Updates: Global Coronavirus
Outbreak}}{Latest Updates: Global Coronavirus Outbreak}}\label{latest-updates-global-coronavirus-outbreak}}

Updated 2020-08-01T17:52:59.169Z

\begin{itemize}
\tightlist
\item
  \href{https://www.nytimes.com/2020/08/01/world/coronavirus-covid-19.html?action=click\&pgtype=Article\&state=default\&region=MAIN_CONTENT_1\&context=storylines_live_updates\#link-3ac56579}{Top
  officials work to break impasse over jobless benefit.}
\item
  \href{https://www.nytimes.com/2020/08/01/world/coronavirus-covid-19.html?action=click\&pgtype=Article\&state=default\&region=MAIN_CONTENT_1\&context=storylines_live_updates\#link-8796723}{The
  virus picks up dangerous speed in the Midwest, and in areas that had
  seen success.}
\item
  \href{https://www.nytimes.com/2020/08/01/world/coronavirus-covid-19.html?action=click\&pgtype=Article\&state=default\&region=MAIN_CONTENT_1\&context=storylines_live_updates\#link-25930521}{Thousands
  in Berlin protest Germany's coronavirus measures.}
\end{itemize}

\href{https://www.nytimes.com/2020/08/01/world/coronavirus-covid-19.html?action=click\&pgtype=Article\&state=default\&region=MAIN_CONTENT_1\&context=storylines_live_updates}{See
more updates}

More live coverage:
\href{https://www.nytimes.com/live/2020/07/31/business/stock-market-today-coronavirus?action=click\&pgtype=Article\&state=default\&region=MAIN_CONTENT_1\&context=storylines_live_updates}{Markets}

He offered an example of an ``armchair epi'' from his own social circle:
Scanning his Facebook feed recently, Dr. Adalja read a high-school
classmate authoritatively sharing information on the coronavirus
fatality rate --- far lower than the flu, the classmate asserted.

Dr. Adalja instantly saw that the calculation was flawed. ``For flu, he
used the denominator of total number of flu infections,'' he said. ``For
coronavirus, he used the denominator for the population of the U.S.''

He resisted the urge to type a hasty correction --- Covid-19 is believed
to have a substantially higher fatality rate than that of the seasonal
flu --- and kept scrolling.

Epidemiologists generally advise against relying on a single data point
to determine the state of the virus in a particular area. Instead, to
assess coronavirus trends, they recommend reviewing the daily case
count, the positivity rate, hospitalization data and the number of
confirmed and probable deaths from the virus. But they also cautioned
that data can be misleading and difficult to interpret, and that it is
perhaps best seen as one piece in a larger, complicated puzzle.

``I view everything with a lot of skepticism,'' said Dr. Natalie E.
Dean, an infectious-disease expert at the University of Florida. ``I try
and triangulate across a lot of different things.''

For many Americans, the numbers are a way to make sense of the pandemic
--- which is spreading in the South, West and much of the Midwest, but
calming in the Northeast --- and to gauge whether things are better or
worse in their own cities.

They often begin with the case count. That is the daily tally of
individuals whose coronavirus infections were confirmed by laboratory
tests, a data point that is frequently quoted, misused and debated.

``If I'm sitting at home and saying, `How is my community doing?' I'd
want to look at daily case counts,'' said Dr. John Swartzberg, an
infectious-disease specialist and a clinical professor emeritus at the
University of California, Berkeley's School of Public Health.

Those numbers are jaw-dropping. In the United States, the cumulative
count of people infected with the coronavirus has surpassed four
million. New daily records tied to the case count have been alarmingly
frequent in recent weeks: At least 16 states have posted single-day case
records last week. On Friday, more than 73,000 new cases were identified
across the country, the second-highest day of the pandemic.

There are several ways to parse the case count number.

Image

Pews were taped off during mass at St. Monica's Catholic Church on
Wednesday in Miami Gardens, Fla.Credit...Saul Martinez for The New York
Times

President Trump and other officials have frequently questioned the
legitimacy of coronavirus case counts, falsely suggesting that a rise in
testing availability is solely responsible for the increase in confirmed
infections. More testing can cause an uptick in new reports of
infections, but data
\href{https://www.nytimes.com/interactive/2020/07/22/us/covid-testing-rising-cases.html?action=click\&module=RelatedLinks\&pgtype=Article}{shows}
that the rise in cases far outpaces the growth in testing.

Experts suggested that the daily case count is better viewed as a rough
measure of whether an outbreak is slowing, expanding or stabilizing. A
decrease in new confirmed cases could also indicate that testing is not
available widely enough, or that there is a backlog of tests that have
not yet been processed and delivered to the local health department.

Time period matters, too. Comparing case counts in July to case counts
in April is misleading, because many people were sick but few people
were tested early in the epidemic. But comparing case count to a more
recent period, when testing was relatively constant, is a useful
measure.

Another frequently cited number is the positivity rate, the percentage
of coronavirus tests that have returned with a positive result.

``The positivity number is one of the first places I go to,'' said Gov.
Mike DeWine of Ohio, who wakes up each morning to a fresh PowerPoint
presentation from his staff, which he reads on his iPad before 8 a.m.
``That's what I zero in on.''

\href{https://www.nytimes.com/news-event/coronavirus?action=click\&pgtype=Article\&state=default\&region=MAIN_CONTENT_3\&context=storylines_faq}{}

\hypertarget{the-coronavirus-outbreak-}{%
\subsubsection{The Coronavirus Outbreak
›}\label{the-coronavirus-outbreak-}}

\hypertarget{frequently-asked-questions}{%
\paragraph{Frequently Asked
Questions}\label{frequently-asked-questions}}

Updated July 27, 2020

\begin{itemize}
\item ~
  \hypertarget{should-i-refinance-my-mortgage}{%
  \paragraph{Should I refinance my
  mortgage?}\label{should-i-refinance-my-mortgage}}

  \begin{itemize}
  \tightlist
  \item
    \href{https://www.nytimes.com/article/coronavirus-money-unemployment.html?action=click\&pgtype=Article\&state=default\&region=MAIN_CONTENT_3\&context=storylines_faq}{It
    could be a good idea,} because mortgage rates have
    \href{https://www.nytimes.com/2020/07/16/business/mortgage-rates-below-3-percent.html?action=click\&pgtype=Article\&state=default\&region=MAIN_CONTENT_3\&context=storylines_faq}{never
    been lower.} Refinancing requests have pushed mortgage applications
    to some of the highest levels since 2008, so be prepared to get in
    line. But defaults are also up, so if you're thinking about buying a
    home, be aware that some lenders have tightened their standards.
  \end{itemize}
\item ~
  \hypertarget{what-is-school-going-to-look-like-in-september}{%
  \paragraph{What is school going to look like in
  September?}\label{what-is-school-going-to-look-like-in-september}}

  \begin{itemize}
  \tightlist
  \item
    It is unlikely that many schools will return to a normal schedule
    this fall, requiring the grind of
    \href{https://www.nytimes.com/2020/06/05/us/coronavirus-education-lost-learning.html?action=click\&pgtype=Article\&state=default\&region=MAIN_CONTENT_3\&context=storylines_faq}{online
    learning},
    \href{https://www.nytimes.com/2020/05/29/us/coronavirus-child-care-centers.html?action=click\&pgtype=Article\&state=default\&region=MAIN_CONTENT_3\&context=storylines_faq}{makeshift
    child care} and
    \href{https://www.nytimes.com/2020/06/03/business/economy/coronavirus-working-women.html?action=click\&pgtype=Article\&state=default\&region=MAIN_CONTENT_3\&context=storylines_faq}{stunted
    workdays} to continue. California's two largest public school
    districts --- Los Angeles and San Diego --- said on July 13, that
    \href{https://www.nytimes.com/2020/07/13/us/lausd-san-diego-school-reopening.html?action=click\&pgtype=Article\&state=default\&region=MAIN_CONTENT_3\&context=storylines_faq}{instruction
    will be remote-only in the fall}, citing concerns that surging
    coronavirus infections in their areas pose too dire a risk for
    students and teachers. Together, the two districts enroll some
    825,000 students. They are the largest in the country so far to
    abandon plans for even a partial physical return to classrooms when
    they reopen in August. For other districts, the solution won't be an
    all-or-nothing approach.
    \href{https://bioethics.jhu.edu/research-and-outreach/projects/eschool-initiative/school-policy-tracker/}{Many
    systems}, including the nation's largest, New York City, are
    devising
    \href{https://www.nytimes.com/2020/06/26/us/coronavirus-schools-reopen-fall.html?action=click\&pgtype=Article\&state=default\&region=MAIN_CONTENT_3\&context=storylines_faq}{hybrid
    plans} that involve spending some days in classrooms and other days
    online. There's no national policy on this yet, so check with your
    municipal school system regularly to see what is happening in your
    community.
  \end{itemize}
\item ~
  \hypertarget{is-the-coronavirus-airborne}{%
  \paragraph{Is the coronavirus
  airborne?}\label{is-the-coronavirus-airborne}}

  \begin{itemize}
  \tightlist
  \item
    The coronavirus
    \href{https://www.nytimes.com/2020/07/04/health/239-experts-with-one-big-claim-the-coronavirus-is-airborne.html?action=click\&pgtype=Article\&state=default\&region=MAIN_CONTENT_3\&context=storylines_faq}{can
    stay aloft for hours in tiny droplets in stagnant air}, infecting
    people as they inhale, mounting scientific evidence suggests. This
    risk is highest in crowded indoor spaces with poor ventilation, and
    may help explain super-spreading events reported in meatpacking
    plants, churches and restaurants.
    \href{https://www.nytimes.com/2020/07/06/health/coronavirus-airborne-aerosols.html?action=click\&pgtype=Article\&state=default\&region=MAIN_CONTENT_3\&context=storylines_faq}{It's
    unclear how often the virus is spread} via these tiny droplets, or
    aerosols, compared with larger droplets that are expelled when a
    sick person coughs or sneezes, or transmitted through contact with
    contaminated surfaces, said Linsey Marr, an aerosol expert at
    Virginia Tech. Aerosols are released even when a person without
    symptoms exhales, talks or sings, according to Dr. Marr and more
    than 200 other experts, who
    \href{https://academic.oup.com/cid/article/doi/10.1093/cid/ciaa939/5867798}{have
    outlined the evidence in an open letter to the World Health
    Organization}.
  \end{itemize}
\item ~
  \hypertarget{what-are-the-symptoms-of-coronavirus}{%
  \paragraph{What are the symptoms of
  coronavirus?}\label{what-are-the-symptoms-of-coronavirus}}

  \begin{itemize}
  \tightlist
  \item
    Common symptoms
    \href{https://www.nytimes.com/article/symptoms-coronavirus.html?action=click\&pgtype=Article\&state=default\&region=MAIN_CONTENT_3\&context=storylines_faq}{include
    fever, a dry cough, fatigue and difficulty breathing or shortness of
    breath.} Some of these symptoms overlap with those of the flu,
    making detection difficult, but runny noses and stuffy sinuses are
    less common.
    \href{https://www.nytimes.com/2020/04/27/health/coronavirus-symptoms-cdc.html?action=click\&pgtype=Article\&state=default\&region=MAIN_CONTENT_3\&context=storylines_faq}{The
    C.D.C. has also} added chills, muscle pain, sore throat, headache
    and a new loss of the sense of taste or smell as symptoms to look
    out for. Most people fall ill five to seven days after exposure, but
    symptoms may appear in as few as two days or as many as 14 days.
  \end{itemize}
\item ~
  \hypertarget{does-asymptomatic-transmission-of-covid-19-happen}{%
  \paragraph{Does asymptomatic transmission of Covid-19
  happen?}\label{does-asymptomatic-transmission-of-covid-19-happen}}

  \begin{itemize}
  \tightlist
  \item
    So far, the evidence seems to show it does. A widely cited
    \href{https://www.nature.com/articles/s41591-020-0869-5}{paper}
    published in April suggests that people are most infectious about
    two days before the onset of coronavirus symptoms and estimated that
    44 percent of new infections were a result of transmission from
    people who were not yet showing symptoms. Recently, a top expert at
    the World Health Organization stated that transmission of the
    coronavirus by people who did not have symptoms was ``very rare,''
    \href{https://www.nytimes.com/2020/06/09/world/coronavirus-updates.html?action=click\&pgtype=Article\&state=default\&region=MAIN_CONTENT_3\&context=storylines_faq\#link-1f302e21}{but
    she later walked back that statement.}
  \end{itemize}
\end{itemize}

A rising positivity rate can point to an uncontrolled outbreak; it can
also indicate that not enough testing is occurring.

Mr. DeWine is an avid reader of the daily PowerPoint presentation, which
he calls the Situation Update. It started small in the early days of the
pandemic. It has grown to at least 31 slides of numbers, charts and
graphs --- every day.

He said he also focused closely on the number of Ohioans who have been
hospitalized for the coronavirus, a data point that is difficult to spin
or misinterpret. Last week, the pandemic approached an alarming
\href{https://www.nytimes.com/interactive/2020/07/23/us/coronavirus-hospitalizations-us.html}{milestone}:
About as many people in the United States are now hospitalized with the
coronavirus as at any other time in the pandemic, including during an
earlier surge in the New York region in the spring.

``Hospitalization is a hard number,'' Mr. DeWine said. ``There's no
fudge on it.''

Image

Nurses walk out of SSM Health St. Anthony Hospital in Oklahoma City last
week.Credit...Nick Oxford for The New York Times

Yet even that measure has caveats. Hospitalizations do not reflect how
many people are sick at home and experiencing mild symptoms ---
particularly younger people --- but who could still be infecting others.

Dr. Tara C. Smith, a professor of epidemiology at Kent State University
who studies infectious diseases, said that viewed individually, much of
the available coronavirus data can only offer a glimpse of the state of
the pandemic.

``I think people tend to cherry-pick what they want to see, to confirm
their biases,'' she said.

She has been hesitant to place much stock in statistics on deaths caused
by coronavirus, for instance. ``I see a lot of use of the fatality
statistics, which are incomplete,'' Dr. Smith said. ``You do have deaths
from coronavirus, but we know those are undercounted. For me, at least,
that is not a particularly useful metric. But those are the type of
statistics that some people grab onto.''

Perhaps the most telling numbers are trend data --- examining which
direction a community or state seems to be heading, said Michael T.
Osterholm, director of the University of Minnesota's Center for
Infectious Disease Research and Policy.

``There's no magic number for any of this,'' Dr. Osterholm said. ``This
is more like a windshield where you're looking at everything in front of
you. It's not one piece of data. It's all of it coming together.''

In 1918, newspapers in cities across the United States published daily
tallies of the sick and the dead from the flu pandemic, said John M.
Barry, the author of ``The Great Influenza,'' and public health
officials made policy decisions accordingly, based on the data.

Today's elected officials have far more granular data to consider.

In Chicago, Dr. Arwady, the city health commissioner, has a call with
Mayor Lori Lightfoot every morning, discussing the city's total cases,
deaths, the seven-day average for testing and detailed hospitalization
numbers, among other metrics.

``Data to me is one of the best ways to make it real for people,'' Dr.
Arwady said. She often tries to steer Chicagoans to look at coronavirus
numbers broken down by ZIP code, so that they understand the risk they
face in their own neighborhoods. ``Mostly, I want people to feel like
Covid is in their lives.''

Advertisement

\protect\hyperlink{after-bottom}{Continue reading the main story}

\hypertarget{site-index}{%
\subsection{Site Index}\label{site-index}}

\hypertarget{site-information-navigation}{%
\subsection{Site Information
Navigation}\label{site-information-navigation}}

\begin{itemize}
\tightlist
\item
  \href{https://help.nytimes.com/hc/en-us/articles/115014792127-Copyright-notice}{©~2020~The
  New York Times Company}
\end{itemize}

\begin{itemize}
\tightlist
\item
  \href{https://www.nytco.com/}{NYTCo}
\item
  \href{https://help.nytimes.com/hc/en-us/articles/115015385887-Contact-Us}{Contact
  Us}
\item
  \href{https://www.nytco.com/careers/}{Work with us}
\item
  \href{https://nytmediakit.com/}{Advertise}
\item
  \href{http://www.tbrandstudio.com/}{T Brand Studio}
\item
  \href{https://www.nytimes.com/privacy/cookie-policy\#how-do-i-manage-trackers}{Your
  Ad Choices}
\item
  \href{https://www.nytimes.com/privacy}{Privacy}
\item
  \href{https://help.nytimes.com/hc/en-us/articles/115014893428-Terms-of-service}{Terms
  of Service}
\item
  \href{https://help.nytimes.com/hc/en-us/articles/115014893968-Terms-of-sale}{Terms
  of Sale}
\item
  \href{https://spiderbites.nytimes.com}{Site Map}
\item
  \href{https://help.nytimes.com/hc/en-us}{Help}
\item
  \href{https://www.nytimes.com/subscription?campaignId=37WXW}{Subscriptions}
\end{itemize}
