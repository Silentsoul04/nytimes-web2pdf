Sections

SEARCH

\protect\hyperlink{site-content}{Skip to
content}\protect\hyperlink{site-index}{Skip to site index}

\href{https://www.nytimes.com/section/politics}{Politics}

\href{https://myaccount.nytimes.com/auth/login?response_type=cookie\&client_id=vi}{}

\href{https://www.nytimes.com/section/todayspaper}{Today's Paper}

\href{/section/politics}{Politics}\textbar{}National Guard Officer Says
Police Used `Excessive' Force at White House Clash

\url{https://nyti.ms/2X05zTB}

\begin{itemize}
\item
\item
\item
\item
\item
\end{itemize}

\href{https://www.nytimes.com/news-event/george-floyd-protests-minneapolis-new-york-los-angeles?action=click\&pgtype=Article\&state=default\&region=TOP_BANNER\&context=storylines_menu}{Race
and America}

\begin{itemize}
\tightlist
\item
  \href{https://www.nytimes.com/2020/07/26/us/protests-portland-seattle-trump.html?action=click\&pgtype=Article\&state=default\&region=TOP_BANNER\&context=storylines_menu}{Protesters
  Return to Other Cities}
\item
  \href{https://www.nytimes.com/2020/07/24/us/portland-oregon-protests-white-race.html?action=click\&pgtype=Article\&state=default\&region=TOP_BANNER\&context=storylines_menu}{Portland
  at the Center}
\item
  \href{https://www.nytimes.com/2020/07/23/podcasts/the-daily/portland-protests.html?action=click\&pgtype=Article\&state=default\&region=TOP_BANNER\&context=storylines_menu}{Podcast:
  Showdown in Portland}
\item
  \href{https://www.nytimes.com/interactive/2020/07/16/us/black-lives-matter-protests-louisville-breonna-taylor.html?action=click\&pgtype=Article\&state=default\&region=TOP_BANNER\&context=storylines_menu}{45
  Days in Louisville}
\end{itemize}

Advertisement

\protect\hyperlink{after-top}{Continue reading the main story}

Supported by

\protect\hyperlink{after-sponsor}{Continue reading the main story}

\hypertarget{national-guard-officer-says-police-used-excessive-force-at-white-house-clash}{%
\section{National Guard Officer Says Police Used `Excessive' Force at
White House
Clash}\label{national-guard-officer-says-police-used-excessive-force-at-white-house-clash}}

An Army National Guard officer at Lafayette Square plans to tell
lawmakers that the Park Police unleashed an ``unprovoked escalation'' on
peaceful protesters last month.

\includegraphics{https://static01.nyt.com/images/2020/07/27/us/politics/27dc-lafayette/merlin_173090928_2641b589-0068-488c-8782-2433e250b400-articleLarge.jpg?quality=75\&auto=webp\&disable=upscale}

\href{https://www.nytimes.com/by/catie-edmondson}{\includegraphics{https://static01.nyt.com/images/2019/11/20/us/politics/catie-edmonson-twitter-chatblog/catie-edmonson-twitter-chatblog-thumbLarge.png}}

By \href{https://www.nytimes.com/by/catie-edmondson}{Catie Edmondson}

\begin{itemize}
\item
  July 27, 2020
\item
  \begin{itemize}
  \item
  \item
  \item
  \item
  \item
  \end{itemize}
\end{itemize}

WASHINGTON --- An Army National Guard officer who was called in to
enforce the crackdown on protests in
\href{https://www.nytimes.com/2020/07/28/us/politics/lafayette-square-park-police-protests.html}{Lafayette
Square} last month will tell lawmakers that the demonstrators were
peaceful and ``subjected to an unprovoked escalation and excessive use
of force,'' according to written testimony made public on Monday.

Maj. Adam DeMarco, an Iraq war veteran who currently serves in the
District of Columbia National Guard, will testify on Tuesday before a
House panel investigating the clash, giving the latest account of how
Park Police and Secret Service officers violently cleared protesters
away from the White House. He intends to testify that the harsh actions
were taken without provocation or adequate warning just before President
Trump walked through the area with senior administration officials to
\href{https://www.nytimes.com/2020/06/01/us/politics/trump-st-johns-church-bible.html}{stage
a photo event} in front of a historic church.

``From my observation, those demonstrators --- our fellow American
citizens --- were engaged in the peaceful expression of their First
Amendment rights,'' Major DeMarco will say, according to the advance
text of his remarks. ``Yet they were subjected to an unprovoked
escalation and excessive use of force.''

The
\href{https://www.nytimes.com/2020/06/02/us/politics/trump-walk-lafayette-square.html}{shocking
clash} on June 1, yards from the White House, produced stunning images
as mounted police and riot officers routed demonstrators with smoke,
flash grenades and tear gas, minutes after Mr. Trump declared himself
``your president of law and order'' and ``an ally of all peaceful
protesters.'' Those events have prompted lawmakers, infuriated by the
violent scene, to investigate who ordered the attack on protesters and
why.

The reckoning has been particularly acute in the military. Gen. Mark A.
Milley, the chairman of the Joint Chiefs of Staff and the nation's top
military officer,
\href{https://www.nytimes.com/2020/06/11/us/politics/trump-milley-military-protests-lafayette-square.html}{publicly
apologized} for taking part in the president's photo op. Members of the
District of Columbia National Guard --- a majority of whose personnel
are people of color --- have both
\href{https://www.nytimes.com/2020/06/10/us/politics/national-guard-protests.html}{publicly
and privately lamented} their role in the protests.

The District of Columbia National Guard, typically deployed to help
after natural disasters or to assist with managing crowds and logistics
support for large public events in the capital, was called in with Guard
units from other states to help respond to the growing protests in front
of the White House in the aftermath of the
\href{https://www.nytimes.com/news-event/george-floyd-protests-minneapolis-new-york-los-angeles}{police
killing of George Floyd} and other Black Americans.

The Guard's job on June 1 was not to clear protesters, Major DeMarco
will say, according to the text, but to ``hold a static line,''
establishing a new security perimeter around the White House.

But he was taken aback, according to his testimony, when the Park Police
began issuing orders to protesters to evacuate the park 40 minutes
before the city's curfew began.

``From where I was standing, approximately 20 yards from the
demonstrators, the announcements were barely audible,'' Major DeMarco
says, ``and I saw no indication that the demonstrators were cognizant of
the warnings to disperse.''

Major DeMarco said that a liaison officer for the Park Police told him
that no tear gas was being deployed against the protesters, but that he
felt ``irritation in my eyes and nose'' that he identified as tear gas.
Later, he saw tear gas canisters on the street. The Park Police
initially denied that it had been used against protesters, and then
claimed the statement was a ``mistake.''

Major DeMarco also noted that the equipment to build a
\href{https://www.nytimes.com/2020/06/05/us/politics/white-house-security.html}{barrier
around the White House} --- the stated reason for clearing the
protesters --- did not arrive until 9 p.m., more than two hours after
the evacuation order was given. He called the episode ``deeply
disturbing.''

Gregory T. Monahan, the acting chief of the Park Police, will testify on
Tuesday before the House Committee on Natural Resources, along with
Major DeMarco.

The episode at Lafayette Square is the latest controversy for the Park
Police, which prosecutors and defense lawyers say has a reputation for
\href{https://www.nytimes.com/2020/06/18/us/politics/park-police-gregory-monahan.html}{fostering
a culture of recklessness}. As a U.S. Park Police patrol officer nearly
two decades ago, Mr. Monahan was accused of conducting unlawful body
cavity searches and providing unreliable testimony.

A spokeswoman for the National Park Service said the allegations were
investigated and ``determined to be unfounded.''

Advertisement

\protect\hyperlink{after-bottom}{Continue reading the main story}

\hypertarget{site-index}{%
\subsection{Site Index}\label{site-index}}

\hypertarget{site-information-navigation}{%
\subsection{Site Information
Navigation}\label{site-information-navigation}}

\begin{itemize}
\tightlist
\item
  \href{https://help.nytimes.com/hc/en-us/articles/115014792127-Copyright-notice}{©~2020~The
  New York Times Company}
\end{itemize}

\begin{itemize}
\tightlist
\item
  \href{https://www.nytco.com/}{NYTCo}
\item
  \href{https://help.nytimes.com/hc/en-us/articles/115015385887-Contact-Us}{Contact
  Us}
\item
  \href{https://www.nytco.com/careers/}{Work with us}
\item
  \href{https://nytmediakit.com/}{Advertise}
\item
  \href{http://www.tbrandstudio.com/}{T Brand Studio}
\item
  \href{https://www.nytimes.com/privacy/cookie-policy\#how-do-i-manage-trackers}{Your
  Ad Choices}
\item
  \href{https://www.nytimes.com/privacy}{Privacy}
\item
  \href{https://help.nytimes.com/hc/en-us/articles/115014893428-Terms-of-service}{Terms
  of Service}
\item
  \href{https://help.nytimes.com/hc/en-us/articles/115014893968-Terms-of-sale}{Terms
  of Sale}
\item
  \href{https://spiderbites.nytimes.com}{Site Map}
\item
  \href{https://help.nytimes.com/hc/en-us}{Help}
\item
  \href{https://www.nytimes.com/subscription?campaignId=37WXW}{Subscriptions}
\end{itemize}
