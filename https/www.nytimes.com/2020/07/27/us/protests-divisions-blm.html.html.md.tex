Sections

SEARCH

\protect\hyperlink{site-content}{Skip to
content}\protect\hyperlink{site-index}{Skip to site index}

\href{https://www.nytimes.com/section/us}{U.S.}

\href{https://myaccount.nytimes.com/auth/login?response_type=cookie\&client_id=vi}{}

\href{https://www.nytimes.com/section/todayspaper}{Today's Paper}

\href{/section/us}{U.S.}\textbar{}Peaceful Protesters With `Room for
Rage' Sympathize With Aggressive Tactics

\url{https://nyti.ms/3jObCEg}

\begin{itemize}
\item
\item
\item
\item
\item
\item
\end{itemize}

\href{https://www.nytimes.com/news-event/george-floyd-protests-minneapolis-new-york-los-angeles?action=click\&pgtype=Article\&state=default\&region=TOP_BANNER\&context=storylines_menu}{Race
and America}

\begin{itemize}
\tightlist
\item
  \href{https://www.nytimes.com/2020/07/26/us/protests-portland-seattle-trump.html?action=click\&pgtype=Article\&state=default\&region=TOP_BANNER\&context=storylines_menu}{Protesters
  Return to Other Cities}
\item
  \href{https://www.nytimes.com/2020/07/24/us/portland-oregon-protests-white-race.html?action=click\&pgtype=Article\&state=default\&region=TOP_BANNER\&context=storylines_menu}{Portland
  at the Center}
\item
  \href{https://www.nytimes.com/2020/07/23/podcasts/the-daily/portland-protests.html?action=click\&pgtype=Article\&state=default\&region=TOP_BANNER\&context=storylines_menu}{Podcast:
  Showdown in Portland}
\item
  \href{https://www.nytimes.com/interactive/2020/07/16/us/black-lives-matter-protests-louisville-breonna-taylor.html?action=click\&pgtype=Article\&state=default\&region=TOP_BANNER\&context=storylines_menu}{45
  Days in Louisville}
\end{itemize}

Advertisement

\protect\hyperlink{after-top}{Continue reading the main story}

Supported by

\protect\hyperlink{after-sponsor}{Continue reading the main story}

\hypertarget{peaceful-protesters-with-room-for-rage-sympathize-with-aggressive-tactics}{%
\section{Peaceful Protesters With `Room for Rage' Sympathize With
Aggressive
Tactics}\label{peaceful-protesters-with-room-for-rage-sympathize-with-aggressive-tactics}}

A forceful campaign by federal law enforcement in Portland, Ore., has
prompted an escalation in tactics from some protesters, while others
wonder whether they are losing their focus.

\includegraphics{https://static01.nyt.com/images/2020/07/27/us/27PROTESTS-DIVIDE01/27PROTESTS-DIVIDE01-articleLarge.jpg?quality=75\&auto=webp\&disable=upscale}

By \href{https://www.nytimes.com/by/kate-conger}{Kate Conger},
\href{https://www.nytimes.com/by/thomas-fuller}{Thomas Fuller} and
\href{https://www.nytimes.com/by/mike-baker}{Mike Baker}

\begin{itemize}
\item
  Published July 27, 2020Updated July 29, 2020
\item
  \begin{itemize}
  \item
  \item
  \item
  \item
  \item
  \item
  \end{itemize}
\end{itemize}

PORTLAND, Ore. --- Minutes before midnight on Sunday, the first firework
of the evening screeched low over the heads of protesters gathered in
Portland, Ore., sprinkling them with white flecks of light and
ricocheting into the courthouse that has become a symbol of an unwelcome
federal incursion into a local outcry over police brutality.

Some ducked under
\href{https://www.nytimes.com/2020/07/22/us/portland-protest-tactics.html}{makeshift
shields} to protect themselves from the falling sparks, while others
cheered at the sight of the blazing projectile hurtling toward the
courthouse and the federal law enforcement agents inside: ``This is what
democracy looks like!''

In recent weeks, protesters in Portland have pointed laser beams, lobbed
water bottles and trash bags and, in one case,
\href{https://www.portlandoregon.gov/police/news/read.cfm?id=250945}{according
to the Portland Police Bureau}, hurled an open pocketknife at the
officers guarding the courthouse; they have used power tools, crowbars
and bolt cutters to yank down a fence. In Seattle, demonstrators over
the weekend
\href{https://www.nytimes.com/2020/07/25/us/protests-seattle-portland.html}{set
fire to several construction trailers} at a youth detention facility,
and protests in Richmond, Va., Los Angeles and Oakland, Calif., were
also marked by fires.

Yet the nightly assault
\href{https://www.nytimes.com/2020/07/22/us/portland-protests-courthouse.html}{on
the federal courthouse} has been part of a much wider peaceful
resistance --- high school students, military veterans, off-duty
lawyers, lines of mothers who call themselves the ``Wall of Moms'' ---
that began assembling nearly two months ago in the wake of George
Floyd's death at the hands of the Minneapolis police. The aim, as it has
been in other cities, was to assemble for sweeping police reform and
racial justice.

The raucous escalation in recent days, brought about by the deployment
of federal law enforcement officers and the harsh tactics they have used
against protesters, has prompted new debates among the protesters over
their own tactics and goals.

Now battling nightly rounds of pepper spray and impact munitions fired
by federal forces, some activists worry that the nightly clashes are
distracting from their demands for defunding or reforming local police
departments.

\includegraphics{https://static01.nyt.com/images/2020/07/27/us/27PROTESTS-DIVIDE02/merlin_174932754_9998e60c-898f-4fdf-998e-56bf8f16d482-articleLarge.jpg?quality=75\&auto=webp\&disable=upscale}

``To see people standing in Portland destroying property and not
actually doing the work of advocating for Black people was disturbing,''
said Rachelle Dixon, the vice chair of the Multnomah County Democrats
and an organizer in the Black community. ``I think they're a distraction
from the everyday needs of people of color, especially Black people. My
life is not going to improve because you broke the glass at the Louis
Vuitton store.''

During the protests Sunday night in Portland, organizers tried to gently
coax protesters away from the courthouse, calling them to the nearby
Multnomah County Justice Center to listen to speeches. ``We're family
now,'' said one of the speakers, a local activist and artist who
performs under the stage name Itchy Trigga. ``We can't allow the feds to
break up our family.''

But as the protest ticked closer to its 60th consecutive day, protesters
regrouped in front of the courthouse. Federal agents responded to the
fireworks from the crowd with tear gas, sending protesters reeling, and
later began a pursuit through Portland's streets.

On Monday, the U.S. attorney in Portland, Billy J. Williams, appealed
for an end to the nightly clashes. ``I ask all Portlanders to join us in
working with community leaders, faith leaders and business leaders to
find an end to this,'' he said. ``The violence is wearing this city
out.''

Mayor Ted Wheeler and Jo Ann Hardesty, a city commissioner, said in a
statement that they wanted to ``discuss a cease-fire and removal of
heightened federal forces'' with the Department of Homeland Security,
including Chad F. Wolf, its acting secretary.

Earlier in the day, federal authorities announced that they had
identified 100 additional U.S. Marshals Service personnel to send to
Oregon if needed to relieve or supplement the current force protecting
federal property there. ``We are also determined to reduce the violence
aimed at the federal courthouse in Portland by violent extremists,''
Drew Wade, a spokesman for the Marshals Service, said in a statement.

Yet even some of the demonstrators who fear that the federal presence
has distracted from the original Black Lives Matter message say it is
important for the community to voice its opposition to the dispatching
of federal agents to a city whose leaders have opposed the deployment.
And protesting the militarized federal presence, they say, is not far
off message from the long-running protests against the local police.

And some of those who are not engaged in the more aggressive tactics
being employed find themselves sympathetic to those who are; the federal
government, they say, is repeatedly shooting at protesters with tear
gas, pepper balls and other exploding devices, tactics that have sent
demonstrators to hospitals.

``There may be people throwing water bottles at officers. I'm not going
to do that because I don't see the point,'' said Jennifer Kristiansen, a
family-law attorney who joined the Wall of Moms last week and was later
arrested by federal agents. ``But if people want to express their
frustration in that way, I'm not going to stop them.''

``There is room for chanting and dancing and joyful noises and there is
also room for rage. We make that space for each other,'' she said.

That sentiment has been echoed by some of those in other cities who
joined weekend protests that also opposed the deployment of federal
agents in Portland.

Cat Brooks, a racial justice organizer in Oakland and the co-founder of
the Anti Police-Terror Project, said Black Lives Matter protests and the
movement to oust federal forces from cities were ``one connected
struggle.''

The debate among organizers, she said, is the tactics that protesters
should use. Her own view is that protesters cannot be blamed for
responding forcefully when confronted with rubber bullets and pepper
spray, as they have been in Portland.

``I don't consider property destruction violence,'' she said. ``Violence
is when you attack a person or another living, breathing creature on
this planet. Windows don't cry and they can't die.''

Organizers in Oakland, which has a long tradition of loud protests, are
watching closely whether federal forces will be deployed there.

Image

A protester fanned flames outside the federal courthouse on
Thursday.Credit...Octavio Jones for The New York Times

``If the feds come to Oakland, it's going to make Portland look like
Disneyland,'' Ms. Brooks said.

She rejected the notion, put forward by Mayor Libby Schaaf of Oakland,
among others, that violence and property destruction reinforce President
Trump's message that anarchists were taking over the country's streets.

``We could sit there and sing `Kumbaya' and suck lollipops between now
and November, but if Donald Trump thinks it makes sense to hit Oakland
and Detroit and some of the other cities with large Black populations
--- then that's what Donald Trump is going to do,'' Ms. Brooks said.

In Seattle on Saturday, as a large crowd marching through the city
stopped in front of a new youth detention center, some went and knocked
over a nearby construction trailer while others lit fire to the
construction buildings, drawing cheers from the crowd.

Further down the street, some in the crowd smashed the windows of
buildings, including a Starbucks, where a fire was lit inside. As smoke
came out of the broken windows, people on the street called up to the
residents living above the coffee shop, suggesting that they evacuate.

Jamie Boudreau, who runs a bar a block away, and who described himself
as a ``100 percent'' supporter of the Black Lives Matter movement, said
he was texting his wife as the crowd went by. Someone confronted him,
accusing him of taking a video of the crowd. He said it escalated until
people were punching and spitting on him and an employee. His storefront
windows were smashed.

``I was like, `Guys, you are totally targeting the wrong person right
now,''' Mr. Boudreau said. ``We've been on the marches. It's just so
bizarre.''

In Richmond, a burst of violence over the weekend took residents by
surprise, and broke what Mayor Levar Stoney said had been 24 days of
peaceful gatherings in Virginia's capital.

The gathering began peacefully in Monroe Park on Saturday night. Several
hundred protesters then left the park and began to march through the
city. But when they reached the Police Department, some in the crowd
``became very aggressive verbally toward the officers there,'' the chief
of police, Gerald M. Smith, said at a news conference. He said ``the
rioters in the crowd'' threw bricks, batteries and rocks at the police.
He said the police took action to disperse them after some set a city
dump truck on fire.

Mr. Stoney noted that bricks were lobbed at firefighters who were
attempting to extinguish the blaze. He said he suspected that white
supremacists were behind the violence.

``We have identified some individuals who have been seen with the
Boogaloo boys and some antifa groups around the area,'' Chief Smith
said.

``People broke windows and spray-painted private property with hateful
language,'' he said. ``Frankly, it was disgusting.''

Kate Conger reported from Portland, Thomas Fuller from Oakland, Calif.,
and Mike Baker from Seattle. Nicholas Bogel-Burroughs contributed
reporting from New York, and Zolan Kanno-Youngs and Sabrina Tavernise
from Washington.

Advertisement

\protect\hyperlink{after-bottom}{Continue reading the main story}

\hypertarget{site-index}{%
\subsection{Site Index}\label{site-index}}

\hypertarget{site-information-navigation}{%
\subsection{Site Information
Navigation}\label{site-information-navigation}}

\begin{itemize}
\tightlist
\item
  \href{https://help.nytimes.com/hc/en-us/articles/115014792127-Copyright-notice}{©~2020~The
  New York Times Company}
\end{itemize}

\begin{itemize}
\tightlist
\item
  \href{https://www.nytco.com/}{NYTCo}
\item
  \href{https://help.nytimes.com/hc/en-us/articles/115015385887-Contact-Us}{Contact
  Us}
\item
  \href{https://www.nytco.com/careers/}{Work with us}
\item
  \href{https://nytmediakit.com/}{Advertise}
\item
  \href{http://www.tbrandstudio.com/}{T Brand Studio}
\item
  \href{https://www.nytimes.com/privacy/cookie-policy\#how-do-i-manage-trackers}{Your
  Ad Choices}
\item
  \href{https://www.nytimes.com/privacy}{Privacy}
\item
  \href{https://help.nytimes.com/hc/en-us/articles/115014893428-Terms-of-service}{Terms
  of Service}
\item
  \href{https://help.nytimes.com/hc/en-us/articles/115014893968-Terms-of-sale}{Terms
  of Sale}
\item
  \href{https://spiderbites.nytimes.com}{Site Map}
\item
  \href{https://help.nytimes.com/hc/en-us}{Help}
\item
  \href{https://www.nytimes.com/subscription?campaignId=37WXW}{Subscriptions}
\end{itemize}
