Sections

SEARCH

\protect\hyperlink{site-content}{Skip to
content}\protect\hyperlink{site-index}{Skip to site index}

\href{https://www.nytimes.com/section/well/eat}{Eat}

\href{https://myaccount.nytimes.com/auth/login?response_type=cookie\&client_id=vi}{}

\href{https://www.nytimes.com/section/todayspaper}{Today's Paper}

\href{/section/well/eat}{Eat}\textbar{}With Eating Disorders, Looks Can
Be Deceiving

\url{https://nyti.ms/3f3WBL1}

\begin{itemize}
\item
\item
\item
\item
\item
\item
\end{itemize}

\href{https://www.nytimes.com/news-event/coronavirus?action=click\&pgtype=Article\&state=default\&region=TOP_BANNER\&context=storylines_menu}{The
Coronavirus Outbreak}

\begin{itemize}
\tightlist
\item
  live\href{https://www.nytimes.com/2020/08/01/world/coronavirus-covid-19.html?action=click\&pgtype=Article\&state=default\&region=TOP_BANNER\&context=storylines_menu}{Latest
  Updates}
\item
  \href{https://www.nytimes.com/interactive/2020/us/coronavirus-us-cases.html?action=click\&pgtype=Article\&state=default\&region=TOP_BANNER\&context=storylines_menu}{Maps
  and Cases}
\item
  \href{https://www.nytimes.com/interactive/2020/science/coronavirus-vaccine-tracker.html?action=click\&pgtype=Article\&state=default\&region=TOP_BANNER\&context=storylines_menu}{Vaccine
  Tracker}
\item
  \href{https://www.nytimes.com/interactive/2020/07/29/us/schools-reopening-coronavirus.html?action=click\&pgtype=Article\&state=default\&region=TOP_BANNER\&context=storylines_menu}{What
  School May Look Like}
\item
  \href{https://www.nytimes.com/live/2020/07/31/business/stock-market-today-coronavirus?action=click\&pgtype=Article\&state=default\&region=TOP_BANNER\&context=storylines_menu}{Economy}
\end{itemize}

Advertisement

\protect\hyperlink{after-top}{Continue reading the main story}

Supported by

\protect\hyperlink{after-sponsor}{Continue reading the main story}

Personal Health

\hypertarget{with-eating-disorders-looks-can-be-deceiving}{%
\section{With Eating Disorders, Looks Can Be
Deceiving}\label{with-eating-disorders-looks-can-be-deceiving}}

Distorted eating behaviors occur in young people irrespective of their
weight, gender, race, ethnicity or sexual orientation.

\includegraphics{https://static01.nyt.com/images/2020/07/28/science/WEL-BRODY-YOUTHEATINGDISORDER/WEL-BRODY-YOUTHEATINGDISORDER-articleLarge.jpg?quality=75\&auto=webp\&disable=upscale}

\href{https://www.nytimes.com/by/jane-e-brody}{\includegraphics{https://static01.nyt.com/images/2018/06/12/multimedia/jane-e-brody/jane-e-brody-thumbLarge.png}}

By \href{https://www.nytimes.com/by/jane-e-brody}{Jane E. Brody}

\begin{itemize}
\item
  July 27, 2020
\item
  \begin{itemize}
  \item
  \item
  \item
  \item
  \item
  \item
  \end{itemize}
\end{itemize}

Appearances, as I'm sure you know, can be deceiving. In one
all-too-common example, adolescents and young adults with disordered
eating habits or outright eating disorders often go unrecognized by both
parents and physicians because their appearance defies common beliefs:
they don't look like they have an eating problem.

One such belief is that people with anorexia always look scrawny and
malnourished when in fact they may be of normal weight or even
overweight, according to recent research at the University of
California, San Francisco.

The researchers, led by Dr. Jason M. Nagata, a specialist in adolescent
medicine at the university's Benioff Children's Hospital,
\href{https://onlinelibrary.wiley.com/doi/full/10.1002/eat.23094}{found
in a national survey} that distorted eating behaviors occur in young
people irrespective of their weight, gender, race, ethnicity or sexual
orientation. And it's not just about losing weight.

The survey revealed that among young adults aged 18 to 24, 22 percent of
males and 5 percent of females were striving to gain weight or build
muscle by relying on eating habits that may appear to be healthy but
that the researchers categorized as risky. These practices include
overconsuming protein and avoiding fats and carbohydrates. The use of
poorly tested dietary supplements and anabolic steroids was also common
among those surveyed.

The Covid-19 pandemic has likely exacerbated the problem for many
teenagers whose daily routines have been disrupted and who now find
themselves at home all day with lots of food being hoarded in kitchens
and pantries, Dr. Nagata said in an interview. ``We're seeing more
patients and referrals for eating disorders and their complications,''
he said.

Without a proper diagnosis and intervention, young people with distorted
eating behaviors can jeopardize their growth and long-term health and
may even create a substance abuse problem. The findings suggest that
abnormal behavior with regard to food and exercise is often overlooked,
misunderstood, ignored or perhaps viewed as a passing phase of
adolescence.

This is especially true among teenage boys. One-third of the high school
boys surveyed said they were trying to gain weight and bulk up, and many
were using risky methods to achieve their goals, Dr. Nagata told me.
Sixty percent of the girls surveyed said they were trying to lose
weight. Some consumed unbalanced diets that can jeopardize their growth
and long-term health; others resorted to induced vomiting or abused
laxatives, diuretics, diet pills or engaged in other hazardous behaviors
like fasting or excessive exercise.

\hypertarget{latest-updates-global-coronavirus-outbreak}{%
\section{\texorpdfstring{\href{https://www.nytimes.com/2020/08/01/world/coronavirus-covid-19.html?action=click\&pgtype=Article\&state=default\&region=MAIN_CONTENT_1\&context=storylines_live_updates}{Latest
Updates: Global Coronavirus
Outbreak}}{Latest Updates: Global Coronavirus Outbreak}}\label{latest-updates-global-coronavirus-outbreak}}

Updated 2020-08-02T06:58:18.835Z

\begin{itemize}
\tightlist
\item
  \href{https://www.nytimes.com/2020/08/01/world/coronavirus-covid-19.html?action=click\&pgtype=Article\&state=default\&region=MAIN_CONTENT_1\&context=storylines_live_updates\#link-34047410}{The
  U.S. reels as July cases more than double the total of any other
  month.}
\item
  \href{https://www.nytimes.com/2020/08/01/world/coronavirus-covid-19.html?action=click\&pgtype=Article\&state=default\&region=MAIN_CONTENT_1\&context=storylines_live_updates\#link-780ec966}{Top
  U.S. officials work to break an impasse over the federal jobless
  benefit.}
\item
  \href{https://www.nytimes.com/2020/08/01/world/coronavirus-covid-19.html?action=click\&pgtype=Article\&state=default\&region=MAIN_CONTENT_1\&context=storylines_live_updates\#link-2bc8948}{Its
  outbreak untamed, Melbourne goes into even greater lockdown.}
\end{itemize}

\href{https://www.nytimes.com/2020/08/01/world/coronavirus-covid-19.html?action=click\&pgtype=Article\&state=default\&region=MAIN_CONTENT_1\&context=storylines_live_updates}{See
more updates}

More live coverage:
\href{https://www.nytimes.com/live/2020/07/31/business/stock-market-today-coronavirus?action=click\&pgtype=Article\&state=default\&region=MAIN_CONTENT_1\&context=storylines_live_updates}{Markets}

Over all, distorted eating was more than twice as common among females
than males. It was also reported more often among those who described
themselves as Asian/Pacific Islanders, gay, lesbian or bisexual.

The survey was conducted among a national sample of 14,891 young adults
who were followed for seven years, starting at an average age of 15. The
goal was to see if the youngsters' perceptions and habits surrounding
food and exercise could serve as warning signs of behaviors that could
injure them physically and emotionally.

Among the boys in the study, those most at risk worried that their
bodies were puny --- too small, too skinny or insufficiently muscular,
prompting them to consume unbalanced diets, exercise obsessively and
take supplements or steroids that are a hazard to their health. When
overly fixated on building their bodies, they can become socially
withdrawn and depressed and develop a disorder called muscle dysmorphia
that can lead to heart failure, resulting from insufficient calories and
overexertion.

A missed diagnosis is likely when a young person's relatives or doctors
have preconceived notions about how someone with an eating disorder
looks or behaves. For example, Dr. Nagata said, a teenage girl or young
woman who is anorexic can fall under the medical radar because her
weight is normal or even overweight.

Although diagnosis of an eating disorder like anorexia or bulimia was
twice as common among the young adults whose weight was normal or
underweight, the fact that these disorders also exist in heavier young
adults is often overlooked, Dr. Nagata said.

``Almost half of those with anorexia nervosa are at or above normal
weight,'' he said. ``Young people with atypical anorexia have the same
body image distortions and severe psychological distress as those with
regular anorexia. They're at high medical risk and just as likely to be
hospitalized for complications caused by their distorted eating
behaviors.''

Dr. Nagata's colleague and co-author of the study, Dr. Kirsten
Bibbins-Domingo, an internist at the university, said in an interview,
``Physicians who care for young adults should think about patterns of
eating that are harmful, and not just among very thin women. Young
adults with abnormal eating habits too often fall between the cracks
because physicians think of them as healthy. However, abnormal eating
patterns are not uncommon in adolescence and young adulthood, and that's
when patterns of behavior related to later health and disease are
established and solidified.''

\href{https://www.nytimes.com/news-event/coronavirus?action=click\&pgtype=Article\&state=default\&region=MAIN_CONTENT_3\&context=storylines_faq}{}

\hypertarget{the-coronavirus-outbreak-}{%
\subsubsection{The Coronavirus Outbreak
›}\label{the-coronavirus-outbreak-}}

\hypertarget{frequently-asked-questions}{%
\paragraph{Frequently Asked
Questions}\label{frequently-asked-questions}}

Updated July 27, 2020

\begin{itemize}
\item ~
  \hypertarget{should-i-refinance-my-mortgage}{%
  \paragraph{Should I refinance my
  mortgage?}\label{should-i-refinance-my-mortgage}}

  \begin{itemize}
  \tightlist
  \item
    \href{https://www.nytimes.com/article/coronavirus-money-unemployment.html?action=click\&pgtype=Article\&state=default\&region=MAIN_CONTENT_3\&context=storylines_faq}{It
    could be a good idea,} because mortgage rates have
    \href{https://www.nytimes.com/2020/07/16/business/mortgage-rates-below-3-percent.html?action=click\&pgtype=Article\&state=default\&region=MAIN_CONTENT_3\&context=storylines_faq}{never
    been lower.} Refinancing requests have pushed mortgage applications
    to some of the highest levels since 2008, so be prepared to get in
    line. But defaults are also up, so if you're thinking about buying a
    home, be aware that some lenders have tightened their standards.
  \end{itemize}
\item ~
  \hypertarget{what-is-school-going-to-look-like-in-september}{%
  \paragraph{What is school going to look like in
  September?}\label{what-is-school-going-to-look-like-in-september}}

  \begin{itemize}
  \tightlist
  \item
    It is unlikely that many schools will return to a normal schedule
    this fall, requiring the grind of
    \href{https://www.nytimes.com/2020/06/05/us/coronavirus-education-lost-learning.html?action=click\&pgtype=Article\&state=default\&region=MAIN_CONTENT_3\&context=storylines_faq}{online
    learning},
    \href{https://www.nytimes.com/2020/05/29/us/coronavirus-child-care-centers.html?action=click\&pgtype=Article\&state=default\&region=MAIN_CONTENT_3\&context=storylines_faq}{makeshift
    child care} and
    \href{https://www.nytimes.com/2020/06/03/business/economy/coronavirus-working-women.html?action=click\&pgtype=Article\&state=default\&region=MAIN_CONTENT_3\&context=storylines_faq}{stunted
    workdays} to continue. California's two largest public school
    districts --- Los Angeles and San Diego --- said on July 13, that
    \href{https://www.nytimes.com/2020/07/13/us/lausd-san-diego-school-reopening.html?action=click\&pgtype=Article\&state=default\&region=MAIN_CONTENT_3\&context=storylines_faq}{instruction
    will be remote-only in the fall}, citing concerns that surging
    coronavirus infections in their areas pose too dire a risk for
    students and teachers. Together, the two districts enroll some
    825,000 students. They are the largest in the country so far to
    abandon plans for even a partial physical return to classrooms when
    they reopen in August. For other districts, the solution won't be an
    all-or-nothing approach.
    \href{https://bioethics.jhu.edu/research-and-outreach/projects/eschool-initiative/school-policy-tracker/}{Many
    systems}, including the nation's largest, New York City, are
    devising
    \href{https://www.nytimes.com/2020/06/26/us/coronavirus-schools-reopen-fall.html?action=click\&pgtype=Article\&state=default\&region=MAIN_CONTENT_3\&context=storylines_faq}{hybrid
    plans} that involve spending some days in classrooms and other days
    online. There's no national policy on this yet, so check with your
    municipal school system regularly to see what is happening in your
    community.
  \end{itemize}
\item ~
  \hypertarget{is-the-coronavirus-airborne}{%
  \paragraph{Is the coronavirus
  airborne?}\label{is-the-coronavirus-airborne}}

  \begin{itemize}
  \tightlist
  \item
    The coronavirus
    \href{https://www.nytimes.com/2020/07/04/health/239-experts-with-one-big-claim-the-coronavirus-is-airborne.html?action=click\&pgtype=Article\&state=default\&region=MAIN_CONTENT_3\&context=storylines_faq}{can
    stay aloft for hours in tiny droplets in stagnant air}, infecting
    people as they inhale, mounting scientific evidence suggests. This
    risk is highest in crowded indoor spaces with poor ventilation, and
    may help explain super-spreading events reported in meatpacking
    plants, churches and restaurants.
    \href{https://www.nytimes.com/2020/07/06/health/coronavirus-airborne-aerosols.html?action=click\&pgtype=Article\&state=default\&region=MAIN_CONTENT_3\&context=storylines_faq}{It's
    unclear how often the virus is spread} via these tiny droplets, or
    aerosols, compared with larger droplets that are expelled when a
    sick person coughs or sneezes, or transmitted through contact with
    contaminated surfaces, said Linsey Marr, an aerosol expert at
    Virginia Tech. Aerosols are released even when a person without
    symptoms exhales, talks or sings, according to Dr. Marr and more
    than 200 other experts, who
    \href{https://academic.oup.com/cid/article/doi/10.1093/cid/ciaa939/5867798}{have
    outlined the evidence in an open letter to the World Health
    Organization}.
  \end{itemize}
\item ~
  \hypertarget{what-are-the-symptoms-of-coronavirus}{%
  \paragraph{What are the symptoms of
  coronavirus?}\label{what-are-the-symptoms-of-coronavirus}}

  \begin{itemize}
  \tightlist
  \item
    Common symptoms
    \href{https://www.nytimes.com/article/symptoms-coronavirus.html?action=click\&pgtype=Article\&state=default\&region=MAIN_CONTENT_3\&context=storylines_faq}{include
    fever, a dry cough, fatigue and difficulty breathing or shortness of
    breath.} Some of these symptoms overlap with those of the flu,
    making detection difficult, but runny noses and stuffy sinuses are
    less common.
    \href{https://www.nytimes.com/2020/04/27/health/coronavirus-symptoms-cdc.html?action=click\&pgtype=Article\&state=default\&region=MAIN_CONTENT_3\&context=storylines_faq}{The
    C.D.C. has also} added chills, muscle pain, sore throat, headache
    and a new loss of the sense of taste or smell as symptoms to look
    out for. Most people fall ill five to seven days after exposure, but
    symptoms may appear in as few as two days or as many as 14 days.
  \end{itemize}
\item ~
  \hypertarget{does-asymptomatic-transmission-of-covid-19-happen}{%
  \paragraph{Does asymptomatic transmission of Covid-19
  happen?}\label{does-asymptomatic-transmission-of-covid-19-happen}}

  \begin{itemize}
  \tightlist
  \item
    So far, the evidence seems to show it does. A widely cited
    \href{https://www.nature.com/articles/s41591-020-0869-5}{paper}
    published in April suggests that people are most infectious about
    two days before the onset of coronavirus symptoms and estimated that
    44 percent of new infections were a result of transmission from
    people who were not yet showing symptoms. Recently, a top expert at
    the World Health Organization stated that transmission of the
    coronavirus by people who did not have symptoms was ``very rare,''
    \href{https://www.nytimes.com/2020/06/09/world/coronavirus-updates.html?action=click\&pgtype=Article\&state=default\&region=MAIN_CONTENT_3\&context=storylines_faq\#link-1f302e21}{but
    she later walked back that statement.}
  \end{itemize}
\end{itemize}

The problem of disordered eating behaviors among teens and young adults
is often encouraged or compounded by participation in certain
competitive sports and other activities that overemphasize a particular
body weight and physique. Among these are gymnastics, wrestling, dance,
figure skating, weight lifting and bodybuilding.

Social media, with its heavy focus on appearance, has fostered the
problem as well, Dr. Nagata said. Even toys, like Barbie dolls and
action figures, have made a contribution. ``A study of male action
figures found that they have become bigger, more muscular and more
extreme in their appearance over a 30-year period,'' he said.

``If youngsters are obsessed with an idealized body image, their
thinking and behavior become disordered and can take over their lives,''
he said. ``The detrimental effects can be subtle. Prior to the pandemic,
they may have rejected going out with friends so they could spend more
time in the gym. It's a warning sign when they withdraw from normal
activities and become preoccupied with their appearance.''

Unhealthy weight control methods can predispose people to eating
disorders and actually lead to weight gain, not loss. I struggled with
weight gain in my early 20s and, having failed to control my weight any
other way, I eventually resorted to fasting all day until supper. But
once I started eating, I couldn't stop and ended up gaining even more
weight. I had developed a binge-eating disorder that resolved only when
I stopped trying to diet and returned to
\href{https://www.nytimes.com/2018/03/05/well/jane-brodys-personal-secrets-to-lasting-weight-loss.html}{eating
three wholesome meals a day}, including one small snack so I didn't feel
deprived.

Dr. Bibbins-Domingo wants doctors to be proactive in asking about eating
and exercise habits when treating adolescents and young adults. ``They
should have a conversation about what these young people are eating,
when they're eating and how they're eating, and be able to give advice
about healthy eating patterns.

``Without making a value judgment about body size, they can open the
door to a discussion about eating and exercise habits,'' Dr.
Bibbins-Domingo suggested. ``The physician might ask, `What did you eat
yesterday, and where, and what do you think about the choices you made?'
or `Do you want to address weight issues?'''

The pandemic may offer one silver lining, Dr. Nagata said. ``With more
families eating meals together, it's easier for parents to monitor what
their kids are eating.'' Having family meals together is one of the
basic tenets of therapy for eating disorders, he said.

Advertisement

\protect\hyperlink{after-bottom}{Continue reading the main story}

\hypertarget{site-index}{%
\subsection{Site Index}\label{site-index}}

\hypertarget{site-information-navigation}{%
\subsection{Site Information
Navigation}\label{site-information-navigation}}

\begin{itemize}
\tightlist
\item
  \href{https://help.nytimes.com/hc/en-us/articles/115014792127-Copyright-notice}{©~2020~The
  New York Times Company}
\end{itemize}

\begin{itemize}
\tightlist
\item
  \href{https://www.nytco.com/}{NYTCo}
\item
  \href{https://help.nytimes.com/hc/en-us/articles/115015385887-Contact-Us}{Contact
  Us}
\item
  \href{https://www.nytco.com/careers/}{Work with us}
\item
  \href{https://nytmediakit.com/}{Advertise}
\item
  \href{http://www.tbrandstudio.com/}{T Brand Studio}
\item
  \href{https://www.nytimes.com/privacy/cookie-policy\#how-do-i-manage-trackers}{Your
  Ad Choices}
\item
  \href{https://www.nytimes.com/privacy}{Privacy}
\item
  \href{https://help.nytimes.com/hc/en-us/articles/115014893428-Terms-of-service}{Terms
  of Service}
\item
  \href{https://help.nytimes.com/hc/en-us/articles/115014893968-Terms-of-sale}{Terms
  of Sale}
\item
  \href{https://spiderbites.nytimes.com}{Site Map}
\item
  \href{https://help.nytimes.com/hc/en-us}{Help}
\item
  \href{https://www.nytimes.com/subscription?campaignId=37WXW}{Subscriptions}
\end{itemize}
