Sections

SEARCH

\protect\hyperlink{site-content}{Skip to
content}\protect\hyperlink{site-index}{Skip to site index}

\href{https://www.nytimes.com/section/well/family}{Family}

\href{https://myaccount.nytimes.com/auth/login?response_type=cookie\&client_id=vi}{}

\href{https://www.nytimes.com/section/todayspaper}{Today's Paper}

\href{/section/well/family}{Family}\textbar{}The Pandemic's Toll on
Children With Special Needs and Their Parents

\url{https://nyti.ms/39uuh3k}

\begin{itemize}
\item
\item
\item
\item
\item
\item
\end{itemize}

\href{https://www.nytimes.com/news-event/coronavirus?action=click\&pgtype=Article\&state=default\&region=TOP_BANNER\&context=storylines_menu}{The
Coronavirus Outbreak}

\begin{itemize}
\tightlist
\item
  live\href{https://www.nytimes.com/2020/08/01/world/coronavirus-covid-19.html?action=click\&pgtype=Article\&state=default\&region=TOP_BANNER\&context=storylines_menu}{Latest
  Updates}
\item
  \href{https://www.nytimes.com/interactive/2020/us/coronavirus-us-cases.html?action=click\&pgtype=Article\&state=default\&region=TOP_BANNER\&context=storylines_menu}{Maps
  and Cases}
\item
  \href{https://www.nytimes.com/interactive/2020/science/coronavirus-vaccine-tracker.html?action=click\&pgtype=Article\&state=default\&region=TOP_BANNER\&context=storylines_menu}{Vaccine
  Tracker}
\item
  \href{https://www.nytimes.com/interactive/2020/07/29/us/schools-reopening-coronavirus.html?action=click\&pgtype=Article\&state=default\&region=TOP_BANNER\&context=storylines_menu}{What
  School May Look Like}
\item
  \href{https://www.nytimes.com/live/2020/07/31/business/stock-market-today-coronavirus?action=click\&pgtype=Article\&state=default\&region=TOP_BANNER\&context=storylines_menu}{Economy}
\end{itemize}

Advertisement

\protect\hyperlink{after-top}{Continue reading the main story}

Supported by

\protect\hyperlink{after-sponsor}{Continue reading the main story}

The Checkup

\hypertarget{the-pandemics-toll-on-children-with-special-needs-and-their-parents}{%
\section{The Pandemic's Toll on Children With Special Needs and Their
Parents}\label{the-pandemics-toll-on-children-with-special-needs-and-their-parents}}

Missing social contacts and altered routines, disturbed sleep and eating
habits can be particularly intense for the kids with developmental
challenges.

\includegraphics{https://static01.nyt.com/images/2020/07/27/well/27klass-special/27klass-special-articleLarge.jpg?quality=75\&auto=webp\&disable=upscale}

By \href{https://www.nytimes.com/by/perri-klass-md}{Perri Klass, M.D.}

\begin{itemize}
\item
  July 27, 2020
\item
  \begin{itemize}
  \item
  \item
  \item
  \item
  \item
  \item
  \end{itemize}
\end{itemize}

Franscheska Eliza has a 9-year-old son with autism spectrum disorder,
attention deficit hyperactivity disorder, anxiety and sensory issues.
Before the pandemic, he was in a program in the Bedford, Mass., public
schools designed for children with autism.

This meant her son, Rafael, was in a special classroom, but also was a
member of the regular third-grade class, and could join them for morning
meeting or some academic subjects. He had a dedicated aide who worked
with him when he got anxious. This was his first year in the school, and
the beginning of the year was tough, but by March, things were going
well.

Then came Covid-19. At first, Ms. Eliza said, her son was excited to be
home. ``He was happy that he didn't have to get up early for school,''
she said.

It took a while to get the special education classes up and running
virtually, but when they were available, he initially did well. He was
happy to see his teachers and very interested in the computer
technology. But when the teachers started giving his mother extra work
to do with him, things got harder.

``He started becoming very aggressive,'' Ms. Eliza said. ``He didn't
want to work with me. He didn't want me to help him with the problems;
he wanted me to give him the answers.'' Problem behaviors from when he
was much younger started to come back: ``He didn't see me as a teacher
--- he would become very aggressive with me and start hitting me.''

Many children had academic and social issues being at home, but ``for
kids with developmental challenges, the challenges got exacerbated by
Covid,'' said Dr. Marilyn Augustyn, a developmental and behavioral
pediatrician at Boston Medical Center and professor of pediatrics at
Boston University School of Medicine. Some children aren't getting their
therapy, some miss being in the classroom, and some parents decided to
stop medications, or couldn't get refills.

Dr. Jenny Radesky, a developmental behavioral pediatrician who is an
assistant professor of pediatrics at Michigan Medicine C.S. Mott
Children's Hospital in Ann Arbor, said, ``some children were showing
angry outbursts, intense crying episodes, signs they were emotionally
dysregulated.''

\hypertarget{latest-updates-global-coronavirus-outbreak}{%
\section{\texorpdfstring{\href{https://www.nytimes.com/2020/08/01/world/coronavirus-covid-19.html?action=click\&pgtype=Article\&state=default\&region=MAIN_CONTENT_1\&context=storylines_live_updates}{Latest
Updates: Global Coronavirus
Outbreak}}{Latest Updates: Global Coronavirus Outbreak}}\label{latest-updates-global-coronavirus-outbreak}}

Updated 2020-08-01T18:23:51.652Z

\begin{itemize}
\tightlist
\item
  \href{https://www.nytimes.com/2020/08/01/world/coronavirus-covid-19.html?action=click\&pgtype=Article\&state=default\&region=MAIN_CONTENT_1\&context=storylines_live_updates\#link-3ac56579}{Top
  officials work to break impasse over jobless benefit.}
\item
  \href{https://www.nytimes.com/2020/08/01/world/coronavirus-covid-19.html?action=click\&pgtype=Article\&state=default\&region=MAIN_CONTENT_1\&context=storylines_live_updates\#link-8796723}{The
  virus picks up dangerous speed in the Midwest, and in areas that had
  seen success.}
\item
  \href{https://www.nytimes.com/2020/08/01/world/coronavirus-covid-19.html?action=click\&pgtype=Article\&state=default\&region=MAIN_CONTENT_1\&context=storylines_live_updates\#link-25930521}{Thousands
  in Berlin protest Germany's coronavirus measures.}
\end{itemize}

\href{https://www.nytimes.com/2020/08/01/world/coronavirus-covid-19.html?action=click\&pgtype=Article\&state=default\&region=MAIN_CONTENT_1\&context=storylines_live_updates}{See
more updates}

More live coverage:
\href{https://www.nytimes.com/live/2020/07/31/business/stock-market-today-coronavirus?action=click\&pgtype=Article\&state=default\&region=MAIN_CONTENT_1\&context=storylines_live_updates}{Markets}

{[}\textbf{\href{https://www.nytimes.com/newsletters/well-family}{\emph{Sign
up for the Well Family newsletter}}}{]}

``I'm seeing a lot of stalling of developmental progress,'' said
Rafael's pediatrician, Dr. Eileen Costello, the chief of ambulatory
pediatrics at Boston Medical Center (we are co-authors of the
forthcoming book
``\href{https://shop.aap.org/quirky-kids-2nd-edition-paperback/}{Quirky
Kids: Understanding and Supporting Your Child With Developmental
Differences}\emph{''}). ``It's hard for kids to interact in ways we know
are important.''

Rafael was fortunate --- his mother has only praise for the special
education team at his school, and he was able to continue his
occupational therapy virtually, and even his social skills group, and
meet with the therapeutic mentor he considers his best friend. But other
kids stopped showing up for morning meeting, which wasn't required. ``He
just became very sad because he could never physically interact with
these people anymore,'' his mother said.

Dr. Radesky said, ``Teachers are a huge attachment for kids, especially
when teachers really get it, really click with kids.'' There is so much
more contextual social information when children are in the classroom,
she said, and that matters even more for children who struggle with
social interactions.

It was new for her son to express his feelings verbally, Ms. Eliza said,
but he said, ``I'm lonely, nobody wants to play with me anymore, this
virus took everything from me.'' To find comfort, ``He would go to his
room and put a lot of blankets over himself, and just kind of stare off
into space.'' When his mother went in to keep him company, he would ask
her, when is this virus going to go away?

His mother was able to adjust the school expectations, and academically,
things got better, which helped diminish the aggression and anxiety. But
she still had to be there with him for everything he was doing and
learning virtually, keeping him focused, offering incentives.

``Parents are utterly burned out,'' Dr. Costello said. ``The toll this
is taking on both kids and parents cannot be underestimated.''
Sleep-wake cycles are off, she said, programs and camps are canceled ---
including the camps that are designed to help keep kids with special
needs from losing the progress they've made over the course of the
summer.

``I'm getting more requests for medication even from parents who
traditionally were reluctant to medicate their kids,'' she said.

Dr. Augustyn said that some families are ``finally feeling a tiny bit
encouraged,'' now that the school year is over. ``I feel there's a lot
of strength out there, parents know what they want, and kids, they're
totally reading their parents,'' Dr. Augustyn said. ``The parents'
response, especially for kids with developmental disabilities, is so
important.''

The general advice to parents is to build as much structure and
consistency in as possible; these tend to be children who really do
better with set times for sleep and for meals, for activities and
therapies and learning.

``We know these kids are rigid, they respond to structures,'' Dr.
Costello said. She has parents at home writing the daily schedule on a
whiteboard, she says, as teachers do in class. ``Try to get up at same
time every day, try to keep structure --- but it's really hard,
especially if people have other children.''

Many school programs do offer some kind of summer catch-up or
reinforcement, which can be a way for children to maintain a little
remote social contact. And looking forward to next year, Dr. Radesky
said, parents need to build on what they've witnessed at home. ``I'm
encouraging parents to be really vocal advocates, to contact the school
principal and the special ed team in August, say, `For my son or
daughter, here's what did and did not work well.'''

\href{https://www.nytimes.com/news-event/coronavirus?action=click\&pgtype=Article\&state=default\&region=MAIN_CONTENT_3\&context=storylines_faq}{}

\hypertarget{the-coronavirus-outbreak-}{%
\subsubsection{The Coronavirus Outbreak
›}\label{the-coronavirus-outbreak-}}

\hypertarget{frequently-asked-questions}{%
\paragraph{Frequently Asked
Questions}\label{frequently-asked-questions}}

Updated July 27, 2020

\begin{itemize}
\item ~
  \hypertarget{should-i-refinance-my-mortgage}{%
  \paragraph{Should I refinance my
  mortgage?}\label{should-i-refinance-my-mortgage}}

  \begin{itemize}
  \tightlist
  \item
    \href{https://www.nytimes.com/article/coronavirus-money-unemployment.html?action=click\&pgtype=Article\&state=default\&region=MAIN_CONTENT_3\&context=storylines_faq}{It
    could be a good idea,} because mortgage rates have
    \href{https://www.nytimes.com/2020/07/16/business/mortgage-rates-below-3-percent.html?action=click\&pgtype=Article\&state=default\&region=MAIN_CONTENT_3\&context=storylines_faq}{never
    been lower.} Refinancing requests have pushed mortgage applications
    to some of the highest levels since 2008, so be prepared to get in
    line. But defaults are also up, so if you're thinking about buying a
    home, be aware that some lenders have tightened their standards.
  \end{itemize}
\item ~
  \hypertarget{what-is-school-going-to-look-like-in-september}{%
  \paragraph{What is school going to look like in
  September?}\label{what-is-school-going-to-look-like-in-september}}

  \begin{itemize}
  \tightlist
  \item
    It is unlikely that many schools will return to a normal schedule
    this fall, requiring the grind of
    \href{https://www.nytimes.com/2020/06/05/us/coronavirus-education-lost-learning.html?action=click\&pgtype=Article\&state=default\&region=MAIN_CONTENT_3\&context=storylines_faq}{online
    learning},
    \href{https://www.nytimes.com/2020/05/29/us/coronavirus-child-care-centers.html?action=click\&pgtype=Article\&state=default\&region=MAIN_CONTENT_3\&context=storylines_faq}{makeshift
    child care} and
    \href{https://www.nytimes.com/2020/06/03/business/economy/coronavirus-working-women.html?action=click\&pgtype=Article\&state=default\&region=MAIN_CONTENT_3\&context=storylines_faq}{stunted
    workdays} to continue. California's two largest public school
    districts --- Los Angeles and San Diego --- said on July 13, that
    \href{https://www.nytimes.com/2020/07/13/us/lausd-san-diego-school-reopening.html?action=click\&pgtype=Article\&state=default\&region=MAIN_CONTENT_3\&context=storylines_faq}{instruction
    will be remote-only in the fall}, citing concerns that surging
    coronavirus infections in their areas pose too dire a risk for
    students and teachers. Together, the two districts enroll some
    825,000 students. They are the largest in the country so far to
    abandon plans for even a partial physical return to classrooms when
    they reopen in August. For other districts, the solution won't be an
    all-or-nothing approach.
    \href{https://bioethics.jhu.edu/research-and-outreach/projects/eschool-initiative/school-policy-tracker/}{Many
    systems}, including the nation's largest, New York City, are
    devising
    \href{https://www.nytimes.com/2020/06/26/us/coronavirus-schools-reopen-fall.html?action=click\&pgtype=Article\&state=default\&region=MAIN_CONTENT_3\&context=storylines_faq}{hybrid
    plans} that involve spending some days in classrooms and other days
    online. There's no national policy on this yet, so check with your
    municipal school system regularly to see what is happening in your
    community.
  \end{itemize}
\item ~
  \hypertarget{is-the-coronavirus-airborne}{%
  \paragraph{Is the coronavirus
  airborne?}\label{is-the-coronavirus-airborne}}

  \begin{itemize}
  \tightlist
  \item
    The coronavirus
    \href{https://www.nytimes.com/2020/07/04/health/239-experts-with-one-big-claim-the-coronavirus-is-airborne.html?action=click\&pgtype=Article\&state=default\&region=MAIN_CONTENT_3\&context=storylines_faq}{can
    stay aloft for hours in tiny droplets in stagnant air}, infecting
    people as they inhale, mounting scientific evidence suggests. This
    risk is highest in crowded indoor spaces with poor ventilation, and
    may help explain super-spreading events reported in meatpacking
    plants, churches and restaurants.
    \href{https://www.nytimes.com/2020/07/06/health/coronavirus-airborne-aerosols.html?action=click\&pgtype=Article\&state=default\&region=MAIN_CONTENT_3\&context=storylines_faq}{It's
    unclear how often the virus is spread} via these tiny droplets, or
    aerosols, compared with larger droplets that are expelled when a
    sick person coughs or sneezes, or transmitted through contact with
    contaminated surfaces, said Linsey Marr, an aerosol expert at
    Virginia Tech. Aerosols are released even when a person without
    symptoms exhales, talks or sings, according to Dr. Marr and more
    than 200 other experts, who
    \href{https://academic.oup.com/cid/article/doi/10.1093/cid/ciaa939/5867798}{have
    outlined the evidence in an open letter to the World Health
    Organization}.
  \end{itemize}
\item ~
  \hypertarget{what-are-the-symptoms-of-coronavirus}{%
  \paragraph{What are the symptoms of
  coronavirus?}\label{what-are-the-symptoms-of-coronavirus}}

  \begin{itemize}
  \tightlist
  \item
    Common symptoms
    \href{https://www.nytimes.com/article/symptoms-coronavirus.html?action=click\&pgtype=Article\&state=default\&region=MAIN_CONTENT_3\&context=storylines_faq}{include
    fever, a dry cough, fatigue and difficulty breathing or shortness of
    breath.} Some of these symptoms overlap with those of the flu,
    making detection difficult, but runny noses and stuffy sinuses are
    less common.
    \href{https://www.nytimes.com/2020/04/27/health/coronavirus-symptoms-cdc.html?action=click\&pgtype=Article\&state=default\&region=MAIN_CONTENT_3\&context=storylines_faq}{The
    C.D.C. has also} added chills, muscle pain, sore throat, headache
    and a new loss of the sense of taste or smell as symptoms to look
    out for. Most people fall ill five to seven days after exposure, but
    symptoms may appear in as few as two days or as many as 14 days.
  \end{itemize}
\item ~
  \hypertarget{does-asymptomatic-transmission-of-covid-19-happen}{%
  \paragraph{Does asymptomatic transmission of Covid-19
  happen?}\label{does-asymptomatic-transmission-of-covid-19-happen}}

  \begin{itemize}
  \tightlist
  \item
    So far, the evidence seems to show it does. A widely cited
    \href{https://www.nature.com/articles/s41591-020-0869-5}{paper}
    published in April suggests that people are most infectious about
    two days before the onset of coronavirus symptoms and estimated that
    44 percent of new infections were a result of transmission from
    people who were not yet showing symptoms. Recently, a top expert at
    the World Health Organization stated that transmission of the
    coronavirus by people who did not have symptoms was ``very rare,''
    \href{https://www.nytimes.com/2020/06/09/world/coronavirus-updates.html?action=click\&pgtype=Article\&state=default\&region=MAIN_CONTENT_3\&context=storylines_faq\#link-1f302e21}{but
    she later walked back that statement.}
  \end{itemize}
\end{itemize}

Ms. Eliza said that she and her son's therapist have discussed his
depression: ``We've been worried about the time he's really down ---
he's usually a pretty happy-go-lucky kid, and there are times he's
really quiet, doesn't speak.''

For many children, the emotional issues and anxieties manifest in
sleeping problems and eating problems. Rafael, who generally has very
restrictive eating patterns, lost seven pounds during the pandemic, and
eventually Dr. Costello started him on a medication to stimulate his
appetite. His mother also sees him looking for the sensory stimulation
that comforts him --- the blankets, even when it's hot, or the TV turned
up very loud (his hearing is normal).

Dr. Costello said, ``Kids tell me in the office, I miss my friends, even
kids who are quirky, for whom going to school is being around other
kids, whether they're their friends or not.''

Parents can look for online social opportunities, which may be available
through parent advocacy and support groups, if not through schools. Even
socially distanced contacts may help children feel a little more
connected.

Recently, Ms. Eliza said, Rafael met some other children in the
neighborhood, with parents supervising social distancing. ``Now he says
hi to them. It's been really nice to have the kids say hi back --- that
makes him happy.''

``There are some children who, just a little bit of social time online
seems to be meeting their needs right now,'' said Dr. Mark Bertin, a
developmental pediatrician in Pleasantville, N.Y., who
\href{https://www.nytimes.com/2020/05/13/well/family/coronavirus-shutdowns-children-special-needs-adhd-autism.html?searchResultPosition=1}{wrote
in May} about the demands that the pandemic put on parents of children
with special needs. ``Some kids with more significant disabilities,
their only social life is going to school, and that's a very hard gap to
bridge.''

There are families, he said, ``whose lives are swamped and the social
emotional side of this is overwhelming, and kids are really struggling
being out of school.'' On the other hand, ``There are some kids who must
have found school quite stressful and seem quite content to be at home
learning at their own pace.''

Dr. Radesky said in an email, ``The fact that virus cases are rising at
a time when we hoped we'd be planning for school reopenings feels like a
complete disaster to many families raising kids with special needs. Most
simply can't access the special education supports they deserve by law
unless they can be taught and receive therapies in person. So parents
are put in the impossible position of choosing their child's
developmental progress or their health, and the health of their
teachers.''

``It's difficult to manage a child with a disability full time on your
own,'' Dr. Costello said. ``This is exposing all the cracks, the stress
of raising a child with a disability.''

Advertisement

\protect\hyperlink{after-bottom}{Continue reading the main story}

\hypertarget{site-index}{%
\subsection{Site Index}\label{site-index}}

\hypertarget{site-information-navigation}{%
\subsection{Site Information
Navigation}\label{site-information-navigation}}

\begin{itemize}
\tightlist
\item
  \href{https://help.nytimes.com/hc/en-us/articles/115014792127-Copyright-notice}{©~2020~The
  New York Times Company}
\end{itemize}

\begin{itemize}
\tightlist
\item
  \href{https://www.nytco.com/}{NYTCo}
\item
  \href{https://help.nytimes.com/hc/en-us/articles/115015385887-Contact-Us}{Contact
  Us}
\item
  \href{https://www.nytco.com/careers/}{Work with us}
\item
  \href{https://nytmediakit.com/}{Advertise}
\item
  \href{http://www.tbrandstudio.com/}{T Brand Studio}
\item
  \href{https://www.nytimes.com/privacy/cookie-policy\#how-do-i-manage-trackers}{Your
  Ad Choices}
\item
  \href{https://www.nytimes.com/privacy}{Privacy}
\item
  \href{https://help.nytimes.com/hc/en-us/articles/115014893428-Terms-of-service}{Terms
  of Service}
\item
  \href{https://help.nytimes.com/hc/en-us/articles/115014893968-Terms-of-sale}{Terms
  of Sale}
\item
  \href{https://spiderbites.nytimes.com}{Site Map}
\item
  \href{https://help.nytimes.com/hc/en-us}{Help}
\item
  \href{https://www.nytimes.com/subscription?campaignId=37WXW}{Subscriptions}
\end{itemize}
