Sections

SEARCH

\protect\hyperlink{site-content}{Skip to
content}\protect\hyperlink{site-index}{Skip to site index}

\href{https://www.nytimes.com/section/world/europe}{Europe}

\href{https://myaccount.nytimes.com/auth/login?response_type=cookie\&client_id=vi}{}

\href{https://www.nytimes.com/section/todayspaper}{Today's Paper}

\href{/section/world/europe}{Europe}\textbar{}Of Wine, Hand Sanitizer
and Heartbreak

\url{https://nyti.ms/32Z5WBF}

\begin{itemize}
\item
\item
\item
\item
\item
\item
\end{itemize}

\href{https://www.nytimes.com/news-event/coronavirus?action=click\&pgtype=Article\&state=default\&region=TOP_BANNER\&context=storylines_menu}{The
Coronavirus Outbreak}

\begin{itemize}
\tightlist
\item
  live\href{https://www.nytimes.com/2020/08/04/world/coronavirus-cases.html?action=click\&pgtype=Article\&state=default\&region=TOP_BANNER\&context=storylines_menu}{Latest
  Updates}
\item
  \href{https://www.nytimes.com/interactive/2020/us/coronavirus-us-cases.html?action=click\&pgtype=Article\&state=default\&region=TOP_BANNER\&context=storylines_menu}{Maps
  and Cases}
\item
  \href{https://www.nytimes.com/interactive/2020/science/coronavirus-vaccine-tracker.html?action=click\&pgtype=Article\&state=default\&region=TOP_BANNER\&context=storylines_menu}{Vaccine
  Tracker}
\item
  \href{https://www.nytimes.com/2020/08/02/us/covid-college-reopening.html?action=click\&pgtype=Article\&state=default\&region=TOP_BANNER\&context=storylines_menu}{College
  Reopening}
\item
  \href{https://www.nytimes.com/live/2020/08/04/business/stock-market-today-coronavirus?action=click\&pgtype=Article\&state=default\&region=TOP_BANNER\&context=storylines_menu}{Economy}
\end{itemize}

Advertisement

\protect\hyperlink{after-top}{Continue reading the main story}

Supported by

\protect\hyperlink{after-sponsor}{Continue reading the main story}

France Dispatch

\hypertarget{of-wine-hand-sanitizer-and-heartbreak}{%
\section{Of Wine, Hand Sanitizer and
Heartbreak}\label{of-wine-hand-sanitizer-and-heartbreak}}

Between the coronavirus and the Trump tariffs, the French wine market
has collapsed. So winemakers are --- sadly --- sending their excess
product off to another life as hand sanitizer.

\includegraphics{https://static01.nyt.com/images/2020/07/24/world/00france-wine1/merlin_174871056_ae254e73-15d3-440c-997b-65cde45a173f-articleLarge.jpg?quality=75\&auto=webp\&disable=upscale}

\href{https://www.nytimes.com/by/adam-nossiter}{\includegraphics{https://static01.nyt.com/images/2018/10/15/multimedia/author-adam-nossiter/author-adam-nossiter-thumbLarge.png}}

By \href{https://www.nytimes.com/by/adam-nossiter}{Adam Nossiter}

\begin{itemize}
\item
  July 27, 2020
\item
  \begin{itemize}
  \item
  \item
  \item
  \item
  \item
  \item
  \end{itemize}
\end{itemize}

\href{https://www.nytimes.com/es/2020/07/28/espanol/mundo/vino-blanco-alsacia-coronavirus.html}{Leer
en español}

HUNAWIHR, France --- The tanker-truck pulled up and it was time to let
it go. The decision to send the wine to the distillery had been made
weeks ago. It still hurt. Soon the wine would be sanitizing hand gel.

``We've got to load it up now,'' said Jérôme Mader, a 38-year-old
winemaker, muttering to himself. ``OK, I am not even going to think
about it anymore,'' he said quietly. ``It's over.''

Head down, he dragged the hoses out through his shed, affixed them to
the truck's valves with the help of the driver, walked up to his cool
cellar, and turned on the pumps. The wine --- good Alsace white wine,
drinkable wine --- coursed through the hoses and into the truck's belly.
Its fate didn't bear thinking about.

Across the emerald Alsace wine country, now carpeted in deep-green vines
--- and across France's other wine regions as well --- thousands of
winemakers, famous and obscure, are facing similar moments of
heartbreak.

The economic crisis brought on by the coronavirus, combined with the
Trump administration's 25 percent tax on French wines in the trade war
dispute with Europe, has collapsed the wine market.

Mr. Mader, whose high-quality Rieslings and Gewürztraminers are sent to
fancy restaurants and shops on both sides of the Atlantic, has lost half
his sales since December.

``Covid is a catastrophe for us,'' he said.

\includegraphics{https://static01.nyt.com/images/2020/07/24/world/00france-wine2/merlin_174871080_804e4be1-60d0-478c-8585-aa8d55bcafae-articleLarge.jpg?quality=75\&auto=webp\&disable=upscale}

And so some of the succulent and subtle white wine for which this region
is famous, nurtured on the stony, sunbathed Alsace slopes, will wind up
as hand sanitizer.

Like other winemakers, Mr. Mader has no room in his cellar to stock
unsold wine. ``We can't keep stocking what we haven't sold,'' he said.

The precocious 2020 harvest, blessed by abundant sunshine, is barely a
month away. The wine vats must be emptied for the new production. The
distillery, for modest compensation, is the only option.

\hypertarget{latest-updates-global-coronavirus-outbreak}{%
\section{\texorpdfstring{\href{https://www.nytimes.com/2020/08/04/world/coronavirus-cases.html?action=click\&pgtype=Article\&state=default\&region=MAIN_CONTENT_1\&context=storylines_live_updates}{Latest
Updates: Global Coronavirus
Outbreak}}{Latest Updates: Global Coronavirus Outbreak}}\label{latest-updates-global-coronavirus-outbreak}}

Updated 2020-08-04T18:14:55.559Z

\begin{itemize}
\tightlist
\item
  \href{https://www.nytimes.com/2020/08/04/world/coronavirus-cases.html?action=click\&pgtype=Article\&state=default\&region=MAIN_CONTENT_1\&context=storylines_live_updates\#link-4d1eafa8}{N.Y.C.'s
  health commissioner resigns after clashing with the mayor over the
  virus.}
\item
  \href{https://www.nytimes.com/2020/08/04/world/coronavirus-cases.html?action=click\&pgtype=Article\&state=default\&region=MAIN_CONTENT_1\&context=storylines_live_updates\#link-18bf040e}{Public
  and private schools are dividing over in-person instruction in
  Maryland and elsewhere.}
\item
  \href{https://www.nytimes.com/2020/08/04/world/coronavirus-cases.html?action=click\&pgtype=Article\&state=default\&region=MAIN_CONTENT_1\&context=storylines_live_updates\#link-6b644638}{`Long
  days, long nights': Washington prepares for a prolonged fight over
  virus relief.}
\end{itemize}

\href{https://www.nytimes.com/2020/08/04/world/coronavirus-cases.html?action=click\&pgtype=Article\&state=default\&region=MAIN_CONTENT_1\&context=storylines_live_updates}{See
more updates}

More live coverage:
\href{https://www.nytimes.com/live/2020/08/04/business/stock-market-today-coronavirus?action=click\&pgtype=Article\&state=default\&region=MAIN_CONTENT_1\&context=storylines_live_updates}{Markets}

The driver from the distillery had been making the rounds of winemakers
all morning. ``Some of them are taking this quite badly, because this
wine has commercial value,'' the driver, Lucas Neret, noted dryly.

``We're producing more than we can sell,'' said Thibaut Specht, a
winemaker in nearby Mittelwihr. ``We have no choice.''

Image

Vineyards near Reichsfeld, France.Credit...Dmitry Kostyukov for The New
York Times

Marion Borès's family business, Domaine Borès, in Reichsfeld, is sending
30 percent of its production --- 19,000 liters. ``It's like you are
saying goodbye to somebody who is very dear to you,'' she said.

``This is not exactly the destination we had in mind, when we made this
wine,'' the 27-year-old winemaker added.

The old wine is ending up in the towering steel silos of the nearby
Romann distillery, where it will be boiled down to alcohol.

In Alsace alone, over six million liters of wine, or about 1.5 million
gallons, will end up like this. Mr. Mader is sending 15 percent of his
production, wine he calls ``Edelzwicker,'' or ``noble blend'' in
Alsatian dialect. Usually sold wholesale, ``it's still pretty good,''
Mr. Mader said.

At the distillery, the odor of boiled-down wine, like the essence of a
rich beef burgundy sauce, hung heavy over the establishment on a warm
morning this week.

Image

Erwin Brouard at the Romann distillery in Sigolsheim,
France.Credit...Dmitry Kostyukov for The New York Times

``We're continuously distilling,'' said Erwin Brouard, the company's
director. ``It's something that's very sad for the winemakers. Their
stocks are too big. They've got to make space. And the harvest is early
this year.''

The French government, anxious to protect its precious wine heritage, is
subsidizing the operation, compensating the some 5,000 winemakers who
have signed up so far at a fraction of the wine's value, less than \$1
per liter, in what the government calls Crisis Distillation.

``My cellar is bursting, ''said Guillaume Klauss, who owns a nearby
winery. ``If I don't send it off, I don't eat. Clearly this is tearing
me up. It's three years of work, and we're not even paid properly.''

Alsace is having to resort to Crisis Distillation for the first time in
its history although it is not unknown in other wine regions. The last
time this happened was in 2009, after the financial collapse.

Image

The tanker-truck making its way in Hunawihr.Credit...Dmitry Kostyukov
for The New York Times

``A very big majority have been battered by this crisis,'' said Francis
Backert, head of the Independent Winemakers Association of Alsace.
``These people are really hurting.'' he said.

\href{https://www.nytimes.com/news-event/coronavirus?action=click\&pgtype=Article\&state=default\&region=MAIN_CONTENT_3\&context=storylines_faq}{}

\hypertarget{the-coronavirus-outbreak-}{%
\subsubsection{The Coronavirus Outbreak
›}\label{the-coronavirus-outbreak-}}

\hypertarget{frequently-asked-questions}{%
\paragraph{Frequently Asked
Questions}\label{frequently-asked-questions}}

Updated August 4, 2020

\begin{itemize}
\item ~
  \hypertarget{i-have-antibodies-am-i-now-immune}{%
  \paragraph{I have antibodies. Am I now
  immune?}\label{i-have-antibodies-am-i-now-immune}}

  \begin{itemize}
  \tightlist
  \item
    As of right
    now,\href{https://www.nytimes.com/2020/07/22/health/covid-antibodies-herd-immunity.html?action=click\&pgtype=Article\&state=default\&region=MAIN_CONTENT_3\&context=storylines_faq}{that
    seems likely, for at least several months.} There have been
    frightening accounts of people suffering what seems to be a second
    bout of Covid-19. But experts say these patients may have a
    drawn-out course of infection, with the virus taking a slow toll
    weeks to months after initial exposure. People infected with the
    coronavirus typically
    \href{https://www.nature.com/articles/s41586-020-2456-9}{produce}
    immune molecules called antibodies, which are
    \href{https://www.nytimes.com/2020/05/07/health/coronavirus-antibody-prevalence.html?action=click\&pgtype=Article\&state=default\&region=MAIN_CONTENT_3\&context=storylines_faq}{protective
    proteins made in response to an
    infection}\href{https://www.nytimes.com/2020/05/07/health/coronavirus-antibody-prevalence.html?action=click\&pgtype=Article\&state=default\&region=MAIN_CONTENT_3\&context=storylines_faq}{.
    These antibodies may} last in the body
    \href{https://www.nature.com/articles/s41591-020-0965-6}{only two to
    three months}, which may seem worrisome, but that's perfectly normal
    after an acute infection subsides, said Dr. Michael Mina, an
    immunologist at Harvard University. It may be possible to get the
    coronavirus again, but it's highly unlikely that it would be
    possible in a short window of time from initial infection or make
    people sicker the second time.
  \end{itemize}
\item ~
  \hypertarget{im-a-small-business-owner-can-i-get-relief}{%
  \paragraph{I'm a small-business owner. Can I get
  relief?}\label{im-a-small-business-owner-can-i-get-relief}}

  \begin{itemize}
  \tightlist
  \item
    The
    \href{https://www.nytimes.com/article/small-business-loans-stimulus-grants-freelancers-coronavirus.html?action=click\&pgtype=Article\&state=default\&region=MAIN_CONTENT_3\&context=storylines_faq}{stimulus
    bills enacted in March} offer help for the millions of American
    small businesses. Those eligible for aid are businesses and
    nonprofit organizations with fewer than 500 workers, including sole
    proprietorships, independent contractors and freelancers. Some
    larger companies in some industries are also eligible. The help
    being offered, which is being managed by the Small Business
    Administration, includes the Paycheck Protection Program and the
    Economic Injury Disaster Loan program. But lots of folks have
    \href{https://www.nytimes.com/interactive/2020/05/07/business/small-business-loans-coronavirus.html?action=click\&pgtype=Article\&state=default\&region=MAIN_CONTENT_3\&context=storylines_faq}{not
    yet seen payouts.} Even those who have received help are confused:
    The rules are draconian, and some are stuck sitting on
    \href{https://www.nytimes.com/2020/05/02/business/economy/loans-coronavirus-small-business.html?action=click\&pgtype=Article\&state=default\&region=MAIN_CONTENT_3\&context=storylines_faq}{money
    they don't know how to use.} Many small-business owners are getting
    less than they expected or
    \href{https://www.nytimes.com/2020/06/10/business/Small-business-loans-ppp.html?action=click\&pgtype=Article\&state=default\&region=MAIN_CONTENT_3\&context=storylines_faq}{not
    hearing anything at all.}
  \end{itemize}
\item ~
  \hypertarget{what-are-my-rights-if-i-am-worried-about-going-back-to-work}{%
  \paragraph{What are my rights if I am worried about going back to
  work?}\label{what-are-my-rights-if-i-am-worried-about-going-back-to-work}}

  \begin{itemize}
  \tightlist
  \item
    Employers have to provide
    \href{https://www.osha.gov/SLTC/covid-19/standards.html}{a safe
    workplace} with policies that protect everyone equally.
    \href{https://www.nytimes.com/article/coronavirus-money-unemployment.html?action=click\&pgtype=Article\&state=default\&region=MAIN_CONTENT_3\&context=storylines_faq}{And
    if one of your co-workers tests positive for the coronavirus, the
    C.D.C.} has said that
    \href{https://www.cdc.gov/coronavirus/2019-ncov/community/guidance-business-response.html}{employers
    should tell their employees} -\/- without giving you the sick
    employee's name -\/- that they may have been exposed to the virus.
  \end{itemize}
\item ~
  \hypertarget{should-i-refinance-my-mortgage}{%
  \paragraph{Should I refinance my
  mortgage?}\label{should-i-refinance-my-mortgage}}

  \begin{itemize}
  \tightlist
  \item
    \href{https://www.nytimes.com/article/coronavirus-money-unemployment.html?action=click\&pgtype=Article\&state=default\&region=MAIN_CONTENT_3\&context=storylines_faq}{It
    could be a good idea,} because mortgage rates have
    \href{https://www.nytimes.com/2020/07/16/business/mortgage-rates-below-3-percent.html?action=click\&pgtype=Article\&state=default\&region=MAIN_CONTENT_3\&context=storylines_faq}{never
    been lower.} Refinancing requests have pushed mortgage applications
    to some of the highest levels since 2008, so be prepared to get in
    line. But defaults are also up, so if you're thinking about buying a
    home, be aware that some lenders have tightened their standards.
  \end{itemize}
\item ~
  \hypertarget{what-is-school-going-to-look-like-in-september}{%
  \paragraph{What is school going to look like in
  September?}\label{what-is-school-going-to-look-like-in-september}}

  \begin{itemize}
  \tightlist
  \item
    It is unlikely that many schools will return to a normal schedule
    this fall, requiring the grind of
    \href{https://www.nytimes.com/2020/06/05/us/coronavirus-education-lost-learning.html?action=click\&pgtype=Article\&state=default\&region=MAIN_CONTENT_3\&context=storylines_faq}{online
    learning},
    \href{https://www.nytimes.com/2020/05/29/us/coronavirus-child-care-centers.html?action=click\&pgtype=Article\&state=default\&region=MAIN_CONTENT_3\&context=storylines_faq}{makeshift
    child care} and
    \href{https://www.nytimes.com/2020/06/03/business/economy/coronavirus-working-women.html?action=click\&pgtype=Article\&state=default\&region=MAIN_CONTENT_3\&context=storylines_faq}{stunted
    workdays} to continue. California's two largest public school
    districts --- Los Angeles and San Diego --- said on July 13, that
    \href{https://www.nytimes.com/2020/07/13/us/lausd-san-diego-school-reopening.html?action=click\&pgtype=Article\&state=default\&region=MAIN_CONTENT_3\&context=storylines_faq}{instruction
    will be remote-only in the fall}, citing concerns that surging
    coronavirus infections in their areas pose too dire a risk for
    students and teachers. Together, the two districts enroll some
    825,000 students. They are the largest in the country so far to
    abandon plans for even a partial physical return to classrooms when
    they reopen in August. For other districts, the solution won't be an
    all-or-nothing approach.
    \href{https://bioethics.jhu.edu/research-and-outreach/projects/eschool-initiative/school-policy-tracker/}{Many
    systems}, including the nation's largest, New York City, are
    devising
    \href{https://www.nytimes.com/2020/06/26/us/coronavirus-schools-reopen-fall.html?action=click\&pgtype=Article\&state=default\&region=MAIN_CONTENT_3\&context=storylines_faq}{hybrid
    plans} that involve spending some days in classrooms and other days
    online. There's no national policy on this yet, so check with your
    municipal school system regularly to see what is happening in your
    community.
  \end{itemize}
\end{itemize}

``All the outlets are blocked,'' he added. ``Export is blocked. Trump,
Covid. There's very little going on outside France. The American market,
blocked.''

Wholesale wine traders are facing losses of 70 percent, he said.

But the monetary losses are one thing. There is also the psychological
blow.

``Look, these people have a great deal of circumspection, and shame,''
Mr. Backert said. ``They just don't want to talk about it. Obviously,
this is breaking their heart.''

Some winemakers in the region refused to be interviewed on the subject.

The relationship to their vines, and what is produced from them, is
personal as much as financial. Many live in modest houses, carrying on a
family trade that often goes back centuries. The date carved above the
original Borès cellar is 1723.

On the sun-battered schist and sandstone slopes above Reichsfeld, Ms.
Borès patrolled vines she has worked in since the age of 10, plucking
out dead leaves and pulling shriveled grapes. Her touch was light.

Image

``It's like you are saying goodbye to somebody who is very dear to
you,'' said Marion Borès, a winemaker.Credit...Dmitry Kostyukov for The
New York Times

``These are vines that we fuss over the whole year round,'' said her
mother, Marie-Claire. ``We do everything by hand. And now, this.
Terrible.''

Climbing the steep slope, Marion said, ``We played in these vines,''
adding that she takes part in the harvest herself.

``The schist is magical,'' she said. ``It is what makes the wine
dynamic. There are moments when you are really glad to be alone in these
vines.''

In his career, Mr. Mader has won prizes, and has faced the opposite
problem he has today --- not having enough wine to satisfy demand.

``To have imagined, a few years ago, that a truck would be passing by
one day \ldots{} it's unimaginable,'' he said, his voice trailing off.

For days he put off making a decision about the distillery.

``I hesitated,'' he said. ``I thought we would get over it. I waited
until the last day to decide. I always think the next day will be
better.''

But the decision couldn't be postponed; the government was pressing with
its sign-up deadline.

Afterward, to console himself and colleagues, he said, ``I called up a
friend and we drank a couple of bottles.''

``As long as the wine is good, there is always hope,'' he added.

Orders have recently picked up, a little. Besides, ``the grapes this
year are truly magnificent,'' he said.

Image

For winemakers in Alsace, it's a time of heartbreak. A view of Colmar,
in Alsace.Credit...Dmitry Kostyukov for The New York Times

Advertisement

\protect\hyperlink{after-bottom}{Continue reading the main story}

\hypertarget{site-index}{%
\subsection{Site Index}\label{site-index}}

\hypertarget{site-information-navigation}{%
\subsection{Site Information
Navigation}\label{site-information-navigation}}

\begin{itemize}
\tightlist
\item
  \href{https://help.nytimes.com/hc/en-us/articles/115014792127-Copyright-notice}{©~2020~The
  New York Times Company}
\end{itemize}

\begin{itemize}
\tightlist
\item
  \href{https://www.nytco.com/}{NYTCo}
\item
  \href{https://help.nytimes.com/hc/en-us/articles/115015385887-Contact-Us}{Contact
  Us}
\item
  \href{https://www.nytco.com/careers/}{Work with us}
\item
  \href{https://nytmediakit.com/}{Advertise}
\item
  \href{http://www.tbrandstudio.com/}{T Brand Studio}
\item
  \href{https://www.nytimes.com/privacy/cookie-policy\#how-do-i-manage-trackers}{Your
  Ad Choices}
\item
  \href{https://www.nytimes.com/privacy}{Privacy}
\item
  \href{https://help.nytimes.com/hc/en-us/articles/115014893428-Terms-of-service}{Terms
  of Service}
\item
  \href{https://help.nytimes.com/hc/en-us/articles/115014893968-Terms-of-sale}{Terms
  of Sale}
\item
  \href{https://spiderbites.nytimes.com}{Site Map}
\item
  \href{https://help.nytimes.com/hc/en-us}{Help}
\item
  \href{https://www.nytimes.com/subscription?campaignId=37WXW}{Subscriptions}
\end{itemize}
