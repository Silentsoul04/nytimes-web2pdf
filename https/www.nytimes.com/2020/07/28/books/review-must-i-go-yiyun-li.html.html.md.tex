Sections

SEARCH

\protect\hyperlink{site-content}{Skip to
content}\protect\hyperlink{site-index}{Skip to site index}

\href{https://www.nytimes.com/section/books}{Books}

\href{https://myaccount.nytimes.com/auth/login?response_type=cookie\&client_id=vi}{}

\href{https://www.nytimes.com/section/todayspaper}{Today's Paper}

\href{/section/books}{Books}\textbar{}In Yiyun Li's Latest, a Grieving
Mother Desperately Clings to Memory

\url{https://nyti.ms/3hF52hu}

\begin{itemize}
\item
\item
\item
\item
\item
\end{itemize}

Advertisement

\protect\hyperlink{after-top}{Continue reading the main story}

Supported by

\protect\hyperlink{after-sponsor}{Continue reading the main story}

\href{/column/books-of-the-times}{Books of The Times}

\hypertarget{in-yiyun-lis-latest-a-grieving-mother-desperately-clings-to-memory}{%
\section{In Yiyun Li's Latest, a Grieving Mother Desperately Clings to
Memory}\label{in-yiyun-lis-latest-a-grieving-mother-desperately-clings-to-memory}}

By \href{https://www.nytimes.com/by/parul-sehgal}{Parul Sehgal}

\begin{itemize}
\item
  July 28, 2020
\item
  \begin{itemize}
  \item
  \item
  \item
  \item
  \item
  \end{itemize}
\end{itemize}

\includegraphics{https://static01.nyt.com/images/2020/07/29/books/28BOOKLI1/28BOOKLI1-articleLarge.png?quality=75\&auto=webp\&disable=upscale}

Buy Book ▾

\begin{itemize}
\tightlist
\item
  \href{https://www.amazon.com/gp/search?index=books\&tag=NYTBSREV-20\&field-keywords=Must+I+Go+Yiyun+Li}{Amazon}
\item
  \href{https://du-gae-books-dot-nyt-du-prd.appspot.com/buy?title=Must+I+Go\&author=Yiyun+Li}{Apple
  Books}
\item
  \href{https://www.anrdoezrs.net/click-7990613-11819508?url=https\%3A\%2F\%2Fwww.barnesandnoble.com\%2Fw\%2F\%3Fean\%3D9780399589126}{Barnes
  and Noble}
\item
  \href{https://www.anrdoezrs.net/click-7990613-35140?url=https\%3A\%2F\%2Fwww.booksamillion.com\%2Fp\%2FMust\%2BI\%2BGo\%2FYiyun\%2BLi\%2F9780399589126}{Books-A-Million}
\item
  \href{https://bookshop.org/a/3546/9780399589126}{Bookshop}
\item
  \href{https://www.indiebound.org/book/9780399589126?aff=NYT}{Indiebound}
\end{itemize}

When you purchase an independently reviewed book through our site, we
earn an affiliate commission.

Yiyun Li's new novel, ``Must I Go,'' was for many years a book
interrupted.

The award-winning author has been acclaimed for her haunting portrayals
of the Communist China of her youth. Her latest, set in America, follows
Lilia, a retiree who is annotating the posthumously published diaries of
a former lover. (He gets a lot wrong, as you might imagine.) But as Li
began delving into Lilia's past --- how, at 44, she lost a child to
suicide --- she abruptly and inexplicably abandoned the project.

At the time, Li was 44 herself. Shortly after, in an appalling
coincidence, her own child --- her 16-year-old son, Vincent --- killed
himself, in 2017.

``Was I writing to prepare myself?'' Li has wondered aloud in
interviews. At first, she did not return to ``Must I Go.'' She not only
shelved the novel but, in dramatic fashion, dismantled her own style.
For her, the pleasure in writing had always come from precision and
revision --- unsurprising perhaps, for a mathematical prodigy who had
once trained as an immunologist. In the months following her son's
death, Li wrote
\href{https://www.nytimes.com/2019/01/22/books/review-where-reasons-end-yiyun-li.html}{``Where
Reasons End''} in one furious draft.

That novel is a series of ragged, recursive conversations between a
mother and the ghost of her dead son --- shockingly autobiographical for
a writer so famously leery of self-disclosure. She could scarcely abide
the pronoun ``I,'' she wrote in her memoir,
\href{https://www.nytimes.com/2017/02/15/books/review/dear-friend-from-my-life-i-write-to-you-in-your-life-yiyun-li.html}{``Dear
Friend, From My Life I Write to You in Your Life,''} itself a
masterpiece of reticence, with its oblique depictions of the writer's
own history of suicide attempts and hospitalizations.

No word, no notion in Li's work or life seems as necessary or as prized
as privacy. It even governed her choice of language. She mastered and
adopted English in her 20s, drawn to the idea of working in a tongue she
could imagine was hers alone, untainted by personal history or the
Communist Party's degradation of language. ``What marks our era,'' a
character in her first novel, ``The Vagrants,'' says, ``is the moaning
of our bones crushed beneath the weight of empty words.''

Now, Li has finally published ``Must I Go,'' a book that was scarily
prescient. What does Lilia, that other grieving mother, tell us? ``I
haven't stopped arguing with Lucy for 37 years,'' she writes of her dead
daughter. ``Everything in my life is a part of that long argument.''

Image

Yiyun Li, whose new novel is ``Must I Go.''Credit...Agence Opale-Alamy

There's an echo of the epigraph from ``Where Reasons End,'' a line from
Elizabeth Bishop: ``Argue argue argue with me.'' The books bleed into
each other. Their titles could run together in a single despairing
sentence, a mission statement of sorts: \emph{Where reasons end, must I
go}. Where the previous book is stripped down, a bundle of exposed
nerves, ``Must I Go'' is upholstered with the nested narratives,
intricate back stories and details of a historical novel. For all their
differences, their concerns are knotted together. They reach into realms
that the author and characters feel are unspeakable: What is this
perplexing obligation to endure? What are the limits and consolations of
language? What is the self that can survive the death of a beloved
child? Do we still call that existence life?

They are among the loneliest books I've ever read --- if they are merely
books. At times they seem more like ruins; the chipped sentences and
broken structures let you see all the devastated, discarded certainties.

``I am writing with my burnt hand about the nature of fire,'' the
novelist Ingeborg Bachmann once wrote. With Li too, there is that
feeling --- her books are documents of survival but they bear wounds as
a body might.

``Must I Go'' is less immediately autobiographical, although there are
little hints scattered throughout: The echoes of Li's own name in
``Lilia,'' to say nothing of the character's unfortunate, heavy-handed
last name: ``Imbody.'' Her lover Roland's middle name belongs to Li's
son. They are small sparks of continuity and connection in an expansive
plot --- or so it first seems.

Lilia has led a full life. She married three times and outlived all
three husbands. She bore five children, buried one (Lucy) and raised
Lucy's child, Katherine, as her own. When we meet Lilia, she is readying
a version of Roland's diaries to present to Katherine (Roland was Lucy's
father), accompanied by a crippling amount of life advice. Much of the
action of the book is just this: Lilia repetitively, even compulsively
explaining to Katherine, and by extension the reader, her philosophy of
survival, a harsh and doughty stoicism.

Little happens, but I've always found the openness, the near
shapelessness of Li's work to be part of its beauty. Her characters are
never coerced; they are patiently observed, they are allowed to live,
allowed to disappoint.

The core of ``Must I Go'' is the same as that of ``Where Reasons End'':
Again, we see a mother desperately trying to prolong her conversation
with the dead, to keep her child close. The new book is bloated and
unwieldy, however; it lacks the blunt power of its predecessor, which
was stark and swift, flensed of artifice. There is a strange feeling of
watching Li retreating into a form and narrative structure she has
outgrown and outpaced.

There is an image that has always haunted me from Li's early work. In
the short story ``Kindness,'' a young girl buys a small chick from the
market. It gets sick and dies. The girl cannot accept its death. She
goes to the kitchen, cracks open an egg and drains it. She tries to
squash the dead chick into the empty shell. \emph{Begin again, begin
again}, I imagine her thinking. \emph{Let's start again}. So too in
these narratives, we feel these desperate resurrections, this attempt to
return to the beginning. \emph{What did I miss with you?} \emph{Where
did I go wrong?} the mothers wonder. It's worth noting that the title of
this new novel is taken from Roland's diaries; it's the one question the
mothers don't ask. They sermonize and theorize, lecture and filibuster.
\emph{Don't go just yet; let's start again.} Lilia talks and talks ---
to Lucy, to Katherine. And then, one day, as an old woman, she hears
someone suddenly mention her daughter's name --- ``Lucy? Isn't she
dead?'' She opens her mouth and no words come.

Advertisement

\protect\hyperlink{after-bottom}{Continue reading the main story}

\hypertarget{site-index}{%
\subsection{Site Index}\label{site-index}}

\hypertarget{site-information-navigation}{%
\subsection{Site Information
Navigation}\label{site-information-navigation}}

\begin{itemize}
\tightlist
\item
  \href{https://help.nytimes.com/hc/en-us/articles/115014792127-Copyright-notice}{©~2020~The
  New York Times Company}
\end{itemize}

\begin{itemize}
\tightlist
\item
  \href{https://www.nytco.com/}{NYTCo}
\item
  \href{https://help.nytimes.com/hc/en-us/articles/115015385887-Contact-Us}{Contact
  Us}
\item
  \href{https://www.nytco.com/careers/}{Work with us}
\item
  \href{https://nytmediakit.com/}{Advertise}
\item
  \href{http://www.tbrandstudio.com/}{T Brand Studio}
\item
  \href{https://www.nytimes.com/privacy/cookie-policy\#how-do-i-manage-trackers}{Your
  Ad Choices}
\item
  \href{https://www.nytimes.com/privacy}{Privacy}
\item
  \href{https://help.nytimes.com/hc/en-us/articles/115014893428-Terms-of-service}{Terms
  of Service}
\item
  \href{https://help.nytimes.com/hc/en-us/articles/115014893968-Terms-of-sale}{Terms
  of Sale}
\item
  \href{https://spiderbites.nytimes.com}{Site Map}
\item
  \href{https://help.nytimes.com/hc/en-us}{Help}
\item
  \href{https://www.nytimes.com/subscription?campaignId=37WXW}{Subscriptions}
\end{itemize}
