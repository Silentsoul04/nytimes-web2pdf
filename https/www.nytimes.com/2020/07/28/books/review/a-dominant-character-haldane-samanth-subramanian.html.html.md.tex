Sections

SEARCH

\protect\hyperlink{site-content}{Skip to
content}\protect\hyperlink{site-index}{Skip to site index}

\href{https://www.nytimes.com/section/books/review}{Book Review}

\href{https://myaccount.nytimes.com/auth/login?response_type=cookie\&client_id=vi}{}

\href{https://www.nytimes.com/section/todayspaper}{Today's Paper}

\href{/section/books/review}{Book Review}\textbar{}The Groundbreaking
Scientist Who Risked All in Pursuit of His Beliefs

\url{https://nyti.ms/330FLKJ}

\begin{itemize}
\item
\item
\item
\item
\item
\end{itemize}

Advertisement

\protect\hyperlink{after-top}{Continue reading the main story}

Supported by

\protect\hyperlink{after-sponsor}{Continue reading the main story}

Nonfiction

\hypertarget{the-groundbreaking-scientist-who-risked-all-in-pursuit-of-his-beliefs}{%
\section{The Groundbreaking Scientist Who Risked All in Pursuit of His
Beliefs}\label{the-groundbreaking-scientist-who-risked-all-in-pursuit-of-his-beliefs}}

\includegraphics{https://static01.nyt.com/images/2020/07/28/books/review/28Weiner/28Weiner-articleLarge.jpg?quality=75\&auto=webp\&disable=upscale}

Buy Book ▾

\begin{itemize}
\tightlist
\item
  \href{https://www.amazon.com/gp/search?index=books\&tag=NYTBSREV-20\&field-keywords=A+Dominant+Character\%3A+The+Radical+Science+and+Restless+Politics+of+J.B.S.+Haldane+Samanth+Subramanian}{Amazon}
\item
  \href{https://du-gae-books-dot-nyt-du-prd.appspot.com/buy?title=A+Dominant+Character\%3A+The+Radical+Science+and+Restless+Politics+of+J.B.S.+Haldane\&author=Samanth+Subramanian}{Apple
  Books}
\item
  \href{https://www.anrdoezrs.net/click-7990613-11819508?url=https\%3A\%2F\%2Fwww.barnesandnoble.com\%2Fw\%2F\%3Fean\%3D9780393634242}{Barnes
  and Noble}
\item
  \href{https://www.anrdoezrs.net/click-7990613-35140?url=https\%3A\%2F\%2Fwww.booksamillion.com\%2Fp\%2FA\%2BDominant\%2BCharacter\%253A\%2BThe\%2BRadical\%2BScience\%2Band\%2BRestless\%2BPolitics\%2Bof\%2BJ.B.S.\%2BHaldane\%2FSamanth\%2BSubramanian\%2F9780393634242}{Books-A-Million}
\item
  \href{https://bookshop.org/a/3546/9780393634242}{Bookshop}
\item
  \href{https://www.indiebound.org/book/9780393634242?aff=NYT}{Indiebound}
\end{itemize}

When you purchase an independently reviewed book through our site, we
earn an affiliate commission.

By Jonathan Weiner

\begin{itemize}
\item
  July 28, 2020
\item
  \begin{itemize}
  \item
  \item
  \item
  \item
  \item
  \end{itemize}
\end{itemize}

\textbf{A DOMINANT CHARACTER}\\
\textbf{The Radical Science and Restless Politics of J. B. S. Haldane}\\
By Samanth Subramanian

``Suffer'' is the family motto of the Haldane clan, and, along with a
line from Aeschylus --- ``We suffer into knowledge'' --- it is the
epigraph of Samanth Subramanian's ``A Dominant Character,'' a
fascinating biography of the British biologist J. B. S. Haldane. In
England, Haldane was a scientific celebrity as famous as Einstein. One
of his contemporaries called him ``the last man who might know all there
was to be known.'' But he lived a life of epic suffering, much of it
self-inflicted, and the political dimension of his story speaks to the
pain that we are living through now.

The self-infliction began early. His father, John Scott Haldane, an
eminent physiologist, often used himself as a human guinea pig in his
research, and his son, known as Jack, served as assistant guinea pig.
When Jack was 13, in 1906, his father sent him over the side of a
torpedo gunboat in an ill-fitting diving suit weighing 155 pounds and
then raised and lowered him by stages to test the best method to avoid
``the bends.'' On another ship, John killed plague rats with
experimental doses of sulfur dioxide, while Jack dashed down into the
hold to help collect the dead ones.

It was Jack's father who invented the idea of taking canaries into
mines. Along with the canary, there was young Jack. Deep in an old coal
pit in Staffordshire, he recited ``Friends, Romans, countrymen\ldots{}''
for his father until the methane in the air made the boy collapse.

\includegraphics{https://static01.nyt.com/images/2020/07/06/books/review/Weiner2/Weiner2-articleLarge.jpg?quality=75\&auto=webp\&disable=upscale}

Haldane seems to have enjoyed it: the science, the risk, the suffering
for a cause. Brilliant and intellectually omnivorous, he studied the
classics and mathematics at Oxford while doing genetics experiments on
the side. When the outbreak of World War I interrupted his schooling, he
worked on his first genetics paper in the trenches. He made himself an
expert in hand-thrown bombs and discovered that he loved killing. One
general called him ``the bravest and dirtiest soldier in my army.''
Wartime was one of the best times in Haldane's life; forever after, he
would dream sweet dreams of the front lines.

When the Great War was over, he threw himself just as fiercely into
research. Darwin's theory of evolution by natural selection had been
weakened by his ignorance of the laws of inheritance --- he never read
Mendel's papers on the subject. And Mendel, breeding peas and bees in
his monastery garden, seems to have missed the full significance of his
own studies for Darwinism. Haldane helped bring their work together in
what is known as the ``modern synthesis.'' Thanks in good part to the
insights of Haldane and a few other gifted, mathematically-minded
researchers, evolutionary biology became a powerful science that
embraces a vast range of temporal and physical scales, from the passage
of geological ages to the collision of a single sperm and egg, from the
biosphere to the gene.

In Mendelian genetics, a ``dominant character'' is a trait (such as
smoothness in peas) that invariably prevails over another (in this case,
wrinkledness) when genes for both are passed on to an offspring. Haldane
was dominant, never recessive. He lived ``a boisterous life,''
Subramanian writes, ``stocked with enough danger and drama for a dozen
ordinary humans.'' A reporter once described him as ``a large woolly
rhinoceros of uncertain temper.'' He made an arresting figure at the
podium --- as Subramanian puts it: ``a king-size man in rumpled clothes,
his mustache so thick and his head so large and bare that it was as if a
bird had built a nest at the base of a boulder.''

As his father's assistant, he'd risked his neck and his lungs to improve
the working conditions of divers, sailors and miners. As a soldier, he'd
relished not only the savagery of combat but also the comradeship of the
trenches. He wanted science to help everyone, not just his own
privileged, patrician class. He thought the Soviet experiment promised
to do that, and he became a communist --- eventually chairman of the
editorial board of The Daily Worker.

Subramanian, a journalist and regular contributor to The Guardian, is a
strong writer, and he recounts Haldane's communist adventures with brio:
the hoarse, roaring speeches in Trafalgar Square; the admiring trip to
Stalin's Soviet Union; the tour of the front lines of the Spanish Civil
War, where Haldane kicked around a bit with Hemingway and Martha
Gellhorn. (He almost got them all blown up.) In London during the Blitz,
Haldane designed a giant, inexpensive underground bomb shelter that he
argued could save thousands of lives. These ``Haldane Shelters'' were
never built. To Haldane that was another crime of capitalist society,
and in this telling, at least, he had a case. British intelligence kept
him under surveillance for more than 20 years on the suspicion ---
probably unfounded --- that he was a Soviet spy. (In a way, as
Subramanian says, MI5 was Haldane's first biographer.)

Image

A national conference for British-Soviet unity held in London in 1946.
J.B.S. Haldane, a committed communist, championed the Soviet Union long
after Stalin began slaughtering his people.Credit...The New York Times

``I am a man of violence by temperament and training,'' Haldane once
declared. He liked to claim that he was descended from Pedro the Cruel,
the king of Castile and Léon. With his passion, his iconoclasm and his
willingness to shock, his celebrity grew and grew. People liked his
sensational stories of self-experimentation. (He continued his father's
physiological research; a fit of convulsions in one self-designed
chamber of horrors broke his back.) They loved reading about his
scandalous first marriage to Charlotte Burghes, a journalist, which made
the British tabloids and almost got him kicked out of Cambridge, where
he had taken a position as a reader in biochemistry. She was married
when they met; her divorce was ugly; their own union was
unconventionally loose. (His next marriage was to Helen Spurway, a
biologist who was 22 years younger than he was.)

In lectures --- which drew large crowds --- and in pubs, Haldane tossed
off important and futuristic ideas like firecrackers. He wrote a
revolutionary paper that helped transform the way biologists think about
the origin of life. His vision of what would become known as test-tube
babies helped inspire Aldous Huxley's ``Brave New World.'' Haldane was a
terrific writer in his own right. His political essays were ``like razor
blades in print,'' Subramanian says. His science essays were superb. In
one of his best, ``On Being the Right Size,'' Haldane writes, ``You can
drop a mouse down a thousand-yard mine shaft; and, on arriving at the
bottom, it gets a slight shock and walks away, provided that the ground
is fairly soft. A rat is killed, a man is broken, a horse splashes.''
What a sentence. That last word shocks you every time --- and you can
hear in it more than a hint of his genetic inheritance from Pedro the
Cruel.

At his best, Haldane was a heroic example of the scientist as activist,
humanist and idealist. ``He felt, as we now feel afresh in our
century,'' Subramanian writes, ``that nations were held rapt by the
wealthy, that they were warmongering and venal, that they placed the
narrow interests of the powerful above the well-being of the
powerless.'' Many of his views on class and race have aged well. But he
picked petty fights wherever he went; and he championed the Soviet Union
long after Stalin began slaughtering his people and murdering his
geneticists. Haldane put himself through disgraceful intellectual
contortions to defend Stalinist pseudoscience.

Marx thought a single theory would someday cover everything from the
laws of physics to the laws of human progress: ``There will be
\emph{one} science.'' It was Haldane's great accomplishment to help make
biology one science, with the modern synthesis. To do more, to explain
the tragic messiness of history --- that kind of synthesis continues to
elude us. It's hard enough to sum up the good and the bad in one human
being.

``A Dominant Character'' is the best Haldane biography yet. With science
so politicized in this country and abroad, the book could be an allegory
for every scientist who wants to take a stand. ``In the past few
years,'' Subramanian writes, ``as we've witnessed deliberate assaults on
fact and truth and as we've realized the failures of the calm weight of
scientific evidence to influence government policy, the need for
scientists to find their voice has grown even more urgent.'' Haldane's
political principles were ``unbending and forthright,'' as Subramanian
says, and his science illuminated all of life. In both these ways, for
all his failings, he was ``deeply attractive during a time of shifting,
murky moralities.''

Advertisement

\protect\hyperlink{after-bottom}{Continue reading the main story}

\hypertarget{site-index}{%
\subsection{Site Index}\label{site-index}}

\hypertarget{site-information-navigation}{%
\subsection{Site Information
Navigation}\label{site-information-navigation}}

\begin{itemize}
\tightlist
\item
  \href{https://help.nytimes.com/hc/en-us/articles/115014792127-Copyright-notice}{©~2020~The
  New York Times Company}
\end{itemize}

\begin{itemize}
\tightlist
\item
  \href{https://www.nytco.com/}{NYTCo}
\item
  \href{https://help.nytimes.com/hc/en-us/articles/115015385887-Contact-Us}{Contact
  Us}
\item
  \href{https://www.nytco.com/careers/}{Work with us}
\item
  \href{https://nytmediakit.com/}{Advertise}
\item
  \href{http://www.tbrandstudio.com/}{T Brand Studio}
\item
  \href{https://www.nytimes.com/privacy/cookie-policy\#how-do-i-manage-trackers}{Your
  Ad Choices}
\item
  \href{https://www.nytimes.com/privacy}{Privacy}
\item
  \href{https://help.nytimes.com/hc/en-us/articles/115014893428-Terms-of-service}{Terms
  of Service}
\item
  \href{https://help.nytimes.com/hc/en-us/articles/115014893968-Terms-of-sale}{Terms
  of Sale}
\item
  \href{https://spiderbites.nytimes.com}{Site Map}
\item
  \href{https://help.nytimes.com/hc/en-us}{Help}
\item
  \href{https://www.nytimes.com/subscription?campaignId=37WXW}{Subscriptions}
\end{itemize}
