Sections

SEARCH

\protect\hyperlink{site-content}{Skip to
content}\protect\hyperlink{site-index}{Skip to site index}

\href{https://www.nytimes.com/section/books/review}{Book Review}

\href{https://myaccount.nytimes.com/auth/login?response_type=cookie\&client_id=vi}{}

\href{https://www.nytimes.com/section/todayspaper}{Today's Paper}

\href{/section/books/review}{Book Review}\textbar{}Aimee Bender's Latest
Is a Proustian Reverie

\url{https://nyti.ms/3hMRAIJ}

\begin{itemize}
\item
\item
\item
\item
\item
\end{itemize}

\href{https://www.nytimes.com/spotlight/at-home?action=click\&pgtype=Article\&state=default\&region=TOP_BANNER\&context=at_home_menu}{At
Home}

\begin{itemize}
\tightlist
\item
  \href{https://www.nytimes.com/2020/08/03/well/family/the-benefits-of-talking-to-strangers.html?action=click\&pgtype=Article\&state=default\&region=TOP_BANNER\&context=at_home_menu}{Talk:
  To Strangers}
\item
  \href{https://www.nytimes.com/2020/08/01/at-home/coronavirus-make-pizza-on-a-grill.html?action=click\&pgtype=Article\&state=default\&region=TOP_BANNER\&context=at_home_menu}{Make:
  Grilled Pizza}
\item
  \href{https://www.nytimes.com/2020/07/31/arts/television/goldbergs-abc-stream.html?action=click\&pgtype=Article\&state=default\&region=TOP_BANNER\&context=at_home_menu}{Watch:
  'The Goldbergs'}
\item
  \href{https://www.nytimes.com/interactive/2020/at-home/even-more-reporters-editors-diaries-lists-recommendations.html?action=click\&pgtype=Article\&state=default\&region=TOP_BANNER\&context=at_home_menu}{Explore:
  Reporters' Google Docs}
\end{itemize}

Advertisement

\protect\hyperlink{after-top}{Continue reading the main story}

Supported by

\protect\hyperlink{after-sponsor}{Continue reading the main story}

Fiction

\hypertarget{aimee-benders-latest-is-a-proustian-reverie}{%
\section{Aimee Bender's Latest Is a Proustian
Reverie}\label{aimee-benders-latest-is-a-proustian-reverie}}

\includegraphics{https://static01.nyt.com/images/2020/07/10/books/review/Brockmeier1/Brockmeier1-articleLarge.jpg?quality=75\&auto=webp\&disable=upscale}

Buy Book ▾

\begin{itemize}
\tightlist
\item
  \href{https://www.amazon.com/gp/search?index=books\&tag=NYTBSREV-20\&field-keywords=The+Butterfly+Lampshade+Aimee+Bender}{Amazon}
\item
  \href{https://du-gae-books-dot-nyt-du-prd.appspot.com/buy?title=The+Butterfly+Lampshade\&author=Aimee+Bender}{Apple
  Books}
\item
  \href{https://www.anrdoezrs.net/click-7990613-11819508?url=https\%3A\%2F\%2Fwww.barnesandnoble.com\%2Fw\%2F\%3Fean\%3D9780385534871}{Barnes
  and Noble}
\item
  \href{https://www.anrdoezrs.net/click-7990613-35140?url=https\%3A\%2F\%2Fwww.booksamillion.com\%2Fp\%2FThe\%2BButterfly\%2BLampshade\%2FAimee\%2BBender\%2F9780385534871}{Books-A-Million}
\item
  \href{https://bookshop.org/a/3546/9780385534871}{Bookshop}
\item
  \href{https://www.indiebound.org/book/9780385534871?aff=NYT}{Indiebound}
\end{itemize}

When you purchase an independently reviewed book through our site, we
earn an affiliate commission.

By Kevin Brockmeier

\begin{itemize}
\item
  July 28, 2020
\item
  \begin{itemize}
  \item
  \item
  \item
  \item
  \item
  \end{itemize}
\end{itemize}

\textbf{THE BUTTERFLY LAMPSHADE}\\
By Aimee Bender

If it is defensible to divide works of fiction into those that prize
stillness and those that prize motion, then ``The Butterfly Lampshade,''
Aimee Bender's compact surrealist memory box of a novel, sets its store
firmly by stillness. Her earlier books were dedicated, as this one is,
to transfiguring the American domestic landscape by way of magic,
fantasy, bewitchment, peculiarization. But where those books were also
propelled by motion, to change, ``The Butterfly Lampshade'' by contrast
stakes its ground early, and remains there. It resists becoming
something other than what its opening pages suggest it's going to be.
Yet its particular quality of stillness hums with so much mystery and
intensity that the book never feels static. It is a measure of the
book's success that as I reached the conclusion, I felt considerably
more altered by the experience than I often am by novels that travel
much further from their beginnings.

The music critic Walter Holland recently proposed that ``active musical
listening is part patient attention to the moment and part predictive
attention to the possible futures that that moment suggests.'' If the
same holds true for reading --- and I think it does --- then Bender's
success here might be explained by the ease and simplicity with which
she commingles these two forms of attention. After all, how often does a
novel that seems poised to reward your immersive attention diminish in
its power, conspicuously and all at once, as soon as it tries to engage
your predictive attention? Everything begins so promisingly, but then
the plot takes hold and the book becomes smaller, more desiccated, as
you realize the predictive attention the writer is applying to the
material is so much more meager than it could have been, or than your
own was. ``The Butterfly Lampshade'' never makes that swerve. Instead it
retraces the path it has already established, gradually filling in its
textures, looking both back and deeper. In this way, it evades the
stiffness of those stories that are able to move forward only by
hardening into their possibilities.

Early in the novel, before her obsessions saturate her life, the
narrator, Francie, holds a managerial position at a framing store. It is
a telling job for someone so aware of the need to maintain the borders
between things. ``My great love then --- and still --- is delineation,''
she says, recollecting a few formative days when, at the age of 8, she
lost her mother to psychosis and ``the scrim of meaning had floated off
of everything.''

After a premonitory glimpse of the book's cast, we learn that three
times during Francie's childhood she witnessed a sort of mystic
reification, when a picture to which she'd given her attention opened
out from its surface, discharging itself into object being: a butterfly
from a lampshade, a beetle from a worksheet, and a rose from a window
curtain.

Image

The novel is a kind of small-scale, supernatural Proustian reverie:
Proust if what Proust had been trying to recover was not luminous,
ordinary reality, but a rupture therein.

She spends the rest of the novel trying to comprehend these visitations.
The memory of them remains vivid in her mind, coloring even her most
mundane moments: ``I ate my bag of potato chips and sat next to the
small succulent plant in its terra-cotta pot left behind by a previous
tenant, and for a moment felt myself living inside both times at once.''
This double attention, which lies at the center of her experiences,
turns the novel into a kind of small-scale, supernatural Proustian
reverie: Proust if what Proust had been trying to recover was not
luminous, ordinary reality, but a rupture in luminous ordinary reality;
Proust if his childhood had been broken open by Arthur Machen or Lord
Dunsany.

Bender's concern with evoking the inwardness of objects, however, is
less common to fiction than to poetry. I think of Francis Ponge,
Gertrude Stein, Martha Ronk --- the great thing-masons of literature,
describing pebbles, bowls and buttons with rigor and exultation. In some
respects, Bender's ongoing fascination with the border separating
objects from people makes ``The Butterfly Lampshade'' a mirror image of
her 2010 novel, ``The Particular Sadness of Lemon Cake'': In that
earlier novel, the narrator's brother was a living being who took on the
quiddity of an object; in this one, objects take on the quiddities of
living beings.

But such an occurrence, though ``fun to imagine in a story,'' Francie
insists, ``is terrifying in real life.'' An object can be magical
without becoming an amulet or a charm. The butterfly ``had to gain
internal functions and an external structure, had to come out of an
entirely different plane of existence to make itself, but somehow it
did,'' she thinks. ``It was an active psychosis.''

Increasingly, Francie feels ``as if everything --- hamburger, cartoon
dog, letters --- might be on the verge of popping into the world.'' And
indeed, late in the book, she witnesses one last puncture in reality:
two humans, or human-shapes, who request ``tickets'' from her, ``a
paper,'' as if ``speaking in another language that was still pretending
to be our language.'' Reading these final pages, it is hard not to feel
as Francie does: that anything and anyone might be a two-way street,
capable of passing from our side into theirs by means of illustration
--- or from their side into ours by means of emanation.

One finishes the novel with the eerie sense that we too are objects who
have slipped accidentally into being and that, like the butterfly, the
beetle and the dried rose, we really ought not to be here.

Advertisement

\protect\hyperlink{after-bottom}{Continue reading the main story}

\hypertarget{site-index}{%
\subsection{Site Index}\label{site-index}}

\hypertarget{site-information-navigation}{%
\subsection{Site Information
Navigation}\label{site-information-navigation}}

\begin{itemize}
\tightlist
\item
  \href{https://help.nytimes.com/hc/en-us/articles/115014792127-Copyright-notice}{©~2020~The
  New York Times Company}
\end{itemize}

\begin{itemize}
\tightlist
\item
  \href{https://www.nytco.com/}{NYTCo}
\item
  \href{https://help.nytimes.com/hc/en-us/articles/115015385887-Contact-Us}{Contact
  Us}
\item
  \href{https://www.nytco.com/careers/}{Work with us}
\item
  \href{https://nytmediakit.com/}{Advertise}
\item
  \href{http://www.tbrandstudio.com/}{T Brand Studio}
\item
  \href{https://www.nytimes.com/privacy/cookie-policy\#how-do-i-manage-trackers}{Your
  Ad Choices}
\item
  \href{https://www.nytimes.com/privacy}{Privacy}
\item
  \href{https://help.nytimes.com/hc/en-us/articles/115014893428-Terms-of-service}{Terms
  of Service}
\item
  \href{https://help.nytimes.com/hc/en-us/articles/115014893968-Terms-of-sale}{Terms
  of Sale}
\item
  \href{https://spiderbites.nytimes.com}{Site Map}
\item
  \href{https://help.nytimes.com/hc/en-us}{Help}
\item
  \href{https://www.nytimes.com/subscription?campaignId=37WXW}{Subscriptions}
\end{itemize}
