Sections

SEARCH

\protect\hyperlink{site-content}{Skip to
content}\protect\hyperlink{site-index}{Skip to site index}

\href{https://www.nytimes.com/section/business}{Business}

\href{https://myaccount.nytimes.com/auth/login?response_type=cookie\&client_id=vi}{}

\href{https://www.nytimes.com/section/todayspaper}{Today's Paper}

\href{/section/business}{Business}\textbar{}Malls Seek to Evolve as the
Pandemic Hastens a Retail Overhaul

\url{https://nyti.ms/30UfgUv}

\begin{itemize}
\item
\item
\item
\item
\item
\end{itemize}

\href{https://www.nytimes.com/news-event/coronavirus?action=click\&pgtype=Article\&state=default\&region=TOP_BANNER\&context=storylines_menu}{The
Coronavirus Outbreak}

\begin{itemize}
\tightlist
\item
  live\href{https://www.nytimes.com/2020/08/01/world/coronavirus-covid-19.html?action=click\&pgtype=Article\&state=default\&region=TOP_BANNER\&context=storylines_menu}{Latest
  Updates}
\item
  \href{https://www.nytimes.com/interactive/2020/us/coronavirus-us-cases.html?action=click\&pgtype=Article\&state=default\&region=TOP_BANNER\&context=storylines_menu}{Maps
  and Cases}
\item
  \href{https://www.nytimes.com/interactive/2020/science/coronavirus-vaccine-tracker.html?action=click\&pgtype=Article\&state=default\&region=TOP_BANNER\&context=storylines_menu}{Vaccine
  Tracker}
\item
  \href{https://www.nytimes.com/interactive/2020/07/29/us/schools-reopening-coronavirus.html?action=click\&pgtype=Article\&state=default\&region=TOP_BANNER\&context=storylines_menu}{What
  School May Look Like}
\item
  \href{https://www.nytimes.com/live/2020/07/31/business/stock-market-today-coronavirus?action=click\&pgtype=Article\&state=default\&region=TOP_BANNER\&context=storylines_menu}{Economy}
\end{itemize}

Advertisement

\protect\hyperlink{after-top}{Continue reading the main story}

Supported by

\protect\hyperlink{after-sponsor}{Continue reading the main story}

Square Feet

\hypertarget{malls-seek-to-evolve-as-the-pandemic-hastens-a-retail-overhaul}{%
\section{Malls Seek to Evolve as the Pandemic Hastens a Retail
Overhaul}\label{malls-seek-to-evolve-as-the-pandemic-hastens-a-retail-overhaul}}

Operators have spent billions positioning themselves for a future with
few or no department stores by reconfiguring their properties for other
options, including housing, health care and logistics.

\includegraphics{https://static01.nyt.com/images/2020/07/28/business/28Virus-Malls/merlin_174989943_7ad78d70-1827-49fa-b5c8-34f4c173aeb8-articleLarge.jpg?quality=75\&auto=webp\&disable=upscale}

By Joe Gose

\begin{itemize}
\item
  July 28, 2020
\item
  \begin{itemize}
  \item
  \item
  \item
  \item
  \item
  \end{itemize}
\end{itemize}

In southwest San Francisco near San Francisco State University and
championship golf courses, Brookfield Properties is redeveloping its
\href{https://www.stonestowngalleria.com/en.html}{Stonestown Galleria}
in perhaps the most disruptive retail environment in modern times.

Macy's vacated the mall in 2018, and Nordstrom followed about 18 months
later. Consequently, Brookfield, a global real estate developer and
manager, is spending \$149 million to reconfigure the
804,000-square-foot property, adding a Whole Foods, a health care
provider and a Sports Basement sporting goods store, while expanding an
existing Target and relocating a stand-alone Regal Cinemas to inside the
mall.

Plans to build apartments are also in the works at the property, which
after reopening in June was ordered to close again as
\href{https://www.nytimes.com/interactive/2020/us/california-coronavirus-cases.html}{coronavirus
cases surged in California}.

More than just a routine overhaul, the improvements represent an effort
to stay relevant as growth in online shopping combined with shifts in
retailing practices and consumer tastes have for years
\href{https://www.nytimes.com/2020/07/05/business/coronavirus-malls-department-stores-bankruptcy.html}{dismantled
the traditional shopping mall model}.

Image

A shuttered entrance to Nordstrom, a former anchor at Stonestown
Galleria.Credit...Jason Henry for The New York Times

Image

After losing Nordstrom, the mall's owner expanded an existing
Target.Credit...Jason Henry for The New York Times

``If the old mall design was one of two to four department stores with
connecting retail space, in my mind what we're doing at Stonestown
Galleria is closest to the modern footprint going forward,'' said Adam
Tritt, executive vice president of development for Brookfield. ``We
don't necessarily see the pandemic or shifting role of department stores
as new. Retail is always changing.''

Indeed, Brookfield, Simon Property Group, Unibail-Rodamco-Westfield and
other mall operators have spent billions positioning themselves for a
future with few or no department stores after the struggles of
traditional anchors like
\href{https://www.nytimes.com/2018/10/14/business/sears-bankruptcy-filing-chapter-11.html}{Sears}
and Macy's. Under pressure caused by pandemic-induced lockdowns,
\href{https://www.nytimes.com/2020/05/15/business/jc-penney-bankruptcy-coronavirus.html}{J.C.
Penney} and
\href{https://www.nytimes.com/2020/05/07/business/neiman-marcus-bankruptcy.html}{Neiman
Marcus} are the latest distressed mall anchors
\href{https://www.nytimes.com/2020/05/14/business/coronavirus-retail-bankruptcies-private-equity.html}{to
declare bankruptcy and close stores}. Malls were also hit by the loss of
other retailers that have fallen during the pandemic, like
\href{https://www.nytimes.com/2020/07/08/business/brooks-brothers-chapter-11-bankruptcy.html}{Brooks
Brothers} and
\href{https://www.nytimes.com/2020/07/23/business/ascena-bankruptcy-ann-taylor-lane-bryant.html}{Ascena
Retail Group}, the owner of Ann Taylor and Lane Bryant.

Developers have replaced the vacant big boxes with a mix of retail,
dining, entertainment, fitness, co-working and health care options. They
have also added apartments, hotels and offices to the properties ---
often to make better use of vacant parking lots and create built-in
traffic generators --- and they are beginning to create distribution and
self-storage hubs at malls as more people purchase their goods online.

\includegraphics{https://static01.nyt.com/images/2020/07/29/business/28Virus-Malls-04/merlin_174989964_fc847846-096a-43d2-b330-cc0dd57bfa6a-articleLarge.jpg?quality=75\&auto=webp\&disable=upscale}

In June, Washington Prime Group agreed to turn a former Sears location
at its Morgantown Mall into a logistics, distribution and fulfillment
center for WVU Medicine, the health care network serving West Virginia
University. The move comes after Washington Prime's recently announced
Fulventory initiative, through which the company is providing logistics
and warehouse solutions at its properties.

\hypertarget{latest-updates-economy}{%
\section{\texorpdfstring{\href{https://www.nytimes.com/live/2020/07/31/business/stock-market-today-coronavirus?action=click\&pgtype=Article\&state=default\&region=MAIN_CONTENT_1\&context=storylines_live_updates}{Latest
Updates:
Economy}}{Latest Updates: Economy}}\label{latest-updates-economy}}

\href{https://www.nytimes.com/live/2020/07/31/business/stock-market-today-coronavirus?action=click\&pgtype=Article\&state=default\&region=MAIN_CONTENT_1\&context=storylines_live_updates\#kodaks-chief-executive-was-given-stock-options-then-the-share-price-spiked-1000-percent}{29h
ago}

\href{https://www.nytimes.com/live/2020/07/31/business/stock-market-today-coronavirus?action=click\&pgtype=Article\&state=default\&region=MAIN_CONTENT_1\&context=storylines_live_updates\#kodaks-chief-executive-was-given-stock-options-then-the-share-price-spiked-1000-percent}{Kodak's
chief executive was given stock options. Then the share price spiked
1,000 percent.}

\href{https://www.nytimes.com/live/2020/07/31/business/stock-market-today-coronavirus?action=click\&pgtype=Article\&state=default\&region=MAIN_CONTENT_1\&context=storylines_live_updates\#fitch-ratings-downgrades-its-outlook-on-us-debt}{32h
ago}

\href{https://www.nytimes.com/live/2020/07/31/business/stock-market-today-coronavirus?action=click\&pgtype=Article\&state=default\&region=MAIN_CONTENT_1\&context=storylines_live_updates\#fitch-ratings-downgrades-its-outlook-on-us-debt}{Fitch
Ratings downgrades its outlook on U.S. debt.}

\href{https://www.nytimes.com/live/2020/07/31/business/stock-market-today-coronavirus?action=click\&pgtype=Article\&state=default\&region=MAIN_CONTENT_1\&context=storylines_live_updates\#us-sanctions-more-chinese-officials-over-human-rights-violations-as-tensions-flare}{39h
ago}

\href{https://www.nytimes.com/live/2020/07/31/business/stock-market-today-coronavirus?action=click\&pgtype=Article\&state=default\&region=MAIN_CONTENT_1\&context=storylines_live_updates\#us-sanctions-more-chinese-officials-over-human-rights-violations-as-tensions-flare}{U.S.
sanctions more Chinese officials over human rights violations as
tensions flare}

\href{https://www.nytimes.com/live/2020/07/31/business/stock-market-today-coronavirus?action=click\&pgtype=Article\&state=default\&region=MAIN_CONTENT_1\&context=storylines_live_updates}{See
more updates}

More live coverage:
\href{https://www.nytimes.com/2020/08/01/world/coronavirus-covid-19.html?action=click\&pgtype=Article\&state=default\&region=MAIN_CONTENT_1\&context=storylines_live_updates}{Global}

``As more department stores become vacant, we do need to re-envision the
future of mall properties,'' said Greg Maloney, president and chief
executive of the Americas retail unit of Jones Lang LaSalle. ``Will it
be 100 percent retail? No, but its success still comes down to
location.''

Long before \href{https://www.nytimes.com/news-event/coronavirus}{the
coronavirus} arrived in the United States, many malls, often
overburdened with debt and struggling with vacancy and declining values,
were fighting to stay alive.

The number of malls has declined to less than 1,000 today from 3,000 at
the turn of the century, according to Nick Egelanian, president of
SiteWorks, a shopping center and retail consultant in Annapolis, Md.
And, he predicts, only about 200 of the strongest malls with the best
locations will be left by the end of the decade, if not sooner.

But to thrive, most must adapt, said Mr. Egelanian, who has long argued
that the deconstruction of department stores --- and therefore malls ---
began 40 years ago with the dawn of big-box stores, or ``category
killers.''

``The true mall of the future will incorporate a mix of uses,'' he said,
``and the retail will be downsized: If it has 2 million square feet
today, it may only need 1 million square feet tomorrow. But it's going
to be painful getting there, and the ones that survive are going to need
a lot of capital.''

The ideal, he said, will mirror properties like Tysons Corner Center, a
1.9-million-square-foot mall in Virginia owned by Macerich that is
surrounded by offices, high-rise residences and hotels.

Some malls will emphasize luxury and cater to the affluent, observers
add, while others will focus on middle-market consumers and continue to
replace former anchors with off-price tenants, such as Dick's Sporting
Goods, Burlington and T.J. Maxx, which have traditionally operated in
strip centers or lower-tier malls.

The off-price strategy is one that malls of higher quality previously
avoided, said Vince Tibone, a senior analyst covering retail for Green
Street Advisors.

``In the minds of the owners of top malls, there was a higher and better
use for their properties,'' he said. ``But there are a lot fewer options
to backfill space today, and even those malls are looking to just get
tenants into vacancies.''

To better position itself for the future, one middle-market mall owner,
Pennsylvania Real Estate Investment Trust, has sold 18 malls with
significant department store exposure over the past five years and is
reinvesting more than \$885 million in proceeds into its core assets.

Image

The Pennsylvania Real Estate Investment Trust spent \$210 million to
renovate the Fashion District mall in downtown
Philadelphia.Credit...PREIT

Image

The new Fashion District has new food and entertainment tenants like an
AMC Theaters cineplex.Credit...Patrick Darby Photography / Tonic Photo
Studios

Among other projects, the real estate trust sank \$210 million into the
Fashion District, a 1.5 million-square-foot mall in downtown
Philadelphia that it owns with Macerich. A number of new tenants joined
the roster when it reopened in 2019, including an AMC Theaters cineplex
and the flexible-office provider Industrious.

Other nonretail companies continue to show interest in the company's
properties, said Joseph F. Coradino, chief executive of Pennsylvania
Real Estate Investment Trust, which is based in Philadelphia and owns 21
malls primarily on the East Coast.

``When you step back and look at malls, they typically have phenomenal
locations at major intersections and highways,'' Mr. Coradino said.
``Certainly the mall business today is different than it was a year ago,
or even six months ago. But I don't think the success of malls is a
question of apocalypse or death. I think it's really an evolution.''

Yet like many malls that have been repositioned in the past decade,
Fashion District boasts a high percentage of food and entertainment
tenants. Plagued by questions over how soon consumers will feel safe
returning to them, these businesses are operating at partial capacity
nationwide, and some are closing.

``The pandemic certainly gives us a cautionary tale on the value of
tenant diversification and how we can best utilize our real estate
footprint,'' Mr. Tritt said.

Image

The flexible-office provider Industrious has moved into the Fashion
District mall.Credit...Matterport

Observers anticipate that entertainment and restaurants will continue to
generate traffic over the long term, although they acknowledge that full
recovery depends on a coronavirus vaccine.

Still, restaurants and entertainment have become so ubiquitous that they
have failed to live up to the expectations of many mall owners, said
Scott Y. Stuart, chief executive of the Turnaround Management
Association, an organization in Chicago that represents restructuring
professionals.

``While the uses have filled gaps in malls, they're beginning to look
like short-term solutions,'' he said. ``Consumers have a choice to go to
a mall or some other setting to experience them.''

Some owners, however, will be forced to recognize that their locations
no longer fit in the retail world, Mr. Egelanian said. But that may
produce opportunities to start new industrial, housing, office or
mixed-use developments from scratch.

``There may not be any time in the last 100 years when so many 100-acre
sites located at that perfect intersection have been available for
redevelopment within such a short period of time,'' he suggested. ``They
will have value for many uses and could be big economic generators for
their communities.''

Advertisement

\protect\hyperlink{after-bottom}{Continue reading the main story}

\hypertarget{site-index}{%
\subsection{Site Index}\label{site-index}}

\hypertarget{site-information-navigation}{%
\subsection{Site Information
Navigation}\label{site-information-navigation}}

\begin{itemize}
\tightlist
\item
  \href{https://help.nytimes.com/hc/en-us/articles/115014792127-Copyright-notice}{©~2020~The
  New York Times Company}
\end{itemize}

\begin{itemize}
\tightlist
\item
  \href{https://www.nytco.com/}{NYTCo}
\item
  \href{https://help.nytimes.com/hc/en-us/articles/115015385887-Contact-Us}{Contact
  Us}
\item
  \href{https://www.nytco.com/careers/}{Work with us}
\item
  \href{https://nytmediakit.com/}{Advertise}
\item
  \href{http://www.tbrandstudio.com/}{T Brand Studio}
\item
  \href{https://www.nytimes.com/privacy/cookie-policy\#how-do-i-manage-trackers}{Your
  Ad Choices}
\item
  \href{https://www.nytimes.com/privacy}{Privacy}
\item
  \href{https://help.nytimes.com/hc/en-us/articles/115014893428-Terms-of-service}{Terms
  of Service}
\item
  \href{https://help.nytimes.com/hc/en-us/articles/115014893968-Terms-of-sale}{Terms
  of Sale}
\item
  \href{https://spiderbites.nytimes.com}{Site Map}
\item
  \href{https://help.nytimes.com/hc/en-us}{Help}
\item
  \href{https://www.nytimes.com/subscription?campaignId=37WXW}{Subscriptions}
\end{itemize}
