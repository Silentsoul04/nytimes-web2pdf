Sections

SEARCH

\protect\hyperlink{site-content}{Skip to
content}\protect\hyperlink{site-index}{Skip to site index}

\href{https://myaccount.nytimes.com/auth/login?response_type=cookie\&client_id=vi}{}

\href{https://www.nytimes.com/section/todayspaper}{Today's Paper}

How to Stop Biting Your Nails

\url{https://nyti.ms/3hJ8702}

\begin{itemize}
\item
\item
\item
\item
\item
\end{itemize}

Advertisement

\protect\hyperlink{after-top}{Continue reading the main story}

Supported by

\protect\hyperlink{after-sponsor}{Continue reading the main story}

\href{/column/magazine-tip}{Tip}

\hypertarget{how-to-stop-biting-your-nails}{%
\section{How to Stop Biting Your
Nails}\label{how-to-stop-biting-your-nails}}

\includegraphics{https://static01.nyt.com/images/2020/08/02/magazine/02Mag-Tip-01/02Mag-Tip-01-articleLarge-v2.jpg?quality=75\&auto=webp\&disable=upscale}

By Malia Wollan

\begin{itemize}
\item
  July 28, 2020
\item
  \begin{itemize}
  \item
  \item
  \item
  \item
  \item
  \end{itemize}
\end{itemize}

``Learn to resist the urge,'' says Tara S. Peris, an associate professor
of psychiatry and biobehavioral sciences at the University of
California, Los Angeles, where she is co-director of the Child O.C.D.,
Anxiety and Tic Disorders Program. Psychiatrists consider nail biting a
``body focused repetitive behavior,'' along with things like hair
pulling and skin picking. Nail biting tends to begin in childhood and
adolescence, but researchers estimate that as much as 30 percent of
Americans are chronic nail biters. Often a form of self-soothing, the
disorder can, over time, disrupt the functioning of a brain's reward
circuitry. An occasional nibble probably isn't concerning, but if you
gnaw until you injure yourself --- if your fingers are bloody or
infected --- or if the biting distracts or shames you, you should know
that you can get help.

A treatment established in the early 1970s called habit reversal therapy
can break the cycle in as little as eight to 12 weeks. ``First become
very aware of the behavior,'' Peris says. Keep a written log. Focus
attention inward. What sensation do you experience just before you start
biting your nails? What mood accompanies the biting? Then turn outward
to your surroundings. Are you more likely to chew your hands in certain
rooms? In the car? When watching TV or reading? This first stage of
treatment, awareness training, typically takes about a week or two.
``Next you'll learn what we call a competing response,'' Peris says.
When you feel a nail bite coming, you'll do something else instead, like
clasp your hands or pinch your thumb and index finger and hold it for
one minute, or until the impulse subsides. Try modifying your
environment --- by, for example, doing your homework at the kitchen
table, rather than where you tend to bite more --- and then practice
catching and replacing the behavior over and over again.

Keep in mind that putting your hands in your mouth during a viral
pandemic increases your infection risk. ``During times of high stress,
you might see symptoms pop up or worsen,'' Peris says. ``That's normal
and you'll just need to practice those competing behavior skills
again.''

Advertisement

\protect\hyperlink{after-bottom}{Continue reading the main story}

\hypertarget{site-index}{%
\subsection{Site Index}\label{site-index}}

\hypertarget{site-information-navigation}{%
\subsection{Site Information
Navigation}\label{site-information-navigation}}

\begin{itemize}
\tightlist
\item
  \href{https://help.nytimes.com/hc/en-us/articles/115014792127-Copyright-notice}{©~2020~The
  New York Times Company}
\end{itemize}

\begin{itemize}
\tightlist
\item
  \href{https://www.nytco.com/}{NYTCo}
\item
  \href{https://help.nytimes.com/hc/en-us/articles/115015385887-Contact-Us}{Contact
  Us}
\item
  \href{https://www.nytco.com/careers/}{Work with us}
\item
  \href{https://nytmediakit.com/}{Advertise}
\item
  \href{http://www.tbrandstudio.com/}{T Brand Studio}
\item
  \href{https://www.nytimes.com/privacy/cookie-policy\#how-do-i-manage-trackers}{Your
  Ad Choices}
\item
  \href{https://www.nytimes.com/privacy}{Privacy}
\item
  \href{https://help.nytimes.com/hc/en-us/articles/115014893428-Terms-of-service}{Terms
  of Service}
\item
  \href{https://help.nytimes.com/hc/en-us/articles/115014893968-Terms-of-sale}{Terms
  of Sale}
\item
  \href{https://spiderbites.nytimes.com}{Site Map}
\item
  \href{https://help.nytimes.com/hc/en-us}{Help}
\item
  \href{https://www.nytimes.com/subscription?campaignId=37WXW}{Subscriptions}
\end{itemize}
