Pollution Is Killing Black Americans. This Community Fought Back.

\url{https://nyti.ms/2Bzkhcu}

\begin{itemize}
\item
\item
\item
\item
\item
\item
\end{itemize}

\includegraphics{https://static01.nyt.com/images/2020/08/02/magazine/02mag-philadelphia/02mag-philadelphia-articleLarge.jpg?quality=75\&auto=webp\&disable=upscale}

Sections

\protect\hyperlink{site-content}{Skip to
content}\protect\hyperlink{site-index}{Skip to site index}

Feature

\hypertarget{pollution-is-killing-black-americans-this-community-fought-back}{%
\section{Pollution Is Killing Black Americans. This Community Fought
Back.}\label{pollution-is-killing-black-americans-this-community-fought-back}}

African-Americans are 75 percent more likely than others to live near
facilities that produce hazardous waste. Can a grass-roots
environmental-justice movement make a difference?

The Philadelphia Energy Solutions refinery.Credit...Hannah Price for The
New York Times

Supported by

\protect\hyperlink{after-sponsor}{Continue reading the main story}

By Linda Villarosa

\begin{itemize}
\item
  July 28, 2020
\item
  \begin{itemize}
  \item
  \item
  \item
  \item
  \item
  \item
  \end{itemize}
\end{itemize}

\hypertarget{listen-to-this-article}{%
\subsubsection{Listen to This Article}\label{listen-to-this-article}}

Audio Recording by Audm

\emph{To hear more audio stories from publishers like The New York
Times, download}
\emph{\href{https://www.audm.com/?utm_source=nytmag\&utm_medium=embed\&utm_campaign=refinery_next_door\%09\%09\%09\%09\%09\%09https://www.audm.com/?utm_source=nytmag\&utm_medium=embed\&utm_campaign=refinery_next_door}{Audm
for iPhone or Android}.}

When Kilynn Johnson walks out the door of the house her parents bought
in 1972, where she grew up and lives to this day, she steps into the
warm embrace of a community where neighbors feel more like kin. Her home
sits across the street from Stinger Square Park, where Johnson passed
long days of her childhood playing alongside her siblings and cousins
and friends. But by age 8, diagnosed with asthma, she spent more time
sitting on the sidelines, watching the other children tumble on
playground equipment or rip and run through the park. Once in a while a
neighbor, Ms. Sylvia or any number of Black mother figures whom Johnson
and everyone knew never to call by just their first names, might come by
and check on her. ``You doing all right, Kilynn?'' they would ask the
quiet little girl.

Near the end of 2015, Johnson felt short of breath and wondered whether
the asthma that plagued her when she was a child had flared up once
again. By the last week of December, she was able to leave her house on
the corner of Dickinson Street and South 32nd Street, in the Grays Ferry
neighborhood of South Philadelphia, only once, to drag herself to church
on New Year's Eve. Three nights later, she began vomiting
uncontrollably. At sunrise, she managed to call her former partner,
Tony, and could get out only one word: ``Hospital.''

Several hours and a battery of tests later, doctors at the Hospital of
the University of Pennsylvania in West Philadelphia, across the
Schuylkill from Grays Ferry, told Johnson that she needed surgery to
remove a tumor from her gallbladder --- but that she was also suffering
from such a severe infection that she would require IV antibiotics and a
week in intensive care before doctors could operate. The surgery
revealed gallbladder cancer that had spread; the doctors removed her
gallbladder, seven lymph nodes and part of her liver. She needed six
weeks of both radiation and chemotherapy. ``They didn't know if I was
going to make it,'' Johnson said.

Shy and reserved by nature, Johnson was slow to tell anyone about the
cancer. ``I held it to myself,'' Johnson recalls. ``In the beginning it
was private, so I preferred to open up a little at a time.'' One day in
the spring of 2016, Johnson went out for some fresh air. Leaning heavily
on a walker, she passed the familiar rowhouses on Dickinson Street. As
she made her way with the walker, she met Sylvia Bennett, whom Johnson
still called Ms. Sylvia, and who lived three doors down on the same
block.

Bennett, 76, a retired behavioral-health specialist, had raised five
children in the tight-knit community of Grays Ferry. Bennett's youngest
daughter was just a little older than Kilynn Johnson; Ms. Sylvia had
watched Johnson grow up and raise a family of her own. Now, observing
her frail neighbor and the walker, she asked Johnson in her most gentle
voice: ``Where you been? Haven't seen you for a while.'' ``I think I
told her, `I been sick,''' Johnson says, recalling her reticence.
Bennett knew not to pry. This went on for months, until the summer day
when Bennett asked, ``How you doing?'' and Johnson told her, ``Ms.
Sylvia, I have cancer.''

After she recovered from the initial shock of her diagnosis, Johnson
began to wonder why she had such an unusual cancer. The Centers for
Disease Control and Prevention estimates that only about 3,700 Americans
find out they have gallbladder cancer each year; breast cancer is the
most frequently diagnosed cancer in the country, with more than 276,000
new cases annually. Because Johnson's disease was so uncommon, doctors
at University Hospital had to formulate a special treatment plan.
Gallbladder cancer occurs mainly in older people, and 72 is the average
age at diagnosis. Johnson was 46. ``I started thinking, What was I doing
with this?''

Bennett had an answer for her. ``Look across the highway,'' she said,
pointing toward the massive 150-year-old refinery, owned by Philadelphia
Energy Solutions since 2012, that was so familiar to Grays Ferry
residents that it seemed like part of the landscape.

\includegraphics{https://static01.nyt.com/images/2020/08/02/magazine/02mag-philadelphia-02/02mag-philadelphia-02-articleLarge-v2.jpg?quality=75\&auto=webp\&disable=upscale}

Over the next year, Bennett and Johnson began to tally the diseases all
around them suffered by the people they loved. Johnson's father's
brother, her uncle Robert, who also lived in the neighborhood, died of
prostate cancer in 2010, and three of his children, Kilynn's first
cousins, had also had different forms of cancer --- four out of six
people in one household. Those three cousins learned they had cancer
earlier than age 66, the average age of a diagnosis. Bennett's daughters
Ladeania and Wanda, found out they had breast cancer several months
apart and when they were both in their 50s; Wanda then came down with
multiple myeloma, a cancer of the blood. ``And now me,'' Johnson said.

Between the two of them, Johnson and Bennett knew two dozen family
members, friends and neighbors, a number of them under 50, who'd had
cancer. As they tallied their sick and their dead, the two women
wondered, ``What we gonna do?''

\textbf{Black communities like} Grays Ferry shoulder a disproportionate
burden of the nation's pollution --- from foul water in Flint, Mich., to
dangerous chemicals that have poisoned a corridor of Louisiana known as
Cancer Alley --- which scientists and policymakers have known for
decades.
\href{https://www.naacp.org/wp-content/uploads/2017/11/Fumes-Across-the-Fence-Line_NAACP-and-CATF-Study.pdf}{A
2017 report from the N.A.A.C.P. and the Clean Air Task Force provided
more evidence}. It showed that African-Americans are 75 percent more
likely than other Americans to live in so-called fence-line communities,
defined as areas situated near facilities that produce hazardous waste.

A study conducted by the Environmental Protection Agency's National
Center for Environmental Assessment and
\href{https://mail.google.com/mail/u/0/\#inbox/FMfcgxwJXCCfnFTwVGjRcSPbJCghDlCL}{published
in 2018 in the American Journal of Public Health} examined facilities
emitting air pollution along with the racial and economic profiles of
surrounding communities. It found that Black Americans are subjected to
higher levels of air pollution than white Americans --- regardless of
their income level. Black Americans are exposed to 1.5 times as much of
the sooty pollution that comes from burning fossil fuels as the
population at large. This dirty air is associated with lung disease,
including asthma, as well as heart disease, premature death and now
Covid-19.

Philadelphia, which is 44 percent Black, received a warning from
\href{http://www.stateoftheair.org/city-rankings/states/pennsylvania/philadelphia.html}{the
American Lung Association in 2019}: ``If you live in Philadelphia
County, the air you breathe may put your health at risk.'' According to
2016 E.P.A. data, the refinery that looms over Grays Ferry was
responsible for the bulk of toxic air emissions in the city. The E.P.A.
found that the refinery had been out of compliance with the Clean Air
Act nine of the past 12 quarters through 2019 with little recourse. From
2014 to 2019, P.E.S. was fined almost \$650,000 for violating air, water
and waste-disposal rules.

Though Black communities bear disproportionate hardships of the
environmental crisis, they historically have been left out of the
environmental movement.
\href{http://orgs.law.harvard.edu/els/files/2014/02/FullReport_Green2.0_FINALReducedSize.pdf}{A
2018 survey conducted by Dorceta Taylor}, a professor at the University
of Michigan School for Environment and Sustainability, found that white
people made up 85 percent of the staffs and 80 percent of the boards of
2,057 environmental nonprofits. Last year,
\href{https://www.diversegreen.org/wp-content/uploads/2019/06/Green_2.0_Retention_Report.pdf}{a
report released by Green 2.0,} an independent advocacy campaign that
examines the intersection of environmental issues and race, showed that
people of color made up only 20 percent of the staffs of 40
environmental nongovernmental organizations. The face of the
environmental movement is more likely to be someone like Greta Thunberg,
the Swedish teenager who was Time magazine's 2019 person of the year,
than someone like Kilynn Johnson living environmental injustice on the
ground. Protests and movement conferences are filled with a sea of
mostly young white people and generally not Black people whose families
have lived near polluting facilities for generations, their bodies
ravaged by the effects of toxic emissions.

The urgency of this environmental crisis has been hastened by climate
change and has now gathered speed and attention as a result of the
coronavirus pandemic and the current racial-justice movement. The racial
disparities that have exposed Black Americans to a disproportionate
share of air pollution have risen to the surface to lethal effect during
the current pandemic. \href{https://projects.iq.harvard.edu/covid-pm}{A
study of more than 3,000 U.S. counties released in April} but not yet
published shows a statistical connection between death rates from
Covid-19 and long-term exposure to air pollution. The researchers, from
the Harvard T.H. Chan School of Public Health, noted that even a small
increase in particulate matter --- tiny airborne particles emitted from
power plants, industrial facilities and vehicles --- corresponded to a
significant increase in Covid-19 mortality. Each increased microgram of
this kind of pollution per cubic meter of air is associated with an 8
percent increase in death from Covid-19.

The death rate for the city's Black patients is 50 percent higher than
for white patients. ``You can't understand environmental racism without
understanding the legacy and the history of residential segregation,
which created the disinvestment that has happened in communities in
Philadelphia like Grays Ferry for decades,'' says Sharrelle Barber, an
assistant research professor of epidemiology and biostatistics at Drexel
University's Dornsife School of Public Health in Philadelphia.

Image

The Grays Ferry neighborhood in Philadelphia, where residents say a
nearby oil refinery had catastrophic effects on their health, even
before a fire there in 2019.Credit...Hannah Price for The New York Times

``The compounded effect of racism is really showing up in the
interlocking systems of structural inequality operating in this moment
to increase exposure, transmission, severity and the likelihood of death
from Covid-19 in communities like Grays Ferry, which have already
experienced such devastating environmental racism for so many years,''
says Barber, who is the daughter of the Rev. Dr. William Barber, the
civil rights activist, and a national adviser for the Covid-19
health-justice advisory committee of his Poor People's Campaign. ``This
has all been brought to the surface at this moment.''

\textbf{Across the highway} from Grays Ferry, the immense P.E.S.
refinery, with its lattice of rusting pipes, smokestacks streaked with
soot and mammoth holding tanks, swallows up 1,300 acres of land on the
banks of the Schuylkill. It is a city in itself, encircled by a
chain-link fence topped with barbed wire --- nearly the size of Central
Park and Arlington National Cemetery combined. For decades, when the sun
set, the facility looked like its own vast metropolis, lights flickering
throughout the night. The site was first used as a storage facility in
Philadelphia a year after the Civil War ended and began refining oil
shortly after that. By 1891, half the world's lighting fuel and more
than a third of U.S. petroleum exports came from the refinery.

The Industrial Revolution and the invention of cars drove an insatiable
hunger for oil, which became the dominant fuel of the 20th century. As
the refinery continued to be a powerhouse in oil production on the East
Coast and expanded operations, Philadelphia experienced a significant
demographic shift. During the Great Migration, the Black population
exploded with waves of new arrivals from the South, and white people
moved out of the city. The city's African-American community went from
251,000 in 1940 to 376,000 in 1950, and peaked at 654,000 residents in
1970.

In 1934 South Philadelphia was redlined: given a D rating --- the lowest
--- by the Home Owners' Loan Corporation, which outlined the community
in red on maps used to determine loan eligibility. Agents of the loan
group noted ``Negro encroachment in certain neighborhoods.'' The Federal
Housing Administration later relied on these maps, and its own
underwriting manuals pointed to the condition of housing and the race or
ethnicity of residents as characteristics that increased the risk of a
community receiving a low rating from the agency. As a result, lending
institutions issued fewer mortgages in these areas than in other parts
of the city, creating entrenched segregation, disinvestment and decay.
In South Philly, the proximity of residential areas to factories,
including the refinery, most likely contributed to the neighborhood
receiving the lowest grade and a label as ``hazardous,'' making it
difficult for residents to get approved for loans to buy homes.

Public housing filled the void. In 1940 the city completed the Tasker
Street Homes Project, 125 barracks-like buildings with 1,000 units,
taking up 40 acres to the southwest of 30th and Tasker Streets. More
followed: Philadelphia received federal funding in 1949 for more than
20,000 low-income public-housing units. The city built Wilson Park, a
650-unit complex across the highway from P.E.S. in 1953, and continued
to expand. According to the book ``Public Housing, Race and Renewal:
Urban Planning in Philadelphia, 1920-1974,'' by John F. Bauman, from
1956 to 1967 all of this public housing landed in poor or transitional
communities. This included more than a thousand additional units in
South Philadelphia. ``Black leaders accused the {[}housing{]} authority
of warehousing as well as ghettoizing the Black poor,'' Bauman, the
author of several books about urban planning, wrote.

In 1969, when Johnson, the last of nine children, was born, her family
lived in the Tasker Street Homes housing project. Her parents had good,
stable jobs: Troy as a mechanic for SEPTA, the city's
public-transportation system, Elizabeth as a custodian for the school
district. When the couple heard about a good deal on a four-bedroom
rowhouse not far away on Dickinson Street with a basement and a yard,
they decided to make a move. Troy Johnson's brother Robert and his wife
also bought a home nearby. Sylvia Bennett and her husband, who also
lived in the Tasker Street Homes, landed on Dickinson Street as well. At
that time, the neighborhood was less than one-third Black; it is now
majority Black.

The ``hazardous'' label the government stamped onto the Johnsons' and
Bennetts' community 86 years ago now has a different meaning. The legacy
of 150 years of pollution from heavy industry has mounted. Local people
have grown used to the poor air quality. Gloria C. Endres, a lifelong
resident, described the constant cough and runny nose as the ``South
Philly postnasal drip'' in a letter to The South Philly Review, a local
publication. Derek Hixon joked that the South Philadelphia High
basketball team ``always has home-court advantage because opposing
players find it hard to breathe.'' More ominous are the disturbingly
frequent accounts of cancer.

According to data collected by the National Cancer Institute, each year
501 people in every 100,000 in Philadelphia will get cancer, compared
with 449 in the United States and 485 in Pennsylvania. Data from the
E.P.A.'s Toxics Release Inventory shows that contaminants released from
the P.E.S. refinery include benzene, hydrogen cyanide, toluene and other
hazardous chemicals.
\href{https://kleinmanenergy.upenn.edu/paper/beyond-bankruptcy}{An
analysis by the University of Pennsylvania's Kleinman Center for Energy
Policy} notes that the soil and groundwater at the site of P.E.S. have
been contaminated with a number of toxic substances, including benzene,
a known carcinogen.

Image

Sylvia Bennett in Stinger Square Park. She and Kilynn Johnson tracked
the illnesses suffered by their families and neighbors and became active
in a local environmental-justice organization.Credit...Hannah Price for
The New York Times

Despite the data, it's difficult to link individual cases of cancer to
the documented dumping of carcinogenic substances into the air and soil
in the community adjacent to the refinery. But the danger has long been
apparent. ``The refinery has a very long history of environmental
regulation problems and really old technology,'' says Peter DeCarlo, a
former professor at Drexel University who lived less than two miles from
the refinery for eight years and is now an associate professor of
environmental health and engineering at Johns Hopkins University. ``It
sits very close to a densely populated area. If a refinery were trying
to get a permit to operate where it is currently, today, right now, it
would never be given.''

\textbf{Three years after} Kilynn Johnson's diagnosis, she had battled
back from the aftereffects of the cancer and its harsh treatments ---
including the loss of her hair, energy, mobility and fragments of her
memory --- and was in remission. Now she was determined to understand
how the refinery across the highway might have contributed to what
happened to her. In January 2019, Sylvia Bennett persuaded Johnson to
overcome her shyness and attend a meeting of Philly Thrive, a small but
energetic local environmental-justice organization. Co-founded by Alexa
Ross, a young organizer who moved to Philadelphia in 2013 after
graduating from Swarthmore College, the group was determined to rally
residents and make a more explicit connection between P.E.S. and the
negative health impacts in the surrounding community.

Johnson stayed close to Bennett as they walked into a brightly lit room
in a co-working space near the University of Pennsylvania for Philly
Thrive's first monthly gathering of the year. She looked around at the
swell of people of all ages, most of them Black and some of whom she
knew from the neighborhood. Carol White, a retired mental-health worker
who lives in Wilson Park, the South Philadelphia public-housing complex
adjacent to I-76 and P.E.S., was the first to share. ``I got 13
grandchildren, and most of them have asthma; I have inhalers all over
the house for when they come to visit,'' she said. ``Then I started
thinking about my mother, who had cancer. I looked over at the refinery
across the road from my house, and I started thinking, How long do I
have to live?''

Bennett stood up. ``Both my daughters got breast cancer,'' she said.
``They are in remission from the breast cancer, but now one of them has
been diagnosed with blood cancer.'' Tears pooled in her eyes. ``This
refinery, I call it a silent killer.'' She looked down at Johnson. ``You
want to speak?'' Johnson shook her head.

``My eyes were opening,'' Johnson recalled later, ``but I wasn't ready
to speak.'' By the end of the meeting, the Thrivers had decided to focus
on blocking the construction of a new \$60 million plant in southwest
Philly capable of producing 120,000 gallons of liquefied natural gas a
day on city-owned land close to P.E.S. Though accidents at
liquefied-natural-gas plants are infrequent, a 2009 report by the U.S.
Congressional Research Service warned that spills can release
combustible vapor clouds and trigger fires or explosions.

Many of those who attended that January meeting may not have realized
that they were joining a long tradition of on-the-ground environmental
activism. The first stirrings of the Black-led environmental-justice
movement began in the late 1970s as a convergence of a growing interest
in environmental issues and the civil rights and Black-power movements.
Alarmed and angry community members began raising concerns about the
placement of facilities that contaminate the air, water and soil ---
including incinerators, oil refineries, smelters, sewage-treatment
plants, landfills and chemical plants --- near communities of color and,
as in the case of Grays Ferry, placing housing that would be mainly
occupied by Black citizens close to such facilities.

In 1978, a lawyer named Linda McKeever Bullard brought a lawsuit against
the health departments of Houston, Harris County and Texas in federal
court, charging these government agencies, as well as a now-defunct
private waste-management company, with racial discrimination in the
siting of the Whispering Pines municipal landfill in the predominantly
middle-class Black neighborhood of Northwood Manor in suburban Houston.
Her husband, Robert Bullard, was then a young professor of sociology at
Texas Southern University. ``My wife said, `For this lawsuit, I need
somebody who can find out and put on a map where all the landfills,
solid-waste facilities and incinerators are in the city,''' recalls
Bullard, 73, a distinguished professor of urban planning and
environmental policy at T.S.U., who is now regarded as the father of the
environmental-justice movement.

Bullard and his students combed state and city records on paper and
microfiche and walked through neighborhoods using census-tract maps to
locate the waste facilities in the city. They discovered that all five
municipal dumps, six of eight city-operated garbage incinerators and
three of four private landfills were located in Black communities ---
though African-Americans made up only 25 percent of the population at
the time. ``What the data showed was a pattern of racist decisions over
years and years by city officials,'' Bullard says. ``In the case of
Whispering Pines, it was the height of disrespect compounded by the fact
that the landfill was 1,300 feet from a high school in a Black school
district and with at least a half-dozen elementary schools in a two-mile
radius. It gets hot in Houston. How can kids learn if they're smelling
garbage? That's the kind of racism that permeated that particular
case.''

\textbf{In 1978, North Carolina} residents noticed dark streaks along
the shoulders of more than 200 miles of roadway. Over that summer, the
Ward Transformer Company dumped more than 30,000 gallons of oil thick
with polychlorinated biphenyl (PCBs) --- which can cause birth defects,
liver and skin disorders and cancer --- in the middle of the night, in
order to avoid the cost of proper disposal. One of the so-called
midnight dumpers went to prison, along with the head of the company,
leaving state officials and the E.P.A. to decide where to place 60,000
tons of contaminated soil. They chose Warren County, a predominantly
African-American part of the state. The community began to mobilize.

Four years later,
\href{https://timeline.com/warren-county-dumping-race-4d8fe8de06cb}{hundreds
of Warren County residents and environmental and civil rights activists
were arrested} as they rallied to stop construction of the landfill. A
line of protesters lay in the street, blocking dump trucks full of the
toxic soil. A group of mostly women and children clung to each other
while being wrenched apart and dragged into buses by state troopers who
had been summoned to break up the rallies. The evening news featured
video of Black leaders, flanked by highway-patrol officers, marching arm
and arm with the local organizers and singing ``Ain't No Stoppin' Us
Now'' to the tune of the old protest song ``Which Side Are You On?''

The rallies, marches, arrests and media attention weren't enough to stop
the landfill, but they did galvanize a growing movement against
environmental racism, a term coined by the Rev. Dr. Benjamin Chavis, a
leader of the protest in North Carolina. The following year, the U.S.
General Accounting Office examined hazardous-waste-landfill placement
and found that Black residents made up a majority in three of the four
communities with hazardous-waste landfills in the eight Southern states
that make up E.P.A. Region IV.

In 1987, the United Church of Christ Commission for Racial Justice, then
headed by Chavis, issued a report,
\href{http://d3n8a8pro7vhmx.cloudfront.net/unitedchurchofchrist/legacy_url/13567/toxwrace87.pdf}{``Toxic
Wastes and Race in the United States,''} that was the first to examine
race, class and the environment on a national level. The study revealed
that three out of five Black and Hispanic-Americans, or more than 23
million people, resided in communities blighted by toxic-waste sites and
found that while socioeconomic status was an important correlation, race
was the most significant factor.

Bullard continued his research after the Whispering Pines lawsuit in
Houston, finding the same correlation. In his 1990 book, ``Dumping in
Dixie: Race, Class and Environmental Quality,'' using case studies
including Sumter County, Ala., the site of the nation's largest
hazardous-waste landfill, Bullard argued that pollution from solid-waste
facilities, hazardous-waste landfills, toxic-waste dumps and chemical
emissions from industrial facilities was exacting a heavy toll on Black
communities across the country. His book became a bible for the nascent
environmental-justice movement.

In 2007, the United Church of Christ updated its research, this time
with Bullard as a principal author, in ``Toxic Wastes and Race at
Twenty: 1987-2007,'' finding that racial disparities in the location of
toxic-waste facilities were ``greater than previously reported.'' People
of color made up a majority of the population in communities within 1.8
miles of a polluting facility, and race --- not income or property
values --- was the most significant predictor. The following year, a
study by two University of Colorado social scientists published in the
journal Sociological Perspectives found that African-American families
with incomes of \$50,000 to \$60,000 were more likely to live in
environmentally polluted neighborhoods than white households with
incomes below \$10,000.

As more research established such disparities, frustration grew with the
mainstream environmental movement. In March 1990, more than 100
grass-roots activists, almost all of them people of color,
\href{https://www.ejnet.org/ej/swop.pdf}{signed an accusatory letter to
10 of the most prominent environmental groups.} ``Racism is a root cause
of your inaction around addressing environmental problems in our
communities,'' they wrote, demanding that the organizations increase
staffing of people of color to 35 to 40 percent (the demand was not
met). The following year, more than 500 people gathered in Washington,
D.C., for the First National People of Color Environmental Leadership
Summit, dispelling the assumption that Black and brown people are not
interested in or involved with environmental issues.

The federal government was shamed into action. Early in 1990, the
Congressional Black Caucus met with E.P.A. officials to discuss the
polluting of communities of color and why the government agency was not
addressing the needs of their constituents. In November 1992, the E.P.A.
created the Office of Environmental Equity (later changed to
Environmental Justice). In 1994, President Bill Clinton issued an
executive order to address adverse health and environmental conditions
in minority and low-income populations. The government also established
a multimillion-dollar grant program to support grass-roots organizations
working on environmental-justice issues. A local nonprofit in
Spartanburg, S.C., leveraged an initial grant of \$20,000 in 1997 into
\$270 million to clean up and revitalize three neighborhoods near an
operating chemical-fertilizer manufacturing plant, two Superfund sites
and six brownfield sites.

Image

Alexa Ross, co-founder of Philly Thrive, a local environmental-justice
organization.Credit...Hannah Price for The New York Times

The changes at the E.P.A. dovetailed with the growing
environmental-justice movement on the ground. Mustafa Ali, then a young
Black staff member in the Office of Environmental Justice, had a foot in
both worlds. ``It was an exciting time, because there was so much
energy,'' Ali recalls. ``It was a paradigm shift, but it was also tough
back then. There were still folks in senior positions in the
Environmental Protection Agency and other places who believed that the
impacts that were happening in these communities weren't real, that
these folks had to be making this stuff up. They were also uncomfortable
using the federal space to honor the voices and the innovation coming
out of the communities.''

In 2008, Ali was named the associate director of the Office of
Environmental Justice and senior adviser to the E.P.A. administrator on
environmental-justice issues. The E.P.A. was criticized during this time
for not doing enough to combat environmental disparities in communities
of color and the Flint water catastrophe unfolded as well, but Ali and
his colleagues also assisted 1,500 communities with small grants to
address local environmental issues.

When Donald Trump's administration arrived in 2017, his new E.P.A.
administrator, Scott Pruitt, was a climate-change denier and an ally of
the fossil-fuel industry who, as Oklahoma's attorney general, sued the
E.P.A. several times. Pruitt proposed gutting the agency's budget by 25
percent, to just under \$6 billion from \$8 billion. As reported in The
Oregonian newspaper, an internal memo called for dismantling the Office
of Environmental Justice and reducing related funding by 79 percent, to
\$1.5 million from \$6.7 million. Most painful for Ali, the proposed
budget eliminated the small-grants program. ``When I saw them talking
about the elimination of certain air and clean-power-plant programs and
cutting dollars to deal with lead, I knew how it would play out in our
communities,'' he says. ``I knew I couldn't be a part of what was
happening.''

In March 2017, Ali resigned, just short of 25 years at the agency,
forfeiting his full government pension, and now serves as vice president
for environmental justice, climate and community revitalization for the
National Wildlife Federation. His three-page
\href{https://www.documentcloud.org/documents/3514958-Final-Resignation-Letter-for-Administrator.html}{resignation
letter to Pruitt pleaded for the E.P.A.} not to turn its back on
marginalized communities. ``Communities have shared with me over the
past two decades how important the enforcement work at the Agency is in
protecting their often forgotten and overlooked communities,'' he wrote.
``By ensuring that there is equal protection and enforcement in these
communities, E.P.A. plays a significant role in addressing unintended
impacts and improving some of the public health disparities that often
exist from exposure to pollution.''

\textbf{On June 1, 2019,} about 60 Philly Thrive members gathered in
front of P.E.S. as tanker trucks passed in and out of the facility's
gates. For the past four months, the group had attended planning
meetings, spoken at City Hall and circulated petitions opposing the
proposed South Philadelphia gas plant. Kilynn Johnson joined Alexa Ross,
Sylvia Bennett, Carol White and others to distribute hundreds of fliers
throughout Grays Ferry for the protest they organized for that day, two
weeks before the City Council vote.

Holding a sign with her mother's name on it, Johnson stepped forward to
the front of the assembly. Like the others, she wore Philly Thrive's
signature T-shirt, bright yellow with two sunflowers bursting with
kaleidoscopic colors. Since attending that first Thrive meeting in
January, she had gone to more environmental-justice gatherings,
participated in a public-speaking workshop and finally got up her nerve
to address those assembled at the rally --- her first time ever speaking
before a crowd. She looked over at Bennett, wearing sunglasses and
holding a sign with her daughter Wanda's name on it, who nodded. ``Many
of you may not know about the dangers of the oil refinery, with so many
illnesses caused by air pollution,'' Johnson began, reading haltingly
from a sheath of papers that she held before her face. ``I was
nonchalant about the refinery, but then Alexa was mentioning things like
asthma. And I'm like, `Check.' And cancer, and I'm like, `Check,''' she
continued. ``That made me more aware of how the refinery is making our
people not just sick --- but killing our communities all over a
dollar.''

She asked the crowd to join her in a chant: ``We're fired up! Can't take
it no more!'' As the sun got hotter and some of the older folks began to
wilt, the protesters marched behind a banner that read ``Philly Thrive
Right to Breathe'' as the refinery's security guards eyed them. There
was little coverage of the protest. ``Where were the TV crews?'' Bennett
asked after the rally. ``What do we have to do to get anybody to pay
attention? Why doesn't anybody care?''

In mid-June, the Philadelphia City Council voted 13 to 4 in favor of
developing the gas plant. But even as Johnson, Bennett and the other
Philly Thrivers nursed their defeat in the days afterward and feared for
the future, a more imminent danger was at hand.

Image

Irene Russell, the president of the nonprofit group Friends of Stinger
Square. She tapes up memorials for deceased residents.Credit...Hannah
Price for The New York Times

Just one week later, on June 21, Johnson was startled awake when she
felt her bed move. She bolted upright, wrestled herself from a snarl of
sheets, reached for her glasses and tried to figure out what was going
on. It wasn't just her bed shaking, but her entire house. Johnson
grabbed hold of the edge of her mattress, dropped her head, closed her
eyes and prayed. ``Father, Lord, God,'' she said out loud. ``Protect my
family, watch over my neighbors. Please help us.''

Johnson's prayers were interrupted by the phone. On the other end of the
line, she heard the panicked voice of her daughter Michelle, who lived
about a mile and a half away in Southwest Philly. Her house was shaking,
too, and she had lost power and was sitting in the dark holding tight to
her two young children. ``Mommy, turn on the news,'' she said, her voice
trembling. ``It's the refinery.''

Johnson would later learn that at 4 that morning, a corroded pipe
fitting appeared to have given way, triggering a series of explosions
that set off a three-alarm inferno that would burn for more than a full
day. A smaller fire erupted 11 days earlier at the refinery, but the
heat this time was so intense that the National Weather Service was able
to capture it on satellite from space, using infrared imagery. Large
chunks of debris tumbled through the air, landing heavily on city
streets as sirens sounded throughout Grays Ferry and the city's
emergency-management department issued a shelter-in-place order for
residents living near the refinery.

By 7 a.m., even with the refinery still engulfed in flames and clouds of
smoke belching into the atmosphere, the shelter-in-place order was
lifted. A few hours later, James Garrow, a spokesman from the
Philadelphia Department of Public Health, released a statement assuring
local residents that the fire posed no ``immediate danger.'' Johnson,
with that asthma diagnosis 40 years earlier, felt skeptical. She made
certain all of her windows were closed to block out the rank odor that
would hang in the air for weeks. And then, as Johnson traded calls with
family and neighbors, watched the news and checked Facebook for updates,
her breathing became more labored. By early afternoon she was
lightheaded and struggling to catch her breath.

An hour later, as she sat on an examining table at Penn's University
Hospital with a breathing mask strapped to her face, she thought of the
thick black smoke that city officials insisted was safe to inhale and
remembered the noxious odor that had singed her nostrils and irritated
her airways. With oxygen filling her lungs through a machine, she
thought about how often she had been in hospital rooms like these,
suffering from asthma throughout her childhood and the rare cancer that
was diagnosed three and a half years earlier. ``I was tired of them
saying that the refinery didn't affect people,'' Johnson says, ``that it
was doing no harm.''

\textbf{Four days after} the explosion, some 100 Thrivers gathered at a
small playground a few blocks from P.E.S. This time, the media was out
in full force, jostling to get comments from members of Philly Thrive
about the blast and fire. ``The chemicals that they use, it's, like,
really killing us,'' Johnson told a reporter from a local radio station.
``It's killing us slowly. That's what it's doing.''

As the Thrivers marched toward the refinery, they were met by a dozen
police officers lined up in front of 17 police cars parked before the
gates of P.E.S., where hard-hatted employees watched behind the metal
fence as the protesters advanced. Chanting ``What do we want? Clean
air!'' the Thrivers held up traffic for a half mile in either direction.
Behind them, a large billboard sponsored by the local chapter of the
United Steelworkers, the union representing the plant workers, rising
over the highway, reminding drivers and neighbors that ``Healthy
communities need good jobs!''

After months attending Philly Thrive meetings and learning about the
environmental dangers created by the refinery, after the explosion and
her emergency trip to the hospital, Johnson had changed. The painful
death of her first cousin Sharon, a longtime Grays Ferry resident, in
late spring from pancreatic cancer was the final blow. This time
Johnson, a yellow flower entwined in her braids, didn't speak from the
edge of the crowd, but stepped straight into the middle. ``I was born in
South Philadelphia, a few blocks over,'' she said firmly. ``The
pollution and chemicals, they have been here 150 years. I have been here
for a half century. I don't know how long asthma has been in my system,
but in 2016 the doctor didn't even know if I was going to make it or
not. They told my family to pray.''

Image

Irene Russell maintains a repository of memorial programs from the
funeral services of local residents, including many, like her brother,
who died prematurely of cancer.Credit...Hannah Price for The New York
Times

Turning in a circle to face all sides of the crowd, she continued, her
voice rising: ``P.E.S. must go. They are taking our people away. By
droves. By droves!'' Johnson seemed to have shed any hint of the social
anxiety that had been with her all her life. ``I used to be a real quiet
person, until I ran into Philly Thrive. Guess what? My voice will carry
for the person down the street, for the person up the street. For the
baby that cannot speak, for the senior citizen who cannot speak. My
voice will travel. They will know my name and they will know my voice.''
As she spoke, the crowd snapped their fingers, clapped and showered her
with amens.

In late June, the chief executive of P.E.S., Mark Smith, announced that
the explosion and fire made it impossible to keep the plant open. A
month later, P.E.S. filed for bankruptcy. The company would receive an
advance of up to \$65 million in bankruptcy financing in order to wind
down current operations and potentially access \$1.25 billion in
insurance coverage. The goal, according to a statement from P.E.S., was
to rebuild the refinery's fire-damaged infrastructure in order to
position it for a sale and restart in the oil-refining business.
(Representatives for the company did not respond to repeated requests
for comment.) The city of Philadelphia formed an advisory group of
environmental experts, business leaders, city officials, organized labor
and community members who would hold six meetings to address the fallout
from the P.E.S. fire, collect information about the future of the
company and the site and hear public comments.

After the refinery closed, some 1,000 employees were dismissed without
severance pay or extended health benefits; P.E.S. executives received
\$4.5 million in retention bonuses. At the third meeting of the city's
advisory group in late August, convened to address labor issues, Philly
Thrive members found themselves outnumbered by recently laid-off P.E.S.
workers, mainly white men, some in tears, pleading for P.E.S. to remain
in business. At the meeting, it was clear the distressed and angry
former refinery employees didn't know the mostly Black Thrivers though
they had coexisted in the same corner of the city, breathing the same
dirty air at work and at home, for years and years. When Sylvia Bennett
stood at the microphone and told the advisory panel about her daughter
Wanda, who was now in so much pain from cancer treatments that she could
no longer walk, one worker shouted, ``If you don't like the refinery,
then move!''

Bennett was hurt deeply by the hostility, but she also recognized that
P.E.S. had caused harm to its workers, too. ``We are not against workers
or against workers having a job to support their families,'' she said.
``What we want is the air cleaned up so we can \emph{all} breathe.''

\textbf{The community of} Grays Ferry, still more Southern than
Northern, is full of people bound together by history, memories,
struggle, dreams, blood, love and death. These residents may have landed
there because of options limited by the structural discrimination
created by redlining. But even as they pray for the sick and count their
dead, they have stayed. The homes that their parents bought or that they
bought, and the families they raised in them, all this is their legacy.

That legacy also remains in their bodies.

\href{https://www.phila.gov/media/20191202091559/refineryreport12219.pdf}{In
a report last October,} the Chemical Safety and Hazard Investigation
Board noted that the P.E.S. explosion released more than 5,000 pounds of
hydrofluoric acid. Ingesting even a thimbleful can prove deadly, and
when discharged into the air in gas form, the chemical can irritate the
eyes, nose and respiratory tract at low concentrations and cause
irregular heartbeat and lung complications at higher levels.

In January 2020,
\href{https://www.nbcnews.com/science/environment/massive-oil-refinery-leaks-toxic-chemical-middle-philadelphia-n1115336}{an
investigation by the environmental and energy-reporting organization
E\&E News, NBC and American University's Investigative Reporting
Workshop} revealed that even before the June explosion, P.E.S. had
released the cancer-causing chemical benzene into the air at 21 times
the federal limit, though the city failed to let the public know. The
report said: ``The fenceline benzene emission data, which E.P.A. began
posting early last year, shows the refinery exceeded the benzene
emissions limit for all but 12 weeks from the end of January 2018 to
late September 2019 --- an 86-week span. That may have exposed thousands
of Philadelphians to troubling levels of benzene, including children
like those who often play in the streets of Grays Ferry.''

In February, a U.S. Bankruptcy Court approved the sale of P.E.S. to
Chicago-based Hilco Redevelopment Partners for \$252 million (the final
sale was for \$225.5 million). The Trump administration made one last
lobbying effort to restart P.E.S.'s oil-refining business. ``Look, these
are great jobs for Philly,'' Peter Navarro, the president's director of
the office of trade and manufacturing policy, told The Philadelphia
Inquirer in January. ``This is a way to advance the energy-policy
agenda, the economic-policy agenda and the national-security agenda. So
we'd love to see that remain as a refinery.''

The community was concerned. But Hilco announced plans to demolish the
refinery, clean up the site and rebuild the property as a mixed-use
industrial park. ``This will be welcome environmental progress for
neighborhoods that have suffered from the effects of the refinery,''
said Roberto Perez, the chief executive of Hilco Redevelopment Partners,
``and an exciting new chapter for Philadelphia.'' The news, however
welcome, could not erase 150 years of pollution or the fears of the
toxins that remain.

The death of P.E.S. cannot bring back Grays Ferry's dead, not those from
cancer and not the 54 residents who lived in Grays Ferry's ZIP codes who
have died of Covid-19, a virus known to prey on those exposed to
long-term air pollution.

Irene Russell, 68, who has lived in Grays Ferry all her life, helps the
community remember. She was raised on South 32nd Street and now lives a
few blocks away on South Napa Street in a rowhouse she bought in 1980.
On 50 white boards, Russell, the president of the nonprofit group
Friends of Stinger Square, has taped memorial programs from the
community's funeral services, six or seven per board. If she doesn't
have a program, she attaches a photograph. Deceased residents, sometimes
their younger selves, smile from the yellowed programs, encircled in
roses or floating in a sea of blue sky and fluffy clouds. They wear
military uniforms, towering hats, graduation caps and gowns or simple
Sunday best.

This spring, Russell rested a lime green fingernail on the face of
George Scott, who died in 2010 at age 57. ``That's my brother,'' she
said softly. ``He died of liver cancer; left behind eight kids.''
Russell's sister Sandy also died of cancer, at age 42. Her son George,
named after her brother, developed lymphoma in his late 20s and
survived. Russell shuffled through the boards until she found Sharon,
Kilynn Johnson's cousin, whose program she taped to a board a few months
earlier. Next to the words ``it is with deep sorrow, that we regret to
inform you of the passing of our beloved Sharon E. Johnson''
superimposed over a rose, Sharon looked off to the side, her lips pursed
as if she were whistling a song.

Russell found out she had uterine cancer in 2018 and had a hysterectomy
in January 2019. Last September her doctor discovered cancer in her
lungs. She tried hard to keep the boards, stored in plastic garbage bags
in her Stinger Square office, up to date, but the pile of memorials
stacked on top of her computer, waiting to be attached, has grown larger
since the coronavirus struck in February. ``Between the cancer and the
Covid, the loss is crazy,'' Russell, who recently finished chemotherapy
treatments for her lung cancer, said in June. ``It's just a lot of
people who have died. It's been kind of devastating, but all we can do
is just keep living. And keep remembering.''

Advertisement

\protect\hyperlink{after-bottom}{Continue reading the main story}

\hypertarget{site-index}{%
\subsection{Site Index}\label{site-index}}

\hypertarget{site-information-navigation}{%
\subsection{Site Information
Navigation}\label{site-information-navigation}}

\begin{itemize}
\tightlist
\item
  \href{https://help.nytimes.com/hc/en-us/articles/115014792127-Copyright-notice}{©~2020~The
  New York Times Company}
\end{itemize}

\begin{itemize}
\tightlist
\item
  \href{https://www.nytco.com/}{NYTCo}
\item
  \href{https://help.nytimes.com/hc/en-us/articles/115015385887-Contact-Us}{Contact
  Us}
\item
  \href{https://www.nytco.com/careers/}{Work with us}
\item
  \href{https://nytmediakit.com/}{Advertise}
\item
  \href{http://www.tbrandstudio.com/}{T Brand Studio}
\item
  \href{https://www.nytimes.com/privacy/cookie-policy\#how-do-i-manage-trackers}{Your
  Ad Choices}
\item
  \href{https://www.nytimes.com/privacy}{Privacy}
\item
  \href{https://help.nytimes.com/hc/en-us/articles/115014893428-Terms-of-service}{Terms
  of Service}
\item
  \href{https://help.nytimes.com/hc/en-us/articles/115014893968-Terms-of-sale}{Terms
  of Sale}
\item
  \href{https://spiderbites.nytimes.com}{Site Map}
\item
  \href{https://help.nytimes.com/hc/en-us}{Help}
\item
  \href{https://www.nytimes.com/subscription?campaignId=37WXW}{Subscriptions}
\end{itemize}
