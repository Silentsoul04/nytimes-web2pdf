Sections

SEARCH

\protect\hyperlink{site-content}{Skip to
content}\protect\hyperlink{site-index}{Skip to site index}

\href{https://www.nytimes.com/section/realestate}{Real Estate}

\href{https://myaccount.nytimes.com/auth/login?response_type=cookie\&client_id=vi}{}

\href{https://www.nytimes.com/section/todayspaper}{Today's Paper}

\href{/section/realestate}{Real Estate}\textbar{}Making a Move During
the Pandemic? Not So Fast

\url{https://nyti.ms/3hDO6YI}

\begin{itemize}
\item
\item
\item
\item
\item
\item
\end{itemize}

\href{https://www.nytimes.com/news-event/coronavirus?action=click\&pgtype=Article\&state=default\&region=TOP_BANNER\&context=storylines_menu}{The
Coronavirus Outbreak}

\begin{itemize}
\tightlist
\item
  live\href{https://www.nytimes.com/2020/08/01/world/coronavirus-covid-19.html?action=click\&pgtype=Article\&state=default\&region=TOP_BANNER\&context=storylines_menu}{Latest
  Updates}
\item
  \href{https://www.nytimes.com/interactive/2020/us/coronavirus-us-cases.html?action=click\&pgtype=Article\&state=default\&region=TOP_BANNER\&context=storylines_menu}{Maps
  and Cases}
\item
  \href{https://www.nytimes.com/interactive/2020/science/coronavirus-vaccine-tracker.html?action=click\&pgtype=Article\&state=default\&region=TOP_BANNER\&context=storylines_menu}{Vaccine
  Tracker}
\item
  \href{https://www.nytimes.com/interactive/2020/07/29/us/schools-reopening-coronavirus.html?action=click\&pgtype=Article\&state=default\&region=TOP_BANNER\&context=storylines_menu}{What
  School May Look Like}
\item
  \href{https://www.nytimes.com/live/2020/07/31/business/stock-market-today-coronavirus?action=click\&pgtype=Article\&state=default\&region=TOP_BANNER\&context=storylines_menu}{Economy}
\end{itemize}

Advertisement

\protect\hyperlink{after-top}{Continue reading the main story}

Supported by

\protect\hyperlink{after-sponsor}{Continue reading the main story}

Sheltering

\hypertarget{making-a-move-during-the-pandemic-not-so-fast}{%
\section{Making a Move During the Pandemic? Not So
Fast}\label{making-a-move-during-the-pandemic-not-so-fast}}

As New York cautiously reopens in the shadow of a potential second wave
of infections, home buyers and sellers are racing to complete the almost
impossible task of closing on time and safely moving.

\includegraphics{https://static01.nyt.com/images/2020/07/28/realestate/28stuck1/28stuck1-articleLarge-v2.jpg?quality=75\&auto=webp\&disable=upscale}

By Tripp Whetsell

\begin{itemize}
\item
  July 28, 2020
\item
  \begin{itemize}
  \item
  \item
  \item
  \item
  \item
  \item
  \end{itemize}
\end{itemize}

Last summer, Steve and Ellen Resnick followed their daughter and
son-in-law from the Harlem condominium where both families lived to a
new life in New Jersey. That was the plan, anyway.

The kids were moving to a house in Glen Rock, N.J., so the couple aimed
to upgrade to a three-bedroom in Horizon House, a luxury co-op in nearby
Fort Lee. To sell their Harlem apartment, the Resnicks hired Josh
Lieberman, a broker at Douglas Elliman who had recently handled their
daughter's condo. He listed theirs for \$1.35 million and, after a few
false starts, sold the unit in early March for the original listing
price.

As the coronavirus began its assault on the East Coast, the Resnicks
forged ahead. With an expected closing date on both places in late April
and a move by mid-May, they began packing up. That's as far as they got.

After Gov. Andrew M. Cuomo issued the New York Pause order on March 20,
closing all nonessential businesses, both of the Resnicks' closings were
delayed indefinitely. Neither building was allowing anyone to move in or
out, even though moving companies were considered an essential business.

``Nobody was telling us anything and we didn't know what was happening
from one day to the next,'' said Ms. Resnick, who is in her mid-60s.
``Our one saving grace was that the moving company had dropped off some
boxes before the shutdown so we could start getting ready, but then we
had no place to go.''

\includegraphics{https://static01.nyt.com/images/2020/07/28/realestate/28stuck2/merlin_174134952_36600900-cb03-4b70-a8dc-21dd9805b435-articleLarge.jpg?quality=75\&auto=webp\&disable=upscale}

The Resnicks eventually arranged a virtual closing on the Harlem condo
and were able to lease it back while they waited for their co-op in Fort
Lee to close, which it did June 29. They moved the next day.

\hypertarget{latest-updates-global-coronavirus-outbreak}{%
\section{\texorpdfstring{\href{https://www.nytimes.com/2020/08/01/world/coronavirus-covid-19.html?action=click\&pgtype=Article\&state=default\&region=MAIN_CONTENT_1\&context=storylines_live_updates}{Latest
Updates: Global Coronavirus
Outbreak}}{Latest Updates: Global Coronavirus Outbreak}}\label{latest-updates-global-coronavirus-outbreak}}

Updated 2020-08-02T07:42:09.613Z

\begin{itemize}
\tightlist
\item
  \href{https://www.nytimes.com/2020/08/01/world/coronavirus-covid-19.html?action=click\&pgtype=Article\&state=default\&region=MAIN_CONTENT_1\&context=storylines_live_updates\#link-34047410}{The
  U.S. reels as July cases more than double the total of any other
  month.}
\item
  \href{https://www.nytimes.com/2020/08/01/world/coronavirus-covid-19.html?action=click\&pgtype=Article\&state=default\&region=MAIN_CONTENT_1\&context=storylines_live_updates\#link-780ec966}{Top
  U.S. officials work to break an impasse over the federal jobless
  benefit.}
\item
  \href{https://www.nytimes.com/2020/08/01/world/coronavirus-covid-19.html?action=click\&pgtype=Article\&state=default\&region=MAIN_CONTENT_1\&context=storylines_live_updates\#link-2bc8948}{Its
  outbreak untamed, Melbourne goes into even greater lockdown.}
\end{itemize}

\href{https://www.nytimes.com/2020/08/01/world/coronavirus-covid-19.html?action=click\&pgtype=Article\&state=default\&region=MAIN_CONTENT_1\&context=storylines_live_updates}{See
more updates}

More live coverage:
\href{https://www.nytimes.com/live/2020/07/31/business/stock-market-today-coronavirus?action=click\&pgtype=Article\&state=default\&region=MAIN_CONTENT_1\&context=storylines_live_updates}{Markets}

The Resnicks' plight has been typical in this year of chaos. ``It's had
a domino effect on everyone,'' said Michael J. Romer, a Manhattan real
estate attorney. ``People weren't looking to close on an apartment they
couldn't move into, and then a lot of sellers couldn't move out
either.''

As New York cautiously reopens in the shadow of a potential second wave
of infections, thousands of buyers and sellers are racing to complete
the almost impossible task of closing on time and safely moving ---
vexed by a patchwork of rules and guidelines handed down by local
governments and the boards of the buildings themselves.

Paul J. Herman, president of Brown Harris Stevens Residential
Management, said that many of the city and state's recommended best
practices for reopening have been confusing, especially as they relate
to apartment buildings. (The city
\href{https://www.nytimes.com/2020/07/20/nyregion/nyc-phase-4-reopening-coronavirus.html}{entered
the fourth and final phase} of its reopening plan on July 20, though
many indoor activities remain prohibited.)

``It's been tough throughout the various phases because the language
about what the specific requirements are coming out of the governor's
office and other state and city agencies hasn't been that clear and
we've had to interpret a lot of it ourselves,'' Mr. Herman said. ``For
example, we know what to do with the movers, but we still don't know the
definition of what a visitor is, whether it's a nanny, a friend or a
delivery person.''

For those dealing with co-ops and their infamously fastidious boards,
what was already an invasive and tedious process has become an expanding
maze of logistical hurdles.

``Each building has been affected differently by the pandemic and no
single approach is likely going to work,'' said attorney Neil B.
Garfinkel, broker counsel to the Real Estate Board of New York, which
issued its own set of guidelines in April. ``In my view, boards have the
right to set their own policies as long as they are acting with
uniformity and in the best interest of shareholders.''

Lailani Moody had just begun moving out of her Upper East Side co-op to
a Westchester retirement community when the pandemic arrived and moves
were stopped, forcing her to leave behind clothes and furniture.

Ms. Moody, 77, a former accountant, decided in December to sell her
co-op studio on 72nd Street, where she had lived since 1997. Despite an
already sluggish residential market, it sold briskly. ``We already had a
signed contract, and everything seemed fine,'' said Ms. Moody's broker,
Harriet Norris of Douglas Elliman.

But the deal got held up when the co-op board demanded additional
financial information from the buyer. Ms. Moody had hoped to stay in the
studio part-time while she waited for the issue to be resolved, but with
her retirement community under quarantine, she wasn't able to return to
Manhattan, nor could she complete the move because her old co-op had
banned all outsiders, which meant no moving companies, cleaning services
or visitors.

As a result, she was saddled with double maintenance fees on top of
moving costs, and many of her possessions were held hostage. ``The major
hurdle was that I had to find someone who could finish packing and clean
up, but they weren't allowing anybody in,'' Ms. Moody said.

Finally, in mid-June, she was permitted to retrieve the rest of her
belongings, and closed virtually on the sale of her co-op a month later
--- a process that Ms. Norris, her broker, said ``was like trying to
push wet spaghetti.''

``What I've found the most frustrating is that I've had several deals in
contract, whether it's buyers wanting to buy or sellers trying to sell
and get their money, but we've run into issues almost every step of the
way,'' Ms. Norris said. ``I've also had instances where managing agents
were trying to get the boards to do the right thing and be
accommodating.''

Image

Lailani Moody on the grounds of her Westchester retirement community.
After selling her Upper East Side co-op,~Ms. Moody was unable to
complete the move because the co-op had banned all outsiders, which
meant no moving companies or cleaning services.Credit...Karsten Moran
for The New York Times

Other co-op sellers have told horror stories about wading through
mounting red tape. Gill Chowdhury, an agent with Warburg Realty,
mentioned one client who got an offer in January for \$790,000 on her
Gramercy co-op. As the economy soured, the co-op requested a second
review of the buyer's finances. Finally, after being approved but
prohibited from moving in, the buyer stalled on the closing as he tried
to renegotiate the sale price, something Mr. Chowdhury said ``wasn't as
surprising as it was shocking, given the circumstances.''

Mr. Chowdhury's client said she would never buy another co-op again.

For their part, co-op boards and managing agents are struggling to
remain accommodating to buyers and sellers as coronavirus cases surge
around the country.

\href{https://www.nytimes.com/news-event/coronavirus?action=click\&pgtype=Article\&state=default\&region=MAIN_CONTENT_3\&context=storylines_faq}{}

\hypertarget{the-coronavirus-outbreak-}{%
\subsubsection{The Coronavirus Outbreak
›}\label{the-coronavirus-outbreak-}}

\hypertarget{frequently-asked-questions}{%
\paragraph{Frequently Asked
Questions}\label{frequently-asked-questions}}

Updated July 27, 2020

\begin{itemize}
\item ~
  \hypertarget{should-i-refinance-my-mortgage}{%
  \paragraph{Should I refinance my
  mortgage?}\label{should-i-refinance-my-mortgage}}

  \begin{itemize}
  \tightlist
  \item
    \href{https://www.nytimes.com/article/coronavirus-money-unemployment.html?action=click\&pgtype=Article\&state=default\&region=MAIN_CONTENT_3\&context=storylines_faq}{It
    could be a good idea,} because mortgage rates have
    \href{https://www.nytimes.com/2020/07/16/business/mortgage-rates-below-3-percent.html?action=click\&pgtype=Article\&state=default\&region=MAIN_CONTENT_3\&context=storylines_faq}{never
    been lower.} Refinancing requests have pushed mortgage applications
    to some of the highest levels since 2008, so be prepared to get in
    line. But defaults are also up, so if you're thinking about buying a
    home, be aware that some lenders have tightened their standards.
  \end{itemize}
\item ~
  \hypertarget{what-is-school-going-to-look-like-in-september}{%
  \paragraph{What is school going to look like in
  September?}\label{what-is-school-going-to-look-like-in-september}}

  \begin{itemize}
  \tightlist
  \item
    It is unlikely that many schools will return to a normal schedule
    this fall, requiring the grind of
    \href{https://www.nytimes.com/2020/06/05/us/coronavirus-education-lost-learning.html?action=click\&pgtype=Article\&state=default\&region=MAIN_CONTENT_3\&context=storylines_faq}{online
    learning},
    \href{https://www.nytimes.com/2020/05/29/us/coronavirus-child-care-centers.html?action=click\&pgtype=Article\&state=default\&region=MAIN_CONTENT_3\&context=storylines_faq}{makeshift
    child care} and
    \href{https://www.nytimes.com/2020/06/03/business/economy/coronavirus-working-women.html?action=click\&pgtype=Article\&state=default\&region=MAIN_CONTENT_3\&context=storylines_faq}{stunted
    workdays} to continue. California's two largest public school
    districts --- Los Angeles and San Diego --- said on July 13, that
    \href{https://www.nytimes.com/2020/07/13/us/lausd-san-diego-school-reopening.html?action=click\&pgtype=Article\&state=default\&region=MAIN_CONTENT_3\&context=storylines_faq}{instruction
    will be remote-only in the fall}, citing concerns that surging
    coronavirus infections in their areas pose too dire a risk for
    students and teachers. Together, the two districts enroll some
    825,000 students. They are the largest in the country so far to
    abandon plans for even a partial physical return to classrooms when
    they reopen in August. For other districts, the solution won't be an
    all-or-nothing approach.
    \href{https://bioethics.jhu.edu/research-and-outreach/projects/eschool-initiative/school-policy-tracker/}{Many
    systems}, including the nation's largest, New York City, are
    devising
    \href{https://www.nytimes.com/2020/06/26/us/coronavirus-schools-reopen-fall.html?action=click\&pgtype=Article\&state=default\&region=MAIN_CONTENT_3\&context=storylines_faq}{hybrid
    plans} that involve spending some days in classrooms and other days
    online. There's no national policy on this yet, so check with your
    municipal school system regularly to see what is happening in your
    community.
  \end{itemize}
\item ~
  \hypertarget{is-the-coronavirus-airborne}{%
  \paragraph{Is the coronavirus
  airborne?}\label{is-the-coronavirus-airborne}}

  \begin{itemize}
  \tightlist
  \item
    The coronavirus
    \href{https://www.nytimes.com/2020/07/04/health/239-experts-with-one-big-claim-the-coronavirus-is-airborne.html?action=click\&pgtype=Article\&state=default\&region=MAIN_CONTENT_3\&context=storylines_faq}{can
    stay aloft for hours in tiny droplets in stagnant air}, infecting
    people as they inhale, mounting scientific evidence suggests. This
    risk is highest in crowded indoor spaces with poor ventilation, and
    may help explain super-spreading events reported in meatpacking
    plants, churches and restaurants.
    \href{https://www.nytimes.com/2020/07/06/health/coronavirus-airborne-aerosols.html?action=click\&pgtype=Article\&state=default\&region=MAIN_CONTENT_3\&context=storylines_faq}{It's
    unclear how often the virus is spread} via these tiny droplets, or
    aerosols, compared with larger droplets that are expelled when a
    sick person coughs or sneezes, or transmitted through contact with
    contaminated surfaces, said Linsey Marr, an aerosol expert at
    Virginia Tech. Aerosols are released even when a person without
    symptoms exhales, talks or sings, according to Dr. Marr and more
    than 200 other experts, who
    \href{https://academic.oup.com/cid/article/doi/10.1093/cid/ciaa939/5867798}{have
    outlined the evidence in an open letter to the World Health
    Organization}.
  \end{itemize}
\item ~
  \hypertarget{what-are-the-symptoms-of-coronavirus}{%
  \paragraph{What are the symptoms of
  coronavirus?}\label{what-are-the-symptoms-of-coronavirus}}

  \begin{itemize}
  \tightlist
  \item
    Common symptoms
    \href{https://www.nytimes.com/article/symptoms-coronavirus.html?action=click\&pgtype=Article\&state=default\&region=MAIN_CONTENT_3\&context=storylines_faq}{include
    fever, a dry cough, fatigue and difficulty breathing or shortness of
    breath.} Some of these symptoms overlap with those of the flu,
    making detection difficult, but runny noses and stuffy sinuses are
    less common.
    \href{https://www.nytimes.com/2020/04/27/health/coronavirus-symptoms-cdc.html?action=click\&pgtype=Article\&state=default\&region=MAIN_CONTENT_3\&context=storylines_faq}{The
    C.D.C. has also} added chills, muscle pain, sore throat, headache
    and a new loss of the sense of taste or smell as symptoms to look
    out for. Most people fall ill five to seven days after exposure, but
    symptoms may appear in as few as two days or as many as 14 days.
  \end{itemize}
\item ~
  \hypertarget{does-asymptomatic-transmission-of-covid-19-happen}{%
  \paragraph{Does asymptomatic transmission of Covid-19
  happen?}\label{does-asymptomatic-transmission-of-covid-19-happen}}

  \begin{itemize}
  \tightlist
  \item
    So far, the evidence seems to show it does. A widely cited
    \href{https://www.nature.com/articles/s41591-020-0869-5}{paper}
    published in April suggests that people are most infectious about
    two days before the onset of coronavirus symptoms and estimated that
    44 percent of new infections were a result of transmission from
    people who were not yet showing symptoms. Recently, a top expert at
    the World Health Organization stated that transmission of the
    coronavirus by people who did not have symptoms was ``very rare,''
    \href{https://www.nytimes.com/2020/06/09/world/coronavirus-updates.html?action=click\&pgtype=Article\&state=default\&region=MAIN_CONTENT_3\&context=storylines_faq\#link-1f302e21}{but
    she later walked back that statement.}
  \end{itemize}
\end{itemize}

``It worries me a lot that there could very possibly be a second wave
here,'' said Daniel J. Wollman, chief executive of property management
firm Gumley Haft, which ``guides boards of cooperatives and condominiums
to become decision-makers,'' according to its website. ``We're trying to
do everything we can to prepare for that by making sure our residents
and staff are safe and stocking up on personal protective equipment.''

Many of the city's more than 6,000 co-ops and condos had already begun
initiating their own protocols in early March. ``Before the governor's
orders came down, we created an internal task force because we didn't
know how this was going to play itself out,'' said Desi Ndreu, chief
operating officer of Charles H. Greenthal \& Co., which runs more than
200 co-ops, condos and rentals around the city.

One of the buildings Ms. Ndreu personally manages is the Victoria, a
465-unit co-op in the Flatiron district. Longtime board member Corinne
Arnold said they were among the first to require masks in elevators and
temperature checks for visitors after the pandemic started.
``Particularly in the beginning, we were all very hands-on-deck,'' she
said. ``I feel extremely confident in our policies and how they've
evolved.''

Such practices, including prohibiting open houses, are likely to remain
the norm at most buildings until a vaccine or another widely accepted
treatment is available, said Barbara Fox of Fox Residential, which
mainly does business in Manhattan and Brooklyn. ``Advancements in
technology have allowed us to facilitate negotiations, closings and open
houses,'' Ms. Fox said. ``While not a substitute for the in-person
experience, we are able to bring property showings into the virtual
space as we collectively navigate the new norm.''

Regardless of their size, most buildings are limiting the number of
moves they allow per day, often to just one, along with assigning
separate crews in the apartment and the basement to minimize traffic.
Most are also requiring buyers and sellers to pay for a professionally
licensed cleaning service, while some management companies, like Gumley
Haft, are requiring moving companies to provide Masonite floor coverings
in halls and elevators.

``For us, it's all been about being proactive about what we're doing and
explaining why we need to be that way,'' said Ms. Ndreu.

Steve Wagner, a Manhattan real estate attorney who represents the
Victoria, said he's seen an increase in moves and closings over the past
few weeks ---~likely a reaction to the possibility of another
coronavirus surge later this year, even if it means jumping through
hoops and selling an apartment at a loss.

``If you have to close down again, you have to close down,'' he said.
``Safety is the main concern. I have no doubt that co-ops and condos
will continue to do whatever they need to do ---~and so will
shareholders.''

Advertisement

\protect\hyperlink{after-bottom}{Continue reading the main story}

\hypertarget{site-index}{%
\subsection{Site Index}\label{site-index}}

\hypertarget{site-information-navigation}{%
\subsection{Site Information
Navigation}\label{site-information-navigation}}

\begin{itemize}
\tightlist
\item
  \href{https://help.nytimes.com/hc/en-us/articles/115014792127-Copyright-notice}{©~2020~The
  New York Times Company}
\end{itemize}

\begin{itemize}
\tightlist
\item
  \href{https://www.nytco.com/}{NYTCo}
\item
  \href{https://help.nytimes.com/hc/en-us/articles/115015385887-Contact-Us}{Contact
  Us}
\item
  \href{https://www.nytco.com/careers/}{Work with us}
\item
  \href{https://nytmediakit.com/}{Advertise}
\item
  \href{http://www.tbrandstudio.com/}{T Brand Studio}
\item
  \href{https://www.nytimes.com/privacy/cookie-policy\#how-do-i-manage-trackers}{Your
  Ad Choices}
\item
  \href{https://www.nytimes.com/privacy}{Privacy}
\item
  \href{https://help.nytimes.com/hc/en-us/articles/115014893428-Terms-of-service}{Terms
  of Service}
\item
  \href{https://help.nytimes.com/hc/en-us/articles/115014893968-Terms-of-sale}{Terms
  of Sale}
\item
  \href{https://spiderbites.nytimes.com}{Site Map}
\item
  \href{https://help.nytimes.com/hc/en-us}{Help}
\item
  \href{https://www.nytimes.com/subscription?campaignId=37WXW}{Subscriptions}
\end{itemize}
