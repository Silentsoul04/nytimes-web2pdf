Sections

SEARCH

\protect\hyperlink{site-content}{Skip to
content}\protect\hyperlink{site-index}{Skip to site index}

\href{https://www.nytimes.com/section/style}{Style}

\href{https://myaccount.nytimes.com/auth/login?response_type=cookie\&client_id=vi}{}

\href{https://www.nytimes.com/section/todayspaper}{Today's Paper}

\href{/section/style}{Style}\textbar{}Can Unsexy Beauty Be Sexy?

\url{https://nyti.ms/2CX9l9f}

\begin{itemize}
\item
\item
\item
\item
\item
\end{itemize}

Advertisement

\protect\hyperlink{after-top}{Continue reading the main story}

Supported by

\protect\hyperlink{after-sponsor}{Continue reading the main story}

skin deep

\hypertarget{can-unsexy-beauty-be-sexy}{%
\section{Can Unsexy Beauty Be Sexy?}\label{can-unsexy-beauty-be-sexy}}

A new guard of beauty brands is using slick packaging and candid
messaging to sell products women were once embarrassed to buy.

\includegraphics{https://static01.nyt.com/images/2020/07/30/fashion/28SKIN-UNSEXY-art/28SKIN-UNSEXY-art-articleLarge.jpg?quality=75\&auto=webp\&disable=upscale}

By Rachel Strugatz

\begin{itemize}
\item
  Published July 28, 2020Updated July 30, 2020
\item
  \begin{itemize}
  \item
  \item
  \item
  \item
  \item
  \end{itemize}
\end{itemize}

Two years ago, when Jules Miller, the co-founder and chief executive of
\href{https://www.thenueco.com/}{the Nue Co.}, a line of vitamin
supplements, tried to get her brand on shelves at Barneys New York, a
buyer at the store was horrified that Ms. Miller wanted to sell products
for bloating and constipation next to high-end skin care and makeup.

It was easy to get behind ingestible beauty: the idea that consuming
capsules or powders could make skin radiant, nails stronger or hair
lustrous. But Barneys, now closed, could not conceive of placing items
promoting gut health, specifically one that claimed to relieve chronic
bloating, on its beauty floor.

``It wasn't always an understanding that retailers had,'' Ms. Miller
said.

Even so, the Nue Co.'s Prebiotic + Probiotic and Debloat Food +
Prebiotic cocktail of digestive enzymes and prebiotics was not a taboo
concept to consumers. Combined, the two products make up almost a third
of the Nue Co.'s sales, according to Ms. Miller.

Barneys, she said, came back a year later and started stocking the line.
``We get it now,'' she said the store told her.

Katie Sturino, the founder of
\href{https://megababebeauty.com/}{Megababe}, has built an entire brand
around products for thigh chafing, breast sweat and melasma mustaches
(skin discoloration from sun exposure on the upper lip). In early days,
Ms. Sturino said she was met with ``giggling'' and ``snickers'' when she
came out with Thigh Rescue, an anti-chafe stick, but now Megababe is
sold in Target and Ulta Beauty stores.

The Nue Co. and Megababe are part of a group of brands that address
unsexy beauty and grooming concerns using slick packaging and candid,
unconventional messaging. These companies are encouraging consumers to
discard the embarrassment or shame we typically feel about butt acne,
dandruff or toe hair.

The key, Ms. Miller and Ms. Sturino believe, is using traditional beauty
brands as a blueprint, at least when it comes to their aesthetic and the
kinds of stores to sell their products. The Nue Co.'s gut health
supplements and Megababe powders that absorb breast sweat (Bust Dust)
give off cool Gen Z vibes. They are not products that remind you of a
doctor's office or GNC.

What sets Ms. Sturino, who is also a plus-size influencer with more than
half a million Instagram followers, apart from some of the biggest names
in beauty is the way she talks about her brand. She approaches Megababe
the same way she does her body: with unabashed positivity, acceptance
and no filter. Ms. Sturino reminds customers that thighs rubbing
together, a breakout on your behind and a post-summer mustache are
normal.

``We had an example of a beauty editor who used our product but wouldn't
write about it because she didn't want to be associated with chafe,''
she said.

\href{https://hellojupiter.com/}{Jupiter}, a new hair-care line started
by Robbie Salter and Ross Goodhart, who call themselves ``lifelong flake
fighters,'' is trying to do something similar. The two are gunning for
Head \& Shoulders' younger customers, armed with the tagline ``Zero
Flakes Given'' and what they describe as a youthful alternative to a
decades-old drugstore aisle product.

Jupiter's Balancing Shampoo contains zinc pyrithione, the active
ingredient in Head \& Shoulders that treats dandruff, and also looks
good in the shower.

``Existing brands have intentionally stigmatized the category,'' Mr.
Goodhart said. ``From our angle, a significant percent of the population
has it, and we say, `Just use our products and don't worry about it.'''

What's going on in these markets is not much different from what
happened with soaps and household cleaning products: taking something
inherently unsexy --- hand soap or all-surface cleaner --- and
repackaging it to appeal to millennials. That is what put
\href{https://methodhome.com/}{Method} soaps on the map. In 2017, Method
was acquired by SC Johnson, the owner of Windex, Scrubbing Bubbles and
Shout.

Soap may be an easier sell for the TikTok generation (and their parents)
than dandruff shampoo, but Kevin Spight, a brand consultant, believes
you can create a multimillion-dollar company around a taboo concept. ­­­

``You need that niche or hero product to carve your space,'' Mr. Spight
said. ``From there, you create your following and your advocates. People
want brands that represent their personal ethos. It's a badge of honor
now.''

\href{https://mybillie.com/}{Billie}, a women's razor line that came out
in 2017, has worked to reduce the stigma associated with women's body
hair, including with ad campaigns with toe hair and a close-up of a
bikini bottom with pubic hair peeking out. In 2018, its Project Body
Hair video amassed millions of views over several months on YouTube and
other platforms.

It took months for Georgina Gooley, a founder of Billie, to figure out
how a razor brand should talk about (and celebrate) body hair. Billie
not only acknowledges that body hair exists, she explained, but also
endorses the belief that shaving is a choice, not an expectation. For
decades, ads for women's razors showed only legs that were a mile long
and completely hairless.

``You couldn't even get a good visual of a product demonstration,'' Ms.
Gooley said. Body hair was so taboo, she said, that commercials didn't
even acknowledge that women had hair.

Decades ago, she said, women sneaked out of bed to put makeup on while
their partners were asleep, pretending that's how they woke up. (Cue
Midge Maisel of ``The Marvelous Mrs. Maisel,'' who waits for her husband
to fall asleep so she can take off her makeup --- only to wake up before
he does so she can apply a fresh face of it.)

``You're going to see that very direct advertising a lot more often
because women have become a lot less embarrassed,'' said Monique
Woodard, the managing director of Cake Ventures, a venture capital firm.

That may be true, but only recently have brands like Billie started to
challenge industry norms. In 2004, Dove's Real Beauty campaign showcased
``real'' women's bodies, and 13 years later Glossier did the same with
Body Hero, but these are exceptions. Much of the beauty industry is
still fueled by marketing that conveys unrealistic physical ideals.

Margaret Hiestand, 33, who works in community relations for the Chicago
White Sox, said that Ms. Sturino is one of a few influencer accounts she
follows on social media because she is plain-spoken about things like
``sweating in weird places.''

``A lot of brands are like: `Here is this beautiful model with this
flawless skin. Please use this product,''' Ms. Hiestand said. ``She does
it differently.''

Ms. Hiestand was referring to videos on Instagram where Ms. Sturino is,
she said, ``borderline naked'' in her bathroom applying Le Tush
clarifying butt mask or ``throwing up a leg'' to apply chafe stick to
her inner thighs.

``There is nothing taboo when it comes to the human body,'' said Dr.
Shereene Idriss, a dermatologist in New York. Melasma and dandruff are
among the most common skin conditions she treats, along with acne. ``I
hope these brands make it mainstream to be human,'' Dr. Idriss said.

Beatrice Dixon, the founder and chief executive of
\href{https://thehoneypot.co/}{the Honey Pot Company}, has no qualms
talking about human issues. Her line makes nothing but what she calls
``vagina products.''

``What people don't want to do or talk about are the things that you
should absolutely be selling,'' said Ms. Dixon, whose brand is sold at
Walmart, Target, CVS and Walgreens. A Sensitive collection of feminine
wash and wipes that ``kiss feeling dry goodbye'' are among the line's
best sellers.

``Infections and odor and all of the things people deem to be these
terrible things, I go out of my way to discuss those things,'' she said.
``Because they're absolutely normal.''

Advertisement

\protect\hyperlink{after-bottom}{Continue reading the main story}

\hypertarget{site-index}{%
\subsection{Site Index}\label{site-index}}

\hypertarget{site-information-navigation}{%
\subsection{Site Information
Navigation}\label{site-information-navigation}}

\begin{itemize}
\tightlist
\item
  \href{https://help.nytimes.com/hc/en-us/articles/115014792127-Copyright-notice}{©~2020~The
  New York Times Company}
\end{itemize}

\begin{itemize}
\tightlist
\item
  \href{https://www.nytco.com/}{NYTCo}
\item
  \href{https://help.nytimes.com/hc/en-us/articles/115015385887-Contact-Us}{Contact
  Us}
\item
  \href{https://www.nytco.com/careers/}{Work with us}
\item
  \href{https://nytmediakit.com/}{Advertise}
\item
  \href{http://www.tbrandstudio.com/}{T Brand Studio}
\item
  \href{https://www.nytimes.com/privacy/cookie-policy\#how-do-i-manage-trackers}{Your
  Ad Choices}
\item
  \href{https://www.nytimes.com/privacy}{Privacy}
\item
  \href{https://help.nytimes.com/hc/en-us/articles/115014893428-Terms-of-service}{Terms
  of Service}
\item
  \href{https://help.nytimes.com/hc/en-us/articles/115014893968-Terms-of-sale}{Terms
  of Sale}
\item
  \href{https://spiderbites.nytimes.com}{Site Map}
\item
  \href{https://help.nytimes.com/hc/en-us}{Help}
\item
  \href{https://www.nytimes.com/subscription?campaignId=37WXW}{Subscriptions}
\end{itemize}
