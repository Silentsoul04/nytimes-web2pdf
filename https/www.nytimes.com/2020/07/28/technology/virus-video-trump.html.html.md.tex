Sections

SEARCH

\protect\hyperlink{site-content}{Skip to
content}\protect\hyperlink{site-index}{Skip to site index}

\href{https://www.nytimes.com/section/technology}{Technology}

\href{https://myaccount.nytimes.com/auth/login?response_type=cookie\&client_id=vi}{}

\href{https://www.nytimes.com/section/todayspaper}{Today's Paper}

\href{/section/technology}{Technology}\textbar{}Misleading Virus Video,
Pushed by the Trumps, Spreads Online

\url{https://nyti.ms/311XFKG}

\begin{itemize}
\item
\item
\item
\item
\item
\end{itemize}

\href{https://www.nytimes.com/news-event/coronavirus?action=click\&pgtype=Article\&state=default\&region=TOP_BANNER\&context=storylines_menu}{The
Coronavirus Outbreak}

\begin{itemize}
\tightlist
\item
  live\href{https://www.nytimes.com/2020/08/04/world/coronavirus-cases.html?action=click\&pgtype=Article\&state=default\&region=TOP_BANNER\&context=storylines_menu}{Latest
  Updates}
\item
  \href{https://www.nytimes.com/interactive/2020/us/coronavirus-us-cases.html?action=click\&pgtype=Article\&state=default\&region=TOP_BANNER\&context=storylines_menu}{Maps
  and Cases}
\item
  \href{https://www.nytimes.com/interactive/2020/science/coronavirus-vaccine-tracker.html?action=click\&pgtype=Article\&state=default\&region=TOP_BANNER\&context=storylines_menu}{Vaccine
  Tracker}
\item
  \href{https://www.nytimes.com/2020/08/02/us/covid-college-reopening.html?action=click\&pgtype=Article\&state=default\&region=TOP_BANNER\&context=storylines_menu}{College
  Reopening}
\item
  \href{https://www.nytimes.com/live/2020/08/04/business/stock-market-today-coronavirus?action=click\&pgtype=Article\&state=default\&region=TOP_BANNER\&context=storylines_menu}{Economy}
\end{itemize}

Advertisement

\protect\hyperlink{after-top}{Continue reading the main story}

Supported by

\protect\hyperlink{after-sponsor}{Continue reading the main story}

\hypertarget{misleading-virus-video-pushed-by-the-trumps-spreads-online}{%
\section{Misleading Virus Video, Pushed by the Trumps, Spreads
Online}\label{misleading-virus-video-pushed-by-the-trumps-spreads-online}}

Social media companies took down the video within hours. But by then, it
had already been viewed tens of millions of times.

\includegraphics{https://static01.nyt.com/images/2020/07/28/business/28virus-disinfo/28virus-disinfo-articleLarge.jpg?quality=75\&auto=webp\&disable=upscale}

By \href{https://www.nytimes.com/by/sheera-frenkel}{Sheera Frenkel} and
\href{https://www.nytimes.com/by/davey-alba}{Davey Alba}

\begin{itemize}
\item
  July 28, 2020
\item
  \begin{itemize}
  \item
  \item
  \item
  \item
  \item
  \end{itemize}
\end{itemize}

In a video posted Monday online, a group of people calling themselves
``America's Frontline Doctors'' and wearing white medical coats spoke
against the backdrop of the Supreme Court in Washington, sharing
misleading claims about the virus, including that hydroxychloroquine was
an effective coronavirus treatment and that masks did not slow the
spread of the virus.

The video did not appear to be anything special. But within six hours,
President Trump and his son Donald Trump Jr. had tweeted versions of it,
and the right-wing news site Breitbart had shared it. It went viral,
shared largely through Facebook groups dedicated to anti-vaccination
movements and conspiracy theories such as QAnon, racking up tens of
millions of views. Multiple versions of the video were uploaded to
YouTube, and links were shared through Twitter.

Facebook, YouTube and
\href{https://www.nytimes.com/2020/08/03/technology/ftc-twitter-privacy-violations.html}{Twitter}
worked feverishly to remove it, but by the time they had, the video had
already become the latest example of
\href{https://www.nytimes.com/2020/05/20/technology/plandemic-movie-youtube-facebook-coronavirus.html}{misinformation
about the virus that has spread widely}.

That was because the video had been designed specifically to appeal to
internet conspiracists and conservatives eager to see the economy
reopen, with a setting and characters to lend authenticity. It showed
that even as social media companies have sped up response time to remove
dangerous virus misinformation within hours of its posting, people have
continued to find new ways around the platforms' safeguards.

``Misinformation about a deadly virus has become political fodder, which
was then spread by many individuals who are trusted by their
constituencies,'' said Lisa Kaplan, founder of Alethea Group, a start-up
that helps fight disinformation. ``If just one person listened to anyone
spreading these falsehoods and they subsequently took an action that
caused others to catch, spread or even die from the virus --- that is
one person too many.''

\hypertarget{latest-updates-economy}{%
\section{\texorpdfstring{\href{https://www.nytimes.com/live/2020/08/04/business/stock-market-today-coronavirus?action=click\&pgtype=Article\&state=default\&region=MAIN_CONTENT_1\&context=storylines_live_updates}{Latest
Updates:
Economy}}{Latest Updates: Economy}}\label{latest-updates-economy}}

\href{https://www.nytimes.com/live/2020/08/04/business/stock-market-today-coronavirus?action=click\&pgtype=Article\&state=default\&region=MAIN_CONTENT_1\&context=storylines_live_updates\#fox-corporations-plunging-profit-is-cushioned-by-fox-news}{10h
ago}

\href{https://www.nytimes.com/live/2020/08/04/business/stock-market-today-coronavirus?action=click\&pgtype=Article\&state=default\&region=MAIN_CONTENT_1\&context=storylines_live_updates\#fox-corporations-plunging-profit-is-cushioned-by-fox-news}{Fox
Corporation's plunging profit is cushioned by Fox News.}

\href{https://www.nytimes.com/live/2020/08/04/business/stock-market-today-coronavirus?action=click\&pgtype=Article\&state=default\&region=MAIN_CONTENT_1\&context=storylines_live_updates\#trading-in-kodak-shares-comes-under-scrutiny}{11h
ago}

\href{https://www.nytimes.com/live/2020/08/04/business/stock-market-today-coronavirus?action=click\&pgtype=Article\&state=default\&region=MAIN_CONTENT_1\&context=storylines_live_updates\#trading-in-kodak-shares-comes-under-scrutiny}{Trading
in Kodak shares comes under scrutiny.}

\href{https://www.nytimes.com/live/2020/08/04/business/stock-market-today-coronavirus?action=click\&pgtype=Article\&state=default\&region=MAIN_CONTENT_1\&context=storylines_live_updates\#disney-lost-4-7-billion-last-quarter-but-its-newest-business-was-a-big-hit}{12h
ago}

\href{https://www.nytimes.com/live/2020/08/04/business/stock-market-today-coronavirus?action=click\&pgtype=Article\&state=default\&region=MAIN_CONTENT_1\&context=storylines_live_updates\#disney-lost-4-7-billion-last-quarter-but-its-newest-business-was-a-big-hit}{Disney
lost \$4.7 billion last quarter, but its newest business was a big hit.}

\href{https://www.nytimes.com/live/2020/08/04/business/stock-market-today-coronavirus?action=click\&pgtype=Article\&state=default\&region=MAIN_CONTENT_1\&context=storylines_live_updates}{See
more updates}

More live coverage:
\href{https://www.nytimes.com/2020/08/04/world/coronavirus-cases.html?action=click\&pgtype=Article\&state=default\&region=MAIN_CONTENT_1\&context=storylines_live_updates}{Global}

One of the speakers in the video, who identified herself as Dr. Stella
Immanuel, said, ``You don't need masks'' to prevent spread of the
coronavirus. She also claimed to be treating hundreds of patients
infected with coronavirus with hydroxychloroquine, and asserted that it
was an effective treatment. The claims have been repeatedly disputed by
the medical establishment.

President Trump repeatedly promoted hydroxychloroquine, a malaria drug,
in the early months of the crisis. In June, he said he was taking it
himself. But that same month, the Food and Drug Administration
\href{https://www.fda.gov/media/138945/download}{revoked emergency
authorization} for the drug for Covid-19 patients and said it was
``unlikely to be effective'' and carried potential risks. The National
Institutes of Health
\href{https://www.nytimes.com/2020/06/20/health/hydroxychloroquine-coronavirus-trial.html}{halted
clinical trials of the drug}.

In addition, studies have repeatedly shown that masks are effective in
curbing the spread of the coronavirus.

The trajectory of Monday's video mirrored that of
``\href{https://www.nytimes.com/2020/05/20/technology/plandemic-movie-youtube-facebook-coronavirus.html}{Plandemic},''
a 26-minute slickly produced narration that spread widely in May and
falsely claimed that a shadowy cabal of elites was using the virus and a
potential vaccine to profit and gain power. In just over a week,
``Plandemic'' was viewed more than eight million times on YouTube,
Facebook, Twitter and Instagram before it was taken down.

But the video posted Monday had more views than ``Plandemic'' within
hours of being posted online, even though it was removed much faster. At
least one version of the video, viewed by The Times on Facebook, was
watched over 16 million times.

Facebook, YouTube, and Twitter deleted several versions of the video on
Monday night. All three companies said the video violated their policies
on sharing misinformation related to the coronavirus.

On Tuesday morning,
\href{https://www.nytimes.com/live/2020/07/28/business/stock-market-today-coronavirus/twitter-limits-donald-trump-jrs-account-after-he-shares-virus-misinformation}{Twitter
also took action} against Donald Trump Jr. after he shared a link to the
video. A spokesman for Twitter said the company had ordered Mr. Trump to
delete the misleading tweet and said it would ``limit some account
functionality for 12 hours.'' Twitter took a similar action against
Kelli Ward, the Arizona Republican Party chairwoman, who also tweeted
the video.

No action was taken against the president, who retweeted multiple clips
of the same video to his 84.2 million followers Monday night. The
original posts have since been removed.

When asked about the video on Tuesday, Mr. Trump continued to defend the
doctors involved and the treatments they are backing.

``For some reason the internet wanted to take them down and took them
off,'' the president said. ``I think they are very respected doctors.
There was a woman who was spectacular in her statements about it, that
she's had tremendous success with it and they took her voice off. I
don't know why they took her off. Maybe they had a good reason, maybe
they didn't.''

Facebook and YouTube did not answer questions about multiple versions of
the video that remained online on Tuesday afternoon. Twitter said it was
``continuing to take action on new and existing tweets with the video.''

The members of the group behind Monday's video say they are physicians
treating patients infected with the coronavirus. But it was unclear
where many of them practice medicine or how many patients they had
actually seen. As early as May, anti-Obamacare conservative activists
called the Tea Party Patriots Action
\href{https://news.bloomberglaw.com/health-law-and-business/hospitals-doctors-get-conservatives-push-for-elective-care}{reportedly}
worked with some of them to advocate loosening states' restrictions on
elective surgeries and nonemergency care. On July 15, the group
registered a website called ``America's Frontline Doctors,''
\href{https://whois.domaintools.com/americasfrontlinedoctors.com}{domain
registration records show}.

One of the first copies of the video that appeared on Monday was posted
to the Tea Party Patriots' YouTube channel, alongside other videos
featuring the members of ``America's Frontline Doctors.''

The doctors have also been
\href{https://twitter.com/daveyalba/status/1287933609433804802}{promoted}
by conservatives like Brent Bozell, founder of the Media Research
Center, a nonprofit media organization.

Advertisement

\protect\hyperlink{after-bottom}{Continue reading the main story}

\hypertarget{site-index}{%
\subsection{Site Index}\label{site-index}}

\hypertarget{site-information-navigation}{%
\subsection{Site Information
Navigation}\label{site-information-navigation}}

\begin{itemize}
\tightlist
\item
  \href{https://help.nytimes.com/hc/en-us/articles/115014792127-Copyright-notice}{©~2020~The
  New York Times Company}
\end{itemize}

\begin{itemize}
\tightlist
\item
  \href{https://www.nytco.com/}{NYTCo}
\item
  \href{https://help.nytimes.com/hc/en-us/articles/115015385887-Contact-Us}{Contact
  Us}
\item
  \href{https://www.nytco.com/careers/}{Work with us}
\item
  \href{https://nytmediakit.com/}{Advertise}
\item
  \href{http://www.tbrandstudio.com/}{T Brand Studio}
\item
  \href{https://www.nytimes.com/privacy/cookie-policy\#how-do-i-manage-trackers}{Your
  Ad Choices}
\item
  \href{https://www.nytimes.com/privacy}{Privacy}
\item
  \href{https://help.nytimes.com/hc/en-us/articles/115014893428-Terms-of-service}{Terms
  of Service}
\item
  \href{https://help.nytimes.com/hc/en-us/articles/115014893968-Terms-of-sale}{Terms
  of Sale}
\item
  \href{https://spiderbites.nytimes.com}{Site Map}
\item
  \href{https://help.nytimes.com/hc/en-us}{Help}
\item
  \href{https://www.nytimes.com/subscription?campaignId=37WXW}{Subscriptions}
\end{itemize}
