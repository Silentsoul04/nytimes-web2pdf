Sections

SEARCH

\protect\hyperlink{site-content}{Skip to
content}\protect\hyperlink{site-index}{Skip to site index}

\href{https://www.nytimes.com/section/politics}{Politics}

\href{https://myaccount.nytimes.com/auth/login?response_type=cookie\&client_id=vi}{}

\href{https://www.nytimes.com/section/todayspaper}{Today's Paper}

\href{/section/politics}{Politics}\textbar{}Here Are the Differences
Between the House and Senate Coronavirus Relief Bills

\url{https://nyti.ms/3hOaZsR}

\begin{itemize}
\item
\item
\item
\item
\item
\end{itemize}

\href{https://www.nytimes.com/news-event/coronavirus?action=click\&pgtype=Article\&state=default\&region=TOP_BANNER\&context=storylines_menu}{The
Coronavirus Outbreak}

\begin{itemize}
\tightlist
\item
  live\href{https://www.nytimes.com/2020/08/01/world/coronavirus-covid-19.html?action=click\&pgtype=Article\&state=default\&region=TOP_BANNER\&context=storylines_menu}{Latest
  Updates}
\item
  \href{https://www.nytimes.com/interactive/2020/us/coronavirus-us-cases.html?action=click\&pgtype=Article\&state=default\&region=TOP_BANNER\&context=storylines_menu}{Maps
  and Cases}
\item
  \href{https://www.nytimes.com/interactive/2020/science/coronavirus-vaccine-tracker.html?action=click\&pgtype=Article\&state=default\&region=TOP_BANNER\&context=storylines_menu}{Vaccine
  Tracker}
\item
  \href{https://www.nytimes.com/interactive/2020/07/29/us/schools-reopening-coronavirus.html?action=click\&pgtype=Article\&state=default\&region=TOP_BANNER\&context=storylines_menu}{What
  School May Look Like}
\item
  \href{https://www.nytimes.com/live/2020/07/31/business/stock-market-today-coronavirus?action=click\&pgtype=Article\&state=default\&region=TOP_BANNER\&context=storylines_menu}{Economy}
\end{itemize}

Advertisement

\protect\hyperlink{after-top}{Continue reading the main story}

Supported by

\protect\hyperlink{after-sponsor}{Continue reading the main story}

\hypertarget{here-are-the-differences-between-the-house-and-senate-coronavirus-relief-bills}{%
\section{Here Are the Differences Between the House and Senate
Coronavirus Relief
Bills}\label{here-are-the-differences-between-the-house-and-senate-coronavirus-relief-bills}}

Lawmakers will have to bridge significant policy gaps to reach an
election-year agreement on how to best provide relief to businesses and
families still reeling from the pandemic.

\includegraphics{https://static01.nyt.com/images/2020/07/27/us/politics/27dc-virus-explainer-sub/merlin_175014996_2f21e529-b224-4294-91e2-c772debcb1a2-articleLarge.jpg?quality=75\&auto=webp\&disable=upscale}

\href{https://www.nytimes.com/by/emily-cochrane}{\includegraphics{https://static01.nyt.com/images/2018/11/28/multimedia/author-emily-cochrane/author-emily-cochrane-thumbLarge-v3.png}}

By \href{https://www.nytimes.com/by/emily-cochrane}{Emily Cochrane}

\begin{itemize}
\item
  July 28, 2020
\item
  \begin{itemize}
  \item
  \item
  \item
  \item
  \item
  \end{itemize}
\end{itemize}

WASHINGTON --- With the economic and public health toll of the
\href{https://www.nytimes.com/interactive/2020/us/coronavirus-us-cases.html}{coronavirus
pandemic continuing to mount across the country}, members of Congress
are debating another round of federal relief for individuals and
businesses.

But it remains unclear how Democrats and Republicans will reconcile
their vastly different proposals. They are staring down a tight
deadline, with tens of millions of Americans slated to
\href{https://www.nytimes.com/2020/07/21/business/economy/coronavirus-unemployment-benefits.html}{lose
their enhanced jobless aid} this week.

Senate Republicans and administration officials on Monday unveiled a \$1
trillion proposal, narrowly tailored to Republican priorities. It
includes slashing by two-thirds the \$600-per-week unemployment payments
that workers have received since April and providing tax cuts and
liability protections for businesses.

House Democrats in May approved a
\href{https://www.congress.gov/116/bills/hr6800/BILLS-116hr6800ih.pdf}{\$3
trillion relief package} that amounts to their opening offer: a sweeping
measure that contains a number of Democratic priorities, including an
extension of the jobless aid, nearly \$200 billion for rental assistance
and mortgage relief, \$3.6 billion to bolster election security and
additional aid for food assistance.

Speaker Nancy Pelosi of California has said that she plans to fight for
even more funding, particularly for schools, in negotiations with
Republicans. Senator Mitch McConnell of Kentucky, the majority leader,
has warned against letting the price tag rise beyond \$1 trillion,
particularly as many Republicans question the merits of approving any
additional aid.

\hypertarget{latest-updates-global-coronavirus-outbreak}{%
\section{\texorpdfstring{\href{https://www.nytimes.com/2020/08/01/world/coronavirus-covid-19.html?action=click\&pgtype=Article\&state=default\&region=MAIN_CONTENT_1\&context=storylines_live_updates}{Latest
Updates: Global Coronavirus
Outbreak}}{Latest Updates: Global Coronavirus Outbreak}}\label{latest-updates-global-coronavirus-outbreak}}

Updated 2020-08-02T05:48:45.291Z

\begin{itemize}
\tightlist
\item
  \href{https://www.nytimes.com/2020/08/01/world/coronavirus-covid-19.html?action=click\&pgtype=Article\&state=default\&region=MAIN_CONTENT_1\&context=storylines_live_updates\#link-34047410}{The
  U.S. reels as July cases more than double the total of any other
  month.}
\item
  \href{https://www.nytimes.com/2020/08/01/world/coronavirus-covid-19.html?action=click\&pgtype=Article\&state=default\&region=MAIN_CONTENT_1\&context=storylines_live_updates\#link-780ec966}{Top
  U.S. officials work to break an impasse over the federal jobless
  benefit.}
\item
  \href{https://www.nytimes.com/2020/08/01/world/coronavirus-covid-19.html?action=click\&pgtype=Article\&state=default\&region=MAIN_CONTENT_1\&context=storylines_live_updates\#link-25930521}{Thousands
  in Berlin protest Germany's coronavirus measures.}
\end{itemize}

\href{https://www.nytimes.com/2020/08/01/world/coronavirus-covid-19.html?action=click\&pgtype=Article\&state=default\&region=MAIN_CONTENT_1\&context=storylines_live_updates}{See
more updates}

More live coverage:
\href{https://www.nytimes.com/live/2020/07/31/business/stock-market-today-coronavirus?action=click\&pgtype=Article\&state=default\&region=MAIN_CONTENT_1\&context=storylines_live_updates}{Markets}

While administration officials have floated the prospect of speeding
through a short-term, narrow measure to address the looming expiration
of unemployment benefits, liability protections and funding for schools,
Democrats have panned that suggestion in favor of a comprehensive
package.

Here are some of the main points of contrast that are likely to emerge
as sticking points.

\begin{center}\rule{0.5\linewidth}{\linethickness}\end{center}

\hypertarget{democrats-want-to-extend-600-weekly-jobless-payments-while-republicans-want-to-slash-them}{%
\subsection{Democrats want to extend \$600 weekly jobless payments,
while Republicans want to slash
them.}\label{democrats-want-to-extend-600-weekly-jobless-payments-while-republicans-want-to-slash-them}}

The \$2.2 trillion stimulus law added a \$600-per-week supplement for
those on unemployment insurance, but conservatives have argued that it
discourages people from returning to work in certain states, because it
exceeds their normal wages. The House bill would extend the full benefit
through January, while the
\href{https://www.nytimes.com/2020/06/24/us/politics/senate-police-bill.html}{Senate
measure} would severely curtail it, scaling it back to \$200 per week.
The lump sum would eventually be replaced with a newly calculated
benefit that, when combined with state benefits, would be
\href{https://www.nytimes.com/2020/07/23/business/economy/unemployment-benefits.html}{capped
at 70 percent of a worker's prior income}.

\hypertarget{republicans-want-liability-shields-for-businesses-while-democrats-are-pressing-for-protections-for-workers}{%
\subsection{Republicans want liability shields for businesses, while
Democrats are pressing for protections for
workers.}\label{republicans-want-liability-shields-for-businesses-while-democrats-are-pressing-for-protections-for-workers}}

House Democrats are again pushing for the Occupational Safety and Health
Administration to establish an enforceable standard, based on guidance
from top federal health agencies, for workplaces to develop
infection-control plans. The House bill would also prevent employers
from retaliating against any employee who reports workplace violations.

In the Senate, Mr. McConnell has repeatedly said that he views
strengthening liability protections for businesses, schools and
hospitals that remain open during the pandemic as a prerequisite for any
aid bill. The Republican proposal would establish a liability shield for
businesses, schools and hospitals from facing claims over episodes
related to the coronavirus.

\hypertarget{democrats-would-allocate-1-trillion-for-state-and-local-governments-republicans-left-them-out-entirely}{%
\subsection{Democrats would allocate \$1 trillion for state and local
governments. Republicans left them out
entirely.}\label{democrats-would-allocate-1-trillion-for-state-and-local-governments-republicans-left-them-out-entirely}}

Funding for state, local and tribal governments is the centerpiece of
the legislation House Democrats approved in May. Democrats argue that
governments will need another major infusion of relief to keep essential
workers on payrolls and make up for the loss of revenue after decreased
tourism and spending during the pandemic.

The bill unveiled by Senate Republicans does not have any aid
specifically set aside for state, local and tribal governments, though
it grants more flexibility for how states spent previously allocated
funds. Several conservative lawmakers note that some of the money
allocated in the March stimulus law has not yet been spent. Others have
warned against states using the coronavirus relief to make up for
pre-existing debt and expenses.

\hypertarget{both-proposals-would-provide-for-another-round-of-direct-payments-to-american-families}{%
\subsection{Both proposals would provide for another round of direct
payments to American
families.}\label{both-proposals-would-provide-for-another-round-of-direct-payments-to-american-families}}

Both proposals would again allocate another round of \$1,200 direct
payments to American families, duplicating a provision in the stimulus
law enacted in March that would phase out the amount of money for
individual incomes above \$75,000.

\href{https://www.nytimes.com/news-event/coronavirus?action=click\&pgtype=Article\&state=default\&region=MAIN_CONTENT_3\&context=storylines_faq}{}

\hypertarget{the-coronavirus-outbreak-}{%
\subsubsection{The Coronavirus Outbreak
›}\label{the-coronavirus-outbreak-}}

\hypertarget{frequently-asked-questions}{%
\paragraph{Frequently Asked
Questions}\label{frequently-asked-questions}}

Updated July 27, 2020

\begin{itemize}
\item ~
  \hypertarget{should-i-refinance-my-mortgage}{%
  \paragraph{Should I refinance my
  mortgage?}\label{should-i-refinance-my-mortgage}}

  \begin{itemize}
  \tightlist
  \item
    \href{https://www.nytimes.com/article/coronavirus-money-unemployment.html?action=click\&pgtype=Article\&state=default\&region=MAIN_CONTENT_3\&context=storylines_faq}{It
    could be a good idea,} because mortgage rates have
    \href{https://www.nytimes.com/2020/07/16/business/mortgage-rates-below-3-percent.html?action=click\&pgtype=Article\&state=default\&region=MAIN_CONTENT_3\&context=storylines_faq}{never
    been lower.} Refinancing requests have pushed mortgage applications
    to some of the highest levels since 2008, so be prepared to get in
    line. But defaults are also up, so if you're thinking about buying a
    home, be aware that some lenders have tightened their standards.
  \end{itemize}
\item ~
  \hypertarget{what-is-school-going-to-look-like-in-september}{%
  \paragraph{What is school going to look like in
  September?}\label{what-is-school-going-to-look-like-in-september}}

  \begin{itemize}
  \tightlist
  \item
    It is unlikely that many schools will return to a normal schedule
    this fall, requiring the grind of
    \href{https://www.nytimes.com/2020/06/05/us/coronavirus-education-lost-learning.html?action=click\&pgtype=Article\&state=default\&region=MAIN_CONTENT_3\&context=storylines_faq}{online
    learning},
    \href{https://www.nytimes.com/2020/05/29/us/coronavirus-child-care-centers.html?action=click\&pgtype=Article\&state=default\&region=MAIN_CONTENT_3\&context=storylines_faq}{makeshift
    child care} and
    \href{https://www.nytimes.com/2020/06/03/business/economy/coronavirus-working-women.html?action=click\&pgtype=Article\&state=default\&region=MAIN_CONTENT_3\&context=storylines_faq}{stunted
    workdays} to continue. California's two largest public school
    districts --- Los Angeles and San Diego --- said on July 13, that
    \href{https://www.nytimes.com/2020/07/13/us/lausd-san-diego-school-reopening.html?action=click\&pgtype=Article\&state=default\&region=MAIN_CONTENT_3\&context=storylines_faq}{instruction
    will be remote-only in the fall}, citing concerns that surging
    coronavirus infections in their areas pose too dire a risk for
    students and teachers. Together, the two districts enroll some
    825,000 students. They are the largest in the country so far to
    abandon plans for even a partial physical return to classrooms when
    they reopen in August. For other districts, the solution won't be an
    all-or-nothing approach.
    \href{https://bioethics.jhu.edu/research-and-outreach/projects/eschool-initiative/school-policy-tracker/}{Many
    systems}, including the nation's largest, New York City, are
    devising
    \href{https://www.nytimes.com/2020/06/26/us/coronavirus-schools-reopen-fall.html?action=click\&pgtype=Article\&state=default\&region=MAIN_CONTENT_3\&context=storylines_faq}{hybrid
    plans} that involve spending some days in classrooms and other days
    online. There's no national policy on this yet, so check with your
    municipal school system regularly to see what is happening in your
    community.
  \end{itemize}
\item ~
  \hypertarget{is-the-coronavirus-airborne}{%
  \paragraph{Is the coronavirus
  airborne?}\label{is-the-coronavirus-airborne}}

  \begin{itemize}
  \tightlist
  \item
    The coronavirus
    \href{https://www.nytimes.com/2020/07/04/health/239-experts-with-one-big-claim-the-coronavirus-is-airborne.html?action=click\&pgtype=Article\&state=default\&region=MAIN_CONTENT_3\&context=storylines_faq}{can
    stay aloft for hours in tiny droplets in stagnant air}, infecting
    people as they inhale, mounting scientific evidence suggests. This
    risk is highest in crowded indoor spaces with poor ventilation, and
    may help explain super-spreading events reported in meatpacking
    plants, churches and restaurants.
    \href{https://www.nytimes.com/2020/07/06/health/coronavirus-airborne-aerosols.html?action=click\&pgtype=Article\&state=default\&region=MAIN_CONTENT_3\&context=storylines_faq}{It's
    unclear how often the virus is spread} via these tiny droplets, or
    aerosols, compared with larger droplets that are expelled when a
    sick person coughs or sneezes, or transmitted through contact with
    contaminated surfaces, said Linsey Marr, an aerosol expert at
    Virginia Tech. Aerosols are released even when a person without
    symptoms exhales, talks or sings, according to Dr. Marr and more
    than 200 other experts, who
    \href{https://academic.oup.com/cid/article/doi/10.1093/cid/ciaa939/5867798}{have
    outlined the evidence in an open letter to the World Health
    Organization}.
  \end{itemize}
\item ~
  \hypertarget{what-are-the-symptoms-of-coronavirus}{%
  \paragraph{What are the symptoms of
  coronavirus?}\label{what-are-the-symptoms-of-coronavirus}}

  \begin{itemize}
  \tightlist
  \item
    Common symptoms
    \href{https://www.nytimes.com/article/symptoms-coronavirus.html?action=click\&pgtype=Article\&state=default\&region=MAIN_CONTENT_3\&context=storylines_faq}{include
    fever, a dry cough, fatigue and difficulty breathing or shortness of
    breath.} Some of these symptoms overlap with those of the flu,
    making detection difficult, but runny noses and stuffy sinuses are
    less common.
    \href{https://www.nytimes.com/2020/04/27/health/coronavirus-symptoms-cdc.html?action=click\&pgtype=Article\&state=default\&region=MAIN_CONTENT_3\&context=storylines_faq}{The
    C.D.C. has also} added chills, muscle pain, sore throat, headache
    and a new loss of the sense of taste or smell as symptoms to look
    out for. Most people fall ill five to seven days after exposure, but
    symptoms may appear in as few as two days or as many as 14 days.
  \end{itemize}
\item ~
  \hypertarget{does-asymptomatic-transmission-of-covid-19-happen}{%
  \paragraph{Does asymptomatic transmission of Covid-19
  happen?}\label{does-asymptomatic-transmission-of-covid-19-happen}}

  \begin{itemize}
  \tightlist
  \item
    So far, the evidence seems to show it does. A widely cited
    \href{https://www.nature.com/articles/s41591-020-0869-5}{paper}
    published in April suggests that people are most infectious about
    two days before the onset of coronavirus symptoms and estimated that
    44 percent of new infections were a result of transmission from
    people who were not yet showing symptoms. Recently, a top expert at
    the World Health Organization stated that transmission of the
    coronavirus by people who did not have symptoms was ``very rare,''
    \href{https://www.nytimes.com/2020/06/09/world/coronavirus-updates.html?action=click\&pgtype=Article\&state=default\&region=MAIN_CONTENT_3\&context=storylines_faq\#link-1f302e21}{but
    she later walked back that statement.}
  \end{itemize}
\end{itemize}

But the Democratic proposal would allow undocumented immigrants to
receive money, undoing language that prohibited payments to anyone who
filed taxes jointly with someone who used an Individual Taxpayer
Identification Number, a common substitute for a Social Security number.
That number is used mostly by immigrants without legal status.

Democrats would also increase the amount of money per child to \$1,200
for up to three children per family. The Republican proposal would
maintain the \$500 amount set in the first stimulus, but also allow
adult dependents to qualify.

\hypertarget{the-republican-proposal-would-condition-some-school-funding-on-reopening-while-democrats-would-not}{%
\subsection{The Republican proposal would condition some school funding
on reopening, while Democrats would
not.}\label{the-republican-proposal-would-condition-some-school-funding-on-reopening-while-democrats-would-not}}

Because the Democratic measure was approved in May, before schools were
contemplating
\href{https://www.nytimes.com/2020/07/11/health/coronavirus-schools-reopen.html}{how
to begin another academic year safely} with the virus still surging
across the country, Ms. Pelosi has said she will now push for more than
the \$100 billion included in the package for education.

The Republican bill would allocate \$105 billion for states to put
toward schools. Of that money, \$5 billion would be set aside for
governors to use at their discretion, and \$30 billion would be set
aside for colleges and universities. The remaining \$70 billion would go
to elementary and secondary schools, with two-thirds of the relief
designated for schools that have begun reopening and holding in-person
classes.

Democrats have so far balked at the prospect of tying federal relief to
reopening.

Advertisement

\protect\hyperlink{after-bottom}{Continue reading the main story}

\hypertarget{site-index}{%
\subsection{Site Index}\label{site-index}}

\hypertarget{site-information-navigation}{%
\subsection{Site Information
Navigation}\label{site-information-navigation}}

\begin{itemize}
\tightlist
\item
  \href{https://help.nytimes.com/hc/en-us/articles/115014792127-Copyright-notice}{©~2020~The
  New York Times Company}
\end{itemize}

\begin{itemize}
\tightlist
\item
  \href{https://www.nytco.com/}{NYTCo}
\item
  \href{https://help.nytimes.com/hc/en-us/articles/115015385887-Contact-Us}{Contact
  Us}
\item
  \href{https://www.nytco.com/careers/}{Work with us}
\item
  \href{https://nytmediakit.com/}{Advertise}
\item
  \href{http://www.tbrandstudio.com/}{T Brand Studio}
\item
  \href{https://www.nytimes.com/privacy/cookie-policy\#how-do-i-manage-trackers}{Your
  Ad Choices}
\item
  \href{https://www.nytimes.com/privacy}{Privacy}
\item
  \href{https://help.nytimes.com/hc/en-us/articles/115014893428-Terms-of-service}{Terms
  of Service}
\item
  \href{https://help.nytimes.com/hc/en-us/articles/115014893968-Terms-of-sale}{Terms
  of Sale}
\item
  \href{https://spiderbites.nytimes.com}{Site Map}
\item
  \href{https://help.nytimes.com/hc/en-us}{Help}
\item
  \href{https://www.nytimes.com/subscription?campaignId=37WXW}{Subscriptions}
\end{itemize}
