Sections

SEARCH

\protect\hyperlink{site-content}{Skip to
content}\protect\hyperlink{site-index}{Skip to site index}

\href{https://www.nytimes.com/section/world/europe}{Europe}

\href{https://myaccount.nytimes.com/auth/login?response_type=cookie\&client_id=vi}{}

\href{https://www.nytimes.com/section/todayspaper}{Today's Paper}

\href{/section/world/europe}{Europe}\textbar{}British-Australian
Academic Jailed in Iran Is Moved to Remote Prison

\url{https://nyti.ms/3333j1K}

\begin{itemize}
\item
\item
\item
\item
\item
\end{itemize}

Advertisement

\protect\hyperlink{after-top}{Continue reading the main story}

Supported by

\protect\hyperlink{after-sponsor}{Continue reading the main story}

\hypertarget{british-australian-academic-jailed-in-iran-is-moved-to-remote-prison}{%
\section{British-Australian Academic Jailed in Iran Is Moved to Remote
Prison}\label{british-australian-academic-jailed-in-iran-is-moved-to-remote-prison}}

Kylie Moore-Gilbert, who has denied charges of espionage, is now in a
facility where many have been infected with the coronavirus, rights
activists say.

\href{https://www.nytimes.com/by/elian-peltier}{\includegraphics{https://static01.nyt.com/images/2019/07/03/reader-center/author-elian-peltier/165383d8b7284129a185b6ca96e2a52e-thumbLarge.png}}

By \href{https://www.nytimes.com/by/elian-peltier}{Elian Peltier}

\begin{itemize}
\item
  July 28, 2020
\item
  \begin{itemize}
  \item
  \item
  \item
  \item
  \item
  \end{itemize}
\end{itemize}

LONDON --- A British-Australian academic serving a 10-year sentence in
Iran for espionage has been moved to a remote prison south of Tehran
that is said to be riddled with coronavirus cases, according to rights
activists, raising further concerns about her deteriorating health.

The academic, Kylie Moore-Gilbert, a Cambridge-educated professor in
Islamic studies at the University of Melbourne, was arrested in 2018 at
the Tehran airport as she tried to leave Iran after a conference. Her
detention was
\href{https://www.nytimes.com/2019/09/11/world/australia/australians-detained-iran.html?rref=collection\%2Fbyline\%2Fmegan-specia\&action=click\&contentCollection=undefined\&region=stream\&module=stream_unit\&version=latest\&contentPlacement=1\&pgtype=collection}{publicly
confirmed a year later} by the Australian authorities.

Ms. Moore-Gilbert has strongly denounced the charges and maintains her
innocence. She was tried in secret and had been detained for the past
two years at Evin prison in Tehran, where friends say she was often
forced to sleep on the floor and sent to solitary confinement.

She was moved on Friday to Qarchak, a notorious and isolated women's
detention facility southeast of the capital, according to Reza Khandan,
an Iranian rights activist who said he spoke to Ms. Moore-Gilbert by
telephone.

``I cannot eat anything,'' Ms. Moore-Gilbert said, according to Mr.
Khandan, who wrote about the conversation on Facebook. ``I don't know,
I'm so disappointed,'' he quoted her as saying. ``I'm so very
depressed.''

Image

Kylie Moore-GilbertCredit...Australian Department of Foreign Affairs and
Trade, via Shutterstock

Dozens of women detained at the Qarchak prison have reportedly been
infected with the coronavirus in recent months, according to rights
activists and prisoners' relatives. Inmates at the prison have
\href{https://iranhumanrights.org/2019/08/prisoners-in-irans-gharchak-prison-for-women-protest-inhumane-living-conditions/}{described}
a lack of accessible drinking water, inedible meals, overcrowding and
insufficient access to medical treatment.

The State Department included the prison in a
\href{https://www.state.gov/report-to-congress-list-of-persons-who-are-responsible-for-or-complicit-in-certain-human-rights-abuses-in-iran/}{list
of entities it deems responsible} ``for extrajudicial killings, torture
or other gross violations of internationally recognized human rights.''

``It is known for unbearable conditions, including regular assaults and
inappropriate behavior of prison guards toward women,'' the State
Department
\href{https://www.state.gov/this-week-in-iran-policy-december-2-6/}{said
in a statement} released in December.

``It's not a positive sign to move her there,'' said Sanam Vakil, a
researcher on Iran at Chatham House, an international affairs research
institute based in London. ``They're looking to isolate her more, to put
more pressure on the Australian government for whatever they're seeking
to obtain,'' she added, referring to Ms. Moore-Gilbert. ``The big
question remains, what do they want?''

Iran has imprisoned dozens of foreign academics and dual citizens on
espionage or national security charges in recent years, with some of
them used as bargaining chips to obtain the repatriation of Iranian
citizens detained abroad. A
\href{https://www.nytimes.com/2020/03/21/world/middleeast/prisoner-swap-france-iran.html}{French
academic was released} in March as part of a prisoner swap.

Ms. Moore-Gilbert was at Evin prison for nearly two years along with
Fariba Adelkhah, a renowned French-Iranian academic who in May was
\href{https://www.nytimes.com/2020/05/16/world/middleeast/fariba-adelkah-iran.html}{sentenced
to six years in prison} on national security charges.

A British-Iranian prisoner whose case has made international headlines,
\href{https://www.nytimes.com/2020/05/20/world/europe/iran-uk-nazanin-zaghari-ratcliffe.html}{Nazanin
Zaghari-Ratcliffe}, was also held at Evin, but she was temporarily
released in May.

It is unusual for foreigners to be incarcerated at Qarchak prison, and
the reason for the transfer is unclear. But the move raised new concerns
for Ms. Moore-Gilbert's mental and physical health.

Ana-Diamond Aaba Atach, a Finnish-Iranian dual citizen who was
imprisoned in Evin for eight months in 2016, speculated that the
transfer of Ms. Moore-Gilbert allowed the Iranian authorities to show
how they could keep a firm hand on highly publicized cases.

``Prisoners describe Evin as a hotel in comparison to Qarchak,'' Ms.
Aaba Atach said. ``And Evin is a horrendous place.''

Inmates at Qarchak have said in testimonies collected by the New
York-based
\href{https://iranhumanrights.org/2019/08/prisoners-in-irans-gharchak-prison-for-women-protest-inhumane-living-conditions/}{Center
for Human Rights in Iran} that the facility did not have enough toilets
or beds, and that some inmates were denied medical treatment based on
their alleged crimes.

They also said that the food is barely edible, and that they had to buy
expensive bottled water because the water they were given was too salty
to be drinkable.

The coronavirus pandemic has worsened the detention conditions in
overcrowded Iranian prisons. Some 85,000 inmates were released in March
in an effort to fight the spread of the virus.

``Prisons were a huge concern in Iran in the early days of the
coronavirus outbreak,'' said Ms. Vakil of Chatham House. ``Since then,
we're back to not knowing things.''

Yet thousands more still behind bars have staged protests in recent
months, saying that they are not being protected from the coronavirus.
\href{https://www.amnesty.org/en/latest/news/2020/04/iran-prisoners-killed-by-security-forces-during-covid19-pandemic-protests/}{Amnesty
International} said in April that some 36 prisoners were feared to have
been killed by security forces in an effort to control the protests.

United Nations experts have urged the Iranian authorities to release
more dual citizens like Ms. Moore-Gilbert.

In letters smuggled out of Evin in 2019 and published by British news
outlets in January, Ms. Moore-Gilbert said she felt ``abandoned and
forgotten'' and described how her health had ``deteriorated
significantly.''

She accused the Islamic Revolutionary Guards Corps, which manages the
prison ward where she was detained, of ``playing an awful game with
me.''

And she again proclaimed her innocence.

``I am not a spy,'' she wrote. ``I have never been a spy, and I have no
interest to work for a spying organization in any country.''

Advertisement

\protect\hyperlink{after-bottom}{Continue reading the main story}

\hypertarget{site-index}{%
\subsection{Site Index}\label{site-index}}

\hypertarget{site-information-navigation}{%
\subsection{Site Information
Navigation}\label{site-information-navigation}}

\begin{itemize}
\tightlist
\item
  \href{https://help.nytimes.com/hc/en-us/articles/115014792127-Copyright-notice}{©~2020~The
  New York Times Company}
\end{itemize}

\begin{itemize}
\tightlist
\item
  \href{https://www.nytco.com/}{NYTCo}
\item
  \href{https://help.nytimes.com/hc/en-us/articles/115015385887-Contact-Us}{Contact
  Us}
\item
  \href{https://www.nytco.com/careers/}{Work with us}
\item
  \href{https://nytmediakit.com/}{Advertise}
\item
  \href{http://www.tbrandstudio.com/}{T Brand Studio}
\item
  \href{https://www.nytimes.com/privacy/cookie-policy\#how-do-i-manage-trackers}{Your
  Ad Choices}
\item
  \href{https://www.nytimes.com/privacy}{Privacy}
\item
  \href{https://help.nytimes.com/hc/en-us/articles/115014893428-Terms-of-service}{Terms
  of Service}
\item
  \href{https://help.nytimes.com/hc/en-us/articles/115014893968-Terms-of-sale}{Terms
  of Sale}
\item
  \href{https://spiderbites.nytimes.com}{Site Map}
\item
  \href{https://help.nytimes.com/hc/en-us}{Help}
\item
  \href{https://www.nytimes.com/subscription?campaignId=37WXW}{Subscriptions}
\end{itemize}
