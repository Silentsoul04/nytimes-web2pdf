Sections

SEARCH

\protect\hyperlink{site-content}{Skip to
content}\protect\hyperlink{site-index}{Skip to site index}

\href{https://www.nytimes.com/section/food}{Food}

\href{https://myaccount.nytimes.com/auth/login?response_type=cookie\&client_id=vi}{}

\href{https://www.nytimes.com/section/todayspaper}{Today's Paper}

\href{/section/food}{Food}\textbar{}What to Cook Right Now

\url{https://nyti.ms/334JGpP}

\begin{itemize}
\item
\item
\item
\item
\item
\end{itemize}

Advertisement

\protect\hyperlink{after-top}{Continue reading the main story}

Supported by

\protect\hyperlink{after-sponsor}{Continue reading the main story}

\href{/column/what-to-cook}{What to Cook}

\hypertarget{what-to-cook-right-now}{%
\section{What to Cook Right Now}\label{what-to-cook-right-now}}

\includegraphics{https://static01.nyt.com/images/2020/08/02/dining/yf-baked-feta/yf-baked-feta-articleLarge.jpg?quality=75\&auto=webp\&disable=upscale}

By \href{https://www.nytimes.com/by/sam-sifton}{Sam Sifton}

\begin{itemize}
\item
  July 29, 2020
\item
  \begin{itemize}
  \item
  \item
  \item
  \item
  \item
  \end{itemize}
\end{itemize}

Good morning. I've been cooking two or three squares a day for 133 days
now, surely the longest run of my cooking life, and if it's sometimes a
chore, I still thrill to the possibilities of an improvisatory session
--- a no-recipe recipe made with what's on hand. I love those
``recipes,'' which are really just outlines, notes that others can make
into dishes that become entirely their own.

So, for instance, my pal
\href{https://www.nytco.com/person/theodore-kim/}{Ted Kim}'s gochujang
spaghetti? He sent me the rough instructions: \emph{Thick spaghetti
cooked al dente + sesame oil + a touch of sugar + generous dollops of
gochujang + romaine lettuce cut into pieces + sesame seeds + pepper +
chopped scallion to garnish. Serve with kimchi if you have it.}

It occurred to me, you could make that dish hot, or make it hot, then
fridge it and serve it cold. You could sauté some ground pork for the
gochujang and make a kind of meat sauce with it, with a splash of soy
sauce, or oyster sauce. You could thin out that sauce with stock or
water. You could ribbon some zucchini in there in place of the lettuce.
You could swap out the lettuce for chopped kale. All of these would be
gochujang spaghetti. Go make the one you want. It's delicious every
time.

I get it, if that guidance is too vague. In these trying times, some
prefer the order of stepped instruction, the glory of a recipe that
holds your hand from beginning to end.

For them, then, for you: Yasmin Fahr's new recipe for
\href{https://cooking.nytimes.com/recipes/1021277-sheet-pan-baked-feta-with-broccolini-tomatoes-and-lemon}{sheet-pan
baked feta with broccolini, tomatoes and lemon} (above). Or the chef
Sean Sherman's slightly older one, for
\href{https://cooking.nytimes.com/recipes/1020563-salmon-with-crushed-blackberries-and-seaweed}{salmon
with crushed blackberries and seaweed}.

You could make Tejal Rao's recipe for
\href{https://cooking.nytimes.com/recipes/1020913-keema-spiced-ground-meat}{keema},
a sauté of spiced ground beef that you might accompany with her
\href{https://cooking.nytimes.com/recipes/1020909-gajjara-kosambari-carrot-salad}{gajjara
kosambari}, a simple carrot salad.

Here's Sarah DiGregorio's recipe for
\href{https://cooking.nytimes.com/recipes/1020818-pressure-cooker-red-beans-and-rice}{pressure
cooker red beans and rice}, a good way to cook quickly and without
heating the kitchen too much while you do. And Sue Li's new recipe for
\href{https://cooking.nytimes.com/recipes/1021228-chicken-and-celery-salad-with-wasabi-tahini-dressing}{chicken
and celery salad with wasabi-tahini dressing}, a worthy dinner and an
incredible sandwich filling at lunch.

Take a look, too, at David Tanis's
\href{https://cooking.nytimes.com/recipes/1019844-sauteed-scallops-with-crushed-peppercorns}{sautéed
scallops with crushed peppercorns}, at Mark Bittman's
\href{https://cooking.nytimes.com/recipes/11380-pasta-with-gorgonzola-and-arugula}{pasta
with Gorgonzola and arugula}, at Melissa Clark's
\href{https://cooking.nytimes.com/recipes/1016659-chilled-corn-soup-with-basil}{chilled
corn soup with basil}.

And recall that for many, Eid al-Adha, the Muslim holiday that honors
Abraham's willingness to sacrifice his son at God's command, begins
Thursday night. We have two new recipes for that, both written by
Nargisse Benkabbou: for
\href{https://cooking.nytimes.com/recipes/1021280-mrouzia-lamb-shanks}{mrouzia},
a Moroccan tagine of lamb shanks; and for
\href{https://cooking.nytimes.com/recipes/1021279-rose-and-almond-ghriba}{rose
and almond ghriba}, crackly-soft cookies of amazing delight. Perhaps you
can make one or both of those, as well. (Here are even
\href{https://cooking.nytimes.com/68861692-nyt-cooking/1688528-what-to-cook-for-eid-al-adha}{more
recipes for the holiday}.)

There are many thousands more recipes to consider cooking waiting for
you on \href{https://cooking.nytimes.com/}{NYT Cooking}. Simply
\href{https://www.nytimes.com/subscription/cooking.html?campaignId=6XQHR}{subscribe
today} to access them all, if you haven't already. I'm bold to ask
because subscriptions support our work. They allow it to continue.

And we will be standing by like lifeguards at the reopened community
pool should anything go wrong with your cooking or our technology. Just
write the team at
\href{mailto:cookingcare@nytimes.com}{\nolinkurl{cookingcare@nytimes.com}}
and someone will get back to you, I promise. (You can always escalate
matters by reaching out to me at
\href{mailto:foodeditor@nytimes.com}{\nolinkurl{foodeditor@nytimes.com}}.
I read every letter sent, and try to help where I can.)

Now, it's a far cry from chocolate chip cookies and discussion of the
very best sort of olive oil, but I think you ought to read
\href{https://www.gq.com/story/americas-best-wedding-band-the-sultans}{Daniel
Riley on the Sultans}, maybe the nation's best wedding band, in GQ.
Weddings! Remember them? It's exciting and fun, a reminder of the world
that was.

I'm late to it, and it's dark, dark, dark, but
``\href{https://www.imdb.com/title/tt2303687/}{Line of Duty},'' the BBC
police procedural you can stream all over the place, is just terrific.

Also old, also transporting:
``\href{https://bookshop.org/books/the-soho-press-book-of-80s-short-fiction/9781616955465}{The
Soho Press Book of '80s Short Fiction}.'' The library may not have it,
but used copies abound online.

Finally, it's Geddy Lee's birthday. He's 67. Here he is with Rush,
playing ``\href{https://www.youtube.com/watch?v=3GS3sQxRciY}{Time and
Motion}'' live in 1996. I'll be back on Friday.

Advertisement

\protect\hyperlink{after-bottom}{Continue reading the main story}

\hypertarget{site-index}{%
\subsection{Site Index}\label{site-index}}

\hypertarget{site-information-navigation}{%
\subsection{Site Information
Navigation}\label{site-information-navigation}}

\begin{itemize}
\tightlist
\item
  \href{https://help.nytimes.com/hc/en-us/articles/115014792127-Copyright-notice}{©~2020~The
  New York Times Company}
\end{itemize}

\begin{itemize}
\tightlist
\item
  \href{https://www.nytco.com/}{NYTCo}
\item
  \href{https://help.nytimes.com/hc/en-us/articles/115015385887-Contact-Us}{Contact
  Us}
\item
  \href{https://www.nytco.com/careers/}{Work with us}
\item
  \href{https://nytmediakit.com/}{Advertise}
\item
  \href{http://www.tbrandstudio.com/}{T Brand Studio}
\item
  \href{https://www.nytimes.com/privacy/cookie-policy\#how-do-i-manage-trackers}{Your
  Ad Choices}
\item
  \href{https://www.nytimes.com/privacy}{Privacy}
\item
  \href{https://help.nytimes.com/hc/en-us/articles/115014893428-Terms-of-service}{Terms
  of Service}
\item
  \href{https://help.nytimes.com/hc/en-us/articles/115014893968-Terms-of-sale}{Terms
  of Sale}
\item
  \href{https://spiderbites.nytimes.com}{Site Map}
\item
  \href{https://help.nytimes.com/hc/en-us}{Help}
\item
  \href{https://www.nytimes.com/subscription?campaignId=37WXW}{Subscriptions}
\end{itemize}
